\subsection{Green's formula}
	確率的偏微分を導入する.$(\phi,\psi)$が$\Phi$の$(W,t)$に関する確率的偏微分であるとは,
	\begin{itemize}
		\item $\phi$と$\psi$は適合過程であって,
		\item $\Phi_{st} = \Phi_{s0} + \int_{0}^{t} \phi_{sv}\ \partial_{2} W_{sv} + \int_{0}^{t} \psi_{sv}\ dv$
	\end{itemize}
	を満たすことである.$\Phi$がマルチンゲールなら$\psi$は消滅する.
	
	\begin{thm}[長方形に対するGreenの公式]
		$\Phi$を適合,可測,二乗可積分有界過程とし,$(\phi,\psi)$を$\Phi$の$(W,t)$に関する確率的偏微分とし,
		$\psi = 0$とし,$A$を長方形とする.このとき
		\begin{align}
			\int_{\partial A} \Phi\ \partial_{1}W = \int_{A} \Phi\ dW + \int_{A} \phi\ dJ.
		\end{align}
		同様に,$(\phi,\psi)$を$\Phi$の$(W,s)$に関する確率的偏微分とし,$\psi = 0$とすると,
		\begin{align}
			\int_{\partial A} \Phi\ \partial_{2}W = -\int_{A} \Phi\ dW - \int_{A} \phi\ dJ.
		\end{align}
	\end{thm}
	
	\begin{align}
		W_{st} = W_{s0} + \int_{0}^{t} \partial_{1}W_{sv}
	\end{align}
	なので,伊藤の公式より
	\begin{align}
		J_{st} = \frac{1}{2}\left({W_{st}}^{2} - st\right) - \int_{R_{st}} W\ dW
	\end{align}
	が得られる.
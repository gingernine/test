\subsection{Wiener Integral and Hermite Functionals}
	$\Set{W_{z}}{z \in T = [0,1]^{n}}$をWiener過程とし,
	$\Set{\mathscr{F}_{z}}{z \in T}$を$\sigma$-加法族の増大系とし,
	$\Set{W_{z},\mathscr{F}_{z}}{z \in T}$がマルチンゲールであるとする.
	$L^{2}(T)$を$\int_{T} \phi^{2}(\zeta)\ d\zeta < \infty$を満たす
	$T$上の実数値関数$\phi$の全体とする.
	$L^{2}(T)$の要素$\phi$に対して,Wiener積分$\int_{T} \phi(\zeta)\ W(d\zeta)$は
	以下の条件により定義される.
	
	\begin{description}
		\item[(a)] $\phi$が$\prod_{i=1}^{n} [0,z_{i})$の指示関数ならば
			$\int_{T} \phi(\zeta)\ W(d\zeta) = W_{z}$.
		
		\item[(b)] $\int_{T} [a\phi(\zeta) + b\psi(\zeta)]\ W(d\zeta)
			= a \int_{T} \phi(\zeta)\ W(d\zeta) + b \int_{T} \psi(\zeta)\ W(d\zeta)$.
		
		\item[(c)] $\int_{T} [\phi_{n}(\zeta) - \phi(\zeta)]^{2}\ d\zeta
			\longrightarrow 0 \Longrightarrow \int_{T} \phi_{n}(\zeta)\ W(d\zeta)
			\longrightarrow \int_{T} \phi(\zeta)\ W(d\zeta)$.
	\end{description}
	
	Wiener積分は重要な性質を持つ:
	\begin{align}
		&\cexp{\int_{T} \phi(\zeta)\ W(d\zeta)}{\mathscr{F}_{z}}
		= \int_{\zeta \prec z} \phi(\zeta)\ W(d\zeta), \\
		&\cexp{\left\{\int_{T} \phi(\zeta)\ W(d\zeta)\right\}^{2}}{\mathscr{F}_{z}}
		= \int_{\zeta \prec z} \phi^{2}(\zeta)\ W(d\zeta).
	\end{align}
	
	$M_{z} = \int_{\zeta \prec z} \phi(\zeta)\ W(d\zeta)$と定めると,
	$\Set{M_{z},\mathscr{F}_{z}}{z \in T}$はマルチンゲールである.さらに,
	連続なバージョンを取ることができる.
	\begin{align}
		M_{iz} = \int_{\zeta \prec z} \phi_{i}(\zeta)\ W(d\zeta),
		\quad i = 1,2,\cdot,m,
	\end{align}
	と定めれば
	\begin{align}
		\inprod<M_{i},M_{j}>_{z} = 
		\int_{\zeta \prec z} \phi_{i}(\zeta) \phi_{j}(\zeta)\ d\zeta = V_{ij}(z)
	\end{align}
	が成り立ち,従って任意の線型結合$\sum_{i=1}^{m} \alpha_{i} M_{iz}$はパス-独立なマルチンゲールである.伊藤の公式は
	\begin{align}
		f\left(M_{\theta(t)},\theta(t)\right) - f\left(M_{\theta(0)},\theta(0)\right)
		&= \sum_{i} \int_{0}^{t} f^{i}\left(M_{\theta(s)},\theta(s)\right)\ dM_{i\theta{s}} \\
		&\quad + \int_{0}^{t} \left[\frac{1}{2} \sum_{i,j} f^{ij}(u,z) \nabla V_{ij}(z)
		+ \nabla f(u,z)\right]_{\substack{z=\theta(s) \\ u=\theta(s)}} \cdot d\theta(s)
		\label{fom:Wiener_Integral_and_Hermite_Functionals_3}
	\end{align}
	となる.明らかに,$f(u,z)$が
	\begin{align}
		\frac{1}{2} \sum_{i,j} f^{ij}(u,z) \nabla V_{ij}(z) + \nabla f(u,z) = 0
		\label{fom:Wiener_Integral_and_Hermite_Functionals_1}
	\end{align}
	を満たすならば,$f(M_{z},z)$は局所マルチンゲールである.
	
	$f_{\alpha}(u,z)$を
	\begin{align}
		f_{\alpha}(u,z) = \exp{\left\{i \sum_{j}u_{j}\alpha_{j} 
		+ \frac{1}{2} \sum_{j,k}\alpha_{j}\alpha_{k}V_{jk}\right\}}
	\end{align}
	で定める.$\R^{m}$の任意の要素$\alpha$に対して$f_{\alpha}$が
	(\refeq{fom:Wiener_Integral_and_Hermite_Functionals_1})を満たすことは簡単に示される.
	ゆえに,$\Set{f_{\alpha}}{\alpha \in \R^{m}}$の要素の任意の線型結合は
	(\refeq{fom:Wiener_Integral_and_Hermite_Functionals_1})を満たし,
	線型結合の適切な収束列の極限も同式を満たす.
	具体例は
	\begin{align}
		f_{k}(u,z)
		= (-i)^{k_{1}+k_{2}+\cdots+k_{m}}\left.\left[
		\frac{\partial^{k_{1}+k_{2}+\cdots+k_{m}}}{\partial \alpha_{1}^{k_{1}}
		\partial \alpha_{2}^{k_{2}}\cdots\partial \alpha_{m}^{k_{m}}}f_{\alpha}(u,z)\right] \right| \alpha_{j} = 0,
		\quad 1 \leq j \leq m
		\label{fom:Wiener_Integral_and_Hermite_Functionals_2}
	\end{align}
	と
	\begin{align}
		f(u,z) = \int_{\R^{m}} f_{\alpha}(u,z)\ d\mu(\alpha)
	\end{align}
	を含む.ここで$\mu$は
	\begin{align}
		\int_{\R^{m}} \Norm{\alpha}{}^{2} f_{\alpha}(u,\zeta)\ d\mu(\zeta) < \infty,
		\quad u \in \R^{m},\ z \in T
	\end{align}
	を満たすBorel測度である.
	
	Cameron Martinによれば,Wiener過程$\Set{W_{z}}{z \in T}$の任意の
	二乗可積分汎関数はHermite汎関数の級数で表現できる.Hermite汎関数とは
	\begin{align}
		\prod_{\nu=1}^{m} H_{p_{\nu}} \int_{T} \phi_{\nu}(\zeta)\ W(d\zeta)
	\end{align}
	であり,この$H_{p}$はHermite多項式である.
	
	\begin{thm}
		任意のHermite汎関数に対して,$f(u,z),\ u \in \R^{m},\ z \in T$で,
		\begin{align}
			\prod_{\nu=1}^{m} H_{p_{\nu}} \int_{T} \phi_{\nu}(\zeta)\ W(d\zeta)
			&= f(M_{1}, 1) \\
			&= \sum_{i} \int_{0}^{1} f^{i}\left(M_{\theta(s)},\theta(s)\right)\ dM_{\theta(s)}
			+ f(0,0)
			\label{fom:Wiener_Integral_and_Hermite_Functionals_4}
		\end{align}
		を満たすものが取れる.ここで$\theta$は$0$と$1$を繋ぐ任意の増大パスで,
		\begin{align}
			M_{\nu z} = \int_{\zeta \prec z} \phi_{\nu}(\zeta)\ W(d\zeta)
		\end{align}
		である.
	\end{thm}
	
	\begin{sketch}
		各$p=(p_{1},p_{2},\cdots,p_{m})$に対して$\prod_{\nu=1}^{m} H_{p_{\nu}}(u_{\nu})$
		は$u=(u_{1},u_{2},\cdots,u_{m})$の$p$次の多項式である.ゆえに,
		(\refeq{fom:Wiener_Integral_and_Hermite_Functionals_2})を満たす
		$f_{k}$を用いて
		\begin{align}
			\prod_{\nu=1}^{m} H_{p_{\nu}}(u_{\nu})
			= \sum_{k \leq p} \beta_{p_{k}} f_{k}(u,1)
		\end{align}
		と書ける.従って,
		\begin{align}
			f(u,z) = \sum_{k \leq p} \beta_{p_{k}} f_{k}(u,z)
		\end{align}
		と(\refeq{fom:Wiener_Integral_and_Hermite_Functionals_2})を満たす関数$f$が取れて
		\begin{align}
			\prod_{\nu=1}^{m} H_{p_{\nu}}(u_{\nu}) = f(u,1).
		\end{align}
		(\refeq{fom:Wiener_Integral_and_Hermite_Functionals_4})は
		(\refeq{fom:Wiener_Integral_and_Hermite_Functionals_3})から直ちに従う.
		\QED
	\end{sketch}
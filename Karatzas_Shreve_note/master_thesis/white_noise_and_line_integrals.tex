\subsection{Brownian sheet}
	$(\Omega,\mathscr{F},P)$を確率空間とし,
	\begin{align}
		\R_{+}^{2} \defeq [0,\infty) \times [0,\infty)
	\end{align}
	とし,$W$を$\borel{\R_{+}^{2}} \times \Omega$上の写像で
	\begin{itemize}
		\item $A \longmapsto W_{A}(\omega)$は有限加法的
		\item $W_{A}$は$N(0,|A|)$に従う確率変数
		\item $A \cap B = \emptyset$ならば$W_{A}$と$W_{B}$は独立
	\end{itemize}
	を満たすものとする.この$W$を2パラメーター Wiener 過程,或いは Brownian sheet と呼ぶ.
	\begin{align}
		W_{st} \defeq W_{[0,s] \times [0,t]}
	\end{align}
	とすると,
	\begin{itemize}
		\item いかなる$t$に対しても$s \longmapsto W_{st}$はWiener過程
		\item いかなる$s$に対しても$t \longmapsto W_{st}$はWiener過程
	\end{itemize}
	である.
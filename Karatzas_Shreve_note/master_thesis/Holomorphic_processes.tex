\subsection{Holomorphic processes}
	$\Phi$がに対して,次を満たす$\phi$を導過程と呼ぶ.
	\begin{itemize}
		\item $\phi$は適合,可測
		\item $\R_{+}^{2}$の任意のコンパクト集合$K$に対して$\left\{E[{\phi_{z}}^{2}]\right\}_{z \in K}$は有界
		\item 任意の$z$及び$0$から$z$への階段路$\Gamma$に対して
			\begin{align}
				\Phi_{z} = \Phi_{0} + \int_{\Gamma} \phi\ \partial W.
			\end{align}
	\end{itemize}
	$\Phi$に対して導過程が取れるとき,$\Phi$を正則過程と呼ぶ.閉階段路上の$\phi$の積分は$0$である(3番目の条件).
	正則過程の構造は正則関数のものと似ている.正則過程は確率積分で定義されているのでマルチンゲールである.
	$W^{2}$や$W^{3}$は正則過程ではないが,そもそも$z^{n}$の確率的類似物は$W^{n}$ではない.
	$H_{n}(W_{z},z)$である.ここで$H_{n}$はHermite多項式である.つまり
	\begin{align}
		H_{n}(x,t) = \frac{(-1)^{n}}{n!} e^{x^{2}/2t} \frac{\partial^{n}}{\partial x^{n}} e^{-x^{2}/2t}
	\end{align}
	である.各$t$に対して$\{H_{n}(x,t)\}_{n \in \Natural}$は重量関数$e^{-x^{2}/2t} dx$に関して完全正規直交系をなす.
	$W_{st}$は$N(0,st)$に従う確率変数なので
	\begin{align}
		E\left[H_{n}(W_{st},st) H_{m}(W_{st},st)\right]
		= \begin{cases}
			0 & \mbox{if } n \neq m \\
			\displaystyle \frac{(st)^{n}}{n!} & \mbox{if } n = m
		\end{cases}
	\end{align}
	が成り立つ.
	\begin{align}
		\frac{\partial}{\partial x} H_{n} = H_{n-1}
	\end{align}
	と
	\begin{align}
		\frac{1}{2} \frac{\partial^{2}}{\partial t^{2}} H_{n} + \frac{\partial}{\partial t} H_{n} = 0
	\end{align}
	であることから,伊藤の公式より
	\begin{align}
		H_{n}(W_{st},st) &= \int_{0}^{s} H_{n-1}(W_{ut},ut)\ \partial_{1} W_{ut} \\
		&= \int_{0}^{t} H_{n-1}(W_{sv},sv)\ \partial_{2} W_{sv}
	\end{align}
	を得る.つまり$H_{n}(W_{st},st)$は正則過程であり,その導過程は$H_{n-1}(W_{st},st)$である.
	
	\begin{thm}
		$\Phi$を正則過程とし,$\Gamma$を閉階段路とするとき,
		\begin{align}
			\int_{\Gamma} \Phi\ \partial W = 0.
		\end{align}
	\end{thm}
	
	実際,$\Gamma$を長方形$A$の外周とすると,Greenの公式より
	\begin{align}
		\int_{\partial A} \Phi\ \partial W 
		= \int_{\partial A} \Phi\ \partial_{1} W + \int_{\partial A} \Phi\ \partial_{2} W 
		= 0
	\end{align}
	が成立する.$\Phi$を導過程とする正則過程$\Psi$は
	\begin{align}
		\Psi_{z} \defeq \int_{0}^{z} \Phi\ \partial W
	\end{align}
	で定められる.正則過程の積分は正則過程である.では正則過程の微分はどうであるか.
	
	\begin{thm}
		$\Phi$を正則過程とし,$\phi$をその導過程とする.このとき$\phi$は正則過程である.
	\end{thm}
	
	$\Phi$の$n$階導過程を$\Phi^{(n)}$と書くと,Taylorの定理を得る.
	
	\begin{thm}
		$\Phi$を正則過程とするとき
		\begin{align}
			\Phi_{st} = \sum_{n=0}^{\infty} \Phi_{0}^{(n)} H_{n}(W_{st},st).
		\end{align}
		右辺は,各$s,t$で$L^{2}$収束する.
	\end{thm}
	
	$\mathscr{F}_{0} = \{\emptyset,\Omega\}$なので$\Phi_{0}^{(n)}$は定数である.
	
	\begin{thm}
		$M$を二乗可積分マルチンゲールとする.$M$が$s$に関しても$t$に関しても確率的偏微分を有するならば,
		$M$は正則過程である.
	\end{thm}
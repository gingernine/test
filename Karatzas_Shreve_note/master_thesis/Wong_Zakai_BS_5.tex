\subsection{Stochastic Integrals}
	$T=[0,1]^{n}$とし,$\Set{W_{z},\mathscr{F}_{z}}{z \in T}$をWiener過程とする.
	適当な条件を満たすランダム関数$\phi$に対して
	\begin{align}
		I_{1}(\phi) = \int_{T} \phi(z)\ W(dz)
		\label{fom:Stochastic_Integrals_1}
	\end{align}
	の形の積分は,伊藤積分の一般化で定義出来る.Cairoliは$n=2$の場合で定義したが,
	それを任意の$n$に問題なく拡張することが出来る.
	まずは(\refeq{fom:Stochastic_Integrals_1})の形の積分を第一形確率積分として紹介する.
	以下に定義と性質のまとめた.
	
	$\phi(\omega,z)$は次の条件を満たす:
	\begin{description}
		\item[$H_{1}$] $\phi(\omega,z)$は$\mathscr{F} \otimes \mathscr{S}$-可測である.
			ここで$\mathscr{S}$は$T$のBorel集合族である.
			
		\item[$H_{2}$] $T$の各要素$z$に対して$\phi_{z}$は$\mathscr{F}_{z}$-可測である.
		
		\item[$H_{3}$] $\int_{T} E{\phi_{z}}^{2} < \infty$.
	\end{description}
	
	$\phi$を単関数とせよ.
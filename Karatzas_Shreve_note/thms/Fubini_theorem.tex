\section{Fubiniの定理}
	$(X,\mathscr{M}),(Y,\mathscr{N})$を可測空間とするとき,
	任意の$x \in X$に対し
	\begin{align}
		p_x:Y \ni y \longmapsto (x,y) \in X \times Y
	\end{align}
	で定める$p_x$は$\mathscr{N}/\mathscr{M} \otimes \mathscr{N}$-可測である.
	実際,$A \in \mathscr{M},\ B \in \mathscr{N}$に対しては
	\begin{align}
		p_x^{-1}(A \times B) = 
		\begin{cases}
			\emptyset, & (x \notin A), \\
			B, & (x \in A),
		\end{cases}
		\in \mathscr{N}
	\end{align}
	より
	\begin{align}
		\Set{A \times B}{A \in \mathscr{M},\ B \in \mathscr{N}}
		\subset \Set{E \in \mathscr{M} \otimes \mathscr{N}}{p_x^{-1}(E) \in \mathscr{N}}
	\end{align}
	が従い$p_x$の$\mathscr{N}/\mathscr{M} \otimes \mathscr{N}$-可測性が出る.
	同様に任意の$y \in Y$に対し
	\begin{align}
		q_y:X \ni x \longmapsto (x,y) \in X \times Y
	\end{align}
	で定める$q_y$は$\mathscr{M}/\mathscr{M} \otimes \mathscr{N}$-可測である.
	
	\begin{screen}
		\begin{lem}[二変数可測写像は片変数で可測]
			$(X,\mathscr{M}),(Y,\mathscr{N}),(Z,\mathscr{L})$を可測空間とするとき,
			写像$f: X \times Y \longmapsto Z$が
			$\mathscr{M}\otimes \mathscr{N}/ \mathscr{L}$-可測であれば
			\begin{align}
				X \ni x \longmapsto f(x,y_0),
				\quad Y \ni y \longmapsto f(x_0,y),
				\quad (\forall x_0 \in X,\ y_0 \in Y)
			\end{align}
			はそれぞれ$\mathscr{M}/\mathscr{L}$-可測,
			$\mathscr{N}/\mathscr{L}$-可測である.
		\end{lem}
	\end{screen}
	
	\begin{prf}
		$X \ni x \longmapsto f(x,y_0)$は$f$と$q_{y_0}$の合成$f \circ q_{y_0}$であり,
		$Y \ni y \longmapsto f(x_0,y)$は$f \circ p_{y_0}$である.
		\QED
	\end{prf}
	
	\begin{screen}
		\begin{dfn}[直積測度]
			$(X_i,\mathscr{M}_i,\mu_i),\ (i=1,\cdots,n)$を測度空間の族とするとき,
			$\prod_{i=1}^n X_i,\ \bigotimes_{i=1}^n \mathscr{M}_i$上の測度$\mu$で
			\begin{align}
				\mu(A_1 \times \cdots \times A_n)
				= \mu_1(A_1) \cdots \mu_n(A_n),
				\quad (A_i \in \mathscr{M}_i,\ i=1,\cdots,n)
			\end{align}
			を満たすものがただ一つ存在する.この$\mu$を$(\mu_i)_{i=1}^n$の
			直積測度と呼び$\bigotimes_{i=1}^n \mu_i = \mu_1 \otimes \cdots \otimes \mu_n$と書く.
		\end{dfn}
	\end{screen}
	
	\begin{screen}
		\begin{lem}
			$(X,\mathscr{M},\mu),(Y,\mathscr{N},\nu)$を$\sigma$-有限測度空間とするとき,
			任意の$Q \in \mathscr{M} \otimes \mathscr{N}$に対し
			\begin{align}
				X \ni x \longmapsto \int_Y \defunc_{Q}(x,y)\ \nu(dy),
				\quad Y \ni y \longmapsto \int_X \defunc_{Q}(x,y)\ \mu(dx),
			\end{align}
			はそれぞれ$\mathscr{M}/\borel{[0,\infty]}$-可測,
			$\mathscr{N}/\borel{[0,\infty]}$-可測であり
			\begin{align}
				\int_X \int_Y \defunc_Q (x,y)\ \nu(dy)\ \mu(dx)
				= (\mu \otimes \nu)(Q)
				= \int_Y \int_X \defunc_Q (x,y)\ \mu(dx)\ \nu(dy)
			\end{align}
			が成立する.
		\end{lem}
	\end{screen}
	
	\begin{prf}\mbox{}
		\begin{description}
			\item[第一段]
				仮定より$X_1 \subset X_2 \subset \cdots \uparrow X,
				\ Y_1 \subset Y_2 \subset \cdots \uparrow Y$で
				$\mu(X_i),\nu(Y_i) < \infty,\ (i=1,2,\cdots)$を満たすものが存在する.
				\begin{align}
					\mathscr{M}_n \coloneqq \Set{A \cap X_n}{A \in \mathscr{M}},
					\quad \mathscr{N}_n \coloneqq \Set{B \cap Y_n}{B \in \mathscr{N}}
				\end{align}
				及び
				\begin{align}
					\mathscr{D}_n \coloneqq
					\Set{E \in \mathscr{M}_n \otimes \mathscr{N}_n}{} 
				\end{align}
				とおけば,
				\begin{align}
					\Set{A \times B}{A \in \mathscr{M}_n,\ B \in \mathscr{N}_n}
					\subset \mathscr{D}_n
				\end{align}
				が従い
		\end{description}
	\end{prf}
	
	$(X_i,\mathscr{M}_i),\ (i=1,\cdots,n)$を可測空間の族とする.
	いま,$0 < k < n$として$\{i_1,\cdots,i_k\} \subset \{1,\cdots,n\}$を取り
	\begin{align}
		Y &\coloneqq \prod_{i=1}^n X_i,
		\quad Y_1 \coloneqq \prod_{j=1}^k X_{i_j},
		\quad Y_2 \coloneqq \prod_{i \in \substack{\{1,\cdots,n\} \\ \backslash \{i_1,\cdots,i_k\}}} X_i, \\
		\mathscr{N} &\coloneqq \bigotimes_{i=1}^n \mathscr{M}_i,
		\quad \mathscr{N}_1 \coloneqq \bigotimes_{j=1}^k \mathscr{M}_{i_j},
		\quad \mathscr{N}_2 \coloneqq \bigotimes_{i \in \substack{\{1,\cdots,n\} \\ \backslash \{i_1,\cdots,i_k\}}} \mathscr{M}_i
	\end{align}
	とおく.このとき,任意の$x_1 = (x_1^{i_1},\cdots,x_1^{i_k}) \in Y_1$に対し
	\begin{align}
		p_{x_1}:X_2 \ni x_2 = (x_2^i)_{i \in \substack{\{1,\cdots,n\} \\ \backslash \{i_1,\cdots,i_k\}}} \longmapsto (x_1^1) \in Y
	\end{align}
	は
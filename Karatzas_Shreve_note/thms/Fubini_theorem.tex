\section{Fubiniの定理}
	$(X,\mathcal{M}),(Y,\mathcal{N})$を可測空間とするとき,
	任意の$x \in X$に対し
	\begin{align}
		p_x:Y \ni y \longmapsto (x,y) \in X \times Y
	\end{align}
	で定める$p_x$は$\mathcal{N}/\mathcal{M} \otimes \mathcal{N}$-可測である.
	実際,$A \in \mathcal{M},\ B \in \mathcal{N}$に対しては
	\begin{align}
		p_x^{-1}(A \times B) = 
		\begin{cases}
			\emptyset, & (x \notin A), \\
			B, & (x \in A),
		\end{cases}
		\in \mathcal{N}
	\end{align}
	となるから,
	\begin{align}
		\Set{A \times B}{A \in \mathcal{M},\ B \in \mathcal{N}}
		\subset \Set{E \in \mathcal{M} \otimes \mathcal{N}}{p_x^{-1}(E) \in \mathcal{N}}
	\end{align}
	が従い$p_x$の$\mathcal{N}/\mathcal{M} \otimes \mathcal{N}$-可測性が出る.
	同様に任意の$y \in Y$に対し
	\begin{align}
		q_y:X \ni x \longmapsto (x,y) \in X \times Y
	\end{align}
	で定める$q_y$は$\mathcal{M}/\mathcal{M} \otimes \mathcal{N}$-可測である.
	
	\begin{screen}
		\begin{lem}[二変数可測写像は片変数で可測]\label{lem:Fubini_lemma_1}
			$(X,\mathcal{M}),(Y,\mathcal{N}),(Z,\mathcal{L})$を可測空間とするとき,
			写像$f: X \times Y \longmapsto Z$が
			$\mathcal{M}\otimes \mathcal{N}/ \mathcal{L}$-可測であれば,
			任意の$x_0 \in X,\ y_0 \in Y$に対し
			\begin{align}
				X \ni x \longmapsto f(x,y_0),
				\quad Y \ni y \longmapsto f(x_0,y)
			\end{align}
			はそれぞれ$\mathcal{M}/\mathcal{L}$-可測,
			$\mathcal{N}/\mathcal{L}$-可測である.
		\end{lem}
	\end{screen}
	
	\begin{prf}
		$X \ni x \longmapsto f(x,y_0)$は$f$と$q_{y_0}$の合成$f \circ q_{y_0}$であり,
		$Y \ni y \longmapsto f(x_0,y)$は$f \circ p_{x_0}$である.
		\QED
	\end{prf}
	
	\begin{screen}
		\begin{lem}\label{lem:Fubini_theorem}
			$(X,\mathcal{M},\mu),(Y,\mathcal{N},\nu)$を$\sigma$-有限な測度空間とするとき,
			任意の$Q \in \mathcal{M} \otimes \mathcal{N}$に対し
			\begin{align}
				\varphi_Q: X \ni x \longmapsto \int_Y \defunc_{Q}\circ p_x\ d\nu,
				\quad \psi_Q: Y \ni y \longmapsto \int_X \defunc_{Q} \circ q_y\ d\mu,
			\end{align}
			はそれぞれ$\mathcal{M}/\borel{[0,\infty]}$-可測,
			$\mathcal{N}/\borel{[0,\infty]}$-可測であり
			\begin{align}
				\int_X \varphi_Q\ d\mu
				= (\mu \otimes \nu)(Q)
				= \int_Y \psi_Q\ d\nu
				\label{eq:lem_Fubini_theorem_1}
			\end{align}
			が成立する.
		\end{lem}
	\end{screen}
	
	\begin{prf}\mbox{}
		\begin{description}
			\item[第一段]
				$\sigma$-有限の仮定より,
				\begin{align}
					\bigcup_{n=1}^\infty X_n = X,
					\quad \bigcup_{n=1}^\infty Y_n = Y,
					\quad \mu(X_n),\ \nu(Y_n) < \infty;
					\ n = 1,2,\cdots
				\end{align}
				を満たす増大列$\{X_n\}_{n=1}^\infty \subset \mathcal{M}$と
				$\{Y_n\}_{n=1}^\infty \subset \mathcal{N}$が存在する.ここで
				\begin{align}
					\mathcal{M}_n \coloneqq \Set{A \cap X_n}{A \in \mathcal{M}},
					\quad \mathcal{N}_n \coloneqq \Set{B \cap Y_n}{B \in \mathcal{N}}
				\end{align}
				により$X_n,Y_n$上の$\sigma$-加法族を定めて
				\begin{align}
					\mathcal{D}_n \coloneqq
					\Set{Q_n \in \mathcal{M}_n \otimes \mathcal{N}_n}{
					\substack{
					\displaystyle \varphi_{Q_n}: X \ni x \longmapsto \int_Y \defunc_{Q_n} \circ p_x\ d\nu \mbox{ が$\mathcal{M}/\borel{[0,\infty]}$-可測},\\
					\displaystyle \psi_{Q_n}: Y \ni y \longmapsto \int_X \defunc_{Q_n} \circ q_y\ d\mu \mbox{ が$\mathcal{N}/\borel{[0,\infty]}$-可測},\\
					\displaystyle \int_X \varphi_{Q_n}\ d\mu
					= (\mu \otimes \nu)(Q_n)
					= \int_Y \psi_{Q_n}\ d\nu}} 
				\end{align}
				とおけば,$\mathcal{D}_n$は$X_n \times Y_n$上のDynkin族であり
				\begin{align}
					\Set{A \times B}{A \in \mathcal{M}_n,\ B \in \mathcal{N}_n}
					\subset \mathcal{D}_n
				\end{align}
				を満たすから$\mathcal{M}_n \otimes \mathcal{N}_n = \mathcal{D}_n$が従う.
			
			\item[第二段]
				$\mathcal{M}_n \otimes \mathcal{N}_n = \Set{Q \cap (X_n \times Y_n)}{Q \in \mathcal{M} \otimes \mathcal{N}}$より,任意の$Q \in \mathcal{M} \otimes \mathcal{N}$に対して
				\begin{align}
					Q_n \coloneqq Q \cap (X_n \times Y_n) \in \mathcal{D}_n,
					\ (\forall n \geq 1),
					\quad Q_1 \subset Q_2 \subset \cdots \longrightarrow Q
				\end{align}
				が従い,単調収束定理より
				\begin{align}
					\varphi_Q(x) = \int_Y \defunc_Q \circ p_x\ d\nu
					= \lim_{n \to \infty} \int_Y \defunc_{Q_n} \circ p_x\ d\nu
					= \lim_{n \to \infty} \varphi_{Q_n}(x),
					\quad (\forall x \in X)
				\end{align}
				となるから$\varphi_Q$の$\mathcal{M}/\borel{[0,\infty]}$-可測性が出る.
				また,
				\begin{align}
					\varphi_{Q_n}(x) = \int_Y \defunc_{Q_n} \circ p_x\ d\nu
					\leq \int_Y \defunc_{Q_{n+1}} \circ p_x\ d\nu
					= \varphi_{Q_{n+1}}(x),
					\quad (n=1,2,\cdots)
				\end{align}
				が満たされているから,再び単調収束定理により
				\begin{align}
					\int_X \varphi_Q\ d\mu
					= \lim_{n \to \infty} \int_X \varphi_{Q_n}\ d\mu
					= \lim_{n \to \infty} (\mu \otimes \nu)(Q_n)
					= (\mu \otimes \nu)(Q)
				\end{align}
				が得られる.同様に$\psi_Q$は$\mathcal{N}/\borel{[0,\infty]}$-可測であり
				(\refeq{eq:lem_Fubini_theorem_1})を満たす.
				\QED
		\end{description}
	\end{prf}
	
	\begin{screen}
		\begin{thm}[Fubini]
			$(X,\mathcal{M},\mu),(Y,\mathcal{N},\nu)$を$\sigma$-有限な測度空間とする.
			\begin{description}
				\item[(1)]
					$f:X \times Y \longrightarrow [0,\infty]$を
					$\mathcal{M} \otimes \mathcal{N}/\borel{[0,\infty]}$-可測写像とするとき,
					\begin{align}
						\varphi: X \ni x \longmapsto \int_Y f \circ p_x\ d\nu,
						\quad \psi: Y \ni y \longmapsto \int_X f \circ q_y\ d\mu
					\end{align}
					により定める$\varphi,\psi$はそれぞれ$\mathcal{M}/\borel{[0,\infty]}$-可測,
					$\mathcal{N}/\borel{[0,\infty]}$-可測であり,
					\begin{align}
						\int_X \varphi\ d\mu
						= \int_{X \times Y} f\ d(\mu \otimes \nu)
						= \int_Y \psi\ d\nu
					\end{align}
					が成立する.
					
				\item[(2)]
					$f:X \times Y \longrightarrow \C$を
					$\mathcal{M} \otimes \mathcal{N}/\borel{\C}$-可測な
					可積分関数とするとき,
			\end{description}
		\end{thm}
	\end{screen}
	
	\begin{screen}
		\begin{thm}[$n$変数関数のFubiniの定理]
			$\left((X_i,\mathcal{M}_i,\mu_i)\right)_{i=1}^n,\ (n \geq 3)$を
			$\sigma$-有限な測度空間の族とし,
			\begin{align}
				\{i_1,\cdots,i_k\} \cup \{j_1,\cdots,j_h\} = \{1,2,\cdots,n\},
				\quad \{i_1,\cdots,i_k\} \cap \{j_1,\cdots,j_h\} = \emptyset
			\end{align}
			を満たす添数列$i_1, \cdots, i_k$と$j_1, \cdots, j_h,\ (1 \leq k,h \leq n-1)$を任意に取り
			\begin{align}
				&Y \coloneqq \prod_{i=1}^n X_i,
				\quad Y_1 \coloneqq \prod_{\ell=1}^k X_{i_\ell},
				\quad Y_2 \coloneqq \prod_{\ell=1}^h X_{j_\ell}, \\
				&\mathcal{N} \coloneqq \bigotimes_{i=1}^n \mathcal{M}_i,
				\quad \mathcal{N}_1 \coloneqq \bigotimes_{\ell=1}^k \mathcal{M}_{i_\ell},
				\quad \mathcal{N}_2 \coloneqq \bigotimes_{\ell=1}^h \mathcal{M}_{j_\ell}, \\
				&\mu \coloneqq \bigotimes_{i=1}^n \mu_i,
				\quad \nu_1 \coloneqq \bigotimes_{\ell=1}^k \mu_{i_\ell},
				\quad \nu_2 \coloneqq \bigotimes_{\ell=1}^h \mu_{j_\ell}
			\end{align}
			とおく.また
			\begin{align}
				p_{y_1}:Y_2 \ni y_2 \longmapsto (y_1,y_2),\ (\forall y_1 \in Y_1),
				\quad q_{y_2}:Y_1 \ni y_1 \longmapsto (y_1,y_2),\ (\forall y_2 \in Y_2)
			\end{align}
			とする.このとき,射影$\pi_1:Y \longrightarrow Y_1,\ \pi_2:Y \longrightarrow Y_2$に対し
			\begin{align}
				\varphi: Y_1 \times Y_2 \ni (y_1,y_2) \longmapsto \pi_1^{-1}(y_1) \cap \pi_2^{-1}(y_2)
			\end{align}
			により$\varphi:Y_1 \times Y_2 \longrightarrow Y$を定めれば
			$\varphi$は$\mathcal{N}_1 \otimes \mathcal{N}_2/\mathcal{N}$-可測であり,
			更に以下が成立する:
			\begin{description}
				\item[(1)] $f:Y \longrightarrow [0,\infty]$が$\mathcal{N}/\borel{[0,\infty]}$-可測なら次が成り立つ:
					\begin{align}
						\int_Y f\ d\mu
						= \int_{Y_1} \int_{Y_2} f \left(\varphi\left(p_{y_1}(y_2)\right)\right)\ \nu_2(dy_2)\ \nu_1(dy_1)
						= \int_{Y_2} \int_{Y_1} f \left(\varphi\left(q_{y_2}(y_1)\right)\right)\ \nu_1(dy_1)\ \nu_2(dy_2).
					\end{align}
			\end{description}
		\end{thm}
	\end{screen}
	
	\begin{prf}\mbox{}
		\begin{description}
			\item[第一段]
				$\varphi$の$\mathcal{N}_1 \otimes \mathcal{N}_2/\mathcal{N}$-可測性を示す.
				実際,$\varphi:Y_1 \times Y_2 \longrightarrow Y$が全単射であることより
				\begin{align}
					\varphi^{-1}(E_1 \times \cdots \times E_n)
					= \prod_{\ell=1}^k E_{i_\ell} \times \prod_{\ell=1}^h E_{j_\ell}
					\in \mathcal{N}_1 \otimes \mathcal{N}_2,
					\quad (\forall E_i \in \mathcal{M}_i,\ i=1,\cdots,n)
					\label{eq:Fubini_theorem_n_variables_1}
				\end{align}
				が成り立つから
				\begin{align}
					\Set{E_1 \times \cdots \times E_n}{E_i \in \mathcal{M}_i,\ i=1,\cdots,n}
					\subset \Set{E \in \mathcal{N}}{\varphi^{-1}(E) \in \mathcal{N}_1 \otimes \mathcal{N}_2}
				\end{align}
				となり,左辺は$\mathcal{N}$を生成するから$\varphi$は$\mathcal{N}_1 \otimes \mathcal{N}_2/\mathcal{N}$-可測である.
				
			\item[第二段]
				$f = \defunc_E\ (E \in \mathcal{N})$に対し
				\begin{align}
					\int_Y f\ d\mu 
					= \int_{Y_1 \times Y_2} f \circ \varphi\ d(\nu_1 \otimes \nu_2)
				\end{align}
				となることを示す.実際,(\refeq{eq:Fubini_theorem_n_variables_1})より
				\begin{align}
					\Set{E_1 \times \cdots \times E_n}{E_i \in \mathcal{M}_i,\ i=1,\cdots,n}
					\subset \Set{E \in \mathcal{N}}{\mu(E) = \nu_1 \otimes \nu_2\left( \varphi^{-1}(E) \right)}
				\end{align}
				となるから,Dinkin族定理より任意の$E \in \mathcal{N}$に対して
				$\mu(E) = \nu_1 \otimes \nu_2\left( \varphi^{-1}(E) \right)$が成立し
				\begin{align}
					\int_Y f\ d\mu
					= \mu(E)
					= \nu_1 \otimes \nu_2\left( \varphi^{-1}(E) \right)
					= \int_{Y_1 \times Y_2} f \circ \varphi\ d(\nu_1 \otimes \nu_2)
				\end{align}
				が従う.
		\end{description}
	\end{prf}
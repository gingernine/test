\subsection{商空間}
	$\left((X,\sigma_X),(\Phi,+,\bullet),s,\mathscr{O}_X\right)$を位相線型空間とし,
	\begin{align}
		0_X
	\end{align}
	を$(X,\sigma_X)$の単位元とする.また
	$\left((N,\sigma_N),(\Phi,+,\bullet),s_N\right)$を$\left((X,\sigma_X),(\Phi,+,\bullet),s\right)$の部分空間とする.
	つまり$N$は$X$の部分集合であって,$\sigma_N$と$s_N$は
	\begin{align}
		\sigma_N \defeq \sigma_X|_{N \times N}
	\end{align}
	および
	\begin{align}
		s_N \defeq s|_{\Phi \times N}
	\end{align}
	によって定められ
	\begin{align}
		\left((N,\sigma_N),(\Phi,+,\bullet),s_N\right)
	\end{align}
	は線型空間をなしている.ここで
	\begin{align}
		\Psi \defeq \Set{(x,y)}{x \in X \wedge y \in X \wedge \sigma_X(x,-y) \in N}
	\end{align}
	とおくと$\Psi$は$X$上の同値関係である.実際
	\begin{align}
		\Psi \subset X \times X
	\end{align}
	であり,また
	\begin{align}
		0_X \in N
	\end{align}
	であるから
	\begin{align}
		\forall x \in X\, \left[\, (x,x) \in \Psi\, \right]
	\end{align}
	が満たさる.
	\begin{align}
		(x,y) \in \Psi
	\end{align}
	なるとき,
	\begin{align}
		\sigma_X(y,-x) = -\sigma_X(x,-y) \in N
	\end{align}
	が成り立つから
	\begin{align}
		(y,x) \in \Psi
	\end{align}
	も満たされる.また
	\begin{align}
		(x,y) \in \Psi \wedge (y,z) \in \Psi
	\end{align}
	なるとき,
	\begin{align}
		\sigma_X(x,-z) = \sigma_X\left(\sigma_X(x,-y),\sigma_X(y,-z)\right) \in N
	\end{align}
	が成り立つから
	\begin{align}
		(x,z) \in \Psi
	\end{align}
	も満たされる.
	
	ここで$\Psi$が定理\ref{thm:quotient_module}の(\refeq{fom:thm_quotient_module_1})と(\refeq{fom:thm_quotient_module_2})
	を満たすことを確認する.
	$x,y,a,b$を$X$の要素とし
	\begin{align}
		(x,a) \in \Psi \wedge (y,b) \in \Psi
	\end{align}
	が成り立っているとする.$\sigma_X$の結合律より
	\begin{align}
		\sigma_X\left(\sigma_X(x,y),-\sigma_X(a,b)\right)
		&= \sigma_X\left(\sigma_X(x,y),\sigma_X(-b,-a)\right) \\
		&= \sigma_X\left(x,\sigma_X\left(y,\sigma_X(-b,-a)\right)\right) \\
		&= \sigma_X\left(x,\sigma_X\left(\sigma_X(y,-b),-a\right)\right) \\
		&= \sigma_X\left(x,\sigma_X\left(-a,\sigma_X(y,-b)\right)\right) \\
		&= \sigma_X\left(\sigma_X(x,-a),\sigma_X(y,-b)\right)
	\end{align}
	が成り立つが,いま
	\begin{align}
		\sigma_X(x,-a) \in N \wedge \sigma_X(y,-b) \in N
	\end{align}
	であるから
	\begin{align}
		\sigma_X\left(\sigma_X(x,-a),\sigma_X(y,-b)\right) \in N
	\end{align}
	が成り立ち
	\begin{align}
		\sigma_X\left(\sigma_X(x,y),-\sigma_X(a,b)\right) \in N
	\end{align}
	が従う.ゆえに
	\begin{align}
		\left(\sigma_X(x,y),\sigma_X(a,b)\right) \in \Psi
	\end{align}
	が成り立つ.ゆえに(\refeq{fom:thm_quotient_module_1})は満たされている.
	
	次に$\alpha$を$\Phi$の要素とし,$x,y$を$X$の要素として
	\begin{align}
		(x,y) \in \Psi
	\end{align}
	が成り立っているとする.スカラ倍の規則より
	\begin{align}
		\sigma_X\left(s(\alpha,x),-s(\alpha,y)\right)
		&= \sigma_X\left(s(\alpha,x),s(\alpha,-y)\right) \\
		&= s\left(\alpha,\sigma_X(x,-y)\right)
	\end{align}
	が成り立つが,いま
	\begin{align}
		\sigma_X(x,-y) \in N
	\end{align}
	であるから
	\begin{align}
		s\left(\alpha,\sigma_X(x,-y)\right) \in N
	\end{align}
	が成り立ち
	\begin{align}
		\sigma_X\left(s(\alpha,x),-s(\alpha,y)\right) \in N
	\end{align}
	が従う.ゆえに
	\begin{align}
		\left(s(\alpha,x),s(\alpha,y)\right) \in \Psi
	\end{align}
	が成り立つ.ゆえに(\refeq{fom:thm_quotient_module_2})も満たされている.よって,
	\begin{align}
		X_q \defeq X/\Psi
	\end{align}
	とおき,$q$を$X$から$X_q$への商写像とし,
	\begin{align}
		\sigma_q \defeq \Set{z}{\exists x,y \in X\, 
		\left[\, z=\left(\left(q(x),q(y)\right),q\left(\sigma_X(x,y)\right)\right)\, \right]}
	\end{align}
	および
	\begin{align}
		s_q \defeq \Set{z}{\exists \alpha \in \Phi\, \exists x \in X\, 
		\left[\, z=\left(\left(\alpha,q(x)\right),q\left(s(\alpha,x)\right)\right)\, \right]}
	\end{align}
	と定めると,定理\ref{thm:quotient_module}より
	\begin{align}
		\left(\left(X_q,\sigma_q\right),\left(\Phi,+,\bullet\right),s_q\right)
	\end{align}
	は線型空間をなす.
	
	本小節の主題は,{\bf $X_q$上の商位相が線型位相である}ということである.つまり
	\begin{align}
		\mathscr{O}_q \defeq \Set{u}{u \subset X_q \wedge q^{-1} \ast u \in \mathscr{O}_X}
	\end{align}
	とおけば
	\begin{align}
		\left(\left(X_q,\sigma_q\right),\left(\Phi,+,\bullet\right),s_q,\mathscr{O}_q\right)
	\end{align}
	は位相線型空間である.
	
	\begin{sketch}\mbox{}
		\begin{description}
			\item[第一段] まず$q$が開写像であることを示す.実際,$v$を$\mathscr{O}_X$の要素とすると
				\begin{align}
					q^{-1} \ast (q \ast v) = \bigcup_{x \in N} \Set{z}{\exists y \in v\, \left(\, z=\sigma_X(x,y)\, \right)}
					\label{fom:thm_quotient_topological_vector_space_1}
				\end{align}
				が成り立つ.これは直感的に書けば
				\begin{align}
					q^{-1} \ast (q \ast v) = \bigcup_{x \in N} x + v
				\end{align}
				が成り立つということであるが,定理\ref{thm:topological_groups_are_invariant}より
				\begin{align}
					\Set{z}{\exists y \in v\, \left(\, z=\sigma_X(x,y)\, \right)} \in \mathscr{O}_X
				\end{align}
				であるから,(\refeq{fom:thm_quotient_topological_vector_space_1})が示されれば
				\begin{align}
					q \ast v \in \mathscr{O}_q
				\end{align}
				が従う.いま$z$を$q^{-1} \ast (q \ast v)$の要素とすると
				\begin{align}
					q(z) = q(y)
				\end{align}
				を満たす$v$の要素$y$が取れる.すなわち
				\begin{align}
					\sigma_X(z,-y) \in N
				\end{align}
				が成り立ち,
				\begin{align}
					z = \sigma_X\left(\sigma_X(z,-y),y\right)
				\end{align}
				も成り立つので
				\begin{align}
					z \in \bigcup_{x \in N} \Set{z}{\exists y \in v\, \left(\, z=\sigma_X(x,y)\, \right)}
				\end{align}
				が従う.逆に集合$z$が
				\begin{align}
					z \in \bigcup_{x \in N} \Set{z}{\exists y \in v\, \left(\, z=\sigma_X(x,y)\, \right)}
				\end{align}
				であるとき,
				\begin{align}
					z = \sigma_X(x,y)
				\end{align}
				を満たす$N$の要素$x$と$v$の要素$y$が取れるが,
				\begin{align}
					x = \sigma_X(z,-y) \in N
				\end{align}
				より
				\begin{align}
					q(z) = q(y)
				\end{align}
				が成り立つので
				\begin{align}
					z \in q^{-1} \ast (q \ast v)
				\end{align}
				が従う.以上で(\refeq{fom:thm_quotient_topological_vector_space_1})が示された.
				
			\item[第二段]
				$\mathscr{B}$を$\left((X,\sigma_X),\mathscr{O}_X\right)$の局所基で全ての要素が
				均衡集合であるものとする.また
				\begin{align}
					0_q \defeq q(0_X)
				\end{align}
				とおく.このとき
				\begin{align}
					\mathscr{B}_q \defeq \Set{q \ast b}{b \in \mathscr{B}}
				\end{align}
				は$\mathscr{O}_q$に関して$0_q$の基本近傍系である.実際
				$q$は開写像であるから$\mathscr{B}_q$の要素は全て$0_q$の近傍であり,また
				$v$を$0_q$の近傍とすれば
				\begin{align}
					q^{-1} \ast v
				\end{align}
				は$0_X$の近傍であるから
				\begin{align}
					b \subset q^{-1} \ast v
				\end{align}
				を満たす$\mathscr{B}$の要素$b$が取れる.ゆえに
				\begin{align}
					q \ast b \subset v
				\end{align}
				が成り立つ.
				
			\item[第三段]
				次は$X$上に近縁系が取れることを示すが,その前にまず$\sigma_q$が$(0_q,0_q)$において
				$\mathscr{O}_q$に関して連続であることを示す.$u$を$0_q$の近傍とすると
				\begin{align}
					q \ast b \subset u
				\end{align}
				を満たす$\mathscr{B}$の要素$b$が取れる.また$\sigma_X$は$(0_X,0_X)$において$\mathscr{O}_X$に関して連続であるから
				\begin{align}
					a \times a \subset {\sigma_X}^{-1} \ast b
				\end{align}
				を満たす$\mathscr{B}$の要素$a$が取れる.このとき$x$と$y$を$q \ast a$の要素とすれば,
				\begin{align}
					x = q(\eta)
				\end{align}
				および
				\begin{align}
					y = q(\xi)
				\end{align}
				を満たす$a$の要素$\eta$と$\xi$が取れて,
				\begin{align}
					\sigma_q(x,y) = q\left(\sigma_X(\eta,\xi)\right) \in q \ast b
				\end{align}
				が従う.ゆえに
				\begin{align}
					(q \ast a) \times (q \ast a) \subset {\sigma_q}^{-1} \ast u
				\end{align}
				が従う.ゆえに$\sigma_q$は$(0_q,0_q)$において$\mathscr{O}_q$に関して連続である.
				
				次に$\mathscr{B}_q$の要素が均衡集合であることを示す.$u$を$\mathscr{B}_q$の要素とする.
				$\alpha$を
				\begin{align}
					|\alpha| \leq 1
				\end{align}
				なる$\Phi$の要素とし,$x$を$u$の要素とするとき
				\begin{align}
					s_q(\alpha,x) \in u
				\end{align}
				が成り立てば良いが,実際
				\begin{align}
					u = q \ast b
				\end{align}
				を満たす$\mathscr{B}$の要素$b$と
				\begin{align}
					x = q(\eta)
				\end{align}
				を満たす$b$の要素$\eta$を取れば,$b$が均衡集合であるので
				\begin{align}
					s(\alpha,\eta) \in b
				\end{align}
				が成り立ち
				\begin{align}
					s_q(\alpha,x) = q\left(s(\alpha,\eta)\right) \in u
				\end{align}
				が従う.ゆえに$u$は均衡集合である.よっていま
				\begin{align}
					\mathscr{U} \defeq \Set{u}{\exists b \in \mathscr{B}_q\, 
					\forall z\, \left[\, z \in u \Longleftrightarrow 
					\exists x,y \in X_q\, \left(\, z=(x,y) \wedge \sigma_q(-x,y) \in b\, \right)\, \right]}
				\end{align}
				と定めて
				\begin{align}
					\mathscr{V} \defeq \Set{v}{v \subset X_q \times X_q \wedge \exists u \in \mathscr{U}\,
					(\, u \subset v\, )}
				\end{align}
				とおくと,定理\ref{thm:summation_is_continuous_at_zero_then_there_exists_a_entourages}より
				$\mathscr{V}$は$X_q$上の近縁系をなす.この$\mathscr{U}$が
				定理\ref{thm:entourages_introducing_vector_topology}の四条件を満たすことを確認する.
				
				
			\item[第四段]
				$\mathscr{V}$で導入する$X$上の一様位相を
				\begin{align}
					\mathscr{O}_{\mathscr{V}}
				\end{align}
				として,最後に
				\begin{align}
					\mathscr{O}_q = \mathscr{O}_{\mathscr{V}}
				\end{align}
				であることを示すが,これは一致する基本近傍系が取れることを見ればよい.いま$x$を$X_q$の要素とする.
				$\mathscr{U}$の要素$u$に対しての
				\begin{align}
					\Set{y}{(x,y) \in u}
				\end{align}
				なる集合を,$\mathscr{U}$のすべての要素に亘って取ったものの全体,つまり
				\begin{align}
					\mathscr{U}(x) \defeq 
					\Set{b}{\exists u \in \mathscr{U}\, 
					\left[\, \forall y\, \left(\, y \in b \Longleftrightarrow (x,y) \in u\, \right) \right]}
				\end{align}
				は$\mathscr{O}_{\mathscr{V}}$に関して$x$の基本近傍系である.このとき
				\begin{align}
					\mathscr{U}(x)
				\end{align}
				が$\mathscr{O}_q$に関しても$x$の基本近傍系であることを示す.$x$に対して
				\begin{align}
					x = q(\eta)
				\end{align}
				を満たす$X$の要素$\eta$を取る.$u$を$\mathscr{U}$の要素とすれば
				\begin{align}
					u = q \ast b
				\end{align}
				を満たす$\mathscr{B}$の要素$b$が取れるが,このとき
				\begin{align}
					\Set{y}{(x,y) \in u} = q \ast \Set{\zeta}{\exists \xi \in b\, 
					\left(\, \zeta = \sigma_X(\eta,\xi)\, \right)}
				\end{align}
				が成り立ち,かつ
				\begin{align}
					\Set{\zeta}{\exists \xi \in b\, \left(\, \zeta = \sigma_X(\eta,\xi)\, \right)}
				\end{align}
				は$\mathscr{O}_X$に関して$\eta$の近傍であるから
				\begin{align}
					q \ast \Set{\zeta}{\exists \xi \in b\, \left(\, \zeta = \sigma_X(\eta,\xi)\, \right)}
				\end{align}
				は$\mathscr{O}_q$に関して$x$の近傍である.また$v$を$x$の$\mathscr{O}_q$に関する近傍とすれば
				\begin{align}
					q^{-1} \ast v
				\end{align}
				は$\eta$の$\mathscr{O}_X$に関する近傍であって,
				\begin{align}
					\Set{\zeta}{\exists \xi \in q^{-1} \ast v\, \left(\, \zeta = \sigma_X(-\eta,\xi)\, \right)}
				\end{align}
				は$0_X$の$\mathscr{O}_X$に関する近傍である.ゆえに
				\begin{align}
					b \subset \Set{\zeta}{\exists \xi \in q^{-1} \ast v\, \left(\, \zeta = \sigma_X(-\eta,\xi)\, \right)}
				\end{align}
				を満たす$\mathscr{B}$の要素$b$が取れて
				\begin{align}
					\Set{\zeta}{\exists \xi \in b\, \left(\, \zeta = \sigma_X(\eta,\xi)\, \right)}
					\subset q^{-1} \ast v
				\end{align}
				が成立する.ゆえに
				\begin{align}
					\Set{y}{(x,y) \in u} \subset v
				\end{align}
				が成立する.以上より$\mathscr{U}(x)$は$\mathscr{O}_q$に関しても$x$の基本近傍系である.
				\QED
		\end{description}
	\end{sketch}
	
	以上の内容をまとめると
	\begin{screen}
		\begin{thm}[位相線型空間を部分空間で割ったときの商位相は線型位相である]
			$\left((X,\sigma_X),(\Phi,+,\bullet),s,\mathscr{O}_X\right)$を位相線型空間とし,
			$\left((N,\sigma_N),(\Phi,+,\bullet),s_N\right)$を$\left((X,\sigma_X),(\Phi,+,\bullet),s\right)$の部分空間とし,
			\begin{align}
				\Psi \defeq \Set{(x,y)}{x \in X \wedge y \in X \wedge \sigma_X(-x,y) \in N}
			\end{align}
			により$X$上の同値関係を定める.そして
			\begin{align}
				X_q \defeq X/\Psi
			\end{align}
			とおき,$q$を$X$から$X_q$への商写像とし,
			\begin{align}
				\sigma_q \defeq \Set{z}{\exists x,y \in X\, 
				\left[\, z=\left(\left(q(x),q(y)\right),q\left(\sigma_X(x,y)\right)\right)\, \right]}
			\end{align}
			および
			\begin{align}
				s_q \defeq \Set{z}{\exists \alpha \in \Phi\, \exists x \in X\, 
				\left[\, z=\left(\left(\alpha,q(x)\right),q\left(s(\alpha,x)\right)\right)\, \right]}
			\end{align}
			および
			\begin{align}
				\mathscr{O}_q \defeq \Set{u}{u \subset X_q \wedge q^{-1} \ast u \in \mathscr{O}_X}
			\end{align}
			と定める.このとき
			\begin{itemize}
				\item $q$は$\mathscr{O}_X$と$\mathscr{O}_q$に関して開写像である.
				
				\item $\left(\left(X_q,\sigma_q\right),\left(\Phi,+,\bullet\right),s_q,\mathscr{O}_q\right)$は位相線型空間である.
				
				\item $\mathscr{B}$を$\left((X,\sigma_X),\mathscr{O}_X\right)$の局所基とすると,
					\begin{align}
						\mathscr{B}_q \defeq \Set{q \ast b}{b \in \mathscr{B}}
					\end{align}
					で定める$\mathscr{B}_q$は$\left((X_q,\sigma_q),\mathscr{O}_q\right)$の局所基である.
				
				\item $N$が$(X,\mathscr{O}_X)$の閉集合で$(X,\mathscr{O}_X)$がHausdorffなら,
					$(X_q,\mathscr{O}_q)$もHausdorffである.
			\end{itemize}
		\end{thm}
	\end{screen}
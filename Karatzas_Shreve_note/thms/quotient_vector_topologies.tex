\subsection{商空間}
	$\left((X,\sigma_X),(\Phi,+,\bullet),s,\mathscr{O}_X\right)$を位相線型空間とし,
	\begin{align}
		0_X
	\end{align}
	を$(X,\sigma_X)$の単位元とする.また
	$\left((N,\sigma_N),(\Phi,+,\bullet),s_N\right)$を$\left((X,\sigma_X),(\Phi,+,\bullet),s\right)$の部分空間とする.
	つまり$N$は$X$の部分集合であって,$\sigma_N$と$s_N$は
	\begin{align}
		\sigma_N \defeq \sigma_X|_{N \times N}
	\end{align}
	および
	\begin{align}
		s_N \defeq s|_{\Phi \times N}
	\end{align}
	によって定められ
	\begin{align}
		\left((N,\sigma_N),(\Phi,+,\bullet),s_N\right)
	\end{align}
	は線型空間をなしている.ここで
	\begin{align}
		\Psi \defeq \Set{(x,y)}{x \in X \wedge y \in X \wedge \sigma_X(-x,y) \in N}
	\end{align}
	とおくと$\Psi$は$X$上の同値関係である.実際
	\begin{align}
		\Psi \subset X \times X
	\end{align}
	であり,また
	\begin{align}
		0_X \in N
	\end{align}
	であるから
	\begin{align}
		\forall x \in X\, \left[\, (x,x) \in \Psi\, \right]
	\end{align}
	が満たさる.
	\begin{align}
		(x,y) \in \Psi
	\end{align}
	なるとき,
	\begin{align}
		\sigma_X(-y,x) = -\sigma_X(-x,y) \in N
	\end{align}
	が成り立つから
	\begin{align}
		(y,x) \in \Psi
	\end{align}
	も満たされる.また
	\begin{align}
		(x,y) \in \Psi \wedge (y,z) \in \Psi
	\end{align}
	なるとき,
	\begin{align}
		\sigma_X(-x,z) = \sigma_X\left(\sigma_X(-x,y),\sigma_X(-y,z)\right) \in N
	\end{align}
	が成り立つから
	\begin{align}
		(x,z) \in \Psi
	\end{align}
	も満たされる.
	
	ここで$\Psi$が定理\ref{thm:quotient_module}の(\refeq{fom:thm_quotient_module_1})と(\refeq{fom:thm_quotient_module_2})
	を満たすことを確認する.
	$x,y,a,b$を$X$の要素とし
	\begin{align}
		(x,a) \in \Psi \wedge (y,b) \in \Psi
	\end{align}
	が成り立っているとすると,
	\begin{align}
		\sigma_X(-x,a) \in N \wedge \sigma_X(-y,b) \in N
	\end{align}
	であるから
	\begin{align}
		\sigma_X\left(-\sigma_X(x,y),\sigma_X(a,b)\right)
		&= \sigma_X\left(\sigma_X(-y,-x),\sigma_X(a,b)\right) \\
		&= \sigma_X\left(-y,\sigma_X\left(-x,\sigma_X(a,b)\right)\right) \\
		&= \sigma_X\left(-y,\sigma_X\left(\sigma_X(-x,a),b\right)\right) \\
		&= \sigma_X\left(-y,\sigma_X\left(b,\sigma_X(-x,a)\right)\right) \\
		&= \sigma_X\left(\sigma_X(-y,b),\sigma_X(-x,a)\right) \\
		& \in N
	\end{align}
	が従う.ゆえに
	\begin{align}
		\left(\sigma_X(x,y),\sigma_X(a,b)\right) \in \Psi
	\end{align}
	が成り立つ.ゆえに(\refeq{fom:thm_quotient_module_1})は満たされている.
	
	次に$\alpha$を$\Phi$の要素とし,$x,y$を$X$の要素として
	\begin{align}
		(x,y) \in \Psi
	\end{align}
	が成り立っているとすると,
	\begin{align}
		\sigma_X(-x,y) \in N
	\end{align}
	であるから
	\begin{align}
		\sigma_X\left(-s(\alpha,x),s(\alpha,y)\right)
		&= \sigma_X\left(s(\alpha,-x),s(\alpha,y)\right) \\
		&= s\left(\alpha,\sigma_X(-x,y)\right) \\
		& \in N
	\end{align}
	が従う.ゆえに
	\begin{align}
		\left(s(\alpha,x),s(\alpha,y)\right) \in \Psi
	\end{align}
	が成り立つ.ゆえに(\refeq{fom:thm_quotient_module_2})も満たされている.よって,
	\begin{align}
		X_q \defeq X/\Psi
	\end{align}
	とおき,$q$を$X$から$X_q$への商写像とし,
	\begin{align}
		\sigma_q \defeq \Set{z}{\exists x,y \in X\, 
		\left[\, z=\left(\left(q(x),q(y)\right),q\left(\sigma_X(x,y)\right)\right)\, \right]}
	\end{align}
	および
	\begin{align}
		s_q \defeq \Set{z}{\exists \alpha \in \Phi\, \exists x \in X\, 
		\left[\, z=\left(\left(\alpha,q(x)\right),q\left(s(\alpha,x)\right)\right)\, \right]}
	\end{align}
	と定めると,定理\ref{thm:quotient_module}より
	\begin{align}
		\left(\left(X_q,\sigma_q\right),\left(\Phi,+,\bullet\right),s_q\right)
	\end{align}
	は線型空間をなす.
	
	本小節の主題は,{\bf $X_q$上の商位相が線型位相である}ということである.つまり
	\begin{align}
		\mathscr{O}_q \defeq \Set{u}{u \subset X_q \wedge q^{-1} \ast u \in \mathscr{O}_X}
	\end{align}
	とおけば
	\begin{align}
		\left(\left(X_q,\sigma_q\right),\left(\Phi,+,\bullet\right),s_q,\mathscr{O}_q\right)
	\end{align}
	は位相線型空間である.
	
	\begin{sketch}\mbox{}
		\begin{description}
			\item
		\end{description}		
	\end{sketch}
	
	以上の内容をまとめると
	\begin{screen}
		\begin{thm}[位相線型空間を部分空間で割ったときの商位相は線型位相である]
			$\left((X,\sigma_X),(\Phi,+,\bullet),s,\mathscr{O}_X\right)$を位相線型空間とし,
			$\left((N,\sigma_N),(\Phi,+,\bullet),s_N\right)$を$\left((X,\sigma_X),(\Phi,+,\bullet),s\right)$の部分空間とし,
			\begin{align}
				\Psi \defeq \Set{(x,y)}{x \in X \wedge y \in X \wedge \sigma_X(-x,y) \in N}
			\end{align}
			により$X$上の同値関係を定める.そして
			\begin{align}
				X_q \defeq X/\Psi
			\end{align}
			とおき,$q$を$X$から$X_q$への商写像とし,
			\begin{align}
				\sigma_q \defeq \Set{z}{\exists x,y \in X\, 
				\left[\, z=\left(\left(q(x),q(y)\right),q\left(\sigma_X(x,y)\right)\right)\, \right]}
			\end{align}
			および
			\begin{align}
				s_q \defeq \Set{z}{\exists \alpha \in \Phi\, \exists x \in X\, 
				\left[\, z=\left(\left(\alpha,q(x)\right),q\left(s(\alpha,x)\right)\right)\, \right]}
			\end{align}
			および
			\begin{align}
				\mathscr{O}_q \defeq \Set{u}{u \subset X_q \wedge q^{-1} \ast u \in \mathscr{O}_X}
			\end{align}
			と定める.このとき
			\begin{itemize}
				\item $\left(\left(X_q,\sigma_q\right),\left(\Phi,+,\bullet\right),s_q,\mathscr{O}_q\right)$は位相線型空間である.
				
				\item $q$は$\mathscr{O}_X$と$\mathscr{O}_q$に関して開写像である.
				
				\item $N$が$(X,\mathscr{O}_X)$の閉集合で$(X,\mathscr{O}_X)$がHausdorffなら,
					$(X_q,\mathscr{O}_q)$もHausdorffである.
			\end{itemize}
		\end{thm}
	\end{screen}
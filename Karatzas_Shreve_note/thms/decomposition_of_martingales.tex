\subsection{Doob分解}
	
	\begin{screen}
		\begin{thm}[有界変動な局所マルチンゲールは定数]
		\label{thm:local_martingale_of_bounded_variation_is_constant}
			$X$を$\mathscr{M}_{c,loc}$の要素とし,$\Omega$の任意の要素$\omega$に対する
			標本路$X_{\bullet}(\omega)$は$[0,1]$上で有界変動であるとする.このとき
			\begin{align}
				\Set{\omega \in \Omega}{\exists t \in [0,1]\,
				\left(\, X_{t}(\omega) \neq 0\, \right)}
			\end{align}
			は$P$-零集合である.
		\end{thm}
	\end{screen}
	
	\begin{sketch}\mbox{}
		\begin{description}
			\item[第一段]
				$X$の総変動過程$|X|$が有界であるとする.つまり,
				$[0,1]$の任意の要素$t$及び$\Omega$の任意の要素$\omega$に対して
				\begin{align}
					|X|_{t}(\omega) \leq b
				\end{align}
				を満たす実数$b$が取れる.このとき$X$も有界であるから,
				\begin{align}
					X \in \mathscr{M}_{c}^{2}
				\end{align}
				が成り立つ.特に
				\begin{align}
					E\left(X_{1}^{2}\right)
					&= E\left[\sum_{k=0}^{n-1}\left(X_{\frac{k+1}{n}}^{2} - X_{\frac{k}{n}}^{2}\right)\right] \\
					&= E\left[\sum_{k=0}^{n-1}\left(X_{\frac{k+1}{n}} - X_{\frac{k}{n}}\right)^{2}\right] \\
					&\leq E\left[|X|_{1} \cdot \sup{k \in n}{\left|X_{\frac{k+1}{n}} - X_{\frac{k}{n}}\right|}\right] \\
					&\leq b \cdot E\left[\sup{k \in n}{\left|X_{\frac{k+1}{n}} - X_{\frac{k}{n}}\right|}\right]
				\end{align}
				が成り立ち,Lebesgueの収束定理より
				\begin{align}
					E\left[\sup{k \in n}{\left|X_{\frac{k+1}{n}} - X_{\frac{k}{n}}\right|}\right] \longrightarrow 0 \quad (n \longrightarrow \infty)
				\end{align}
				となるから
				\begin{align}
					E\left(X_{1}^{2}\right) = 0
				\end{align}
				を得る.そして$X^{2}$は$\{\mathscr{F}_{t}\}_{t \in [0,1]}$-劣マルチンゲール性であるから,$[0,1]$の任意の要素$t$で
				\begin{align}
					E\left(X_{t}^{2}\right) = 0
				\end{align}
				が成立する.
				
			\item[第二段] $\{\mathscr{F}_{t}\}_{t \in [0,1]}$-増大停止時刻列$\sigma$で,
				任意の自然数$n$及び$\Omega$の任意の要素$\omega$に対して
				\begin{align}
					\sigma_{n}(\omega) = 
					\begin{cases}
						\inf{}{\Set{t \in [0,1]}{n \leq |X|_{t}(\omega)}} 
						& \mbox{if } \Set{t \in [0,1]}{n \leq |X|_{t}(\omega)} \neq \emptyset \\
						1 & \mbox{if } \Set{t \in [0,1]}{n \leq |X|_{t}(\omega)} = \emptyset
					\end{cases}
				\end{align}
				を満たすものを取る.このとき任意の自然数$n$,$[0,1]$の任意の要素$t$
				及び$\Omega$の任意の要素$\omega$に対して
				\begin{align}
					\left|X^{\sigma_{n}}\right|_{t}(\omega) = 
					\begin{cases}
						\left|X\right|_{t}(\omega) &\mbox{if } t \leq \sigma_{n}(\omega) \\
						\left|X\right|_{\sigma_{n}(\omega)}(\omega) &\mbox{if } \sigma_{n}(\omega) < t
					\end{cases}
				\end{align}
				が成り立つので,つまり
				\begin{align}
					\left|X^{\sigma_{n}}\right| = |X|^{\sigma_{n}}
				\end{align}
				である.ゆえに前段の結果より,任意の自然数$n$及び
				$[0,1]$の任意の要素$t$に対して
				\begin{align}
					E\left({X^{\sigma_{n}}_{t}}^{2}\right) = 0
				\end{align}
				が成立する.ここで
				\begin{align}
					F \defeq \bigcup_{n \in \Natural} \bigcup_{t \in [0,1]}
					\Set{\omega \in \Omega}{X^{\sigma_{n}}_{t}(\omega) \neq 0}
				\end{align}
				とおくと,$\omega$を$\Omega \backslash F$の任意の要素とし,
				$t$を$[0,1]$の任意の要素とすれば,任意の自然数$n$に対して
				\begin{align}
					X^{\sigma_{n}}_{t}(\omega) = 0
				\end{align}
				が成り立つので
				\begin{align}
					X_{t}(\omega) = 0
				\end{align}
				が従う.
				\QED
		\end{description}
	\end{sketch}
	
	本節の主題は局所マルチンゲールが増大過程と局所マルチンゲールの和に分解できるということである.
	まずは二乗可積分マルチンゲールがマルチンゲールと増大過程に分解できることを示す.
	
	\begin{screen}
		\begin{dfn}[増大過程]
			$\mathbf{T} \times \Omega$上の$\R$値$\{\mathscr{F}_t\}_{t \in \mathbf{T}}$-適合過程$A$で,
			\begin{itemize}
				\item $\Omega$の任意の要素$\omega$に対して$A_0(\omega) = 0$,
				\item $A$のすべてのパスが単調非減少,
				\item $\mathbf{T}$の任意の要素$t$で$E(A_t) < \infty$,
			\end{itemize}
			を満たすものを,$\mathbf{T} \times \Omega$上の
			$\{\mathscr{F}_t\}_{t \in \mathbf{T}}$-{\bf 増大過程}
			\index{ぞうだいかてい@増大過程}{\bf (increasing process)}と呼ぶ.
		\end{dfn}
	\end{screen}
	
	\begin{screen}
		\begin{thm}[局所マルチンゲールの二乗過程は増大過程と局所マルチンゲールに分解できる]
		\label{thm:decomposition_of_local_martingales}
			$X$を$\mathscr{M}_{c}^{2}$の要素とし,$\{\mathscr{F}_{t}\}_{t \in [0,1]}$
			は完備である(定義\ref{thm:completeness_of_filtration})とする.このとき
			\begin{align}
				X^{2} - A \in \mathscr{M}_{c,loc}
			\end{align}
			を満たす$\{\mathscr{F}_{t}\}_{t \in [0,1]}$-増大連続過程$A$が取れる.
		\end{thm}
	\end{screen}
	
	\begin{sketch}\mbox{}
		\begin{description}
			\item[step1-1] $X$が有界であるとする.つまり,$[0,1]$の任意の要素$t$及び$\Omega$の任意の要素$\omega$に対して
				\begin{align}
					|X(t,\omega)| \leq b
				\end{align}
				を満たす実数$b$が取れる.$n$を自然数とし,
				\begin{align}
					(t,\omega) \longmapsto 
					\begin{cases}
						\left( X_{t}(\omega) - X_{0}(\omega) \right)^{2} 
						&\mbox{if } {\displaystyle 0 \leq t < \frac{1}{2^{n}}} \\
						{\displaystyle \sum_{j=0}^{k-1} \left( X_{\frac{j+1}{2^{n}}}(\omega) - X_{\frac{j}{2^{n}}}(\omega) \right)^{2}
						+ \left( X_{t}(\omega) - X_{\frac{k}{2^{n}}}(\omega) \right)^{2}}
						&\mbox{if } {\displaystyle k \in 2^{n} \wedge \frac{k}{2^{n}} \leq t < \frac{k+1}{2^{n}}} \\
						{\displaystyle \sum_{j=0}^{2^{n}-1} \left( X_{\frac{j+1}{2^{n}}}(\omega) - X_{\frac{j}{2^{n}}}(\omega) \right)^{2}}
						&\mbox{if } t = 1
					\end{cases}
				\end{align}
				なる関係で定める$[0,1] \times \Omega$上の写像を$A^{(n)}$とする.このとき
				\begin{align}
					X^2 - A^{(n)} \in \mathscr{M}_{c}^{2}
					\label{fom:thm_decomposition_of_square_integrable_martingales_2}
				\end{align}
				であることを示す.まず$X$の連続性より$A^{(n)}$も連続である.また$t$を$[0,1]$の要素とすれば,
				\begin{align}
					\frac{j}{2^{n}} \leq t
				\end{align}
				を満たす自然数$j$に対して$X_{j/2^{n}}$は$\mathscr{F}_{t}/\borel{\R}$-可測であるから,
				$A^{(n)}$は$\{\mathscr{F}_t\}_{t \in [0,1]}$-適合である.
				そして$[0,1]$の任意の要素$t$及び$\Omega$の任意の要素$\omega$に対して
				\begin{align}
					A^{(n)}_{t}(\omega) 
					\leq \sum_{j=0}^{2^{n}-1} \left( X_{\frac{j+1}{2^{n}}}(\omega) - X_{\frac{j}{2^{n}}}(\omega) \right)^{2} 
					\leq 2^{n+2} \cdot b^{2}
				\end{align}
				が成り立つので${X_{t}}^{2} - A^{(n)}_{t}$は二乗可積分である.いま$t$を
				\begin{align}
					t < 1
				\end{align}
				なる$[0,1]$の要素とし,
				\begin{align}
					\frac{k}{2^{n}} \leq t < \frac{k+1}{2^{n}}
				\end{align}
				を満たす自然数$k$を取る.このとき
				\begin{align}
					A^{(n)}_{1} - A^{(n)}_{t}
					= \sum_{j=k}^{2^{n}-1} \left( X_{\frac{j+1}{2^{n}}} - X_{\frac{j}{2^{n}}} \right)^{2}
					- \left( X_{t} - X_{\frac{k}{2^{n}}} \right)^{2}
				\end{align}
				が成り立つ.ここで$F$を$\mathscr{F}_{t}$の要素とすると,
				\begin{align}
					k = 2^{n} - 1
				\end{align}
				ならば
				\begin{align}
					\int_{F} A^{(n)}_{1} - A^{(n)}_{t}\ dP
					&= \int_{F} {X_{1}}^{2}\ dP - \int_{F} {X_{t}}^{2}\ dP 
					+ 2 \cdot \int_{F} \left(X_{t} - X_{1}\right) \cdot X_{\frac{2^{n}-1}{2^{n}}}\ dP \\
					&= \int_{F} {X_{1}}^{2}\ dP - \int_{F} {X_{t}}^{2}\ dP
					+ 2 \cdot \int_{F} \cexp{\left(X_{t} - X_{1}\right) \cdot X_{\frac{2^{n}-1}{2^{n}}}}{\mathscr{F}_{t}}\ dP \\
					&= \int_{F} {X_{1}}^{2}\ dP - \int_{F} {X_{t}}^{2}\ dP
					+ 2 \cdot \int_{F} \cexp{X_{t} - X_{1}}{\mathscr{F}_{t}} \cdot X_{\frac{2^{n}-1}{2^{n}}}\ dP \\
					&= \int_{F} {X_{1}}^{2}\ dP - \int_{F} {X_{t}}^{2}\ dP
				\end{align}
				が成り立つ.また
				\begin{align}
					k < 2^{n} - 1
				\end{align}
				ならば
				\begin{align}
					A^{(n)}_{1} - A^{(n)}_{t}
					= \sum_{j=k+1}^{2^{n}-1} \left( X_{\frac{j+1}{2^{n}}} - X_{\frac{j}{2^{n}}} \right)^{2}
					+ {X_{\frac{k+1}{2^{n}}}}^{2} - {X_{t}}^{2}
					+ 2 \cdot \left( X_{t} - X_{\frac{k+1}{2^{n}}} \right) \cdot X_{\frac{k}{2^{n}}}
				\end{align}
				が成り立つが,
				\begin{align}
					\int_{F} \sum_{j=k+1}^{2^{n}-1} \left( X_{\frac{j+1}{2^{n}}} - X_{\frac{j}{2^{n}}} \right)^{2}\ dP
					&= \sum_{j=k+1}^{2^{n}-1} \left[\int_{F} {X_{\frac{j+1}{2^{n}}}}^{2}\ dP
					- \int_{F} {X_{\frac{j}{2^{n}}}}^{2}\ dP\right] \\
					&= \int_{F} {X_{1}}^{2}\ dP - \int_{F} {X_{\frac{k+1}{2^{n}}}}^{2}\ dP
				\end{align}
				及び
				\begin{align}
					\int_{F} \left( X_{t} - X_{\frac{k+1}{2^{n}}} \right) \cdot X_{\frac{k}{2^{n}}}\ dP
					= \int_{F} \cexp{X_{t} - X_{\frac{k+1}{2^{n}}}}{\mathscr{F}_{t}} \cdot X_{\frac{k}{2^{n}}}\ dP
					= 0
				\end{align}
				より
				\begin{align}
					\int_{F} A^{(n)}_{1} - A^{(n)}_{t}\ dP = \int_{F} {X_{1}}^{2}\ dP - \int_{F} {X_{t}}^{2}\ dP
				\end{align}
				が成り立つ.つまり,$\mathscr{F}_{t}$の任意の要素$F$に対して
				\begin{align}
					\int_{F} {X_{1}}^{2} - A^{(n)}_{1}\ dP = \int_{F} {X_{t}}^{2} - A^{(n)}_{t}\ dP
				\end{align}
				が成り立つのであるから$X^{2} - A^{(n)}$は$\{\mathscr{F}_{t}\}_{t \in [0,1]}$-マルチンゲールである.
				以上で(\refeq{fom:thm_decomposition_of_square_integrable_martingales_2})が得られた.
				
			\item[step1-2]
				任意の自然数$n$に対して
				\begin{align}
					\int_\Omega \left|{X_{1}}^{2} - A_{1}^{(n)}\right|^{2}\ dP
					\leq 14 \cdot b^{2} \cdot \int_\Omega \left|X_{1}\right|^{2}\ dP
				\end{align}
				が成り立つことを示す.いま
				\begin{align}
					M \defeq X^{2} - A^{(n)}
				\end{align}
				とおけば
				\begin{align}
					\int_\Omega {M_{\frac{j+1}{2^n}}}^{2} - {M_{\frac{j}{2^n}}}^{2}\ dP
					&= \int_\Omega \left(M_{\frac{j+1}{2^n}} - M_{\frac{j}{2^n}} \right)^{2}\ dP \\
					&= \int_\Omega \left\{ {X_{\frac{j+1}{2^n}}}^{2} - {X_{\frac{j}{2^n}}}^{2} -
					\left(A^{(n)}_{\frac{j+1}{2^n}} - A^{(n)}_{\frac{j}{2^n}}\right) \right\}^{2}\ dP \\
					&= \int_\Omega \left\{ {X_{\frac{j+1}{2^n}}}^{2} - {X_{\frac{j}{2^n}}}^{2} -
					\left(X_{\frac{j+1}{2^n}} - X_{\frac{j}{2^n}}\right)^2 \right\}^{2}\ dP
				\end{align}
				が成り立つから,
				\begin{align}
					\int_\Omega \left|M_{1}\right|^{2}\ dP
					&= \sum_{j=0}^{2^n-1} \int_\Omega {M_{\frac{j+1}{2^n}}}^{2} - {M_{\frac{j}{2^n}}}^{2}\ dP \\
					&= \sum_{j=0}^{2^n-1} \int_\Omega \left\{ {X_{\frac{j+1}{2^n}}}^{2} - {X_{\frac{j}{2^n}}}^{2} -
					\left(X_{\frac{j+1}{2^n}} - X_{\frac{j}{2^n}}\right)^2 \right\}^2\ dP \\
					&= \sum_{j=0}^{2^n-1} \int_\Omega \left( {X_{\frac{j+1}{2^n}}}^{2} - {X_{\frac{j}{2^n}}}^{2} \right)^{2}\ dP \\
						&\quad - 2 \cdot \sum_{j=0}^{2^n-1} \int_\Omega \left( {X_{\frac{j+1}{2^n}}}^{2} - {X_{\frac{j}{2^n}}}^{2} \right) \cdot \left(X_{\frac{j+1}{2^n}} - X_{\frac{j}{2^n}}\right)^{2}\ dP \\
						&\quad + \sum_{j=0}^{2^n-1} \int_\Omega \left(X_{\frac{j+1}{2^n}} - X_{\frac{j}{2^n}}\right)^{4}\ dP
					\label{fom:thm_decomposition_of_square_integrable_martingales_1}
				\end{align}
				が成り立つ.ここで
				\begin{align}
					\int_\Omega \left( {X_{\frac{j+1}{2^n}}}^{2} - {X_{\frac{j}{2^n}}}^{2} \right)^{2}\ dP
					\leq 2 \cdot b^{2} \cdot \int_\Omega {X_{\frac{j+1}{2^n}}}^{2} - {X_{\frac{j}{2^n}}}^{2}\ dP
				\end{align}
				かつ
				\begin{align}
					\int_\Omega \left( {X_{\frac{j+1}{2^n}}}^{2} - {X_{\frac{j}{2^n}}}^{2} \right) \cdot \left(X_{\frac{j+1}{2^n}} - X_{\frac{j}{2^n}}\right)^{2}\ dP
					\leq 4 \cdot b^{2} \cdot \int_\Omega {X_{\frac{j+1}{2^n}}}^{2} - {X_{\frac{j}{2^n}}}^{2}\ dP
				\end{align}
				かつ
				\begin{align}
					\int_\Omega \left(X_{\frac{j+1}{2^n}} - X_{\frac{j}{2^n}}\right)^{4}\ dP
					&\leq 4 \cdot b^{2} \cdot \int_\Omega {X_{\frac{j+1}{2^n}}}^{2} - {X_{\frac{j}{2^n}}}^{2}\ dP \\
					&= 4 \cdot b^{2} \cdot \int_\Omega {X_{\frac{j+1}{2^n}}}^{2} - {X_{\frac{j}{2^n}}}^{2}\ dP
				\end{align}
				が成り立つので
				\begin{align}
					(\refeq{fom:thm_decomposition_of_square_integrable_martingales_1})
					&\leq 14 \cdot b^{2} \cdot \sum_{j=0}^{2^n-1} \int_\Omega {X_{\frac{j+1}{2^n}}}^{2} - {X_{\frac{j}{2^n}}}^{2}\ dP \\
					&= 14 \cdot b^{2} \cdot \int_\Omega \left|X_{1}\right|^{2}\ dP
				\end{align}
				が成立する.
				
			\item[step1-3]
				$\mathscr{M}_{c}^{2}$に定理\ref{thm:pseudo_metric_on_square_integrable_martingales}の
				擬距離$d$を定めると,step1-2より
				\begin{align}
					\left\{X^{2} - A^{(n)}\right\}_{n \in \Natural}
				\end{align}
				は$\mathscr{M}_{c}^{2}$の$d$-有界集合である.ゆえにKomlosの補題より
				$\left(\mathscr{M}_{c}^{2},d\right)$のCauchy列$h$で,任意の自然数$n$で
				\begin{align}
					h(n) \in \conv{\Set{X^{2}-A^{(k)}}{k \in \Natural \wedge n \leq k}}
					\label{fom:thm_decomposition_of_square_integrable_martingales_5}
				\end{align}
				を満たすものが取れて,定理\ref{thm:pseudo_metric_on_square_integrable_martingales}より
				$h$の極限$M$が取れる.ここで
				\begin{align}
					B \defeq X^{2} - M
				\end{align}
				とおく.定め方より$B$は連続かつ$\{\mathscr{F}_t\}_{t \in [0,1]}$-適合である.
				あとは$B$の増大性を示せばよい.$\Natural$から$\Natural$への写像$f$で,
				\begin{itemize}
					\item $n$と$m$を任意の自然数とするとき,
						\begin{align}
							n < m
						\end{align}
						ならば
						\begin{align}
							f(n) < f(m).
						\end{align}
						
					\item $n$を任意の自然数とするとき
						\begin{align}
							n \leq f(m)
						\end{align}
						を満たす自然数$m$が取れる.
						
					\item 任意の自然数$n$に対して
						\begin{align}
							d(h(f(n)),M) < \frac{1}{4^{n+1}}
						\end{align}
				\end{itemize}
				を満たすものが取れる.このとき,Doobの劣マルチンゲール不等式より任意の自然数$n$で
				\begin{align}
					\int_\Omega \left\{\sup{t \in [0,1]}{\left|{X_{t}}^{2} - h(f(n))_{t} - B_{t}\right|}\right\}^{2}\ dP
					&= \int_\Omega \left\{\sup{t \in [0,1]}{\left|h(f(n))_{t} - M_{t}\right|}\right\}^{2}\ dP \\
					&\leq 4 \cdot \int_\Omega \left|h(f(n))_{1} - M_{1}\right|^{2}\ dP \\
					&< \frac{1}{8^{n}}
				\end{align}
				が成り立つので,任意の自然数$n$で
				\begin{align}
					P\left(\Set{\omega \in \Omega}{
					\frac{1}{2^{n}} \leq \sup{t \in [0,1]}{\left|{X_{t}(\omega)}^{2} - h(f(n))_{t}(\omega) - B_{t}(\omega) \right|}}\right)
					\leq \frac{1}{2^{n}}
				\end{align}
				が成立する.ゆえに
				\begin{align}
					F \defeq \bigcap_{n \in \Natural} \bigcup_{\substack{k \in \Natural \\ n < k}} 
					\Set{\omega \in \Omega}{\frac{1}{2^{k}} \leq \sup{t \in [0,1]}{\left|{X_{t}(\omega)}^{2} - h(f(k))_{t}(\omega) - B_{t}(\omega) \right|}}
				\end{align}
				により定める$F$は$P$-零集合である.
				いま$\omega$を$\Omega \backslash F$の要素とし,$s$と$t$を
				\begin{align}
					s < t
				\end{align}
				なる$[0,1]$の要素とする.$\epsilon$を任意に与えられた正の実数とすると
				\begin{align}
					\frac{1}{2^{k-1}} < \epsilon
				\end{align}
				及び
				\begin{align}
					\sup{t \in [0,1]}{\left|{X_{t}(\omega)}^{2} - h(f(k))_{t}(\omega) - B_{t}(\omega)\right|} < \frac{1}{2^{k}}
				\end{align}
				を満たす自然数$k$が取れるが,このとき
				\begin{align}
					{X_{s}(\omega)}^{2} - h(f(k))_{s}(\omega)
					\leq {X_{t}(\omega)}^{2} - h(f(k))_{t}(\omega)
					\label{fom:thm_decomposition_of_local_martingales_1}
				\end{align}
				が成立する.実際,
				\begin{align}
					h(f(k)) \in \conv{\Set{X^{2} - A^{(j)}}{j \in \Natural \wedge f(k) \leq j}}
				\end{align}
				であるから,$1$以上の自然数$m$,及び$m$上の写像$c$と$g$で
				\begin{itemize}
					\item $c$は$m$から$\R_{+}$への写像で
						\begin{align}
							\sum_{i=0}^{m-1} c(i) = 1,
						\end{align}
						
					\item $g$は$m$から$\Natural$への写像で,$m$の任意の要素$i$に対して
						\begin{align}
							f(k) \leq g(i)
						\end{align}
				\end{itemize}
				を満たすものが取れて
				\begin{align}
					h(f(k)) = \sum_{i=0}^{m-1} c(i) \cdot \left( X^{2} - A^{(g(i))} \right)
				\end{align}
				が成立する.すなわち
				\begin{align}
					X^{2} - h(f(k)) = \sum_{i=0}^{m-1} c(i) \cdot A^{(g(i))}
				\end{align}
				が成立する.$m$の任意の要素$i$に対して
				\begin{align}
					c(i) \cdot A^{(g(i))}_{s}(\omega) 
					\leq c(i) \cdot A^{(g(i))}_{t}(\omega) 
				\end{align}
				が成り立つので(\refeq{fom:thm_decomposition_of_local_martingales_1})が従う.
				ゆえに
				\begin{align}
					B_{s}(\omega) - \frac{1}{2^{k}}
					< {X_{s}(\omega)}^{2} - h(f(k))_{s}(\omega)
					\leq {X_{t}(\omega)}^{2} - h(f(k))_{t}(\omega)
					< B_{t}(\omega) + \frac{1}{2^{k}}
				\end{align}
				が成り立つから
				\begin{align}
					B_{s}(\omega) - \epsilon < B_{t}(\omega)
				\end{align}
				が従い,$\epsilon$の任意性より
				\begin{align}
					B_{s}(\omega) \leq B_{t}(\omega)
				\end{align}
				が得られる.最後に
				\begin{align}
					(t,\omega) \longmapsto
					\begin{cases}
						B_{t}(\omega) & \mbox{if } \omega \in \Omega \backslash F \\
						0 & \mbox{if } \omega \in F
					\end{cases}
				\end{align}
				なる$[0,1] \times \Omega$上の写像を$A$とおけば,$A$は
				\begin{align}
					X^{2} - A \in \mathscr{M}_{c}^{2}
				\end{align}
				を満たす$\{\mathscr{F}_{t}\}_{t \in [0,1]}$-増大過程である.
				
			\item[step2-1]
				定理\ref{thm:increasing_stopping_times_which_locally_bound_martingale}より
				$\{\mathscr{F}_{t}\}_{t \in [0,1]}$-増大停止時刻列$\tau$が取れて,任意の自然数$n$に対して
				$X^{\tau_{n}}$は有界$\{\mathscr{F}_{t}\}_{t \in [0,1]}$-マルチンゲールをなす.
				すなわちstep1-3と
				定理\ref{thm:direct_product_of_non_empty_sets_is_not_empty}より,
				任意の自然数$n$に対して,
				\begin{align}
					\left(X^{\tau_{n}}\right)^{2} - A_{n} \in \mathscr{M}^{2}_{c}
				\end{align}
				を満たす$\{\mathscr{F}_{t}\}_{t \in [0,1]}$-増大過程$A_{n}$(を対応させる$\Natural$上の写像)が取れる.
				$n$と$m$を
				\begin{align}
					n < m
				\end{align}
				なる自然数とすると,
				\begin{align}
					X^{\tau_{n}} = \left(X^{\tau_{m}}\right)^{\tau_{n}}
				\end{align}
				が成り立つので
				\begin{align}
					\left(X^{\tau_{n}}\right)^{2} - A_{m}^{\tau_{n}} \in \mathscr{M}_{c}^{2}
				\end{align}
				が従う.ゆえに定理\ref{thm:local_martingale_of_bounded_variation_is_constant}より
				\begin{align}
					\Set{\omega \in \Omega}{\exists t \in [0,1]\, \left(\, A_{n}(t,\omega) \neq A_{m}^{\tau_{n}}(t,\omega)\, \right)}
				\end{align}
				は$P$-零集合である.ここで
				\begin{align}
					F \defeq \bigcup_{\substack{n,m \in \Natural \\ n < m}} \Set{\omega \in \Omega}{\exists t \in [0,1]\, \left(\, A_{n}(t,\omega) \neq A_{m}^{\tau_{n}}(t,\omega)\, \right)}
				\end{align}
				とおいて,$A$を
				\begin{align}
					(t,\omega) \longmapsto
					\begin{cases}
						A_{n}(t,\omega) & \mbox{if } \omega \notin F \wedge t \leq \tau_{n}(\omega) \\
						0 & \mbox{if } \omega \in F
					\end{cases}
				\end{align}
				なる$[0,1] \times \Omega$上の写像とすれば(写像であることは次段以降で示す),つまり正式には
				\begin{align}
					A \defeq \{\, x \mid \quad 
					\exists t \in [0,1]\, \exists \omega \in \Omega\, \exists a\,
					&[\, x = ((t,\omega),a) \wedge \\
					&\quad (\, \omega \notin F \Longrightarrow \forall n \in \Natural\, 
					(\, t \leq \tau_{n}(\omega) \Longrightarrow a = A_{n}(t,\omega)\, )\, )\, \\
					&\quad \wedge (\, \omega \in F \Longrightarrow a = 0\, )\, ]\, \}
				\end{align}
				と定めれば,
				\begin{align}
					X^{2} - A \in \mathscr{M}_{c,loc}
				\end{align}
				が成立する.実際,$[0,1]$の任意の要素$t$及び$\Omega \backslash F$の任意の要素$\omega$に対して
				\begin{align}
					{X^{\tau_{n}}(t,\omega)}^{2} - A^{\tau_{n}}(t,\omega)
					= {X^{\tau_{n}}(t,\omega)}^{2} - A_{n}(t,\omega)
				\end{align}
				が成り立つので
				\begin{align}
					\left(X^{\tau_{n}}\right)^{2} - A^{\tau_{n}} \in \mathscr{M}_{c}^{2}
				\end{align}
				が成立する.また$\omega$を$\Omega$の任意の要素とすれば
				\begin{align}
					1 = \tau_{n}(\omega)
				\end{align}
				を満たす自然数$n$が取れるが,このとき
				\begin{align}
					
				\end{align}
				
				
				\begin{align}
					(t,\omega) \longmapsto
					\begin{cases}
						\lim_{n \to \infty} A^n_{t \wedge \tau_n(\omega)}(\omega) & \\
						0
					\end{cases}
				\end{align}
				として$A$を定めると$A$は適合,増大,(右)連続
				\begin{align}
					A_{t \wedge \tau_n} = A^n_{t \wedge \tau_n}
				\end{align}
				が成り立つ.
				
			\item[step2-2]
				$A$がsingle-valuedであることを示す.
				任意の集合$x$と$y$と$z$に対して,
				\begin{align}
					(x,y) \in A
				\end{align}
				かつ
				\begin{align}
					(x,z) \in A
				\end{align}
				ならば
				\begin{align}
					x = (t,\omega)
				\end{align}
				を満たす$[0,1]$の要素$t$と$\Omega$の要素$\omega$が取れる.このとき
				\begin{align}
					\omega \in F
				\end{align}
				ならば
				\begin{align}
					y = 0 = z
				\end{align}
				が成り立ち,
				\begin{align}
					\omega \notin F
				\end{align}
				ならば,
				\begin{align}
					1 = \tau_{n}(\omega)
				\end{align}
				なる自然数$n$を取れば
				\begin{align}
					y = A^{(n)}(t,\omega) = z
				\end{align}
				が成り立つ.ゆえに
				\begin{align}
					\forall x,y,z\, \left[\, 
					(x,y) \in A \wedge (x,z) \in A \Longrightarrow y = z\, \right]
				\end{align}
				が得られた.
				
			\item[step2-3]
				$A$の定義域が$[0,1] \times \Omega$に一致することを示す.
				$x$を$\dom{A}$の要素とすれば,
				\begin{align}
					(x,y) \in A
				\end{align}
				を満たす集合$y$が取れて
				\begin{align}
					(x,y) = ((t,\omega),a)
				\end{align}
				を満たす$[0,1]$の要素$t$,$\Omega$の要素$\omega$,及び集合$a$が取れる.すなわち
				\begin{align}
					x \in [0,1] \times \Omega
				\end{align}
				が従う,逆に$x$を$[0,1] \times \Omega$の要素とすれば
				\begin{align}
					x = (t,\omega)
				\end{align}
				なる$[0,1]$の要素$t$と$\Omega$の要素$\omega$が取れる.
				\begin{align}
					\omega \in F
				\end{align}
				であれば
				\begin{align}
					((t,\omega),0) \in A
				\end{align}
				が成り立つ.すなわち
				\begin{align}
					\exists y\, \left(\, (x,y) \in A\, \right)
				\end{align}
				が成り立つ.
				\begin{align}
					\omega \notin F
				\end{align}
				であるとき,
				\begin{align}
					1 = \tau_{n}(\omega)
				\end{align}
				を満たす自然数$n$を取れば
				\begin{align}
					\left((t,\omega),A^{(n)}(t,\omega)\right) \in A
				\end{align}
				が成り立つ.実際,任意の自然数$m$に対して,
				\begin{align}
					t \leq \tau_{m}(\omega)
				\end{align}
				ならば
				\begin{align}
					A^{(n)}(t,\omega) = A^{(m)}(t,\omega)
				\end{align}
				が成り立つ.ゆえに
				\begin{align}
					x \in [0,1] \times \Omega \Longrightarrow 
					\exists y\, \left(\, (x,y) \in A\, \right)
				\end{align}
				が得られた.
				
			\item[step2-4]
				$A$が$\R$値であることを示す.$y$を$\ran{A}$の要素とすると
				\begin{align}
					(x,y) \in A
				\end{align}
				を満たす集合$x$が取れて,
				\begin{align}
					x = (t,\omega)
				\end{align}
				を満たす$[0,1]$の要素$t$と$\Omega$の要素$\omega$が取れる.このとき
				\begin{align}
					\omega \in F
				\end{align}
				ならば
				\begin{align}
					y = 0
				\end{align}
				が従い,
				\begin{align}
					\omega \notin F
				\end{align}
				ならば,
				\begin{align}
					1 = \tau_{n}(\omega)
				\end{align}
				を満たす自然数$n$を取れば
				\begin{align}
					y = A^{(n)}(t,\omega)
				\end{align}
				が成立するので,
				\begin{align}
					\exists x\, (\, (x,y) \in A\, ) \Longrightarrow y \in \R
				\end{align}
				を得る.
				\QED
		\end{description}
	\end{sketch}
	
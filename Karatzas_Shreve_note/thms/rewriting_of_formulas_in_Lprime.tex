\subsection{式の書き換え}
	\monologue{
		院生「我々は$\mathcal{L}$の式$A$を用いて$\Set{x}{A(x)}$の記法を導入しましたが,
			$\mathcal{L}'$の式$B$に対しても$\Set{x}{B(x)}$の形で書けると便利なことが多いです.
			ただし後者の記法は$B$と同値な$\mathcal{L}$の式$B'$によって
			\begin{align}
				\Set{x}{B(x)} \coloneqq \Set{x}{B'(x)}
			\end{align}
			で定められるものとします.$\mathcal{L}'$の式が与えられたらそれを
			或る手続きで$\mathcal{L}$の式に書き換えていくのですが,
			そこで真価を発揮するのは$\varepsilon$記号です.」
	}
	
	$a$を類とするとき,$a$は$\mathcal{L}$の対象であるか$\Set{x}{A(x)}$の形をしている.そこで,文字$x$に対し
	\begin{itemize}
		\item $a$が$\mathcal{L}$の対象ならば$\varepsilon a(x) \overset{\mathrm{def}}{\Longleftrightarrow} x \in a$,
		\item $a$が$\Set{x}{A(x)}$の形をしていれば$\varepsilon a(x) \overset{\mathrm{def}}{\Longleftrightarrow} A(x)$,
	\end{itemize}
	として記号列$\varepsilon a(x)$を定める.この記法は
	\begin{align}
		\forall x\, (\, \varepsilon a(x) \Longleftrightarrow x \in a\, )
		\label{eq:a_meaning_of_epsilon_notation}
	\end{align}
	を満たすことを意図している.$\varepsilon$記号を用いているのは,
	量化記号に関する推論規則で$\varepsilon$記号を定めたときと導入の動機が似ているためである.
	
	次に$B$を$\mathcal{L}'$の式として,$B$を$\mathcal{L}$の式に書き換える手続きを指定する.
	\begin{description}
		\item[step1] $B$が$\mathcal{L}$の式であるとき,
			\begin{align}
				\mathcal{L}B \overset{\mathrm{def}}{\Longleftrightarrow} B
			\end{align}
			と定める.そうでない場合の対応を以下に示す.
			
		\item[step2] $s,t$を$\mathcal{L}'$の項として,$B$が
			\begin{align}
				s \in t
			\end{align}
			であるとき,
			\begin{itemize}
				\item $s,t$が共に$\mathcal{L}$の項であるとき
					\begin{align}
						\mathcal{L}B \overset{\mathrm{def}}{\Longleftrightarrow}
						s \in t,
					\end{align}
				
				\item $s$が$\mathcal{L}$の項ではなく,$t$が$\mathcal{L}$の項であるとき
					\begin{align}
						\mathcal{L}B \overset{\mathrm{def}}{\Longleftrightarrow}
						\varepsilon x\, (\, s=x\, ) \in t,
					\end{align}
				
				\item $s$が$\mathcal{L}$の項であり,$t$が$\mathcal{L}$の項でないとき
					\begin{align}
						\mathcal{L}B \overset{\mathrm{def}}{\Longleftrightarrow}
						\varepsilon t(s),
					\end{align}
					
				\item $s$も$t$も$\mathcal{L}$の項でないとき
					\begin{align}
						\mathcal{L}B \overset{\mathrm{def}}{\Longleftrightarrow}
						\varepsilon t(\varepsilon x\, (\, s=x\, )),
					\end{align}
			\end{itemize}
			と定める.一方で$B$が
			\begin{align}
				s = t
			\end{align}
			であるとき,
			\begin{itemize}
				\item $s,t$が共に$\mathcal{L}$の項であるとき
					\begin{align}
						\mathcal{L}B \overset{\mathrm{def}}{\Longleftrightarrow}
						s = t,
					\end{align}
				
				\item $s$が$\mathcal{L}$の項ではなく,$t$が$\mathcal{L}$の項であるとき,
					$s$が集合なら
					\begin{align}
						\mathcal{L}B \overset{\mathrm{def}}{\Longleftrightarrow}
						\varepsilon x\, (\, s=x\, ) = t,
					\end{align}
					$s$が真類なら
					\begin{align}
						\mathcal{L}B \overset{\mathrm{def}}{\Longleftrightarrow}
						\forall u\, \left(\, \varepsilon s(u) \Longleftrightarrow u \in t\, \right),
					\end{align}
				
				\item $s$が$\mathcal{L}$の項であり,$t$が$\mathcal{L}$の項でないとき,
					$t$が集合なら
					\begin{align}
						\mathcal{L}B \overset{\mathrm{def}}{\Longleftrightarrow}
						s = \varepsilon x\, (\, t=x\, ),
					\end{align}
					$t$が真類なら
					\begin{align}
						\mathcal{L}B \overset{\mathrm{def}}{\Longleftrightarrow}
						\forall u\, \left(\, u \in s \Longleftrightarrow \varepsilon t(u)\, \right),
					\end{align}
					
				\item $s$も$t$も$\mathcal{L}$の項でないとき,
					$s,t$がどちらも集合なら
					\begin{align}
						\mathcal{L}B \overset{\mathrm{def}}{\Longleftrightarrow}
						\varepsilon x\, (\, s=x\, ) = \varepsilon x\, (\, t=x\, ),
					\end{align}
					$s,t$の一方でも真類なら
					\begin{align}
						\mathcal{L}B \overset{\mathrm{def}}{\Longleftrightarrow}
						\forall u\, \left(\, \varepsilon s(u) \Longleftrightarrow \varepsilon t(u)\, \right),
					\end{align}
			\end{itemize}
			と定める.
			
		\item[step3] $P,Q$を$\mathcal{L}'$の式として,$B$が
			\begin{align}
				P \vee Q,\quad P \wedge Q,\quad P \Longrightarrow Q,\quad \rightharpoondown P
			\end{align}
			であるとき,
			\begin{align}
				\mathcal{L}P,\mathcal{L}Q
			\end{align}
			をそれぞれ$P,Q$から得られた$\mathcal{L}$の式として,各場合に応じて
			\begin{align}
				\mathcal{L}B \overset{\mathrm{def}}{\Longleftrightarrow}
				\begin{cases}
					\mathcal{L}P \vee \mathcal{L}Q & \\
					\mathcal{L}P \wedge \mathcal{L}Q & \\
					\mathcal{L}P \Longrightarrow \mathcal{L}Q & \\
					\rightharpoondown \mathcal{L}P &
				\end{cases}
			\end{align}
			と定める.
	\end{description}
	
	$B$を$\mathcal{L}'$の式とし,$x$を$B$に現れる文字とし,
	$B$に現れる文字で$x$のみが量化されていないとき,
	\begin{align}
		\Set{x}{B(x)} \coloneqq \Set{x}{\mathcal{L}B(x)}
	\end{align}
	と定義する.このとき$y$を$B(x)$に現れない文字とすれば
	\begin{align}
		\forall y\, \left(\, B(y) \Longleftrightarrow y \in \Set{x}{B(x)}\, \right)
	\end{align}
	が成立する.
	
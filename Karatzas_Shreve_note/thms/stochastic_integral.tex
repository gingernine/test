\subsection{確率積分}
	$(\Omega,\mathscr{F},P)$を確率空間とし,$\mathbf{T}$を$[0,T]$とし,
	$\{\mathscr{F}_t\}_{t \in \mathbf{T}}$を$\mathscr{F}$に付随するフィルトレーションとし,
	\begin{align}
		\Set{a}{a \in \mathscr{F} \wedge P(a) = 0} \subset \mathscr{F}_0
	\end{align}
	が満たされているとする.また$M$を$\mathscr{M}^2_{\mathbf{T}}$の要素か,或いは$\mathscr{M}^{2,c}_{\mathbf{T}}$の要素とする.ただし
	\begin{align}
		M \in \mathscr{M}^2_{\mathbf{T}}
	\end{align}
	の場合は$\{\mathscr{F}_t\}_{t \in \mathbf{T}}$は右連続であるとする.
	また$\inprod<M>$の$\omega$に対する標本路(表記法はP. \pageref{def:omega_to_path_mapping})
	\begin{align}
		\inprod<M>_\bullet(\omega)
	\end{align}
	で構成する$\borel{\mathbf{T}}$上のStieltjes測度を
	\begin{align}
		s_{M,\omega}
	\end{align}
	と書く.
	
	本節の始めでは,可予測集合$A$に対して
	\begin{align}
		\int_\Omega \int_{\mathbf{T}} \defunc_A(t,\omega)\ s_{M,\omega}(dt)\ P(d\omega)
	\end{align}
	という積分を正当化する.
	
	いま$A$を$\mathscr{P}_{\mathbf{T}}$の要素とすれば,
	\begin{align}
		A \in \borel{\mathbf{T}} \otimes \mathscr{F}_T
	\end{align}
	であるから$\Omega$の各要素$\omega$で
	\begin{align}
		\mathbf{T} \ni t \longmapsto \defunc_A(t,\omega)
	\end{align}
	なる写像は$\borel{\mathbf{T}}/\borel{\R}$-可測である.ゆえに
	\begin{align}
		\int_{\mathbf{T}} \defunc_A(t,\omega)\ s_{M,\omega}(dt)
	\end{align}
	という積分は各$\omega$で存在している.実際この積分は
	\begin{align}
		\int_{\mathbf{T}} \defunc_A(t,\omega)\ s_{M,\omega}(dt)
		\leq \int_{\mathbf{T}} \defunc_{\mathbf{T} \times \Omega}(t,\omega)\ s_{M,\omega}(dt)
		= \inprod<M>_T(\omega)
	\end{align}
	なる不等式を満たすから実数値で確定している.
	
	測度$s_{M,\omega}$は$\omega$に依存しているため,
	\begin{align}
		\Omega \ni \omega \longmapsto \int_{\mathbf{T}} \defunc_A(t,\omega)\ s_{M,\omega}(dt)
	\end{align}
	の可測性はFubiniの定理の適用では得られない.だが以下が示される.
	
	\begin{itembox}[l]{Stieltjes積分は$\omega$の関数として可測}
		$A$を$\mathscr{P}_{\mathbf{T}}$の要素とすれば
		\begin{align}
			\Omega \ni \omega \longmapsto \int_{\mathbf{T}} \defunc_A(t,\omega)\ s_{M,\omega}(dt)
		\end{align}
		は$\mathscr{F}_T/\borel{\R}$-可測である.
	\end{itembox}
	
	\begin{sketch}
		いま
		\begin{align}
			\mathscr{D} \defeq
			\Set{A \in \mathscr{P}_{\mathbf{T}}}{\mbox{$\omega \longmapsto \int_{\mathbf{T}} \defunc_A(t,\omega)\ s_{M,\omega}(dt)$が$\mathscr{F}_T/\borel{\R}$-可測}}
		\end{align}
		によりDynkin族を定めて,また$A$を$\mathscr{U}_{\mathbf{T}}$の要素とする.$\mathscr{F}_0$の要素$B$によって
		\begin{align}
			A = \{0\} \times B
		\end{align}
		が成り立っているとき,$\Omega$の任意の要素$\omega$に対して
		\begin{align}
			\int_{\mathbf{T}} \defunc_A(t,\omega)\ s_{M,\omega}(dt)
			&= \defunc_B(\omega) \cdot \int_{\mathbf{T}} \defunc_{\{0\}}(t)\ s_{M,\omega}(dt) \\
			&= \defunc_B(\omega) \cdot \left(\inprod<M>_0(\omega) - \inprod<M>_0(\omega)\right) \\
			&= 0
		\end{align}
		が成り立つので
		\begin{align}
			A \in \mathscr{D}
		\end{align}
		が成り立つ.
		\begin{align}
			s < t
		\end{align}
		なる$\mathbf{T}$の要素と$\mathscr{F}_s$の要素$B$によって
		\begin{align}
			A = ]s,t] \times B
		\end{align}
		が成り立っているとき,$\Omega$の任意の要素$\omega$に対して
		\begin{align}
			\int_{\mathbf{T}} \defunc_A(t,\omega)\ s_{M,\omega}(dt)
			&= \defunc_B(\omega) \cdot \int_{\mathbf{T}} \defunc_{]s,t]}(t)\ s_{M,\omega}(dt) \\
			&= \defunc_B(\omega) \cdot \left(\inprod<M>_t(\omega) - \inprod<M>_s(\omega)\right)
		\end{align}
		が成り立つので
		\begin{align}
			A \in \mathscr{D}
		\end{align}
		が成り立つ.ゆえに
		\begin{align}
			\mathscr{U}_{\mathbf{T}} \subset \mathscr{D}
		\end{align}
		が成り立つ.ゆえにDynkin族定理より
		\begin{align}
			\mathscr{P}_{\mathbf{T}} = \mathscr{D}
		\end{align}
		が従う.
		\QED
	\end{sketch}
	
	$\mathscr{P}_{\mathbf{T}}$上の写像$\nu_M$を
	\begin{align}
		\mathscr{P}_{\mathbf{T}} \ni A \longmapsto
		\int_\Omega \int_{\mathbf{T}} \defunc_A(t,\omega)\ s_{M,\omega}(dt)\ P(d\omega)
	\end{align}
	なる関係により定める.
	
	\begin{screen}
		\begin{thm}[二乗可積分マルチンゲールで構成する測度]
			$\nu_M$は$\mathscr{P}_{\mathbf{T}}$上の正値有限測度である.
		\end{thm}
	\end{screen}
	
	\begin{sketch}
		先ず
		\begin{align}
			\int_\Omega \int_{\mathbf{T}} \defunc_{\mathbf{T} \times \Omega}(t,\omega)\ s_{M,\omega}(dt)\ P(d\omega)
			= \int_\Omega \inprod<M>_T\ dP
		\end{align}
		が成り立つが,
		\begin{align}
			M^2 - \inprod<M>
		\end{align}
		はマルチンゲールであり$M_T^2$は可積分なので
		\begin{align}
			\int_\Omega \inprod<M>_T\ dP < \infty
		\end{align}
		が従う.ゆえに$\nu_M$は有限値しか取らない.また
		\begin{align}
			\left\{A_n\right\}_{n \in \Natural}
		\end{align}
		を互いに素な$\mathscr{P}_{\mathbf{T}}$の部分集合とすると,$\Omega$の任意の要素$\omega$に対して
		単調収束定理より
		\begin{align}
			\int_{\mathbf{T}} \defunc_{\bigcup_{n \in \Natural}A_n}(t,\omega)\ s_{M,\omega}(dt)
			= \sum_{n \in \Natural} \int_{\mathbf{T}} \defunc_{A_n}(t,\omega)\ s_{M,\omega}(dt)
		\end{align}
		が成り立ち,再び単調収束定理より
		\begin{align}
			\int_\Omega \sum_{n \in \Natural} \int_{\mathbf{T}} \defunc_{A_n}(t,\omega)\ s_{M,\omega}(dt)\ P(d\omega)
			= \sum_{n \in \Natural} \int_\Omega \int_{\mathbf{T}} \defunc_{A_n}(t,\omega)\ s_{M,\omega}(dt)\ P(d\omega)
		\end{align}
		が成り立つ.ゆえに$\nu_M$は
		\begin{align}
			\bigcup_{n \in \Natural} A_n \longmapsto \sum_{n \in \Natural} \nu_M(A_n)
		\end{align}
		を満たす.ゆえに$\nu_M$は$\mathscr{P}_{\mathbf{T}}$上の正値有限測度である.
		\QED
	\end{sketch}
	
	$\mathscr{L}^1(\mathbf{T} \times \Omega,\mathscr{P}_{\mathbf{T}},\nu_M)$を
	$\mathscr{L}^1(\nu_M)$と略記する.
	
	\begin{screen}
		\begin{thm}[可積分可予測過程の積分表現]
			$f$を$\mathscr{L}^1(\nu_M)$の要素とすると
			\begin{align}
				\int_{\mathbf{T} \times \Omega} f\ d\nu_M 
				= \int_\Omega \int_{\mathbf{T}} f(t,\omega)\ s_{M,\omega}(dt)\ P(d\omega).
			\end{align}
		\end{thm}
	\end{screen}
	
	\begin{sketch}
		
	\end{sketch}
	
	\begin{comment}
	いま,$\mathscr{P}_{\mathbf{T}}$の要素に対する定義関数の全体の線型包を$\mathscr{S}$と定める:
	\begin{align}
		\mathscr{S} \defeq \operatorname{Span}\Set{\defunc_A}{A \in \mathscr{P}_{\mathbf{T}}}.
	\end{align}
	このとき$\mathscr{S}$は$\mathscr{L}^2(\nu_M)$において,セミノルム$\Norm{\cdot}{\mathscr{L}^2(\nu_M)}$に関して稠密となる.
	
	\begin{screen}
		\begin{thm}
			
		\end{thm}
	\end{screen}
	\end{comment}
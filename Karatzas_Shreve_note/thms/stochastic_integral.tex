\subsection{確率積分}
	$(\Omega,\mathscr{F},P)$を確率空間とし,$\mathbf{T}$を$[0,T]$とし,
	$\{\mathscr{F}_t\}_{t \in \mathbf{T}}$を$\mathscr{F}$に付随するフィルトレーションとし,
	\begin{align}
		\Set{a}{a \in \mathscr{F} \wedge P(a) = 0} \subset \mathscr{F}_0
	\end{align}
	が満たされているとする.また$M$を$\mathscr{M}^2_{\mathbf{T}}$の要素か,或いは$\mathscr{M}^{2,c}_{\mathbf{T}}$の要素とする.ただし
	\begin{align}
		M \in \mathscr{M}^2_{\mathbf{T}}
	\end{align}
	の場合は$\{\mathscr{F}_t\}_{t \in \mathbf{T}}$は右連続であるとする.
	また$\inprod<M>$の$\omega$に対する標本路(表記法はP. \pageref{def:omega_to_path_mapping})
	\begin{align}
		\inprod<M>_\bullet(\omega)
	\end{align}
	で構成する$\borel{\mathbf{T}}$上のStieltjes測度を
	\begin{align}
		s_{M,\omega}
	\end{align}
	と書く.
	
	本節の始めでは,可予測集合$A$に対して
	\begin{align}
		\int_\Omega \int_{\mathbf{T}} \defunc_A(t,\omega)\ s_{M,\omega}(dt)\ P(d\omega)
	\end{align}
	という積分を正当化する.
	
	いま$A$を$\mathscr{P}_{\mathbf{T}}$の要素とすれば,
	\begin{align}
		A \in \borel{\mathbf{T}} \otimes \mathscr{F}_T
	\end{align}
	であるから$\Omega$の各要素$\omega$で
	\begin{align}
		\mathbf{T} \ni t \longmapsto \defunc_A(t,\omega)
	\end{align}
	なる写像は$\borel{\mathbf{T}}/\borel{\R}$-可測である.ゆえに
	\begin{align}
		\int_{\mathbf{T}} \defunc_A(t,\omega)\ s_{M,\omega}(dt)
	\end{align}
	という積分は各$\omega$で存在している.実際この積分は
	\begin{align}
		\int_{\mathbf{T}} \defunc_A(t,\omega)\ s_{M,\omega}(dt)
		\leq \int_{\mathbf{T}} \defunc_{\mathbf{T} \times \Omega}(t,\omega)\ s_{M,\omega}(dt)
		= \inprod<M>_T(\omega)
	\end{align}
	なる不等式を満たすから実数値で確定している.
	
	測度$s_{M,\omega}$は$\omega$に依存しているため,
	\begin{align}
		\Omega \ni \omega \longmapsto \int_{\mathbf{T}} \defunc_A(t,\omega)\ s_{M,\omega}(dt)
	\end{align}
	の可測性はFubiniの定理の適用では得られない.だが以下が示される.
	
	\begin{itembox}[l]{Stieltjes積分は$\omega$の関数として可測}
		$A$を$\mathscr{P}_{\mathbf{T}}$の要素とすれば
		\begin{align}
			\Omega \ni \omega \longmapsto \int_{\mathbf{T}} \defunc_A(t,\omega)\ s_{M,\omega}(dt)
		\end{align}
		は$\mathscr{F}_T/\borel{\R}$-可測である.
	\end{itembox}
	
	\begin{sketch}
		いま
		\begin{align}
			\mathscr{D} \defeq
			\Set{A \in \mathscr{P}_{\mathbf{T}}}{\mbox{$\omega \longmapsto \int_{\mathbf{T}} \defunc_A(t,\omega)\ s_{M,\omega}(dt)$が$\mathscr{F}_T/\borel{\R}$-可測}}
		\end{align}
		によりDynkin族を定めて,また$A$を$\mathscr{U}_{\mathbf{T}}$の要素とする.$\mathscr{F}_0$の要素$B$によって
		\begin{align}
			A = \{0\} \times B
		\end{align}
		が成り立っているとき,$\Omega$の任意の要素$\omega$に対して
		\begin{align}
			\int_{\mathbf{T}} \defunc_A(t,\omega)\ s_{M,\omega}(dt)
			&= \defunc_B(\omega) \cdot \int_{\mathbf{T}} \defunc_{\{0\}}(t)\ s_{M,\omega}(dt) \\
			&= \defunc_B(\omega) \cdot \left(\inprod<M>_0(\omega) - \inprod<M>_0(\omega)\right) \\
			&= 0
		\end{align}
		が成り立つので
		\begin{align}
			A \in \mathscr{D}
		\end{align}
		が成り立つ.
		\begin{align}
			s < t
		\end{align}
		なる$\mathbf{T}$の要素と$\mathscr{F}_s$の要素$B$によって
		\begin{align}
			A = ]s,t] \times B
		\end{align}
		が成り立っているとき,$\Omega$の任意の要素$\omega$に対して
		\begin{align}
			\int_{\mathbf{T}} \defunc_A(t,\omega)\ s_{M,\omega}(dt)
			&= \defunc_B(\omega) \cdot \int_{\mathbf{T}} \defunc_{]s,t]}(t)\ s_{M,\omega}(dt) \\
			&= \defunc_B(\omega) \cdot \left(\inprod<M>_t(\omega) - \inprod<M>_s(\omega)\right)
		\end{align}
		が成り立つので
		\begin{align}
			A \in \mathscr{D}
		\end{align}
		が成り立つ.ゆえに
		\begin{align}
			\mathscr{U}_{\mathbf{T}} \subset \mathscr{D}
		\end{align}
		が成り立つ.ゆえにDynkin族定理より
		\begin{align}
			\mathscr{P}_{\mathbf{T}} = \mathscr{D}
		\end{align}
		が従う.
		\QED
	\end{sketch}
	
	$\mathscr{P}_{\mathbf{T}}$上の写像$\nu_M$を
	\begin{align}
		\mathscr{P}_{\mathbf{T}} \ni A \longmapsto
		\int_\Omega \int_{\mathbf{T}} \defunc_A(t,\omega)\ s_{M,\omega}(dt)\ P(d\omega)
	\end{align}
	なる関係により定める.
	
	\begin{screen}
		\begin{thm}[二乗可積分マルチンゲールで構成する測度]
			$\nu_M$は$\mathscr{P}_{\mathbf{T}}$上の正値有限測度である.
		\end{thm}
	\end{screen}
	
	\begin{sketch}
		先ず
		\begin{align}
			\int_\Omega \int_{\mathbf{T}} \defunc_{\mathbf{T} \times \Omega}(t,\omega)\ s_{M,\omega}(dt)\ P(d\omega)
			= \int_\Omega \inprod<M>_T\ dP
		\end{align}
		が成り立つが,
		\begin{align}
			M^2 - \inprod<M>
		\end{align}
		はマルチンゲールであり$M_T^2$は可積分なので
		\begin{align}
			\int_\Omega \inprod<M>_T\ dP < \infty
		\end{align}
		が従う.ゆえに$\nu_M$は有限値しか取らない.また
		\begin{align}
			\left\{A_n\right\}_{n \in \Natural}
		\end{align}
		を互いに素な$\mathscr{P}_{\mathbf{T}}$の部分集合とすると,$\Omega$の任意の要素$\omega$に対して
		単調収束定理より
		\begin{align}
			\int_{\mathbf{T}} \defunc_{\bigcup_{n \in \Natural}A_n}(t,\omega)\ s_{M,\omega}(dt)
			= \sum_{n \in \Natural} \int_{\mathbf{T}} \defunc_{A_n}(t,\omega)\ s_{M,\omega}(dt)
		\end{align}
		が成り立ち,再び単調収束定理より
		\begin{align}
			\int_\Omega \sum_{n \in \Natural} \int_{\mathbf{T}} \defunc_{A_n}(t,\omega)\ s_{M,\omega}(dt)\ P(d\omega)
			= \sum_{n \in \Natural} \int_\Omega \int_{\mathbf{T}} \defunc_{A_n}(t,\omega)\ s_{M,\omega}(dt)\ P(d\omega)
		\end{align}
		が成り立つ.ゆえに$\nu_M$は
		\begin{align}
			\bigcup_{n \in \Natural} A_n \longmapsto \sum_{n \in \Natural} \nu_M(A_n)
		\end{align}
		を満たす.ゆえに$\nu_M$は$\mathscr{P}_{\mathbf{T}}$上の正値有限測度である.
		\QED
	\end{sketch}
	
	$\mathscr{L}^1(\mathbf{T} \times \Omega,\mathscr{P}_{\mathbf{T}},\nu_M)$を
	$\mathscr{L}^1(\nu_M)$と略記する.
	
	\begin{screen}
		\begin{thm}[可積分可予測過程の積分表現]
			$f$を$\mathscr{L}^1(\nu_M)$の要素とすると
			\begin{align}
				\int_{\mathbf{T} \times \Omega} f\ d\nu_M 
				= \int_\Omega \int_{\mathbf{T}} f(t,\omega)\ s_{M,\omega}(dt)\ P(d\omega).
			\end{align}
		\end{thm}
	\end{screen}
	
	\begin{sketch}
		
	\end{sketch}
	
	\begin{comment}
	いま,$\mathscr{P}_{\mathbf{T}}$の要素に対する定義関数の全体の線型包を$\mathscr{S}$と定める:
	\begin{align}
		\mathscr{S} \defeq \operatorname{Span}\Set{\defunc_A}{A \in \mathscr{P}_{\mathbf{T}}}.
	\end{align}
	このとき$\mathscr{S}$は$\mathscr{L}^2(\nu_M)$において,セミノルム$\Norm{\cdot}{\mathscr{L}^2(\nu_M)}$に関して稠密となる.
	
	\begin{screen}
		\begin{thm}
			
		\end{thm}
	\end{screen}
	\end{comment}
	
	\begin{screen}
		\begin{thm}[確立積分に対するヘルダーの不等式]
			$M$と$N$を$\mathscr{M}_{c}^{2}$の要素とし,$X$を$\mathscr{L}^{2}(\nu_{M})$の要素とし,
			$Y$を$\mathscr{L}^{2}(\nu_{N})$の要素とするとき,$P$-零集合$F$が取れて,$\Omega \backslash F$
			の任意の要素$\omega$に対して
			\begin{align}
				\int_{[0,1]} \left|X_{t}(\omega) \cdot Y_{t}(\omega)\right|\ d|\inprod<M,N>|_{t}(\omega)
				\leq \left\{\int_{[0,1]} \left|X_{t}(\omega)\right|^{2}\ d\inprod<M>_{t}(\omega)
				\cdot \int_{[0,1]} \left|Y_{t}(\omega)\right|^{2}\ d\inprod<N>_{t}(\omega)
				\right\}^{\frac{1}{2}}.
			\end{align}
		\end{thm}
	\end{screen}
	
	\begin{sketch}
		いま$\alpha$と$\beta$を実数とすれば,定理??より$P$-零集合$F_{\alpha,\beta}$が取れて,
		$\Omega \backslash F_{\alpha,\beta}$の任意の要素$\omega$及び$[0,1]$の任意の要素$t$に対して
		\begin{align}
			\inprod<\alpha \cdot M + \beta \cdot N>_{t}(\omega)
			= \alpha^{2} \cdot \inprod<M>_{t}(\omega) 
			+ 2 \cdot \alpha \cdot \beta \cdot \inprod<M,N>_{t,\omega}
			+ \beta^{2} \cdot \inprod<N>_{t}(\omega)
		\end{align}
		が成立する.ここで
		\begin{align}
			F \defeq \bigcup_{\alpha, \beta \in \Q} F_{\alpha,\beta}
		\end{align}
		とおき,$\omega$を$\Omega \backslash F$の要素とする.また$\alpha$と$\beta$を
		任意の有理数とする.このとき,
		\begin{align}
			s < t
		\end{align}
		を満たす$[0,1]$の任意の要素$s$と$t$に対して
		\begin{align}
			0 &\leq  \inprod<\alpha \cdot M + \beta \cdot N>_{t}(\omega) 
			- \inprod<\alpha \cdot M + \beta \cdot N>_{s}(\omega) \\
			&= \alpha^{2} \cdot \left[\inprod<M>_{t}(\omega) - \inprod<M>_{s}(\omega)\right]
			+ 2 \cdot \alpha \cdot \beta \cdot 
			\left[\inprod<M,N>_{t}(\omega) - \inprod<M,N>_{s}(\omega)\right]
			+ \beta^{2} \cdot \left[\inprod<N>_{t}(\omega) - \inprod<N>_{s}(\omega)\right]
			\label{fom:Holder_ineqaulity_for_stochastic_integrals_1}
		\end{align}
		が成り立つ.ところで,$\varphi$を
		\begin{align}
			(t,\omega) \longmapsto \frac{1}{2} \cdot 
			\left[\inprod<M>_{t}(\omega) + \inprod<N>_{t}(\omega)\right]
		\end{align}
		なる関係で定める写像とすれば,$\inprod<M>_{\bullet}(\omega)$も
		$\inprod<N>_{\bullet}(\omega)$も$\inprod<M,N>_{\bullet}(\omega)$も
		$\varphi_{\bullet}(\omega)$に対して絶対連続なので,
		\begin{align}
			s < t
		\end{align}
		を満たす$[0,1]$の任意の要素$s$と$t$に対して
		\begin{align}
			\inprod<M>_{t}(\omega) - \inprod<M>_{s}(\omega)
			= \int_{]s,t]} f_{\omega}(u)\ d\varphi_{u}(\omega)
		\end{align}
		を満たす$f_{\omega}$,及び
		\begin{align}
			\inprod<N>_{t}(\omega) - \inprod<N>_{s}(\omega)
			= \int_{]s,t]} g_{\omega}(u)\ d\varphi_{u}(\omega)
		\end{align}
		を満たす$g_{\omega}$,及び
		\begin{align}
			\inprod<M,N>_{t}(\omega) - \inprod<M,N>_{s}(\omega)
			= \int_{]s,t]} h_{\omega}(u)\ d\varphi_{u}(\omega)
		\end{align}
		を満たす$h_{\omega}$が取れる.従って,(\refeq{fom:Holder_ineqaulity_for_stochastic_integrals_1})は
		\begin{align}
			0 \leq \int_{]s,t]} \alpha^{2} \cdot f_{\omega}(u) + 2 \cdot \alpha \cdot \beta
			\cdot h_{\omega}(u) + \beta^{2} \cdot g_{\omega}(u)\ d\varphi_{u}(\omega)
		\end{align}
		と書き換えられる.$s$と$t$は任意であるから,
		\begin{align}
			\int_{T_{\omega}(\alpha,\beta)} d\varphi_{u}(\omega) = 0
		\end{align}
		を満たす$\borel{[0,1]}$の要素$T_{\omega}(\alpha,\beta)$が取れて,
		\begin{align}
			t \notin T_{\omega}(\alpha,\beta)
		\end{align}
		を満たす$[0,1]$の任意の要素$t$に対して
		\begin{align}
			0 \leq \alpha^{2} \cdot f_{\omega}(t) + 2 \cdot \alpha \cdot \beta
			\cdot h_{\omega}(t) + \beta^{2} \cdot g_{\omega}(t)
		\end{align}
		が成立する.ゆえに,
		\begin{align}
			T_{\omega} \defeq \bigcup_{\alpha,\beta \in \Q} T_{\omega}(\alpha,\beta)
		\end{align}
		とおけば,$t$を
		\begin{align}
			t \notin T_{\omega}
		\end{align}
		を満たす$[0,1]$の任意の要素とすれば,任意の有理数$\alpha$と$\beta$に対して
		\begin{align}
			0 \leq \alpha^{2} \cdot f_{\omega}(t) + 2 \cdot \alpha \cdot \beta
			\cdot h_{\omega}(t) + \beta^{2} \cdot g_{\omega}(t)
		\end{align}
		が成立する.すなわち任意の実数$\alpha$と$\beta$に対して
		\begin{align}
			0 \leq \alpha^{2} \cdot f_{\omega}(t) + 2 \cdot \alpha \cdot \beta
			\cdot h_{\omega}(t) + \beta^{2} \cdot g_{\omega}(t)
		\end{align}
		が成立する.
	\end{sketch}
	
	\begin{screen}
		\begin{thm}
			$M$と$N$を$\mathscr{M}_{c}^{2}$の要素とし,$f$を$\mathscr{L}^{2}(\nu_{M})$の要素とするとき,
			\begin{align}
				E\left[I_{M}(f)_{1} \cdot N_{1}\right] = E\int_{[0,1]} f_{s}\ d\inprod<M,N>_{s}.
			\end{align}
		\end{thm}
	\end{screen}
	
	\begin{sketch}
		$u$と$v$を
		\begin{align}
			u < v
		\end{align}
		なる$[0,1]$の要素とし,$g$を$\mathscr{F}_{u}/\borel{\R}$-可測関数とし,
		\begin{align}
			f \defeq g \cdot \defunc_{]u,v]}
		\end{align}
		とおくとき,
		\begin{align}
			E\left[I_{M}(f)_{1} \cdot N_{1}\right] 
			&= E\left[(g \cdot M_{v} - g \cdot M_{u}) \cdot N_{1}\right] \\
			&= E\left[g \cdot M_{v} \cdot N_{1}\right] - E\left[g \cdot M_{u} \cdot N_{1}\right] \\
			&= E\left[g \cdot M_{v} \cdot N_{v}\right] - E\left[g \cdot M_{u} \cdot N_{u}\right]
		\end{align}
		が成り立つ.ここで
		\begin{align}
			E\left[g \cdot M_{v} \cdot N_{v}\right] 
			= E\left[g \cdot \inprod<M,N>_{v}\right]
		\end{align}
		及び
		\begin{align}
			E\left[g \cdot M_{u} \cdot N_{u}\right] 
			= E\left[g \cdot \inprod<M,N>_{u}\right]
		\end{align}
		が成り立つので($g$が単関数の場合はeasy)
		\begin{align}
			E\left[I_{M}(f)_{1} \cdot N_{1}\right] 
			= E\left[g \cdot \left(\inprod<M,N>_{v} - \inprod<M,N>_{u}\right)\right]
			= E \int_{[0,1]} f_{s}\ d\inprod<M,N>_{s}
		\end{align}
		が従う.$f$が$\mathscr{L}^{2}(\nu_{M})$の一般の要素であるとき
		\begin{align}
			E\int_{[0,1]} \left(f_{s} - f^{n}_{s}\right)^{2}\ d\inprod<M>_{s}
			\longrightarrow 0
		\end{align}
		を満たす単過程列$\{f^{n}\}_{n \in \Natural}$が取れる.このとき
		\begin{align}
			E\left|I_{M}(f)_{1} \cdot N_{1} - I_{M}(f^{n})_{1} \cdot N_{1}\right|
			&\leq \left\{ E\left|I_{M}(f - f^{n})_{1}\right|^{2} \cdot E({N_{1}}^{2}) \right\}^{\frac{1}{2}} \\
			&= \left\{ E\int_{[0,1]} \left(f_{s} - f^{n}_{s}\right)^{2}\ d\inprod<M>_{s} \cdot E({N_{1}}^{2}) \right\}^{\frac{1}{2}}
		\end{align}
		及び
		\begin{align}
			E\left|\int_{[0,1]}f_{s}\ d\inprod<M,N>_{s} - \int_{[0,1]} f^{n}_{s}\ d\inprod<M,N>_{s}\right|
			&\leq E\int_{[0,1]} \left| f_{s} - f^{n}_{s} \right|\ d|\inprod<M,N>|_{s} \\
			&\leq E\left[\left\{\int_{[0,1]} (f_{s} - f^{n}_{s})^{2}\ d\inprod<M>_{s}\right\}^{\frac{1}{2}}
			\cdot {\inprod<N>_{1}}^{\frac{1}{2}}\right] \\
			&\leq \left\{E\int_{[0,1]} (f_{s} - f^{n}_{s})^{2}\ d\inprod<M>_{s}
			\cdot E\inprod<N>_{1}\right\}^{\frac{1}{2}}
		\end{align}
		が成り立つので
		\begin{align}
			E\left[I_{M}(f)_{1} \cdot N_{1}\right]
			= E\int_{[0,1]}f_{s}\ d\inprod<M,N>_{s}
		\end{align}
		を得る.
		\QED
	\end{sketch}
	
	\begin{screen}
		\begin{thm}
			$M$と$N$を$\mathscr{M}_{c}^{2}$の要素とし,$f$を$\mathscr{L}^{2}(\nu_{M})$の要素とするとき,
			$P$-零集合$F$が取れて,$[0,1]$の任意の要素$t$及び$\Omega \backslash F$の任意の要素$\omega$
			に対して
			\begin{align}
				\inprod<I_{M}(f),N>_{t}(\omega)
				= \int_{[0,t]} f_{s}(\omega)\ d\inprod<M,N>_{s}(\omega).
			\end{align}
		\end{thm}
	\end{screen}
	
	\begin{sketch}
		$Z$を
		\begin{align}
			(t,\omega) \longmapsto I_{M}(f)_{t}(\omega) \cdot N_{t}(\omega)
			- \int_{[0,t]} f_{s}(\omega)\ d\inprod<M,N>_{s}(\omega)
		\end{align}
		なる関係で定める写像として,$Z$が$\{\mathscr{F}_{t}\}_{t \in [0,1]}$マルチンゲールであることを示す.
		$\tau$を任意の停止時刻とすれば,定理??より
		\begin{align}
			E\left(I_{M}(f)_{1} \cdot N^{\tau}_{1}\right)
			= E\int_{[0,1]} f_{s}\ d\inprod<M,N^{\tau}>_{s}
		\end{align}
		が成り立つが,ここで
		\begin{align}
			E\left(I_{M}(f)_{1} \cdot N^{\tau}_{1}\right)
			= E\left(I_{M}(f)_{\tau} \cdot N_{\tau}\right)
		\end{align}
		及び
		\begin{align}
			E\int_{[0,1]} f_{s}\ d\inprod<M,N^{\tau}>_{s}
			&= E\int_{[0,1]} f_{s}\ d\inprod<M,N>^{\tau}_{s} \\
			&= E\int_{[0,\tau]} f_{s}\ d\inprod<M,N>_{s}
		\end{align}
		が成立するから
		\begin{align}
			E(Z_{\tau}) = 0
		\end{align}
		が従う.ゆえに$Z$は$\{\mathscr{F}_{t}\}_{t \in [0,1]}$マルチンゲールである.
	\end{sketch}
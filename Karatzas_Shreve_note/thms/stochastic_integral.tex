\section{確率積分}
	$(\Omega,\mathscr{F},P)$を確率空間とし,$\mathbf{T}$を$[0,\infty[$か$[0,T]$とし,
	$\{\mathscr{F}_t\}_{t \in \mathbf{T}}$を$\mathscr{F}$に付随するフィルトレーションとし,
	\begin{align}
		\Set{N \in \mathscr{F}}{P(N) = 0} \subset \mathscr{F}_0
	\end{align}
	が満たされているとする.また$M$を$\mathscr{M}_2$の要素か,或いは$\mathscr{M}_2^c$の要素とする.ただし
	$M \in \mathscr{M}_2$の場合は$\{\mathscr{F}_t\}_{t \in \mathbf{T}}$は右連続であるとする.
	また$\inprod<M>$の$\omega$に対する標本路$\inprod<M>_\bullet(\omega)$で構成する
	$\borel{\mathbf{T}}$上のStieltjes測度を$s_{M,\omega}$と書く.
	
	\begin{screen}
		\begin{thm}
			$\omega$を$\Omega$の要素とするとき
			\begin{align}
				s_{M,\omega}(\mathbf{T}) < \infty.
			\end{align}
		\end{thm}
	\end{screen}
	
	\begin{prf}
		$\mathbf{T} = [0,\infty[$の場合,
		\begin{align}
			s_{M,\omega}(\mathbf{T}) = \lim_{n \to \infty} \inprod<M>_n(\omega) < \infty
		\end{align}
		となり,$\mathbf{T} = [0,T]$の場合
		\begin{align}
			s_{M,\omega}(\mathbf{T}) = \inprod<M>_T(\omega) < \infty
		\end{align}
		が成り立つ.
		\QED
	\end{prf}
	
	測度$s_{M,\omega}$は$\omega$に依存しているため,
	\begin{align}
		\Omega \ni \omega \longmapsto \int_{\mathbf{T}} \defunc_A(t,\omega)\ s_{M,\omega}(dt)
		\quad (A \in \mathscr{P}_{\mathbf{T}})
	\end{align}
	の可測性はFubiniの定理の適用では得られない.しかし以下が示される.
	
	\begin{screen}
		\begin{thm}
			$A$を$\mathscr{P}_{\mathbf{T}}$の要素とすれば
			\begin{align}
				\Omega \ni \omega \longmapsto \int_{\mathbf{T}} \defunc_A(t,\omega)\ s_{M,\omega}(dt)
			\end{align}
			は$\mathscr{F}/\borel{\R}$-可測である.
		\end{thm}
	\end{screen}
	
	\begin{prf}
	\end{prf}
	
	$\mathscr{P}_{\mathbf{T}}$上の写像$\nu_M$を
	\begin{align}
		\nu_M \defeq \Set{(A,y)}{A \in \mathscr{P}_{\mathbf{T}} \wedge y = 
		\int_\Omega \int_{\mathbf{T}} \defunc_A(t,\omega)\ s_{M,\omega}(dt)\ P(d\omega)}
	\end{align}
	で定める.つまり$\nu_M$は
	\begin{align}
		A \in \mathscr{P}_{\mathbf{T}} \Longrightarrow
		\nu_M(A) = \int_\Omega \int_{\mathbf{T}} \defunc_A(t,\omega)\ s_{M,\omega}(dt)\ P(d\omega)
	\end{align}
	を満たす.
	
	\begin{screen}
		\begin{thm}[二乗可積分マルチンゲールで構成する測度]
			$\nu_M$は$\mathscr{P}_{\mathbf{T}}$上の正値有限測度である.
		\end{thm}
	\end{screen}
	
	\begin{screen}
		\begin{thm}[可積分可予測過程の積分表現]
			\begin{align}
				f \in \mathscr{L}^1(\nu_M) \Longrightarrow 
				\int_{\mathbf{T} \times \Omega} f\ d\nu_M = \int_\Omega \int_{\mathbf{T}} f(t,\omega)\ s_{M,\omega}(dt)\ dP.
			\end{align}
		\end{thm}
	\end{screen}
	
	いま,$\mathscr{P}_{\mathbf{T}}$の要素に対する定義関数の全体の線型包を$\mathscr{S}$と定める:
	\begin{align}
		\mathscr{S} \defeq \operatorname{Span}\Set{\defunc_A}{A \in \mathscr{P}_{\mathbf{T}}}.
	\end{align}
	このとき$\mathscr{S}$は$\mathscr{L}^2(\nu_M)$において,セミノルム$\Norm{\cdot}{\mathscr{L}^2(\nu_M)}$に関して稠密となる.
	
	\begin{screen}
		\begin{thm}
			
		\end{thm}
	\end{screen}
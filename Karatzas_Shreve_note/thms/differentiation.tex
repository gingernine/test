\subsection{微分}
	複素微分,実微分,微分の線型性,Rollの定理,平均値の定理,中間値の定理,
	位相,$\Re$と$\Im$の連続性,コンパクト性
	
	\begin{screen}
		\begin{thm}[Rolleの定理]\label{Rolle_theorem}
			$a$と$b$を$a<b$なる実数とし,$f$を$[a,b]$上で定義された実連続関数とし,
			$f$は$]a,b[$の各要素で微分可能であるとする.また$f'$を$]a,b[$上の$f$の導関数とする.
			このとき
			\begin{align}
				f(a) = f(b)
			\end{align}
			ならば
			\begin{align}
				a < c < b \wedge f'(c) = 0
			\end{align}
			を満たす実数$c$が取れる.
		\end{thm}
	\end{screen}
	
	\begin{screen}
		\begin{thm}[平均値の定理]
		\label{mean_value_theorem_for_real_valued_differentiable_functions}
			$a$と$b$を$a<b$なる実数とし,$f$を$[a,b]$上で定義された実連続関数とし,
			$f$は$]a,b[$の各要素で微分可能であるとする.また$f'$を$]a,b[$上の$f$の導関数とする.
			このとき
			\begin{align}
				a < c < b
			\end{align}
			かつ
			\begin{align}
				\frac{f(b) - f(a)}{b - a} = f'(c)
			\end{align}
			を満たす実数$c$が取れる.
		\end{thm}
	\end{screen}
	
	\begin{sketch}
		$[a,b]$上の写像$g$を
		\begin{align}
			x \longmapsto f(x) - f(a) - \frac{f(b) - f(a)}{b - a} \cdot (x - a)
		\end{align}
		なる関係により定めれば,Rolleの定理より
		\begin{align}
			a < c < b
		\end{align}
		かつ
		\begin{align}
			g'(c) = 0
		\end{align}
		を満たす実数$c$が取れる.
		\begin{align}
			g'(c) = f'(c) - \frac{f(b) - f(a)}{b - a}
		\end{align}
		であるから定理の主張が得られる.
		\QED
	\end{sketch}
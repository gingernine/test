\section{微分}
	複素微分,実微分,微分の線型性,Rollの定理,平均値の定理,中間値の定理,
	位相,$\Re$と$\Im$の連続性,コンパクト性
	
	いま$\Omega$を$\C$或いは$\R$の開集合とし,$f$を$\Omega$で定義された$\C$値関数とする.
	$\alpha$を$\Omega$の要素とするとき,$f$が$\alpha$で{\bf 微分可能である}\index{びぶんかのう@微分可能}{\bf (differentiable)}ということを
	\begin{align}
		\exists a \in \C\, \forall \epsilon \in \R_+\, \exists \delta \in \R_+\,
		\forall z \in \Omega\, 
		\left(\, 0 < |z - \alpha| < \delta \Longrightarrow 
		\left| \frac{f(z) - f(\alpha)}{z-\alpha} - a\right| < \epsilon\, \right)
	\end{align}
	で定める.
	
	\begin{screen}
		\begin{thm}[微分可能なら連続]
			$\Omega$を$\C$或いは$\R$の開集合とし,$f$を$\Omega$で定義された$\C$値関数とし,$\alpha$を$\Omega$の要素とする.
			このとき$f$が$\alpha$で微分可能であるならば$f$は$\alpha$で連続である.
		\end{thm}
	\end{screen}
	
	\begin{sketch}
		$f$が$\alpha$で微分可能であるとする.このとき,
		\begin{align}
			\forall \epsilon \in \R_+\, \exists \delta \in \R_+\,
			\forall z \in \Omega\, 
			\left(\, 0 < |z - \alpha| < \delta \Longrightarrow 
			\left| \frac{f(z) - f(\alpha)}{z-\alpha} - a\right| < \epsilon\, \right)
		\end{align}
		を満たす複素数$a$が取れる.いま$\epsilon$を任意に与えられた正の実数とすれば
		\begin{align}
			\forall z \in \Omega\, 
			\left(\, 0 < |z - \alpha| < \delta \Longrightarrow 
			\left| \frac{f(z) - f(\alpha)}{z-\alpha} - a\right| < \epsilon\, \right)
		\end{align}
		を満たす正の実数$\delta$が取れるが,このとき
		\begin{align}
			|z - \alpha| < \min\left\{\frac{\epsilon}{|a| + \epsilon}, \delta\right\}
		\end{align}
		を満たす$\Omega$の任意の要素$z$に対して
		\begin{align}
			|f(z) - f(\alpha)| < |a \cdot (z-\alpha)| + \epsilon \cdot |z-\alpha|
			\leq (|a| + \epsilon) \cdot |z-\alpha|
			\leq \epsilon
		\end{align}
		が成立する.ゆえに$f$は$\alpha$において連続である.
		\QED
	\end{sketch}
	
	\begin{screen}
		\begin{dfn}[正則関数]
			$\Omega$を$\C$の開集合とし,$f$を$\Omega$で定義された$\C$値関数とする.
			$f$が$\Omega$の各要素において微分可能であるとき,$f$を$\Omega$上の
			{\bf 正則関数}\index{せいそくかんすう@正則関数}{\bf (holomorphic function)}と呼ぶ.また
			$\Omega$上の正則関数の全体を
			\begin{align}
				\Holomorphic{\Omega}
			\end{align}
			と書く.
		\end{dfn}
	\end{screen}
	
	$f$が$\alpha$で微分可能であるとき,
	\begin{align}
		\exists a \in \C\, \forall \epsilon \in \R_+\, \exists \delta \in \R_+\,
		\forall z \in \Omega\, 
		\left(\, 0 < |z - \alpha| < \delta \Longrightarrow 
		\left| \frac{f(z) - f(\alpha)}{z-\alpha} - a\right| < \epsilon\, \right)
	\end{align}
	を満たす複素数$a$のことを$f$の$\alpha$における{\bf 微分係数}\index{びぶんけいすう@微分係数}{\bf (derivative)}と呼ぶ.
	$(\C,\mathscr{O}_{\C})$はHausdorff空間であるから,微分係数は存在すればただ一つである.
	
	\begin{screen}
		\begin{dfn}[導関数]
			$\Omega$を$\C$或いは$\R$の開集合とし,$f$を$\Omega$で定義された$\C$値関数とする.
			$f$が$\Omega$の各要素において微分可能であるとき,
			$\Omega$の各要素に対して,そこにおける$f$の微分係数を対応させる写像を
			$f$の{\bf 導関数}\index{どうかんすう@導関数}{\bf (derivative function)}と呼び,多くの場合で
			\begin{align}
				f'
			\end{align}
			や
			\begin{align}
				f^{(1)}
			\end{align}
			と表記する.
		\end{dfn}
	\end{screen}
	
	\begin{screen}
		\begin{thm}[連鎖律]
			$\Omega$と$\Phi$を$\C$或いは$\R$の開集合とし,
			$f$を$\Omega$上の$\C$値関数とし,$g$を$\Phi$上の$\C$値関数とし,
			\begin{align}
				f \ast \Omega \subset \Phi
			\end{align}
			であるとする.$\alpha$を$\Omega$の要素とするとき,$f$が$\alpha$において微分可能であって,
			$g$が$f(\alpha)$において微分可能であるならば,$g \circ f$は$\alpha$において微分可能であって,その微分係数は
			\begin{align}
				g'(f(\alpha)) \cdot f'(\alpha)
			\end{align}
			である.ただし$f'(\alpha)$も$g'(f(\alpha))$も導関数の表記を使ってはいるが,これは微分係数を表すため便宜的に用いているだけである.
		\end{thm}
	\end{screen}
	
	\begin{sketch}
		$\epsilon$を任意に与えられた正数とする.ここで
		\begin{align}
			\eta^2 + \left(|f'(\alpha)| + |g'(f(\alpha))| \right) \eta = \epsilon
		\end{align}
		を満たす正数$\eta$を取る.$\eta$に対し,
		\begin{align}
			|z-\alpha| < \delta_1 \Longrightarrow
			\left| (f(z) - f(\alpha)) - f'(\alpha)(z-\alpha) \right| < \eta |z-\alpha|
		\end{align}
		を満たす正数$\delta_1$と,
		\begin{align}
			|w-f(\alpha)| < \delta_2 \Longrightarrow
			\left| (g(w) - g(f(\alpha))) - g'(f(\alpha))(w-f(\alpha)) \right| < \eta |w-f(\alpha)|
		\end{align}
		を満たす正数$\delta_2$を取る.また$f$は$\alpha$で連続であるから
		\begin{align}
			|z-\alpha| < \delta_3 \Longrightarrow \left| f(z) - f(\alpha) \right| < \delta_2
		\end{align}
		を満たす正数$\delta_3$が取れる.このとき
		\begin{align}
			\delta \defeq \operatorname{min}\{\delta_1, \delta_3\}
		\end{align}
		とおけば,
		\begin{align}
			|z-\alpha| < \delta
		\end{align}
		なる$\Omega$の任意の要素$z$に対して
		\begin{align}
			\left| \left(g(f(z)) - g(f(\alpha))\right) - g'(f(\alpha))(f(z)-f(\alpha)) \right| 
			&< \eta |f(z)-f(\alpha)| \\
			&< \eta \left( \eta|z-\alpha| + |f'(\alpha)||z-\alpha| \right),
		\end{align}
		及び
		\begin{align}
			&\left| \left(g(f(z)) - g(f(\alpha))\right) - g'(f(\alpha))(f(z)-f(\alpha)) \right| \\
			&= \left| \left(g(f(z)) - g(f(\alpha))\right) - g'(f(\alpha))f'(\alpha)(z-\alpha)
			- g'(f(\alpha)) \left( (f(z) - f(\alpha)) - f'(\alpha)(z-\alpha) \right) \right|
		\end{align}
		から
		\begin{align}
			&\left| \left(g(f(z)) - g(f(\alpha))\right) - g'(f(\alpha))f'(\alpha)(z-\alpha) \right| \\
			&\leq \left| \left(g(f(z)) - g(f(\alpha))\right) - g'(f(\alpha))(f(z)-f(\alpha)) \right|
			+ \left| g'(f(\alpha)) \right| \left| (f(z) - f(\alpha)) - f'(\alpha)(z-\alpha) \right|
		\end{align}
		が成り立つので,
		\begin{align}
			\left| \left(g(f(z)) - g(f(\alpha))\right) - g'(f(\alpha))f'(\alpha)(z-\alpha) \right|
			< \left[ \eta^2 + \left(|f'(\alpha)| + |g'(f(\alpha))| \right) \eta \right] |z-\alpha|
		\end{align}
		が従う.ゆえに
		\begin{align}
			0 < |z-\alpha| < \delta
			\Longrightarrow \left| \frac{g(f(z)) - g(f(\alpha))}{z-\alpha} - g'(f(\alpha)) \cdot f'(\alpha) \right| < \epsilon
		\end{align}
		が成り立つ.
		\QED
	\end{sketch}
	
	\begin{screen}
		\begin{thm}[Rolleの定理]\label{Rolle_theorem}
			$a$と$b$を$a<b$なる実数とし,$f$を$[a,b]$上で定義された実連続関数とし,
			$f$は$]a,b[$の各要素で微分可能であるとする.また$f'$を$]a,b[$上の$f$の導関数とする.
			このとき
			\begin{align}
				f(a) = f(b)
			\end{align}
			ならば
			\begin{align}
				a < c < b \wedge f'(c) = 0
			\end{align}
			を満たす実数$c$が取れる.
		\end{thm}
	\end{screen}
	
	\begin{screen}
		\begin{thm}[平均値の定理]
		\label{mean_value_theorem_for_real_valued_differentiable_functions}
			$a$と$b$を$a<b$なる実数とし,$f$を$[a,b]$上で定義された実連続関数とし,
			$f$は$]a,b[$の各要素で微分可能であるとする.また$f'$を$]a,b[$上の$f$の導関数とする.
			このとき
			\begin{align}
				a < c < b
			\end{align}
			かつ
			\begin{align}
				\frac{f(b) - f(a)}{b - a} = f'(c)
			\end{align}
			を満たす実数$c$が取れる.
		\end{thm}
	\end{screen}
	
	\begin{sketch}
		$[a,b]$上の写像$g$を
		\begin{align}
			x \longmapsto f(x) - f(a) - \frac{f(b) - f(a)}{b - a} \cdot (x - a)
		\end{align}
		なる関係により定めれば,Rolleの定理より
		\begin{align}
			a < c < b
		\end{align}
		かつ
		\begin{align}
			g'(c) = 0
		\end{align}
		を満たす実数$c$が取れる.
		\begin{align}
			g'(c) = f'(c) - \frac{f(b) - f(a)}{b - a}
		\end{align}
		であるから定理の主張が得られる.
		\QED
	\end{sketch}
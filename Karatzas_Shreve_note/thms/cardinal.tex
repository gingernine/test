	\begin{screen}
		\begin{dfn}[有限・可算・無限]
			
		\end{dfn}
	\end{screen}
	
	\begin{screen}
		\begin{thm}[任意の無限集合は可算集合を含む]
			\begin{align}
				\forall a\ \left(\ \exists \alpha \in \ON \backslash {\bf \omega}\ (\  \alpha \eqp a\ )
				\Longrightarrow \exists b\ (\ b \subset a \wedge {\bf \omega} \eqp b\ )\ \right).
			\end{align}
		\end{thm}
	\end{screen}
	
	\begin{screen}
		\begin{dfn}[対等]
			$a,b$を類とするとき,$a$と$b$が{\bf 対等である}\index{たいとう@対等}{\bf (equipotent)}ということを
			\begin{align}
				a \eqp b \overset{\mathrm{def}}{\Longleftrightarrow} 
				\exists f\, \left(\, f: a \bij b\, \right)
			\end{align}
			で定める.
		\end{dfn}
	\end{screen}
	
	\begin{screen}
		\begin{thm}[対等関係は同値関係]
			$\Univ$上の関係$R$を
			\begin{align}
				R \coloneqq \Set{x}{\exists s,t\, (\, x=(s,t) \wedge s \eqp t\, )}
			\end{align}
			で定めるとき,$R$は$\Univ$上の同値関係となる.
		\end{thm}
	\end{screen}
	
	\begin{screen}
		\begin{dfn}[濃度・基数]
			$a$を類とするとき,$a$と対等な順序数のうち最小のもの,つまり
			\begin{align}
				\card{a} \coloneqq \mu \alpha\, (\, a \eqp \alpha\, )
			\end{align}
			で定める$\card{a}$を$a$の{\bf 濃度}\index{のうど@濃度}{\bf (cardinal)}と呼び,
			\begin{align}
				\card{\alpha} = \alpha
			\end{align}
			を満たす順序数$\alpha$を{\bf 基数}\index{きすう@基数}{\bf (cardinal number)}と呼ぶ.
			また基数の全体を
			\begin{align}
				\CN \coloneqq \Set{x}{\exists \alpha \in \ON\, (\, \card{\alpha} = \alpha \wedge x = \alpha\, )}
			\end{align}
			とおく.
		\end{dfn}
	\end{screen}
	
	\monologue{
		院生「整列可能定理の結果より,全ての集合には濃度が定められるのですね.」
	}
	
	
	
	\begin{screen}
		\begin{thm}[順序数はその濃度より小さくない]\label{thm:ordinal_number_is_not_less_than_its_cardinal}
			\begin{align}
				\forall \alpha \in \ON\, \left(\, \card{\alpha} \leq \alpha\, \right).
			\end{align}
		\end{thm}
	\end{screen}
	
	\begin{sketch}
		$\alpha$上の恒等写像は$\alpha$から$\alpha$への全単射であるから
		\begin{align}
			\alpha \in \Set{\beta}{\beta \in \ON \wedge \beta \eqp \alpha}
		\end{align}
		が満たされる.従って
		\begin{align}
			\card{\alpha} \leq \alpha
		\end{align}
		が成立する.
		\QED
	\end{sketch}
	
	\begin{screen}
		\begin{thm}[濃度は基数]\label{thm:cardinal_is_cardinal_number}
			次が成り立つ:
			\begin{align}
				\forall a\, \left(\, \card{a} = \card{\card{a}}\, \right).
			\end{align}
		\end{thm}
	\end{screen}
	
	\begin{sketch}
		$a$を集合とする.まず定理\ref{thm:ordinal_number_is_not_less_than_its_cardinal}より
		\begin{align}
			\card{\card{a}} \leq \card{a}
		\end{align}
		が満たされる.他方で
		\begin{align}
			a \eqp \card{a} \wedge \card{a} \eqp \card{\card{a}}
		\end{align}
		が成り立っているので
		\begin{align}
			a \eqp \card{\card{a}}
		\end{align}
		が従い
		\begin{align}
			\card{a} \leq \card{\card{a}}
		\end{align}
		も満たされる.
		\QED
	\end{sketch}
	
	\begin{screen}
		\begin{thm}[$\CN$は濃度の全体である]
			次が成り立つ:
			\begin{align}
				\CN = \Set{x}{\exists a\, \left(\, x = \card{a}\, \right)}.
			\end{align}
		\end{thm}
	\end{screen}
	
	\begin{sketch}
		$\alpha$を$\CN$の任意の要素とすれば
		\begin{align}
			\alpha = \card{\alpha}
		\end{align}
		となるから,
		\begin{align}
			\exists a\, \left(\, \alpha = \card{a}\, \right)
		\end{align}
		が満たされ
		\begin{align}
			\alpha \in \Set{x}{\exists a\, \left(\, x = \card{a}\, \right)}
		\end{align}
		が従う.逆に$\alpha$を$\Set{x}{\exists a\, \left(\, x = \card{a}\, \right)}$の任意の要素とすれば,
		或る集合$a$が存在して
		\begin{align}
			\alpha = \card{a}
		\end{align}
		となる.このとき定理\ref{thm:cardinal_is_cardinal_number}より
		\begin{align}
			\card{a} = \card{\card{a}}
		\end{align}
		が成り立つので
		\begin{align}
			\exists \beta \in \ON\, (\, \card{\beta} = \beta \wedge \card{a} = \beta\, )
		\end{align}
		が満たされ
		\begin{align}
			\alpha = \card{a} \in \CN
		\end{align}
		が従う.
		\QED
	\end{sketch}
	
	\begin{screen}
		\begin{thm}
			次が成り立つ:
			\begin{align}
				\forall a\, \forall b\, \left(\, a \eqp b \Longleftrightarrow \card{a} = \card{b}\, \right).
			\end{align}
		\end{thm}
	\end{screen}
	
	\begin{prf}
		$a \eqp b$が成り立っていると仮定する.このとき
		\begin{align}
			\card{a} \eqp a \wedge a \eqp b
		\end{align}
		が成り立つので
		\begin{align}
			\card{a} \eqp b
		\end{align}
		が従い,
		\begin{align}
			\card{b} \leq \card{a}
		\end{align}
		となる.$a$と$b$を入れ替えれば
		\begin{align}
			\card{a} \leq \card{b}
		\end{align}
		も得られ,
		\begin{align}
			\card{a} = \card{b}
		\end{align}
		が成立する.以上より
		\begin{align}
			a \eqp b \Longrightarrow \card{a} = \card{b}
		\end{align}
		が示された.逆に$\card{a} = \card{b}$が成り立っていると仮定する.このとき
		\begin{align}
			f_1 &\coloneqq \varepsilon f\, \left(\, f:a \bij \card{a}\, \right), \\
			f_2 &\coloneqq \Set{x}{\exists s \in \card{a}\, (\,x = (s,s)\, )}, \\
			f_3 &\coloneqq \varepsilon f\, \left(\, f:\card{b} \bij b\, \right)
		\end{align}
		とおけば$f_1,f_2,f_3$はどれも全単射であるから,
		\begin{align}
			g \coloneqq (f_3 \circ f_2) \circ f_1
		\end{align}
		は$a$から$b$への全単射となる.よって$a \eqp b$が従い
		\begin{align}
			\card{a} = \card{b} \Longrightarrow a \eqp b
		\end{align}
		も示された.
		\QED
	\end{prf}
	
	\begin{screen}
		\begin{thm}[集合が大きい方が濃度も大きい]
			\begin{align}
				\forall a\, \forall b\, \left(\, a \subset b \Longrightarrow \card{a} \leq \card{b}\, \right).
			\end{align}
		\end{thm}
	\end{screen}
	
	\begin{sketch}
		$a,b$を集合として$a \subset b$が成り立っていると仮定する.
		\begin{align}
			\beta \coloneqq \card{b}
		\end{align}
		とおいて
		\begin{align}
			f:b \bij \beta
		\end{align}
		なる$f$を取り
		\begin{align}
			c \coloneqq f \ast a
		\end{align}
		とおけば,$c$の順序型である順序数$\alpha$と
		\begin{align}
			g:\alpha \bij c
		\end{align}
		および
		\begin{align}
			\forall \zeta,\eta \in \alpha\,
			\left(\, \zeta < \eta \Longrightarrow g(\zeta) < g(\eta)\, \right)
		\end{align}
		を満たす$g$が取れる.このとき
		\begin{align}
			\forall \zeta \in \alpha\, \left(\, \zeta \leq g(\zeta)\, \right)
		\end{align}
		が成り立つことが超限帰納法から示され,
		\begin{align}
			\alpha \subset \beta
		\end{align}
		が従う.
		\begin{align}
			\card{a} = \card{c} = \card{\alpha} \leq \alpha \leq \beta
		\end{align}
		から
		\begin{align}
			\card{a} \leq \card{b}
		\end{align}
		が得られる.
		\QED
	\end{sketch}
	
	\begin{screen}
		\begin{thm}[真類の濃度は$0$]
			$a$を類とするとき次が成り立つ:
			\begin{align}
				\rightharpoondown \set{a} \Longrightarrow \card{a} = 0.
			\end{align}
		\end{thm}
	\end{screen}
	
	\begin{screen}
		\begin{thm}[対等な集合同士は冪も対等]
			\begin{align}
				\forall a\, \forall b\, \left(\, a \eqp b \Longrightarrow \dirpro{a} \eqp \dirpro{b}\, \right).
			\end{align}
		\end{thm}
	\end{screen}
	
	\begin{sketch}
		$a,b$を集合とし,$a \eqp b$が成り立っているとする.このとき
		\begin{align}
			f:a \bij b
		\end{align}
		なる集合$f$を取り
		\begin{align}
			g \coloneqq \Set{x}{\exists s \in \dirpro{a}\, \left(\, x=(s,f \ast s)\, \right)}
		\end{align}
		で$g$を定めれば,
		\begin{align}
			g:\dirpro{a} \bij \dirpro{b}
		\end{align}
		が成立する.実際,$x,y$を$\dirpro{a}$の任意の要素とすれば
		\begin{align}
			x=y \Longrightarrow f \ast x = f \ast y
		\end{align}
		となるので$g$は写像である.また
		\begin{align}
			f \ast x = f \ast y
		\end{align}
		のとき
		\begin{align}
			x = f^{-1} \ast \left(f \ast x\right)
			= f^{-1} \ast \left(f \ast y\right)
			= y
		\end{align}
		となるので$g$は単射である.そして$z$を$\dirpro{b}$の任意の要素とすれば,
		\begin{align}
			w \coloneqq f^{-1} \ast z
		\end{align}
		とおけば
		\begin{align}
			f \ast w = z
		\end{align}
		となるので$g$は全射である.
		\QED
	\end{sketch}
	
	\begin{screen}
		\begin{thm}[Cantorの定理]
			集合の冪の濃度は集合の濃度より真に大きい:
			\begin{align}
				\forall a\, \left(\, \card{a} < \card{\dirpro{a}}\, \right).
			\end{align}
		\end{thm}
	\end{screen}
	
	\begin{screen}
		\begin{thm}[$\CN$は集合でない]
			\begin{align}
				\rightharpoondown \set{\CN}.
			\end{align}
		\end{thm}
	\end{screen}
	
	\begin{sketch}
		\begin{align}
			S \subset \ON \wedge \set{S} \Longrightarrow \card{\bigcup S} \notin S
		\end{align}
		が成り立つことを示す.
	\end{sketch}
	
	\begin{screen}
		\begin{thm}[自然数は基数]
			次が成立する.
			\begin{align}
				\omg \subset \CN.
			\end{align}
		\end{thm}
	\end{screen}
	
	\begin{screen}
		\begin{thm}[$\omg$は基数]
			次が成立する.
			\begin{align}
				\omg \in \CN.
			\end{align}
		\end{thm}
	\end{screen}
	
	\begin{screen}
		\begin{thm}
			有限基数を抜いた基数の全体を
			\begin{align}
				\InfCN \coloneqq \CN \backslash \omg
			\end{align}
			とおいて(`I' は Infinite の意),$\Univ$上の写像$G$を
			\begin{align}
				G \coloneqq \Set{x}{\exists s\, \left(\, x=(s,\mu \alpha\, (\, \alpha \in \InfCN \backslash \ran{s}\, )\, \right)}
			\end{align}
			で定めるとき,超限帰納法による写像の構成から
			\begin{align}
				\forall \beta \in \ON\, (\, F(\beta) = \mu \alpha\, (\, \alpha \in \InfCN \backslash F \ast \beta\, )\, )
			\end{align}
			を満たす$\ON$上の写像$F$が存在するが,この$F$は$\ON$から$\InfCN$への順序同型となる.つまり
			\begin{align}
				F:\ON \bij \InfCN \wedge \forall \gamma, \delta \in \ON\, (\, \gamma < \delta
				\Longrightarrow F(\gamma) < F(\delta)\, )
			\end{align}
			が成立する.
		\end{thm}
	\end{screen}
	
	\begin{sketch}
		いま$\gamma,\delta$を$\ON$の要素として
		\begin{align}
			\gamma < \delta
		\end{align}
		であると仮定する.このとき
		\begin{align}
			F \ast \gamma \subset F \ast \delta
		\end{align}
		かつ
		\begin{align}
			F(\delta) \in \InfCN \backslash F \ast \delta
		\end{align}
		が満たされるので
		\begin{align}
			F(\delta) \in \InfCN \backslash F \ast \gamma
		\end{align}
		が成立する.従って
		\begin{align}
			F(\gamma) \leq F(\delta)
		\end{align}
		が成立する.一方で
		\begin{align}
			F(\gamma) \in F \ast \delta \wedge
			F(\delta) \in \InfCN \backslash F \ast \delta
		\end{align}
		から
		\begin{align}
			F(\gamma) \neq F(\delta)
		\end{align}
		も満たされるので
		\begin{align}
			F(\gamma) < F(\delta)
		\end{align}
		が従う.以上より
		\begin{align}
			\forall \gamma, \delta \in \ON\, (\, \gamma < \delta \Longrightarrow F(\gamma) < F(\delta)\, )
		\end{align}
		が得られる.またこの結果より$F$が単射であることも従う.
	\end{sketch}
\subsection{濃度}
	\begin{screen}
		\begin{axm}[選択公理]
			任意の類$a$に対し,$a$上の写像$f$で,
			$a$の空でない要素$x$から$f(x)$を選択するもの
			(これを{\bf 選択関数}\index{せんたくかんすう@選択関数}{\bf (choice function)}と呼ぶ)
			が存在する:
			\begin{align}
				\forall a\ \exists f\ \left(\ 
				f:a \longrightarrow V \wedge \forall x \in a\ 
				(\ x \neq \emptyset \Longrightarrow f(x) \in x\ )\ \right). 
			\end{align}
		\end{axm}
	\end{screen}
	
	\begin{screen}
		\begin{thm}[整列可能定理]
			任意の集合は,或る順序数と全単射で結ばれる:
			\begin{align}
				\forall a \in V\ \exists \alpha \in \ON\ 
				\exists f\ \left( f:\alpha \bij a \right).
			\end{align}
		\end{thm}
	\end{screen}
	
	\begin{screen}
		\begin{dfn}[有限・可算・無限]
			
		\end{dfn}
	\end{screen}
	
	\begin{screen}
		\begin{thm}[${\bf \omega}$は最小の無限集合]
		\label{thm:the_principle_of_mathematical_induction}
			${\bf \omega}$は次の意味で最小の無限集合である:
			\begin{align}
				\forall a\ \left(\ \emptyset \in a \wedge \forall x\ 
				(\ x \in a \Longrightarrow x \cup \{x\} \in a\ ) 
				\Longrightarrow {\bf \omega} \subset a\ \right).
			\end{align}
		\end{thm}
	\end{screen}
	
	\begin{prf}
		超限帰納法で示す.いま$a$を
		\begin{align}
			\emptyset \in a \wedge \forall x\ 
			(\ x \in a \Longrightarrow x \cup \{x\} \in a\ )
		\end{align}
		を満たす類とし,また$\alpha$を任意に与えられた順序数とする.
		$\alpha = \emptyset$の場合は$\emptyset \in a$より
		\begin{align}
			\emptyset \in \omega \Longrightarrow \emptyset \in a
		\end{align}
		が成立する.$\alpha \neq \emptyset$の場合,$\alpha$の任意の要素$\beta$に対して
		\begin{align}
			\beta \in {\bf \omega} \Longrightarrow \beta \in a
		\end{align}
		が成り立つと仮定する.このとき,$\alpha \in {\bf \omega}$なら
		$\alpha$は極限数でないから$\alpha = \beta \cup \{\beta\}$を満たす順序数$\beta$が取れて,
		仮定より$\beta \in a$となり$\alpha \in a$が従う.以上で
		\begin{align}
			\forall \alpha \in \ON\ (\ \forall \beta \in \alpha\ (\ \beta \in {\bf \omega} \Longrightarrow \beta \in a\ ) \Longrightarrow (\ \alpha \in {\bf \omega} \Longrightarrow \alpha \in a\ )\ )
		\end{align}
		が得られた.超限帰納法により
		\begin{align}
			\forall \alpha \in \ON\ (\ \alpha \in {\bf \omega} \Longrightarrow \alpha \in a\ )
		\end{align}
		となるから$\omega \subset a$が出る.
		\QED
	\end{prf}
	
	\monologue{
		院生「定理\ref{thm:the_principle_of_mathematical_induction}で示された
			${\bf \omega}$の性質は{\bf 数学的帰納法の原理}
			\index{すうがくてききのうほうのげんり@数学的帰納法の原理}
			{\bf (the principle of mathematical induction)}と呼ばれます.
			高校数学だとドミノ倒しに喩えられる数学的帰納法ですが,
			なぜ数学的帰納法による証明が正しいのか簡単に説明いたしましょう.
			」
	}
	
	\begin{screen}
		\begin{thm}[任意の無限集合は可算集合を含む]
			\begin{align}
				\forall a\ \left(\ \exists \alpha \in \ON \backslash {\bf \omega}\ (\  \alpha \simeq a\ )
				\Longrightarrow \exists b\ (\ b \subset a \wedge {\bf \omega} \simeq b\ )\ \right).
			\end{align}
		\end{thm}
	\end{screen}

	\begin{screen}
		\begin{axm}[選択公理]
			$a$を類とするとき,$a$上の写像$f$で,
			$a$の空でない要素$x$から$f(x)$を選択するもの
			(これを{\bf 選択関数}\index{せんたくかんすう@選択関数}{\bf (choice function)}と呼ぶ)
			が存在する:
			\begin{align}
				\exists f\ \left(\ 
				f:a \longrightarrow V \wedge \forall x \in a\ 
				(\ x \neq \emptyset \Longrightarrow f(x) \in x\ )\ \right). 
			\end{align}
		\end{axm}
	\end{screen}
	
	\begin{screen}
		\begin{thm}[整列可能定理]
			任意の集合は,或る順序数と全単射で結ばれる:
			\begin{align}
				\forall a\ \exists \alpha \in \ON\ 
				\exists f\ \left( f:\alpha \bij a \right).
			\end{align}
		\end{thm}
	\end{screen}
	
	\begin{screen}
		\begin{dfn}[有限・可算・無限]
			
		\end{dfn}
	\end{screen}
	
	\begin{screen}
		\begin{thm}[任意の無限集合は可算集合を含む]
			\begin{align}
				\forall a\ \left(\ \exists \alpha \in \ON \backslash {\bf \omega}\ (\  \alpha \simeq a\ )
				\Longrightarrow \exists b\ (\ b \subset a \wedge {\bf \omega} \simeq b\ )\ \right).
			\end{align}
		\end{thm}
	\end{screen}
	\begin{screen}
		\begin{dfn}[有限・可算・無限]
			
		\end{dfn}
	\end{screen}
	
	\begin{screen}
		\begin{thm}[任意の無限集合は可算集合を含む]
			\begin{align}
				\forall a\ \left(\ \exists \alpha \in \ON \backslash {\bf \omega}\ (\  \alpha \simeq a\ )
				\Longrightarrow \exists b\ (\ b \subset a \wedge {\bf \omega} \simeq b\ )\ \right).
			\end{align}
		\end{thm}
	\end{screen}
	
	\begin{screen}
		\begin{dfn}[対等]
			$a,b$を集合とするとき,$a$から$b$への全単射が存在するとき
			$a$と$b$は対等であるといい,
			\begin{align}
				a \simeq b \overset{\mathrm{def}}{\Longleftrightarrow} 
				\exists f\, \left(\, f: a \bij b\, \right)
			\end{align}
			と定める.
		\end{dfn}
	\end{screen}
	
	\begin{screen}
		\begin{thm}[対等関係は同値関係]
			$\Univ$上の関係$R$を
			\begin{align}
				R \coloneqq \Set{x}{\exists s,t\, (\, x=(s,t) \wedge s \simeq t\, )}
			\end{align}
			で定めるとき,$R$は$\Univ$上の同値関係となる.
		\end{thm}
	\end{screen}
	
	\begin{screen}
		\begin{dfn}[濃度・基数]
			$a$を集合とするとき,$a$と対等な順序数のうち最小のもの,つまり
			\begin{align}
				\# a \coloneqq \mu \alpha\, (\, a \simeq \alpha\, )
			\end{align}
			で定める$\# a$を$a$の濃度と呼び,
			\begin{align}
				\# \alpha = \alpha
			\end{align}
			を満たす順序数$\alpha$を基数と呼ぶ.また
			\begin{align}
				\CN \coloneqq \Set{x}{\exists \alpha \in \ON\, (\, \#\alpha = \alpha\, )}
			\end{align}
			とおく.
		\end{dfn}
	\end{screen}
	
	\begin{screen}
		\begin{thm}[自然数は基数]
			次が成立する.
			\begin{align}
				\omg \subset \CN.
			\end{align}
		\end{thm}
	\end{screen}
	
	\begin{screen}
		\begin{thm}[$\omg$は基数]
			次が成立する.
			\begin{align}
				\omg \in \CN.
			\end{align}
		\end{thm}
	\end{screen}
	
	\begin{screen}
		\begin{thm}
			有限基数を抜いた基数の全体を
			\begin{align}
				\InfCN \coloneqq \CN \backslash \omg
			\end{align}
			とおいて,$\Univ$上の写像$G$を
			\begin{align}
				G \coloneqq \Set{x}{\exists s\, \left(\, x=(s,\mu \alpha\, (\, \alpha \in \InfCN \backslash \ran{s}\, )\, \right)}
			\end{align}
			で定めるとき,超限帰納法による写像の構成から
			\begin{align}
				\forall \beta \in \ON\, (\, F(\beta) = \mu \alpha\, (\, \alpha \in \InfCN \backslash F \ast \beta\, )\, )
			\end{align}
			を満たす$\ON$上の写像$F$が存在するが,この$F$は$\ON$から$\InfCN$への順序同型となる.つまり
			\begin{align}
				F:\ON \bij \InfCN \wedge \forall \gamma, \delta \in \ON\, (\, \gamma < \delta
				\Longrightarrow F(\gamma) < F(\delta)\, )
			\end{align}
			が成立する.
		\end{thm}
	\end{screen}
	
	\begin{prf}
		いま$\gamma,\delta$を$\ON$の要素として
		\begin{align}
			\gamma < \delta
		\end{align}
		であると仮定する.このとき
		\begin{align}
			F \ast \gamma \subset F \ast \delta
		\end{align}
		かつ
		\begin{align}
			F(\delta) \in \InfCN \backslash F \ast \delta
		\end{align}
		が満たされるので
		\begin{align}
			F(\delta) \in \InfCN \backslash F \ast \gamma
		\end{align}
		が成立する.従って
		\begin{align}
			F(\gamma) \leq F(\delta)
		\end{align}
		が成立する.一方で
		\begin{align}
			F(\gamma) \in F \ast \delta \wedge
			F(\delta) \in \InfCN \backslash F \ast \delta
		\end{align}
		から
		\begin{align}
			F(\gamma) \neq F(\delta)
		\end{align}
		も満たされるので
		\begin{align}
			F(\gamma) < F(\delta)
		\end{align}
		が従う.以上より
		\begin{align}
			\forall \gamma, \delta \in \ON\, (\, \gamma < \delta \Longrightarrow F(\gamma) < F(\delta)\, )
		\end{align}
		が得られる.またこの結果より$F$が単射であることも従う.
	\end{prf}
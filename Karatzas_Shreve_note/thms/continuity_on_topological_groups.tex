\subsection{連続性}
	\begin{screen}
		\begin{thm}[単位元で連続な準同型は一様連続である]
			$\left(\left(X,\sigma_X\right),\mathscr{O}_X\right)$と
			$\left(\left(Y,\sigma_Y\right),\mathscr{O}_Y\right)$を位相群とし,
			$0_X$を$\left(X,\sigma_X\right)$の単位元とし,
			$f$を$\left(X,\sigma_X\right)$から$\left(Y,\sigma_Y\right)$への準同型写像とする.
			また$\mathscr{V}_X$を定理\ref{thm:topological_groups_are_uniformazable}の要領で構成する$X$上の近縁系とし,
			$\mathscr{V}_Y$を定理\ref{thm:topological_groups_are_uniformazable}の要領で構成する$Y$上の近縁系とする.
			このとき,$f$は$0_X$で連続ならば$\mathscr{V}_X/\mathscr{V}_Y$-一様連続である.
		\end{thm}
	\end{screen}
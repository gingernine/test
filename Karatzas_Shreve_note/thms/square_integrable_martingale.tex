\section{二乗可積分マルチンゲール}
	本節では$(\Omega,\mathscr{F},P)$を確率空間とし,$\{\mathscr{F}_t\}_{t \in [0,1]}$を$\mathscr{F}$に付随するフィルトレーションとする.
	
	\begin{screen}
		\begin{dfn}[二乗可積分マルチンゲール]
			$(\Omega,\mathscr{F},P)$上の連続な$\{\mathscr{F}_t\}_{t \in [0,1]}$-マルチンゲール$X$で,$[0,1]$の任意の要素$t$で
			\begin{align}
				E\left(X_{t}^{2}\right) < \infty
			\end{align}
			を満たし,かつ$\Omega$の任意の要素$\omega$に対して
			\begin{align}
				X_0(\omega) = 0
			\end{align}
			を満たすものの全体を
			\begin{align}
				\mathscr{M}_{c}^{2}
			\end{align}
			とおく.
		\end{dfn}
	\end{screen}
	
	\begin{screen}
		\begin{thm}[$\mathscr{M}_{c}^{2}$の線型構造]
			$X$と$Y$を$\mathscr{M}_{c}^{2}$の要素とするとき,$(X,Y)$に対して
			\begin{align}
				[0,1] \times \Omega \ni (t,\omega) \longmapsto X(t,\omega) + Y(t,\omega)
			\end{align}
			なる写像を対応させる関係$+_{m}$を$\mathscr{M}_{c}^{2}$上の加法とし,
			また$\alpha$を実数とするときに$(\alpha,X)$に対して
			\begin{align}
				(t,\omega) \longmapsto \alpha \cdot X(t,\omega)
			\end{align}
			なる写像を対応させる関係$\cdot_{m}$をスカラ倍とすれば,
			\begin{align}
				\left(\left(\mathscr{M}_{c}^{2},+_{m} \right),(\R,+,\bullet),\cdot_{m}\right)
			\end{align}
			は線型空間をなす.
		\end{thm}
	\end{screen}
	
	\begin{screen}
		\begin{thm}[Doobの劣マルチンゲール不等式]
		\end{thm}
	\end{screen}
	
	\begin{screen}
		\begin{thm}[$\mathscr{M}_{c}^{2}$は完備な擬距離空間]
		\label{thm:pseudo_metric_on_square_integrable_martingales}
			$\{\mathscr{F}_t\}_{t \in [0,1]}$が完備である(定義\ref{thm:completeness_of_filtration})とするとき,
			\begin{align}
				\mathscr{M}_{c}^{2} \times \mathscr{M}_{c}^{2} \ni (X,Y) \longmapsto
				\left\{E|X_{1}-Y_{1}|^2\right\}^{\frac{1}{2}}
			\end{align}
			なる関係を$d$とすると,$\left(\mathscr{M}_{c}^{2},d\right)$は完備な擬距離空間をなす.
		\end{thm}
	\end{screen}
	
	\begin{sketch}\mbox{}
		\begin{description}
			\item[第一段] 擬距離空間は可算な基本近縁系が取れるので,完備性はCauchy列の収束を示すだけで良い.いま
				\begin{align}
					\Natural \ni n \longmapsto X^{(n)} \in \mathscr{M}^2_{\mathbf{T}}
				\end{align}
				なる関係を$\left(\mathscr{M}^2_{\mathbf{T}},d\right)$のCauchy列とする.すると
				\begin{align}
					\forall k \in \Natural\, \left[\, d\left( X^{(n_k)},X^{(n_{k+1})} \right) < \frac{1}{4^{k+1}}\, \right]
				\end{align}
				を満たす部分列
				\begin{align}
					\Natural \ni k \longmapsto X^{(n_k)}
				\end{align}
				が取れる.このときDoobの劣マルチンゲール不等式から,任意の自然数$k$で
				\begin{align}
					\int_\Omega \left\{ \sup{t \in [0,T]}{\left|X_t^{(n_k)} - X_t^{(n_{k+1})}\right|} \right\}^2\ dP
					\leq 4 \int_\Omega \left|X_T^{(n_k)} - X_T^{(n_{k+1})}\right|^2\ dP
					< \frac{1}{8^k}
					\label{fom:thm_pseudo_metric_on_square_integrable_martingales_2}
				\end{align}
				が成立する.従って,自然数$k$に対して
				\begin{align}
					E_k \defeq \left\{ \frac{1}{2^k} \leq \sup{t \in [0,T]}{\left|X_t^{(n_k)} - X_t^{(n_{k+1})}\right|} \right\}
				\end{align}
				とおけば
				\begin{align}
					P(E_k) < \frac{1}{2^k}
				\end{align}
				が成立するので,Borel-Cantelliの補題より
				\begin{align}
					E \defeq \bigcap_{n \in \Natural} \bigcup_{\substack{k \in \Natural \\ n < k}} E_k
				\end{align}
				で定める$E$は$P$-零集合である.$\omega$を$\Omega \backslash E$の要素とすれば
				\begin{align}
					\forall k \in \Natural\,
					\left[\, N < k \Longrightarrow \sup{t \in [0,T]}{\left|X_t^{(n_k)}(\omega) - X_t^{(n_{k+1})}(\omega)\right|} < \frac{1}{2^k}\, \right]
					\label{fom:thm_pseudo_metric_on_square_integrable_martingales_1}
				\end{align}
				を満たす自然数$N$が取れる.ゆえに,いま$t$を$\mathbf{T}$の要素とすれば
				\begin{align}
					\Natural \ni k \longmapsto X_t^{(n_k)}(\omega)
				\end{align}
				は$\R$のCauchy列であり,$\R$で収束する.ここで
				\begin{align}
					\mathbf{T} \times \Omega \ni (t,\omega) \longmapsto
					\begin{cases}
						\lim_{k \to \infty} X_t^{(n_k)}(\omega) & \mbox{if } \omega \in \Omega \backslash E \\
						0 & \mbox{if } \omega \in E
					\end{cases} 
				\end{align}
				で定める関係を$X$とする.
			
			\item[第二段]
				$X$のパスが右連続(または連続)であることを示す.$\omega$を$\Omega \backslash E$の要素とすれば
				(\refeq{fom:thm_pseudo_metric_on_square_integrable_martingales_1})より
				\begin{align}
					\forall k \in \Natural\,
					\left[\, N < k \Longrightarrow \sup{t \in [0,T]}{\left|X_t^{(n_k)}(\omega) - X_t(\omega)\right|} \leq \frac{1}{2^k}\, \right]
				\end{align}
				を満たす自然数$N$が取れるので,パスは一様収束している.ゆえに
				\begin{align}
					\left\{X^{(n)}\right\}_{n \in \Natural} \subset \mathscr{M}^2_{\mathbf{T}}
				\end{align}
				ならば$X$は右連続であり,
				\begin{align}
					\left\{X^{(n)}\right\}_{n \in \Natural} \subset \mathscr{M}^{2,c}_{\mathbf{T}}
				\end{align}
				ならば$X$は連続である.
			
			\item[第三段]
				$X$が$\{\mathscr{F}_t\}_{t \in \mathbf{T}}$-適合であることを示す.
				$t$を$\mathbf{T}$の任意の要素とすれば
				\begin{align}
					\forall \omega \in \Omega\, \left(\, 
					\lim_{k \to \infty} X_t^{(n_k)}(\omega) \cdot \defunc_{\Omega \backslash E}(\omega) = X_t(\omega)\, \right)
				\end{align}
				が成り立ち,またフィルトレーションの完備性の仮定から
				\begin{align}
					E \in \mathscr{F}_t
				\end{align}
				なので,各自然数$k$で$X_t^{(n_k)} \defunc_{\Omega \backslash E}$は$\mathscr{F}_t/\borel{\R}$-可測である.
				よって定理\ref{lem:measurability_metric_space}より$X_t$は$\mathscr{F}_t/\borel{\R}$-可測である.
				
			\item[第四段]
				$X$が二乗可積分な$\{\mathscr{F}_t\}_{t \in \mathbf{T}}$-マルチンゲールであることを示す.
				$t$を$\mathbf{T}$の任意の要素とすれば,Fatouの補題と
				(\refeq{fom:thm_pseudo_metric_on_square_integrable_martingales_2})より
				任意の自然数$k$で
				\begin{align}
					\int_\Omega \left|X_t-X_t^{(n_k)}\right|^2\ dP
					\leq \sup{n \in \Natural}{\inf{\substack{j \in \Natural \\ n < j}}{
					\int_\Omega \left|X_t^{(n_j)}-X_t^{(n_k)}\right|^2\ dP}}
					\leq \frac{1}{4^k}
					\label{fom:thm_pseudo_metric_on_square_integrable_martingales_3}
 				\end{align}
 				が成立する.ゆえにMinkowskiの不等式から
 				\begin{align}
 					\left\{\int_\Omega |X_t|^2\ dP\right\}^{\frac{1}{2}}
 					\leq \left\{\int_\Omega \left|X_t - X^{(n_k)}_t\right|^2\ dP\right\}^{\frac{1}{2}}
 					+ \left\{\int_\Omega \left|X^{(n_k)}_t\right|^2\ dP\right\}^{\frac{1}{2}}
 					< \infty
 				\end{align}
 				が成立する.またH\Ddot{o}lderの不等式から
 				\begin{align}
 					\int_\Omega \left|X_t-X_t^{(n_k)}\right|\ dP
 					\leq \left\{\int_\Omega \left|X_t - X^{(n_k)}_t\right|^2\ dP\right\}^{\frac{1}{2}}
 					\longrightarrow 0\quad (k \longrightarrow \infty)
 				\end{align}
 				が成り立つ.ゆえに,いま$s$と$t$を
 				\begin{align}
 					s < t
 				\end{align}
 				なる$\mathbf{T}$の要素とすれば,$\mathscr{F}_s$の任意の要素$A$で
 				\begin{align}
 					\int_A X_t\ dP = \lim_{k \to \infty} \int_A X^{(n_k)}_t\ dP
 					= \lim_{k \to \infty} \int_A X^{(n_k)}_s\ dP
 					= \int_A X_s\ dP
 				\end{align}
 				が成り立つ.
 			
 			\item[第五段]
 				以上より
 				\begin{align}
 					X \in \mathscr{M}^2_{\mathbf{T}}
 				\end{align}
 				である.最後に,(\refeq{fom:thm_pseudo_metric_on_square_integrable_martingales_3})より
 				\begin{align}
 					d\left(X,X^{(n_k)}\right) = \int_\Omega \left|X_T-X_T^{(n_k)}\right|^2\ dP
 					\longrightarrow 0\quad (k \longrightarrow \infty)
 				\end{align}
 				が成り立つので$X$は$d$に関して$k \longmapsto X^{(n_k)}$の極限である.部分列の収束から
 				\begin{align}
 					d\left(X,X^{(n)}\right) \longrightarrow 0\quad (n \longrightarrow \infty)
 				\end{align}
 				が従う.
 				\QED
		\end{description}
	\end{sketch}
	
	\begin{screen}
		\begin{thm}[右連続マルチンゲールから停止時刻の増大列が作れる]
		\label{thm:increasing_stopping_times_made_from_continuous_martingales}
			$\mathbf{T} = [0,\infty[$とし,$X$を$\mathscr{M}_{\mathbf{T}}$の要素とする.自然数$n$に対して
			\begin{align}
				\omega \longmapsto 
				\begin{cases}
					\inf{}{\Set{t \in \mathbf{T}}{n \leq |X_t(\omega)|}} & \mbox{if } \Set{t \in \mathbf{T}}{n \leq |X_t(\omega)|} \neq \emptyset \\
					\infty & \mbox{if } \Set{t \in \mathbf{T}}{n \leq |X_t(\omega)|} = \emptyset
				\end{cases}
			\end{align}
			なる写像を$\tau_n$と定めると,$\tau_n$は$\{\mathscr{F}_t\}_{t \in \mathbf{T}}$-停止時刻であって,
			$\Omega$のすべての要素$\omega$に対して
			\begin{align}
				\forall n \in \Natural\, \left(\, \tau_n(\omega) \leq \tau_{n+1}(\omega)\, \right)
			\end{align}
			を満たす.またパスが$RCLL$である$\omega$に対しては
			\begin{align}
				\sup{n \in \Natural}{\tau_n(\omega)} = \infty
			\end{align}
			が成り立つ.またパスが連続である$\omega$に対しては
			\begin{align}
				\sup{t \in \mathbf{T}}{\left|X^{\tau_n}_t(\omega)\right|} \leq n
			\end{align}
			が成り立つ.
		\end{thm}
	\end{screen}
	
	\begin{sketch}
		$\mathbf{T}$の任意の要素$t$に対して
		\begin{align}
			\left\{\tau_n \leq t\right\} = \Set{\omega \in \Omega}{n \leq |X_t(\omega)|}
		\end{align}
		が成り立つので$\tau_n$は$\{\mathscr{F}_t\}_{t \in \mathbf{T}}$-停止時刻である.
		また$RLCC$なパスは有界区間上で有界であるから,パスが$RCLL$である$\omega$に対しては
		\begin{align}
			\sup{n \in \Natural}{\tau_n(\omega)} = \infty
		\end{align}
		が成り立つ.
		\QED
	\end{sketch}
	
	右連続な劣マルチンゲールは殆ど全てのパスが$RCLL$なので,
	上の様に構成する停止時刻の列$\left\{\tau_n\right\}_{n \in \Natural}$は殆どすべての$\omega$に対し
	\begin{align}
		0 = \tau_0(\omega) \leq \tau_1(\omega) \leq \tau_2(\omega) \leq \longrightarrow \infty
	\end{align}
	を満たす.
	
\subsection{Convergence Results}
	\begin{itembox}[l]{Problem 3.16}
		Let $\Set{X_t,\mathscr{F}_t}{0 \leq t < \infty}$ be a right-continuous, nonnegative
		supermatingale; then $X_\infty(\omega) = \lim_{t \to \infty} X_t(\omega)$ exists for
		$P$-a.e. $\omega \in \Omega$, and $\Set{X_t,\mathscr{F}_t}{0 \leq t \leq \infty}$ is a supermartingale.
	\end{itembox}
	
	\begin{prf}
		$\Set{-X_t,\mathscr{F}_t}{0 \leq t < \infty}$は右連続な$(\mathscr{F}_t)$-劣マルチンゲールとなり
		\begin{align}
			\sup{t \geq 0}{E(-X_t)^+} = 0
		\end{align}
		が満たされるから,劣マルチンゲール収束定理により或る$P$-零集合$A$が存在して
		\begin{align}
			Z_\infty \coloneqq \lim_{t \to \infty} (-X_t)\defunc_{\Omega \backslash A}
		\end{align}
		により$\mathscr{F}_\infty/\borel{\R}$-可測な可積分関数$Z_\infty$が定まる.
		すなわち
		\begin{align}
			X_\infty \coloneqq \lim_{t \to \infty} X_t\defunc_{\Omega \backslash A}
		\end{align}
		により$\mathscr{F}_\infty/\borel{\R}$-可測関数が定まり,
		かつ$X_\infty = -Z_\infty$より$X_\infty$は可積分である.
		またFatouの補題により任意の$t \geq 0$及び$A \in \mathscr{F}_t$に対し
		\begin{align}
			\int_A X_\infty\ dP \leq \liminf_{\substack{n \to \infty \\ n > t}} \int_A X_n\ dP \leq \int_A X_t\ dP
		\end{align}
		が成立するから$\Set{X_t,\mathscr{F}_t}{0 \leq t \leq \infty}$は優マルチンゲールである.
		\QED
	\end{prf}
	
	\begin{itembox}[l]{Exercise 3.18}
		Suppose that the filtration $\{\mathscr{F}_t\}$ satisfies the usual conditions.
		Then every right-continuous, uniformly integrable supermartingale $\Set{X_t,\mathscr{F}_t}{0 \leq t < \infty}$
		admits the Riesz decomposition $X_t = M_t + Z_t,\ \mbox{a.s. $P$}$, as the sum
		of a right-continuous, uniformly integrable martingale $\Set{M_t,\mathscr{F}_t}{0 \leq t < \infty}$
		and a potential $\Set{Z_t,\mathscr{F}_t}{0 \leq t < \infty}$.
 	\end{itembox}
 	条件を満たす二つの分解$X_t = M_t + Z_t = M'_t + Z'_t\ \mbox{a.s. $P$}, (\forall t \geq 0)$が存在する場合,
 	次の意味で分解は一意である:
 	\begin{align}
 		P \left( M_t = M'_t,\ Z_t = Z'_t,\ \forall t \geq 0 \right) = 1.
 		\label{eq:chapter_1_Exercise_3_18_4}
 	\end{align}
 	
 	\begin{prf}\mbox{}
		\begin{description}
			\item[第一段] $M$を構成する.いま,$t \geq 0$を固定する.
				$n > t$を満たす$n \in \N$と任意の$A \in \mathscr{F}_t$に対し
 				\begin{align}
 					\int_A \cexp{X_{n+1}}{\mathscr{F}_t}\ dP
 					&= \int_A X_{n+1}\ dP
 					= \int_A \cexp{X_{n+1}}{\mathscr{F}_n}\ dP \\
		 			&\leq \int_A X_n\ dP
 					= \int_A \cexp{X_n}{\mathscr{F}_t}\ dP
 				\end{align}
 				が成り立つから
 				\begin{align}
 					E \coloneqq \bigcup_{n > t}\Set{\omega \in \Omega}{\cexp{X_n}{\mathscr{F}_t}(\omega) < \cexp{X_{n+1}}{\mathscr{F}_t}(\omega)}
 				\end{align}
 				として$P$-零集合が定まる.また,同様に優マルチンゲール性より
 				\begin{align}
 					F \coloneqq \bigcup_{n > t}\Set{\omega \in \Omega}{\cexp{X_n}{\mathscr{F}_t}(\omega) > X_t(\omega)}
 				\end{align}
 				も$P$-零集合である.このとき,単調減少性より
 				\begin{align}
 					X^*_t \coloneqq \lim_{n \to \infty} \cexp{X_n}{\mathscr{F}_t} \defunc_{\Omega \backslash (E \cup F)}
 				\end{align}
 				が$-\infty$まで込めて確定し,$X^*_t$は$\mathscr{F}_t/\borel{[-\infty,\infty]}$-可測であり
 				\begin{align}
 					X_t(\omega) \geq X^*_t(\omega), \quad (\forall \omega \in \Omega \backslash (E \cup F))
 				\end{align}
 				を満たす.単調収束定理と$\sup{n \geq 1}{E|X_n|} < \infty$ (一様可積分性)より
 				\begin{align}
 					E\left( X_t - X^*_t \right)
 					= \int_{\Omega \backslash (E \cup F)} \lim_{n \to \infty} \left( X_t - \cexp{X_n}{\mathscr{F}_t} \right)\ dP
 					= \lim_{n \to \infty} \int_{\Omega \backslash (E \cup F)} X_t - \cexp{X_n}{\mathscr{F}_t}\ dP
 					= E X_t - \lim_{n \to \infty} EX_n < \infty
 				\end{align}
 				が成立するから$X^*_t$は可積分性であり$P$-a.s.に$|X^*_t| <\infty$となる.ここで
 				\begin{align}
 					X^{**}_t \coloneqq X^*_t \defunc_{|X^*_t| < \infty}
 				\end{align}
 				により$\mathscr{F}_t/\borel{\R}$-可測な可積分関数を定めれば,
 				単調収束定理より
 				\begin{align}
 					E X^{**}_t
 					= \lim_{n \to \infty} \int_\Omega \cexp{X_n}{\mathscr{F}_t}\ dP
 					= \lim_{n \to \infty} E X_n
 					\label{eq:chapter_1_Exercise_3_18_1}
 				\end{align}
 				となる.任意の$t \geq 0$に対し$X^{**}_t$を定めれば,任意の$0 \leq s < t$及び$A \in \mathscr{F}_s$に対して
 				\begin{align}
 					\int_A X^{**}_t\ dP
 					= \lim_{n \to \infty} \int_A \cexp{X_n}{\mathscr{F}_t}\ dP 
 					= \lim_{n \to \infty} \int_A \cexp{X_n}{\mathscr{F}_s}\ dP
 					= \int_A X^{**}_s\ dP
 					\label{eq:chapter_1_Exercise_3_18_3}
 				\end{align}
 				が成り立つから$\Set{X^{**}_t,\mathscr{F}_t}{0 \leq t < \infty}$はマルチンゲールである.
 				マルチンゲール性より$[0,\infty) \ni t \longmapsto EX^{**}_t$は定数であるから
 				Theorem 3.13により右連続な修正$\Set{M_t,\mathscr{F}_t}{0 \leq t < \infty}$が存在する.
 		
 			\item[第二段]
 				まず$\lim_{t \to \infty} EX_t$が存在することを示す.
 				任意の単調増大列$(t_k)_{k=1}^\infty,\ t_k \uparrow \infty$に対し優マルチンゲール性より
	 			\begin{align}
	 				\lim_{k \to \infty} EX_{t_k} = \inf{k \geq 1}{EX_{t_k}}
	 			\end{align}
	 			が確定し,任意の$n \in \N$に対し$n < t_k$を満たす$k$が存在するから
	 			\begin{align}
	 				\inf{n \geq 1}{EX_n} \geq \inf{k \geq 1}{EX_{t_k}}
	 			\end{align}
	 			が従う.逆に任意の$t_k$に対し$t_k < n$を満たす$n$が存在するから
	 			\begin{align}
	 				\lim_{n \to \infty} EX_n = \inf{n \geq 1}{EX_n} 
	 				= \inf{k \geq 1}{EX_{t_k}} = \lim_{k \to \infty} EX_{t_k}
	 			\end{align}
	 			が成立し,$(t_k)_{k=1}^\infty$の任意性から$\lim_{t \to \infty} EX_t$が存在して
	 			\begin{align}
	 				\lim_{t \to \infty} EX_t = \lim_{n \to \infty} EX_n
	 				\label{eq:chapter_1_Exercise_3_18_2}
	 			\end{align}
	 			となる.右連続な優マルチンゲール$\Set{Z_t,\mathscr{F}_t}{0 \leq t < \infty}$を
	 			\begin{align}
 					Z_t \coloneqq X_t - M_t,
 					\quad (\forall t \geq 0)
 				\end{align}
 				により定めれば,
 				(\refeq{eq:chapter_1_Exercise_3_18_1})より任意の$t \geq 0$に対し
 				\begin{align}
 					E(X_t - M_t)
 					= E X_t - E M_t
 					= EX_t - \lim_{n \to \infty} E X_n
 				\end{align}
 				が成り立ち,(\refeq{eq:chapter_1_Exercise_3_18_2})より
 				\begin{align}
 					\lim_{t \to \infty} E(X_t - M_t)
 					= \lim_{t \to \infty} E X_t - \lim_{n \to \infty} E X_n
 					= 0
 				\end{align}
 				が満たされるから$\Set{Z_t,\mathscr{F}_t}{0 \leq t < \infty}$はポテンシャルである.
			
			\item[第三段]
				分解の一意性を示す.任意の$t \geq 0$及び$A \in \mathscr{F}_t$に対し,
				(\refeq{eq:chapter_1_Exercise_3_18_3})と$M'$のマルチンゲール性より
				\begin{align}
					\int_A M_t\ dP
					= \lim_{\substack{n \to \infty \\ n > t}} \int_A X_n\ dP
					= \lim_{\substack{n \to \infty \\ n > t}} \left\{ \int_A M'_n - Z'_n\ dP \right\}
					= \lim_{\substack{n \to \infty \\ n > t}} \left\{ \int_A M'_t\ dP - \int_A Z'_n\ dP \right\}
				\end{align}
				が成立する.またポテンシャルは非負であるから
				\begin{align}
					0 \leq \int_A Z'_n\ dP \leq \int_\Omega Z'_n\ dP \longrightarrow 0
					\quad (n \longrightarrow \infty)
				\end{align}
				が成り立ち,$M_t = M'_t\ \mbox{$P$-a.s.}$及び$Z_t = Z'_t\ \mbox{$P$-a.s.}$が従う.パスの右連続性より
				(\refeq{eq:chapter_1_Exercise_3_18_4})が出る.
 				\QED
 		\end{description}
 	\end{prf}
	
	\begin{itembox}[l]{Problem 3.19}
		Assume that $\mathscr{F}_0$ contains all the $P$-negligible events in $\mathscr{F}$ \footnotemark.
		Then the following three conditions are equivalent for a nonnegative, right-continuous 
		submartingale $\Set{X_t,\mathscr{F}_t}{0 \leq t < \infty}$:
		\begin{description}
			\item[(a)] it is a uniformly integrable family of random variables;
			\item[(b)] is converges in $L^1$, as $t \to \infty$;
			\item[(c)] it converges $P$ a.s. (as $t \to \infty$) to an integrable random variable $X_\infty$,
			such that $\Set{X_t,\mathscr{F}_t}{0 \leq t \leq \infty}$ is a submartingale.
		\end{description}
		Observe that the implications (a) $\Rightarrow$ (b) $\Rightarrow$ (c) hold without the assumption of nonnegativity. 
	\end{itembox}
	\footnotetext{
		証明の第二段で出てくる$E$が$\mathscr{F}_\infty$に属していなければならない.
	}
	\begin{prf}\mbox{}
		\begin{description}
			\item[第一段]
				(a) $\Rightarrow$ (b)を示す.実際,一様可積分性の同値条件の補題より
				\begin{align}
					\sup{t \geq 0}{EX_t^+} \leq \sup{t \geq 0}{E|X_t|} < \infty
				\end{align}
				となるから,劣マルチンゲール収束定理より或る$\mathscr{F}_\infty/\borel{\R}$-可測な
				\footnote{
					Theorem 3.15における$X_\infty$は$\pm \infty$も取るが,可積分性より
					$P$-a.s.に$\R$値であるから$X_\infty \defunc_{|X_\infty|<\infty}$を$X_\infty$に置き換えればよい.
				}
				可積分関数$X_\infty$が存在して
				\begin{align}
					\lim_{t \to \infty} X_t = X_\infty
					\quad \mbox{$P$-a.s.}
				\end{align}
				が満たされる.一様可積分性と平均収束の補題より,$t_n \uparrow \infty$となる任意の単調増大列$(t_n)_{n=1}^\infty$に対して
				\begin{align}
					E|X_{t_n} - X_\infty| \longrightarrow 0
					\quad (n \longrightarrow \infty)
				\end{align}
				が成立するから
				\begin{align}
					E|X_t - X_\infty| \longrightarrow 0
					\quad (t \longrightarrow \infty)
				\end{align}
				が従う.
			
			\item[第二段]
				(b) $\Rightarrow$ (c)を示す.(b)の下で,或る可積分関数$X_*$が存在して
				\begin{align}
					E|X_n - X_*| \longrightarrow 0
					\quad (n \longrightarrow \infty)
				\end{align}
				が満たされるから,或る部分列$\left( X_{n_k} \right)_{k=1}^\infty$と$P$-零集合$E$が存在して
				\begin{align}
					\lim_{k \to \infty} X_{n_k}(\omega) = X_*(\omega),
					\quad (\forall \omega \in \Omega \backslash E)
				\end{align}
				となる.$X_{n_k}\defunc_{\Omega \backslash E}$は全て$\mathscr{F}_\infty/\borel{\R}$-可測であるから,
				\begin{align}
					X_\infty \coloneqq \lim_{k \to \infty} X_{n_k} \defunc_{\Omega \backslash E}
				\end{align}
				とおけば$X_\infty$は$\mathscr{F}_\infty/\borel{\R}$-可測,
				かつ$X_\infty = X^*\ \mbox{$P$-a.s.}$より可積分であり
				\begin{align}
					E|X_n - X_\infty| = E|X_n - X_*| \longrightarrow 0
					\quad (n \longrightarrow \infty)
					\label{eq:chapter_1_Problem_3_19_3}
				\end{align}
				を満たす.任意の$t \geq 0$及び$A \in \mathscr{F}_t$に対し
				\begin{align}
					\int_A X_t\ dP \leq \int_A X_n\ dP,
					\quad (\forall n > t)
					\label{eq:chapter_1_Problem_3_19_1}
				\end{align}
				が成り立つから,(\refeq{eq:chapter_1_Problem_3_19_3})より
				\begin{align}
					\int_A X_t\ dP \leq \int_A X_\infty\ dP
					\label{eq:chapter_1_Problem_3_19_2}
				\end{align}
				が出る.
				
			\item[第三段]
				$X_t \geq 0\ (\forall t \geq 0)$を仮定して(c) $\Rightarrow$ (a)を示す.実際,
				劣マルチンゲール性より
				\begin{align}
					\int_{|X_t| > \lambda} |x_t|\ dP
					= \int_{X_t > \lambda} X_t\ dP
					\leq \int_{X_t > \lambda} X_\infty\ dP
				\end{align}
				かつ
				\begin{align}
					P\left( X_t > \lambda \right)
					\leq \frac{1}{\lambda} EX_t
					\leq \frac{1}{\lambda} EX_\infty
				\end{align}
				が成り立ち,$X_\infty$の可積分性より
				\begin{align}
					\sup{t \geq 0}{\int_{|X_t| > \lambda} |x_t|\ dP} 
					\longrightarrow 0
					\quad (\lambda \longrightarrow \infty)
				\end{align}
				となる.
				\QED
		\end{description}
	\end{prf}
	
	\begin{itembox}[l]{Problem 3.20}
		Assume that $\mathscr{F}_0$ contains all the $P$-negligible events in $\mathscr{F}$.
		Then the following four conditions are equivalent for a right-continuous martingale
		$\Set{X_t,\mathscr{F}_t}{0 \leq t < \infty}$:
		\begin{description}
			\item[(a),(b)] as in Problem 3.19;
			\item[(c)] it converges $P$ a.s. (as $t \to \infty$) to an integrable random variable $X_\infty$,
				such that $\Set{X_t,\mathscr{F}_t}{0 \leq t \leq \infty}$ is a martingale;
			\item[(d)] there exists an integrable random variable $Y$, such that $X_t = \cexp{Y}{\mathscr{F}_t}$ a.s. $P$,
				for every $t \geq 0$.
		\end{description}
		Besides, if (d) holds and $X_\infty$ is the random variable in (c), then
		\begin{align}
			\cexp{Y}{\mathscr{F}_\infty} = X_\infty
			\quad \mbox{a.s. $P$}.
			\label{eq:chapter_1_Problem_3_20_1}
		\end{align}
	\end{itembox}
	
	\begin{prf}\mbox{}
		\begin{description}
			\item[第一段] マルチンゲールは劣マルチンゲールであるから,Problem 3.19より(a) $\Rightarrow$ (b)が従う.
				また今の仮定の下では
				(\refeq{eq:chapter_1_Problem_3_19_1})と(\refeq{eq:chapter_1_Problem_3_19_2})
				の不等号が等号に代わり(b) $\Rightarrow$ (c)となる.$Y \coloneqq X_\infty$として(c) $\Rightarrow$ (d)が得られ,
				一様可積分性と条件付き期待値に関する補題(P. \pageref{lem:uniformly_integrability_and_conditional_expectations})
				より(d) $\Rightarrow$ (a)が出る.
				
			\item[第二段]
				(\refeq{eq:chapter_1_Problem_3_20_1})を示す.
				いま,任意の$t \geq 0$及び$A \in \mathscr{F}_t$に対し
				\begin{align}
					\int_A Y\ dP = \int_A X_t\ dP = \int_A X_\infty\ dP
				\end{align}
				が成立するから
				\begin{align}
					\int_A Y\ dP = \int_A X_\infty\ dP,
					\quad (\forall A \in \bigcup_{t \geq 0} \mathscr{F}_t)
				\end{align}
				が従う.$Y$と$X_\infty$の可積分性より
				\begin{align}
					\mathscr{D} \coloneqq
					\Set{A \in \mathscr{F}_\infty}{\int_A Y\ dP = \int_A X_\infty\ dP}
				\end{align}
				はDynkin族をなし乗法族$\bigcup_{t \geq 0} \mathscr{F}_t$を含むから,
				Dynkin族定理より
				\begin{align}
					\int_A Y\ dP = \int_A X_\infty\ dP,
					\quad (\forall A \in \mathscr{F}_\infty)
				\end{align}
				が成立する.
				\QED
		\end{description}
	\end{prf}
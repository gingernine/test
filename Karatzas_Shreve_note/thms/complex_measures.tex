\section{複素測度}
	\begin{screen}
		\begin{dfn}[複素測度]
			$(X,\mathscr{F})$を可測空間とする.
			$\lambda: \mathscr{F} \longrightarrow \C$が
			任意の互いに素な列$(E_i)_{i=1}^{\infty} \subset \mathscr{F}$に対し
			\begin{align}
				\lambda\biggl( \sum_{i=1}^{\infty} E_i \biggr) = \sum_{i=1}^{\infty} \lambda(E_i)
				\label{eq:dfn_complex_measure}
			\end{align}
			を満たすとき,$\lambda$を複素測度(complex measure)という.
		\end{dfn}
	\end{screen}
	
	任意の全単射$\sigma:\N \rightarrow \N$に対し
	\begin{align}
		(E \coloneqq)\ \sum_{i=1}^{\infty}E_i = \sum_{i=1}^{\infty}E_{\sigma(i)}
	\end{align}
	が成り立つから
	\begin{align}
		\sum_{i=1}^{\infty} \lambda(E_i) = \lambda(E) = \sum_{i=1}^{\infty} \lambda(E_{\sigma(i)})
	\end{align}
	が従い,Riemannの級数定理より
	$\sum_{i=1}^{\infty} \lambda(E_i)$は絶対収束する.
	ここで,
	\begin{align}
		|\lambda(E)| \leq \mu(E) \quad (\forall E \in \mathscr{F})
		\label{radon_nikodym_1}
	\end{align}
	を満たすような或る$(X,\mathscr{F})$上の測度$\mu$が存在すると考える.
	このとき$\mu$は
	\begin{align}
		\sum_{i=1}^{\infty} |\lambda(E_i)| \leq \sum_{i=1}^{\infty} \mu(E_i) 
		= \mu\Biggl(\sum_{i=1}^{\infty} E_i\Biggr)
	\end{align}
	を満たすから
	\begin{align}
		\sup{}{\Set{\sum_{i=1}^{\infty} |\lambda(A_i)|}{E = \sum_{i=1}^\infty A_i,\ \{A_i\}_{i=1}^\infty \subset \mathscr{F}}} 
		\leq \mu(E),
		\quad (\forall E \in \mathscr{F})
		\label{radon_nikodym_2}
	\end{align}
	が成立する.実は,
	\begin{align}
		|\lambda|(E) \coloneqq 
		\sup{}{\Set{\sum_{i=1}^{\infty} |\lambda(A_i)|}{E = \sum_{i=1}^\infty A_i,\ \{A_i\}_{i=1}^\infty \subset \mathscr{F}}},
		\quad (\forall E \in \mathscr{F})
		\label{radon_nikodym_3}
	\end{align}
	で定める$|\lambda|$は(\refeq{radon_nikodym_1})を満たす最小の有限測度となる
	(定理\ref{thm:total_variation_measure},定理\ref{thm:total_variation_measure_bounded}).
	
	\begin{screen}
		\begin{dfn}[総変動・総変動測度]
			可測空間$(X,\mathscr{F})$上の複素測度$\lambda$に対し,(\refeq{radon_nikodym_3})で定める
			$|\lambda|$を$\lambda$の総変動測度(total variation measure)といい,$|\lambda|(X)$を
			$\lambda$の総変動(total variation)という.
		\end{dfn}
	\end{screen}
	特に$\lambda$が正値有限測度である場合は$\lambda = |\lambda|$が成り立つ.実際,任意の$E \in \mathscr{F}$に対して
	\begin{align}
		|\lambda|(E) = \sup{}{\Set{\sum_{i=1}^{\infty} |\lambda(A_i)|}{E = \sum_{i=1}^\infty A_i,\ \{A_i\}_{i=1}^\infty \subset \mathscr{F}}}
		= \lambda(E)
	\end{align}
	が成立する.

	\begin{screen}
		\begin{thm}[$|\lambda|$は測度]
			可測空間$(X,\mathscr{F})$上の複素測度$\lambda$に対して,
			(\refeq{radon_nikodym_3})で定める$|\lambda|$は正値測度である.
			\label{thm:total_variation_measure}
		\end{thm}
	\end{screen}
	
	\begin{prf}
		$|\lambda|$の正値性は(\refeq{radon_nikodym_3})より従うから,
		$|\lambda|$の完全加法性を示す.
		いま,互いに素な集合列$E_i \in \mathscr{F}\ (i=1,2,\cdots)$を取り
		$E \coloneqq \sum_{i=1}^{\infty} E_i$とおく.
		このとき,任意の$\epsilon > 0$に対して
		$E_i$の或る分割$(A_{ij})_{j=1}^{\infty} \subset \mathscr{F}$が存在して
		\begin{align}
			|\lambda|(E_i) \geq \sum_{j=1}^{\infty} |\lambda(A_{ij})| 
			> |\lambda|(E_i) - \frac{\epsilon}{2^i}
		\end{align}
		を満たすから,$E = \sum_{i,j=1}^{\infty} A_{ij}$と併せて
		\begin{align}
			|\lambda|(E) \geq \sum_{i,j=1}^{\infty} |\lambda(A_{ij})| \geq \sum_{i=1}^{\infty}\sum_{j=1}^{\infty} |\lambda(A_{ij})| > \sum_{i=1}^{\infty} |\lambda|(E_i) - \epsilon
		\end{align}
		となり,$\epsilon > 0$の任意性より
		\begin{align}
			|\lambda|(E) \geq \sum_{j=1}^{\infty} |\lambda|(E_j)
		\end{align}
		が従う.一方で$E$の任意の分割$(A_j)_{j=1}^{\infty} \subset \mathscr{F}$に対し
		\begin{align}
			\sum_{j=1}^{\infty} |\lambda(A_j)| 
			= \sum_{j=1}^{\infty} \left| \sum_{i=1}^{\infty} \lambda(A_j \cap E_i) \right|
			\leq \sum_{j=1}^{\infty} \sum_{i=1}^{\infty} |\lambda|(A_j \cap E_i)
			\leq \sum_{i=1}^{\infty} |\lambda|(E_i)
		\end{align}
		が成り立つから,$E$の分割について上限を取って
		\begin{align}
			|\lambda|(E) \leq \sum_{i=1} |\lambda|(E_i)
		\end{align}
		を得る.
		\QED
	\end{prf}
	
	\begin{screen}
		\begin{lem}\label{lem:total_variation_measure_bounded}
			$z_1,\cdots,z_N$を複素数とする.このとき,次を満たす或る部分集合$S \subset \{1,\cdots,N\}$が存在する:
			\begin{align}
				\left| \sum_{k \in S} z_k \right| \geq \frac{1}{2\pi} \sum_{k=1}^{N} |z_k|.
			\end{align}
		\end{lem}
	\end{screen}
	
	\begin{prf}
		$i = \sqrt{-1}$として,
		$z_k = |z_k|\exp{i \alpha_k}\ (-\pi \leq \alpha_k < \pi,\ k=1,\cdots,N)$を満たす$\alpha_1,\cdots,\alpha_N$を取り
		\begin{align}
			S(\theta) \coloneqq \Set{k \in \{1,\cdots,N\}}{\cos{(\alpha_k - \theta)}{} > 0},
			\quad (-\pi \leq \theta \leq \pi)
		\end{align}
		とおく.このとき,$\cos{x}{+} \coloneqq 0 \vee \cos{x}{}\ (x \in \R)$とすれば
		\begin{align}
			\left| \sum_{k \in S(\theta)} z_k \right| &= |\exp{-i\theta}|\left| \sum_{k \in S(\theta)} z_k \right| = \left| \sum_{k \in S(\theta)} |z_k|\exp{i(\alpha_k - \theta)} \right| \\
			&\geq \Re{\sum_{k \in S(\theta)} |z_k|\exp{i(\alpha_k - \theta)}} = \sum_{k \in S(\theta)} |z_k| \cos{(\alpha_k - \theta)}{} = \sum_{k=1}^{N} |z_k| \cos{(\alpha_k - \theta)}{+}
		\end{align}
		が成り立ち,最右辺は$\theta$に関して連続であるから最大値を達成する$\theta_0 \in [-\pi,\pi]$が存在する.
		$S \coloneqq S(\theta_0)$として
		\begin{align}
			\left| \sum_{k \in S} z_k \right| \geq \sum_{k=1}^{N} |z_k| \cos{(\alpha_k - \theta_0)}{+} \geq \sum_{k=1}^{N} |z_k| \cos{(\alpha_k - \theta)}{+}
			\quad (\forall \theta \in [-\pi, \pi])
		\end{align}
		となり,積分して
		\begin{align}
			\left| \sum_{k \in S} z_k \right| 
			&\geq \sum_{k=1}^{N} |z_k| \frac{1}{2\pi} \int_{[-\pi,\pi]} \cos{(\alpha_k - \theta)}{+}\ d\theta \\
			&= \sum_{k=1}^{N} |z_k| \frac{1}{2\pi} \int_{[-\pi,\pi]} \cos{\theta}{+}\ d\theta
			= \frac{1}{2\pi} \sum_{k=1}^{N} |z_k|
		\end{align}
		が得られる.
		\QED
	\end{prf}
	
	\begin{screen}
		\begin{thm}[総変動は有限]\label{thm:total_variation_measure_bounded}
			可測空間$(X,\mathscr{F})$上の複素測度$\lambda$の総変動測度$|\lambda|$について次が成り立つ:
			\begin{align}
				|\lambda|(X) < \infty.
			\end{align}
			特に,複素測度は有界である.
		\end{thm}
	\end{screen}

	\begin{prf} $|\lambda|(X) = \infty$と仮定して背理法により定理を導く.
		\begin{description}
		\item[第一段]
			或る$E \in \mathscr{F}$に対し$|\lambda|(E) = \infty$が成り立っているなら,
			\begin{align}
				|\lambda(A)| > 1, \quad |\lambda(B)| > 1, \quad E = A + B
			\end{align}
			を満たす$A,B \in \mathscr{F}$が存在することを示す.いま,$t \coloneqq 2\pi(1 + |\lambda(E)|)$とおけば
			\begin{align}
				\sum_{i=1}^{\infty} |\lambda(E_i)| > t
			\end{align}
			を満たす$E$の分割$(E_i)_{i=1}^{\infty}$が存在する.従って或る$N \in \N$に対し
			\begin{align}
				\sum_{i=1}^{N} |\lambda(E_i)| > t
			\end{align}
			が成り立ち,補題\ref{lem:total_variation_measure_bounded}より
			\begin{align}
				\left| \sum_{k \in S} \lambda(E_k) \right| \geq \frac{1}{2\pi} \sum_{k=1}^{N} |\lambda(E_k)| > \frac{t}{2\pi} > 1
			\end{align}
			を満たす$S \subset \{1,\cdots,N\}$が取れる.ここで$A \coloneqq \sum_{k \in S} E_k,\ B \coloneqq E - A$とおけば,
			$|\lambda(A)| > 1$かつ
			\begin{align}
				|\lambda(B)| = |\lambda(E)-\lambda(A)| \geq |\lambda(A)| - |\lambda(E)| > \frac{t}{2\pi} - |\lambda(E)| = 1
			\end{align}
			が成り立つ.また,
			\begin{align}
				|\lambda|(E) = |\lambda|(A) + |\lambda|(B)
			\end{align}
			より$|\lambda|(A),\ |\lambda|(B)$の少なくとも一方は$\infty$となる.
		
		\item[第二段]
			いま,$|\lambda|(X) = \infty$と仮定すると,前段の結果より
			\begin{align}
				|\lambda|(B_1) = \infty, \quad |\lambda(A_1)| > 1, \quad |\lambda(B_1)| > 1,
				\quad X = A_1 + B_1
			\end{align}
			を満たす$A_1,B_1 \in \mathscr{F}$が存在する.同様に$B_1$に対しても
			\begin{align}
				|\lambda|(B_2) = \infty, \quad |\lambda(A_2)| > 1, \quad |\lambda(B_2)| > 1,
				\quad B_1 = A_2 + B_2
			\end{align}
			を満たす$A_2,B_2 \in \mathscr{F}$が存在する.
			繰り返せば$|\lambda(A_j)| > 1\ (j=1,2,\cdots)$
			を満たす互いに素な集合列$(A_j)_{j=1}^{\infty}$が構成され,
			このとき$\sum_{j=1}^{\infty} |\lambda(A_j)| = \infty$となる.
			一方でRiemannの級数定理より$\sum_{j=1}^{\infty} |\lambda(A_j)| < \infty$
			が成り立つから矛盾が生じ,$|\lambda|(X) < \infty$が出る.
			\QED
		\end{description}
	\end{prf}
	
	\begin{screen}
		\begin{thm}[複素測度全体は線型空間・総変動ノルム]
			可測空間$(X,\mathscr{F})$上の複素測度の全体を$CM(X,\mathscr{F})$と書く.
			\begin{align}
				&(\lambda + \mu)(E) \coloneqq \lambda(E) + \mu(E), \\
				&(c\lambda)(E) \coloneqq c\lambda(E)
				\label{complex_measure_linear}
			\end{align}
			を線型演算として$CM(X,\mathscr{F})$は線形空間となり,また
			\begin{align}
				\Norm{\lambda}{TV} \coloneqq |\lambda|(X) \quad (\lambda \in CM(X,\mathscr{F}))
			\end{align}
			により$CM(X,\mathscr{F})$に総変動ノルム$\Norm{\cdot}{TV}$が定まる.
		\end{thm}
	\end{screen}
	
	\begin{prf}\mbox{}
		$\Norm{\cdot}{TV}$がノルムであることを示す.
		\begin{description}
			\item[第一段]
				$\lambda = 0$なら$\Norm{\lambda}{TV} = |\lambda|(X) = 0$となる.また
				$|\lambda(E)| \leq |\lambda|(E) \leq \Norm{\lambda}{TV}$より
				$\Norm{\lambda}{TV} = 0$なら$\lambda=0$が従う.
			
			\item[第二段]
				任意の$\lambda \in CM(X,\mathscr{F})$と$c \in \C$に対し
				\begin{align}
					\Norm{c\lambda}{TV} = \sup{}{\sum_{i}|(c\lambda)(E_i)|} = \sup{}{\sum_{i}|c\lambda(E_i)|} = |c|\sup{}{\sum_{i}|\lambda(E_i)|} = |c|\Norm{\lambda}{TV}
				\end{align}
				が成り立ち同次性が得られる.
			
			\item[第三段]
				$\lambda,\mu \in CM(X,\mathscr{F})$を任意に取る.このとき,
				$X$の任意の分割$X = \sum_{i=1}^\infty E_i\ (E_i \in \mathscr{F})$に対して
				\begin{align}
					\sum_{i=1}^\infty |(\lambda + \mu)(E_i)| 
					= \sum_{i=1}^\infty |\lambda(E_i) + \mu(E_i)| 
					\leq \sum_{i=1}^\infty |\lambda(E_i)| + \sum_{i=1}^\infty |\mu(E_i)| \leq \Norm{\lambda}{TV} + \Norm{\mu}{TV}
				\end{align}
				が成り立つから$\Norm{\lambda + \mu}{TV} \leq \Norm{\lambda}{TV} + \Norm{\mu}{TV}$が従う.
				\QED
		\end{description}
	\end{prf}
	
	可測空間$(X,\mathscr{F})$において,$\R$にしか値を取らない複素測度を符号付き測度(signed measure)という.
	\begin{screen}
		\begin{dfn}[正変動と負変動・Jordanの分解]
			$(X,\mathscr{F})$を可測空間とする.$(X,\mathscr{F})$上の符号付き測度$\mu$に対し
			\begin{align}
				\mu^+ \coloneqq \frac{1}{2}(|\mu| + \mu) , \quad \mu^- \coloneqq \frac{1}{2}(|\mu| - \mu)
			\end{align}
			として正値有限測度$\mu^+,\mu^-$を定める.
			$\mu^+\ (\mu^-)$を$\mu$の正(負)変動(positive (negative) variation)と呼び,
			\begin{align}
				\mu = \mu^+ - \mu^-
			\end{align}
			を符号付き測度$\mu$のJordan分解(Jordan decomposition)という.同時に$|\mu| = \mu^+ + \mu^-$も成り立つ.
		\end{dfn}
	\end{screen}
	
	\begin{screen}
		\begin{dfn}[絶対連続・特異]
			$(X,\mathscr{F})$を可測空間,
			$\mu$を$\mathscr{F}$上の正値測度
			,$\lambda,\lambda_1,\lambda_2$を$\mathscr{F}$上の任意の測度とする.
			\begin{itemize}
				\item $\mu(E)=0$ならば$\lambda(E)=0$となるとき,
					$\lambda$は$\mu$に関して絶対連続である(absolutely continuous)といい
					\begin{align}
						\lambda \ll \mu
					\end{align}
					と書く.
				
				\item 或る$A \in \mathscr{F}$が存在して
					\begin{align}
						\lambda(E) = \lambda(A \cap E),\quad (\forall E \in \mathscr{F})
					\end{align}
					が成り立つとき,$\lambda$は$A$に集中している(concentrated on A)という.
					$\lambda_1$が$A_1$に,$\lambda_2$が$A_2$に集中し,かつ
					$A_1 \cap A_2 = \emptyset$であるとき,
					$\lambda_1,\lambda_2$は互いに特異である(mutually singular)といい
					\begin{align}
						\lambda_1 \perp \lambda_2
					\end{align}
					と書く.
			\end{itemize}
		\end{dfn}
	\end{screen}
	
	\begin{screen}
		\begin{thm}[絶対連続性の同値条件]\label{thm:equivalent_condition_of_absolute_continuity}
			$\lambda,\mu$をそれぞれ可測空間$(X,\mathscr{F})$上の複素測度,正値測度とするとき,
			次は同値である:
			\begin{description}
				\item[(1)] $\lambda \ll \mu$,
				\item[(2)] $|\lambda| \ll \mu$
				\item[(3)] 任意の$\epsilon > 0$に対し或る$\delta > 0$が存在して
					$\mu(E) < \delta$なら$|\lambda|(E) < \epsilon$となる.
			\end{description}
		\end{thm}
	\end{screen}
	
	\begin{prf}\mbox{}
		\begin{description}
			\item[第一段]
				$(1) \Leftrightarrow (2)$を示す.
				任意の$E \in \mathscr{F}$に対し$|\lambda(E)| \leq |\lambda|(E)$より
				$(2) \Rightarrow (1)$が従う.また$\lambda \ll \mu$のとき,
				\begin{align}
					|\lambda|(E) =
					\sup{}{\Set{\sum_{i=1}^{\infty} |\lambda(A_i)|}{E = \sum_{i=1}^\infty A_i,\ \{A_i\}_{i=1}^\infty \subset \mathscr{F}}},
					\quad (\forall E \in \mathscr{F})
				\end{align}
				より$\mu(E) = 0$なら$\mu(A_i) = 0$となり
				$\lambda(A_i) = 0\ (\forall i \geq 1)$が満たされ
				$(1) \Rightarrow (2)$が従う.
				
			\item[第二段]
				$(2) \Leftrightarrow (3)$を示す.
				実際(3)が満たされているとき,$\mu(E) = 0$なら任意の$\delta > 0$に対し
				$\mu(E) < \delta$となるから$|\lambda|(E) < \epsilon\ (\forall \epsilon > 0)$
				となり$|\mu|(E) = 0$が出る.逆に(3)が満たされていないとき,或る$\epsilon > 0$に対して
				\begin{align}
					\mu(E_n) < \frac{1}{2^{n+1}}, \quad |\lambda|(E_n) \geq \epsilon,
					\quad (n=1,2,\cdots)
				\end{align}
				を満たす$\{E_n\}_{n=1}^\infty \subset \mathscr{F}$が存在する.このとき
				\begin{align}
					A_n \coloneqq \bigcup_{i=n}^\infty E_i,
					\quad A \coloneqq \bigcap_{n=1}^\infty A_n
				\end{align}
				とおけば
				\begin{align}
					\mu(A) = \lim_{n \to \infty} \mu(A_n) 
					\leq \lim_{n \to \infty} \frac{1}{2^n} = 0
				\end{align}
				かつ
				\begin{align}
					|\lambda|(A) = \lim_{n \to \infty} |\lambda|(A_n) 
					\geq \lim_{n \to \infty} |\lambda|(E_n) \geq \epsilon 
				\end{align}
				が成り立ち,対偶を取れば$(2) \Rightarrow (3)$が従う.
				\QED
		\end{description}
	\end{prf}
	
	\begin{screen}
		\begin{lem}\label{lem:Lebesgue_Radon_Nikodym}
			$(X,\mathscr{F},\mu)$を$\sigma$-有限測度空間とするとき,
			$0 < w < 1$を満たす可積分関数$w$が存在する.
		\end{lem}
	\end{screen}
	
	\begin{prf}
		$\mu(X) = 0$なら$w \equiv 1/2$とすればよい.$\mu(X) > 0$の場合,$\sigma$-有限の仮定より
		\begin{align}
			0 < \mu(X_n) < \infty,\ (\forall n \geq 1),
			\quad X = \bigcup_{n=1}^\infty X_n
		\end{align}
		を満たす$\{X_n\}_{n=1}^\infty \subset \mathscr{F}$が存在する.ここで
		\begin{align}
			w_n(x) \coloneqq
			\begin{cases}
				\displaystyle\frac{1}{2^n\left(1+\mu(X_n)\right)}, & x \in X_n, \\
				0, & x \in X \backslash X_n,
			\end{cases}
			\quad n=1,2,\cdots
		\end{align}
		に対して
		\begin{align}
			w \coloneqq \sum_{n=1}^\infty w_n
		\end{align}
		と定めれば,任意の$x \in X$は或る$X_n$に属するから
		\begin{align}
			0 < w_n(x) \leq w(x)
		\end{align}
		が成り立ち,かつ
		\begin{align}
			w(x) = w_1(x) + \sum_{n=2}^\infty w_n(x)
			\leq \frac{1}{2\left(1+\mu(X_1)\right)} + \frac{1}{2}
			< 1,
			\quad (\forall x \in X)
		\end{align}
		が満たされる.また単調収束定理より
		\begin{align}
			\int_X w\ d\mu \leq \sum_{n=1}^\infty \int_X w_n\ d\mu
			\leq \sum_{n=1}^\infty \frac{\mu(X_n)}{2^n\left(1+\mu(X_n)\right)}
			\leq 1
		\end{align}
		となり$w$の可積分性が出る.
		\QED
	\end{prf}
	
	\begin{screen}
		\begin{thm}[Lebesgue-Radon-Nikodym]
			$(X,\mathscr{F})$を可測空間,$\lambda$を$(X,\mathscr{F})$上の複素測度,
			$\mu$を$(X,\mathscr{F})$上の$\sigma$-有限正値測度$(\mu(X)>0)$とするとき,以下が成立する:
			\begin{description}
				\item[Lebesgue分解]
					$\lambda$は$\mu$に関して絶対連続な$\lambda_a$及び$\mu$と互いに特異な
					$\lambda_s$に一意に分解される:
					\begin{align}
						\lambda = \lambda_a + \lambda_s,
						\quad \lambda_a \ll \mu,
						\quad \lambda_s \perp \mu.
					\end{align}
				
				\item[密度関数の存在]
					$\lambda_a$に対し或る$g \in L^1(\mu) = L^1(X,\mathscr{F},\mu)$が唯一つ存在して次を満たす:
					\begin{align}
						\lambda_a(E) = \int_E g\ d\mu,
						\quad (\forall E \in \mathscr{F}).
					\end{align}
			\end{description}
		\end{thm}
	\end{screen}
	
	\begin{prf}\mbox{}
		\begin{description}
			\item[第一段] Lebesgueの分解の一意性を示す.
				$\lambda'_a \ll \mu$と$\lambda'_s \perp \mu$により
				\begin{align}
					\lambda_a + \lambda_s = \lambda'_a + \lambda'_s
				\end{align}
				が成り立つとき,
				\begin{align}
					\Lambda \coloneqq \lambda_a - \lambda'_a = \lambda'_s - \lambda_s,
					\quad \Lambda \ll \mu,
					\quad \Lambda \perp \mu
				\end{align}
				となり$\Lambda = 0$が従い分解の一意性が出る.
			
			\item[第二段] 密度関数の一意性を示す.実際,可積分関数$f$に対して
				\begin{align}
					\int_E f\ d\mu = 0, \quad (\forall E \in \mathscr{F})
				\end{align}
				が成り立つとき,定理\ref{thm:mean_value_of_integral_and_closed_set}より
				$f = 0,\ \mbox{$\mu$-a.e.}$が成り立つ.
				
			\item[第三段] Lebesgueの分解と密度関数の存在を示す.
				
		\end{description}
	\end{prf}
	
	\begin{screen}
		\begin{thm}[Vitali-Hahn-Saks]\label{thm;Vitali_Hahn_Saks}
			$(X,\mathscr{F})$を可測空間,$(\lambda_n)_{n=1}^\infty$をこの上の複素測度の列とするとき,
			\begin{align}
				\lambda(E) \coloneqq \lim_{n \to \infty} \lambda_n(E),
				\quad (\forall E \in \mathscr{F})
				\label{eq:thm_Vitali_Hahn_Saks}
			\end{align}
			が存在すれば$\lambda$もまた$(X,\mathscr{F})$上の複素測度となる.
			つまり$(CM(X,\mathscr{F}),\Norm{\cdot}{TV})$はBanach空間である.
		\end{thm}
	\end{screen}
	
	\begin{prf}$\lambda_n \equiv 0\ (\forall n \geq 1)$なら$\lambda \equiv 0$で複素測度となるから,
		或る$n$と$E \in \mathscr{F}$に対し$\lambda_n(E) \neq 0$と仮定する.
		\begin{description}
			\item[第一段] $(X,\mathscr{F})$上の有限測度を
				\begin{align}
					\mu \coloneqq \sum_{n=1}^\infty \frac{1}{2^n(1 + \Norm{\lambda_n}{TV})} |\lambda_n|
				\end{align}
				により定めるとき,%定理\ref{thm:equivalent_condition_of_absolute_continuity}より
				任意の$\epsilon > 0$に対して或る$\delta > 0$が存在し
				\begin{align}
					\mu(E) < \delta \quad \Rightarrow \quad |\lambda_n|(E) < \epsilon\ (\forall n \geq 1)
					\label{eq:thm_Vitali_Hahn_Saks_2}
				\end{align}
				となることを示す.
				任意の$n \geq 1$に対して$\lambda_n \ll \mu$であるから
				Lebesgue-Radon-Nikodymの定理より
				\begin{align}
					\lambda_n(E) = \int_E g_n\ d\mu,
					\quad (\forall E \in \mathscr{F})
				\end{align}
				を満たす$g_n \in L^1(\mu)$が存在し,このとき
				\begin{align}
					\left| \int_E g_n\ d\mu \right|
					\leq |\lambda_n|(E)
					\leq 2^n(1+\Norm{\lambda_n}{TV})\mu(E),
					\quad (\forall E \in \mathscr{F})
				\end{align}
				が成立するから定理\ref{thm:mean_value_of_integral_and_closed_set}より
				\begin{align}
					\Norm{g_n}{L^\infty(\mu)} \leq 2^n(1+\Norm{\lambda_n}{TV})
				\end{align}
				が従う.いま,任意の$E \in \mathscr{F}$に対し$f_E \coloneqq [\defunc_E]$として
				\begin{align}
					L \coloneqq \Set{f_E}{E \in \mathscr{F}}
				\end{align}
				とおけば,$\mu(X) < \infty$より$L \subset L^1(\mu)$となり,また
				\begin{align}
					d(f_E,f_{E'}) \coloneqq \Norm{f_E - f_{E'}}{L^1(\mu)}
				\end{align}
				で定める距離$d$により$L$は完備距離空間となる.実際,定理\ref{thm:Lp_banach}より
				$L$の任意のCauchy列$\left(f_{E_n}\right)_{n=1}^\infty$に対し極限$f \in L^1(\mu)$が存在し,
				或る部分列$\left(\defunc_{E_{n_k}}\right)_{k=1}^\infty$は或る$\mu$-零集合$A$を除いて各点収束するから
				\begin{align}
					\varphi \coloneqq \lim_{k \to \infty} \defunc_{E_{n_k}} \defunc_{X \backslash A}
				\end{align}
				に対し$E \coloneqq \{\varphi = 1\}$とおけば$f = [\defunc_E] \in L$が満たされる.ここで
				\begin{align}
					\Phi_n: L \ni f_E \longmapsto \int_X |g_n| f_E\ d\mu
				\end{align}
				とおけば,任意の$E \in \mathscr{F}$に対し$|\lambda_n|(E) \leq \Phi_n(f_E)$が満たされ,
				またH\Ddot{o}lderの不等式より
				\begin{align}
					\left| \Phi_n(f_E) - \Phi_n(f_{E'}) \right|
					\leq \int_X |g_n| |f_E - f_{E'}|\ d\mu
					\leq \Norm{g_n}{L^\infty(\mu)} d(f_E,f_{E'}),
					\quad (\forall f_E,f_{E'} \in L)
				\end{align}
				がとなるから$\Phi_n$は$L$上の連続写像である.いま$\epsilon > 0$を任意に取れば,
				$\eta \coloneqq \epsilon/4$に対して
				\begin{align}
					F_n(\eta) 
					\coloneqq \Set{f_E \in L}{\sup{k \geq 1}{\left| \Phi_n(f_E)-\Phi_{n+k}(f_E) \right|} \leq \eta}
					= \bigcap_{k \geq 1} \Set{f_E \in L}{\left| \Phi_n(f_E)-\Phi_{n+k}(f_E) \right| \leq \eta}
				\end{align}
				により定める$F_n(\delta)$は閉集合であり,任意の$f_E \in L$は
				\begin{align}
					\sup{k \geq 1}{\left| \Phi_n(f_E)-\Phi_{n+k}(f_E) \right|}
					&\leq \left| \Phi_n(f_E)-\lambda(E) \right|
						+ \sup{k \geq 1}{\left| \lambda(E)-\Phi_{n+k}(f_E) \right|} \\
					&= \left| \lambda_n(E)-\lambda(E) \right|
						+ \sup{k \geq 1}{\left| \lambda(E)-\lambda_{n+k}(E) \right|} \\
					&\longrightarrow 0 \quad (n \longrightarrow \infty)
				\end{align}
				を満たすから
				\begin{align}
					L = \bigcup_{n=1}^\infty F_n(\eta)
				\end{align}
				が成り立ち,Baireの範疇定理より或る$F_{n_0}(\eta)$は内点$f_{E_0}$を持つ.
				つまり或る$\delta_0 > 0$が存在して
				\begin{align}
					d(f_{E_0},f_E) < \delta_0
					\quad \Rightarrow \quad
					\sup{k \geq 1}{\left| \Phi_n(f_E)-\Phi_{n+k}(f_E) \right|} \leq \eta
				\end{align}
				となる.$\mu(E) < \delta_0$ならば,
				\begin{align}
					E_1 \coloneqq E \cup E_0,
					\quad E_2 \coloneqq E_0 \backslash (E \cap E_0)
				\end{align}
				とすれば$f_E = [\defunc_E] = [\defunc_{E_1} - \defunc_{E_2}]
				= [\defunc_{E_1}] - [\defunc_{E_2}] = f_{E_1} - f_{E_2}$かつ
				\begin{align}
					d(f_{E_0},f_{E_1}) = \mu(E \backslash E_0) < \delta_0,
					\quad d(f_{E_0},f_{E_2}) = \mu(E \cap E_0) < \delta_0
				\end{align}
				が満たされるから,$n > n_0$なら
				\begin{align}
					\left|\Phi_n(f_E)\right| 
					&\leq \left|\Phi_{n_0}(f_E)\right| + \left|\Phi_n(f_E) - \Phi_{n_0}(f_E)\right| \\
					&\leq \left|\Phi_{n_0}(f_E)\right| + \left|\Phi_n(f_{E_1}) - \Phi_{n_0}(f_{E_1})\right|
						+ \left|\Phi_n(f_{E_2}) - \Phi_{n_0}(f_{E_2})\right| \\
					&\leq \left|\Phi_{n_0}(f_E)\right| + 2\eta
				\end{align}
				が従い,一方で$n=1,2,\cdots,n_0$に対しては,
				定理\ref{thm:integrable_intvalue_uniformly_shrinking}より
				或る$\delta_n > 0$が存在して
				\begin{align}
					\mu(E) < \delta_n \Longrightarrow \Phi_n(f_E) = \int_E |g_n|\ d\mu < \frac{\epsilon}{2}
				\end{align}
				が成立し,$\delta \coloneqq \min{}{\{\delta_0,\delta_1,\cdots,\delta_{n_0}\}}$として
				\begin{align}
					\mu(E) < \delta_n \Longrightarrow |\lambda_n|(E) \leq \Phi_n(f_E) < \epsilon,\ (\forall n \geq 1)
				\end{align}
				が得られる.
				
			\item[第二段] $\lambda$の可算加法性を示す.任意の互いに素な$A,B \in \mathscr{F}$を取れば
				\begin{align}
					\lambda(A + B) = \lim_{n \to \infty} \lambda_n(A + B)
					= \lim_{n \to \infty} \lambda_n(A) + \lim_{n \to \infty} \lambda_n(B)
					= \lambda(A) + \lambda(B)
				\end{align}
				となるから$\lambda$は有限加法的であり,このとき任意の互いに素な列$\{E_i\}_{i=1}^\infty \subset \mathscr{F}$に対し
				\begin{align}
					\lambda\Biggl( \sum_{i=1}^\infty E_i \Biggr)
					= \lambda\Biggl( \sum_{i=1}^N E_i \Biggr) + \lambda\Biggl( \sum_{i=N+1}^\infty E_i \Biggr)
					= \sum_{i=1}^N \lambda(E_i) + \lambda\Biggl( \sum_{i=N+1}^\infty E_i \Biggr)
				\end{align}
				が任意の$N \geq 1$について満たされるが,
				\begin{align}
					\mu\Biggl( \sum_{i=N+1}^\infty E_i \Biggr) \longrightarrow 0 \quad (N \longrightarrow \infty)
				\end{align}
				と(\refeq{eq:thm_Vitali_Hahn_Saks_2})より
				\begin{align}
					\lambda\Biggl( \sum_{i=N+1}^\infty E_i \Biggr) \longrightarrow 0 \quad (N \longrightarrow \infty)
				\end{align}
				が従い
				\begin{align}
					\lambda\Biggl( \sum_{i=1}^\infty E_i \Biggr) = \sum_{i=1}^\infty \lambda(E_i)
				\end{align}
				が得られる.よって$\lambda$は複素測度である.
				\QED
		\end{description}
	\end{prf}
	
	\begin{screen}
		\begin{thm}[$L^p$の共役空間]\label{thm:dual_space_of_L_p}
			$1 \leq p < \infty$,$q$を$p$の共役指数とし,また$(X,\mathscr{F},\mu)$を$\sigma$-有限な測度空間とするとき,
			$g \in L^q(\mu)$に対して
			\begin{align}
				\Phi_g: L^p(\mu) \ni f \longmapsto \int_X fg\ d\mu
				\label{eq:thm_dual_space_of_L_p_1}
			\end{align}
			は有界線形作用素となる.また
			\begin{align}
				\Phi: L^q(\mu) \ni g \longmapsto \Phi_g \in \left( L^p(\mu) \right)^*
			\end{align}
			で定める$\Phi$は$\left( L^p(\mu) \right)^*$から$L^q(\mu)$への線型同型であり,
			次の意味で等長である:
			\begin{align}
				\Norm{g}{L^q(\mu)} = \Norm{\Phi_g}{\left( L^p(\mu) \right)^*}.
				\label{eq:thm_dual_space_of_L_p_asseretion_2}
			\end{align}
			$p=\infty$の場合,$\mu(X) < \infty$かつ$\varphi \in \left( L^\infty(\mu) \right)^*$に対し
			$\mathscr{F} \ni A \longmapsto \varphi(\defunc_A)$が可算加法的ならば,
			$\varphi$に対し或る$g \in L^1(\mu)$が唯一つ存在して
			$\varphi = \Phi_g$と(\refeq{eq:thm_dual_space_of_L_p_asseretion_2})を満たす.
		\end{thm}
	\end{screen}
	
	\begin{prf}\mbox{}
		\begin{description}
			\item[第一段]
				$\Phi_g$が(\refeq{eq:thm_dual_space_of_L_p_1})で与えられていれば,H\Ddot{o}lderの不等式より
				\begin{align}
					\left|\Phi_g(f)\right| \leq \Norm{g}{L^q(\mu)}\Norm{f}{L^p(\mu)}
				\end{align}
				が成り立つから
				\begin{align}
					\Norm{\Phi_g}{\left( L^p(\mu) \right)^*} \leq \Norm{g}{L^q(\mu)}
					\label{eq:thm_dual_space_of_L_p_3}
				\end{align}
				が従う.よって$\Phi_g \in \left( L^p(\mu) \right)^*$となる.
			
			\item[第二段]
				$\varphi \in \left( L^p(\mu) \right)^*$に対して
				$\Phi(g) = \varphi$を満たす$g \in L^q(\mu)$が存在するとき,
				$g$が$\varphi$に対して一意に決まることを示す.$\sigma$-有限の仮定より
				\begin{align}
					\mu(X_n) < \infty,\ (\forall n \geq 1);
					\quad X = \bigcup_{n=1}^\infty X_n
					\label{eq:thm_dual_space_of_L_p_6}
				\end{align}
				を満たす$\{X_n\}_{n=1}^\infty \subset \mathscr{F}$が存在する.
				いま,$g,g' \in L^q(\mu)$に対して
				\begin{align}
					\int_X fg\ d\mu = \int_X fg'\ d\mu,
					\quad (\forall f \in L^p(\mu))
				\end{align}
				が成り立っているとすれば,任意の$E \in \mathscr{F}$に対して
				$\defunc_{E \cap X_n} \in L^p(\mu)$であるから
				\begin{align}
					\int_{E \cap X_n} g-g'\ d\mu = 0,
					\quad (\forall n \geq 1)
				\end{align}
				となり,Lebesgueの収束定理より
				\begin{align}
					\int_E g-g'\ d\mu = 0
				\end{align}
				が従い$L^q(\mu)$で$g = g'$が成立する.
				
			\item[第三段]
				$1 \leq p < \infty$の場合,$\mu(X) < \infty$なら
				任意の$\varphi \in \left( L^p(\mu) \right)^*$に対して
				$\Phi(g) = \varphi$を満たす$g \in L^q(\mu)$が存在することを示す.
				\begin{align}
					\lambda(E) \coloneqq \varphi(\defunc_E)
					\label{eq:thm_dual_space_of_L_p_7}
				\end{align}
				により$\lambda$を定めれば
				\begin{align}
					\lambda(A + B) = \varphi(\defunc_{A+B}) = \varphi(\defunc_A + \defunc_B)
					= \varphi(\defunc_A) + \varphi(\defunc_B)
					= \lambda(A) + \lambda(B)
				\end{align}
				となり$\lambda$の加法性が出る.また
				任意の互いに素な$\{E_n\}_{n=1}^\infty \in \mathscr{F}$に対して
				\begin{align}
					A_k \coloneqq \sum_{n=1}^k E_n,
					\quad A \coloneqq \sum_{n=1}^\infty E_n
				\end{align}
				とおけば
				\begin{align}
					\left| \lambda(A) - \sum_{n=1}^k \lambda(E_n) \right|
					&= \left| \lambda(A) - \lambda(A_k) \right|
					= \left| \varphi(\defunc_A - \defunc_{A_k}) \right| \\
					&\leq \Norm{\varphi}{\left( L^p(\mu) \right)^*} \Norm{\defunc_A - \defunc_{A_k}}{L^p(\mu)}
					= \Norm{\varphi}{\left( L^p(\mu) \right)^*} \mu(A - A_k)^{1/p}
					\longrightarrow 0
					\quad (k \longrightarrow \infty)
				\end{align}
				が成り立つから$\lambda$は複素測度である.また
				\begin{align}
					|\lambda(E)| \leq \Norm{\varphi}{\left( L^p(\mu) \right)^*} \mu(E)^{1/p}
				\end{align}
				より$\lambda \ll \mu$となるから,Lebesgue-Radon-Nikodymの定理より
				\begin{align}
					\varphi(\defunc_E) = \lambda(E) = \int_X \defunc_E g\ d\mu,
					\quad (\forall E \in \mathscr{F})
					\label{eq:thm_dual_space_of_L_p_8}
				\end{align}
				を満たす$g \in L^1(\mu)$が存在する.$\varphi$の線型性より
				任意の単関数の同値類$f$に対して
				\begin{align}
					\varphi(f) = \int_X fg\ d\mu
					\label{eq:thm_dual_space_of_L_p_2}
				\end{align}
				が成立し,特に$f \in L^\infty(\mu)$に対しては
				\begin{align}
					B \coloneqq \Set{x \in X}{|f(x)| > \Norm{f}{L^\infty(\mu)}}
				\end{align}
				とおけば$\mu(B) = 0$となり,有界可測関数$f \defunc_{X \backslash B}$を
				一様に近似する単関数列$(f_n)_{n=1}^\infty$が存在して
				\begin{align}
					\left| \varphi(f) - \int_X fg\ d\mu \right|
					&\leq \left| \varphi(f) - \varphi(f_n) \right| + \left| \int_X f_ng\ d\mu - \int_X fg\ d\mu \right| \\
					&\leq \Norm{\varphi}{\left( L^p(\mu) \right)^*} \Norm{f - f_n}{L^p(\mu)}
						+ \int_X |f_n - f||g|\ d\mu \\
					&\longrightarrow 0 \quad (n \longrightarrow \infty)
				\end{align}
				となるから(\refeq{eq:thm_dual_space_of_L_p_2})が成立する.
			
			\item[第四段]
				$p = \infty,\ \mu(X) < \infty$の場合,
				$\varphi \in \left( L^p(\mu) \right)^*$に対して
				$\mathscr{F} \ni A \longmapsto \varphi(\defunc_A)$が可算加法的ならば
				(\refeq{eq:thm_dual_space_of_L_p_7})で定める$\lambda$は複素測度となり,
				前段と同じ理由で(\refeq{eq:thm_dual_space_of_L_p_8})を満たす$g \in L^1(\mu)$が存在し
				\begin{align}
					\varphi(f) = \int_X fg\ d\mu,
					\quad (\forall f \in L^\infty(\mu))
				\end{align}
				が成立する.すなわち$\varphi = \Phi_g$であり,
				このとき$f \coloneqq \defunc_{\{g \neq 0\}}\overline{g}/g \in L^\infty(\mu)$
				に対して
				\begin{align}
					\Norm{g}{L^1(\mu)} = \int_X fg\ d\mu = \varphi(f) 
					\leq \Norm{\varphi}{\left( L^\infty(\mu) \right)^*} 
				\end{align}
				となるから,(\refeq{eq:thm_dual_space_of_L_p_3})と併せて
				(\refeq{eq:thm_dual_space_of_L_p_asseretion_2})が満たされる.
				以降は$p < \infty$とする.
				
			\item[第五段]
				$g \in L^q(\mu)$であることを示す.$p = 1$の場合,
				任意の$E \in \mathscr{F}$に対して$f = \defunc_E$とすれば,
				(\refeq{eq:thm_dual_space_of_L_p_2})より
				\begin{align}
					\left| \int_E g\ d\mu \right| = \left| \varphi(\defunc_E) \right|
					\leq \Norm{\varphi}{\left( L^p(\mu) \right)^*} \mu(E)
				\end{align}
				が成立し
				\begin{align}
					\Norm{g}{L^q(\mu)} \leq \Norm{\varphi}{\left( L^p(\mu) \right)^*}
					\label{eq:thm_dual_space_of_L_p_4}
				\end{align}
				が従う.$1 < p < \infty$の場合は
				$\alpha \coloneqq \defunc_{\{g \neq 0\}}\overline{g}/g$と
				\begin{align}
					E_n \coloneqq \Set{x \in X}{|g(x)| \leq n},
					\quad (n=1,2,\cdots)
				\end{align}
				に対して$f \coloneqq \defunc_{E_n} |g|^{q-1} \alpha$とおけば,
				\begin{align}
					fg = \defunc_{E_n} |g|^q = |f|^p
				\end{align}
				が成り立ち$|f|^p \in L^\infty(\mu)$となるから(\refeq{eq:thm_dual_space_of_L_p_2})より
				\begin{align}
					\int_X \defunc_{E_n} |g|^q\ d\mu
					= \int_X fg\ d\mu
					= \varphi(f)
					\leq \Norm{\varphi}{\left( L^p(\mu) \right)^*} \Norm{f}{L^p(\mu)}
					= \Norm{\varphi}{\left( L^p(\mu) \right)^*} \left\{ \int_X \defunc_{E_n} |g|^q\ d\mu \right\}^{1/p}
				\end{align}
				が従い
				\begin{align}
					\left\{ \int_X \defunc_{E_n} |g|^q\ d\mu \right\}^{1/q} \leq \Norm{\varphi}{\left( L^p(\mu) \right)^*}
				\end{align}
				が得られ,単調収束定理より
				\begin{align}
					\Norm{g}{L^q(\mu)} \leq \Norm{\varphi}{\left( L^p(\mu) \right)^*}
					\label{eq:thm_dual_space_of_L_p_5}
				\end{align}
				が出る.
				
			\item[第六段]
				任意の$f \in L^p(\mu)$に対して,単関数近似列$(f_n)_{n=1}^\infty$は(\refeq{eq:thm_dual_space_of_L_p_2})を満たすから,
				H\Ddot{o}lderの不等式とLebesgueの収束定理より
				\begin{align}
					\left| \varphi(f) - \int_X fg\ d\mu \right|
					&\leq \left| \varphi(f) - \varphi(f_n) \right| + \left| \int_X f_ng\ d\mu - \int_X fg\ d\mu \right| \\
					&\leq \Norm{\varphi}{\left( L^p(\mu) \right)^*} \Norm{f - f_n}{L^p(\mu)}
						+ \Norm{f - f_n}{L^p(\mu)}\Norm{g}{L^q(\mu)} \\
					&\longrightarrow 0 \quad (n \longrightarrow \infty)
				\end{align}
				となり
				\begin{align}
					\varphi = \Phi(g)
				\end{align}
				が成り立つ.また,このとき(\refeq{eq:thm_dual_space_of_L_p_3})と(\refeq{eq:thm_dual_space_of_L_p_4})或は
				(\refeq{eq:thm_dual_space_of_L_p_5})より
				\begin{align}
					\Norm{g}{L^q(\mu)} = \Norm{\varphi}{\left( L^p(\mu) \right)^*}
				\end{align}
				が満たされる.
				
			\item[第七段]
				$\mu(X) = \infty$の場合,補題\ref{lem:Lebesgue_Radon_Nikodym}の関数$w$を用いて
				\begin{align}
					\tilde{\mu}(E) \coloneqq \int_E w\ d\mu,
					\quad (\forall E \in \mathscr{F})
				\end{align}
				により有限測度$\tilde{\mu}$を定める.このとき
				任意の$f \in L^p(\mu)$に対して
				\begin{align}
					F \coloneqq w^{-1/p} f
				\end{align}
				とおけば
				\begin{align}
					\int_X |F|^p\ d\tilde{\mu} = \int_X |F|^p w\ d\mu = \int_X |f|^p\ d\mu
					\label{eq:thm_dual_space_of_L_p_6}
				\end{align}
				が成立し,
				\begin{align}
					L^p \ni f \longmapsto w^{-1/p} f \in L^p(\tilde{\mu})
				\end{align}
				は等長な線型同型となる.ここで任意の$\varphi \in \left( L^p(\mu) \right)^*$に対して
				\begin{align}
					\Psi(F) \coloneqq \varphi\left( w^{1/p} F \right),
					\quad (\forall F \in L^p(\tilde{\mu}))
				\end{align}
				で線形作用素$\Psi$を定めれば
				\begin{align}
					\left| \Psi(F) \right| = \left| \varphi\left( w^{1/p} F \right) \right|
					\leq \Norm{\varphi}{\left( L^p(\mu) \right)^*}\Norm{w^{1/p} F}{L^p(\mu)}
					= \Norm{\varphi}{\left( L^p(\mu) \right)^*}\Norm{F}{L^p(\tilde{\mu})}
				\end{align}
				より$\Psi \in \left( L^p(\tilde{\mu}) \right)^*$が満たされ,かつ
				任意の$f \in L^p(\mu)$に対して
				\begin{align}
					\left| \varphi(f) \right| = \left| \Psi\left( w^{-1/p} f \right) \right|
					\leq \Norm{\Psi}{\left( L^p(\mu) \right)^*}\Norm{w^{-1/p} f}{L^p(\tilde{\mu})}
					= \Norm{\Psi}{\left( L^p(\tilde{\mu}) \right)^*}\Norm{f}{L^p(\mu)}
				\end{align}
				も成り立ち
				\begin{align}
					\Norm{\varphi}{\left( L^p(\mu) \right)^*} = \Norm{\Psi}{\left( L^p(\tilde{\mu}) \right)^*}
				\end{align}
				が得られる.前段までの結果より$\Psi$に対し或る$G \in L^q(\tilde{\mu})$が存在して
				\begin{align}
					\Psi(F) = \int_X FG\ d\tilde{\mu}
				\end{align}
				が成立するから,任意の$f \in L^p(\mu)$に対して
				\begin{align}
					\varphi(f) = \Psi\left( w^{-1/p} f \right)
					= \int_X w^{-1/p} f G w\ d\mu
					= \begin{cases}
						\displaystyle\int_X f G\ d\mu, & (p = 1), \\
						\displaystyle\int_X f w^{1/q} G\ d\mu, & (1 < p < \infty)
					\end{cases}
				\end{align}
				が従い,
				\begin{align}
					g \coloneqq
					\begin{cases}
						G, & (p = 1), \\
						w^{1/q} G, & (1 < p < \infty)
					\end{cases}
				\end{align}
				とおけば(\refeq{eq:thm_dual_space_of_L_p_6})より$g \in L^q(\mu)$となり,
				$\varphi = \Phi(g)$かつ
				\begin{align}
					\Norm{\varphi}{\left( L^p(\mu) \right)^*} = \Norm{\Psi}{\left( L^p(\tilde{\mu}) \right)^*}
					= \Norm{G}{L^q(\tilde{\mu})}
					= \Norm{g}{L^q(\mu)}
				\end{align}
				が満たされる.
				\QED
		\end{description}
	\end{prf}
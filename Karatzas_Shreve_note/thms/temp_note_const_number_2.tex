\subsubsection{実数}
	実数体の構成はなかなかうまくいかない.
	Artin-Schreier理論によれば順序体には実閉包が存在し,特に有理数体の実閉包が実数体として定められる.
	実閉包の存在の証明にはZornの補題が使われる.
	他方でDedeind切断による実数の構成は選択公理を使わないので,
	この方法で実数体を構成すれば複素数体の構成までは選択公理なしで記述できる.
	しかしDedeind切断による方法は,厚顔無恥な言い方をすれば泥臭い.
	しかしZornの補題はまだ使いたくない.
	Artin-Schreierの定理は任意の順序体に対しての実閉包の存在を主張しているが,
	例えばArchimedes的順序体の実閉包の存在はZornの補題なしで,華麗に証明できるのか?見通しが立たない.
	
	もう一つ問題がある.実閉体がleast upper bound propertyを満たすかどうかがまだわからない. 
	いかなる実閉体もleast upper bound propertyを満たすのか,実閉体がArchimedes的ならばleast upper bound propertyを満たすのか,
	どういう状況でどう証明すれば良いのかまだ把握していない.
	
\subsubsection{複素数}
	実数が構成できたとすれば,複素数体は$\R$を単純拡大して得られる.
	そうして得られた複素数は
	\begin{align}
		\alpha + \beta i
	\end{align}
	なる形で一意に表される.単純拡大とは別に$\R \times \R$に適当な算法を導入して複素数体(に同型な体)を構成する方法もあるが,
	単純拡大の方がエレガントに感じられる.

\newpage
\subsection{実数}
	実数体の構成はなかなかうまくいかない.
	Artin-Schreier理論によれば順序体には実閉包が存在し,特に有理数体の実閉包が実数体として定められる.
	実閉包の存在の証明にはZornの補題が使われる.
	他方でDedeind切断による実数の構成は選択公理を使わないので,
	この方法で実数体を構成すれば複素数体の構成までは選択公理なしで記述できる.
	しかしDedeind切断による方法は,厚顔無恥な言い方をすれば泥臭い.
	しかしZornの補題はまだ使いたくない.
	Artin-Schreierの定理は任意の順序体に対しての実閉包の存在を主張しているが,
	例えばArchimedes的順序体の実閉包の存在はZornの補題なしで,華麗に証明できるのか?見通しが立たない.
	
	もう一つ問題がある.実閉体がleast upper bound propertyを満たすかどうかがまだわからない. 
	いかなる実閉体もleast upper bound propertyを満たすのか,実閉体がArchimedes的ならばleast upper bound propertyを満たすのか,
	どういう状況でどう証明すれば良いのかまだ把握していない.
	
	\begin{itembox}[l]{実閉体メモ}
		体が実体であることと順序付け可能であることは同値.
		
		体$F$について,$F$が実閉体であること,$F$のいかなる要素$x$にも$\sqrt{x}$か$\sqrt{-x}$が存在してかつ奇数次の多項式が$F$で解を持つこと,
		平方の全体を正の要素として順序付け可能であること,
		$F$が実体で$F(\sqrt{-1})$が代数閉体であること,
		は同値.
	\end{itembox}

\subsection{多項式環}
	$(R,\sigma_R,\mu_R)$を可換環とする.また$\zeta_R$を$R$の零元とし,
	$\epsilon_R$を$R$の単位元とする.
	$P$を$\Natural$上の$R$-値写像で有限個の自然数を除いて
	$\zeta_R$に張り付くものの全体とする.つまり$P$は
	\begin{align}
		P \defeq \Set{f}{f:\Natural \longrightarrow R \wedge
		\exists n \in \Natural\, \forall m \in \Natural\,
		(\, n < m \Longrightarrow f(m) = \zeta_R\, )}
	\end{align}
	で定められる.$f$を$P$の要素とするとき
	\begin{align}
		\forall m \in \Natural\, (\, n < m \Longrightarrow f(m) = \zeta_R\, )
	\end{align}
	を満たす最小の自然数$n$が取れるが,その自然数を$f$の次数と呼び
	\begin{align}
		\deg_{(R,\sigma_R,\mu_R)}{f}
	\end{align}
	と書く.さらに言えば,$\deg_{(R,\sigma_R,\mu_R)}$とは$f$に対して
	\begin{align}
		\mu n\, \left[\, \forall m \in \Natural\, (\, n < m \Longrightarrow f(m) = \zeta_R\, )\, \right]
	\end{align}
	を対応させる写像である.$f$と$g$を$P$の要素とするとき
	\begin{align}
		\Natural \ni n \longmapsto \sigma_R(f(n),g(n))
	\end{align}
	なる写像を対応させる関係を$P$の加法として定め,
	\begin{align}
		\Natural \ni n \longmapsto \sum_{i \in n+1} \mu_R(f(i),g(n-i))
	\end{align}
	なる写像を対応させる関係を$P$の乗法として定める.$P$の加法を
	\begin{align}
		\sigma_P
	\end{align}
	と書き,$P$の乗法を
	\begin{align}
		\mu_P
	\end{align}
	と書く.$P$の特別な要素として,
	\begin{align}
		\Natural \ni n \longmapsto
		\begin{cases}
			\epsilon_R & \mbox{if } n = 1 \\
			\zeta_R & \mbox{if } n \neq 1
		\end{cases}
	\end{align}
	なる写像を$X$と定める.このとき自然数$i$に対して$X^i$は
	\begin{align}
		\Natural \ni n \longmapsto
		\begin{cases}
			\epsilon_R & \mbox{if } n = i \\
			\zeta_R & \mbox{if } n \neq i
		\end{cases}
	\end{align}
	なる写像であって,さらに
	\begin{align}
		\Natural \ni n \defeq \deg_{(R,\sigma_R,\mu_R)}{f}
	\end{align}
	とおけば
	\begin{align}
		f = \sum_{i \in n} \mu_P \left(\varphi(f(i)),X^i\right)
	\end{align}
	が成立する.
	
	ここで$R$の要素$r$に対して
	\begin{align}
		\Natural \ni n \longmapsto
		\begin{cases}
			r & \mbox{if } n = 0 \\
			\zeta_R & \mbox{if } n \neq 0
		\end{cases}
	\end{align}
	なる写像を対応させる関係を$\varphi$とすると,$\varphi$は$R$から$P$への埋め込みとなる.そして
	\begin{align}
		R[X] \defeq \left(P \backslash \varphi \ast R\right) \cup R
	\end{align}
	と定める.
	\begin{align}
		P \ni f \longmapsto
		\begin{cases}
			\varphi^{-1}(f) & \mbox{if } f \in \varphi \ast R \\
			f & \mbox{if } f \notin \varphi \ast R
		\end{cases}
	\end{align}
	なる写像を$h$とすると
	\begin{align}
		h:P \bij R[X]
	\end{align}
	が成立して,この$h$によって$P$の算法を$R[X]$に移すことが出来る.つまり,
	$f$と$g$を$R[X]$の要素とするとき
	\begin{align}
		h\left( \sigma_P\left(h^{-1}(f),h^{-1}(g)\right) \right)
	\end{align}
	なる要素を対応させる関係を$R[X]$の加法として定め,
	\begin{align}
		h\left( \mu_P\left(h^{-1}(f),h^{-1}(g)\right) \right)
	\end{align}
	なる要素を対応させる関係を$R[X]$の乗法として定める.$R[X]$の加法を
	\begin{align}
		\sigma_X
	\end{align}
	と書き,$R[X]$の乗法を
	\begin{align}
		\mu_X
	\end{align}
	と書くことにする.すると$h$は$(P,\sigma_P,\mu_P)$から$(R[X],\sigma_X,\mu_X)$への環同型となり,しかも
	\begin{align}
		R \subset R[X]
	\end{align}
	かつ
	\begin{align}
		\sigma_R \subset \sigma_X
	\end{align}
	かつ
	\begin{align}
		\mu_R \subset \mu_X
	\end{align}
	が満たされる.
	
	$f$を$P$の要素とすると,適当な自然数$n$を取って
	\begin{align}
		f = \sum_{i \in n} \mu_P \left(\varphi(f(i)),X^i\right)
	\end{align}
	と書けたわけであるが,このとき
	\begin{align}
		h(f) = \sum_{i \in n} \mu_X \left(f(i),X^i\right)
	\end{align}
	が成立する.
	
	$(R,\sigma_R,\mu_R)$が整域ならば$(R[X],\sigma_X,\mu_X)$もまた整域である.

\subsection{単拡大}
	前小節の記号を継承する.$(R,\sigma_R,\mu_R)$が体であるとき,
	$(R[X],\sigma_X,\mu_X)$はEuclid整域となる.従って,
	$q$を$R[X]$の既約多項式とし,$q$の生成するイデアルを
	\begin{align}
		(q)
	\end{align}
	と書くことにすると,商環
	\begin{align}
		E \defeq R[X]/(q)
	\end{align}
	は体となる.ただし商環の算法を
	\begin{align}
		\sigma_E
	\end{align}
	及び
	\begin{align}
		\mu_E
	\end{align}
	と書く.そして$\psi$を$R[X]$から$E$への商写像とすると,
	$\psi$によって$R$は$E$に埋め込まれる.
	\begin{align}
		F \defeq \left(E \backslash (\psi \ast R)\right) \cup R
	\end{align}
	とおいて,
	\begin{align}
		E \ni x \longmapsto
		\begin{cases}
			\psi^{-1}(x) & \mbox{if } x \in \psi \ast R \\
			x & \mbox{if } x \notin \psi \ast R
		\end{cases}
	\end{align}
	なる写像を$J$とすると
	\begin{align}
		J:E \bij F
	\end{align}
	が成立して,この$J$によって$E$の算法を$F$に移すことが出来る.つまり,
	$x$と$y$を$E$の要素とするとき
	\begin{align}
		J\left( \sigma_E\left(h^{-1}(x),h^{-1}(y)\right) \right)
	\end{align}
	なる要素を対応させる関係を$F$の加法として定め,
	\begin{align}
		h\left( \mu_E\left(h^{-1}(x),h^{-1}(y)\right) \right)
	\end{align}
	なる要素を対応させる関係を$F$の乗法として定める.$F$の加法を
	\begin{align}
		\sigma_F
	\end{align}
	と書き,$F$の乗法を
	\begin{align}
		\mu_F
	\end{align}
	と書くことにする.すると$h$は$(E,\sigma_E,\mu_E)$から$(F,\sigma_F,\mu_F)$への体同型となり,しかも
	\begin{align}
		R \subset F
	\end{align}
	かつ
	\begin{align}
		\sigma_R \subset \sigma_F
	\end{align}
	かつ
	\begin{align}
		\mu_R \subset \mu_F
	\end{align}
	が満たされる.
	
	ここで$q$は,適当な自然数$n$を取って
	\begin{align}
		\sum_{i \in n} \mu_X \left(q(i),X^i\right)
	\end{align}
	と表される.
	\begin{align}
		\alpha \defeq \psi(X)
	\end{align}
	とおくと,
	\begin{align}
	\end{align}
	
\subsection{複素数}
	複素数体は$(\R,+_\R,\cdot_\R)$を単拡大して得られる.
	$(\R[X],+_X,\cdot_X)$を$\R$上の多項式環とすれば
	\begin{align}
		1 +_X X^2
	\end{align}
	(中置記法)は既約多項式となり,単拡大の際にはこれを用いる.$\C$の要素でこの多項式の解となるものを
	\begin{align}
		\imag
	\end{align}
	と書くと,つまり$\imag$とは
	\begin{align}
		0 = 1 +_\C \imag^2
	\end{align}
	を満たす$\C$の要素であるが,複素数は全て
	\begin{align}
		x +_\C (y \cdot_\C \imag)
	\end{align}
	なる形で一意に表される.これでは若干見づらいので,
	$+_\C$と$\cdot_\C$を順序数のそれと同じ記号に直して
	\begin{align}
		x + y \cdot \imag
	\end{align}
	と表記する.

\newpage
\subsection{複素数の代数的性質}

	複素数の全体を
	\begin{align}
		\C
	\end{align}
	と書く.また実数の全体を
	\begin{align}
		\R
	\end{align}
	と書く.$\C \times \C$から$\C$への写像を$\C$上の{\bf 算法}{\bf (operation)}と呼ぶ.
	
	\begin{itembox}[l]{$\C$の加法}
		$\C$上には{\bf 加法}と呼ばれる算法$+$が定まっていて,加法は次の性質を満たす:
		\begin{description}
			\item[結合律] $\alpha$と$\beta$と$\gamma$を複素数とするとき
				\begin{align}
					+\left(+(\alpha,\beta),\gamma\right) = +\left(\alpha,+(\beta,\gamma)\right).
				\end{align}
				
			\item[可換律]  $\alpha$と$\beta$を複素数とするとき
				\begin{align}
					+(\alpha,\beta) = +(\beta,\alpha).
				\end{align}
				
			\item[加法に関する逆元の存在] $\alpha$を複素数とするとき
				\begin{align}
					+(\alpha,\beta) = 0
				\end{align}
				を満たす複素数$\beta$が取れる.この$\beta$を$\alpha$の加法に関する{\bf 逆元}と呼ぶ.
				
			\item[$0$は加法の単位元] $\alpha$を複素数とするとき
				\begin{align}
					+(\alpha,0) = \alpha.
				\end{align}
		\end{description}
	\end{itembox}
	
	$\alpha$と$\beta$を複素数とするとき,
	\begin{align}
		+(\alpha,\beta)
	\end{align}
	を$\alpha$と$\beta$の{\bf 和}{\bf (sum)}と呼び
	\begin{align}
		\alpha + \beta
	\end{align}
	と書く.これは{\bf 中置記法}と呼ばれる.中置記法によって先の規則を書き直すと
	\begin{description}
			\item[結合律] $\alpha$と$\beta$と$\gamma$を複素数とするとき
				\begin{align}
					(\alpha+\beta)+\gamma = \alpha+(\beta+\gamma).
				\end{align}
				
			\item[可換律]  $\alpha$と$\beta$を複素数とするとき
				\begin{align}
					\alpha + \beta = \beta + \alpha.
				\end{align}
				
			\item[逆元の存在] $\alpha$を複素数とするとき
				\begin{align}
					\alpha + \beta = 0
				\end{align}
				を満たす複素数$\beta$が取れる.
				
			\item[$0$は単位元] $\alpha$を複素数とするとき
				\begin{align}
					\alpha + 0 = \alpha.
				\end{align}
	\end{description}
	となる.
	
	\begin{screen}
		\begin{thm}[加法に関する逆元はただ一つ]
			複素数$\alpha$に対して
			\begin{align}
				\alpha + \beta = 0
			\end{align}
			を満たす複素数$\beta$はただ一つである.
		\end{thm}
	\end{screen}
	
	つまり{\bf $\alpha$の逆元はただ一つであるということである.}
	
	\begin{sketch}
		複素数$\beta$と$\gamma$に対して
		\begin{align}
			\alpha + \beta = 0
		\end{align}
		かつ
		\begin{align}
			\alpha + \gamma = 0
		\end{align}
		が成り立っているとする.このとき
		\begin{align}
			\gamma &= \gamma + 0 &&\mbox{$0$が単位元であるから} \\
			&= \gamma + (\alpha + \beta) &&\mbox{$\alpha + \beta = 0$} \\
			&= (\gamma + \alpha) + \beta &&\mbox{結合律} \\
			&= 0 + \beta &&\mbox{$\alpha + \gamma = 0$と可換律} \\
			&= \beta
		\end{align}
		が成り立つ.
		\QED
	\end{sketch}
	
	$\alpha$を複素数とするとき,その加法に関する逆元を
	\begin{align}
		- \alpha
	\end{align}
	と書く.また
	\begin{align}
		a + (-b)
	\end{align}
	なる複素数を
	\begin{align}
		a - b
	\end{align}
	とも略記する.
	
	\begin{screen}
		\begin{thm}[和の逆元は逆元の和]
		\label{thm:inverse_of_sum}
			$a$と$b$を複素数とするとき
			\begin{align}
				-(a + b) = (-b) + (-a).
			\end{align}
		\end{thm}
	\end{screen}
	
	\begin{sketch}
		$a$と$b$を複素数とするとき,
		\begin{align}
			(a+b) + ((-b) + (-a))
			&= (a + (b + (-b))) + (-a) \\
			&= (a + 0) + (-a) \\
			&= a + (-a) \\
			&= 0 
		\end{align}
		かつ
		\begin{align}
			((-b) + (-a)) + (a+b)
			&= ((-b) + ((-a) + a)) + b \\
			&= ((-b) + 0) + b \\
			&= (-b) + b \\
			&= 0
		\end{align}
		が成り立つので
		\begin{align}
			-(a + b) = (-b) + (-a)
		\end{align}
		が得られた.
		\QED
	\end{sketch}
	
	\begin{itembox}[l]{$\C$の乗法}
		$\C$上には{\bf 乗法}と呼ばれる算法$\cdot$が定まっていて,乗法は次の性質を満たす:
		\begin{description}
			\item[結合律] $\alpha$と$\beta$と$\gamma$を複素数とするとき
				\begin{align}
					\cdot\ \left(\cdot\ (\alpha,\beta),\gamma\right) = \cdot\ \left(\alpha,\cdot\ (\beta,\gamma)\right).
				\end{align}
				
			\item[可換律]  $\alpha$と$\beta$を複素数とするとき
				\begin{align}
					\cdot\ (\alpha,\beta) = \cdot\ (\beta,\alpha).
				\end{align}
				
			\item[乗法に関する逆元の存在] $\alpha$を複素数とするとき,$\alpha \neq 0$ならば
				\begin{align}
					\cdot\ (\alpha,\beta) = 1
				\end{align}
				を満たす複素数$\beta$が取れる.この$\beta$を$\alpha$の乗法に関する逆元と呼ぶ.
				
			\item[$1$は乗法の単位元] $\alpha$を複素数とするとき
				\begin{align}
					\cdot\ (\alpha,1) = \alpha.
				\end{align}
				
			\item[左分配律] $\alpha$と$\beta$と$\gamma$を複素数とするとき
				\begin{align}
					\cdot\ \left(\alpha,(\beta + \gamma)\right) = \cdot\ (\alpha,\beta) + \cdot\ (\alpha,\gamma).
				\end{align}
				
			\item[右分配律] $\alpha$と$\beta$と$\gamma$を複素数とするとき
				\begin{align}
					\cdot\ \left((\alpha + \beta),\gamma\right) = \cdot\ (\alpha,\gamma) + \cdot\ (\beta,\gamma).
				\end{align}
		\end{description}
	\end{itembox}
	
	$\alpha$と$\beta$を複素数とするとき,
	\begin{align}
		\cdot\ (\alpha,\beta)
	\end{align}
	を$\alpha$と$\beta$の{\bf 積}{\bf (product)}と呼び
	\begin{align}
		\alpha \cdot \beta
	\end{align}
	と書く.これも中置記法である.中置記法によって先の規則を書き直すと
	\begin{description}
			\item[結合律] $\alpha$と$\beta$と$\gamma$を複素数とするとき
				\begin{align}
					(\alpha \cdot \beta) \cdot \gamma = \alpha \cdot (\beta \cdot \gamma).
				\end{align}
				
			\item[可換律]  $\alpha$と$\beta$を複素数とするとき
				\begin{align}
					\alpha \cdot \beta = \beta \cdot \alpha.
				\end{align}
				
			\item[逆元の存在] $\alpha$を複素数とするとき,$\alpha \neq 0$ならば
				\begin{align}
					\alpha \cdot \beta = 1
				\end{align}
				を満たす複素数$\beta$が取れる.
				
			\item[$1$は単位元] $\alpha$を複素数とするとき
				\begin{align}
					\alpha \cdot 1 = \alpha.
				\end{align}
				
			\item[左分配律] $\alpha$と$\beta$と$\gamma$を複素数とするとき
				\begin{align}
					\alpha \cdot (\beta + \gamma) = \alpha \cdot \beta + \alpha \cdot \gamma.
				\end{align}
				
			\item[右分配律] $\alpha$と$\beta$と$\gamma$を複素数とするとき
				\begin{align}
					(\alpha + \beta) \cdot \gamma = \alpha \cdot \gamma + \beta \cdot \gamma.
				\end{align}
	\end{description}
	となる.また{\bf 演算の順序は括弧内を優先するが,括弧が付いてない場合は乗法優先である.}
	
	常識であるが$\C$には
	\begin{align}
		1 + \isym^2 = 0
	\end{align}
	を満たす複素数
	\begin{align}
		\isym
	\end{align}
	が存在している.これは{\bf 虚数単位}と呼ばれ
	\begin{align}
		\sqrt{-1}
	\end{align}
	とも書かれる.$\isym$は複素数の表示において見慣れたものであるが,$z$を複素数とすれば
	\begin{align}
		z = x + \isym \cdot y
	\end{align}
	を満たす$\R^2$の要素$(x,y)$が唯一つ取れる.さらに言えば
	\begin{align}
		\R^2 \ni (x,y) \longmapsto x + \isym \cdot y \in \C
	\end{align}
	なる関係は$\R^2$から$\C$への全単射であるのだが,このことは複素数を構成する際に導かれる定理であるが,
	ここではその証明まで踏み込むことは出来ない.
	
	複素数$z$が与えられたときに
	\begin{align}
		z = x + \isym \cdot y
	\end{align}
	を満たす$x$を対応させる写像を
	\begin{align}
		\Re
	\end{align}
	と書いて,
	\begin{align}
		\Re{z}
	\end{align}
	のことを$z$の{\bf 実部}{\bf (real part)}と呼ぶ.また$z$に対して$y$を対応させる写像を
	\begin{align}
		\Im
	\end{align}
	と書いて,
	\begin{align}
		\Im{z}
	\end{align}
	のことを$z$の{\bf 虚部}{\bf (imaginary part)}と呼ぶ.
	
	複素数$z$に対して
	\begin{align}
		\Re{z} - \isym \cdot \Im{z}
	\end{align}
	なる複素数を$z$の{\bf 複素共役}\index{ふくそきょうやく@複素共役}{\bf (complex conjugate)}と呼び,これを
	\begin{align}
		\overline{z}
	\end{align}
	と表記する.
	
	ところで暗黙の裡に$\R$は$\C$の部分集合であるとして扱ってきたが,
	それが成り立つように数を構成していない文献が案外多いので注意が必要である.
	特に$\C = \R^2$と考えるのは不正確で,複素数と実数の順序対には一対一の対応はあるが別物である.
	$\R$は$\R^2$の部分集合ではない.
	
	$(\C,+,\cdot\ )$の組は複素数体と呼ばれる閉じた世界である.
	$\R$も$\C$と同様に加法と乗法で閉じていて,
	つまり$(\R,+,\cdot\ )$も実数体と呼ばれる体であるので,実数同士の演算は必ず実数である.
	
	\begin{screen}
		\begin{thm}[$0$は乗法に関して逆元を持たない]
		\label{thm:zero_multiplication_is_zero}
			任意の複素数$\alpha$に対して
			\begin{align}
				0 \cdot \alpha = 0.
			\end{align}
		\end{thm}
	\end{screen}
	
	\begin{sketch}
		$0 = 0+0$と右分配律から
		\begin{align}
			0 \cdot \alpha = (0+0) \cdot \alpha
			= 0 \cdot \alpha + 0 \cdot \alpha
		\end{align}
		が成り立つ.両辺に$-(0 \cdot \alpha)$と足せば
		\begin{align}
			0 &= -(0 \cdot \alpha) + 0 \cdot \alpha \\
			&= -(0 \cdot \alpha) + (0 \cdot \alpha + 0 \cdot \alpha) \\
			&= (-(0 \cdot \alpha) + 0 \cdot \alpha) + 0 \cdot \alpha \\
			&= 0 + 0 \cdot \alpha \\
			&= 0 \cdot \alpha
		\end{align}
		が成り立つ.
		\QED
	\end{sketch}
	
	\begin{screen}
		\begin{thm}[乗法に関する逆元が取れるなら$0$でない]
		\label{thm:if_has_inverse_wrt_multiplication_then_not_zero}
			$\alpha$を複素数とするとき
			\begin{align}
				\exists \beta \in \C\, (\, \alpha \cdot \beta = 1\, )
				\Longrightarrow \alpha \neq 0.
			\end{align}
		\end{thm}
	\end{screen}
	
	\begin{sketch}
		前定理より
		\begin{align}
			\alpha = 0 \Longrightarrow \forall \beta \in \C\, (\, \alpha \cdot \beta = 0\, ),
		\end{align}
		すなわち
		\begin{align}
			\alpha = 0 \Longrightarrow \forall \beta \in \C\, (\, \alpha \cdot \beta \neq 1\, )
		\end{align}
		が成り立つ.この対偶を取れば
		\begin{align}
			\exists \beta \in \C\, (\, \alpha \cdot \beta = 1\, )
			\Longrightarrow \alpha \neq 0
		\end{align}
		が得られる.
		\QED
	\end{sketch}
	
	\begin{screen}
		\begin{thm}[乗法に関する逆元はただ一つ]
			$0$でない複素数$\alpha$に対して
			\begin{align}
				\alpha \cdot \beta = 0
			\end{align}
			を満たす複素数$\beta$はただ一つである.
		\end{thm}
	\end{screen}
	
	\begin{sketch}
		複素数$\beta$と$\gamma$に対して
		\begin{align}
			\alpha \cdot \beta = 0
		\end{align}
		かつ
		\begin{align}
			\alpha \cdot \gamma = 0
		\end{align}
		が成り立っているとする.このとき
		\begin{align}
			\gamma &= \gamma \cdot 1 &&\mbox{$1$が単位元であるから} \\
			&= \gamma \cdot (\alpha \cdot \beta) &&\mbox{$\alpha \cdot \beta = 1$} \\
			&= (\gamma \cdot \alpha) \cdot \beta &&\mbox{結合律} \\
			&= 1 \cdot \beta &&\mbox{$\alpha \cdot \gamma = 1$と可換律} \\
			&= \beta
		\end{align}
		が成り立つ.
		\QED
	\end{sketch}
	
	$\alpha$を$0$でない複素数とするとき,その乗法に関する逆元を
	\begin{align}
		\alpha^{-1}
	\end{align}
	や
	\begin{align}
		1/\alpha
	\end{align}
	や
	\begin{align}
		\frac{1}{\alpha}
	\end{align}
	と書く.また$\alpha \cdot \beta$の加法に関する逆元を
	\begin{align}
		- \alpha \cdot \beta
	\end{align}
	と書く.紛らわしいが,これと$(-\alpha) \cdot \beta$とは結果としては同じ数であるが導入は別である.
	
	\begin{screen}
		\begin{thm}[積の逆元は逆元の積]
		\label{thm:inverse_of_product}
			$a$と$b$を複素数とするとき
			\begin{align}
				-a \cdot b = (-a) \cdot b = a \cdot (-b).
			\end{align}
		\end{thm}
	\end{screen}
	
	\begin{sketch}
		$a$と$b$を複素数とするとき,
		\begin{align}
			a \cdot b + (-a) \cdot b
			&= (a + (-a)) \cdot b \\
			&= 0 \cdot b \\
			&= 0
		\end{align}
		かつ
		\begin{align}
			(-a) \cdot b + a \cdot b
			&= ((-a) + a) \cdot b \\
			&= 0 \cdot b \\
			&= 0
		\end{align}
		が成り立つので
		\begin{align}
			-(a \cdot b) = (-a) \cdot b
		\end{align}
		が得られた.積の可換律より
		\begin{align}
			-a \cdot b = -(b \cdot a) = (-b) \cdot a = a \cdot (-b)
		\end{align}
		が得られた.
		\QED
	\end{sketch}
	
	\begin{screen}
		\begin{thm}[$0$でない数同士の積は乗法に関して逆元を持つ]
		\label{thm:C_has_no_zero_divisor}
			$a$と$b$を$0$でない複素数とするとき,$a \cdot b$も$0$ではない.またこのとき
			\begin{align}
				(a \cdot b)^{-1} = b^{-1} \cdot a^{-1}.
			\end{align}
		\end{thm}
	\end{screen}
	
	\begin{sketch}
		$a$と$b$を$0$でない複素数とするとき,
		\begin{align}
			(a \cdot b) \cdot (b^{-1} \cdot a^{-1})
			&= \left[a \cdot (b \cdot b^{-1})\right] \cdot a^{-1} \\
			&= (a \cdot 1) \cdot a^{-1} \\
			&= a \cdot a^{-1} \\
			&= 1
		\end{align}
		かつ
		\begin{align}
			(b^{-1} \cdot a^{-1}) \cdot (a \cdot b) 
			&= \left[b^{-1} \cdot (a^{-1} \cdot a)\right] \cdot b \\
			&= (b^{-1} \cdot 1) \cdot b \\
			&= b^{-1} \cdot b \\
			&= 1
		\end{align}
		が成り立つので
		\begin{align}
			(a \cdot b)^{-1} = b^{-1} \cdot a^{-1}
		\end{align}
		が得られた.また定理\ref{thm:if_has_inverse_wrt_multiplication_then_not_zero}から
		\begin{align}
			a \cdot b \neq 0
		\end{align}
		も従う.
		\QED
	\end{sketch}
	
	\begin{screen}
		\begin{thm}[積の共役は共役の積]
		\label{thm:conjugate_of_product_is_product_of_conjugates}
			$a$と$b$を複素数とするとき,
			\begin{align}
				\overline{a \cdot b} = \overline{a} \cdot \overline{b}.
			\end{align}
		\end{thm}
	\end{screen}
	
	\begin{sketch}
		$\alpha,\beta,\gamma,\delta$を実数とするとき,
		\begin{align}
			(\alpha + \isym \cdot \beta) \cdot (\gamma + \isym \cdot \delta)
			&= (\alpha + \isym \cdot \beta) \cdot \gamma
			+ (\alpha + \isym \cdot \beta) \cdot (\isym \cdot \delta) \\
			&= \left[\alpha \cdot \gamma + \isym \cdot (\beta \cdot \gamma) \right]
			+ \left[\isym \cdot (\alpha \cdot \delta) + (-1) \cdot (\beta \cdot \delta)\right] \\
			&= \left[\alpha \cdot \gamma + (-1) \cdot (\beta \cdot \delta)\right]
			+ \left[\isym \cdot (\beta \cdot \gamma) + \isym \cdot (\alpha \cdot \delta)\right] \\
			&= (\alpha \cdot \gamma - \beta \cdot \delta) + \isym \cdot (\beta \cdot \gamma + \alpha \cdot \delta)
		\end{align}
		が成立する.よって
		\begin{align}
			a \cdot b 
			&= (\Re{a} + \isym \cdot \Im{a}) \cdot (\Re{b} + \isym \cdot \Im{b}) \\
			&= (\Re{a} \cdot \Re{b} - \Im{a} \cdot \Im{b}) 
			+ \isym \cdot (\Im{a} \cdot \Re{b} + \Re{a} \cdot \Im{b})
		\end{align}
		および
		\begin{align}
			\overline{a} \cdot \overline{b}
			&= (\Re{a} - \isym \cdot \Im{a}) \cdot (\Re{b} - \isym \cdot \Im{b}) \\
			&= (\Re{a} + \isym \cdot (-\Im{a})) \cdot (\Re{b} + \isym \cdot (-\Im{b})) \\
			&= (\Re{a} \cdot \Re{b} - \Im{a} \cdot \Im{b}) 
			+ \isym \cdot \left[(-\Im{a}) \cdot \Re{b} + \Re{a} \cdot (-\Im{b})\right] \\
			&= (\Re{a} \cdot \Re{b} - \Im{a} \cdot \Im{b}) 
			+ \isym \cdot \left[-(\Im{a} \cdot \Re{b} + \Re{a} \cdot \Im{b})\right] \\
			&= (\Re{a} \cdot \Re{b} - \Im{a} \cdot \Im{b}) 
			- \isym \cdot (\Im{a} \cdot \Re{b} + \Re{a} \cdot \Im{b})
		\end{align}
		が成立する.よって
		\begin{align}
			\overline{a \cdot b} = \overline{a} \cdot \overline{b}
		\end{align}
		である.
		\QED
	\end{sketch}
	
	$z$を複素数とするとき,$z$の整数乗を定義する.まずは
	\begin{align}
		z^0 \defeq 1
	\end{align}
	と定め,次に
	\begin{align}
		z^1 \defeq z
	\end{align}
	と定め,次に
	\begin{align}
		z^2 \defeq z \cdot z
	\end{align}
	と定め,次も同様に
	\begin{align}
		z^3 \defeq (z \cdot z) \cdot z = z^2 \cdot z
	\end{align}
	と定める.この調子で
	\begin{align}
		z^4 &\defeq z^3 \cdot z \\
		z^5 &\defeq z^4 \cdot z \\
		z^6 &\defeq z^5 \cdot z \\
	\end{align}
	と定めていくと,任意の自然数$n$に対して
	\begin{align}
		z^n
	\end{align}
	なる複素数が得られる.累乗の厳密な定義は帰納法を使った再帰的手法を用いるがここでは直感的な導入で終える.
	ただし一つ言っておくが,$n$を自然数とするとき
	\begin{align}
		z^{n+1} = z^n \cdot z
	\end{align}
	は定理である.
	
	$n$を負の整数とするとき,$-n$は自然数であるから
	\begin{align}
		z^{-n}
	\end{align}
	なる複素数は既に得られている.そこで
	\begin{align}
		z^n \defeq (z^{-n})^{-1}
		\label{fom:def_of_integer_exponentiation}
	\end{align}
	により$z$の$n$乗を定める.
	
	\begin{screen}
		\begin{thm}[積の累乗は累乗の積]
		\label{thm:law_of_exponentiation_for_multiplication}
			$a$と$b$を複素数とし,$n$を整数とするとき,
			\begin{align}
				(a \cdot b)^n = a^n \cdot b^n.
			\end{align}
		\end{thm}
	\end{screen}
	
	\begin{sketch}
		まず
		\begin{align}
			1 = (a \cdot b)^0 = a^0 = b^0
		\end{align}
		より
		\begin{align}
			(a \cdot b)^0 = a^0 \cdot b^0
		\end{align}
		が成り立つ.また$n$を自然数とするとき
		\begin{align}
			(a \cdot b)^n = a^n \cdot b^n
		\end{align}
		が成り立っているとすると,
		\begin{align}
			(a \cdot b)^{n+1}
			&= (a \cdot b)^n \cdot (a \cdot b) \\
			&= (a^n \cdot b^n) \cdot (a \cdot b) \\
			&= (a^n \cdot b^n) \cdot (b \cdot a) \\
			&= \left[a^n \cdot (b^n \cdot b)\right] \cdot a \\
			&= (a^n \cdot b^{n+1}) \cdot a \\
			&= (b^{n+1} \cdot a^n) \cdot a \\
			&= b^{n+1} \cdot (a^n \cdot a) \\
			&= b^{n+1} \cdot a^{n+1} \\
			&= a^{n+1} \cdot b^{n+1}
		\end{align}
		が成立する.ゆえに数学的帰納法の原理より任意の自然数$n$で
		\begin{align}
			(a \cdot b)^n = a^n \cdot b^n
		\end{align}
		が成立する.$n$を負の整数とするとき,
		\begin{align}
			(a \cdot b)^{-n} = a^{-n} \cdot b^{-n}
		\end{align}
		が成り立つから
		\begin{align}
			(a \cdot b)^n &= ((a \cdot b)^n)^{-1} \\
			&= (a^{-n} \cdot b^{-n})^{-1} \\
			&= (b^{-n})^{-1} \cdot (a^{-n})^{-1} \\
			&= b^n \cdot a^n \\
			&= a^n \cdot b^n
		\end{align}
		が従う.
		\QED
	\end{sketch}
	
	\begin{screen}
		\begin{thm}[指数法則(指数が和の場合)]
		\label{thm:exponential_law_of_complex_numbers}
			$z$を複素数とし,$n$と$m$を整数とするとき,
			\begin{align}
				z^{n+m} = z^n \cdot z^m.
			\end{align}
		\end{thm}
	\end{screen}
	
	\begin{sketch}\mbox{}
		\begin{description}
			\item[第一段]
				$n$を整数とするとき
				\begin{align}
					z^{n} \cdot z^{-n} = 1
					\label{fom:thm_exponential_law_of_complex_numbers_1}
				\end{align}
				が成り立つ.実際,$n$が$0$なら
				\begin{align}
					z^{0} =  z^{-0} = 1
				\end{align}
				が成り立つ.$n$が正の自然数なら,累乗の定め方より
				\begin{align}
					z^{-n} = (z^{n})^{-1}
				\end{align}
				であるから
				\begin{align}
					z^{n} \cdot z^{-n} = z^{n} \cdot (z^{n})^{-1} = 1
				\end{align}
				が成り立つ.$n$が負の整数なら,累乗の定め方より
				\begin{align}
					z^{n} = (z^{-n})^{-1}
				\end{align}
				であるから
				\begin{align}
					z^{n} \cdot z^{-n} = (z^{-n})^{-1} \cdot z^{-n} = 1
				\end{align}
				が成り立つ.
				
			\item[第二段]
				$n$を整数とするとき
				\begin{align}
					z^{n+1} = z^{n} \cdot z
					\label{fom:thm_exponential_law_of_complex_numbers_2}
				\end{align}
				が成り立つ.$n$が自然数であるときはこれは定理であるが,
				$n$が負の整数であるとき,
				\begin{align}
					n+1 \leq 0
				\end{align}
				であるから
				\begin{align}
					-(n+1) \in \Natural
				\end{align}
				であり,よって
				\begin{align}
					z^{-(n+1) + 1} = z^{-(n+1)} \cdot z
				\end{align}
				が成立する.ここで
				\begin{align}
					z^{-(n+1) + 1} = z^{(-n) + ((-1) + 1)} = z^{-n}
				\end{align}
				より
				\begin{align}
					z^{-n} = z^{-(n+1)} \cdot z
				\end{align}
				が成立し,両辺に$z^{n} \cdot z^{n+1}$を掛ければ前段の結果より
				\begin{align}
					z^{n+1} &= z^{n+1} \cdot (z^{n} \cdot z^{-n}) \\
					&= (z^{n+1} \cdot z^{n}) \cdot z^{-n} \\
					&= (z^{n} \cdot z^{n+1}) \cdot z^{-n} \\
					&= (z^{n} \cdot z^{n+1}) \cdot (z^{-(n+1)} \cdot z) \\
					&= \left[z^{n} \cdot (z^{n+1} \cdot z^{-(n+1)})\right] \cdot z \\
					&= z^{n} \cdot z
				\end{align}
				が従う.
				
			\item[第三段]
				$n$を整数とし,$m$を自然数とするとき,
				\begin{align}
					z^{n+m} = z^{n} \cdot z^{m}
				\end{align}
				が成立する.実際,まず
				\begin{align}
					z^{n+0} = z^{n} = z^{n} \cdot z^{0}
				\end{align}
				が成り立ち,また自然数$m$に対して
				\begin{align}
					z^{n+m} = z^{n} \cdot z^{m}
				\end{align}
				が成り立っているとすると,前段の結果より
				\begin{align}
					z^{n+(m+1)} &= z^{(n+m)+1} \\
					&= z^{n+m} \cdot z \\
					&= (z^{n} \cdot z^{m}) \cdot z \\
					&= z^{n} \cdot (z^{m} \cdot z) \\
					&= z^{n} \cdot z^{m+1}
				\end{align}
				が成立する.ゆえに数学的帰納法の原理より任意の自然数$m$に対して
				\begin{align}
					z^{n+m} = z^{n} \cdot z^{m}
				\end{align}
				が満たされる.
				
			\item[第四段]
				最後に,$n$と$m$を整数とするとき
				\begin{align}
					z^{n+m} = z^{n} \cdot z^{m}
				\end{align}
				が成立することを示す.$m$が負の整数であるとき,前段の結果より
				\begin{align}
					z^{-(n+m)} = z^{(-n) + (-m)} = z^{-n} \cdot z^{-m}
				\end{align}
				が成立する.よって
				\begin{align}
					z^{n+m} &= (z^{-(n+m)})^{-1} \\
					&= (z^{-n} \cdot z^{-m})^{-1} \\
					&= (z^{-n})^{-1} \cdot (z^{-m})^{-1} \\
					&= z^{n} \cdot z^{m}
				\end{align}
				が従う.
				\QED
		\end{description}
	\end{sketch}
	
	特に,$z$を任意に与えられた複素数とし,$n$を任意に与えられた整数とすると,
	\begin{align}
		z^{n} \cdot z^{-n} = z^{0} = 1
	\end{align}
	が成り立つから
	\begin{align}
		z^{-n} = (z^{n})^{-1}
	\end{align}
	ということになる.
	
	\begin{screen}
		\begin{thm}[指数法則(指数が積の場合)]
		\label{thm:exponential_law_of_complex_numbers_2}
			$z$を複素数とし,$n$と$m$を整数とするとき,
			\begin{align}
				z^{n \cdot m} = (z^n)^m.
			\end{align}
		\end{thm}
	\end{screen}
	
	\begin{sketch}
		まず
		\begin{align}
			z^{n \cdot 0} = z^{0} = 1 = (z^n)^0 
		\end{align}
		が成り立つ.また自然数$m$に対して
		\begin{align}
			z^{n \cdot m} = (z^n)^m
		\end{align}
		が成り立っているならば,定理\ref{thm:exponential_law_of_complex_numbers}より
		\begin{align}
			z^{n \cdot (m + 1)} = z^{n \cdot m + n}
			= z^{n \cdot m} \cdot z^n
			= (z^n)^m \cdot z^n
			= (z^n)^{m+1}
		\end{align}
		が従うので,数学的帰納法の原理より任意の自然数$m$で
		\begin{align}
			z^{n \cdot m} = (z^n)^m
		\end{align}
		が成立する.$m$が負の整数であるときは
		\begin{align}
			z^{-n \cdot m} = z^{n \cdot (-m)} = (z^{n})^{-m} = ((z^{n})^{m})^{-1}
		\end{align}
		が成り立つから,
		\begin{align}
			z^{n \cdot m} = (z^{-n \cdot m})^{-1} = (z^{n})^{m}
		\end{align}
		が得られる.
		\QED
	\end{sketch}
	
	\begin{screen}
		\begin{thm}[二乗が等しい数同士は一致するか逆元である]
		\label{thm:numbers_whose_squares_are_coincide_are_same_or_inverse}
			$a$と$b$を複素数とするとき,
			\begin{align}
				a^2 = b^2 \Longleftrightarrow 
				a = b \vee a = -b.
			\end{align}
		\end{thm}
	\end{screen}
	
	\begin{sketch}
		$a$と$b$を複素数とする.ここでまず
		\begin{align}
			(a + b) \cdot (a - b)
			&= (a+b) \cdot a + (a+b) \cdot (-b) \\
			&= (a^2 + b \cdot a) + \left(a \cdot (-b) + b \cdot (-b)\right) \\
			&= (a^2 + b \cdot a) + \left(a \cdot (-b) + b \cdot (-b)\right) \\
			&= \left[a^2 + (b \cdot a + a \cdot (-b)) \right] + b \cdot (-b) \\
			&= \left[a^2 + (b \cdot a - a \cdot b) \right] - b^2 \\
			&= (a^2 + 0) - b^2 \\
			&= a^2 - b^2
		\end{align}
		が成り立つ.よって
		\begin{align}
			a^2 = b^2 \Longleftrightarrow (a + b) \cdot (a - b) = 0
		\end{align}
		が成り立つが,他方で定理\ref{thm:zero_multiplication_is_zero}と
		定理\ref{thm:C_has_no_zero_divisor}から
		\begin{align}
			(a + b) \cdot (a - b) = 0
			\Longleftrightarrow a+b = 0 \vee a - b = 0
			\Longleftrightarrow a = -b \vee a = b
		\end{align}
		も成立するので
		\begin{align}
			a^2 = b^2 \Longleftrightarrow a = -b \vee a = b
		\end{align}
		が従う.
		\QED
	\end{sketch}
	
	次の定理は証明なしに認める.
	\begin{screen}
		\begin{thm}[非負実数には平方根が存在する]
			$\alpha$を$0$以上の実数とするとき,
			\begin{align}
				\alpha = \beta^2
			\end{align}
			を満たす実数$\beta$が取れる.
		\end{thm}
	\end{screen}
	
	$\alpha$を$0$以上の実数とするとき,
	\begin{align}
		\alpha = \beta^2
	\end{align}
	を満たす実数$\beta$を取ると,定理\ref{thm:numbers_whose_squares_are_coincide_are_same_or_inverse}より
	\begin{align}
		(-\beta)^2 = \alpha
	\end{align}
	も成り立つ.
	\begin{align}
		0 \leq \beta \vee 0 \leq -\beta
	\end{align}
	が成り立つので(ここでは未証明だが認める),すなわち
	{\bf 二乗が$\alpha$となる実数として非負であるものを取ることが出来る.}また
	実数$\gamma$が
	\begin{align}
		\gamma^2 = \alpha
	\end{align}
	を満たすなら
	\begin{align}
		\gamma = \beta \vee \gamma = -\beta
	\end{align}
	が成り立つので,二乗が$\alpha$となる実数は$\beta$と$-\beta$に限られる.
	
	\begin{screen}
		\begin{dfn}[平方根]
			$\alpha$を$0$以上の実数とするとき,二乗が$\alpha$となる
			実数を$\alpha$の{\bf 平方根}\index{へいほうこん@平方根}{\bf (square root)}と呼ぶ.
			また,そのような実数のうち非負であるものを
			\begin{align}
				\sqrt{\alpha}
			\end{align}
			と書く.
		\end{dfn}
	\end{screen}
	
	\begin{screen}
		\begin{dfn}[絶対値]
			$z$を複素数とするとき,
			\begin{align}
				\sqrt{(\Re{z})^2 + (\Im{z})^2}
			\end{align}
			なる実数を$z$の{\bf 絶対値}\index{ぜったいち@絶対値}{\bf (absolute value)}と呼び
			\begin{align}
				|z|
			\end{align}
			と書く.
		\end{dfn}
	\end{screen}
	
	\begin{screen}
		\begin{thm}[絶対値の二乗は複素共役との積に等しい]
		\label{thm:square_of_absolute_value_is_product_with_conjugate}
			$z$を複素数とするとき
			\begin{align}
				|z|^2 = z \cdot \overline{z}.
			\end{align}
		\end{thm}
	\end{screen}
	
	\begin{sketch}
		算法の分配律と定理\ref{thm:law_of_exponentiation_for_multiplication}
		および定理\ref{thm:inverse_of_product}を用いれば
		\begin{align}
			z \cdot \overline{z} 
			&= (\Re{z} + \isym \cdot \Im{z}) \cdot (\Re{z} - \isym \cdot \Im{z}) \\
			&= (\Re{z})^2 - (\isym \cdot \Im{z})^2 \\
			&= (\Re{z})^2 - \isym^2 \cdot (\Im{z})^2 \\
			&= (\Re{z})^2 - (-1) \cdot (\Im{z})^2 \\
			&= (\Re{z})^2 + (\Im{z})^2 \\
			&= |z|^2
		\end{align}
		が成立する.
		\QED
	\end{sketch}
	
	\begin{screen}
		\begin{thm}[積の絶対値は絶対値の積]
		\label{thm:absolute_value_of_product_is_product_of_absolute_values}
			$a$と$b$を複素数とするとき
			\begin{align}
				|a \cdot b| = |a| \cdot |b|.
			\end{align}
		\end{thm}
	\end{screen}
	
	\begin{sketch}
		定理\ref{thm:conjugate_of_product_is_product_of_conjugates}より
		\begin{align}
			|a \cdot b|^2 = (a \cdot b) \cdot \overline{a \cdot b}
			= (a \cdot b) \cdot (\overline{a} \cdot \overline{b})
			= (a \cdot \overline{a}) \cdot (b \cdot \overline{b})
			= |a|^2 \cdot |b|^2
			= (|a| \cdot |b|)^2
		\end{align}
		が成り立つので,定理\ref{thm:numbers_whose_squares_are_coincide_are_same_or_inverse}より
		\begin{align}
			|a \cdot b| = |a| \cdot |b| \vee |a \cdot b| = -|a| \cdot |b|
		\end{align}
		が従う.ここで非負の数の積は非負であるから(これも未証明だが認める)
		\begin{align}
			|a \cdot b| = |a| \cdot |b|
		\end{align}
		が成り立つ.
		\QED
	\end{sketch}
	
	\begin{screen}
		\begin{thm}[逆元の絶対値は等しい]
			$z$を複素数とするとき
			\begin{align}
				|-z| = |z|.
			\end{align}
		\end{thm}
	\end{screen}
	
	\begin{sketch}
		定理\ref{thm:inverse_of_product}より
		\begin{align}
			-z = -(1 \cdot z) = (-1) \cdot z
		\end{align}
		が成り立つから,定理\ref{thm:absolute_value_of_product_is_product_of_absolute_values}より
		\begin{align}
			|-z| = |-1| \cdot |z|
		\end{align}
		が成り立つ.
		\begin{align}
			|-1| = \sqrt{(-1)^2} = \sqrt{1} = 1
		\end{align}
		であるから
		\begin{align}
			|-z| = |-1| \cdot |z| = 1 \cdot |z| = |z|
		\end{align}
		が従う.
		\QED
	\end{sketch}
	
	\begin{screen}
		\begin{thm}[劣加法性]
			$a$と$b$を複素数とするとき
			\begin{align}
				|a + b| \leq |a| + |b|.
			\end{align}
		\end{thm}
	\end{screen}
	
	\begin{sketch}
		定理\ref{thm:square_of_absolute_value_is_product_with_conjugate}より
		\begin{align}
			|a + b|^2 &= (a + b) \cdot (\overline{a} + \overline{b}) \\
			&= (a \cdot \overline{a} + b \cdot \overline{a})
			+ (a \cdot \overline{b} + b \cdot \overline{b}) \\
			&= |a|^2 + |b|^2 + 2 \cdot \Re{(a \cdot \overline{b})}
		\end{align}
		が成り立つ.
		\begin{align}
			\Re{(a \cdot \overline{b})}
			\leq |a \cdot \overline{b}|
			= |a| \cdot |b|
		\end{align}
		が成り立つので
		\begin{align}
			|a + b|^2 \leq (|a| + |b|)^2
		\end{align}
		が従い
		\begin{align}
			|a + b| \leq |a| + |b|
		\end{align}
		が得られる.
		\QED
	\end{sketch}
	
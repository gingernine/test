\subsection{代数的性質}
	\begin{itembox}[l]{認めること}
		\begin{itemize}
			\item 零因子が存在しないこと.
			\item 正の実数には平方根が取れること.
			\item $a - b$は$a + (-b)$の略.
			\item $a + (b \cdot c)$は$a + b \cdot c$と同じ.
		\end{itemize}
	\end{itembox}
	
	$z$を複素数とするとき,$z$の整数乗を定義する.まずは
	\begin{align}
		z^0 \defeq 1
	\end{align}
	と定め,次に
	\begin{align}
		z^1 \defeq z
	\end{align}
	と定め,次に
	\begin{align}
		z^2 \defeq z \cdot z
	\end{align}
	と定め,次も同様に
	\begin{align}
		z^3 \defeq (z \cdot z) \cdot z = z^2 \cdot z
	\end{align}
	と定める.この調子で
	\begin{align}
		z^4 &\defeq z^3 \cdot z \\
		z^5 &\defeq z^4 \cdot z \\
		z^6 &\defeq z^5 \cdot z \\
	\end{align}
	と定めていくと,任意の自然数$n$に対して
	\begin{align}
		z^n
	\end{align}
	なる複素数が得られる.累乗の厳密な定義は帰納法を用いるがここでは直感的な導入で終える.
	ただし一つ言っておくが,$n$を任意の自然数とするとき
	\begin{align}
		z^{n+1} = z^n \cdot z
	\end{align}
	は定理である.
	
	$n$を負の整数とするときは,
	\begin{align}
		-n \in \Natural
	\end{align}
	であり,すでに
	\begin{align}
		z^{-n}
	\end{align}
	なる複素数は得られている.そこで
	\begin{align}
		z^n \defeq (z^{-n})^{-1}
	\end{align}
	により$z$の$n$乗を定める.
	
	\begin{screen}
		\begin{thm}[積の累乗は累乗の積]
			$a$と$b$を複素数とし,$n$を整数とするとき,
			\begin{align}
				(a \cdot b)^n = a^n \cdot b^n.
			\end{align}
		\end{thm}
	\end{screen}
	
	\begin{sketch}
		
	\end{sketch}
	
	\begin{screen}
		\begin{thm}[指数法則]
			$z$を複素数とし,$n$と$m$を整数とするとき,
			\begin{align}
				z^{n+m} = z^n \cdot z^m.
			\end{align}
		\end{thm}
	\end{screen}
	
	\begin{sketch}
		
	\end{sketch}
	
	\begin{screen}
		\begin{thm}[二乗が等しい数同士は一致するか逆元である]
			$a$と$b$を複素数とするとき,
			\begin{align}
				a^2 = b^2 \Longleftrightarrow 
				a = b \vee a = -b.
			\end{align}
		\end{thm}
	\end{screen}
	
	\begin{sketch}
		$a$と$b$を複素数とする.ここでまず
		\begin{align}
			(a + b) \cdot (a - b)
			&= (a+b) \cdot a + (a+b) \cdot (-b) \\
			&= (a^2 + b \cdot a) + \left(a \cdot (-b) + b \cdot (-b)\right) \\
			&= (a^2 + b \cdot a) + \left(a \cdot (-b) + b \cdot (-b)\right) \\
			&= \left[a^2 + (b \cdot a + a \cdot (-b)) \right] + b \cdot (-b) \\
			&= \left[a^2 + (b \cdot a - a \cdot b) \right] - b^2 \\
			&= (a^2 + 0) - b^2 \\
			&= a^2 - b^2
		\end{align}
		が成り立つ.従って
		\begin{align}
			a^2 = b^2 \Longleftrightarrow (a + b) \cdot (a - b) = 0
		\end{align}
		が成り立つが,他方で$\C$は零因子を持たないので
		\begin{align}
			(a + b) \cdot (a - b) = 0
			\Longleftrightarrow a+b = 0 \vee a - b = 0
			\Longleftrightarrow a = -b \vee a = b
		\end{align}
		も成立する.以上を併せて定理の主張を得る.
		\QED
	\end{sketch}
	
	\begin{screen}
		\begin{dfn}[平方根]
			$\alpha$を$0$以上の実数とするとき,
			\begin{align}
				
			\end{align}
		\end{dfn}
	\end{screen}
	
	\begin{screen}
		\begin{dfn}[絶対値]
			$z$を複素数とするとき,
			\begin{align}
				\sqrt{(\Re{z})^2 + (\Im{z})^2}
			\end{align}
			なる実数を$z$の{\bf 絶対値}\index{ぜったいち@絶対値}{\bf (absolute value)}と呼び
			\begin{align}
				|z|
			\end{align}
			と書く.
		\end{dfn}
	\end{screen}
	
	$a$と$b$を複素数とするとき,
	\begin{align}
		(a+b) + ((-b) + (-a))
		&= (a + (b + (-b))) + (-a) \\
		&= (a + 0) + (-a) \\
		&= a + (-a) \\
		&= 0 
	\end{align}
	かつ
	\begin{align}
		((-b) + (-a)) + (a+b)
		&= ((-b) + ((-a) + a)) + b \\
		&= ((-b) + 0) + b \\
		&= (-b) + b \\
		&= 0
	\end{align}
	が成り立つので
	\begin{align}
		-(a + b) = (-b) + (-a)
	\end{align}
	である.
	
	\begin{screen}
		\begin{thm}[積の絶対値は絶対値の積]
		\label{thm:absolute_value_of_product_is_product_of_absolute_values}
			$a$と$b$を複素数とするとき
			\begin{align}
				|a \cdot b| = |a| \cdot |b|.
			\end{align}
		\end{thm}
	\end{screen}
	
	\begin{sketch}
		
	\end{sketch}
	
	\begin{screen}
		\begin{thm}[逆元の絶対値は等しい]
			$z$を複素数とするとき
			\begin{align}
				|-z| = |z|.
			\end{align}
		\end{thm}
	\end{screen}
	
	\begin{sketch}
		定理より
		\begin{align}
			-z = (-1) \cdot z
		\end{align}
		であるから,定理\ref{thm:absolute_value_of_product_is_product_of_absolute_values}より
		\begin{align}
			|-z| = |-1| \cdot |z|
		\end{align}
		が成り立つ.
		\begin{align}
			|-1| = \sqrt{(-1)^2} = \sqrt{1} = 1
		\end{align}
		であるから
		\begin{align}
			|-z| = |-1| \cdot |z| = 1 \cdot |z| = |z|
		\end{align}
		が従う.
		\QED
	\end{sketch}
	
	\begin{screen}
		\begin{thm}[劣加法性]
			$a$と$b$を複素数とするとき
			\begin{align}
				|a + b| = |a| + |b|.
			\end{align}
		\end{thm}
	\end{screen}
	
	\begin{sketch}
		
	\end{sketch}
	
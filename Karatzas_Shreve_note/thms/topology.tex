\subsection{位相}
	\begin{screen}
		\begin{dfn}[位相]
			$S$を集合とし,$\mathscr{O}$を$\power{S}$の部分集合とする.
			$\mathscr{O}$が以下の三カ条を満たすとき,$\mathscr{O}$を$S$上の{\bf 位相}
			\index{いそう@位相}{\bf (topology)}と呼ぶ:
			\begin{description}
				\item[(O1)] $\mathscr{O}$は$S$と空集合を要素に持つ:
					\begin{align}
						\emptyset \in \mathscr{O} \wedge S \in \mathscr{O}.
					\end{align}
				\item[(O2)] $\mathscr{O}$は交叉で閉じる:
					\begin{align}
						\forall u,v\, \left(\, u,v \in \mathscr{O} \Longrightarrow u \cap v \in \mathscr{O}\, \right).
					\end{align}
				\item[(O3)] $\mathscr{O}$は部分集合の合併で閉じる:
					\begin{align}
						\forall U\, \left(\, U \subset \mathscr{O} \Longrightarrow \bigcup U \in \mathscr{O} \right).
					\end{align}
			\end{description}
			そして対
			\begin{align}
				(S,\mathscr{O})
			\end{align}
			を{\bf 位相空間}\index{いそうくうかん@位相空間}{\bf (topological space)}と呼ぶ.
		\end{dfn}
	\end{screen}
	
	\begin{screen}
		\begin{dfn}[開集合・閉集合]
			$(S,\mathscr{O})$を位相空間とするとき,$\mathscr{O}$の要素を$S$の{\bf 開集合}
			\index{かいしゅうごう@開集合}{\bf (open set)}と呼び,
			補集合が開である$S$の部分集合,つまり
			\begin{align}
				A \subset S \wedge S \backslash A \in \mathscr{O}
			\end{align}
			なる$A$を$S$の{\bf 閉集合}\index{へいしゅうごう@閉集合}{\bf (closed set)}と呼ぶ.
		\end{dfn}
	\end{screen}
	
	$\C$の部分集合族を
	\begin{align}
		\mathscr{O}_\C \defeq \Set{O}{O \subset \C \wedge \forall x \in O\, \exists r \in \R_+\, 
		\left(\, \forall y \in \C\, (\, |y-x| < r \Longrightarrow y \in O\, )\, \right)}
	\end{align}
	で定めると,これは$\C$上の位相となる.以降は$\mathscr{O}_\C$を$\C$の通常の位相として考える.
	つまり,$\C$の部分集合$O$は,$O$の要素$x$が与えられたときに
	\begin{align}
		\Set{y \in \C}{|x-y| < r} \subset O
	\end{align}
	なる正の実数$r$が取れるなら$\C$の通常の開集合と見做される.
	
	\begin{screen}
		\begin{dfn}[内部・閉包]
			位相空間の部分集合$A$に対し,
			$A$に含まれる最大の開集合を$A$の{\bf 内部}\index{ないぶ@内部}{\bf (interior)}と呼び
			$A^{\mathrm{o}}$や$A^i$で表す.また
			$A$を含む最大の閉集合を$A$の{\bf 閉包}\index{へいほう@閉包}{\bf (closure)}と呼び
			$\overline{A}$や$A^a$で表す.特に,
			\begin{align}
				\mbox{$A$が開}\ \Longleftrightarrow\ 
				A = A^\mathrm{o},
				\quad \mbox{$A$が閉}\ \Longleftrightarrow\ 
				A = \overline{A}.
				\label{eq:dfn_interior_closure}
			\end{align}
		\end{dfn}
	\end{screen}
	
	\begin{screen}
		\begin{thm}[内部の補集合は補集合の閉包]
		\label{thm:topology_note_closure_interior}
			$A$を位相空間の部分集合とするとき次が成り立つ.
			\begin{align}
				A^{ic} = A^{ca},
				\quad A^{cic} = A^a,
				\quad A^{ci} = A^{ac}.
			\end{align}
		\end{thm}
	\end{screen}
	
	\begin{prf}
		$A^i \subset A$より$A^{ic} \supset A^c$が従い,
		$A^{ic}$が閉であるから$A^{ic} \supset A^{ca}$となる.
		一方で$A^c \subset A^{ca}$より$A \supset A^{cac}$が従い,
		$A^{cac}$は開であるから$A^i \supset A^{cac}$すなわち
		$A^{ic} \subset A^{ca}$となる.
		$A$を$A^c$に替えれば残りの関係も得られる.
		\QED
	\end{prf}
	
	\begin{screen}
		\begin{dfn}[近傍・基本近傍系]
			空でない位相空間$S$において,$x \in S$と$U \subset S$に対し
			\begin{align}
				x \in U^{\mathrm{o}}
			\end{align}
			が満たされるとき$U$は$x$の{\bf 近傍}\index{きんぼう@近傍}
			{\bf (neighborhood)}であるという.
			同様に$A \subset S$と$V \subset S$に対し
			\begin{align}
				A \subset V^{\mathrm{o}}
			\end{align}
			が満たされるとき,$V$は$A$の近傍であるという.
			点$x$の近傍全体({\bf 近傍系}\index{きんぼうけい@近傍系}
			{\bf (neighborhood system)}と呼ぶ)を$\mathscr{V}(x)$と書くとき,
			$S$は$x$の最大の近傍であるから$\mathscr{V}(x)$は空ではない.
			また$\mathscr{V}(x)$の空でない部分集合$\mathscr{U}(x)$が
			\begin{align}
				\forall V \in \mathscr{V}(x),
				\quad \exists U \in \mathscr{U}(x),
				\quad U \subset V
			\end{align}
			を満たすとき,$\mathscr{U}(x)$を$x$の{\bf 基本近傍系}
			\index{きほんきんぼうけい@基本近傍系}{\bf (local base)}と呼ぶ.
		\end{dfn}
	\end{screen}
	
	\begin{screen}
		\begin{thm}[基本近傍系は開集合を決定する]\label{thm:local_base_defines_open_sets}
			$S$を空でない位相空間,
			$\mathscr{U}(x)$を点$x$の基本近傍系とすれば
			\begin{align}
				\mbox{$O$が$S$の開集合} \quad \Longleftrightarrow \quad 
				\mbox{$O = \emptyset$,或は任意の$x \in O$に対し
				$U \subset O$を満たす$U \in \mathscr{U}(x)$が存在する}
			\end{align}
			が成立する.すなわち,$\{\mathscr{U}(x)\}_{x \in S}$を基本近傍系とする$S$の位相は唯一つである.
		\end{thm}
	\end{screen}
	
	\begin{prf}
		空でない部分集合$O$が開集合なら任意の$x \in O$に対し$O$は$x$の近傍となるから,
		或る$U \in \mathscr{U}(x)$が存在して$U \subset O$を満たす.
		逆に任意の$x \in O$に対し$U \subset O$を満たす$U \in \mathscr{U}(x)$が存在するとき,
		\begin{align}
			x \in U^{\mathrm{o}} \subset O^{\mathrm{o}}
		\end{align}
		となり$O = O^{\mathrm{o}}$が成立するから$O$は開集合である.
		\QED
	\end{prf}
	
	\begin{screen}
		\begin{thm}[基本近傍系は位相を復元する]
		\label{thm:a_local_base_restores_the_topology}\mbox{}
			\begin{description}
				\item[(1)] 
					$(S,\mathscr{O})$を空でない位相空間とし,各点
					$x \in S$に対し$\mathscr{U}(x)$を基本近傍系とすれば以下が成り立つ:
					\begin{description}
						\item[(LB1)] $\mathscr{U}(x)$は空ではなく,また任意の$U \in \mathscr{U}(x)$は$x \in U$を満たす.
						\item[(LB2)] 任意の$U,V \in \mathscr{U}(x)$に対し或る$W \in \mathscr{U}(x)$
							が存在して$W \subset U \cap V$を満たす.
						\item[(LB3)] 任意の$U \in \mathscr{U}(x)$に対し或る$V \in \mathscr{U}(x)$が存在し,
							$V \subset U$かつ任意の$y \in V$に対し$W_y \subset U$を満たす$W_y \in \mathscr{U}(y)$が取れる.
					\end{description}
				\item[(2)]
					空でない集合$S$の各点$x$に対し(LB1)(LB2)(LB3)を満たす部分集合族$\mathscr{U}(x)$が与えられれば,
					\begin{align}
						\mathscr{O} \coloneqq
						\Set{O \subset S}{\mbox{$O = \emptyset$,或は任意の$x \in O$に対し
						$U \subset O$を満たす$U \in \mathscr{U}(x)$が存在する}}
					\end{align}
					により$S$に位相が定まり,$\{\mathscr{U}(x)\}_{x \in S}$は
					$(S,\mathscr{O})$において基本近傍系となる.
				\item[(3)] 空でない位相空間$(S,\mathscr{O})$から基本近傍系
					$\{\mathscr{U}(x)\}_{x \in S}$を得れば,
					$\{\mathscr{U}(x)\}_{x \in S}$を基本近傍系とする位相
					を(2)の手続きで構成することにより$\mathscr{O}$を復元できる.
			\end{description}
		\end{thm}
	\end{screen}
	
	\begin{prf}\mbox{}
		\begin{description}
			\item[(1)] 任意の$U \in \mathscr{U}(x)$は$x$の近傍であるから
				$(LB1)$が満たされる.また$U,V \in \mathscr{U}(x)$に対し
				\begin{align}
					x \in U^{\mathrm{o}} \cap V^{\mathrm{o}} = (U \cap V)^{\mathrm{o}}
				\end{align}
				となるから$U \cap V$は$x$の近傍であり(LB2)も従う.
				任意に$U \in \mathscr{U}(x)$を取れば,
				$U^{\mathrm{o}}$は$x$の開近傍であるから
				或る$V \in \mathscr{U}(x)$で$V \subset U^{\mathrm{o}}$
				を満たすものが存在する.このとき任意の$y \in V$に対し
				$U^{\mathrm{o}}$は$y$の開近傍となるから
				\begin{align}
					W_y \subset U^{\mathrm{o}} \subset U
				\end{align}
				を満たす$W_y \in \mathscr{U}(y)$が取れる.従って(LB3)も得られる.
			
			\item[(2)] 
				$\mathscr{U}(x)$は空ではないから$S \in \mathscr{O}$となる.
				また$O_1,O_2 \in \mathscr{O}$を取れば,
				任意の$x \in O_1 \cap O_2$に対し
				\begin{align}
					x \in U_1 \subset O_1,
					\quad x \in U_2 \subset O_2
				\end{align}
				を満たす$U_1,U_2 \in \mathscr{U}(x)$が存在し,
				(LB2)より或る$U_3 \in \mathscr{U}(x)$に対して
				\begin{align}
					U_3 \subset U_1 \cap U_2 \subset O_1 \cap O_2
				\end{align}
				が成り立つから$O_1 \cap O_2 \in \mathscr{O}$となる.
				任意に$\mathscr{G} \subset \mathscr{O}$を取れば
				任意の$x \in \bigcup \mathscr{G}$は或る$G \in \mathscr{G}$の点であるから,
				\begin{align}
					U \subset G \subset \bigcup \mathscr{G}
				\end{align}
				を満たす$U \in \mathscr{U}(x)$が存在し$\bigcup \mathscr{G} \in \mathscr{O}$が従う.
				よって$\mathscr{O}$は位相である.
				ところで,任意の$U \in \mathscr{U}(x)$に対し
				\begin{align}
					U^{\mathrm{o}} = 
					\Set{y \in U}{\mbox{或る$W_y \in \mathscr{U}(y)$が存在して
					$W_y \subset U$となる}} \eqqcolon \tilde{U}
					\label{eq:thm_a_local_base_restores_the_topology_0}
				\end{align}
				が成立する.実際$\mathscr{O}$の定義より
				\begin{align}
					y \in U^{\mathrm{o}} \quad \Longrightarrow \quad
					\mbox{或る$W_y \in \mathscr{U}(y)$で
					$W_y \subset U^{\mathrm{o}}$}
				\end{align}
				となるから$U^{\mathrm{o}}\subset\tilde{U}$が従い,
				逆に$y \in \tilde{U}$については,
				(\refeq{eq:thm_a_local_base_restores_the_topology_0})の$W_y$に対して
				(LB3)より或る$X_y \in \mathscr{U}(y)$が
				\begin{align}
					X_y \subset W_y,\quad 
					z \in X_y \ \Longrightarrow \
					\mbox{或る$Y_z \in \mathscr{U}(z)$で$Y_z \subset X_y \subset U$}
				\end{align}
				を満たすから$X_y \subset \tilde{U}$が従う.
				すなわち$\tilde{U}$は開集合であり,$U^{\mathrm{o}}\subset\tilde{U}$
				と併せて(\refeq{eq:thm_a_local_base_restores_the_topology_0})
				を得る.(LB3)より
				\begin{align}
					V \subset U, \quad y \in V \ \Longrightarrow \
					\mbox{或る$W_y \in \mathscr{U}(y)$で$W_y \subset U$}
				\end{align}
				を満たす$V \in \mathscr{U}(x)$が存在し,(LB1)と併せて
				\begin{align}
					x \in V \subset \tilde{U} = U^{\mathrm{o}}
				\end{align}
				が成り立つから任意の$U \in \mathscr{U}(x)$は$x$の近傍である.
				そして$W$を$x$の任意の近傍とすれば,
				$\mathscr{O}$の定め方より或る$U \in \mathscr{U}(x)$が
				$U \subset W^{\mathrm{o}}$を満たすから
				$\mathscr{U}(x)$は$x$の基本近傍系である.
			
			\item[(3)] 
				定理\ref{thm:local_base_defines_open_sets}より
				$\{\mathscr{U}(x)\}_{x \in S}$を基本近傍系とする位相は唯一つであるから
				主張が従う.
				\QED
		\end{description}
	\end{prf}
	
	\begin{screen}
		\begin{dfn}[集積点・密集点]
			位相空間$S$の点$x$と部分集合$A$について,
			$x$の任意の近傍$U$に対し
			\begin{align}
				(U \backslash \{x\}) \cap A \neq \emptyset
			\end{align}
			となるとき,$x$は$A$の{\bf 集積点}\index{しゅうせきてん@集積点}
			{\bf (accumulation point)}であるという.
			同様に$x$の任意の近傍$U$に対し
			\begin{align}
				U \cap A \neq \emptyset
			\end{align}
			となるとき,$x$は$A$の{\bf 密集点}\index{みっしゅうてん@密集点}
			{\bf (cluster point)}であるという.
		\end{dfn}
	\end{screen}
	
	集積点と密集点の明確な違いは$T_1$空間(後述)において現れる.
	\begin{screen}
		\begin{thm}[閉である一点集合は集積点を持たない]
		\label{thm:closed_singleton_has_no_accumulation_point}
			位相空間において,閉じている一点集合は集積点を持たない.特に
			$\{x\}$が閉であるとき,$x$は$\{x\}$の密集点ではあるが集積点ではない.
		\end{thm}
	\end{screen}
	
	\begin{prf}
		一点集合$\{x\}$が閉であるとする.このとき$y \neq x$なら
		$U \coloneqq \{x\}^c$は$y$の開近傍となり
		\begin{align}
			(U \backslash \{y\}) \cap \{x\} = \emptyset
		\end{align}
		を満たすから$y$は$\{x\}$の集積点ではない.
		$x$は$\{x\}$の集積点となりえないから$\{x\}$は集積点を持たない.
		\QED
	\end{prf}
	
	\begin{screen}
		\begin{thm}[閉集合は密集点集合]
		\label{thm:belongs_to_closure_iff_clusters}
			位相空間$S$の点$x$と部分集合$A$について次が成り立つ:
			\begin{align}
				x \in \overline{A} \quad \Longleftrightarrow \quad
				\mbox{$x$は$A$の密集点である}.
				\label{eq:thm_belongs_to_closure_iff_clusters}
			\end{align}
			特に,$A$が閉であることと$A$の密集点全体が$A$に一致することは同値になる.
		\end{thm}
	\end{screen}
	
	\begin{prf}
		$x$の或る近傍$U$が$U \cap A = \emptyset$を満たすとき,
		定理\ref{thm:topology_note_closure_interior}より
		\begin{align}
			x \in U^i \subset A^{ci} = A^{ac}
		\end{align}
		となり$x \notin \overline{A}$が従う.逆に
		$x \notin \overline{A}$なら
		$\overline{A}^c$は$A$と交わらない$x$の開近傍となるから
		(\refeq{eq:thm_belongs_to_closure_iff_clusters})が出る.
		また(\refeq{eq:dfn_interior_closure})より
		\begin{align}
			\mbox{$A$が閉} \quad \Longleftrightarrow \quad A = \overline{A}
			\quad \Longleftrightarrow \quad
			\mbox{$A$の密集点全体が$A$に一致}
		\end{align}
		が成立する.
		\QED
	\end{prf}
	
	\begin{screen}
		\begin{thm}[$x \in \overline{A \backslash \{x\}}$$\Longleftrightarrow$$x$が$A$の集積点]
			位相空間$S$の点$x$と部分集合$A$について次が成り立つ:
			\begin{align}
				x \in \overline{A \backslash \{x\}} \quad \Longleftrightarrow \quad
				\mbox{$x$は$A$の集積点である}.
			\end{align}
		\end{thm}
	\end{screen}
	
	\begin{prf}
		$x$の任意の近傍$U$に対し
		$U \cap (A \backslash \{x\}) = (U \backslash \{x\}) \cap A$となるから,
		定理\ref{thm:belongs_to_closure_iff_clusters}と併せて
		\begin{align}
			x \in \overline{A \backslash \{x\}} 
			&\quad \Longleftrightarrow \quad
			\mbox{$x$の任意の近傍$U$に対し$U \cap (A \backslash \{x\}) \neq \emptyset$} \\
			&\quad \Longleftrightarrow \quad
			\mbox{$x$の任意の近傍$U$に対し$(U \backslash \{x\}) \cap A \neq \emptyset$}
			\quad \Longleftrightarrow \quad
			\mbox{$x$は$A$の集積点}
		\end{align}
		が成立する.
		\QED
	\end{prf}
	
	\begin{screen}
		\begin{dfn}[相対位相]
			$(S,\mathscr{O})$を位相空間,$M \subset S$を部分集合,
			$i:M \longrightarrow S$を恒等写像とするとき,
			\begin{align}
				\mathscr{O}_M \coloneqq 
				\Set{i^{-1}(O) = O \cap M}{O \in \mathscr{O}}
			\end{align}
			で定める$\mathscr{O}_M$を$M$の{\bf 相対位相}
			\index{そうたいいそう@相対位相}{\bf (relative topology)}と呼ぶ.
			また相対位相が定まった部分集合をもとの空間に対し{\bf 部分位相空間}
			\index{ぶぶんいそうくうかん@部分位相空間}{\bf (topological subspace)}と呼び,
			紛れが無ければ単に{\bf 部分空間}\index{ぶぶんくうかん@部分空間}とも呼ぶ.
		\end{dfn}
	\end{screen}
	
	\begin{screen}
		\begin{dfn}[$\R$上の位相]
			$\R$上の位相は$\C$上の位相の相対位相として定める:
			\begin{align}
				\mathscr{O}_\R \defeq \Set{O \cap \R}{O \in \mathscr{O}_\C}.
			\end{align}
		\end{dfn}
	\end{screen}
	
	\begin{screen}
		\begin{thm}[$\R$の開集合はボールから成る]
			$O$を$\R$の部分集合とするとき,
			\begin{align}
				O \in \mathscr{O}_\R \Longleftrightarrow
				\forall x \in O\, \exists r \in \R_+\, \left(\, \Set{y \in \R}{|x-y| < r} \subset O\, \right).
			\end{align}
		\end{thm}
	\end{screen}
	
	\begin{screen}
		\begin{dfn}[被覆・コンパクト・相対コンパクト・局所コンパクト・$\sigma$-コンパクト]\mbox{}
			\begin{itemize}
				\item
					集合$S$の部分集合族$\mathscr{B}$が
					$S$の{\bf 被覆}\index{ひふく@被覆}{\bf (cover)}であるとは,
					\begin{align}
						S = \bigcup \mathscr{B}
					\end{align}
					を満たすことをいう.また可算(有限)個の部分集合から成る被覆を
					{\bf 可算(有限)被覆}\index{かさんひふく@可算被覆}
					\index{ゆうげんひふく@有限被覆}と呼ぶ.
					特に,位相空間において開集合のみから成る被覆を
					{\bf 開被覆}\index{かいひふく@開被覆}{\bf (open cover)}と呼ぶ.
				
				\item 集合$S$の被覆$\mathscr{B}$に対し,その部分集合で
					$S$の被覆となるものを$\mathscr{B}$の{\bf 部分被覆}
					\index{ぶぶんひふく@部分被覆}{\bf (subcover)}と呼ぶ.
					部分被覆が有限(可算)集合であるときは有限(可算)部分被覆と呼ぶ.
				\item 
					位相空間において任意の開被覆が有限部分被覆を持つとき,
					その空間は{\bf コンパクト}\index{こんぱくと@コンパクト}である
					{\bf (compact)}という.
					位相空間の部分集合は,その相対位相でコンパクト空間となるとき
					{\bf コンパクト部分集合}と呼ばれる.
				
				\item 位相空間の部分集合で,その閉包がコンパクトであるものを
					{\bf 相対コンパクト}\index{そうたいこんぱくと@相対コンパクト}な
					{\bf (relatively compact)}部分集合という.
				
				\item 位相空間の任意の点がコンパクトな近傍を持つとき,
					その空間は{\bf 局所コンパクト}である
					\index{きょくしょこんぱくと@局所コンパクト}{\bf (locally compact)}という.
					
				\item 位相空間においてコンパクト集合から成る可算被覆が存在するとき,
					その空間は{\bf $\sigma$-コンパクト}
					\index{しぐまこんぱくと@$\sigma$-コンパクト}であるという.
			\end{itemize}
		\end{dfn}
	\end{screen}
	
	集合$S$とその部分集合$A$に対し,$S$の部分集合族$\mathscr{B}$で
	$A \subset \bigcup \mathscr{B}$を満たすものを
	$A$の`{\bf $S$における被覆}'と呼ぶ.$\mathscr{B}$の構成要素が$S$の開集合である場合は
	`{\bf $S$における開被覆}'と呼び,他に`{\bf $S$における部分被覆}'や`{\bf $S$における有限被覆}'といった言い方もする.
	
	\begin{screen}
		\begin{thm}[部分集合のコンパクト性]
		\label{thm:subset_is_compact_iff_every_original_open_cover_contains_finite_subcover}
			$A$を位相空間$S$の部分集合とするとき次が成り立つ:
			\begin{align}
				\mbox{$A$がコンパクト部分集合} \quad \Longleftrightarrow \quad
				\mbox{$A$の$S$における任意の開被覆が($S$における)有限部分被覆を含む}.
			\end{align}
		\end{thm}
	\end{screen}
	
	\begin{prf}
		$A$がコンパクト部分集合であるとき,$\mathscr{B}$を$A$の$S$における開被覆とすれば
		\begin{align}
			\Set{B \cap A}{B \in \mathscr{B}}
		\end{align}
		は部分空間$A$における開被覆となり,有限個の$B_1,B_2,\cdots,B_n \in \mathscr{B}$により
		\begin{align}
			A = \bigcup_{i=1}^n (B_i \cap A) \subset \bigcup_{i=1}^n B_i
		\end{align}
		となるから$\Longrightarrow$が従う.逆に右辺が満たされているとき,
		$\mathscr{A}$を$A$の相対開集合から成る$A$の被覆として
		\begin{align}
			\mathscr{C} \coloneqq \Set{C \subset S}{\mbox{$C$は$S$の開集合で$C \cap A \in \mathscr{A}$}}
		\end{align}
		とおけば,
		\begin{align}
			\mathscr{A} = \Set{C \cap A}{C \in \mathscr{C}}
		\end{align}
		が満たされる.このとき$\mathscr{C}$は$A$を覆うから有限個の
		$C_1,C_2,\cdots,C_m \in \mathscr{C}$で$A \subset \bigcup_{j=1}^m C_j$となり,
		\begin{align}
			A = \bigcup_{j=1}^m (A \cap C_j)
		\end{align}
		かつ$A \cap C_j \in \mathscr{A}$が成り立つから$A$はコンパクトである.
		\QED
	\end{prf}
	
	\begin{screen}
		\begin{thm}[コンパクト集合の閉部分集合はコンパクト]
		\label{thm:closed_subset_of_compact_set_is_compact_on_Hausdorff_space}
			$S$を位相空間,$K,F$をそれぞれ$S$のコンパクト部分集合,閉集合とするとき,
			$K \cap F$は$S$のコンパクト部分集合である.
		\end{thm}
	\end{screen}
	
	\begin{prf}
		$K \cap F$の任意の($S$における)開被覆に$S \backslash F$を加えれば
		$K$の($S$における)開被覆となるから,そのうち$K$の有限部分被覆を取ることができる.
		$S \backslash F$を除けば$K \cap F$の有限被覆が残り
		$K \cap F$のコンパクト性が出る.
		\QED
	\end{prf}
	
	\begin{screen}
		\begin{dfn}[有限交叉性]
			集合$S$の部分集合族$\mathscr{S}$について,その任意の
			有限部分族$\mathscr{T} \subset \mathscr{S}$が
			$\bigcap \mathscr{T} \neq \emptyset$を満たすとき
			$\mathscr{S}$は{\bf 有限交叉性}\index{ゆうげんこうさせい@有限交叉性}
			{\bf (finite intersection property)}を持つという.
		\end{dfn}
	\end{screen}
	
	\begin{screen}
		\begin{thm}[コンパクト$\Longleftrightarrow$閉集合族が有限交叉的]
		\label{thm:compact_iff_closed_sets_family_finitely_intersect}
			$S$を位相空間,$A$を$S$の部分集合とするとき,
			\begin{align}
				&\mbox{$A$がコンパクト部分集合} \quad \Longleftrightarrow \\ 
				&\quad \mbox{任意の$S$の閉集合族$\mathscr{F}$に対し,
				$\Set{F \cap A}{F \in \mathscr{F}}$が有限交叉性を持つなら
				$A \cap \bigcap \mathscr{F} \neq \emptyset.$}
			\end{align}
		\end{thm}
	\end{screen}
	
	\begin{prf}
		定理\ref{thm:subset_is_compact_iff_every_original_open_cover_contains_finite_subcover}より
		\begin{align}
			&\mbox{$A$がコンパクト部分集合} \\
			&\Longleftrightarrow \quad \mbox{$A$の$S$における任意の開被覆が($S$における)有限部分被覆を含む} \\
			&\Longleftrightarrow \quad \mbox{任意の$S$の閉集合族$\mathscr{F}$に対し,
			$A \cap \bigcap \mathscr{F} = \emptyset$なら或る有限族$\mathscr{M} \subset \mathscr{F}$で
			$A \cap \bigcap \mathscr{M} = \emptyset$} \\
			&\Longleftrightarrow \quad \mbox{任意の$S$の閉集合族$\mathscr{F}$に対し,
			$\Set{F \cap A}{F \in \mathscr{F}}$が有限交叉性を持つなら
			$A \cap \bigcap \mathscr{F} \neq \emptyset$}
		\end{align}
		が従う.
		\QED
	\end{prf}
	
	\begin{screen}
		\begin{dfn}[連続・同相・開写像]
			$f$を位相空間$S$から位相空間$T$への写像とする.
			\begin{itemize}
				\item
					$x \in S$において$f(x)$の任意の任意の近傍の
					$f$による引き戻しが$x$の近傍となるとき,
					$f$は{\bf 点$x$で連続}\index{れんぞく@連続}である
					{\bf (continuous at a point $x$)}という.
					
				\item $T$の任意の開集合の$f$による引き戻しが$S$の開集合となるとき,
					$f$を{\bf 連続写像}\index{れんぞくしゃぞう@連続写像}
					{\bf (continuous mapping)}と呼ぶ.
					
				\item $f$に逆写像$f^{-1}$が存在し,$f,f^{-1}$が共に連続であるとき,
					$f$を{\bf 同相写像}\index{どうそうしゃぞう@同相写像}{\bf (homeomorphism)}
					や{\bf 位相同型写像}\index{いそうどうけいしゃぞう@位相同型写像},
					或は単に{\bf 同相}や{\bf 位相同型}と呼ぶ.
					また$S,T$間に同相写像が存在するとき$S$と$T$は
					{\bf 同相}\index{どうそう@同相}である{\bf (homeomorphic)},
					或は{\bf 位相同型}であるという.
					
				\item $S$の任意の開集合の$f$による像が$T$の開集合となるとき,
					$f$を{\bf 開写像}\index{かいしゃぞう@開写像}{\bf (open mapping)}と呼ぶ.
			\end{itemize}
		\end{dfn}
	\end{screen}
	
	\begin{screen}
		\begin{thm}[コンパクト集合の連続写像による像はコンパクト]
		\end{thm}
	\end{screen}
	
	\begin{screen}
		\begin{thm}[各点連続$\Longleftrightarrow$連続]
		\label{thm:continuous_on_every_point_iff_continuous}
			$f$を位相空間$S$から位相空間$T$への写像とするとき次が成り立つ:
			\begin{align}
				\mbox{$f$が連続} \quad \Longleftrightarrow \quad
				\mbox{$f$が$S$の各点で連続}.
			\end{align}
		\end{thm}
	\end{screen}
	
	\begin{prf}
		$f$が連続であるとき,各点$x \in S$で$f(x)$の任意の近傍$U$に対し
		$f(x) \in U^{\mathrm{o}}$が満たされるから
		$f^{-1}(U^{\mathrm{o}})$は$x$を含む開集合となる.
		$f^{-1}(U^{\mathrm{o}})$は$f^{-1}(U)$に含まれる開集合であるから
		\begin{align}
			x \in f^{-1}(U^{\mathrm{o}}) \subset f^{-1}(U)^{\mathrm{o}}
		\end{align}
		が成り立ち,従って$f$は$x$で連続である.
		逆に$f$が各点連続であるとき,
		$T$の任意の開集合$O$に対し
		$f^{-1}(O)$は任意の$x \in f^{-1}(O)$の近傍となるから
		定理\ref{thm:local_base_defines_open_sets}より
		$f^{-1}(O)$は開集合である.よって$f$は連続である.
		\QED
	\end{prf}
	
	\begin{screen}
		\begin{thm}[部分空間と制限写像の連続性]
			$S,T$を位相空間,$f$を$S$から$T$への写像とする.
			また$g:S \longrightarrow f(S)$を
			$f$の終集合を$f(S)$へ制限した写像とする.このとき次が成り立つ:
			\begin{align}
				\mbox{$f:S \longrightarrow T$が連続である} 
				\quad \Longleftrightarrow \quad
				\mbox{$g:S \longrightarrow f(S)$が($f(S)$の相対位相に関して)連続である}.
			\end{align}
		\end{thm}
	\end{screen}
	
	\begin{prf}
		$U \coloneqq f(S)$とおけば$T$の任意の開集合$O$に対し
		\begin{align}
			g^{-1}(U \cap O) = f^{-1}(U \cap O) = f^{-1}(O)
		\end{align}
		が成り立つから,$f$と$g$の連続性は一致する.
		\QED
	\end{prf}
	
	\begin{screen}
		\begin{thm}[位相の生成]
			$S$を集合,$\mathscr{M}$を$S$の部分集合の族として
			\begin{align}
				\mathscr{A} \coloneqq
				\Set{\bigcap \mathscr{F}}{\mbox{$\mathscr{F}$は$\mathscr{M}$の有限部分集合}}
			\end{align}
			とおくとき,$\mathscr{M}$を含む最小の位相は
			\begin{align}
				\mathscr{O} \coloneqq
				\Set{\bigcup \Lambda}{\Lambda \subset \mathscr{A}}
				\cup \{S\}
			\end{align}
			で与えられる.この$\mathscr{O}$を$\mathscr{M}$が生成する$S$の位相と呼ぶ.
		\end{thm}
	\end{screen}
	
	\begin{prf}
		$\mathscr{O}$は定め方より$S$と$\emptyset$を含む.また
		任意の$O_1 = \bigcup \Lambda_1,\ O_2=\bigcup \Lambda_2 \in \mathscr{O},\ 
		(\Lambda_1,\Lambda_2 \subset \mathscr{A})$に対し
		\begin{align}
			\Lambda \coloneqq
			\Set{I \cap J}{I \in \Lambda_1,\ J \in \Lambda_2} \subset \mathscr{A}
		\end{align}
		となるから
		\begin{align}
			O_1 \cap O_2 = \bigcup_{I \in \Lambda_1,\ J \in \Lambda_2} I \cap J
			= \bigcup \Lambda \in \mathscr{O}
		\end{align}
		が成立する.任意に$\emptyset \neq \mathscr{U} \subset \mathscr{O}$を取れば,
		各$U \in \mathscr{U}$に$U = \bigcup \Lambda_U$を満たす
		$\Lambda_U \subset \mathscr{A}$が対応し,このとき
		\begin{align}
			\bigcup_{U \in \mathscr{U}} \Lambda_U \subset \mathscr{A}
		\end{align}
		となるから
		\begin{align}
			\bigcup \mathscr{U} = \bigcup \Biggl(\bigcup_{U \in \mathscr{U}} \Lambda_U\Biggr)
			\in \mathscr{O}
		\end{align}
		が従う.$\mathscr{M}$を含む任意の位相は$\mathscr{A}$を含みかつその任意和で閉じるから$\mathscr{O}$を含む.
		\QED
	\end{prf}
	
	\begin{screen}
		\begin{thm}[Alexanderの定理]
		\end{thm}
	\end{screen}
	
	\begin{screen}
		\begin{dfn}[始位相]
			$f \in \mathscr{F}$を集合$S$から位相空間$(T_f,\mathscr{O}_f)$への写像とするとき,
			全ての$f \in \mathscr{F}$を連続にする最弱の位相を$S$の$\mathscr{F}$-始位相
			(initial topology)と呼ぶ.$\mathscr{F}$-始位相は次が生成する位相である:
			\begin{align}
				\bigcup_{f \in \mathscr{F}} \Set{f^{-1}(O)}{O \in \mathscr{O}_f}.
			\end{align}
		\end{dfn}
	\end{screen}
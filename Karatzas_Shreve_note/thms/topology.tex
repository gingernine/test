\subsection{位相}
	\begin{screen}
		\begin{dfn}[位相]
			集合$S$の部分集合族$\mathscr{O}$が
			以下を満たすとき,$\mathscr{O}$を$S$の位相(topology),或は開集合系と呼ぶ:
			\begin{description}
				\item[(O1)] $\emptyset, S \in \mathscr{O}$,
				\item[(O2)] $O_1,O_2 \in \mathscr{O} 
					\quad \Longrightarrow \quad O_1 \cap O_2 \in \mathscr{O}$,
				\item[(O3)] $\displaystyle\mathscr{U} \subset \mathscr{O}
					\quad \Longrightarrow \quad \bigcup \mathscr{U} = 
					\bigcup_{U \in \mathscr{U}} U \in \mathscr{O}$.
			\end{description}
			また$\mathscr{O}$の元を$S$の開集合(open set)と呼び,
			補集合が開である集合を閉集合(closed set)と呼ぶ.
		\end{dfn}
	\end{screen}
	
	\begin{screen}
		\begin{dfn}[内部・閉包]
			位相空間の部分集合$A$に対し,
			$A$に含まれる最大の開集合を$A$の内部(interior)と呼び
			$A^{\mathrm{o}}$や$A^i$で表す.また
			$A$を含む最大の閉集合を$A$の閉包(closure)と呼び
			$\overline{A}$や$A^a$で表す.特に,
			\begin{align}
				\mbox{$A$が開} \quad \Longleftrightarrow \quad
				A = A^\mathrm{o},
				\quad \mbox{$A$が閉} \quad \Longleftrightarrow \quad
				A = \overline{A}.
			\end{align}
		\end{dfn}
	\end{screen}
	
	\begin{screen}
		\begin{thm}[内部の補集合は補集合の閉包]
		\label{thm:topology_note_closure_interior}
			$A$を位相空間の部分集合とするとき次が成り立つ.
			\begin{align}
				A^{ic} = A^{ca}.
			\end{align}
		\end{thm}
	\end{screen}
	
	\begin{prf}
		$A^i \subset A$より$A^{ic} \supset A^c$が従い,
		$A^{ic}$が閉であるから$A^{ic} \supset A^{ca}$となる.
		一方で$A^c \subset A^{ca}$より$A \supset A^{cac}$が従い,
		$A^{cac}$は開であるから$A^i \supset A^{cac}$すなわち
		$A^{ic} \subset A^{ca}$となる.
		\QED
	\end{prf}
	
	\begin{screen}
		\begin{dfn}[近傍・基本近傍系]
			空でない位相空間$S$において,$x \in S$と$U \subset S$に対し
			\begin{align}
				x \in U^{\mathrm{o}}
			\end{align}
			が満たされるとき$U$は$x$の近傍(neighborhood)であるという.
			同様に$A \subset S$と$V \subset S$に対し
			\begin{align}
				A \subset V^{\mathrm{o}}
			\end{align}
			が満たされるとき,$V$は$A$の近傍であるという.
			点$x$の近傍全体を$\mathscr{V}(x)$と書くとき,
			$S$は$x$の最大の近傍であるから$\mathscr{V}(x)$は空ではない.
			また$\mathscr{V}(x)$の空でない部分集合$\mathscr{U}(x)$が
			\begin{align}
				\forall V \in \mathscr{V}(x),
				\quad \exists U \in \mathscr{U}(x),
				\quad U \subset V
			\end{align}
			を満たすとき,$\mathscr{U}(x)$を$x$の基本近傍系(local base of a point $x$)と呼ぶ.
		\end{dfn}
	\end{screen}
	
	\begin{screen}
		\begin{thm}[基本近傍系は開集合を決定する]\label{thm:local_base_defines_open_sets}
			$S$を空でない位相空間,
			$\mathscr{U}(x)$を点$x$の基本近傍系とすれば
			\begin{align}
				\mbox{$O$が$S$の開集合} \quad \Longleftrightarrow \quad 
				\mbox{$O = \emptyset$,或は任意の$x \in O$に対し
				$U \subset O$を満たす$U \in \mathscr{U}(x)$が存在する}
			\end{align}
			が成立する.すなわち,$\{\mathscr{U}(x)\}_{x \in S}$を基本近傍系とする$S$の位相は唯一つである.
		\end{thm}
	\end{screen}
	
	\begin{prf}
		空でない部分集合$O$が開集合なら任意の$x \in O$に対し$O$は$x$の近傍となるから,
		或る$U \in \mathscr{U}(x)$が存在して$U \subset O$を満たす.
		逆に任意の$x \in O$に対し$U \subset O$を満たす$U \in \mathscr{U}(x)$が存在するとき,
		\begin{align}
			x \in U^{\mathrm{o}} \subset O^{\mathrm{o}}
		\end{align}
		となり$O = O^{\mathrm{o}}$が成立するから$O$は開集合である.
		\QED
	\end{prf}
	
	\begin{screen}
		\begin{thm}[基本近傍系は位相を復元する]\mbox{}
			\begin{description}
				\item[(1)] 
					$(S,\mathscr{O})$を空でない位相空間とし,各点
					$x \in S$に対し$\mathscr{U}(x)$を基本近傍系とすれば以下が成り立つ:
					\begin{description}
						\item[(LB1)] $\mathscr{U}(x)$は空ではなく,また任意の$U \in \mathscr{U}(x)$は$x \in U$を満たす.
						\item[(LB2)] 任意の$U,V \in \mathscr{U}(x)$に対し或る$W \in \mathscr{U}(x)$
							が存在して$W \subset U \cap V$を満たす.
						\item[(LB3)] 任意の$U \in \mathscr{U}(x)$に対し或る$V \in \mathscr{U}(x)$が存在し,
							任意の$y \in V$に対し$W_y \subset V$を満たす$W_y \in \mathscr{U}(y)$が取れる.
					\end{description}
				\item[(2)]
					空でない集合$S$の各点$x$に対し(LB1)(LB2)(LB3)を満たす部分集合族$\mathscr{U}(x)$が与えられれば,
					\begin{align}
						\mathscr{O} \coloneqq
						\Set{O \subset S}{\mbox{$O = \emptyset$,或は任意の$x \in O$に対し
						$U \subset O$を満たす$U \in \mathscr{U}(x)$が存在する}}
					\end{align}
					により$S$に位相が定まり,$\{\mathscr{U}(x)\}_{x \in S}$は
					$(S,\mathscr{O})$において基本近傍系となる.
				\item[(3)] 空でない位相空間$(S,\mathscr{O})$から基本近傍系
					$\{\mathscr{U}(x)\}_{x \in S}$を得れば,
					$\{\mathscr{U}(x)\}_{x \in S}$を基本近傍系とする位相
					を(2)の手続きで構成することにより$\mathscr{O}$を復元できる.
			\end{description}
		\end{thm}
	\end{screen}
	
	\begin{prf}\mbox{}
		\begin{description}
			\item[(1)] 任意の$U \in \mathscr{U}(x)$は$x$の近傍であるから
				$(LB1)$が満たされる.また$U,V \in \mathscr{U}(x)$に対し
				\begin{align}
					x \in U^{\mathrm{o}} \cap V^{\mathrm{o}} = (U \cap V)^{\mathrm{o}}
				\end{align}
				となるから$U \cap V$は$x$の近傍であり(LB2)も従う.
				任意の$U \in \mathscr{U}(x)$に対し$V \coloneqq U^{\mathrm{o}}$とおけば,
				$V$は任意の$y \in V$の開近傍となるから(LB3)も得られる.
			
			\item[(2)] 
				$\mathscr{U}(x)$は空ではないから$S \in \mathscr{O}$となる.
				また$O_1,O_2 \in \mathscr{O}$を取れば,
				任意の$x \in O_1 \cap O_2$に対し
				\begin{align}
					x \in U_1 \subset O_1,
					\quad x \in U_2 \subset O_2
				\end{align}
				を満たす$U_1,U_2 \in \mathscr{U}(x)$が存在し,
				(LB2)より或る$U_3 \in \mathscr{U}(x)$に対して
				\begin{align}
					U_3 \subset U_1 \cap U_2 \subset O_1 \cap O_2
				\end{align}
				が成り立つから$O_1 \cap O_2 \in \mathscr{O}$となる.
				任意に$\mathscr{G} \subset \mathscr{O}$を取れば
				任意の$x \in \bigcup \mathscr{G}$は或る$G \in \mathscr{G}$の点であるから,
				\begin{align}
					U \subset G \subset \bigcup \mathscr{G}
				\end{align}
				を満たす$U \in \mathscr{U}(x)$が存在し$\bigcup \mathscr{G} \in \mathscr{O}$が従う.
				よって$\mathscr{O}$は位相である.
				また(LB3)の$V$は$\mathscr{O}$の元であり
				\begin{align}
					x \in V \subset U^{\mathrm{o}}
				\end{align}
				が成り立つから任意の$U \in \mathscr{U}(x)$は$x$の近傍である.
				そして$W$を$x$の任意の近傍とすれば,$\mathscr{O}$の定め方より或る$U \in \mathscr{U}(x)$が
				$U \subset W^{\mathrm{o}}$を満たすから$\mathscr{U}(x)$は$x$の基本近傍系である.
			
			\item[(3)] 
				定理\ref{thm:local_base_defines_open_sets}より
				$\{\mathscr{U}(x)\}_{x \in S}$を基本近傍系とする位相は唯一つであるから
				主張が従う.
				\QED
		\end{description}
	\end{prf}
	
	\begin{screen}
		\begin{dfn}[集積点・密集点]
			位相空間$S$の点$x$と部分集合$A$について,
			$x$の任意の近傍$U$に対し
			\begin{align}
				(U \backslash \{x\}) \cap A \neq \emptyset
			\end{align}
			となるとき,$x$は$A$の集積点(accumulation point)であるという.
			同様に$x$の任意の近傍$U$に対し
			\begin{align}
				U \cap A \neq \emptyset
			\end{align}
			となるとき,$x$は$A$の密集点(cluster point)であるという.
		\end{dfn}
	\end{screen}
	
	集積点と密集点の明確な違いは$T_1$空間(後述)において現れる.
	\begin{screen}
		\begin{thm}[閉である一点集合は集積点を持たない]
		\label{thm:closed_singleton_has_no_accumulation_point}
			位相空間において,閉じている一点集合は集積点を持たない.特に
			$\{x\}$が閉であるとき,$x$は$\{x\}$の密集点ではあるが集積点ではない.
		\end{thm}
	\end{screen}
	
	\begin{prf}
		一点集合$\{x\}$が閉であるとする.このとき$y \neq x$なら
		$U \coloneqq \{x\}^c$は$y$の開近傍となり
		\begin{align}
			(U \backslash \{y\}) \cap \{x\} = \emptyset
		\end{align}
		を満たすから$y$は$\{x\}$の集積点ではない.
		$x$は$\{x\}$の集積点となりえないから$\{x\}$は集積点を持たない.
		\QED
	\end{prf}
	
	\begin{screen}
		\begin{thm}[閉集合は密集点集合]
		\label{thm:belongs_to_closure_iff_clusters}
			位相空間$S$の点$x$と部分集合$A$について次が成り立つ:
			\begin{align}
				x \in \overline{A} \quad \Longleftrightarrow \quad
				\mbox{$x$は$A$の密集点である}.
			\end{align}
			特に,$A$が閉であることと$A$の密集点全体が$A$に一致することは同値になる.
		\end{thm}
	\end{screen}
	
	\begin{prf}
		$x$の或る近傍$U$が$U \cap A = \emptyset$を満たすとき,
		定理\ref{thm:topology_note_closure_interior}より
		\begin{align}
			x \in U^i \subset A^{ci} = A^{ac}
		\end{align}
		が成り立ち$x \notin \overline{A}$が従う.逆に
		$x \notin \overline{A}$のとき,
		$\overline{A}^c$は$A$と交わらない$x$の開近傍となる.
		\QED
	\end{prf}
	
	\begin{screen}
		\begin{thm}[$x \in \overline{A \backslash \{x\}}$$\Longleftrightarrow$$x$が$A$の集積点]
			位相空間$S$の点$x$と部分集合$A$について次が成り立つ:
			\begin{align}
				x \in \overline{A \backslash \{x\}} \quad \Longleftrightarrow \quad
				\mbox{$x$は$A$の集積点である}.
			\end{align}
		\end{thm}
	\end{screen}
	
	\begin{prf}
		$x$の任意の近傍$U$に対し
		$U \cap (A \backslash \{x\}) = (U \backslash \{x\}) \cap A$となるから,
		定理\ref{thm:belongs_to_closure_iff_clusters}と併せて
		\begin{align}
			x \in \overline{A \backslash \{x\}} 
			&\quad \Longleftrightarrow \quad
			\mbox{$x$の任意の近傍$U$に対し$U \cap (A \backslash \{x\}) \neq \emptyset$} \\
			&\quad \Longleftrightarrow \quad
			\mbox{$x$の任意の近傍$U$に対し$(U \backslash \{x\}) \cap A \neq \emptyset$}
			\quad \Longleftrightarrow \quad
			\mbox{$x$は$A$の集積点}
		\end{align}
		が成立する.
		\QED
	\end{prf}
	
	\begin{screen}
		\begin{dfn}[相対位相]
			$(S,\mathscr{O})$を位相空間,$M \subset S$を部分集合,
			$i:M \longrightarrow S$を恒等写像とするとき,
			\begin{align}
				\mathscr{O}_M \coloneqq 
				\Set{i^{-1}(O) = O \cap M}{O \in \mathscr{O}}
			\end{align}
			で定まる$i$による$\mathscr{O}$の引き戻しを$M$の相対位相(relative topology)と呼ぶ.
		\end{dfn}
	\end{screen}
	
	\begin{screen}
		\begin{dfn}[被覆・コンパクト・局所コンパクト・$\sigma$-コンパクト]\mbox{}
			\begin{description}
				\item[(1)]
					集合$S$の部分集合族$\mathscr{B}$が
					$S$の被覆(cover)であるとは,
					\begin{align}
						S = \bigcup \mathscr{B}
					\end{align}
					を満たすことをいう.
					$S$が位相空間であるとき,任意の開被覆(開集合から成る被覆)$\mathscr{B}$に対し
					\begin{align}
						S = \bigcup \mathscr{B}'
					\end{align}
					を満たす有限部分集合$\mathscr{B}' \subset \mathscr{B}$が取れるとき,
					$S$はコンパクトである(compact)という.
					$S$の部分集合$A$がコンパクトであるとは,
					$A$がその相対位相でコンパクト位相空間となることを指す.
				
				\item[(2)] 位相空間の任意の点がコンパクトな近傍を持つとき,
					その空間は局所コンパクト(locally compact)であるという.
					
				\item[(3)] 位相空間においてコンパクト集合の可算被覆が存在するとき,
					その空間は$\sigma$-コンパクトであるという.
			\end{description}
		\end{dfn}
	\end{screen}
	
	\begin{screen}
		\begin{thm}[コンパクト部分集合]
		\end{thm}
	\end{screen}
	
	\begin{screen}
		\begin{dfn}[連続・同相・開写像]
			$f$を位相空間$S$から位相空間$T$への写像とする.
			\begin{description}
				\item[(1)]
					$x \in S$において$f(x)$の任意の近傍$U$に対し
					$f^{-1}(U)$が$x$の近傍となるとき,
					$f$は$x$で連続である(continuous)という.
					
				\item[(2)] $T$の任意の開集合$O$に対し$f^{-1}(O)$が$S$の開集合となるとき,
					$f$は連続であるという.
					
				\item[(3)] $f$に逆写像$f^{-1}$が存在し,$f,f^{-1}$が共に連続であるとき,
					$f$を同相写像(homeomorphism),或は位相同型と呼び,
					$S,T$間に同相写像が存在するとき$S$と$T$は同相である,或は位相同型であるという.
					
				\item[(4)] $S$の任意の開集合の$f$による像が$T$の開集合であるとき,
					$f$を開写像(open mapping)と呼ぶ.
			\end{description}
		\end{dfn}
	\end{screen}
	
	\begin{screen}
		\begin{thm}[各点連続$\Longleftrightarrow$連続]
			$f$を位相空間$S$から位相空間$T$への写像とするとき次が成り立つ:
			\begin{align}
				\mbox{$f$が連続} \quad \Longleftrightarrow \quad
				\mbox{$f$が$S$の各点で連続}.
			\end{align}
		\end{thm}
	\end{screen}
	
	\begin{prf}
		$f$が連続であるとき,各点$x \in S$で$f(x)$の任意の近傍$U$に対し
		$f(x) \in U^{\mathrm{o}}$が満たされるから
		$f^{-1}(U^{\mathrm{o}})$は$x$を含む開集合となる.
		$f^{-1}(U^{\mathrm{o}})$は$f^{-1}(U)$に含まれる開集合であるから
		\begin{align}
			x \in f^{-1}(U^{\mathrm{o}}) \subset f^{-1}(U)^{\mathrm{o}}
		\end{align}
		が成り立ち,従って$f$は$x$で連続である.
		逆に$f$が各点連続であるとき,
		$T$の任意の開集合$O$に対し
		$f^{-1}(O)$は任意の$x \in f^{-1}(O)$の近傍となるから
		定理\ref{thm:local_base_defines_open_sets}より
		$f^{-1}(O)$は開集合である.よって$f$は連続である.
		\QED
	\end{prf}
	
	\begin{screen}
		\begin{thm}[部分空間と制限写像の連続性]
			$S,T$を位相空間,$f$を$S$から$T$への写像とする.
			また$U \coloneqq f(S)$として$f'$を
			$S$から$U$への$f$の制限とする.このとき次が成り立つ:
			\begin{align}
				\mbox{$f:S \longrightarrow T$が連続である} 
				\quad \Longleftrightarrow \quad
				\mbox{$f':S \longrightarrow U$が($U$の相対位相に関して)連続である}.
			\end{align}
		\end{thm}
	\end{screen}
	
	\begin{prf}
		$T$の任意の開集合$O$に対し
		\begin{align}
			f'^{-1}(U \cap O) = f^{-1}(U \cap O) = f^{-1}(O)
		\end{align}
		が成り立つから,$f$と$f'$の連続性は一致する.
		\QED
	\end{prf}
	
	\begin{screen}
		\begin{thm}[位相の生成]
			$S$を集合,$\mathcal{P}(S)$を冪集合として
			任意に$M \subset \mathcal{P}(S)$を取り
			\begin{align}
				\mathscr{A} \coloneqq
				\Set{\bigcap_{i=1}^n I_i}{I_i \in M,\ n = 1,2,\cdots}
			\end{align}
			とおくとき,$M$を含む最小の位相は
			\begin{align}
				\mathscr{O} \coloneqq
				\Set{\bigcup \Lambda}{\Lambda \subset \mathscr{A}}
				\cup \{S\}
			\end{align}
			で与えられる.この$\mathscr{O}$を$M$が生成する$S$の位相と呼ぶ.
		\end{thm}
	\end{screen}
	
	\begin{prf}
		$\mathscr{O}$は定め方より$S$と$\emptyset$を含む.また
		任意の$O_1 = \bigcup \Lambda_1,\ O_2=\bigcup \Lambda_2 \in \mathscr{O},\ 
		(\Lambda_1,\Lambda_2 \subset \mathscr{A})$に対し
		\begin{align}
			\Set{I \cap J}{I \in \Lambda_1,\ J \in \Lambda_2} \subset \mathscr{A}
		\end{align}
		となるから
		\begin{align}
			O_1 \cap O_2 = \bigcup_{I \in \Lambda_1,\ J \in \Lambda_2} I \cap J \in \mathscr{O}
		\end{align}
		が成立する.任意に$\emptyset \neq \mathscr{U} \subset \mathscr{O}$を取れば,
		各$U \in \mathscr{U}$に$U = \bigcup \Lambda_U$を満たす
		$\Lambda_U \subset \mathscr{A}$が対応し,このとき
		\begin{align}
			\bigcup_{U \in \mathscr{U}} \Lambda_U \subset \mathscr{A}
		\end{align}
		となるから
		\begin{align}
			\bigcup \mathscr{U} = \bigcup \Biggl(\bigcup_{U \in \mathscr{U}} \Lambda_U\Biggr)
			\in \mathscr{O}
		\end{align}
		が従う.$M$を含む任意の位相は$\mathscr{A}$を含みかつその任意和で閉じるから$\mathscr{O}$を含む.
		\QED
	\end{prf}
	
	\begin{screen}
		\begin{dfn}[始位相]
			$f \in \mathscr{F}$を集合$S$から位相空間$(T_f,\mathscr{O}_f)$への写像とするとき,
			全ての$f \in \mathscr{F}$を連続にする最弱の位相を$S$の$\mathscr{F}$-始位相
			(initial topology)と呼ぶ.$\mathscr{F}$-始位相は次が生成する位相である:
			\begin{align}
				\bigcup_{f \in \mathscr{F}} \Set{f^{-1}(O)}{O \in \mathscr{O}_f}.
			\end{align}
		\end{dfn}
	\end{screen}
	
\subsection{分離公理}
	\begin{screen}
		\begin{dfn}[位相的に識別可能・分離]
			$S$を位相空間とする.
			\begin{itemize}
				\item $x,y \in S$に対し$x \notin \overline{\{y\}}$
					或は$y \notin \overline{\{x\}}$が満たされるとき,
					$x$と$y$は位相的に識別可能(topologically distinguishable)であるという.
				\item $A,B \subset S$に対し$\overline{A} \cap B = \emptyset$
					或は$A \cap \overline{B} = \emptyset$が満たされるとき,
					$A$と$B$は分離される(separeted)という.点と点,点と集合の分離は一点集合を考える.
				\item $A,B \subset S$が近傍で分離される(separated by neighborhoods)とは,
					$A,B$が互いに交わらない近傍を持つことをいう.
				\item 閉集合$A,B \subset S$が関数で分離される(separated by a function)とは,
					或る連続関数$f:S \longrightarrow [0,1]$によって$f(A) = \{0\},\ f(B) = \{1\}$
					が満たされることをいう.
				\item 閉集合$A,B \subset S$が関数でちょうど分離される
					(precisely separated by a function)とは,
					或る連続関数$f:S \longrightarrow [0,1]$によって
					$A = f^{-1}(\{0\}),\ B = f^{-1}(\{1\})$が満たされることをいう.
			\end{itemize}
		\end{dfn}
	\end{screen}
	
	\begin{screen}
		\begin{thm}[位相的に識別可能な二点は相異なる]
			$S$を位相空間とするとき,任意の$x,y \in S$に対し
			\begin{align}
				\mbox{$x$と$y$が位相的に識別可能} \quad \Longrightarrow \quad
				x \neq y .
			\end{align}
		\end{thm}
	\end{screen}
	
	\begin{prf}
		$x = y$なら$\overline{\{x\}} = \overline{\{y\}}$となる.
		後述の$T_0$空間とは,この逆が満たされる位相空間である.
		\QED
	\end{prf}
	
	\begin{screen}
		\begin{thm}[分離される集合は他方を含まない近傍を持つ]
		\label{thm:the_equivalent_condition_of_separatedness}
			位相空間$S$において,$A,B \subset S$が分離されることと
			\begin{align}
				A \subset U,\quad B \subset V,\quad 
				A \cap V = \emptyset,
				\quad B \cap U = \emptyset
				\label{eq:thm_the_equivalent_condition_of_separatedness}
			\end{align}
			を満たす開集合$U,V$が存在することは同値である.
		\end{thm}
	\end{screen}
	
	\begin{prf}
		$A,B \subset S$が分離されるとき,$U \coloneqq \overline{B}^c,\ V \coloneqq \overline{A}^c$
		とおけば(\refeq{eq:thm_the_equivalent_condition_of_separatedness})が成立する.
		逆に$A,B$に対し(\refeq{eq:thm_the_equivalent_condition_of_separatedness})を満たす
		開集合$U,V$が存在するとき,$\closure{A} \subset V^c \subset B^c$及び
		$\closure{B} \subset U^c \subset A^c$となるから$A,B$は分離される.
		\QED
	\end{prf}
	
	\begin{screen}
		\begin{dfn}[分離公理]\mbox{}
			\begin{itemize}
				\item 任意の二点が位相的に識別可能である位相空間を$T_0$空間,或はKolmogorov空間という.
				\item 任意の二点が分離される位相空間を$T_1$空間という.
				\item 任意の二点が近傍で分離される位相空間を$T_2$空間,或はHausdorff空間という.
				\item 任意の交わらない点と閉集合が近傍で分離される位相空間を
					正則(regular)空間という.
				\item $T_0$かつ正則な位相空間を$T_3$空間,或は正則Hausdorff空間という.
				\item 任意の交わらない二つの閉集合が近傍で分離される位相空間を正規(normal)空間という.
				\item $T_1$かつ正規な位相空間を$T_4$空間,或は正規Hausdorff空間という.
				\item 任意の部分位相空間が正規である位相空間は全部分正規(completely normal)であるという.
				\item $T_1$かつ全部分正規な位相空間を$T_5$空間,或は全部分正規Hausdorff空間という.
				\item 任意の交わらない二つの閉集合が関数でちょうど分離される位相空間は完全正規(perfectly normal)であるという.
				\item $T_1$かつ完全正規な位相空間を$T_6$空間,或は完全正規Hausdorff空間という.
			\end{itemize}
		\end{dfn}
	\end{screen}
	
	\begin{screen}
		\begin{thm}[$T_1$空間とは一点集合が閉である空間]
			位相空間$S$に対し,
			\begin{align}
				\mbox{$S$が$T_1$}
				&\quad \Longleftrightarrow \quad \mbox{$S$は$T_0$かつ位相的に識別可能な任意の二点が分離される} \\
				&\quad \Longleftrightarrow \quad \mbox{$S$の任意の一点集合は閉} \\
				&\quad \Longleftrightarrow \quad \mbox{$x \in S$が$A \subset S$の集積点であることと$x$の任意の開近傍が$A$と交わることは同値}.
			\end{align}
		\end{thm}
	\end{screen}
	
	\begin{prf}
		$x$が$A$の集積点であるとき,任意に$x$の近傍$U$を取る.
		いま,$x$の或る開近傍$U_{n-1}$と$x_{n-1} \in U_{n-1},\ (x \neq x_{n-1})$
		が取れたとして,
		\begin{align}
			U_n \coloneqq U_{n-1} \cap (S \backslash \{x_{n-1}\})
		\end{align}
		は$x$の開近傍となり或る$x_n \in (U_{n-1} \backslash \{x\}) \cap A$が取れる.
		$U_0 \coloneqq U^{\mathrm{o}},\ 
		x_0 \in (U^{\mathrm{o}} \backslash \{x\}) \cap A$を出発点とすれば
		$A$は$U$の無限集合$\{x_n\}_{n=1}^\infty$を含む.
	\end{prf}
	
	\begin{screen}
		\begin{thm}[Hausdorff空間のコンパクト部分集合は閉]
			Hausdorff空間のコンパクト部分集合は閉である.
		\end{thm}
	\end{screen}
	
	\begin{prf}
		$S$をHausdorff空間,$K \subset S$をコンパクト部分集合とするとき,
		任意に$x \in S \backslash K,\ y \in K$を取れば
		\begin{align}
			x \in U_y,\quad y \in V_y, \quad U_y \cap V_y = \emptyset
		\end{align}
		を満たす開集合$U_y,V_y$が取れる.或る$\{y_i\}_{i=1}^n \subset K$に対し
		$K \subset \bigcup_{i=1}^n V_{y_i}$となるから,
		$U \coloneqq \bigcap_{i=1}^n U_{y_i}$とおけば
		\begin{align}
			x \in U,\quad U \subset \bigcap_{i=1}^n \left(S\backslash V_{y_i}\right)
			\subset S \backslash K
		\end{align}
		が成立する.従って$S \backslash K$は開集合であり,$K$は閉集合である.
		\QED
	\end{prf}
	
	\begin{screen}
		\begin{thm}[Hausdorff空間においてコンパクト集合の閉部分集合はコンパクト]
			$S$をHausdorff空間,$K \subset S$をコンパクト部分集合,$F \subset S$を閉集合とするとき,
			$K \cap F$はコンパクトである.
		\end{thm}
	\end{screen}
	
	\begin{prf}
		$K \cap F$の任意の開被覆に$S \backslash F$を加えれば
		$K$の開被覆となるから,そのうち$K$の有限被覆を取ることができる.
		$S \backslash F$を除けば$K \cap F$の有限被覆が残り
		$K \cap F$のコンパクト性が出る.
		\QED
	\end{prf}
	
	\begin{screen}
		\begin{thm}[Hausdorff空間とは交わらない二つのコンパクト集合が近傍で分離される空間]
		\label{thm:Hausdorff_space_two_disjoint_compact_sets_are_separated_by_nbh}
			位相空間において,Hausdorff性と,交わらない二つのコンパクト集合が近傍で分離されることは同値である.
		\end{thm}
	\end{screen}
	
	\begin{prf}
		$A,B$をHausdorff空間の交わらないコンパクト集合とするとき,
		任意の$p \in A$に対し
		\begin{align}
			p \in U_p,\quad B \subset V_p,\quad U_p \cap V_p = \emptyset
			\label{eq:thm_Hausdorff_space_two_disjoint_compact_sets_are_separated_by_nbh_1}
		\end{align}
		を満たす開集合$U_p,V_p$が存在する.実際
		任意の$q \in B$に対し
		\begin{align}
			p \in U_p(q),\quad q \in V_p(q),\quad U_p(q) \cap U_p(q) = \emptyset
		\end{align}
		を満たす開集合$U_p(q), U_p(q)$が取れ,$B$のコンパクト性より
		或る$\{q_i\}_{i=1}^n \subset B$で$B \subset \bigcup_{i=1}^n U_p(q_i)$となるから,
		\begin{align}
			U_p \coloneqq \bigcap_{i=1}^n U_p(q_i),
			\quad V_p \coloneqq \bigcup_{i=1}^n V_p(q_i)
		\end{align}
		とおけば(\refeq{eq:thm_Hausdorff_space_two_disjoint_compact_sets_are_separated_by_nbh_1})
		が成立する.$A$のコンパクト性より或る$\{p_j\}_{j=1}^m \subset A$で
		$A \subset \bigcup_{j=1}^m U_{p_j}$となるから,
		\begin{align}
			U \coloneqq \bigcup_{j=1}^m U_{p_j},
			\quad V \coloneqq \bigcap_{j=1}^m V_{p_j}
		\end{align}
		とおけば$A$と$B$は$U,V$により分離される.
		逆の主張は一点集合がコンパクトであることより従う.
		\QED
	\end{prf}
	
	\begin{screen}
		\begin{thm}[Hausdorff空間値連続写像の等価域は閉]
			$S$を位相空間,$T$をHausdorff空間,$f,g$を
			$S$から$T$への連続写像とするとき,$E \coloneqq \Set{x \in S}{f(x) = g(x)}$は$S$で閉じている.
			特に,$E$が$X$で稠密なら$f=g$となる.
		\end{thm}
	\end{screen}
	
	\begin{prf}
		任意に$x \in \Set{x \in S}{f(x) \neq g(x)}$を取れば,Hausdorff性より
		\begin{align}
			f(x) \in A,\quad g(x) \in B,\quad A \cap B = \emptyset
		\end{align}
		を満たす$T$の開集合$A,B$が存在する.
		$f^{-1}(A) \cap g^{-1}(B)$は$x$の開近傍であり,
		\begin{align}
			f^{-1}(A) \cap g^{-1}(B) \subset \Set{x \in S}{f(x) \neq g(x)}
		\end{align}
		となるから$\Set{x \in S}{f(x) \neq g(x)}$は$S$の開集合である.
		従って$E$は閉である.
		\QED
	\end{prf}
	
	\begin{screen}
		\begin{thm}[正則空間とは交わらないコンパクト集合と閉集合が近傍で分離できる空間]
		\label{thm:each_point_in_regular_space_has_closesd_local_base}\mbox{}
			\begin{description}
				\item[(1)] 位相空間において,正則性と,交わらないコンパクト集合と閉集合が近傍で分離されることは同値である.
					
				\item[(2)]
					$K,W$をそれぞれ局所コンパクトな正則空間のコンパクト集合,開集合とするとき,
					閉包がコンパクトな開集合$U$が存在して次を満たす:
					\begin{align}
						K \subset U \subset \overline{U} \subset W.
						\label{eq:thm_each_point_in_regular_space_has_closesd_local_base}
					\end{align}
			\end{description}
		\end{thm}
	\end{screen}
	
	\begin{prf}\mbox{}
		\begin{description}
			\item[(1)]
				$K,F$を正則空間のコンパクト集合,閉集合とするとき,
				$K \cap F = \emptyset$なら任意の点$x \in K$に対して
				\begin{align}
					x \in U_x,\ \quad F \subset V_x,
					\quad U_x \cap V_x = \emptyset
				\end{align}
				を満たす開集合$U_x,V_x$が取れる.
				$K$はコンパクトであるから或る$\{x_i\}_{i=1}^n \subset K$で
				$K \subset \bigcup_{i=1}^n U_{x_i}$となり
				\begin{align}
					K \subset U \coloneqq \bigcup_{i=1}^n U_{x_i},
					\quad F \subset V \coloneqq \bigcap_{i=1}^n V_{x_i},
					\quad U \cap V = \emptyset
				\end{align}
				が成立する.逆の主張は一点集合がコンパクトであることにより従う.
			\item[(2)]
				任意の$x \in K$に対し,$\overline{U_x} \subset W$
				となる開近傍$U_x$と閉包がコンパクトな開近傍$C_x$が存在するから,
				\begin{align}
					K \subset (C_{y_1} \cap U_{y_1}) \cup \cdots \cup (C_{y_m} \cap U_{y_m})
				\end{align}
				を満たす$\{y_i\}_{i=1}^m \subset K$に対し
				$U \coloneqq \bigcup_{i=1}^m C_{y_i} \cap U_{y_i}$
				とおけば,$\overline{U}$はコンパクトであり
				(\refeq{eq:thm_each_point_in_regular_space_has_closesd_local_base})を満たす.
				\QED
		\end{description}
	\end{prf}
	
	\begin{screen}
		\begin{thm}[局所コンパクトなら$T_2$と$T_3$は同値]
		\label{thm:T_2_equals_to_T_3_in_locally_compact_spaces}
			局所コンパクト位相空間において,$T_2 \Longleftrightarrow T_3$である.
		\end{thm}
	\end{screen}
	
	\begin{prf}
		$T_3$ならば$T_2$であるから$\Longleftarrow$を得る.
		逆に$S$を局所コンパクトHausdorff空間とし,点$x$と閉集合$F$が$x \notin F$を満たしているとする.
		$x$のコンパクトな近傍$K$を取れば,Hausdorff性より$K \cap F$はコンパクトであるから
		\begin{align}
			U_0 \cap V_0 = \emptyset, \quad x \in U_0,  \quad K \cap F \subset V_0
		\end{align}
		を満たす開集合$U_0,V_0$が存在する.このとき,
		\begin{align}
			U \coloneqq U_0 \cap K^{\mathrm{o}},
			\quad V \coloneqq V_0 \cup (S \backslash K)
		\end{align}
		により開集合$U,V$を定めれば
		\begin{align}
			U \cap V = \emptyset,
			\quad x \in U,
			\quad F \subset V
		\end{align}
		が成立し,$S$の正則性が出る.$S$は$T_0$空間でもあるから$T_3$である.
		\QED
	\end{prf}
	
	\begin{screen}
		\begin{thm}[正規空間とは交わらない二つの閉集合が関数で分離される空間(Urysohnの補題)]
		\label{thm:Urysohn_lemma}
			位相空間において,正規性と,任意の交わらない二つの閉集合が関数で分離されることは同値である.
		\end{thm}
	\end{screen}
	
	\begin{screen}
		\begin{dfn}[$G_\delta$集合・$F_\sigma$集合]
			位相空間の部分集合で,開集合の可算交叉で表されるものを$G_\delta$集合,
			閉集合の可算和で表されるものを$F_\sigma$集合と呼ぶ.
			特に,任意の閉集合が$G_\delta$である空間では任意の開集合が$F_\sigma$となる.
		\end{dfn}
	\end{screen}
	
	\begin{screen}
		\begin{thm}[完全正規空間とは正規かつ閉集合が全て$G_\delta$である空間]
		\label{thm:perfectly_normal_Hausdorff_is_normal_and_closed_is_G_delta}\mbox{}
			\begin{description}
				\item[(1)]
					$F$を完全正規空間の閉集合とすれば,次を満たす閉集合系$(F_n)_{n=1}^\infty$が存在する:
					\begin{align}
						F = \bigcap_{n=1}^\infty F_n,
						\quad F_n^{\mathrm{o}} \supset F_{n+1}. 
					\end{align}
					
				\item[(2)]
					位相空間において,完全正規であることと,正規かつ任意の閉集合が$G_\delta$であることは同値である.
			\end{description}
		\end{thm}
	\end{screen}
	
	\begin{prf}
		$S$を完全正規空間,$A,B$を互いに交わらない$S$の閉集合とすれば,
		$A=f^{-1}(\{0\}),\ B = f^{-1}(\{1\})$を満たす連続関数
		$f:S \longrightarrow \R$が存在する.このとき
		$U \coloneqq f^{-1}([0,1/2)),\ V \coloneqq f^{-1}((1/2,1])$
		で開集合$U,V$を定めれば
		\begin{align}
			A \subset U,\quad B \subset V,\quad U \cap V = \emptyset
		\end{align}
		となるから$S$は正規である.また$F$を閉集合とすれば
		或る連続関数$g:S \longrightarrow \R,\ (\emptyset = g^{-1}(\{1\}))$により
		\begin{align}
			F = g^{-1}(\{0\}) 
			= g^{-1}\Biggl(\bigcap_{n=1}^\infty\left[0,n^{-1}\right)\Biggr)
			= \bigcap_{n=1}^\infty g^{-1}\left(\left[0,n^{-1}\right)\right)
		\end{align}
		が成立するから$F$は$G_\delta$である.特に,このとき
		$F_n \coloneqq g^{-1}\left(\left[0,n^{-1}\right]\right)$とおけば
		\begin{align}
			F = \bigcap_{n=1}^\infty g^{-1}\left(\left[0,n^{-1}\right]\right)
			= \bigcap_{n=1}^\infty F_n,
			\quad F_n^{\mathrm{o}} \supset g^{-1}\left(\left[0,n^{-1}\right)\right)
			\supset g^{-1}\left(\left[0,(n+1)^{-1}\right]\right)
			= F_{n+1}
		\end{align}
		となり(1)の主張が得られる.逆に$S$が正規かつ
		閉集合が全て$G_\delta$であるとき,任意の交わらない閉集合$A,B$に対し
		$A = \bigcap_{n=1}^\infty U_n,\ B = \bigcap_{n=1}^\infty V_n$
		を満たす開集合系$(U_n)_{n=1}^\infty,\ (V_n)_{n=1}^\infty$が取れて,
		定理\ref{thm:Urysohn_lemma}より各$n \geq 1$で
		\begin{align}
			f_n(A) = \{0\},\quad f_n(S \backslash U_n) = \{1\},
			\quad g_n(B) = \{0\},\quad g_n(S \backslash V_n) = \{1\}
		\end{align}
		を満たす連続写像$f_n,g_n:S \longrightarrow [0,1]$が存在する.
		ここで連続写像を$f \coloneqq \sum_{n=1}^\infty 2^{-n} f_n,\ 
		g \coloneqq \sum_{n=1}^\infty 2^{-n} g_n$で定めれば
		\begin{align}
			\begin{cases}
				f(x) = 0, & (x \in A), \\
				f(x) > 0, & (x \notin A),
			\end{cases}
			\quad \begin{cases}
				g(x) = 0, & (x \in B), \\
				g(x) > 0, & (x \notin B),
			\end{cases}
		\end{align}
		となり,$h \coloneqq f/(f+g)$とおけば$A = h^{-1}(\{0\}),\ B = h^{-1}(\{1\})$が成立する.
		従って$S$は完全正規である.
		\QED
	\end{prf}
	
	\begin{screen}
		\begin{thm}[連続な単射の引き戻しによる分離性の遺伝]
			$S,T$を位相空間とする.$S$から$T$への連続単射が存在するとき,
			$T$が$T_k$-空間$(k=0,1,\cdots,6)$なら
			$S$もまた$T_k$-空間となる.
		\end{thm}
	\end{screen}
	
	\begin{prf}
		任意に異なる二点$s_1,s_2 \in S$を取れば単射性より$f(s_1) \neq f(s_2)$となる.
		$T$の分離性より
	\end{prf}
	
\subsection{可算公理}
	\begin{screen}
		\begin{thm}[可算コンパクト性の同値条件]
		\end{thm}
	\end{screen}
	
	\begin{screen}
		\begin{dfn}[開基]
			位相空間$(S,\mathscr{O})$において,
			$\mathscr{O}$の部分集合$\mathscr{B}$で
			\begin{align}
				\mathscr{O}
				= \Set{\bigcup \mathscr{U}}{\mathscr{U} \subset \mathscr{B}}
			\end{align}
			を満たすもの,ただし$\bigcup \emptyset = \emptyset$,
			を$\mathscr{O}$の開基や基底,基(base)と呼ぶ.基底は一意に定まるものではない.
			$S \neq \emptyset$のときは$\mathscr{B}$の任意の元は空でないとする.
		\end{dfn}
	\end{screen}
	
	\begin{screen}
		\begin{dfn}[可算公理]
			位相空間$S$において,任意の点が高々可算な基本近傍系を持つとき
			$S$は第一可算公理(the first axiom of countability)を満たす,或は
			$S$は第一可算であるといい,
			$S$が高々可算な基底を持つとき
			$S$は第二可算公理(the second axiom of countability)を満たす,或は
			$S$は第二可算であるという.
		\end{dfn}
	\end{screen}
	空集合(要素数0)を含む任意の有限位相空間は,その冪集合が有限集合であるから
	第二可算公理を満たす.
	
	\begin{screen}
		\begin{thm}[第二可算なら第一可算]
			空でない第二可算空間は第一可算である.
		\end{thm}
	\end{screen}
	
	\begin{prf}
		$\mathscr{B}$を空でない第二可算空間$S$の可算基とするとき,任意の$x \in S$に対して
		\begin{align}
			\mathscr{U}(x) \coloneqq
			\Set{B \in \mathscr{B}}{x \in B}
		\end{align}
		で可算な基本近傍系が定まる.実際
		$x$の任意の近傍$U$に対し或る$B \in \mathscr{B}$で
		\begin{align}
			x \in B \subset U^{\mathrm{o}}
		\end{align}
		が成立し,定義より$B \in \mathscr{U}(x)$が満たされる.
		\QED
	\end{prf}
	
	\begin{screen}
		\begin{dfn}[稠密・可分]
			位相空間$S$において,$\overline{M} = S$を満たすような部分集合$M$を
			$S$で稠密な(dense)部分集合と呼ぶ.
			また高々可算かつ稠密な部分集合$M$が存在するとき$S$は可分である(separable)という.
		\end{dfn}
	\end{screen}
	
	\begin{screen}
		\begin{thm}[第二可算なら可分]\label{thm:second_countable_then_separable}
			第二可算位相空間は可分である.
		\end{thm}
	\end{screen}
	
	\begin{prf}
		$\mathscr{B}$を第二可算空間$S$の可算基とするとき,
		$S = \emptyset$なら$\emptyset$は$S$の唯一の部分集合であり,
		要素数$0$かつ$\overline{\emptyset} = \emptyset = S$を満たすから
		$S$は可分である.$S \neq \emptyset$のとき,
		選択関数$\Phi \in \prod \mathscr{B} = \prod_{B \in \mathscr{B}} B$を取り
		\begin{align}
			M \coloneqq \Set{\Phi(B)}{B \in \mathscr{B}}
		\end{align}
		で可算集合を定めれば,任意の$x \in S$及び$x$の任意の近傍$U$に対し
		$x \in B \subset U^{\mathrm{o}}$を満たす
		$B \in \mathscr{B}$が存在して
		\begin{align}
			\Phi(B) \in B \cap M \subset U \cap M
		\end{align}
		となるから,定理\ref{thm:belongs_to_closure_iff_clusters}より
		$S = \overline{M}$が成立する.
		\QED
	\end{prf}
	
	\begin{screen}
		\begin{dfn}[局所有限]
			位相空間$S$の部分集合族$\mathscr{F}$が局所有限
			\index{きょくしょゆうげん@局所有限}
			(locally finite)であるとは,
			任意の$x \in S$が$\mathscr{F}$の高々有限個の元としか交叉しない近傍を持つことである.
			局所有限な部分集合族の可算和で表される部分集合族を$\sigma$-局所有限な族という.
		\end{dfn}
	\end{screen}
	
	\begin{screen}
		\begin{thm}[$\sigma$-局所有限な基底が存在すれば第一可算]
			$\sigma$-局所有限な基底が存在する空でない位相空間は第一可算である.
		\end{thm}
	\end{screen}
	
	\begin{prf}
		$S$を空でない位相空間,$\mathscr{B} = \bigcup_{n=1}^\infty \mathscr{B}_n$を
		$\sigma$-局所有限な基底とする(各$\mathscr{B}_n$は局所有限).
		任意の$x \in S$で
		\begin{align}
			\mathscr{U}_n(x) \coloneqq \Set{B \in \mathscr{B}_n}{x \in B},
			\quad \mathscr{U}(x) \coloneqq \bigcup_{n=1}^\infty \mathscr{U}_n(x)
		\end{align}
		と定めれば,局所有限性より$\mathscr{U}_n(x)$は有限集合であるから
		$\mathscr{U}(x)$は可算集合である.また$x$の任意の近傍$U$に対し
		\begin{align}
			x \in B \subset U^{\mathrm{o}}
		\end{align}
		を満たす$B \in \mathscr{B}$が存在し,定義より$B \in \mathscr{U}(x)$
		が成り立つから$\mathscr{U}(x)$は$x$の可算な基本近傍系をなす.
		\QED
	\end{prf}
	
	\begin{screen}
		\begin{thm}[可分空間の局所有限な開集合族は高々有限集合]
		\label{thm:locally_finite_family_of_open_sets_is_countable_in_separable_space}
			$S$を空でない可分位相空間,
			$M$を$S$で稠密な高々可算集合,$\mathscr{B}$を
			$S$の空でない開集合から成る局所有限な族とするとき,
			\begin{align}
				\mathscr{B} = \bigcup_{m \in M} \Set{B \in \mathscr{B}}{m \in B}
			\end{align}
			が成立する.すなわち$\mathscr{B}$は高々可算集合である.
		\end{thm}
	\end{screen}
	
	\begin{prf}
		局所有限性より各$m \in M$で$\Set{B \in \mathscr{B}}{m \in B}$は有限集合であるから,
		\begin{align}
			\mathscr{U} \coloneqq \bigcup_{m \in M} \Set{B \in \mathscr{B}}{m \in B}
		\end{align}
		で高々可算集合が定まる.稠密性より任意の$E \in \mathscr{B}$は
		$E \cap M \neq \emptyset$を満たすから,$m \in E \cap M$に対し
		\begin{align}
			E \in \Set{B \in \mathscr{B}}{m \in B}
		\end{align}
		となり$\mathscr{B} = \mathscr{U}$が従う.
		\QED	
	\end{prf}
	
	\begin{screen}
		\begin{thm}[$\sigma$-局所有限な基底が存在すれば,可分$\Longleftrightarrow$第二可算]
			$\sigma$-局所有限な基底が存在する空でない位相空間において,
			可分性と第二可算性は一致する.
		\end{thm}
	\end{screen}
	
	\begin{prf}
		空でない可分位相空間において$\sigma$-局所有限な基底が存在するとき,
		定理\ref{thm:locally_finite_family_of_open_sets_is_countable_in_separable_space}
		よりその基底は高々可算集合であるから第二可算性が満たされる.
		逆に第二可算なら可分であるから定理の主張を得る.
		\QED
	\end{prf}
	
	\begin{screen}
		\begin{thm}[第二可算空間の任意の基底は可算基を内包する]\label{thm:countable_base_of_second_countable_space}
			$\mathscr{B}$を第二可算空間$S$の任意の基底とするとき,或る可算部分集合
			$\mathscr{B}_0 \subset \mathscr{B}$もまた$S$の基底となる.
			すなわち第二可算空間はLindel\Ddot{o}f性を持つ.
		\end{thm}
	\end{screen}
	
	\begin{prf}
		$\mathscr{D}$を$S$の可算基とする.
		任意の開集合$U$に対し或る$\mathscr{B}_U \subset \mathscr{B}$が存在して
		$U = \bigcup_{V \in \mathscr{B}_U}V$を満たすから,
		\begin{align}
			\mathscr{D}_U \coloneqq
			\Set{W \in \mathscr{D}}{W \subset V,\ V \in \mathscr{B}_U}
			\label{eq:thm_countable_base_of_second_countable_space_1}
		\end{align}
		とおけば$U = \bigcup_{V \in \mathscr{B}_U} V
			= \bigcup_{V \in \mathscr{B}_U} \bigcup_{\substack{W \in \mathscr{D}_U \\ W \subset V}} W
			\subset \bigcup_{W \in \mathscr{D}_U} W
			\subset U$より
		\begin{align}
			U = \bigcup_{W \in \mathscr{D}_U} W
			\label{eq:thm_countable_base_of_second_countable_space_2}
		\end{align}
		が成り立つ.ここで(\refeq{eq:thm_countable_base_of_second_countable_space_1})より
		任意の$W \in \mathscr{D}_U$に対して
		$\Set{V \in \mathscr{B}}{W \subset V} \neq \emptyset$であるから
		\begin{align}
			\Phi_U \in \prod_{W \in \mathscr{D}_U} \Set{V \in \mathscr{B}}{W \subset V}
		\end{align}
		が取れる.$\mathscr{B}_U' \coloneqq \Set{\Phi_U(W)}{W \in \mathscr{D}_U}$とすれば
		$U = \bigcup_{W \in \mathscr{D}_U} W \subset \bigcup_{W \in \mathscr{D}_U} \Phi(W)
		\subset \bigcup_{V \in \mathscr{B}_U'} V \subset U$より
		\begin{align}
			U = \bigcup_{V \in \mathscr{B}_U'} V
			\label{eq:thm_countable_base_of_second_countable_space_3}
		\end{align}
		が満たされ,
		\begin{align}
			\mathscr{B}_0 \coloneqq \bigcup_{W \in \mathscr{D}} \mathscr{B}_W'
		\end{align}
		と定めれば$\mathscr{B}_0$は求める$S$の可算基となる.実際,任意の開集合$U$に対し
		(\refeq{eq:thm_countable_base_of_second_countable_space_2})と
		(\refeq{eq:thm_countable_base_of_second_countable_space_3})より
		\begin{align}
			U = \bigcup_{W \in \mathscr{D}_U} W
			= \bigcup_{W \in \mathscr{D}_U} \bigcup_{V \in \mathscr{B}_W'} V
		\end{align}
		となる.
		\QED
	\end{prf}
	
	\begin{screen}
		\begin{thm}[局所コンパクトHausdorff空間が第二可算なら$\sigma$-コンパクト]\label{thm:second_countable_Hausdorff_sigma_compact}
			$S$が第二可算性をもつ局所コンパクトHausdorff空間なら,
			次を満たすコンパクト部分集合の列$(K_n)_{n=1}^\infty$が存在する:
			\begin{align}
				K_n \subset K_{n+1}^{\mathrm{o}},
				\quad S = \bigcup_{n=1}^\infty K_n.
			\end{align}
		\end{thm}
	\end{screen}
	
	\begin{prf}
		任意の$x \in S$に対して閉包がコンパクトな開近傍$U_x$を取っておく.
		$\mathscr{O}$を$S$の開集合系として
		\begin{align}
			\mathscr{B} \coloneqq
			\Set{U \in \mathscr{O}}{\mbox{$\overline{U}$がコンパクト}}
		\end{align}
		とおけば,$\mathscr{B}$は$\mathscr{O}$の基底となる.実際,
		任意の$O \in \mathscr{O}$に対し$O \cap U_x \in \mathscr{B}$かつ
		\begin{align}
			O = \bigcup_{x \in O} O \cap U_x
		\end{align}
		となる.従って定理\ref{thm:countable_base_of_second_countable_space}より
		或る可算部分集合$\{U_n\}_{n=1}^\infty \subset \mathscr{B}$が
		$\mathscr{O}$の基底となる.いま,$K_1 \coloneqq \overline{U_1}$として,
		またコンパクト集合$K_n$が選ばれたとして,
		$K_n$の有限被覆$\mathscr{U}_n \subset \mathscr{B}_0$を取り
		\begin{align}
			K_{n+1} \coloneqq \overline{U_{n+1}} \cup \bigcup_{V \in \mathscr{U}_n} \overline{V}
		\end{align}
		とすれば,$K_{n+1}$はコンパクトであり$K_n \subset K_{n+1}^{\mathrm{o}}$を満たす.
		この操作で$(K_n)_{n=1}^\infty$を構成すれば
		\begin{align}
			S = \bigcup_{n=1}^\infty U_n \subset \bigcup_{n=1}^\infty K_n \subset S
		\end{align}
		が成立する.
		\QED
	\end{prf}
	
\subsection{商位相}
	\begin{screen}
		\begin{thm}[商位相]
			位相空間$(S,\mathscr{O})$に同値関係$\sim$が定まっているとき,
			$x \in S$からその同値類$\pi(x)$への対応
			\begin{align}
				\pi: S \ni x \longmapsto \pi(x) \in S/\sim
			\end{align}
			を商写像(quotient mapping)という.
			すなわち商写像は
			\begin{align}
				x \sim y \quad \Longleftrightarrow \quad
				\pi(x) = \pi(y)
			\end{align}
			を満たす.また,商写像を連続にする$S/\sim$の最強の位相,つまり
			\begin{align}
				\mathscr{O}(S/\sim) \coloneqq
				\Set{V \subset S/\sim}{\pi^{-1}(V) \in \mathscr{O}}
			\end{align}
			で定まる位相を$S/\sim$の商位相(quotient topology)という.
		\end{thm}
	\end{screen}
	
	\begin{screen}
		\begin{thm}[商空間が$T_1 \Longleftrightarrow$同値類が元の空間で閉じている]
		\label{thm:quotient_space_T_1_iff_each_equivalence_class_closed}
			$S$を位相空間,$\sim$を$S$上の同値関係,$\pi:S \longrightarrow S/\sim$を商写像
			とする.このとき次が成り立つ:
			\begin{align}
				\mbox{$S/\sim$が$T_1$空間である}
				\quad \Longleftrightarrow \quad
				\mbox{任意の$x \in S$に対し$\pi(x)$が$S$の閉集合である}.
			\end{align}
		\end{thm}
	\end{screen}
	
	\begin{prf}
		任意の$F \subset S/\sim$に対し
		\begin{align}
			\mbox{$F$が閉} \quad \Longleftrightarrow \quad
			\mbox{$\pi^{-1}(F^c) = \pi^{-1}(F)^c$が開} \quad \Longleftrightarrow \quad
			\mbox{$\pi^{-1}(F)$が閉}
		\end{align}
		となる.いま任意の$x \in S$に対し
		$\pi(x) = \pi^{-1}(\pi(x))$が満たされているから定理の主張を得る.
		\QED
	\end{prf}
	
	\begin{screen}
		\begin{thm}[商写像が開なら,商空間がHausdorff$\Longleftrightarrow$対角線集合が閉]
		\label{thm:quotient_space_Hausdorff_iff_diagonal_set_closed}
			$S$を位相空間,$\sim$を$S$上の同値関係,$\pi:S \longrightarrow S/\sim$を商写像
			とする.このとき,$\pi$が開写像であれば次が成立する:
			\begin{align}
				\mbox{$S/\sim$がHausdorff} \quad \Longleftrightarrow \quad
				\mbox{$\Set{(x,y) \in S \times S}{x \sim y}$が閉}.
			\end{align}
		\end{thm}
	\end{screen}
	
	\begin{prf}
		$S/\sim$がHausdorffであるとき,$x \not\sim y$を満たす$(x,y) \in S \times S$に対し
		$\pi(x) \neq \pi(y)$となるから
		\begin{align}
			\pi(x) \in U,\quad \pi(y) \in V,\quad U \cap V = \emptyset
		\end{align}
		を満たす$S/\sim$の開集合$U,V$が取れる.このとき
		$\pi^{-1}(U) \times \pi^{-1}(V)$は$S \times S$の開集合であり
		\begin{align}
			(x,y) \in \pi^{-1}(U) \times \pi^{-1}(V)
			\subset \Set{(s,t) \in S \times S}{s \not\sim t}
		\end{align}
		が成り立つから$\Longrightarrow$が得られる.
		逆に$\Set{(s,t) \in S \times S}{s \not\sim t}$が開集合であるとき,
		$\pi(x) \neq \pi(y)$なら
		\begin{align}
			(x,y) \in U \times V \subset \Set{(s,t) \in S \times S}{s \not\sim t}
		\end{align}
		を満たす$S$の開集合$U,V$が存在し,このとき
		\begin{align}
			\pi(x) \in \pi(U),\quad \pi(y) \in \pi(V),
			\quad \pi(U) \cap \pi(V) = \emptyset
		\end{align}
		となりかつ$\pi$が開写像であるから$\Longleftarrow$が従う.
		\QED
	\end{prf}
	
	\begin{screen}
		\begin{cor}[Hausdorff$\Longleftrightarrow$対角線集合が閉]
		\label{cor:quotient_space_Hausdorff_iff_diagonal_set_closed}
			$S$を位相空間とするとき,
			\begin{align}
				\mbox{$S$がHausdorffである}
				\quad \Longleftrightarrow \quad
				\mbox{$\Set{(x,x)}{x \in S}$が$S \times S$で閉じている}.
			\end{align}
		\end{cor}
	\end{screen}
	
	\begin{prf}
		等号$=$を同値関係と見れば$S$と$S/=$は商写像により同相となるから,
		定理\ref{thm:quotient_space_Hausdorff_iff_diagonal_set_closed}より
		\begin{align}
			\mbox{$S$がHausdorff} \quad \Longleftrightarrow \quad
			\mbox{$S/=$がHausdorff} \quad \Longleftrightarrow \quad
			\mbox{$\Set{(x,x)}{x \in S}$が閉}
		\end{align}
		が成立する.
		\QED
	\end{prf}
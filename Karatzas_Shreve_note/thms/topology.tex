\subsection{位相}
	\begin{screen}
		\begin{dfn}[位相]
			集合$S$の部分集合族$\mathscr{O}$が
			以下を満たすとき,$\mathscr{O}$を$S$の位相
			\index{いそう@位相}(topology),或は開集合系
			\index{かいしゅうごうけい@開集合系}と呼ぶ:
			\begin{description}
				\item[(O1)] $\emptyset, S \in \mathscr{O}$,
				\item[(O2)] $O_1,O_2 \in \mathscr{O} 
					\quad \Longrightarrow \quad O_1 \cap O_2 \in \mathscr{O}$,
				\item[(O3)] $\displaystyle\mathscr{U} \subset \mathscr{O}
					\quad \Longrightarrow \quad \bigcup \mathscr{U} = 
					\bigcup_{U \in \mathscr{U}} U \in \mathscr{O}$.
			\end{description}
			また$\mathscr{O}$の元を$S$の開集合
			\index{かいしゅうごう@開集合}(open set)と呼び,
			補集合が開である集合を閉集合\index{へいしゅうごう@閉集合}(closed set)と呼ぶ.
		\end{dfn}
	\end{screen}
	
	$\R\mbox{ (resp. $\C$)}$において,部分集合$O$で
	\begin{itemize}
		\item $O = \emptyset$,又は任意の$x \in O$に対し或る$r_x > 0$が存在して
			$\Set{y \in \R \mbox{ (resp. $\C$)}}{|x-y| < r_x} \subset O$となる
	\end{itemize}
	を満たすものの全体を$\mathscr{O}$とおくと,$\mathscr{O}$は$\R\mbox{ (resp. $\C$)}$
	の位相となる.通常$\R$と$\C$には暗黙の裡にこの位相が入る.
	
	\begin{screen}
		\begin{dfn}[内部・閉包]
			位相空間の部分集合$A$に対し,
			$A$に含まれる最大の開集合を$A$の内部\index{ないぶ@内部}(interior)と呼び
			$A^{\mathrm{o}}$や$A^i$で表す.また
			$A$を含む最大の閉集合を$A$の閉包\index{へいほう@閉包}(closure)と呼び
			$\overline{A}$や$A^a$で表す.特に,
			\begin{align}
				\mbox{$A$が開}\ \Longleftrightarrow\ 
				A = A^\mathrm{o},
				\quad \mbox{$A$が閉}\ \Longleftrightarrow\ 
				A = \overline{A}.
				\label{eq:dfn_interior_closure}
			\end{align}
		\end{dfn}
	\end{screen}
	
	\begin{screen}
		\begin{thm}[内部の補集合は補集合の閉包]
		\label{thm:topology_note_closure_interior}
			$A$を位相空間の部分集合とするとき次が成り立つ.
			\begin{align}
				A^{ic} = A^{ca},
				\quad A^{cic} = A^a,
				\quad A^{ci} = A^{ac}.
			\end{align}
		\end{thm}
	\end{screen}
	
	\begin{prf}
		$A^i \subset A$より$A^{ic} \supset A^c$が従い,
		$A^{ic}$が閉であるから$A^{ic} \supset A^{ca}$となる.
		一方で$A^c \subset A^{ca}$より$A \supset A^{cac}$が従い,
		$A^{cac}$は開であるから$A^i \supset A^{cac}$すなわち
		$A^{ic} \subset A^{ca}$となる.
		$A$を$A^c$に替えれば残りの関係も得られる.
		\QED
	\end{prf}
	
	\begin{screen}
		\begin{dfn}[近傍・基本近傍系]
			空でない位相空間$S$において,$x \in S$と$U \subset S$に対し
			\begin{align}
				x \in U^{\mathrm{o}}
			\end{align}
			が満たされるとき$U$は$x$の近傍\index{きんぼう@近傍}(neighborhood)であるという.
			同様に$A \subset S$と$V \subset S$に対し
			\begin{align}
				A \subset V^{\mathrm{o}}
			\end{align}
			が満たされるとき,$V$は$A$の近傍であるという.
			点$x$の近傍全体を$\mathscr{V}(x)$と書くとき,
			$S$は$x$の最大の近傍であるから$\mathscr{V}(x)$は空ではない.
			また$\mathscr{V}(x)$の空でない部分集合$\mathscr{U}(x)$が
			\begin{align}
				\forall V \in \mathscr{V}(x),
				\quad \exists U \in \mathscr{U}(x),
				\quad U \subset V
			\end{align}
			を満たすとき,$\mathscr{U}(x)$を$x$の基本近傍系
			\index{きほんきんぼうけい@基本近傍系}(local base of a point $x$)と呼ぶ.
		\end{dfn}
	\end{screen}
	
	\begin{screen}
		\begin{thm}[基本近傍系は開集合を決定する]\label{thm:local_base_defines_open_sets}
			$S$を空でない位相空間,
			$\mathscr{U}(x)$を点$x$の基本近傍系とすれば
			\begin{align}
				\mbox{$O$が$S$の開集合} \quad \Longleftrightarrow \quad 
				\mbox{$O = \emptyset$,或は任意の$x \in O$に対し
				$U \subset O$を満たす$U \in \mathscr{U}(x)$が存在する}
			\end{align}
			が成立する.すなわち,$\{\mathscr{U}(x)\}_{x \in S}$を基本近傍系とする$S$の位相は唯一つである.
		\end{thm}
	\end{screen}
	
	\begin{prf}
		空でない部分集合$O$が開集合なら任意の$x \in O$に対し$O$は$x$の近傍となるから,
		或る$U \in \mathscr{U}(x)$が存在して$U \subset O$を満たす.
		逆に任意の$x \in O$に対し$U \subset O$を満たす$U \in \mathscr{U}(x)$が存在するとき,
		\begin{align}
			x \in U^{\mathrm{o}} \subset O^{\mathrm{o}}
		\end{align}
		となり$O = O^{\mathrm{o}}$が成立するから$O$は開集合である.
		\QED
	\end{prf}
	
	\begin{screen}
		\begin{thm}[基本近傍系は位相を復元する]
		\label{thm:a_local_base_restores_the_topology}\mbox{}
			\begin{description}
				\item[(1)] 
					$(S,\mathscr{O})$を空でない位相空間とし,各点
					$x \in S$に対し$\mathscr{U}(x)$を基本近傍系とすれば以下が成り立つ:
					\begin{description}
						\item[(LB1)] $\mathscr{U}(x)$は空ではなく,また任意の$U \in \mathscr{U}(x)$は$x \in U$を満たす.
						\item[(LB2)] 任意の$U,V \in \mathscr{U}(x)$に対し或る$W \in \mathscr{U}(x)$
							が存在して$W \subset U \cap V$を満たす.
						\item[(LB3)] 任意の$U \in \mathscr{U}(x)$に対し或る$V \in \mathscr{U}(x)$が存在し,
							$V \subset U$かつ任意の$y \in V$に対し$W_y \subset U$を満たす$W_y \in \mathscr{U}(y)$が取れる.
					\end{description}
				\item[(2)]
					空でない集合$S$の各点$x$に対し(LB1)(LB2)(LB3)を満たす部分集合族$\mathscr{U}(x)$が与えられれば,
					\begin{align}
						\mathscr{O} \coloneqq
						\Set{O \subset S}{\mbox{$O = \emptyset$,或は任意の$x \in O$に対し
						$U \subset O$を満たす$U \in \mathscr{U}(x)$が存在する}}
					\end{align}
					により$S$に位相が定まり,$\{\mathscr{U}(x)\}_{x \in S}$は
					$(S,\mathscr{O})$において基本近傍系となる.
				\item[(3)] 空でない位相空間$(S,\mathscr{O})$から基本近傍系
					$\{\mathscr{U}(x)\}_{x \in S}$を得れば,
					$\{\mathscr{U}(x)\}_{x \in S}$を基本近傍系とする位相
					を(2)の手続きで構成することにより$\mathscr{O}$を復元できる.
			\end{description}
		\end{thm}
	\end{screen}
	
	\begin{prf}\mbox{}
		\begin{description}
			\item[(1)] 任意の$U \in \mathscr{U}(x)$は$x$の近傍であるから
				$(LB1)$が満たされる.また$U,V \in \mathscr{U}(x)$に対し
				\begin{align}
					x \in U^{\mathrm{o}} \cap V^{\mathrm{o}} = (U \cap V)^{\mathrm{o}}
				\end{align}
				となるから$U \cap V$は$x$の近傍であり(LB2)も従う.
				任意に$U \in \mathscr{U}(x)$を取れば,
				$U^{\mathrm{o}}$は$x$の開近傍であるから
				或る$V \in \mathscr{U}(x)$で$V \subset U^{\mathrm{o}}$
				を満たすものが存在する.このとき任意の$y \in V$に対し
				$U^{\mathrm{o}}$は$y$の開近傍となるから
				\begin{align}
					W_y \subset U^{\mathrm{o}} \subset U
				\end{align}
				を満たす$W_y \in \mathscr{U}(y)$が取れる.従って(LB3)も得られる.
			
			\item[(2)] 
				$\mathscr{U}(x)$は空ではないから$S \in \mathscr{O}$となる.
				また$O_1,O_2 \in \mathscr{O}$を取れば,
				任意の$x \in O_1 \cap O_2$に対し
				\begin{align}
					x \in U_1 \subset O_1,
					\quad x \in U_2 \subset O_2
				\end{align}
				を満たす$U_1,U_2 \in \mathscr{U}(x)$が存在し,
				(LB2)より或る$U_3 \in \mathscr{U}(x)$に対して
				\begin{align}
					U_3 \subset U_1 \cap U_2 \subset O_1 \cap O_2
				\end{align}
				が成り立つから$O_1 \cap O_2 \in \mathscr{O}$となる.
				任意に$\mathscr{G} \subset \mathscr{O}$を取れば
				任意の$x \in \bigcup \mathscr{G}$は或る$G \in \mathscr{G}$の点であるから,
				\begin{align}
					U \subset G \subset \bigcup \mathscr{G}
				\end{align}
				を満たす$U \in \mathscr{U}(x)$が存在し$\bigcup \mathscr{G} \in \mathscr{O}$が従う.
				よって$\mathscr{O}$は位相である.
				ところで,任意の$U \in \mathscr{U}(x)$に対し
				\begin{align}
					U^{\mathrm{o}} = 
					\Set{y \in U}{\mbox{或る$W_y \in \mathscr{U}(y)$が存在して
					$W_y \subset U$となる}} \eqqcolon \tilde{U}
					\label{eq:thm_a_local_base_restores_the_topology_0}
				\end{align}
				が成立する.実際$\mathscr{O}$の定義より
				\begin{align}
					y \in U^{\mathrm{o}} \quad \Longrightarrow \quad
					\mbox{或る$W_y \in \mathscr{U}(y)$で
					$W_y \subset U^{\mathrm{o}}$}
				\end{align}
				となるから$U^{\mathrm{o}}\subset\tilde{U}$が従い,
				逆に$y \in \tilde{U}$については,
				(\refeq{eq:thm_a_local_base_restores_the_topology_0})の$W_y$に対して
				(LB3)より或る$X_y \in \mathscr{U}(y)$が
				\begin{align}
					X_y \subset W_y,\quad 
					z \in X_y \ \Longrightarrow \
					\mbox{或る$Y_z \in \mathscr{U}(z)$で$Y_z \subset X_y \subset U$}
				\end{align}
				を満たすから$X_y \subset \tilde{U}$が従う.
				すなわち$\tilde{U}$は開集合であり,$U^{\mathrm{o}}\subset\tilde{U}$
				と併せて(\refeq{eq:thm_a_local_base_restores_the_topology_0})
				を得る.(LB3)より
				\begin{align}
					V \subset U, \quad y \in V \ \Longrightarrow \
					\mbox{或る$W_y \in \mathscr{U}(y)$で$W_y \subset U$}
				\end{align}
				を満たす$V \in \mathscr{U}(x)$が存在し,(LB1)と併せて
				\begin{align}
					x \in V \subset \tilde{U} = U^{\mathrm{o}}
				\end{align}
				が成り立つから任意の$U \in \mathscr{U}(x)$は$x$の近傍である.
				そして$W$を$x$の任意の近傍とすれば,
				$\mathscr{O}$の定め方より或る$U \in \mathscr{U}(x)$が
				$U \subset W^{\mathrm{o}}$を満たすから
				$\mathscr{U}(x)$は$x$の基本近傍系である.
			
			\item[(3)] 
				定理\ref{thm:local_base_defines_open_sets}より
				$\{\mathscr{U}(x)\}_{x \in S}$を基本近傍系とする位相は唯一つであるから
				主張が従う.
				\QED
		\end{description}
	\end{prf}
	
	\begin{screen}
		\begin{dfn}[集積点・密集点]
			位相空間$S$の点$x$と部分集合$A$について,
			$x$の任意の近傍$U$に対し
			\begin{align}
				(U \backslash \{x\}) \cap A \neq \emptyset
			\end{align}
			となるとき,$x$は$A$の集積点\index{しゅうせきてん@集積点}
			(accumulation point)であるという.
			同様に$x$の任意の近傍$U$に対し
			\begin{align}
				U \cap A \neq \emptyset
			\end{align}
			となるとき,$x$は$A$の密集点\index{みっしゅうてん@密集点}(cluster point)であるという.
		\end{dfn}
	\end{screen}
	
	集積点と密集点の明確な違いは$T_1$空間(後述)において現れる.
	\begin{screen}
		\begin{thm}[閉である一点集合は集積点を持たない]
		\label{thm:closed_singleton_has_no_accumulation_point}
			位相空間において,閉じている一点集合は集積点を持たない.特に
			$\{x\}$が閉であるとき,$x$は$\{x\}$の密集点ではあるが集積点ではない.
		\end{thm}
	\end{screen}
	
	\begin{prf}
		一点集合$\{x\}$が閉であるとする.このとき$y \neq x$なら
		$U \coloneqq \{x\}^c$は$y$の開近傍となり
		\begin{align}
			(U \backslash \{y\}) \cap \{x\} = \emptyset
		\end{align}
		を満たすから$y$は$\{x\}$の集積点ではない.
		$x$は$\{x\}$の集積点となりえないから$\{x\}$は集積点を持たない.
		\QED
	\end{prf}
	
	\begin{screen}
		\begin{thm}[閉集合は密集点集合]
		\label{thm:belongs_to_closure_iff_clusters}
			位相空間$S$の点$x$と部分集合$A$について次が成り立つ:
			\begin{align}
				x \in \overline{A} \quad \Longleftrightarrow \quad
				\mbox{$x$は$A$の密集点である}.
				\label{eq:thm_belongs_to_closure_iff_clusters}
			\end{align}
			特に,$A$が閉であることと$A$の密集点全体が$A$に一致することは同値になる.
		\end{thm}
	\end{screen}
	
	\begin{prf}
		$x$の或る近傍$U$が$U \cap A = \emptyset$を満たすとき,
		定理\ref{thm:topology_note_closure_interior}より
		\begin{align}
			x \in U^i \subset A^{ci} = A^{ac}
		\end{align}
		となり$x \notin \overline{A}$が従う.逆に
		$x \notin \overline{A}$なら
		$\overline{A}^c$は$A$と交わらない$x$の開近傍となるから
		(\refeq{eq:thm_belongs_to_closure_iff_clusters})が出る.
		また(\refeq{eq:dfn_interior_closure})より
		\begin{align}
			\mbox{$A$が閉} \quad \Longleftrightarrow \quad A = \overline{A}
			\quad \Longleftrightarrow \quad
			\mbox{$A$の密集点全体が$A$に一致}
		\end{align}
		が成立する.
		\QED
	\end{prf}
	
	\begin{screen}
		\begin{thm}[$x \in \overline{A \backslash \{x\}}$$\Longleftrightarrow$$x$が$A$の集積点]
			位相空間$S$の点$x$と部分集合$A$について次が成り立つ:
			\begin{align}
				x \in \overline{A \backslash \{x\}} \quad \Longleftrightarrow \quad
				\mbox{$x$は$A$の集積点である}.
			\end{align}
		\end{thm}
	\end{screen}
	
	\begin{prf}
		$x$の任意の近傍$U$に対し
		$U \cap (A \backslash \{x\}) = (U \backslash \{x\}) \cap A$となるから,
		定理\ref{thm:belongs_to_closure_iff_clusters}と併せて
		\begin{align}
			x \in \overline{A \backslash \{x\}} 
			&\quad \Longleftrightarrow \quad
			\mbox{$x$の任意の近傍$U$に対し$U \cap (A \backslash \{x\}) \neq \emptyset$} \\
			&\quad \Longleftrightarrow \quad
			\mbox{$x$の任意の近傍$U$に対し$(U \backslash \{x\}) \cap A \neq \emptyset$}
			\quad \Longleftrightarrow \quad
			\mbox{$x$は$A$の集積点}
		\end{align}
		が成立する.
		\QED
	\end{prf}
	
	\begin{screen}
		\begin{dfn}[相対位相]
			$(S,\mathscr{O})$を位相空間,$M \subset S$を部分集合,
			$i:M \longrightarrow S$を恒等写像とするとき,
			\begin{align}
				\mathscr{O}_M \coloneqq 
				\Set{i^{-1}(O) = O \cap M}{O \in \mathscr{O}}
			\end{align}
			で定まる$i$による$\mathscr{O}$の引き戻しを$M$の相対位相(relative topology)と呼ぶ.
		\end{dfn}
	\end{screen}
	
	\begin{screen}
		\begin{dfn}[被覆・コンパクト・相対コンパクト・局所コンパクト・$\sigma$-コンパクト]\mbox{}
			\begin{itemize}
				\item
					集合$S$の部分集合族$\mathscr{B}$が
					$S$の{\bf 被覆}\index{ひふく@被覆}{\bf (cover)}であるとは,
					\begin{align}
						S = \bigcup \mathscr{B}
					\end{align}
					を満たすことをいう.また可算個の部分集合から成る被覆を{\bf 可算被覆}
					\index{かさんひふく@可算被覆}と呼ぶ.
					特に,位相空間において開集合のみから成る被覆を
					{\bf 開被覆}\index{かいひふく@開被覆}{\bf (open cover)}と呼ぶ.
				
				\item 集合$S$の被覆$\mathscr{B}$に対し,その部分集合で
					$S$の被覆となるものを$\mathscr{B}$の{\bf 部分被覆}
					\index{ぶぶんひふく@部分被覆}{\bf (subcover)}と呼ぶ.
					部分被覆が有限(可算)集合であるときは有限(可算)部分被覆と呼ぶ.
				\item 
					位相空間において任意の開被覆が有限部分被覆を持つとき,
					その空間は{\bf コンパクト}\index{こんぱくと@コンパクト}である
					{\bf (compact)}という.
					位相空間の部分集合は,その相対位相でコンパクト空間となるとき
					{\bf コンパクト部分集合}と呼ばれる.
				
				\item 位相空間の部分集合で,その閉包がコンパクトであるものを
					{\bf 相対コンパクト}\index{そうたいこんぱくと@相対コンパクト}な
					{\bf (relatively compact)}部分集合という.
				
				\item 位相空間の任意の点がコンパクトな近傍を持つとき,
					その空間は{\bf 局所コンパクト}である
					\index{きょくしょこんぱくと@局所コンパクト}{\bf (locally compact)}という.
					
				\item 位相空間においてコンパクト集合から成る可算被覆が存在するとき,
					その空間は{\bf $\sigma$-コンパクト}
					\index{しぐまこんぱくと@$\sigma$-コンパクト}であるという.
			\end{itemize}
		\end{dfn}
	\end{screen}
	
	集合$S$とその部分集合$A$に対し,$S$の部分集合族$\mathscr{B}$で
	$A \subset \bigcup \mathscr{B}$を満たすものを
	$A$の`$S$における被覆'と呼ぶ.$\mathscr{B}$の構成要素が$S$の開集合である場合は
	`$S$における開被覆'と呼び,他に`$S$における部分被覆'や`$S$における有限被覆'といった言い方もする.
	
	\begin{screen}
		\begin{thm}[部分集合のコンパクト性]
		\label{thm:subset_is_compact_iff_every_original_open_cover_contains_finite_subcover}
			$A$を位相空間$S$の部分集合とするとき次が成り立つ:
			\begin{align}
				\mbox{$A$がコンパクト部分集合} \quad \Longleftrightarrow \quad
				\mbox{$A$の$S$における任意の開被覆が($S$における)有限部分被覆を含む}.
			\end{align}
		\end{thm}
	\end{screen}
	
	\begin{prf}
		$A$がコンパクト部分集合であるとき,$\mathscr{B}$を$A$の$S$における開被覆とすれば
		\begin{align}
			\Set{B \cap A}{B \in \mathscr{B}}
		\end{align}
		は部分空間$A$における開被覆となり,有限個の$B_1,B_2,\cdots,B_n \in \mathscr{B}$により
		\begin{align}
			A = \bigcup_{i=1}^n (B_i \cap A) \subset \bigcup_{i=1}^n B_i
		\end{align}
		となるから$\Longrightarrow$が従う.逆に右辺が満たされているとき,
		$\mathscr{A}$を$A$の相対開集合から成る$A$の被覆とすれば
		\begin{align}
			\mathscr{A} = \Set{C \cap A}{C \in \mathscr{C}}
		\end{align}
		を満たす$S$の開集合族$\mathscr{C}$が存在する.実際任意の$U \in \mathscr{A}$に対し
		\begin{align}
			\mathscr{C}_U \coloneqq \Set{C \subset S}{\mbox{$C$は$S$の開集合で$C \cap A = U$}}
		\end{align}
		とおけば$\mathscr{C}_U \neq \emptyset$であるから,
		一つ$\Phi \in \prod_{U \in \mathscr{A}} \mathscr{C}_U$を取り
		\begin{align}
			\mathscr{C} \coloneqq \Set{\Phi(U)}{U \in \mathscr{A}}
		\end{align}
		と定めればよい.このとき有限個の$C_1,C_2,\cdots,C_m \in \mathscr{C}$により
		$A \subset \bigcup_{j=1}^m C_j$となり,
		\begin{align}
			A = \bigcup_{j=1}^m (A \cap C_j)
		\end{align}
		かつ$A \cap C_j \in \mathscr{A}$が成り立つから$A$はコンパクトである.
		\QED
	\end{prf}
	
	\begin{screen}
		\begin{dfn}[有限交叉性]
			集合$S$の部分集合族$\mathscr{S}$について,その任意の
			有限部分族$\mathscr{T} \subset \mathscr{S}$が
			$\bigcap \mathscr{T} \neq \emptyset$を満たすとき
			$\mathscr{S}$は{\bf 有限交叉性}\index{ゆうげんこうさせい@有限交叉性}
			{\bf (finite intersection property)}を持つという.
		\end{dfn}
	\end{screen}
	
	\begin{screen}
		\begin{thm}[コンパクト$\Longleftrightarrow$閉集合族が有限交叉的]
		\label{thm:compact_iff_closed_sets_family_finitely_intersect}
			$S$を位相空間,$A$を$S$の部分集合とするとき,
			\begin{align}
				&\mbox{$A$がコンパクト部分集合} \quad \Longleftrightarrow \\ 
				&\quad \mbox{任意の$S$の閉集合族$\mathscr{F}$に対し,
				$\Set{F \cap A}{F \in \mathscr{F}}$が有限交叉性を持つなら
				$A \cap \bigcap \mathscr{F} \neq \emptyset.$}
			\end{align}
		\end{thm}
	\end{screen}
	
	\begin{prf}
		定理\ref{thm:subset_is_compact_iff_every_original_open_cover_contains_finite_subcover}より
		\begin{align}
			&\mbox{$A$がコンパクト部分集合} \\
			&\Longleftrightarrow \quad \mbox{$A$の$S$における任意の開被覆が($S$における)有限部分被覆を含む} \\
			&\Longleftrightarrow \quad \mbox{任意の$S$の閉集合族$\mathscr{F}$に対し,
			$A \cap \bigcap \mathscr{F} = \emptyset$なら或る有限族$\mathscr{M} \subset \mathscr{F}$で
			$A \cap \bigcap \mathscr{M} = \emptyset$} \\
			&\Longleftrightarrow \quad \mbox{任意の$S$の閉集合族$\mathscr{F}$に対し,
			$\Set{F \cap A}{F \in \mathscr{F}}$が有限交叉性を持つなら
			$A \cap \bigcap \mathscr{F} \neq \emptyset$}
		\end{align}
		が従う.
		\QED
	\end{prf}
	
	\begin{screen}
		\begin{dfn}[連続・同相・開写像]
			$f$を位相空間$S$から位相空間$T$への写像とする.
			\begin{itemize}
				\item
					$x \in S$において$f(x)$の任意の近傍$U$に対し
					$f^{-1}(U)$が$x$の近傍となるとき,
					$f$は$x$で連続\index{れんぞく@連続}である(continuous at a point $x$)という.
					
				\item $T$の任意の開集合$O$に対し$f^{-1}(O)$が$S$の開集合となるとき,
					$f$は連続である(continuous)という.
					
				\item $f$に逆写像$f^{-1}$が存在し,$f,f^{-1}$が共に連続であるとき,
					$f$を同相写像\index{どうそうしゃぞう@同相写像}(homeomorphism),
					或は位相同型\index{いそうどうけい@位相同型}と呼ぶ.また
					$S,T$間に同相写像が存在するとき$S$と$T$は
					同相\index{どうそう@同相}である(homeomorphic),或は位相同型であるという.
					
				\item $S$の任意の開集合の$f$による像が$T$の開集合であるとき,
					$f$を開写像\index{かいしゃぞう@開写像}(open mapping)と呼ぶ.
			\end{itemize}
		\end{dfn}
	\end{screen}
	
	\begin{screen}
		\begin{thm}[各点連続$\Longleftrightarrow$連続]
		\label{thm:continuous_on_every_point_iff_continuous}
			$f$を位相空間$S$から位相空間$T$への写像とするとき次が成り立つ:
			\begin{align}
				\mbox{$f$が連続} \quad \Longleftrightarrow \quad
				\mbox{$f$が$S$の各点で連続}.
			\end{align}
		\end{thm}
	\end{screen}
	
	\begin{prf}
		$f$が連続であるとき,各点$x \in S$で$f(x)$の任意の近傍$U$に対し
		$f(x) \in U^{\mathrm{o}}$が満たされるから
		$f^{-1}(U^{\mathrm{o}})$は$x$を含む開集合となる.
		$f^{-1}(U^{\mathrm{o}})$は$f^{-1}(U)$に含まれる開集合であるから
		\begin{align}
			x \in f^{-1}(U^{\mathrm{o}}) \subset f^{-1}(U)^{\mathrm{o}}
		\end{align}
		が成り立ち,従って$f$は$x$で連続である.
		逆に$f$が各点連続であるとき,
		$T$の任意の開集合$O$に対し
		$f^{-1}(O)$は任意の$x \in f^{-1}(O)$の近傍となるから
		定理\ref{thm:local_base_defines_open_sets}より
		$f^{-1}(O)$は開集合である.よって$f$は連続である.
		\QED
	\end{prf}
	
	\begin{screen}
		\begin{thm}[部分空間と制限写像の連続性]
			$S,T$を位相空間,$f$を$S$から$T$への写像とする.
			また$U \coloneqq f(S)$として$g:S \longrightarrow U$を
			$f$の値域を$U$へ制限した写像とする.このとき次が成り立つ:
			\begin{align}
				\mbox{$f:S \longrightarrow T$が連続である} 
				\quad \Longleftrightarrow \quad
				\mbox{$g:S \longrightarrow U$が($U$の相対位相に関して)連続である}.
			\end{align}
		\end{thm}
	\end{screen}
	
	\begin{prf}
		$T$の任意の開集合$O$に対し
		\begin{align}
			g^{-1}(U \cap O) = f^{-1}(U \cap O) = f^{-1}(O)
		\end{align}
		が成り立つから,$f$と$g$の連続性は一致する.
		\QED
	\end{prf}
	
	\begin{screen}
		\begin{thm}[位相の生成]
			$S$を集合,$\mathcal{P}(S)$を冪集合として
			任意に$M \subset \mathcal{P}(S)$を取り
			\begin{align}
				\mathscr{A} \coloneqq
				\Set{\bigcap_{i=1}^n I_i}{I_i \in M,\ n = 1,2,\cdots}
			\end{align}
			とおくとき,$M$を含む最小の位相は
			\begin{align}
				\mathscr{O} \coloneqq
				\Set{\bigcup \Lambda}{\Lambda \subset \mathscr{A}}
				\cup \{S\}
			\end{align}
			で与えられる.この$\mathscr{O}$を$M$が生成する$S$の位相と呼ぶ.
		\end{thm}
	\end{screen}
	
	\begin{prf}
		$\mathscr{O}$は定め方より$S$と$\emptyset$を含む.また
		任意の$O_1 = \bigcup \Lambda_1,\ O_2=\bigcup \Lambda_2 \in \mathscr{O},\ 
		(\Lambda_1,\Lambda_2 \subset \mathscr{A})$に対し
		\begin{align}
			\Lambda \coloneqq
			\Set{I \cap J}{I \in \Lambda_1,\ J \in \Lambda_2} \subset \mathscr{A}
		\end{align}
		となるから
		\begin{align}
			O_1 \cap O_2 = \bigcup_{I \in \Lambda_1,\ J \in \Lambda_2} I \cap J
			= \bigcup \Lambda \in \mathscr{O}
		\end{align}
		が成立する.任意に$\emptyset \neq \mathscr{U} \subset \mathscr{O}$を取れば,
		各$U \in \mathscr{U}$に$U = \bigcup \Lambda_U$を満たす
		$\Lambda_U \subset \mathscr{A}$が対応し,このとき
		\begin{align}
			\bigcup_{U \in \mathscr{U}} \Lambda_U \subset \mathscr{A}
		\end{align}
		となるから
		\begin{align}
			\bigcup \mathscr{U} = \bigcup \Biggl(\bigcup_{U \in \mathscr{U}} \Lambda_U\Biggr)
			\in \mathscr{O}
		\end{align}
		が従う.$M$を含む任意の位相は$\mathscr{A}$を含みかつその任意和で閉じるから$\mathscr{O}$を含む.
		\QED
	\end{prf}
	
	\begin{screen}
		\begin{dfn}[始位相]
			$f \in \mathscr{F}$を集合$S$から位相空間$(T_f,\mathscr{O}_f)$への写像とするとき,
			全ての$f \in \mathscr{F}$を連続にする最弱の位相を$S$の$\mathscr{F}$-始位相
			(initial topology)と呼ぶ.$\mathscr{F}$-始位相は次が生成する位相である:
			\begin{align}
				\bigcup_{f \in \mathscr{F}} \Set{f^{-1}(O)}{O \in \mathscr{O}_f}.
			\end{align}
		\end{dfn}
	\end{screen}
	
\subsection{分離公理}
	\begin{screen}
		\begin{dfn}[位相的に識別可能・分離]
			$S$を位相空間とする.
			\begin{itemize}
				\item $x,y \in S$に対し$x \notin \overline{\{y\}}$
					或は$y \notin \overline{\{x\}}$が満たされるとき,
					$x$と$y$は{\bf 位相的に識別可能}
					\index{いそうてきにしきべつかのう@位相的に識別可能}である
					{\bf (topologically distinguishable)}という.
				\item $A,B \subset S$に対し$\overline{A} \cap B = \emptyset$
					或は$A \cap \overline{B} = \emptyset$が満たされるとき,
					$A$と$B$は{\bf 分離される}
					\index{ぶんりされる@(集合が)分離される}
					{\bf (separeted)}という.点と点,点と集合の分離は一点集合を考える.
				\item $A,B \subset S$が{\bf 近傍で分離される}
					\index{きんぼうでぶんりされる@近傍で分離される}
					{\bf (separated by neighborhoods)}とは,
					$A,B$が互いに交わらない近傍を持つことをいう.
				\item 閉集合$A,B \subset S$が
					{\bf 関数で分離される}
					\index{かんすうでぶんりされる@関数で分離される}
					{\bf (separated by a function)}とは,
					或る連続関数$f:S \longrightarrow [0,1]$によって$f(A) = \{0\},\ f(B) = \{1\}$
					が満たされることをいう.
				\item 閉集合$A,B \subset S$が
					{\bf 関数でちょうど分離される}
					\index{かんすうでちょうどぶんりされる@関数でちょうど分離される}
					{\bf (precisely separated by a function)}とは,
					或る連続関数$f:S \longrightarrow [0,1]$によって
					$A = f^{-1}(\{0\}),\ B = f^{-1}(\{1\})$が満たされることをいう.
			\end{itemize}
		\end{dfn}
	\end{screen}
	
	\begin{screen}
		\begin{thm}[位相的に識別可能な二点は相異なる]
			$S$を位相空間とするとき,任意の$x,y \in S$に対し
			\begin{align}
				\mbox{$x$と$y$が位相的に識別可能} \quad \Longrightarrow \quad
				x \neq y .
			\end{align}
		\end{thm}
	\end{screen}
	
	\begin{prf}
		$x = y$なら$y \in \overline{\{x\}}$かつ$x \in \overline{\{y\}}$となる.
		後述の$T_0$空間とは,この逆が満たされる位相空間である.
		\QED
	\end{prf}
	
	\begin{screen}
		\begin{thm}[分離される集合は他方を含まない近傍を持つ]
		\label{thm:the_equivalent_condition_of_separatedness}
			位相空間$S$において,$A,B \subset S$が分離されることと
			\begin{align}
				A \subset U,\quad B \subset V,\quad 
				A \cap V = \emptyset,
				\quad B \cap U = \emptyset
				\label{eq:thm_the_equivalent_condition_of_separatedness}
			\end{align}
			を満たす開集合$U,V$が存在することは同値である.
		\end{thm}
	\end{screen}
	
	\begin{prf}
		$A,B \subset S$が分離されるとき,$U \coloneqq \overline{B}^c,\ V \coloneqq \overline{A}^c$
		とおけば(\refeq{eq:thm_the_equivalent_condition_of_separatedness})が成立する.
		逆に$A,B$に対し(\refeq{eq:thm_the_equivalent_condition_of_separatedness})を満たす
		開集合$U,V$が存在するとき,$\closure{A} \subset V^c \subset B^c$及び
		$\closure{B} \subset U^c \subset A^c$となるから$A,B$は分離される.
		\QED
	\end{prf}
	
	\begin{screen}
		\begin{dfn}[分離公理]\mbox{}
			\begin{itemize}
				\item 任意の二点が位相的に識別可能である位相空間を{\bf $T_0$空間}
					\index{$T_0$くうかん@$T_0$空間},
					或は{\bf Kolmogorov空間}という.
				\item 任意の二点が分離される位相空間を{\bf $T_1$空間}
					\index{$T_1$くうかん@$T_1$空間}という.
				\item 任意の二点が近傍で分離される位相空間を{\bf $T_2$空間}
					\index{$T_2$くうかん@$T_2$空間},
					或は{\bf Hausdorff空間}\index{Hausdorffくうかん@Hausdorff空間}という.
				\item 任意の交わらない点と閉集合が近傍で分離される位相空間を
					{\bf 正則(regular)空間}\index{せいそくくうかん@正則空間}という.
				\item $T_0$かつ正則な位相空間を{\bf $T_3$空間}
					\index{$T_3$くうかん@$T_3$空間},
					或は{\bf 正則Hausdorff空間}
					\index{せいそくHausdorffくうかん@正則Hausdorff空間}という.
				\item 任意の交わらない点と閉集合が関数で分離される位相空間を
					{\bf 完全正則(completely regular)空間}
					\index{かんぜんせいそくくうかん@完全正則空間}という.
				\item $T_0$かつ完全正則な位相空間を{\bf $T_{3{}^1{\mskip -5mu/\mskip -3mu}_2}$空間}
					\index{$T_{3{}^1{\mskip -5mu/\mskip -3mu}_2}$くうかん@$T_{3{}^1{\mskip -5mu/\mskip -3mu}_2}$空間}や
					{\bf 完全正則Hausdorff空間}
					\index{かんぜんせいそくHausdorffくうかん@完全正則Hausdorff空間},
					或は{\bf Tychonoff空間}\index{Tychonoffくうかん@Tychonoff空間}という.
				\item 任意の交わらない二つの閉集合が近傍で分離される位相空間を
					{\bf 正規(normal)空間}\index{せいきくうかん@正規空間}という.
				\item $T_1$かつ正規な位相空間を{\bf $T_4$空間}
					\index{$T_4$くうかん@$T_4$空間},
					或は{\bf 正規Hausdorff空間}
					\index{せいきHausdorffくうかん@正規Hausdorff空間}という.
				\item 任意の部分位相空間が正規である位相空間を
					{\bf 全部分正規(completely normal)空間}
					\index{ぜんぶぶんせいきくうかん@全部分正規空間}という.
				\item $T_1$かつ全部分正規な位相空間を{\bf $T_5$空間}
					\index{$T_5$くうかん@$T_5$空間},
					或は{\bf 全部分正規Hausdorff空間}
					\index{ぜんぶぶんせいきHausdorffくうかん@全部分正規Hausdorff空間}という.
				\item 任意の交わらない二つの閉集合が関数でちょうど分離される位相空間を
					{\bf 完全正規(perfectly normal)空間}
					\index{かんぜんせいきくうかん@完全正規空間}という.
				\item $T_1$かつ完全正規な位相空間を{\bf $T_6$空間}
					\index{$T_6$くうかん@$T_6$空間},
					或は{\bf 完全正規Hausdorff空間}
					\index{かんぜんせいきHausdorffくうかん@完全正規Hausdorff空間}という.
			\end{itemize}
		\end{dfn}
	\end{screen}
	
	\begin{screen}
		\begin{thm}[$T_1$空間とは一点集合が閉である空間]
			位相空間$S$に対し,
			\begin{align}
				\mbox{$S$が$T_1$}
				&\quad \Longleftrightarrow \quad \mbox{$S$は$T_0$かつ位相的に識別可能な任意の二点が分離される} \\
				&\quad \Longleftrightarrow \quad \mbox{$S$の任意の一点集合は閉} \\
				&\quad \Longleftrightarrow \quad \mbox{$x \in S$が$A \subset S$の集積点であることと$x$の任意の開近傍が$A$と交わることは同値}.
			\end{align}
		\end{thm}
	\end{screen}
	
	\begin{prf}
		$x$が$A$の集積点であるとき,任意に$x$の近傍$U$を取る.
		いま,$x$の或る開近傍$U_{n-1}$と$x_{n-1} \in U_{n-1},\ (x \neq x_{n-1})$
		が取れたとして,
		\begin{align}
			U_n \coloneqq U_{n-1} \cap (S \backslash \{x_{n-1}\})
		\end{align}
		は$x$の開近傍となり或る$x_n \in (U_{n-1} \backslash \{x\}) \cap A$が取れる.
		$U_0 \coloneqq U^{\mathrm{o}},\ 
		x_0 \in (U^{\mathrm{o}} \backslash \{x\}) \cap A$を出発点とすれば
		$A$は$U$の無限集合$\{x_n\}_{n=1}^\infty$を含む.
	\end{prf}
	
	\begin{screen}
		\begin{thm}[Hausdorff空間のコンパクト部分集合は閉]
			Hausdorff空間のコンパクト部分集合は閉である.
		\end{thm}
	\end{screen}
	
	\begin{prf}
		$S$をHausdorff空間,$K \subset S$をコンパクト部分集合とするとき,
		任意に$x \in S \backslash K,\ y \in K$を取れば
		\begin{align}
			x \in U_y,\quad y \in V_y, \quad U_y \cap V_y = \emptyset
		\end{align}
		を満たす開集合$U_y,V_y$が取れる.或る$\{y_i\}_{i=1}^n \subset K$に対し
		$K \subset \bigcup_{i=1}^n V_{y_i}$となるから,
		$U \coloneqq \bigcap_{i=1}^n U_{y_i}$とおけば
		\begin{align}
			x \in U,\quad U \subset \bigcap_{i=1}^n \left(S\backslash V_{y_i}\right)
			\subset S \backslash K
		\end{align}
		が成立する.従って$S \backslash K$は開集合であり,$K$は閉集合である.
		\QED
	\end{prf}
	
	\begin{screen}
		\begin{thm}[Hausdorff空間においてコンパクト集合の閉部分集合はコンパクト]
			$S$をHausdorff空間,$K \subset S$をコンパクト部分集合,$F \subset S$を閉集合とするとき,
			$K \cap F$はコンパクトである.
		\end{thm}
	\end{screen}
	
	\begin{prf}
		$K \cap F$の任意の($S$における)開被覆に$S \backslash F$を加えれば
		$K$の($S$における)開被覆となるから,そのうち$K$の有限部分被覆を取ることができる.
		$S \backslash F$を除けば$K \cap F$の有限被覆が残り
		$K \cap F$のコンパクト性が出る.
		\QED
	\end{prf}
	
	\begin{screen}
		\begin{thm}[Hausdorff空間とは交わらない二つのコンパクト集合が近傍で分離される空間]
		\label{thm:Hausdorff_space_two_disjoint_compact_sets_are_separated_by_nbh}
			位相空間において,Hausdorffであることと,
			交わらない二つのコンパクト部分集合が近傍で分離されることは同値である.
		\end{thm}
	\end{screen}
	
	\begin{prf}
		$A,B$をHausdorff空間の交わらないコンパクト集合とするとき,
		任意の$p \in A$に対し
		\begin{align}
			p \in U_p,\quad B \subset V_p,\quad U_p \cap V_p = \emptyset
			\label{eq:thm_Hausdorff_space_two_disjoint_compact_sets_are_separated_by_nbh_1}
		\end{align}
		を満たす開集合$U_p,V_p$が存在する.実際
		任意の$q \in B$に対し
		\begin{align}
			p \in U_p(q),\quad q \in V_p(q),\quad U_p(q) \cap U_p(q) = \emptyset
		\end{align}
		を満たす開集合$U_p(q), U_p(q)$が取れ,$B$のコンパクト性より
		或る$\{q_i\}_{i=1}^n \subset B$で$B \subset \bigcup_{i=1}^n U_p(q_i)$となるから,
		\begin{align}
			U_p \coloneqq \bigcap_{i=1}^n U_p(q_i),
			\quad V_p \coloneqq \bigcup_{i=1}^n V_p(q_i)
		\end{align}
		とおけば(\refeq{eq:thm_Hausdorff_space_two_disjoint_compact_sets_are_separated_by_nbh_1})
		が成立する.$A$のコンパクト性より或る$\{p_j\}_{j=1}^m \subset A$で
		$A \subset \bigcup_{j=1}^m U_{p_j}$となるから,
		\begin{align}
			U \coloneqq \bigcup_{j=1}^m U_{p_j},
			\quad V \coloneqq \bigcap_{j=1}^m V_{p_j}
		\end{align}
		とおけば$A$と$B$は$U,V$により分離される.
		逆の主張は一点集合がコンパクトであることより従う.
		\QED
	\end{prf}
	
	\begin{screen}
		\begin{thm}[Hausdorff空間値連続写像の等価域は閉]
			$S$を位相空間,$T$をHausdorff空間,$f,g$を
			$S$から$T$への連続写像とするとき,$E \coloneqq \Set{x \in S}{f(x) = g(x)}$は$S$で閉じている.
			特に,$E$が$X$で稠密なら$f=g$となる.
		\end{thm}
	\end{screen}
	
	\begin{prf}
		任意に$x \in \Set{x \in S}{f(x) \neq g(x)}$を取れば,Hausdorff性より
		\begin{align}
			f(x) \in A,\quad g(x) \in B,\quad A \cap B = \emptyset
		\end{align}
		を満たす$T$の開集合$A,B$が存在する.
		$f^{-1}(A) \cap g^{-1}(B)$は$x$の開近傍であり,
		\begin{align}
			f^{-1}(A) \cap g^{-1}(B) \subset \Set{x \in S}{f(x) \neq g(x)}
		\end{align}
		となるから$\Set{x \in S}{f(x) \neq g(x)}$は$S$の開集合である.
		従って$E$は閉である.
		\QED
	\end{prf}
	
	\begin{screen}
		\begin{thm}[正則空間とは交わらないコンパクト集合と閉集合が近傍で分離される空間]
		\label{thm:each_point_in_regular_space_has_closesd_local_base}\mbox{}
			\begin{description}
				\item[(1)] 位相空間において,正則性と,交わらないコンパクト集合と閉集合が近傍で分離されることは同値である.
					
				\item[(2)]
					$K,W$をそれぞれ局所コンパクトな正則空間のコンパクト集合,開集合とするとき,
					相対コンパクトな開集合$U$が存在して次を満たす:
					\begin{align}
						K \subset U \subset \overline{U} \subset W.
						\label{eq:thm_each_point_in_regular_space_has_closesd_local_base}
					\end{align}
			\end{description}
		\end{thm}
	\end{screen}
	
	\begin{prf}\mbox{}
		\begin{description}
			\item[(1)]
				$K,F$を正則空間のコンパクト集合,閉集合とするとき,
				$K \cap F = \emptyset$なら任意の点$x \in K$に対して
				\begin{align}
					x \in U_x,\ \quad F \subset V_x,
					\quad U_x \cap V_x = \emptyset
				\end{align}
				を満たす開集合$U_x,V_x$が取れる.
				$K$はコンパクトであるから或る$\{x_i\}_{i=1}^n \subset K$で
				$K \subset \bigcup_{i=1}^n U_{x_i}$となり
				\begin{align}
					K \subset U \coloneqq \bigcup_{i=1}^n U_{x_i},
					\quad F \subset V \coloneqq \bigcap_{i=1}^n V_{x_i},
					\quad U \cap V = \emptyset
				\end{align}
				が成立する.逆の主張は一点集合がコンパクトであることにより従う.
			\item[(2)]
				任意の$x \in K$に対し,$\overline{U_x} \subset W$
				となる開近傍$U_x$と閉包がコンパクトな開近傍$C_x$が存在するから,
				\begin{align}
					K \subset (C_{y_1} \cap U_{y_1}) \cup \cdots \cup (C_{y_m} \cap U_{y_m})
				\end{align}
				を満たす$\{y_i\}_{i=1}^m \subset K$に対し
				$U \coloneqq \bigcup_{i=1}^m C_{y_i} \cap U_{y_i}$
				とおけば,$\overline{U}$はコンパクトであり
				(\refeq{eq:thm_each_point_in_regular_space_has_closesd_local_base})を満たす.
				\QED
		\end{description}
	\end{prf}
	
	\begin{screen}
		\begin{thm}[局所コンパクトなら$T_2$と$T_3$は同値]
		\label{thm:T_2_equals_to_T_3_in_locally_compact_spaces}
			局所コンパクト位相空間において,$T_2 \Longleftrightarrow T_3$である.
		\end{thm}
	\end{screen}
	
	\begin{prf}
		$T_3$ならば$T_2$であるから$\Longleftarrow$を得る.
		逆に$S$を局所コンパクトHausdorff空間とし,点$x$と閉集合$F$が$x \notin F$を満たしているとする.
		$x$のコンパクトな近傍$K$を取れば,Hausdorff性より$K \cap F$はコンパクトであるから
		\begin{align}
			U_0 \cap V_0 = \emptyset, \quad x \in U_0,  \quad K \cap F \subset V_0
		\end{align}
		を満たす開集合$U_0,V_0$が存在する.このとき,
		\begin{align}
			U \coloneqq U_0 \cap K^{\mathrm{o}},
			\quad V \coloneqq V_0 \cup (S \backslash K)
		\end{align}
		により開集合$U,V$を定めれば
		\begin{align}
			U \cap V = \emptyset,
			\quad x \in U,
			\quad F \subset V
		\end{align}
		が成立し,$S$の正則性が出る.$S$は$T_0$空間でもあるから$T_3$である.
		\QED
	\end{prf}
	
	\begin{screen}
		\begin{thm}[完全正則空間とは交わらないコンパクト集合と閉集合が関数で分離される空間]
			位相空間において,完全正則であることと,交わらないコンパクト集合と閉集合が
			関数で分離されることは同値である.
		\end{thm}
	\end{screen}
	
	\begin{prf}
		$K,C$をそれぞれ完全正則空間$S$のコンパクト部分集合と閉集合とする.
		任意の$x \in K$に対し
		\begin{align}
			f_x(y) = 
			\begin{cases}
				0, & (y=x), \\
				1, & (y \in C)
			\end{cases} 
		\end{align}
		を満たす連続写像$f_x:S \longrightarrow [0,1]$が存在し,
		$K$のコンパクト性より或る$x_1,x_2,\cdots,x_n \in K$で
		\begin{align}
			K \subset \bigcup_{i=1}^n \Set{x \in K}{f_{x_i}(x) < \frac{1}{2}}
		\end{align}
		が成り立つ.$x \in K$なら$\prod_{i=1}^n f_{x_i}(x) < 1/2$,
		$x \in C$なら$\prod_{i=1}^n f_{x_i}(x) = 1$となるから,
		$f \coloneqq \prod_{i=1}^n f_{x_i}$として
		\begin{align}
			g(x) \coloneqq 2 \operatorname{max}\left\{f(x),\frac{1}{2}\right\} - 1
		\end{align}
		により連続写像$g:S \longrightarrow [0,1]$を定めれば
		\begin{align}
			g(x) = 
			\begin{cases}
				0, & (x \in K), \\
				1, & (x \in C)
			\end{cases}
		\end{align}
		が従う.すなわち$K,C$は$g$で分離される.
		一点はコンパクトであるから逆の主張も得られる.
		\QED
	\end{prf}
	
	\begin{screen}
		\begin{thm}[実連続写像の族が生成する始位相は完全正則]
		\label{thm:initial_topology_of_continuous_functions_is_completely_regular}
			$S$を空でない位相空間とし,$S$から$\R$への連続写像の族を$\mathscr{C}$で表す.このとき
			$S$は$\mathscr{C}$-始位相により完全正則空間となる.
		\end{thm}
	\end{screen}
	
	\begin{prf}
		$S$に$\mathscr{C}$-始位相を入れるとき,任意の$x \in S$と$x$を含まない(空でない)始位相の閉集合$F$に対して
		\begin{align}
			x \in \bigcap_{i=1}^n f_i^{-1}(O_i) \subset S \backslash F
		\end{align}
		を満たす$f_i \in \mathscr{C}$と$\R$の開集合$O_i,\ (i=1,\cdots,n)$が取れる.
		ここで$j$を$f_j(x) \in O_j$を満たす添数とすれば
		\begin{align}
			g_j(f_j(x)) = 1;\quad g_i(f_i(y)) = 0,\quad (y \in S \backslash f_i^{-1}(O_i),
			\ i=1,\cdots,n)
		\end{align}
		を満たす連続写像$g_i:\R \longrightarrow [0,1]$が存在する($\R$は完全正則である).
		\begin{align}
			h \coloneqq \operatorname{max}\{g_1 \circ f_1,\, g_2 \circ f_2,\cdots,g_n \circ f_n\}
		\end{align}
		により写像$h:S \longrightarrow [0,1]$を定めれば,
		$h$は$\mathscr{C}$-始位相に関して連続であり,
		$h(x)=1$かつ$h$は$C$上で$0$となる.
		従って$S$は完全正則となる.
		\QED
	\end{prf}
	
	\begin{screen}
		\begin{thm}[完全正則空間の位相は実連続写像全体の始位相に一致する]
			$(S,\mathscr{O})$を空でない位相空間,$C(S)$を実連続写像の全体とし,
			$\mathscr{Z} \coloneqq \Set{f^{-1}(\{0\})}{f \in C(S)}$とおくとき,
			以下は同値となる:
			\begin{description}
				\item[(a)] $S$が完全正則である.
				\item[(b)] $S$の$C(S)$-始位相が$\mathscr{O}$に一致する.
				\item[(c)] $S$の閉集合全体と
					$\Set{\bigcap \mathscr{F}}{\mathscr{F} \subset \mathscr{Z}}$が一致する
					(ただし$\bigcap \emptyset = \emptyset$).
			\end{description}
		\end{thm}
	\end{screen}
		
	\begin{prf}\mbox{}
		\begin{description}
			\item[$(a) \Longrightarrow (c)$]
				$S$が完全正則であるとき,$C=\emptyset$なら
				$f:S \longrightarrow \{1\}$により,
				$C = S$なら$\tilde{f}:S \longrightarrow \{0\}$により
				$C = f^{-1}(\{0\})$となる.$C$が$\emptyset$でも$S$でもない閉集合であるとき,
				任意の$x \in S \backslash C$に対し或る$f_x \in C(S)$で
				\begin{align}
					f_x(y) = \begin{cases}
						1, & (y=x),\\
						0, & (y \in C)
					\end{cases}
				\end{align}
				を満たすものが存在する.このとき$C \subset \bigcup_{x \in S \backslash C} f_x^{-1}(\{0\})$となるが,
				一方で$x \notin C$なら$x \notin f_x^{-1}(\{0\})$より
				\begin{align}
					C = \bigcap_{x \in S \backslash C} f_x^{-1}(\{0\})
					\in \Set{\bigcap \mathscr{F}}{\mathscr{F} \subset \mathscr{Z}}
				\end{align}
				が成り立つ.一方で$f \in C(S)$に対し$f^{-1}(\{0\})$は閉であるから
				$\Set{\bigcap \mathscr{F}}{\mathscr{F} \subset \mathscr{Z}}$
				は$S$の閉集合の族であり,$(c)$が満たされる.
				
			\item[$(c) \Longrightarrow (b)$]
				$C(S)$-始位相は
				\begin{align}
					\bigcup_{f \in C(S)} \Set{f^{-1}(O)}{\mbox{$O$は$\R$の開集合}}
				\end{align}
				で生成されるから$\mathscr{O}$より弱い位相である.
				一方で$(c)$が満たされているとき,任意の空でない$O \in \mathscr{O}$に対し
				或る$\mathscr{F} \subset \Set{f^{-1}(\{0\})}{f \in C(S)}$が存在して
				\begin{align}
					O = \bigcup_{f \in \mathscr{F}} f^{-1}(\R \backslash \{0\})
				\end{align}
				となるから,$O$は$C(S)$-始位相においても開集合となり$(b)$が出る.
			
			\item[$(b) \Longrightarrow (a)$] 
				定理\ref{thm:initial_topology_of_continuous_functions_is_completely_regular}
				より従う.
				\QED
		\end{description}
	\end{prf}
	
	\begin{screen}
		\begin{thm}[正規空間とは交わらない二つの閉集合が関数で分離される空間(Urysohnの補題)]
		\label{thm:Urysohn_lemma}
			位相空間において,正規性と,任意の交わらない二つの閉集合が関数で分離されることは同値である.
		\end{thm}
	\end{screen}
	
	\begin{screen}
		\begin{dfn}[$G_\delta$集合・$F_\sigma$集合]
			位相空間の部分集合で,開集合の可算交叉で表されるものを$G_\delta$集合,
			閉集合の可算和で表されるものを$F_\sigma$集合と呼ぶ.
			特に,任意の閉集合が$G_\delta$である空間では任意の開集合が$F_\sigma$となる.
		\end{dfn}
	\end{screen}
	
	\begin{screen}
		\begin{thm}[完全正規空間とは正規かつ閉集合が全て$G_\delta$である空間]
		\label{thm:perfectly_normal_Hausdorff_is_normal_and_closed_is_G_delta}\mbox{}
			\begin{description}
				\item[(1)]
					$F$を完全正規空間の閉集合とすれば,次を満たす閉集合系$(F_n)_{n=1}^\infty$が存在する:
					\begin{align}
						F = \bigcap_{n=1}^\infty F_n,
						\quad F_n^{\mathrm{o}} \supset F_{n+1}. 
					\end{align}
					
				\item[(2)]
					位相空間において,完全正規であることと,正規かつ任意の閉集合が$G_\delta$であることは同値である.
			\end{description}
		\end{thm}
	\end{screen}
	
	\begin{prf}
		$S$を完全正規空間,$A,B$を互いに交わらない$S$の閉集合とすれば,
		$A=f^{-1}(\{0\}),\ B = f^{-1}(\{1\})$を満たす連続関数
		$f:S \longrightarrow \R$が存在する.このとき
		$U \coloneqq f^{-1}([0,1/2)),\ V \coloneqq f^{-1}((1/2,1])$
		で開集合$U,V$を定めれば
		\begin{align}
			A \subset U,\quad B \subset V,\quad U \cap V = \emptyset
		\end{align}
		となるから$S$は正規である.また$F$を閉集合とすれば
		或る連続関数$g:S \longrightarrow \R,\ (\emptyset = g^{-1}(\{1\}))$により
		\begin{align}
			F = g^{-1}(\{0\}) 
			= g^{-1}\Biggl(\bigcap_{n=1}^\infty\left[0,n^{-1}\right)\Biggr)
			= \bigcap_{n=1}^\infty g^{-1}\left(\left[0,n^{-1}\right)\right)
		\end{align}
		が成立するから$F$は$G_\delta$である.特に,このとき
		$F_n \coloneqq g^{-1}\left(\left[0,n^{-1}\right]\right)$とおけば
		\begin{align}
			F = \bigcap_{n=1}^\infty g^{-1}\left(\left[0,n^{-1}\right]\right)
			= \bigcap_{n=1}^\infty F_n,
			\quad F_n^{\mathrm{o}} \supset g^{-1}\left(\left[0,n^{-1}\right)\right)
			\supset g^{-1}\left(\left[0,(n+1)^{-1}\right]\right)
			= F_{n+1}
		\end{align}
		となり(1)の主張が得られる.逆に$S$が正規かつ
		閉集合が全て$G_\delta$であるとき,任意の交わらない閉集合$A,B$に対し
		$A = \bigcap_{n=1}^\infty U_n,\ B = \bigcap_{n=1}^\infty V_n$
		を満たす開集合系$(U_n)_{n=1}^\infty,\ (V_n)_{n=1}^\infty$が取れて,
		定理\ref{thm:Urysohn_lemma}より各$n \geq 1$で
		\begin{align}
			f_n(A) = \{0\},\quad f_n(S \backslash U_n) = \{1\},
			\quad g_n(B) = \{0\},\quad g_n(S \backslash V_n) = \{1\}
		\end{align}
		を満たす連続写像$f_n,g_n:S \longrightarrow [0,1]$が存在する.
		ここで連続写像を$f \coloneqq \sum_{n=1}^\infty 2^{-n} f_n,\ 
		g \coloneqq \sum_{n=1}^\infty 2^{-n} g_n$で定めれば
		\begin{align}
			\begin{cases}
				f(x) = 0, & (x \in A), \\
				f(x) > 0, & (x \notin A),
			\end{cases}
			\quad \begin{cases}
				g(x) = 0, & (x \in B), \\
				g(x) > 0, & (x \notin B),
			\end{cases}
		\end{align}
		となり,$h \coloneqq f/(f+g)$とおけば$A = h^{-1}(\{0\}),\ B = h^{-1}(\{1\})$が成立する.
		従って$S$は完全正規である.
		\QED
	\end{prf}
	
	\begin{screen}
		\begin{thm}[連続な単射の引き戻しによる分離性の遺伝]
			$S,T$を位相空間とする.$S$から$T$への連続単射が存在するとき,
			$T$が$T_k$-空間$(k=0,1,\cdots,6)$なら
			$S$もまた$T_k$-空間となる.
		\end{thm}
	\end{screen}
	
	\begin{prf}
		任意に異なる二点$s_1,s_2 \in S$を取れば単射性より$f(s_1) \neq f(s_2)$となる.
		$T$の分離性より
	\end{prf}
	
\subsection{可算公理}
	\begin{screen}
		\begin{thm}[可算コンパクト性の同値条件]
		\end{thm}
	\end{screen}
	
	\begin{screen}
		\begin{dfn}[開基]
			位相空間$(S,\mathscr{O})$において,
			$\mathscr{O}$の部分集合$\mathscr{B}$で
			\begin{align}
				\mathscr{O}
				= \Set{\bigcup \mathscr{U}}{\mathscr{U} \subset \mathscr{B}}
			\end{align}
			を満たすもの,ただし$\bigcup \emptyset = \emptyset$,
			を$\mathscr{O}$の開基\index{かいき@開基}や
			基底\index{きてい@基底},基\index{き@基}(base)と呼ぶ.基底は一意に定まるものではない.
			$S \neq \emptyset$のときは$\mathscr{B}$の任意の元は空集合でないとする.
		\end{dfn}
	\end{screen}
	
	\begin{screen}
		\begin{dfn}[可算公理]
			位相空間$S$において,任意の点が高々可算な基本近傍系を持つとき
			$S$は第一可算公理
			\index{だいいちかさんこうり@第一可算公理}
			(the first axiom of countability)を満たす,或は
			$S$は第一可算であるといい,
			$S$が高々可算な基底を持つとき
			$S$は第二可算公理
			\index{だいにかさんこうり@第二可算公理}
			(the second axiom of countability)を満たす,或は
			$S$は第二可算であるという.
		\end{dfn}
	\end{screen}
	空集合(要素数0)を含む任意の有限位相空間は,その冪集合が有限集合であるから
	第二可算公理を満たす.
	
	\begin{screen}
		\begin{thm}[第二可算なら第一可算]
			空でない第二可算空間は第一可算である.
		\end{thm}
	\end{screen}
	
	\begin{prf}
		$\mathscr{B}$を空でない第二可算空間$S$の可算基とするとき,任意の$x \in S$に対して
		\begin{align}
			\mathscr{U}(x) \coloneqq
			\Set{B \in \mathscr{B}}{x \in B}
		\end{align}
		で可算な基本近傍系が定まる.実際
		$x$の任意の近傍$U$に対し或る$B \in \mathscr{B}$で
		\begin{align}
			x \in B \subset U^{\mathrm{o}}
		\end{align}
		が成立し,定義より$B \in \mathscr{U}(x)$が満たされる.
		\QED
	\end{prf}
	
	\begin{screen}
		\begin{dfn}[稠密・可分]
			位相空間$S$において,$\overline{M} = S$を満たすような部分集合$M$を
			$S$で稠密\index{ちゅうみつ@稠密}な(dense)部分集合と呼ぶ.
			また高々可算かつ稠密な部分集合$M$が存在するとき$S$は可分
			\index{かぶん@可分}である(separable)という.
		\end{dfn}
	\end{screen}
	
	\begin{screen}
		\begin{thm}[第二可算なら可分]\label{thm:second_countable_then_separable}
			第二可算位相空間は可分である.
		\end{thm}
	\end{screen}
	
	\begin{prf}
		$\mathscr{B}$を第二可算空間$S$の可算基とするとき,
		$S = \emptyset$なら$\emptyset$は$S$の唯一の部分集合であり,
		要素数$0$かつ$\overline{\emptyset} = \emptyset = S$を満たすから
		$S$は可分である.$S \neq \emptyset$のとき,
		選択関数$\Phi \in \prod \mathscr{B} = \prod_{B \in \mathscr{B}} B$を取り
		\begin{align}
			M \coloneqq \Set{\Phi(B)}{B \in \mathscr{B}}
		\end{align}
		で可算集合を定めれば,任意の$x \in S$及び$x$の任意の近傍$U$に対し
		$x \in B \subset U^{\mathrm{o}}$を満たす
		$B \in \mathscr{B}$が存在して
		\begin{align}
			\Phi(B) \in B \cap M \subset U \cap M
		\end{align}
		となるから,定理\ref{thm:belongs_to_closure_iff_clusters}より
		$S = \overline{M}$が成立する.
		\QED
	\end{prf}
	
	\begin{screen}
		\begin{dfn}[局所有限]
			位相空間$S$の部分集合族$\mathscr{F}$が局所有限
			\index{きょくしょゆうげん@局所有限}
			(locally finite)であるとは,
			任意の$x \in S$が$\mathscr{F}$の高々有限個の元としか交叉しない近傍を持つことである.
			局所有限な部分集合族の可算和で表される部分集合族を$\sigma$-局所有限
			\index{しぐまきょくしょゆうげん@$\sigma$-局所有限}な族という.
		\end{dfn}
	\end{screen}
	
	\begin{screen}
		\begin{dfn}[細分・パラコンパクト]\mbox{}
			\begin{description}
				\item[(1)] $\mathscr{A}$と$\mathscr{B}$を或る集合の被覆とする.
					任意の$B \in \mathscr{B}$に対し$B \subset A$を満たす
					$A \in \mathscr{A}$が存在するとき,
					$\mathscr{B}$を$\mathscr{A}$の細分
					\index{さいぶん@細分}(refinement)と呼ぶ.
					位相空間において,被覆の細分で元が全て開集合であるものを
					開細分\index{かいさいぶん@開細分}と呼ぶ.
					
				\item[(2)] 任意の開被覆が局所有限な開細分を持つ
					位相空間はパラコンパクト\index{ぱらこんぱくと@パラコンパクト}
					(paracompact)であるという.
			\end{description}
		\end{dfn}
	\end{screen}
	
	\begin{screen}
		\begin{thm}[$\sigma$-局所有限な基底が存在すれば第一可算]
			$\sigma$-局所有限な基底が存在する空でない位相空間は第一可算である.
		\end{thm}
	\end{screen}
	
	\begin{prf}
		$S$を空でない位相空間,$\mathscr{B} = \bigcup_{n=1}^\infty \mathscr{B}_n$を
		$\sigma$-局所有限な基底とする(各$\mathscr{B}_n$は局所有限).
		任意の$x \in S$で
		\begin{align}
			\mathscr{U}_n(x) \coloneqq \Set{B \in \mathscr{B}_n}{x \in B},
			\quad \mathscr{U}(x) \coloneqq \bigcup_{n=1}^\infty \mathscr{U}_n(x)
		\end{align}
		と定めれば,局所有限性より$\mathscr{U}_n(x)$は有限集合であるから
		$\mathscr{U}(x)$は可算集合である.また$x$の任意の近傍$U$に対し
		\begin{align}
			x \in B \subset U^{\mathrm{o}}
		\end{align}
		を満たす$B \in \mathscr{B}$が存在し,定義より$B \in \mathscr{U}(x)$
		が成り立つから$\mathscr{U}(x)$は$x$の可算な基本近傍系をなす.
		\QED
	\end{prf}
	
	\begin{screen}
		\begin{thm}[可分空間の局所有限な開集合族は高々可算集合]
		\label{thm:locally_finite_family_of_open_sets_is_countable_in_separable_space}
			$S$を空でない可分位相空間,
			$M$を$S$で稠密な高々可算集合,$\mathscr{B}$を
			$S$の空でない開集合から成る族とするとき,
			\begin{align}
				\mathscr{B} = \bigcup_{m \in M} \Set{B \in \mathscr{B}}{m \in B}
				\label{eq:thm_locally_finite_family_of_open_sets_is_countable_in_separable_space}
			\end{align}
			が成立する.特に$\mathscr{B}$が局所有限なら$\mathscr{B}$は高々可算集合である.
		\end{thm}
	\end{screen}
	
	\begin{prf}
		稠密性より任意の$E \in \mathscr{B}$は
		$E \cap M \neq \emptyset$を満たすから,$m \in E \cap M$で$
		E \in \Set{B \in \mathscr{B}}{m \in B}$となり
		(\refeq{eq:thm_locally_finite_family_of_open_sets_is_countable_in_separable_space})が出る.
		$\mathscr{B}$が局所有限なら$\Set{B \in \mathscr{B}}{m \in B}$は全て有限集合となり
		$\mathscr{B}$は高々可算集合となる.
		\QED	
	\end{prf}
	
	\begin{screen}
		\begin{thm}[$\sigma$-局所有限な基底が存在すれば,可分$\Longleftrightarrow$第二可算]
			$\sigma$-局所有限な基底が存在する空でない位相空間において,
			可分であることと第二可算であることは同値になる.
		\end{thm}
	\end{screen}
	
	\begin{prf}
		空でない可分位相空間において$\sigma$-局所有限な基底が存在するとき,
		定理\ref{thm:locally_finite_family_of_open_sets_is_countable_in_separable_space}
		よりその基底は高々可算集合であるから第二可算性が満たされる.
		逆に第二可算なら可分であるから定理の主張を得る.
		\QED
	\end{prf}
	
	\begin{screen}
		\begin{thm}[第二可算空間の任意の基底は可算基を内包する]\label{thm:countable_base_of_second_countable_space}
			$\mathscr{B}$を第二可算空間$S$の任意の基底とするとき,或る可算部分集合
			$\mathscr{B}_0 \subset \mathscr{B}$もまた$S$の基底となる.
			すなわち第二可算空間はLindel\Ddot{o}f性を持つ.
		\end{thm}
	\end{screen}
	
	\begin{prf}
		$\mathscr{D}$を$S$の可算基とする.
		任意の開集合$U$に対し或る$\mathscr{B}_U \subset \mathscr{B}$が存在して
		$U = \bigcup_{V \in \mathscr{B}_U}V$を満たすから,
		\begin{align}
			\mathscr{D}_U \coloneqq
			\Set{W \in \mathscr{D}}{W \subset V,\ V \in \mathscr{B}_U}
			\label{eq:thm_countable_base_of_second_countable_space_1}
		\end{align}
		とおけば$U = \bigcup_{V \in \mathscr{B}_U} V
			= \bigcup_{V \in \mathscr{B}_U} \bigcup_{\substack{W \in \mathscr{D}_U \\ W \subset V}} W
			\subset \bigcup_{W \in \mathscr{D}_U} W
			\subset U$より
		\begin{align}
			U = \bigcup_{W \in \mathscr{D}_U} W
			\label{eq:thm_countable_base_of_second_countable_space_2}
		\end{align}
		が成り立つ.ここで(\refeq{eq:thm_countable_base_of_second_countable_space_1})より
		任意の$W \in \mathscr{D}_U$に対して
		$\Set{V \in \mathscr{B}}{W \subset V} \neq \emptyset$であるから
		\begin{align}
			\Phi_U \in \prod_{W \in \mathscr{D}_U} \Set{V \in \mathscr{B}}{W \subset V}
		\end{align}
		が取れる.$\mathscr{B}_U' \coloneqq \Set{\Phi_U(W)}{W \in \mathscr{D}_U}$とすれば
		$U = \bigcup_{W \in \mathscr{D}_U} W \subset \bigcup_{W \in \mathscr{D}_U} \Phi(W)
		\subset \bigcup_{V \in \mathscr{B}_U'} V \subset U$より
		\begin{align}
			U = \bigcup_{V \in \mathscr{B}_U'} V
			\label{eq:thm_countable_base_of_second_countable_space_3}
		\end{align}
		が満たされ,
		\begin{align}
			\mathscr{B}_0 \coloneqq \bigcup_{W \in \mathscr{D}} \mathscr{B}_W'
		\end{align}
		と定めれば$\mathscr{B}_0$は求める$S$の可算基となる.実際,任意の開集合$U$に対し
		(\refeq{eq:thm_countable_base_of_second_countable_space_2})と
		(\refeq{eq:thm_countable_base_of_second_countable_space_3})より
		\begin{align}
			U = \bigcup_{W \in \mathscr{D}_U} W
			= \bigcup_{W \in \mathscr{D}_U} \bigcup_{V \in \mathscr{B}_W'} V
		\end{align}
		となる.
		\QED
	\end{prf}
	
	\begin{screen}
		\begin{thm}[局所コンパクトHausdorff空間が第二可算なら$\sigma$-コンパクト]\label{thm:second_countable_Hausdorff_sigma_compact}
			$S$が第二可算性をもつ局所コンパクトHausdorff空間なら,
			次を満たすコンパクト部分集合の列$(K_n)_{n=1}^\infty$が存在する:
			\begin{align}
				K_n \subset K_{n+1}^{\mathrm{o}},
				\quad S = \bigcup_{n=1}^\infty K_n.
			\end{align}
		\end{thm}
	\end{screen}
	
	\begin{prf}
		任意の$x \in S$に対して閉包がコンパクトな開近傍$U_x$を取っておく.
		$\mathscr{O}$を$S$の開集合系として
		\begin{align}
			\mathscr{B} \coloneqq
			\Set{U \in \mathscr{O}}{\mbox{$\overline{U}$がコンパクト}}
		\end{align}
		とおけば,$\mathscr{B}$は$\mathscr{O}$の基底となる.実際,
		任意の$O \in \mathscr{O}$に対し$O \cap U_x \in \mathscr{B}$かつ
		\begin{align}
			O = \bigcup_{x \in O} O \cap U_x
		\end{align}
		となる.従って定理\ref{thm:countable_base_of_second_countable_space}より
		或る可算部分集合$\{U_n\}_{n=1}^\infty \subset \mathscr{B}$が
		$\mathscr{O}$の基底となる.いま,$K_1 \coloneqq \overline{U_1}$として,
		またコンパクト集合$K_n$が選ばれたとして,
		$K_n$の有限被覆$\mathscr{U}_n \subset \mathscr{B}_0$を取り
		\begin{align}
			K_{n+1} \coloneqq \overline{U_{n+1}} \cup \bigcup_{V \in \mathscr{U}_n} \overline{V}
		\end{align}
		とすれば,$K_{n+1}$はコンパクトであり$K_n \subset K_{n+1}^{\mathrm{o}}$を満たす.
		この操作で$(K_n)_{n=1}^\infty$を構成すれば
		\begin{align}
			S = \bigcup_{n=1}^\infty U_n \subset \bigcup_{n=1}^\infty K_n \subset S
		\end{align}
		が成立する.
		\QED
	\end{prf}
	
\subsection{商位相}
	\begin{screen}
		\begin{thm}[商位相]
			位相空間$(S,\mathscr{O})$に同値関係$\sim$が定まっているとき,
			$x \in S$からその同値類$\pi(x)$への対応
			\begin{align}
				\pi: S \ni x \longmapsto \pi(x) \in S/\sim
			\end{align}
			を商写像\index{しょうしゃぞう@商写像}(quotient mapping)という.
			すなわち商写像は
			\begin{align}
				x \sim y \quad \Longleftrightarrow \quad
				\pi(x) = \pi(y)
			\end{align}
			を満たす.また,商写像を連続にする$S/\sim$の最強の位相,つまり
			\begin{align}
				\mathscr{O}(S/\sim) \coloneqq
				\Set{V \subset S/\sim}{\pi^{-1}(V) \in \mathscr{O}}
			\end{align}
			で定まる位相を$S/\sim$の商位相
			\index{しょういそう@商位相}(quotient topology)という.
		\end{thm}
	\end{screen}
	
	\begin{screen}
		\begin{thm}[商空間が$T_1 \Longleftrightarrow$同値類が元の空間で閉じている]
		\label{thm:quotient_space_T_1_iff_each_equivalence_class_closed}
			$S$を位相空間,$\sim$を$S$上の同値関係,$\pi:S \longrightarrow S/\sim$を商写像
			とする.このとき次が成り立つ:
			\begin{align}
				\mbox{$S/\sim$が$T_1$空間である}
				\quad \Longleftrightarrow \quad
				\mbox{任意の$x \in S$に対し$\pi(x)$が$S$の閉集合である}.
			\end{align}
		\end{thm}
	\end{screen}
	
	\begin{prf}
		任意の$F \subset S/\sim$に対し
		\begin{align}
			\mbox{$F$が閉} \quad \Longleftrightarrow \quad
			\mbox{$\pi^{-1}(F^c) = \pi^{-1}(F)^c$が開} \quad \Longleftrightarrow \quad
			\mbox{$\pi^{-1}(F)$が閉}
		\end{align}
		となる.いま任意の$x \in S$に対し
		$\pi(x) = \pi^{-1}(\pi(x))$が満たされているから定理の主張を得る.
		\QED
	\end{prf}
	
	\begin{screen}
		\begin{thm}[商写像が開なら,商空間がHausdorff
		$\Longleftrightarrow$対角線集合が閉]
		\label{thm:quotient_space_Hausdorff_iff_diagonal_set_closed}
			$S$を位相空間,$\sim$を$S$上の同値関係,$\pi:S \longrightarrow S/\sim$を商写像
			とする.このとき,$\pi$が開写像であれば次が成立する:
			\begin{align}
				\mbox{$S/\sim$がHausdorff} \quad \Longleftrightarrow \quad
				\mbox{$\Set{(x,y) \in S \times S}{x \sim y}$が閉}.
			\end{align}
		\end{thm}
	\end{screen}
	
	\begin{prf}
		$S/\sim$がHausdorffであるとき,$x \not\sim y$を満たす$(x,y) \in S \times S$に対し
		$\pi(x) \neq \pi(y)$となるから
		\begin{align}
			\pi(x) \in U,\quad \pi(y) \in V,\quad U \cap V = \emptyset
		\end{align}
		を満たす$S/\sim$の開集合$U,V$が取れる.このとき
		$\pi^{-1}(U) \times \pi^{-1}(V)$は$S \times S$の開集合であり
		\begin{align}
			(x,y) \in \pi^{-1}(U) \times \pi^{-1}(V)
			\subset \Set{(s,t) \in S \times S}{s \not\sim t}
		\end{align}
		が成り立つから$\Longrightarrow$が得られる.
		逆に$\Set{(s,t) \in S \times S}{s \not\sim t}$が開集合であるとき,
		$\pi(x) \neq \pi(y)$なら
		\begin{align}
			(x,y) \in U \times V \subset \Set{(s,t) \in S \times S}{s \not\sim t}
		\end{align}
		を満たす$S$の開集合$U,V$が存在し,このとき
		\begin{align}
			\pi(x) \in \pi(U),\quad \pi(y) \in \pi(V),
			\quad \pi(U) \cap \pi(V) = \emptyset
		\end{align}
		となりかつ$\pi$が開写像であるから$\Longleftarrow$が従う.
		\QED
	\end{prf}
	
	\begin{screen}
		\begin{cor}[Hausdorff
		$\Longleftrightarrow$対角線集合が閉]
		\label{cor:quotient_space_Hausdorff_iff_diagonal_set_closed}
			$S$を位相空間とするとき,
			\begin{align}
				\mbox{$S$がHausdorffである}
				\quad \Longleftrightarrow \quad
				\mbox{$\Set{(x,x)}{x \in S}$が$S \times S$で閉じている}.
			\end{align}
		\end{cor}
	\end{screen}
	
	\begin{prf}
		等号$=$を同値関係と見れば$S$と$S/=$は商写像により同相となるから,
		定理\ref{thm:quotient_space_Hausdorff_iff_diagonal_set_closed}より
		\begin{align}
			\mbox{$S$がHausdorff} \quad \Longleftrightarrow \quad
			\mbox{$S/=$がHausdorff} \quad \Longleftrightarrow \quad
			\mbox{$\Set{(x,x)}{x \in S}$が閉}
		\end{align}
		が成立する.
		\QED
	\end{prf}
\subsection{位相}
	位相とは,集合から二つの要素が与えられたとき,その要素同士がどれだけ``近い''のかを測るためのモノサシたる構造である.
	位相構造が与えられると,例えば次のような概念が生まれる.
	
	$a$と$b$を集合として,それぞれに位相構造が定まっているとする.そして$f$を$a$から$b$への写像とし,$x$を$a$の要素とする.
	いま$a$と$b$には位相構造が定まっているので,$x$と$f(x)$のそれぞれに対して``近所''が把握できるわけであるが,
	ここで$f(x)$の近所をどんなに``狭く''取ったとしても,$x$の近所で,そこの任意の要素$y$で
	\begin{align}
		f(y) \in \mbox{$f(x)$の近所}
	\end{align}
	となるものが取れるとする.つまりイメージとしては,$x$の近所をどんどん狭く取って$f$で移していくと,
	その像は$f(x)$を目指して縮んでいく.このとき,$f$の値は$f(x)$の近くでズレがいくらでも抑えられるので,
	まるで$f(x)$の周りで値が連続的に繋がっているという意味で``$f$は$x$で連続である''と言われる.
	
	\begin{screen}
		\begin{dfn}[位相]
			$S$を集合とし,$\mathscr{O}$を$\power{S}$の部分集合とする.
			$\mathscr{O}$が以下の三カ条を満たすとき,$\mathscr{O}$を$S$上の{\bf 位相}
			\index{いそう@位相}{\bf (topology)}や{\bf 位相構造}
			\index{いそうこうぞう@位相構造}{\bf (topological structure)}と呼ぶ:
			\begin{description}
				\item[(O1)] $\mathscr{O}$は$S$と空集合を要素に持つ:
					\begin{align}
						\emptyset \in \mathscr{O} \wedge S \in \mathscr{O}.
					\end{align}
				\item[(O2)] $\mathscr{O}$は要素の対の交叉で閉じる.つまり,$u$と$v$を$\mathscr{O}$の要素とすると
					\begin{align}
						u \cap v \in \mathscr{O}.
					\end{align}
				\item[(O3)] $\mathscr{O}$は部分集合の合併で閉じる.つまり,$w$を$\mathscr{O}$の部分集合とすると
					\begin{align}
						\bigcup w \in \mathscr{O}.
					\end{align}
			\end{description}
			そして対$(S,\mathscr{O})$を{\bf 位相空間}\index{いそうくうかん@位相空間}{\bf (topological space)}と呼ぶ.
		\end{dfn}
	\end{screen}
	
	$S$を集合とするとき,$\power{S}$と$\{\emptyset,S\}$は共に$S$上の極端な位相構造であり,
	それぞれ{\bf 離散位相}\index{りさんいそう@離散位相}{\bf (discrete topology)}及び
	{\bf 密着位相}\index{みっちゃくいそう@密着位相}{\bf (indiscrete topology)}と呼ばれる.
	通常このような位相を考えることはないが,位相構造は少なくとも二つは取れるという証拠であるため
	以下に続く話が空論ではないと安心できる.
	
	\begin{screen}
		\begin{dfn}[開集合と閉集合]
			$(S,\mathscr{O})$を位相空間とするとき,$\mathscr{O}$の要素を
			$\mathscr{O}$-{\bf 開集合}\index{かいしゅうごう@開集合}{\bf (open set)}と呼び,
			補集合が開である$S$の部分集合,つまり
			\begin{align}
				S \backslash a \in \mathscr{O}
			\end{align}
			なる$S$の部分集合$a$を$\mathscr{O}$-{\bf 閉集合}\index{へいしゅうごう@閉集合}{\bf (closed set)}と呼ぶ.
		\end{dfn}
	\end{screen}
	
	\begin{screen}
		\begin{thm}[閉集合の全体は要素の対の合併と空でない部分集合の交叉で閉じる]
		\label{thm:union_of_two_closed_sets_is_closed_and_intersection_of_closed_sets_is_closed}
			$(S,\mathscr{O})$を位相空間とし,$\mathscr{O}$-閉集合の全体を
			\begin{align}
				\mathscr{A} \defeq \Set{a}{a \subset S \wedge S \backslash a \in \mathscr{O}}
			\end{align}
			とおく.このとき,$u$と$v$を$\mathscr{A}$の要素とすると
			\begin{align}
				u \cup v \in \mathscr{A}.
			\end{align}
			かつ,$w$を$\mathscr{A}$の部分集合とすると
			\begin{align}
				w \neq \emptyset \Longrightarrow \bigcap w \in \mathscr{A}.
			\end{align}
		\end{thm}
	\end{screen}
	
	ちなみに$w = \emptyset$の場合は$\bigcap w = \Univ$となる(定理\ref{thm:union_of_the_emptyset_is_the_Universe}と
	定理\ref{thm:intersections_of_equal_classes_are_equal}).
	
	\begin{sketch}
		$u$と$v$を$\mathscr{A}$の要素とする.定理\ref{thm:difference_of_union_of_two_classes_is_intersection_of_two_differences}より
		\begin{align}
			S \backslash (u \cup v) = (S \backslash u) \cap (S \backslash v)
		\end{align}
		が成り立ち,
		\begin{align}
			(S \backslash u) \cap (S \backslash v) \in \mathscr{O}
		\end{align}
		であるから
		\begin{align}
			u \cup v \in \mathscr{A}
		\end{align}
		が成立する.また$w$を$\mathscr{A}$の空でない部分集合とすると,
		定理\ref{thm:difference_of_intersection_is_union_of_differences_of_elements}より
		\begin{align}
			S \backslash \bigcap w = \bigcup \Set{S \backslash a}{a \in w}
		\end{align}
		が成り立ち,かつ
		\begin{align}
			\Set{S \backslash a}{a \in w} \subset \mathscr{O}
		\end{align}
		が成り立つので,
		\begin{align}
			\bigcup \Set{S \backslash a}{a \in w} \in \mathscr{O}
		\end{align}
		が成り立ち
		\begin{align}
			\bigcap w \in \mathscr{A}
		\end{align}
		が従う.
		\QED
	\end{sketch}
	
	$\C$の部分集合族を
	\begin{align}
		\mathscr{O}_\C \defeq \Set{u}{u \subset \C \wedge \forall x \in u\, \exists r \in \R_+\, 
				\left[\, \forall y\, \left(\, y \in \C \wedge |y-x| < r \Longrightarrow y \in u\, \right)\, \right]}
	\end{align}
	で定めると,これは$\C$上の位相となる.以降は$\mathscr{O}_\C$を$\C$の通常の位相として考える.
	つまり,$\C$の部分集合$O$は,$O$の要素$x$が与えられたときに
	\begin{align}
		\Set{y \in \C}{|x-y| < r} \subset O
	\end{align}
	なる正の実数$r$が取れるなら$\C$の通常の開集合と見做される.
	
	\begin{screen}
		\begin{dfn}[$\C$の位相]
			$\C$の位相構造を
			\begin{align}
				\mathscr{O}_\C \defeq \Set{u}{u \subset \C \wedge \forall x \in u\, \exists r \in \R_+\, 
				\left[\, \forall y\, \left(\, y \in \C \wedge |y-x| < r \Longrightarrow y \in u\, \right)\, \right]}
			\end{align}
			で定める.
		\end{dfn}
	\end{screen}
	
	\begin{screen}
		\begin{dfn}[開核・閉包]
			$(S,\mathscr{O})$を位相空間とし,$\mathscr{A}$を$(S,\mathscr{O})$の閉集合の全体とし,$A$を$S$の部分集合とする.
			このとき,$A$に含まれる$(S,\mathscr{O})$の開集合の全体の合併を
			$A$の$\mathscr{O}$に関する{\bf 開核}\index{かいかく@開核}{\bf (interior)}と呼び,
			$A$を含む$(S,\mathscr{O})$の閉集合の全体の交叉を
			$A$の$\mathscr{O}$に関する{\bf 閉包}\index{へいほう@閉包}{\bf (closure)}と呼ぶ.つまり,
			\begin{align}
				\bigcup \Set{u \in \mathscr{O}}{u \subset A}
			\end{align}
			が$A$の$\mathscr{O}$に関する開核であり,
			\begin{align}
				\bigcap \Set{a \in \mathscr{A}}{A \subset a}
			\end{align}
			が$A$の$\mathscr{O}$に関する閉包である.
		\end{dfn}
	\end{screen}
	
	$(S,\mathscr{O})$を位相空間とし,$\mathscr{A}$を$(S,\mathscr{O})$の閉集合の全体とし,$A$を$S$の部分集合とするとき,
	$A$の$\mathscr{O}$に関する開核を
	\begin{align}
		A^{\mathrm{o}}
	\end{align}
	と書くと
	\begin{align}
		A^{\mathrm{o}} \in \mathscr{O} \wedge A^{\mathrm{o}} \subset A \wedge 
		\forall v \in \mathscr{O}\, \left(\, v \subset A \Longrightarrow v \subset A^{\mathrm{o}}\, \right)
		\label{fom:interior_is_the_largest_open_subset}
	\end{align}
	が成立する.実際,
	\begin{align}
		\Set{u \in \mathscr{O}}{u \subset A} \subset \mathscr{O}
	\end{align}
	であるから
	\begin{align}
		A^{\mathrm{o}} \in \mathscr{O}
	\end{align}
	となり,また定理\ref{thm:union_of_subsets_is_subclass}より
	\begin{align}
		A^{\mathrm{o}} \subset A
	\end{align}
	が成立し,また$v$を
	\begin{align}
		v \subset A
	\end{align}
	なる$(S,\mathscr{O})$の開集合とすれば
	\begin{align}
		v \in \Set{u \in \mathscr{O}}{u \subset A}
	\end{align}
	となるので,定理\ref{thm:union_is_bigger_than_any_member}より
	\begin{align}
		v \subset A^{\mathrm{o}}
	\end{align}
	が成立する.つまり,
	\begin{itembox}[l]{開核の特徴づけ}
		$A$の$\mathscr{O}$に関する開核は,$A$に含まれる$(S,\mathscr{O})$の開集合のうちで包含関係に関して最大のものである.
	\end{itembox}
	
	また$A$の$\mathscr{O}$に関する閉包を
	\begin{align}
		\overline{A}
	\end{align}
	と書くと
	\begin{align}
		\overline{A} \in \mathscr{A} \wedge A \subset \overline{A} \wedge
		\forall v \in \mathscr{A}\, \left(\, A \subset v \Longrightarrow \overline{A} \subset v\, \right)
		\label{fom:closure_is_the_smallest_closed_set}
	\end{align}
	が成立する.実際,
	\begin{align}
		S \in \Set{a \in \mathscr{A}}{A \subset a}
	\end{align}
	であるから
	\begin{align}
		\Set{a \in \mathscr{A}}{A \subset a} \neq \emptyset
	\end{align}
	であって,かつ
	\begin{align}
		\Set{a \in \mathscr{A}}{A \subset a} \subset \mathscr{A}
	\end{align}
	なので,定理\ref{thm:union_of_two_closed_sets_is_closed_and_intersection_of_closed_sets_is_closed}より
	\begin{align}
		\overline{A} \in \mathscr{A}
	\end{align}
	が成立する.また定理\ref{thm:if_obtained_by_all_elements_then_obtained_by_intersection}より
	\begin{align}
		A \subset \overline{A}
	\end{align}
	が成立する.また$v$を
	\begin{align}
		A \subset v
	\end{align}
	なる$(S,\mathscr{O})$の閉集合とすれば
	\begin{align}
		v \in \Set{a \in \mathscr{A}}{A \subset a}
	\end{align}
	となるので,定理\ref{thm:intersection_is_obtained_by_all_elements}より
	\begin{align}
		\overline{A} \subset v
	\end{align}
	が成立する.つまり,
	\begin{itembox}[l]{閉包の特徴づけ}
		$A$の$\mathscr{O}$に関する閉包は,$A$を含む$(S,\mathscr{O})$の閉集合のうちで包含関係に関して最小のものである.
	\end{itembox}
	
	開集合と同様に,閉集合も閉包によって特徴づけられる.
	\begin{screen}
		\begin{thm}[閉集合は自身の閉包に一致する集合である]
		\label{thm:closed_set_coincides_with_its_closure}
			$(S,\mathscr{O})$を位相空間とし,$a$を$S$の部分集合とする.
			$a$の$\mathscr{O}$-閉包を$\overline{a}$と書くとき,
			\begin{align}
				S \backslash a \in \mathscr{O} \Longleftrightarrow a = \overline{a}.
			\end{align}
		\end{thm}
	\end{screen}
	
	\begin{sketch}
		$a$が$\mathscr{O}$-閉集合であれば(\refeq{fom:closure_is_the_smallest_closed_set})より
		\begin{align}
			\overline{a} \subset a
		\end{align}
		が成り立つので
		\begin{align}
			a = \overline{a}
		\end{align}
		が成り立つ.$\mathscr{O}$-閉包は$\mathscr{O}$-閉集合であるから
		\begin{align}
			a = \overline{a} \Longrightarrow S \backslash a \in \mathscr{O}
		\end{align}
		も成り立つ.
		\QED
	\end{sketch}
	
	\begin{screen}
		\begin{thm}[開核の補集合は補集合の閉包]
		\label{thm:topology_note_closure_interior}
			$(S,\mathscr{O})$を位相空間とする.また$A$を$S$の部分集合とするとき,
			\begin{align}
				A^c \defeq S \backslash A
			\end{align}
			として,$A$の$\mathscr{O}$に関する開核を$A^i$と書き,
			$A$の$\mathscr{O}$に関する閉包を$A^a$と書き,
			$(A^i)^c$など連鎖する場合は括弧を省略して$A^{ic}$と書く.このとき次が成り立つ.
			\begin{itemize}
				\item $A^{ic} = A^{ca}.$
				\item $A^{cic} = A^a.$
				\item $A^{ci} = A^{ac}.$
			\end{itemize}
		\end{thm}
	\end{screen}
	
	\begin{prf}
		(\refeq{fom:interior_is_the_largest_open_subset})より
		\begin{align}
			A^i \subset A
		\end{align}
		が成り立つので
		\begin{align}
			A^c \subset A^{ic}
		\end{align}
		が従い,$A^{ic}$は$(S,\mathscr{O})$の閉集合であるから(\refeq{fom:closure_is_the_smallest_closed_set})より
		\begin{align}
			A^{ca} \subset A^{ic} 
		\end{align}
		が成立する.一方で
		\begin{align}
			A^c \subset A^{ca}
		\end{align}
		であるから
		\begin{align}
			A^{cac} \subset A
		\end{align}
		が成り立ち,$A^{cac}$は$(S,\mathscr{O})$の開集合なので(\refeq{fom:interior_is_the_largest_open_subset})より
		\begin{align}
			A^{cac} \subset A^i
		\end{align}
		となる.すなわち
		\begin{align}
			A^{ic} \subset A^{ca}
		\end{align}
		となる.以上で
		\begin{align}
			A^{ic} = A^{ca}
		\end{align}
		が得られた.この式で$A$を$A^c$に替えれば
		\begin{align}
			A^{cic} = A^a
		\end{align}
		が得られ,ゆえに
		\begin{align}
			A^{ci} = A^{ac}
		\end{align}
		も得られる.
		\QED
	\end{prf}
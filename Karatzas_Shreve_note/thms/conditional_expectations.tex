\subsection{条件付期待値}
	\begin{screen}
		\begin{dfn}[条件付期待値]
			$(X,\mathscr{F},\mu)$を測度空間,$f \in L^1(\mu)$とする.
			部分$\sigma$-加法族$\mathscr{G} \subset \mathscr{F}$に対し
			$\nu \coloneqq \left. \mu \right|_{\mathscr{G}}$が$\sigma$-有限であるとき,
			\begin{align}
				\lambda(A) \coloneqq \int_A f\ d\mu,
				\quad (\forall A \in \mathscr{G})
			\end{align}
			により$(X,\mathscr{G})$上に複素測度$\lambda$が定まり,$\lambda \ll \nu$であるから
			Lebesgue-Radon-Nikodymの定理より
			\begin{align}
				\lambda(A) = \int_A g\ d\nu,
				\quad (\forall A \in \mathscr{G})
			\end{align}
			を満たす$g \in L^1(\nu) = L^1\left(X,\mathscr{G},\nu\right)$
			が唯一つ存在する.この$g$を$\mathscr{G}$で条件付けた$f$の条件付期待値と呼び
			\begin{align}
				g = \cexp{f}{\mathscr{G}}
			\end{align}
			と書く.
		\end{dfn}
	\end{screen}
	
	$f \in \mathscr{L}^1(\mu)$が$\mu$-a.e.に$\R$値なら$\lambda$は正値測度となるから,
	定理\ref{thm:mean_value_of_integral_and_closed_set}より
	$\cexp{f}{\mathscr{G}}$も$\nu$-a.e.に$\R$値となる.
	
	\begin{screen}
		\begin{lem}[凸関数の片側微係数の存在]
			任意の凸関数$\varphi:\R \longrightarrow \R$には
			各点で左右の微係数が存在する.特に,凸関数は連続であり,すなわちBorel可測である.
		\end{lem}
	\end{screen}
	
	\begin{prf}
		凸性より任意の$x < y < z$に対して
		\begin{align}
			\frac{\varphi(y) - \varphi(x)}{y - x} 
			\leq \frac{\varphi(z) - \varphi(x)}{z - x}
			\leq \frac{\varphi(z) - \varphi(y)}{z - y}
			\label{ineq:lem:convex_function_measurability_1}
		\end{align}
		が満たされる.従って,$x$を固定すれば,$x$に単調減少に近づく任意の点列$(x_n)_{n=1}^{\infty}$に対し
		 \begin{align}
		 	\left(\frac{f(x_n)-f(x)}{x_n-x}\right)_{n=1}^{\infty} 
		 	\label{seq:lem:convex_function_measurability_2}
		 \end{align}
		 は下に有界な単調減少列となり下限が存在する.$x$に単調減少に近づく別の点列$(y_k)_{k=1}^{\infty}$を取れば
		 \begin{align}
		 	\inf{k \in \Natural}{\frac{f(y_k)-f(x)}{y_k-x}} \leq \frac{f(x_n)-f(x)}{x_n-x} \quad (n=1,2,\cdots)
		 \end{align}
		 より
		 \begin{align}
		 	\inf{k \in \Natural}{\frac{f(y_k)-f(x)}{y_k-x}} \leq \inf{n \in \Natural}{\frac{f(x_n)-f(x)}{x_n-x}}
		 \end{align}
		 が成立し,$(x_n),(y_k)$の立場を変えれば逆向きの不等号も得られる.
		 すなわち極限は点列に依らず確定し,$\varphi$は$x$で右側微係数を持つ.
		 同様に左側微係数も存在し,特に$\varphi$の連続性及びBorel可測性が従う.
		 \QED
	\end{prf}
	
	\begin{screen}
		\begin{thm}[Jensenの不等式]\label{thm:Jensen_inequality_for_convex_functions}
			$(X,\mathscr{F},\mu)$を測度空間,
			$\mathscr{G} \subset \mathscr{F}$を部分$\sigma$-加法族とし,
			$\left. \mu \right|_{\mathscr{G}}$が$\sigma$-有限であるとする.
			このとき,任意の$\mathscr{F}/\borel{\R}$-可測関数$f$と
			凸関数$\varphi:\R \longrightarrow \R$に対し,
			$f,\varphi \circ f$が$\mu$-可積分なら次が成立する:
			\begin{align}
				\varphi \circ \cexp{f}{\mathscr{G}}
				\leq \cexp{\varphi \circ f}{\mathscr{G}}.
			\end{align}
		\end{thm}
	\end{screen}
	
	\begin{prf}
			$\varphi$は各点$x \in \R$で右側接線を持つから,
			それを$\R \ni t \longmapsto a_x t + b_x$と表せば,
			\begin{align}
				\varphi(t) = \sup{r \in \Q}{\left\{ a_r t + b_r \right\}} \quad (\forall t \in \R)
				\label{eq:prp_properties_of_expanded_conditional_expectation_1}
			\end{align}
			が成立する.
			よって任意の$r \in \Q$に対して
			\begin{align}
				\varphi(f(x)) \geq a_r f(x) + b_r
			\end{align}
			が満たされるから
			\begin{align}
				\cexp{\varphi(f)}{\mathscr{G}}
				\geq a_r \cexp{f}{\mathscr{G}} + b_r 
				\quad \mbox{$\mu$-a.e.},
				\quad \forall r \in \Q 
			\end{align}
			が従い,各$r \in \Q$に対し
			\begin{align}
				N_r \coloneqq \Set{x \in X}{\cexp{\varphi(f)}{\mathscr{G}}(x)
				< a_r \cexp{f}{\mathscr{G}}(x) + b_r}
			\end{align}
			とおけば$\mu(N_r) = 0$かつ
			\begin{align}
				\cexp{\varphi(f)}{\mathscr{G}}(x)
				\geq a_r \cexp{f}{\mathscr{G}}(x) + b_r, 
				\quad \forall r \in \Q,\ x \notin \bigcup_{r \in \Q} N_r
			\end{align}
			となる.$r$の任意性と(\refeq{eq:prp_properties_of_expanded_conditional_expectation_1})より
			\begin{align}
				\cexp{\varphi(f)}{\mathscr{G}} \geq \varphi\left( \cexp{f}{\mathscr{G}} \right),
				\quad \mbox{$\mu$-a.e.}
			\end{align}
			が得られる.
			\QED
	\end{prf}
	
	\begin{screen}
		\begin{thm}[条件付き期待値の性質]\label{thm:properties_of_conditional_expectations}
			$(X,\mathscr{F},\mu)$を測度空間,$\mathscr{H},\mathscr{G}$を$\mathscr{H} \subset \mathscr{G}$を満たす
			$\mathscr{F}$の部分$\sigma$-加法族とし,$\theta \coloneqq \left. \mu \right|_{\mathscr{H}},
			\gamma \coloneqq \left. \mu \right|_{\mathscr{G}}$
			がそれぞれ$\sigma$-有限測度であるとする.このとき以下が成立する:
			\begin{description}
				\item[(1)] $\cexp{\cdot}{\mathscr{G}}$は$L^1(X,\mathscr{F},\mu)$から
					$L^1\left(X,\mathscr{G},\gamma\right)$への有界線形作用素であり,
					次を満たす:
					\begin{align}
						|\cexp{f}{\mathscr{G}}| \leq \cexp{|f|}{\mathscr{G}},
						\quad (\forall f \in L^1(\mu)).
						\label{eq:thm_properties_of_conditional_expectations_2}
					\end{align}
				
				\item[(2)] $f \in L^1(\mu),\ g \in L^0(\gamma)$に対して,
					$gf \in L^1(\mu)$なら$g \cexp{f}{\mathscr{G}} \in L^1(\gamma)$であり
					\begin{align}
						\cexp{gf}{\mathscr{G}} = g\cexp{f}{\mathscr{G}}.
						\label{eq:thm_properties_of_conditional_expectations_4}
					\end{align}
					
				\item[(3)] $f \in L^1(\mu)$に対して
					\begin{align}
						\cexp{\cexp{f}{\mathscr{G}}}{\mathscr{H}} = \cexp{f}{\mathscr{H}}.
					\end{align}
					
				\item[(4)] $f \in L^1(\mu) \cap L^p(\mu)$に対し,
					$1 \leq p < \infty$のとき
					\begin{align}
						\left| \cexp{f}{\mathscr{G}} \right|^p
						\leq \cexp{|f|^p}{\mathscr{G}}
					\end{align}
					が満たされ,$1 \leq p \leq \infty$のとき
					\begin{align}
						\Norm{\cexp{f}{\mathscr{G}}}{L^p(\gamma)}
						\leq \Norm{f}{L^p(\mu)}
						\label{eq:thm_properties_of_conditional_expectations_3}
					\end{align}
					も成立する.すなわち$\cexp{\cdot}{\mathscr{G}}$は
					$L^1(\mu) \cap L^p(\mu)$から$L^1(\gamma) \cap L^p(\gamma)$
					への有界線形作用素である.
			\end{description}
		\end{thm}
	\end{screen}
	
	\begin{prf}\mbox{}
		\begin{description}
			\item[(1)]
				任意の$\alpha_1,\alpha_2 \in \C,\ f_1,f_2 \in L^1(\mu)$と$A \in \mathscr{G}$に対して
				\begin{align}
					&\int_A \cexp{\alpha_1 f_1 + \alpha_2 f_2}{\mathscr{G}}\ d\gamma
					= \int_A \alpha_1 f_1 + \alpha_2 f_2\ d\mu
					= \alpha_1 \int_A f_1\ d\mu + \alpha_2 \int_A f_2\ d\mu \\
					&\qquad = \alpha_1 \int_A \cexp{f_1}{\mathscr{G}}\ d\gamma 
						+ \alpha_2 \int_A \cexp{f_2}{\mathscr{G}}\ d\gamma
					= \int_A \alpha_1 \cexp{f_1}{\mathscr{G}} + \alpha_2 \cexp{f_2}{\mathscr{G}}\ d\gamma
				\end{align}
				が成立するから,$L^1(\nu)$で
				\begin{align}
					\cexp{\alpha_1 f_1 + \alpha_2 f_2}{\mathscr{G}}
					= \alpha_1 \cexp{f_1}{\mathscr{G}} + \alpha_2 \cexp{f_2}{\mathscr{G}}
				\end{align}
				となり$\cexp{\cdot}{\mathscr{G}}$の線型性が出る.
				いま,$f \in L^1(\mu),\ g \in L^\infty(\gamma)$に対して
				\begin{align}
					\cexp{gf}{\mathscr{G}} = g\cexp{f}{\mathscr{G}}.
					\label{eq:thm_properties_of_conditional_expectations_1}
				\end{align}
				が成り立つことを示す.実際,任意の$A,B \in \mathscr{G}$に対して
				\begin{align}
					\int_A \defunc_B f\ d\mu
					= \int_{A \cap B} f\ d\mu
					= \int_{A \cap B} \cexp{f}{\mathscr{G}}\ d\gamma
					= \int_A \defunc_B \cexp{f}{\mathscr{G}}\ d\gamma
				\end{align}
				となるから,$g$の単関数近似列$(g_n)_{n=1}^\infty,\ \left(g_n \in L^\infty(\gamma),|g_n| \leq |g|\right)$に対して
				\begin{align}
					\int_A g_n f\ d\mu = \int_A g_n \cexp{f}{\mathscr{G}}\ d\gamma,
					\quad (\forall n \geq 1)
				\end{align}
				が成り立ち,$gf \in L^1(\mu)$かつ$g\cexp{f}{\mathscr{G}} \in L^1(\gamma)$であるから
				Lebesgueの収束定理より
				\begin{align}
					\int_A g \cexp{f}{\mathscr{G}}\ d\gamma
					= \int_A g f\ d\mu
					= \int_A \cexp{g f}{\mathscr{G}}\ d\gamma
				\end{align}
				が従い(\refeq{eq:thm_properties_of_conditional_expectations_1})が得られる.
				ここで$f \in L^1(\mu)$に対し
				\begin{align}
					\alpha \coloneqq \defunc_{\left\{\cexp{f}{\mathscr{G}} \neq 0\right\}} 
						\frac{\overline{\cexp{f}{\mathscr{G}}}}{|\cexp{f}{\mathscr{G}}|}
				\end{align}
				により$\alpha \in L^\infty(\gamma)$を定めれば,任意の$A \in \mathscr{G}$に対して
				\begin{align}
					\int_A |\cexp{f}{\mathscr{G}}|\ d\gamma
					&= \int_A \alpha \cexp{f}{\mathscr{G}}\ d\gamma
					= \int_A \cexp{\alpha f}{\mathscr{G}}\ d\gamma \\
					&= \int_A \alpha f\ d\mu
					\leq \int_A |f|\ d\mu
					= \int_A \cexp{|f|}{\mathscr{G}}\ d\gamma
				\end{align}
				が成り立つから,(\refeq{eq:thm_properties_of_conditional_expectations_2})及び
				$\cexp{\cdot}{\mathscr{G}}$の有界性が得られる.
				
			\item[(2)]
				$(g_n)_{n=1}^\infty$を$g$の単関数近似列とすれば,
				単調収束定理と(\refeq{eq:thm_properties_of_conditional_expectations_1})より
				\begin{align}
					\int_X |g|\cexp{|f|}{\mathscr{G}}\ d\gamma
					= \lim_{n \to \infty} \int_X |g_n|\cexp{|f|}{\mathscr{G}}\ d\gamma
					= \lim_{n \to \infty} \int_X |g_n||f|\ d\mu
					= \int_X |g||f|\ d\mu
				\end{align}
				となり$g\cexp{f}{\mathscr{G}}$の可積分性が従う.従って,
				Lebesgueの収束定理より任意の$A \in \mathscr{G}$に対して
				\begin{align}
					\int_A g\cexp{f}{\mathscr{G}}\ d\gamma
					= \lim_{n \to \infty} \int_A g_n\cexp{f}{\mathscr{G}}\ d\gamma
					= \lim_{n \to \infty} \int_A g_n f\ d\mu
					= \int_A gf\ d\mu
				\end{align}
				が成り立ち(\refeq{eq:thm_properties_of_conditional_expectations_4})が得られる.
				
			\item[(3)]
				任意の$A \in \mathscr{H}$に対して
				\begin{align}
					\int_A \cexp{f}{\mathscr{H}}\ d\theta
					= \int_A f\ d\mu
					= \int_A \cexp{f}{\mathscr{G}}\ d\gamma
					= \int_A \cexp{\cexp{f}{\mathscr{G}}}{\mathscr{H}}\ d\theta
				\end{align}
				が成立する.
				
			\item[(4)] 
				$1 \leq p < \infty$の場合,
				(\refeq{eq:thm_properties_of_conditional_expectations_2})とJensenの不等式より
				\begin{align}
					|\cexp{f}{\mathscr{G}}|^p \leq \cexp{|f|}{\mathscr{G}}^p
					\leq \cexp{|f|^p}{\mathscr{G}}
				\end{align}
				が成り立つ.$p = \infty$の場合は任意の$A \in \mathscr{G}$に対して
				\begin{align}
					\int_A \left| \cexp{f}{\mathscr{G}} \right|\ d\gamma
					\leq \int_A |f|\ d\mu
					\leq \mu(A) \Norm{f}{L^\infty}
					= \gamma(A) \Norm{f}{L^\infty}
				\end{align}
				となり,$1 \leq p < \infty$の場合も込めて(\refeq{eq:thm_properties_of_conditional_expectations_3})が従う.
				\QED
		\end{description}
	\end{prf}
	
	$\varphi:\R \longrightarrow \R$なる写像$\varphi$で,$-\varphi$が凸であるものを
	{\bf 凹関数}\index{おうかんすう@凹関数}{\bf (concave function)}と呼ぶ.
	
	\begin{screen}
		\begin{thm}[凹関数に対するJensenの不等式]\label{thm:Jensen_inequality_for_concave_functions}
			$(X,\mathscr{F},\mu)$を測度空間,
			$\mathscr{G} \subset \mathscr{F}$を部分$\sigma$-加法族とし,
			$\left. \mu \right|_{\mathscr{G}}$が$\sigma$-有限であるとする.
			このとき,任意の$\mathscr{F}/\borel{\R}$-可測関数$f$と
			凹関数$\varphi:\R \longrightarrow \R$に対し,
			$f,\varphi \circ f$が$\mu$-可積分なら次が成立する:
			\begin{align}
				\cexp{\varphi \circ f}{\mathscr{G}}
				\leq \varphi \circ \cexp{f}{\mathscr{G}}.
			\end{align}
		\end{thm}
	\end{screen}
	
	\begin{prf}
		定理\ref{thm:Jensen_inequality_for_concave_functions}より
		$\mu$-a.e.の$x( \in X)$で
		\begin{align}
			-\varphi\left(\cexp{f}{\mathscr{G}}(x)\right) \leq \cexp{-\varphi \circ f}{\mathscr{G}}(x)
		\end{align}
		が成立する.条件付き期待値は線形作用素であるから
		\begin{align}
			\cexp{\varphi \circ f}{\mathscr{G}}(x) \leq \varphi\left(\cexp{f}{\mathscr{G}}(x)\right)
		\end{align}
		が従う.
		\QED
	\end{prf}
	
	\begin{screen}
		\begin{thm}
			\begin{description}
				\item[(1)]
					$X_n \leq X_{n+1}$
					$X_n \longrightarrow X\ a.s.P$
					$\cexp{X_n}{\mathscr{G}} \longrightarrow \cexp{X}{\mathscr{G}}\ a.s.P$
				\item[(2)]
					$X_n \geq 0$
					$\cexp{\liminf X_n}{\mathscr{G}} \leq \liminf \cexp{X_n}{\mathscr{G}}$
				\item[(3)]
					$|X_n| \leq Y$ $X_n \longrightarrow X\ a.s.P$
					$\cexp{X_n}{\mathscr{G}} \longrightarrow \cexp{X}{\mathscr{G}}\ a.s.P$
			\end{description}
		\end{thm}
	\end{screen}
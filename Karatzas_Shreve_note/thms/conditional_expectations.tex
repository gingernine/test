\section{条件付き期待値}
	\begin{screen}
		\begin{lem}
			$(X,\mathscr{F},\mu)$を$\sigma$-有限測度空間 $(\mu(X) > 0)$とするとき,
			$0 < w < 1$を満たす可積分関数$w$が存在する.
		\end{lem}
	\end{screen}
	
	\begin{prf}
		$\sigma$-有限の仮定より
		\begin{align}
			0 < \mu(X_n) < \infty,\ (\forall n \geq 1),
			\quad X = \bigcup_{n=1}^\infty X_n
		\end{align}
		を満たす$\{X_n\}_{n=1}^\infty \subset \mathscr{F}$が存在する.ここで
		\begin{align}
			w_n(x) \coloneqq
			\begin{cases}
				\displaystyle\frac{1}{2^n\left(1+\mu(X_n)\right)}, & x \in X_n, \\
				0, & x \in X \backslash X_n,
			\end{cases}
			\quad n=1,2,\cdots
		\end{align}
		に対して
		\begin{align}
			w \coloneqq \sum_{n=1}^\infty w_n
		\end{align}
		と定めれば,任意の$x \in X$は或る$X_n$に属するから
		\begin{align}
			0 < w_n(x) \leq w(x)
		\end{align}
		が成り立ち,かつ
		\begin{align}
			w(x) = w_1(x) + \sum_{n=2}^\infty w_n(x)
			\leq \frac{1}{2\left(1+\mu(X_1)\right)} + \frac{1}{2}
			< 1,
			\quad (\forall x \in X)
		\end{align}
		が満たされる.また単調収束定理より
		\begin{align}
			\int_X w\ d\mu \leq \sum_{n=1}^\infty \int_X w_n\ d\mu
			\leq \sum_{n=1}^\infty \frac{\mu(X_n)}{2^n\left(1+\mu(X_n)\right)}
			\leq 1
		\end{align}
		となり$w$の可積分性が出る.
		\QED
	\end{prf}
	
	\begin{screen}
		\begin{thm}[Lebesgue-Radon-Nikodym]
			$(X,\mathscr{F})$を可測空間,$\lambda$を$(X,\mathscr{F})$上の複素測度,
			$\mu$を$(X,\mathscr{F})$上の$\sigma$-有限正値測度とするとき,以下が成立する:
			\begin{description}
				\item[Lebesgue分解]
					$\lambda$は$\mu$に関して絶対連続な$\lambda_a$及び$\mu$と互いに特異な
					$\lambda_s$に一意に分解される:
					\begin{align}
						\lambda = \lambda_a + \lambda_s,
						\quad \lambda_a \ll \mu,
						\quad \lambda_s \perp \mu.
					\end{align}
				
				\item[密度関数の存在]
					$\lambda_a$に対し或る$g \in L^1(\mu) = L^1(X,\mathscr{F},\mu)$が唯一つ存在して次を満たす:
					\begin{align}
						\lambda_a(E) = \int_E g\ d\mu,
						\quad (\forall E \in \mathscr{F}).
					\end{align}
			\end{description}
		\end{thm}
	\end{screen}
	
	\begin{prf}\mbox{}
		\begin{description}
			\item[第一段] Lebesgueの分解の一意性を示す.
				$\lambda'_a \ll \mu$と$\lambda'_s \perp \mu$により
				\begin{align}
					\lambda_a + \lambda_s = \lambda'_a + \lambda'_s
				\end{align}
				が成り立つとき,
				\begin{align}
					\Lambda \coloneqq \lambda_a - \lambda'_a = \lambda'_s - \lambda_s,
					\quad \Lambda \ll \mu,
					\quad \Lambda \perp \mu
				\end{align}
				となり$\Lambda = 0$が従い分解の一意性が出る.
			
			\item[第二段] 密度関数の一意性は
			\item[第三段] Lebesgueの分解と密度関数の存在を示す.
		\end{description}
	\end{prf}
	
	\begin{screen}
		\begin{dfn}[条件付き期待値]
			$(X,\mathscr{F},\mu)$を測度空間,$f \in L^1(\mu)$とする.
			部分$\sigma$-加法族$\mathscr{G} \subset \mathscr{F}$に対し
			$\nu \coloneqq \left. \mu \right|_{\mathscr{G}}$が$\sigma$-有限であるとき,
			\begin{align}
				\lambda(A) \coloneqq \int_A f\ d\mu,
				\quad (\forall A \in \mathscr{G})
			\end{align}
			により$(X,\mathscr{G})$上に複素測度$\lambda$が定まり,$\lambda \ll \nu$であるから
			Lebesgue-Radon-Nikodymの定理より
			\begin{align}
				\lambda(A) = \int_A g\ d\nu,
				\quad (\forall A \in \mathscr{G})
			\end{align}
			を満たす$g \in L^1(\nu) = L^1\left(X,\mathscr{G},\nu\right)$
			が存在する.この$g$を$\mathscr{G}$で条件付けた$f$の条件付き期待値と呼び
			\begin{align}
				g = \cexp{f}{\mathscr{G}}
			\end{align}
			と書く.
		\end{dfn}
	\end{screen}
	
	\begin{screen}
		\begin{thm}
			\begin{description}
				\item[(1)]
					$X_n \leq X_{n+1}$
					$X_n \longrightarrow X\ a.s.P$
					$\cexp{X_n}{\mathscr{G}} \longrightarrow \cexp{X}{\mathscr{G}}\ a.s.P$
				\item[(2)]
					$X_n \geq 0$
					$\cexp{\liminf X_n}{\mathscr{G}} \leq \liminf \cexp{X_n}{\mathscr{G}}$
				\item[(3)]
					$|X_n| \leq Y$ $X_n \longrightarrow X\ a.s.P$
					$\cexp{X_n}{\mathscr{G}} \longrightarrow \cexp{X}{\mathscr{G}}\ a.s.P$
			\end{description}
		\end{thm}
	\end{screen}
	
	\begin{screen}
	\begin{lem}[凸関数の片側微係数の存在]
		任意の凸関数$\varphi:\R \longrightarrow \R$には
		各点で左右の微係数が存在する.特に,凸関数は連続であり,すなわちBorel可測である.
	\end{lem}
	\end{screen}
	
	\begin{prf}
		凸性より任意の$x < y < z$に対して
		\begin{align}
			\frac{\varphi(y) - \varphi(x)}{y - x} 
			\leq \frac{\varphi(z) - \varphi(x)}{z - x}
			\leq \frac{\varphi(z) - \varphi(y)}{z - y}
			\label{ineq:lem:convex_function_measurability_1}
		\end{align}
		が満たされる.従って,$x$を固定すれば,$x$に単調減少に近づく任意の点列$(x_n)_{n=1}^{\infty}$に対し
		 \begin{align}
		 	\left(\frac{f(x_n)-f(x)}{x_n-x}\right)_{n=1}^{\infty} 
		 	\label{seq:lem:convex_function_measurability_2}
		 \end{align}
		 は下に有界な単調減少列となり下限が存在する.$x$に単調減少に近づく別の点列$(y_k)_{k=1}^{\infty}$を取れば
		 \begin{align}
		 	\inf{k \in \N}{\frac{f(y_k)-f(x)}{y_k-x}} \leq \frac{f(x_n)-f(x)}{x_n-x} \quad (n=1,2,\cdots)
		 \end{align}
		 より
		 \begin{align}
		 	\inf{k \in \N}{\frac{f(y_k)-f(x)}{y_k-x}} \leq \inf{n \in \N}{\frac{f(x_n)-f(x)}{x_n-x}}
		 \end{align}
		 が成立し,$(x_n),(y_k)$の立場を変えれば逆向きの不等号も得られる.
		 すなわち極限は点列に依らず確定し,$\varphi$は$x$で右側微係数を持つ.
		 同様に左側微係数も存在し,特に$\varphi$の連続性及びBorel可測性が従う.
		 \QED
	\end{prf}
	
	\begin{screen}
	\begin{thm}[Jensenの不等式]
		$(X,\mathscr{F},\mu)$を測度空間,
		$\mathscr{G} \subset \mathscr{F}$を部分$\sigma$-加法族とし,
		$\left. \mu \right|_{\mathscr{G}}$が$\sigma$-有限であるとする.
		このとき,任意の可積分関数
		$f:X \longrightarrow \R$と
		凸関数$\varphi:\R \longrightarrow \R$に対し,
		$\varphi(f)$が可積分なら次が成立する:
		\begin{align}
			\varphi\left(\cexp{f}{\mathscr{G}} \right)
			\leq \cexp{\varphi(f)}{\mathscr{G}},
			\quad \mbox{$\mu$-a.e.}
		\end{align}
	\end{thm}
	\end{screen}
	
	\begin{prf}
			$\varphi$は各点$x \in \R$で右側接線を持つから,
			それを$\R \ni t \longmapsto a_x t + b_x$と表せば,
			\begin{align}
				\varphi(t) = \sup{r \in \Q}{\left\{ a_r t + b_r \right\}} \quad (\forall t \in \R)
				\label{eq:prp_properties_of_expanded_conditional_expectation_1}
			\end{align}
			が成立する.
			よって任意の$r \in \Q$に対して
			\begin{align}
				\varphi(f(x)) \geq a_r f(x) + b_r
			\end{align}
			が満たされるから
			\begin{align}
				\cexp{\varphi(f)}{\mathscr{G}}
				\geq a_r \cexp{f}{\mathscr{G}} + b_r 
				\quad \mbox{$\mu$-a.e.},
				\quad \forall r \in \Q 
			\end{align}
			が従い,各$r \in \Q$に対し
			\begin{align}
				N_r \coloneqq \Set{x \in X}{\cexp{\varphi(f)}{\mathscr{G}}(x)
				< a_r \cexp{f}{\mathscr{G}}(x) + b_r}
			\end{align}
			とおけば$\mu(N_r) = 0$かつ
			\begin{align}
				\cexp{\varphi(f)}{\mathscr{G}}(x)
				\geq a_r \cexp{f}{\mathscr{G}}(x) + b_r, 
				\quad \forall r \in \Q,\ x \notin \bigcup_{r \in \Q} N_r
			\end{align}
			となる.$r$の任意性と(\refeq{eq:prp_properties_of_expanded_conditional_expectation_1})より
			\begin{align}
				\cexp{\varphi(f)}{\mathscr{G}} \geq \varphi\left( \cexp{f}{\mathscr{G}} \right),
				\quad \mbox{$\mu$-a.e.}
			\end{align}
			が得られる.
			\QED
	\end{prf}
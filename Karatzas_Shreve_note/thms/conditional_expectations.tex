\section{条件付き期待値}
	\begin{screen}
		\begin{lem}\label{lem:Lebesgue_Radon_Nikodym}
			$(X,\mathscr{F},\mu)$を$\sigma$-有限測度空間とするとき,
			$0 < w < 1$を満たす可積分関数$w$が存在する.
		\end{lem}
	\end{screen}
	
	\begin{prf}
		$\mu(X) = 0$なら$w \equiv 1/2$とすればよい.$\mu(X) > 0$の場合,$\sigma$-有限の仮定より
		\begin{align}
			0 < \mu(X_n) < \infty,\ (\forall n \geq 1),
			\quad X = \bigcup_{n=1}^\infty X_n
		\end{align}
		を満たす$\{X_n\}_{n=1}^\infty \subset \mathscr{F}$が存在する.ここで
		\begin{align}
			w_n(x) \coloneqq
			\begin{cases}
				\displaystyle\frac{1}{2^n\left(1+\mu(X_n)\right)}, & x \in X_n, \\
				0, & x \in X \backslash X_n,
			\end{cases}
			\quad n=1,2,\cdots
		\end{align}
		に対して
		\begin{align}
			w \coloneqq \sum_{n=1}^\infty w_n
		\end{align}
		と定めれば,任意の$x \in X$は或る$X_n$に属するから
		\begin{align}
			0 < w_n(x) \leq w(x)
		\end{align}
		が成り立ち,かつ
		\begin{align}
			w(x) = w_1(x) + \sum_{n=2}^\infty w_n(x)
			\leq \frac{1}{2\left(1+\mu(X_1)\right)} + \frac{1}{2}
			< 1,
			\quad (\forall x \in X)
		\end{align}
		が満たされる.また単調収束定理より
		\begin{align}
			\int_X w\ d\mu \leq \sum_{n=1}^\infty \int_X w_n\ d\mu
			\leq \sum_{n=1}^\infty \frac{\mu(X_n)}{2^n\left(1+\mu(X_n)\right)}
			\leq 1
		\end{align}
		となり$w$の可積分性が出る.
		\QED
	\end{prf}
	
	\begin{screen}
		\begin{thm}[Lebesgue-Radon-Nikodym]
			$(X,\mathscr{F})$を可測空間,$\lambda$を$(X,\mathscr{F})$上の複素測度,
			$\mu$を$(X,\mathscr{F})$上の$\sigma$-有限正値測度とするとき,以下が成立する:
			\begin{description}
				\item[Lebesgue分解]
					$\lambda$は$\mu$に関して絶対連続な$\lambda_a$及び$\mu$と互いに特異な
					$\lambda_s$に一意に分解される:
					\begin{align}
						\lambda = \lambda_a + \lambda_s,
						\quad \lambda_a \ll \mu,
						\quad \lambda_s \perp \mu.
					\end{align}
				
				\item[密度関数の存在]
					$\lambda_a$に対し或る$g \in L^1(\mu) = L^1(X,\mathscr{F},\mu)$が唯一つ存在して次を満たす:
					\begin{align}
						\lambda_a(E) = \int_E g\ d\mu,
						\quad (\forall E \in \mathscr{F}).
					\end{align}
			\end{description}
		\end{thm}
	\end{screen}
	
	\begin{prf}\mbox{}
		\begin{description}
			\item[第一段] Lebesgueの分解の一意性を示す.
				$\lambda'_a \ll \mu$と$\lambda'_s \perp \mu$により
				\begin{align}
					\lambda_a + \lambda_s = \lambda'_a + \lambda'_s
				\end{align}
				が成り立つとき,
				\begin{align}
					\Lambda \coloneqq \lambda_a - \lambda'_a = \lambda'_s - \lambda_s,
					\quad \Lambda \ll \mu,
					\quad \Lambda \perp \mu
				\end{align}
				となり$\Lambda = 0$が従い分解の一意性が出る.
			
			\item[第二段] 密度関数の一意性を示す.実際,可積分関数$f$に対して
				\begin{align}
					\int_E f\ d\mu = 0, \quad (\forall E \in \mathscr{F})
				\end{align}
				が成り立つとき,$u \coloneqq \Re{f}$に対して
				\begin{align}
					E_1 \coloneqq \{u > 0\},
					\quad E_2 \coloneqq \{u < 0\}
				\end{align}
				とおけば
				\begin{align}
					0 = \left| \int_{E_j} f\ d\mu \right| \geq \left| \int_{E_j} u\ d\mu \right|,
					\quad (j=1,2)
				\end{align}
				より$\mu(E_1) = \mu(E_2) = 0$となり,同様に$\Im{f} = 0\ \mbox{$\mu$-a.e.}$も従うから
				$f = 0,\ \mbox{$\mu$-a.e.}$が成り立つ.
				
			\item[第三段] Lebesgueの分解と密度関数の存在を示す.
				
		\end{description}
	\end{prf}
	
	\begin{screen}
		\begin{thm}[$L^p$の共役空間]\label{thm:dual_space_of_L_p}
			$1 \leq p < \infty$,$q$を$p$の共役指数とし,また$(X,\mathscr{F},\mu)$を$\sigma$-有限な測度空間とするとき,
			$g \in L^q(\mu)$に対して
			\begin{align}
				\Phi_g: L^p(\mu) \ni f \longmapsto \int_X fg\ d\mu
				\label{eq:thm_dual_space_of_L_p_1}
			\end{align}
			は有界線形作用素となる.また
			\begin{align}
				\Phi: L^q(\mu) \ni g \longmapsto \Phi_g \in \left( L^p(\mu) \right)^*,
				\quad \left( \Phi_g(f) = \int_X fg\ d\mu,\quad \forall f \in L^p(\mu) \right)
			\end{align}
			で定める$\Phi$は$\left( L^p(\mu) \right)^*$から$L^q(\mu)$への線型同型であり,
			次の意味で等長である:
			\begin{align}
				\Norm{g}{L^q(\mu)} = \Norm{\Phi_g}{\left( L^p(\mu) \right)^*}.
				\label{eq:thm_dual_space_of_L_p_asseretion_2}
			\end{align}
			$p=\infty$の場合,$\mu(X) < \infty$かつ$\varphi \in \left( L^\infty(\mu) \right)^*$に対し
			$\mathscr{F} \ni A \longmapsto \varphi(\defunc_A)$が可算加法的ならば,
			$\varphi$に対し或る$g \in L^1(\mu)$が唯一つ存在して
			$\varphi = \Phi_g$と(\refeq{eq:thm_dual_space_of_L_p_asseretion_2})を満たす.
		\end{thm}
	\end{screen}
	
	\begin{prf}\mbox{}
		\begin{description}
			\item[第一段]
				$\Phi_g$が(\refeq{eq:thm_dual_space_of_L_p_1})で与えられていれば,H\Ddot{o}lderの不等式より
				\begin{align}
					\left|\Phi_g(f)\right| \leq \Norm{g}{L^q(\mu)}\Norm{f}{L^p(\mu)}
				\end{align}
				が成り立つから
				\begin{align}
					\Norm{\Phi_g}{\left( L^p(\mu) \right)^*} \leq \Norm{g}{L^q(\mu)}
					\label{eq:thm_dual_space_of_L_p_3}
				\end{align}
				が従う.よって$\Phi_g \in \left( L^p(\mu) \right)^*$となる.
			
			\item[第二段]
				$\varphi \in \left( L^p(\mu) \right)^*$に対して
				$\Phi(g) = \varphi$を満たす$g \in L^q(\mu)$が存在するとき,
				$g$が$\varphi$に対して一意に決まることを示す.$\sigma$-有限の仮定より
				\begin{align}
					\mu(X_n) < \infty,\ (\forall n \geq 1);
					\quad X = \bigcup_{n=1}^\infty X_n
					\label{eq:thm_dual_space_of_L_p_6}
				\end{align}
				を満たす$\{X_n\}_{n=1}^\infty \subset \mathscr{F}$が存在する.
				いま,$g,g' \in L^q(\mu)$に対して
				\begin{align}
					\int_X fg\ d\mu = \int_X fg'\ d\mu,
					\quad (\forall f \in L^p(\mu))
				\end{align}
				が成り立っているとすれば,任意の$E \in \mathscr{F}$に対して
				$\defunc_{E \cap X_n} \in L^p(\mu)$であるから
				\begin{align}
					\int_{E \cap X_n} g-g'\ d\mu = 0,
					\quad (\forall n \geq 1)
				\end{align}
				となり,Lebesgueの収束定理より
				\begin{align}
					\int_E g-g'\ d\mu = 0
				\end{align}
				が従い$L^q(\mu)$で$g = g'$が成立する.
				
			\item[第三段]
				$1 \leq p < \infty$の場合,$\mu(X) < \infty$なら
				任意の$\varphi \in \left( L^p(\mu) \right)^*$に対して
				$\Phi(g) = \varphi$を満たす$g \in L^q(\mu)$が存在することを示す.
				\begin{align}
					\lambda(E) \coloneqq \varphi(\defunc_E)
					\label{eq:thm_dual_space_of_L_p_7}
				\end{align}
				により$\lambda$を定めれば
				\begin{align}
					\lambda(A + B) = \varphi(\defunc_{A+B}) = \varphi(\defunc_A + \defunc_B)
					= \varphi(\defunc_A) + \varphi(\defunc_B)
					= \lambda(A) + \lambda(B)
				\end{align}
				となり$\lambda$の加法性が出る.また
				任意の互いに素な$\{E_n\}_{n=1}^\infty \in \mathscr{F}$に対して
				\begin{align}
					A_k \coloneqq \sum_{n=1}^k E_n,
					\quad A \coloneqq \sum_{n=1}^\infty E_n
				\end{align}
				とおけば
				\begin{align}
					\left| \lambda(A) - \sum_{n=1}^k \lambda(E_n) \right|
					&= \left| \lambda(A) - \lambda(A_k) \right|
					= \left| \varphi(\defunc_A - \defunc_{A_k}) \right| \\
					&\leq \Norm{\varphi}{\left( L^p(\mu) \right)^*} \Norm{\defunc_A - \defunc_{A_k}}{L^p(\mu)}
					= \Norm{\varphi}{\left( L^p(\mu) \right)^*} \mu(A - A_k)^{1/p}
					\longrightarrow 0
					\quad (k \longrightarrow \infty)
				\end{align}
				が成り立つから$\lambda$は複素測度である.また
				\begin{align}
					|\lambda(E)| \leq \Norm{\varphi}{\left( L^p(\mu) \right)^*} \mu(E)^{1/p}
				\end{align}
				より$\lambda \ll \mu$となるから,Lebesgue-Radon-Nikodymの定理より
				\begin{align}
					\varphi(\defunc_E) = \lambda(E) = \int_X \defunc_E g\ d\mu,
					\quad (\forall E \in \mathscr{F})
					\label{eq:thm_dual_space_of_L_p_8}
				\end{align}
				を満たす$g \in L^1(\mu)$が存在する.$\varphi$の線型性より
				任意の単関数の同値類$f$に対して
				\begin{align}
					\varphi(f) = \int_X fg\ d\mu
					\label{eq:thm_dual_space_of_L_p_2}
				\end{align}
				が成立し,特に$f \in L^\infty(\mu)$に対しては
				\begin{align}
					B \coloneqq \Set{x \in X}{|f(x)| > \Norm{f}{L^\infty(\mu)}}
				\end{align}
				とおけば$\mu(B) = 0$となり,有界可測関数$f \defunc_{X \backslash B}$を
				一様に近似する単関数列$(f_n)_{n=1}^\infty$が存在して
				\begin{align}
					\left| \varphi(f) - \int_X fg\ d\mu \right|
					&\leq \left| \varphi(f) - \varphi(f_n) \right| + \left| \int_X f_ng\ d\mu - \int_X fg\ d\mu \right| \\
					&\leq \Norm{\varphi}{\left( L^p(\mu) \right)^*} \Norm{f - f_n}{L^p(\mu)}
						+ \int_X |f_n - f||g|\ d\mu \\
					&\longrightarrow 0 \quad (n \longrightarrow \infty)
				\end{align}
				となるから(\refeq{eq:thm_dual_space_of_L_p_2})が成立する.
			
			\item[第四段]
				$p = \infty,\ \mu(X) < \infty$の場合,
				$\varphi \in \left( L^p(\mu) \right)^*$に対して
				$\mathscr{F} \ni A \longmapsto \varphi(\defunc_A)$が可算加法的ならば
				(\refeq{eq:thm_dual_space_of_L_p_7})で定める$\lambda$は複素測度となり,
				前段と同じ理由で(\refeq{eq:thm_dual_space_of_L_p_8})を満たす$g \in L^1(\mu)$が存在し
				\begin{align}
					\varphi(f) = \int_X fg\ d\mu,
					\quad (\forall f \in L^\infty(\mu))
				\end{align}
				が成立する.すなわち$\varphi = \Phi_g$であり,
				このとき$f \coloneqq \defunc_{\{g \neq 0\}}\overline{g}/g \in L^\infty(\mu)$
				に対して
				\begin{align}
					\Norm{g}{L^1(\mu)} = \int_X fg\ d\mu = \varphi(f) 
					\leq \Norm{\varphi}{\left( L^\infty(\mu) \right)^*} 
				\end{align}
				となるから,(\refeq{eq:thm_dual_space_of_L_p_3})と併せて
				(\refeq{eq:thm_dual_space_of_L_p_asseretion_2})が満たされる.
				以降は$p < \infty$とする.
				
			\item[第五段]
				$g \in L^q(\mu)$であることを示す.$p = 1$の場合,
				任意の$E \in \mathscr{F}$に対して$f = \defunc_E$とすれば,
				(\refeq{eq:thm_dual_space_of_L_p_2})より
				\begin{align}
					\left| \int_E g\ d\mu \right| = \left| \varphi(\defunc_E) \right|
					\leq \Norm{\varphi}{\left( L^p(\mu) \right)^*} \mu(E)
				\end{align}
				が成立し
				\begin{align}
					\Norm{g}{L^q(\mu)} \leq \Norm{\varphi}{\left( L^p(\mu) \right)^*}
					\label{eq:thm_dual_space_of_L_p_4}
				\end{align}
				が従う.$1 < p < \infty$の場合は
				$\alpha \coloneqq \defunc_{\{g \neq 0\}}\overline{g}/g$と
				\begin{align}
					E_n \coloneqq \Set{x \in X}{|g(x)| \leq n},
					\quad (n=1,2,\cdots)
				\end{align}
				に対して$f \coloneqq \defunc_{E_n} |g|^{q-1} \alpha$とおけば,
				\begin{align}
					fg = \defunc_{E_n} |g|^q = |f|^p
				\end{align}
				が成り立ち$|f|^p \in L^\infty(\mu)$となるから(\refeq{eq:thm_dual_space_of_L_p_2})より
				\begin{align}
					\int_X \defunc_{E_n} |g|^q\ d\mu
					= \int_X fg\ d\mu
					= \varphi(f)
					\leq \Norm{\varphi}{\left( L^p(\mu) \right)^*} \Norm{f}{L^p(\mu)}
					= \Norm{\varphi}{\left( L^p(\mu) \right)^*} \left\{ \int_X \defunc_{E_n} |g|^q\ d\mu \right\}^{1/p}
				\end{align}
				が従い
				\begin{align}
					\left\{ \int_X \defunc_{E_n} |g|^q\ d\mu \right\}^{1/q} \leq \Norm{\varphi}{\left( L^p(\mu) \right)^*}
				\end{align}
				が得られ,単調収束定理より
				\begin{align}
					\Norm{g}{L^q(\mu)} \leq \Norm{\varphi}{\left( L^p(\mu) \right)^*}
					\label{eq:thm_dual_space_of_L_p_5}
				\end{align}
				が出る.
				
			\item[第六段]
				任意の$f \in L^p(\mu)$に対して,単関数近似列$(f_n)_{n=1}^\infty$は(\refeq{eq:thm_dual_space_of_L_p_2})を満たすから,
				H\Ddot{o}lderの不等式とLebesgueの収束定理より
				\begin{align}
					\left| \varphi(f) - \int_X fg\ d\mu \right|
					&\leq \left| \varphi(f) - \varphi(f_n) \right| + \left| \int_X f_ng\ d\mu - \int_X fg\ d\mu \right| \\
					&\leq \Norm{\varphi}{\left( L^p(\mu) \right)^*} \Norm{f - f_n}{L^p(\mu)}
						+ \Norm{f - f_n}{L^p(\mu)}\Norm{g}{L^q(\mu)} \\
					&\longrightarrow 0 \quad (n \longrightarrow \infty)
				\end{align}
				となり
				\begin{align}
					\varphi = \Phi(g)
				\end{align}
				が成り立つ.また,このとき(\refeq{eq:thm_dual_space_of_L_p_3})と(\refeq{eq:thm_dual_space_of_L_p_4})或は
				(\refeq{eq:thm_dual_space_of_L_p_5})より
				\begin{align}
					\Norm{g}{L^q(\mu)} = \Norm{\varphi}{\left( L^p(\mu) \right)^*}
				\end{align}
				が満たされる.
				
			\item[第七段]
				$\mu(X) = \infty$の場合,補題\ref{lem:Lebesgue_Radon_Nikodym}の関数$w$を用いて
				\begin{align}
					\tilde{\mu}(E) \coloneqq \int_E w\ d\mu,
					\quad (\forall E \in \mathscr{F})
				\end{align}
				により有限測度$\tilde{\mu}$を定める.このとき
				任意の$f \in L^p(\mu)$に対して
				\begin{align}
					F \coloneqq w^{-1/p} f
				\end{align}
				とおけば
				\begin{align}
					\int_X |F|^p\ d\tilde{\mu} = \int_X |F|^p w\ d\mu = \int_X |f|^p\ d\mu
					\label{eq:thm_dual_space_of_L_p_6}
				\end{align}
				が成立し,
				\begin{align}
					L^p \ni f \longmapsto w^{-1/p} f \in L^p(\tilde{\mu})
				\end{align}
				は等長な線型同型となる.ここで任意の$\varphi \in \left( L^p(\mu) \right)^*$に対して
				\begin{align}
					\Psi(F) \coloneqq \varphi\left( w^{1/p} F \right),
					\quad (\forall F \in L^p(\tilde{\mu}))
				\end{align}
				で線形作用素$\Psi$を定めれば
				\begin{align}
					\left| \Psi(F) \right| = \left| \varphi\left( w^{1/p} F \right) \right|
					\leq \Norm{\varphi}{\left( L^p(\mu) \right)^*}\Norm{w^{1/p} F}{L^p(\mu)}
					= \Norm{\varphi}{\left( L^p(\mu) \right)^*}\Norm{F}{L^p(\tilde{\mu})}
				\end{align}
				より$\Psi \in \left( L^p(\tilde{\mu}) \right)^*$が満たされ,かつ
				任意の$f \in L^p(\mu)$に対して
				\begin{align}
					\left| \varphi(f) \right| = \left| \Psi\left( w^{-1/p} f \right) \right|
					\leq \Norm{\Psi}{\left( L^p(\mu) \right)^*}\Norm{w^{-1/p} f}{L^p(\tilde{\mu})}
					= \Norm{\Psi}{\left( L^p(\tilde{\mu}) \right)^*}\Norm{f}{L^p(\mu)}
				\end{align}
				も成り立ち
				\begin{align}
					\Norm{\varphi}{\left( L^p(\mu) \right)^*} = \Norm{\Psi}{\left( L^p(\tilde{\mu}) \right)^*}
				\end{align}
				が得られる.前段までの結果より$\Psi$に対し或る$G \in L^q(\tilde{\mu})$が存在して
				\begin{align}
					\Psi(F) = \int_X FG\ d\tilde{\mu}
				\end{align}
				が成立するから,任意の$f \in L^p(\mu)$に対して
				\begin{align}
					\varphi(f) = \Psi\left( w^{-1/p} f \right)
					= \int_X w^{-1/p} f G w\ d\mu
					= \begin{cases}
						\displaystyle\int_X f G\ d\mu, & (p = 1), \\
						\displaystyle\int_X f w^{1/q} G\ d\mu, & (1 < p < \infty)
					\end{cases}
				\end{align}
				が従い,
				\begin{align}
					g \coloneqq
					\begin{cases}
						G, & (p = 1), \\
						w^{1/q} G, & (1 < p < \infty)
					\end{cases}
				\end{align}
				とおけば(\refeq{eq:thm_dual_space_of_L_p_6})より$g \in L^q(\mu)$となり,
				$\varphi = \Phi(g)$かつ
				\begin{align}
					\Norm{\varphi}{\left( L^p(\mu) \right)^*} = \Norm{\Psi}{\left( L^p(\tilde{\mu}) \right)^*}
					= \Norm{G}{L^q(\tilde{\mu})}
					= \Norm{g}{L^q(\mu)}
				\end{align}
				が満たされる.
				\QED
		\end{description}
	\end{prf}
	
	\begin{screen}
		\begin{dfn}[条件付き期待値]
			$(X,\mathscr{F},\mu)$を測度空間,$f \in L^1(\mu)$とする.
			部分$\sigma$-加法族$\mathscr{G} \subset \mathscr{F}$に対し
			$\nu \coloneqq \left. \mu \right|_{\mathscr{G}}$が$\sigma$-有限であるとき,
			\begin{align}
				\lambda(A) \coloneqq \int_A f\ d\mu,
				\quad (\forall A \in \mathscr{G})
			\end{align}
			により$(X,\mathscr{G})$上に複素測度$\lambda$が定まり,$\lambda \ll \nu$であるから
			Lebesgue-Radon-Nikodymの定理より
			\begin{align}
				\lambda(A) = \int_A g\ d\nu,
				\quad (\forall A \in \mathscr{G})
			\end{align}
			を満たす$g \in L^1(\nu) = L^1\left(X,\mathscr{G},\nu\right)$
			が唯一つ存在する.この$g$を$\mathscr{G}$で条件付けた$f$の条件付き期待値と呼び
			\begin{align}
				g = \cexp{f}{\mathscr{G}}
			\end{align}
			と書く.
		\end{dfn}
	\end{screen}
	
	\begin{screen}
		\begin{lem}[凸関数の片側微係数の存在]
			任意の凸関数$\varphi:\R \longrightarrow \R$には
			各点で左右の微係数が存在する.特に,凸関数は連続であり,すなわちBorel可測である.
		\end{lem}
	\end{screen}
	
	\begin{prf}
		凸性より任意の$x < y < z$に対して
		\begin{align}
			\frac{\varphi(y) - \varphi(x)}{y - x} 
			\leq \frac{\varphi(z) - \varphi(x)}{z - x}
			\leq \frac{\varphi(z) - \varphi(y)}{z - y}
			\label{ineq:lem:convex_function_measurability_1}
		\end{align}
		が満たされる.従って,$x$を固定すれば,$x$に単調減少に近づく任意の点列$(x_n)_{n=1}^{\infty}$に対し
		 \begin{align}
		 	\left(\frac{f(x_n)-f(x)}{x_n-x}\right)_{n=1}^{\infty} 
		 	\label{seq:lem:convex_function_measurability_2}
		 \end{align}
		 は下に有界な単調減少列となり下限が存在する.$x$に単調減少に近づく別の点列$(y_k)_{k=1}^{\infty}$を取れば
		 \begin{align}
		 	\inf{k \in \N}{\frac{f(y_k)-f(x)}{y_k-x}} \leq \frac{f(x_n)-f(x)}{x_n-x} \quad (n=1,2,\cdots)
		 \end{align}
		 より
		 \begin{align}
		 	\inf{k \in \N}{\frac{f(y_k)-f(x)}{y_k-x}} \leq \inf{n \in \N}{\frac{f(x_n)-f(x)}{x_n-x}}
		 \end{align}
		 が成立し,$(x_n),(y_k)$の立場を変えれば逆向きの不等号も得られる.
		 すなわち極限は点列に依らず確定し,$\varphi$は$x$で右側微係数を持つ.
		 同様に左側微係数も存在し,特に$\varphi$の連続性及びBorel可測性が従う.
		 \QED
	\end{prf}
	
	\begin{screen}
		\begin{thm}[Jensenの不等式]
			$(X,\mathscr{F},\mu)$を測度空間,
			$\mathscr{G} \subset \mathscr{F}$を部分$\sigma$-加法族とし,
			$\left. \mu \right|_{\mathscr{G}}$が$\sigma$-有限であるとする.
			このとき,任意の可積分関数$f:X \longrightarrow \R$と
			凸関数$\varphi:\R \longrightarrow \R$に対し,
			$\varphi(f)$が可積分なら次が成立する:
			\begin{align}
				\varphi\left(\cexp{f}{\mathscr{G}} \right)
				\leq \cexp{\varphi(f)}{\mathscr{G}}.
			\end{align}
		\end{thm}
	\end{screen}
	
	\begin{prf}
			$\varphi$は各点$x \in \R$で右側接線を持つから,
			それを$\R \ni t \longmapsto a_x t + b_x$と表せば,
			\begin{align}
				\varphi(t) = \sup{r \in \Q}{\left\{ a_r t + b_r \right\}} \quad (\forall t \in \R)
				\label{eq:prp_properties_of_expanded_conditional_expectation_1}
			\end{align}
			が成立する.
			よって任意の$r \in \Q$に対して
			\begin{align}
				\varphi(f(x)) \geq a_r f(x) + b_r
			\end{align}
			が満たされるから
			\begin{align}
				\cexp{\varphi(f)}{\mathscr{G}}
				\geq a_r \cexp{f}{\mathscr{G}} + b_r 
				\quad \mbox{$\mu$-a.e.},
				\quad \forall r \in \Q 
			\end{align}
			が従い,各$r \in \Q$に対し
			\begin{align}
				N_r \coloneqq \Set{x \in X}{\cexp{\varphi(f)}{\mathscr{G}}(x)
				< a_r \cexp{f}{\mathscr{G}}(x) + b_r}
			\end{align}
			とおけば$\mu(N_r) = 0$かつ
			\begin{align}
				\cexp{\varphi(f)}{\mathscr{G}}(x)
				\geq a_r \cexp{f}{\mathscr{G}}(x) + b_r, 
				\quad \forall r \in \Q,\ x \notin \bigcup_{r \in \Q} N_r
			\end{align}
			となる.$r$の任意性と(\refeq{eq:prp_properties_of_expanded_conditional_expectation_1})より
			\begin{align}
				\cexp{\varphi(f)}{\mathscr{G}} \geq \varphi\left( \cexp{f}{\mathscr{G}} \right),
				\quad \mbox{$\mu$-a.e.}
			\end{align}
			が得られる.
			\QED
	\end{prf}
	
	\begin{screen}
		\begin{thm}[条件付き期待値の性質]\label{thm:properties_of_conditional_expectations}
			$(X,\mathscr{F},\mu)$を測度空間,$\mathscr{H},\mathscr{G}$を$\mathscr{H} \subset \mathscr{G}$を満たす
			$\mathscr{F}$の部分$\sigma$-加法族とし,$\theta \coloneqq \left. \mu \right|_{\mathscr{H}},
			\gamma \coloneqq \left. \mu \right|_{\mathscr{G}}$
			がそれぞれ$\sigma$-有限測度であるとする.このとき以下が成立する:
			\begin{description}
				\item[(1)] $f \in L^1(\mu),\ g \in L^\infty(\gamma)$に対して
					\begin{align}
						\cexp{gf}{\mathscr{G}} = g\cexp{f}{\mathscr{G}}.
						\label{eq:thm_properties_of_conditional_expectations_1}
					\end{align}
					
				\item[(2)] $\cexp{\cdot}{\mathscr{G}}$は$L^1(X,\mathscr{F},\mu)$から
					$L^1\left(X,\mathscr{G},\gamma\right)$への有界線形作用素であり,
					次を満たす:
					\begin{align}
						|\cexp{f}{\mathscr{G}}| \leq \cexp{|f|}{\mathscr{G}},
						\quad (\forall f \in L^1(\mu)).
						\label{eq:thm_properties_of_conditional_expectations_2}
					\end{align}
				
				\item[(3)] $f \in L^1(\mu),\ g \in L^0(\gamma)$に対して,
					$gf \in L^1(\mu)$なら$g \cexp{f}{\mathscr{G}} \in L^1(\gamma)$であり
					\begin{align}
						\cexp{gf}{\mathscr{G}} = g\cexp{f}{\mathscr{G}}.
						\label{eq:thm_properties_of_conditional_expectations_4}
					\end{align}
					
				\item[(4)] $f \in L^1(\mu)$に対して
					\begin{align}
						\cexp{\cexp{f}{\mathscr{G}}}{\mathscr{H}} = \cexp{f}{\mathscr{H}}.
					\end{align}
					
				\item[(5)] $p \in [1,\infty]$に対し$f \in L^1(\mu) \cap L^p(\mu)$なら
					\begin{align}
						\left| \cexp{f}{\mathscr{G}} \right|^p
						\leq \cexp{|f|^p}{\mathscr{G}},\quad (p \in [0,\infty))
					\end{align}
					かつ
					\begin{align}
						\Norm{\cexp{f}{\mathscr{G}}}{L^p(\mu)}
						\leq \Norm{f}{L^p(\mu)},\quad (p \in [0,\infty]).
						\label{eq:thm_properties_of_conditional_expectations_3}
					\end{align}
			\end{description}
		\end{thm}
	\end{screen}
	
	\begin{prf}\mbox{}
		\begin{description}
			\item[(1)]
				任意の$A,B \in \mathscr{G}$に対して
				\begin{align}
					\int_A \defunc_B f\ d\mu
					= \int_{A \cap B} f\ d\mu
					= \int_{A \cap B} \cexp{f}{\mathscr{G}}\ d\gamma
					= \int_A \defunc_B \cexp{f}{\mathscr{G}}\ d\gamma
				\end{align}
				となるから,$g$の単関数近似列$(g_n)_{n=1}^\infty,\ \left(g_n \in L^\infty(\gamma),|g_n| \leq |g|\right)$に対して
				\begin{align}
					\int_A g_n f\ d\mu = \int_A g_n \cexp{f}{\mathscr{G}}\ d\gamma,
					\quad (\forall n \geq 1)
				\end{align}
				が成り立つ.$gf \in L^1(\mu)$かつ$g\cexp{f}{\mathscr{G}} \in L^1(\gamma)$であるから
				Lebesgueの収束定理より
				\begin{align}
					\int_A g \cexp{f}{\mathscr{G}}\ d\gamma
					= \int_A g f\ d\mu
					= \int_A \cexp{g f}{\mathscr{G}}\ d\gamma
				\end{align}
				が従い(\refeq{eq:thm_properties_of_conditional_expectations_2})が得られる.
				
			\item[(2)]
				任意の$\alpha_1,\alpha_2 \in \C,\ f_1,f_2 \in L^1(\mu)$と$A \in \mathscr{G}$に対して
				\begin{align}
					&\int_A \cexp{\alpha_1 f_1 + \alpha_2 f_2}{\mathscr{G}}\ d\gamma
					= \int_A \alpha_1 f_1 + \alpha_2 f_2\ d\mu
					= \alpha_1 \int_A f_1\ d\mu + \alpha_2 \int_A f_2\ d\mu \\
					&\qquad = \alpha_1 \int_A \cexp{f_1}{\mathscr{G}}\ d\gamma 
						+ \alpha_2 \int_A \cexp{f_2}{\mathscr{G}}\ d\gamma
					= \int_A \alpha_1 \cexp{f_1}{\mathscr{G}} + \alpha_2 \cexp{f_2}{\mathscr{G}}\ d\gamma
				\end{align}
				が成立するから$L^1(\nu)$で
				\begin{align}
					\cexp{\alpha_1 f_1 + \alpha_2 f_2}{\mathscr{G}}
					= \alpha_1 \cexp{f_1}{\mathscr{G}} + \alpha_2 \cexp{f_2}{\mathscr{G}}
				\end{align}
				となり$\cexp{\cdot}{\mathscr{G}}$の線型性が出る.また$f \in L^1(\mu)$に対し
				\begin{align}
					\alpha \coloneqq \defunc_{\left\{\cexp{f}{\mathscr{G}} \neq 0\right\}} 
						\frac{\overline{\cexp{f}{\mathscr{G}}}}{|\cexp{f}{\mathscr{G}}|}
				\end{align}
				により$\alpha \in L^\infty(\gamma)$を定めれば,任意の$A \in \mathscr{G}$に対して
				\begin{align}
					\int_A |\cexp{f}{\mathscr{G}}|\ d\gamma
					&= \int_A \alpha \cexp{f}{\mathscr{G}}\ d\gamma
					= \int_A \cexp{\alpha f}{\mathscr{G}}\ d\gamma \\
					&= \int_A \alpha f\ d\mu
					\leq \int_A |f|\ d\mu
					= \int_A \cexp{|f|}{\mathscr{G}}\ d\gamma
				\end{align}
				が成り立つから,(\refeq{eq:thm_properties_of_conditional_expectations_2})及び
				$\cexp{\cdot}{\mathscr{G}}$の有界性が得られる.
				
			\item[(3)]
				$(g_n)_{n=1}^\infty$を$g$の単関数近似列とすれば,
				単調収束定理と(\refeq{eq:thm_properties_of_conditional_expectations_1})より
				\begin{align}
					\int_X |g|\cexp{|f|}{\mathscr{G}}\ d\gamma
					= \lim_{n \to \infty} \int_X |g_n|\cexp{|f|}{\mathscr{G}}\ d\gamma
					= \lim_{n \to \infty} \int_X |g_n||f|\ d\mu
					= \int_X |g||f|\ d\mu
				\end{align}
				となり$g\cexp{f}{\mathscr{G}}$の可積分性が従う.従って,
				Lebesgueの収束定理より任意の$A \in \mathscr{G}$に対して
				\begin{align}
					\int_A g\cexp{f}{\mathscr{G}}\ d\gamma
					= \lim_{n \to \infty} \int_A g_n\cexp{f}{\mathscr{G}}\ d\gamma
					= \lim_{n \to \infty} \int_A g_n f\ d\mu
					= \int_A gf\ d\mu
				\end{align}
				が成り立ち(\refeq{eq:thm_properties_of_conditional_expectations_4})が得られる.
				
			\item[(4)]
				任意の$A \in \mathscr{H}$に対して
				\begin{align}
					\int_A \cexp{f}{\mathscr{H}}\ d\theta
					= \int_A f\ d\mu
					= \int_A \cexp{f}{\mathscr{G}}\ d\gamma
					= \int_A \cexp{\cexp{f}{\mathscr{G}}}{\mathscr{H}}\ d\theta
				\end{align}
				が成立する.
				
			\item[(5)] 
				$1 \leq p < \infty$の場合,
				(\refeq{eq:thm_properties_of_conditional_expectations_2})とJensenの不等式より
				\begin{align}
					|\cexp{f}{\mathscr{G}}|^p \leq \cexp{|f|}{\mathscr{G}}^p
					\leq \cexp{|f|^p}{\mathscr{G}}
				\end{align}
				が成り立つ.$p = \infty$の場合は任意の$A \in \mathscr{G}$に対して
				\begin{align}
					\int_A \left| \cexp{f}{\mathscr{G}} \right|\ d\gamma
					\leq \int_A |f|\ d\mu
					\leq \mu(A) \Norm{f}{L^\infty}
					= \gamma(A) \Norm{f}{L^\infty}
				\end{align}
				となり,$1 \leq p < \infty$の場合も込めて(\refeq{eq:thm_properties_of_conditional_expectations_3})が従う.
				\QED
		\end{description}
	\end{prf}
	
	\begin{screen}
		\begin{thm}
			\begin{description}
				\item[(1)]
					$X_n \leq X_{n+1}$
					$X_n \longrightarrow X\ a.s.P$
					$\cexp{X_n}{\mathscr{G}} \longrightarrow \cexp{X}{\mathscr{G}}\ a.s.P$
				\item[(2)]
					$X_n \geq 0$
					$\cexp{\liminf X_n}{\mathscr{G}} \leq \liminf \cexp{X_n}{\mathscr{G}}$
				\item[(3)]
					$|X_n| \leq Y$ $X_n \longrightarrow X\ a.s.P$
					$\cexp{X_n}{\mathscr{G}} \longrightarrow \cexp{X}{\mathscr{G}}\ a.s.P$
			\end{description}
		\end{thm}
	\end{screen}
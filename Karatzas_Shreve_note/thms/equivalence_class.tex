\section{同値類}
	\begin{screen}
		\begin{dfn}[商集合]
			$a$を集合とし,$R$を$a$上の同値関係とする.$x$を$a$の要素とするとき
			\begin{align}
				\Set{y}{(y,x) \in R}
			\end{align}
			を$x$の$R$に関する{\bf 同値類}\index{どうちるい@同値類}{\bf (equivalence class)}と呼び,
			$[x]$などで表す.また
			\begin{align}
				a / R \coloneqq \Set{x}{\exists y \in a\ \forall z\ (\ (y,z) \in R \Longleftrightarrow z \in x\ )}
			\end{align}
			で定められる類$a/R$を,$a$を$R$で割った
			{\bf 商集合}\index{しょうしゅうごう@商集合}{\bf (quotient set)}と呼ぶ.
		\end{dfn}
	\end{screen}
	
	$a$が空であれば$R$も$a/R$も空となる.
	
	\monologue{
		商``集合''と名前を付けましたが,集合であることは後で示します.また上の定義の設定の下では
		\begin{align}
			a/R = \Set{x}{\exists y \in a\ (\ x = [y]\ )}
		\end{align}
		が成り立ちます.これも後で証明しますが,つまり商集合とは同値類を全て集めたものであるということです.
	}
	
	\begin{screen}
		\begin{thm}[同値類の性質]
			$a$を集合とし,$R$を$a$上の同値関係として,$y$を$a$の要素とするとき$y$の$R$に関する同値類を
			$[y]$で表す.このとき次が成り立つ:
			\begin{description}
				\item[(1)] $\forall y \in a\ \left(\ [y] \subset a\ \right)$
				\item[(2)] $\forall y \in a\ \left(\ y \in [y]\ \right)$
				\item[(3)] $\forall y,z \in a\ \left(\ (y,z) \in R \Longleftrightarrow [y] = [z]\ \right)$
				\item[(4)] $\forall y,z \in a\ \left(\ (y,z) \notin R \Longleftrightarrow [y] \cap [z] = \emptyset\ \right)$
			\end{description}
		\end{thm}
	\end{screen}
	
	\begin{prf} $a$が空であれば空虚な真より(1)(2)(3)(4)は全て成立する.以下では$a \neq \emptyset$として証明する.
		\begin{description}
			\item[(1)] $s,t$を$\mathcal{L}$の任意の対象とするとき,$s \in a$であれば
				\begin{align}
					t \in [s] \Longrightarrow (s,t) \in R
				\end{align}
				が成り立つ.$R \subset a \times a$より
				\begin{align}
					(s,t) \in R \Longrightarrow t \in a
				\end{align}
				が従い
				\begin{align}
					t \in [s] \Longrightarrow t \in a
				\end{align}
				が得られる.$t$の任意性より
				\begin{align}
					[s] \subset a
				\end{align}
				となり,$s$の任意性より(1)が出る.
				
			\item[(2)] $t$を$\mathcal{L}$の任意の対象とするとき,$t \in a$であれば
				\begin{align}
					(t,t) \in R
				\end{align}
				となるから$t \in [t]$が成立する.$t$の任意性より
				\begin{align}
					\forall y \in a\ \left(\ y \in [y]\ \right)
				\end{align}
				が得られる.
				
			\item[(3)] $s,t$を$\mathcal{L}$の任意の対象として,$s,t \in a$であると仮定する.
				\begin{align}
					(s,t) \in R
				\end{align}
				が成り立っているとき,$\tau$を$\mathcal{L}$の任意の対象とすれば
				\begin{align}
					\tau \in [s] \Longleftrightarrow (\tau,s) \in R
				\end{align}
				となり,$R$の推移律より$(\tau,s) \in R$ならば$(\tau,t) \in R$となるから
				\begin{align}
					\tau \in [s] \Longrightarrow \tau \in [t]
				\end{align}
				が従う.同様に$\tau \in [t] \Longrightarrow \tau \in [s]$も成り立つので
				$[s] = [t]$となり
				\begin{align}
					(s,t) \in R \Longrightarrow [s] = [t]
				\end{align}
				が得られる.逆に$[s] = [t]$が成り立っているとき,$s \in [s]$より$s \in [t]$が従い
				\begin{align}
					[s] = [t] \Longrightarrow (s,t) \in R
				\end{align}
				も得られる.$s,t$の任意性より(2)が出る.
			
			\item[(4)] $s,t$を$\mathcal{L}$の任意の対象として,$s,t \in a$であると仮定する.
				\begin{align}
					[s] \cap [t] \neq \emptyset
				\end{align}
				が成り立っているとき,$[s] \cap [t]$の要素を$u$とすれば
				\begin{align}
					(s,u) \in R \wedge (u,t) \in R
				\end{align}
				となるので$(s,t) \in R$が従う.ゆえに
				\begin{align}
					(s,t) \notin R \Longrightarrow [s] \cap [t] = \emptyset
				\end{align}
				が得られる.逆に$(s,t) \in R$が成り立っているとき,(2)より$[s] = [t]$となるから
				\begin{align}
					[s] \cap [t] \neq \emptyset \Longrightarrow (s,t) \in R
				\end{align}
				も得られる.$s,t$の任意性より(3)が出る.
				\QED
		\end{description}
	\end{prf}
	
	\monologue{
		(1)の主張は{\bf 同値類は空でない}ということです.
		(2)の主張は{\bf 同値な要素の同値類は一致する}ということで,
		(2)と(3)を併せれば{\bf 同値類同士は一致していなければ交わらない}と言えます.
	}
	
	\begin{screen}
		\begin{dfn}[商写像]
			$a$を集合とし,$R$を$a$上の同値関係として,$y$を$a$の要素とするとき$y$の$R$に関する同値類を
			$[y]$で表す.このとき
			\begin{align}
				f \coloneqq \Set{x}{\exists t \in a\ \left(\ x=(t,[t])\ \right)}
			\end{align}
			で定められる$f$を{\bf 商写像}\index{しょうしゃぞう@商写像}{\bf (quotient mapping)}と呼ぶ.
		\end{dfn}
	\end{screen}
	
	\monologue{
		$f$が写像であることを述べる前に商写像と名前を付けましたが,以下に示す通り
		$f$は$a$から$a/R$への全射となっています.また商写像は
		{\bf 自然な全射}\index{しぜんなぜんしゃ@自然な全射}{\bf (natural surjection)}や
		{\bf 標準的全射}\index{ひょうじゅんてきぜんしゃ@標準的全射}{\bf (canonical surjection)}
		とも呼ばれます.
	}
	
	\begin{screen}
		\begin{thm}[商写像は全射である]\label{thm:quotient_mapping_is_a_surjection}
			$a$を集合とし,$R$を$a$上の同値関係として,$y$を$a$の要素とするとき$y$の$R$に関する同値類を
			$[y]$で表す.このとき次が成り立つ:
			\begin{align}
				\forall x\ \left(\ x \in a/R \Longleftrightarrow \exists y \in a\ (\ x=[y]\ )\ \right).
				\label{eq:thm_quotient_mapping_is_a_surjection}
			\end{align}
			特に$a$から$a/R$への商写像は写像であり,さらに言えば全射である.
		\end{thm}
	\end{screen}
	
	\begin{prf}
		$a$が空である場合は$a/R$が空となるので,空虚な真より(\refeq{eq:thm_quotient_mapping_is_a_surjection})
		が成り立つ.また商写像も空となり,空写像は空集合から空集合への全単射であるから
		主張は全て従う.以下では$a$が空でない場合で証明する.
	\end{prf}
	
	\begin{screen}
		\begin{thm}[商集合の性質]
			$a$を集合とし,$R$を$a$上の同値関係として,$y$を$a$の要素とするとき$y$の$R$に関する同値類を
			$[y]$で表す.このとき次が成り立つ:
			\begin{description}
				\item[(1)] $a/R \in \Univ$
				\item[(2)] $a = \bigcup (a/R)$
			\end{description}
		\end{thm}
	\end{screen}
	
	\begin{prf} $a$が空であれば$a/R$は空となり(1)が成立する.また$\emptyset = \bigcup \emptyset$より
		(2)も成立する.以下では$a$が空でない場合で証明する.
		\begin{description}
			\item[(1)] $a/R$は$a$から$a/R$への商写像の値域であるから,置換公理より$a/R \in \Univ$が従う.
			\item[(2)] $\tau$を$\mathcal{L}$の任意の対象とすれば,$\tau \in a$ならば
				\begin{align}
					\tau \in [\tau]
				\end{align}
				となるから$\tau \in \bigcup (a/R)$が成立する.ゆえに
				\begin{align}
					\tau \in a \Longrightarrow \tau \in \bigcup (a/R)
				\end{align}
				が得られる.逆に$\tau \in \bigcup (a/R)$が成り立っているとすれば
				$\tau$に対して$a$の或る要素$y$が取れて$\tau \in [y]$となるが,
				$[y] \subset a$より$\tau \in a$が従うので
				\begin{align}
					\tau \in \bigcup (a/R) \Longrightarrow \tau \in a
				\end{align}
				も得られる.$\tau$の任意性より$a = \bigcup (a/R)$が出る.
				\QED
		\end{description}
	\end{prf}
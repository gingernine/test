\section{同値類}
	\begin{screen}
		\begin{dfn}[商集合]
			$a$を集合とし,$R$を$a$上の同値関係とする.$x$を$a$の要素とするとき
			\begin{align}
				\Set{y}{(y,x) \in R}
			\end{align}
			を$x$の$R$に関する{\bf 同値類}\index{どうちるい@同値類}{\bf (equivalence class)}と呼ぶ.また
			\begin{align}
				a / R \defeq \Set{x}{\exists y \in a\, \forall z\, \left(\, (y,z) \in R \Longleftrightarrow z \in x\, \right)}
			\end{align}
			で定める類を,$a$を$R$で割った{\bf 商集合}\index{しょうしゅうごう@商集合}{\bf (quotient set)}と呼ぶ.
		\end{dfn}
	\end{screen}
	
	$a$が空であれば$R$も$a/R$も空となる.
	
	\begin{screen}
		\begin{dfn}[商写像]
			$a$を集合とし,$R$を$a$上の同値関係とする.このとき,
			\begin{align}
				a \ni x \longmapsto \Set{y}{(y,x) \in R}
			\end{align}
			なる関係を$a$から$a/R$への{\bf 商写像}\index{しょうしゃぞう@商写像}{\bf (quotient mapping)}と呼ぶ.
		\end{dfn}
	\end{screen}
	
	\monologue{
		写像であることを述べる前に商写像と名前を付けましたが,以下に示す通り
		商写像は$a$から$a/R$への全射となっています.また商写像は
		{\bf 自然な全射}\index{しぜんなぜんしゃ@自然な全射}{\bf (natural surjection)}や
		{\bf 標準的全射}\index{ひょうじゅんてきぜんしゃ@標準的全射}{\bf (canonical surjection)}
		とも呼ばれます.
	}
	
	\begin{screen}
		\begin{thm}[商写像は全射である]\label{thm:quotient_mapping_is_a_surjection}
			$a$を集合とし,$R$を$a$上の同値関係とする.このとき
			$a$から$a/R$への商写像は全射である.
		\end{thm}
	\end{screen}
	
	\begin{prf}
		$a$が空である場合は$a/R = \emptyset$も空となる.この場合$a$上には空写像しか定まらないが,
		定理\ref{thm:emptyset_is_a_mapping}より空写像は$a$から$a/R$への全射である.
		\begin{align}
			a \neq \emptyset
		\end{align}
		であるとし,また$q$を$a$から$a/R$への商写像とする.
		\begin{description}
		\item[第一段] $q$が写像であることを示す.
		$x,y,z$を集合として
		\begin{align}
			(x,y) \in q \wedge (x,z) \in q
		\end{align}
		が成り立っているとすると,
		\begin{align}
			s = x \wedge y = \Set{\tau}{(\tau,s) \in R}
		\end{align}
		なる$a$の要素$s$と,
		\begin{align}
			t = x \wedge y = \Set{\tau}{(\tau,t) \in R}
		\end{align}
		なる$a$の要素$t$が取れる.このとき
		\begin{align}
			s = t
		\end{align}
		が成り立つから
		\begin{align}
			\Set{\tau}{(\tau,s) \in R} = \Set{\tau}{(\tau,t) \in R}
		\end{align}
		が従い
		\begin{align}
			y = z
		\end{align}
		が成立する.ゆえに$q$は写像である.
		
		\item[第二段] $q$の定義域が$a$に等しいことを示す.
		$x$を集合とするとき,
		\begin{align}
			x \in \dom{q}
		\end{align}
		ならば
		\begin{align}
			(x,y) \in q
		\end{align}
		なる集合$y$が取れるが,このとき
		\begin{align}
			s = x \wedge y = \Set{\tau}{(\tau,s) \in R}
		\end{align}
		なる$a$の要素$s$が取れて
		\begin{align}
			x \in a
		\end{align}
		が従う.$x$が
		\begin{align}
			x \in a
		\end{align}
		を満たすならば,
		\begin{align}
			y \defeq \Set{\tau}{(\tau,x) \in R}
		\end{align}
		とおけば
		\begin{align}
			(x,y) \in q
		\end{align}
		が成り立つので
		\begin{align}
			x \in \dom{q}
		\end{align}
		となる.$x$の任意性ゆえに
		\begin{align}
			a = \dom{q}
		\end{align}
		が従う.ゆえに$q$は$a$上の写像である.
		
		\item[第三段] $q$が$a/R$への写像であることを示す.
			$x$を$a$の要素とすると
			\begin{align}
				t \defeq q(x)
			\end{align}
			とおくと,
			\begin{align}
				t = \Set{y}{(y,x) \in R}
			\end{align}
			が成り立つ.すると
			\begin{align}
				\forall z\, \left(\, (z,x) \in R \Longleftrightarrow z \in t\, \right)
			\end{align}
			が成り立って
			\begin{align}
				\exists y \in a\, \forall z\, \left(\, (z,y) \in R \Longleftrightarrow z \in t\, \right)
			\end{align}
			が満たされる.ゆえに
			\begin{align}
				t \in a/R
			\end{align}
			が成立し
			\begin{align}
				q(x) \in a/R
			\end{align}
			が従う.ゆえに$q$は$a/R$への写像である.
			
		\item[第四段] $q$が全射であることを示す.
			$y$を$a/R$の要素とすると
			\begin{align}
				s \in a\, \forall t\, \left(\, (s,t) \in R \Longleftrightarrow t \in y\, \right)
			\end{align}
			を満たす集合$s$が取れて,外延性の公理より
			\begin{align}
				y = \Set{t}{(s,t) \in R}
			\end{align}
			が成立する.ゆえに
			\begin{align}
				(s,y) \in q
			\end{align}
			が満たされる.よって$q$は全射である.
			\QED
		\end{description}
	\end{prf}
	
	\begin{screen}
		\begin{thm}[要素は自分の同値類に属する]\label{thm:element_is_contained_in_its_equivalence_class}
			$a$を集合とし,$R$を$a$上の同値関係とし,$q$を$a$から$a/R$への商写像とする.このとき
			\begin{align}
				\forall x \in a\, (\, x \in q(x)\, ).
			\end{align}
		\end{thm}
	\end{screen}
	
	\begin{sketch}
		$x$を$a$の要素とすると,
		\begin{align}
			q(x) = \Set{y}{(y,x) \in R}
		\end{align}
		が成立する.$R$は同値関係なので
		\begin{align}
			(x,x) \in R
		\end{align}
		が成り立ち,
		\begin{align}
			x \in \Set{y}{(y,x) \in R}
		\end{align}
		が従う.ゆえに
		\begin{align}
			x \in q(x)
		\end{align}
		が成立する.
		\QED
	\end{sketch}
	
	\begin{screen}
		\begin{thm}[商集合の合併は割る前の集合に一致する]
		\label{thm:union_of_quotient_set_is_the_original_set}
			$a$を集合とし,$R$を$a$上の同値関係とする.このとき$a/R$は集合であって,また
			\begin{align}
				a = \bigcup (a/R).
			\end{align}
		\end{thm}
	\end{screen}
	
	\begin{prf}
		$q$を$a$から$a/R$への商写像とすると,定理\ref{thm:quotient_mapping_is_a_surjection}より
		\begin{align}
			q \ast a = a/R
		\end{align}
		が成り立つから,置換公理より
		\begin{align}
			\set{a/R}
		\end{align}
		が成立する.$x$を集合とするとき,
		\begin{align}
			x \in a
		\end{align}
		ならば
		\begin{align}
			x \in q(x)
		\end{align}
		が成り立つから
		\begin{align}
			x \in \bigcup (a/R)
		\end{align}
		が従う.逆に
		\begin{align}
			x \in \bigcup (a/R)
		\end{align}
		ならば
		\begin{align}
			x \in q(y)
		\end{align}
		を満たす集合$y$が取れるが,このとき
		\begin{align}
			(x,y) \in R
		\end{align}
		が成立するので
		\begin{align}
			x \in a
		\end{align}
		が成立する.$x$の任意性より
		\begin{align}
			a = \bigcup (a/R)
		\end{align}
		となる.
		\QED
	\end{prf}
	
	\begin{screen}
		\begin{thm}[同値な要素同士の同値類は一致する]\label{thm:equivalence_classes_of_equivalent_elements_coincide}
			$a$を集合とし,$R$を$a$上の同値関係とし,$q$を$a$から$a/R$への商写像とする.このとき
			\begin{align}
				\forall x,y \in a\, \left(\, (x,y) \in R \Longleftrightarrow q(x) = q(y)\, \right).
			\end{align}
		\end{thm}
	\end{screen}
	
	\begin{sketch}
		$a$が空であれば空虚な真より定理は成立する.以下では$a$は空でないとして証明する.
		$x$と$y$を$a$の要素とする.
		\begin{align}
			(x,y) \in R
		\end{align}
		が成り立っているとすると,$z$を
		\begin{align}
			z \in q(x)
		\end{align}
		なる集合とすれば
		\begin{align}
			(z,x) \in R
		\end{align}
		が成り立ち,同値関係の推移性より
		\begin{align}
			(z,y) \in R
		\end{align}
		が成り立つ.ゆえに
		\begin{align}
			z \in q(y)
		\end{align}
		が成立し,$z$の任意性より
		\begin{align}
			q(x) \subset q(y)
		\end{align}
		となる.$x$と$y$の立場を入れ替えれば
		\begin{align}
			q(y) \subset q(x)
		\end{align}
		も成り立つから
		\begin{align}
			q(x) = q(y)
		\end{align}
		が成り立つ.ゆえに
		\begin{align}
			(x,y) \in R \Longrightarrow q(x) = q(y)
		\end{align}
		が成り立つ.逆に
		\begin{align}
			q(x) = q(y)
		\end{align}
		が成り立っているとすると,定理\ref{thm:element_is_contained_in_its_equivalence_class}より
		\begin{align}
			x \in q(x)
		\end{align}
		が成り立つので
		\begin{align}
			x \in q(y)
		\end{align}
		となる.ゆえに
		\begin{align}
			(x,y) \in R
 		\end{align}
 		が従う.ゆえに
 		\begin{align}
			q(x) = q(y) \Longrightarrow (x,y) \in R
		\end{align}
		が成り立つ.
		\QED
	\end{sketch}
	
	\begin{screen}
		\begin{thm}[同値類は一致していなければ交わらない]
		\label{thm:equivalence_classes_of_not_equivalent_elements_are_disjoint}
			$a$を集合とし,$R$を$a$上の同値関係とし,$q$を$a$から$a/R$への商写像とする.このとき
			\begin{align}
				\forall x,y \in a\, \left(\, q(x) \neq q(y) \Longleftrightarrow q(x) \cap q(y) = \emptyset\, \right).
			\end{align}
		\end{thm}
	\end{screen}
	
	\begin{sketch}
		$a$が空であれば空虚な真より定理は成立する.以下では$a$は空でないとして証明する.
		$x$と$y$を$a$の要素とする.
		\begin{align}
			q(x) = q(y)
		\end{align}
		が成り立っているとすると,定理\ref{thm:element_is_contained_in_its_equivalence_class}より
		\begin{align}
			x \in q(x)
		\end{align}
		が成り立つので
		\begin{align}
			x \in q(y)
		\end{align}
		となる.ゆえに
		\begin{align}
			q(x) \cap q(y) \neq \emptyset
		\end{align}
		が成立する.ゆえに
		\begin{align}
			q(x) = q(y) \Longrightarrow q(x) \cap q(y) \neq \emptyset
		\end{align}
		が成立する.逆に
		\begin{align}
			q(x) \cap q(y) \neq \emptyset
		\end{align}
		が成り立っているとすると,
		\begin{align}
			z \in q(x) \cap q(y)
		\end{align}
		を満たす集合$z$が取れて
		\begin{align}
			(x,z) \in R
		\end{align}
		かつ
		\begin{align}
			(y,z) \in R
		\end{align}
		が成り立つ.このとき同値関係の可換律より
		\begin{align}
			(z,y) \in R
		\end{align}
		が成り立って,同値関係の推移律より
		\begin{align}
			(x,y) \in R
		\end{align}
		が従い,定理\ref{thm:equivalence_classes_of_equivalent_elements_coincide}より
		\begin{align}
			q(x) = q(y)
		\end{align}
		が成立する.ゆえに
		\begin{align}
			q(x) \cap q(y) \neq \emptyset \Longrightarrow q(x) = q(y)
		\end{align}
		が成立する.
		\QED
	\end{sketch}
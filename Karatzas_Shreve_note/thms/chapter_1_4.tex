\section{The Doob-Meyer Decomposition}
	\begin{itembox}[l]{martingale transform}
		If $A = \Set{A_n,\mathscr{F}_n}{n=0,1,\cdots}$ is predictable with $E|A_n|<\infty$ for every $n$,
		and if $\Set{M_n,\mathscr{F}_n}{n=0,1,\cdots}$ is bounded martingale, then the martingale transform of $A$
		by $M$ defined by
		\begin{align}
			Y_0 = 0 \quad \mbox{and} \quad
			Y_n = \sum_{k=1}^n A_k (M_k - M_{k-1});
			\quad n \geq 1, 
		\end{align}
		is itself a martingale.
	\end{itembox}
	
	\begin{prf}
		$A_k(M_k - M_{k-1})\ (k \leq n)$は$\mathscr{F}_n/\borel{\R}$-可測であるから
		$(Y_n)_{n=1}^\infty$は$(\mathscr{F}_n)$-適合である.また
		\begin{align}
			E|Y_n| = E\left| \sum_{k=1}^n A_k (M_k - M_{k-1}) \right|
			\leq \sum_{k=1}^n \left\{\esssup{\omega \in \Omega}{\left(|M_k(\omega)|+|M_{k-1}(\omega)|\right)}\right\} E|A_k| < \infty
		\end{align}
		が成り立つ.更に任意の$n \geq 0$に対し
		\begin{align}
			\cexp{Y_{n+1} - Y_n}{\mathscr{F}_n}
			= \cexp{A_{n+1}(M_{n+1} - M_n)}{\mathscr{F}_n}
			= A_{n+1}\cexp{M_{n+1} - M_n}{\mathscr{F}_n}
			= 0,
			\quad \mbox{a.s. $P$}
		\end{align}
		が満たされる.
		\QED
	\end{prf}
	
	\begin{itembox}[l]{Doob's decomposition}\label{lem:Doob_decomposition}
		Any submartingale $\Set{X_n,\mathscr{F}_n}{n=0,1,\cdots}$ admits the unique decomposition
		$X_n = M_n + A_n$ as the summation of a martingale $\{M_n,\mathscr{F}_n\}$ and an 
		predictable and increasing sequence $\{A_n,\mathscr{F}_n\}$, where
		\begin{align}
			A_n = \sum_{k=0}^{n-1}\cexp{X_{k+1}-X_k}{\mathscr{F}_k},
			\quad \mbox{a.s. $P$},\ n \geq 1.
		\end{align}
	\end{itembox}
	
	\begin{prf}\mbox{}
		\begin{description}
			\item[第一段]
				Doob分解が存在するとして,分解の一意性を示す.
				実際,分解が存在すれば
				\begin{align}
					A_{n+1} - A_n = \cexp{A_{n+1}-A_n}{\mathscr{F}_{n}}
					= \cexp{X_{n+1}-X_n}{\mathscr{F}_{n}} - \cexp{M_{n+1}-M_n}{\mathscr{F}_{n}}
					= \cexp{X_{n+1}-X_n}{\mathscr{F}_{n}},
					\quad \mbox{a.s. $P$}
				\end{align}
				が成立し,$A_n\ (n \geq 1)$は
				\begin{align}
					A_n = \sum_{k=0}^{n-1} \cexp{X_{k+1}-X_k}{\mathscr{F}_{k}},
					\quad \mbox{a.s. $P$}
				\end{align}
				を満たすことになり分解の一意性が出る.
				
			\item[第二段]
				分解可能性を示す.
				\begin{align}
					A_0 \coloneqq 0,
					\quad A_n \coloneqq \sum_{k=0}^{n-1} \cexp{X_{k+1}-X_k}{\mathscr{F}_{k}},
					\quad (n=1,2,\cdots)
				\end{align}
				と定めれば$(A_n)$は可予測かつ可積分であり,
				\begin{align}
					A_{n+1} - A_n = \cexp{X_{k+1}-X_k}{\mathscr{F}_{k}} \geq 0,
					\quad \mbox{a.s. $P$}
					\ (\forall n \geq 1)
				\end{align}
				より増大過程である.また$M_n \coloneqq X_n - A_n$により$(\mathscr{F}_n)$-適合かつ可積分な過程を定めれば,
				\begin{align}
					\cexp{M_{n+1} - M_n}{\mathscr{F}_n}
					&= \cexp{(X_{n+1} - X_n)-(A_{n+1}-A_n)}{\mathscr{F}_n} \\
					&= \cexp{X_{n+1} - X_n}{\mathscr{F}_n} - \cexp{\cexp{X_{n+1} - X_n}{\mathscr{F}_n}}{\mathscr{F}_n}
					= 0,
					\quad \mbox{a.s. $P$}
				\end{align}
				が成り立つから$\{M_n,\mathscr{F}_n\}$はマルチンゲールである.
				\QED
		\end{description}
	\end{prf}
	
	\begin{itembox}[l]{Proposition 4.3 修正}
		An increasing random sequence $A$ \textcolor{red}{has a predictable modification}
		if and only if it is natural.
	\end{itembox}
	
	\begin{prf}
		$A$が可予測な修正$\tilde{A}$を持つとき,任意の有界マルチンゲール$M$に対して
		\begin{align}
			\tilde{Y}_0 \coloneqq 0,
			\quad \tilde{Y}_n \coloneqq \sum_{k=1}^n \tilde{A}_k(M_k - M_{k-1}); \quad n \geq 1
		\end{align}
		は$(\mathscr{F}_n)$-マルチンゲールとなる.
		このとき$M_n \tilde{A}_n$と$\sum_{k=1}^n M_{k-1}(\tilde{A}_k - \tilde{A}_{k-1})$は可積分であり
		\begin{align}
			0 = E \tilde{Y}_n = E\left[ M_n \tilde{A}_n - \sum_{k=1}^n M_{k-1}(\tilde{A}_k - \tilde{A}_{k-1}) \right]
			= E(M_n A_n) - E\sum_{k=1}^n M_{k-1}(A_k - A_{k-1}),
			\quad (\forall n \geq 1)
		\end{align}
		が成り立つから$A$はナチュラルである.逆に$A$がナチュラルであるとき,
		有界マルチンゲール$M$に対して
		\begin{align}
			0 &= E\left[ M_n A_n - \sum_{k=1}^n M_{k-1}(A_k - A_{k-1}) \right] \\
			&= E\left[ A_n(M_n-M_{n-1}) \right] - E\left[ M_{n-1} A_{n-1} - \sum_{k=1}^{n-1} M_{k-1}(A_k - A_{k-1}) \right] \\
			&= E\left[ A_n(M_n-M_{n-1}) \right],
			\quad (\forall n \geq 1)
		\end{align}
		が成り立つ.一方で
		\begin{align}
			E\left[ M_{n-1}(A_n-\cexp{A_n}{\mathscr{F}_{n-1}}) \right]
			&= E\left[ \cexp{M_{n-1} (A_n-\cexp{A_n}{\mathscr{F}_{n-1}})}{\mathscr{F}_{n-1}} \right] \\
			&= E\left[ M_{n-1} \cexp{A_n-\cexp{A_n}{\mathscr{F}_{n-1}}}{\mathscr{F}_{n-1}} \right]
			= 0,
			\quad (\forall n \geq 1)
		\end{align}
		及び
		\begin{align}
			E\left[ \cexp{A_n}{\mathscr{F}_{n-1}}(M_n-M_{n-1}) \right]
			&= E\left[ \cexp{ \cexp{A_n}{\mathscr{F}_{n-1}}(M_n-M_{n-1})}{\mathscr{F}_{n-1}} \right] \\
			&= E\left[ \cexp{A_n}{\mathscr{F}_{n-1}}\cexp{M_n-M_{n-1}}{\mathscr{F}_{n-1}} \right]
			= 0,
			\quad (\forall n \geq 1)
		\end{align}
		となるから
		\begin{align}
			E\left[ M_n(A_n - \cexp{A_n}{\mathscr{F}_{n-1}}) \right]
			&= E\left[ A_n(M_n-M_{n-1}) \right] \\
			&\quad	+ E\left[ M_{n-1}(A_n-\cexp{A_n}{\mathscr{F}_{n-1}}) \right] \\
			&\quad	- E\left[ \cexp{A_n}{\mathscr{F}_{n-1}}(M_n-M_{n-1}) \right] \\
			&= 0,
			\quad (\forall n \geq 1)
		\end{align}
		が従う.ここで各$n \geq 1$に対し,
		$\borel{\R}/\borel{\R}$-可測関数$\operatorname{sgn} = \defunc_{(0,\infty)} - \defunc_{(-\infty,0)}$を用いて
		\begin{align}
			M^{(n)}_k \coloneqq 
			\begin{cases}
				\sgn{A_n - \cexp{A_n}{\mathscr{F}_{n-1}}}, & (k \geq n), \\
				\cexp{\sgn{A_n - \cexp{A_n}{\mathscr{F}_{n-1}}}}{\mathscr{F}_k}, & (0 \leq k < n)
			\end{cases}
		\end{align}
		により有界マルチンゲール$M^{(n)} = \Set{M^{(n)}_k,\mathscr{F}_k}{k=0,1,\cdots}$を定めれば,
		\begin{align}
			0 = E\left[ M^{(n)}_n(A_n - \cexp{A_n}{\mathscr{F}_{n-1}}) \right] 
			= E\left| A_n - \cexp{A_n}{\mathscr{F}_{n-1}} \right|,
			\quad (\forall n \geq 1)
		\end{align}
		が得られ
		\begin{align}
			\tilde{A}_0 \coloneqq 0,
			\quad \tilde{A}_n \coloneqq \cexp{A_n}{\mathscr{F}_{n-1}}; \quad n \geq 1
		\end{align}
		は$A$の可予測な修正となる.
		\QED
	\end{prf}
	
	\begin{itembox}[l]{区別不能性によるパスの同値類}
		区間\footnotemark $I \subset [0,\infty)$
		の上で右連続な確率過程の全体を$RCSP(I)$と書く.また$RCSP([0,\infty))$は$RCSP$と書く.
		任意の$M = \Set{M_t}{t \in I},N = \Set{N_t}{t \in I} \in RCSP(I)$に対し,
		\begin{align}
			\{M_t = N_t,\ \forall t \in I\} = 
			\begin{cases}
				\displaystyle \bigcap_{r \in (I \cap \Q) \cup \{\sup{}{I}\}}\{M_r = N_r\}, & (\sup{}{I} \in I), \\
				\displaystyle \bigcap_{r \in I \cap \Q}\{M_r = N_r\}, & (\sup{}{I} \notin I)
			\end{cases}
		\end{align}
		となるから$\{M_t = N_t,\ \forall t \in I\}$は可測であり,
		このとき,
		\begin{align}
			M \sim N \quad \overset{\mathrm{def}}{\Longleftrightarrow} \quad 
			P(M_t = N_t,\ \forall t \in I) = 1
			\label{eq:equivalence_with_respect_to_path}
		\end{align}
		により同値関係$\sim$が定まる.
	\end{itembox}
	\footnotetext{
		この場合区間は$[a,b],(a,b),[a,b),(a,b],[a,\infty),(a,\infty),\ (0 \leq a < b < \infty)$のいずれかと考える.
	}
	
	\begin{itembox}[l]{Definition 4.4 修正}
		\textcolor{red}{Let $I \subset [0,\infty)$ be an interval.}
		An adapted process \textcolor{red}{$A = \Set{A_t,\mathscr{F}_t}{t \in I}$} 
		is called increasing if \textcolor{red}{for all $\omega \in \Omega$} we have
		\begin{description}
			\item[(a)] $A_0(\omega) = 0$
			\item[(b)] $t \longmapsto A_t(\omega)$ is nondecreasing, right-continuous function,
		\end{description}
		and $E(A_t) < \infty$ holds for every \textcolor{red}{$t \in I$}.
		An increasing process is called integrable if \textcolor{red}{$E\left(A_{\infty}\right) < \infty$,
		where $A_{\infty} = \lim_{t \to \sup{}{I}} A_t$.
		Since $A$ is nondecreasing, $A_{\infty} = A_{(\sup{}{I})-}$ if $\sup{}{I} \in I$.}
	\end{itembox}
	
	\begin{itembox}[l]{Definition 4.5 修正}
		\textcolor{red}{Let $I \subset [0,\infty)$ be an interval and $\alpha \coloneqq \inf{}{I}$.}
		An increasing processs \textcolor{red}{$A = \Set{A_t,\mathscr{F}_t}{t \in I}$} 
		is called natural if for every bounded, 
		\textcolor{red}{$RCLL$ martingale $\Set{M_t,\mathscr{F}_t}{t \in I}$} we have
		\begin{align}
			E \int_{(\alpha,t]} M_s\ dA_s = E \int_{(\alpha,t]} M_{s-}\ dA_s,
			\quad \mbox{for every $t \in (\alpha,\infty) \cap I$}.
		\end{align}
		\textcolor{red}{Let us denote the subset of $RCSP(I)$ as
		\begin{align}
			NAT(I) \coloneqq
			\Set{A \in RCSP(I)}{\mbox{natural}},
			\quad NAT \coloneqq NAT([0,\infty))
		\end{align}
		and the equivalent class of $A \in NAT$
		in the meaning of (\refeq{eq:equivalence_with_respect_to_path}) as $[A]_{NAT}\ ( \subset NAT)$.}
	\end{itembox}
	
	プロセスが$RCLL$とは全てのパスが$RCLL$であるということである.Theorem 3.8によれば
	右連続な劣マルチンゲールはa.e.のパスが$RCLL$であるから,
	(\refeq{eq:equivalence_with_respect_to_path})の意味で同値である.
	$A$も全てのパスが右連続かつ単調非減少であるから,
	全ての$\omega \in \Omega$に対し$\int_{(0,t]} M_s(\omega)\ dA_s(\omega)$と
	$\int_{(0,t]} M_{s-}(\omega)\ dA_s(\omega)$が定義される.
	たぶん余計な煩雑さを回避できる.
		
	\begin{itembox}[l]{$RCLL$なパスの不連続点は高々可算個}
		$(S,d)$を距離空間とする.写像$f:[0,\infty) \longrightarrow S$について
		各点$t \in [0,\infty)$で右連続かつ各点$t \in (0,\infty)$で左極限が存在するとき,
		$f$の不連続点は存在しても高々可算個である.
	\end{itembox}
	
	\begin{prf}
		各点$t > 0$における$f$の左極限を$f(t-)$と書けば
		\begin{align}
			\mbox{$f$が$t \in (0,\infty)$で不連続}
			\quad \Leftrightarrow \quad
			\mbox{$d(f(t),f(t-)) > 0$}
		\end{align}
		が成立するから,任意に$T > 0$を選び固定して
		\begin{align}
			D(n) \coloneqq \Set{t \in (0,T]}{\frac{1}{n+1} \leq d(f(t),f(t-)) < \frac{1}{n}},
			\quad E(n) \coloneqq \Set{t \in (0,T]}{n \leq d(f(t),f(t-)) < n+1}
		\end{align}
		とおけば
		\begin{align}
			D_T \coloneqq \Set{t \in (0,T]}{\mbox{$f$が$t \in (0,\infty)$で不連続}}
			= \bigcup_{n=1}^\infty D(n) \cup E(n)
		\end{align}
		となる.このとき$D(n),E(n)$は全て有限集合である.実際,或る$n$に対し$D(n)$が無限集合なら
		\begin{align}
			\left\{ t_k \right\}_{k=1}^\infty \subset D(n),
			\quad t_k \neq t_j\ (k \neq j)
		\end{align}
		を満たす可算集合が存在し,$[0,T]$のコンパクト性より
		或る部分列$\left( t_{k_m} \right)_{m=1}^\infty$は
		或る$y \in [0,T]$に収束する.
		$y=0$の場合,右連続の仮定より$1/2(n+1) > \epsilon > 0$に対し或る$\delta > 0$が存在して
		\begin{align}
			d(f(0),f(t)) < \epsilon, \quad (\forall 0 < t < \delta)
		\end{align}
		が成り立つが,一方で$0 < t_{k_m} < \delta$を満たす$t_{k_m}$が存在して
		\begin{align}
			\frac{1}{n+1} - \epsilon < d(f(t_{k_m}),f(t_{k_m}-)) - d(f(0),f(t_{k_m}-))
			\leq d(f(0),f(t_{k_m})) < \epsilon 
		\end{align}
		となり矛盾が生じる.
		$y > 0$の場合も,$1/2(n+1) > \epsilon > 0$に対し或る$\delta > 0$が存在して
		\begin{align}
			d(f(y-),f(t)) < \epsilon, \quad (\forall t \in (y-\delta,y))
		\end{align}
		となるが,$f$が$y$で右連続であるから(或は$y=T$のとき) $y-\delta < t_{k_m} \leq y$を満たす$t_{k_m}$が存在して
		\begin{align}
			\frac{1}{n+1} - \epsilon < 
			d(f(t_{k_m}-),f(t_{k_m})) - d(f(t_{k_m}-),f(y-)) \leq d(f(y-),f(t_{k_m})) < \epsilon
		\end{align}
		が従い矛盾が生じる.よって任意の$n \geq 1$に対して$D(n)$は有限集合であり,同様に
		$E(n)$も有限集合であるから$D_T$は高々可算集合である.
		$f$の不連続点の全体は$\bigcup_{T=1}^\infty D_T$に一致するから高々可算個である.
		\QED
	\end{prf}
	
	\begin{itembox}[l]{Remarks 4.6 (i) 修正}
		If $A$ is an increasing and $X$ a measurable process, then with $\omega \in \Omega$ fixed,
		the sample path $\Set{X_t(\omega)}{0 \leq t < \infty}$ is a measurable function from $[0,\infty)$
		into $\R$. It follows that the Lebesgue-Stieltjes integrals
		\begin{align}
			I^{\pm}_t(\omega) \coloneqq
			\int_{(0,t]} X^\pm_s(\omega)\ dA_s(\omega)
		\end{align}
		are well defined. \textcolor{red}{If $X$ is bounded, right-continuous and adapted
		to the filtration $(\mathscr{F}_t)$, then $I$ is finite, right-continuous and 
		$(\mathscr{F}_t)$-progressively measurable.}
	\end{itembox}
	
	\begin{prf}
		$X$が$\borel{[0,\infty)} \otimes \mathscr{F}/\borel{\R}$-可測なら,
		補題\ref{lem:Fubini_lemma_1} (P. \pageref{lem:Fubini_lemma_1})より
		$[0,\infty) \ni t \longmapsto X_t(\omega)$は
		$\borel{[0,\infty)}/\borel{\R}$-可測である.
		また全ての$\omega \in \Omega$に対し$t \longmapsto A_t(\omega)$は右連続非減少であるから
		\begin{align}
			\mu_\omega((a,b]) = A_b(\omega) - A_a(\omega),
			\quad (\forall (a,b] \subset [0,\infty)),
			\quad \mu_\omega(\{0\}) = 0
		\end{align}
		を満たす$\left([0,\infty),\borel{[0,\infty)}\right)$上の$\sigma$-有限測度が唯一つ存在して
		\begin{align}
			I^\pm_t(\omega) = \int_{(0,t]} X^\pm_s(\omega)\ dA_s(\omega)
			\coloneqq \int_{(0,t]} X^\pm_s(\omega)\ \mu_\omega(ds),
			\quad (0 < t < \infty)
		\end{align}
		及び$I_t \coloneqq I^+_t - I^-_t$が定義される.
		特に$\sup{s \in (0,t]}{|X^\pm_s|} \leq B < \infty$なら
		\begin{align}
			\left|I^\pm_t\right| \leq B A_t
		\end{align}
		となるから$I^\pm_t$は有限確定する.$X$が有界かつ右連続$(\mathscr{F}_t)$-適合であるとき,
		$t>0$を固定し$t^{(n)}_j \coloneqq tj/2^n$として
		\begin{align}
			X^{(n)\pm}_s \coloneqq X_0 \defunc_{\{0\}}(s) + 
				\sum_{j=0}^{2^n-1} X_{t^{(n)}_{j+1}} 
				\defunc_{\left(t^{(n)}_j,t^{(n)}_{j+1}\right]}(s)
		\end{align}
		とおけば右連続性より$X^{(n)\pm}_s \longrightarrow X^\pm_s,\ (\forall s \in [0,t])$が成立し,かつ
		\begin{align}
			I^{(n)\pm}_t \coloneqq \int_{(0,t]} X^{(n)\pm}_s\ dA_s
			= \sum_{j=0}^{2^n-1} X_{t^{(n)}_{j+1}} \left(A_{t^{(n)}_j} - A_{t^{(n)}_{j+1}}\right)
		\end{align}
		となり$I^{(n)\pm}_t$の$\mathscr{F}_t/\borel{\R}$-可測性が得られる.
		$X$が有界であるからLebesgueの収束定理より
		\begin{align}
			I^{\pm}_t = \lim_{n \to \infty} \int_{(0,t]} X^{(n)\pm}_s\ dA_s
			= \lim_{n \to \infty} I^{(n)\pm}_t
		\end{align}
		が成り立ち,定理\ref{lem:measurability_metric_space}より
		$I^{\pm}_t$の$\mathscr{F}_t/\borel{\R}$-可測性が従う.
		また$t<T$及び$\{t_n\}_{n=1}^\infty \subset (t,T],\ t_n \downarrow t$に対して,Lebesgueの収束定理より
		\begin{align}
			\lim_{n \to \infty} I^\pm_{t_n}
			= \lim_{n \to \infty} \int_{(0,T]} \defunc_{(0,t_n]}(s)X^\pm_s\ dA_s
			= \int_{(0,T]} \defunc_{(0,t]}(s)X^\pm_s\ dA_s
			= I^\pm_t
		\end{align}
		が成立し$t \longmapsto I_t(\omega)$の右連続性が出る.$I$は右連続$(\mathscr{F}_t)$-適合過程であるから
		$(\mathscr{F}_t)$-発展的可測である.
		\QED
	\end{prf}
	
	\begin{itembox}[l]{Remark 4.6 (ii) 修正}
		Every continuous, increasing process is natural. Indeed then, for \textcolor{red}{every} $\omega \in \Omega$
		we have
		\begin{align}
			\int_{(0,t]} (M_s(\omega)-M_{s-}(\omega))\ dA_s(\omega) = 0
			\quad \mbox{for every $0 < t < \infty$},
		\end{align}
		because every path $\Set{M_s(\omega)}{0 \leq s < \infty}$ has only countably many discontinuities
		(Theorem 3.8(v)).
	\end{itembox}
	
	\begin{prf}
		$RCLL$なパスの不連続点は高々可算個であり,
		連続な$A$で作る測度に対し一点集合は零集合となる.
		\QED
	\end{prf}
	
	\begin{itembox}[l]{Lemma 4.7 修正}
		If $A$ is an increasing process and $\Set{M_t,\mathscr{F}_t}{0 \leq t < \infty}$ is a bounded,
		\textcolor{red}{$RCLL$} martingale, then
		\begin{align}
			E(M_t A_t) = E \int_{(0,t]} M_s\ dA_s, \quad (\forall t > 0).
			\label{eq:chapter_1_lemma_4_7}
		\end{align}
	\end{itembox}
	
	\begin{prf}
		$t_j^{(n)} \coloneqq jt/2^n,\ (j=0,1,\cdots,2^n)$として
		\begin{align}
			M^{(n)}_s \coloneqq \sum_{j=1}^{2^n} \defunc_{\left(t^{(n)}_{j-1},t^{(n)}_j\right]}(s) M_{t^{(n)}_j},
			\quad (\forall s \in (0,t])
		\end{align}
		とおけば,$M$のパスの右連続性より任意の$s \in (0,t]$で$\lim_{n \to \infty} M^{(n)}_s = M_s$となる.また
		\begin{align}
			E \left[ A_{t^{(n)}_{j-1}} \left( M_{t^{(n)}_j} - M_{t^{(n)}_{j-1}} \right) \right]
			= E \left[ A_{t^{(n)}_{j-1}} \cexp{M_{t^{(n)}_j} - M_{t^{(n)}_{j-1}}}{\mathscr{F}_{t_{j-1}}} \right]
			= 0,
			\quad (j=1,\cdots,2^n)
		\end{align}
		が満たされるから任意の$n \geq 1$で
		\begin{align}
			E\int_{(0,t]} M^{(n)}_s\ dA_s
			&= E \sum_{j=1}^{2^n} M_{t^{(n)}_j} \left( A_{t^{(n)}_j} - A_{t^{(n)}_{j-1}} \right) \\
			&= E(M_t A_t) - \sum_{j=1}^{2^n} E \left[ A_{t^{(n)}_{j-1}} \left( M_{t^{(n)}_j} - M_{t^{(n)}_{j-1}} \right) \right] \\
			&= E(M_t A_t)
		\end{align}
		が成立する.仮定より$\sup{s \geq 0}{|M_s|} \leq b < \infty$を満たす$b$が存在して
		\begin{align}
			\left| \int_{(0,t]} M^{(n)}_s\ dA_s \right| \leq b (A_t - A_0) = b A_t,
			\quad (\forall n \geq 1)
		\end{align}
		となり,$A_t$の可積分性とLebesgueの収束定理より
		\begin{align}
			\lim_{n \to \infty} E \int_{(0,t]} M^{(n)}_s\ dA_s = E \lim_{n \to \infty} \int_{(0,t]} M^{(n)}_s\ dA_s
			= E \int_{(0,t]} M_s\ dA_s 
		\end{align}
		が従い(\refeq{eq:chapter_1_lemma_4_7})を得る.
		\QED
	\end{prf}
	
	\begin{itembox}[l]{Definition 4.8 修正}
		Let us consider the class $\mathscr{S}(\mathscr{S}_a)$ such as
		\begin{align}
			\mathscr{S} \coloneqq \Set{T:\mbox{stopping time of $(\mathscr{F}_t)$}}{\textcolor{red}{T < \infty}},
			\quad \mathscr{S}_a \coloneqq \Set{T:\mbox{stopping time of $(\mathscr{F}_t)$}}{\textcolor{red}{T \leq a}},\ (a > 0).
		\end{align}
		The right-continuous process $\Set{X_t,\mathscr{F}_t}{0 \leq t < \infty}$ is said to be 
		of class $D$, if the family $\{X_T\}_{T \in \mathscr{S}}$ is uniformly integrable;
		of class $DL$, if the family $\{X_T\}_{T \in \mathscr{S}_a}$ is uniformly integrable,
		for every $0 < a < \infty$.
	\end{itembox}
	$T \in \mathscr{S}(\mbox{resp. } \mathscr{S}_a)$, then 
	$T(\omega) < \infty\ (\mbox{resp. } \leq a)$
	for all $\omega \in \Omega$, not $P$-a.s. $\omega$.
	
	\begin{itembox}[l]{Problem 4.9 修正}
		$X = \Set{X_t,\mathscr{F}_t}{0 \leq t < \infty}$ is a right-continuous submartingale.
		Show that under any one of the following conditions, $X$ is of class $DL$.
		\begin{description}
			\item[(a)] $X_t \geq 0$ a.s. for every $t \geq 0$.
			\item[(b)] $X$ has the special form
				\begin{align}
					X_t = M_t + A_t, \quad 0 \leq t < \infty
				\end{align}
				suggested by the Doob-Meyer decomposition, where $\Set{M_t,\mathscr{F}_t}{0 \leq t < \infty}$
				is a martingale and $\Set{A_t,\mathscr{F}_t}{0 \leq t < \infty}$ is an increasing process.
		\end{description}
		Show also that if \textcolor{red}{$\mathscr{F}_0$ contains all the $P$-negligible events in $\mathscr{F}$} and
		$X$ is a uniformly integrable martingale, then it is of class $D$.
	\end{itembox}
	
	\begin{prf}\mbox{}
		\begin{description}
			\item[(a)]
				任意の$T \in \mathscr{S}_a$に対して
				$X_T$は$\mathscr{F}_T/\borel{\R}$-可測であるから
				(Proposition 2.18 修正),任意抽出定理より
				\begin{align}
					\int_{\{X_T > \lambda\}} X_T\ dP
					\leq \int_{\{X_T > \lambda\}} X_a\ dP,
					\quad (\forall \lambda > 0)
				\end{align}
				及び
				\begin{align}
					P\left( X_T > \lambda \right)
					\leq \frac{EX_T}{\lambda}
					\leq \frac{EX_a}{\lambda},
					\quad (\forall \lambda > 0)
				\end{align}
				が成立する.$X_a$が可積分であるから
				\begin{align}
					\sup{T \in \mathscr{S}_a}{\int_{\{X_T > \lambda\}} X_T\ dP}
					\longrightarrow 0
					\quad (\lambda \longrightarrow \infty)
				\end{align}
				となり,$(X_T)_{T \in \mathscr{S}_a}$の一様可積分性が得られる.
				
			\item[(b)]
				$a > 0$とすれば,任意抽出定理より
				\begin{align}
					M_T = \cexp{M_a}{\mathscr{F}_T},\ \mbox{a.s. $P$,}
					\quad (\forall T \in \mathscr{S}_a)
				\end{align}
				が成り立つから,定理\ref{lem:uniformly_integrability_and_conditional_expectations}
				(P. \pageref{lem:uniformly_integrability_and_conditional_expectations})より
				$(M_T)_{T \in \mathscr{S}_a}$は一様可積分である.このとき
				\begin{align}
					\int_{\{|X_T| > \lambda\}} |X_T|\ dP
					&\leq 2\int_{\{|M_T| > \lambda/2\}} |M_T|\ dP + 2\int_{\{|A_T| > \lambda/2\}} |A_T|\ dP \\
					&\leq 2\sup{T \in \mathscr{S}_a}{\int_{\{|M_T| > \lambda/2\}} |M_T|\ dP} + 2\int_{\{A_a > \lambda/2\}} A_a\ dP \\
					&\longrightarrow 0 \quad (\lambda \longrightarrow \infty)
				\end{align}
				が従い$(X_T)_{T \in \mathscr{S}_a}$の一様可積分性が出る.
		\end{description}
		$X$が一様可積分なマルチンゲールであるとき,Problem 3.20より
		\begin{align}
			X_t = \cexp{X_\infty}{\mathscr{F}_t},\ \mbox{a.s. $P$},
			\quad (\forall t \geq 0)
		\end{align}
		を満たす$\mathscr{F}_\infty/\borel{\R}$-可測可積分関数$X_\infty$が存在し,任意抽出定理より
		\begin{align}
			X_T = \cexp{X_\infty}{\mathscr{F}_T},\ \mbox{a.s. $P$},
			\quad (\forall T \in \mathscr{S})
		\end{align}
		が成り立つから$X$はクラス$DL$に属する.
		\QED
	\end{prf}
	
	\begin{itembox}[l]{Problem 4.11 修正}
		Let $(X,\mathscr{F},\mu)$ be a measure space and  
		$\left\{f_n\right\}_{n=1}^\infty$ be a sequence of integrable complex functions on $(X,\mathscr{F},\mu)$
		which converges weakly in $L^1$ to an integrable complex function $f$.
		Then for each $\sigma$-field $\mathscr{G} \subset \mathscr{F}$
		where $(X,\mathscr{G},\left.\mu\right|_{\mathscr{G}})$ is $\sigma$-finite,
		the sequence $\cexp{f_n}{\mathscr{G}}$ converges to $\cexp{f}{\mathscr{G}}$ weakly in $L^1$.
	\end{itembox}
	
	\begin{prf}
		$\nu \coloneqq \left.\mu\right|_{\mathscr{G}}$とおく.
		定理\ref{thm:properties_of_conditional_expectations}より
		任意の$g \in L^\infty(\mu)$と$F \in L^1(\mu)$に対して
		\begin{align}
			\int_X g\cexp{F}{\mathscr{G}}\ d\mu
			&= \int_X \cexp{g\cexp{F}{\mathscr{G}}}{\mathscr{G}}\ d\nu \\
			&= \int_X \cexp{g}{\mathscr{G}}\cexp{F}{\mathscr{G}}\ d\nu \\
			&= \int_X \cexp{\cexp{g}{\mathscr{G}}F}{\mathscr{G}}\ d\nu \\
			&= \int_X \cexp{g}{\mathscr{G}}F\ d\mu
		\end{align}
		と$\Norm{\cexp{g}{\mathscr{G}}}{L^\infty(\nu)} \leq \Norm{g}{L^\infty(\mu)}$が成り立ち
		\begin{align}
			\lim_{n \to \infty} \int_X g\cexp{f_n}{\mathscr{G}}\ d\mu
			= \lim_{n \to \infty} \int_X \cexp{g}{\mathscr{G}}f_n\ d\mu
			= \int_X \cexp{g}{\mathscr{G}}f\ d\mu
			= \int_X g\cexp{f}{\mathscr{G}}\ d\mu
		\end{align}
		となるから$\cexp{f_n}{\mathscr{G}}$は$\cexp{f}{\mathscr{G}}$に$L^1(\mu)$で弱収束する.
		\QED
	\end{prf}
	
	\begin{itembox}[l]{Lemma for theorem 4.10}\label{lem:uniqueness_of_Doob_Meyer_decomposition}
		Let $I \subset [0,\infty)$ be an interval and 
		$\Set{M_t,\mathscr{F}_t}{t \in I}$ be a right-continuous martingale,
		where the filtration $(\mathscr{F}_t)_{t \in I}$ is usual.
		If $M$ is a difference of two natural processes 
		$\Set{A_t,\mathscr{F}_t}{t \in I}$
		and $\Set{B_t,\mathscr{F}_t}{t \in I}$, namely
		\begin{align}
			M_t = A_t - B_t; \quad \forall t \in I,
		\end{align}
		then $P\Set{M_t = 0}{\forall t \in I} = 1$.
	\end{itembox}
	
	\begin{prf}
		$a_0 \coloneqq \inf{}{I}$として任意に$a \in I \cap (a_0,\infty)$を取り,
		\begin{align}
			t^{(n)}_j \coloneqq a_0 + \frac{j}{2^n}(a-a_0), \quad (j=0,1,\cdots,2^n)
		\end{align}
		とおく.任意の有界かつ$RCLL$なマルチンゲール$\xi = \Set{\xi_t,\mathscr{F}_t}{t \in I}$に対し
		\begin{align}
			\xi^{(n)}_t \coloneqq \sum_{j=1}^{2^n} \defunc_{\left(t_{j-1}^{(n)},t_j^{(n)}\right]}(t)\ \xi_{t^{(n)}_{j-1}},
			\quad (\forall t \in (a_0,a])
		\end{align}
		とおけば,任意の$\omega \in \Omega$と$t \in (a_0,a]$で
		\begin{align}
			\lim_{n \to \infty} \xi^{(n)}_t(\omega) = \xi_{t-}(\omega)
		\end{align}
		が満たされるからLebesgueの収束定理より
		\begin{align}
			&\lim_{n \to \infty} \int_{(a_0,a]} \xi^{(n)}_t(\omega)\ dA_t(\omega) 
				= \int_{(a_0,a]} \xi_{t-}(\omega)\ dA_t(\omega), \\
			&\lim_{n \to \infty} \int_{(a_0,a]} \xi^{(n)}_t(\omega)\ dB_t(\omega) 
				= \int_{(a_0,a]} \xi_{t-}(\omega)\ dB_t(\omega)
		\end{align}
		が成立する.また$A_a,B_a$の可積性と$\xi$の有界性により,再びLebesgueの収束定理を適用すれば
		\begin{align}
			E\left[ \xi_a\left( A_a - B_a \right) \right]
			&= E\left[ \xi_a A_a \right] -  E\left[ \xi_a B_a \right]
			= E \int_{(a_0,a]} \xi_{t-}\ dA_t - E\int_{(a_0,a]} \xi_{t-}\ dB_t \\
			&= E \left[ \lim_{n \to \infty} \int_{(a_0,a]} \xi^{(n)}_t\ dA_t \right]
				- E \left[ \lim_{n \to \infty} \int_{(a_0,a]} \xi^{(n)}_t\ dB_t \right] \\
			&= \lim_{n \to \infty} E\left[ \sum_{j=1}^{2^n}\xi_{t^{(n)}_{j-1}}\left( A_{t^{(n)}_j} - A_{t^{(n)}_{j-1}} \right) \right]
				-  \lim_{n \to \infty} E \left[ \sum_{j=1}^{2^n}\xi_{t^{(n)}_{j-1}}\left( B_{t^{(n)}_j} - B_{t^{(n)}_{j-1}} \right) \right] \\
			&= \lim_{n \to \infty} E \left[ \sum_{j=1}^{2^n}\xi_{t^{(n)}_{j-1}}\left( M_{t^{(n)}_j} - M_{t^{(n)}_{j-1}} \right) \right]
		\end{align}
		が従い,このとき右辺は$M$のマルチンゲール性より
		\begin{align}
			E\xi_{t^{(n)}_{j-1}}\left( M_{t^{(n)}_j} - M_{t^{(n)}_{j-1}} \right)
			= E \left[\cexp{\xi_{t^{(n)}_{j-1}}\left( M_{t^{(n)}_j} - M_{t^{(n)}_{j-1}} \right)}{\mathscr{F}_{t^{(n)}_{j-1}}} \right]
			= E \left[ \xi_{t^{(n)}_{j-1}}\cexp{M_{t^{(n)}_j} - M_{t^{(n)}_{j-1}}}{\mathscr{F}_{t^{(n)}_{j-1}}} \right]
			= 0 
		\end{align}
		となるから
		\begin{align}
			E\left[ \xi_a\left( A_a - B_a \right) \right] = 0
		\end{align}
		が得られる.$\xi$を有界マルチンゲール
		$\Set{\cexp{\sgn{A_a - B_a}}{\mathscr{F}_t},\mathscr{F}_t}{t \in I}$
		の$RCLL$な修正とすれば(usual条件よりTheorem 3.13を適用)
		\begin{align}
			0 = E\left[ \xi_a\left( A_a - B_a \right) \right]
			= E\left[ \sgn{A_a - B_a}\left( A_a - B_a \right) \right]
			= E\left| A_a - B_a \right|
		\end{align}
		が成り立ち,$a > 0$の任意性及び$A,B$のパスの右連続性より
		\begin{align}
			P\left[ \Set{A_t = B_t}{t \in I}\right] =
			\begin{cases}
				\displaystyle P\Biggl( \bigcap_{r \in (I \cap \Q) \cup \{\sup{}{I}\}}\{A_r = B_r\} \Biggr) = 1, 
					& (\sup{}{I} \in I), \\
				\displaystyle P\Biggl( \bigcap_{r \in I \cap \Q}\{A_r = B_r\} \Biggr) = 1, & (\sup{}{I} \notin I)
			\end{cases}
		\end{align}
		が出る.
		\QED
	\end{prf}
	
	\begin{itembox}[l]{Theorem 4.10 (Doob-Meyer Decomposition) 修正}
		Let $\{\mathscr{F}_t\}$ satisfy the usual conditions. If the right-continuous
		submartingale $X = \Set{X_t,\mathscr{F}_t}{0 \leq t < \infty}$ is of class $DL$, then
		\textcolor{red}{there exists a unique $[A]_{NAT}$ where $X - A'$ is right-continuous martingale
		for every $A' \in [A]_{NAT}$.}
		Further, if $X$ is of class $D$, then $M$ is a uniformly integrable martingale 
		and $A$ is integrable.	
	\end{itembox}
	
	\begin{prf}\mbox{}
		\begin{description}
			\item[第一段]
				$[A]_{NAT}$の一意性を示す.二つの右連続マルチンゲール$M,M'$とナチュラルな$A,A'$により
				\begin{align}
					X_t = M_t + A_t = M'_t + A'_t,
					\quad \forall t \geq 0
				\end{align}
				と書けるとき,
				\begin{align}
					B=\Set{B_t \coloneqq A_t - A'_t = M'_t - M_t,\mathscr{F}_t}{0 \leq t < \infty}
				\end{align}
				はLemmaの仮定を満たすマルチンゲールとなるから$[A]_{NAT} = [A']_{NAT}$が従う.
				
			\item[第二段]
				任意の区間$[0,a]$上で分解の存在を示せば
				$[0,\infty)$での分解が得られる.
				実際任意の$n \geq 1$に対し
				\begin{align}
					X_t = M^n_t + A^n_t, \quad (t \in [0,n])
				\end{align}
				と分解されるなら,$m > n$に対して
				\begin{align}
					M^n_t + A^n_t = X_t = M^m_t + A^m_t, \quad (t \in [0,n])
				\end{align}
				となり,Lemmaより或る$P$-零集合$E_{n,m}$が存在して,任意の$\omega \in \Omega \backslash E_{n,m}$で
				\begin{align}
					A^n_t(\omega) = A^m_t(\omega), \quad (\forall t \in [0,n])
				\end{align}
				が成立し,かつ$[0,n) \ni t \longmapsto A^n_t(\omega)$が右連続非減少となる.ここで
				\begin{align}
					E \coloneqq \bigcup_{\substack{n,m \in \N \\ n<m}} E_{n,m}
				\end{align}
				により$P$-零集合を定めれば,任意の$\omega \in \Omega \backslash E$及び$t \geq 0$に対して
				\begin{align}
					A^n_t(\omega) = A^m_t(\omega), \quad (\forall m > n > t)
				\end{align}
				となり$\lim_{n \to \infty} A^n_t(\omega)$が確定する.
				usual条件より$E \in \mathscr{F}_0$だから$A^n_t \defunc_{\Omega \backslash E}\ (n > t)$は
				$\mathscr{F}_t/\borel{\R}$-可測であり,
				\begin{align}
					A_t \coloneqq  \lim_{n \to \infty} A^n_t \defunc_{\Omega \backslash E},
					\quad (\forall t \geq 0)
				\end{align}
				で$A_t$を定めれば$A_t$は$\mathscr{F}_t/\borel{\R}$-可測となる.また
				任意の$n \geq 1$で
				\begin{align}
					A_t = A^n_t \defunc_{\Omega \backslash E}, \quad (\forall t \in [0,n))
				\end{align}
				が成り立つから$A_t$は可積分であり,$[0,\infty) \ni t \longmapsto A_t(\omega)$は右連続かつ非減少である.
				$\Set{\xi_t,\mathscr{F}_t}{0 \leq t < \infty}$を有界$RCLL$マルチンゲールとすれば
				任意の$t > 0$で
				\begin{align}
					E \int_{(0,t]} \xi_s\ dA_s = E \int_{(0,t]} \xi_s\ dA^n_s 
					= E \int_{(0,t]} \xi_{s-}\ dA^n_s = E \int_{(0,t]} \xi_{s-}\ dA_s,
					\quad (t < n)
				\end{align}
				が成立する.
				\begin{align}
					M \coloneqq X - A
				\end{align}
				とおけば$(M_t)_{t \geq 0}$は$(\mathscr{F}_t)$-適合かつ可積分であり,
				任意の$0 \leq s < t$及び$t < n$に対して
				\begin{align}
					M_t = X_t - A^n_t \defunc_{\Omega \backslash E} = M^n_t,
					\quad M_s = X_s - A^n_s \defunc_{\Omega \backslash E} = M^n_s,
					\quad \mbox{a.s. $P$}
				\end{align}
				となるから$\cexp{M_t}{\mathscr{F}_s} = M_s\ \mbox{a.s. $P$}$が満たされる.
				次段以降で$[0,a]$上で分解の存在を示す.
			
			\item[第三段]
				%\footnote{
				%	$X_\infty$が定義され
				%	$\Set{X_t,\mathscr{F}_t}{0 \leq t \leq \infty}$が劣マルチンゲールの場合に$a=\infty$とする.
				%}
				$\Set{Z_t,\mathscr{F}_t}{0 \leq t < \infty}$
				を$\Set{\cexp{X_a}{\mathscr{F}_t},\mathscr{F}_t}{0 \leq t < \infty}$の
				右連続な修正として(Theorem 3.13),
				\begin{align}
					Y_t \coloneqq X_t - Z_t,
					\quad (t \in [0,a])
				\end{align}
				により非正値の劣マルチンゲール$\Set{Y_t,\mathscr{F}_t}{0 \leq t \leq a}$を定め
				\begin{align}
					\Set{Y_{t^{(n)}_j},\mathscr{F}_{t^{(n)}_j}}{t^{(n)}_j = \frac{j}{2^n}a,\ j=0,1,\cdots,2^n},
					\quad n=1,2,\cdots,
					%\quad (\mbox{$a = \infty$の場合は$t^{(n)}_j = j/2^n$},\ j \in \N_0)
				\end{align}
				で離散化すれば,離散時のDoob分解 (P. \pageref{lem:Doob_decomposition})より
				\begin{align}
					&A^{(n)}_0 \coloneqq 0,
					\quad A^{(n)}_{t^{(n)}_j} \coloneqq \sum_{k=0}^{j-1} \cexp{Y_{t^{(n)}_{k+1}} - Y_{t^{(n)}_k}}{\mathscr{F}_{t^{(n)}_k}}; \\
					%\ A^{(n)}_\infty \coloneqq \sum_{k=0}^\infty \cexp{Y_{t^{(n)}_{k+1}} - Y_{t^{(n)}_k}}{\mathscr{F}_{t^{(n)}_k}}; \\
					&M^{(n)}_{t^{(n)}_j} \coloneqq Y_{t^{(n)}_j} - A^{(n)}_{t^{(n)}_j}
				\end{align}
				により可予測な増大過程$A^{(n)}$とマルチンゲール$M^{(n)}$に分解され,
				$Y_a = 0\ \mbox{a.s. $P$}$であるから
				\begin{align}
					Y_{t^{(n)}_j} = A^{(n)}_{t^{(n)}_j} +  M^{(n)}_{t^{(n)}_j}
					= A^{(n)}_{t^{(n)}_j} + \cexp{M^{(n)}_a}{\mathscr{F}_{t^{(n)}_j}}
					= A^{(n)}_{t^{(n)}_j} - \cexp{A^{(n)}_a}{\mathscr{F}_{t^{(n)}_j}},
					\quad \mbox{a.s. $P$},
					\quad j=0,1,\cdots,2^n
				\end{align}
				となる.

			\item[第四段]
				$(Y_T)_{T \in \mathscr{S}_a}$が一様可積分であることを示す.
				先ず任意の$T \in \mathscr{S}_a$に対し
				\begin{align}
					Z_T = \cexp{X_a}{\mathscr{F}_T},\quad \mbox{a.s. $P$}
					\label{eq:chapter_1_theorem_4_10_2}
				\end{align}
				が成立する.実際,任意抽出定理より
				\begin{align}
					\int_A Z_T\ dP = \int_A Z_a\ dP
					= \int_A X_a\ dP
					= \int_A \cexp{X_a}{\mathscr{F}_T}\ dP,
					\quad (\forall A \in \mathscr{F}_T)
				\end{align}
				が従い(\refeq{eq:chapter_1_theorem_4_10_2})が得られる.
				$\left(\cexp{X_a}{\mathscr{F}_T}\right)_{T \in \mathscr{S}_a}$は
				定理\ref{lem:uniformly_integrability_and_conditional_expectations}より一様可積分であるから
				$\left(Z_T\right)_{T \in \mathscr{S}_a}$も一様可積分であり,
				また$X$がクラス$DL$に属しているので$(Y_T)_{T \in \mathscr{S}_a}$の一様可積分性が従う.
				
			\item[第五段]
				$\left( A^{(n)}_a \right)_{n=1}^\infty$が一様可積分であることを示す.任意に$\lambda > 0$を取り
				\begin{align}
					T_\lambda^{(n)} \coloneqq
					a \wedge \min{}{\Set{t^{(n)}_{j-1}}{A^{(n)}_{t^{(n)}_j} > \lambda \mbox{ for some } j,\ 1 \leq j \leq 2^n}}
				\end{align}
				とおけば,$A^{(n)}$の可予測性より任意の$t \geq 0$で
				\begin{align}
					\left\{ T_\lambda^{(n)} \leq t \right\}
					= \bigcup_{j\, :\, t^{(n)}_{j-1} \leq t} \left\{ T_\lambda^{(n)} = t^{(n)}_{j-1} \right\}
					= \bigcup_{j\, :\, t^{(n)}_{j-1} \leq t}
						\left[ \bigcap_{k=1}^{j-1} \left\{ A^{(n)}_{t^{(n)}_k} \leq \lambda \right\} \right] 
						\cap \left\{ A^{(n)}_{t^{(n)}_j} > \lambda \right\}
					\in \mathscr{F}_t
				\end{align}
				が成り立つから$T_\lambda^{(n)} \in \mathscr{S}_a$が満たされ,また
				\begin{align}
					\mu < \lambda
					\quad \Longrightarrow \quad
					\left\{T^{(n)}_\lambda < a\right\} \subset \left\{T^{(n)}_\mu < a\right\}
					\label{eq:chapter_1_theorem_4_10_6}
				\end{align}
				及び
				\begin{align}
					T^{(n)}_\lambda(\omega) < a
					\quad \Longrightarrow \quad
					A^{(n)}_{T^{(n)}_\lambda}(\omega) \leq \lambda
					\label{eq:chapter_1_theorem_4_10_3}
				\end{align}
				も満たされる.
				\begin{align}
					N \coloneqq \bigcup_{k=1}^{2^n} \left\{ \cexp{Y_{t^{(n)}_k} - Y_{t^{(n)}_{k-1}}}{\mathscr{F}_{t^{(n)}_{k-1}}} < 0 \right\}
				\end{align}
				により$P$-零集合を定めれば,$\Omega \backslash N$の上で
				$A^{(n)}_0 \leq A^{(n)}_{t^{(n)}_1} \leq \cdots \leq A^{(n)}_a$となるから
				\begin{align}
					\left\{T^{(n)}_\lambda < a\right\} \cap (\Omega \backslash N)
					= \left\{A^{(n)}_a > \lambda\right\} \cap (\Omega \backslash N)
					\label{eq:chapter_1_theorem_4_10_1}
				\end{align}
				が従う.任意に$\Lambda \in \mathscr{F}_{T^{(n)}_\lambda}$を取れば,
				$\Lambda \cap \left\{T^{(n)}_\lambda=t^{(n)}_{j-1}\right\} \in \mathscr{F}_{t^{(n)}_{j-1}},
				\ (j=1,\cdots,2^n)$より
				\begin{align}
					\int_\Lambda Y_{T^{(n)}_\lambda}\ dP = 
					\sum_{j=1}^{2^n} \int_{\Lambda \cap \left\{T^{(n)}_\lambda=t^{(n)}_{j-1}\right\}} Y_{t^{(n)}_{j-1}}\ dP
					&= \sum_{j=1}^{2^n} \int_{\Lambda \cap \left\{T^{(n)}_\lambda=t^{(n)}_{j-1}\right\}} 
						A^{(n)}_{t^{(n)}_{j-1}} - \cexp{A^{(n)}_a}{\mathscr{F}_{t^{(n)}_{j-1}}}\ dP \\
					&= \sum_{j=1}^{2^n} \int_{\Lambda \cap \left\{T^{(n)}_\lambda=t^{(n)}_{j-1}\right\}} 
						A^{(n)}_{T^{(n)}_\lambda} - A^{(n)}_a\ dP \\
					&= \int_\Lambda A^{(n)}_{T^{(n)}_\lambda} - A^{(n)}_a\ dP
					\label{eq:chapter_1_theorem_4_10_5}
				\end{align}
				が成立するから,(\refeq{eq:chapter_1_theorem_4_10_3})と(\refeq{eq:chapter_1_theorem_4_10_1})と併せて
				\begin{align}
					\int_{\left\{A^{(n)}_a > \lambda\right\}} A^{(n)}_a\ dP
					= \int_{\left\{T^{(n)}_\lambda < a\right\}} A^{(n)}_{T^{(n)}_\lambda}\ dP
						- \int_{\left\{T^{(n)}_\lambda < a\right\}} Y_{T^{(n)}_\lambda}\ dP
					\leq \lambda P\left(T^{(n)}_\lambda < a\right) 
						- \int_{\left\{T^{(n)}_\lambda < a\right\}} Y_{T^{(n)}_\lambda}\ dP
				\end{align}
				となる.一方で(\refeq{eq:chapter_1_theorem_4_10_6}),(\refeq{eq:chapter_1_theorem_4_10_3}),
				(\refeq{eq:chapter_1_theorem_4_10_1}),(\refeq{eq:chapter_1_theorem_4_10_5})より
				\begin{align}
					\int_{\left\{T^{(n)}_{\lambda/2} < a\right\}} Y_{T^{(n)}_{\lambda/2}}\ dP
					&= \int_{\left\{T^{(n)}_{\lambda/2} < a\right\}} A^{(n)}_{T^{(n)}_{\lambda/2}} - A^{(n)}_a\ dP \\
					&\leq \int_{\left\{T^{(n)}_{\lambda} < a\right\}} A^{(n)}_{T^{(n)}_{\lambda/2}} - A^{(n)}_a\ dP \\
					&\leq -\frac{\lambda}{2} P\left(T^{(n)}_{\lambda} < a\right)
				\end{align}
				が成立するから
				\begin{align}
					\int_{\left\{A^{(n)}_a > \lambda\right\}} A^{(n)}_a\ dP
					\leq -2 \int_{\left\{T^{(n)}_{\lambda/2} < a\right\}} Y_{T^{(n)}_{\lambda/2}}\ dP
						- \int_{\left\{T^{(n)}_\lambda < a\right\}} Y_{T^{(n)}_\lambda}\ dP
				\end{align}
				となる.ここで
				\begin{align}
					P\left(T^{(n)}_{\lambda} < a\right)
					= P\left(A^{(n)}_a > \lambda\right)
					\leq \frac{E A^{(n)}_a}{\lambda}
					= \frac{- E M^{(n)}_a}{\lambda}
					= \frac{- E M^{(n)}_0}{\lambda}
					= \frac{- E Y_0}{\lambda}
				\end{align}
				より$P\left(T^{(n)}_{\lambda} < a\right)$は$\lambda$のみに依存して
				0に収束し,定理\ref{thm:appendix_uniform_integrability_equivalence}と$(Y_T)_{T \in \mathscr{S}_a}$の
				一様可積分性により
				\begin{align}
					\sup{n \in \N}{\int_{\left\{A^{(n)}_a > \lambda\right\}} A^{(n)}_a\ dP}
					\leq 2 \sup{n \in \N}{\int_{\left\{T^{(n)}_{\lambda/2} < a\right\}} \left|Y_{T^{(n)}_{\lambda/2}}\right|\ dP}
					+ \sup{n \in \N}{\int_{\left\{T^{(n)}_{\lambda} < a\right\}} \left|Y_{T^{(n)}_{\lambda}}\right|\ dP}
					\longrightarrow 0
					\quad (\lambda \longrightarrow \infty)
				\end{align}
				が従い$\left( A^{(n)}_a \right)_{n=1}^\infty$が一様可積分性が出る.
				
			\item[第六段]
				Dunford-Pettisの定理より$\left( A^{(n)}_a \right)_{n=1}^\infty$の或る部分列
				$\left( A^{(n_k)}_a \right)_{k=1}^\infty$は$L^1(P)$で弱収束する.つまり
				或る$A_a \in L^1(P)$が存在して
				任意の$\xi \in L^\infty(P)$に対し
				\begin{align}
					E \left( \xi A^{(n_k)}_a \right) \longrightarrow E (\xi A_a)
					\quad (k \longrightarrow \infty)
				\end{align}
				が成立する.
				\begin{align}
					\Pi_n \coloneqq \Set{t^{(n)}_j}{t^{(n)}_j = \frac{j}{2^n}a,\ j=0,1,\cdots,2^n},
					\quad \Pi \coloneqq \bigcup_{n=1}^\infty \Pi_n
				\end{align}
				とすれば,任意の$t \in \Pi$に対し或る$K \geq 1$が存在して
				$t \in \Pi_{n_k}\ (\forall k > K)$となり,Problem 4.11より
				\begin{align}
					E \left( \xi A^{(n_k)}_t \right)
					= E \xi\left\{ Y_t + \cexp{A^{(n_k)}_a}{\mathscr{F}_t} \right\}
					\longrightarrow E \xi\left\{ Y_t + \cexp{A_a}{\mathscr{F}_t} \right\}
					\quad (k > K,\ k \longrightarrow \infty)
					\label{eq:chapter_1_theorem_4_10_7}
				\end{align}
				が成り立つから$A^{(n_k)}_t$は$Y_t + \cexp{A_a}{\mathscr{F}_t}$に弱収束する.
				ここで
				\begin{align}
					\tilde{A}_t \coloneqq Y_t + \cexp{A_a}{\mathscr{F}_t},
					\quad (t \in [0,a])
				\end{align}
				と定めれば$\Set{\tilde{A}_t,\mathscr{F}_t}{0 \leq t \leq a}$は
				劣マルチンゲールとなり,$\Set{X_t,\mathscr{F}_t}{0 \leq t <\infty}$の右連続性より
				\begin{align}
					[0,a] \ni t \longmapsto E\left[ Y_t + \cexp{A_a}{\mathscr{F}_t} \right]
					= E X_t - E X_a + E A_a
				\end{align}
				は右連続であるから(Theorem 3.13),$\tilde{A}$の右連続な修正$\Set{A_t,\mathscr{F}_t}{0 \leq t \leq a}$
				が得られる.
			
			\item[第七段]
				$t \longmapsto A_t(\omega)$がa.s.に0出発かつ非減少であることを示す.
				実際,$\xi = \sgn{A_0}$として,(\refeq{eq:chapter_1_theorem_4_10_7})より
				\begin{align}
					E |A_0| = E \xi A_0 = E \xi \tilde{A}_0 = \lim_{k \to \infty} E \xi A^{(n_k)}_0 = 0
				\end{align}
				が成り立つから$A_0 = 0\ \mbox{a.s. $P$}$が従う.また任意に$s,t \in \Pi,\ (s<t)$を取れば
				或る$K \geq 1$が存在して$s,t \in \Pi_{n_k}\ (\forall k > K)$が満たされ,
				$A^{(n_k)}$は増大過程であるから$\xi = \defunc_{\{A_s > A_t\}}$として
				\begin{align}
					E \xi (A_t - A_s) = E \xi \left( \tilde{A}_t - \tilde{A}_s \right)
					= \lim_{k \to \infty} E \xi \left( A^{(n_k)}_t - A^{(n_k)}_s \right) \geq 0 
				\end{align}
				となり$P(A_s > A_t) = 0$が成り立つ.$t \longmapsto A_t$が右連続性であるから,$P$-零集合を
				\begin{align}
					N \coloneqq \Biggl(\bigcup_{\substack{s,t \in \Pi \\ s < t}} \{A_s > A_t\}\Biggr) \cup \{A_0 \neq 0\}
				\end{align}
				で定めれば$\Omega \backslash N$上で$t \longmapsto A_t$は0出発非減少となり,
				$N$上で$A \equiv 0$と修正すれば$A$は増大過程となる.
				
			\item[第八段]
				$A$がナチュラルであることを示す.$\xi = \Set{\xi_t,\mathscr{F}_t}{0 \leq t \leq a}$を有界な$RCLL$マルチンゲールとすれば
				\begin{align}
					E \xi_a A^{(n_k)}_a 
					&= E\left[ \sum_{j=1}^{2^{n_k}}\xi_{t^{(n_k)}_{j-1}} \left( A^{(n_k)}_{t^{(n_k)}_j} - A^{(n_k)}_{t^{(n_k)}_{j-1}} \right) \right] \\
					&= E\left[ \sum_{j=1}^{2^{n_k}}\xi_{t^{(n_k)}_{j-1}} \left( Y_{t^{(n_k)}_j} - Y_{t^{(n_k)}_{j-1}} \right) \right]
						+ E\left[ \sum_{j=1}^{2^{n_k}}\xi_{t^{(n_k)}_{j-1}} \left( \cexp{A^{(n_k)}_a}{\mathscr{F}_{t^{(n_k)}_j}} - \cexp{A^{(n_k)}_a}{\mathscr{F}_{t^{(n_k)}_{j-1}}} \right) \right] \\
					&= E\left[ \sum_{j=1}^{2^{n_k}}\xi_{t^{(n_k)}_{j-1}} \left( A_{t^{(n_k)}_j} - A_{t^{(n_k)}_{j-1}} \right) \right]
				\end{align}
				が任意の$k \geq 1$で成り立ち(Proposition 4.3),$k \longrightarrow \infty$として
				\begin{align}
					E \xi_a A_a = E \int_{(0,a]} \xi_{s-}\ dA_s
				\end{align}
				が得られる.任意の$t \in (0,a]$に対し
				$\xi^{(t)} = \Set{\xi^{(t)}_s \coloneqq \xi_{t \wedge s},\mathscr{F}_s}{0 \leq s \leq a}$
				も$RCLL$マルチンゲールであり
				\begin{align}
					\xi^{(t)}_{s-} &= \xi_{s-},\quad (\forall s \in (0,t]), \\
					\xi^{(t)}_{s-} &= \xi_t, \quad (\forall s \in (t,a])
				\end{align}
				より
				\begin{align}
					E \xi_t A_t + E \xi_t(A_a - A_t) = E \xi^{(t)}_a A_a 
					= E \int_{(0,a]} \xi^{(t)}_{s-}\ dA_s
					= E \int_{(0,t]} \xi_{s-}\ dA_s + E \xi_t (A_a - A_t)
				\end{align}
				となり
				\begin{align}
					E \xi_t A_t = E \int_{(0,t]} \xi_{s-}\ dA_s,
					\quad (\forall t \in (0,a])
				\end{align}
				が成立する.よって$A$はナチュラルである.
				
			\item[第九段]
				$\Set{M_t \coloneqq X_t - A_t, \mathscr{F}_t}{0 \leq t \leq a}$がマルチンゲールであることを示す.
				$M$の適合性と可積分性は$X,A$のそれより従い,また任意に
				$0 \leq s \leq t \leq a$を取れば,任意の$A \in \mathscr{F}_s$で
				\begin{align}
					\int_A M_s\ dP = \int_A X_s - A_s\ dP
					&= \int_A X_s - \left(Y_s - \cexp{A_a}{\mathscr{F}_s}\right)\ dP \\
					&= \int_A X_s - \left(X_s - Z_s - \cexp{A_a}{\mathscr{F}_s}\right)\ dP \\
					&= \int_A Z_t + \cexp{A_a}{\mathscr{F}_t}\ dP \\
					&= \int_A X_t - \left(X_t - Z_t - \cexp{A_a}{\mathscr{F}_t}\right)\ dP \\
					&= \int_A M_t\ dP
				\end{align}
				が成立する.
				\QED
		\end{description}
	\end{prf}
	
	\begin{itembox}[l]{Problem 4.13}
		Verify that a continuous, nonnegative submartingale is regular. 
	\end{itembox}
	
	\begin{prf}
		Problem 4.9 より$(X_{T_n})_{n=1}^\infty$は一様可積分であり,またパスの連続性より
		$X_{T_n} \longrightarrow X_T\ (n \longrightarrow \infty)$
		となるから,定理\ref{lem:uniformly_integrable_and_convergence_in_mean}より
		$\lim_{n \to \infty} EX_{T_n} = EX_T$が成立する.
		\QED
	\end{prf}
	
	\begin{itembox}[l]{Theorem 4.14 修正}
		Suppose that $X = \Set{X_t}{0 \leq t < \infty}$ is a right-continuous submartingale
		of class $DL$ with respect to the filtration $\{\mathscr{F}_t\}$, which
		satisfies the usual conitions, and 
		\textcolor{red}{let $[A]_{NAT}$ be of the Doob-Meyer decomposition of $X$.
		There exists a continuous version of $A$ in $[A]_{NAT}$ if and only if $X$ is regular.}
	\end{itembox}
	
	\begin{prf}\mbox{}
		\begin{description}
			\item[第一段] $A$が連続であるとき,
				増大列$\{T_n\}_{n=1}^\infty \subset \mathscr{S}_a$と
				$T \coloneqq \lim_{n \to \infty} \in T_n \mathscr{S}_a$に対し
				単調収束定理より
				\begin{align}
					\lim_{n \to \infty} EA_{T_n}
					= E \lim_{n \to \infty} A_{T_n}
					= EA_T
				\end{align}
				が成立する.また任意抽出定理より
				\begin{align}
					E(X_{T_n} - A_{T_n}) = E(X_{T} - A_{T}),
					\quad (\forall n \geq 1)
				\end{align}
				となるから
				\begin{align}
					\lim_{n \to \infty} EX_{T_n} 
					= \lim_{n \to \infty} E(X_{T_n} - A_{T_n}) + \lim_{n \to \infty} EA_{T_n} 
					= E(X_{T} - A_{T}) + EA_T
					= EX_T
				\end{align}
				が従う.
				
			\item[第二段]
				以降$X$がレギュラーであるとする.このとき任意の有界な停止時刻の増大列
				$(T_n)$と$T \coloneqq \lim T_n$に対し,$X-A$のマルチンゲール性と任意抽出定理,
				及び$X$のレギュラリティより
				\begin{align}
					EA_{T_n} &= EX_{T_n} - E(X_{T_n} - A_{T_n})
					= EX_{T_n} - E(X_T - A_T) \\
					&\qquad \longrightarrow EX_T - E(X_T - A_T)
					= EA_T
					\quad (n \longrightarrow \infty)
					\label{eq:chapter_1_theorem_4_14_1}
				\end{align}
				が得られる.いま,任意に$a \in \N$を取り
				\begin{align}
					\Pi_n \coloneqq 
					\Set{t^{(n)}_j}{t^{(n)}_j = \frac{j}{2^n}a,\ j=0,1,\cdots,2^n},
					\quad \Pi \coloneqq \bigcup_{n=1}^\infty \Pi_n
				\end{align}
				とおく.また任意に$\lambda \in \N$を取り,各$j = 0,1,\cdots,2^n$に対し
				\begin{align}
					Y^{(n),j}_t \coloneqq
					\cexp{\lambda \wedge A_{t^{(n)}_{j+1}}}{\mathscr{F}_t},
					\quad (\forall t \geq 0)
				\end{align}
				によりマルチンゲール$\Set{Y^{(n),j}_t,\mathscr{F}_t}{0 \leq t < \infty}$を定めれば,
				\begin{align}
					[0,\infty) \ni t \longmapsto EY^{(n),j}_t 
					= E\left(\lambda \wedge A_{t^{(n)}_{j+1}}\right)
				\end{align}
				と Theorem 3.13 より$RCLL$な修正$\tilde{Y}^{(n),j}$が存在する.このとき
				各$t \geq 0$で
				\begin{align}
					\int_A \tilde{Y}^{(n),j}_t\ dP 
					= \int_A \lambda \wedge A_{t^{(n)}_{j+1}}\ dP
					\leq \lambda P(A),
					\quad (\forall A \in \mathscr{F}_t)
				\end{align}
				となり,一方で各$t \in \left[t^{(n)}_j, t^{(n)}_{j+1} \right)$で
				\begin{align}
					\int_A \tilde{Y}^{(n),j}_t\ dP
					= \int_A \lambda \wedge A_{t^{(n)}_{j+1}}\ dP
					\geq \int_A \lambda \wedge A_t\ dP,
					\quad (\forall A \in \mathscr{F}_t) 
				\end{align}
				となるから,各$j$で
				\begin{align}
					E_j &\coloneqq \Set{\tilde{Y}^{(n),j}_t > \lambda}{\exists t \geq 0} 
						\cup \Set{\tilde{Y}^{(n),j}_t < \lambda \wedge A_t}{\exists t \in \left[t^{(n)}_j, t^{(n)}_{j+1} \right)} \\
					&= \left[ \bigcup_{r \in [0,\infty)\cap\Q}\left\{\tilde{Y}^{(n),j}_r > \lambda\right\} \right]
					\bigcup \left[ \bigcup_{r \in \left[t^{(n)}_j, t^{(n)}_{j+1} \right)\cap\Q}\left\{\tilde{Y}^{(n),j}_r < \lambda \wedge A_r\right\} \right]
				\end{align}
				とおけば$P$-零集合$E \coloneqq \bigcup_{j=0}^{2^n} E_j$が定まる.usual条件より$E \in \mathscr{F}_0$であるから
				\begin{align}
					\Set{Z^{(n),j}_t \coloneqq \tilde{Y}^{(n),j}_t \defunc_{\Omega \backslash E},
					\mathscr{F}_t}{0 \leq t < \infty}
				\end{align}
				で定める$Y^{(n),j}$のバージョン$Z^{(n),j}$は
				\begin{align}
					\omega \in \Omega \backslash E
					\quad \Longrightarrow \quad
					\begin{cases}
						Z^{(n),j}_t(\omega) \leq \lambda, & \forall t \geq 0, \\
						Z^{(n),j}_t(\omega) \geq \lambda \wedge A_t(\omega), & \forall t \in \left[t^{(n)}_j, t^{(n)}_{j+1} \right)
					\end{cases}
				\end{align}
				を満たす$RCLL$かつ有界なマルチンゲールとなり,
				\begin{align}
					\eta^{(n)}_t \coloneqq
					\sum_{j=0}^{2^n-1} Z^{(n),j}_t \defunc_{\left[t^{(n)}_j,t^{(n)}_{j+1}\right)}(t)
						+ (\lambda \wedge A_a) \defunc_{[a,\infty)}(t),
					\quad (t \geq 0)
				\end{align}
				とおけば
				\begin{align}
					\omega \in \Omega \backslash E
					\quad \Longrightarrow \quad
					\begin{cases}
						\eta^{(n)}_t(\omega) \leq \lambda, & (\forall t \geq 0), \\
						\eta^{(n)}_t(\omega) \geq \lambda \wedge A_t(\omega), & (\forall t \in [0,a])
					\end{cases}
					\label{eq:chapter_1_theorem_4_14_4}
				\end{align}
				が成り立つ.また$\eta^{(n)}$の右連続性,Corollary2.4,Problem2.5 及びusual条件より
				\begin{align}
					T^{(n)}_\epsilon \coloneqq
					a \wedge \inf{}{\Set{t \geq 0}{\eta^{(n)}_t - (\lambda \wedge A_t)  > \epsilon}}
				\end{align}
				は$\mathscr{S}_a$に属する停止時刻となり,このとき
				\begin{align}
					\varphi_n(t) \coloneqq 
					\begin{cases}
						t^{(n)}_{j+1}, & t^{(n)}_j \leq t < t^{(n)}_{j+1},\ j=0,1,\cdots,2^n-1 \\
						a, & t = a
					\end{cases}
				\end{align}
				を用いれば,任意抽出定理より
				\begin{align}
					E\left( \eta_{T^{(n)}_\epsilon} \right)
					&= \sum_{j=0}^{2^n-1} \int_{\left\{t^{(n)}_j \leq T^{(n)}_\epsilon < t^{(n)}_{j+1}\right\}} Z^{(n),j}_{T^{(n)}_\epsilon}\ dP
						+ \int_{\left\{T^{(n)}_\epsilon = a\right\}} \lambda \wedge A_a\ dP \\
					&= \sum_{j=0}^{2^n-1} \int_{\left\{t^{(n)}_j \leq T^{(n)}_\epsilon < t^{(n)}_{j+1}\right\}} \cexp{Z^{(n),j}_{t^{(n)}_{j+1}}}{\mathscr{F}_{T^{(n)}_\epsilon}}\ dP
						+ \int_{\left\{T^{(n)}_\epsilon = a\right\}} \lambda \wedge A_a\ dP \\
					&= \sum_{j=0}^{2^n-1} \int_{\left\{t^{(n)}_j \leq T^{(n)}_\epsilon < t^{(n)}_{j+1}\right\}} Z^{(n),j}_{t^{(n)}_{j+1}}\ dP
						+ \int_{\left\{T^{(n)}_\epsilon = a\right\}} \lambda \wedge A_a\ dP \\
					&= \sum_{j=0}^{2^n-1} \int_{\left\{t^{(n)}_j \leq T^{(n)}_\epsilon < t^{(n)}_{j+1}\right\}} \lambda \wedge A_{t^{(n)}_{j+1}}\ dP
						+ \int_{\left\{T^{(n)}_\epsilon = a\right\}} \lambda \wedge A_a\ dP \\
					&= \sum_{j=0}^{2^n-1} \int_{\left\{t^{(n)}_j \leq T^{(n)}_\epsilon < t^{(n)}_{j+1}\right\}} \lambda \wedge A_{\varphi_n\left(T^{(n)}_\epsilon\right)}\ dP
						+ \int_{\left\{T^{(n)}_\epsilon = a\right\}} \lambda \wedge A_{\varphi_n\left(T^{(n)}_\epsilon\right)}\ dP \\
					&= E\left(\lambda \wedge A_{\varphi_n\left(T^{(n)}_\epsilon\right)}\right)
				\end{align}
				が従う.また$t \longmapsto \eta^{(n)}_t - (\lambda \wedge A_t)$の右連続性より
				\begin{align}
					T^{(n)}_\epsilon(\omega) < a
					\quad \Longrightarrow
					\quad \eta^{(n)}_{T^{(n)}_\epsilon}(\omega) - \left(\lambda \wedge A_{T^{(n)}_\epsilon}(\omega)\right)
						\geq \epsilon
				\end{align}
				となるから
				\begin{align}
					E\left(\lambda \wedge A_{\varphi_n\left(T^{(n)}_\epsilon\right)}
						- \lambda \wedge A_{T^{(n)}_\epsilon} \right)
					&= E\left(\eta^{(n)}_{T^{(n)}_\epsilon}
						- \lambda \wedge A_{T^{(n)}_\epsilon} \right) \\
					&= E\defunc_{\left\{T^{(n)}_\epsilon < a\right\}}\left(\eta^{(n)}_{T^{(n)}_\epsilon}
						- \lambda \wedge A_{T^{(n)}_\epsilon} \right)
					\geq \epsilon P\left(T^{(n)}_\epsilon < a\right)
					\label{eq:chapter_1_theorem_4_14_5}
				\end{align}
				が成立する.
				
			\item[第三段]
				$\left( \eta^{(n)} \right)_{n=1}^\infty$は$n$に関して
				$P$-a.s. に減少していく.実際,任意の$t \in [0,a)$に対し
				\begin{align}
					t \in \left[t^{(n)}_j, t^{(n)}_{j+1}\right)
				\end{align}
				を満たす$0 \leq j \leq 2^n-1$を取れば
				$t \in \left[t^{(n+1)}_{2j}, t^{(n+1)}_{2j+1}\right)$或は
				$t \in \left[t^{(n+1)}_{2j+1}, t^{(n+1)}_{2j+2}\right)$となるから,
				任意の$A \in \mathscr{F}_t$で
				\begin{align}
					\int_A \eta^{(n)}_t\ dP
					= \int_A \lambda \wedge A_{t^{(n)}_{j+1}}\ dP
					\begin{cases}
						\displaystyle= \int_A \lambda \wedge A_{t^{(n+1)}_{2j+2}}\ dP \\
						\displaystyle\geq \int_A \lambda \wedge A_{t^{(n+1)}_{2j+1}}\ dP
					\end{cases}
					= \int_A \eta^{(n+1)}_t\ dP
				\end{align}
				が成り立ち$\eta^{(n)}_t \geq \eta^{(n+1)}_t,\ \mbox{a.s. $P$}$が従う.
				$\eta^{(n)},\eta^{(n+1)}$のパスは右連続であるから
				\begin{align}
					F_n \coloneqq \Set{\eta^{(n)}_t < \eta^{(n+1)}_t}{\exists t \in [0,a)}
					= \bigcup_{r \in [0,a) \cap \Q} \left\{\eta^{(n)}_r < \eta^{(n+1)}_r\right\}
				\end{align}
				で$P$-零集合が定まり,$F \coloneqq \bigcup_{n=1}^\infty F_n$とおけば
				任意の$\omega \in \Omega \backslash F$と$t \in [0,a]$で
				$\left( \eta^{(n)}_t(\omega) \right)_{n=1}^\infty$は減少し
				\begin{align}
					T^{(1)}_\epsilon \defunc_{\Omega \backslash F} 
					\leq T^{(2)}_\epsilon \defunc_{\Omega \backslash F} \leq \cdots \leq a
					\label{eq:chapter_1_theorem_4_14_2}
				\end{align}
				となる.usual条件より$F \in \mathscr{F}_0$であるから
				\begin{align}
					\left\{ T^{(n)}_\epsilon \defunc_{\Omega \backslash F} \leq t \right\}
					= \left\{ T^{(n)}_\epsilon \leq t \right\} \cap (\Omega \backslash F) + F
					\in \mathscr{F}_t,\quad (\forall t \geq 0)
				\end{align}
				が成り立つので$T^{(n)}_\epsilon \defunc_{\Omega \backslash F} \in \mathscr{S}_a$となり,
				単調増大性より
				\begin{align}
					T_\epsilon \coloneqq \lim_{n \to \infty} T^{(n)}_\epsilon \defunc_{\Omega \backslash F}
				\end{align}
				と定めれば$T_\epsilon \in \mathscr{S}_a$も満たされる.
				一方$\varphi_n\left(T^{(n)}_\epsilon\right)$についても
				\begin{align}
					\left\{\varphi_n\left(T^{(n)}_\epsilon\right) \leq t\right\}
					= \bigcup_{j\, :\, t^{(n)}_{j+1} \leq t} \left\{t^{(n)}_j \leq T^{(n)}_\epsilon < t^{(n)}_{j+1}\right\}
					\in \mathscr{F}_t,
					\quad (\forall t \geq 0)
				\end{align}
				より$\varphi_n\left(T^{(n)}_\epsilon\right) \in \mathscr{S}_a$が従い,
				また$\varphi_n(t) \geq t$と$t \longmapsto \varphi_n(t)$の増大性より
				\begin{align}
					T^{(n)}_\epsilon(\omega)
					\leq \varphi_n\left(T^{(n)}_\epsilon(\omega)\right)
					\leq \varphi_n\left(T_\epsilon(\omega)\right),
					\quad (\forall \omega \in \Omega \backslash F)
				\end{align}
				が成立し,$A$のパスの増大性と併せて
				\begin{align}
					E \left( \lambda \wedge A_{T^{(n)}_\epsilon} \right)
					\leq E \left( \lambda \wedge A_{\varphi_n\left(T^{(n)}_\epsilon\right)} \right)
					\leq E \left( \lambda \wedge A_{\varphi_n\left(T_\epsilon\right)} \right)
				\end{align}
				が満たされる.このとき(\refeq{eq:chapter_1_theorem_4_14_1})より
				\begin{align}
					\lim_{n \to \infty} E \left( \lambda \wedge A_{T^{(n)}_\epsilon} \right)
					= E \left( \lambda \wedge A_{T_\epsilon} \right)
				\end{align}
				が成り立ち,右辺も$\varphi_n(t) \downarrow t$と$A$のパスの右連続性
				及びLebesgueの収束定理より$E \left( \lambda \wedge A_{T_\epsilon} \right)$に収束するから
				\begin{align}
					\lim_{n \to \infty} E \left( \lambda \wedge A_{\varphi_n\left(T^{(n)}_\epsilon\right)} \right) 
					= E \left( \lambda \wedge A_{T_\epsilon} \right)
				\end{align}
				が得られる.
			
			\item[第五段]
				任意の$\omega \in \Omega$と$n \geq 1$に対し
				\begin{align}
					T^{(n)}_\epsilon(\omega) < a
					\quad \Longleftrightarrow \quad
					\sup{0 \leq t \leq a}{\left\{(\lambda \wedge A_t(\omega)) - \eta^{(n)}_t(\omega)\right\}} > \epsilon
				\end{align}
				が満たされ,また(\refeq{eq:chapter_1_theorem_4_14_4})より
				$\Omega \backslash E$の上で$\eta^{(n)}_t - (\lambda \wedge A_t) \geq 0,\ (\forall t \in [0,a])$だから,
				(\refeq{eq:chapter_1_theorem_4_14_5})と前段の結果と併せて
				\begin{align}
					&P\left(\sup{0 \leq t \leq a}{\left|\eta^{(n)}_t - (\lambda \wedge A_t)\right|} > \epsilon\right)
					= P\left(T^{(n)}_\epsilon < a\right) \\
					&\qquad \leq \frac{1}{\epsilon} E\left(\lambda \wedge A_{\varphi_n\left(T^{(n)}_\epsilon\right)} 
						- \lambda \wedge A_{T^{(n)}_\epsilon} \right)
					\longrightarrow \frac{1}{\epsilon} E\left(\lambda \wedge A_{T_\epsilon}
						- \lambda \wedge A_{T_\epsilon} \right) = 0 \quad (n \longrightarrow \infty)
				\end{align}
				が得られる.従って定理\ref{thm:convergence_in_measure_then_convergence_almost_everywhere}より
				或る部分列$(n_k)_{k=1}^\infty$と$P$-零集合$G$が存在して
				\begin{align}
					\sup{0 \leq t \leq a}{\left|\eta^{(n_k)}_t(\omega) - (\lambda \wedge A_t(\omega))\right|}
					\longrightarrow 0
					\quad (k \longrightarrow \infty),
					\quad (\forall \omega \in \Omega \backslash G)
					\label{eq:chapter_1_theorem_4_14_6}
				\end{align}
				が成立する.
				
			\item[第六段]
				$A$はナチュラルであり,$Z^{(n),j}$は有界かつ$RCLL$なマルチンゲールであるから
				\begin{align}
					E\int_{\left(t^{(n)}_j,t^{(n)}_{j+1}\right]} Z^{(n),j}_s\ dA_s
					&= E\int_{\left(0,t^{(n)}_{j+1}\right]} Z^{(n),j}_s\ dA_s
						- E\int_{\left(0,t^{(n)}_j\right]} Z^{(n),j}_s\ dA_s \\
					&= E\int_{\left(0,t^{(n)}_{j+1}\right]} Z^{(n),j}_{s-}\ dA_s
						- E\int_{\left(0,t^{(n)}_j\right]} Z^{(n),j}_{s-}\ dA_s \\
					&= E\int_{\left(t^{(n)}_j,t^{(n)}_{j+1}\right]} Z^{(n),j}_{s-}\ dA_s
				\end{align}
				が成立する.従って
				\begin{align}
					\xi^{(n)}_t \coloneqq
					\sum_{j=0}^{2^n-1} Z^{(n),j}_t \defunc_{\left(t^{(n)}_j,t^{(n)}_{j+1}\right]}(t),
					\quad (t \geq 0)
				\end{align}
				とおけば任意の$t \in (0,a]$で$\xi^{(n)}_{t-}$が存在し
				\begin{align}
					E\int_{(0,a]} \xi^{(n)}_s\ dA_s
					= \sum_{j=0}^{2^n-1} E\int_{\left(t^{(n)}_j,t^{(n)}_{j+1}\right]} Z^{(n),j}_s\ dA_s
					= \sum_{j=0}^{2^n-1} E\int_{\left(t^{(n)}_j,t^{(n)}_{j+1}\right]} Z^{(n),j}_{s-}\ dA_s
					= E\int_{(0,a]} \xi^{(n)}_{s-}\ dA_s
				\end{align}
				が成立する.一方で$t \notin \Pi$で$\xi^{(n)}_t = \eta^{(n)}_t,\ (\forall n \geq 1)$
				であるから(\refeq{eq:chapter_1_theorem_4_14_6})より
				\begin{align}
					\sup{t \in (0,a]\backslash\Pi}{\left|\xi^{(n_k)}_t(\omega) - \lambda \wedge A_t(\omega)\right|}
					\longrightarrow 0 \quad (k \longrightarrow \infty),
					\quad (\forall \omega \in \Omega \backslash G)
				\end{align}
				が従い,これにより
				\begin{align}
					\sup{t \in (0,a]}{\left|\xi^{(n_k)}_{t-}(\omega) - \lambda \wedge A_{t-}(\omega)\right|}
					\longrightarrow 0 \quad (k \longrightarrow \infty),
					\quad (\forall \omega \in \Omega \backslash G)
				\end{align}
				も出る.実際,$\omega \in \Omega \backslash G$を固定すれば,
				任意の$\epsilon > 0$に対し或る$K = K(\omega,\epsilon) \geq 1$が存在して
				\begin{align}
					\sup{t \in (0,a]\backslash\Pi}{\left|\xi^{(n_k)}_t(\omega) - \lambda \wedge A_t(\omega)\right|} 
					< \epsilon,\quad (\forall k \geq K)
				\end{align}
				となり,このとき任意の$t \in (0,a]$と$k \geq K$で
				\begin{align}
					&\left|\xi^{(n_k)}_{t-}(\omega) - \lambda \wedge A_{t-}(\omega)\right| \\
					&\quad \leq \left|\xi^{(n_k)}_{t-}(\omega) - \xi^{(n_k)}_{s}(\omega)\right|
						+ \left|\xi^{(n_k)}_{s}(\omega) - \lambda \wedge A_{s}(\omega)\right|
						+ \left|\lambda \wedge A_{s}(\omega) - \lambda \wedge A_{t-}(\omega)\right| \\
					&\quad < \epsilon
				\end{align}
				を満たす$s = s(t,k) \in (0,a]\backslash\Pi,\ (s < t)$が取れるから
				\begin{align}
					\sup{t \in (0,a]}{\left|\xi^{(n_k)}_{t-}(\omega) - \lambda \wedge A_{t-}(\omega)\right|} 
					\leq \epsilon,\quad (\forall k \geq K)
				\end{align}
				が成立する.$t \in \Pi$なら或る$N = N(t)$で$t \in \Pi_N$となるから
				$\xi^{(n)}_t = \lambda \wedge A_t,\ \mbox{$P$-a.s.},\ (\forall n \geq N)$となり
				\begin{align}
					H_t \coloneqq \bigcup_{n \geq N} \left\{\xi^{(n)}_t \neq \lambda \wedge A_t\right\},
					\quad H \coloneqq \bigcup_{t \in \Pi} H_t
				\end{align}
				により$P$-零集合$H$を定めれば任意の$t \in [0,a]$で
				\begin{align}
					\lim_{k \to \infty} \xi^{(n_k)}_t(\omega) = \lambda \wedge A_t(\omega),
					\quad (\forall \omega \in \Omega \backslash (G \cup H))
				\end{align}
				となる.Lebesgueの収束定理より
				\begin{align}
					E\int_{(0,a]} \lambda \wedge A_t\ dA_t
					= E\int_{(0,a]} \lambda \wedge A_{t-}\ dA_t
				\end{align}
				が得られ,$A$の単調非減少性より$A_{t-} \leq A_t$であるから
				或る$P$-零集合$U_a$が存在し,任意の$\omega \in \Omega \backslash U_a$で
				\begin{align}
					\int_{(0,a]} (\lambda \wedge A_t(\omega)) 
					- (\lambda \wedge A_{t-}(\omega))\ dA_t(\omega) = 0
				\end{align}
				が成立し$(0,a] \ni t \longmapsto \lambda \wedge A_t(\omega)$
				の連続性が出る.$a$の任意性より
				$V_\lambda \coloneqq \bigcup_{a=1}^\infty U_a$とおけば
				\begin{align}
					(0,\infty) \ni t \longmapsto \lambda \wedge A_t(\omega),
					 \quad (\forall \omega \in \Omega \backslash V_\lambda)
				\end{align}
				は連続となり,$\lambda$も任意であるから
				$V \coloneqq \bigcup_{\lambda=1}^\infty V_\lambda$として
				\begin{align}
					(0,\infty) \ni t \longmapsto A_t(\omega),
					\quad (\forall \omega \in \Omega \backslash V)
				\end{align}
				は連続となる.$\tilde{A} \coloneqq A \defunc_{\Omega \backslash V} \in [A]_{NAT}$
				が求める$A$のバージョンである.
				\QED
		\end{description}
	\end{prf}
	
	\begin{itembox}[l]{Problem 4.15}
		Let $X = \Set{X_t,\mathscr{F}_t}{0 \leq t < \infty}$ be a continuous, nonnegative process
		with $X_0 = 0$ a.s., and $A = \Set{A_t,\mathscr{F}_t}{0 \leq t < \infty}$ any continuous,
		increasing process for which
		\begin{align}
			E(X_T) \leq E(A_T)
		\end{align}
		holds for every bounded stopping time $T$ of $\{\mathscr{F}_t\}$. Introduce the process
		$V_t \coloneqq \max{0 \leq s \leq t}{X_s}$, consider a continuous, increasing function $F$
		on $[0,\infty)$ with $F(0) = 0$, and define \textcolor{red}{$G(x) \coloneqq 2F(x) + x\int_{(x,\infty)} u^{-1}\ dF(u);
		\ 0 < x < \infty$.} Establish the inequalities
		\begin{description}
			\item[(4.14)] $\displaystyle P[V_T \geq \epsilon] \leq \frac{E(A_T)}{\epsilon};\quad \forall \epsilon > 0$
			\textcolor{red}{\item[(4.15) (Lenglart inequality)] $\displaystyle P[V_T \geq \epsilon] 
				\leq \frac{E(\delta \wedge A_T)}{\epsilon} + P[A_T \geq \delta];\quad \forall \epsilon > 0,\ \delta > 0$}
			\item[(4.16)] $EF(V_T) \leq EG(A_T)$
		\end{description}
		for any stopping time $T$ of $\{\mathscr{F}_t\}$.
	\end{itembox}
	
	\begin{prf}\mbox{}
		\begin{description}
			\item[(1)] $X$のパスの連続性とProblem 2.7より
				\begin{align}
					H_\epsilon \coloneqq \inf{}{\Set{t \geq 0}{X_t \geq \epsilon}}
				\end{align}
				で$(\mathscr{F}_t)$-停止時刻が定まる.このとき
				\begin{align}
					V_T(\omega) \geq \epsilon 
					&\quad \Longrightarrow \quad
					X_t(\omega) \geq \epsilon, \quad \exists t \in [0,T(\omega)] \\
					&\quad \Longrightarrow \quad
					H_\epsilon(\omega) \leq t \leq T(\omega)
				\end{align}
				が成立するから,$\{X_0 = 0\} \cap \{V_T \geq \epsilon\}$上で
				$\epsilon = X_{H_\epsilon} = X_{T \wedge H_\epsilon}$となり
				\begin{align}
					\epsilon P(V_T \geq \epsilon)
					= \int_{\{V_T \geq \epsilon\}} X_{T \wedge H_\epsilon}\ dP
					\leq EX_{T \wedge H_\epsilon}
					\leq EA_{T \wedge H_\epsilon}
					\leq EA_T
				\end{align}
				が得られる.
				
			\item[(2)] $S_\delta \coloneqq \inf{}{\Set{t \geq 0}{A_t \geq \delta}}$により$(\mathscr{F}_t)$-停止時刻を定めれば,
				$A_{S_\delta} = \delta$と$t \longmapsto A_t(\omega)$の増大性より
				\begin{align}
					A_T(\omega) < \delta \quad \Longleftrightarrow \quad
					T(\omega) < S_\delta(\omega)
				\end{align}
				となるから
				\begin{align}
					P\left( V_T \geq \epsilon,\ A_T < \delta \right)
					&= P\left( V_{T \wedge S_\delta} \geq \epsilon,\ A_T < \delta \right)
					\leq P\left( V_{T \wedge S_\delta} \geq \epsilon \right) \\
					&\leq \frac{E(A_{S_\delta \wedge T})}{\epsilon}
					= \frac{E(A_{S_\delta} \wedge A_T)}{\epsilon}
					= \frac{E(\delta \wedge A_T)}{\epsilon}
				\end{align}
				が成立し,両辺に$P\left( V_T \geq \epsilon,\ A_T \geq \delta \right)$を加えて
				Lenglartの不等式を得る.
				
			\item[(3)] $F$は連続かつ非減少であるからLebesgue-Stieltjes積分が構成され,任意の$x \in [0,\infty)$に対し
				\begin{align}
					F(x) = \int_{[0,\infty)} \defunc_{(0,x]}(u)\ dF(u)
				\end{align}
				が満たされる.
				\begin{align}
					(\omega,u) \longmapsto \defunc_{(0,V_T(\omega)]}(u)
				\end{align}
				は,$u$の関数として左連続であり,また$\omega$の関数としては$\mathscr{F}/\borel{\R}$-可測であるから(Problem 1.16),
				二変数関数として$\mathscr{F} \otimes \borel{[0,\infty)}/\borel{\R}$-可測であり,このときFubiniの定理より
				\begin{align}
					E F(V_T) &= \int_{[0,\infty)} E\left( \defunc_{[u,\infty)}(V_T) \right)\ dF(u) \\
					&= \int_{[0,\infty)} P(V_T \geq u)\ dF(u) \\
					&\leq \int_{[0,\infty)} \frac{E(u \wedge A_T)}{u} + P(A_T \geq u)\ dF(u) \\
					&= \int_{[0,\infty)} \frac{E(u \wedge A_T \defunc_{\{A_T \geq u\}})}{u} + 
						\frac{E(u \wedge A_T \defunc_{\{A_T < u\}})}{u} + P(A_T \geq u)\ dF(u) \\
					&= \int_{[0,\infty)} 2 P(A_T \geq u) + \frac{E(A_T \defunc_{\{A_T < u\}})}{u}\ dF(u) \\
					&=  E\left(2F(A_T)\right) 
						+ E\left[A_T \int_{[0,\infty)} \frac{1}{u} \defunc_{(A_T,\infty)}(u)\ dF(u)\right] \\
					&= EG(A_T)
				\end{align}
				が得られる.
				\QED
		\end{description}
	\end{prf}
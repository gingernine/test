\section{The Doob-Meyer Decomposition}
	\begin{itembox}[l]{martingale transform}
		If $A = \Set{A_n,\mathscr{F}_n}{n=0,1,\cdots}$ is predictable with $E|A_n|<\infty$ for every $n$,
		and if $\Set{M_n,\mathscr{F}_n}{n=0,1,\cdots}$ is bounded martingale, then the martingale transform of $A$
		by $M$ defined by
		\begin{align}
			Y_0 = 0 \quad \mbox{and} \quad
			Y_n = \sum_{k=1}^n A_k (M_k - M_{k-1});
			\quad n \geq 1, 
		\end{align}
		is itself a martingale.
	\end{itembox}
	
	\begin{prf}
		$A_k(M_k - M_{k-1})\ (k \leq n)$は$\mathscr{F}_n/\borel{\R}$-可測であるから
		$(Y_n)_{n=1}^\infty$は$(\mathscr{F}_n)$-適合である.また
		\begin{align}
			E|Y_n| = E\left| \sum_{k=1}^n A_k (M_k - M_{k-1}) \right|
			\leq \sum_{k=1}^n \left\{\esssup{\omega \in \Omega}{\left(|M_k(\omega)|+|M_{k-1}(\omega)|\right)}\right\} E|A_k| < \infty
		\end{align}
		が成り立つ.更に任意の$n \geq 1$に対し
		\begin{align}
			\cexp{Y_n}{\mathscr{F}_{n-1}}
			&= \cexp{A_n(M_n - M_{n-1}) + Y_{n-1}}{\mathscr{F}_{n-1}} \\
			&= A_n(\cexp{M_n}{\mathscr{F}_{n-1}} - M_{n-1}) + Y_{n-1}
			= Y_{n-1},
			\quad \mbox{a.s. $P$}
		\end{align}
		が満たされる.
		\QED
	\end{prf}
	
	\begin{itembox}[l]{Doob's decomposition}
		Any submartingale $\Set{X_n,\mathscr{F}_n}{n=0,1,\cdots}$ admits the Doob decomposition
		$X_n = M_n + A_n$ as the summation of a martingale $\{M_n,\mathscr{F}_n\}$ and an 
		predictable and increasing sequence $\{A_n,\mathscr{F}_n\}$.
		This decomposition is unique.
	\end{itembox}
	
	\begin{prf}\mbox{}
		\begin{description}
			\item[第一段]
				Doob分解が存在するとして,分解の一意性を示す.
				実際,分解が存在すれば
				\begin{align}
					A_{n+1} - A_n = \cexp{A_{n+1}-A_n}{\mathscr{F}_{n}}
					= \cexp{X_{n+1}-X_n}{\mathscr{F}_{n}} - \cexp{M_{n+1}-M_n}{\mathscr{F}_{n}}
					= \cexp{X_{n+1}-X_n}{\mathscr{F}_{n}},
					\quad \mbox{a.s. $P$}
				\end{align}
				が成立し,$A_n\ (n \geq 1)$は
				\begin{align}
					A_n = \sum_{k=0}^{n-1} \cexp{X_{k+1}-X_k}{\mathscr{F}_{k}},
					\quad \mbox{a.s. $P$}
				\end{align}
				を満たすことになり分解の一意性が出る.
				
			\item[第二段]
				分解可能性を示す.
				\begin{align}
					A_0 \coloneqq 0,
					\quad A_n \coloneqq \sum_{k=0}^{n-1} \cexp{X_{k+1}-X_k}{\mathscr{F}_{k}},
					\quad (n=1,2,\cdots)
				\end{align}
				と定めれば$(A_n)$は可予測かつ可積分であり,
				\begin{align}
					A_{n+1} - A_n = \cexp{X_{k+1}-X_k}{\mathscr{F}_{k}} \geq 0,
					\quad \mbox{a.s. $P$}
					\ (\forall n \geq 1)
				\end{align}
				より増大過程である.また$M_n \coloneqq X_n - A_n$により$(\mathscr{F}_n)$-適合かつ可積分な過程を定めれば,
				\begin{align}
					\cexp{M_{n+1} - M_n}{\mathscr{F}_n}
					&= \cexp{(X_{n+1} - X_n)-(A_{n+1}-A_n)}{\mathscr{F}_n} \\
					&= \cexp{X_{n+1} - X_n}{\mathscr{F}_n} - \cexp{\cexp{X_{n+1} - X_n}{\mathscr{F}_n}}{\mathscr{F}_n}
					= 0,
					\quad \mbox{a.s. $P$}
				\end{align}
				が成り立つから$\{M_n,\mathscr{F}_n\}$はマルチンゲールである.
				\QED
		\end{description}
	\end{prf}
	
	\begin{itembox}[l]{Proposition 4.3 修正}
		An increasing random sequence $A$ has a predictable modification
		if and only if it is natural.
	\end{itembox}
	
	\begin{prf}
		$A$が可予測な修正$\tilde{A}$を持つとき,任意の有界マルチンゲール$M$に対して
		\begin{align}
			\tilde{Y}_0 \coloneqq 0,
			\quad \tilde{Y}_n \coloneqq \sum_{k=1}^n \tilde{A}_k(M_k - M_{k-1}); \quad n \geq 1
		\end{align}
		は$(\mathscr{F}_n)$-マルチンゲールとなる.
		このとき$M_n \tilde{A}_n$と$\sum_{k=1}^n M_{k-1}(\tilde{A}_k - \tilde{A}_{k-1})$は可積分であり
		\begin{align}
			0 = E \tilde{Y}_n = E\left[ M_n \tilde{A}_n - \sum_{k=1}^n M_{k-1}(\tilde{A}_k - \tilde{A}_{k-1}) \right]
			= E(M_n A_n) - E\sum_{k=1}^n M_{k-1}(A_k - A_{k-1}),
			\quad (\forall n \geq 1)
		\end{align}
		が成り立つから$A$はナチュラルである.逆に$A$がナチュラルであるとき,
		有界マルチンゲール$M$に対して
		\begin{align}
			0 &= E\left[ M_n A_n - \sum_{k=1}^n M_{k-1}(A_k - A_{k-1}) \right] \\
			&= E\left[ A_n(M_n-M_{n-1}) \right] - E\left[ M_{n-1} A_{n-1} - \sum_{k=1}^{n-1} M_{k-1}(A_k - A_{k-1}) \right] \\
			&= E\left[ A_n(M_n-M_{n-1}) \right],
			\quad (\forall n \geq 1)
		\end{align}
		が成り立つ.一方で
		\begin{align}
			E\left[ M_{n-1}(A_n-\cexp{A_n}{\mathscr{F}_{n-1}}) \right]
			&= E\left[ \cexp{M_{n-1} (A_n-\cexp{A_n}{\mathscr{F}_{n-1}})}{\mathscr{F}_{n-1}} \right] \\
			&= E\left[ M_{n-1} \cexp{A_n-\cexp{A_n}{\mathscr{F}_{n-1}}}{\mathscr{F}_{n-1}} \right]
			= 0,
			\quad (\forall n \geq 1)
		\end{align}
		及び
		\begin{align}
			E\left[ \cexp{A_n}{\mathscr{F}_{n-1}}(M_n-M_{n-1}) \right]
			&= E\left[ \cexp{ \cexp{A_n}{\mathscr{F}_{n-1}}(M_n-M_{n-1})}{\mathscr{F}_{n-1}} \right] \\
			&= E\left[ \cexp{A_n}{\mathscr{F}_{n-1}}\cexp{M_n-M_{n-1}}{\mathscr{F}_{n-1}} \right]
			= 0,
			\quad (\forall n \geq 1)
		\end{align}
		となるから
		\begin{align}
			E\left[ M_n(A_n - \cexp{A_n}{\mathscr{F}_{n-1}}) \right]
			&= E\left[ A_n(M_n-M_{n-1}) \right] \\
			&\quad	+ E\left[ M_{n-1}(A_n-\cexp{A_n}{\mathscr{F}_{n-1}}) \right] \\
			&\quad	- E\left[ \cexp{A_n}{\mathscr{F}_{n-1}}(M_n-M_{n-1}) \right] \\
			&= 0,
			\quad (\forall n \geq 1)
		\end{align}
		が従う.ここで各$n \geq 1$に対し,
		$\borel{\R}/\borel{\R}$-可測関数$\operatorname{sgn} = \defunc_{(0,\infty)} - \defunc_{(-\infty,0)}$を用いて
		\begin{align}
			M^{(n)}_k \coloneqq 
			\begin{cases}
				\sgn{A_n - \cexp{A_n}{\mathscr{F}_{n-1}}}, & (k \geq n), \\
				\cexp{\sgn{A_n - \cexp{A_n}{\mathscr{F}_{n-1}}}}{\mathscr{F}_k}, & (0 \leq k < n)
			\end{cases}
		\end{align}
		により有界マルチンゲール$M^{(n)} = \Set{M^{(n)}_k,\mathscr{F}_k}{k=0,1,\cdots}$を定めれば,
		\begin{align}
			0 = E\left[ M^{(n)}_n(A_n - \cexp{A_n}{\mathscr{F}_{n-1}}) \right] 
			= E\left| A_n - \cexp{A_n}{\mathscr{F}_{n-1}} \right|,
			\quad (\forall n \geq 1)
		\end{align}
		が得られ
		\begin{align}
			\tilde{A}_0 \coloneqq 0,
			\quad \tilde{A}_n \coloneqq \cexp{A_n}{\mathscr{F}_{n-1}}; \quad n \geq 1
		\end{align}
		は$A$の可予測な修正となる.
		\QED
	\end{prf}
	
	\begin{itembox}[l]{Definition 4.5 について}
		$A$に対し或る$P$-零集合$N \in \mathscr{F}$が存在して,$\omega \in \Omega \backslash N$なら
		$t \longmapsto A_t(\omega)$は右連続非減少であるからLebesgue-Stieltjes積分を定義できる.このとき
		\begin{align}
			\left( \int_{(0,t]} M_s\ dA_s \right)(\omega)
			= \begin{cases}
				\displaystyle\int_{(0,t]} M_s(\omega)\ dA_s(\omega), & (\omega \in \Omega \backslash N), \\
				0, & (\omega \in N)
			\end{cases}
		\end{align}
		と理解する.
	\end{itembox}
	
	\begin{itembox}[l]{RCLL関数の不連続点は高々可算個}
		$(S,d)$を距離空間とする.写像$X:[0,\infty) \longrightarrow S$について
		各点$t \in [0,\infty)$で右連続かつ各点$t \in (0,\infty)$で左極限が存在するとき,
		$X$の不連続点は存在しても高々可算個である.
	\end{itembox}
	
	\begin{prf}
		各点$t > 0$における$f$の左極限を$f(t-)$と書けば
		\begin{align}
			\mbox{$f$が$t \in (0,\infty)$で不連続}
			\quad \Leftrightarrow \quad
			\mbox{$d(f(t),f(t-)) > 0$}
		\end{align}
		が成立するから,任意に$T > 0$を選び固定して
		\begin{align}
			D(n) \coloneqq \Set{t \in (0,T]}{\frac{1}{n+1} \leq d(f(t),f(t-)) < \frac{1}{n}},
			\quad E(n) \coloneqq \Set{t \in (0,T]}{n \leq d(f(t),f(t-)) < n+1}
		\end{align}
		とおけば
		\begin{align}
			D_T \coloneqq \Set{t \in (0,T]}{\mbox{$f$が$t \in (0,\infty)$で不連続}}
			= \bigcup_{n=1}^\infty D(n) \cup E(n)
		\end{align}
		となる.このとき$D(n),E(n)$は全て有限集合である.実際,或る$n$に対し$D(n)$が無限集合なら
		\begin{align}
			\left\{ t_k \right\}_{k=1}^\infty \subset D(n),
			\quad t_k \neq t_j\ (k \neq j)
		\end{align}
		を満たす可算集合が存在し,$[0,T]$のコンパクト性より
		或る部分列$\left( t_{k_m} \right)_{m=1}^\infty$は
		或る$y \in [0,T]$に収束する.
		$y=0$の場合,右連続の仮定より$1/2(n+1) > \epsilon > 0$に対し或る$\delta > 0$が存在して
		\begin{align}
			d(f(0),f(t)) < \epsilon, \quad (\forall 0 < t < \delta)
		\end{align}
		が成り立つが,一方で$0 < t_{k_m} < \delta$を満たす$t_{k_m}$が存在して
		\begin{align}
			\frac{1}{n+1} - \epsilon < d(f(t_{k_m}),f(t_{k_m}-)) - d(f(0),f(t_{k_m}-))
			\leq d(f(0),f(t_{k_m})) < \epsilon 
		\end{align}
		となり矛盾が生じる.
		$y > 0$の場合も,$1/2(n+1) > \epsilon > 0$に対し或る$\delta > 0$が存在して
		\begin{align}
			d(f(y-),f(t)) < \epsilon, \quad (\forall t \in (y-\delta,y))
		\end{align}
		となるが,$f$が$y$で右連続であるから(或は$y=T$のとき) $y-\delta < t_{k_m} \leq y$を満たす$t_{k_m}$が存在して
		\begin{align}
			\frac{1}{n+1} - \epsilon < 
			d(f(t_{k_m}-),f(t_{k_m})) - d(f(t_{k_m}-),f(y-)) \leq d(f(y-),f(t_{k_m})) < \epsilon
		\end{align}
		が従い矛盾が生じる.よって任意の$n \geq 1$に対して$D(n)$は有限集合であり,同様に
		$E(n)$も有限集合であるから$D_T$は高々可算集合である.
		$f$の不連続点の全体は$\bigcup_{T=1}^\infty D_T$に一致するから高々可算個である.
		\QED
	\end{prf}
	
	\begin{itembox}[l]{Remarks 4.6 (i) 修正}
		If $A$ is an increasing and $X$ a measurable process, then with $\omega \in \Omega$ fixed,
		the sample path $\Set{X_t(\omega)}{0 \leq t < \infty}$ is a measurable function from $[0,\infty)$
		into $\R$. It follows that the Lebesgue-Stieltjes integrals
		\begin{align}
			I^{\pm}_t(\omega) \coloneqq
			\int_{(0,t]} X^\pm_s(\omega)\ dA_s(\omega)
		\end{align}
		are well defined. If $\mathscr{F}_0$ contains all the $P$-negligible events in $\mathscr{F}$ 
		and $X$ is progressively measurable (e.g., right-continuous and adapted), and
		if $I_t = I^+_t - I^-_t$ is well defined and finite for all $t > 0$, then $I$ is right-continuous,
		progressively measurable.
	\end{itembox}
	
	\begin{prf}
		$X$が$\borel{[0,\infty)} \otimes \mathscr{F}/\borel{\R}$-可測なら,
		補題\ref{lem:Fubini_lemma_1} (P. \pageref{lem:Fubini_lemma_1})より
		$[0,\infty) \ni t \longmapsto X_t(\omega)$は
		$\borel{[0,\infty)}/\borel{\R}$-可測である.
		また,或る$P$-零集合$N \in \mathscr{F}$が存在して$\omega \in \Omega \backslash N$に対し
		$t \longmapsto A_t(\omega)$は右連続非減少であり,このとき
		\begin{align}
			\mu_\omega((a,b]) = A_b(\omega) - A_a(\omega),
			\quad (\forall (a,b] \subset [0,\infty))
		\end{align}
		を満たす$\left([0,\infty),\borel{[0,\infty)}\right)$上の$\sigma$-有限測度が唯一つ存在して
		\begin{align}
			\int_{(0,t]} X^\pm_s(\omega)\ dA_s(\omega)
			\coloneqq \int_{(0,t]} X^\pm_s(\omega)\ \mu_\omega(ds),
			\quad (0 < t < \infty)
		\end{align}
		が定義される.ここで$I_t = I^+_t - I^-_t$を
		\begin{align}
			I^\pm_t(\omega) \coloneqq
			\begin{cases}
				\displaystyle\int_{(0,t]} X_s(\omega)\ \mu_\omega(ds), & (\omega \in \Omega \backslash N), \\
				0, & (\omega \in N)
			\end{cases}
		\end{align}
		により定める.$I_t\ (\forall t > 0)$が有限確定し$X$が$(\mathscr{F}_t)$-発展的可測であるとき,
		$t>0$と$\omega \in \Omega \backslash N$に対して
		\begin{align}
			\mu_{t,\omega} \coloneqq \left.\mu_\omega\right|_{\borel{[0,t]}}
		\end{align}
		とすれば$\left([0,t],\borel{[0,t]},\mu_{t,\omega}\right)$は測度空間となり
		\begin{align}
			I^{\pm}_t(\omega) =
			\int_{(0,t]} \left.X^\pm\right|_{[0,t] \times \Omega}(s,\omega)\ \mu_{t,\omega}(ds),
			\quad (\forall \omega \in \Omega \backslash N)
		\end{align}
		が成り立つから,
		\begin{align}
			\mathscr{F}^N_t \coloneqq \Set{E \cap (\Omega \backslash N)}{E \in \mathscr{F}_t}
		\end{align}
		とおけばFubiniの定理より$\left.I^\pm_t\right|_{\Omega \backslash N}$は
		$\mathscr{F}^N_t/\borel{\R}$-可測であり,
		$N \in \mathscr{F}_t$より$I^\pm_t$は$\mathscr{F}_t/\borel{\R}$-可測である.
		また$t<T$及び$\{t_n\}_{n=1}^\infty \subset (t,T],\ t_n \downarrow t$に対して,Lebesgueの収束定理より
		\begin{align}
			\lim_{n \to \infty} I^\pm_{t_n}(\omega)
			= \lim_{n \to \infty} \int_{(0,T]} \defunc_{(0,t_n]}(s)X^\pm_s(\omega)\ \mu_\omega(ds)
			= \int_{(0,T]} \defunc_{(0,t]}(s)X^\pm_s(\omega)\ \mu_\omega(ds)
			= I^\pm_t(\omega)
		\end{align}
		が成立し$t \longmapsto I_t(\omega)$の右連続性が出る.$I$は右連続$(\mathscr{F}_t)$-適合過程であるから
		$(\mathscr{F}_t)$-発展的可測である.
		\QED
	\end{prf}
	
	\begin{itembox}[l]{Remark 4.6 (ii)}
		Every continuous, increasing process is natural. Indeed then, for $P$-a.e. $\omega \in \Omega$
		we have
		\begin{align}
			\int_{(0,t]} (M_s(\omega)-M_{s-}(\omega))\ dA_s(\omega) = 0
			\quad \mbox{for every $0 < t < \infty$},
		\end{align}
		because every path $\Set{M_s(\omega)}{0 \leq s < \infty}$ has only countably many discontinuities
		(Theorem 3.8(v)).
	\end{itembox}
	
	\begin{prf}
		$P$-a.e. $\omega \in \Omega$に対しLebesgue-Stieltjes積分が定義され,さらに
		$\Set{M_s(\omega)}{0 \leq s < \infty}$はRCLLである.
		RCLLパスの不連続点は高々可算個であり,
		連続な$A$で作る測度に対し一点集合は零集合となる.
		\QED
	\end{prf}
	
	\begin{itembox}[l]{Problem 4.9 修正}
		Suppose $\mathscr{F}_0$ contains all the $P$-negligible events in $\mathscr{F}$ and
		$X = \Set{X_t,\mathscr{F}_t}{0 \leq t < \infty}$ is a right-continuous submartingale,
		and set 
		\begin{align}
			X_U(\omega) \coloneqq 0, \quad (\forall \omega \in \{U=\infty\})
		\end{align}
		for any random time $U$. Show that under any one of the following conditions, $X$ is of class $DL$.
		\begin{description}
			\item[(a)] $X_t \geq 0$ a.s. for every $t \geq 0$.
			\item[(b)] $X$ has the special form
				\begin{align}
					X_t = M_t + A_t, \quad 0 \leq t < \infty
				\end{align}
				suggested by the Doob decomposition, where $\Set{M_t,\mathscr{F}_t}{0 \leq t < \infty}$
				is a martingale and $\Set{A_t,\mathscr{F}_t}{0 \leq t < \infty}$ is an increasing process.
		\end{description}
		Show also that if $X$ is a uniformly integrable martingale, then it is of class $D$.
	\end{itembox}
	
	\begin{prf}\mbox{}
		\begin{description}
			\item[(a)]
				任意の$T \in \mathscr{S}_a\ (a > 0)$に対して
				$X_T$は$\mathscr{F}_T/\borel{\R}$-可測であるから,任意抽出定理より
				\begin{align}
					\int_{\{X_T > \lambda\}} X_T\ dP
					\leq \int_{\{X_T > \lambda\}} X_a\ dP,
					\quad (\forall \lambda > 0)
				\end{align}
				及び
				\begin{align}
					P\left( X_T > \lambda \right)
					\leq \frac{EX_T}{\lambda}
					\leq \frac{EX_a}{\lambda},
					\quad (\forall \lambda > 0)
				\end{align}
				が成立する.$X_a$が可積分であるから
				\begin{align}
					\sup{T \in \mathscr{S}_a}{\int_{\{X_T > \lambda\}} X_T\ dP}
					\longrightarrow 0
					\quad (\lambda \longrightarrow \infty)
				\end{align}
				となり,$(X_T)_{T \in \mathscr{S}_a}$の一様可積分性が得られる.
				
			\item[(b)]
				$a > 0$とすれば,任意抽出定理より
				\begin{align}
					M_T = \cexp{M_a}{\mathscr{F}_T},\ \mbox{a.s. $P$,}
					\quad (\forall T \in \mathscr{S}_a)
				\end{align}
				が成り立つから,一様可積分性と条件付き期待値の補題
				(P. \pageref{lem:uniformly_integrability_and_conditional_expectations})より
				$(M_T)_{T \in \mathscr{S}_a}$は一様可積分である.このとき
				\begin{align}
					\int_{\{|X_T| > \lambda\}} |X_T|\ dP
					&\leq 2\int_{\{|M_T| > \lambda/2\}} |M_T|\ dP + 2\int_{\{|A_T| > \lambda/2\}} |A_T|\ dP \\
					&\leq 2\sup{T \in \mathscr{S}_a}{\int_{\{|M_T| > \lambda/2\}} |M_T|\ dP} + 2\int_{\{A_a > \lambda/2\}} A_a\ dP \\
					&\longrightarrow 0 \quad (\lambda \longrightarrow \infty)
				\end{align}
				が従い$(X_T)_{T \in \mathscr{S}_a}$の一様可積分性が出る.
		\end{description}
		$X$が一様可積分なマルチンゲールであるとき,Problem 3.20より
		\begin{align}
			X_t = \cexp{X_\infty}{\mathscr{F}_t},\ \mbox{a.s. $P$},
			\quad (\forall t \geq 0)
		\end{align}
		を満たす$\mathscr{F}_\infty/\borel{\R}$-可測可積分関数$X_\infty$が存在し,任意抽出定理より
		\begin{align}
			X_T = \cexp{X_\infty}{\mathscr{F}_T},\ \mbox{a.s. $P$},
			\quad (\forall T \in \mathscr{S})
		\end{align}
		が成り立つから$X$はクラス$DL$に属する.
		\QED
	\end{prf}
	
	\begin{itembox}[l]{Problem 4.11 修正}
		Let $(X,\mathscr{F},\mu)$ be a $\sigma$-finite measurable space and  
		$\left\{f_n\right\}_{n=1}^\infty$ be a sequence of integrable complex functions on $(X,\mathscr{F},\mu)$
		which converges weakly in $L^1$ to an integrable complex function $f$.
		Then for each $\sigma$-field $\mathscr{G} \subset \mathscr{F}$
		where $(X,\mathscr{G},\left.\mu\right|_{\mathscr{G}})$ is $\sigma$-finite,
		the sequence $\cexp{f_n}{\mathscr{G}}$ converges to $\cexp{f}{\mathscr{G}}$ weakly in $L^1$.
	\end{itembox}
	
	\begin{prf}
		任意の$g \in \mathscr{L}^\infty(X,\mathscr{F},\mu)$に対して
		\begin{align}
			\int_X g\cexp{f}{\mathscr{G}}\ dP
			&= \int_X \cexp{g\cexp{f}{\mathscr{G}}}{\mathscr{G}}\ dP \\
			&= \int_X \cexp{g}{\mathscr{G}}\cexp{f}{\mathscr{G}}\ dP \\
			&= \int_X \cexp{\cexp{g}{\mathscr{G}}f}{\mathscr{G}}\ dP \\
			&= \int_X \cexp{g}{\mathscr{G}}f\ dP
		\end{align}
		が成り立つから
		\begin{align}
			\lim_{n \to \infty} \int_X g\cexp{f_n}{\mathscr{G}}\ dP
			= \lim_{n \to \infty} \int_X \cexp{g}{\mathscr{G}}f_n\ dP
			= \int_X \cexp{g}{\mathscr{G}}f\ dP
			= \int_X g\cexp{f}{\mathscr{G}}\ dP
		\end{align}
		となる.
		\QED
	\end{prf}
	
	\begin{itembox}[l]{Theorem 4.10 (Doob-Meyer Decomposition)}
		Let $\{\mathscr{F}_t\}$ satisfy the usual conditions. If the right-continuous
		submartingale $X = \Set{X_t,\mathscr{F}_t}{0 \leq t < \infty}$ is of class $DL$, then it admits
		the decomposition (4.6) as the summation of a right-continuous martingale $M = \Set{M_t,\mathscr{F}_t}{0 \leq t < \infty}$
		and an increasing process $A = \Set{A_t,\mathscr{F}_t}{0 \leq t < \infty}$. The latter can be taken to be natural;
		under this additional condition, the decomposition (4.6) is unique (up to indistinguishability).
		Further, if $X$ is of class $D$, then $M$ is a uniformly integrable martingale and $A$ is integrable.	
	\end{itembox}
	
	\begin{prf}\mbox{}
		\begin{description}
			\item[第一段]
				分解の一意性を示す.
		\end{description}
	\end{prf}
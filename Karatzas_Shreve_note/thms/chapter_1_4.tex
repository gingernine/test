\section{The Doob-Meyer Decomposition}
	\begin{itembox}[l]{martingale transform}
		If $A = \Set{A_n,\mathscr{F}_n}{n=0,1,\cdots}$ is predictable with $E|A_n|<\infty$ for every $n$,
		and if $\Set{M_n,\mathscr{F}_n}{n=0,1,\cdots}$ is bounded martingale, then the martingale transform of $A$
		by $M$ defined by
		\begin{align}
			Y_0 = 0 \quad \mbox{and} \quad
			Y_n = \sum_{k=1}^n A_k (M_k - M_{k-1});
			\quad n \geq 1, 
		\end{align}
		is itself a martingale.
	\end{itembox}
	
	\begin{prf}
		$A_k(M_k - M_{k-1})\ (k \leq n)$は$\mathscr{F}_n/\borel{\R}$-可測であるから
		$(Y_n)_{n=1}^\infty$は$(\mathscr{F}_n)$-適合である.また
		\begin{align}
			E|Y_n| = E\left| \sum_{k=1}^n A_k (M_k - M_{k-1}) \right|
			\leq \sum_{k=1}^n \left\{\esssup{\omega \in \Omega}{\left(|M_k(\omega)|+|M_{k-1}(\omega)|\right)}\right\} E|A_k| < \infty
		\end{align}
		が成り立つ.更に任意の$n \geq 0$に対し
		\begin{align}
			\cexp{Y_{n+1} - Y_n}{\mathscr{F}_n}
			= \cexp{A_{n+1}(M_{n+1} - M_n)}{\mathscr{F}_n}
			= A_{n+1}\cexp{M_{n+1} - M_n}{\mathscr{F}_n}
			= 0,
			\quad \mbox{a.s. $P$}
		\end{align}
		が満たされる.
		\QED
	\end{prf}
	
	\begin{itembox}[l]{Doob's decomposition}\label{lem:Doob_decomposition}
		Any submartingale $\Set{X_n,\mathscr{F}_n}{n=0,1,\cdots}$ admits the unique decomposition
		$X_n = M_n + A_n$ as the summation of a martingale $\{M_n,\mathscr{F}_n\}$ and an 
		predictable and increasing sequence $\{A_n,\mathscr{F}_n\}$, where
		\begin{align}
			A_n = \sum_{k=0}^{n-1}\cexp{X_{k+1}-X_k}{\mathscr{F}_k},
			\quad \mbox{a.s. $P$},\ n \geq 1.
		\end{align}
	\end{itembox}
	
	\begin{prf}\mbox{}
		\begin{description}
			\item[第一段]
				Doob分解が存在するとして,分解の一意性を示す.
				実際,分解が存在すれば
				\begin{align}
					A_{n+1} - A_n = \cexp{A_{n+1}-A_n}{\mathscr{F}_{n}}
					= \cexp{X_{n+1}-X_n}{\mathscr{F}_{n}} - \cexp{M_{n+1}-M_n}{\mathscr{F}_{n}}
					= \cexp{X_{n+1}-X_n}{\mathscr{F}_{n}},
					\quad \mbox{a.s. $P$}
				\end{align}
				が成立し,$A_n\ (n \geq 1)$は
				\begin{align}
					A_n = \sum_{k=0}^{n-1} \cexp{X_{k+1}-X_k}{\mathscr{F}_{k}},
					\quad \mbox{a.s. $P$}
				\end{align}
				を満たすことになり分解の一意性が出る.
				
			\item[第二段]
				分解可能性を示す.
				\begin{align}
					A_0 \coloneqq 0,
					\quad A_n \coloneqq \sum_{k=0}^{n-1} \cexp{X_{k+1}-X_k}{\mathscr{F}_{k}},
					\quad (n=1,2,\cdots)
				\end{align}
				と定めれば$(A_n)$は可予測かつ可積分であり,
				\begin{align}
					A_{n+1} - A_n = \cexp{X_{k+1}-X_k}{\mathscr{F}_{k}} \geq 0,
					\quad \mbox{a.s. $P$}
					\ (\forall n \geq 1)
				\end{align}
				より増大過程である.また$M_n \coloneqq X_n - A_n$により$(\mathscr{F}_n)$-適合かつ可積分な過程を定めれば,
				\begin{align}
					\cexp{M_{n+1} - M_n}{\mathscr{F}_n}
					&= \cexp{(X_{n+1} - X_n)-(A_{n+1}-A_n)}{\mathscr{F}_n} \\
					&= \cexp{X_{n+1} - X_n}{\mathscr{F}_n} - \cexp{\cexp{X_{n+1} - X_n}{\mathscr{F}_n}}{\mathscr{F}_n}
					= 0,
					\quad \mbox{a.s. $P$}
				\end{align}
				が成り立つから$\{M_n,\mathscr{F}_n\}$はマルチンゲールである.
				\QED
		\end{description}
	\end{prf}
	
	\begin{itembox}[l]{Proposition 4.3 修正}
		An increasing random sequence $A$ has a predictable modification
		if and only if it is natural.
	\end{itembox}
	
	\begin{prf}
		$A$が可予測な修正$\tilde{A}$を持つとき,任意の有界マルチンゲール$M$に対して
		\begin{align}
			\tilde{Y}_0 \coloneqq 0,
			\quad \tilde{Y}_n \coloneqq \sum_{k=1}^n \tilde{A}_k(M_k - M_{k-1}); \quad n \geq 1
		\end{align}
		は$(\mathscr{F}_n)$-マルチンゲールとなる.
		このとき$M_n \tilde{A}_n$と$\sum_{k=1}^n M_{k-1}(\tilde{A}_k - \tilde{A}_{k-1})$は可積分であり
		\begin{align}
			0 = E \tilde{Y}_n = E\left[ M_n \tilde{A}_n - \sum_{k=1}^n M_{k-1}(\tilde{A}_k - \tilde{A}_{k-1}) \right]
			= E(M_n A_n) - E\sum_{k=1}^n M_{k-1}(A_k - A_{k-1}),
			\quad (\forall n \geq 1)
		\end{align}
		が成り立つから$A$はナチュラルである.逆に$A$がナチュラルであるとき,
		有界マルチンゲール$M$に対して
		\begin{align}
			0 &= E\left[ M_n A_n - \sum_{k=1}^n M_{k-1}(A_k - A_{k-1}) \right] \\
			&= E\left[ A_n(M_n-M_{n-1}) \right] - E\left[ M_{n-1} A_{n-1} - \sum_{k=1}^{n-1} M_{k-1}(A_k - A_{k-1}) \right] \\
			&= E\left[ A_n(M_n-M_{n-1}) \right],
			\quad (\forall n \geq 1)
		\end{align}
		が成り立つ.一方で
		\begin{align}
			E\left[ M_{n-1}(A_n-\cexp{A_n}{\mathscr{F}_{n-1}}) \right]
			&= E\left[ \cexp{M_{n-1} (A_n-\cexp{A_n}{\mathscr{F}_{n-1}})}{\mathscr{F}_{n-1}} \right] \\
			&= E\left[ M_{n-1} \cexp{A_n-\cexp{A_n}{\mathscr{F}_{n-1}}}{\mathscr{F}_{n-1}} \right]
			= 0,
			\quad (\forall n \geq 1)
		\end{align}
		及び
		\begin{align}
			E\left[ \cexp{A_n}{\mathscr{F}_{n-1}}(M_n-M_{n-1}) \right]
			&= E\left[ \cexp{ \cexp{A_n}{\mathscr{F}_{n-1}}(M_n-M_{n-1})}{\mathscr{F}_{n-1}} \right] \\
			&= E\left[ \cexp{A_n}{\mathscr{F}_{n-1}}\cexp{M_n-M_{n-1}}{\mathscr{F}_{n-1}} \right]
			= 0,
			\quad (\forall n \geq 1)
		\end{align}
		となるから
		\begin{align}
			E\left[ M_n(A_n - \cexp{A_n}{\mathscr{F}_{n-1}}) \right]
			&= E\left[ A_n(M_n-M_{n-1}) \right] \\
			&\quad	+ E\left[ M_{n-1}(A_n-\cexp{A_n}{\mathscr{F}_{n-1}}) \right] \\
			&\quad	- E\left[ \cexp{A_n}{\mathscr{F}_{n-1}}(M_n-M_{n-1}) \right] \\
			&= 0,
			\quad (\forall n \geq 1)
		\end{align}
		が従う.ここで各$n \geq 1$に対し,
		$\borel{\R}/\borel{\R}$-可測関数$\operatorname{sgn} = \defunc_{(0,\infty)} - \defunc_{(-\infty,0)}$を用いて
		\begin{align}
			M^{(n)}_k \coloneqq 
			\begin{cases}
				\sgn{A_n - \cexp{A_n}{\mathscr{F}_{n-1}}}, & (k \geq n), \\
				\cexp{\sgn{A_n - \cexp{A_n}{\mathscr{F}_{n-1}}}}{\mathscr{F}_k}, & (0 \leq k < n)
			\end{cases}
		\end{align}
		により有界マルチンゲール$M^{(n)} = \Set{M^{(n)}_k,\mathscr{F}_k}{k=0,1,\cdots}$を定めれば,
		\begin{align}
			0 = E\left[ M^{(n)}_n(A_n - \cexp{A_n}{\mathscr{F}_{n-1}}) \right] 
			= E\left| A_n - \cexp{A_n}{\mathscr{F}_{n-1}} \right|,
			\quad (\forall n \geq 1)
		\end{align}
		が得られ
		\begin{align}
			\tilde{A}_0 \coloneqq 0,
			\quad \tilde{A}_n \coloneqq \cexp{A_n}{\mathscr{F}_{n-1}}; \quad n \geq 1
		\end{align}
		は$A$の可予測な修正となる.
		\QED
	\end{prf}
	
	\begin{itembox}[l]{区別不能性によるパスの同値類}
		$I \subset [0,\infty)$を区間とし,$I$上で右連続な確率過程
		($s=\sup{}{I} \in I$の場合は$s$での右連続性は考えない)
		の全体を$RCSP(I)$と書けば,$M = \Set{M_t}{t \in I},N = \Set{N_t}{t \in I} \in RCSP(I)$に対し
		\begin{align}
			M \sim N \quad \overset{\mathrm{def}}{\Longleftrightarrow} \quad 
			P(M_t = N_t,\ \forall t \in I) = 1
			\label{eq:equivalence_with_respect_to_path}
		\end{align}
		により同値関係$\sim$が定まる.ここで
		右連続性より
		\begin{align}
			\{M_t = N_t,\ \forall t \in I\} = 
			\begin{cases}
				\displaystyle \bigcap_{r \in (I \cap \Q) \cup \{\sup{}{I}\}}\{M_r = N_r\}, & (\sup{}{I} \in I), \\
				\displaystyle \bigcap_{r \in I \cap \Q}\{M_r = N_r\}, & (\sup{}{I} \notin I)
			\end{cases}
		\end{align}
		となるから$\{M_t = N_t,\ \forall t \in I\}$は可測である.
		$\sim$による$M \in RCSP(I)$の同値類を$[M]_{RCSP}$と書く.
	\end{itembox}
	
	\begin{itembox}[l]{Definition 4.4 修正}
		An adapted process $A$ is called increasing if \textcolor{red}{for all $\omega \in \Omega$} we have
		\begin{description}
			\item[(a)] $A_0(\omega) = 0$
			\item[(b)] $t \longmapsto A_t(\omega)$ is nondecreasing, right-continuous function,
		\end{description}
		and $E(A_t) < \infty$ holds for every $t \in [0,\infty)$.
		An increasing process is called integrable if $E(A_\infty) < \infty$,
		where $A_\infty = \lim_{t \to \infty} A_t$.
		\textcolor{red}{Let us denote the subspace of $RCSP[0,\infty)$ as
		\begin{align}
			Inc[0,\infty) \coloneqq 
			\Set{A \in RCSP[0,\infty)}{increasing},
		\end{align}
		and the equivalent class of $A \in Inc[0,\infty)$ as $[A]_{Inc}$.
		By definition, $[A]_{Inc} \subset [A]_{RCSP}$ for every $A \in Inc[0,\infty)$.}
	\end{itembox}
	
	\begin{itembox}[l]{Definition 4.5 修正}
		An increasing processs $A$ is called natural if for every bounded, 
		\textcolor{red}{$RCLL$} martingale $\Set{M_t,\mathscr{F}_t}{0 \leq t < \infty}$ we have
		\begin{align}
			E \int_{(0,t]} M_s\ dA_s = E \int_{(0,t]} M_{s-}\ dA_s,
			\quad \mbox{for every $0 < t < \infty$}.
		\end{align}
		プロセスが$RCLL$とは全てのパスが$RCLL$であるということである.Theorem 3.8によれば
		右連続な劣マルチンゲールはa.e.のパスが$RCLL$であるから,
		(\refeq{eq:equivalence_with_respect_to_path})の意味で同値である.
		$A$も全てのパスが右連続かつ単調非減少であるから,
		全ての$\omega \in \Omega$に対し$\int_{(0,t]} M_s(\omega)\ dA_s(\omega)$と
		$\int_{(0,t]} M_{s-}(\omega)\ dA_s(\omega)$が定義される.また
		\begin{align}
			Nat[0,\infty) \coloneqq
			\Set{A \in RCSP[0,\infty)}{natural},
		\end{align}
		と定め,$A \in Nat[0,\infty)$の同値類を$[A]_{Nat}$と書けば,
		$[A]_{Nat} \subset [A]_{Inc} \subset [A]_{RCSP}$となる.
	\end{itembox}
	
	\begin{itembox}[l]{$RCLL$なパスの不連続点は高々可算個}
		$(S,d)$を距離空間とする.写像$X:[0,\infty) \longrightarrow S$について
		各点$t \in [0,\infty)$で右連続かつ各点$t \in (0,\infty)$で左極限が存在するとき,
		$X$の不連続点は存在しても高々可算個である.
	\end{itembox}
	
	\begin{prf}
		各点$t > 0$における$f$の左極限を$f(t-)$と書けば
		\begin{align}
			\mbox{$f$が$t \in (0,\infty)$で不連続}
			\quad \Leftrightarrow \quad
			\mbox{$d(f(t),f(t-)) > 0$}
		\end{align}
		が成立するから,任意に$T > 0$を選び固定して
		\begin{align}
			D(n) \coloneqq \Set{t \in (0,T]}{\frac{1}{n+1} \leq d(f(t),f(t-)) < \frac{1}{n}},
			\quad E(n) \coloneqq \Set{t \in (0,T]}{n \leq d(f(t),f(t-)) < n+1}
		\end{align}
		とおけば
		\begin{align}
			D_T \coloneqq \Set{t \in (0,T]}{\mbox{$f$が$t \in (0,\infty)$で不連続}}
			= \bigcup_{n=1}^\infty D(n) \cup E(n)
		\end{align}
		となる.このとき$D(n),E(n)$は全て有限集合である.実際,或る$n$に対し$D(n)$が無限集合なら
		\begin{align}
			\left\{ t_k \right\}_{k=1}^\infty \subset D(n),
			\quad t_k \neq t_j\ (k \neq j)
		\end{align}
		を満たす可算集合が存在し,$[0,T]$のコンパクト性より
		或る部分列$\left( t_{k_m} \right)_{m=1}^\infty$は
		或る$y \in [0,T]$に収束する.
		$y=0$の場合,右連続の仮定より$1/2(n+1) > \epsilon > 0$に対し或る$\delta > 0$が存在して
		\begin{align}
			d(f(0),f(t)) < \epsilon, \quad (\forall 0 < t < \delta)
		\end{align}
		が成り立つが,一方で$0 < t_{k_m} < \delta$を満たす$t_{k_m}$が存在して
		\begin{align}
			\frac{1}{n+1} - \epsilon < d(f(t_{k_m}),f(t_{k_m}-)) - d(f(0),f(t_{k_m}-))
			\leq d(f(0),f(t_{k_m})) < \epsilon 
		\end{align}
		となり矛盾が生じる.
		$y > 0$の場合も,$1/2(n+1) > \epsilon > 0$に対し或る$\delta > 0$が存在して
		\begin{align}
			d(f(y-),f(t)) < \epsilon, \quad (\forall t \in (y-\delta,y))
		\end{align}
		となるが,$f$が$y$で右連続であるから(或は$y=T$のとき) $y-\delta < t_{k_m} \leq y$を満たす$t_{k_m}$が存在して
		\begin{align}
			\frac{1}{n+1} - \epsilon < 
			d(f(t_{k_m}-),f(t_{k_m})) - d(f(t_{k_m}-),f(y-)) \leq d(f(y-),f(t_{k_m})) < \epsilon
		\end{align}
		が従い矛盾が生じる.よって任意の$n \geq 1$に対して$D(n)$は有限集合であり,同様に
		$E(n)$も有限集合であるから$D_T$は高々可算集合である.
		$f$の不連続点の全体は$\bigcup_{T=1}^\infty D_T$に一致するから高々可算個である.
		\QED
	\end{prf}
	
	\begin{itembox}[l]{Remarks 4.6 (i) 修正}
		If $A$ is an increasing and $X$ a measurable process, then with $\omega \in \Omega$ fixed,
		the sample path $\Set{X_t(\omega)}{0 \leq t < \infty}$ is a measurable function from $[0,\infty)$
		into $\R$. It follows that the Lebesgue-Stieltjes integrals
		\begin{align}
			I^{\pm}_t(\omega) \coloneqq
			\int_{(0,t]} X^\pm_s(\omega)\ dA_s(\omega)
		\end{align}
		are well defined. \textcolor{red}{If $X$ is bounded, right-continuous and adapted, 
		then $I$ is right-continuous, progressively measurable.}
	\end{itembox}
	
	\begin{prf}
		$X$が$\borel{[0,\infty)} \otimes \mathscr{F}/\borel{\R}$-可測なら,
		補題\ref{lem:Fubini_lemma_1} (P. \pageref{lem:Fubini_lemma_1})より
		$[0,\infty) \ni t \longmapsto X_t(\omega)$は
		$\borel{[0,\infty)}/\borel{\R}$-可測である.
		また全ての$\omega \in \Omega$に対し$t \longmapsto A_t(\omega)$は右連続非減少であるから
		\begin{align}
			\mu_\omega((a,b]) = A_b(\omega) - A_a(\omega),
			\quad (\forall (a,b] \subset [0,\infty)),
			\quad \mu_\omega(\{0\}) = 0
		\end{align}
		を満たす$\left([0,\infty),\borel{[0,\infty)}\right)$上の$\sigma$-有限測度が唯一つ存在して
		\begin{align}
			I^\pm_t(\omega) = \int_{(0,t]} X^\pm_s(\omega)\ dA_s(\omega)
			\coloneqq \int_{(0,t]} X^\pm_s(\omega)\ \mu_\omega(ds),
			\quad (0 < t < \infty)
		\end{align}
		及び$I_t \coloneqq I^+_t - I^-_t$が定義される.
		$X$が有界かつ右連続$(\mathscr{F}_t)$-適合であるとき,
		$t>0$を固定し$t^{(n)}_j \coloneqq tj/2^n$とおいて
		\begin{align}
			X^{(n)\pm}_s \coloneqq X_0 \defunc_{\{0\}}(s) + 
				\sum_{j=0}^{2^n-1} X_{t^{(n)}_{j+1}} 
				\defunc_{\left(t^{(n)}_j,t^{(n)}_{j+1}\right]}(s)
		\end{align}
		とおけば$X^{(n)\pm}_s \longrightarrow X^\pm_s,\ (\forall s \in [0,t])$が成立し,かつ
		\begin{align}
			I^{(n)\pm}_t \coloneqq \int_{(0,t]} X^{(n)\pm}_s\ dA_s
			= \sum_{j=0}^{2^n-1} X_{t^{(n)}_{j+1}} \left(A_{t^{(n)}_j} - A_{t^{(n)}_{j+1}}\right)
		\end{align}
		となり$I^{(n)\pm}_t$の$\mathscr{F}_t/\borel{\R}$-可測性が得られる.
		$X$が有界であるからLebesgueの収束定理より
		\begin{align}
			I^{\pm}_t = \lim_{n \to \infty} \int_{(0,t]} X^{(n)\pm}_s\ dA_s
			= \lim_{n \to \infty} I^{(n)\pm}_t
		\end{align}
		が成り立ち,定理\ref{lem:measurability_metric_space}より
		$I^{\pm}_t$の$\mathscr{F}_t/\borel{\R}$-可測性が従う.
		また$t<T$及び$\{t_n\}_{n=1}^\infty \subset (t,T],\ t_n \downarrow t$に対して,Lebesgueの収束定理より
		\begin{align}
			\lim_{n \to \infty} I^\pm_{t_n}
			= \lim_{n \to \infty} \int_{(0,T]} \defunc_{(0,t_n]}(s)X^\pm_s\ dA_s
			= \int_{(0,T]} \defunc_{(0,t]}(s)X^\pm_s\ dA_s
			= I^\pm_t
		\end{align}
		が成立し$t \longmapsto I_t(\omega)$の右連続性が出る.$I$は右連続$(\mathscr{F}_t)$-適合過程であるから
		$(\mathscr{F}_t)$-発展的可測である.
		\QED
	\end{prf}
	
	\begin{itembox}[l]{Remark 4.6 (ii)}
		Every continuous, increasing process is natural. Indeed then, for $P$-a.e. $\omega \in \Omega$
		we have
		\begin{align}
			\int_{(0,t]} (M_s(\omega)-M_{s-}(\omega))\ dA_s(\omega) = 0
			\quad \mbox{for every $0 < t < \infty$},
		\end{align}
		because every path $\Set{M_s(\omega)}{0 \leq s < \infty}$ has only countably many discontinuities
		(Theorem 3.8(v)).
	\end{itembox}
	
	\begin{prf}
		$RCLL$パスの不連続点は高々可算個であり,
		連続な$A$で作る測度に対し一点集合は零集合となる.
		\QED
	\end{prf}
	
	\begin{itembox}[l]{Definition 4.8 修正}
		Let us consider the class $\mathscr{S}(\mathscr{S}_a)$ such as
		\begin{align}
			\mathscr{S} \coloneqq \Set{T:\mbox{stopping time of $(\mathscr{F}_t)$}}{\textcolor{red}{T < \infty}},
			\quad \mathscr{S}_a \coloneqq \Set{T:\mbox{stopping time of $(\mathscr{F}_t)$}}{\textcolor{red}{T \leq a}},\ (a > 0).
		\end{align}
		The right-continuous process $\Set{X_t,\mathscr{F}_t}{0 \leq t < \infty}$ is said to be 
		of class $D$, if the family $\{X_T\}_{T \in \mathscr{S}}$ is uniformly integrable;
		of class $DL$, if the family $\{X_T\}_{T \in \mathscr{S}_a}$ is uniformly integrable,
		for every $0 < a < \infty$.
	\end{itembox}
	
	\begin{itembox}[l]{Problem 4.9 修正}
		$X = \Set{X_t,\mathscr{F}_t}{0 \leq t < \infty}$ is a right-continuous submartingale.
		Show that under any one of the following conditions, $X$ is of class $DL$.
		\begin{description}
			\item[(a)] $X_t \geq 0$ a.s. for every $t \geq 0$.
			\item[(b)] $X$ has the special form
				\begin{align}
					X_t = M_t + A_t, \quad 0 \leq t < \infty
				\end{align}
				suggested by the Doob decomposition, where $\Set{M_t,\mathscr{F}_t}{0 \leq t < \infty}$
				is a martingale and $\Set{A_t,\mathscr{F}_t}{0 \leq t < \infty}$ is an increasing process.
		\end{description}
		Show also that if \textcolor{red}{$\mathscr{F}_0$ contains all the $P$-negligible events in $\mathscr{F}$} and
		$X$ is a uniformly integrable martingale, then it is of class $D$.
	\end{itembox}
	
	\begin{prf}\mbox{}
		\begin{description}
			\item[(a)]
				任意の$T \in \mathscr{S}_a$に対して
				$X_T$は$\mathscr{F}_T/\borel{\R}$-可測であるから
				(Proposition 2.18 修正),任意抽出定理より
				\begin{align}
					\int_{\{X_T > \lambda\}} X_T\ dP
					\leq \int_{\{X_T > \lambda\}} X_a\ dP,
					\quad (\forall \lambda > 0)
				\end{align}
				及び
				\begin{align}
					P\left( X_T > \lambda \right)
					\leq \frac{EX_T}{\lambda}
					\leq \frac{EX_a}{\lambda},
					\quad (\forall \lambda > 0)
				\end{align}
				が成立する.$X_a$が可積分であるから
				\begin{align}
					\sup{T \in \mathscr{S}_a}{\int_{\{X_T > \lambda\}} X_T\ dP}
					\longrightarrow 0
					\quad (\lambda \longrightarrow \infty)
				\end{align}
				となり,$(X_T)_{T \in \mathscr{S}_a}$の一様可積分性が得られる.
				
			\item[(b)]
				$a > 0$とすれば,任意抽出定理より
				\begin{align}
					M_T = \cexp{M_a}{\mathscr{F}_T},\ \mbox{a.s. $P$,}
					\quad (\forall T \in \mathscr{S}_a)
				\end{align}
				が成り立つから,定理\ref{lem:uniformly_integrability_and_conditional_expectations}
				(P. \pageref{lem:uniformly_integrability_and_conditional_expectations})より
				$(M_T)_{T \in \mathscr{S}_a}$は一様可積分である.このとき
				\begin{align}
					\int_{\{|X_T| > \lambda\}} |X_T|\ dP
					&\leq 2\int_{\{|M_T| > \lambda/2\}} |M_T|\ dP + 2\int_{\{|A_T| > \lambda/2\}} |A_T|\ dP \\
					&\leq 2\sup{T \in \mathscr{S}_a}{\int_{\{|M_T| > \lambda/2\}} |M_T|\ dP} + 2\int_{\{A_a > \lambda/2\}} A_a\ dP \\
					&\longrightarrow 0 \quad (\lambda \longrightarrow \infty)
				\end{align}
				が従い$(X_T)_{T \in \mathscr{S}_a}$の一様可積分性が出る.
		\end{description}
		$X$が一様可積分なマルチンゲールであるとき,Problem 3.20より
		\begin{align}
			X_t = \cexp{X_\infty}{\mathscr{F}_t},\ \mbox{a.s. $P$},
			\quad (\forall t \geq 0)
		\end{align}
		を満たす$\mathscr{F}_\infty/\borel{\R}$-可測可積分関数$X_\infty$が存在し,任意抽出定理より
		\begin{align}
			X_T = \cexp{X_\infty}{\mathscr{F}_T},\ \mbox{a.s. $P$},
			\quad (\forall T \in \mathscr{S})
		\end{align}
		が成り立つから$X$はクラス$DL$に属する.
		\QED
	\end{prf}
	
	\begin{itembox}[l]{Problem 4.11 修正}
		Let $(X,\mathscr{F},\mu)$ be a measure space and  
		$\left\{f_n\right\}_{n=1}^\infty$ be a sequence of integrable complex functions on $(X,\mathscr{F},\mu)$
		which converges weakly in $L^1$ to an integrable complex function $f$.
		Then for each $\sigma$-field $\mathscr{G} \subset \mathscr{F}$
		where $(X,\mathscr{G},\left.\mu\right|_{\mathscr{G}})$ is $\sigma$-finite,
		the sequence $\cexp{f_n}{\mathscr{G}}$ converges to $\cexp{f}{\mathscr{G}}$ weakly in $L^1$.
	\end{itembox}
	
	\begin{prf}
		$\nu \coloneqq \left.\mu\right|_{\mathscr{G}}$とおく.
		定理\ref{thm:properties_of_conditional_expectations}より
		任意の$g \in L^\infty(\mu)$と$F \in L^1(\mu)$に対して
		\begin{align}
			\int_X g\cexp{F}{\mathscr{G}}\ d\mu
			&= \int_X \cexp{g\cexp{F}{\mathscr{G}}}{\mathscr{G}}\ d\nu \\
			&= \int_X \cexp{g}{\mathscr{G}}\cexp{F}{\mathscr{G}}\ d\nu \\
			&= \int_X \cexp{\cexp{g}{\mathscr{G}}F}{\mathscr{G}}\ d\nu \\
			&= \int_X \cexp{g}{\mathscr{G}}F\ d\mu
		\end{align}
		が成り立ち,また定理\ref{thm:mean_value_of_integral_and_closed_set}と
		\begin{align}
			\int_A \cexp{g}{\mathscr{G}}\ d\nu = \int_A g\ d\mu \leq \mu(A) \Norm{g}{L^\infty(\mu)}
			= \nu(A) \Norm{g}{L^\infty(\mu)},
			\quad (\forall A \in \mathscr{G})
		\end{align}
		より$\Norm{\cexp{g}{\mathscr{G}}}{L^\infty(\nu)} \leq \Norm{g}{L^\infty(\mu)}$が従い
		\begin{align}
			\lim_{n \to \infty} \int_X g\cexp{f_n}{\mathscr{G}}\ d\mu
			= \lim_{n \to \infty} \int_X \cexp{g}{\mathscr{G}}f_n\ d\mu
			= \int_X \cexp{g}{\mathscr{G}}f\ d\mu
			= \int_X g\cexp{f}{\mathscr{G}}\ d\mu
		\end{align}
		となるから$\cexp{f_n}{\mathscr{G}}$は$\cexp{f}{\mathscr{G}}$に$L^1(\mu)$で弱収束する.
		\QED
	\end{prf}
	
	\begin{itembox}[l]{Lemma for Theorem 4.10}
		$a > 0$,$\mathscr{F}_a/\borel{\R}$-可測写像$A$と$T \in \mathscr{S}_a$に対して次が成り立つ:
		\begin{align}
			\left. \cexp{A}{\mathscr{F}_t} \right|_{t = T} = \cexp{A}{\mathscr{F}_T}
			\quad \mbox{a.s. $P$}.
			\label{eq:lemma_for_theorem_4_10}
		\end{align}
	\end{itembox}
	
	\begin{prf}
		$\Set{Y_t \coloneqq \cexp{A}{\mathscr{F}_t},\mathscr{F}_t}{0 \leq t < \infty}$はマルチンゲールであり,
		任意抽出定理より
		\begin{align}
			\cexp{Y_t}{\mathscr{F}_T} = Y_{t \wedge T},
			\quad \mbox{a.s. $P$}
		\end{align}
		が成り立つ.$Y_a = A,\ \mbox{a.s. $P$}$かつ$P(T \leq a)=1$より$t=a$として(\refeq{eq:lemma_for_theorem_4_10})を得る.
		\QED
	\end{prf}
	
	\begin{itembox}[l]{Theorem 4.10 (Doob-Meyer Decomposition)}
		Let $\{\mathscr{F}_t\}$ satisfy the usual conditions. If the right-continuous
		submartingale $X = \Set{X_t,\mathscr{F}_t}{0 \leq t < \infty}$ is of class $DL$, then
		there exists unique $[A]_{Nat}$ where $M \coloneqq X - A'$ is right-continuous martingale
		for every $A' \in [A]_{Nat}$.
		Further, if $X$ is of class $D$, then $M$ is a uniformly integrable martingale 
		and $A$ is integrable.	
	\end{itembox}
	
	\begin{prf}\mbox{}
		\begin{description}
			\item[第一段]
				分解の一意性を示す.
				二つのマルチンゲール$\{M_t,\mathscr{F}_t\},\{M'_t,\mathscr{F}_t\}$と
				ナチュラルな$\{A_t,\mathscr{F}_t\},\{A'_t,\mathscr{F}_t\}$により
				\begin{align}
					X_t = M_t + A_t = M'_t + A'_t,
					\quad \forall t \geq 0
				\end{align}
				と書けるとき,
				\begin{align}
					B=\Set{B_t \coloneqq A_t - A'_t = M'_t - M_t,\mathscr{F}_t}{0 \leq t < \infty}
				\end{align}
				は(a.s.のパスが有界変動な)マルチンゲールとなる.いま任意に$a > 0$を取り
				\begin{align}
					\Pi_n \coloneqq \left\{t_0^{(n)},\cdots,t_{m_n}^{(n)}\right\},
					\quad \max{1 \leq j \leq m_n}{\left| t_j^{(n)} - t_{j-1}^{(n)}\right|} \longrightarrow 0\ (n \longrightarrow \infty)
				\end{align}
				を$[0,a]$の細分列として,
				任意の有界な右連続マルチンゲール$\xi = \Set{\xi_t,\mathscr{F}_t}{0 \leq t < \infty}$に対し
				\begin{align}
					\xi^{(n)}_s \coloneqq \sum_{j=1}^{m_n} \defunc_{\left(t_{j-1}^{(n)},t_j^{(n)}\right]}(s)\ \xi_{t^{(n)}_{j-1}},
					\quad (\forall s \in [0,a])
				\end{align}
				とおけば,a.s.のパスの正則性(Theorem 3.8(v))より或る$P$-零集合$N_1$が存在して
				\begin{align}
					\lim_{n \to \infty} \xi^{(n)}_s(\omega) = \xi_{s-}(\omega),\ \forall s \in (0,a];
					\quad (\forall \omega \in \Omega \backslash N_1)
				\end{align}
				が満たされる.また或る$P$-零集合$N_2$が存在して$\omega \in \Omega \backslash N_2$に対する
				$A,A'$のパスが右連続かつ非減少となり,このときLebesgueの収束定理より
				任意の$\omega \in \Omega \backslash (N_1 \cup N_2)$に対して
				\begin{align}
					\lim_{n \to \infty} \int_{(0,a]} \xi^{(n)}_s(\omega)\ dA_s(\omega) = \int_{(0,a]} \xi_{s-}(\omega)\ dA_s(\omega),
					\quad \lim_{n \to \infty} \int_{(0,a]} \xi^{(n)}_s(\omega)\ dA'_s(\omega) = \int_{(0,a]} \xi_{s-}(\omega)\ dA'_s(\omega)
				\end{align}
				が成立する.ここで$A_a,A'_a$が可積分で$\xi$が有界であるから,再びLebesgueの収束定理より
				\begin{align}
					E\left[ \xi_a\left( A_a - A'_a \right) \right]
					&= E\left[ \xi_a A_a \right] -  E\left[ \xi_a A'_a \right]
					= E \int_{(0,a]} \xi_{s-}\ dA_s - E\int_{(0,a]} \xi_{s-}\ dA'_s \\
					&= E \left[ \lim_{n \to \infty} \int_{(0,a]} \xi^{(n)}_s\ dA_s \right]
						- E \left[ \lim_{n \to \infty} \int_{(0,a]} \xi^{(n)}_s\ dA'_s \right] \\
					&= \lim_{n \to \infty} E\left[ \sum_{j=1}^{m_n}\xi_{t^{(n)}_{j-1}}\left( A_{t^{(n)}_j} - A_{t^{(n)}_{j-1}} \right) \right]
						-  \lim_{n \to \infty} E \left[ \sum_{j=1}^{m_n}\xi_{t^{(n)}_{j-1}}\left( A'_{t^{(n)}_j} - A'_{t^{(n)}_{j-1}} \right) \right] \\
					&= \lim_{n \to \infty} E \left[ \sum_{j=1}^{m_n}\xi_{t^{(n)}_{j-1}}\left( B_{t^{(n)}_j} - B_{t^{(n)}_{j-1}} \right) \right]
				\end{align}
				が成立し,$B$のマルチンゲール性より
				\begin{align}
					E\xi_{t^{(n)}_{j-1}}\left( B_{t^{(n)}_j} - B_{t^{(n)}_{j-1}} \right)
					= E \left[\cexp{\xi_{t^{(n)}_{j-1}}\left( B_{t^{(n)}_j} - B_{t^{(n)}_{j-1}} \right)}{\mathscr{F}_{t^{(n)}_{j-1}}} \right]
					= E \left[ \xi_{t^{(n)}_{j-1}}\cexp{B_{t^{(n)}_j} - B_{t^{(n)}_{j-1}}}{\mathscr{F}_{t^{(n)}_{j-1}}} \right]
					= 0 
				\end{align}
				となるから
				\begin{align}
					E\left[ \xi_a\left( A_a - A'_a \right) \right] = 0
				\end{align}
				が得られる.ここで$\xi$を有界実確率変数とすれば
				$\Set{\cexp{\xi}{\mathscr{F}_t},\mathscr{F}_t}{0 \leq t < \infty}$はマルチンゲールとなり
				\begin{align}
					E\left[ \cexp{\xi}{\mathscr{F}_t} \right] = E \xi,
					\quad (\forall t \geq 0)
				\end{align}
				が満たされ,Theorem 3,13より右連続な修正$\Set{\xi_t,\mathscr{F}_t}{0 \leq t < \infty}$が得られる.
				このとき
				\begin{align}
					0 = E\left[ \xi_a\left( A_a - A'_a \right) \right]
					= E\left[ \cexp{\xi}{\mathscr{F}_a}\left( A_a - A'_a \right) \right]
					= E\left[ \cexp{\xi\left( A_a - A'_a \right)}{\mathscr{F}_a} \right]
					= E\left[ \xi\left( A_a - A'_a \right) \right]
				\end{align}
				が成り立ち,$\xi = \sgn{A_a - A'_a}$として
				\begin{align}
					E\left| A_a - A'_a \right| = 0
				\end{align}
				が従う.$a > 0$の任意性及び$A,A'$のパスの右連続性より
				\begin{align}
					P\left[ \Set{A_t = A'_t}{0 \leq t < \infty} \cap (\Omega \backslash N_2) \right]
					= P\left[ \bigcap_{a \in [0,\infty) \cap \Q} \{A_a=A'_a\} \right]
					= 1
				\end{align}
				となり分解の一意性が出る.
				
			\item[第二段]
				任意の区間$[0,a]$上で分解の存在を示せば
				$[0,\infty)$での分解が得られる.
				実際任意の$n \geq 1$に対し
				\begin{align}
					X_t = M^n_t + A^n_t, \quad (t \in [0,n])
				\end{align}
				と分解されるなら,$m > n$に対して
				\begin{align}
					M^n_t + A^n_t = M^m_t + A^m_t, \quad (t \in [0,n])
				\end{align}
				となり,前段の結果より或る$P$-零集合$E_{n,m}$が存在して,任意の$\omega \in \Omega \backslash E_{n,m}$に対して
				\begin{align}
					A^n_t(\omega) = A^m_t(\omega), \quad (\forall t \in [0,n])
				\end{align}
				が成立し,かつ$[0,n) \ni t \longmapsto A^n_t(\omega)$が右連続非減少となる.ここで
				\begin{align}
					E \coloneqq \bigcup_{\substack{n,m \in \N \\ n<m}} E_{n,m}
				\end{align}
				により$P$-零集合を定めれば,任意の$\omega \in \Omega \backslash E$及び$t \geq 0$に対して
				\begin{align}
					A^n_t(\omega) = A^m_t(\omega), \quad (\forall m > n > t)
				\end{align}
				となり$\lim_{n \to \infty} A^n_t(\omega)$が確定する.
				usual条件より$E \in \mathscr{F}_0$だから$A^n_t \defunc_{\Omega \backslash E}\ (n > t)$は
				$\mathscr{F}_t/\borel{\R}$-可測であり,
				\begin{align}
					A_t \coloneqq  \lim_{n \to \infty} A^n_t \defunc_{\Omega \backslash E},
					\quad (\forall t \geq 0)
				\end{align}
				で$A_t$を定めれば$A_t$は$\mathscr{F}_t/\borel{\R}$-可測かつ可積分となる.また
				任意の$\omega \in \Omega$と$n \geq 1$に対し
				\begin{align}
					A_t(\omega) = A^n_t(\omega) \defunc_{\Omega \backslash E}, \quad (\forall t \in [0,n))
				\end{align}
				が成り立つから$[0,\infty) \ni t \longmapsto A_t(\omega)$は右連続かつ非減少である.
				$\Set{\xi_t,\mathscr{F}_t}{0 \leq t < \infty}$を右連続なマルチンゲールとすれば
				或る$P$-零集合$C$が存在して$\Omega \backslash C$上でパスはRCLLとなるから,任意の$t > 0$に対し
				\begin{align}
					\left( \int_{(0,t]} \xi_{s(-)}\ dA_s \right)(\omega) &\coloneqq
					\begin{cases}
						\displaystyle\int_{(0,t]} \xi_{s(-)}(\omega)\ dA_s(\omega), & (\omega \in \Omega \backslash (E \cup C)), \\
						0, & (\omega \in E \cup C)
					\end{cases}, \\
					\left( \int_{(0,t]} \xi_{s(-)}\ dA^n_s \right)(\omega) &\coloneqq
					\begin{cases}
						\displaystyle\int_{(0,t]} \xi_{s(-)}(\omega)\ dA^n_s(\omega), & (\omega \in \Omega \backslash (E \cup C)), \\
						0, & (\omega \in E \cup C)
					\end{cases},
					\quad (n > t)
				\end{align}
				とおけば
				\begin{align}
					E \int_{(0,t]} \xi_s\ dA_s = E \int_{(0,t]} \xi_s\ dA^n_s 
					= E \int_{(0,t]} \xi_{s-}\ dA^n_s = E \int_{(0,t]} \xi_{s-}\ dA_s
				\end{align}
				が成立する.
				\begin{align}
					M \coloneqq X - A
				\end{align}
				とおけば$(M_t)_{t \geq 0}$は$(\mathscr{F}_t)$-適合かつ可積分であり,
				任意の$0 \leq s < t$及び$t < n$に対して
				\begin{align}
					M_t = X_t - A^n_t \defunc_{\Omega \backslash E} = M^n_t,
					\quad M_s = X_s - A^n_s \defunc_{\Omega \backslash E} = M^n_s,
					\quad \mbox{a.s. $P$}
				\end{align}
				となるから$\cexp{M_t}{\mathscr{F}_s} = M_s\ \mbox{a.s. $P$}$が満たされる.
			
			\item[第三段]
				以下,$a > 0$として$[0,a]$上で分解の存在を示す.
				\begin{align}
					Y_t \coloneqq X_t - \cexp{X_a}{\mathscr{F}_t},
					\quad (t \in [0,a])
				\end{align}
				とおけば$\Set{Y_t,\mathscr{F}_t}{0 \leq t \leq a}$は劣マルチンゲールであり,
				\begin{align}
					\Set{Y_{t^{(n)}_j},\mathscr{F}_{t^{(n)}_j}}{t^{(n)}_j = \frac{j}{2^n}a,\ j=0,1,\cdots,2^n},
					\quad n=1,2,\cdots
				\end{align}
				で離散化すれば,Doob分解の補題 (P. \pageref{lem:Doob_decomposition})より
				\begin{align}
					A^{(n)}_0 \coloneqq 0,
					\quad A^{(n)}_{t^{(n)}_j} \coloneqq \sum_{k=0}^{j-1} \cexp{Y_{t^{(n)}_{k+1}} - Y_{t^{(n)}_k}}{\mathscr{F}_{t^{(n)}_k}},
					\ (k=1,\cdots,2^n);
					\quad M^{(n)}_{t^{(n)}_j} \coloneqq Y_{t^{(n)}_j} - A^{(n)}_{t^{(n)}_j},
					\ (k=0,1,\cdots,2^n)
				\end{align}
				により可予測な増大過程$A^{(n)}$とマルチンゲール$M^{(n)}$に分解され,
				$Y_a = 0\ \mbox{a.s. $P$}$であるから
				\begin{align}
					Y_{t^{(n)}_j} = A^{(n)}_{t^{(n)}_j} +  M^{(n)}_{t^{(n)}_j}
					= A^{(n)}_{t^{(n)}_j} + \cexp{M^{(n)}_a}{\mathscr{F}_{t^{(n)}_j}}
					= A^{(n)}_{t^{(n)}_j} - \cexp{A^{(n)}_a}{\mathscr{F}_{t^{(n)}_j}},
					\quad \mbox{a.s. $P$},
					\quad j=0,1,\cdots,2^n
				\end{align}
				となる.
				
			\item[第四段]
				$\left( A^{(n)}_a \right)_{n=1}^\infty$が一様可積分であることを示す.
				\begin{align}
					f
				\end{align}
				
			\item[第五段]
				Dunford-Pettisの定理より$\left( A^{(n)}_a \right)_{n=1}^\infty$の或る部分列
				$\left( A^{(n_k)}_a \right)_{k=1}^\infty$は$L^1(P)$で弱収束する.つまり
				或る$A_a \in L^1(P)$が存在して
				任意の$\xi \in L^\infty(P)$に対し
				\begin{align}
					E \xi A^{(n_k)}_a \longrightarrow E \xi A_a
					\quad (k \longrightarrow \infty)
				\end{align}
				が成立する.
				\begin{align}
					\Pi_n \coloneqq \Set{t^{(n)}_j}{t^{(n)}_j = \frac{j}{2^n}a,\ j=0,1,\cdots,2^n},
					\quad \Pi \coloneqq \bigcup_{n=1}^\infty \Pi_n
				\end{align}
				とすれば,任意の$t \in \Pi$に対し或る$K \geq 1$が存在して
				$t \in \Pi_{n_k}\ (\forall k > K)$となり,Problem 4.11より
				\begin{align}
					E \xi A^{(n_k)}_t
					= E \xi\left\{ Y_t + \cexp{A^{(n_k)}_a}{\mathscr{F}_t} \right\}
					\longrightarrow E \xi\left\{ Y_t + \cexp{A_a}{\mathscr{F}_t} \right\}
					\quad (k > K,\ k \longrightarrow \infty)
				\end{align}
				が成り立つから$A^{(n_k)}_t$は$Y_t + \cexp{A_a}{\mathscr{F}_t}$に弱収束する.
				ここで
				\begin{align}
					\tilde{A}_t \coloneqq Y_t + \cexp{A_a}{\mathscr{F}_t},
					\quad (t \in [0,a])
				\end{align}
				と定めれば$\Set{\tilde{A}_t,\mathscr{F}_t}{0 \leq t \leq a}$は
				劣マルチンゲールとなり,$\Set{X_t,\mathscr{F}_t}{0 \leq t <\infty}$の右連続性より
				\begin{align}
					[0,a] \ni t \longmapsto E\left[ Y_t + \cexp{A_a}{\mathscr{F}_t} \right]
					= E X_t - E X_a + E A_a
				\end{align}
				は右連続であるから(Theorem 3.13),$\tilde{A}$の右連続な修正$\Set{A_t,\mathscr{F}_t}{0 \leq t \leq a}$
				が得られる.
			
			\item[第六段]
				$t \longmapsto A_t(\omega)$がa.s.に0出発かつ非減少であることを示す.
				実際,$\xi = \sgn{A_0}$として
				\begin{align}
					E |A_0| = E \xi A_0 = E \xi \tilde{A}_0 = \lim_{k \to \infty} E \xi A^{(n_k)}_0 = 0
				\end{align}
				が成り立つから$A_0 = 0\ \mbox{a.s. $P$}$が従う.また任意に$s,t \in \Pi,\ (s<t)$を取れば
				或る$K \geq 1$が存在して$s,t \in \Pi_{n_k}\ (\forall k > K)$が満たされ,
				$A^{(n_k)}$は増大過程であるから$\xi = \defunc_{\{A_s > A_t\}}$として
				\begin{align}
					E \xi (A_t - A_s) = E \xi \left( \tilde{A}_t - \tilde{A}_s \right)
					= \lim_{k \to \infty} E \xi \left( A^{(n_k)}_t - A^{(n_k)}_s \right) \geq 0 
				\end{align}
				となり$P(A_s > A_t) = 0$が成り立つ.
				\begin{align}
					N \coloneqq \Biggl(\bigcup_{\substack{s,t \in \Pi \\ s < t}} \{A_s > A_t\}\Biggr) \cup \{A_0 \neq 0\}
				\end{align}
				により$P$-零集合を定めれば,$t \longmapsto A_t$の右連続性より
				$\Omega \backslash N$上で$t \longmapsto A_t$は0出発非減少となる.
				
			\item[第七段]
				$A$がナチュラルであることを示す.$\xi = \Set{\xi_t,\mathscr{F}_t}{0 \leq t \leq a}$を有界な右連続マルチンゲールとすれば
				\begin{align}
					E \xi_a A^{(n_k)}_a 
					&= E\left[ \sum_{j=1}^{2^n}\xi_{t^{(n_k)}_{j-1}} \left( A^{(n_k)}_{t^{(n_k)}_j} - A^{(n_k)}_{t^{(n_k)}_{j-1}} \right) \right]\\
					&= E\left[ \sum_{j=1}^{2^n}\xi_{t^{(n_k)}_{j-1}} \left( Y_{t^{(n_k)}_j} - Y_{t^{(n_k)}_{j-1}} \right) \right]
						+ E\left[ \sum_{j=1}^{2^n}\xi_{t^{(n_k)}_{j-1}} \left( \cexp{A^{(n_k)}_a}{\mathscr{F}_{t^{(n_k)}_j}} - \cexp{A^{(n_k)}_a}{\mathscr{F}_{t^{(n_k)}_{j-1}}} \right) \right] \\
					&= E\left[ \sum_{j=1}^{2^n}\xi_{t^{(n_k)}_{j-1}} \left( A_{t^{(n_k)}_j} - A_{t^{(n_k)}_{j-1}} \right) \right]
				\end{align}
				が任意の$k \geq 1$で成り立ち,或る$P$-零集合$N'$が存在して
				$\Omega \backslash N'$上で$\xi$のパスがRCLLとなるから,
				\begin{align}
					\left(\int_{(0,a]} \xi_{s(-)}\ dA_s\right)(\omega)
					\coloneqq 
					\begin{cases}
						\displaystyle\int_{(0,a]} \xi_{s(-)}(\omega)\ dA_s(\omega), & (\omega \in \Omega \backslash (N \cup N')), \\
						0, & (\omega \in N \cup N')
					\end{cases}
				\end{align}
				と定めれば$k \longrightarrow \infty$として
				\begin{align}
					E \xi_a A_a = E \int_{(0,a]} \xi_{s-}\ dA_s
				\end{align}
				が得られる.任意の$t \in (0,a]$に対し
				$\xi^t = \Set{\xi^t_s \coloneqq \xi_{t \wedge s},\mathscr{F}_s}{0 \leq s \leq a}$
				も連続マルチンゲールであり
				\begin{align}
					\xi^t_{s-} &= \xi_{s-},\quad (\forall s \in (0,t]), \\
					\xi^t_{s-} &= \xi_t, \quad (\forall s \in (t,a])
				\end{align}
				より
				\begin{align}
					E \xi_t A_t + E \xi_t(A_a - A_t) = E \xi^t_a A_a 
					= E \int_{(0,a]} \xi^t_{s-}\ dA_s
					= E \int_{(0,t]} \xi_{s-}\ dA_s + E \xi_t (A_a - A_t)
				\end{align}
				となり
				\begin{align}
					E \xi_t A_t = E \int_{(0,t]} \xi_{s-}\ dA_s,
					\quad (\forall t \in (0,a])
				\end{align}
				が成立する.よって$A$はナチュラルである.
		\end{description}
	\end{prf}
	
	\begin{itembox}[l]{Problem 4.13}
		Verify that a continuous, nonnegative submartingale is regular. 
	\end{itembox}
	
	\begin{prf}
		$(X_{T_n})_{n=1}^\infty$は一様可積分である.実際,
		\begin{align}
			\int_{X_{T_n} > \lambda} X_{T_n}\ dP
			\leq \int_{X_{T_n} > \lambda} X_T\ dP
		\end{align}
		かつ
		\begin{align}
			P\left( X_{T_n} > \lambda \right)
			\leq \frac{EX_{T_n}}{\lambda}
			\leq \frac{EX_T}{\lambda}
		\end{align}
		が成り立ち,$X_T$の可積分性(任意抽出定理)より一様可積分性が従う.またパスの連続性より
		\begin{align}
			X_{T_n} \longrightarrow X_T
			\quad (n \longrightarrow \infty)
		\end{align}
		となるから,一様可積分性と平均収束の補題 (P. \pageref{lem:uniformly_integrable_and_convergence_in_mean})より
		\begin{align}
			\lim_{n \to \infty} EX_{T_n} = EX_T
		\end{align}
		が成立する.
		\QED
	\end{prf}
	
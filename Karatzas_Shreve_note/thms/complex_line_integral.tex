\subsection{複素線積分}
	
	$\C$上で定義された関数を積分するには,$\C$を台とする可測空間に測度を導入するか,
	何らかの方法で$\R$上のLebesgue積分に持ち込むかすれば良い.複素積分は後者の方法を採る.
	直感的なイメージとしては``複素平面上に現れた線に沿って''関数を積分するのが複素積分であり,
	その``線''とは実区間上の写像の像である.
	
	\begin{screen}
		\begin{dfn}[路]
			$\R$の有界閉区間上で定義された連続有界変動な$\C$値関数を{\bf 路}\index{ろ@路}{\bf (contour)}と呼ぶ.
		\end{dfn}
	\end{screen}
	
	いま$\gamma$を路とし,$\alpha$と$\beta$を
	\begin{align}
		[\alpha,\beta] = \dom{\gamma}
	\end{align}
	を満たす実数とする.$\gamma$で作る$\borel{[\alpha,\beta]}$上の複素Stieltjes測度を
	\begin{align}
		\mu_{\gamma}
	\end{align}
	とする.$f$を
	\begin{align}
		\gamma^* \defeq \ran{\gamma}
	\end{align}
	上の$\C$値連続写像とするとき,Stieltjes積分
	\begin{align}
		\int_{[\alpha,\beta]} f \circ \gamma\ d\mu_{\gamma}
	\end{align}
	を$f$の$\gamma$に関する{\bf 複素線積分}\index{ふくそせんせきぶん@複素線積分}{\bf (complex contour integral)}と呼び,便宜上
	\begin{align}
		\int_{\gamma} f(z)\ dz
	\end{align}
	や
	\begin{align}
		\int_{\gamma} f
	\end{align}
	と書く.特に$\gamma$が$[\alpha,\beta]$上で絶対連続なら,
	$\lambda$を一次元Lebesgue測度とすれば,微分積分学の基本定理より
	\begin{align}
		\int_{[\alpha,\beta]} f \circ \gamma\ d\mu_\gamma = \int_{[\alpha,\beta]}f \circ \gamma \cdot \gamma'\ d\lambda
	\end{align}
	が成立する.
	
	$\gamma$は$[\alpha,\beta]$上の写像であるが,$\gamma$に関する線積分はパラメータ区間を変項することが出来る.
	
	\begin{itembox}[l]{線積分のパラメータ区間変項}
		$\sigma$と$\tau$を$\sigma < \tau$を満たす実数とし,$\varphi$を
		\begin{align}
			[\sigma,\tau] \ni t \longmapsto \alpha + \frac{t - \sigma}{\tau - \sigma} \cdot (\beta - \alpha)
		\end{align}
		なる$[\sigma,\tau]$上の写像とする.このとき
		\begin{align}
			\eta \defeq \gamma \circ \varphi
		\end{align}
		で定める$\eta$は,$[\sigma,\tau]$上の連続な有界変動関数であって,
		\begin{align}
			\mu_{\gamma}(E) = \mu_{\eta} \left(\varphi^{-1} \ast E\right)
		\end{align}
		が$\borel{[\alpha,\beta]}$の任意の要素$E$で成立する.
	\end{itembox}
	
	これが示されれば,$\gamma^*$上の任意の$\C$値連続写像$f$に対して
	\begin{align}
		\int_{\gamma} f = \int_{[\alpha,\beta]} f \circ \gamma\ d\mu_{\gamma}
		= \int_{[\sigma,\tau]} f \circ \eta\ d\mu_{\eta} = \int_{\eta} f
	\end{align}
	が成立する.実際,
	\begin{align}
		\borel{\gamma^*} \defeq \Set{E \cap \gamma^*}{E \in \borel{C}}
	\end{align}
	とおけば,$\gamma^*$は$\C$のコンパクト部分集合すなわち閉集合であるから
	\begin{align}
		\borel{\gamma^*} \subset \borel{C}
	\end{align}
	が成り立つ.よって$E$を$\borel{\gamma^*}$の要素とすれば
	\begin{align}
		\mu_{\gamma}\left(\gamma^{-1} \ast E\right)
		= \mu_{\eta}\left(\varphi^{-1} \ast (\gamma^{-1} \ast E)\right)
		= \mu_{\eta}\left(\eta^{-1} \ast E\right)
	\end{align}
	が成立する.ゆえに$f$が$\borel{\gamma^*}/\borel{\C}$-可測単関数である場合は
	\begin{align}
		\int_{[\alpha,\beta]} f \circ \gamma\ d\mu_{\gamma} = \int_{[\sigma,\tau]} f \circ \eta\ d\mu_{\eta}
	\end{align}
	が成り立つから,Lebesgueの収束定理より$f$が連続である場合も
	\begin{align}
		\int_{[\alpha,\beta]} f \circ \gamma\ d\mu_{\gamma} = \int_{[\sigma,\tau]} f \circ \eta\ d\mu_{\eta}
	\end{align}
	が成り立つ.特に{\bf 複素線積分は全て$[0,1]$上の路に関する積分に書き直すことが出来る.}
	
	\begin{sketch}\mbox{}
		\begin{description}
			\item[第一段]
				$v$を$\gamma$の$[\alpha,\beta]$上の総変動とする.
				いま$t$を$[\sigma,\tau]$の分割とする.つまり$1$以上の自然数$n$で
				\begin{align}
					t: n+1 \longrightarrow [\sigma,\tau]
				\end{align}
				を満たすものが取れて,
				\begin{align}
					\begin{cases}
						t_{0} = \sigma & \\
						t_{j} < t_{j+1} & \mbox{if } j \in n \wedge 1 \leq j \\
						t_{n} = \tau, &
					\end{cases}
				\end{align}
				つまり
				\begin{align}
					\sigma = t_{0} < t_{1} < \cdots < t_{n} = \tau
				\end{align}
				が成り立っている.このとき
				\begin{align}
					\varphi \circ t
				\end{align}
				は$[\alpha,\beta]$の分割であって
				\begin{align}
					\sum_{j=1}^{n} \left|\eta(t_j) - \eta(t_{j-1})\right|
					= \sum_{j=1}^{n} \left|\gamma(\varphi(t_j)) - \gamma(\varphi(t_{j-1}))\right|
					\leq v
				\end{align}
				が成立する.これは$[\sigma,\tau]$の分割の取り方に依らないから$\eta$は有界変動である.
				$\eta$の連続性は$\gamma$と$\varphi$の連続性から従う.
				
			\item[第二段]
				$s$と$t$を
				\begin{align}
					s < t
				\end{align}
				を満たす$[\alpha,\beta]$の要素とするとき,
				\begin{align}
					\mu_{\eta}\left(\varphi^{-1} \ast ]s,t]\right)
					&= \mu_{\eta}\left(\left]\varphi^{-1}(s), \varphi^{-1}(t)\right]\right) \\
					&= \eta(\varphi^{-1}(t)) - \eta(\varphi^{-1}(s)) \\
					&= \gamma(t) - \gamma(s) \\
					&= \mu_{\gamma}(]s,t])
				\end{align}
				が成立する.$\mu_{\gamma}$も$\mu_{\eta}$も一点の測度は$0$であるから,
				測度の一致の定理より$\borel{[\alpha,\beta]}$の任意の要素$E$で
				\begin{align}
					\mu_{\gamma}(E) = \mu_{\eta} \left(\varphi^{-1} \ast E\right)
				\end{align}
				が成立する.
				\QED
		\end{description}
	\end{sketch}
	
	次に`逆向き'の路に関する積分を考える.
	
	\begin{screen}
		\begin{dfn}[逆路]
			$\gamma$を路とし,$\alpha$と$\beta$を
			\begin{align}
				[\alpha,\beta] = \dom{\gamma}
			\end{align}
			を満たす実数とする.このとき
			\begin{align}
				[\alpha, \beta] \ni t \longmapsto \gamma(\alpha + \beta - t) 
			\end{align}
			なる関係で定める$[\alpha,\beta]$上の写像を$\gamma$の{\bf 逆路}\index{ぎゃくろ@逆路}{\bf (inverse contour)}と呼ぶ.
		\end{dfn}
	\end{screen}
	
	先ほどから扱っている$\gamma$に対して,$\rho$をその逆路とする.このとき$\rho$は有界変動かつ連続である.
	実際,$\psi$を
	\begin{align}
		[\alpha,\beta] \ni t \longmapsto \alpha + \beta - t
	\end{align}
	なる写像とすれば
	\begin{align}
		\rho = \gamma \circ \psi
	\end{align}
	が成り立つので,$\rho$の連続性は$\gamma$と$\psi$の連続性から従う.
	また$t$を$[\sigma,\tau]$の分割とすれば,
	$1$以上の自然数$n$で
	\begin{align}
		t: n+1 \longrightarrow [\sigma,\tau]
	\end{align}
	かつ
	\begin{align}
		\begin{cases}
			t_{0} = \sigma & \\
			t_{j} < t_{j+1} & \mbox{if } j \in n \wedge 1 \leq j \\
			t_{n} = \tau, &
		\end{cases}
	\end{align}
	を満たすものが取れるが,このとき$s$を
	\begin{align}
		n+1 \ni j \longmapsto \psi(t_{n-j})
	\end{align}
	なる写像とすれば$s$もまた$[\alpha,\beta]$の分割であって
	\begin{align}
		\sum_{j=1}^{n} \left|\rho(t_{j}) - \rho(t_{j-1})\right|
		= \sum_{j=1}^{n} \left|\gamma(s_{n-j}) - \gamma(s_{n-j+1})\right|
		= \sum_{j=1}^{n} \left|\gamma(s_{j}) - \gamma(s_{j-1})\right|
	\end{align}
	が成り立つ.右辺は$\gamma$の総変動で抑えられるので$\rho$の総変動も有限値である.以上より
	$\rho$で複素Stieltjes測度を構成できる.
	
	\begin{screen}
		\begin{thm}[逆路に関する積分は正負が逆転する]
			$\gamma$を路とし,$\rho$をその逆路とする.このとき,$f$を$\ran{\gamma}$上の$\C$値連続関数とすれば
			\begin{align}
				\int_{\gamma} f = -\int_{\rho} f
			\end{align}
			が成立する.
		\end{thm}
	\end{screen}
	
	\begin{sketch}
		$\alpha$と$\beta$を
		\begin{align}
			[\alpha,\beta] = \dom{\gamma}
		\end{align}
		を満たす実数とする.また$\psi$を
		\begin{align}
			[\alpha,\beta] \ni t \longmapsto \alpha + \beta - t
		\end{align}
		なる写像とする.まず$\borel{[\alpha,\beta]}$の任意の要素$E$で
		\begin{align}
			\mu_{\gamma}(E) = -\mu_{\rho}\left(\psi^{-1} \ast E\right)
		\end{align}
		が成り立つことを示す.$s$と$t$を
		\begin{align}
			s < t
		\end{align}
		を満たす$[\alpha,\beta]$の要素とすれば,
		\begin{align}
			\mu_{\rho}\left(\psi^{-1} \ast ]s,t[\right)
			&= \mu_{\rho}\left(]\psi^{-1}(t),\psi^{-1}(s)[\right) \\
			&= \rho(\psi^{-1}(s)) - \rho(\psi^{-1}(t)) \\
			&= \gamma(s) - \gamma(t) \\
			&= -\mu_{\gamma}(]s,t[)
		\end{align}
		が成立する.$\mu_{\gamma}$も$\mu_{\rho}$も一点の測度は$0$であるから,
		測度の一致の定理より$\borel{[\alpha,\beta]}$の任意の要素$E$で
		\begin{align}
			\mu_{\gamma}(E) = -\mu_{\rho}\left(\psi^{-1} \ast E\right)
		\end{align}
		が成立する.次に$f$を$\ran{\gamma}$上の$\C$値連続関数として
		\begin{align}
			\int_{\rho} f = -\int_{\gamma} f
		\end{align}
		であることを示す.ここで
		\begin{align}
			\borel{\gamma^*} \defeq \Set{E \cap \ran{\gamma}}{E \in \borel{C}}
		\end{align}
		とおくと,$\borel{\gamma^*}$の任意の要素$E$に対しては
		\begin{align}
			\mu_{\gamma}\left(\gamma^{-1} \ast E\right) = -\mu_{\rho}\left(\psi^{-1} \ast (\gamma^{-1} \ast E)\right)
			= -\mu_{\rho}\left(\rho^{-1} \ast E\right)
		\end{align}
		が成り立つので,$f$が$\borel{\gamma^*}/\borel{\C}$-可測単関数である場合は
		\begin{align}
			\int_{[\alpha,\beta]} f \circ \gamma\ d\mu_{\gamma} = -\int_{[\alpha,\beta]} f \circ \rho\ d\mu_{\rho}
		\end{align}
		が成り立つ.Lebesgueの収束定理より$f$が連続である場合も
		\begin{align}
			\int_{[\alpha,\beta]} f \circ \gamma\ d\mu_{\gamma} = -\int_{[\alpha,\beta]} f \circ \rho\ d\mu_{\rho}
		\end{align}
		が成り立つ.
		\QED
	\end{sketch}
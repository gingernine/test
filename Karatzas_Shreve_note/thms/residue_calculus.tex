\subsection{留数解析}
	
	\begin{screen}
		\begin{thm}[線積分の微分積分学の基本定理]
		\label{thm:fundamental_theorem_of_calculus_for_contour_integral}
			$\Omega$を$\C$の開集合とし,$f$を$\Omega$上の正則関数とし,
			$\gamma$を$[0,1]$から$\Omega$への路とする.このとき
			\begin{align}
				\int_{\gamma} f' = f(\gamma(1)) - f(\gamma(0)).
			\end{align}
		\end{thm}
	\end{screen}
	
	\begin{sketch}
		$\epsilon$を任意に与えられた正の実数とし,
		\begin{align}
			L \defeq |\mu_{\gamma}|([0,1])
		\end{align}
		とおき,
		\begin{align}
			r \defeq \inf{}{\Set{|\gamma(t) - z|}{t \in [0,1] \wedge z \in \C \backslash \Omega}}
		\end{align}
		とおく.このとき
		\begin{align}
			N \defeq \bigcup_{t \in [0,1]} \disc{\gamma(t)}{r/2}
		\end{align}
		とおけば,$\overline{N}$はコンパクトであって
		\begin{align}
			\overline{N} \subset \Omega
		\end{align}
		を満たし,$f'$は$\overline{N}$の上で連続であるから,
		\begin{align}
			\delta < \frac{r}{2}
		\end{align}
		なる正の実数$\delta$で
		\begin{align}
			\forall z,w \in \overline{N}\,
			\left[\, |z-w| < \delta \Longrightarrow |f'(z) - f'(w)| < \frac{\epsilon}{L}\, \right]
		\end{align}
		を満たすもの取れる.一方で,自然数$n$で
		\begin{align}
			\left|\int_{[0,1]} f' \circ \gamma\ d\mu_{\gamma}
			- \sum_{k=0}^{n-1} f'(\gamma(k/n)) \cdot \left(\gamma((k+1)/n) - \gamma(k/n)\right)\right|
			< \epsilon
		\end{align}
		及び,$n$より小さい任意の自然数$k$に対して
		\begin{align}
			|\gamma((k+1)/n) - \gamma(k/n)| < \delta
		\end{align}
		を満たすものが取れる.ここで
		\begin{align}
			[0,1] \ni t \longmapsto \sum_{k=0}^{n-1} \gamma(k/n) + \frac{t - k/n}{1/n} \cdot \left(\gamma((k+1)/n) - \gamma(k/n)\right)
		\end{align}
		なる路を$\eta$とおけば,微分積分学の基本定理より
		\begin{align}
			\int_{[0,1]} f' \circ \eta\ d\mu_{\eta} = f(\eta(1)) - f(\eta(0))
			= f(\gamma(1)) - f(\gamma(0))
		\end{align}
		が成り立ち,他方で
		\begin{align}
			&\left|\int_{[0,1]} f' \circ \eta\ d\mu_{\eta}
			- \sum_{k=0}^{n-1} f'(\gamma(k/n)) \cdot \left(\gamma((k+1)/n) - \gamma(k/n)\right)\right| \\
			&= \left|\sum_{k=0}^{n-1} \int_{\left(\frac{k}{n},\frac{k+1}{n}\right]} f' \circ \eta\ d\mu_{\eta}
			- \sum_{k=0}^{n-1} \int_{\left(\frac{k}{n},\frac{k+1}{n}\right]} f'(\eta(k/n))\ d\mu_{\eta}\right| \\
			&\leq \sum_{k=0}^{n-1} \int_{\left(\frac{k}{n},\frac{k+1}{n}\right]}
			\left|f' \circ \eta - f'(\eta(k/n))\right|\ d|\mu_{\eta}| \\
			&\leq \frac{\epsilon}{L} \cdot \sum_{k=0}^{n-1} |\mu_{\eta}|\left(\left(k/n,(k+1)/n\right]\right) \\
			&= \frac{\epsilon}{L} \cdot \sum_{k=0}^{n-1} \left|\gamma((k+1)/n) - \gamma(k/n)\right| \\
			&\leq \epsilon
		\end{align}
		が成立する.以上から
		\begin{align}
			\left|\int_{[0,1]} f'\circ\gamma\ d\mu_{\gamma}
			- \left[f(\gamma(1)) - f(\gamma(0))\right]\right|
			\leq 2 \cdot \epsilon
		\end{align}
		が従い,$\epsilon$の任意性から
		\begin{align}
			\int_{[0,1]} f'\circ\gamma\ d\mu_{\gamma} = f(\gamma(1)) - f(\gamma(0))
		\end{align}
		が得られる.
		\QED
	\end{sketch}
	
	いま$R$を正の実数とし,$f$を
	\begin{align}
		\Omega \defeq \disc{0}{R} \backslash \{0\}
	\end{align}
	上の正則関数とし,$p$と$q$を複素数列とし,$\Omega$の各要素$z$で
	\begin{align}
		f(z) = \sum_{n=0}^{\infty} \frac{p_{n}}{z^{n+1}} + \sum_{n=0}^{\infty} q_{n} \cdot z^{n}
	\end{align}
	が満たされているとする.つまり$\ran{\gamma}$の各要素$z$に対して,数列$\left(\sum_{k=0}^{n} p_{k}/z^{k+1}\right)_{n \in \Natural}$と
	$\left(\sum_{k=0}^{n} q_{k} \cdot z^{k}\right)_{n \in \Natural}$が$\C$で収束しているということである.このとき,$\gamma$を
	\begin{align}
		\ran{\gamma} \subset \Omega
	\end{align}
	を満たす閉路とすれば
	\begin{align}
		p_{n} \cdot \Ind_{\gamma}(0) = \frac{1}{2\cdot\pi\cdot\isym} \cdot \int_{\gamma} f(z) \cdot z^{n}\ dz
		\label{fom:Laurent_series_0_1}
	\end{align}
	及び
	\begin{align}
		q_{n} \cdot \Ind_{\gamma}(0) = \frac{1}{2\cdot\pi\cdot\isym} \cdot \int_{\gamma} \frac{f(z)}{z^{n+1}}\ dz
		\label{fom:Laurent_series_0_2}
	\end{align}
	が任意の自然数$n$で成立する.
	
	実際,$n$を自然数とすれば
	\begin{align}
		\int_{\gamma} f(z) \cdot z^{n}\ dz
		&= \int_{\gamma} \sum_{k=0}^{\infty} \frac{p_{k}}{z^{k+1-n}}\ dz
			+ \int_{\gamma} \sum_{k=0}^{\infty} q_{k} \cdot z^{k+n}\ dz \\
		&= \sum_{k=0}^{\infty} p_{k} \cdot \int_{\gamma} \frac{1}{z^{k+1-n}}\ dz
			+ \sum_{k=0}^{\infty} q_{k} \cdot \int_{\gamma} z^{k+n}\ dz
	\end{align}
	となるが,定理\ref{thm:fundamental_theorem_of_calculus_for_contour_integral}より$n$でない任意の自然数$k$で
	\begin{align}
		\int_{\gamma} \frac{1}{z^{k+1-n}}\ dz = 0
	\end{align}
	が成り立ち,また任意の自然数$k$で
	\begin{align}
		\int_{\gamma} z^{k+n}\ dz = 0
	\end{align}
	が成り立つので,
	\begin{align}
		\int_{\gamma} f \cdot z^{n}\ dz = p_{n} \cdot \int_{\gamma} \frac{1}{z}\ dz
	\end{align}
	が従う.これで(\refeq{fom:Laurent_series_0_1})が得られた.同様に
	\begin{align}
		\int_{\gamma} \frac{f(z)}{z^{n+1}}\ dz
		&= \int_{\gamma} \sum_{k=0}^{\infty} \frac{p_{k}}{z^{n+k+2}}\ dz
			+ \int_{\gamma} \sum_{k=0}^{\infty} q_{k} \cdot z^{k-n-1}\ dz \\
		&= \sum_{k=0}^{\infty} p_{k} \cdot \int_{\gamma} \frac{1}{z^{n+k+2}}\ dz
			+ \sum_{k=0}^{\infty} q_{k} \cdot \int_{\gamma} z^{k-n-1}\ dz
	\end{align}
	となるが,任意の自然数$k$で
	\begin{align}
		\int_{\gamma} \frac{1}{z^{n+k+2}}\ dz = 0
	\end{align}
	が成り立ち,$n$でない任意の自然数$k$で
	\begin{align}
		\int_{\gamma} z^{k-n-1}\ dz = 0
	\end{align}
	が成り立つので,
	\begin{align}
		\int_{\gamma} \frac{f(z)}{z^{n+1}}\ dz
		= q_{n} \cdot \int_{\gamma} \frac{1}{z}\ dz
	\end{align}
	が従う.これで(\refeq{fom:Laurent_series_0_2})も得られた.ところで
	\begin{align}
		\int_{\gamma} f(z)\ dz = p_{0} \cdot \Ind_{\gamma}(0)
	\end{align}
	が成り立つので,$f$の$\gamma$に関する線積分は$a_{0}$さえ得られれば計算できてしまう.
	
	\begin{screen}
		\begin{dfn}[留数]
			$\Omega$を$\mathscr{O}_{\C}$-開集合とし,$a$を$\Omega$の要素とし,
			$f$を$\Omega \backslash \{a\}$上の正則関数とする.また$R$を
			\begin{align}
				\disc{a}{R} \subset \Omega
			\end{align}
			を満たす正の実数とし,$p$と$q$を複素数列とし,$\disc{a}{R} \backslash \{a\}$の各要素$z$で
			\begin{align}
				f(z) = \sum_{n=0}^{\infty} \frac{p_{n}}{z^{n+1}} + \sum_{n=0}^{\infty} q_{n} \cdot z^{n}
			\end{align}
			が満たされているとする.このとき$p_{0}$を$f$の$a$における{\bf 留数}\index{りゅうすう@留数}{\bf (residue)}と呼び
			\begin{align}
				\Res{f}{a}
			\end{align}
			と書く.
		\end{dfn}
	\end{screen}
	
	
	
	\begin{screen}
		\begin{thm}[留数定理]
		\end{thm}
	\end{screen}
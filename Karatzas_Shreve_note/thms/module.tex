	この節では,$\left(X,\sigma_X\right)$を群とするとき,$X$の要素$x$の逆元を
	\begin{align}
		-x
	\end{align}
	と書く.また$\left(R,\sigma_R,\mu_R\right)$を環とするとき,$R$の要素$x$の$\sigma_R$に関する逆元を
	\begin{align}
		-x
	\end{align}
	と書き,$\mu_R$に関する逆元を
	\begin{align}
		x^{-1}
	\end{align}
	と書く.逆元の書き方が同じだが混乱する危険はおそらく無い.
	
	\begin{screen}
		\begin{dfn}[加群]
			$(X,\sigma_X)$をAbel群とし,$(R,\sigma_R,\mu_R)$を環とし,$1_R$を
			$\mu_R$に関する単位元とする.また$s$を
			\begin{align}
				s:R \times X \longrightarrow X
			\end{align}
			なる写像で
			\begin{itemize}
				\item $\forall \alpha,\beta \in R\, \forall x \in X\, \left(\, s(\sigma_R(\alpha,\beta),x)
					= \sigma_X(s(\alpha,x),s(\beta,x))\, \right)$
				\item $\forall \alpha \in R\, \forall x,y \in X\, \left(\, s(\alpha,\sigma_X(x,y))
					= \sigma_X(s(\alpha,x),s(\alpha,y))\, \right)$
				\item $\forall \alpha,\beta \in R\, \forall x \in X\, \left(\, s(\mu_R(\alpha,\beta),x)
					= s(\alpha,s(\beta,x))\, \right)$
				\item $\forall x \in X\, \left(\, s(1_R,x) = x\, \right)$
			\end{itemize}
			を満たすものとする.このとき
			\begin{align}
				\left((X,\sigma_X),(R,\sigma_R,\mu_R),s\right)
			\end{align}
			を{\bf 加群}\index{かぐん@加群}{\bf (module)}と呼ぶ.
			$(R,\sigma_R,\mu_R)$が体である場合はこの3つ組を{\bf 線型空間}\index{せんけいくうかん@線型空間}{\bf (vector space)}と呼ぶ.
		\end{dfn}
	\end{screen}
	
	$(X,\sigma_X)$と$(R,\sigma_R,\mu_R)$がそれぞれ$(\R,+)$と$(\R,+,\bullet)$であるとき,$s$も$\bullet$とすれば
	$s$についての規則は
	\begin{itemize}
		\item $\forall \alpha,\beta \in \R\, \forall x \in \R\, \left(\, (\alpha + \beta) \cdot x
			= \alpha \cdot x + \beta \cdot x\, \right)$
		\item $\forall \alpha \in \R\, \forall x,y \in \R\, \left(\, \alpha \cdot (x+y)
			= \alpha \cdot x + \alpha \cdot y\, \right)$
		\item $\forall \alpha,\beta \in \R\, \forall x \in \R\, \left(\, (\alpha \cdot \beta) \cdot x
			= \alpha \cdot (\beta \cdot x)\, \right)$
		\item $\forall x \in \R\, \left(\, 1 \cdot x = x\, \right)$
	\end{itemize}
	と見やすくなる.$s$は{\bf スカラ倍}\index{すからばい@スカラ倍}{\bf (scalar multiplication)}と呼ばれるが,
	異なる算法の間に分配則と結合則を既定しているに過ぎない.
	
	\begin{screen}
		\begin{thm}[$0$倍すればゼロ]\label{thm:zero_multiplication_is_zero}
			$\left((X,\sigma_X),(R,\sigma_R,\mu_R),s\right)$を加群とし,
			$0_X$を$(X,\sigma_X)$の単位元とし,$0_R$を$\sigma_R$に関する単位元とする.このとき,$X$の任意の要素$x$に対して
			\begin{align}
				s\left(0_R,x\right) = 0_X.
			\end{align}
		\end{thm}
	\end{screen}
	
	\begin{sketch}
		$0_R$は$\sigma_R$に関する単位元なので
		\begin{align}
			\sigma_R\left(0_R,0_R\right) = 0_R
		\end{align}
		が成り立つ.ゆえに$x$を$X$の要素とすると
		\begin{align}
			s\left(0_R,x\right) &= s\left(\sigma_R\left(0_R,0_R\right),x\right) \\
			&= \sigma_X\left(s\left(0_R,x\right),s\left(0_R,x\right)\right)
		\end{align}
		が成り立つ.ゆえに
		\begin{align}
			0_X &= \sigma_X\left(-s\left(0_R,x\right),s\left(0_R,x\right)\right) \\
			&= \sigma_X\left(-s\left(0_R,x\right),\sigma_X\left(s\left(0_R,x\right),s\left(0_R,x\right)\right)\right) \\
			&= \sigma_X\left(\sigma_X\left(-s\left(0_R,x\right),s\left(0_R,x\right)\right),s\left(0_R,x\right)\right) \\
			&= \sigma_X\left(0_X,s\left(0_R,x\right)\right) \\
			&= s\left(0_R,x\right)
		\end{align}
		が成り立つ.
		\QED
	\end{sketch}
	
	\begin{screen}
		\begin{thm}[ゼロには何を掛けてもゼロ]\label{thm:zero_multiplied_is_zero}
			$\left((X,\sigma_X),(R,\sigma_R,\mu_R),s\right)$を加群とし,
			$0_X$を$(X,\sigma_X)$の単位元とする.このとき,$R$の任意の要素$r$に対して
			\begin{align}
				s\left(r,0_X\right) = 0_X.
			\end{align}
		\end{thm}
	\end{screen}
	
	\begin{sketch}
		$0_X$は$(X,\sigma_X)$の単位元なので
		\begin{align}
			\sigma_X\left(0_X,0_X\right) = 0_X
		\end{align}
		が成り立つ.ゆえに$r$を$R$の要素とすると
		\begin{align}
			s\left(r,0_X\right) &= s\left(r,\sigma_X\left(0_X,0_X\right)\right) \\
			&= \sigma_X\left(s\left(r,0_X\right),s\left(r,0_X\right)\right)
		\end{align}
		が成り立つ.ゆえに
		\begin{align}
			0_X &= \sigma_X\left(-s\left(r,0_X\right),s\left(r,0_X\right)\right) \\
			&= \sigma_X\left(-s\left(r,0_X\right),\sigma_X\left(s\left(r,0_X\right),s\left(r,0_X\right)\right)\right) \\
			&= \sigma_X\left(\sigma_X\left(-s\left(r,0_X\right),s\left(r,0_X\right)\right),s\left(r,0_X\right)\right) \\
			&= \sigma_X\left(0_X,s\left(r,0_X\right)\right) \\
			&= s\left(r,0_X\right)
		\end{align}
		が成り立つ.
		\QED
	\end{sketch}
	
	\begin{screen}
		\begin{thm}[逆元は$-1$倍に等しい]\label{thm:inverse_element_equals_to_its_minus}
			$\left((X,\sigma_X),(R,\sigma_R,\mu_R),s\right)$を加群とし,
			$1_R$を$\mu_R$に関する単位元とする.このとき,$X$の任意の要素$x$に対して
			\begin{align}
				-x = s\left(-1_R,x\right).
			\end{align}
		\end{thm}
	\end{screen}
	
	\begin{sketch}
		$\sigma_R$に関する単位元を
		\begin{align}
			0_R
		\end{align}
		と書く.$x$を$X$の要素とすると,
		\begin{align}
			x = s\left(1_R,x\right)
		\end{align}
		が成り立つので
		\begin{align}
			\sigma_X\left(x,s\left(-1_R,x\right)\right) &= \sigma_X\left(s\left(1_R,x\right),s\left(-1_R,x\right)\right) \\
			&= s\left(\sigma_R\left(1_R,-1_R\right),x\right) \\
			&= s\left(0_R,x\right) \\
			&= x
		\end{align}
		が成り立つ.同様にして
		\begin{align}
			\sigma_X\left(s\left(-1_R,x\right),x\right) = x
		\end{align}
		も成り立つので
		\begin{align}
			-x = s\left(-1_R,x\right)
		\end{align}
		が従う.
		\QED
	\end{sketch}
	
	\begin{screen}
		\begin{dfn}[加群準同型]
			
		\end{dfn}
	\end{screen}
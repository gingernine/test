	\begin{screen}
		\begin{dfn}[加群]
			$(X,\sigma_X)$をAbel群とし,$(R,\sigma_R,\mu_R)$を環とし,$\epsilon_R$を
			$(R,\sigma_R,\mu_R)$の零元とする.また$s$を
			\begin{align}
				s:R \times X \longrightarrow X
			\end{align}
			なる写像で
			\begin{itemize}
				\item $\forall \alpha,\beta \in R\, \forall x \in X\, \left(\, s(\sigma_R(\alpha,\beta),x)
					= \sigma_X(s(\alpha,x),s(\beta,x))\, \right)$
				\item $\forall \alpha \in R\, \forall x,y \in X\, \left(\, s(\alpha,\sigma_X(x,y))
					= \sigma_X(s(\alpha,x),s(\alpha,y))\, \right)$
				\item $\forall \alpha,\beta \in R\, \forall x \in X\, \left(\, s(\mu_R(\alpha,\beta),x)
					= s(\alpha,s(\beta,x))\, \right)$
				\item $\forall x \in X\, \left(\, s(\epsilon_R,x) = x\, \right)$
			\end{itemize}
			を満たすものとする.このとき
			\begin{align}
				\left((X,\sigma_X),(R,\sigma_R,\mu_R),s\right)
			\end{align}
			を{\bf 加群}\index{かぐん@加群}{\bf (module)}と呼ぶ.
			$(R,\sigma_R,\mu_R)$が体である場合はこの3つ組を{\bf 線型空間}\index{せんけいくうかん@線型空間}{\bf (vector space)}と呼ぶ.
		\end{dfn}
	\end{screen}
	
	$(X,\sigma_X)$と$(R,\sigma_R,\mu_R)$がそれぞれ$(\R,+)$と$(\R,+,\bullet)$であるとき,$s$も$\bullet$とすれば
	$s$についての規則は
	\begin{itemize}
		\item $\forall \alpha,\beta \in \R\, \forall x \in \R\, \left(\, (\alpha + \beta) \cdot x
			= \alpha \cdot x + \beta \cdot x\, \right)$
		\item $\forall \alpha \in \R\, \forall x,y \in \R\, \left(\, \alpha \cdot (x+y)
			= \alpha \cdot x + \alpha \cdot y\, \right)$
		\item $\forall \alpha,\beta \in \R\, \forall x \in \R\, \left(\, (\alpha \cdot \beta) \cdot x
			= \alpha \cdot (\beta \cdot x)\, \right)$
		\item $\forall x \in \R\, \left(\, 1 \cdot x = x\, \right)$
	\end{itemize}
	と見やすくなる.$s$は{\bf スカラ倍}\index{すからばい@スカラ倍}{\bf (scalar multiplication)}と呼ばれるが,
	異なる算法の間に分配則と結合則を既定しているに過ぎない.
	
	\begin{screen}
		\begin{dfn}[加群準同型]
			
		\end{dfn}
	\end{screen}
\subsection{近傍}
	\begin{screen}
		\begin{dfn}[近傍]
			$(S,\mathscr{O})$を位相空間とし,$S$は空でないとし,$x$を$S$の要素とする.
			$u$を$S$の部分集合とするとき,
			\begin{align}
				x \in o \subset u
			\end{align}
			を満たす$\mathscr{O}$-開集合$o$が取れるならば$u$を$x$の$\mathscr{O}$-{\bf 近傍}
			\index{きんぼう@近傍}{\bf (neighborhood)}と呼ぶ.
			$x$の$\mathscr{O}$-近傍の全体を$\mathscr{O}$-{\bf 近傍系}
			\index{きんぼうけい@近傍系}{\bf (neighborhood system)}と呼ぶが,
			これを$\mathscr{V}$と書くとき,$\mathscr{V}$の空でない部分集合$\mathscr{U}$で
			\begin{align}
				\forall v \in \mathscr{V}\, \exists u \in \mathscr{U}\,
				(\, u \subset v\, )
			\end{align}
			を満たすものを$x$の$\mathscr{O}$-{\bf 基本近傍系}
			\index{きほんきんぼうけい@基本近傍系}{\bf (fundamental system of neighbourhoods)}と呼ぶ.
		\end{dfn}
	\end{screen}
	
	上の定義において,$S$は$x$の$\mathscr{O}$-近傍であるから$\mathscr{V}$は空ではない.
	また$\mathscr{V}$自体が$\mathscr{O}$-基本近傍系であるから
	$\mathscr{O}$-基本近傍系は少なくとも一つは取れる.ゆえに,{\bf $S$上の写像で,
	$S$の要素に対してその$\mathscr{O}$-基本近傍系を対応させるものが取れる(選択公理).}
	ちなみに$\mathscr{V}$とは
	\begin{align}
		\Set{v}{v \subset S \wedge \exists o \in \mathscr{O}\, (\, x \in o \wedge o \subset v\, )}
	\end{align}
	なる集合であり,$x$の$\mathscr{O}$-基本近傍系の全体は
	\begin{align}
		\{\, u \mid \quad &\forall w \in u\, [\, w \subset S \wedge \exists o \in \mathscr{O}\, (\, x \in o \wedge o \subset w\, )\, ] \wedge \\
		&\forall v\, \left[\, v \subset S \wedge \exists o \in \mathscr{O}\, (\, x \in o \wedge o \subset v\, )
		\Longrightarrow \exists w \in u\, (\, v \subset w\, )\, \right]\, \}
	\end{align}
	である.また$S$の要素に対してその$\mathscr{O}$-基本近傍系の全体を対応させる写像は
	\begin{align}
		\{\, (x,y) \mid \quad x \in S \wedge
		\forall u\, [\, u \in y \Longleftrightarrow 
		&\forall w \in u\, [\, w \subset S \wedge \exists o \in \mathscr{O}\, (\, x \in o \wedge o \subset w\, )\, ] \wedge \\
		&\forall v\, \left[\, v \subset S \wedge \exists o \in \mathscr{O}\, (\, x \in o \wedge o \subset v\, )
		\Longrightarrow \exists w \in u\, (\, v \subset w\, )\, \right]\, ]\, \}
	\end{align}
	である.
	
	\begin{screen}
		\begin{thm}[基本近傍系による開集合の特徴づけ]
		\label{thm:local_base_defines_open_sets}
			$(S,\mathscr{O})$を位相空間とし,$S$は空でないとし,$u$を$S$上の写像で,
			$S$の要素に対してその$\mathscr{O}$-基本近傍系を対応させるものとする.
			このとき
			\begin{align}
				\forall o\, \left[\, o \in \mathscr{O} \Longleftrightarrow
				o \subset S \wedge \forall x \in o\, \exists w \in u_{x}\, (\, w \subset o\, )\, \right].
			\end{align}
		\end{thm}
	\end{screen}
	
	\begin{sketch}
		いま
		\begin{align}
			o \in \mathscr{O}
		\end{align}
		が成り立っているとする.この下で$x$を集合として
		\begin{align}
			x \in o
		\end{align}
		が成り立っているとすれば,$o$は$x$の$\mathscr{O}$-近傍であるから
		\begin{align}
			w \subset o
		\end{align}
		を満たす$u_{x}$の要素$w$が取れる.ゆえに
		\begin{align}
			o \in \mathscr{O} \Longrightarrow
			\forall x \in o\, \exists w \in u_{x}\, (\, w \subset o\, )
		\end{align}
		が得られた.次に
		\begin{align}
			\forall x \in o\, \exists w \in u_{x}\, (\, w \subset o\, )
		\end{align}
		が成り立っているとする.この下で$x$を集合として
		\begin{align}
			x \in o
		\end{align}
		が成り立っているとすれば
		\begin{align}
			w \subset o
		\end{align}
		を満たす$u_{x}$の要素$w$が取れるが,このとき
		\begin{align}
			x \in v \subset w
		\end{align}
		を満たす$\mathscr{O}$-開集合$v$が取れるので
		\begin{align}
			x \in o^{\mathrm{o}}
		\end{align}
		が成立する.ただし$o^{\mathrm{o}}$は$o$の$\mathscr{O}$-開核である.ゆえに
		\begin{align}
			o = o^{\mathrm{o}}
		\end{align}
		が成り立つので
		\begin{align}
			o \in \mathscr{O}
		\end{align}
		が従う.以上で
		\begin{align}
			\forall x \in o\, \exists w \in u_{x}\, (\, w \subset o\, )
			\Longrightarrow o \in \mathscr{O}
		\end{align}
		も得られた.
		\QED
	\end{sketch}
	
	次に基本近傍系の持つ性質を分析する.$(S,\mathscr{O})$を位相空間とし,$S$は空でないとする.また
	$u$を$S$上の写像で,$S$の要素に対してその$\mathscr{O}$-基本近傍系を対応させるものとする.
	このとき$u$は以下の性質を持つ:
	\begin{description}
		\item[(LB1)] $u_{x}$は空ではなく,また$u_{x}$の任意の要素は$x$を要素に持つ:
			\begin{align}
				&\forall x \in S\, (\, u_{x} \neq \emptyset\, ) \wedge\\
				&\forall x \in S\, \forall w \in u_{x}\, (\, x \in w\, ).
			\end{align}

		\item[(LB2)] $o$と$w$を$u_{x}$の要素とすれば,$v \subset o \cap w$を満たす
			$u_{x}$の要素$v$が取れる:
			\begin{align}
				\forall x \in S\, 
				\left[\, \forall o,w \in u_{x}\, \exists w \in u_{x}\,
				\forall t\, \left(\, t \in v \Longrightarrow t \in o \wedge t \in w\, \right)\, \right].
			\end{align}
			
		\item[(LB3)] $w$を$u_{x}$の要素とすれば,$u_{x}$の要素$v$が取れて
			$v \subset w$が成り立ち,かつ$v$の任意の要素$y$に対して
			$o \subset w$を満たす$u(y)$の要素$o$が取れる:
			\begin{align}
				\forall x \in S\, \forall w \in u_{x}\, \exists v \in u_{x}\,
				\left[\, v \subset w \wedge \left(\, \forall y \in v\,
				\exists o \in u(y)\, (\, o \subset w\, )\, \right)\, \right].
			\end{align}
	\end{description}
	
	\begin{screen}
		\begin{thm}[与えられた集合を基本近傍系とする位相の生成]
		\label{thm:a_local_base_restores_the_topology}
			$S$を空でない集合とし,$u$を(LB1)(LB2)(LB3)を満たす$S$上の写像とするとき,
			\begin{align}
				\mathscr{O} \defeq
				\Set{o}{o \subset S \wedge \forall x \in o\, \exists w \in u_{x}\, (\, w \subset o\, )}
			\end{align}
			で定める$\mathscr{O}$は$S$上の位相構造であって,
			$S$の各要素$x$に対して$u_{x}$は$x$の$\mathscr{O}$-基本近傍系である.
		\end{thm}
	\end{screen}
	
	\begin{prf}\mbox{}
		\begin{description}
			\item[(1)] 任意の$U \in \mathscr{U}(x)$は$x$の近傍であるから
				$(LB1)$が満たされる.また$U,V \in \mathscr{U}(x)$に対し
				\begin{align}
					x \in U^{\mathrm{o}} \cap V^{\mathrm{o}} = (U \cap V)^{\mathrm{o}}
				\end{align}
				となるから$U \cap V$は$x$の近傍であり(LB2)も従う.
				任意に$U \in \mathscr{U}(x)$を取れば,
				$U^{\mathrm{o}}$は$x$の開近傍であるから
				或る$V \in \mathscr{U}(x)$で$V \subset U^{\mathrm{o}}$
				を満たすものが存在する.このとき任意の$y \in V$に対し
				$U^{\mathrm{o}}$は$y$の開近傍となるから
				\begin{align}
					W_y \subset U^{\mathrm{o}} \subset U
				\end{align}
				を満たす$W_y \in \mathscr{U}(y)$が取れる.従って(LB3)も得られる.
			
			\item[(2)] 
				$\mathscr{U}(x)$は空ではないから$S \in \mathscr{O}$となる.
				また$O_1,O_2 \in \mathscr{O}$を取れば,
				任意の$x \in O_1 \cap O_2$に対し
				\begin{align}
					x \in U_1 \subset O_1,
					\quad x \in U_2 \subset O_2
				\end{align}
				を満たす$U_1,U_2 \in \mathscr{U}(x)$が存在し,
				(LB2)より或る$U_3 \in \mathscr{U}(x)$に対して
				\begin{align}
					U_3 \subset U_1 \cap U_2 \subset O_1 \cap O_2
				\end{align}
				が成り立つから$O_1 \cap O_2 \in \mathscr{O}$となる.
				任意に$\mathscr{G} \subset \mathscr{O}$を取れば
				任意の$x \in \bigcup \mathscr{G}$は或る$G \in \mathscr{G}$の点であるから,
				\begin{align}
					U \subset G \subset \bigcup \mathscr{G}
				\end{align}
				を満たす$U \in \mathscr{U}(x)$が存在し$\bigcup \mathscr{G} \in \mathscr{O}$が従う.
				よって$\mathscr{O}$は位相である.
				ところで,任意の$U \in \mathscr{U}(x)$に対し
				\begin{align}
					U^{\mathrm{o}} = 
					\Set{y \in U}{\mbox{或る$W_y \in \mathscr{U}(y)$が存在して
					$W_y \subset U$となる}} \eqqcolon \tilde{U}
					\label{eq:thm_a_local_base_restores_the_topology_0}
				\end{align}
				が成立する.実際$\mathscr{O}$の定義より
				\begin{align}
					y \in U^{\mathrm{o}} \quad \Longrightarrow \quad
					\mbox{或る$W_y \in \mathscr{U}(y)$で
					$W_y \subset U^{\mathrm{o}}$}
				\end{align}
				となるから$U^{\mathrm{o}}\subset\tilde{U}$が従い,
				逆に$y \in \tilde{U}$については,
				(\refeq{eq:thm_a_local_base_restores_the_topology_0})の$W_y$に対して
				(LB3)より或る$X_y \in \mathscr{U}(y)$が
				\begin{align}
					X_y \subset W_y,\quad 
					z \in X_y \ \Longrightarrow \
					\mbox{或る$Y_z \in \mathscr{U}(z)$で$Y_z \subset X_y \subset U$}
				\end{align}
				を満たすから$X_y \subset \tilde{U}$が従う.
				すなわち$\tilde{U}$は開集合であり,$U^{\mathrm{o}}\subset\tilde{U}$
				と併せて(\refeq{eq:thm_a_local_base_restores_the_topology_0})
				を得る.(LB3)より
				\begin{align}
					V \subset U, \quad y \in V \ \Longrightarrow \
					\mbox{或る$W_y \in \mathscr{U}(y)$で$W_y \subset U$}
				\end{align}
				を満たす$V \in \mathscr{U}(x)$が存在し,(LB1)と併せて
				\begin{align}
					x \in V \subset \tilde{U} = U^{\mathrm{o}}
				\end{align}
				が成り立つから任意の$U \in \mathscr{U}(x)$は$x$の近傍である.
				そして$W$を$x$の任意の近傍とすれば,
				$\mathscr{O}$の定め方より或る$U \in \mathscr{U}(x)$が
				$U \subset W^{\mathrm{o}}$を満たすから
				$\mathscr{U}(x)$は$x$の基本近傍系である.
				\QED
		\end{description}
	\end{prf}
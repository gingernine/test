\subsection{近傍}
	\begin{screen}
		\begin{dfn}[近傍]
			$(S,\mathscr{O})$を位相空間とし,$S$は空でないとし,$x$を$S$の要素とする.
			$u$を$S$の部分集合とするとき,
			\begin{align}
				x \in o \subset u
			\end{align}
			を満たす$\mathscr{O}$-開集合$o$が取れるならば$u$を$x$の$\mathscr{O}$-{\bf 近傍}
			\index{きんぼう@近傍}{\bf (neighborhood)}と呼ぶ.
			$x$の$\mathscr{O}$-近傍の全体を$\mathscr{O}$-{\bf 近傍系}
			\index{きんぼうけい@近傍系}{\bf (neighborhood system)}と呼ぶが,
			これを$\mathcal{V}$と書くとき,$\mathcal{V}$の空でない部分集合$\mathcal{U}$で
			\begin{align}
				\forall v \in \mathcal{V}\, \exists u \in \mathcal{U}\,
				(\, u \subset v\, )
			\end{align}
			を満たすものを$x$の$\mathscr{O}$-{\bf 基本近傍系}
			\index{きほんきんぼうけい@基本近傍系}{\bf (fundamental system of neighbourhoods)}と呼ぶ.
		\end{dfn}
	\end{screen}
	
	上の定義において,$S$は$x$の$\mathscr{O}$-近傍であるから$\mathcal{V}$は空ではない.
	また$\mathcal{V}$自体が$\mathscr{O}$-基本近傍系であるから
	$\mathscr{O}$-基本近傍系は少なくとも一つは取れる.ゆえに,{\bf $S$上の写像で,
	$S$の要素に対してその$\mathscr{O}$-基本近傍系を対応させるものが取れる(選択公理).}
	ちなみに$\mathcal{V}$とは
	\begin{align}
		\Set{v}{v \subset S \wedge \exists o \in \mathscr{O}\, (\, x \in o \wedge o \subset v\, )}
	\end{align}
	なる集合であり,$x$の$\mathscr{O}$-基本近傍系の全体は
	\begin{align}
		\left\{\, \mathcal{U} \mid \quad \right.
		&\mathcal{U} \neq \emptyset \wedge \\
		&\forall u \in \mathcal{U}\, [\, u \subset S \wedge \exists o \in \mathscr{O}\, (\, x \in o \wedge o \subset u\, )\, ] \wedge \\
		&\forall v\, \left.\left[\, v \subset S \wedge \exists o \in \mathscr{O}\, (\, x \in o \wedge o \subset v\, )
		\Longrightarrow \exists u \in \mathcal{U}\, (\, u \subset v\, )\, \right]\, \right\}
	\end{align}
	である.また$S$の要素に対してその$\mathscr{O}$-基本近傍系の全体を対応させる写像は
	\begin{align}
		\left\{\, (x,y) \mid \quad x \in S \wedge
		\forall \mathcal{U}\, \left[\, \mathcal{U} \in y \Longleftrightarrow \right.\right. 
		&\mathcal{U} \neq \emptyset \wedge \\
		&\forall u \in \mathcal{U}\, \left[\, u \subset S \wedge \exists o \in \mathscr{O}\, (\, x \in o \wedge o \subset u\, )\, \right] \wedge \\
		&\forall v\, \left.\left.\left[\, v \subset S \wedge \exists o \in \mathscr{O}\, (\, x \in o \wedge o \subset v\, )
		\Longrightarrow \exists u \in \mathcal{U}\, (\, u \subset v\, )\, \right]\, \right]\, \right\}
	\end{align}
	である.
	
	\begin{screen}
		\begin{thm}[基本近傍系は位相を復元する]
		\label{thm:local_base_defines_open_sets}
			$(S,\mathscr{O})$を位相空間とし,$S$は空でないとし,$\mathcal{U}$を$S$上の写像で,
			$S$の要素に対してその$\mathscr{O}$-基本近傍系を対応させるものとする.
			このとき
			\begin{align}
				\forall o\, \left[\, o \in \mathscr{O} \Longleftrightarrow
				o \subset S \wedge \forall x \in o\, \exists u \in \mathcal{U}_{x}\, (\, u \subset o\, )\, \right].
			\end{align}
		\end{thm}
	\end{screen}
	
	\begin{sketch}
		いま
		\begin{align}
			o \in \mathscr{O}
		\end{align}
		が成り立っているとする.この下で$x$を集合として
		\begin{align}
			x \in o
		\end{align}
		が成り立っているとすれば,$o$は$x$の$\mathscr{O}$-近傍であるから
		\begin{align}
			u \subset o
		\end{align}
		を満たす$\mathcal{U}_{x}$の要素$u$が取れる.ゆえに
		\begin{align}
			o \in \mathscr{O} \Longrightarrow
			\forall x \in o\, \exists u \in \mathcal{U}_{x}\, (\, u \subset o\, )
		\end{align}
		が得られた.次に
		\begin{align}
			\forall x \in o\, \exists u \in \mathcal{U}_{x}\, (\, u \subset o\, )
		\end{align}
		が成り立っているとする.この下で$x$を集合として
		\begin{align}
			x \in o
		\end{align}
		が成り立っているとすれば
		\begin{align}
			u \subset o
		\end{align}
		を満たす$\mathcal{U}_{x}$の要素$u$が取れるが,このとき
		\begin{align}
			x \in w \subset u
		\end{align}
		を満たす$\mathscr{O}$-開集合$w$が取れるので
		\begin{align}
			x \in o^{\mathrm{o}}
		\end{align}
		が成立する.ただし$o^{\mathrm{o}}$は$o$の$\mathscr{O}$-開核である.ゆえに
		\begin{align}
			o = o^{\mathrm{o}}
		\end{align}
		が成り立つので
		\begin{align}
			o \in \mathscr{O}
		\end{align}
		が従う.以上で
		\begin{align}
			\forall x \in o\, \exists u \in \mathcal{U}_{x}\, (\, u \subset o\, )
			\Longrightarrow o \in \mathscr{O}
		\end{align}
		も得られた.
		\QED
	\end{sketch}
	
	次に基本近傍系の持つ性質を分析する.$(S,\mathscr{O})$を位相空間とし,$S$は空でないとする.また
	$\mathcal{U}$を$S$上の写像で,$S$の要素に対してその$\mathscr{O}$-基本近傍系を対応させるものとする.
	このとき$\mathcal{U}$は以下の性質を持つ:
	\begin{description}
		\item[(LB1)] $\mathcal{U}_{x}$は空ではなく,また$\mathcal{U}_{x}$の任意の要素は$x$を要素に持つ$S$の部分集合である:
			\begin{align}
				\forall x \in S\, \left[\, \mathcal{U}_{x} \neq \emptyset 
				\wedge \forall u \in \mathcal{U}_{x}\, (\, x \in u \wedge u \subset S\, )\, \right].
			\end{align}

		\item[(LB2)] $u$と$v$を$\mathcal{U}_{x}$の要素とすれば,$w \subset u \cap v$を満たす
			$\mathcal{U}_{x}$の要素$w$が取れる:
			\begin{align}
				\forall x \in S\, 
				\left[\, \forall u,v \in \mathcal{U}_{x}\, \exists w \in \mathcal{U}_{x}\,
				\forall t\, \left(\, t \in w \Longrightarrow t \in u \wedge t \in v\, \right)\, \right].
			\end{align}
			
		\item[(LB3)] $u$を$\mathcal{U}_{x}$の要素とすれば,$v \subset u$なる
			$\mathcal{U}_{x}$の要素$v$が取れて,$v$の任意の要素$y$に対して
			$w \subset v$を満たす$\mathcal{U}_{y}$の要素$w$が取れる:
			\begin{align}
				\forall x \in S\, \forall u \in \mathcal{U}_{x}\, \exists v \in \mathcal{U}_{x}\,
				\left[\, v \subset u \wedge \left(\, \forall y \in v\,
				\exists w \in \mathcal{U}_{y}\, (\, w \subset u\, )\, \right)\, \right].
			\end{align}
	\end{description}
	
	任意の$U \in \mathcal{U}(x)$は$x$の近傍であるから
				$(LB1)$が満たされる.また$U,V \in \mathcal{U}(x)$に対し
				\begin{align}
					x \in U^{\mathrm{o}} \cap V^{\mathrm{o}} = (U \cap V)^{\mathrm{o}}
				\end{align}
				となるから$U \cap V$は$x$の近傍であり(LB2)も従う.
				任意に$U \in \mathcal{U}(x)$を取れば,
				$U^{\mathrm{o}}$は$x$の開近傍であるから
				或る$V \in \mathcal{U}(x)$で$V \subset U^{\mathrm{o}}$
				を満たすものが存在する.このとき任意の$y \in V$に対し
				$U^{\mathrm{o}}$は$y$の開近傍となるから
				\begin{align}
					W_y \subset U^{\mathrm{o}} \subset U
				\end{align}
				を満たす$W_y \in \mathcal{U}(y)$が取れる.従って(LB3)も得られる.
				
	\begin{screen}
		\begin{thm}[与えられたシステムを基本近傍系とする位相の生成]
		\label{thm:a_local_base_restores_the_topology}
			$S$を空でない集合とし,$\mathcal{U}$を(LB1)(LB2)(LB3)を満たす$S$上の写像とするとき,
			\begin{align}
				\mathscr{O} \defeq
				\Set{o}{o \subset S \wedge \forall x \in o\, \exists u \in \mathcal{U}_{x}\, (\, u \subset o\, )}
			\end{align}
			で定める$\mathscr{O}$は$S$上の位相構造であって,
			$S$の各要素$x$に対して$\mathcal{U}_{x}$は$x$の$\mathscr{O}$-基本近傍系である.
		\end{thm}
	\end{screen}
	
	\begin{prf}\mbox{}
		\begin{description}
			\item[第一段]
				$\mathscr{O}$が位相であることを示す.
				\begin{itemize}
					\item $x$を$S$の要素$x$にとすれば,$\mathcal{U}_{x}$は空でないので
						その要素$u$が取れて,(LB1)より
						\begin{align}
							u \subset S
						\end{align}
						が成り立つ.つまり$S$は
						\begin{align}
							\forall x \in S\, \exists u \in \mathcal{U}_{x}\, (\, u \subset S\, )
						\end{align}
						を満たすので
						\begin{align}
							S \in \mathscr{O}
						\end{align}
						が従う.また空虚な真より
						\begin{align}
							\forall x \in \emptyset\, \exists u \in \mathcal{U}_{x}\, (\, u \subset \emptyset\, )
						\end{align}
						が成り立つので
						\begin{align}
							\emptyset \in \mathscr{O}
						\end{align}
						も成り立つ.
						
					\item $a$と$b$を$\mathscr{O}$の要素とし,$x$を$a \cap b$の要素とする.このとき
						\begin{align}
							u \subset a
						\end{align}
						を満たす$\mathcal{U}_{x}$の要素$u$と
						\begin{align}
							v \subset b
						\end{align}
						を満たす$\mathcal{U}_{x}$の要素$v$が取れるが,(LB2)より
						\begin{align}
							w \subset u \cap v
						\end{align}
						を満たす$\mathcal{U}_{x}$の要素$w$が取れて
						\begin{align}
							w \subset a \cap b
						\end{align}
						が成立するので
						\begin{align}
							a \cap b \in \mathscr{O}
						\end{align}
						が従う.
						
					\item $\mathscr{A}$を$\mathscr{O}$の部分集合とする.$x$を$\bigcup \mathscr{A}$の要素とすれば
						\begin{align}
							x \in a
						\end{align}
						を満たす$\mathscr{A}$の要素$a$が取れて,
						\begin{align}
							u \subset a
						\end{align}
						を満たす$\mathcal{U}_{x}$の要素$u$が取れる.このとき
						\begin{align}
							u \subset \bigcup \mathscr{A}
						\end{align}
						が成り立つ.つまり
						\begin{align}
							\forall x \in \bigcup \mathscr{A}\, \exists u \in \mathcal{U}_{x}\, \left(\, u \subset \bigcup \mathscr{A}\, \right)
						\end{align}
						が成立する.ゆえに
						\begin{align}
							\bigcup \mathscr{A} \in \mathscr{O}
						\end{align}
						である.
				\end{itemize}
				
			\item[第二段]
				$x$を$S$の要素として,$\mathcal{U}_x$が$x$の$\mathscr{O}$-基本近傍系であることを示す.
				$u$を$\mathcal{U}_{x}$の要素とするとき,
				\begin{align}
					v \defeq \Set{y \in u}{\exists w \in \mathcal{U}_{y}\, (\, w \subset u\, )}
				\end{align}
				により定める$v$は$\mathscr{O}$-開集合である.実際,$y$を$v$の要素とすれば
				\begin{align}
					w \subset u
				\end{align}
				を満たす$\mathcal{U}_{y}$の要素$w$が取れるが,このとき(LB3)より
				\begin{align}
					a \subset w
				\end{align}
				かつ
				\begin{align}
					\forall t \in a\, \exists b \in \mathcal{U}_{t}\, (\, b \subset w\, )
				\end{align}
				を満たす$\mathcal{U}_{y}$の要素$a$が取れる.$t$を$a$の要素とすれば
				\begin{align}
					\exists b \in \mathcal{U}_{t}\, (\, b \subset u\, )
				\end{align}
				が成り立つので
				\begin{align}
					t \in v
				\end{align}
				が成り立つ.つまり
				\begin{align}
					a \subset v
				\end{align}
				が成り立つ.以上より
				\begin{align}
					\forall y \in v\, \exists a \in \mathcal{U}_{y}\, (\, a \subset v\, )
				\end{align}
				が成り立つので,$v$は$\mathscr{O}$-開集合である.そして
				\begin{align}
					x \in v \wedge v \subset u
				\end{align}
				が満たされるので$u$は$x$の$\mathscr{O}$-近傍である.
				$c$を$x$の近傍とすれば
				\begin{align}
					x \in o \wedge o \subset c
				\end{align}
				を満たす$\mathscr{O}$-開集合$o$が取れるが,開集合の定義より
				\begin{align}
					u \subset o
				\end{align}
				を満たす$\mathcal{U}_{x}$の要素$u$が取れる.ゆえに$\mathcal{U}_{x}$は$x$の$\mathscr{O}$-基本近傍系である.
				\QED
		\end{description}
	\end{prf}
\subsection{有限加法的測度の拡張}
		\begin{screen}
			\begin{thm}[有限加法的な正値測度空間の生成]\label{thm:forming_finitely_additive_class}
				$X$を集合,$\mathcal{A}$を集合$X$上の乗法族で$X$を含むものとする.
				\begin{align}
					\mathcal{B} \coloneqq \Set{\sum_{i=1}^n I_i}{I_i \in \mathcal{A},\ n=1,2,\cdots}
				\end{align}
				は$X\backslash I \in \mathcal{B},\ (\forall I \in \mathcal{A})$のとき
				$X$上の有限加法族となる.
				また$I \in \mathcal{A}$が$I = \sum_{j=1}^m J_j,\ (J_j \in \mathcal{A},\ m \in \N)$
				で表されるとき$m(I) = \sum_{j=1}^m m(J_j)$を満たすような
				写像$m:\mathcal{A} \longrightarrow [0,\infty],\ (m(\emptyset)=0)$が与えられれば,
				\begin{align}
					\mu(B) \coloneqq \sum_{i=1}^n m(I_i),
					\quad (B=I_1 + \cdots + I_n \in \mathcal{B})
				\end{align}
				で定める$\mu$はwell-derinedであり$\mathcal{B}$の上の有限加法的な正値測度となる.
			\end{thm}
		\end{screen}
		
		\begin{prf} $X \backslash I \in \mathcal{B},\ (\forall I \in \mathcal{A})$及び
			$m:\mathcal{A} \longrightarrow [0,\infty]$が与えられたとする.このとき
			$\emptyset = X \backslash X \in \mathcal{B}$より$\emptyset \in \mathcal{A}$である.
			\begin{description}
				\item[第一段]
					$\mathcal{B}$が有限加法族であることを示す.
					$\mathcal{A} \subset \mathcal{B}$より$X \in \mathcal{B}$となる.$A,B \in \mathcal{B}$が
					\begin{align}
						A = I_1 + I_2 + \cdots + I_n,
						\quad B = J_1 + J_2 + \cdots + J_m
						\label{eq:thm_forming_finitely_additive_class_4}
					\end{align}
					と表されているとき,$A \cap B = \emptyset$なら
					\begin{align}
						A + B = I_1 + I_2 + \cdots + I_n + J_1 + J_2 + \cdots + J_m \in \mathcal{A}
						\label{eq:thm_forming_finitely_additive_class_1}
					\end{align}
					となり,そうでない場合$I_i \cap J_j \in \mathcal{A}$より
					\begin{align}
						A \cap B = \sum_{i=1}^n\sum_{j=1}^m I_i \cap J_i \in \mathcal{B}
						\label{eq:thm_forming_finitely_additive_class_2}
					\end{align}
					となるから$\mathcal{B}$は交演算で閉じ,$X \backslash I_i \in \mathcal{B}$であるから
					\begin{align}
						X \backslash A = (X \backslash I_1) \cap \cdots \cap (X \backslash I_n) \in \mathcal{B}
						\label{eq:thm_forming_finitely_additive_class_3}
					\end{align}
					が従う.(\refeq{eq:thm_forming_finitely_additive_class_1}),
					(\refeq{eq:thm_forming_finitely_additive_class_2}),
					(\refeq{eq:thm_forming_finitely_additive_class_3})
					より$A \cup B = A + B \cap (X \backslash A) \in \mathcal{B}$が成り立ち,
					$\mathcal{B}$は集合和でも閉じる.
		
				\item[第二段]
					$\mu$がwell-definedかつ有限加法的であることを示す.実際
					(\refeq{eq:thm_forming_finitely_additive_class_4})の$A,B \in \mathcal{B}$に対して,
					$A = B$のとき
					\begin{align}
						\sum_{i=1}^n m(I_i)
						= \sum_{i=1}^n \sum_{j=1}^m m(I_i \cap J_j)
						= \sum_{j=1}^m \sum_{i=1}^n m(I_i \cap J_j)
						= \sum_{j=1}^m m(J_j)
					\end{align}
					が成り立つから$\mu$はwell-definedであり,また$A \cap B = \emptyset$のとき
					\begin{align}
						\mu(A + B)
						=  m(I_1) + \cdots + m(I_n) + m(J_1) + \cdots + m(J_m)
						= \mu(A) + \mu(B)
					\end{align}
					となり$\mu$の有限加法性が出る.$\mu$は$m$の拡張であるから$\mu(\emptyset) = 0$も従う.
					\QED
			\end{description}
		\end{prf}
		
		\begin{screen}
			\begin{dfn}[外測度]\label{def:outer_measure}
				$X$の冪集合$\mathcal{P}(X)$の上で定義される$[0,\infty]$-値写像$\mu$が
				\begin{description}
					\item[(OM1)] $\mu(\emptyset) = 0,\quad 0 \leq \mu(A) \leq \infty,\ (\forall A \subset X)$,
					\item[(OM2) (単調性)] $A \subset B \quad \Longrightarrow \quad \mu(A) \leq \mu(B)$,
					\item[(OM3) (劣加法性)] $\mu\left( \bigcup_{n=1}^\infty A_n \right) \leq \sum_{n=1}^\infty \mu(A_n),\ (A_n \subset X,\ n=1,2,\cdots)$
				\end{description}
				を満たすとき,$\mu$を$X$の外測度(outer measure)と呼ぶ.また$A \subset X$が
				\begin{align}
					\mu(W) = \mu(A \cap W) + \mu(A^c \cap W), \quad (\forall W \in \mathcal{P}(X))
				\end{align}
				を満たすとき$A$は$\mu$-可測集合であるという.
			\end{dfn}
		\end{screen}
		
		\begin{screen}
			\begin{thm}[Caratheodoryの拡張定理]\label{thm:Caratheodory_extension_theorem}
				$\mu$を集合$X$の外測度とし,$\mathcal{B}^*$を$\mu$-可測集合の全体とする.
				このとき$\mathcal{B}^*$は$\mathcal{B}$を含む$\sigma$-加法族であり,
				また$\mu^* \coloneqq \left. \mu \right|_{\mathcal{B}^*}$は$\mathcal{B}^*$の上の完備測度となる.
			\end{thm}
		\end{screen}
		
		\begin{prf}\mbox{}
			\begin{description}
				\item[第一段]
					$\mathcal{B}^*$が$\sigma$-加法族であることを示す.
				
				\item[第二段]
					任意の$B \in \mathcal{B}$が$\mu$-可測であること,つまり任意の$W \subset X$に対し
					\begin{align}
						\mu(W) \geq \mu(W \cap B) + \mu(W \cap B^c)
						\label{eq:thm_Caratheodory_extension_theorem}
					\end{align}
					となることを示せば$\mathcal{B} \subset \mathcal{B}^*$が従う.
					任意の$W \subset X,\ \epsilon > 0$に対し
					\begin{align}
						W \subset \bigcup_{n=1}^\infty B_n,
						\quad \sum_{n=1}^\infty \mu_0(B_n) < \mu(W) + \epsilon
					\end{align}
					を満たす$\{B_n\}_{n=1}^\infty \subset \mathcal{B}$が存在する.
					このとき$W \cap B \subset \bigcup_{n=1}^\infty (B_n \cap B)
					,\ W \cap B^c \subset \bigcup_{n=1}^\infty (B_n \cap B^c)$より
					\begin{align}
						\mu(W \cap B) \leq \sum_{n=1}^\infty \mu_0(B_n \cap B),
						\quad \mu(W \cap B^c) \leq \sum_{n=1}^\infty \mu_0(B_n \cap B^c)
					\end{align}
					となるから
					\begin{align}
						\mu(W) + \epsilon
						&\geq \sum_{n=1}^\infty \mu_0(B_n)
						= \sum_{n=1}^\infty \left\{ \mu_0(B_n \cap B) + \mu_0(B_n \cap B^c) \right\} \\
						&= \sum_{n=1}^\infty \mu_0(B_n \cap B) + \sum_{n=1}^\infty \mu_0(B_n \cap B^c) \\
						&\geq \mu(W \cap B) + \mu(W \cap B^c)
					\end{align}
					が成り立つ.$\epsilon$の任意性より(\refeq{eq:thm_Caratheodory_extension_theorem})が出る.
					
				\item[第三段]
					$\mu^*$が完備測度であることを示す.
					\QED
			\end{description}
		\end{prf}
		
		\begin{screen}
			\begin{thm}[有限加法的な正値測度により定まる外測度]\label{thm:outer_measure_finitely_additive_measure}
				$(X,\mathcal{B},\mu_0)$を有限加法的な正値測度空間とするとき,
				\begin{align}
					\mu(A) \coloneqq \inf{}{}\Set{\sum_{n=1}^\infty \mu_0(B_n)}{B_n \in \mathcal{B},\ A \subset \bigcup_{n=1}^\infty B_n},
					\quad (\forall A \subset X)
				\end{align}
				で定める$\mu$は$X$の外測度である.また$\mu_0$が$\mathcal{B}$の上で完全加法的ならば
				$\mu$は$\mu_0$の拡張となる.
			\end{thm}
		\end{screen}
		
		\begin{prf}\mbox{}
			\begin{description}
				\item[第一段]
					$\mu$が定義\ref{def:outer_measure}の(OM1)(OM2)(OM3)を満たすことを示す.
					$\mu_0$の正値性より$\mu$の正値性が従い,また
					\begin{align}
						\mu(\emptyset) \leq \sum_{n=1}^\infty \mu_0(B_n) = 0,
						\quad (B_n = \emptyset,\ n=1,2,\cdots)
					\end{align}
					となるから(OM1)を得る.$X$の部分集合$A,B,\ (A \subset B)$に対し
					\begin{align}
						\Set{\{B_n\}_{n=1}^\infty}{B_n \in \mathcal{B},\ B \subset \bigcup_{n=1}^\infty B_n}
						\subset \Set{\{B_n\}_{n=1}^\infty}{B_n \in \mathcal{B},\ A \subset \bigcup_{n=1}^\infty B_n}
					\end{align}
					となるから$\mu(A) \leq \mu(B)$が成り立ち(OM2)も得られる.
					
				\item[第二段]
					$\mu_0$が$\mathcal{B}$の上で完全加法的であるとする.
					任意に$B \in \mathcal{B}$を取れば
					$B \subset B \cup \emptyset \cup \emptyset \cup \cdots$より
					\begin{align}
						\mu(B) \leq \mu_0(B)
					\end{align}
					が成り立つ.一方で
					$B \subset \bigcup_{n=1}^\infty B_n$を満たす$\{B_n\}_{n=1}^\infty \subset \mathcal{B}$に対し
					\begin{align}
						B = \sum_{n=1}^\infty \Biggl( B \cap \Biggl( B_n \backslash \bigcup_{k=1}^{n-1}B_k \Biggr) \Biggr)
					\end{align}
					かつ$B \cap \left( B_n \backslash \bigcup_{k=1}^{n-1}B_k \right) \in \mathcal{B},\ (\forall n \geq 1)$
					が満たされるから,$\mu_0$の完全加法性より
					\begin{align}
						\mu_0(B) = \sum_{n=1}^\infty \mu_0\Biggl( B \cap \Biggl( B_n \backslash \bigcup_{k=1}^{n-1}B_k \Biggr) \Biggr)
						\leq \sum_{n=1}^\infty \mu_0(B_n)
					\end{align}
					が成り立ち$\mu_0(B) \leq \mu(B)$が従う.よって任意の$B \in \mathcal{B}$で$\mu_0(B) = \mu(B)$となる.
					\QED
			\end{description}
		\end{prf}
		
		
		\begin{screen}
			\begin{thm}[測度の一致の定理]\label{thm:identity_theorem_of_measures}
				$(X,\mathcal{B})$を可測空間,
				$\mathcal{A}$を$\mathcal{B}$を生成する乗法族とし,
				$(X,\mathcal{B})$上の測度$\mu_1,\mu_2$が
				$\mathcal{A}$上で一致しているとする.このとき,
				\begin{align}
					\mu_1(X_n) < \infty,
					\quad \bigcup_{n=1}^\infty X_n = X
				\end{align}
				を満たす増大列$\{X_n\}_{n=1}^\infty \subset \mathcal{A}$が存在すれば
				$\mu_1 = \mu_2$が成り立つ.
			\end{thm}
		\end{screen}
		
		\begin{prf}
			任意の$n = 1,2,\cdots$に対して
			\begin{align}
				\mathscr{D}_n \coloneqq \Set{B \in \mathcal{B}}{\mu_1(B \cap X_n) = \mu_2(B \cap X_n)}
			\end{align}
			とおけば,$\mathscr{D}_n$は$\mathcal{A}$を含むDynkin族であるから,Dynkin族定理より
			\begin{align}
				\mathscr{D}_n = \mathcal{B},\quad (\forall n \geq 1)
			\end{align}
			となり
			\begin{align}
				\mu_1(B) = \lim_{n \to \infty} \mu_1(B \cap X_n)
				= \lim_{n \to \infty} \mu_2(B \cap X_n) = \mu_2(B),
				\quad (\forall B \in \mathcal{B})
			\end{align}
			が従う.
			\QED
		\end{prf}
		
		\begin{screen}
			\begin{thm}[Kolmogorov-Hopf]\label{thm:appendix_Kolmogorov_Hopf}
				$(X,\mathcal{B},\mu_0)$を有限加法的な正値測度空間とすれば,
				定理\ref{thm:outer_measure_finitely_additive_measure}より
				\begin{align}
					\tilde{\mu}(A) \coloneqq \inf{}{}\Set{\sum_{n=1}^\infty \mu_0(B_n)}{B_n \in \mathcal{B},\ A \subset \bigcup_{n=1}^\infty B_n},
					\quad (\forall A \subset X)
				\end{align}
				により$X$上に外測度が定まる.このとき$\mathcal{B}^*$を$\tilde{\mu}$-可測集合として
				$\mu^* \coloneqq\left.\tilde{\mu}\right|_{\mathcal{B}^*},
				\ \mu \coloneqq \left.\tilde{\mu}\right|_{\sigma[\mathcal{B}]}$おけば以下が成り立つ:
				\begin{description}
					\item[(1)] $\sigma[\mathcal{B}] \subset \mathcal{B}^*$が成り立つ.
					
					\item[(2)] $\mu_0$が$\mathcal{B}$上で$\sigma$-加法的なら
						$\mu$は$\mu_0$の拡張となっている:
						\begin{align}
							\mu_0(B) = \mu(B),\quad (\forall B \in \mathcal{B}).
							\label{eq:appendix_finite_additive_measure_expansion_1}
						\end{align}
						
					\item[(3)] $\mu_0$が$\mathcal{B}$上で$\sigma$-有限的であるとき,
						(\refeq{eq:appendix_finite_additive_measure_expansion_1})を満たすような
						$\left( X,\sigma[\mathcal{B}] \right)$上の測度は存在しても唯一つである.
					
					\item[(4)] $\mu_0$が$\mathcal{B}$上で$\sigma$-加法的かつ$\sigma$-有限的ならば,
						$\mu$は$\mu_0$の$\left( X,\sigma[\mathcal{B}] \right)$への唯一つの拡張測度であり,
						このとき$\left( X,\mathcal{B}^*,\mu^* \right)$は$(X,\sigma[\mathcal{B}],\mu)$の
						Lebesgue拡大に一致する:
						\begin{align}
							\left( X,\mathcal{B}^*,\mu^* \right) 
							= \left( X,\overline{\sigma[\mathcal{B}]},\overline{\mu} \right).
							\label{eq:appendix_finite_additive_measure_expansion_5}
						\end{align}
				\end{description}
			\end{thm}
		\end{screen}
		
		\begin{prf}\mbox{}
			\begin{description}
				\item[(1)の証明]
					定理\ref{thm:Caratheodory_extension_theorem}より
					$\mathcal{B}^*$は$\mathcal{B}$を含む$\sigma$-加法族であるから
					$\sigma[\mathcal{B}] \subset \mathcal{B}^*$となる.
				
				\item[(2)の証明]
					定理\ref{thm:outer_measure_finitely_additive_measure}より任意の
					$B \in \mathcal{B}$で$\mu_0(B) = \tilde{\mu}(B) = \mu(B)$が成り立つ.
				
				\item[(3)の証明]
					$\sigma$-有限の仮定より,或る増大列$X_1 \subset X_2 \subset \cdots
					,\ \{X_n\}_{n=1}^\infty \subset \mathcal{B}$が存在して
					\begin{align}
						\mu_0 (X_n) < \infty \quad \bigcup_{n=1}^\infty X_n = X
						\label{eq:appendix_finite_additive_measure_expansion_3}
					\end{align}
					が成り立つ.一致の定理より,(\refeq{eq:appendix_finite_additive_measure_expansion_1})を満たす
					$\left( X,\sigma[\mathcal{B}] \right)$上の測度は存在しても一つのみである.
					
				\item[(4)の証明]
					(2)と(3)の結果より$\mu$は$\mu_0$の唯一つの拡張測度である.次に
					\begin{align}
						\mathcal{B}^* = \overline{\sigma[\mathcal{B}]}
						\label{eq:appendix_finite_additive_measure_expansion_4}
					\end{align}
					を示す.$E \in \overline{\sigma[\mathcal{B}]}$なら
					或る$B_1,B_2 \in \sigma[\mathcal{B}]$が存在して
					\begin{align}
						B_1 \subset E \subset B_2, \quad \mu(B_2 - B_1) = 0
					\end{align}
					を満たす.このとき(1)より
					$\mu^*(B_2 - B_1) = 0$であり,$\left( X,\mathcal{B}^*,\mu^* \right)$の完備性より
					$E \backslash B_1 \in \mathcal{B}^*$が満たされ
					\begin{align}
						E = B_1 + E \backslash B_1 \in \mathcal{B}^*
					\end{align}
					が従う.いま,(\refeq{eq:appendix_finite_additive_measure_expansion_3})を満たす
					$\{X_n\}_{n=1}^\infty \subset \mathcal{B}$を取り,
					$E \in \mathcal{B}^*$に対して$E_n \coloneqq E \cap X_n$とおく.このとき
					\begin{align}
						\mu^*(E_n) \leq \mu^*(X_n) = \mu_0(X_n) < \infty
					\end{align}
					となるから,任意の$k = 1,2,\cdots$に対して
					\begin{align}
						E_n \subset \bigcup_{j=1}^\infty B^n_{k,j},
						\quad
						\sum_{j=1}^\infty \mu_0\left( B^n_{k,j} \right)
						< \mu^*(E_n) + \frac{1}{k}
					\end{align}
					を満たす$\left\{B^n_{k,j}\right\}_{j=1}^\infty \subset \mathcal{B}$が取れる.
					\begin{align}
						B_{2,n} \coloneqq \bigcap_{k=1}^\infty \bigcup_{j=1}^\infty B^n_{k,j}
					\end{align}
					とおけば$E_n \subset B_{2,n} \in \sigma[\mathcal{B}]$であり,
					任意の$k = 1,2,\cdots$に対して
					\begin{align}
						&\mu^*(B_{2,n} - E_n) = \mu^*(B_{2,n}) - \mu^*(E_n)
						\leq \mu^*\Biggl( \bigcup_{j=1}^\infty B^n_{k,j} \Biggr) - \mu^*(E_n) \\
						&\qquad \leq \sum_{j=1}^\infty \mu^*\left( B^n_{k,j} \right) - \mu^*(E_n)
						< \mu^*(E_n) + \frac{1}{k} - \mu^*(E_n)
						= \frac{1}{k}
					\end{align}
					が成り立つから$\mu^*(B_{2,n} - E_n) = 0$となる.
					$E_n$を$B_{2,n} - E_n$に替えれば
					\begin{align}
						B_{2,n} - E_n \subset N_n, \quad \mu(N_n) = 0
					\end{align}
					を満たす$N_n \in \sigma[\mathcal{B}]$が取れる.
					\begin{align}
						B_{1,n} \coloneqq B_{2,n} \cap N_n^c
					\end{align}
					とおけば,$B_{1,n} \subset B_{2,n} \cap (B_{2,n} - E_n)^c = E_n$より
					\begin{align}
						B_{1,n} \subset E_n \subset B_{2,n},
						\quad \mu(B_{2,n} - B_{1,n}) \leq \mu(N_n) = 0
					\end{align}
					が成り立つから,
					\begin{align}
						B_1 \coloneqq \bigcup_{n=1}^\infty B_{1,n},
						\quad B_2 \coloneqq \bigcup_{n=1}^\infty B_{2,n}
					\end{align}
					として
					\begin{align}
						B_1 \subset E \subset B_2,
						\quad \mu(B_2 - B_1) \leq \mu\Biggl( \bigcup_{n=1}^\infty(B_{2,n} - B_{1,n}) \Biggr) = 0
					\end{align}
					が満たされ,$E \in \overline{\sigma[\mathcal{B}]}$が従い
					(\refeq{eq:appendix_finite_additive_measure_expansion_4})が得られる.
					同時に
					\begin{align}
						\overline{\mu}(E) = \mu(B_1) = \mu^*(B_1)
						\leq \mu^*(E) \leq \mu^*(B_2) = \mu(B_2) = \overline{\mu}(E)
					\end{align}
					が成立するから,$\overline{\mu} = \mu^*$となり
					(\refeq{eq:appendix_finite_additive_measure_expansion_5})が出る.
					\QED
			\end{description}
		\end{prf}
	
	本稿では`Cartesian 積'と`直積'は分けていて,
	Cartesian 積は$\times$で表される集合とし,直積は写像の集合としている.
	`Cartesian 積'と`直積'で別に積$\sigma$-加法族を定義する必要がある.
	\begin{screen}
		\begin{dfn}[Cartesian 積上の$\sigma$-加法族]
			$(A,\mathscr{F}_A),(B,\mathscr{F}_B)$を可測空間とするとき,
			\begin{align}
				\mathscr{F}_A \otimes \mathscr{F}_B
				\coloneqq \sigma\left(\Set{x}{
				\exists a\, \exists b\, 
				\left(\, x = a \times b \wedge a \in \mathscr{F}_A \wedge
				b \in \mathscr{F}_B\, \right)} \right)
			\end{align}
			と定め,これを$\mathscr{F}_A$と$\mathscr{F}_B$の{\bf 積$\sigma$-加法族}
			\index{せきしぐまかほうぞく@積$\sigma$-加法族}{\bf (product sigma-algebra)}と呼ぶ.
		\end{dfn}
	\end{screen}
	
	\begin{align}
		\mathscr{F}_A \otimes \mathscr{F}_B \otimes \mathscr{F}_C
		&\coloneqq \left( \mathscr{F}_A \otimes \mathscr{F}_B  \right) \otimes \mathscr{F}_C, \\
		\mathscr{F}_A \otimes \mathscr{F}_B \otimes \mathscr{F}_C \otimes \mathscr{F}_D
		&\coloneqq \left( \mathscr{F}_A \otimes \mathscr{F}_B \otimes \mathscr{F}_C \right) \otimes \mathscr{F}_D, \\
		\mathscr{F}_A \otimes \mathscr{F}_B \otimes \mathscr{F}_C \otimes \mathscr{F}_D \otimes \mathscr{F}_E
		&\coloneqq \left( \mathscr{F}_A \otimes \mathscr{F}_B \otimes \mathscr{F}_C \otimes \mathscr{F}_D \right) \otimes \mathscr{F}_E, \\
		&\vdots
	\end{align}
	
	\begin{screen}
		\begin{dfn}[直積上の$\sigma$-加法族]		
			$\Lambda$を集合とし,
			$\left\{(X_\lambda,\mathscr{F}_\lambda)\right\}_{\lambda \in \Lambda}$を可測空間の族とし,
			\begin{align}
				X \coloneqq \prod_{\lambda \in \Lambda} X_\lambda
			\end{align}
			とおく.$p_\lambda$を$\lambda$射影とするとき,
			\begin{align}
				\bigotimes_{\lambda \in \Lambda} \mathscr{F}_\lambda \coloneqq
				\sigma\left(\Set{x}{\exists \lambda \in \Lambda\, \exists A \in \mathscr{F}_\lambda\,
				\left(\, x=p_\lambda^{-1} \ast A\, \right)}\right)
			\end{align}
			と定め,これを$\{\mathscr{F}_\lambda\}_{\lambda \in \Lambda}$の積$\sigma$-加法族と呼ぶ.
		\end{dfn}
	\end{screen}
	
	\begin{screen}
		\begin{thm}[積$\sigma$-加法族を生成する加法族]
			$\Lambda$を集合とし、
			$\left\{(X_\lambda,\mathscr{F}_\lambda)\right\}_{\lambda \in \Lambda}$を可測空間の族とし,
			$X \coloneqq \prod_{\lambda \in \Lambda} X_\lambda$とおく.
			$\mathscr{B}_\lambda$を$X_\lambda$を含み$\mathscr{F}_\lambda$を生成する乗法族とし,
			$\lambda$射影を$p_\lambda$と書くとき,
			\begin{align}
				\mathcal{A} \coloneqq
				\Set{\bigcap_{\lambda \in \Lambda'}p_\lambda^{-1}(A_\lambda)}{A_\lambda \in \mathscr{B}_\lambda,\ \mbox{$\Lambda'$: $\Lambda$の有限部分集合}}
			\end{align}
			とおけば$\mathcal{A}$は乗法族となり,$\mathcal{A}$の要素の有限直和の全体を
			$\mathcal{B}$とすればこれは有限加法族となる.そして次が成り立つ:
			\begin{align}
				\bigotimes_{\lambda \in \Lambda} \mathscr{F}_\lambda
				= \sigma\left[ \mathcal{A} \right]
				= \sigma\left[ \mathcal{B} \right].
			\end{align}
		\end{thm}
	\end{screen}
	
	\begin{prf}
		
	\end{prf}
	
	\begin{screen}
		\begin{thm}[第二可算空間の直積Borel集合族]\label{thm:Borel_algebra_of_products_of_second_countable_spaces}
			$\Lambda$を空でない高々可算集合,
			$(S_\lambda)_{\lambda \in \Lambda}$を空でない第二可算空間の族とする.
			$S \coloneqq \prod_{\lambda \in \Lambda} S_\lambda$とおいて
			直積位相を導入すれば,$S$も第二可算空間となりまた次が成立する:
			\begin{align}
				\borel{S} = \bigotimes_{\lambda \in \Lambda} \borel{S_\lambda}.
				\label{eq:lem1_for_chap_1_exercise_1_8_1}
			\end{align}
		\end{thm}
	\end{screen}

\begin{prf}
	各$S_\lambda$の開集合系及び可算基を$\mathscr{O}_\lambda,\ \mathscr{B}_\lambda$,
	$S$の開集合系を$\mathscr{O}$とし,
	また$\lambda$射影を$p_\lambda:S \longrightarrow S_\lambda$と書く.
	先ず,任意の$O_\lambda \in \mathscr{O}_\lambda$に対して
	$p_\lambda^{-1}(O_\lambda) \in \mathscr{O}$
	が満たされるから
	\begin{align}
		\mathscr{O}_\lambda 
		\subset \Set{A_\lambda \in \borel{S_\lambda}}{p_\lambda^{-1}(A_\lambda) \in \borel{S}}
	\end{align}
	が従い,右辺が$\sigma$-加法族であるから
	\begin{align}
		\bigotimes_{\lambda \in \Lambda} \borel{S_\lambda}
		= \sgmalg{\Set{p_\lambda^{-1}(A_\lambda)}{A_\lambda \in \borel{S_\lambda},\ \lambda \in \Lambda}} \subset \borel{S}
	\end{align}
	を得る.一方で
	\begin{align}
		\mathscr{B} \coloneqq 
		\Set{\bigcap_{\lambda \in \Lambda'} p_\lambda^{-1}(B_\lambda)}{B_\lambda \in \mathscr{B}_\lambda,\ \Lambda' \subset \Lambda:finite\ subset}
	\end{align}
	は$\mathscr{O}$の基底の一つである.実際,任意に$O \in \mathscr{O}$を取れば,
	任意の$x \in O$に対し或る有限集合$\Lambda' \subset \Lambda$が存在して
	\begin{align}
		x \in \bigcap_{\lambda \in \Lambda'} p_\lambda^{-1} (O_\lambda) \subset O
	\end{align}
	が成立するが,更に$S_\lambda$の第二可算性より或る
	$\mathscr{B}'_\lambda \subset \mathscr{B}_\lambda\ (\lambda \in \Lambda')$が存在して
	\begin{align}
		x \in \bigcap_{\lambda \in \Lambda'} p_\lambda^{-1} (O_\lambda)
		= \bigcap_{\lambda \in \Lambda'} \bigcup_{B_{\lambda} \in \mathscr{B}'_\lambda} p_\lambda^{-1} (B_{\lambda})
	\end{align}
	が満たされる.すなわち,任意の$O \in \mathscr{O}$は
	\begin{align}
		O = \bigcup_{E \in \mathscr{B}'} E,
		\quad (\exists\mathscr{B}' \subset \mathscr{B})
	\end{align}
	と表される.$\mathscr{B}$は高々可算の濃度を持ち
	\footnote{
		$L_0 \coloneqq 
		\Set{\Lambda'}{\Lambda' \subset \Lambda:finite\ subset}$は
		高々可算集合である.実際,$\Lambda_n \coloneqq \Lambda \times \cdots \times \Lambda\ 
		(\mbox{$n$ copies of $\Lambda$})$として
		$L \coloneqq \bigcup_{n=1}^{\# \Lambda} \Lambda_n$とおき,
		$(x_1,\cdots,x_n) \in L$に対し$\{x_1,\cdots,x_n\} \in L_0$を対応させる
		$f:L \longrightarrow L_0$を考えれば全射であるから
		$\card{L_0} \leq \card{L} \leq \aleph_0$が従う.
	},
	$\mathscr{B} \subset \prod_{\lambda \in \Lambda} \borel{S_\lambda}$が満たされるから
	\begin{align}
		\mathscr{O} \subset \bigotimes_{\lambda \in \Lambda} \borel{S_\lambda}
	\end{align}
	が従い(\refeq{eq:lem1_for_chap_1_exercise_1_8_1})を得る.
	\QED
\end{prf}
		
		\begin{screen}
			\begin{thm}[積測度]
				測度空間の族$\left((X_i,\mathscr{F}_i,\mu_i)\right)_{i=1}^n$に対し,
				$\left( \prod_{i=1}^n X_i,\ \bigotimes_{i=1}^n \mathscr{F}_i \right)$上の
				測度$\mu$で
				\begin{align}
					\mu(A_1 \times \cdots \times A_n)
					= \mu_1(A_1) \cdots \mu_n(A_n),
					\quad (\forall A_i \in \mathscr{F}_i,\ i=1,\cdots,n)
				\end{align}
				を満たすものが存在する.この$\mu$を$(\mu_i)_{i=1}^n$の
				積測度と呼び$\bigotimes_{i=1}^n \mu_i = \mu_1 \otimes \cdots \otimes \mu_n$と書く.
				全ての$\mu_i$が$\sigma$-有限なら積測度$\mu$は唯一つ存在し$\sigma$-有限となる.
			\end{thm}
		\end{screen}
		
		\begin{prf}
			$\bigotimes_{i=1}^n \mathscr{F}_i$を生成する乗法族を
			\begin{align}
				\mathcal{A} \coloneqq
				\Set{A_1 \times \cdots \times A_n}{A_i \in \mathscr{F}_i,\ i=1,\cdots,n}
			\end{align}
			とおけば,定理\ref{thm:forming_finitely_additive_class}より
			\begin{align}
				\mathcal{B} \coloneqq \Set{\sum_{i=1}^n I_i}{I_i \in \mathcal{A},\ n=1,2,\cdots}
			\end{align}
			は$\prod_{i=1}^n X_i$の上の加法族となり$\bigotimes_{i=1}^n \mathscr{F}_i$を生成する.
		\end{prf}
		
		\begin{screen}
			\begin{thm}[完備測度空間の直積空間]\label{thm:product_space_of_complete_measure_space}
				$\left( (X_i,\mathcal{B}_i,\mu_i) \right)_{i=1}^n$を$\sigma$-有限な測度空間の族とし,
				$(X_i,\mathcal{B}_i,\mu_i)$のLebesgue拡大を$\left( X_i,\mathfrak{M}_i,m_i \right)$と書く.
				このとき次が成り立つ:
				\begin{align}
					\left( X_1 \times \cdots \times X_n, \overline{\mathcal{B}_1 \otimes \cdots \otimes \mathcal{B}_n}, \overline{\mu_1 \otimes \cdots \otimes \mu_n} \right)
					= \left( X_1 \times \cdots \times X_n, \overline{\mathfrak{M}_1 \otimes \cdots \otimes \mathfrak{M}_n}, \overline{m_1 \otimes \cdots \otimes m_n} \right).
				\end{align}
			\end{thm}
		\end{screen}
		
		\begin{prf} $X \coloneqq \prod_{i=1}^n X_i,\ 
			\mathcal{B} \coloneqq \bigotimes_{i=1}^n \mathcal{B}_i,\ 
			\mathfrak{M} \coloneqq \bigotimes_{i=1}^n \mathfrak{M}_i,\ 
			\mu \coloneqq \bigotimes_{i=1}^n \mu_i,\ 
			m \coloneqq  \bigotimes_{i=1}^n m_i$と表記する.
			\begin{description}
				\item[第一段]
					$\mathcal{B}_i \subset \mathfrak{M}_i,\ (i=1,\cdots,n)$より
					$\mathcal{B} \subset \mathfrak{M}$
					が従う.このとき
					\begin{align}
						\mu(B) 
						= m(B),
						\quad (\forall B \in \mathcal{B})
						\label{eq:thm_product_space_of_complete_measure_space_1}
					\end{align}
					が成り立つことを示す.いま,$\sigma$-有限の仮定により各$i$に対し
					\begin{align}
						\mu_i(X^k_i) < \infty,
						\quad X^k_i \in \mathcal{B}_i,
						\ (\forall k = 1,2,\cdots),
						\quad X^k_1 \subset X^k_2 \subset \cdots
					\end{align}
					を満たす増大列$\left( X^k_i \right)_{k=1}^\infty$が存在し,
					\begin{align}
						X^k \coloneqq X^k_1 \times \cdots \times X^k_n,
						\quad (k=1,2,\cdots)
					\end{align}
					により$\mathcal{B}$の増大列$(X^k)_{k=1}^\infty$を定めれば
					\begin{align}
						X = \bigcup_{k=1}^\infty X^k,
						\quad \mu(X^k)
						= \mu_1(X^k_1) \cdots \mu_n(X^k_n) < \infty;
						\quad (k=1,2,\cdots)
					\end{align}
					が満たされる.ここで$\mathcal{B}$を生成する乗法族を
					\begin{align}
						\mathcal{A} \coloneqq
						\Set{B_1 \times \cdots \times B_n}{B_i \in \mathcal{B}_i,\ i=1,\cdots,n}
					\end{align}
					とおけば,$\mathcal{A}$は$\left\{ X^k \right\}_{k=1}^\infty$を含み,かつ
					任意の$B_1 \times \cdots \times B_n \in \mathcal{A}$に対して
					\begin{align}
						\mu(B_1 \times \cdots \times B_n)
						= \mu_1(B_1) \cdots \mu_n(B_n)
						= m_1(B_1) \cdots m_n(B_n)
						= m(B_1 \times \cdots \times B_n)
					\end{align}
					となるから,定理\ref{thm:identity_theorem_of_measures}より
					(\refeq{eq:thm_product_space_of_complete_measure_space_1})が出る.
					
				\item[第二段]
					この段と次の段で$\left(X, \overline{\mathcal{B}}, \overline{\mu} \right)$
					が$\left( X, \mathfrak{M}, m \right)$の完備拡張であることを示す.まず
					\begin{align}
						\mathfrak{M}
						\subset \overline{\mathcal{B}}
						\label{eq:thm_product_space_of_complete_measure_space_2}
					\end{align}
					が成り立つことを示す.任意の$E_j \in \mathfrak{M}_j$に対し,
					\begin{align}
						B^1_j \subset E_j \subset B^2_j,
						\quad \mu_j\left( B^2_j - B^1_j \right) = 0
					\end{align}
					を満たす$B^1_j,B^2_j \in \mathcal{B}_j$が存在する.
					このとき,$X$から$X_j$への射影を$p_j$と書けば
					\begin{align}
						p_j^{-1}\left(B^1_j\right) \subset p_j^{-1}(E_j) \subset p_j^{-1}\left(B^2_j\right),
						\quad p_j^{-1}\left(B^1_j\right),p_j^{-1}\left(B^2_j\right) \in \mathcal{B}
					\end{align}
					及び
					\begin{align}
						\mu\left(p_j^{-1}\left(B^2_j\right)-p_j^{-1}\left(B^1_j\right)\right)
						= \mu_1(X_1) \cdots \mu_j\left(B^2_j - B^1_j\right) \cdots \mu_n(X_n)
						= 0
					\end{align}
					が成り立つから
					\begin{align}
						p_j^{-1}(E_j) \in \overline{\mathcal{B}}
					\end{align}
					が従い,
					\begin{align}
						E_1 \times \cdots \times E_n
						= \bigcap_{i=1}^n p_i^{-1}(E_i),
						\quad (E_i \in \mathfrak{M}_i,\ i=1,\cdots,n)
					\end{align}
					と積$\sigma$-加法族の定義より(\refeq{eq:thm_product_space_of_complete_measure_space_2})が得られる.
				
				\item[第三段]
					任意の$E \in \mathfrak{M}$に対し
					\begin{align}
						m(E)
						= \overline{\mu}(E)
					\end{align}
					が成り立つことを示す.実際,(\refeq{eq:thm_product_space_of_complete_measure_space_2})より
					$E \in \mathfrak{M}$なら$E \in \overline{\mathcal{B}}$
					となるから,
					\begin{align}
						B_1 \subset E \subset B_2,
						\quad \mu(B_2-B_1) = 0,
						\quad \overline{\mu}(E)
						= \mu(B_1)
						\label{eq:thm_product_space_of_complete_measure_space_3}
					\end{align}
					を満たす$B_1,B_2 \in \mathcal{B}$が存在し,このとき
					(\refeq{eq:thm_product_space_of_complete_measure_space_1})より
					\begin{align}
						m(E-B_1)
						\leq m(B_2-B_1)
						= \mu(B_2-B_1)
						= 0
					\end{align}
					が成立し
					\begin{align}
						m(E)
						= m(B_1)
						= \mu(B_1)
						= \overline{\mu}(E)
					\end{align}
					が得られる.
					
				\item[第四段]
					前段の結果より
					$\left(X, \overline{\mathcal{B}}, \overline{\mu} \right)$
					は$\left( X, \mathfrak{M}, m \right)$
					の完備拡張であるから,
					\begin{align}
						\overline{\mathfrak{M}}
						\supset \overline{\mathcal{B}}
					\end{align}
					を示せば定理の主張を得る.実際,
					任意の$E \in \overline{\mathcal{B}}$に対し
					(\refeq{eq:thm_product_space_of_complete_measure_space_3})を満たす
					$B_1,B_2 \in \mathcal{B}$を取れば,
					\begin{align}
						m(B_2-B_1)
						= \mu(B_2-B_1)
						= 0
					\end{align}
					が成り立ち$E \in \overline{\mathfrak{M}}$が従う.
					\QED
			\end{description}
		\end{prf}
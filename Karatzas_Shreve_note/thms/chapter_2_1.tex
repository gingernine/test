\begin{itembox}[l]{Dynkin system theorem}
		Let $\mathscr{C}$ be a collection of subsets of $\Omega$ 
		which is closed under pairwise intersection. If $\mathscr{D}$ is 
		a Dynkin system containing $\mathscr{C}$, then $\mathscr{D}$ also 
		contains the $\sigma$-field $\sigma(\mathscr{C})$ generated by $\mathscr{C}$.
\end{itembox}

\begin{prf}
	定理\ref{thm:Dynkin_system_theorem}より
	$\sigma(\mathscr{C}) = \delta(\mathscr{C}) \subset \mathscr{D}$となる.
	\QED
\end{prf}

\begin{itembox}[l]{Problem 1.4}
\label{thm:application_dynkin_system_theorem_to_independence}
		Let $X = \Set{X_t}{0 \leq t < \infty}$ be a stochastic process 
		for which $X_0,X_{t_1} - X_{t_0}, \cdots, X_{t_n} - X_{t_{n-1}}$ are 
		independent random variables, for every integer $n \geq 1$ and indices 
		$0 = t_0 < t_1 < \cdots < t_n < \infty$. Then for any fixed $0 \leq s < t < \infty$, 
		the increment $X_t - X_s$ is independent of $\mathscr{F}^X_s$.
\end{itembox}
この主張の逆も成立する:
\begin{prf}
	先ず任意の$s \leq t \leq r$に対し$\sigma(X_t - X_s) \subset \mathscr{F}^X_r$が成り立つ.実際,
	\begin{align}
		\Phi:\R^d \times \R^d \ni (x,y) \longmapsto x - y
	\end{align}
	の連続性と$\borel{\R^d \times \R^d} = \borel{\R^d} \otimes \borel{\R^d}$より,
	任意の$E \in \borel{\R^d}$に対して
	\begin{align}
		(X_t - X_s)^{-1}(E) 
		= \left\{ \left( X_t,X_s \right) \in \Phi^{-1}(E) \right\}
		\in \sigma(X_s,X_t) \subset \mathscr{F}^X_r
		\label{eq:thm_application_dynkin_system_theorem_to_independence_1}
	\end{align}
	が満たされる.よって任意に$A_0 \in \sigma(X_0),\ A_i \in \sigma(X_{t_i} - X_{t_{i-1}})$を取れば,
	$X_{t_n} - X_{t_{n-1}}$が$\mathscr{F}^X_{t_{n-1}}$と独立であるから
	\begin{align}
		P(A_0 \cap A_1 \cap \cdots \cap A_n)
		= P(A_0 \cap A_1 \cap \cdots \cap A_{n-1}) P(A_n)
	\end{align}
	が成立する.帰納的に
	\begin{align}
		P(A_0 \cap A_1 \cap \cdots \cap A_n)
		= P(A_0) P(A_1) \cdots P(A_n)
	\end{align}
	が従い$X_0,X_{t_1} - X_{t_0}, \cdots, X_{t_n} - X_{t_{n-1}}$の独立性を得る.
	\QED
\end{prf}

\begin{prf}[Problem 1.4]\mbox{}
	\begin{description}
		\item[第一段]
			Dynkin族を次で定める:
			\begin{align}
				\mathscr{D} \coloneqq
				\Set{A \in \mathscr{F}}{P(A \cap B) = P(A)P(B),\ \forall B \in \sigma(X_t - X_s)}.
			\end{align}
			いま,任意に$0 = s_0 < \cdots < s_n = s$を取り固定し
			\begin{align}
				\mathscr{A}_{s_0, \cdots, s_n} \coloneqq
				\Set{\bigcap_{i=0}^n A_i}{A_0 \in \sigma(X_0),\ A_i \in \sigma(X_{s_i} - X_{s_j}),\ i=1,\cdots,n}
			\end{align}
			により乗法族を定めれば,仮定より$\sigma(X_{s_i} - X_{s_{i-1}})$と$\sigma(X_t - X_s)$が独立であるから
			\begin{align}
				\mathscr{A}_{s_0, \cdots, s_n}
				\subset \mathscr{D}
			\end{align}
			が成立し,Dynkin族定理により
			\begin{align}
				\sigma(X_{s_0},X_{s_1}-X_{s_0},\cdots,X_{s_n} - X_{s_{n-1}})
				= \sgmalg{\mathscr{A}_{s_0, \cdots, s_n}}
				\subset \mathscr{D}
				\label{eq:thm_application_dynkin_system_theorem_to_independence_2}
			\end{align}
			が従う.
		
		\item[第二段]
			$\sigma(X_{s_0},X_{s_1}-X_{s_0},\cdots,X_{s_n} - X_{s_{n-1}})$の全体が
			$\mathscr{F}^X_s$を生成することを示す.先ず,
			(\refeq{eq:thm_application_dynkin_system_theorem_to_independence_1})より
			\begin{align}
				\bigcup_{\substack{n \geq 1 \\ s_0 < \cdots < s_n}} 
				\sigma(X_{s_0},X_{s_1}-X_{s_0},\cdots,X_{s_n} - X_{s_{n-1}})
				\subset \mathscr{F}^X_s
				\label{eq:thm_application_dynkin_system_theorem_to_independence_3}
			\end{align}
			が成立する.一方で,任意の
			$X_r^{-1}(E)\ (\forall E \in \borel{\R^d},\ 0 < r \leq s)$について,
			\begin{align}
				\Psi:\R^d \times \R^d \ni (x,y) \longmapsto x + y
			\end{align}
			で定める連続写像を用いれば
			\begin{align}
				X_r^{-1}(E)
				= \left( X_r - X_0 + X_0 \right)^{-1}(E)
				= \left\{\left( X_r - X_0, X_0\right) \in \Psi^{-1}(E) \right\}
			\end{align}
			となり,$X_r^{-1}(E) \in \sigma(X_0, X_r - X_0)$が満たされ
			\begin{align}
				\sigma(X_r) \subset \sigma(X_0, X_r - X_0)
				\subset \sigma(X_0, X_r - X_0,X_s - X_r)
				\label{eq:thm_application_dynkin_system_theorem_to_independence_4}
			\end{align}
			が出る.
			$\sigma(X_0) \subset \sigma(X_0,X_s - X_0)$
			も成り立ち
			\begin{align}
				\bigcup_{0 \leq r \leq s} \sigma(X_r) \subset 
				\bigcup_{\substack{n \geq 1 \\ s_0 < \cdots < s_n}} \sigma(X_{s_0},X_{s_1}-X_{s_0},\cdots,X_{s_n} - X_{s_{n-1}})
			\end{align}
			が従うから,(\refeq{eq:thm_application_dynkin_system_theorem_to_independence_3})
			と併せて
			\begin{align}
				\mathscr{F}^X_s
				= \sgmalg{\bigcup_{\substack{n \geq 1 \\ s_0 < \cdots < s_n}} \sigma(X_{s_0},X_{s_1}-X_{s_0},\cdots,X_{s_n} - X_{s_{n-1}})}
				\label{eq:thm_application_dynkin_system_theorem_to_independence_5}
			\end{align}
			が得られる.
		
		\item[第三段]
			任意の$0 = s_0 < s_1 < \cdots < s_n = s$に対し,
			(\refeq{eq:thm_application_dynkin_system_theorem_to_independence_1})と
			(\refeq{eq:thm_application_dynkin_system_theorem_to_independence_4})より
			\begin{align}
				\sigma(X_{s_0},X_{s_1}-X_{s_0},\cdots,X_{s_n} - X_{s_{n-1}})
				= \sigma(X_{s_0},X_{s_1},\cdots,X_{s_n})
				\label{eq:thm_application_dynkin_system_theorem_to_independence_6}
			\end{align}
			が成り立つ.
		
		\item[第四段]
			二つの節点$0 = s_0 < \cdots < s_n = s$と$0 = r_0 < \cdots < r_m = s$
			の合併を$0 = u_0 < \cdots < u_k = s$と書けば
			\begin{align}
				\sigma(X_{s_0},\cdots,X_{s_n})
				\cup \sigma(X_{r_0},\cdots,X_{r_m})
				\subset \sigma(X_{u_0},\cdots,X_{u_k})
			\end{align}
			が成り立つから
			\begin{align}
				\bigcup_{\substack{n \geq 1 \\ s_0 < \cdots < s_n}} \sigma(X_{s_0},X_{s_1},\cdots,X_{s_n})
			\end{align}
			は交演算で閉じている.従って
			(\refeq{eq:thm_application_dynkin_system_theorem_to_independence_2}),
			(\refeq{eq:thm_application_dynkin_system_theorem_to_independence_5}),
			(\refeq{eq:thm_application_dynkin_system_theorem_to_independence_6})及び
			Dynkin族定理により
			\begin{align}
				\mathscr{F}^X_s 
				= \sgmalg{\bigcup_{\substack{n \geq 1 \\ s_0 < \cdots < s_n}} \sigma(X_{s_0},X_{s_1}-X_{s_0},\cdots,X_{s_n} - X_{s_{n-1}})}
				= \sgmalg{\bigcup_{\substack{n \geq 1 \\ s_0 < \cdots < s_n}} \sigma(X_{s_0},X_{s_1},\cdots,X_{s_n})}
				\subset \mathscr{D}
			\end{align}
			が従い定理の主張を得る.
			\QED
	\end{description}
\end{prf}
\section{The Consistency Theorem}
	\begin{itembox}[l]{Daniell, Kolmogorov}
			Let $(Q_t)_{t \in T}$ be a consistent family of finite-dimensional 
			distributions. Then there is a probability measure $P$ on 
			$\left( \R^{d,[0,\infty)}, \borel{\R^{d,[0,\infty)}} \right)$, such that 
			(2.2) holds for every $t \in T$.
	\end{itembox}
	
	\begin{prf}\mbox{}
		\begin{description}
			\item[第一段]
				$C \in \mathscr{C}$について,
				$C = \Set{\omega \in \R^{d,[0,\infty)}}{(\omega(t_1),\cdots,\omega(t_n)) \in A}$
				という表示に対し
				\begin{align}
					Q(C) \coloneqq Q_t(A)
				\end{align}
				により$Q(C)$を定めると,$Q(C)$はwell-definedであり,$Q$は$\mathscr{C}$上の初等確率測度となる:
				いま,
				\begin{align}
					C 
					&= \Set{\omega \in \R^{d,[0,\infty)}}{(\omega(t_1),\cdots,\omega(t_n)) \in A} \\
					&= \Set{\omega \in \R^{d,[0,\infty)}}{(\omega(s_1),\cdots,\omega(s_m)) \in B} 
				\end{align}
				という二つの表示があるとする.
				\begin{description}
					\item[case1]
						$s$が$t$の並び替えである場合,
						或る$\{1,\cdots,n\}$の全単射$\varphi$が存在して
						\begin{align}
							(s_1,\cdots,s_n) 
							= (t_{\varphi(1)},\cdots,t_{\varphi(n)})
						\end{align}
						と書ける.
						\begin{align}
							F:(\R^d)^n \ni x \longmapsto x \circ \varphi \in (\R^d)^n
						\end{align}
						により線型同型写像$F$を定めれば
						\begin{align}
							\omega \circ t \in A
							\quad \Leftrightarrow \quad \omega \circ t \circ \varphi 
							= F(\omega \circ t) \in F(A),
							\quad (\omega \in \R^{d,[0,\infty)})
						\end{align}
						が成り立つから
						\begin{align}
							\Set{\omega \in \R^{d,[0,\infty)}}{(\omega(t_1),\cdots,\omega(t_n)) \in A}
							&= \Set{\omega \in \R^{d,[0,\infty)}}{(\omega(t_{\varphi(1)}),\cdots,\omega(t_{\varphi(n)})) \in F(A)} \\
							&= \Set{\omega \in \R^{d,[0,\infty)}}{(\omega(s_1),\cdots,\omega(s_m)) \in B} 
						\end{align}
						が従い$B = F(A)$が出る.
						$F$は逆写像も込めて有限次元空間上の線型同型であるから同相であり
						\begin{align}
							F(A) \in \borel{(\R^d)^n}
						\end{align}
						が満たされ,特に矩形集合に対しては
						\begin{align}
							F(E_1 \times \cdots \times E_n) = E_{\varphi(1)} \times \cdots \times  E_{\varphi(n)},
							\quad (E_i \in \borel{\R^d})
						\end{align}
						が成立する.$(Q_t)_{t \in T}$のconsistencyにより
						\begin{align}
							\Set{E_1 \times \cdots \times E_n}{E_i \in \borel{\R^d},\ i=1,\cdots,n}
							\subset \Set{E \in \borel{(\R^d)^n}}{Q_t(E) = Q_s(F(E))}
						\end{align}
						が成り立ち,Dynkin族定理より$Q_t(A) = Q_s (B)$を得る.
					
					\item[case2]
						$n < m$かつ$\{t_1,\cdots,t_n\} \subset \{s_1,\cdots,s_m\}$の場合,
						或る$\{1,\cdots,m\}$上の全単射$\psi$が存在して
						\begin{align}
							(s_{\psi(1)},\cdots, s_{\psi(m)})
							= (t_1, \cdots, t_n,s_{\psi(n+1)},\cdots,s_{\psi(m)})
						\end{align}
						を満たす.
						\begin{align}
							G:(\R^d)^m \ni x \longmapsto x \circ \psi \in (\R^d)^m
						\end{align}
						により線型同型かつ同相な写像$G$を定めれば
						\begin{align}
							\omega \circ s \in B
							\quad \Leftrightarrow \quad \omega \circ s \circ \psi 
							= G(\omega \circ s) \in G(B),
							\quad (\omega \in \R^{d,[0,\infty)})
						\end{align}
						が成り立つから
						\begin{align}
							&\Set{\omega \in \R^{d,[0,\infty)}}{(\omega(t_1),\cdots,\omega(t_n)) \in A}
							= \Set{\omega \in \R^{d,[0,\infty)}}{(\omega(s_1),\cdots,\omega(s_m)) \in B} \\
							&\qquad= \Set{\omega \in \R^{d,[0,\infty)}}{(\omega(t_1),\cdots,\omega(t_n),\omega(s_{\psi(n+1)}),\cdots,\omega(s_{\psi(m)})) \in G(B)} 
						\end{align}
						が従い$G(B) = A \times (\R^d)^{m-n}$が出る.
						$(Q_t)_{t \in T}$のconsistencyとcase1の結果を併せて
						\begin{align}
							Q_t(A) = Q_{s \circ \psi}(A \times (\R^d)^{m-n}) = Q_{s \circ \psi}(G(B))
							= Q_s(B)
						\end{align}
						を得る.
						
					\item[case3]
						$\{u_1,\cdots,u_k\} \coloneqq \{t_1,\cdots,t_n\} \cup \{s_1,\cdots,s_m\}$により
						$u \in T$を定める.$n \vee m=k$なら
						case1或はcase2に該当するので,$n \vee m < k$の場合を考える.
						このとき$\{1,\cdots,k\}$上の全単射$\varphi_1,\varphi_2$が存在して
						\begin{align}
							(u_{\varphi_1(1)},\cdots, u_{\varphi_1(k)})
							&= (t_1, \cdots, t_n,u_{\varphi_1(n+1)},\cdots,u_{\varphi_1(k)}), \\
							(u_{\varphi_2(1)},\cdots, u_{\varphi_2(k)})
							&= (s_1, \cdots, s_m,u_{\varphi_2(m+1)},\cdots,u_{\varphi_2(k)})
						\end{align}
						を満たす.いま,
						\begin{align}
							H:(\R^d)^m \ni x \longmapsto x \circ \varphi_2^{-1} \circ \varphi_1 \in (\R^d)^m
						\end{align}
						により線型同型かつ同相な写像$H$を定めれば
						\begin{align}
							\omega \circ u \circ \varphi_2 \in B \times (\R^d)^{k-m}
							\quad \Leftrightarrow \quad 
							\omega \circ u \circ \varphi_1 
							= H(\omega \circ u \circ \varphi_2) \in H\left( B \times (\R^d)^{k-m} \right),
							\quad (\omega \in \R^{d,[0,\infty)})
						\end{align}
						が成り立つから,case1の結果より
						\begin{align}
							H\left( B \times (\R^d)^{k-m} \right) = A \times (\R^d)^{k-n}
						\end{align}
						かつ
						\begin{align}
							Q_t(A) 
							= Q_{u \circ \varphi_1} \left( A \times (\R^d)^{k-n} \right)
							= Q_{u \circ \varphi_1} \left( H\left( B \times (\R^d)^{k-m} \right) \right)
							= Q_{u \circ \varphi_2} \left( B \times (\R^d)^{k-m} \right)
							= Q_s(B)
						\end{align}
						が出る.
				\end{description}
				以上より$Q$はwell-definedである.次に$Q$が初等確率測度であることを示す.
				実際,
				\begin{align}
					\R^{d,[0,\infty)} = \Set{\omega \in \R^{d,[0,\infty)}}{(\omega(t_1),\cdots,\omega(t_n)) \in (\R^d)^n}
				\end{align}
				により$Q(\R^{d,[0,\infty)}) = Q_t((\R^d)^n) = 1$が従い,また
				$C_1,C_2 \in \mathscr{C},\ C_1 \cap C_2 = \emptyset$に対して
				\begin{align}
					C_1 &= \Set{\omega \in \R^{d,[0,\infty)}}{(\omega(t_1),\cdots,\omega(t_n)) \in A_1}, \\
					C_2 &= \Set{\omega \in \R^{d,[0,\infty)}}{(\omega(s_1),\cdots,\omega(s_m)) \in A_2}
				\end{align}
				という表示があるとして,case1$\sim$3の証明と同様にすれば$Q(C_1 + C_2) = Q(C_1) + Q(C_2)$が得られる.
		\end{description}
	\end{prf}
	
	\begin{itembox}[l]{Problem 1.5}
			Show that the just defined family $(Q_{t})_{t \in T}$ is consistent.
	\end{itembox}
	
	\begin{prf}
		Karatzas-Shreveの解答が間違っているので訂正する.
		Karatzas-Shreveの解答では$(Q_{t})_{t \in T}$が整合性条件を満たしていることを
		証明する前にKolmogorovの拡張定理を適用し,$(\R^{[0,\infty)},\borel{\R^{[0,\infty)}})$
		上の確率測度$P$によって$(Q_{t})_{t \in T}$の整合性条件を示しているが,
		これは論理が転倒している.$(Q_{t})_{t \in T}$の構成から
		やり直さなければならない.まず任意に
		\begin{align}
			t \in T,
			\quad (t = (t_1,\cdots,t_n)) 
		\end{align}
		を取る.このとき$t_1,\cdots,t_n$は昇順に並んでいるとは限らないが,
		これを昇順にした並び替え
		\begin{align}
			t_{i_1} < t_{i_2} < \cdots < t_{i_n}
		\end{align}
		がただ一つ存在する.ここで並び替え$i$は$\{1,\cdots,n\}$上の全単射とみなせる.
		\begin{align}
			s \coloneqq (t_{i_1},\cdots,t_{i_n}),
			\quad J_{ts} : \R^n \ni x \longmapsto x \circ i \in \R^n
		\end{align}
		により$s$と$J_{ts}$を定めれば,$J_{ts}$は線型同型($\R^n$は有限次元なので同相,すなわちBorel同型)
		であり,
		\begin{align}
			J_{ts} t = t \circ i = (t_{i_1},\cdots,t_{i_n}) = s
		\end{align}
		となる.
		分布$Q_{s}$は式(2.5)の分布関数により定めることができるから,
		$Q_{t}$については
		\begin{align}
			Q_{t}(A) \coloneqq Q_{s}(J_{ts}A),
			\quad (\forall A \in \borel{\R^n})
		\end{align}
		で定義する.このとき,$t$と$s$に対して整合性条件の(a)が成立する:
		\begin{align}
			Q_{t} (A_1 \times \cdots \times A_n)
			= Q_{s}(A_{i_1} \times \cdots \times A_{i_n}),
			\quad (\forall A_i \in \borel{\R}).
		\end{align}
		$s$を媒介することにより,
		$t$の任意の並び替え$u$に対する$Q_{u}$も定まり,
		$t,u$間の整合性条件(a)が満たされる.実際,
		$J_{us}$と$J_{tu}$を$J_{ts}$と同様に定めれば$J_{us}^{-1}J_{ts} = J_{tu}$
		の関係があるので
		\begin{align}
			Q_{t}(A) = Q_{s}(J_{ts}A)
			= Q_{s}(J_{us}J_{us}^{-1}J_{ts}A)
			= Q_{u}(J_{us}^{-1}J_{ts}A)
			= Q_{u}(J_{tu}A),
			\quad (\forall A \in \borel{\R^n})
		\end{align}
		が得られる.以上で分布族$(Q_{t})_{t \in T}$を構成する.
		あとは$(Q_{t})_{t \in T}$が整合性条件の(b)を満たせばよい.
		$t$の拡張
		\begin{align}
			t' \coloneqq (t_1,\cdots,t_n,t_{n+1})
		\end{align}
		に対して,その昇順並び替えを$s'$と書くとき,
		\begin{align}
			Q_{t'}(A_1 \times \cdots \times A_n \times \R)
			= Q_{s'}(J_{t's'}(A_1 \times \cdots \times A_n \times \R))
			= Q_{s} (J_{ts} (A_1 \times \cdots \times A_n))
			= Q_{t} (A_1 \times \cdots \times A_n)
		\end{align}
		が成立する.実際,昇順に並び替える全単射$i:\{1,\cdots,n+1\} \longrightarrow \{1,\cdots,n+1\}$により
		\begin{align}
			s' = (t_{i_1},\cdots,t_{i_{n+1}})
		\end{align}
		と表せるとき,$j \coloneqq i^{-1}(n+1)$とおいて,
		$n$次元Lebesgue測度を$\mu_n$と書けば,Fubiniの定理より
		\begin{align}
			&Q_{t'}(A_1 \times \cdots \times A_n \times \R) \\
			&=Q_{s'}(A_{i_1} \times \cdots \times A_{i_{j-1}} \times \R \times A_{i_{j+1}} \times \cdots \times A_{i_{n+1}}) \\
			&= \int_{A_{i_1} \times \cdots \times \R \times \cdots \times A_{i_{n+1}}}
			p(s_{i_1};0,y_1)p(s_{i_2} -s_{i_1};y_1,y_2)\cdots p(s_{i_{n+1}}-s_{i_n};y_n,y_{n+1})\ \mu_{n+1}(dy_1 \cdots dy_1) \\
			&= \int_{A_{i_1}}\cdots\int_{A_{i_{n+1}}} p(s_{i_1};0,y_1) \cdots
			\left[ \int_{\R} p(s_{i_{j+1}}-s_{i_j};y_i,y_{i+1}) p(s_{i_j}-s_{i_{j-1}};y_{i-1},y_i)\ dy_j\right]dy_{n+1}\cdots dy_1 \\
			&= \int_{A_{i_1}}\cdots\int_{A_{i_{n+1}}} p(s_{i_1};0,y_1) \cdots p(s_{i_{n+1}}-s_{i_n};y_n,y_{n+1})
			\left[p(s_{i_{j+1}}-s_{i_{j-1}};y_{i-1},y_{i+1})\right]dy_{n+1}\cdots dy_1 \\
			&= \int_{A_{i_1} \times \cdots \times A_{i_{n+1}}} p(s_{i_1};0,y_1) \cdots p(s_{i_{j+1}}-s_{i_{j-1}};y_{i-1},y_{i+1}) \cdots p(s_{i_{n+1}}-s_{i_n};y_n,y_{n+1})\ \mu_n(dy_1 \cdots dy_{n+1}) \\
			&= Q_{s}(A_{i_1} \times \cdots \times A_{i_{j-1}} \times A_{i_{j+1}} \times \cdots \times A_{i_{n+1}}) \\
			&= Q_{t} (A_1 \times \cdots \times A_n)
		\end{align}
		を得る.これにより
		\begin{align}
			\Set{A_1 \times \cdots \times A_n}{A_i \in \borel{\R},\ i=1,\cdots,n}
			\subset \Set{A \in \borel{\R^n}}{Q_{t'}(A \times \R) = Q_{t}(A)}
		\end{align}
		が従い,Dynkin族定理より整合性条件の(b)が満たされる.
		\QED
	\end{prf}
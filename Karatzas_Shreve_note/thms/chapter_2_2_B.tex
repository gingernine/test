\section{The Kolmogorov-\v{C}entsov Theorem}
	\begin{itembox}[l]{Exercise 2.7}
		The only $\borel{(\R^d)^{[0,\infty)}}$-measurable set contained 
		in $C[0,\infty)^d$ is the empty set.
	\end{itembox}
	
	\begin{prf}\mbox{}
		\begin{description}
			\item[第一段]
				$\borel{(\R^d)^{[0,\infty)}} = \sigma(B_t;\ 0 \leq t < \infty)$
				が成り立つことを示す.先ず,任意の$C \in \mathscr{C}$は
				\begin{align}
					C &= \Set{\omega \in (\R^d)^{[0,\infty)}}{(\omega(t_1),\cdots,\omega(t_n)) \in A} \\
					&=  \Set{\omega \in (\R^d)^{[0,\infty)}}{(B_{t_1}(\omega),\cdots,B_{t_n}(\omega)) \in A},
					\quad (A \in \borel{(\R^d)^n})
				\end{align}
				の形で表されるから$\mathscr{C} \subset \sigma(B_t;\ 0 \leq t < \infty)$
				が従い$\borel{(\R^d)^{[0,\infty)}} \subset \sigma(B_t;\ 0 \leq t < \infty)$
				を得る.逆に
				\begin{align}
					\sigma(B_t) \subset \mathscr{C},
					\quad (\forall t \geq 0)
				\end{align}
				より$\borel{(\R^d)^{[0,\infty)}} \supset \sigma(B_t;\ 0 \leq t < \infty)$
				も成立し$\borel{(\R^d)^{[0,\infty)}} = \sigma(B_t;\ 0 \leq t < \infty)$
				が出る.
				
			\item[第二段]
				高々可算集合$S = \{t_1,t_2,\cdots\} \subset [0,\infty)$に対して
				\begin{align}
					\mathcal{E}_S \coloneqq \Set{\Set{\omega \in (\R^d)^{[0,\infty)}}{(\omega(t_1),\omega(t_2),\cdots) \in A}}{A \in \borel{(\R^d)^{\# S}}}
				\end{align}
				とおけば
				\footnote{
					$S$が可算無限なら$(\R^d)^{\# S} = \R^\infty$.
				},座標過程$B$は
				$(\omega(t_1),\omega(t_2),\cdots) = (B_{t_1}(\omega),B_{t_2}(\omega),\cdots)$
				を満たすから
				\begin{align}
					\mathcal{E}_S = \Set{\left\{(B_{t_1},B_{t_2},\cdots) \in A\right\}}{A \in \borel{(\R^d)^{\# S}}} \eqqcolon \mathcal{F}^B_S
				\end{align}
				が成立する.従って第一章のLemma3 for Exercise 1.8と前段の結果より
				\begin{align}
					\borel{(\R^d)^{[0,\infty)}}
					&= \sigma(B_t;\ 0 \leq t < \infty)
					= \mathcal{F}^B_{[0,\infty)}
					= \bigcup_{S \subset [0,\infty):at\ most\ countable} \mathcal{F}^B_S\\
					&= \bigcup_{S \subset [0,\infty):at\ most\ countable} \mathcal{E}_S
				\end{align}
				を得る.すなわち,$\borel{(\R^d)^{[0,\infty)}}$の任意の元は
				$\Set{\omega \in (\R^d)^{[0,\infty)}}{(\omega(t_1),\omega(t_2),\cdots) \in A}$
				の形で表現され,$A \neq \emptyset$ならば
				$\Set{\omega \in (\R^d)^{[0,\infty)}}{(\omega(t_1),\omega(t_2),\cdots) \in A} \not\subset C[0,\infty)^d$となり主張が従う.
				\QED
		\end{description}
	\end{prf}
	
	\begin{itembox}[l]{Theorem 2.8 and Problem 2.9}
		Suppose that a process $X = \Set{X_t}{t \in  [0,T]^d}$ ($d \geq 1$)
		on a probability space $(\Omega,\mathscr{F},P)$ satisfies the condition
		\begin{align}
			\forall s,t \in [0,T]^d,\quad
			E|X_t - X_s|^\alpha \leq C\Norm{t-s}{}^{d + \beta},
			\quad \mbox{where $\Norm{\cdot}{}$ is max norm}
		\end{align}
		for some positive constants $\alpha,\beta$, and $C$. Then there exists a 
		continuous modification $\tilde{X} = \Set{\tilde{X}_t}{t \in [0,T]^d}$ of $X$, 
		which is locally H\Ddot{o}lder-continuous with exponent $\gamma$ for every 
		$\gamma \in (0,\beta/\alpha)$. \textcolor{red}{More precisely, for every $\gamma \in (0,\beta/\alpha)$,
		\begin{align}
			\forall \omega \in \Omega^*, \quad \sup{\substack{0 < \Norm{t-s}{} < h(\omega) \\ s,t \in [0,T]^d}}{\frac{\left| \tilde{X}_t(\omega) - \tilde{X}_s(\omega) \right|}{\Norm{t-s}{}^\gamma}} \leq \frac{2}{1-2^{-\gamma}}
		\end{align}
		for some $\Omega^* \in \mathscr{F}$ with $P(\Omega^*)=1$ and 
		positive random variable $h$, where $\Omega^*$ and $h$ depend on $\gamma$.}
	\end{itembox}
	
	\begin{prf}\mbox{}
		\begin{description}
			\item[第一段]
				$\N$の任意の要素$n$に対して
				\begin{align}
					L_n = \Set{\frac{kT}{2^n}}{k=0,1,2,\cdots,2^n-1}
				\end{align}
				として$L = \bigcup_{n \in \N} L_n$とおく.$L$は$[0,T]^d$において稠密である.
				$L_n$の要素$s$に対して
				\begin{align}
					R_n(s) = \Set{t \in L_n}{\Norm{t-s}{} \leq T2^{-n}}
				\end{align}
				とおく.つまり,$R_n(s)$とは$s$の各成分を最大$T2^{-n}$だけ動かした順序対の集合である.
				いま,$L_n$の要素数は$2^{nd}$,$L_n$の各要素$s$に対して
				$R_n(s)$の要素数は$3^d$である.
				Chebyshevの不等式より,任意の正数$\epsilon$に対して
				\begin{align}
					P\left(|X_t-X_s|\geq\epsilon\right)
					\leq \epsilon^{-\alpha}E|X_t-X_s|^\alpha
					\leq C\epsilon^{-\alpha}\Norm{t-s}{}^{d+\beta}
				\end{align}
				となり,特に$\epsilon = 2^{-\gamma n}$かつ
				$\Norm{t-s}{} \leq T2^{-n}$の場合は
				\begin{align}
					P\left(|X_t-X_s|\geq2^{-\gamma n}\right)
					\leq C 2^{-n(d+\beta - \alpha \gamma)}
				\end{align}
				が成り立つから,
				\begin{align}
					P\left(\operatorname*{max}_{s \in L_n \wedge t \in R_n(s)}
					|X_t-X_s|\geq2^{-\gamma n}\right)
					= P\left(\bigcup_{s \in L_n} \bigcup_{t \in R_n(s)}
					\{|X_t-X_s|\geq2^{-\gamma n}\}\right)
					\leq 3^d C T^{d+\beta} 2^{-n(\beta - \alpha \gamma)}
				\end{align}
				が成り立つ.$A_n = \left\{\operatorname*{max}_{s \in L_n \wedge t \in R_n(s)}|X_t-X_s|\geq2^{-\gamma n}\right\}$とおけば,$\beta - \alpha \gamma > 0$より
				$\sum_{n \in \N}P(A_n) < \infty$となるから,Borel-Cantelliの補題より
				\begin{align}
					N = \bigcap_{n \in \N} \bigcup_{k \geq n} A_k
				\end{align}
				は$P$-零集合となり,
				\begin{align}
					\forall \omega \in \Omega \backslash N,\
					\exists N \in \N,\
					\forall n \in \N,\quad
					N \leq n \Longrightarrow \operatorname*{max}_{s \in L_n \wedge t \in R_n(s)}
					|X_t(\omega) - X_s(\omega)| < 2^{-\gamma n}
					\label{eq:chapter_2_theorem_2_8_1}
				\end{align}
				が満たされる.
				
			\item[第二段]
				$\Omega \backslash N$の要素$\omega$に対して,
				(\refeq{eq:chapter_2_theorem_2_8_1})を満たす自然数$N$のうち
				最小なもの与える写像を$n^*$と書く(順序数の整列性).つまり$n^*$は
				\begin{align}
					n^* = \Biggl\{\ (\omega,n)\ \, : \quad &\omega \in \Omega \wedge n \in \N \wedge \Biggr.\\
					&\forall m \in \N\left[n \leq m \Longrightarrow \operatorname*{max}_{s \in L_m \wedge t \in R_m(s)}
						|X_t(\omega) - X_s(\omega)| < 2^{-\gamma m}\right] \wedge \\
					&\forall N \in \N
					\Biggl. \left[\forall m \in \N\left[N \leq m \Longrightarrow \operatorname*{max}_{s \in L_m \wedge t \in R_m(s)}
						|X_t(\omega) - X_s(\omega)| < 2^{-\gamma m}\right] \Longrightarrow n \leq N\right]\ \Biggr\}
				\end{align}
				で与えられる写像である.写像$n^*$は$\mathscr{F}/\borel{\R}$-可測性を持つ.実際,任意の自然数$\ell$に対して
				\begin{align}
					{n^*}^{-1}(\ell) = \left\{ \bigcap_{n = \ell}^\infty A_n^c \right\} \cap \left\{ \bigcap_{1 \leq j \leq \ell-1} \bigcap_{n = j}^\infty A_n \right\}
				\end{align}
				を満たす.$n^*$の定め方より
				\begin{align}
					\forall \omega \in \Omega \backslash N,\
					\forall n \in \N,\quad 
					n^*(\omega) \leq n \Longrightarrow \operatorname*{max}_{s \in L_n \wedge t \in R_n(s)}|X_t(\omega) - X_s(\omega)| < 2^{-\gamma n}
					\label{eq:chapter_2_theorem_2_8_2}
				\end{align}
				が成立する.
				
			\item[第三段]
				いま,$\Omega \backslash N$の任意の要素$\omega$と$n^*(\omega) \leq n$を満たす
				任意の自然数$n$が与えられたとすると,
				\begin{align}
					\forall m \in \N,\quad
					n < m \Longrightarrow \forall s,t \in L_m\left[
					\Norm{t-s}{} < 2^{-n} \Longrightarrow 
					|X_t(\omega) - X_s(\omega)|<2\sum_{j=n+1}^m 2^{-\gamma j}\right]
				\end{align}
				が成立する.実際,$m=n+1$ならば,$L_{n+1}$の任意の要素$s,t$に対し
				$\Norm{t-s}{} < 2^{-n}$となるのは
				$t \in R_{n+1}(s)$の場合のみであるから,(\refeq{eq:chapter_2_theorem_2_8_2})
				より
				\begin{align}
					\Norm{t-s}{} < 2^{-n} \Longrightarrow 
					|X_t(\omega) - X_s(\omega)| < 2^{-\gamma(n+1)}
					< 2\sum_{j=n+1}^m 2^{-\gamma j}
				\end{align}
				となる.$m=M-1$のとき
				\begin{align}
					\forall s,t \in L_m\left[
					\Norm{t-s}{} < 2^{-n} \Longrightarrow 
					|X_t(\omega) - X_s(\omega)|<2\sum_{j=n+1}^m 2^{-\gamma j}\right]
				\end{align}
				が成立すると仮定すると,$\Norm{t-s}{} < 2^{-n}$を満たす$L_M$の任意の要素$s,t$に対し,
				\begin{itemize}
					\item $s_i < t_i$ならば
						$s_i \leq s'_i \leq t'_i \leq t_i$を満たす
						$\Set{Tk/2^{M-1}}{k\in\{0,1,\cdots,2^{M-1}\}}$の要素$s'_i,t'_i$を取り,
					
					\item $s_i = t_i$のときは,$s_i \in \Set{Tk/2^{M-1}}{k\in\{0,1,\cdots,2^{M-1}\}}$ならば
						$s'_i = s_i,\ t'_i = s_i$とおき,
						$s_i \notin \Set{Tk/2^{M-1}}{k\in\{0,1,\cdots,2^{M-1}\}}$ならば
						$s'_i = s_i-1/2^M,\ t'_i = s_i-1/2^M$とおく
				\end{itemize}
				として$s'=(s'_1,\cdots,s'_d)$と$t'=(t'_1,\cdots,t'_d)$を定めれば,
				\begin{align}
					s',t' \in L_{M-1}\wedge
					\Norm{s-s'}{} < 1/2^M \wedge \Norm{t-t'}{} < 1/2^M
					\wedge \Norm{t'-s'}{} < 1/2^n
				\end{align}
				が満たされるから
				\begin{align}
					|X_s(\omega) - X_{s'}(\omega)| < 2^{-\gamma M},
					\quad |X_{t'}(\omega) - X_t(\omega)| < 2^{-\gamma M},
					\quad |X_{t'}(\omega) - X_{s'}(\omega)|  < 2\sum_{j=n+1}^{M-1} 2^{-\gamma j}
				\end{align}
				となり
				\begin{align}
					|X_t(\omega) - X_s(\omega)|<2\sum_{j=n+1}^M 2^{-\gamma j}
				\end{align}
				が成立する.
			
			\item[第四段]
				実確率変数$h$を
				\begin{align}
					h(\omega) =
					\begin{cases}
						2^{-n^*(\omega)}, & (\omega \in \Omega \backslash N), \\
						0, & (\omega \in N)
					\end{cases}
				\end{align}
				で定める.このとき
				\begin{align}
					\forall \omega \in \Omega \backslash N,\
					\forall s,t \in L,\quad
					0 < \Norm{t-s}{} < h(\omega)
					\Longrightarrow |X_t(\omega)-X_s(\omega)| < \frac{2}{1-2^{-\gamma}}
					\Norm{t-s}{}^\gamma
					\label{eq:chapter_2_theorem_2_8_3}
				\end{align}
				が成立する.いま$\Omega \backslash N$の要素$\omega$と
				$L$の要素$s,t$が任意に与えられたとして,$0 < \Norm{t-s}{} < h(\omega)$のとき,
				$\Norm{t-s}{} < 2^{-n}$を満たす最大の自然数$n$を取れば
				\begin{align}
					n^*(\omega) \leq n \wedge 2^{-(n+1)} \leq \Norm{t-s}{} < 2^{-n}
				\end{align}
				が成立し,前段の結果より
				\begin{align}
					|X_t(\omega) - X_s(\omega)| < 2\sum_{j=n+1}^\infty 2^{-\gamma j}
					= 2 \frac{2^{-\gamma(n+1)}}{1-2^{-\gamma}}
					\leq \frac{2}{1-2^{-\gamma}} \Norm{t-s}{}^\gamma
				\end{align}
				が従う.
				
			\item[第五段]
				$X$の連続な修正$\tilde{X}$を構成する.$\Omega \backslash N$の要素
				$\omega$と$[0,T]^d$の要素$t$が任意に与えられたとき,
				$t$に収束する$L$の列$\{s_n\}_{n\in\N}$を取れば
				(\refeq{eq:chapter_2_theorem_2_8_3})より$\left\{X_{s_n}(\omega)\right\}_{n\in\N}$はCauchy列をなす.
				従って$\lim_{n \in \N}X_{s_n}(\omega)$が存在するが,
				(\refeq{eq:chapter_2_theorem_2_8_3})より$\lim_{n \in \N}X_{s_n}(\omega)$
				は列$\{s_n\}_{n\in\N}$の取り方に依らずに確定する.
				この極限を$\tilde{X}_t(\omega)$と書いて
				\begin{align}
					\tilde{X} = \Biggl\{\ (x,y)\ \, : \quad \exists \omega \in \Omega \exists t \in [0,T]^d [&x=(t,\omega) \wedge \\
					&\omega \in \Omega \backslash N \Longrightarrow y = \tilde{X}_t(\omega) \wedge \\
					&\omega \in N \Longrightarrow y=0] \Biggr\}
				\end{align}
				で写像$\tilde{X}$を定めれば,$\tilde{X}$は
				\begin{align}
					\forall \omega \in \Omega \backslash N,\
					\forall s,t \in L,\quad
					0 < \Norm{t-s}{} < h(\omega)
					\Longrightarrow |\tilde{X}_t(\omega)-\tilde{X}_s(\omega)| < \frac{2}{1-2^{-\gamma}}
					\Norm{t-s}{}^\gamma
				\end{align}
				を満たし,ゆえに$t$の写像と見て連続であり,かつ$[0,T]^d$の任意の要素$t$において,
				任意の正数$\epsilon$に対し
				\begin{align}
					P\left(\left|X_t - \tilde{X}_t\right| > \epsilon\right)
					\leq P\left(\left|X_t - X_{s_n}\right| > \epsilon/2\right) 
						+ P\left(\left|X_{s_n} - \tilde{X}_t\right| > \epsilon/2\right)
					\longrightarrow 0 \quad (n \longrightarrow \infty)
				\end{align}
				が成立するから$P(X_t \neq \tilde{X}_t) = 1$となる.
				\QED
		\end{description}
	\end{prf}
	
	\begin{itembox}[l]{Corollary to Theorem 2.8}
		There is a probability measure $P$ on $(\R^{[0,\infty)},\mathscr{B}(\R^{[0,\infty)}))$,
		and a stochastic process $W = \Set{W_t,\mathscr{F}_t^W}{t \geq 0}$ on the same space,
		such that under $P$, $W$ is a Brownian motion.
	\end{itembox}
	
	\begin{prf}\mbox{}
		\begin{description}
			\item[第一段]
				Corollary to Theorem 2.2より,$(\R^{[0,\infty)},\mathscr{B}(\R^{[0,\infty)}))$にただ一つの確率測度$P$が存在して,
				\begin{align}
					B = \Set{(x,y)}{\exists t \in [0,\infty) \exists \omega \in \R^{[0,\infty)}
					\left( x=(t,\omega) \wedge y=\omega(t)-\omega(0) \right)}
				\end{align}
				で定める写像$B$が$P$の下で
				\begin{itemize}
					\item $\R^{[0,\infty)}$の任意の要素$\omega$に対して
						$B_0(\omega) = 0$,
					\item 任意の実数$s,t$に対し,$0 \leq s < t$ならば
						$B_t - B_s$は$\mathscr{F}_s$と独立,
					\item 任意の実数$s,t$に対し,$0 \leq s < t$ならば
						$P(B_t - B_s)^{-1}$は平均0で分散が$t-s$の正規分布
				\end{itemize}
				となる.Theorem2.8 と Problem2.10 により,1以上の任意の自然数$N$に対し,
				$[0,N]$上で$B$の修正$W^N$が存在する.
				\begin{align}
					\Omega_N &= \Set{\omega \in \R^{[0,\infty)}}{\forall t \in [0,N] \cap \Q,
					\quad W_t^N(\omega) = B_t(\omega)} \\
					&= \bigcap_{t \in [0,N] \cap \Q} \Set{\omega \in \R^{[0,\infty)}}{W_t^N(\omega) = B_t(\omega)}
				\end{align}
				とおけば,$W^N$は$B$の修正であるから$P(\Omega_N)=1$.ここで
				$\tilde{\Omega} = \bigcap_{N \in \N}\Omega_N$とおく.
				0以上の実数$t$と$\tilde{\Omega}$の要素$\omega$が任意に与えられたとき,
				$t < N$を満たす自然数$N$を取れば,$N$以上の任意の自然数$n$で
				\begin{align}
					\forall s \in [0,N] \cap \Q, \quad
					B_s(\omega) = W^N_s(\omega) \wedge B_s(\omega) = W^n_s(\omega)
				\end{align}
				となり,$W^N(\omega)$と$W^n(\omega)$の連続性と定理
				\ref{thm:equivalence_set_of_two_mappings_into_Hausdorff_space_is_closed}より
				$W^N(\omega)$と$W^n(\omega)$は$[0,N]$上で一致する.すなわち
				\begin{align}
					\forall n \in \N,\quad N \leq n \Longrightarrow W^n_t(\omega)
					= W^N_t(\omega)
				\end{align}
				が成り立つから,このとき$\lim_{n \to \infty} W^n_t(\omega)$が確定する.
				\begin{align}
					W_t(\omega) = 
					\begin{cases}
						\lim_{n \to \infty} W^n_t(\omega), & (\omega \in \tilde{\Omega}), \\
						0, & (\omega \in \R^{[0,\infty)} \backslash \tilde{\Omega})
					\end{cases}
				\end{align}
				で$W$を定めれば,$W$は$B$の修正となる.実際,0以上の任意の実数$t$に対し,
				$t < N$を満たす自然数$N$を取れば
				\begin{align}
					\forall \omega \in \tilde{\Omega},\quad 
					W_t(\omega) = W^N_t(\omega)
				\end{align}
				となり,$W^N$が$B$の修正であるから
				\begin{align}
					P(W_t \neq B_t) \leq P(W_t \neq W^N_t) + P(W^N_t \neq B_t) = 0
				\end{align}
				が成立する.またこの$t$において,$W^N(\omega)$の連続性から$W(\omega)$の$t$での連続性が従う.
				
			\item[第二段]
				前段で定めた$W$が$(\R^{[0,\infty)},\mathscr{B}(\R^{[0,\infty)}),P)$の上の
				Brown運動であることを示す.まず$P$-a.s.に$W_0 = B_0$である.また
				$0 \leq s < t$を満たす任意の実数$s,t$に対し,
				\begin{align}
					\Omega' = \Set{\omega \in \R^{[0,\infty)}}{W_s(\omega) \neq B_s(\omega) \wedge W_t(\omega) \neq B_t(\omega)}
				\end{align}
				とおく.$\borel{\R}$の要素$E,F$が任意に与えられたとして,
				\begin{align}
					W_s^{-1}(F) \cap \Omega' = B_s^{-1}(F) \cap \Omega',
					\quad
					(W_t-W_s)^{-1}(E) \cap \Omega' = (B_t-B_s)^{-1}(E) \cap \Omega'
				\end{align}
				が成り立ち,かつ$P(\Omega') = 1$であるから
				\begin{align}
					P\left( W_s^{-1}(F) \right)
					= P\left( W_s^{-1}(F) \cap \Omega' \right)
					&= P\left( B_s^{-1}(F) \cap \Omega' \right)
					= P\left( B_s^{-1}(F) \right), \\
					P\left( (W_t-W_s)^{-1}(E) \right) &= P\left( (B_t-B_s)^{-1}(E) \right), \label{eq:chapter_2_Corollary_to_Theorem_2_8} \\
					P\left( W_s^{-1}(F) \cap (W_t-W_s)^{-1}(E) \right)
					&= P\left( B_s^{-1}(F) \cap (B_t-B_s)^{-1}(E) \right)
				\end{align}
				が従い,$B$の独立増分性と併せて
				\begin{align}
					P\left( W_s^{-1}(F) \cap (W_t-W_s)^{-1}(E) \right)
					&= P\left( B_s^{-1}(F) \cap (B_t-B_s)^{-1}(E) \right) \\
					&= P\left( B_s^{-1}(F) \right) P\left( (B_t-B_s)^{-1}(E) \right) \\
					&= P\left( W_s^{-1}(F) \right) P\left( (W_t-W_s)^{-1}(E) \right) \\
				\end{align}
				となる.以上で$W$の独立増分性が示された.また
				(\refeq{eq:chapter_2_Corollary_to_Theorem_2_8})から
				$W_t-W_s$の分布は$B_t-B_s$の分布に一致する.
				\QED
		\end{description}
	\end{prf}
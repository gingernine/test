\subsection{合併}
	\begin{screen}
		\begin{dfn}[合併]
			$a$を類とするとき
			\begin{align}
				\bigcup a \coloneqq \Set{x}{\exists t \in a\, (\, x \in t\, )}
				\label{eq:definition_of_union_1}
			\end{align}
			で$\bigcup a$を定め,これを$a$の{\bf 合併}\index{がっぺい@合併}{\bf (union)}と呼ぶ.
		\end{dfn}
	\end{screen}
	
	類$a$の合併も
	\begin{align}
		\bigcup a \coloneqq \Set{x}{\exists t\, (\, \varepsilon a(t) \wedge x \in t\, )}
	\end{align}
	と定めるのが本式である.しかし
	\begin{align}
		\forall x\, \left(\, \exists t\, (\, \varepsilon a(t) \wedge x \in t\, )
		\Longleftrightarrow \exists t \in a\, (\, x \in t\, )\, \right)
		\label{eq:definition_of_union_2}
	\end{align}
	が成立するので式(\refeq{eq:definition_of_union_1})を受け入れているのである.
	実際,式(\refeq{eq:a_meaning_of_epsilon_notation})より
	\begin{align}
		\forall t\, \left(\, \varepsilon a(t) \Longleftrightarrow t \in a\, \right)
	\end{align}
	が満たされるので,$\chi$と$\tau$を$\mathcal{L}$の任意の対象とすれば同値関係の遺伝性質より
	\begin{align}
		\varepsilon a(\tau) \wedge \chi \in \tau \Longleftrightarrow \tau \in a \wedge \chi \in \tau
	\end{align}
	が成立する.このとき$\tau$の任意性と推論法則\ref{metathm:properties_of_quantifiers}より
	\begin{align}
		\exists t\, (\, \varepsilon a(t) \wedge \chi \in t\, )
		\Longleftrightarrow \exists t\, (\, t \in a \wedge \chi \in t\, )
	\end{align}
	が成立し,$\chi$の任意性と推論法則\ref{metathm:fundamental_law_of_universal_quantifier}より
	(\refeq{eq:definition_of_union_2})が得られる.正確さも大切だが,やはり判りやすい方が良い.
	
	\begin{screen}
		\begin{axm}[合併の公理]
			集合の合併は集合である.つまり,$a$を類とするとき次が成り立つ:
			\begin{align}
				\set{a} \Longrightarrow \set{\bigcup a}.
			\end{align}
		\end{axm}
	\end{screen}
	
	\begin{screen}
		\begin{thm}[空集合の合併は空]\label{thm:the_union_of_the_emptyset_is_empty}
			次が成立する:
			\begin{align}
				\bigcup \emptyset = \emptyset.
			\end{align}
		\end{thm}
	\end{screen}
	
	\begin{prf}
				$\chi$と$\tau$を$\mathcal{L}$の任意の対象とすれば,
				空集合の公理と推論法則\ref{metathm:fundamental_law_of_universal_quantifier}より
				$\chi \notin \emptyset$が成立し,さらに$\vee$の導入より
				\begin{align}
					\chi \notin \emptyset \vee \tau \notin \chi
				\end{align}
				が成立する.
				ここで$\chi$の任意性と推論法則\ref{metathm:fundamental_law_of_universal_quantifier}より
				\begin{align}
					\forall x\, (\, x \notin \emptyset \vee \tau \notin \emptyset\, )
				\end{align}
				が成り立つ.ここで推論法則\ref{metathm:properties_of_quantifiers}とDe Morganの法則より
				\begin{align}
					\forall x\, (\, x \notin \emptyset \vee \tau \notin \emptyset\, )
					&\Longleftrightarrow \forall x\, \rightharpoondown (\, x \in \emptyset \wedge \tau \in \emptyset\, ) \\
					&\Longleftrightarrow\, \rightharpoondown \exists x\, (\, x \in \emptyset \wedge \tau \in \emptyset\, )
				\end{align}
				が成立するので,三段論法より$\rightharpoondown \exists x\, (\, x \in \emptyset \wedge \tau \in \emptyset\, )$が成立する.
				他方で合併の定義の対偶を取れば
				\begin{align}
					\rightharpoondown \exists x\, (\, x \in \emptyset \wedge \tau \in \emptyset\, )
					\Longleftrightarrow \tau \notin \bigcup \emptyset
				\end{align}
				が満たされるので,再び三段論法より$\tau \notin \bigcup \emptyset$が成立する.
				$\tau$の任意性と推論法則\ref{metathm:fundamental_law_of_universal_quantifier}より
				\begin{align}
					\forall t\, (\, t \notin \bigcup \emptyset\, )
				\end{align}
				が成立し,定理\ref{thm:uniqueness_of_emptyset}より
				\begin{align}
					\bigcup \emptyset = \emptyset
				\end{align}
				が従う.
				\QED
	\end{prf}
	
	\begin{screen}
		\begin{thm}[合併は任意の要素より大きい]\label{thm:union_is_bigger_than_any_member}
			$a$を類とするとき次が成立する:
			\begin{align}
				\forall x\, (\, x \in a \Longrightarrow x \subset \bigcup a\, ).
			\end{align}
		\end{thm}
	\end{screen}
	
	$a,b$を類とするとき,その対の合併を
	\begin{align}
		a \cup b \overset{\mathrm{def}}{\Longleftrightarrow} \bigcup \{a,b\}
	\end{align}
	と書く.
	
	\begin{screen}
		\begin{thm}
			$a,b$を類とするとき次が成立する:
			\begin{align}
				\forall x\, (\, x \in a \cup b \Longrightarrow x \in a \vee x \in b\, ).
			\end{align}
		\end{thm}
	\end{screen}
	
	\begin{screen}
		\begin{thm}[合併の可換律]
			$a,b$を類とするとき次が成立する:
			\begin{align}
				a \cup b = b \cup a.
			\end{align}
		\end{thm}
	\end{screen}
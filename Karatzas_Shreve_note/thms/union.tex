\section{合併}
	\begin{screen}
		\begin{dfn}[合併]
			$a$を類とするとき,$a$の{\bf 合併}\index{がっぺい@合併}{\bf (union)}を
			\begin{align}
				\bigcup a \defeq \Set{x}{\exists t \in a\, (\, x \in t\, )}
				\label{eq:definition_of_union_1}
			\end{align}
			で定める.
		\end{dfn}
	\end{screen}
	
	\begin{screen}
		\begin{axm}[合併の公理]
			集合の合併は集合である.つまり,$a$を類とするとき次が成り立つ:
			\begin{align}
				\set{a} \Longrightarrow \set{\bigcup a}.
			\end{align}
		\end{axm}
	\end{screen}
	
	\begin{screen}
		\begin{thm}[空集合の合併は空]\label{thm:the_union_of_the_emptyset_is_empty}
			次が成立する:
			\begin{align}
				\bigcup \emptyset = \emptyset.
			\end{align}
		\end{thm}
	\end{screen}
	
	\begin{prf}
		$\chi$と$\tau$を$\mathcal{L}$の任意の対象とすれば,定理\ref{thm:emptyset_has_nothing}より
		\begin{align}
			\chi \notin \emptyset
		\end{align}
		が成り立つので
		\begin{align}
			\chi \notin \emptyset \vee \tau \notin \chi
		\end{align}
		が成立し,$\chi$の任意性と推論法則\ref{logicalthm:fundamental_law_of_universal_quantifier}より
		\begin{align}
			\forall x\, (\, x \notin \emptyset \vee \tau \notin x\, )
		\end{align}
		が成り立つ.ここで推論法則\ref{logicalthm:properties_of_quantifiers}より
		\begin{align}
			\forall x\, (\, x \notin \emptyset \vee \tau \notin x\, )
			&\Longleftrightarrow \forall x\, \rightharpoondown (\, x \in \emptyset \wedge \tau \in x\, ) \\
			&\Longleftrightarrow\, \rightharpoondown \exists x\, (\, x \in \emptyset \wedge \tau \in x\, )
		\end{align}
		が成立するので,三段論法より
		\begin{align}
			\rightharpoondown \exists x\, (\, x \in \emptyset \wedge \tau \in x\, )
		\end{align}
		が従う.他方で合併の定義から
		\begin{align}
			\rightharpoondown \exists x\, (\, x \in \emptyset \wedge \tau \in x\, )
			\Longleftrightarrow \tau \notin \bigcup \emptyset
		\end{align}
		が満たされているので,再び三段論法より
		\begin{align}
			\tau \notin \bigcup \emptyset
		\end{align}
		が従う.$\tau$の任意性と推論法則\ref{logicalthm:fundamental_law_of_universal_quantifier}より
		\begin{align}
			\forall t\, (\, t \notin \bigcup \emptyset\, )
		\end{align}
		が成立し,定理\ref{thm:uniqueness_of_emptyset}より
		\begin{align}
			\bigcup \emptyset = \emptyset
		\end{align}
		が従う.
		\QED
	\end{prf}
	
	\begin{screen}
		\begin{thm}[合併は任意の要素より大きい]\label{thm:union_is_bigger_than_any_member}
			$a$を類とするとき次が成立する:
			\begin{align}
				\forall x\, (\, x \in a \Longrightarrow x \subset \bigcup a\, ).
			\end{align}
		\end{thm}
	\end{screen}
	
	\begin{sketch}
		$\chi$を$\mathcal{L}$の任意の対象として
		\begin{align}
			\chi \in a
			\label{fom:thm_union_is_bigger_than_any_member_1}
		\end{align}
		であるとする.また$\tau$も$\mathcal{L}$の任意の対象として
		\begin{align}
			\tau \in \chi
		\end{align}
		であるとする.このとき
		\begin{align}
			\chi \in a \wedge \tau \in \chi
		\end{align}
		が成立するので,存在記号の規則より
		\begin{align}
			\exists x\, \left(\, x \in a \wedge \tau \in x\, \right)
		\end{align}
		が成り立ち
		\begin{align}
			\tau \in \bigcup a
		\end{align}
		が従う.$\tau$は任意に与えられていたので,(\refeq{fom:thm_union_is_bigger_than_any_member_1})の下で
		\begin{align}
			\forall t\, (\, t \in \chi \Longrightarrow t \in \bigcup a\, )
		\end{align}
		すなわち
		\begin{align}
			\chi \subset \bigcup a
		\end{align}
		が成り立つ.ゆえに
		\begin{align}
			\chi \in a \Longrightarrow \chi \subset \bigcup a
		\end{align}
		が従い,$\chi$も任意に与えられていたので
		\begin{align}
			\forall x\, (\, x \in a \Longrightarrow x \subset \bigcup a\, )
		\end{align}
		が得られる.
		\QED
	\end{sketch}
	
	$a,b$を類とするとき,その対の合併を
	\begin{align}
		a \cup b \defeq \bigcup \{a,b\}
	\end{align}
	と書く.
	
	\begin{screen}
		\begin{thm}
			$a,b$を類とするとき次が成立する:
			\begin{align}
				\forall x\, (\, x \in a \cup b \Longrightarrow x \in a \vee x \in b\, ).
			\end{align}
		\end{thm}
	\end{screen}
	
	\begin{screen}
		\begin{thm}[合併の可換律]
			$a,b$を類とするとき次が成立する:
			\begin{align}
				a \cup b = b \cup a.
			\end{align}
		\end{thm}
	\end{screen}
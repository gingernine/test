\subsection{距離化可能性}
	\begin{screen}
		\begin{thm}[全射・単射・像・原像]\label{projective_injective_image_preimage}
			$f$を集合$A$から集合$B$への写像とするとき,
			\begin{description}
				\item[(1)] 任意の$U \subset A$に対し$f^{-1}\left(f(U)\right) \supset U$が成立し,
					特に$f$が単射なら$f^{-1}\left(f(U)\right) = U$となる.
				\item[(2)] 任意の$V \subset B$に対し$f\left(f^{-1}(V)\right) \subset V$が成立し,
					特に$f$が全射なら$f\left(f^{-1}(V)\right) = V$となる.
			\end{description}
		\end{thm}
	\end{screen}
	
	\begin{prf}\mbox{}
		\begin{description}
			\item[(1)] 任意の$x \in U$で$f(x) \in f(U)$となるから
				$x \in f^{-1}\left(f(U)\right)$が成立する.
				$f$が単射であれば,任意の$x \in f^{-1}\left(f(U)\right)$に対し
				$f(x) \in f(U)$となるから或る$x_1 \in U$で$f(x) = f(x_1)$となり,
				単射性より$x = x_1 \in U$が成り立つ.
				
			\item[(2)] 任意に$y \in f\left(f^{-1}(V)\right)$を取れば,
				或る$x \in f^{-1}(V)$で$y = f(x) \in V$となる.$f$が全射であるとき,
				任意の$y \in V$に対し或る$x \in A$が$y = f(x)$を満たすから,
				$x \in f^{-1}(V)$となり$y \in f\left(f^{-1}(V)\right)$が従う.
				\QED
		\end{description}
	\end{prf}
	
	\begin{screen}
		\begin{dfn}[距離化可能]
			位相空間において,その位相と一致する距離位相を定める距離が存在するとき,
			その空間は距離化可能(metrizable)であるという.
		\end{dfn}
	\end{screen}
	
	\begin{screen}
		\begin{thm}[連続単射な開写像による距離化可能性の遺伝]\label{thm:heredity_of_metrizability}
			$X,Y$を位相空間,$f:X \longrightarrow Y$を連続単射な開写像とする.
			$X$が距離$d_X$で距離化可能なら,$f(X)$の相対位相は次で定める$d_Y$により距離化可能である:
			\begin{align}
				d_Y\left(f(x),f(y)\right) = d_X(x,y),
				\quad (\forall x,y \in X).
				\label{eq:thm_heredity_of_metrizability}
			\end{align}
			逆に$f(X)$の相対位相が或る距離$d_Y$で距離化可能であるとき,
			(\refeq{eq:thm_heredity_of_metrizability})で定める$d_X$により$X$は距離化可能である.
		\end{thm}
	\end{screen}
	
	\begin{prf}
		$X$に距離$d_X$が定まっているとき,或は$f(X)$に距離$d_Y$が定まっているとき,
		(\refeq{eq:thm_heredity_of_metrizability})で$d_Y$或は$d_X$を定めれば
		いずれも距離となる.このとき任意の$f(x_0) = y_0$と$r > 0$に対し
		\begin{align}
			B^X_r(x_0) \coloneqq \Set{x \in X}{d_X(x_0,x) < r},
			\quad B^Y_r(y_0) \coloneqq \Set{y \in f(X)}{d_Y(y_0,y) < r}
		\end{align}
		とおけば
		\begin{align}
			f\left(B^X_r(x_0)\right) = B^Y_r(y_0),
			\quad B^X_r(x_0) = f^{-1}\left(B^Y_r(y_0)\right)
			\label{eq:thm_heredity_of_metrizability_2}
		\end{align}
		が成立する.$X$が距離化可能であるとき,$U$を$f(X)$の相対開集合とすれば
		$f^{-1}(U)$は$X$の開集合であるから
		\begin{align}
			f^{-1}(U) = \bigcup_{x \in f^{-1}(U)}B^X_{r_x}(x)
		\end{align}
		と表され,定理\ref{projective_injective_image_preimage}と
		(\refeq{eq:thm_heredity_of_metrizability_2})より
		\begin{align}
			U = f\Biggl(\bigcup_{x \in f^{-1}(U)}B^X_{r_x}(x)\Biggr)
			= \bigcup_{x \in f^{-1}(U)}B^Y_{r_x}(f(x))
		\end{align}
		となるから$U$は$d_Y$による距離位相の開集合である.逆に$V$を$d_Y$による距離位相の開集合とすれば,
		(\refeq{eq:thm_heredity_of_metrizability_2})より$f^{-1}(V)$は
		$d_X$による開球の和で書けるから$X$の開集合であり,$f$が全射開写像であるから
		$V = f\left(f^{-1}(V)\right)$は$f(X)$の相対開集合である.
		後半の主張は$f$の逆写像$f^{-1}:f(X) \longrightarrow X$に対し前半の結果を当てはめて得られる.
		\QED
	\end{prf}
	
	\begin{screen}
		\begin{dfn}[$\sigma$-局所有限]
			位相空間$X$の部分集合族$\mathscr{F}$が局所有限(locally finite)であるとは,
			任意の$x \in X$が$\mathscr{F}$の高々有限個の元としか交叉しない近傍を持つことである.
			位相空間が$\sigma$-局所有限であるとは,その基底が局所有限な部分集合族の可算和で表されることである.
		\end{dfn}
	\end{screen}
	
	\begin{screen}
		\begin{thm}[Nagata-Smirnov]\label{thm:Nagata_Smirnov_metrizability}
			位相空間が距離化可能であるための必要十分条件は,$T_3$かつ$\sigma$-局所有限であることである.
		\end{thm}
	\end{screen}
	
	\begin{prf} $X$を位相空間とする.\mbox{}
		\begin{description}
			\item[第一段] $X$が$T_3$かつ$\sigma$-局所有限であれば,
				$X$は$T_4$かつ全ての閉集合は$G_\delta$となる.
				$X$は局所有限な部分集合族$\mathscr{B}_n$の合併
				$\bigcup_{n=1}^\infty \mathscr{B}_n$で表せる基底を有する.
				任意の$n \in \Z_+$及び$B \in \mathscr{B}_n$に対し
				\begin{align}
					f_{n,B}(x)
					\begin{cases}
						> 0, & (x \in B), \\
						= 0, & (x \in X \backslash B)
					\end{cases}
				\end{align}
		\end{description}
		を満たす連続写像$f_{n,B}:X \longrightarrow [0,1/n]$が存在する.
		\begin{align}
			J \coloneqq \Set{(n,B)}{n \in \Z_+,\ B \in \mathscr{B}_n}
		\end{align}
		とおいて
		\begin{align}
			F(x) \coloneqq \left(f_{n,B}(x)\right)_{(n,B) \in J}
		\end{align}
		により$F$を定めれば,$F$は単射となる.また
		\begin{align}
			d\left((x_j)_{j \in J},(y_j)_{j \in J}\right)
			\coloneqq \sup{j \in J}{|x_j - y_j|}
		\end{align}
		により$[0,1]^J$に距離を定めれば,
		$F$は$X$から$[0,1]^J$への開写像かつ連続写像となる.
		任意の$x_0 \in X$と$\epsilon > 0$に対して
		\begin{align}
			x \in W \quad \Longrightarrow \quad d(F(x),F(x_0)) < \epsilon
			\label{eq:thm_Nagata_Smirnov_metrizability_1}
		\end{align}
		を満たす$x_0$の開近傍$W$が存在する.$n$を固定すれば
		或る$x_0$の近傍$U_n$は$\mathscr{B}_n$の高々有限個の元としか交わらない.
		従って或る開近傍$V_n \subset U_n$が存在し,すべての$B \in \mathscr{B}_n$で
		\begin{align}
			x \in V_n \quad \Longrightarrow \quad d(F(x),F(x_0)) < \epsilon
			\label{eq:thm_Nagata_Smirnov_metrizability_2}
		\end{align}
		が満たされる,$1/N < \epsilon/2$を満たす$N \in \Z_+$を取り
		\begin{align}
			W \coloneqq V_1 \cap \cdots \cap U_N
		\end{align}
		とおけば,$n \leq N$の場合任意の$B \in \mathscr{B}_n$で
		(\refeq{eq:thm_Nagata_Smirnov_metrizability_2})が成立し,
		$n > N$の場合任意の$B \in \mathscr{B}_n$で
		\begin{align}
			d(F(x),F(x_0)) \leq \frac{2}{n} < \epsilon, \quad (\forall x \in X)
		\end{align}
		となるから(\refeq{eq:thm_Nagata_Smirnov_metrizability_1})が成り立つ.
	\end{prf}
\section{Stopping Times}
	\begin{itembox}[l]{$[0,\infty]$の位相}
		$[0,\infty]$の位相は拡張実数$[-\infty,\infty]$の相対位相である.
		$O \subset [-\infty,\infty]$が開集合であるとは,
		任意の$x \in O$に対し,
		\begin{description}
			\item[(O1)] $x \in \R$なら或る$\epsilon > 0$が存在して
				$B_\epsilon(x) \subset O$が満たされる,
			
			\item[(O2)] $x = \infty$なら或る$a \in \R$が存在して
				$(a,\infty] \subset O$が満たされる,
			
			\item[(O3)] $x = -\infty$なら或る$a \in \R$が存在して
				$[-\infty,a) \subset O$が満たされる,
		\end{description}
		で定義される.この性質を満たす$O$の全体に$\emptyset$を加えたものが
		$[-\infty,\infty]$の位相であり,
		\begin{align}
			[-\infty,r),\quad (r,r'), \quad (r,\infty],
			\quad (r,r' \in \Q)
		\end{align}
		の全体が可算開基となる.従って$[0,\infty]$の位相の可算開基は
		\begin{align}
			[0,r),\quad (r,r'), \quad (r,\infty],
			\quad (r,r' \in \Q \cap [0,\infty])
		\end{align}
		の全体であり,写像$\tau:\Omega \longrightarrow [0,\infty]$が
		$\mathscr{F}/\borel{[0,\infty]}$-可測性を持つかどうかを調べるには
		\begin{align}
			\{\tau < a\} = \tau^{-1}([0,a)) \in \mathscr{F},
			\quad (\forall a \in (0,\infty))
		\end{align}
		が満たされているかどうかを確認すれば十分である.
	\end{itembox}
	
	\begin{itembox}[l]{Problem 2.2}
		Let $X$ be a stochastic process and $T$ a stopping time of 
		$\left\{ \mathscr{F}^X_t \right\}$. Suppose that for some pair $\omega,\omega' \in \Omega$, 
		we have $X_t(\omega) = X_t(\omega')$ for all $t \in [0,T(\omega)] \cap [0,\infty)$. 
		Show that $T(\omega) = T(\omega')$. 
	\end{itembox}
	
	\begin{prf}[参照元:\cite{key3}]
		$\omega,\omega'$を分離しない集合族$\mathscr{H}$を
		\begin{align}
			\mathscr{H} \coloneqq \Set{A \subset \Omega}{\{\omega,\omega'\} \subset A,\ or\ \{\omega,\omega'\} \subset \Omega \backslash A}
		\end{align}
		により定めれば,$\mathscr{H}$は$\sigma$-加法族である.このとき,
		$\{T = T(\omega)\} \in \mathscr{H}$を示せばよい.
		\begin{description}
			\item[case1]
				$T(\omega) = \infty$の場合,
				任意の$A \in \borel{\R^d}$及び$0 \leq t < \infty$に対して,
				仮定より
				\begin{align}
					\omega \in X_t^{-1}(A) \quad \Leftrightarrow \quad
					\omega' \in X_t^{-1}(A)
				\end{align}
				が成り立ち
				\begin{align}
					\sigma(X_t;\ 0 \leq t < \infty) \subset \mathscr{H}
				\end{align}
				となる.任意の$t \geq 0$に対し$\{T \leq t\} \in \mathscr{F}^X_t \subset 
				\sigma(X_t;\ 0 \leq t < \infty)$が満たされるから
				\begin{align}
					\{T = \infty\} = \bigcap_{n=1}^\infty \{T \leq n\}^c
					\in \sigma(X_t;\ 0 \leq t < \infty) \subset \mathscr{H}
				\end{align}
				が成立し,$\omega \in \{T = \infty\}$より$\omega' \in \{T = \infty\}$が従い
				$T(\omega) = T(\omega')$を得る.
				
			\item[case2]
				$T(\omega) < \infty$の場合,
				case1と同様に任意の$0 \leq t \leq T(\omega)$に対し
				$\sigma(X_t) \subset \mathscr{H}$が満たされるから
				\begin{align}
					\mathscr{F}^X_{T(\omega)} \subset \mathscr{H}
				\end{align}
				が成り立つ.$\{T = T(\omega)\} \in \mathscr{F}^X_{T(\omega)}$より
				$\omega' \in \{T = T(\omega)\}$が従い$T(\omega) = T(\omega')$を得る.
				\QED
		\end{description}
	\end{prf}
	
	\begin{itembox}[l]{Lemma for Proposition 2.3}
		$(\mathscr{F}_t)_{t \geq 0}$を可測空間
		$(\Omega,\mathscr{F})$のフィルトレーションとするとき,
		任意の$t \geq 0$及び任意の点列$s_1  > s_2 > \cdots > t, (s_n \downarrow t)$
		に対して次が成立する:
		\begin{align}
			\bigcap_{s>t} \mathscr{F}_s = \bigcap_{n=1}^\infty \mathscr{F}_{s_n}.
		\end{align}
	\end{itembox}
	
	\begin{prf}
		先ず任意の$n \geq 1$に対して
		\begin{align}
			\bigcap_{s > t} \mathscr{F}_s \subset \mathscr{F}_{s_n}
		\end{align}
		が成り立つから
		\begin{align}
			\bigcap_{s > t} \mathscr{F}_s \subset \bigcap_{n=1}^\infty \mathscr{F}_{s_n}
		\end{align}
		を得る.一方で,任意の$s > t$に対し$s \geq s_n$を満たす$n$が存在するから,
		\begin{align}
			\mathscr{F}_s \supset  \mathscr{F}_{s_n}
			\supset \bigcap_{n=1}^\infty \mathscr{F}_{s_n}
		\end{align}
		が成立し
		\begin{align}
			\bigcap_{s > t} \mathscr{F}_s \supset \bigcap_{n=1}^\infty \mathscr{F}_{s_n}
		\end{align}
		が従う.
		\QED
	\end{prf}
	
	$(\mathscr{F}_{t+})_{t \geq 0}$は右連続である.実際,任意の$t \geq 0$で
	\begin{align}
		\bigcap_{s > t} \mathscr{F}_{s+} = \bigcap_{s > t} \bigcap_{u > s} \mathscr{F}_u
		= \bigcap_{s > t} \mathscr{F}_s
		= \mathscr{F}_{t+}
	\end{align}
	が成立する.
	
	\begin{itembox}[l]{Corollary 2.4}\label{chapter_1_Corollary_2_4}
		$T$ is an optional time of the filtration $\{\mathscr{F}_t\}$ if and only if 
		it is a stopping time of the (right-continuous!) filtration $\{\mathscr{F}_{t+}\}$.
	\end{itembox}
	言い換えれば,確率時刻$T$に対し
	\begin{align}
		\{T < t\} \in \mathscr{F}_t,\ \forall t \geq 0
		\quad \Leftrightarrow \quad
		\{T \leq t\} \in \mathscr{F}_{t+},\ \forall t \geq 0
	\end{align}
	が成り立つことを主張している.
	\begin{prf}
		$T$が$(\mathscr{F}_{t+})$-停止時刻であるとき,
		任意の$n \geq 1$に対して$\{T \leq t - 1/n\} \in \mathscr{F}_{(t-1/n)+} \subset \mathscr{F}_t$
		が満たされるから
		\begin{align}
			\{T < t\} = \bigcup_{n=1}^\infty \left\{T \leq t - \frac{1}{n}\right\} \in \mathscr{F}_t
		\end{align}
		が従う.逆に$T$が$(\mathscr{F}_t)$-弱停止時刻
		\footnote{
			optional time の訳語がわからないので弱停止時刻と呼ぶ.
		}
		のとき,任意の$m \geq 1$に対し
		\begin{align}
			\{T \leq t\} = \bigcap_{n=m}^\infty \left\{T < t+\frac{1}{n} \right\}
			\in \mathscr{F}_{t + 1/m}
		\end{align}
		が成立するから
		\begin{align}
			\{T \leq t\} \in \bigcap_{n=1}^\infty \mathscr{F}_{t + 1/n} = \mathscr{F}_{t+}
		\end{align}
		を得る.
		\QED
	\end{prf}
	
	\begin{itembox}[l]{Problem 2.6}
		If the set $\Gamma$ in Example 2.5 is open, show that $H_\Gamma$ is 
		an optional time.
	\end{itembox}
	
	\begin{prf}
		$\{H_\Gamma < 0\}=\emptyset$であるから,以下$t > 0$とする.
		$H_\Gamma(\omega) < t \Leftrightarrow \exists s < t,\ X_s(\omega) \in \Gamma$より
		\begin{align}
			\{H_\Gamma < t\} = \bigcup_{0 \leq s < t} \{X_s \in \Gamma\}
		\end{align}
		となる.また全てのパスが右連続であることと$\Gamma$が開集合であることにより
		\begin{align}
			\bigcup_{0 \leq s < t} \{X_s \in \Gamma\}
			= \bigcup_{\substack{0 \leq r < t \\ r \in \Q}} \{X_r \in \Gamma\}
		\end{align}
		が成り立ち$\{H_\Gamma < t\} \in \mathscr{F}_t$が従う.
		\QED
	\end{prf}
	
	\begin{itembox}[l]{Problem 2.7}
		If the set $\Gamma$ in Example 2.5 is closed and the sample paths of the 
		process $X$ are continuous, then $H_\Gamma$ is a stopping time.
	\end{itembox}
	
	\begin{prf}\mbox{}
		\begin{description}
			\item[第一段]
				$\R^d$上のEuclid距離を$\rho$で表し,
				\begin{align}
					\rho(x,\Gamma) \coloneqq \inf{y \in \Gamma}{\rho(x, y)},
					\quad \Gamma_n \coloneqq \Set{x \in \R^d}{\rho(x,\Gamma) < \frac{1}{n}},
					\quad (x \in \R^d,\ n=1,2,\cdots)
				\end{align}
				とおく.$\R^d \ni x \longmapsto \rho(x,\Gamma)$の連続性より$\Gamma_n$は開集合であるから,
				Problem 2.6の結果より$T_n \coloneqq H_{\Gamma_n}$で定める$T_n,\ n=1,2,\cdots$は
				$(\mathscr{F}_t)$-弱停止時刻であり,
				また$H \coloneqq H_\Gamma$とおけば次の(1)と(2)が成立する:
				\begin{description}
					\setlength{\leftskip}{3.0cm}
					\item[(1)] $\{H = 0\} = \{X_0 \in \Gamma\}$,
					
					\setlength{\leftskip}{3.0cm}
					\item[(2)] $H(\omega) \leq t 
					\quad \Leftrightarrow \quad 
					T_n(\omega) < t,\ \forall n=1,2,\cdots,
					\quad (\forall \omega \in \{H>0\},\ \forall t>0)$.
				\end{description}
				(1)と(2)及び$T_n,\ n=1,2,\cdots$が$(\mathscr{F}_t)$-弱停止時刻であることにより
				\begin{align}
					\{H \leq t\}
					= \{H \leq t\} \cap \{H > 0\} + \{H = 0\}
					= \left\{ \bigcap_{n=1}^\infty \{T_n < t\} \right\} \cap \{H > 0\} + \{H = 0\}
					\in \mathscr{F}_t,
					\quad (\forall t \geq 0)
				\end{align}
				が成立するから$H$は$(\mathscr{F}_t)$-停止時刻である.
			
			\item[第二段]
				(1)を示す.実際,
				$X_0(\omega) \in \Gamma$なら$H(\omega) = 0$であり,
				$X_0(\omega) \notin \Gamma$なら,$\Gamma$が閉であることと
				パスの連続性より
				\begin{align}
					X_t(\omega) \notin \Gamma,
					\quad (0 \leq t \leq h)
				\end{align}
				を満たす$h > 0$が存在して$H(\omega) \geq h > 0$となる.
		
			\item[第三段]
				$\omega \in \{H>0\},\ t > 0$として(2)を示す.まずパスの連続性より
				\begin{align}
					T_n(\omega) < t \quad \Leftrightarrow \quad
					\exists s \leq t, \quad X_s(\omega) \in \Gamma_n
				\end{align}
				が成り立つ.$H(\omega) \leq t$の場合,
				$\beta \coloneqq H(\omega)$とおけば,$\Gamma$が閉であることと
				パスの連続性より
				\begin{align}
					X_\beta(\omega) \in \Gamma \subset \Gamma_n,
					\quad (\forall n=1,2,\cdots)
				\end{align}
				が満たされ$T_n(\omega) < t\ (\forall n \geq 1)$が従う.
				逆に,$H(\omega) > t$のとき
				\begin{align}
					X_s(\omega) \notin \Gamma,
					\quad (\forall s \in [0,t])
				\end{align}
				が満たされ,パスの連続性と$\rho$の連続性より
				$[0,t] \ni s \longmapsto \rho(X_s(\omega),\Gamma)$
				は連続であるから,
				\begin{align}
					d \coloneqq \min{s \in [0,t]}{\rho(X_s(\omega),\Gamma)} > 0
				\end{align}
				が定まる.このとき$1/n < d/2$を満たす$n \geq 1$を一つ取れば
				\begin{align}
					X_s(\omega) \notin \Gamma_n,
					\quad (\forall s \in [0,t])
				\end{align}
				が成立する.実際,任意の$s \in [0,t],\ x \in \Gamma_n$に対し
				\begin{align}
					\rho(X_s(\omega),x)
					\geq \rho(X_s(\omega),\Gamma) - \rho(x,\Gamma)
					\geq d - \frac{d}{2}
					= \frac{d}{2}
					> \frac{1}{n}
				\end{align}
				が満たされる.従って$T_n(\omega) \geq t$となる.
				\QED
		\end{description}
	\end{prf}
	
	\begin{itembox}[l]{Lemma 2.9 の式変形について}
		第一の式変形は
		\begin{align}
			\{T + S > t\}
			&= \{T = 0,\ T+S > t\} + \{0 < T < t,\ T+S > t\} + \{T \geq t,\ T+S > t\} \\
			&= \{T = 0,\ T+S > t\} + \{0 < T < t,\ T+S > t\} + \{T \geq t,\ T+S > t,\ S = 0\} \\
				&\quad+ \{T \geq t,\ T+S > t,\ S > 0\} \\
			&= \{T = 0,\ S > t\} + \{0 < T < t,\ T+S > t\} + \{T > t,\ S = 0\}
				+ \{T \geq t,\ S > 0\}
		\end{align}
		である.
	\end{itembox}
	
	\begin{itembox}[l]{Problem 2.10}
		Let $T,S$ be optional times; then $T + S$ is optional. 
		It is a stopping time, if one of the following conditions holds:
		\begin{description}
			\item[(i)] $T > 0,\ S > 0$;
			\item[(ii)] $T > 0,$ $T$ is a stopping time.
		\end{description}
	\end{itembox}
	
	\begin{prf}
		$T,S$が$(\mathscr{F}_t)$-弱停止時刻であるとすれば,
		任意の$t > 0$に対し
		\begin{align}
			\{T + S < t\}
			&= \{T = 0,\ T + S < t\} + \{0 < T < t,\ T + S < t\} \\
			&= \{T = 0,\ S < t\} + \bigcup_{\substack{0 < r < t \\ r \in \Q}} \{0 < T < r,\ S < t-r\} \\
			&\in \mathscr{F}_t
		\end{align}
		が成り立つから$T + S$も$(\mathscr{F}_t)$-弱停止時刻である.
		\begin{description}
			\item[(i)] この場合$\{T + S \leq 0\} = \emptyset$である.また$t > 0$なら
				\begin{align}
					\{T + S > t\} = \{0 < T < t,\ T + S > t\} + \{T \geq t,\ T + S > t \}
					= \bigcup_{\substack{0 < r < t \\ r \in \Q}} \{r < T < t,\ S > t-r\} + \{T \geq t\} \in \mathscr{F}_t
				\end{align}
				が成立する.
				
			\item[(ii)]
				この場合も$\{T + S \leq 0\} = \emptyset$であり,また$t > 0$のとき
				\begin{align}
					\{T + S > t\} &= \{0 < T < t,\ T + S > t\} + \{T \geq t,\ T + S > t \} \\
					&= \{0 < T < t,\ T + S > t\} + \{T \geq t,\ T + S > t,\ S=0 \} + \{T \geq t,\ T + S > t,\ S>0 \} \\
					&= \{0 < T < t,\ T + S > t\} + \{T > t,\ S=0 \} + \{T \geq t,\ S>0 \} \\
					&\in \mathscr{F}_t
				\end{align}
				が成立する.
				\QED
		\end{description}
	\end{prf}
	
	\begin{itembox}[l]{Problem 2.13}
		Verify that $\mathscr{F}_T$ is actually a $\sigma$-field and $T$ is 
		$\mathscr{F}_T$-measurable. Show that if $T(\omega) = t$ for some constant 
		$t \geq 0$ and every $\omega \in \Omega$, then $\mathscr{F}_T = \mathscr{F}_t$.
	\end{itembox}
	
	\begin{prf}\mbox{}
		\begin{description}
			\item[第一段]
				$\mathscr{F}_T$が$\sigma$-加法族であることを示す.実際,
				$\Omega \cap \{T \leq t\} = \{T \leq t\} \in \mathscr{F}_t,\ (\forall t \geq 0)$
				より$\Omega \in \mathscr{F}_T$が従い,また
				\begin{align}
					A^c \cap \{T \leq t\} = \{T \leq t\} - A \cap \{T \leq t\},
					\quad \left\{ \bigcup_{n=1}^\infty A_n \right\} \cap \{T \leq t\}
					= \bigcup_{n=1}^\infty \left( A_n \cap \{T \leq t\} \right)
				\end{align}
				より$\mathscr{F}_T$は補演算と可算和で閉じる.
				
			\item[第二段]
				任意の$\alpha \geq 0$に対し
				\begin{align}
					\{T \leq \alpha \} \cap \{T \leq t\}
					= \{T \leq \alpha \wedge t\}
					\in \mathscr{F}_{\alpha \wedge t} \subset \mathscr{F}_t
				\end{align}
				が成立し$T$の$\mathscr{F}_T/\borel{[0,\infty]}$-可測性が出る.
				
			\item[第三段]
				$A \in \mathscr{F}_T$なら$A = A \cap \{T \leq t\} \in \mathscr{F}_t$となり,
				$A \in \mathscr{F}_t$については,任意の$s \geq 0$に対し
				$s \geq t$なら
				\begin{align}
					A \cap \{T \leq s\} = A \in \mathscr{F} \subset \mathscr{F}_s,
				\end{align}
				$s < t$なら
				\begin{align}
					A \cap \{T \leq s\} = \emptyset \in \mathscr{F}_s
				\end{align}
				が成り立ち$A \in \mathscr{F}_T$が従う.
				\QED
		\end{description}
	\end{prf}
	
	\begin{itembox}[l]{Exercise 2.14}
		Let $T$ be a stopping time and $S$ a random time such that $S \geq T$ 
		on $\Omega$. If $S$ is $\mathscr{F}_T$-measurable, then it is also a stopping time.
	\end{itembox}
	
	\begin{prf}
		任意の$t \geq 0$に対し
		\begin{align}
			\{S \leq t\} = \{S \leq t\} \cap \{T \leq t\} \in \mathscr{F}_t
		\end{align}
		が成立する.
		\QED
	\end{prf}
	
	\begin{itembox}[l]{Problem 2.17 修正}\label{chapter_1_Problem_2_17}
		Let $T,S$ be stopping times and $Z$ an $\mathscr{F}/\borel{\R}$-measurable, 
		integrable random variable. Then
		\begin{align}
			A \in \mathscr{F}_T \quad \Rightarrow \quad A \cap \{T \leq S\}, A \cap \{T < S\} \in \mathscr{F}_{S \wedge T},
		\end{align}
		and we have
		\begin{description}
			\item[(i)] $\defunc_{\{T \leq S\}} \cexp{Z}{\mathscr{F}_T} = \defunc_{\{T \leq S\}} \cexp{Z}{\mathscr{F}_{S \wedge T}},\ \mbox{$P$-a.s.}$
			\item[(ii)] $\defunc_{\{T < S\}} \cexp{Z}{\mathscr{F}_T} = \defunc_{\{T < S\}} \cexp{Z}{\mathscr{F}_{S \wedge T}},\ \mbox{$P$-a.s.}$
			\item[(iii)] $\cexp{\cexp{Z}{\mathscr{F}_T}}{\mathscr{F}_S} = \cexp{Z}{\mathscr{F}_{S \wedge T}},\ \mbox{$P$-a.s.}$
		\end{description}
	\end{itembox}
	
	\begin{prf}\mbox{}
		\begin{description}
			\item[第一段]
				任意の$A \in \mathscr{F}_T$に対し$A \cap \{T \leq S\} \in \mathscr{F}_{S \wedge T}$
				が成り立つ.実際,
				\begin{align}
					A \cap \{T \leq S\} \cap \{S \wedge T \leq t\}
					= \biggl[ A \cap \{T \leq t\} \biggr] \cap \{T \leq S\} \cap \{S \wedge T \leq t\}
					\in \mathscr{F}_t,
					\quad (\forall t \geq 0)
				\end{align}
				が成立する.同様に$A \cap \{T < S\} \in \mathscr{F}_{S \wedge T}$も得られる.
				
			\item[第二段]
				任意の$A \in \mathscr{F}_T$に対し,前段の結果より
				\begin{align}
					\int_{A \cap \{T \leq S\}} Z\ dP
					= \int_{A \cap \{T \leq S\}} \cexp{Z}{\mathscr{F}_{S \wedge T}}\ dP
				\end{align}
				が従う.$\defunc_{\{T \leq S\}} \cexp{Z}{\mathscr{F}_{S \wedge T}}$
				も$\mathscr{F}_T/\borel{\R}$-可測であるから(i)が得られ,同様に(ii)も出る.
			
			\item[第三段]
				任意の$B \in \mathscr{F}_S$に対し,第一段と第二段の結果により
				\begin{align}
					\int_B \cexp{\cexp{Z}{\mathscr{F}_T}}{\mathscr{F}_S}\ dP
					&= \int_B \cexp{Z}{\mathscr{F}_T}\ dP
					= \int_{B\cap\{S < T\}} \cexp{Z}{\mathscr{F}_T}\ dP
						+ \int_{B\cap\{T \leq S\}} \cexp{Z}{\mathscr{F}_T}\ dP \\
					&= \int_{B \cap \{S < T\}} Z\ dP
						+ \int_{B\cap\{T \leq S\}} \cexp{Z}{\mathscr{F}_{S \wedge T}}\ dP \\
					&= \int_{B \cap \{S < T\}} \cexp{Z}{\mathscr{F}_{S \wedge T}}\ dP
						+ \int_{B\cap\{T \leq S\}} \cexp{Z}{\mathscr{F}_{S \wedge T}}\ dP \\
					&= \int_B \cexp{Z}{\mathscr{F}_{S \wedge T}}\ dP
				\end{align}
				が成り立つ.$\cexp{Z}{\mathscr{F}_{S \wedge T}}$も$\mathscr{F}_S/\borel{\R}$-可測
				であるから(iii)を得る.
				\QED
		\end{description}
	\end{prf}
	
	\begin{itembox}[l]{Proposition 2.18}\label{chapter_1_Problem_2_18}
		Let $X = \Set{X_t,\mathscr{F}_t}{0 \leq t < \infty}$ be a progressively measurable 
		process, and let $T$ be a stopping time of the filtration $\{\mathscr{F}_t\}$. 
		Then the random variable $X_T$ of Definition 1.15, defined on the set 
		$\{T < \infty\} \in \mathscr{F}_T$, is $\mathscr{F}_T$-measurable, and
		the ``stopped process'' $\Set{X_{T \wedge t},\mathscr{F}_t}{0 \leq t < \infty}$
		is progressively measurable.
	\end{itembox}
	
	\begin{prf}\mbox{}
		\begin{description}
			\item[第一段]
				停止過程の発展的可測性を示す.$t \geq 0$を固定する.
				このとき,全ての$\omega \in \Omega$に対して
				$[0,t] \ni s \longmapsto T(\omega) \wedge s$は連続であり,かつ
				全ての$s \in [0,t]$に対し$\Omega \ni \omega \longmapsto T(\omega) \wedge s$は
				$\mathscr{F}_t/\borel{[0,t]}$-可測であるから,
				$[0,t] \times \Omega \ni (s,\omega) \longmapsto T(\omega) \wedge s$
				は$\borel{[0,t]} \otimes \mathscr{F}_t/\borel{[0,t]}$-可測である.
				従って,任意の$A \in \borel{[0,t]}$と$B \in \mathscr{F}_t$に対し
				\begin{align}
					\Set{(s,\omega) \in [0,t] \times \Omega}{(T(\omega) \wedge s,\omega) \in A \times B}
					&= \Set{(s,\omega) \in [0,t] \times \Omega}{T(\omega) \wedge s \in A}
					\cap ([0,t] \times B) \\
					&\in \borel{[0,t]} \otimes \mathscr{F}_t
				\end{align}
				が成り立つから,任意の$E \in \borel{[0,t]} \otimes \mathscr{F}_t$
				に対して
				\begin{align}
					\Set{(s,\omega) \in [0,t] \times \Omega}{(T(\omega) \wedge s,\omega) \in E} 
					\in \borel{[0,t]} \otimes \mathscr{F}_t
				\end{align}
				が満たされ$(s,\omega) \longmapsto (T(\omega) \wedge s,\omega)$の
				$\borel{[0,t]} \otimes \mathscr{F}_t/\borel{[0,t]} \otimes \mathscr{F}_t$-可測性を得る.
				\begin{align}
					X(s,\omega) = X|_{[0,t] \times \Omega}(s,\omega),
					\quad (\forall (s,\omega) \in [0,t] \times \Omega)
				\end{align}
				かつ$X|_{[0,t] \times \Omega}$は$\borel{[0,t]} \otimes \mathscr{F}_t/\borel{\R^d}$-可測であるから,
				$[0,t] \times \Omega \ni (s,\omega) \longmapsto X(T(\omega) \wedge s,\omega) 
				= X|_{[0,t] \times \Omega}(T(\omega) \wedge s,\omega)$の
				$\borel{[0,t]} \otimes \mathscr{F}_t/\borel{\R^d}$-可測性が出る.
				
			\item[第二段]
				定理\ref{lem:Fubini_lemma_1} (P. \pageref{lem:Fubini_lemma_1})より
				$\omega \longmapsto X(T(\omega) \wedge t,\omega)$
				は$\mathscr{F}_t/\borel{\R^d}$であるから,
				任意の$B \in \borel{\R^d}$に対し
				\begin{align}
					\left\{ X_T \defunc_{\{T < \infty\}} \in B \right\} \cap \{T \leq t\}
					= \left\{ X_{T \wedge t} \in B \right\} \cap \{T \leq t\}
					\in \mathscr{F}_t,
					\quad (\forall t \geq 0)
				\end{align}
				が成立し$X_T \defunc_{\{T < \infty\}}$の$\mathscr{F}_T/\borel{\R^d}$-可測性を得る.
				\QED
		\end{description}
	\end{prf}
	
	\begin{itembox}[l]{Problem 2.19}
		Under the same assumption as in Proposition 2.18, and with 
		$f(t,x);[0,\infty) \times \R^d \longrightarrow \R$ a bounded,
		$\borel{[0,\infty)} \otimes \borel{\R^d}$-measurable function,
		show that the process $Y_t = \int_0^t f(s,X_s)\ ds;\ t \geq 0$ is
		progressively measurable with respect to $\{\mathscr{F}_t\}$, 
		and  $Y_T$ is an $\mathscr{F}_T$-measurable random variable.
	\end{itembox}
	
	\begin{prf}
		$[0,t] \times \Omega \ni (s,\omega) \longmapsto f(s,X_s(\omega))$
		が$\borel{[0,t]} \otimes \mathscr{F}_t/\borel{\R}$-可測であれば,
		Fuiniの定理より$\Set{Y_t,\mathscr{F}_t}{0 \leq t < \infty}$は
		適合過程となり,可積分性より$t \longmapsto Y_t(\omega),\ (\forall \omega \in \Omega)$が
		連続であるから$Y$の発展的可測性が従う.実際,
		\begin{align}
			[0,t] \times \Omega \ni (s,\omega)
			\longmapsto \left( s,X_s(\omega)\right) 
			= \left( s,X|_{[0,t] \times \Omega}(s,\omega) \right)
		\end{align}
		による$A \times B,\ (A \in \borel{[0,\infty)},\ B \in \borel{\R^d})$の引き戻しは
		\begin{align}
			\left\{ ([0,t] \cap A) \times \Omega \right\} \cap
			X|_{[0,t] \times \Omega}^{-1}(B)
			\in \borel{[0,t]} \otimes \mathscr{F}_t
		\end{align}
		となるから,$[0,t] \times \Omega \ni (s,\omega) \longmapsto f(s,X_s(\omega))$
		は$\borel{[0,t]} \otimes \mathscr{F}_t/\borel{\R}$-可測である.
		\QED
	\end{prf}
	
	\begin{itembox}[l]{Problem 2.21}
		Verify that the class $\mathscr{F}_{T+}$ is indeed a $\sigma$-field
		with respect to which $T$ is measurable, that it coincides with
		$\Set{A \in \mathscr{F}}{A \cap \{T < t\} \in \mathscr{F}_t,\ \forall t \geq 0}$,
		and that if $T$ is a stopping time (so that both $\mathscr{F}_T,\mathscr{F}_{T+}$
		are defined), then $\mathscr{F}_T \subset \mathscr{F}_{T+}$.
	\end{itembox}
	
	\begin{prf}\mbox{}
		\begin{description}
			\item[第一段]
				$\Omega \cap \{T \leq t\} = \{T \leq t\} \in \mathscr{F}_t,\ (\forall t \geq 0)$
				より$\Omega \in \mathscr{F}_{T+}$が従い,また
				\begin{align}
					A^c \cap \{T \leq t\} = \{T \leq t\} - A \cap \{T \leq t\},
					\quad \left\{ \bigcup_{n=1}^\infty A_n \right\} \cap \{T \leq t\}
					= \bigcup_{n=1}^\infty \left( A_n \cap \{T \leq t\} \right)
				\end{align}
				より$\mathscr{F}_{T+}$は補演算と可算和で閉じるから
				$\mathscr{F}_{T+}$は$\sigma$-加法族である.また,
				\begin{align}
					\{T < \alpha\} \cap \{T \leq t\}
					= \begin{cases}
						\{T < \alpha\}, & (\alpha \leq t), \\
						\{T \leq t\}, & (\alpha > t),
					\end{cases}
					\in \mathscr{F}_{t+},
					\quad (\forall t \geq 0)
				\end{align}
				より$(\mathscr{F}_t)$-弱停止時刻$T$は
				$\mathscr{F}_{T+}/\borel{[0,\infty]}$-可測である.
				
			\item[第二段]
				任意の$t \geq 0$に対し
				\begin{align}
					A \cap \{T < t\} = \bigcup_{n=1}^\infty A \cap \left\{T \leq t - \frac{1}{n}\right\},
					\quad A \cap \{T \leq t\} = \bigcap_{n=1}^\infty A \cap \left\{T < t + \frac{1}{n}\right\}
				\end{align}
				が成り立ち$\mathscr{F}_{T+} = \Set{A \in \mathscr{F}}{A \cap \{T < t\} \in \mathscr{F}_t,\ \forall t \geq 0}$
				が従う.
			
			\item[第三段]
				$T$が$(\mathscr{F}_t)$-停止時刻であるとき,
				任意の$A \in \mathscr{F}_T$に対し
				\begin{align}
					A \cap \{T \leq t\} \in \mathscr{F}_t \subset \mathscr{F}_{t+},
					\quad (\forall t \geq 0)
				\end{align}
				となり$\mathscr{F}_T \subset \mathscr{F}_{T+}$が成り立つ.
				\QED
		\end{description}
	\end{prf}
	
	\begin{itembox}[l]{Lemma: 弱停止時刻の可測性}
		$T$を$(\mathscr{F}_t)$-弱停止時刻とすれば,任意の
		$t \geq 0$に対し$T \wedge t$は$\mathscr{F}_t/\borel{[0,\infty)}$-可測である.
	\end{itembox}
	
	\begin{prf}
		任意の$\alpha \geq 0$に対し
		\begin{align}
			\{T \wedge t \leq \alpha\} = 
			\begin{cases}
			\Omega, & (t \leq \alpha), \\
			\{T \leq \alpha \}, & (t > \alpha),
			\end{cases}
			 \in \mathscr{F}_t
		\end{align}
		が成立する.
		\QED
	\end{prf}
	
	\begin{itembox}[l]{Probelem 2.22}
		Verify that analogues of Lemmas 2.15 and 2.16 hold if $T$ and
		$S$ are assumed to be optional and $\mathscr{F}_T,\ \mathscr{F}_S$
		and $\mathscr{F}_{T \wedge S}$ are replaced by $\mathscr{F}_{T+},\ \mathscr{F}_{S+}$
		and $\mathscr{F}_{(T \wedge S)+}$, respectively. Prove that if $S$ is 
		an optional time and $T$ is a positive stopping time with $S \leq T$,
		and $S < T$ on $\{S < \infty\}$, then $\mathscr{F}_{S+} \subset \mathscr{F}_T$.
	\end{itembox}
	
	\begin{prf}\mbox{}
		\begin{description}
			\item[第一段]
				$T \wedge t,\ S \wedge t$は
				$\mathscr{F}_t/\borel{[0,\infty)}$-可測であるから、
				任意の$A \in \mathscr{F}_{S+}$に対して
				\begin{align}
					A \cap \{S \leq T\} \cap \{T \leq t\}
					= (A \cap \{S \leq t\}) \cap \{S \wedge t \leq T \wedge t\} \cap \{T \leq t\}
					\in \mathscr{F}_{t+},
					\quad (\forall t \geq 0)
				\end{align}
				となり$A \cap \{S \leq T\} \in \mathscr{F}_{T+}$が成立する.
				特に,$\Omega$上で$S \leq T$なら$\mathscr{F}_{S+} \subset \mathscr{F}_{T+}$が従う.
				
			\item[第二段]
				前段の結果より$\mathscr{F}_{(T \wedge S)+} \subset \mathscr{F}_{T+} \cap \mathscr{F}_{S+}$
				が満たされる.一方で,任意の$A \in \mathscr{F}_{T+} \cap \mathscr{F}_{S+}$に対し
				\begin{align}
					A \cap \{T \wedge S \leq t\}
					= \left( A \cap \{T \leq t\} \right) \cup \left( A \cap \{S \leq t\} \right)
					\in \mathscr{F}_{t+},
					\quad (\forall t \geq 0)
				\end{align}
				が成り立ち$\mathscr{F}_{(T \wedge S)+} = \mathscr{F}_{T+} \cap \mathscr{F}_{S+}$を得る.
				また
				\begin{align}
					\{S < T\} \cap \{T \wedge S \leq t\}
					= \Biggl( \bigcup_{\substack{0 \leq r \leq t \\ r \in \Q \cup \{t\}}} \{S \leq r\} \cap \{r < T\} \Biggr) 
					\cap \{S \leq t\}
					\in \mathscr{F}_{t+},
					\quad (\forall t \geq 0)
				\end{align}
				により$\{S < T\} \in \mathscr{F}_{(T \wedge S)+}$及び
				$\{T < S\} \in \mathscr{F}_{(T \wedge S)+}$となり,
				$\{T \leq S\},\{S \leq T\},\{T = S\} \in \mathscr{F}_{(T \wedge S)+}$が従う.
			
			\item[第三段]
				$T$が停止時刻で$\{T < \infty\}$上で$S < T$
				が満たされているとき.任意の$A \in \mathscr{F}_{S+}$に対し
				\begin{align}
					A \cap \{T \leq t\}
					= A \cap \{S < t\} \cap \{T \leq t\}
					\in \mathscr{F}_t,
					\quad (\forall t \geq 0)
				\end{align}
				が成り立り$\mathscr{F}_{S+} \subset \mathscr{F}_T$となる.
				\QED
		\end{description}
	\end{prf}
	
	\begin{itembox}[l]{Problem 2.23}
		Show that if $\{T_n\}_{n=1}^\infty$ is a sequence of optional times
		and $T = \inf{n \geq 1}{T_n}$, then $\mathscr{F}_{T+} = \bigcap_{n=1}^\infty \mathscr{F}_{T_n+}$.
		Besides, if each $T_n$ is a positive stopping time and $T < T_n$ on
		$\{T < \infty\}$, then we have $\mathscr{F}_{T+} = \bigcap_{n=1}^\infty \mathscr{F}_{T_n}$.
	\end{itembox}
	
	\begin{prf}
		$T \leq T_n,\ (\forall n \geq 1)$より
		$\mathscr{F}_{T+} \subset \bigcap_{n=1}^\infty \mathscr{F}_{T_n+}$
		が成り立つ.一方で$A \in \bigcap_{n=1}^\infty \mathscr{F}_{T_n+}$に対し
		\begin{align}
			A \cap \{T < t\}
			= \bigcup_{n=1}^\infty A \cap \{T_n < t\}
			\in \mathscr{F}_t,
			\quad (\forall t > 0)
			\label{eq:chapter_1_problem_2_23}
		\end{align}
		が成り立つから,Problem 2.21より$A \in \mathscr{F}_{T+}$が従う.
		また$\{T < \infty\}$上で$T < T_n,\ (\forall n \geq 1)$であるとき,
		Problem 2.22より$\mathscr{F}_{T+} \subset \bigcap_{n=1}^\infty \mathscr{F}_{T_n}$
		が従い,また$T_n,\ n \geq 1$が停止時刻の場合も(\refeq{eq:chapter_1_problem_2_23})は成立するので
		$\mathscr{F}_{T+} = \bigcap_{n=1}^\infty \mathscr{F}_{T_n}$が出る.
		\QED
	\end{prf}
	
	\begin{itembox}[l]{Problem 2.24 修正}\label{chapter_1_Problem_2_24}
		Given an optional time $T$ of the filtration $\{\mathscr{F}_t\}$,
		consider the sequence $\{T_n\}_{n=1}^\infty$ of random times given by
		\begin{align}
			T_n(\omega) = 
			\begin{cases}
				+\infty; & \mbox{on $\Set{\omega}{T(\omega) \geq n}$} \\
				\displaystyle\frac{k}{2^n}; & \mbox{on $\Set{\omega}{\frac{k-1}{2^n} \leq T(\omega) < \frac{k}{2^n}}$ for $k=1,\cdots,n2^n$},
			\end{cases}
		\end{align}
		for $n \geq 1$. Obviously $T_n \geq T_{n+1} \geq T$,
		for every $n \geq 1$. Show that each $T_n$ is a stopping time,
		that $\lim_{n \to \infty} T_n = T$, and that for every $A \in \mathscr{F}_{T+}$
		we have $A \cap \left\{ T_n = (k/2^n) \right\} \in \mathscr{F}_{k/2^n};\ n \geq 1, 1 \leq k \leq n2^n$.
	\end{itembox}
	
	\begin{prf}\mbox{}
		\begin{description}
			\item[第一段]
				$T_n(\omega)<\infty$を満たす$\omega \in \Omega$に対し,
				或る$1 \leq j \leq (n+1)2^{n+1},\ 1\leq k \leq n2^n$が存在して
				\begin{align}
					\frac{j-1}{2^{n+1}} \leq T(\omega) < \frac{j}{2^{n+1}},
					\quad \frac{k-1}{2^n} \leq T(\omega) < \frac{k}{2^n}
				\end{align}
				となる.このとき
				\begin{align}
					\frac{2k-2}{2^{n+1}} \leq T(\omega) < \frac{2k-1}{2^{n+1}}
				\end{align}
				または
				\begin{align}
					\frac{2k-1}{2^{n+1}} \leq T(\omega) < \frac{2k}{2^{n+1}}
				\end{align}
				のどちらかであるから,すなわち$j=2k-1$或は$j=2k$であり
				\begin{align}
					T(\omega) < \frac{j}{2^{n+1}} = T_{n+1}(\omega)
					\leq \frac{2k}{2^{n+1}} = T_n(\omega)
				\end{align}
				が成立する.$T_n(\omega) = \infty$の場合も併せて$T_n \geq T_{n+1} \geq T\ (\forall n \geq 1)$を得る.
			
			\item[第二段]
				任意の$t \geq 0$に対して
				\begin{align}
					\{T_n \leq t\}
					= \bigcup_{k/2^n \leq n \wedge t} \Set{\omega}{\frac{k-1}{2^n} \leq T(\omega) < \frac{k}{2^n}}
					\in \mathscr{F}_t,
					\quad (\forall t \geq 0)
				\end{align}
				が成り立つから$T_n$は$(\mathscr{F}_t)$-停止時刻である.また$\{T < \infty\}$上では
				$T(\omega) < n$のとき
				\begin{align}
					0 < T_n(\omega) - T(\omega) \leq \frac{1}{2^n} \longrightarrow 0
					\quad (n \longrightarrow \infty)
				\end{align}
				となる.
			
			\item[第三段]
				任意の$A \in \mathscr{F}_{T+}$に対して,Problem 2.21より
				\begin{align}
					A \cap \left\{T_n = \frac{k}{2^n}\right\}
					= A \cap \left\{T < \frac{k}{2^n}\right\}
					- A \cap \left\{T < \frac{k-1}{2^n}\right\}
					\in \mathscr{F}_{k/2^n}
				\end{align}
				が成り立つ.
				\QED
		\end{description}
	\end{prf}
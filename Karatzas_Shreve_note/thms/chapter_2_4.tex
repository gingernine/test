\section{The Space $C[0,\infty)$, Weak Convergence, and the Wiener Measure}
	\begin{itembox}[l]{Problem 4.1}
		Show that $\rho$ defined by (4.1) is a metric on $C[0,\infty)^d$ and, under $\rho$, 
		$C[0,\infty)^d$ is a complete, separable metric space.
	\end{itembox}
	以下,$C[0,\infty)^d$には$\rho$により広義一様収束位相を導入する.
	
	\begin{prf}
		付録の定理\ref{thm:appendix_complete_separability_of_spaces_of_continuous_functions}により従う.
		\QED
	\end{prf}

\begin{itembox}[l]{Problem 4.2}\label{chapter_2_problem_4_2}
	Let $\mathscr{C}(\mathscr{C}_t)$ be the collection of finite-dimensional cylinder sets of the form (2.1); i.e.,
	\begin{align}
		C = \Set{\omega \in C[0,\infty)^d}{(\omega(t_1),\cdots,\omega(t_n)) \in A};
		\quad n \geq 1,\ A \in \borel{(\R^d)^n},
	\end{align}
	where, for all $i=1,\cdots,n,\ t_i \in [0,\infty)$ (respectively, $t_i \in [0,t]$).
	Denote by $\mathscr{G}(\mathscr{G}_t)$ the smallest $\sigma$-field containing $\mathscr{C}(\mathscr{C}_t)$.
	Show that $\mathscr{G} = \borel{C[0,\infty)^d}$, the Borel $\sigma$-field generated by the open sets in
	$C[0,\infty)^d$, and that $\mathscr{G}_t = \varphi_t^{-1}\left( \borel{C[0,\infty)^d} \right) \eqqcolon
	\mathscr{B}_t\left( C[0,\infty)^d \right)$, where $\varphi_t:C[0,\infty)^d \longrightarrow C[0,\infty)^d$ is the
	mapping $(\varphi_t\omega)(s) = \omega(t \wedge s);\ 0 \leq s < \infty$.
\end{itembox}

\begin{prf}\mbox{}
	\begin{description}
		\item[第一段]
			$w_0 \in C[0,\infty)^d$とする.任意に$w \in C[0,\infty)^d$を取れば,$w$の連続性により$d(w_0,w)$の各項について
			\begin{align}
				\sup{t \leq n}{|w_0(t) - w(t)|} = \sup{r \in [0,n]\cap\Q}{|w_0(r) - w(r)|} \quad (n = 1,2,\cdots)
			\end{align}
			とできる.いま,任意に実数$\alpha \in \R$を取れば
			\begin{align}
				\Set{w \in C[0,\infty)^d}{\sup{r \in [0,n]\cap\Q}{|w_0(r) - w(r)|} \leq \alpha}
				= \bigcap_{r \in [0,n]\cap\Q} \Set{w \in C[0,\infty)^d}{|w_0(r) - w(r)| \leq \alpha}
			\end{align}
			が成立し,右辺の各集合は
			$\mathscr{C}$に属するから$\mbox{左辺} \in \sgmalg{\mathscr{C}}$となる.従って
			\begin{align}
				\psi_n : C[0,\infty)^d \ni w \longmapsto \sup{r \in [0,n]\cap\Q}{|w_0(r) - w(r)|} \in \R, \quad (n = 1,2,\cdots)
			\end{align}
			で定める$\psi_n$は可測$\sgmalg{\mathscr{C}}/\borel{\R}$である.
			$x \longmapsto x \wedge 1$の連続性より$\psi_n \wedge 1$
			も$\sgmalg{\mathscr{C}}/\borel{\R}$-可測性を持ち,
			\begin{align}
				d(w_0,w) = \sum_{n=1}^{\infty}\frac{1}{2^n} \left( \psi_n(w) \wedge 1 \right)
			\end{align}
			により$C[0,\infty)^d \ni w \longmapsto d(w_0,w) \in \R$の
			$\sgmalg{\mathscr{C}}/\borel{\R}$-可測性が出るから,任意の$\epsilon > 0$に対する球について
			\begin{align}
				\Set{w \in C[0,\infty)^d}{d(w_0,w) < \epsilon} \in \sgmalg{\mathscr{C}}
			\end{align}
			が成り立つ.$C[0,\infty)^d$は第二可算公理を満たし,可算基底は上式の形の球で構成されるから,
			$\open{C[0,\infty)^d} \subset \sgmalg{\mathscr{C}}$が従い$\borel{C[0,\infty)^d} \subset \sgmalg{\mathscr{C}}$を得る.
			次に逆の包含関係を示す.いま任意に$n \in \Z_+$と$t_1 < \cdots < t_n$を選んで
			\begin{align}
				\phi : C[0,\infty)^d \ni w \longmapsto (w(t_1),\cdots, w(t_n)) \in (\R^d)^n
			\end{align}
			で定める写像は連続である.
			実際,任意の一点$w_0$での連続性を考えると,任意の$\epsilon > 0$に対して$t_n \leq N$を満たす$N \in \N$を取れば,
			$d(w_0,w) < \epsilon/(n2^N)$ならば
			$\sum_{i=1}^{n}|w_0(t_i) - w(t_i)| < \epsilon$が成り立つ.よって
			$\phi$は$w_0$で連続であり
			\begin{align}
				\borel{(\R^d)^n} \subset \Set{A \in \borel{(\R^d)^n}}{\phi^{-1}(A) \in \borel{C[0,\infty)^d}}
			\end{align}
			が出る.任意の$C \in \mathscr{C}$は,$n \in \N$と時点$t_1 < \cdots < t_n$によって決まる写像$\phi$によって
			$C = \phi^{-1}(B)\ (\exists B \in \borel{(\R^d)^n})$と表現できるから,
			$\mathscr{C} \subset \borel{C[0,\infty)^d}$が成り立ち
			$\sgmalg{\mathscr{C}} \subset \borel{C[0,\infty)^d}$が得られる.
			
		\item[第二段]
			$t \geq 0$とする.$C[0,\infty)^d$の位相を$\open{C[0,\infty)^d}$と書けば
			\begin{align}
				\varphi_t^{-1}\left( \borel{C[0,\infty)^d} \right)
				= \sgmalg{\Set{\varphi_t^{-1}(O)}{O \in \open{C[0,\infty)^d}}}
			\end{align}
			が成り立つ.任意の$\alpha \in \R$と$r \geq 0$に対して
			\begin{align}
				&\Set{w \in C[0,\infty)^d}{|w_0(r) - (\varphi_t w)(r)| \leq \alpha} \\
				&\qquad = \begin{cases}
					\Set{w \in C[0,\infty)^d}{|w_0(r) - (\varphi_t w)(r)| \leq \alpha}, & (r \leq t), \\
					\Set{w \in C[0,\infty)^d}{|w_0(r) - (\varphi_t w)(t)| \leq \alpha}, & (r > t),
				\end{cases}
				\quad \in \mathscr{C}_t
			\end{align}
			となるから
			\begin{align}
				\psi^t_n : C[0,\infty)^d \ni w \longmapsto \sup{r \in [0,n]\cap\Q}{|w_0(r) - (\varphi_t w)(r)|} \in \R, \quad (n = 1,2,\cdots)
			\end{align}
			で定める$\psi^t_n$は可測$\sgmalg{\mathscr{C}_t}/\borel{\R}$である.
			$x \longmapsto x \wedge 1$の連続性より$\psi^t_n \wedge 1$
			も$\sgmalg{\mathscr{C}_t}/\borel{\R}$-可測性を持ち,
			\begin{align}
				d(w_0,\varphi_t w) = \sum_{n=1}^{\infty}\frac{1}{2^n} \left( \psi^t_n(w) \wedge 1 \right)
			\end{align}
			により$C[0,\infty)^d \ni w \longmapsto d(w_0,\varphi_t w) \in \R$の
			$\sgmalg{\mathscr{C}_t}/\borel{\R}$-可測性が出るから,任意の$\epsilon > 0$に対する球について
			\begin{align}
				\Set{w \in C[0,\infty)^d}{d(w_0,\varphi_t w) < \epsilon} \in \sgmalg{\mathscr{C}_t}
			\end{align}
			が成り立つ.特に
			\begin{align}
				\varphi_t^{-1}\left( \Set{w \in C[0,\infty)^d}{d(w_0,w) < \epsilon} \right)
				= \Set{w \in C[0,\infty)^d}{d(w_0,\varphi_t w) < \epsilon}
			\end{align}
			が満たされ,$C[0,\infty)^d$の第二可算性より
			\begin{align}
				\varphi_t^{-1}(O) \in \sgmalg{\mathscr{C}_t},
				\quad (\forall O \in \open{C[0,\infty)^d})
			\end{align}
			が従う.ゆえに$\varphi_t^{-1}\left( \borel{C[0,\infty)^d} \right) \subset \sgmalg{\mathscr{C}_t}$となる.
			\QED
	\end{description}
\end{prf}

\begin{comment}
次の事柄は後の定理の証明で使うからここで証明しておく.
\begin{screen}
	\begin{thm}[$\mathscr{C}$は乗法族である]
		$\mathscr{C}$は交演算について閉じている.
	\end{thm}
\end{screen}
\begin{prf}
	任意に$A_1, A_2 \in \mathscr{C}$を取れば,$A_1,\ A_2$それぞれに対し
	$n_1,n_2 \in \N,\ C_1 \in \borel{(\R^d)^{n_1}},\ C_2 \in \borel{(\R^d)^{n_2}},\ t_1<\cdots<t_{n_1}$それから
	$s_1<\cdots<s_{n_2}$が決まっていて,
	\begin{align}
		A_1 &= \left\{\ w \in C[0,\infty)^d\quad |\quad (w(t_1), \cdots, w(t_{n_1})) \in C_1\ \right\} \\
		A_2 &= \left\{\ w \in C[0,\infty)^d\quad |\quad (w(s_1), \cdots, w(s_{n_2})) \in C_2\ \right\}
	\end{align}
	と表されている.$A_1,A_2$の時点に重複があるかないかで場合分けして示す.
	\begin{description}
	\item[時点に重複がない場合]
		集合を次のように同値な表記に直す:
		\begin{align}
			A_1 &= \left\{\ w \in C[0,\infty)^d\quad |\quad (w(t_1), \cdots, w(t_{n_1}),w(s_1), \cdots, w(s_{n_2})) \in C_1 \times (\R^d)^{n_2}\ \right\} \\
			A_2 &= \left\{\ w \in C[0,\infty)^d\quad |\quad (w(t_1), \cdots, w(t_{n_1}),w(s_1), \cdots, w(s_{n_2})) \in (\R^d)^{n_1} \times C_2\ \right\}
		\end{align}
		表現を変えれば乗法を考えやすくなり,上の場合は
		\begin{align}
			A_1 \cap A_2 = \left\{\ w \in C[0,\infty)^d\quad |\quad (w(t_1), \cdots, w(t_{n_1}),w(s_1), \cdots, w(s_{n_2})) \in C_1 \times C_2\ \right\}
		\end{align}
		と表現できる.$t_1,\cdots,s_{n_2}$の並びが気になるなら,この時点の並びを昇順に変換する$(dn_1 + dn_2) \times (dn_1 + dn_2)$行列$J_1$
		を用いて($J_1$は連続,線型,全単射),
		\begin{align}
			A_1 \cap A_2 &= \left\{\ w \in C[0,\infty)^d\quad |\quad J_1\Vector{w} \in J_1(C_1 \times C_2)\ \right\} \\
			\left(\Vector{w} \right.&= {}^{T}(w(t_1), \cdots, \left.w(t_{n_1}),w(s_1), \cdots, w(s_{n_2}))\right)
		\end{align}
		とすれば,$J(C_1 \times C_2) \in \borel{(\R^d)^{n_1 + n_2}}$であるから,$A_1 \cap A_2 \in \mathscr{C}$であることが明確になる.
	
	\item[時点に重複がある場合]
		$(r_{k_1},\cdots,r_{k_l}) \subset (t_1,\cdots,t_{n_1})$が重複時点であるとき,$A_1,A_2$の同値な表記は次のようにすればよい:
		\begin{align}
			A_1 &= \left\{\ w \in C[0,\infty)^d\quad |\quad (w(t_1),.,w(r_{k_1}),.,w(r_{k_l}),.,w(t_{n_1}),\mbox{\scriptsize($s_1, \cdots, s_{n_2}$から$r_{k_1},\cdots,r_{k_l}$を抜いたものを並べる)}) \in C_1 \times (\R^d)^{n_2 - l}\ \right\} \\
			A_2 &= \left\{\ w \in C[0,\infty)^d\quad |\quad (w(s_1),.,w(r_{k_1}),.,w(r_{k_l}),.,w(s_{n_2}),\mbox{\scriptsize($t_1, \cdots, t_{n_1}$から$r_{k_1},\cdots,r_{k_l}$を抜いたものを並べる)}) \in C_2 \times (\R^d)^{n_1 - l}\ \right\}
		\end{align}
		$A_2$について,条件中の時点の並びを変換し$A_1$の条件の順番に合わせる行列$J_2$(連続,線型,全単射)を用いて
		\begin{align}
			A_2 = \left\{\ w \in C[0,\infty)^d\quad |\quad (w(t_1),.,w(r_{k_1}),.,w(r_{k_l}),.,w(t_{n_1}),\mbox{\scriptsize($s_1, \cdots, s_{n_2}$から$r_{k_1},\cdots,r_{k_l}$を抜いたものを並べる)}) \in J_2(C_2 \times (\R^d)^{n_1 - l})\ \right\}
		\end{align}
		と書き直せば,$A_1 \cap A_2$は前段の様に表現可能であり,前段と同様に最後に時点を昇順に変換する行列を用いることで$A_1 \cap A_2 \in \mathscr{C}$となることが明確に判る.
	\end{description}
	\QED
\end{prf}
\end{comment}
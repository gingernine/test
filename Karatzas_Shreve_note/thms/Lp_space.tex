\section{$L^p$空間}

測度空間を$(X,\mathscr{F},\mu)$とする.$\mathscr{F}/\borel{\C}$-可測関数$f$に対して
\begin{align}
	\Norm{f}{\mathscr{L}^p} \coloneqq
	\begin{cases}
		\inf{}{\Set{r \in \C}{|f(x)| \leq r\quad \mbox{$\mu$-a.e.}x \in X}} & (p = \infty) \\
		\displaystyle\left( \int_{X} |f(x)|^p\ \mu(dx) \right)^{1/p} & (0 < p < \infty)
	\end{cases}
\end{align}
により$\Norm{\cdot}{\mathscr{L}^p}$を定め,
\begin{align}
	\mathscr{L}^p(X,\mathscr{F},\mu) \coloneqq \Set{f:X \rightarrow \C}{f:\mbox{可測}\mathscr{F}/\borel{\C},\ \Norm{f}{\mathscr{L}^p} < \infty} \quad (1 \leq p \leq \infty)
\end{align}
で空間$\mathscr{L}^p(X,\mathscr{F},\mu)$を定義する.$\mathscr{L}^p(\mu)$とも略記する.

\begin{screen}
	\begin{lem}\label{lem:holder_inequality}
		任意の$f \in \mathscr{L}^\infty(X,\mathscr{F},\mu)$に対して次が成り立つ:
		\begin{align}
			|f| \leq \Norm{f}{\mathscr{L}^\infty} \quad \mbox{$\mu$-a.e.}
		\end{align}
	\end{lem}
\end{screen}

\begin{prf}
	$\mathscr{L}^\infty(X,\mathscr{F},\mu)$の定義より任意の実数$\alpha > \Norm{f}{\mathscr{L}^\infty}$に対して
	\begin{align}
		\mu\left( \Set{x \in X}{|f(x)| > \alpha} \right) = 0
	\end{align}
	が成り立つから,
	\begin{align}
		\Set{x \in X}{|f(x)| > \Norm{f}{\mathscr{L}^\infty}} = \bigcup_{n =1}^{\infty} \Set{x \in X}{|f(x)| > \Norm{f}{\mathscr{L}^\infty} + \frac{1}{n}}
	\end{align}
	の右辺は$\mu$-零集合であり主張が従う.
	\QED
\end{prf}

\begin{screen}
	\begin{thm}[H\Ddot{o}lderの不等式]\label{thm:holder_inequality}
		$1 \leq p, q \leq \infty$,$p + q = pq\ (p = \infty$なら$q = 1)$とする.このとき
		任意の$\mathscr{F}/\borel{\C}$-可測関数$f,g$に対して次が成り立つ:
		\begin{align}
			\int_{X} |fg|\ d\mu \leq \Norm{f}{\mathscr{L}^p} \Norm{g}{\mathscr{L}^q}. \label{ineq:holder}
		\end{align}
	\end{thm}
\end{screen}

\begin{prf}
	$\Norm{f}{\mathscr{L}^p} = \infty$又は$\Norm{g}{\mathscr{L}^q} = \infty$なら(\refeq{ineq:holder})
		は成り立つから,$\Norm{f}{\mathscr{L}^p} < \infty$かつ$\Norm{g}{\mathscr{L}^q} < \infty$とする.
	\begin{description}
		\item[$p = \infty,\ q = 1$の場合]
			補題\ref{lem:holder_inequality}により或る零集合$A$が存在して
			\begin{align}
				|f(x)g(x)| \leq \Norm{f}{\mathscr{L}^\infty}|g(x)| \quad (\forall x \in X \backslash A).
			\end{align}
			が成り立つから,
			\begin{align}
				\int_{X} |fg|\ d\mu = \int_{X \backslash A} |fg|\ d\mu
				\leq \Norm{f}{\mathscr{L}^\infty} \int_{X \backslash A} |g|\ d\mu 
				= \Norm{f}{\mathscr{L}^\infty} \Norm{g}{\mathscr{L}^1}
			\end{align}
			が従い不等式(\refeq{ineq:holder})を得る.
		
		\item[$1 < p,q < \infty$の場合]
			$\Norm{f}{\mathscr{L}^p} = 0$のとき
			\begin{align}
				B \coloneqq \Set{x \in X}{|f(x)| > 0}
			\end{align}
			は零集合であるから,
			\begin{align}
				\int_{X} |fg|\ d\mu = \int_{X \backslash B} |fg|\ d\mu = 0
			\end{align}
			となり(\refeq{ineq:holder})を得る.$\Norm{g}{\mathscr{L}^q} = 0$の場合も同じである.
			次に$0 < \Norm{f}{\mathscr{L}^p},\Norm{g}{\mathscr{L}^q} < \infty$の場合を示す.
			実数値対数関数$(0,\infty) \ni t \longmapsto -\Log{t}$は凸であるから,$1/p + 1/q = 1$に対して
			\begin{align}
				-\Log{\left( \frac{s}{p} + \frac{t}{q} \right)} \leq \frac{1}{p}(-\Log{s}) + \frac{1}{q}(-\Log{t}) \quad (\forall s,t > 0)
			\end{align}
			を満たし
			\begin{align}
				s^{1/p}t^{1/q} \leq \frac{s}{p} + \frac{t}{q} \quad (\forall s,t > 0)
			\end{align}
			が従う.ここで
			\begin{align}
				F \coloneqq \frac{|f|^p}{\Norm{f}{\mathscr{L}^p}^p},
				\quad G \coloneqq \frac{|g|^q}{\Norm{g}{\mathscr{L}^q}^q}
			\end{align}
			により可積分関数$F,G$を定めれば,
			\begin{align}
				F(x)^{1/p}G(x)^{1/q} \leq \frac{1}{p}F(x) + \frac{1}{q}G(x) \quad (\forall x \in X)
			\end{align}
			が成り立つから
			\begin{align}
				\frac{1}{\Norm{f}{\mathscr{L}^p}\Norm{g}{\mathscr{L}^q}}\int_{X} |fg|\ d\mu
				= \int_{X} F^{1/p}G^{1/q}\ d\mu
				\leq \frac{1}{p} \int_{X} F\ d\mu + \frac{1}{q} \int_{X} G\ d\mu
				= \frac{1}{p} + \frac{1}{q} = 1
			\end{align}
			が従い,$\Norm{f}{\mathscr{L}^p}\Norm{g}{\mathscr{L}^q}$を移項して不等式(\refeq{ineq:holder})を得る.
			\QED
	\end{description}
\end{prf}

\begin{screen}
	\begin{thm}[Minkowskiの不等式]\label{thm:minkowski_inequality}
		$1 \leq p \leq \infty$のとき,
		任意の$\mathscr{F}/\borel{\C}$-可測関数$f,g$に対して次が成り立つ:
		\begin{align}
			\Norm{f+g}{\mathscr{L}^p} \leq \Norm{f}{\mathscr{L}^p} + \Norm{g}{\mathscr{L}^p}. \label{ineq:minkowski}
		\end{align}
	\end{thm}
\end{screen}

\begin{prf}
	$\Norm{f+g}{\mathscr{L}^p} = 0,\ \Norm{f}{\mathscr{L}^p} = \infty,\ \Norm{g}{\mathscr{L}^p} = \infty$
	のいずれかが満たされていれば(\refeq{ineq:minkowski})は成り立つから,
	$\Norm{f+g}{\mathscr{L}^p} > 0$かつ$\Norm{f}{\mathscr{L}^p} < \infty$かつ$\Norm{g}{\mathscr{L}^p} < \infty$
	の場合を考える.
	\begin{description}
		\item[$p = \infty$の場合]
			補題\ref{lem:holder_inequality}により
			\begin{align}
				C \coloneqq \Set{x \in X}{|f(x)| > \Norm{f}{\mathscr{L}^\infty}} \cup \Set{x \in X}{|g(x)| > \Norm{g}{\mathscr{L}^\infty}}
			\end{align}
			は零集合であり,
			\begin{align}
				|f(x) + g(x)| \leq |f(x)| + |g(x)| \leq \Norm{f}{\mathscr{L}^\infty} + \Norm{g}{\mathscr{L}^\infty} \quad (\forall x \in X \backslash C)
			\end{align}
			が成り立ち(\refeq{ineq:minkowski})が従う.
		
		\item[$p = 1$の場合]
			\begin{align}
				\int_X |f + g|\ d\mu \leq \int_X |f| + |g|\ d\mu = \Norm{f}{\mathscr{L}^1} + \Norm{g}{\mathscr{L}^1}
			\end{align}
			より(\refeq{ineq:minkowski})が従う.
		
		\item[$1 < p < \infty$の場合]
			$q$を$p$の共役指数とする.
			\begin{align}
				|f+g|^p = |f+g||f+g|^{p-1} \leq |f||f+g|^{p-1} + |g||f+g|^{p-1}
			\end{align}
			が成り立つから,H\Ddot{o}lderの不等式より
			\begin{align}
				\Norm{f+g}{\mathscr{L}^p}^p &= \int_{X} |f+ g|^p\ d\mu \\
				&\leq \int_{X} |f||f+g|^{p-1}\ d\mu + \int_{X} |g||f+g|^{p-1}\ d\mu \\
				&\leq \Norm{f}{\mathscr{L}^p}\Norm{f+g}{\mathscr{L}^p}^{p-1} + \Norm{g}{\mathscr{L}^p}\Norm{f+g}{\mathscr{L}^p}^{p-1}
				\label{Minkowski_1}
			\end{align}
			が得られる.また$|f|^p,|g|^p$の可積性と
			\begin{align}
				|f + g|^p \leq 2^p \left( |f|^p + |g|^p \right)
			\end{align}
			により$\Norm{f+g}{\mathscr{L}^p} < \infty$が従うから,
			(\refeq{Minkowski_1})の両辺を$\Norm{f+g}{\mathscr{L}^p}^{p-1}$で割って(\refeq{ineq:minkowski})を得る.
			\QED
	\end{description}
\end{prf}

以上の結果より$\mathscr{L}^p(X,\mathscr{F},\mu)$は線形空間となる.実際線型演算は
\begin{align}
	(f+g)(x) \coloneqq f(x) + g(x), \quad (\alpha f)(x) \coloneqq \alpha f(x),
	\quad (\forall x \in X,\ f,g \in \mathscr{L}^p(\mu),\ \alpha \in \C)
\end{align}
により定義され,Minkowskiの不等式により加法について閉じている.

\begin{screen}
	\begin{lem}
		$1 \leq p \leq \infty$に対し,$\Norm{\cdot}{\mathscr{L}^p}$は線形空間$\mathscr{L}^p(X,\mathscr{F},\mu)$のセミノルムである.
	\end{lem}
\end{screen}

\begin{prf}\mbox{}
	\begin{description}
	\item[半正値性] $\Norm{\cdot}{\mathscr{L}^p}$が正値であることは定義による.
		一方で,$E \neq \emptyset$を満たす$\mu$-零集合$E$が存在するとき,
		\begin{align}
			f(x) \coloneqq
			\begin{cases}
				1 & (x \in E) \\
				0 & (x \in \Omega \backslash E)
			\end{cases}
		\end{align}
		で定める$f$は零写像ではないが$\Norm{f}{\mathscr{L}^p} = 0$となる.
		
	\item[同次性] 
		任意に$\alpha \in \C,\ f \in \mathscr{L}^p(\mu)$を取る.
		$1 \leq p < \infty$の場合は
		\begin{align}
			\left( \int_{X} |\alpha f|^p\ d\mu \right)^{1/p} = \left( |\alpha|^p \int_{X} |f|^p\ d\mu \right)^{1/p} 
			= |\alpha| \left( \int_{X} |f|^p\ d\mu \right)^{1/p}
		\end{align}
		により,$p = \infty$の場合は
		\begin{align}
			\inf{}{\Set{r \in \R}{|\alpha f(x)| \leq r \quad \mbox{$\mu$-a.e.}x \in X}} = |\alpha|\inf{}{\Set{r \in \R}{|f(x)|  \leq r \quad \mbox{$\mu$-a.e.}x \in X}}
		\end{align}
		により$\Norm{\alpha f}{\mathscr{L}^p} = |\alpha|\Norm{f}{\mathscr{L}^p}$が成り立つ.
		
	\item[三角不等式] Minkowskiの不等式より従う.
	\QED
	\end{description}
\end{prf}

$\mathscr{L}^p$はノルム空間ではないが,同値類でまとめることによりノルム空間となる.
\begin{description}
	\item[可測関数全体の商集合]
		$\mathscr{F}/\borel{\C}$-可測関数全体の集合を
		\begin{align}
			\mathscr{L}^0(X,\mathscr{F},\mu) \coloneqq \Set{f:X \rightarrow \C}{f:\mbox{可測}\mathscr{F}/\borel{\C}}
		\end{align}
		とおく.$f,g \in \mathscr{L}^0(X,\mathscr{F},\mu)$に対し
		\begin{align}
			 f \sim g \quad \overset{\mathrm{def}}{\Longleftrightarrow} \quad f = g \quad \mbox{$\mu$-a.e.}
		\end{align}
		により定める$\sim$は同値関係であり,$\sim$による$\mathscr{L}^0(X,\mathscr{F},\mu)$の商集合を
		$L^0(X,\mathscr{F},\mu)$と表す.
	
	\item[商集合における算法]
		$L^0(\mu)$の元である関数類(同値類)を$[f]\ $($f$は関数類の代表)と表せば,$L^0(\mu)$は
		\begin{align}
			[f] + [g] \coloneqq [f+g],
			\quad \alpha [f] \coloneqq [\alpha f], \quad ([f],[g] \in L^0(\mu),\ \alpha \in \C).
		\end{align}
		を線型演算として$\C$上の線形空間となる.また
		\begin{align}
			[f][g] \coloneqq [fg] \quad \left([f],[g] \in L^{0}(\mu) \right).
		\end{align}
		を乗法として$L^0(\mu)$は環となる.$L^0(\mu)$の零元は零写像の関数類でありこれを[0]と書く.また
		単位元は恒等的に$1$を取る関数の関数類でありこれを[1]と書く.
		減法は
		\begin{align}
			[f] - [g] \coloneqq [f] + (-[g]) = [f] + [-g] = [f - g]
		\end{align}
		により定める.
	
	\item[関数類の順序]
		$[f],[g] \in L^0(\mu)$に対して次の関係$<(>)$を定める:
		\begin{align}
			[f] < [g]\ \left( [g] > [f] \right) \quad
			\overset{\mathrm{def}}{\Longleftrightarrow}
			\quad f < g \quad \mbox{$\mu$-a.s.} \label{dfn:equiv_class_order}
		\end{align}
		この定義はwell-definedである.実際任意の$f' \in [f],g' \in [g]$に対して
		\begin{align}
			\left\{ f' \geq g' \right\} \subset \left\{ f \neq f' \right\} \cup \left\{ f \geq g \right\} \cup \left\{ g \neq g' \right\}
		\end{align}
		の右辺は零集合であるから
		\begin{align}
			[f] < [g] \Leftrightarrow [f'] < [g']
		\end{align}
		が従う.$<(>)$または$=$であることを$\leq(\geq)$と書くとき,任意の$[f],[g],[h] \in L^0(\mu)$に対し,
		\begin{itemize}
			\item $[f] \leq [f]$が成り立つ.
			\item $[f] \leq [g]$かつ$[g] \leq [f]$ならば$[f] = [g]$が成り立つ.
			\item $[f] \leq [g],\ [g] \leq [h]$ならば$[f] \leq [h]$が成り立つ.
		\end{itemize}
		が満たされるから$\leq$は$L^0(\mu)$における順序となる.
\end{description}

\begin{screen}
	\begin{dfn}[商空間におけるノルムの定義]
		\begin{align}
			\Norm{[f]}{L^p} \coloneqq \Norm{f}{\mathscr{L}^p} 
			\quad (f \in \mathscr{L}^p(X,\mathscr{F},\mu),\ 1 \leq p \leq \infty)
		\end{align}
		により定める$\Norm{\cdot}{L^p}:L^0(X,\mathscr{F},\mu) \rightarrow \R$は関数類の代表に依らずに値が確定する.
		そして
		\begin{align}
			L^p(X,\mathscr{F},\mu) \coloneqq \Set{[f] \in L^0(X,\mathscr{F},\mu)}{\Norm{[f]}{L^p} < \infty} \quad (1 \leq p \leq \infty)
		\end{align}
		として定める空間は$\Norm{\cdot}{L^p}$をノルムとしてノルム空間となる.
	\end{dfn}
\end{screen}

\begin{screen}
	\begin{thm}[$L^p$はBanach空間]\label{thm:Lp_banach}
		ノルム空間$L^p(X,\mathscr{F},\mu)\ (1 \leq p \leq \infty)$の任意のCauchy列$\left( [f_n] \right)_{n=1}^\infty$
		に対してノルム収束極限$[f] \in L^p(\mu)$が存在する.
		また,このとき或る部分列$\left( \left[f_{n_k}\right] \right)_{k=1}^\infty$の代表は
		$f$に概収束する:
		\begin{align}
			\lim_{k \to \infty} f_{n_k} = f, \quad \mbox{$\mu$-a.e.}
		\end{align}
	\end{thm}
\end{screen}

\begin{prf}
	任意にCauchy列$[f_n] \in L^p(\mu)\ (n=1,2,3,\cdots)$を取れば,
	或る$N_1 \in \N$が存在して
	\begin{align}
		\Norm{[f_n]-[f_m]}{L^p} < \frac{1}{2}
		\quad (\forall n > m \geq N_1)
	\end{align}
	を満たす.ここで$m > N_1$を一つ選び$n_1$とおく.
	同様に$N_2 > N_1$を満たす$N_2 \in \N$が存在して
	\begin{align}
		\Norm{[f_n]-[f_m]}{L^p} < \frac{1}{2^2}
		\quad (\forall n > m \geq N_2)
	\end{align}
	を満たすから,$m > N_2$を一つ選び$n_2$とおけば
	\begin{align}
		\Norm{\left[f_{n_1}\right] - \left[f_{n_2}\right]}{L^p} < \frac{1}{2}
	\end{align}
	が成り立つ.同様の操作を繰り返して
	\begin{align}
		\Norm{\left[f_{n_k}\right] - \left[f_{n_{k+1}}\right]}{L^p} < \frac{1}{2^k} 
		\quad (n_k < n_{k+1},\ k=1,2,3,\cdots) \label{ineq:Lp_banach_2}
	\end{align}
	を満たす部分添数列$(n_k)_{k=1}^{\infty}$を構成する.
	\begin{description}
		\item[$p = \infty$の場合]
			$\left[f_{n_k}\right]$の代表$f_{n_k}\ (k=1,2,\cdots)$に対して
			\begin{align}
				A_k &\coloneqq \Set{x \in X}{\left| f_{n_k}(x) \right| > \Norm{f_{n_k}}{\mathscr{L}^\infty}}, \\
				A^k &\coloneqq \Set{x \in X}{\left| f_{n_k}(x) - f_{n_{k+1}}(x) \right| > \Norm{f_{n_k} - f_{n_{k+1}}}{\mathscr{L}^\infty}}
			\end{align}
			とおけば,補題\ref{lem:holder_inequality}より$\mu(A_k) = \mu(A^k) = 0\ (k=1,2,\cdots)$が成り立つ.
			\begin{align}
				A_\circ \coloneqq \bigcup_{k=1}^{\infty} A_k,
				\quad A^\circ \coloneqq \bigcup_{k=1}^{\infty}A^k,
				\quad A \coloneqq A_\circ \cup A^\circ
			\end{align}
			として$\mu$-零集合$A$を定めて
			\begin{align}
				\hat{f}_{n_k} \coloneqq f_{n_k} \defunc_{X \backslash A}
				\quad (\forall k=1,2,\cdots)
			\end{align}
			とおけば
			各$\hat{f}_{n_k}$は$\left[\hat{f}_{n_k}\right] = \left[f_{n_k}\right]$を満たす有界可測関数であり,
			(\refeq{ineq:Lp_banach_2})より
			\begin{align}
				\sup{x \in X}{\left|\hat{f}_{n_k}(x) - \hat{f}_{n_{k+1}}(x)\right|}
				\leq \Norm{\hat{f}_{n_k} - \hat{f}_{n_{k+1}}}{\mathscr{L}^\infty} < \frac{1}{2^k} \quad (k=1,2,3,\cdots) 
				\label{ineq:Lp_banach_1}
			\end{align}
			が成り立つ.
			このとき任意の$\epsilon > 0$に対し$1/2^N < \epsilon$を満たす$N \in \N$を取れば,$\ell > k > N$なら
			\begin{align}
				\left|\hat{f}_{n_k}(x) - \hat{f}_{n_{\ell}}(x)\right| 
				\leq \sum_{j=k}^{\ell-1}\left|\hat{f}_{n_j}(x) - \hat{f}_{n_{j+1}}(x)\right| 
				< \sum_{k > N} \frac{1}{2^k} = \frac{1}{2^N} < \epsilon
				\quad (\forall x \in X)
			\end{align}
			となるから,各点$x \in X$で$\left( \hat{f}_{n_k}(x) \right)_{k=1}^{\infty}$は$\C$のCauchy列となり収束する.
			\begin{align}
				\hat{f}(x) \coloneqq \lim_{k \to \infty} \hat{f}_{n_k}(x)
				\quad (\forall x \in X)
			\end{align}
			として$\hat{f}$を定めれば,$\hat{f}$は可測$\mathscr{F}/\borel{\C}$であり,且つ任意に$k \in \N$を取れば
			\begin{align}
				\sup{x \in X}{|\hat{f}_{n_k}(x) - \hat{f}(x)|} \leq \frac{1}{2^{k-1}} \label{ineq:Lp_banach_3}
			\end{align}
			を満たす.実際或る$y \in X$で$\alpha \coloneqq |\hat{f}_{n_k}(y) - \hat{f}(y)| > 1/2^{k-1}$が成り立つと仮定すれば,
			\begin{align}
				\left| \hat{f}_{n_k}(y) - \hat{f}_{n_\ell}(y) \right|
				\leq \sum_{j=k}^{\ell-1} \sup{x \in X}{\left|\hat{f}_{n_j}(x) - \hat{f}_{n_{j+1}}(x)\right|}
				< \sum_{j=k}^{\infty} \frac{1}{2^j}
				= \frac{1}{2^{k-1}}
				\quad (\forall \ell > k)
			\end{align}
			より
			\begin{align}
				0 < \alpha - \frac{1}{2^{k-1}} < \left| \hat{f}_{n_k}(y) - \hat{f}(y) \right| - \left| \hat{f}_{n_k}(y) - \hat{f}_{n_\ell}(y) \right|
				\leq \left| \hat{f}(y) - \hat{f}_{n_\ell}(y) \right|
				\quad (\forall \ell > k)
			\end{align}
			が従い各点収束に反する.不等式(\refeq{ineq:Lp_banach_3})により
			\begin{align}
				\sup{x \in X}{\left| \hat{f}(x) \right|} 
				< \sup{x \in X}{\left| \hat{f}(x) - \hat{f}_{n_k}(x) \right|} + \sup{x \in X}{\left| \hat{f}_{n_k}(x) \right|} 
				\leq \frac{1}{2^{k-1}} + \Norm{\hat{f}_{n_k}}{\mathscr{L}^\infty}
			\end{align}
			が成り立つから$\left[\hat{f}\right] \in L^\infty(\mu)$が従い,
			\begin{align}
				\Norm{\left[f_{n_k}\right] - \left[\hat{f}\right]}{L^\infty}
				= \Norm{\left[\hat{f}_{n_k}\right] - \left[\hat{f}\right]}{L^\infty}
				\leq \sup{x \in X}{|\hat{f}_{n_k}(x) - \hat{f}(x)|}
				\longrightarrow 0 \quad (k \longrightarrow \infty)
			\end{align}
			により部分列$\left( \left[f_{n_k}\right] \right)_{k=1}^{\infty}$が$\left[\hat{f}\right]$に収束するから
			元のCauchy列も$\left[\hat{f}\right]$に収束する.
			
		\item[$1 \leq p < \infty$の場合]
			$\left[f_{n_k}\right]$の代表$f_{n_k}\ (k=1,2,\cdots)$は
			\begin{align}	
				f_{n_k}(x) = f_{n_1}(x) + \sum_{j=1}^{k}\left( f_{n_j}(x) - f_{n_{j-1}}(x) \right) \quad (\forall x \in X)
				\label{eq:Lp_banach_3}
			\end{align}
			を満たし,これに対して
			\begin{align}
				g_k(x) &\coloneqq \left| f_{n_1}(x) \right| + \sum_{j=1}^{k} \left| f_{n_j}(x) - f_{n_{j-1}}(x) \right|
				\quad (\forall x \in X,\ k=1,2,\cdots)
			\end{align}
			により単調非減少な可測関数列$(g_k)_{k=1}^{\infty}$を定めれば,Minkowskiの不等式と(\refeq{ineq:Lp_banach_2})により
			\begin{align}
				\Norm{g_k}{\mathscr{L}^p} \leq \Norm{f_{n_1}}{\mathscr{L}^p} + \sum_{j=1}^{k}\Norm{f_{n_j} - f_{n_{j-1}}}{\mathscr{L}^p}
				< \Norm{f_{n_1}}{\mathscr{L}^p} + 1 < \infty
				\quad (k = 1,2,\cdots)
				\label{eq:thm_Lp_banach_1}
			\end{align}
			が成り立つ.ここで
			\begin{align}
				B_N \coloneqq \bigcap_{k=1}^{\infty} \Set{x \in X}{g_k(x) \leq N},
				\quad B \coloneqq \bigcup_{N=1}^{\infty} B_N
			\end{align}
			とおけば$(g_k)_{k=1}^{\infty}$は$B$上で各点収束し$X \backslash B$上では発散するが,
			$X \backslash B$は零集合である.実際
			\begin{align}
				\int_X g_k^p\ d\mu
				= \int_B g_k^p\ d\mu + \int_{X \backslash B} g_k^p\ d\mu
				\leq \left( \Norm{f_{n_1}}{\mathscr{L}^p} + 1 \right)^p,
				\quad (k=1,2,\cdots)
			\end{align}
			が満たされているから,単調収束定理より
			\begin{align}
				\int_B \lim_{k \to \infty} g_k^p\ d\mu + \int_{X \backslash B} \lim_{k \to \infty} g_k^p\ d\mu
				\leq \left( \Norm{f_{n_1}}{\mathscr{L}^p} + 1 \right)^p
			\end{align}
			が成り立ち$\mu(X \backslash B) = 0$が従う.$\mathscr{F}/\borel{\C}$-可測関数$g,f$を
			\begin{align}
				g \coloneqq \lim_{k \to \infty} g_k \defunc_B,
				\quad f \coloneqq \lim_{k \to \infty} f_{n_k} \defunc_B
			\end{align}
			で定義すれば,$|f| \leq g$と
			$g^p$の可積分性により$\left[f\right] \in L^p(\mu)$が成り立つ.
			また$\left|f_{n_k} - f\right|^p \leq 2^p g^p\ (\forall k=1,2,\cdots)$が満たされているから,
			Lebesgueの収束定理により
			\begin{align}
				\lim_{k \to \infty}\Norm{\left[f_{n_k}\right] - \left[f\right]}{L^p}^p
				= \lim_{k \to \infty} \int_X \left| f_{n_k} - f \right|^p\ d\mu = 0
			\end{align}
			が従い,部分列の収束により元のCauchy列も$\left[f\right]$に収束する.
			\QED
	\end{description}
\end{prf}

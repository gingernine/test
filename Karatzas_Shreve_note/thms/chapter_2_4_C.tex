\section{Convergence of Finite-Dimensional Distributions}
	\begin{itembox}[l]{記号の修正:標本路の表記(テキスト本文2行目)}
		Suppose that $X$ is a continuous process on some $(\Omega,\mathscr{F},P)$.
		For each $\omega$, the function $t \longmapsto X_t(\omega)$ is a member of
		\textcolor{red}{$C[0,\infty)^d$}, which we denote by \textcolor{red}{$X_\bullet(\omega)$}.
	\end{itembox}
	
	\begin{align}
		0 \leq t_1 < t_2 < \cdots < t_n < \infty
	\end{align}
	なる$t_1,t_2,\cdots,t_n$に対し,
	\begin{align}
		\pi_{t_1,\cdots,t_n}(\omega) = (\omega(t_1),\omega(t_2),\cdots,\omega(t_n))
	\end{align}
	で$C[0,\infty)^d$から$(\R^d)^n$への写像$\pi_{t_1,\cdots,t_n}$を定める.
	このとき
	\begin{align}
		C = \Set{\omega \in C[0,\infty)}{(\omega(t_1),\cdots,\omega(t_n)) \in A}
	\end{align}
	なる形のシリンダー集合は
	\begin{align}
		\pi_{t_1,\cdots,t_n}^{-1}(A)
	\end{align}
	に等しい.ここで
	\begin{align}
		\mathscr{C}' \coloneqq \Set{\pi_t^{-1}(A)}{t \in [0,\infty),\ A \in \borel{\R^d}}
	\end{align}
	と定める.$\mathscr{C}'$とは一次元シリンダー集合の全体である.
	
	\begin{itembox}[l]{$\borel{C[0,\infty)^d}$を生成する一次元シリンダー集合(テキスト本文3行目)}
		\textcolor{red}{$\borel{C[0,\infty)^d}$} is generated by the 
		one-dimensional cylinder sets
	\end{itembox}
	
	\begin{sketch}
		$\mathscr{C}' \subset \mathscr{C}$は満たされているので
		\begin{align}
			\mathscr{C} \subset \sigma\left(\mathscr{C}' \right)
			\label{eq:chapter_2_one_dimensional_cylinder_set_1}
		\end{align}
		が成り立つことを示せばよい.$C$を$\mathscr{C}$の任意の要素とすれば,
		\begin{align}
		0 \leq t_1 < t_2 < \cdots < t_n < \infty
		\end{align}
		なる$t_1,t_2,\cdots,t_n$と$\borel{(\R^d)^n}$の要素$A$を適当に取ることにより
		\begin{align}
			C = \pi_{t_1,\cdots,t_n}^{-1}(A)
		\end{align}
		となる.このとき
		\begin{align}
			C \in \sigma\left(\mathscr{C}' \right)
			\label{eq:chapter_2_one_dimensional_cylinder_set_2}
		\end{align}
		を言うために,
		$\pi_{t_1,\cdots,t_n}$が$\sigma\left(\mathscr{C}' \right)/\borel{(\R^d)^n}$-可測であることを示す.
		\begin{align}
			A_1 \times A_2 \times \cdots \times A_n,\quad (A_i \in \borel{\R^d},\ i=1,2,\cdots,n)
		\end{align}
		に対しては
		\begin{align}
			\pi_{t_1,\cdots,t_n}^{-1}(A_1 \times A_2 \times \cdots \times A_n)
			= \bigcap_{i=1}^n \pi_{t_i}^{-1}(A_i)
		\end{align}
		となるので
		\begin{align}
			\pi_{t_1,\cdots,t_n}^{-1}(A_1 \times A_2 \times \cdots \times A_n)
			\in \sigma\left(\mathscr{C}' \right)
		\end{align}
		が成立する.従って
		\begin{align}
			\Set{A_1 \times A_2 \times \cdots \times A_n}{A_i \in \borel{\R^d},\ i=1,2,\cdots,n}
			\subset \Set{B \in \borel{(\R^d)^n}}{\pi_{t_1,\cdots,t_n}^{-1}(B)
			\in \sigma\left(\mathscr{C}' \right)}
		\end{align}
		が成り立ち,右辺は$\sigma$-加法族であり左辺は$\borel{(\R^d)^n}$を生成するので
		\begin{align}
			\borel{(\R^d)^n} = 
			\Set{B \in \borel{(\R^d)^n}}{\pi_{t_1,\cdots,t_n}^{-1}(B)
			\in \sigma\left(\mathscr{C}' \right)}
		\end{align}
		が成立する.すなわち(\refeq{eq:chapter_2_one_dimensional_cylinder_set_2})
		が成り立ち,$C$の任意性から(\refeq{eq:chapter_2_one_dimensional_cylinder_set_1})が従う.
		\QED
	\end{sketch}
	
	\begin{itembox}[l]{標本路の可測性(テキスト本文4行目)}
		the random function \textcolor{red}{$X_\bullet:\Omega \longrightarrow C[0,\infty)$}
		is \textcolor{red}{$\mathscr{F}/\borel{C[0,\infty)^d}$}-measurable.
	\end{itembox}
	
	\begin{prf}
		任意に$C \in \mathscr{C}$を取れば$C = \Set{w \in C[0,\infty)^d}{(w(t_1),\cdots,w(t_n)) \in B}, \ (B \in \borel{(\R^d)^n})$
		と表されるから
		\begin{align}
			\Set{\omega \in \Omega}{X_{\bullet}(\omega) \in C}
			= \Set{\omega \in \Omega}{\left(X_{t_1}(\omega), \cdots, X_{t_n}(\omega)\right) \in B}
		\end{align}
		が成り立つ.右辺は$\mathcal{F}$に属するから
		\begin{align}
			\mathscr{C} \subset \Set{C \in \sgmalg{\mathscr{C}}}{(X_{\bullet})^{-1}(C) \in \mathcal{F}}
		\end{align}
		が従い,右辺は$\sigma$加法族であるから$X_{\bullet}$の$\mathcal{F}/\sgmalg{\mathscr{C}}$-可測性,
		つまり$\mathcal{F}/\borel{C[0,\infty)^d}$-可測性が出る.
		\QED
	\end{prf}

	\begin{itembox}[l]{}
	\end{itembox}
	
	\begin{itembox}[l]{Theorem 4.15 修正}
		
	\end{itembox}
	
	\begin{sketch}\mbox{}
		\begin{description}
			\item[第一段]
				$\{P_n\}_{n=1}^\infty$は緊密であり,
				全ての$n$で$P_n$は$\borel{C[0,\infty)}$上の確率測度であり,
				$(C[0,\infty),\rho)$は完備可分距離空間であるから,
				Prohorovの定理より$(P_n)_{n=1}^\infty$は弱収束する部分列
				$\left(P_{n_i}\right)_{i=1}^\infty$を持つ.その弱極限を$P$と書く.
				
			\item[第二段]
				$(P_n)_{n=1}^\infty$の任意の部分列が$P$に弱収束する部分列を含むなら,
				$(P_n)_{n=1}^\infty$は$P$に弱収束する.これは列の収束の一般論である.
				実際,$(P_n)_{n=1}^\infty$が$P$に弱収束しないとすれば,
				$P$の或る(弱位相に関する)近傍が存在して,その近傍に入らない$P_n$が無限個取れる.
				そうして取った部分列のいかなる部分列も$P$に収束し得ない.
				
			\item[第三段]
				$(P_n)_{n=1}^\infty$の任意の部分列が$P$に弱収束する部分列を含むことを示す.
				$\left(P_{n(k,1)}\right)_{k=1}^\infty$を$(P_n)_{n=1}^\infty$の部分列とする.
				このときProhorovの定理より弱収束する部分列$\left(P_{n(k,2)}\right)_{k=1}^\infty$および
				その極限$Q$が取れる.あとは
				\begin{align}
					P = Q
				\end{align}
				が成立すれば良いが,これが成り立つには
				\begin{align}
					\forall C \in \mathscr{C}\, (\, P(C) = Q(C)\, )
				\end{align}
				が成り立てば十分である.実際,$\mathscr{C}$は乗法族であるからDynkin族定理より
				\begin{align}
					\delta(\mathscr{C}) = \sigma(\mathscr{C}) = \borel{C[0,\infty)}
				\end{align}
				が成立し($\delta(\mathscr{C})$は$\mathscr{C}$を含む最小のDynkin族),他方で
				\begin{align}
					\mathscr{D} \coloneqq \Set{C \in \borel{C[0,\infty)}}{P(C) = Q(C)}
				\end{align}
				はDynkin族であるから
				\begin{align}
					\mathscr{C} \subset \mathscr{D} \Longrightarrow \delta(\mathscr{C}) \subset \mathscr{D}
				\end{align}
				が成立し,併せれば
				\begin{align}
					\forall C \in \mathscr{C}\, (\, P(C) = Q(C)\, )
					\Longrightarrow \borel{C[0,\infty)} = \mathscr{D} 
				\end{align}
				となる.いま$C$を$\mathscr{C}$の任意の要素とすれば,
				\begin{align}
					0 \leq t_1 < t_2 < \cdots < t_d < \infty
				\end{align}
				なる$t_1,t_2,\cdots,t_d$と$\borel{R^d}$の要素$A$によって
				\begin{align}
					C = \Set{\omega \in C[0,\infty)}{(\omega(t_1),\omega(t_2),\cdots,\omega(t_d)) \in A}
				\end{align}
				となる.書き換えれば
				\begin{align}
					C = \pi_{t_1,\cdots,t_d}^{-1}(A)
				\end{align}
				が成り立つ.すなわち
				\begin{align}
					P\pi_{t_1,\cdots,t_d}^{-1} = Q\pi_{t_1,\cdots,t_d}^{-1}
					\Longrightarrow P(C) = Q(C)
				\end{align}
				が成り立ち,$C$の任意性より
				\begin{align}
					P\pi_{t_1,\cdots,t_d}^{-1} = Q\pi_{t_1,\cdots,t_d}^{-1}
					\Longrightarrow \forall C \in \mathscr{C}\, (\, P(C) = Q(C)\, )
				\end{align}
				が従う.なので最終的に
				\begin{align}
					P\pi_{t_1,\cdots,t_d}^{-1} = Q\pi_{t_1,\cdots,t_d}^{-1}
					\label{eq:chapter_2_Theorem_4_15_1}
				\end{align}
				を示せば$P=Q$を得るのに十分であることが分かった.
				
			\item[第四段]
				(\refeq{eq:chapter_2_Theorem_4_15_1})を示す.$f$を
				\begin{align}
					f:R^d \longrightarrow \R
				\end{align}
				なる有界連続写像とすれば,$f \circ \pi_{t_1,\cdots,t_d}$は
				\begin{align}
					f \circ \pi_{t_1,\cdots,t_d}:C[0,\infty) \longrightarrow \R
				\end{align}
				なる有界連続写像となる.従って
				\begin{align}
					\int_{C[0,\infty)} f \circ \pi_{t_1,\cdots,t_d}\ dP_{n_i}
						&\longrightarrow \int_{C[0,\infty)} f \circ \pi_{t_1,\cdots,t_d}\ dP
						\quad (i \longrightarrow \infty), \\
					\int_{C[0,\infty)} f \circ \pi_{t_1,\cdots,t_d}\ dP_{n(k,2)}
						&\longrightarrow \int_{C[0,\infty)} f \circ \pi_{t_1,\cdots,t_d}\ dQ
						\quad (k \longrightarrow \infty)
				\end{align}
				が満たされるが,$f$は任意に選ばれていたので
				$\left(P_{n_i}\pi_{t_1,\cdots,t_d}^{-1}\right)_{i=1}^\infty$
				と$\left(P_{n(k,2)}\pi_{t_1,\cdots,t_d}^{-1}\right)_{k=1}^\infty$
				はそれぞれ$P\pi_{t_1,\cdots,t_d}^{-1}$と$Q\pi_{t_1,\cdots,t_d}^{-1}$に弱収束する.
				ところで定理の仮定(確率ベクトルの分布収束)から
				\begin{align}
					\int_{R^d} f\ dP^{(n)}\pi_{t_1,\cdots,t_d} (X^{(n)})^{-1}
					\longrightarrow \int_{R^d} f\ dP^*
					\quad (n \longrightarrow \infty)
				\end{align}
				を満たす$\borel{R^d}$上の確率測度$P^*$が存在する.すなわち
				\begin{align}
					\int_{R^d} f \circ \pi_{t_1,\cdots,t_d}\ dP_n
					&= \int_{R^d} f \circ \pi_{t_1,\cdots,t_d}\ dP^{(n)} {X^{(n)}}^{-1} \\
					&\longrightarrow \int_{R^d} f\ dP^* \quad (n \longrightarrow \infty)
				\end{align}
				が成立し,収束列の部分列は同じ極限に収束するから
				\begin{align}
					\int_{C[0,\infty)} f \circ \pi_{t_1,\cdots,t_d}\ dP_{n_i}
						&\longrightarrow \int_{R^d} f\ dP^*
						\quad (i \longrightarrow \infty), \\
					\int_{C[0,\infty)} f \circ \pi_{t_1,\cdots,t_d}\ dP_{n(k,2)}
						&\longrightarrow \int_{R^d} f\ dP^*
						\quad (k \longrightarrow \infty)
				\end{align}
				が成立する.$f$は任意に選ばれていたので,
				$\left(P_{n_i}\pi_{t_1,\cdots,t_d}^{-1}\right)_{i=1}^\infty$
				と$\left(P_{n(k,2)}\pi_{t_1,\cdots,t_d}^{-1}\right)_{k=1}^\infty$
				が$P^*$に弱収束することが示された.弱極限の一意性より
				\begin{align}
					P\pi_{t_1,\cdots,t_d}^{-1} = P^* = Q\pi_{t_1,\cdots,t_d}^{-1}
				\end{align}
				が成立する.以上で(\refeq{eq:chapter_2_Theorem_4_15_1})が示された.
				そして$(P_n)_{n=1}^\infty$が$P$に弱収束することも示された.
			
			\item[第五段]
				
		\end{description}
	\end{sketch}
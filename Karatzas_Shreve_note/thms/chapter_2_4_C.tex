\section{Convergence of Finite-Dimensional Distributions}
	\begin{itembox}[l]{標本路の表記の修正(テキスト本文2行目)}
		Suppose that $X$ is a continuous process on some $(\Omega,\mathscr{F},P)$.
		For each $\omega$, the function $t \longmapsto X_t(\omega)$ is a member of
		\textcolor{red}{$C[0,\infty)^d$}, which we denote by \textcolor{red}{$X_\bullet(\omega)$}.
	\end{itembox}
	
	\begin{align}
		0 \leq t_1 < t_2 < \cdots < t_n < \infty
	\end{align}
	なる$t_1,t_2,\cdots,t_n$に対し,
	\begin{align}
		\pi_{t_1,\cdots,t_n}(w) = (w(t_1),w(t_2),\cdots,w(t_n)),
		\quad (w \in C[0,\infty)^d)
	\end{align}
	で$C[0,\infty)^d$から$(\R^d)^n$への写像$\pi_{t_1,\cdots,t_n}$を定める.
	このとき
	\begin{align}
		C = \Set{w \in C[0,\infty)^d}{(w(t_1),\cdots,w(t_n)) \in A}
	\end{align}
	なる形のシリンダー集合は
	\begin{align}
		\pi_{t_1,\cdots,t_n}^{-1}(A)
	\end{align}
	に等しい.ここで座標過程$W$を
	\begin{align}
		W_t(w) = w(t),\quad (t \in [0,\infty),\ w \in C[0,\infty)^d)
	\end{align}
	で定め
	\begin{align}
		\mathscr{C}' \coloneqq \Set{W_t^{-1}(A)}{t \in [0,\infty),\ A \in \borel{\R^d}}
	\end{align}
	と定める.
	\begin{align}
		W_t^{-1}(A) &= \Set{w \in C[0,\infty)}{W_t(w) \in A} \\
		&= \Set{w \in C[0,\infty)}{w(t) \in A}
	\end{align}
	であるから$\mathscr{C}'$は一次元シリンダー集合の全体である.
	
	\begin{itembox}[l]{$\mathscr{C}'$は$\borel{C[0,\infty)^d}$を生成する(テキスト本文3行目)}
		\textcolor{red}{$\borel{C[0,\infty)^d}$} is generated by the 
		one-dimensional cylinder sets
	\end{itembox}
	
	\begin{sketch}
		$\mathscr{C}' \subset \mathscr{C}$は満たされているので
		\begin{align}
			\mathscr{C} \subset \sigma\left(\mathscr{C}' \right)
			\label{eq:chapter_2_one_dimensional_cylinder_set_1}
		\end{align}
		が成り立つことを示せばよい.$C$を$\mathscr{C}$の任意の要素とすれば,
		\begin{align}
		0 \leq t_1 < t_2 < \cdots < t_n < \infty
		\end{align}
		なる$t_1,t_2,\cdots,t_n$と$\borel{(\R^d)^n}$の要素$A$を適当に取ることにより
		\begin{align}
			C = \pi_{t_1,\cdots,t_n}^{-1}(A)
		\end{align}
		となる.このとき
		\begin{align}
			C \in \sigma\left(\mathscr{C}' \right)
			\label{eq:chapter_2_one_dimensional_cylinder_set_2}
		\end{align}
		を言うために,
		$\pi_{t_1,\cdots,t_n}$が$\sigma\left(\mathscr{C}' \right)/\borel{(\R^d)^n}$-可測であることを示す.
		\begin{align}
			A_1 \times A_2 \times \cdots \times A_n,\quad (A_i \in \borel{\R^d},\ i=1,2,\cdots,n)
		\end{align}
		に対しては
		\begin{align}
			\pi_{t_1,\cdots,t_n}^{-1}(A_1 \times A_2 \times \cdots \times A_n)
			= \bigcap_{i=1}^n W_{t_i}^{-1}(A_i)
		\end{align}
		となるので
		\begin{align}
			\pi_{t_1,\cdots,t_n}^{-1}(A_1 \times A_2 \times \cdots \times A_n)
			\in \sigma\left(\mathscr{C}' \right)
		\end{align}
		が成立する.従って
		\begin{align}
			\Set{A_1 \times A_2 \times \cdots \times A_n}{A_i \in \borel{\R^d},\ i=1,2,\cdots,n}
			\subset \Set{B \in \borel{(\R^d)^n}}{\pi_{t_1,\cdots,t_n}^{-1}(B)
			\in \sigma\left(\mathscr{C}' \right)}
		\end{align}
		が成り立ち,右辺は$\sigma$-加法族であり左辺は$\borel{(\R^d)^n}$を生成するので
		\begin{align}
			\borel{(\R^d)^n} = 
			\Set{B \in \borel{(\R^d)^n}}{\pi_{t_1,\cdots,t_n}^{-1}(B)
			\in \sigma\left(\mathscr{C}' \right)}
		\end{align}
		が成立する.すなわち(\refeq{eq:chapter_2_one_dimensional_cylinder_set_2})
		が成り立ち,$C$の任意性から(\refeq{eq:chapter_2_one_dimensional_cylinder_set_1})が従う.
		\QED
	\end{sketch}
	
	\begin{itembox}[l]{標本路の可測性(テキスト本文4行目)}
		the random function \textcolor{red}{$X_\bullet:\Omega \longrightarrow C[0,\infty)^d$}
		is \textcolor{red}{$\mathscr{F}/\borel{C[0,\infty)^d}$}-measurable.
	\end{itembox}
	
	\begin{prf}
		$C$を$\mathscr{C}'$の任意の要素とする.このとき
		$0 \leq t < \infty$なる或る$t$と$\borel{\R^d}$の或る要素$A$によって
		\begin{align}
			C = W_t^{-1}(A)
		\end{align}
		となる.ここで
		\begin{align}
			C = \Set{w \in C[0,\infty)}{w(t) \in A}
		\end{align}
		より
		\begin{align}
			\forall \omega \in \Omega\,
			\left(\, X_\bullet(\omega) \in C \Longleftrightarrow X_t(\omega) \in A\, \right)
		\end{align}
		が成り立つので
		\begin{align}
			\Set{\omega \in \Omega}{X_\bullet(\omega) \in C}
			= \Set{\omega \in \Omega}{X_t(\omega) \in A}
		\end{align}
		が成り立ち,$X_t$の$\mathscr{F}/\borel{\R^d}$-可測性より
		\begin{align}
			\Set{\omega \in \Omega}{X_t(\omega) \in A} \in \mathscr{F}
		\end{align}
		が成り立つから
		\begin{align}
			X_{\bullet}^{-1}(C) \in \mathscr{F}
		\end{align}
		が従う.$C$は任意に選ばれていたので
		\begin{align}
			\mathscr{C}' \subset \Set{C \in \sgmalg{\mathscr{C}}}{X_{\bullet}^{-1}(C) \in \mathscr{F}}
		\end{align}
		が成立し,右辺は$\sigma$-加法族であるから$X_\bullet$の$\mathscr{F}/\sgmalg{\mathscr{C}'}$-可測性が従う.
		\begin{align}
			\sgmalg{\mathscr{C}'} = \sgmalg{\mathscr{C}} = \borel{C[0,\infty)}
		\end{align}
		より$X_\bullet$の$\mathscr{F}/\borel{C[0,\infty)^d}$-可測性が示された.
		\QED
	\end{prf}
	
	\begin{itembox}[l]{Theorem 4.15 修正}
		Let $\left\{X^{(n)}\right\}_{n=1}^\infty$ be a tight sequence of continuous processes,
		\textcolor{red}{with each $X^{(n)}$ defined on $\left(\Omega^{(n)},\mathscr{F}^{(n)},P^{(n)}\right)$},
		with the property that, whenever $0 \leq t_1 < \cdots < t_d < \infty$, then
		\textcolor{red}{the sequence
		\begin{align}
			\left( P^{(n)}\pi_{t_1,\cdots,t_d}(X_\bullet^{(n)})^{-1} \right)_{n=1}^\infty
		\end{align}
		weakly converges to some probability measure on $\borel{\R^d}$}.
		Let \textcolor{red}{
		\begin{align}
			P_n \coloneqq P^{(n)} {X_\bullet^{(n)}}^{-1}
		\end{align}
		}. Then $\{P_n\}_{n=1}^\infty$ converges weakly to a measure $P$
		\textcolor{red}{on $\borel{C[0,\infty)}$}, under which the coordinate mapping process
		$W_t(\omega) \coloneqq \omega(t)$ on $C[0,\infty)$ satisfies
		\textcolor{red}{
		\begin{align}
			P^{(n)}\pi_{t_1,\cdots,t_d}(X_\bullet^{(n)})^{-1}
			\overset{weak}{\longrightarrow} P\pi_{t_1,\cdots,t_d}(W_\bullet)^{-1},
			0 \leq t_1, \cdots < t_d < \infty,\ d \geq 1.
			\label{eq:chapter_2_Theorem_4_15_6}
		\end{align}
		}
	\end{itembox}
	
	\begin{sketch}\mbox{}
		\begin{description}
			\item[第一段]
				$\{P_n\}_{n=1}^\infty$は緊密であり,
				全ての$n$で$P_n$は$\borel{C[0,\infty)}$上の確率測度であり,
				$(C[0,\infty),\rho)$は完備可分距離空間であるから,
				Prohorovの定理より$(P_n)_{n=1}^\infty$は弱収束する部分列
				$\left(P_{n_i}\right)_{i=1}^\infty$を持つ.その弱極限を$P$と書く.
				
			\item[第二段]
				$(P_n)_{n=1}^\infty$の任意の部分列が$P$に弱収束する部分列を含むなら,
				$(P_n)_{n=1}^\infty$は$P$に弱収束する.これは列の収束の一般論である.
				実際,$(P_n)_{n=1}^\infty$が$P$に弱収束しないとすれば,
				$P$の或る(弱位相に関する)近傍が存在して,その近傍に入らない$P_n$が無限個取れる.
				そうして取った部分列のいかなる部分列も$P$に収束し得ない.
				
			\item[第三段]
				$(P_n)_{n=1}^\infty$の任意の部分列が$P$に弱収束する部分列を含むことを示す.
				$\left(P_{n(k,1)}\right)_{k=1}^\infty$を$(P_n)_{n=1}^\infty$の部分列とする.
				このときProhorovの定理より弱収束する部分列$\left(P_{n(k,2)}\right)_{k=1}^\infty$および
				その極限$Q$が取れる.あとは
				\begin{align}
					P = Q
				\end{align}
				が成立すれば良いが,これが成り立つには
				\begin{align}
					\forall C \in \mathscr{C}\, (\, P(C) = Q(C)\, )
					\label{eq:chapter_2_Theorem_4_15_2}
				\end{align}
				が成り立てば十分である.実際,$\mathscr{C}$は乗法族であるからDynkin族定理より
				\begin{align}
					\delta(\mathscr{C}) = \sigma(\mathscr{C}) = \borel{C[0,\infty)}
				\end{align}
				が成立し($\delta(\mathscr{C})$は$\mathscr{C}$を含む最小のDynkin族),他方で
				\begin{align}
					\mathscr{D} \coloneqq \Set{C \in \borel{C[0,\infty)}}{P(C) = Q(C)}
				\end{align}
				はDynkin族であるから
				\begin{align}
					\mathscr{C} \subset \mathscr{D} \Longrightarrow \delta(\mathscr{C}) \subset \mathscr{D}
				\end{align}
				が成立し,併せれば
				\begin{align}
					\forall C \in \mathscr{C}\, (\, P(C) = Q(C)\, )
					\Longrightarrow \borel{C[0,\infty)} = \mathscr{D} 
				\end{align}
				となる.いま$C$を$\mathscr{C}$の任意の要素とすれば,
				\begin{align}
					0 \leq t_1 < t_2 < \cdots < t_d < \infty
				\end{align}
				なる$t_1,t_2,\cdots,t_d$と$\borel{R^d}$の要素$A$によって
				\begin{align}
					C = \Set{\omega \in C[0,\infty)}{(\omega(t_1),\omega(t_2),\cdots,\omega(t_d)) \in A}
				\end{align}
				となる.書き換えれば
				\begin{align}
					C = \pi_{t_1,\cdots,t_d}^{-1}(A)
				\end{align}
				となるので
				\begin{align}
					P\pi_{t_1,\cdots,t_d}^{-1} = Q\pi_{t_1,\cdots,t_d}^{-1}
					\Longrightarrow P(C) = Q(C)
				\end{align}
				が成り立つ.すなわち,時点$t_1,\cdots,t_d$のあらゆる取り方に対して
				\begin{align}
					P\pi_{t_1,\cdots,t_d}^{-1} = Q\pi_{t_1,\cdots,t_d}^{-1}
					\label{eq:chapter_2_Theorem_4_15_1}
				\end{align}
				を示せば(\refeq{eq:chapter_2_Theorem_4_15_2})が従う.
				
			\item[第四段]
				(\refeq{eq:chapter_2_Theorem_4_15_1})を示す.$f$を
				\begin{align}
					f:R^d \longrightarrow \R
				\end{align}
				なる有界連続写像とすれば,$f \circ \pi_{t_1,\cdots,t_d}$は
				\begin{align}
					f \circ \pi_{t_1,\cdots,t_d}:C[0,\infty) \longrightarrow \R
				\end{align}
				なる有界連続写像となる.従って
				\begin{align}
					\int_{C[0,\infty)} f \circ \pi_{t_1,\cdots,t_d}\ dP_{n_i}
						&\longrightarrow \int_{C[0,\infty)} f \circ \pi_{t_1,\cdots,t_d}\ dP
						\quad (i \longrightarrow \infty), 
						\label{eq:chapter_2_Theorem_4_15_3} \\
					\int_{C[0,\infty)} f \circ \pi_{t_1,\cdots,t_d}\ dP_{n(k,2)}
						&\longrightarrow \int_{C[0,\infty)} f \circ \pi_{t_1,\cdots,t_d}\ dQ
						\quad (k \longrightarrow \infty)
				\end{align}
				が満たされるが,$f$は任意に選ばれていたので
				$\left(P_{n_i}\pi_{t_1,\cdots,t_d}^{-1}\right)_{i=1}^\infty$
				と$\left(P_{n(k,2)}\pi_{t_1,\cdots,t_d}^{-1}\right)_{k=1}^\infty$
				はそれぞれ$P\pi_{t_1,\cdots,t_d}^{-1}$と$Q\pi_{t_1,\cdots,t_d}^{-1}$に弱収束することになる.
				ところで定理の仮定から
				\begin{align}
					\int_{R^d} f\ dP^{(n)}\pi_{t_1,\cdots,t_d} (X_\bullet^{(n)})^{-1}
					\longrightarrow \int_{R^d} f\ dP^*
					\quad (n \longrightarrow \infty)
					\label{eq:chapter_2_Theorem_4_15_4}
				\end{align}
				を満たす$\borel{R^d}$上の確率測度$P^*$が存在する.すなわち
				\begin{align}
					\int_{C[0,\infty)} f \circ \pi_{t_1,\cdots,t_d}\ dP_n
					&= \int_{C[0,\infty)} f \circ \pi_{t_1,\cdots,t_d}\ dP^{(n)} {X_\bullet^{(n)}}^{-1} \\
					&\longrightarrow \int_{R^d} f\ dP^* \quad (n \longrightarrow \infty)
				\end{align}
				が成立し,収束列の部分列は同じ極限に収束するから
				\begin{align}
					\int_{C[0,\infty)} f \circ \pi_{t_1,\cdots,t_d}\ dP_{n_i}
						&\longrightarrow \int_{R^d} f\ dP^*
						\quad (i \longrightarrow \infty),
						\label{eq:chapter_2_Theorem_4_15_5} \\
					\int_{C[0,\infty)} f \circ \pi_{t_1,\cdots,t_d}\ dP_{n(k,2)}
						&\longrightarrow \int_{R^d} f\ dP^*
						\quad (k \longrightarrow \infty)
				\end{align}
				が成立する.$f$は任意に選ばれていたので,
				$\left(P_{n_i}\pi_{t_1,\cdots,t_d}^{-1}\right)_{i=1}^\infty$
				と$\left(P_{n(k,2)}\pi_{t_1,\cdots,t_d}^{-1}\right)_{k=1}^\infty$
				が$P^*$に弱収束することが示された.弱極限の一意性より
				\begin{align}
					P\pi_{t_1,\cdots,t_d}^{-1} = P^* = Q\pi_{t_1,\cdots,t_d}^{-1}
				\end{align}
				が成立する.以上で(\refeq{eq:chapter_2_Theorem_4_15_1})が示された.
				そして$(P_n)_{n=1}^\infty$が$P$に弱収束することも示された.
				他方で,(\refeq{eq:chapter_2_Theorem_4_15_3})と(\refeq{eq:chapter_2_Theorem_4_15_5})から
				\begin{align}
					\int_{C[0,\infty)} f \circ \pi_{t_1,\cdots,t_d}\ dP
					= \int_{R^d} f\ dP^*
				\end{align}
				が成立し,これと(\refeq{eq:chapter_2_Theorem_4_15_4})を併せれば
				\begin{align}
					\int_{R^d} f\ dP^{(n)}\pi_{t_1,\cdots,t_d} (X_\bullet^{(n)})^{-1}
					\longrightarrow \int_{C[0,\infty)} f \circ \pi_{t_1,\cdots,t_d}\ dP
					\quad (n \longrightarrow \infty)
				\end{align}
				が成立する.ここで$W_\bullet$は$C[0,\infty)$上の恒等写像であるから
				\begin{align}
					f \circ \pi_{t_1,\cdots,t_d} = f \circ \pi_{t_1,\cdots,t_d} \circ W_\bullet
				\end{align}
				が成立し,
				\begin{align}
					\int_{R^d} f\ dP^{(n)}\pi_{t_1,\cdots,t_d} (X_\bullet^{(n)})^{-1}
					\longrightarrow \int_{C[0,\infty)} f\ dP\pi_{t_1,\cdots,t_d} (W_\bullet)^{-1}
					\quad (n \longrightarrow \infty)
				\end{align}
				が従う.$f$は任意に選ばれていたので(\refeq{eq:chapter_2_Theorem_4_15_6})が示された.
				\QED
		\end{description}
	\end{sketch}
\section{二次変分}
	\begin{screen}
		\begin{dfn}[二乗可積分マルチンゲール]
			$(\Omega,\mathscr{F},P)$を確率空間とし,$\mathbf{T} \defeq [0,\infty[$か$[0,T]$とし,
			$\{\mathscr{F}_t\}_{t \in \mathbf{T}}$を$\mathscr{F}$に付随するフィルトレーションとする.
			$X$を$(\Omega,\mathscr{F},P)$上の$\{\mathscr{F}_t\}_{t \in \mathbf{T}}$-マルチンゲールで,
			\begin{align}
				\forall t \in \mathbf{T}\, \left(\, E\left(|X_t|^2\right) < \infty\, \right)
			\end{align}
			かつ
			\begin{align}
				\forall \omega \in \Omega\, \left(\, X(0,\omega) = 0\, \right)
			\end{align}
			を満たすものとするとき,$X$を二乗可積分な$\{\mathscr{F}_t\}_{t \in \mathbf{T}}$-マルチンゲールと呼ぶ.また
			$(\Omega,\mathscr{F},P)$上の二乗可積分な右連続$\{\mathscr{F}_t\}_{t \in \mathbf{T}}$-マルチンゲールの全体を
			\begin{align}
				\mathscr{M}^{\mathbf{T}}
			\end{align}
			として,同様に$(\Omega,\mathscr{F},P)$上の二乗可積分な連続$\{\mathscr{F}_t\}_{t \in \mathbf{T}}$-マルチンゲールの全体を
			\begin{align}
				\mathscr{M}^{\mathbf{T}}_c
			\end{align}
			とする.
		\end{dfn}
	\end{screen}
	
	\begin{screen}
		\begin{thm}[Doobの劣マルチンゲール不等式]
		\end{thm}
	\end{screen}
	
	\begin{screen}
		\begin{thm}[$\mathscr{M}_2$の線型構造]
			$(\Omega,\mathscr{F},P)$を確率空間とし,$\mathbf{T} \defeq [0,\infty[$か$[0,T]$とし,
			$\{\mathscr{F}_t\}_{t \in \mathbf{T}}$を$\mathscr{F}$に付随するフィルトレーションとし,
			$\{\mathscr{F}_t\}_{t \in \mathbf{T}}$は完備であるとする.
		\end{thm}
	\end{screen}
	
	\begin{screen}
		\begin{thm}[$\mathscr{M}_2$の擬距離構造]\label{thm:pseudo_metric_on_square_integrable_martingales}
			$(\Omega,\mathscr{F},P)$を確率空間とし,$\mathbf{T} \defeq [0,\infty[$か$[0,T]$とし,
			$\{\mathscr{F}_t\}_{t \in \mathbf{T}}$を$\mathscr{F}$に付随するフィルトレーションとし,
			$\{\mathscr{F}_t\}_{t \in \mathbf{T}}$は完備であるとする.
			\begin{align}
				\mathbf{T} = [0,T]
			\end{align}
			であるとき
			\begin{align}
				\mathscr{M}^{\mathbf{T}} \times \mathscr{M}^{\mathbf{T}} \ni (X,Y) \longmapsto
				\Norm{X_T - Y_T}{2}
			\end{align}
			なる関係を$d$とし,
			\begin{align}
				\mathbf{T} = [0,\infty[
			\end{align}
			であるとき
			\begin{align}
				\mathscr{M}^{\mathbf{T}} \times \mathscr{M}^{\mathbf{T}} \ni (X,Y) \longmapsto
				\sum_{n \in \Natural} \frac{1}{2^n} \cdot \min \left\{ \Norm{X_n - Y_n}{2},1 \right\}
			\end{align}
			なる関係を$d$とすると,いずれの場合も$\left(\mathscr{M}^{\mathbf{T}},d\right)$は完備な擬距離空間となる.
			また$\mathscr{M}^{\mathbf{T}}_c$はその閉集合となる.
		\end{thm}
	\end{screen}
	
	\begin{sketch}\mbox{}
		\begin{description}
			\item[第一段] $T = [0,T]$であるとする.いま
				\begin{align}
					\Natural \ni n \longmapsto X^{(n)} \in \mathscr{M}^{\mathbf{T}}
				\end{align}
				なる関係を$\left(\mathscr{M}^{\mathbf{T}},d\right)$のCauchy列とする.このとき
				\begin{align}
					\forall k \in \Natural\, \left[\, d\left( X^{(n_k)},X^{(n_{k+1})} \right) < \frac{1}{2^{k+1}}\, \right]
				\end{align}
				を満たす部分列
				\begin{align}
					\Natural \ni k \longmapsto X^{(n_k)}
				\end{align}
				が取れる.このときDoobの劣マルチンゲール不等式から,任意の自然数$k$で
				\begin{align}
					\int_\Omega \left\{ \sup{t \in [0,T]}{\left|X_t^{(n_k)} - X_t^{(n_{k+1})}\right|} \right\}^2\ dP
					\leq 4 \int_\Omega \left|X_T^{(n_k)} - X_T^{(n_{k+1})}\right|^2\ dP
					< \frac{1}{4^k}
					\label{fom:thm_pseudo_metric_on_square_integrable_martingales_2}
				\end{align}
				が成立する.自然数$k$に対して
				\begin{align}
					E_k \defeq \left\{ \frac{1}{2^k} \leq \sup{t \in [0,T]}{\left|X_t^{(n_k)} - X_t^{(n_{k+1})}\right|} \right\}
				\end{align}
				とおけば
				\begin{align}
					P(E_k) < \frac{1}{2^k}
				\end{align}
				が成立するので,Borel-Cantelliの補題より
				\begin{align}
					E \defeq \bigcap_{n \in \Natural} \bigcup_{\substack{k \in \Natural \\ n < k}} E_k
				\end{align}
				で定める$E$は$P$-零集合である.$\omega$を$\Omega \backslash E$の要素とすれば
				\begin{align}
					\forall k \in \Natural\,
					\left[\, N < k \Longrightarrow \sup{t \in [0,T]}{\left|X_t^{(n_k)}(\omega) - X_t^{(n_{k+1})}(\omega)\right|} < \frac{1}{2^k}\, \right]
					\label{fom:thm_pseudo_metric_on_square_integrable_martingales_1}
				\end{align}
				を満たす自然数$N$が取れる.ゆえに,いま$t$を$\mathbf{T}$の要素とすれば
				\begin{align}
					\Natural \ni k \longmapsto X_t^{(n_k)}(\omega)
				\end{align}
				は$\R$のCauchy列であり,$\R$で収束する.ここで
				\begin{align}
					\mathbf{T} \times \Omega \ni (t,\omega) \longmapsto
					\begin{cases}
						\lim_{k \to \infty} X_t^{(n_k)}(\omega) & \mbox{if } \omega \in \Omega \backslash E \\
						0 & \mbox{if } \omega \in E
					\end{cases} 
				\end{align}
				で定める関係を$X$とする.
				
				$X$のパスが右連続(または連続)であることを示す.$\omega$を$\Omega \backslash E$の要素とすれば
				(\refeq{fom:thm_pseudo_metric_on_square_integrable_martingales_1})より
				\begin{align}
					\forall k \in \Natural\,
					\left[\, N < k \Longrightarrow \sup{t \in [0,T]}{\left|X_t^{(n_k)}(\omega) - X_t(\omega)\right|} \leq \frac{1}{2^k}\, \right]
				\end{align}
				を満たす自然数$N$が取れるので,パスは一様収束している.ゆえに
				\begin{align}
					\left\{X^{(n)}\right\} \subset \mathscr{M}^{\mathbf{T}}
				\end{align}
				ならば$X$は右連続であり,
				\begin{align}
					\left\{X^{(n)}\right\} \subset \mathscr{M}_c^{\mathbf{T}}
				\end{align}
				ならば$X$は連続である.
				
				$X$が$\{\mathscr{F}_t\}_{t \in \mathbf{T}}$-適合であることを示す.
				$t$を$\mathbf{T}$の任意の要素とすれば
				\begin{align}
					\forall \omega \in \Omega\, \left(\, 
					\lim_{k \to \infty} X_t^{(n_k)}(\omega) \cdot \defunc_{\Omega \backslash E}(\omega) = X_t(\omega)\, \right)
				\end{align}
				が成り立ち,またフィルトレーションの完備性の仮定から
				\begin{align}
					E \in \mathscr{F}_t
				\end{align}
				なので,各自然数$k$で$X_t^{(n_k)} \defunc_{\Omega \backslash E}$は$\mathscr{F}_t/\borel{\R}$-可測である.
				よって定理\ref{lem:measurability_metric_space}より$X_t$は$\mathscr{F}_t/\borel{\R}$-可測である.
				
				$X$が二乗可積分な$\{\mathscr{F}_t\}_{t \in \mathbf{T}}$-マルチンゲールであることを示す.
				$t$を$\mathbf{T}$の任意の要素とすれば,Fatouの補題と
				(\refeq{fom:thm_pseudo_metric_on_square_integrable_martingales_1})より
				任意の自然数$k$で
				\begin{align}
					\int_\Omega \left|X_t-X_t^{(n_k)}\right|^2\ dP
					\leq \sup{n \in \Natural}{\inf{\substack{j \in \Natural \\ n < j}}{
					\int_\Omega \left|X_t^{(n_j)}-X_t^{(n_k)}\right|^2\ dP}}
					\leq \frac{1}{4^k}
					\label{fom:thm_pseudo_metric_on_square_integrable_martingales_3}
 				\end{align}
 				が成立する.ゆえにMinkowskiの不等式から
 				\begin{align}
 					\Norm{X_t}{2} \leq \Norm{X_t - X_t^{(n_k)}}{2} + \Norm{X_t^{(n_k)}}{2} < \infty
 				\end{align}
 				が成立する.またH\Ddot{o}lderの不等式から
 				\begin{align}
 					\int_\Omega \left|X_t-X_t^{(n_k)}\right|\ dP
 					\leq \Norm{X_t - X_t^{(n_k)}}{2} \longrightarrow 0\quad (k \longrightarrow \infty)
 				\end{align}
 				が成り立つ.ゆえに,いま$s$と$t$を
 				\begin{align}
 					s < t
 				\end{align}
 				なる$\mathbf{T}$の要素とすれば,$\mathscr{F}_s$の任意の要素$A$で
 				\begin{align}
 					\int_A X_t\ dP = \lim_{k \to \infty} \int_A X^{(n_k)}_t\ dP
 					= \lim_{k \to \infty} \int_A X^{(n_k)}_s\ dP
 					= \int_A X_s\ dP
 				\end{align}
 				が成り立つ.
 				
 				以上より
 				\begin{align}
 					X \in \mathscr{M}^{\mathbf{T}}
 				\end{align}
 				である.最後に,(\refeq{fom:thm_pseudo_metric_on_square_integrable_martingales_3})より
 				\begin{align}
 					d\left(X,X^{(n_k)}\right) = \int_\Omega \left|X_T-X_T^{(n_k)}\right|^2\ dP
 					\longrightarrow 0\quad (k \longrightarrow \infty)
 				\end{align}
 				が成り立つので$X$は$d$に関して$k \longmapsto X^{(n_k)}$の極限である.ゆえに
 				\begin{align}
 					d\left(X,X^{(n)}\right) \longrightarrow 0\quad (n \longrightarrow \infty)
 				\end{align}
 				が従う.
 				
			\item[第二段] $T = [0,\infty[$であるとする.
				
		\end{description}
	\end{sketch}
	
	\begin{screen}
		\begin{dfn}[局所マルチンゲール]
		\end{dfn}
	\end{screen}
	
	\begin{screen}
		\begin{thm}
			\begin{align}
				\mathscr{M}^{\mathbf{T}} \subset \mathscr{M}^{\mathbf{T}}_{loc}.
			\end{align}
		\end{thm}
	\end{screen}
	
	\begin{screen}
		\begin{thm}[マルチンゲールの二乗過程は増大過程とマルチンゲールに分解できる]
		\end{thm}
	\end{screen}
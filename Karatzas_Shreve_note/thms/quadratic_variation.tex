\section{二次変分}
	$X$と$Y$を$\mathscr{M}_{c,loc}$の要素とすれば,
	\begin{align}
		X + Y \in \mathscr{M}_{c,loc}
	\end{align}
	及び
	\begin{align}
		X - Y \in \mathscr{M}_{c,loc}
	\end{align}
	であるから,
	\begin{align}
		(X + Y)^{2} - A \in \mathscr{M}_{c,loc}
	\end{align}
	を満たす連続な$\{\mathscr{F}_t\}_{t \in [0,1]}$-増大過程$A$と
	\begin{align}
		(X - Y)^{2} - B \in \mathscr{M}_{c,loc}
	\end{align}
	を満たす連続な$\{\mathscr{F}_t\}_{t \in [0,1]}$-増大過程$B$が取れる.
	このとき
	\begin{align}
		C \defeq \frac{1}{4} \cdot (A - B)
	\end{align}
	とおけば
	\begin{align}
		X \cdot Y - C = \frac{1}{4} \cdot 
		\left\{ \left[(X + Y)^{2} - A\right] - \left[(X - Y)^{2} - B\right] \right\}
	\end{align}
	が成り立つので
	\begin{align}
		X \cdot Y - C \in \mathscr{M}_{c,loc}
	\end{align}
	が成り立つ.ところで$C$は次を満たす:
	\begin{itemize}
		\item $\Omega$のすべての要素$\omega$で$C_{0}(\omega) = 0$.
		\item $C$は$[0,1] \times \Omega$上の$\{\mathscr{F}_{t}\}_{t \in [0,1]}$-適合過程.
		\item $\Omega$のすべての要素$\omega$で$C_{\bullet}(\omega)$は
			$\mathscr{O}_{[0,1]}/\mathscr{O}_{\R}$-連続かつ有界変動.
	\end{itemize}
	
	\begin{screen}
		\begin{dfn}[有界変動過程]
			$[0,1] \times \Omega$上の$\{\mathscr{F}_{t}\}_{t \in [0,1]}$-適合過程$B$で
			\begin{itemize}
				\item $\Omega$のすべての要素$\omega$で$B_{0}(\omega) = 0$,
				\item $\Omega$のすべての要素$\omega$で$B_{\bullet}(\omega)$は有界変動,
			\end{itemize}
			を満たすものを$\{\mathscr{F}_{t}\}_{t \in [0,1]}$-{\bf 有界変動過程}
			\index{ゆうかいへんどうかてい@有界変動過程}と呼ぶことにする.
		\end{dfn}
	\end{screen}
	
	定理\ref{thm:direct_product_of_non_empty_sets_is_not_empty}より,
	$\mathscr{M}_{c,loc} \times \mathscr{M}_{c,loc}$上の写像$v$で,
	$\mathscr{M}_{c,loc}$の任意の要素$X$と$Y$に対して
	\begin{itemize}
		\item $X = Y$ならば$v(X,Y)$は連続な$\{\mathscr{F}_{t}\}_{t \in [0,1]}$-増大過程で
			\begin{align}
				X \cdot Y - v(X,Y) \in \mathscr{M}_{c,loc},
			\end{align}
			
		\item $X \neq Y$ならば$v(X,Y)$は連続な$\{\mathscr{F}_{t}\}_{t \in [0,1]}$-有界変動過程で
			\begin{align}
				X \cdot Y - v(X,Y) \in \mathscr{M}_{c,loc},
			\end{align}
	\end{itemize}
	を満たすものが取れる.
	
	\begin{screen}
		\begin{dfn}[二次変分]
			$QV$を次を満たす$\mathscr{M}_{c,loc} \times \mathscr{M}_{c,loc}$上の写像とする.
			\begin{itemize}
				\item $\mathscr{M}_{c,loc}$の任意の要素$X$と$Y$に対して
					\begin{align}
						\inprod<X,Y> \defeq QV(X,Y)
					\end{align}
					と書き,これを$X$と$Y$の{\bf 共変分}\index{きょうへんぶん@共変分}
					{\bf (cross variation)}と呼ぶ.特に
					\begin{align}
						\inprod<X> \defeq QV(X,X)
					\end{align}
					と書いて,これを$X$の{\bf 二次変分}\index{にじへんぶん@二次変分}
					{\bf (quadratic variation)}と呼ぶ.
					
				\item $X = Y$ならば$\inprod<X,Y>$は連続な$\{\mathscr{F}_{t}\}_{t \in [0,1]}$-増大過程であり,
				$X \neq Y$ならば$\inprod<X,Y>$は連続な$\{\mathscr{F}_{t}\}_{t \in [0,1]}$-有界変動過程であり,いずれの場合も
					\begin{align}
						X \cdot Y - \inprod<X,Y> \in \mathscr{M}_{c,loc}.
					\end{align}
			\end{itemize}
		\end{dfn}
	\end{screen}
	
	\begin{screen}
		\begin{thm}[共変分の対称性]
			$X$と$Y$を$\mathscr{M}_{c,loc}$の要素とするとき,$P$-零集合$F$が取れて,
			$[0,1]$の任意の要素$t$及び$\Omega \backslash F$の任意の要素$\omega$に対して
			\begin{align}
				\inprod<X,Y>_{t}(\omega) = \inprod<Y,X>_{t}(\omega).
			\end{align}
		\end{thm}
	\end{screen}
	
	\begin{sketch}
	\end{sketch}
	
	\begin{screen}
		\begin{thm}[和の共変分]
			$X$と$Y$と$Z$を$\mathscr{M}_{c,loc}$の要素とするとき,$P$-零集合$F$が取れて,
			$[0,1]$の任意の要素$t$及び$\Omega \backslash F$の任意の要素$\omega$に対して
			\begin{align}
				\inprod<X+Y,Z>_{t}(\omega) 
				= \inprod<X,Z>_{t}(\omega) + \inprod<Y,Z>_{t}(\omega).
			\end{align}
		\end{thm}
	\end{screen}
	
	\begin{sketch}
	\end{sketch}
	
	\begin{screen}
		\begin{thm}[スカラ倍の共変分]
			$X$と$Y$を$\mathscr{M}_{c,loc}$の要素とし,$\alpha$を実数とするとき,
			$P$-零集合$F$が取れて,
			$[0,1]$の任意の要素$t$及び$\Omega \backslash F$の任意の要素$\omega$に対して
			\begin{align}
				\inprod<\alpha \cdot X,Y>_{t}(\omega) = \alpha \cdot \inprod<X,Y>_{t}(\omega).
			\end{align}
		\end{thm}
	\end{screen}
	
	\begin{sketch}
	\end{sketch}
	
	\begin{screen}
		\begin{thm}[二乗可積分マルチンゲールの二次変分は可積分]
		\label{thm:square_integrable_iff_quadratic_variation_integrable}
			$X$を$\mathscr{M}_{c,loc}$の要素とするとき
			\begin{align}
				X \in \mathscr{M}_{c}^{2} \Longleftrightarrow E\inprod<X>_{1} < \infty.
			\end{align}
		\end{thm}
	\end{screen}
	
	\begin{sketch}
		定理\ref{thm:increasing_stopping_times_which_locally_bound_martingale}より,
		$\{\mathscr{F}_{t}\}_{t \in [0,1]}$-増大停止時刻列$\tau$で,
		任意の自然数$n$に対して
		\begin{align}
			\left(X^{\tau_{n}}\right)^{2} - \inprod<X>^{\tau_{n}}
		\end{align}
		を有界な$\{\mathscr{F}_{t}\}_{t \in [0,1]}$-マルチンゲールとするものが取れる.
		このとき任意の自然数$n$及び$[0,1]$の任意の要素$t$に対して
		\begin{align}
			E\left(X^{\tau_{n}}_{t}\right)^{2} = E\inprod<X>^{\tau_{n}}_{t}
			\label{fom:square_integrable_iff_quadratic_variation_integrable}
		\end{align}
		が成り立つので,
		\begin{align}
			X \in \mathscr{M}_{c}^{2}
		\end{align}
		ならば単調収束定理より
		\begin{align}
			E{X_{1}}^{2} = E\inprod<X>_{1}
		\end{align}
		が従い
		\begin{align}
			E\inprod<X>_{1} < \infty
		\end{align}
		を得る.他方で(\refeq{fom:square_integrable_iff_quadratic_variation_integrable})
		より任意の自然数$n$で
		\begin{align}
			E\left(X^{\tau_{n}}_{t}\right)^{2} \leq E\inprod<X>_{1}
		\end{align}
		が成り立つので,
		\begin{align}
			E\inprod<X>_{1} < \infty \Longrightarrow X \in \mathscr{M}_{c}^{2}
		\end{align}
		が得られる.
		\QED
	\end{sketch}
	
	\begin{screen}
		\begin{thm}[二次関数が非負であるための条件]
		\label{thm:nonnegative_condition_of_quadratic_function}
			$a,b,c$を実数とし,$a$は非負であるとする.このとき$\R$の任意の要素$x$で
			\begin{align}
				0 \leq a \cdot x^{2} + b \cdot x + c
				\label{fom:nonnegative_condition_of_quadratic_function}
			\end{align}
			が成り立つならば
			\begin{align}
				b^{2} \leq 4 \cdot a \cdot c.
			\end{align}
		\end{thm}
	\end{screen}
	
	\begin{sketch}
		いま(\refeq{fom:nonnegative_condition_of_quadratic_function})が満たされているとする.
		\begin{align}
			a=0
		\end{align}
		のとき,
		\begin{align}
			b \neq 0
		\end{align}
		ならば
		\begin{align}
			b \cdot -\frac{c+1}{b} + c = -1
		\end{align}
		が成り立つので
		\begin{align}
			\exists x \in \R\, (\, b \cdot x + c < 0\, )
		\end{align}
		が従う.ゆえに$a=0$のときは
		\begin{align}
			b = 0
		\end{align}
		及び
		\begin{align}
			b^{2} \leq 4 \cdot a \cdot c
		\end{align}
		が満たされる.
		\begin{align}
			0 < a
		\end{align}
		であるとすると,
		\begin{align}
			a \cdot x^{2} + b \cdot x + c
			= a \cdot \left(x+\frac{b}{2 \cdot a}\right)^{2} - \frac{b^{2} - 4 \cdot a \cdot c}{4 \cdot a}
		\end{align}
		から
		\begin{align}
			\frac{b^{2} - 4 \cdot a \cdot c}{4 \cdot a}
			\leq a \cdot \left(x+\frac{b}{2 \cdot a}\right)^{2}
		\end{align}
		が従う.特に
		\begin{align}
			x = -\frac{b}{2 \cdot a}
		\end{align}
		のとき
		\begin{align}
			\frac{b^{2} - 4 \cdot a \cdot c}{4 \cdot a} \leq 0
		\end{align}
		が成り立つ.すなわちこのとき
		\begin{align}
			b^{2} \leq 4 \cdot a \cdot c
		\end{align}
		が成り立つ.
		\QED
	\end{sketch}
	
	\begin{screen}
		\begin{thm}[二次変分に対するSchwartzの不等式]
		\label{thm:Schwartz_inequality_for_quadratic_variations}
			$X$と$Y$を$\mathscr{M}_{c,loc}$の要素とするとき,
			$P$-零集合$F$が取れて,$\Omega \backslash F$の任意の要素$\omega$と
			$s \leq t$なる$[0,1]$の任意の要素$s$と$t$に対して
			\begin{align}
				\left[\inprod<X,Y>_{t}(\omega) - \inprod<X,Y>_{s}(\omega)\right]^{2}
				\leq \left[\inprod<X>_{t}(\omega) - \inprod<X>_{s}(\omega)\right] 
				\cdot \left[\inprod<Y>_{t}(\omega) - \inprod<Y>_{s}(\omega)\right].
			\end{align}
		\end{thm}
	\end{screen}
	
	\begin{sketch}
		$\R$の任意の要素$x$及び$[0,1]$の任意の要素$t$に対して
		\begin{align}
			\Set{\omega \in \Omega}{\inprod<X+x \cdot Y>_{t}(\omega) \neq
			\inprod<X>_{t}(\omega) + 2 \cdot x \cdot \inprod<X,Y>_{t}(\omega)
			+ x^{2} \cdot \inprod<Y>_{t}(\omega)}
		\end{align}
		は$P$-零集合であるから,
		\begin{align}
			F \defeq \bigcup_{x \in \Q} \bigcup_{t \in [0,1]}
			\Set{\omega \in \Omega}{\inprod<X+x \cdot Y>_{t}(\omega) \neq
			\inprod<X>_{t}(\omega) + 2 \cdot x \cdot \inprod<X,Y>_{t}(\omega)
			+ x^{2} \cdot \inprod<Y>_{t}(\omega)}
		\end{align}
		とおけば$F$は$P$-零集合である.$\omega$を$\Omega \backslash F$の任意の要素とし,
		$s$と$t$を
		\begin{align}
			s \leq t
		\end{align}
		なる$[0,1]$の任意の要素とすれば,任意の有理数$x$に対して
		\begin{align}
			\inprod<X>_{s}(\omega) + 2 \cdot x \cdot \inprod<X,Y>_{s}(\omega)
			+ x^{2} \cdot \inprod<Y>_{s}(\omega)
			&= \inprod<X+x \cdot Y>_{s}(\omega) \\
			&\leq \inprod<X+x \cdot Y>_{t}(\omega) \\
			&= \inprod<X>_{t}(\omega) + 2 \cdot x \cdot \inprod<X,Y>_{t}(\omega)
			+ x^{2} \cdot \inprod<Y>_{t}(\omega)
		\end{align}
		が成り立つ.ゆえに任意の実数$x$に対して
		\begin{align}
			0 &\leq x^{2} \cdot \left[\inprod<Y>_{t}(\omega) - \inprod<Y>_{s}(\omega) \right] \\
			&\quad + 2 \cdot x \cdot \left[\inprod<X,Y>_{t}(\omega) - \inprod<X,Y>_{s}(\omega) \right] \\
			&\quad + \left[\inprod<X>_{t}(\omega) - \inprod<X>_{s}(\omega) \right]
		\end{align}
		が成り立つ.ゆえに定理\ref{thm:nonnegative_condition_of_quadratic_function}より
		\begin{align}
			\left[\inprod<X,Y>_{t}(\omega) - \inprod<X,Y>_{s}(\omega)\right]^{2}
			\leq \left[\inprod<X>_{t}(\omega) - \inprod<X>_{s}(\omega)\right] \cdot
			\left[\inprod<Y>_{t}(\omega) - \inprod<Y>_{s}(\omega)\right]
		\end{align}
		が成り立つ.
		\QED
	\end{sketch}
	
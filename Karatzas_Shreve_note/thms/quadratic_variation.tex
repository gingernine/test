\section{二乗可積分マルチンゲール}
	本節では$(\Omega,\mathscr{F},P)$を確率空間とし,$\mathbf{T} \defeq [0,\infty[$か$[0,T]$とし,
	$\{\mathscr{F}_t\}_{t \in \mathbf{T}}$を$\mathscr{F}$に付随するフィルトレーションとする.
	また$\{\mathscr{F}_t\}_{t \in \mathbf{T}}$が完備であるとは
	\begin{align}
		\Set{a}{a \in \mathscr{F} \wedge P(a) = 0} \subset \mathscr{F}_0
	\end{align}
	が成り立っていることである.
	
	\begin{screen}
		\begin{dfn}[二乗可積分マルチンゲール]
			$(\Omega,\mathscr{F},P)$上の右連続な$\{\mathscr{F}_t\}_{t \in \mathbf{T}}$-マルチンゲールで,
			$\Omega$の任意の要素$\omega$に対して$(0,\omega)$に$0$を対応させるものの全体を
			\begin{align}
				\mathscr{M}_{\mathbf{T}}
			\end{align}
			とおく.$\mathscr{M}_{\mathbf{T}}$の要素のうち連続であるものの全体を
			\begin{align}
				\mathscr{M}^c_{\mathbf{T}}
			\end{align}
			とおく.$X$を$(\Omega,\mathscr{F},P)$上の$\{\mathscr{F}_t\}_{t \in \mathbf{T}}$-マルチンゲールで
			\begin{align}
				\forall t \in \mathbf{T}\, \left(\, \int_\Omega |X_t|^2\ dP < \infty\, \right)
			\end{align}
			を満たすものとするとき,$X$を{\bf 二乗可積分}{\bf (square integrable)}な
			$\{\mathscr{F}_t\}_{t \in \mathbf{T}}$-マルチンゲールと呼ぶ.
			$\mathscr{M}_{\mathbf{T}}$の要素のうち二乗可積分であるものの全体を
			\begin{align}
				\mathscr{M}^2_{\mathbf{T}}
			\end{align}
			とおき,また
			\begin{align}
				\mathscr{M}^{2,c}_{\mathbf{T}} \defeq \mathscr{M}^c_{\mathbf{T}} \cap \mathscr{M}^2_{\mathbf{T}}
			\end{align}
			とおく.
		\end{dfn}
	\end{screen}
	
	\begin{screen}
		\begin{thm}[Doobの劣マルチンゲール不等式]
		\end{thm}
	\end{screen}
	
	\begin{screen}
		\begin{thm}[$\mathscr{M}^2_{\mathbf{T}}$の線型構造]
			$X$と$Y$を$\mathscr{M}^2_{\mathbf{T}}$の要素とするとき,$(X,Y)$に対して
			\begin{align}
				(t,\omega) \longmapsto X(t,\omega) + Y(t,\omega)
			\end{align}
			なる写像を対応させる関係$+_m$を$\mathscr{M}^2_{\mathbf{T}}$上の加法とし,
			また$\alpha$を実数とするときに$(\alpha,X)$に対して
			\begin{align}
				(t,\omega) \longmapsto \alpha \cdot X(t,\omega)
			\end{align}
			なる写像を対応させる関係$s$をスカラ倍とすれば,
			\begin{align}
				\left(\left(\mathscr{M}^2_{\mathbf{T}},+_m \right),(\R,+,\bullet),s\right)
			\end{align}
			は線型空間である.
		\end{thm}
	\end{screen}
	
	\begin{screen}
		\begin{thm}[$\mathscr{M}^2_{\mathbf{T}}$は完備な擬距離空間]
		\label{thm:pseudo_metric_on_square_integrable_martingales}
			$\mathbf{T} = [0,T]$とし,$\{\mathscr{F}_t\}_{t \in \mathbf{T}}$は完備であるとする.このとき
			\begin{align}
				\mathscr{M}^2_{\mathbf{T}} \times \mathscr{M}^2_{\mathbf{T}} \ni (X,Y) \longmapsto
				\left\{\int_\Omega |X_T-Y_T|^2\ dP\right\}^{\frac{1}{2}}
			\end{align}
			なる関係を$d$とすると,$\left(\mathscr{M}^2_{\mathbf{T}},d\right)$は完備な擬距離空間である.
			また$\mathscr{M}^{2,c}_{\mathbf{T}}$はその完備な部分集合である.
		\end{thm}
	\end{screen}
	
	\begin{sketch}\mbox{}
		\begin{description}
			\item[第一段] いま
				\begin{align}
					\Natural \ni n \longmapsto X^{(n)} \in \mathscr{M}^2_{\mathbf{T}}
				\end{align}
				なる関係を$\left(\mathscr{M}^2_{\mathbf{T}},d\right)$のCauchy列とする.すると
				\begin{align}
					\forall k \in \Natural\, \left[\, d\left( X^{(n_k)},X^{(n_{k+1})} \right) < \frac{1}{4^{k+1}}\, \right]
				\end{align}
				を満たす部分列
				\begin{align}
					\Natural \ni k \longmapsto X^{(n_k)}
				\end{align}
				が取れる.このときDoobの劣マルチンゲール不等式から,任意の自然数$k$で
				\begin{align}
					\int_\Omega \left\{ \sup{t \in [0,T]}{\left|X_t^{(n_k)} - X_t^{(n_{k+1})}\right|} \right\}^2\ dP
					\leq 4 \int_\Omega \left|X_T^{(n_k)} - X_T^{(n_{k+1})}\right|^2\ dP
					< \frac{1}{8^k}
					\label{fom:thm_pseudo_metric_on_square_integrable_martingales_2}
				\end{align}
				が成立する.従って,自然数$k$に対して
				\begin{align}
					E_k \defeq \left\{ \frac{1}{2^k} \leq \sup{t \in [0,T]}{\left|X_t^{(n_k)} - X_t^{(n_{k+1})}\right|} \right\}
				\end{align}
				とおけば
				\begin{align}
					P(E_k) < \frac{1}{2^k}
				\end{align}
				が成立するので,Borel-Cantelliの補題より
				\begin{align}
					E \defeq \bigcap_{n \in \Natural} \bigcup_{\substack{k \in \Natural \\ n < k}} E_k
				\end{align}
				で定める$E$は$P$-零集合である.$\omega$を$\Omega \backslash E$の要素とすれば
				\begin{align}
					\forall k \in \Natural\,
					\left[\, N < k \Longrightarrow \sup{t \in [0,T]}{\left|X_t^{(n_k)}(\omega) - X_t^{(n_{k+1})}(\omega)\right|} < \frac{1}{2^k}\, \right]
					\label{fom:thm_pseudo_metric_on_square_integrable_martingales_1}
				\end{align}
				を満たす自然数$N$が取れる.ゆえに,いま$t$を$\mathbf{T}$の要素とすれば
				\begin{align}
					\Natural \ni k \longmapsto X_t^{(n_k)}(\omega)
				\end{align}
				は$\R$のCauchy列であり,$\R$で収束する.ここで
				\begin{align}
					\mathbf{T} \times \Omega \ni (t,\omega) \longmapsto
					\begin{cases}
						\lim_{k \to \infty} X_t^{(n_k)}(\omega) & \mbox{if } \omega \in \Omega \backslash E \\
						0 & \mbox{if } \omega \in E
					\end{cases} 
				\end{align}
				で定める関係を$X$とする.
			
			\item[第二段]
				$X$のパスが右連続(または連続)であることを示す.$\omega$を$\Omega \backslash E$の要素とすれば
				(\refeq{fom:thm_pseudo_metric_on_square_integrable_martingales_1})より
				\begin{align}
					\forall k \in \Natural\,
					\left[\, N < k \Longrightarrow \sup{t \in [0,T]}{\left|X_t^{(n_k)}(\omega) - X_t(\omega)\right|} \leq \frac{1}{2^k}\, \right]
				\end{align}
				を満たす自然数$N$が取れるので,パスは一様収束している.ゆえに
				\begin{align}
					\left\{X^{(n)}\right\} \subset \mathscr{M}^2_{\mathbf{T}}
				\end{align}
				ならば$X$は右連続であり,
				\begin{align}
					\left\{X^{(n)}\right\} \subset \mathscr{M}^{2,c}_{\mathbf{T}}
				\end{align}
				ならば$X$は連続である.
			
			\item[第三段]
				$X$が$\{\mathscr{F}_t\}_{t \in \mathbf{T}}$-適合であることを示す.
				$t$を$\mathbf{T}$の任意の要素とすれば
				\begin{align}
					\forall \omega \in \Omega\, \left(\, 
					\lim_{k \to \infty} X_t^{(n_k)}(\omega) \cdot \defunc_{\Omega \backslash E}(\omega) = X_t(\omega)\, \right)
				\end{align}
				が成り立ち,またフィルトレーションの完備性の仮定から
				\begin{align}
					E \in \mathscr{F}_t
				\end{align}
				なので,各自然数$k$で$X_t^{(n_k)} \defunc_{\Omega \backslash E}$は$\mathscr{F}_t/\borel{\R}$-可測である.
				よって定理\ref{lem:measurability_metric_space}より$X_t$は$\mathscr{F}_t/\borel{\R}$-可測である.
				
			\item[第四段]
				$X$が二乗可積分な$\{\mathscr{F}_t\}_{t \in \mathbf{T}}$-マルチンゲールであることを示す.
				$t$を$\mathbf{T}$の任意の要素とすれば,Fatouの補題と
				(\refeq{fom:thm_pseudo_metric_on_square_integrable_martingales_1})より
				任意の自然数$k$で
				\begin{align}
					\int_\Omega \left|X_t-X_t^{(n_k)}\right|^2\ dP
					\leq \sup{n \in \Natural}{\inf{\substack{j \in \Natural \\ n < j}}{
					\int_\Omega \left|X_t^{(n_j)}-X_t^{(n_k)}\right|^2\ dP}}
					\leq \frac{1}{4^k}
					\label{fom:thm_pseudo_metric_on_square_integrable_martingales_3}
 				\end{align}
 				が成立する.ゆえにMinkowskiの不等式から
 				\begin{align}
 					\left\{\int_\Omega |X_t|^2\ dP\right\}^{\frac{1}{2}}
 					\leq \left\{\int_\Omega \left|X_t - X^{(n_k)}_t\right|^2\ dP\right\}^{\frac{1}{2}}
 					+ \left\{\int_\Omega \left|X^{(n_k)}_t\right|^2\ dP\right\}^{\frac{1}{2}}
 					< \infty
 				\end{align}
 				が成立する.またH\Ddot{o}lderの不等式から
 				\begin{align}
 					\int_\Omega \left|X_t-X_t^{(n_k)}\right|\ dP
 					\leq \left\{\int_\Omega \left|X_t - X^{(n_k)}_t\right|^2\ dP\right\}^{\frac{1}{2}}
 					\longrightarrow 0\quad (k \longrightarrow \infty)
 				\end{align}
 				が成り立つ.ゆえに,いま$s$と$t$を
 				\begin{align}
 					s < t
 				\end{align}
 				なる$\mathbf{T}$の要素とすれば,$\mathscr{F}_s$の任意の要素$A$で
 				\begin{align}
 					\int_A X_t\ dP = \lim_{k \to \infty} \int_A X^{(n_k)}_t\ dP
 					= \lim_{k \to \infty} \int_A X^{(n_k)}_s\ dP
 					= \int_A X_s\ dP
 				\end{align}
 				が成り立つ.
 			
 			\item[第五段]
 				以上より
 				\begin{align}
 					X \in \mathscr{M}^2_{\mathbf{T}}
 				\end{align}
 				である.最後に,(\refeq{fom:thm_pseudo_metric_on_square_integrable_martingales_3})より
 				\begin{align}
 					d\left(X,X^{(n_k)}\right) = \int_\Omega \left|X_T-X_T^{(n_k)}\right|^2\ dP
 					\longrightarrow 0\quad (k \longrightarrow \infty)
 				\end{align}
 				が成り立つので$X$は$d$に関して$k \longmapsto X^{(n_k)}$の極限である.ゆえに
 				\begin{align}
 					d\left(X,X^{(n)}\right) \longrightarrow 0\quad (n \longrightarrow \infty)
 				\end{align}
 				が従う.
 				\QED
		\end{description}
	\end{sketch}
	
	\begin{screen}
		\begin{dfn}[増大過程]
			$A$が$\mathbf{T} \times \Omega$上の$\R$値写像であって,かつ
			\begin{itemize}
				\item $\forall \omega \in \Omega\, (\, A_0(\omega) = 0\, )$
				\item $A$は$\{\mathscr{F}_t\}_{t \in \mathbf{T}}$-適合
				\item $A$のすべてのパスが右連続かつ単調非減少
			\end{itemize}
			を満たすとき,$A$を$(\Omega,\mathscr{F},P)$上の$\{\mathscr{F}_t\}_{t \in \mathbf{T}}$-{\bf 増大過程}
			\index{ぞうだいかてい@増大過程}{\bf (increasing process)}と呼ぶ.
			$(\Omega,\mathscr{F},P)$上の$\{\mathscr{F}_t\}_{t \in \mathbf{T}}$-増大過程の全体を
			\begin{align}
				\mathscr{A}^+_{\mathbf{T}}
			\end{align}
			で表し,区別不能性で類別した商集合を
			\begin{align}
				\mathfrak{A}^+_{\mathbf{T}}
			\end{align}
			で表す.
		\end{dfn}
	\end{screen}
	
	\begin{comment}
	\begin{screen}
		\begin{dfn}[ナチュラル]
			$(\Omega,\mathscr{F},P)$を確率空間とし,$\mathbf{T}$を$[0,\infty[$か$[0,T]$とし,
			$\{\mathscr{F}_t\}_{t \in \mathbf{T}}$を$\mathscr{F}$に付随するフィルトレーションとする.
			$A$が$(\Omega,\mathscr{F},P)$上の$\{\mathscr{F}_t\}_{t \in \mathbf{T}}$-増大過程であって,
			かつ$M$を任意に与えられた$(\Omega,\mathscr{F},P)$上の有界かつ$RCLL$な
			$\{\mathscr{F}_t\}_{t \in \mathbf{T}}$-マルチンゲールとするときに
			\begin{align}
				\forall t \in \mathbf{T}\, 
				\left[\, E(M_t A_t) = E \int_{(0,t]} M_{s-}\ dA_s\, \right]
			\end{align}
			を満たすならば,$A$は{\bf ナチュラル}\index{ナチュラル}{\bf (natural)}であるという.
		\end{dfn}
	\end{screen}
	\end{comment}
	
	\begin{screen}
		\begin{thm}[局所マルチンゲールの二乗過程は増大過程と局所マルチンゲールに分解できる]
		\label{thm:decomposition_of_local_martingales}
			$\mathbf{T} = [0,T]$とし,$\{\mathscr{F}_t\}_{t \in \mathbf{T}}$は完備であるとする.
			このとき,$X$を$\mathscr{M}^{loc}_{\mathbf{T}}$の要素とすれば,
			$\mathfrak{A}^+_{\mathbf{T}}$の要素$Q$が唯一つ取れて,
			$Q$の任意の要素$A$に対して
			\begin{align}
				X^2 - A \in \mathscr{M}^{loc}_{\mathbf{T}}
			\end{align}
			となる.特に
			\begin{align}
				X \in \mathscr{M}^{c,loc}_{\mathbf{T}}
			\end{align}
			であれば$Q$の中から連続なものが取れる.
		\end{thm}
	\end{screen}
	
	\begin{sketch}\mbox{}
		\begin{description}
			\item[step1-1] $X$が有界であるとする.つまりいま
				\begin{align}
					\forall t \in \mathbf{T}\, \forall \omega \in \Omega\,
					\left(\, |X(t,\omega)| \leq b\, \right)
				\end{align}
				を満たす実数$b$が取れる.
			
			\item[step1-2] $n$を自然数とし,
				\begin{align}
					(t,\omega) \longmapsto \sum_{j=0}^{2^n-1} \left(X_{\min\left\{t,\frac{j+1}{2^n}T\right\}}(\omega)
					- X_{\min\left\{t,\frac{j}{2^n}T\right\}}(\omega)\right)^2
				\end{align}
				なる写像を$A^{(n)}$とする.このとき
				\begin{align}
					X^2 - A^{(n)} \in \mathscr{M}^2_{\mathbf{T}}
					\label{fom:thm_decomposition_of_local_martingales_2}
				\end{align}
				であることを示す.まず$X$の右連続性より$A^{(n)}$も右連続である.また全ての$j$で
				\begin{align}
					X_{\min\left\{t,\frac{j}{2^n}T\right\}}
				\end{align}
				は$\mathscr{F}_t/\borel{\R}$-可測なので,$X^2-A^{(n)}$は$\{\mathscr{F}_t\}_{t \in \mathbf{T}}$-適合である.
				いまは$X$も$A^{(n)}$も有界であるから,$\mathbf{T}$の任意の要素$t$で
				\begin{align}
					\int_{\Omega} \left|X_t^2-A_t^{(n)}\right|^2\ dP < \infty
				\end{align}
				が成立する.$s$と$t$を
				\begin{align}
					s < t
				\end{align}
				なる$\mathbf{T}$の要素とする.そして
				\begin{align}
					\frac{k}{2^n}T \leq s < \frac{k+1}{2^n}T
				\end{align}
				を満たす$k$を取る.このとき
				\begin{align}
					A^{(n)}_t - A^{(n)}_s
					&= \sum_{j=k}^{2^n-1} \left\{
					\left(X_{\min\left\{t,\frac{j+1}{2^n}T\right\}} - X_{\min\left\{t,\frac{j}{2^n}T\right\}}\right)^2
					- \left(X_{\min\left\{s,\frac{j+1}{2^n}T\right\}} - X_{\min\left\{s,\frac{j}{2^n}T\right\}}\right)^2\right\} \\
					&= \sum_{j=k+1}^{2^n-1} \left(X_{\min\left\{t,\frac{j+1}{2^n}T\right\}} - X_{\min\left\{t,\frac{j}{2^n}T\right\}}\right)^2
					+ \left(X_{\min\left\{t,\frac{k+1}{2^n}T\right\}} - X_{\frac{k}{2^n}T}\right)^2
					- \left(X_s - X_{\frac{k}{2^n}T}\right)^2
				\end{align}
				が成り立つ.この各項について,
				\begin{align}
					j \in \{k+1,K+2,\cdots,2^n-1\}
				\end{align}
				なる各$j$で$P$-a.s.に
				\begin{align}
					\cexp{ \left(X_{\min\left\{t,\frac{j+1}{2^n}T\right\}} - X_{\min\left\{t,\frac{j}{2^n}T\right\}}\right)^2}{\mathscr{F}_s}
					= \cexp{X^2_{\min\left\{t,\frac{j+1}{2^n}T\right\}}}{\mathscr{F}_s}
					 - \cexp{X^2_{\min\left\{t,\frac{j}{2^n}T\right\}}}{\mathscr{F}_s}
				\end{align}
				が成り立ち,また$P$-a.s.に
				\begin{align}
					&\cexp{\left(X_{\min\left\{t,\frac{k+1}{2^n}T\right\}} - X_{\frac{k}{2^n}T}\right)^2}{\mathscr{F}_s} \\
					&= \cexp{X_{\min\left\{t,\frac{k+1}{2^n}T\right\}}^2}{\mathscr{F}_s}
					- 2 \cexp{X_{\min\left\{t,\frac{k+1}{2^n}T\right\}}X_{\frac{k}{2^n}T}}{\mathscr{F}_s}
					+ \cexp{X_{\frac{k}{2^n}T}^2}{\mathscr{F}_s} \\
					&= \cexp{X_{\min\left\{t,\frac{k+1}{2^n}T\right\}}^2}{\mathscr{F}_s}
					- 2 X_{\frac{k}{2^n}T}^2
					+ \cexp{X_{\frac{k}{2^n}T}^2}{\mathscr{F}_s} \\
				\end{align}
				も成り立つので,
				\begin{align}
					&\cexp{A^{(n)}_t - A^{(n)}_s}{\mathscr{F}_s} \\
					&= \cexp{X^2_t}{\mathscr{F}_s}
					- \cexp{X^2_{\min\left\{t,\frac{k+1}{2^n}T\right\}}}{\mathscr{F}_s}
					+ \cexp{\left(X_{\min\left\{t,\frac{k+1}{2^n}T\right\}} - X_{\frac{k}{2^n}T}\right)^2}{\mathscr{F}_s}
					- \left(X_s - X_{\frac{k}{2^n}T}\right)^2 \\
					&= \cexp{X^2_t}{\mathscr{F}_s} - X_s^2
				\end{align}
				が従う.ゆえに$P$-a.s.に
				\begin{align}
					\cexp{X^2_t - A^{(n)}_t}{\mathscr{F}_s} = X^2_s - A^{(n)}_s
				\end{align}
				となる.以上で(\refeq{fom:thm_decomposition_of_local_martingales_2})が得られたが,
				$X$が連続である場合は$A^{(n)}$も連続であるから
				\begin{align}
					X^2 - A^{(n)} \in \mathscr{M}^{2,c}_{\mathbf{T}}
					\label{fom:thm_decomposition_of_local_martingales_3}
				\end{align}
				が成立する.
				
			\item[step1-3]
				任意の自然数$n$に対して
				\begin{align}
					\int_\Omega \left|X_T^2 - A_T^{(n)}\right|^2\ dP
					\leq 14 \cdot b^2 \cdot \int_\Omega \left|X_T\right|^2\ dP
				\end{align}
				が成り立つことを示す.いま
				\begin{align}
					M \defeq X^2 - A^{(n)}
				\end{align}
				とおけば
				\begin{align}
					\int_\Omega M_{\frac{j+1}{2^n}T}^2 - M_{\frac{j}{2^n}T}^2\ dP
					&= \int_\Omega \left(M_{\frac{j+1}{2^n}T} - M_{\frac{j}{2^n}T} \right)^2\ dP \\
					&= \int_\Omega \left\{ X^2_{\frac{j+1}{2^n}T} - X^2_{\frac{j}{2^n}T} -
					\left(A^{(n)}_{\frac{j+1}{2^n}T} - A^{(n)}_{\frac{j}{2^n}T}\right) \right\}^2\ dP \\
					&= \int_\Omega \left\{ X^2_{\frac{j+1}{2^n}T} - X^2_{\frac{j}{2^n}T} -
					\left(X_{\frac{j+1}{2^n}T} - X_{\frac{j}{2^n}T}\right)^2 \right\}^2\ dP
				\end{align}
				が成り立つから,
				\begin{align}
					\int_\Omega \left|M_T\right|^2\ dP
					&= \sum_{j=0}^{2^n-1} \int_\Omega M_{\frac{j+1}{2^n}T}^2 - M_{\frac{j}{2^n}T}^2\ dP \\
					&= \sum_{j=0}^{2^n-1} \int_\Omega \left\{ X^2_{\frac{j+1}{2^n}T} - X^2_{\frac{j}{2^n}T} -
					\left(X_{\frac{j+1}{2^n}T} - X_{\frac{j}{2^n}T}\right)^2 \right\}^2\ dP \\
					&= \sum_{j=0}^{2^n-1} \int_\Omega \left\{ X^2_{\frac{j+1}{2^n}T} - X^2_{\frac{j}{2^n}T} \right\}^2\ dP \\
						&\quad - 2 \cdot \sum_{j=0}^{2^n-1} \int_\Omega \left( X^2_{\frac{j+1}{2^n}T} - X^2_{\frac{j}{2^n}T} \right) \left(X_{\frac{j+1}{2^n}T} - X_{\frac{j}{2^n}T}\right)^2\ dP \\
						&\quad + \sum_{j=0}^{2^n-1} \int_\Omega \left(X_{\frac{j+1}{2^n}T} - X_{\frac{j}{2^n}T}\right)^4\ dP
					\label{fom:thm_decomposition_of_local_martingales_1}
				\end{align}
				が成り立つ.ここで
				\begin{align}
					\int_\Omega \left\{ X^2_{\frac{j+1}{2^n}T} - X^2_{\frac{j}{2^n}T} \right\}^2\ dP
					\leq 2 \cdot b^2 \cdot \int_\Omega X^2_{\frac{j+1}{2^n}T} - X^2_{\frac{j}{2^n}T}\ dP
				\end{align}
				かつ
				\begin{align}
					\int_\Omega \left( X^2_{\frac{j+1}{2^n}T} - X^2_{\frac{j}{2^n}T} \right) \left(X_{\frac{j+1}{2^n}T} - X_{\frac{j}{2^n}T}\right)^2\ dP
					\leq 4 \cdot b^2 \cdot \int_\Omega X^2_{\frac{j+1}{2^n}T} - X^2_{\frac{j}{2^n}T}\ dP
				\end{align}
				かつ
				\begin{align}
					\int_\Omega \left(X_{\frac{j+1}{2^n}T} - X_{\frac{j}{2^n}T}\right)^4\ dP
					&\leq 4 \cdot b^2 \cdot \int_\Omega X^2_{\frac{j+1}{2^n}T} - X^2_{\frac{j}{2^n}T}\ dP \\
					&= 4 \cdot b^2 \cdot \int_\Omega X^2_{\frac{j+1}{2^n}T} - X^2_{\frac{j}{2^n}T}\ dP
				\end{align}
				が成り立つので
				\begin{align}
					(\refeq{fom:thm_decomposition_of_local_martingales_1})
					&\leq 14 \cdot b^2 \cdot \sum_{j=0}^{2^n-1} \int_\Omega X^2_{\frac{j+1}{2^n}T} - X^2_{\frac{j}{2^n}T}\ dP \\
					&= 14 \cdot b^2 \cdot \int_\Omega \left|X_T\right|^2\ dP
				\end{align}
				が成立する.
				
			\item[step1-4]
				$\mathscr{M}^2_{\mathbf{T}}$に定理\ref{thm:pseudo_metric_on_square_integrable_martingales}の
				擬距離$d$を定めるとき,step1-3より
				\begin{align}
					\left\{X^2 - A^{(n)}\right\}_{n \in \Natural}
				\end{align}
				は$\left(\mathscr{M}^2_{\mathbf{T}},d\right)$で有界である.ゆえにKomlosの補題より
				\begin{align}
					h:\Natural \longrightarrow \mathscr{M}^2_{\mathbf{T}}
				\end{align}
				なる$\left(\mathscr{M}^2_{\mathbf{T}},d\right)$のCauchy列$h$で,任意の自然数$n$で
				\begin{align}
					h(n) \in \conv{\Set{X^2-A^{(k)}}{k \in \Natural \wedge n \leq k}}
				\end{align}
				を満たすものが取れて,定理\ref{thm:pseudo_metric_on_square_integrable_martingales}より
				\begin{align}
					d(h(n),M) \longrightarrow 0\quad (n \longrightarrow \infty)
				\end{align}
				なる$\mathscr{M}^2_{\mathbf{T}}$の要素$M$が取れる.特に$X$が連続である場合は,
				(\refeq{fom:thm_decomposition_of_local_martingales_3})より
				\begin{align}
					\left\{X^2 - A^{(n)}\right\}_{n \in \Natural} \subset \mathscr{M}^{2,c}_{\mathbf{T}}
				\end{align}
				が成り立つから,極限$M$を
				\begin{align}
					M \in \mathscr{M}^{2,c}_{\mathbf{T}}
				\end{align}
				なるものとして取れる.ここで
				\begin{align}
					A \defeq X^2 - M
				\end{align}
				とおく.定め方より$A$は$\{\mathscr{F}_t\}_{t \in \mathbf{T}}$-適合であって,右連続でもあり,特に$X$が連続なら$A$も連続である.
				あとは$A$の増大性を示せばよい.いま任意の自然数$k$に対し
				\begin{align}
					d(h(n_k),M) < \frac{1}{2^{k+1}}
				\end{align}
				を満たす部分列
				\begin{align}
					\Natural \ni k \longmapsto h(n_k)
				\end{align}
				を取ると,Doobの劣マルチンゲール不等式より任意の自然数$k$で
				\begin{align}
					\int_\Omega \left\{\sup{t \in [0,T]}{\left|\left(X^2_t - h(n_k)_t\right) - A_t\right|}\right\}^2\ dP
					&= \int_\Omega \left\{\sup{t \in [0,T]}{\left|h(n_k)_t - M_t\right|}\right\}^2\ dP \\
					&\leq 4 \cdot \int_\Omega \left|h(n_k)_T - M_T\right|^2\ dP \\
					&< \frac{1}{4^k}
				\end{align}
				が成り立つ.
				
		\end{description}
	\end{sketch}
	
	\begin{screen}
		\begin{dfn}[局所マルチンゲール]
		\end{dfn}
	\end{screen}
	
	\begin{screen}
		\begin{thm}[二乗可積分マルチンゲールは局所マルチンゲール]
			\begin{align}
				\mathscr{M}^2_{\mathbf{T}} \subset \mathscr{M}^{loc}_{\mathbf{T}}.
			\end{align}
		\end{thm}
	\end{screen}
	
	\begin{sketch}
		
	\end{sketch}
	
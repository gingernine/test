\section{二乗可積分マルチンゲール}
	本節では$(\Omega,\mathscr{F},P)$を確率空間とし,$\mathbf{T} \defeq [0,\infty[$か$[0,T]$とし,
	$\{\mathscr{F}_t\}_{t \in \mathbf{T}}$を$\mathscr{F}$に付随するフィルトレーションとする.
	
	
	\begin{screen}
		\begin{dfn}[二乗可積分マルチンゲール]
			$(\Omega,\mathscr{F},P)$上の右連続な$\{\mathscr{F}_t\}_{t \in \mathbf{T}}$-マルチンゲールで,
			$\Omega$の任意の要素$\omega$に対して$(0,\omega)$に$0$を対応させる
			ものの全体を
			\begin{align}
				\mathscr{M}_{\mathbf{T}}
			\end{align}
			とおく.$\mathscr{M}_{\mathbf{T}}$の要素のうち連続であるものの全体を
			\begin{align}
				\mathscr{M}^c_{\mathbf{T}}
			\end{align}
			とおく.$X$を$(\Omega,\mathscr{F},P)$上の$\{\mathscr{F}_t\}_{t \in \mathbf{T}}$-マルチンゲールで
			\begin{align}
				\forall t \in \mathbf{T}\, \left(\, \int_\Omega |X_t|^2\ dP < \infty\, \right)
			\end{align}
			を満たすものとするとき,$X$を{\bf 二乗可積分}{\bf (square integrable)}な
			$\{\mathscr{F}_t\}_{t \in \mathbf{T}}$-マルチンゲールと呼ぶ.
			$\mathscr{M}_{\mathbf{T}}$の要素のうち二乗可積分であるものの全体を
			\begin{align}
				\mathscr{M}^2_{\mathbf{T}}
			\end{align}
			とおき,また
			\begin{align}
				\mathscr{M}^{2,c}_{\mathbf{T}} \defeq \mathscr{M}^c_{\mathbf{T}} \cap \mathscr{M}^2_{\mathbf{T}}
			\end{align}
			とおく.
		\end{dfn}
	\end{screen}
	
	つまり$X$を$\mathscr{M}_{\mathbf{T}}$の要素とすれば,$X$は
	\begin{align}
		\forall \omega \in \Omega\, \left(\, X_0(\omega) = 0\, \right)
	\end{align}
	を満たす右連続な$\{\mathscr{F}_t\}_{t \in \mathbf{T}}$-マルチンゲールである.
	
	\begin{screen}
		\begin{thm}[Doobの劣マルチンゲール不等式]
		\end{thm}
	\end{screen}
	
	\begin{screen}
		\begin{thm}[$\mathscr{M}^2_{\mathbf{T}}$の線型構造]
			$X$と$Y$を$\mathscr{M}^2_{\mathbf{T}}$の要素とするとき,$(X,Y)$に対して
			\begin{align}
				(t,\omega) \longmapsto X(t,\omega) + Y(t,\omega)
			\end{align}
			なる写像を対応させる関係$+_m$を$\mathscr{M}^2_{\mathbf{T}}$上の加法とし,
			また$\alpha$を実数とするときに$(\alpha,X)$に対して
			\begin{align}
				(t,\omega) \longmapsto \alpha \cdot X(t,\omega)
			\end{align}
			なる写像を対応させる関係$s$をスカラ倍とすれば,
			\begin{align}
				\left(\left(\mathscr{M}^2_{\mathbf{T}},+_m \right),(\R,+,\bullet),s\right)
			\end{align}
			は線型空間である.また
			\begin{align}
				\left(\left(\mathscr{M}^{2,c}_{\mathbf{T}},+_m \right),(\R,+,\bullet),s\right)
			\end{align}
			はその線型部分空間である.
		\end{thm}
	\end{screen}
	
	\begin{screen}
		\begin{thm}[$\mathscr{M}^2_{\mathbf{T}}$は完備な擬距離空間]
		\label{thm:pseudo_metric_on_square_integrable_martingales}
			$\mathbf{T} = [0,T]$とし,$\{\mathscr{F}_t\}_{t \in \mathbf{T}}$は完備であるとする.このとき
			\begin{align}
				\mathscr{M}^2_{\mathbf{T}} \times \mathscr{M}^2_{\mathbf{T}} \ni (X,Y) \longmapsto
				\left\{\int_\Omega |X_T-Y_T|^2\ dP\right\}^{\frac{1}{2}}
			\end{align}
			なる関係を$d$とすると,$\left(\mathscr{M}^2_{\mathbf{T}},d\right)$は完備な擬距離空間である.
			また$\mathscr{M}^{2,c}_{\mathbf{T}}$はその完備な部分集合である.
		\end{thm}
	\end{screen}
	
	\begin{sketch}\mbox{}
		\begin{description}
			\item[第一段] 擬距離空間は可算な基本近縁系が取れるので,完備性はCauchy列の収束を示すだけで良い.いま
				\begin{align}
					\Natural \ni n \longmapsto X^{(n)} \in \mathscr{M}^2_{\mathbf{T}}
				\end{align}
				なる関係を$\left(\mathscr{M}^2_{\mathbf{T}},d\right)$のCauchy列とする.すると
				\begin{align}
					\forall k \in \Natural\, \left[\, d\left( X^{(n_k)},X^{(n_{k+1})} \right) < \frac{1}{4^{k+1}}\, \right]
				\end{align}
				を満たす部分列
				\begin{align}
					\Natural \ni k \longmapsto X^{(n_k)}
				\end{align}
				が取れる.このときDoobの劣マルチンゲール不等式から,任意の自然数$k$で
				\begin{align}
					\int_\Omega \left\{ \sup{t \in [0,T]}{\left|X_t^{(n_k)} - X_t^{(n_{k+1})}\right|} \right\}^2\ dP
					\leq 4 \int_\Omega \left|X_T^{(n_k)} - X_T^{(n_{k+1})}\right|^2\ dP
					< \frac{1}{8^k}
					\label{fom:thm_pseudo_metric_on_square_integrable_martingales_2}
				\end{align}
				が成立する.従って,自然数$k$に対して
				\begin{align}
					E_k \defeq \left\{ \frac{1}{2^k} \leq \sup{t \in [0,T]}{\left|X_t^{(n_k)} - X_t^{(n_{k+1})}\right|} \right\}
				\end{align}
				とおけば
				\begin{align}
					P(E_k) < \frac{1}{2^k}
				\end{align}
				が成立するので,Borel-Cantelliの補題より
				\begin{align}
					E \defeq \bigcap_{n \in \Natural} \bigcup_{\substack{k \in \Natural \\ n < k}} E_k
				\end{align}
				で定める$E$は$P$-零集合である.$\omega$を$\Omega \backslash E$の要素とすれば
				\begin{align}
					\forall k \in \Natural\,
					\left[\, N < k \Longrightarrow \sup{t \in [0,T]}{\left|X_t^{(n_k)}(\omega) - X_t^{(n_{k+1})}(\omega)\right|} < \frac{1}{2^k}\, \right]
					\label{fom:thm_pseudo_metric_on_square_integrable_martingales_1}
				\end{align}
				を満たす自然数$N$が取れる.ゆえに,いま$t$を$\mathbf{T}$の要素とすれば
				\begin{align}
					\Natural \ni k \longmapsto X_t^{(n_k)}(\omega)
				\end{align}
				は$\R$のCauchy列であり,$\R$で収束する.ここで
				\begin{align}
					\mathbf{T} \times \Omega \ni (t,\omega) \longmapsto
					\begin{cases}
						\lim_{k \to \infty} X_t^{(n_k)}(\omega) & \mbox{if } \omega \in \Omega \backslash E \\
						0 & \mbox{if } \omega \in E
					\end{cases} 
				\end{align}
				で定める関係を$X$とする.
			
			\item[第二段]
				$X$のパスが右連続(または連続)であることを示す.$\omega$を$\Omega \backslash E$の要素とすれば
				(\refeq{fom:thm_pseudo_metric_on_square_integrable_martingales_1})より
				\begin{align}
					\forall k \in \Natural\,
					\left[\, N < k \Longrightarrow \sup{t \in [0,T]}{\left|X_t^{(n_k)}(\omega) - X_t(\omega)\right|} \leq \frac{1}{2^k}\, \right]
				\end{align}
				を満たす自然数$N$が取れるので,パスは一様収束している.ゆえに
				\begin{align}
					\left\{X^{(n)}\right\}_{n \in \Natural} \subset \mathscr{M}^2_{\mathbf{T}}
				\end{align}
				ならば$X$は右連続であり,
				\begin{align}
					\left\{X^{(n)}\right\}_{n \in \Natural} \subset \mathscr{M}^{2,c}_{\mathbf{T}}
				\end{align}
				ならば$X$は連続である.
			
			\item[第三段]
				$X$が$\{\mathscr{F}_t\}_{t \in \mathbf{T}}$-適合であることを示す.
				$t$を$\mathbf{T}$の任意の要素とすれば
				\begin{align}
					\forall \omega \in \Omega\, \left(\, 
					\lim_{k \to \infty} X_t^{(n_k)}(\omega) \cdot \defunc_{\Omega \backslash E}(\omega) = X_t(\omega)\, \right)
				\end{align}
				が成り立ち,またフィルトレーションの完備性の仮定から
				\begin{align}
					E \in \mathscr{F}_t
				\end{align}
				なので,各自然数$k$で$X_t^{(n_k)} \defunc_{\Omega \backslash E}$は$\mathscr{F}_t/\borel{\R}$-可測である.
				よって定理\ref{lem:measurability_metric_space}より$X_t$は$\mathscr{F}_t/\borel{\R}$-可測である.
				
			\item[第四段]
				$X$が二乗可積分な$\{\mathscr{F}_t\}_{t \in \mathbf{T}}$-マルチンゲールであることを示す.
				$t$を$\mathbf{T}$の任意の要素とすれば,Fatouの補題と
				(\refeq{fom:thm_pseudo_metric_on_square_integrable_martingales_2})より
				任意の自然数$k$で
				\begin{align}
					\int_\Omega \left|X_t-X_t^{(n_k)}\right|^2\ dP
					\leq \sup{n \in \Natural}{\inf{\substack{j \in \Natural \\ n < j}}{
					\int_\Omega \left|X_t^{(n_j)}-X_t^{(n_k)}\right|^2\ dP}}
					\leq \frac{1}{4^k}
					\label{fom:thm_pseudo_metric_on_square_integrable_martingales_3}
 				\end{align}
 				が成立する.ゆえにMinkowskiの不等式から
 				\begin{align}
 					\left\{\int_\Omega |X_t|^2\ dP\right\}^{\frac{1}{2}}
 					\leq \left\{\int_\Omega \left|X_t - X^{(n_k)}_t\right|^2\ dP\right\}^{\frac{1}{2}}
 					+ \left\{\int_\Omega \left|X^{(n_k)}_t\right|^2\ dP\right\}^{\frac{1}{2}}
 					< \infty
 				\end{align}
 				が成立する.またH\Ddot{o}lderの不等式から
 				\begin{align}
 					\int_\Omega \left|X_t-X_t^{(n_k)}\right|\ dP
 					\leq \left\{\int_\Omega \left|X_t - X^{(n_k)}_t\right|^2\ dP\right\}^{\frac{1}{2}}
 					\longrightarrow 0\quad (k \longrightarrow \infty)
 				\end{align}
 				が成り立つ.ゆえに,いま$s$と$t$を
 				\begin{align}
 					s < t
 				\end{align}
 				なる$\mathbf{T}$の要素とすれば,$\mathscr{F}_s$の任意の要素$A$で
 				\begin{align}
 					\int_A X_t\ dP = \lim_{k \to \infty} \int_A X^{(n_k)}_t\ dP
 					= \lim_{k \to \infty} \int_A X^{(n_k)}_s\ dP
 					= \int_A X_s\ dP
 				\end{align}
 				が成り立つ.
 			
 			\item[第五段]
 				以上より
 				\begin{align}
 					X \in \mathscr{M}^2_{\mathbf{T}}
 				\end{align}
 				である.最後に,(\refeq{fom:thm_pseudo_metric_on_square_integrable_martingales_3})より
 				\begin{align}
 					d\left(X,X^{(n_k)}\right) = \int_\Omega \left|X_T-X_T^{(n_k)}\right|^2\ dP
 					\longrightarrow 0\quad (k \longrightarrow \infty)
 				\end{align}
 				が成り立つので$X$は$d$に関して$k \longmapsto X^{(n_k)}$の極限である.部分列の収束から
 				\begin{align}
 					d\left(X,X^{(n)}\right) \longrightarrow 0\quad (n \longrightarrow \infty)
 				\end{align}
 				が従う.
 				\QED
		\end{description}
	\end{sketch}
	
	\begin{screen}
		\begin{thm}[右連続マルチンゲールから停止時刻の増大列が作れる]
			$\mathbf{T} = [0,\infty[$とし,$X$を$\mathscr{M}_{\mathbf{T}}$の要素とする.自然数$n$に対して
			\begin{align}
				\omega \longmapsto 
				\begin{cases}
					\inf{}{\Set{t \in \mathbf{T}}{n \leq |X_t(\omega)|}} & \mbox{if } \Set{t \in \mathbf{T}}{n \leq |X_t(\omega)|} \neq \emptyset \\
					\infty & \mbox{if } \Set{t \in \mathbf{T}}{n \leq |X_t(\omega)|} = \emptyset
				\end{cases}
			\end{align}
			なる写像を$\tau_n$と定めると,$\tau_n$は$\{\mathscr{F}_t\}_{t \in \mathbf{T}}$-停止時刻であって,
			$\Omega$のすべての要素$\omega$に対して
			\begin{align}
				\forall n \in \Natural\, \left(\, \tau_n(\omega) \leq \tau_{n+1}(\omega)\, \right)
			\end{align}
			を満たす.またパスが$RCLL$である$\omega$に対しては
			\begin{align}
				\sup{n \in \Natural}{\tau_n(\omega)} = \infty
			\end{align}
			が成り立つ.またパスが連続である$\omega$に対しては
			\begin{align}
				\sup{t \in \mathbf{T}}{\left|X^{\tau_n}_t(\omega)\right|} \leq n
			\end{align}
			が成り立つ.
		\end{thm}
	\end{screen}
	
	\begin{sketch}
		$\mathbf{T}$の任意の要素$t$に対して
		\begin{align}
			\left\{\tau_n \leq t\right\} = \Set{\omega \in \Omega}{n \leq |X_t(\omega)|}
		\end{align}
		が成り立つので$\tau_n$は$\{\mathscr{F}_t\}_{t \in \mathbf{T}}$-停止時刻である.
		また$RLCC$なパスは有界区間上で有界であるから,パスが$RCLL$である$\omega$に対しては
		\begin{align}
			\sup{n \in \Natural}{\tau_n(\omega)} = \infty
		\end{align}
		が成り立つ.
		\QED
	\end{sketch}
	
	右連続な劣マルチンゲールは殆ど全てのパスが$RCLL$なので,
	上の様に構成する停止時刻の列$\left\{\tau_n\right\}_{n \in \Natural}$は殆どすべての$\omega$に対し
	\begin{align}
		0 = \tau_0(\omega) \leq \tau_1(\omega) \leq \tau_2(\omega) \leq \longrightarrow \infty
	\end{align}
	を満たす.
	
	\begin{screen}
		\begin{dfn}[増大過程]
			$A$が$\mathbf{T} \times \Omega$上の$\R$値写像であって,かつ
			\begin{itemize}
				\item $\forall \omega \in \Omega\, (\, A_0(\omega) = 0\, )$
				\item $A$は$\{\mathscr{F}_t\}_{t \in \mathbf{T}}$-適合
				\item $A$のすべてのパスが右連続かつ単調非減少
				\item $\mathbf{T}$の任意の要素$t$で$E(A_t) < \infty$
			\end{itemize}
			を満たすとき,$A$を$(\Omega,\mathscr{F},P)$上の$\{\mathscr{F}_t\}_{t \in \mathbf{T}}$-{\bf 増大過程}
			\index{ぞうだいかてい@増大過程}{\bf (increasing process)}と呼ぶ.
		\end{dfn}
	\end{screen}
	
	\begin{screen}
		\begin{dfn}[ナチュラル]
			$A$を$(\Omega,\mathscr{F},P)$上の$\{\mathscr{F}_t\}_{t \in \mathbf{T}}$-増大過程とする.
			$M$を任意に与えられた$(\Omega,\mathscr{F},P)$上の有界かつ$RCLL$な
			$\{\mathscr{F}_t\}_{t \in \mathbf{T}}$-マルチンゲールとするとき
			\begin{align}
				\forall t \in \mathbf{T}\, 
				\left[\, E(M_t A_t) = E \int_{(0,t]} M_{s-}\ dA_s\, \right]
			\end{align}
			が成り立つならば,$A$は{\bf ナチュラル}\index{ナチュラル}{\bf (natural)}であるという.
			$(\Omega,\mathscr{F},P)$上のナチュラル過程の全体を
			\begin{align}
				\mathscr{N}_{\mathbf{T}}
			\end{align}
			で表し,区別不能性で類別した商集合を
			\begin{align}
				\mathfrak{N}_{\mathbf{T}}
			\end{align}
			で表す.同様に,$(\Omega,\mathscr{F},P)$上の連続なナチュラル過程の全体を
			\begin{align}
				\mathscr{N}^c_{\mathbf{T}}
			\end{align}
			で表し,区別不能性で類別した商集合を
			\begin{align}
				\mathfrak{N}^c_{\mathbf{T}}
			\end{align}
			で表す.
		\end{dfn}
	\end{screen}
	
	\begin{screen}
		\begin{thm}[連続な増大過程はナチュラル]
		\end{thm}
	\end{screen}
	
	\begin{screen}
		\begin{thm}[有界変動なマルチンゲールは殆ど全てのパスが定値]
		\label{thm:martingale_with_bounded_variation_is_constant}
			$\mathbf{T}$を$[0,\infty[$か$[0,T]$とし,$A$と$B$を$\mathscr{N}_{\mathbf{T}}$の要素とし,
			\begin{align}
				A - B \in \mathscr{M}_{\mathbf{T}}
			\end{align}
			が成り立っているとする.このとき,
			$\{\mathscr{F}_t\}_{t \in \mathbf{T}}$が右連続である場合,
			或いは$A$と$B$が共に連続である場合,
			\begin{align}
				\forall t \in \mathbf{T}\, \forall \omega \in \Omega \backslash E\,
				\left(\, A(t,\omega) = B(t,\omega)\, \right)
			\end{align}
			を満たす$P$-零集合$E$が取れる.
		\end{thm}
	\end{screen}
	
	\begin{sketch}\mbox{}
		\begin{description}
			\item[]
		\end{description}
	\end{sketch}
	
	\begin{screen}
		\begin{thm}[二乗可積分マルチンゲールは増大過程とマルチンゲールに分解できる]
		\label{thm:decomposition_of_square_integrable_martingales}
			$\mathbf{T}$を$[0,\infty[$か$[0,T]$とし,
			$X$を$\mathscr{M}^2_{\mathbf{T}}$の要素とする.このとき,
			\begin{itemize}
				\item $\{\mathscr{F}_t\}_{t \in \mathbf{T}}$が右連続かつ完備ならば,
					$\mathfrak{N}_{\mathbf{T}}$の要素$Q$が唯一つ取れて,
					$Q$の任意の要素$A$に対して
					\begin{align}
						X^2 - A \in \mathscr{M}_{\mathbf{T}}.
					\end{align}
					
				\item $X$が連続であるとき,$\{\mathscr{F}_t\}_{t \in \mathbf{T}}$が完備ならば
					$\mathfrak{N}^c_{\mathbf{T}}$の要素$Q$が唯一つ取れて,
					$Q$の任意の要素$A$に対して
					\begin{align}
						X^2 - A \in \mathscr{M}_{\mathbf{T}}.
					\end{align}
			\end{itemize}
		\end{thm}
	\end{screen}
	
	\begin{sketch}
			\textcolor{red}{この証明は怪しい.$X$が右連続の場合はDoob-Meyer分解から示した方が無難か...
			$X$が連続の場合は問題ないのだが.何とかして右連続の場合と連続の場合を包括的に証明したかったのだがうまくいかない...}
		\begin{description}
			\item[step1-1] $\mathbf{T}=[0,T]$とし,$X$が有界であるとする.つまりいま
				\begin{align}
					\forall t \in \mathbf{T}\, \forall \omega \in \Omega\,
					\left(\, |X(t,\omega)| \leq b\, \right)
				\end{align}
				を満たす実数$b$が取れる.
			
			\item[step1-2] $n$を自然数とし,
				\begin{align}
					(t,\omega) \longmapsto \sum_{j=0}^{2^n-1} \left(X_{\min\left\{t,\frac{j+1}{2^n}T\right\}}(\omega)
					- X_{\min\left\{t,\frac{j}{2^n}T\right\}}(\omega)\right)^2
				\end{align}
				なる写像を$A^{(n)}$とする.このとき
				\begin{align}
					X^2 - A^{(n)} \in \mathscr{M}^2_{\mathbf{T}}
					\label{fom:thm_decomposition_of_square_integrable_martingales_2}
				\end{align}
				であることを示す.まず$X$の右連続性より$A^{(n)}$も右連続である.また全ての$j$で
				\begin{align}
					X_{\min\left\{t,\frac{j}{2^n}T\right\}}
				\end{align}
				は$\mathscr{F}_t/\borel{\R}$-可測なので,$X^2-A^{(n)}$は$\{\mathscr{F}_t\}_{t \in \mathbf{T}}$-適合である.
				いまは$X$も$A^{(n)}$も有界であるから,$\mathbf{T}$の任意の要素$t$で
				\begin{align}
					\int_{\Omega} \left|X_t^2-A_t^{(n)}\right|^2\ dP < \infty
				\end{align}
				が成立する.$s$と$t$を
				\begin{align}
					s < t
				\end{align}
				なる$\mathbf{T}$の要素とする.そして
				\begin{align}
					\frac{k}{2^n}T \leq s < \frac{k+1}{2^n}T
				\end{align}
				を満たす$k$を取る.このとき
				\begin{align}
					A^{(n)}_t - A^{(n)}_s
					&= \sum_{j=k}^{2^n-1} \left\{
					\left(X_{\min\left\{t,\frac{j+1}{2^n}T\right\}} - X_{\min\left\{t,\frac{j}{2^n}T\right\}}\right)^2
					- \left(X_{\min\left\{s,\frac{j+1}{2^n}T\right\}} - X_{\min\left\{s,\frac{j}{2^n}T\right\}}\right)^2\right\} \\
					&= \sum_{j=k+1}^{2^n-1} \left(X_{\min\left\{t,\frac{j+1}{2^n}T\right\}} - X_{\min\left\{t,\frac{j}{2^n}T\right\}}\right)^2
					+ \left(X_{\min\left\{t,\frac{k+1}{2^n}T\right\}} - X_{\frac{k}{2^n}T}\right)^2
					- \left(X_s - X_{\frac{k}{2^n}T}\right)^2
				\end{align}
				が成り立つ.この各項について,
				\begin{align}
					j \in \{k+1,K+2,\cdots,2^n-1\}
				\end{align}
				なる各$j$で$P$-a.s.に
				\begin{align}
					\cexp{ \left(X_{\min\left\{t,\frac{j+1}{2^n}T\right\}} - X_{\min\left\{t,\frac{j}{2^n}T\right\}}\right)^2}{\mathscr{F}_s}
					= \cexp{X^2_{\min\left\{t,\frac{j+1}{2^n}T\right\}}}{\mathscr{F}_s}
					 - \cexp{X^2_{\min\left\{t,\frac{j}{2^n}T\right\}}}{\mathscr{F}_s}
				\end{align}
				が成り立ち,また$P$-a.s.に
				\begin{align}
					&\cexp{\left(X_{\min\left\{t,\frac{k+1}{2^n}T\right\}} - X_{\frac{k}{2^n}T}\right)^2}{\mathscr{F}_s} \\
					&= \cexp{X_{\min\left\{t,\frac{k+1}{2^n}T\right\}}^2}{\mathscr{F}_s}
					- 2 \cexp{X_{\min\left\{t,\frac{k+1}{2^n}T\right\}}X_{\frac{k}{2^n}T}}{\mathscr{F}_s}
					+ \cexp{X_{\frac{k}{2^n}T}^2}{\mathscr{F}_s} \\
					&= \cexp{X_{\min\left\{t,\frac{k+1}{2^n}T\right\}}^2}{\mathscr{F}_s}
					- 2 X_{\frac{k}{2^n}T}^2
					+ \cexp{X_{\frac{k}{2^n}T}^2}{\mathscr{F}_s} \\
				\end{align}
				も成り立つので,
				\begin{align}
					&\cexp{A^{(n)}_t - A^{(n)}_s}{\mathscr{F}_s} \\
					&= \cexp{X^2_t}{\mathscr{F}_s}
					- \cexp{X^2_{\min\left\{t,\frac{k+1}{2^n}T\right\}}}{\mathscr{F}_s}
					+ \cexp{\left(X_{\min\left\{t,\frac{k+1}{2^n}T\right\}} - X_{\frac{k}{2^n}T}\right)^2}{\mathscr{F}_s}
					- \left(X_s - X_{\frac{k}{2^n}T}\right)^2 \\
					&= \cexp{X^2_t}{\mathscr{F}_s} - X_s^2
				\end{align}
				が従う.ゆえに$P$-a.s.に
				\begin{align}
					\cexp{X^2_t - A^{(n)}_t}{\mathscr{F}_s} = X^2_s - A^{(n)}_s
				\end{align}
				となる.以上で(\refeq{fom:thm_decomposition_of_square_integrable_martingales_2})が得られたが,
				$X$が連続である場合は$A^{(n)}$も連続であるから
				\begin{align}
					X^2 - A^{(n)} \in \mathscr{M}^{2,c}_{\mathbf{T}}
					\label{fom:thm_decomposition_of_square_integrable_martingales_3}
				\end{align}
				が成立する.
				
			\item[step1-3]
				任意の自然数$n$に対して
				\begin{align}
					\int_\Omega \left|X_T^2 - A_T^{(n)}\right|^2\ dP
					\leq 14 \cdot b^2 \cdot \int_\Omega \left|X_T\right|^2\ dP
				\end{align}
				が成り立つことを示す.いま
				\begin{align}
					M \defeq X^2 - A^{(n)}
				\end{align}
				とおけば
				\begin{align}
					\int_\Omega M_{\frac{j+1}{2^n}T}^2 - M_{\frac{j}{2^n}T}^2\ dP
					&= \int_\Omega \left(M_{\frac{j+1}{2^n}T} - M_{\frac{j}{2^n}T} \right)^2\ dP \\
					&= \int_\Omega \left\{ X^2_{\frac{j+1}{2^n}T} - X^2_{\frac{j}{2^n}T} -
					\left(A^{(n)}_{\frac{j+1}{2^n}T} - A^{(n)}_{\frac{j}{2^n}T}\right) \right\}^2\ dP \\
					&= \int_\Omega \left\{ X^2_{\frac{j+1}{2^n}T} - X^2_{\frac{j}{2^n}T} -
					\left(X_{\frac{j+1}{2^n}T} - X_{\frac{j}{2^n}T}\right)^2 \right\}^2\ dP
				\end{align}
				が成り立つから,
				\begin{align}
					\int_\Omega \left|M_T\right|^2\ dP
					&= \sum_{j=0}^{2^n-1} \int_\Omega M_{\frac{j+1}{2^n}T}^2 - M_{\frac{j}{2^n}T}^2\ dP \\
					&= \sum_{j=0}^{2^n-1} \int_\Omega \left\{ X^2_{\frac{j+1}{2^n}T} - X^2_{\frac{j}{2^n}T} -
					\left(X_{\frac{j+1}{2^n}T} - X_{\frac{j}{2^n}T}\right)^2 \right\}^2\ dP \\
					&= \sum_{j=0}^{2^n-1} \int_\Omega \left\{ X^2_{\frac{j+1}{2^n}T} - X^2_{\frac{j}{2^n}T} \right\}^2\ dP \\
						&\quad - 2 \cdot \sum_{j=0}^{2^n-1} \int_\Omega \left( X^2_{\frac{j+1}{2^n}T} - X^2_{\frac{j}{2^n}T} \right) \left(X_{\frac{j+1}{2^n}T} - X_{\frac{j}{2^n}T}\right)^2\ dP \\
						&\quad + \sum_{j=0}^{2^n-1} \int_\Omega \left(X_{\frac{j+1}{2^n}T} - X_{\frac{j}{2^n}T}\right)^4\ dP
					\label{fom:thm_decomposition_of_square_integrable_martingales_1}
				\end{align}
				が成り立つ.ここで
				\begin{align}
					\int_\Omega \left\{ X^2_{\frac{j+1}{2^n}T} - X^2_{\frac{j}{2^n}T} \right\}^2\ dP
					\leq 2 \cdot b^2 \cdot \int_\Omega X^2_{\frac{j+1}{2^n}T} - X^2_{\frac{j}{2^n}T}\ dP
				\end{align}
				かつ
				\begin{align}
					\int_\Omega \left( X^2_{\frac{j+1}{2^n}T} - X^2_{\frac{j}{2^n}T} \right) \left(X_{\frac{j+1}{2^n}T} - X_{\frac{j}{2^n}T}\right)^2\ dP
					\leq 4 \cdot b^2 \cdot \int_\Omega X^2_{\frac{j+1}{2^n}T} - X^2_{\frac{j}{2^n}T}\ dP
				\end{align}
				かつ
				\begin{align}
					\int_\Omega \left(X_{\frac{j+1}{2^n}T} - X_{\frac{j}{2^n}T}\right)^4\ dP
					&\leq 4 \cdot b^2 \cdot \int_\Omega X^2_{\frac{j+1}{2^n}T} - X^2_{\frac{j}{2^n}T}\ dP \\
					&= 4 \cdot b^2 \cdot \int_\Omega X^2_{\frac{j+1}{2^n}T} - X^2_{\frac{j}{2^n}T}\ dP
				\end{align}
				が成り立つので
				\begin{align}
					(\refeq{fom:thm_decomposition_of_square_integrable_martingales_1})
					&\leq 14 \cdot b^2 \cdot \sum_{j=0}^{2^n-1} \int_\Omega X^2_{\frac{j+1}{2^n}T} - X^2_{\frac{j}{2^n}T}\ dP \\
					&= 14 \cdot b^2 \cdot \int_\Omega \left|X_T\right|^2\ dP
				\end{align}
				が成立する.
				
			\item[step1-4]
				$\mathscr{M}^2_{\mathbf{T}}$に定理\ref{thm:pseudo_metric_on_square_integrable_martingales}の
				擬距離$d$を定めるとき,step1-3より
				\begin{align}
					\left\{X^2 - A^{(n)}\right\}_{n \in \Natural}
				\end{align}
				は$\left(\mathscr{M}^2_{\mathbf{T}},d\right)$で有界である.ゆえにKomlosの補題より
				\begin{align}
					h:\Natural \longrightarrow \mathscr{M}^2_{\mathbf{T}}
				\end{align}
				なる$\left(\mathscr{M}^2_{\mathbf{T}},d\right)$のCauchy列$h$で,任意の自然数$n$で
				\begin{align}
					h(n) \in \conv{\Set{X^2-A^{(k)}}{k \in \Natural \wedge n \leq k}}
					\label{fom:thm_decomposition_of_square_integrable_martingales_5}
				\end{align}
				を満たすものが取れて,定理\ref{thm:pseudo_metric_on_square_integrable_martingales}より
				\begin{align}
					d(h(n),M) \longrightarrow 0\quad (n \longrightarrow \infty)
				\end{align}
				なる$\mathscr{M}^2_{\mathbf{T}}$の要素$M$が取れる.特に$X$が連続である場合は
				(\refeq{fom:thm_decomposition_of_square_integrable_martingales_3})より
				\begin{align}
					\left\{h(n)\right\}_{n \in \Natural} \subset \mathscr{M}^{2,c}_{\mathbf{T}}
				\end{align}
				が成り立つから,極限$M$を
				\begin{align}
					M \in \mathscr{M}^{2,c}_{\mathbf{T}}
				\end{align}
				なるものとして取れる.ここで
				\begin{align}
					A \defeq X^2 - M
				\end{align}
				とおく.定め方より$A$は$\{\mathscr{F}_t\}_{t \in \mathbf{T}}$-適合であって,右連続でもあり,特に$X$が連続なら$A$も連続である.
				あとは$A$の増大性を示せばよい.いま任意の自然数$k$に対し
				\begin{align}
					d(h(n_k),M) < \frac{1}{4^{k+1}}
				\end{align}
				を満たす部分列
				\begin{align}
					\Natural \ni k \longmapsto h(n_k)
				\end{align}
				を取ると,Doobの劣マルチンゲール不等式より任意の自然数$k$で
				\begin{align}
					\int_\Omega \left\{\sup{t \in [0,T]}{\left|\left(X^2_t - h(n_k)_t\right) - A_t\right|}\right\}^2\ dP
					&= \int_\Omega \left\{\sup{t \in [0,T]}{\left|h(n_k)_t - M_t\right|}\right\}^2\ dP \\
					&\leq 4 \cdot \int_\Omega \left|h(n_k)_T - M_T\right|^2\ dP \\
					&< \frac{1}{8^k}
				\end{align}
				が成り立つ.従って,自然数$k$に対して
				\begin{align}
					E_k \defeq \left\{ \frac{1}{2^k} \leq \sup{t \in [0,T]}{\left|\left(X^2_t - h(n_k)_t\right) - A_t\right|} \right\}
				\end{align}
				とおけば
				\begin{align}
					P(E_k) < \frac{1}{2^k}
				\end{align}
				が成立するので,Borel-Cantelliの補題より
				\begin{align}
					E \defeq \bigcap_{n \in \Natural} \bigcup_{\substack{k \in \Natural \\ n < k}} E_k
				\end{align}
				で定める$E$は$P$-零集合である.いま$\omega$を$\Omega \backslash E$の要素とする.すると
				\begin{align}
					\forall k \in \Natural\,
					\left[\, K < k \Longrightarrow \sup{t \in [0,T]}{\left|\left(X^2_t(\omega) - h(n_k)_t(\omega)\right) - A_t(\omega)\right|} < \frac{1}{2^k}\, \right]
					\label{fom:thm_decomposition_of_square_integrable_martingales_4}
				\end{align}
				を満たす自然数$K$が取れる.ここで
				\begin{align}
					D_k \defeq \Set{\frac{j}{2^{n_k}}T}{j \in \{0,1,\cdots,2^{n_k}\}}
				\end{align}
				とおき
				\begin{align}
					D \defeq \bigcup_{k \in \Natural} D_k
				\end{align}
				とおく.$s$と$t$を
				\begin{align}
					s < t
				\end{align}
				なる$D$の要素とするとき
				\begin{align}
					s \in D_N \wedge t \in D_N
				\end{align}
				を満たす自然数$N$が取れるが,このとき
				\begin{align}
					N \leq k
				\end{align}
				なる任意の自然数$k$に対して
				\begin{align}
					X^2_s(\omega) - h(n_k)_s(\omega) \leq X^2_t(\omega) - h(n_k)_t(\omega)
				\end{align}
				が成り立つ.なぜならば(\refeq{fom:thm_decomposition_of_square_integrable_martingales_5})より
				\begin{align}
					X^2 - h(n_k) \in \conv{\Set{A^{(j)}}{j \in \Natural \wedge n_k \leq j}}
				\end{align}
				となるためである.そして(\refeq{fom:thm_decomposition_of_square_integrable_martingales_4})と併せて
				\begin{align}
					A_s(\omega) \leq A_t(\omega)
				\end{align}
				が従う.$s$と$t$が
				\begin{align}
					s < t
				\end{align}
				なる$\mathbf{T}$の要素であるときは,
				\begin{align}
					\{t_n\}_{n \in \Natural} \subset [t,T] \cap D
				\end{align}
				かつ
				\begin{align}
					t_n \downarrow t
				\end{align}
				を満たす$\{t_n\}_{n \in \Natural}$と
				\begin{align}
					\{s_n\}_{n \in \Natural} \subset [s,t) \cap D
				\end{align}
				かつ
				\begin{align}
					s_n \downarrow s
				\end{align}
				を満たす$\{s_n\}_{n \in \Natural}$を取れば
				\begin{align}
					\forall n \in \Natural\, \left(\, A_{s_n}(\omega) \leq A_{t_n}(\omega)\, \right)
				\end{align}
				が成り立ち,また$A$は右連続であるから
				\begin{align}
					A_s(\omega) \leq A_t(\omega)
				\end{align}
				が従う.ちなみに$\Omega$の任意の要素$\omega$で
				\begin{align}
					X^2_0(\omega) - h(n_k)_0(\omega) = 0
				\end{align}
				が成り立つので(\refeq{fom:thm_decomposition_of_square_integrable_martingales_4})から
				\begin{align}
					\omega \in \Omega \backslash E \Longrightarrow A_0(\omega) = 0
				\end{align}
				も満たされる.最後に,
				\begin{align}
					(t,\omega) \longmapsto
					\begin{cases}
						A_t(\omega) & \mbox{if } \omega \in \Omega \backslash E \\
						0 & \mbox{if } \omega \in E
					\end{cases}
				\end{align}
				なる写像を$\tilde{A}$とすれば,
				\begin{align}
					\tilde{A} \in \mathscr{A}^+_{\mathbf{T}}
				\end{align}
				かつ
				\begin{align}
					X^2 - \tilde{A} \in \mathscr{M}^2_{\mathbf{T}}
				\end{align}
				が成立する.とくに$X$が連続である場合は$\tilde{A}$も連続である.
			
			\item[step1-5]
				前段で得られた$\tilde{A}$がナチュラルであることを示す.
				あれ...示せない...
				
			\item[step2]
				$\mathbf{T}=[0,\infty[$とし,$X$が有界であるとする.
				$N$を$0$でない自然数とすれば,step1の結果より
				\begin{align}
					\left(X|_{[0,N] \times \Omega}\right)^2 - A^N \in \mathscr{M}^2_{[0,N]}
				\end{align}
				かつ
				\begin{align}
					A^N \in \mathscr{A}^+_{[0,N]}
				\end{align}
				を満たす$A^N$が取れる.特に$X$が連続であれば$A^N$は連続なものとして取れる.$M$を
				\begin{align}
					N < M
				\end{align}
				なる自然数とすれば
				\begin{align}
					\left(X|_{[0,N] \times \Omega}\right)^2 - A^M|_{[0,N] \times \Omega} \in \mathscr{M}^2_{[0,N]}
				\end{align}
				が成立するので,
				\begin{align}
					E_{N,M} \defeq \Set{\omega \in \Omega}{\exists t \in [0,N]\,
					\left(\, A^N_t(\omega) \neq A^M_t(\omega)\, \right)}
				\end{align}
				で定める集合は$P$-零集合である.
				\begin{align}
					E \defeq \bigcup_{\substack{N,M \in \Natural \\ N < M}} E_{N,M}
				\end{align}
				により$P$-零集合を定めて
				\begin{align}
					(t,\omega) \longmapsto
					\begin{cases}
						A^N(t,\omega) & \mbox{if } N \in \Natural \wedge t \leq N \wedge \omega \in \Omega \backslash E \\
						0 & \mbox{if } \omega \in E
					\end{cases}	
				\end{align}
				なる関係を$A$と定めれば
				\begin{align}
					A:\mathbf{T} \times \Omega \longrightarrow \R
				\end{align}
				が成立する.そして任意の自然数$N$に対して
				\begin{align}
					A|_{[0,N] \times \Omega} = A^N \defunc_{[0,N] \times \Omega \backslash E}
				\end{align}
				が成立する.ゆえに
				\begin{align}
					A \in \mathscr{A}^+_{\mathbf{T}}
				\end{align}
				が成立し,特に$X$が連続ならば$A$も連続である.
				$t$を$\mathbf{T}$の任意の要素とすれば,
				\begin{align}
					t \leq N
				\end{align}
				なる自然数$N$に対して
				\begin{align}
					E\left|X^2_t - A_t\right| = E\left|X^2_t - A^N_t\right| < \infty
				\end{align}
				が成立し,また$s$と$t$を$\mathbf{T}$の任意の要素として
				\begin{align}
					t \leq N
				\end{align}
				なる自然数$N$を取れば,$\mathscr{F}_s$の任意の要素$A$に対して
				\begin{align}
					\int_A X^2_t - A_t\ dP
					= \int_A X^2_t - A^N_t\ dP
					= \int_A X^2_s - A^N_s\ dP
					= \int_A X^2_s - A_s\ dP
				\end{align}
				が成立する.ゆえに
				\begin{align}
					X^2 - A \in \mathscr{M}_{\mathbf{T}}
				\end{align}
				である.
				
			\item[step3]
				$\mathbf{T}=[0,\infty[$とし,$X$を$\mathscr{M}^2_{\mathbf{T}}$の要素とする.
				\begin{align}
					\left(X^{\tau_n}\right)^2 - A^n \in \mathscr{M}^2_{\mathbf{T}}
				\end{align}
				となる.そして
				\begin{align}
					(t,\omega) \longmapsto
					\begin{cases}
						\lim_{n \to \infty} A^n_{t \wedge \tau_n(\omega)}(\omega) & \\
						0
					\end{cases}
				\end{align}
				として$A$を定めると$A$は適合,増大,(右)連続
				\begin{align}
					A_{t \wedge \tau_n} = A^n_{t \wedge \tau_n}
				\end{align}
				が成り立つ.そして
				\begin{align}
					\int_\Omega \left(X^{\tau_n}_t\right)^2\ dP = \int_\Omega A_{t \wedge \tau_n}\ dP
				\end{align}
				から
				\begin{align}
					\int_\Omega X_t^2\ dP = \int_\Omega A_t\ dP
				\end{align}
				となるので
				\begin{align}
					X_t^2 - A_t
				\end{align}
				は可積分.また
				\begin{align}
					\int_A X_t^2 - A_t\ dP
					= \lim_{n \to \infty} \int_A \left(X^{\tau_n}_t\right)^2 - A^n_t\ dP
					= \lim_{n \to \infty} \int_A \left(X^{\tau_n}_s\right)^2 - A^n_s\ dP
					= \int_A X_s^2 - A_s\ dP
				\end{align}
				が成り立つ.
				
			\item[step4]
				$\mathbf{T}=[0,T]$とし,$X$を$\mathscr{M}^2_{\mathbf{T}}$の要素とする.
				\begin{align}
					(t,\omega) \longmapsto
					\begin{cases}
						X(t,\omega) & \mbox{if } t \leq T \\
						X(T,\omega) & \mbox{if } T < t
					\end{cases}
				\end{align}
				なる写像を$Y$として,また
				\begin{align}
					[0,\infty[ \ni t \longmapsto 
					\begin{cases}
						\mathscr{F}_t & \mbox{if } t \leq T \\
						\mathscr{F}_T & \mbox{if } T < t
					\end{cases}
				\end{align}
				なる写像でフィルトレーション
				\begin{align}
					\{\mathscr{F}_t\}_{t \in [0,\infty[}
				\end{align}
				を定めれば
				\begin{align}
					Y \in \mathscr{M}^2_{[0,\infty[}
				\end{align}
				が成り立つ.ゆえにstep3の結果より
				\begin{align}
					B \in \mathscr{A}^+_{[0,\infty[}
				\end{align}
				かつ
				\begin{align}
					Y^2 - B \in \mathscr{M}_{[0,\infty[}
				\end{align}
				を満たす$B$が取れる.
				\begin{align}
					A \defeq B|_{[0,T] \times \Omega}
				\end{align}
				で$A$を定めれば,
				\begin{align}
					A \in \mathscr{A}^+_{\mathbf{T}}
				\end{align}
				かつ
				\begin{align}
					X^2 - A \in \mathscr{M}_{\mathbf{T}}
				\end{align}
				が成立する.
				\QED
		\end{description}
	\end{sketch}
	
	\begin{screen}
		\begin{dfn}[局所マルチンゲール]
		\end{dfn}
	\end{screen}
	
	\begin{screen}
		\begin{thm}[マルチンゲールは局所マルチンゲール]
			\begin{align}
				\mathscr{M}_{\mathbf{T}} \subset \mathscr{M}^{loc}_{\mathbf{T}}.
			\end{align}
		\end{thm}
	\end{screen}
	
	\begin{sketch}
		
	\end{sketch}
	
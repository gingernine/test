\section{停止時刻}
	$(\Omega,\mathscr{F},P)$を確率空間とし,$\mathbf{T} \defeq [0,\infty[$か$[0,T]$とし,
	$\{\mathscr{F}_t\}_{t \in \mathbf{T}}$を$\mathscr{F}$に付随するフィルトレーションとする.
	
	$\tau$を
	\begin{align}
		\tau:\Omega \longrightarrow [0,\infty]
	\end{align}
	なる写像で
	\begin{align}
		\forall t \in \mathbf{T}\, \left(\, \{\tau \leq t\} \in \mathscr{F}_t\, \right)
	\end{align}
	を満たすものとするとき,これを$\{\mathscr{F}_t\}_{t \in \mathbf{T}}$-{\bf 停止時刻}\index{ていしじこく@停止時刻}{\bf (stopping time)}と呼ぶ.
	
	\begin{screen}
		\begin{thm}[発展的可測過程と停止時刻の合成の可測性]
		\label{thm:composition_of_progressively_measurable_process_and_stopping_time}
			$f$を$\{\mathscr{F}_t\}_{t \in \mathbf{T}}$-発展的可測過程とし,
			$\tau$を$\{\mathscr{F}_t\}_{t \in \mathbf{T}}$-停止時刻とし,
			$f_\tau$を
			\begin{align}
				\Omega \ni \omega \longmapsto f(\tau(\omega),\omega)
			\end{align}
			なる関係で定める写像とする.このとき$f_\tau$は$\mathscr{F}_\tau/\borel{\R}$-可測である.
		\end{thm}
	\end{screen}
	
	\begin{sketch}
		$t$を$\mathbf{T}$から任意に選ばれた要素とし,$E$を$\borel{\R}$から任意に選ばれた要素とする.
		\begin{align}
			f_\tau^{-1} \ast E = \Set{\omega \in \Omega}{f(\tau(\omega),\omega) \in E} 
		\end{align}
		なので
		\begin{align}
			\Set{\omega \in \Omega}{f(\tau(\omega),\omega) \in E} \cap \{\tau \leq t\}
			\in \mathscr{F}_t
		\end{align}
		が成り立てばよい.ところで左辺は
		\begin{align}
			\Set{\omega \in \Omega}{f|_{[0,t] \times \Omega}(\tau(\omega) \wedge t,\omega) \in E} \cap \{\tau \leq t\}
		\end{align}
		に一致する.$\tau \wedge t$は$\mathscr{F}_t/\borel{[0,t]}$-可測なので
		\begin{align}
			\Omega \ni \omega \longmapsto (\tau(\omega) \wedge t,\omega)
		\end{align}
		は$\mathscr{F}_t/\borel{[0,t]} \otimes \mathscr{F}_t$-可測であり,
		$f|_{[0,t] \times \Omega}$は$\borel{[0,t]} \otimes \mathscr{F}_t/\borel{\R}$-可測である.
		ゆえに
		\begin{align}
			\Set{\omega \in \Omega}{f|_{[0,t] \times \Omega}(\tau(\omega) \wedge t,\omega) \in E} \in \mathscr{F}_t
		\end{align}
		が成り立つ.ゆえに$f_\tau$は$\mathscr{F}_\tau/\borel{\R}$-可測である.
		\QED
	\end{sketch}
	
	\begin{screen}
		\begin{thm}[任意抽出定理]
			
		\end{thm}
	\end{screen}
	
\subsection{The Optional Sampling Theorem}
	\begin{itembox}[l]{Lemma: 離散時間の任意抽出定理}\label{lem:optional_sampling_theorem}
		$0 = t_0 < t_1 < \cdots < t_n < \infty$とし,
		$\Set{X_{t_n},\mathscr{F}_{t_n}}{n=0,\cdots,n}$を劣マルチンゲール,
		$S,T:\Omega \longrightarrow \{t_0,t_1,\cdots,t_n,\infty\}$を$(\mathscr{F}_{t_n})$-停止時刻,
		$Y$を$\mathscr{F}/\borel{\R}$-可測関数として
		\begin{align}
			X_T(\omega) \coloneqq Y(\omega), \quad (\forall \omega \in \{T=\infty\}), \quad
			X_S(\omega) \coloneqq Y(\omega), \quad (\forall \omega \in \{S=\infty\})
		\end{align}
		とおく.このとき,
		\begin{description}
			\item[(a)] $S,T < \infty,\ \mbox{a.s. $P$}$.
			\item[(b)] $Y$が可積分かつ$\cexp{Y}{\mathscr{F}_{t_n}} \geq X_{t_n}\ \mbox{a.s. $P$},\ (n=0,\cdots,n)$.
		\end{description}
		のいずれかのが満たされていれば次が成り立つ:
		\begin{align}
			\cexp{X_T}{\mathscr{F}_S} \geq X_{S \wedge T}
			\quad \mbox{a.s. $P$}.
			\label{eq:lem_optional_sampling_theorem_1}
		\end{align}
	\end{itembox}
	
	\begin{prf}\mbox{}
		\begin{description}
			\item[第一段]
				$X_S$が$\mathscr{F}_S/\borel{\R}$-可測であることを示す.任意の$t \geq 0$に対して
				\begin{align}
					\{X_S \in B\} \cap \{S \leq t\} = \{X_{S \wedge t} \in B\} \cap \{S \leq t\},
					\quad (\forall B \in \borel{\R})
				\end{align}
				となるから$X_{S \wedge t}$の$\mathscr{F}_t/\borel{\R}$-可測性を言えばよい.
				$t_m \leq t < t_{m+1}$の場合($m=n$なら$t_{m+1} = \infty$),
				\begin{align}
					X_{S \wedge t} = \sum_{t_i \leq t} X_{t_i} \defunc_{\{S=t_i\}}
					= \sum_{i=0}^m X_{t_i} \defunc_{\{S=t_i\}}
				\end{align}
				と分解できる.
				連続写像$\varphi:\R^2 \ni (x,y) \longmapsto xy$と
				$\psi:\R^{m+1} \ni (x_0,x_1,\cdots,x_m) \longmapsto x_0+x_1+\cdots+x_m$を用いれば,
				\begin{align}
					\left\{ X_{t_i}\defunc_{\{S=t_i\}} \in B \right\}
					= \left\{ \left( X_{t_i}, \defunc_{\{S=t_i\}} \right) \in \varphi^{-1}(B) \right\},
					\quad (\forall B \in \borel{\R})
				\end{align}
				かつ
				\begin{align}
					\left\{ X_{S \wedge t} \in B \right\}
					= \left\{ \left( X_{t_0}\defunc_{\{S=t_0\}},\cdots,X_{t_m}\defunc_{\{S=t_m\}} \right) \in \psi^{-1}(B) \right\},
					\quad (\forall B \in \borel{\R})
				\end{align}
				が成り立つ.いま,$\R$の第二可算性より
				$\borel{\R^2} = \borel{\R} \otimes \borel{\R}$
				が満たされ,かつ任意の$E,F \in \borel{\R}$に対して
				\begin{align}
					\left\{ \left( X_{t_i}, \defunc_{\{S=t_i\}} \right) \in E \times F \right\}
					= X_{t_i}^{-1}(E) \cap \left\{ \defunc_{\{S=t_i\}} \in F \right\}
					\in \mathscr{F}_t,
					\quad (\forall t_i \leq t)
				\end{align}
				となるから$X_{t_i} \defunc_{\{S=t_i\}}$の$\mathscr{F}_t/\borel{\R}$-可測性が従う.
				同様に
				$\borel{\R^{m+1}} = \borel{\R} \otimes \cdots \otimes \borel{\R}$
				と
				\begin{align}
					\left\{ \left( X_{t_0}\defunc_{\{S=t_0\}},\cdots,X_{t_m}\defunc_{\{S=t_m\}} \right) \in E_0 \times \cdots \times E_n \right\}
					= \bigcap_{i=0}^m \left\{X_{t_i}\defunc_{\{S=t_i\}} \in E_i\right\}
					\in \mathscr{F}_t,
					\quad (\forall E_i \in \borel{\R},\ i=0,\cdots,m)
				\end{align}
				より$X_{S \wedge t}$は$\mathscr{F}_t/\borel{\R}$-可測である.
				これより$X_T$の$\mathscr{F}_T/\borel{\R}$-可測性及び$X_{S \wedge T}$の
				$\mathscr{F}_{S \wedge T}/\borel{\R}$-可測性も出る.
				
			\item[第二段]
				$S \leq T$と仮定して(\refeq{eq:lem_optional_sampling_theorem_1})を示す.先ず
				\begin{align}
					\int_\Omega |X_S|\ dP
					= \sum_{i=0}^n \int_{\{S=t_i\}} |X_{t_i}|\ dP
						+ \int_{\{S=\infty\}} |Y|\ dP
				\end{align}
				より(a),(b)いずれの場合も$X_S,X_T$は可積分である.
				また,劣マルチンゲール性より任意の$A \in \mathscr{F}_S$に対して
				\begin{align}
					\int_{A \cap \{S=t_i\}} X_{t_i}\ dP
					&= \int_{A \cap \{S=t_i\} \cap \{T=t_i\}} X_{t_i}\ dP
						+ \int_{A \cap \{S=t_i\} \cap \{T>t_i\}} X_{t_i}\ dP \\
					&\leq \int_{A \cap \{S=t_i\} \cap \{T=t_i\}} X_T\ dP
						+ \int_{A \cap \{S=t_i\} \cap \{T>t_i\}} X_{t_{i+1}}\ dP \\
					&= \int_{A \cap \{S=t_i\} \cap \{T=t_i\}} X_T\ dP
						+ \int_{A \cap \{S=t_i\} \cap \{T=t_{i+1}\}} X_T\ dP
						+ \int_{A \cap \{S=t_i\} \cap \{T>t_{i+1}\}} X_{t_{i+1}}\ dP \\
					&\cdots \\
					&\leq \sum_{j=i}^n \int_{A \cap \{S=t_i\} \cap \{T=t_j\}} X_T\ dP
						+ \int_{A \cap \{S=t_i\} \cap \{T>t_n\}} X_{t_n}\ dP
				\end{align}
				及び
				\begin{align}
					\int_{A \cap \{S=\infty\}} X_S\ dP
					= \int_{A \cap \{S=\infty\}} Y\ dP
					= \int_{A \cap \{S=\infty\}} X_T\ dP
				\end{align}
				が成り立つから,(a)の場合は
				\begin{align}
					\int_{A \cap \{S=t_i\}} X_{t_i}\ dP \leq
					\sum_{j=i}^n \int_{A \cap \{S=t_i\} \cap \{T=t_j\}} X_T\ dP
					= \int_{A \cap \{S=t_i\}} X_T\ dP,
				\end{align}
				(b)の場合は
				\begin{align}
					\int_{A \cap \{S=t_i\}} X_{t_i}\ dP
					&\leq \sum_{j=i}^n \int_{A \cap \{S=t_i\} \cap \{T=t_j\}} X_T\ dP
						+ \int_{A \cap \{S=t_i\} \cap \{T>t_n\}} X_{t_n}\ dP \\
					&\leq \sum_{j=i}^n \int_{A \cap \{S=t_i\} \cap \{T=t_j\}} X_T\ dP
						+ \int_{A \cap \{S=t_i\} \cap \{T>t_n\}} Y\ dP \\
					&= \sum_{j=i}^n \int_{A \cap \{S=t_i\} \cap \{T=t_j\}} X_T\ dP
						+ \int_{A \cap \{S=t_i\} \cap \{T=\infty\}} Y\ dP \\
					&= \int_{A \cap \{S=t_i\}} X_T\ dP
				\end{align}
				となり,いずれの場合も
				\begin{align}
					\int_A X_S\ dP
					= \sum_{i=0}^n \int_{A \cap \{S=t_i\}} X_{t_i}\ dP
						+ \int_{A \cap \{S=\infty\}} X_S\ dP
					\leq \sum_{i=0}^n \int_{A \cap \{S=t_i\}} X_T\ dP + \int_{A \cap \{S=\infty\}} X_T\ dP
					= \int_A X_T\ dP
				\end{align}
				が成立する.$X_S$の$\mathscr{F}_S/\borel{\R}$-可測性より(\refeq{eq:lem_optional_sampling_theorem_1})を得る.
			
			\item[第三段]
				一般の$S,T$に対して(\refeq{eq:lem_optional_sampling_theorem_1})を示す.
				任意の$A \in \mathscr{F}_S$に対し,Problem 2.17 (P. \pageref{chapter_1_Problem_2_17})
				と前段の結果より
				\begin{align}
					\int_A \cexp{X_T}{\mathscr{F}_S}\ dP
					&= \int_{A \cap \{S \leq T\}} \cexp{X_T}{\mathscr{F}_S}\ dP
						+ \int_{A \cap \{S > T\}} \cexp{X_T}{\mathscr{F}_S}\ dP \\
					&= \int_{A \cap \{S \leq T\}} \cexp{X_T}{\mathscr{F}_{S \wedge T}}\ dP
						+ \int_{A \cap \{S > T\}} X_T\ dP \\
					&\geq \int_{A \cap \{S \leq T\}} X_{S \wedge T}\ dP
					 	+ \int_{A \cap \{S > T\}} X_{S \wedge T}\ dP \\
					&= \int_A X_{S \wedge T}\ dP
				\end{align}
				となる.$X_{S \wedge T}$の$\mathscr{F}_{S \wedge T}/\borel{\R}$-可測性より
				(\refeq{eq:lem_optional_sampling_theorem_1})が出る.
				\QED
		\end{description}
	\end{prf}
	
	\begin{itembox}[l]{Theorem 3.22 修正}
		Let $\Set{X_t,\mathscr{F}_t}{0 \leq t \leq \infty}$ be a right-continuous submartingale with a last element $X_\infty$,
		and let $S \leq T$ be two optional times of the filtration $\{\mathscr{F}_t\}$. We have
		\begin{align}
			\cexp{X_T}{\mathscr{F}_{S+}} \geq X_S
			\quad \mbox{a.s. $P$}.
		\end{align}
		If $S$ is a stopping time, then $\mathscr{F}_S$ can replace $\mathscr{F}_{S+}$ above.
		In particular, $EX_T \geq EX_0$, and for a martingale with a last element we have $EX_T = EX_0$.
	\end{itembox}
	
	任意の停止時刻或は弱停止時刻$U$に対し,$U(\omega) = \infty$のとき$X_U(\omega) = X_\infty(\omega)$と考える.
	
	\begin{prf}\mbox{}
		\begin{description}
			\item[第一段]
				$X_S$の$\mathscr{F}_{S+}/\borel{\R}$-可測性を示す.
				Corollary 2.4 より$S$は$(\mathscr{F}_{t+})$-停止時刻であり,
				$\Set{X_t,\mathscr{F}_{t+}}{0 \leq t < \infty}$は
				発展的可測である.従って
				Proposition 2.18 (P. \pageref{chapter_1_Problem_2_18})より
				任意の$t \geq 0$に対し$X_{S \wedge t}$は$\mathscr{F}_{t+}/\borel{\R}$-可測であり,
				\begin{align}
					\{X_S \in B\} \cap \{S \leq t\}
					= \{X_{S \wedge t} \in B\} \cap \{S \leq t\}
					\in \mathscr{F}_{t+},
					\quad (\forall B \in \borel{\R})
				\end{align}
				より$X_S$の$\mathscr{F}_{S+}/\borel{\R}$-可測性が出る.
				$S$が$(\mathscr{F}_t)$-停止時刻のときは,
				$X_{S \wedge t}$は$\mathscr{F}_t/\borel{\R}$-可測性を持ち
				\begin{align}
					\{X_S \in B\} \cap \{S \leq t\}
					= \{X_{S \wedge t} \in B\} \cap \{S \leq t\}
					\in \mathscr{F}_t,
					\quad (\forall B \in \borel{\R})
				\end{align}
				が従うから$X_S$は$\mathscr{F}_S/\borel{\R}$-可測である.
			
			\item[第二段]
				任意の$n \geq 1$に対し
				\begin{align}
					S_n(\omega) \coloneqq
					\begin{cases}
						\infty & \mbox{if $S(\omega) \geq n$}, \\
						\displaystyle\frac{k}{2^n} & \mbox{if $\displaystyle\frac{k-1}{2^n} \leq S(\omega) < \frac{k}{2^n}$ for $k=1,\cdots,n2^n$},
					\end{cases}
				\end{align}
				により停止時刻列$(S_n)_{n=1}^\infty$が定まる(Problem 2.24 修正版, P. \pageref{chapter_1_Problem_2_24}).
				同様に$(T_n)_{n=1}^\infty$も構成すれば,補題より
				\begin{align}
					\int_A X_{T_n}\ dP \geq \int_A X_{S_n \wedge T_n}\ dP,
					\quad (\forall A \in \mathscr{F}_{S_n},\ \forall n \geq 1)
				\end{align}
				が成立する.
		\end{description}
	\end{prf}
	
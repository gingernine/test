\subsection{The Optional Sampling Theorem}
	\begin{itembox}[l]{Lemma: 離散時間の任意抽出定理}
		$0 = t_0 < t_1 < \cdots < t_n < \infty$とし,
		$\Set{X_{t_n},\mathscr{F}_{t_n}}{n=0,\cdots,n}$を劣マルチンゲール,
		$S,T:\Omega \longrightarrow \{t_0,t_1,\cdots,t_n,\infty\}$を$(\mathscr{F}_{t_n})$-停止時刻とする.
		また或る$\mathscr{F}/\borel{\R}$-可測関数$Y$が存在して
		\begin{align}
			X_T(\omega) \coloneqq Y(\omega)\ (\forall \omega \in \{T=\infty\}),
			\quad X_S(\omega) \coloneqq Y(\omega)\ (\forall \omega \in \{S=\infty\})
		\end{align}
		を満たしているとする.このとき,
		\begin{description}
			\item[(a)] $S,T < \infty$.
			\item[(b)] $Y$が可積分で$\cexp{Y}{\mathscr{F}_{t_n}} \geq X_{t_n}\ \mbox{a.s. $P$},\ (n=0,\cdots,n)$を満たす.
		\end{description}
		のいずれかの場合次が成り立つ:
		\begin{align}
			\cexp{X_T}{\mathscr{F}_S} \geq X_{S \wedge T}
			\quad \mbox{a.s. $P$}.
			\label{eq:lem_optional_sampling_theorem_1}
		\end{align}
	\end{itembox}
	
	\begin{prf}\mbox{}
		\begin{description}
			\item[第一段]	
				$S \leq T$と仮定して(\refeq{eq:lem_optional_sampling_theorem_1})を示す.先ず
				\begin{align}
					\int_\Omega |X_S|\ dP
					= \sum_{i=0}^n \int_{\{S=t_i\}} |X_{t_i}|\ dP
						+ \int_{\{S=\infty\}} |Y|\ dP
				\end{align}
				より(a),(b)いずれの場合も$X_S,X_T$は可積分である.
				また,劣マルチンゲール性より任意の$A \in \mathscr{F}_S$に対して
				\begin{align}
					\int_{A \cap \{S=t_i\}} X_{t_i}\ dP
					&= \int_{A \cap \{S=t_i\} \cap \{T=t_i\}} X_{t_i}\ dP
						+ \int_{A \cap \{S=t_i\} \cap \{T>t_i\}} X_{t_i}\ dP \\
					&\leq \int_{A \cap \{S=t_i\} \cap \{T=t_i\}} X_T\ dP
						+ \int_{A \cap \{S=t_i\} \cap \{T>t_i\}} X_{t_{i+1}}\ dP \\
					&= \int_{A \cap \{S=t_i\} \cap \{T=t_i\}} X_T\ dP
						+ \int_{A \cap \{S=t_i\} \cap \{T=t_{i+1}\}} X_T\ dP
						+ \int_{A \cap \{S=t_i\} \cap \{T>t_{i+1}\}} X_{t_{i+1}}\ dP \\
					&\cdots \\
					&\leq \sum_{j=i}^n \int_{A \cap \{S=t_i\} \cap \{T=t_j\}} X_T\ dP
						+ \int_{A \cap \{S=t_i\} \cap \{T>t_n\}} X_{t_n}\ dP
				\end{align}
				及び
				\begin{align}
					\int_{A \cap \{S=\infty\}} X_S\ dP
					= \int_{A \cap \{S=\infty\}} Y\ dP
					= \int_{A \cap \{S=\infty\}} X_T\ dP
				\end{align}
				が成り立つから,(a)の場合は
				\begin{align}
					\int_{A \cap \{S=t_i\}} X_{t_i}\ dP \leq
					\sum_{j=i}^n \int_{A \cap \{S=t_i\} \cap \{T=t_j\}} X_T\ dP
					= \int_{A \cap \{S=t_i\}} X_T\ dP,
				\end{align}
				(b)の場合は
				\begin{align}
					\int_{A \cap \{S=t_i\}} X_{t_i}\ dP
					&\leq \sum_{j=i}^n \int_{A \cap \{S=t_i\} \cap \{T=t_j\}} X_T\ dP
						+ \int_{A \cap \{S=t_i\} \cap \{T>t_n\}} X_{t_n}\ dP \\
					&\leq \sum_{j=i}^n \int_{A \cap \{S=t_i\} \cap \{T=t_j\}} X_T\ dP
						+ \int_{A \cap \{S=t_i\} \cap \{T>t_n\}} Y\ dP \\
					&= \int_{A \cap \{S=t_i\}} X_T\ dP
				\end{align}
				となり,いずれの場合も
				\begin{align}
					\int_A X_S\ dP
					= \sum_{i=0}^n \int_{A \cap \{S=t_i\}} X_{t_i}\ dP
						+ \int_{A \cap \{S=\infty\}} X_S\ dP
					\leq \sum_{i=0}^n \int_{A \cap \{S=t_i\}} X_T\ dP + \int_{A \cap \{S=\infty\}} X_T\ dP
					= \int_A X_T\ dP
				\end{align}
				が成立する.
			
			\item[第二段]
				一般の$S,T$に対して(\refeq{eq:lem_optional_sampling_theorem_1})を示す.
				任意の$A \in \mathscr{F}_S$に対し,Problem 2.17 (P. \pageref{chapter_1_Problem_2_17})
				と前段の結果より
				\begin{align}
					\int_A \cexp{X_T}{\mathscr{F}_S}\ dP
					&= \int_{A \cap \{S \leq T\}} \cexp{X_T}{\mathscr{F}_S}\ dP
						+ \int_{A \cap \{S > T\}} \cexp{X_T}{\mathscr{F}_S}\ dP \\
					&= \int_{A \cap \{S \leq T\}} \cexp{X_T}{\mathscr{F}_{S \wedge T}}\ dP
						+ \int_{A \cap \{S > T\}} X_T\ dP \\
					&\geq \int_{A \cap \{S \leq T\}} X_{S \wedge T}\ dP
					 	+ \int_{A \cap \{S > T\}} X_{S \wedge T}\ dP \\
					&= \int_A X_{S \wedge T}\ dP
				\end{align}
				となる.
		\end{description}
	\end{prf}
\subsection{The Optional Sampling Theorem}
	\begin{itembox}[l]{Lemma: 離散時間の任意抽出定理}\label{lem:optional_sampling_theorem}
		$0 = t_0 < t_1 < \cdots < t_n < \infty$とし,
		$\Set{X_{t_i},\mathscr{F}_{t_i}}{i=0,\cdots,n}$を劣マルチンゲール,
		$S,T:\Omega \longrightarrow \{t_0,t_1,\cdots,t_n,\infty\}$を$(\mathscr{F}_{t_n})$-停止時刻,
		$Y$を$\mathscr{F}/\borel{\R}$-可測関数として
		\begin{align}
			X_T(\omega) \coloneqq Y(\omega), \quad (\forall \omega \in \{T=\infty\}), \quad
			X_S(\omega) \coloneqq Y(\omega), \quad (\forall \omega \in \{S=\infty\})
		\end{align}
		とおく.このとき,
		\begin{description}
			\item[(a)] $S,T < \infty,\ \mbox{a.s. $P$}$.
			\item[(b)] $Y$が可積分かつ$\cexp{Y}{\mathscr{F}_{t_n}} \geq X_{t_n}\ \mbox{a.s. $P$},\ (n=0,\cdots,n)$.
		\end{description}
		のいずれかのが満たされていれば次が成り立つ:
		\begin{align}
			\cexp{X_T}{\mathscr{F}_S} \geq X_{S \wedge T}
			\quad \mbox{a.s. $P$}.
			\label{eq:lem_optional_sampling_theorem_1}
		\end{align}
	\end{itembox}
	
	\begin{prf}\mbox{}
		\begin{description}
			\item[第一段]
				$X_S$が$\mathscr{F}_S/\borel{\R}$-可測であることを示す.任意の$t \geq 0$に対して
				\begin{align}
					\{X_S \in B\} \cap \{S \leq t\} = \{X_{S \wedge t} \in B\} \cap \{S \leq t\},
					\quad (\forall B \in \borel{\R})
				\end{align}
				となるから$X_{S \wedge t}$の$\mathscr{F}_t/\borel{\R}$-可測性を言えばよい.
				$t_m \leq t < t_{m+1}$の場合($m=n$なら$t_{m+1} = \infty$),
				\begin{align}
					X_{S \wedge t} = \sum_{t_i \leq t} X_{t_i} \defunc_{\{S=t_i\}}
					= \sum_{i=0}^m X_{t_i} \defunc_{\{S=t_i\}}
				\end{align}
				と分解できる.
				連続写像$\varphi:\R^2 \ni (x,y) \longmapsto xy$と
				$\psi:\R^{m+1} \ni (x_0,x_1,\cdots,x_m) \longmapsto x_0+x_1+\cdots+x_m$を用いれば,
				\begin{align}
					\left\{ X_{t_i}\defunc_{\{S=t_i\}} \in B \right\}
					= \left\{ \left( X_{t_i}, \defunc_{\{S=t_i\}} \right) \in \varphi^{-1}(B) \right\},
					\quad (\forall B \in \borel{\R})
				\end{align}
				かつ
				\begin{align}
					\left\{ X_{S \wedge t} \in B \right\}
					= \left\{ \left( X_{t_0}\defunc_{\{S=t_0\}},\cdots,X_{t_m}\defunc_{\{S=t_m\}} \right) \in \psi^{-1}(B) \right\},
					\quad (\forall B \in \borel{\R})
				\end{align}
				が成り立つ.いま,$\R$の第二可算性より
				$\borel{\R^2} = \borel{\R} \otimes \borel{\R}$
				が満たされ,かつ任意の$E,F \in \borel{\R}$に対して
				\begin{align}
					\left\{ \left( X_{t_i}, \defunc_{\{S=t_i\}} \right) \in E \times F \right\}
					= X_{t_i}^{-1}(E) \cap \left\{ \defunc_{\{S=t_i\}} \in F \right\}
					\in \mathscr{F}_t,
					\quad (\forall t_i \leq t)
				\end{align}
				となるから$X_{t_i} \defunc_{\{S=t_i\}}$の$\mathscr{F}_t/\borel{\R}$-可測性が従う.
				同様に
				$\borel{\R^{m+1}} = \borel{\R} \otimes \cdots \otimes \borel{\R}$
				と
				\begin{align}
					\left\{ \left( X_{t_0}\defunc_{\{S=t_0\}},\cdots,X_{t_m}\defunc_{\{S=t_m\}} \right) \in E_0 \times \cdots \times E_m \right\}
					= \bigcap_{i=0}^m \left\{X_{t_i}\defunc_{\{S=t_i\}} \in E_i\right\}
					\in \mathscr{F}_t,
					\quad (\forall E_i \in \borel{\R},\ i=0,\cdots,m)
				\end{align}
				より$X_{S \wedge t}$は$\mathscr{F}_t/\borel{\R}$-可測である.
				これより$X_T$の$\mathscr{F}_T/\borel{\R}$-可測性及び$X_{S \wedge T}$の
				$\mathscr{F}_{S \wedge T}/\borel{\R}$-可測性も出る.
				
			\item[第二段]
				$S \leq T$と仮定して(\refeq{eq:lem_optional_sampling_theorem_1})を示す.先ず
				\begin{align}
					\int_\Omega |X_S|\ dP
					= \sum_{i=0}^n \int_{\{S=t_i\}} |X_{t_i}|\ dP
						+ \int_{\{S=\infty\}} |Y|\ dP
				\end{align}
				より(a),(b)いずれの場合も$X_S,X_T$は可積分である.
				また,劣マルチンゲール性より任意の$A \in \mathscr{F}_S$に対して
				\begin{align}
					\int_{A \cap \{S=t_i\}} X_{t_i}\ dP
					&= \int_{A \cap \{S=t_i\} \cap \{T=t_i\}} X_{t_i}\ dP
						+ \int_{A \cap \{S=t_i\} \cap \{T>t_i\}} X_{t_i}\ dP \\
					&\leq \int_{A \cap \{S=t_i\} \cap \{T=t_i\}} X_T\ dP
						+ \int_{A \cap \{S=t_i\} \cap \{T>t_i\}} X_{t_{i+1}}\ dP \\
					&= \int_{A \cap \{S=t_i\} \cap \{T=t_i\}} X_T\ dP
						+ \int_{A \cap \{S=t_i\} \cap \{T=t_{i+1}\}} X_T\ dP
						+ \int_{A \cap \{S=t_i\} \cap \{T>t_{i+1}\}} X_{t_{i+1}}\ dP \\
					&\cdots \\
					&\leq \sum_{j=i}^n \int_{A \cap \{S=t_i\} \cap \{T=t_j\}} X_T\ dP
						+ \int_{A \cap \{S=t_i\} \cap \{T>t_n\}} X_{t_n}\ dP
				\end{align}
				及び
				\begin{align}
					\int_{A \cap \{S=\infty\}} X_S\ dP
					= \int_{A \cap \{S=\infty\}} Y\ dP
					= \int_{A \cap \{S=\infty\}} X_T\ dP
				\end{align}
				が成り立つから,(a)の場合は
				\begin{align}
					\int_{A \cap \{S=t_i\}} X_{t_i}\ dP \leq
					\sum_{j=i}^n \int_{A \cap \{S=t_i\} \cap \{T=t_j\}} X_T\ dP
					= \int_{A \cap \{S=t_i\}} X_T\ dP,
				\end{align}
				(b)の場合は
				\begin{align}
					\int_{A \cap \{S=t_i\}} X_{t_i}\ dP
					&\leq \sum_{j=i}^n \int_{A \cap \{S=t_i\} \cap \{T=t_j\}} X_T\ dP
						+ \int_{A \cap \{S=t_i\} \cap \{T>t_n\}} X_{t_n}\ dP \\
					&\leq \sum_{j=i}^n \int_{A \cap \{S=t_i\} \cap \{T=t_j\}} X_T\ dP
						+ \int_{A \cap \{S=t_i\} \cap \{T>t_n\}} Y\ dP \\
					&= \sum_{j=i}^n \int_{A \cap \{S=t_i\} \cap \{T=t_j\}} X_T\ dP
						+ \int_{A \cap \{S=t_i\} \cap \{T=\infty\}} Y\ dP \\
					&= \int_{A \cap \{S=t_i\}} X_T\ dP
				\end{align}
				となり,いずれの場合も
				\begin{align}
					\int_A X_S\ dP
					= \sum_{i=0}^n \int_{A \cap \{S=t_i\}} X_{t_i}\ dP
						+ \int_{A \cap \{S=\infty\}} X_S\ dP
					\leq \sum_{i=0}^n \int_{A \cap \{S=t_i\}} X_T\ dP + \int_{A \cap \{S=\infty\}} X_T\ dP
					= \int_A X_T\ dP
				\end{align}
				が成立する.$X_S$の$\mathscr{F}_S/\borel{\R}$-可測性より(\refeq{eq:lem_optional_sampling_theorem_1})を得る.
			
			\item[第三段]
				一般の$S,T$に対して(\refeq{eq:lem_optional_sampling_theorem_1})を示す.
				任意の$A \in \mathscr{F}_S$に対し,Problem 2.17 (P. \pageref{chapter_1_Problem_2_17})
				と前段の結果より
				\begin{align}
					\int_A \cexp{X_T}{\mathscr{F}_S}\ dP
					&= \int_{A \cap \{S \leq T\}} \cexp{X_T}{\mathscr{F}_S}\ dP
						+ \int_{A \cap \{S > T\}} \cexp{X_T}{\mathscr{F}_S}\ dP \\
					&= \int_{A \cap \{S \leq T\}} \cexp{X_T}{\mathscr{F}_{S \wedge T}}\ dP
						+ \int_{A \cap \{S > T\}} X_T\ dP \\
					&\geq \int_{A \cap \{S \leq T\}} X_{S \wedge T}\ dP
					 	+ \int_{A \cap \{S > T\}} X_{S \wedge T}\ dP \\
					&= \int_A X_{S \wedge T}\ dP
				\end{align}
				となり,$X_{S \wedge T}$の$\mathscr{F}_{S \wedge T}/\borel{\R}$-可測性より
				(\refeq{eq:lem_optional_sampling_theorem_1})が出る.
				\QED
		\end{description}
	\end{prf}
	
	\begin{itembox}[l]{Theorem 3.22 修正}
		Let $\Set{X_t,\mathscr{F}_t}{0 \leq t < \infty}$ be a right-continuous submartingale,
		$S, T$ be two optional times of the filtration $\{\mathscr{F}_t\}$,
		and $Y$ be a $\mathscr{F}/\borel{\R}$-measurable function, and set
		\begin{align}
			X_U(\omega) \coloneqq Y(\omega)
			\quad (\forall \omega \in \{U = \infty\}).
		\end{align}
		for any random time $U$. Then, under either of the following two conditions;
		\begin{description}
			\item[(a)] There exists an $N \in \N$ such that $S,T < N\ \mbox{a.s. $P$}$,
			\item[(b)] $Y$ is integrable and $X_t \leq \cexp{Y}{\mathscr{F}_t}\ \mbox{a.s. $P$}$, for every $t \geq 0$,
		\end{description}
		we have
		\begin{align}
			\cexp{X_T}{\mathscr{F}_{S+}} \geq X_{S \wedge T}
			\quad \mbox{a.s. $P$}.
		\end{align}
		If $S$ is a stopping time, then $\mathscr{F}_S$ can replace $\mathscr{F}_{S+}$ above.
		In particular, $EX_T \geq EX_0$, and for a martingale with a last element we have $EX_T = EX_0$.
	\end{itembox}
	この修正によりProblem 3.23 と Problem 3.24の主張が従う.
	\begin{prf}\mbox{}
		\begin{description}
			\item[第一段]
				$X_S$の$\mathscr{F}_{S+}/\borel{\R}$-可測性を示す.
				Corollary 2.4 より$S$は$(\mathscr{F}_{t+})$-停止時刻であり,
				$\Set{X_t,\mathscr{F}_{t+}}{0 \leq t < \infty}$は
				発展的可測である.従って
				Proposition 2.18 (P. \pageref{chapter_1_Problem_2_18})より
				任意の$t \geq 0$に対し$X_{S \wedge t}$は$\mathscr{F}_{t+}/\borel{\R}$-可測であり,
				\begin{align}
					\{X_S \in B\} \cap \{S \leq t\}
					= \{X_{S \wedge t} \in B\} \cap \{S \leq t\}
					\in \mathscr{F}_{t+},
					\quad (\forall B \in \borel{\R})
				\end{align}
				より$X_S$の$\mathscr{F}_{S+}/\borel{\R}$-可測性が出る.
				$S$が$(\mathscr{F}_t)$-停止時刻のときは,
				$X_{S \wedge t}$は$\mathscr{F}_t/\borel{\R}$-可測性を持ち
				\begin{align}
					\{X_S \in B\} \cap \{S \leq t\}
					= \{X_{S \wedge t} \in B\} \cap \{S \leq t\}
					\in \mathscr{F}_t,
					\quad (\forall B \in \borel{\R})
				\end{align}
				が従うから$X_S$は$\mathscr{F}_S/\borel{\R}$-可測である.
			
			\item[第二段]
				任意の$n \geq N$に対し
				\begin{align}
					S_n(\omega) \coloneqq
					\begin{cases}
						\infty & \mbox{if $S(\omega) \geq n$}, \\
						\displaystyle\frac{k}{2^n} & \mbox{if $\displaystyle\frac{k-1}{2^n} \leq S(\omega) < \frac{k}{2^n}$ for $k=1,\cdots,n2^n$},
					\end{cases}
				\end{align}
				により停止時刻$S_n$が定まる(Problem 2.24 修正版, P. \pageref{chapter_1_Problem_2_24}).
				同様に$(T_n)_{n \geq N}$も構成すれば,補題より
				\begin{align}
					\int_A X_{T_n}\ dP \geq \int_A X_{S_n \wedge T_n}\ dP,
					\quad (\forall A \in \mathscr{F}_{S_n},\ \forall n \geq N)
				\end{align}
				が成立する.また$S(\omega) < \infty$なら$S_n(\omega) \downarrow S(\omega)$,
				かつ$S(\omega) = \infty$なら$S_n(\omega) = \infty$であるから
				\begin{align}
					S = \inf{n \geq N}{S_n}
				\end{align}
				が満たされ,Problem 2.23 より
				\begin{align}
					\mathscr{F}_{S+} = \bigcap_{n \geq N} \mathscr{F}_{S_n}
				\end{align}
				となり
				\begin{align}
					\int_A X_{T_n}\ dP \geq \int_A X_{S_n \wedge T_n}\ dP,
					\quad (\forall A \in \mathscr{F}_{S+},\ \forall n \geq N)
					\label{eq:chapter_1_Theorem_3_22_1}
				\end{align}
				が成立する.$S$が停止時刻の場合は
				$\mathscr{F}_S \subset \mathscr{F}_{S+}$
				であるから,(\refeq{eq:chapter_1_Theorem_3_22_1})を
				$\mathscr{F}_S$に置き換えて成立する.
			
			\item[第三段]
				$(S_n)_{n \geq N},(T_n)_{n \geq N}$は単調減少列であるから
				$\left( \mathscr{F}_{T_n} \right)_{n \geq N}$と
				$\left( \mathscr{F}_{S_n \wedge T_n} \right)_{n \geq N}$も単調減少列であり,
				$\Set{X_{T_n},\mathscr{F}_{T_n}}{n \geq N}$及び
				$\Set{X_{S_n \wedge T_n},\mathscr{F}_{S_n \wedge T_n}}{n \geq N}$
				は後退劣マルチンゲールとなる.かつ
				\begin{align}
					\quad EX_{T_n} \downarrow EX_0,
					\quad EX_{S_n \wedge T_n} \downarrow EX_0
				\end{align}
				が満たされているから,Problem 3.11 より
				$\left( X_{T_n} \right)_{n \geq N},\left( X_{S_n \wedge T_n} \right)_{n \geq N}$
				は一様可積分である.また$\{X_t\}$の右連続性より
				\begin{align}
					X_{T_n}(\omega) \longrightarrow X_T(\omega),
					\quad X_{S_n \wedge T_n}(\omega) \longrightarrow X_{S \wedge T}(\omega),
					\quad (\forall \omega \in \Omega)
				\end{align}
				が成り立つから,一様可積分性と平均収束の補題(P. \pageref{lem:uniformly_integrable_and_convergence_in_mean})より
				$X_T,X_{S \wedge T}$の可積分性及び
				\begin{align}
					E\left| X_T - X_{T_n} \right| \longrightarrow 0,
					\quad E\left| X_{S \wedge T} - X_{S_n \wedge T_n} \right| \longrightarrow 0,
					\quad (n \longrightarrow \infty)
				\end{align}
				が従い
				\begin{align}
					\int_A X_T\ dP \geq \int_A X_{S \wedge T}\ dP,
					\quad (\forall A \in \mathscr{F}_{S+})
				\end{align}
				が得られる.$S$が停止時刻の場合は$\mathscr{F}_{S+}$を
				$\mathscr{F}_S$に置き換えて成立する.
				\QED
		\end{description}
	\end{prf}
	
	\begin{itembox}[l]{Problem 3.25}
		A submartingale of constant expectation, i.e., with $E(X_t) = E(X_0)$ for every $t \geq 0$, is a martingale. 
	\end{itembox}
	
	\begin{prf}
		任意の$0 \leq s < t$に対し,
		\begin{align}
			\cexp{X_t}{\mathscr{F}_s} - X_s \geq 0,
			\quad \mbox{a.s. $P$}
		\end{align}
		かつ
		\begin{align}
			E\left( \cexp{X_t}{\mathscr{F}_s} - X_s \right)
			= EX_t - EX_s
			= EX_0 - EX_0
			= 0
		\end{align}
		より
		\begin{align}
			\cexp{X_t}{\mathscr{F}_s} - X_s = 0,
			\quad \mbox{a.s. $P$}
		\end{align}
		が従う.
		\QED
	\end{prf}
	
	\begin{itembox}[l]{Problem 3.26}
		A right-continuous process $\Set{X_t,\mathscr{F}_t}{0 \leq t < \infty}$ with $E|X_t| < \infty;\ 0 \leq t < \infty$
		is a submartingale if and only if for every pair $S \leq T$ of bounded stopping times of 
		the filtration $\{\mathscr{F}_t\}$ we have
		\begin{align}
			E(X_T) \geq E(X_S).
			\label{chapter_1_Problem_3_26_1}
		\end{align}
	\end{itembox}
	
	\begin{prf}
		$(\Rightarrow)$は任意抽出定理より従う.$(\Leftarrow)$を示す.
		任意の$0 \leq s < t$及び$A \in \mathscr{F}_s$に対し,
		\begin{align}
			T(\omega) \coloneqq t,
			\quad 
			S(\omega) \coloneqq
			\begin{cases}
				s, & (\omega \in A), \\
				t, & (\omega \in \Omega \backslash A)
			\end{cases},
			\quad (\forall \omega \in \Omega)
		\end{align}
		により$(\mathscr{F}_t)$-停止時刻$S \leq T$を定めれば,(\refeq{chapter_1_Problem_3_26_1})より
		\begin{align}
			\int_A X_t\ dP = \int_\Omega X_T\ dP - \int_{\Omega \backslash A} X_t\ dP
			\geq \int_\Omega X_S\ dP - \int_{\Omega \backslash A} X_t\ dP
			= \int_A X_s\ dP
		\end{align}
		が成り立ち,$A \in \mathscr{F}_s$の任意性より$\cexp{X_t}{\mathscr{F}_s} \geq X_s\ \mbox{a.s. $P$}$となる.
		\QED
	\end{prf}
	
	\begin{itembox}[l]{Problem 3.27}
		Let $T$ be a bounded stopping time of the filtration $\{\mathscr{F}_t\}$, which satisfies
		the usual conditions, and define $\tilde{\mathscr{F}}_t = \mathscr{F}_{T+t};\ t \geq 0$.
		Then $\{\tilde{\mathscr{F}_t}\}$ also satisfies the usual conditions.
		\begin{description}
			\item[(i)] If $X = \Set{X_t,\mathscr{F}_t}{0 \leq t < \infty}$ is a right-continuous submartingale,
				then so is $\tilde{X} = \Set{\tilde{X}_t \coloneqq X_{T+t} - X_T,\tilde{\mathscr{F}}_t}{0 \leq t < \infty}$.
			\item[(ii)] $\tilde{X} = \Set{\tilde{X}_t, \tilde{\mathscr{F}}_t}{0 \leq t < \infty}$ is a right-continuous submartingale,
				with $\tilde{X}_0 = 0,\ \mbox{a.s. $P$}$, then
				$X = \Set{X_t \coloneqq \tilde{X}_{(t-T) \vee 0},\mathscr{F}_t}{0 \leq t < \infty}$ is also a submartingale.
		\end{description}
	\end{itembox}
	
	\begin{prf}\mbox{}
		\begin{description}
			\item[第一段]
				$\{\tilde{\mathscr{F}}_t\}$が通常の条件(usual conditions)を満たすことを示す.
				実際,
				\begin{align}
					\Set{N \in \mathscr{F}}{P(N)=0} 
					\subset \mathscr{F}_0 \subset \mathscr{F}_{T+t},
					\quad (\forall t \geq 0)
				\end{align}
				より$\{\tilde{\mathscr{F}}_t\}$は完備であり,また
				任意の$t \geq 0$に対して$T+t = \inf{n \geq 1}{(T+t+1/n)}$より
				\begin{align}
					\tilde{\mathscr{F}}_{t+}
					= \bigcap_{n=1}^\infty \tilde{\mathscr{F}}_{t+\frac{1}{n}}
					= \bigcap_{n=1}^\infty \mathscr{F}_{T+t+\frac{1}{n}}
					= \mathscr{F}_{(T+t)+}
				\end{align}
				となるが(Problem 2.23),$\{\mathscr{F}_t\}$の右連続性より
				\begin{align}
					A \cap \{T \leq t\} \in \mathscr{F}_{t+},
					\ \forall t \geq 0
					\quad \Leftrightarrow \quad
					A \cap \{T \leq t\} \in \mathscr{F}_t,
					\ \forall t \geq 0
				\end{align}
				が成立するから$\mathscr{F}_{(T+t)+} = \mathscr{F}_{T+t}$が満たされ
				\begin{align}
					\tilde{\mathscr{F}}_{t+} = \mathscr{F}_{T+t} = \tilde{\mathscr{F}}_{t}
				\end{align}
				を得る.
				
			\item[第二段]
				$X$の右連続性より$\tilde{X}$は右連続である.また
				任意抽出定理より$X_{T+t}$は$\mathscr{F}_{T+t}/\borel{\R}$-可測かつ可積分であり
				\begin{align}
					\cexp{\tilde{X}_t}{\tilde{\mathscr{F}}_s}
					= \cexp{X_{T+t} - X_T}{\mathscr{F}_{T+s}}
					\geq X_{T+s} - X_T
					= \tilde{X}_s,
					\quad \mbox{a.s. $P$},
					\ (0 \leq s < t)
				\end{align}
				が成立するから,$\tilde{X}$は右連続劣マルチンゲールである.
				
			\item[第三段]
				$S_1 \leq S_2$を有界な$(\mathscr{F}_t)$-停止時刻とすれば
				$S_j - T\ (j=1,2)$は$(\tilde{\mathscr{F}}_t)$-停止時刻である.実際,
				\begin{align}
					\{S_j - T \leq t\} \cap \{T+t \leq u\}
					= \{S_j \wedge u \leq (T+t) \wedge u\} \cap \{T+t \leq u\}
					\in \mathscr{F}_u,
					\quad (\forall u \geq 0)
				\end{align}
				より$\{S_j - T \leq t\} \in \mathscr{F}_{T+t} = \tilde{\mathscr{F}}_t\ (\forall t \geq 0)$が成立する.
				Problem 3.26より
				\begin{align}
					EX_{S_1} 
					= E \tilde{X}_{(S_1 - T) \vee 0} 
					\leq E \tilde{X}_{(S_2 - T) \vee 0}
					= EX_{S_2}
				\end{align}
				が満たされ,同じくProblem 3.26より$X$の劣マルチンゲール性が従う.
				\QED
		\end{description}
	\end{prf}
	
	\begin{itembox}[l]{Problem 3.28}
		Let $Z = \Set{Z_t,\mathscr{F}_t}{0 \leq t < \infty}$ be a continuous, nonnegative martingale with
		$Z_\infty \coloneqq \lim_{t \to \infty} Z_t = 0,\ \mbox{a.s. $P$}$. Then for every $s \geq 0,\ b > 0$:
		\begin{description}
			\item[(i)] $\displaystyle P\left[ \sup{t>s}{Z_t \geq b}\ \middle|\ \mathscr{F}_s \right] = \frac{1}{b} Z_s,
				\quad \mbox{a.s. on $\{Z_s < b\}$}$.
			\item[(ii)] $\displaystyle P\left[ \sup{t \geq s}{Z_t \geq b} \right] = P[Z_s \geq b] + \frac{1}{b} E[Z_s \defunc_{\{Z_s < b\}}]$.
		\end{description}
	\end{itembox}
	
	\begin{prf}\mbox{}
		\begin{description}
			\item[第一段] $\inf{}{\Set{t \in [s,\infty)}{Z_t(\omega) = b}} = \inf{}{\Set{t \in [0,\infty)}{Z_{t+s}(\omega) = b}} + s$と
				Problem 2.7より
				\begin{align}
					T(\omega) \coloneqq \inf{}{\Set{t \in [s,\infty)}{Z_t(\omega) = b}},
					\quad (\forall \omega \in \Omega)
				\end{align}
				により$(\mathscr{F}_t)$-停止時刻が定まる.このとき
				\begin{align}
					Z_T(\omega) = b,
					\quad (\forall \omega \in \{T < \infty\} \cap \{Z_s < b\})
					\label{eq:chapter_1_Problem_3_28_1}
				\end{align}
				と
				\begin{align}
					T(\omega) < \infty
					\quad \Leftrightarrow \quad
					\sup{t > s}{Z_t(\omega)} \geq b,
					\quad (\mbox{a.s.}\omega \in \{Z_s < b\})
					\label{eq:chapter_1_Problem_3_28_2}
				\end{align}
				が成立する.実際,$\omega \in \{T < \infty\} \cap \{Z_s < b\}$に対し,
				$Z_T(\omega) < b$なら
				\begin{align}
					\sup{s \leq t \leq T(\omega)}{Z_t(\omega)} < b
				\end{align}
				となり,$t \longmapsto Z_t(\omega)$の連続性より$T(\omega) < T(\omega)$が従い矛盾が生じる.
				逆に$Z_T(\omega) > b$なら中間値の定理より
				\begin{align}
					Z_t(\omega) = b,
					\quad s < \exists t < T(\omega)
				\end{align}
				となるから,$T(\omega) \leq t < T(\omega)$という矛盾が生じ,(\refeq{eq:chapter_1_Problem_3_28_1})が出る.
				これにより,$\omega \in \{Z_s < b\}$に対し
				\begin{align}
					T(\omega) < \infty \quad \Rightarrow \quad b = Z_T(\omega) \leq \sup{t > s}{Z_t(\omega)}
				\end{align}
				が成立する.一方で$\mbox{a.s.}\omega \in \{Z_s < b\}$で$Z_t(\omega) \longrightarrow 0$となるから,
				$0 < \epsilon < b$に対し或る$t_0$が存在して
				\begin{align}
					Z_t(\omega) < \epsilon,
					\quad (\forall t > t_0)
				\end{align}
				が満たされる.この場合
				\begin{align}
					\sup{t > s}{Z_t(\omega)} \geq b
					\quad \Rightarrow \quad
					\sup{t \in [s,t_0]}{Z_t(\omega)} \geq b
				\end{align}
				となるから,連続性より$Z_t(\omega) = b$を満たす$t \in (s,t_0)$が存在し,
				$T(\omega) \leq t$が従い(\refeq{eq:chapter_1_Problem_3_28_2})が出る.
				
			\item[第二段]
				(i)を示す.任意の$A \in \mathscr{F}_s$と$n > s$に対し,任意抽出定理と
				(\refeq{eq:chapter_1_Problem_3_28_1})より
				\begin{align}
					\int_{A \cap \{Z_s < b\}} Z_s\ dP
					&= \int_{A \cap \{Z_s < b\}} Z_{T \wedge n}\ dP \\
					&= \int_{A \cap \{Z_s < b\}} Z_T \defunc_{\{T \leq n\}}\ dP
						+ \int_{A \cap \{Z_s < b\}} Z_n \defunc_{\{T > n\}}\ dP \\
					&= b P\left[ A \cap \{Z_s < b\} \cap \{T \leq n\} \right]
						+ \int_{A \cap \{Z_s < b\}} Z_n \defunc_{\{T > n\}}\ dP
				\end{align}
				が成立する.ここで
				\begin{align}
					P\left[ A \cap \{Z_s < b\} \cap \{T \leq n\} \right]
					\longrightarrow 
					P\left[ A \cap \{Z_s < b\} \cap \{T < \infty\} \right],
					\quad (n \longrightarrow \infty)
				\end{align}
				かつ$Z_n \defunc_{\{T > n\} \cap \{Z_s < b\}} < b,\ (\forall n > s)$及び
				\begin{align}
					Z_n \defunc_{\{T > n\}} \longrightarrow Z_\infty \defunc_{\{T=\infty\}},
					\quad (n \longrightarrow \infty)
				\end{align}
				が成り立つから,$Z_\infty = 0\ \mbox{a.s. $P$}$とLebesgueの収束定理より
				\begin{align}
					\int_{A \cap \{Z_s < b\}} Z_s\ dP
					= b P\left[ A \cap \{Z_s < b\} \cap \{T < \infty\} \right]
					= b \int_{A \cap \{Z_s < b\}} \defunc_{\{T < \infty\}}\ dP
				\end{align}
				が得られる.更に(\refeq{eq:chapter_1_Problem_3_28_2})より
				\begin{align}
					\int_{A \cap \{Z_s < b\}} Z_s\ dP
					= b \int_{A \cap \{Z_s < b\}} \defunc_{\{\sup{t > s}{Z_t(\omega)} \geq b\}}\ dP
					= b \int_{A \cap \{Z_s < b\}} P\left[ \sup{t>s}{Z_t \geq b}\ \middle|\ \mathscr{F}_s \right]\ dP
				\end{align}
				となるから,$A \in \mathscr{F}_s$の任意性より
				\begin{align}
					P\left[ \sup{t>s}{Z_t \geq b}\ \middle|\ \mathscr{F}_s \right] \defunc_{\{Z_s < b\}}
					= \frac{1}{b} Z_s \defunc_{\{Z_s < b\}},
					\quad \mbox{a.s. $P$}
				\end{align}
				が出る.
				
			\item[第三段]
				(iii)を示す.$t \longmapsto Z_t(\omega)$の連続性より
				\begin{align}
					\sup{t > s}{Z_t(\omega)} \geq b 
					\quad \Leftrightarrow \quad
					\sup{t \geq s}{Z_t(\omega)} \geq b,
					\quad (\forall \omega \in \{Z_s < b\})
				\end{align}
				となるから
				\begin{align}
					P\left[ \sup{t \geq s}{Z_t \geq b} \right]
					&= \int_\Omega P\left[ \sup{t \geq s}{Z_t \geq b}\ \middle|\ \mathscr{F}_s \right]\ dP \\
					&= \int_{\{Z_s \geq b\}} P\left[ \sup{t \geq s}{Z_t \geq b}\ \middle|\ \mathscr{F}_s \right]\ dP
						+ \int_{\{Z_s < b\}} P\left[ \sup{t > s}{Z_t \geq b}\ \middle|\ \mathscr{F}_s \right]\ dP \\
					&= \int_{\{Z_s \geq b\}} \defunc_{\left\{ \sup{t \geq s}{Z_t \geq b} \right\}}\ dP
						+ \frac{1}{b} \int_{\{Z_s < b\}} Z_s\ dP \\
					&= P[Z_s \geq b] + \frac{1}{b} E[Z_s \defunc_{\{Z_s < b\}}]
				\end{align}
				が成立する.
				\QED
		\end{description}
	\end{prf}
	
	\begin{itembox}[l]{Problem 3.29}
		Let $\Set{X_t,\mathscr{F}_t}{0 \leq t < \infty}$ be a continuous, nonnegative supermartingale
		and $T = \inf{}{\Set{t \geq 0}{X_t = 0}}$. Show that
		\begin{align}
			X_{T + t} = 0; \quad 0 \leq t < \infty \quad \mbox{hold a.s. on $\{T < \infty\}$}.
		\end{align}
	\end{itembox}
	
	\begin{prf}
		
	\end{prf}
	
	\begin{itembox}[l]{Exercise 3.30}
		Suppose that the filtration $\{\mathscr{F}_t\}$ satisfies the usual conditions and let 
		$X^{(n)} = \Set{X^{(n)}_t,\mathscr{F}_t}{0 \leq t < \infty},\ n \geq 1$ be an increasing sequence
		of right-continuous supermartingales, such that the random variable $\xi_t \coloneqq \lim_{n \to \infty} X^{(n)}_t$
		is nonnegative and integrable for every $0 \leq t < \infty$. Then there exists an RCLL supermartingale
		$X = \Set{X_t,\mathscr{F}_t}{0 \leq t < \infty}$ which is a modification of the process 
		$\xi = \Set{\xi_t,\mathscr{F}_t}{0 \leq t < \infty}$.
	\end{itembox}
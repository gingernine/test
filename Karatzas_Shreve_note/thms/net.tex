\subsection{有向点族}
	\begin{screen}
		\begin{dfn}[有向集合]
			空でない集合$\Lambda$において
			\begin{description}
				\item[(D1)] $\lambda \leq \lambda,\quad (\forall \lambda \in \Lambda)$,
				\item[(D2)] $\lambda \leq \mu,\ \mu \leq \nu 
					\quad \Longrightarrow \quad \lambda \leq \nu,\quad 
					(\forall \lambda,\mu,\nu \in \Lambda)$,
				\item[(D3)] 任意の二元$\lambda,\mu \in \Lambda$に対し
					或る$\nu \in \Lambda$が存在して
					$\lambda \leq \nu,\ \mu \leq \nu$を満たす,
			\end{description}
			を満たす二項関係$\leq$が定まっているとき,
			対$(\Lambda,\leq)$を有向集合
			\index{ゆうこうしゅうごう@有向集合}(directed set)と呼ぶ.
		\end{dfn}
	\end{screen}
	自然数全体$\N$や実数全体$\R$は,通常の順序により
	有向集合となっている.また位相空間の一点の近傍全体も
	\begin{align}
		U \leq V \quad \overset{\mathrm{def}}{\Longleftrightarrow} \quad
		U \supset V
	\end{align}
	により有向集合となる.
	
	\begin{screen}
		\begin{dfn}[有向点族]
			$X$を空でない集合,$\Lambda$を有向集合とするとき,
			写像$x:\Lambda \longrightarrow X\ (\lambda \longmapsto x_\lambda)$を
			有向点族\index{ゆうこうてんぞく@有向点族}(net)と呼ぶ.
			$x$は$(x_\lambda)_{\lambda \in \Lambda}$や$(x_\lambda)$とも書く.
			特に$\Lambda=\N$に対する有向点族を点列\index{てんれつ@点列}(sequence)と呼ぶ.
		\end{dfn}
	\end{screen}
	
	\begin{screen}
		\begin{dfn}[有向点族の収束]
			$x = (x_\lambda)$を位相空間$X$と有向集合$(\Lambda,\leq)$で定まる有向点族
			とする.点$a \in X$において,$a$の任意の近傍$U$に対し或る
			$\lambda_0 \in \Lambda$が存在して
			\begin{align}
				\lambda_0 \leq \lambda \quad \Longrightarrow \quad
				x_\lambda \in U
			\end{align}
			となるとき,$(x_\lambda)$は$a$に収束する(converge)といい
			$\lim x_\lambda = a$と書く.
		\end{dfn}
	\end{screen}
	
	\begin{screen}
		\begin{thm}
		\end{thm}
	\end{screen}
\subsection{有向点族}
	第一可算性が仮定された空間では
	可算個の点族(点列)の収束を用いることでいくつかの位相的概念を記述できるが,
	一般に位相空間では近傍が`多すぎる'ため位相概念を記述するのに点列では間に合わない.
	有向点族の理論では,非可算個の集合に或る種の`向き'を与えることで
	それを添数集合とする点族に収束の概念が定式化され,
	一般の位相空間における閉包や連続性,コンパクト性の概念を点族の収束により記述することが可能となる.
	
	\begin{screen}
		\begin{dfn}[有向集合]
			空でない集合$\Lambda$において
			任意の有限部分集合が上界を持つような前順序が定まっているとき,
			つまり次を満たす二項関係$\leq$が定まっているとき,
			対$(\Lambda,\leq)$を有向集合\index{ゆうこうしゅうごう@有向集合}(directed set)と呼ぶ:
			\begin{description}
				\item[(反射律)] $\lambda \leq \lambda,\quad (\forall \lambda \in \Lambda)$,
				\item[(推移律)] $\lambda \leq \mu,\ \mu \leq \nu 
					\quad \Longrightarrow \quad \lambda \leq \nu,\quad 
					(\forall \lambda,\mu,\nu \in \Lambda)$,
				\item[(有向律)] 
					$M \subset \Lambda$が有限なら
					$\mu \leq \lambda,\ (\forall \mu \in M)$を満たす
					$\lambda \in \Lambda$が存在する.
			\end{description}
			また$\lambda < \mu \overset{\mathrm{def}}{\Longleftrightarrow} 
			\mbox{$\lambda \leq \mu$かつ$\lambda \neq \mu$}$と定める.
		\end{dfn}
	\end{screen}
	正の自然数全体$\N$や実数全体$\R$は,通常の順序により
	有向集合となっている.また位相空間の一点の近傍全体も
	\begin{align}
		U \leq V \quad \overset{\mathrm{def}}{\Longleftrightarrow} \quad
		U \supset V
	\end{align}
	により有向集合となる.
	
	\begin{screen}
		\begin{dfn}[有向点族]
			有向集合を添数集合とする点族
			(P. \pageref{dfn:family_collection})
			を有向点族\index{ゆうこうてんぞく@有向点族}(net)と呼ぶ.
			$(\Lambda,\leq),\ (\Gamma,\preceq)$を有向集合,
			$(x_\lambda)_{\lambda \in \Lambda}$を有向点族とするとき,
			共終かつ序列を保つ写像$f:\Gamma \longrightarrow \Lambda$:
			つまり
			\begin{description}
				\item[(単調性)] $\gamma \preceq \xi \quad \Longrightarrow \quad
					f(\gamma) \leq f(\xi),\quad (\forall \gamma,\xi \in \Gamma)$,
				\item[(共終性)] $f(\Gamma)$が非有界:
					任意の$\lambda \in \Lambda$に対し
					$\lambda \leq f(\gamma)$を満たす$\gamma \in \Gamma$が存在する
			\end{description}
			を満たす写像$f$に対して,$\left(x_{f(\gamma)}\right)_{\gamma \in \Gamma}$を
			$(x_\lambda)$の部分有向点族\index{ぶぶんゆうこうてんぞく@部分有向点族}
			(subnet)と呼ぶ:
		\end{dfn}
	\end{screen}
	特に$\N$を有向集合とする有向点族を点列\index{てんれつ@点列}(sequence)と呼ぶ.
	また点列$(x_n)_{n \in \N}$に対し
	\begin{align}
		f:\N \ni k \longmapsto n_k \in \N,
		\quad (n_1 < n_2 < n_3 < \cdots)
	\end{align}
	で定まる部分有向点族$\left(x_{n_k}\right)_{k \in \N}$
	を部分列\index{ぶぶんれつ@部分列}(subsequence)と呼ぶ.
	一般の部分有向点族ではそれを定める写像$f$に単射性を仮定していないが
	(cf. Tychonoff plank),部分列は$k < j$なら$n_k < n_j$が満たされるものと約束する.
	従って点列の部分有向点族といってもそれが部分列となっているとは限らない.
	
	\begin{screen}
		\begin{dfn}[有向点族の収束\index{ゆうこうてんぞくのしゅうそく@有向点族の収束}]
			$x = (x_\lambda)_{\lambda \in \Lambda}$を位相空間$S$と
			有向集合$(\Lambda,\leq)$で定まる有向点族とする.
			点$a \in S$において,$a$の任意の近傍$U$に対し或る
			$\lambda_0 \in \Lambda$が存在して
			\begin{align}
				\lambda_0 \leq \lambda \quad \Longrightarrow \quad
				x_\lambda \in U
			\end{align}
			となるとき,$(x_\lambda)$は$a$に収束する(converge)といい
			$\lim x_\lambda = a$や$\lim_{\lambda} x_\lambda = a$と書く.
			また$(x_\lambda)_{\lambda \in \Lambda}$が
			部分集合$A$上の有向点族である場合,$(x_\lambda)_{\lambda \in \Lambda}$が
			$A$の点に収束するとき$(x_\lambda)_{\lambda \in \Lambda}$は`$A$で収束する'という.
		\end{dfn}
	\end{screen}
	
	\begin{screen}
		\begin{thm}[有向点族が収束する$\Longleftrightarrow$任意の部分点族が収束する]
		\label{thm:a_net_converges_iff_every_subnet_converges}
			$(x_\lambda)_{\lambda \in \Lambda}$を位相空間$S$
			と有向集合$(\Lambda,\leq)$で定まる有向点族とし,
			また$a$を$S$の任意の点とするとき
			\begin{align}
				\mbox{$(x_\lambda)_{\lambda \in \Lambda}$が$a$に収束する}
				\quad \Longleftrightarrow \quad
				\mbox{$(x_\lambda)_{\lambda \in \Lambda}$
				の任意の部分有向点族が$a$に収束する}
				\label{eq:thm_a_net_converges_iff_every_subnet_converges_2}
			\end{align}
			が成立する.特に$(x_\lambda)_{\lambda \in \Lambda}$が点列であるとき,
			右辺で部分有向点族を部分列に替えても同値関係は成立する.
		\end{thm}
	\end{screen}
	
	\begin{prf}
				$(x_\lambda)_{\lambda \in \Lambda}$が$a$に収束するとき,
				$a$の任意の近傍$U$に対し或る$\lambda_0 \in \Lambda$が存在して
				\begin{align}
					\lambda_0 \leq \lambda
					\quad \Longrightarrow \quad
					x_\lambda \in U
				\end{align}
				を満たす.$(y_\gamma)_{\gamma \in \Gamma}$
				を$(x_\lambda)_{\lambda \in \Lambda}$
				の部分有向点族とするとき,つまりこのとき或る有向集合$(\Gamma,\preceq)$と
				$f:\Gamma \longrightarrow \Lambda$により
				$y_\gamma = x_{f(\gamma)}$と表せるが,$f$の共終性から
				$\lambda_0 \leq f(\gamma_0)$を満たす$\gamma_0 \in \Gamma$が存在し,
				$f$の単調性と$\leq$の推移律より
				\begin{align}
					\gamma_0 \preceq \gamma
					\quad \Longrightarrow \quad
					f(\gamma_0) \leq f(\gamma)
					\quad \Longrightarrow \quad
					\lambda_0 \leq f(\gamma)
					\quad \Longrightarrow \quad
					y_\gamma = x_{f(\gamma)} \in U
				\end{align}
				が従うから$(y_\lambda)_{\gamma \in \Gamma}$は$a$に収束する.
				逆に$(x_\lambda)_{\lambda \in \Lambda}$が$a$に収束しないとき,
				$a$の或る近傍$V$では任意の$\lambda \in \Lambda$に対し
				\begin{align}
					\lambda \leq \mu,
					\quad x_\mu \notin V
					\label{eq:thm_a_net_converges_iff_every_subnet_converges_1}
				\end{align}
				を満たす$\mu \in \Lambda$が取れる.
				ここで
				\begin{align}
					\Gamma \coloneqq \Set{\lambda \in \Lambda}{x_\lambda \notin U}
				\end{align}
				とおけば,任意の有限集合$M \subset \Gamma$に対し
				$\Lambda$における上界$\lambda$が存在するが,
				(\refeq{eq:thm_a_net_converges_iff_every_subnet_converges_1})
				より$\lambda \leq \mu$を満たす
				$\mu \in \Gamma$が取れるから$(\Gamma,\leq)$は有向集合となる.
				恒等写像$\Gamma \longrightarrow \Lambda$は単調性と共終性を満たし,
				この場合の部分有向点族$(x_\gamma)_{\gamma \in \Gamma}$は$a$に収束しないから
				(\refeq{eq:thm_a_net_converges_iff_every_subnet_converges_2})が出る.
				$(x_\lambda)_{\lambda \in \Lambda}$が$a$に収束しない点列であるとき,
				任意の$n \in \N$に対して
				\begin{align}
					\inprod<n> \coloneqq
					\Set{m \in \N}{n < m,\ x_m \notin U}
				\end{align}
				は空ではない.
				$\N$の空でない部分集合の全体を$\mathscr{N}$として
				選択関数$\Phi \in \prod \mathscr{N}$を取り
				\begin{align}
					n_1 &\coloneqq \Phi(\inprod<1>), \\
					n_2 &\coloneqq \Phi(\inprod<n_1>), \\
					n_3 &\coloneqq \Phi(\inprod<n_2>), \\
					&\vdots
				\end{align}
				で$\{n_k\}_{k \in \N}$を定めれば,
				$(x_{n_k})_{k \in \N}$は$a$に収束しない部分列となる.
				\QED
	\end{prf}
	
	\begin{screen}
		\begin{thm}[有向点族の密集点に対する収束部分点族の存在]
			$(x_\lambda)_{\lambda \in \Lambda}$を位相空間$S$
			と有向集合$(\Lambda,\leq)$で定まる有向点族,$a$を$S$の点とするとき
			\begin{align}
				\mbox{$a$が$\{x_\lambda\}_{\lambda \in \Lambda}$の密集点である}
				\quad \Longleftrightarrow \quad
				\mbox{$a$に収束する$(x_\lambda)_{\lambda \in \Lambda}$の部分有向列が存在する}.
			\end{align}
			特に$\Lambda = \N$かつ
			$a$が可算な基本近傍系を持つ場合,右辺の部分有向点族を部分列に替えて
			同値関係が成立する.
		\end{thm}
	\end{screen}
	
	\begin{prf}
		$a$が$\{x_\lambda\}_{\lambda \in \Lambda}$の密集点であるとき,
		$\mathscr{U}$を$a$の基本近傍系とすれば
		任意の$U \in \mathscr{U}$に対し
		或る$\lambda \in \Lambda$が存在して$x_\lambda \in U$となるから,
		選択関数$\Phi \in \prod_{U \in \mathscr{U}} 
		\Set{\lambda \in \Lambda}{x_\lambda \in U}$を取り
		\begin{align}
			\Gamma \coloneqq \Set{(\Phi(U),U)}{U \in \mathscr{U}}
		\end{align}
		と定める.$\Lambda = \N$かつ$\mathscr{U} = \{U_k\}_{k \in \N}$と書けるときは
		\begin{align}
			\Gamma \coloneqq \Set{(n_k,U_k)}{k \in \N,\ x_{n_k} \in U_k,\ n_1 < n_2 < \cdots}
		\end{align}
		で定める.$\Gamma$において二項関係$\preceq$を
		\begin{align}
			(\lambda,U) \preceq (\mu,V) 
			\quad \overset{\mathrm{def}}{\Longleftrightarrow} \quad
			\mbox{$\lambda \leq \mu$かつ$U \supset V$}
		\end{align}
		で定めれば$(\Gamma,\preceq)$は有向集合となる.実際
		$\lambda \leq \lambda$かつ$U \supset U$より
		$(\lambda,U) \preceq (\lambda,U),\ (\forall (\lambda,U) \in \Gamma)$となり,
		\begin{align}
			(\lambda_1,U_1) \preceq (\lambda_2,U_2),\
			(\lambda_2,U_2) \preceq (\lambda_3,U_3) 
			&\quad \Longrightarrow \quad
			\lambda_1 \leq \lambda_2,\ \lambda_2 \leq \lambda_3,
			\ U_1 \supset U_2,\ U_2 \supset U_3 \\
			&\quad \Longrightarrow \quad
			\lambda_1 \leq \lambda_3,\ U_1 \supset U_3 \\
			&\quad \Longrightarrow \quad
			(\lambda_1,U_1) \preceq (\lambda_3,U_3)
		\end{align}
		より推移律も出る.また任意の有限個の$(\lambda_i,U_i) \in \Gamma,\ (i=1,2,\cdots,n)$
		に対し或る$\lambda \in \Lambda$と$U \in \mathscr{U}$が存在して
		\begin{align}
			\lambda_i \leq \lambda,\ (1 \leq i \leq n);
			\quad \bigcap_{i=1}^n U_i \supset U
		\end{align}
		を満たすが,$x_\mu \in U$を満たす$\mu \in \Lambda$を取れば
		或る$\nu \in \Lambda$で$\lambda,\mu \leq \nu$となり
		
		,$\Gamma$の任意の有限部分集合は上界を持つ.そして次の写像
		\begin{align}
			f:\Gamma \ni (\lambda,U) \longmapsto \lambda \in \Lambda
		\end{align}
		は単調かつ共終であるから$(x_{f(\gamma)})_{\gamma \in \Gamma}$は部分有向点族となる.
		そして任意に$U_0 \in \mathscr{U}$を取り
		$\lambda_0 \coloneqq \Phi(U_0)$とおけば
		\begin{align}
			(\lambda_0,U_0) \preceq (\lambda,U)
			\quad \Longrightarrow \quad
			x_{f(\lambda,U)} = x_\lambda \in U \subset U_0
		\end{align}
		となるから$(x_{f(\gamma)})_{\gamma \in \Gamma}$は$a$に収束する.
		逆に$a$に収束する$(x_\lambda)$の部分有向族(又は部分列)$(y_\gamma)$が存在するとき,
		$\{y_\gamma\}$は$\{x_\lambda\}$の部分集合でありかつ
		$a$の任意の近傍と交わるから,$\{x_\lambda\}$も$a$の任意の近傍と交叉する.
		\QED
	\end{prf}
	
	\begin{screen}
		\begin{thm}[コンパクト$\Longleftrightarrow$任意の有向点族が収束部分有向点族を持つ]
			位相空間$S$の部分集合$A$に対し,
			\begin{align}
				\mbox{$A$がコンパクト部分集合}
				\quad \Longleftrightarrow \quad
				\mbox{$A$上の任意の有向点族が$A$で収束する部分有向点族を持つ}.
			\end{align}
		\end{thm}
	\end{screen}
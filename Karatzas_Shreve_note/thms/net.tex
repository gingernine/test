\subsection{有向点族}
	第一可算性が仮定された空間では
	可算個の点族(点列)の収束を用いることでいくつかの位相的概念を記述できるが,
	一般に位相空間では近傍が`多すぎる'ため位相概念を記述するのに点列では間に合わない.
	有向点族の理論では,非可算個の集合に或る種の`向き'を与えることで
	それを添数集合とする点族に収束の概念が定式化され,
	一般の位相空間における閉包や連続性,コンパクト性の概念を点族の収束により記述することが可能となる.
	
	\begin{screen}
		\begin{dfn}[有向集合]
			空でない集合$\Lambda$において
			任意の有限部分集合が上界を持つような前順序が定まっているとき,
			つまり次を満たす二項関係$\leq$が定まっているとき,
			対$(\Lambda,\leq)$を{\bf 有向集合}\index{ゆうこうしゅうごう@有向集合}
			{\bf (directed set)}と呼ぶ:
			\begin{description}
				\item[(反射律)] $\lambda \leq \lambda,\quad (\forall \lambda \in \Lambda)$,
				\item[(推移律)] $\lambda \leq \mu,\ \mu \leq \nu 
					\quad \Longrightarrow \quad \lambda \leq \nu,\quad 
					(\forall \lambda,\mu,\nu \in \Lambda)$,
				\item[(有向律)] 
					$M \subset \Lambda$が有限集合なら
					$\mu \leq \lambda,\ (\forall \mu \in M)$を満たす
					$\lambda \in \Lambda$が取れる.
			\end{description}
			また$\lambda < \mu \overset{\mathrm{def}}{\Longleftrightarrow} 
			\mbox{$\lambda \leq \mu$かつ$\lambda \neq \mu$}$と定める.
		\end{dfn}
	\end{screen}
	正の自然数全体$\N$や実数全体$\R$は,通常の順序により
	有向集合となっている.また位相空間の一点の近傍全体も
	\begin{align}
		U \leq V \quad \overset{\mathrm{def}}{\Longleftrightarrow} \quad
		U \supset V
	\end{align}
	により有向集合となる.
	
	\begin{screen}
		\begin{dfn}[有向点族]
			有向集合を添数集合とする点族
			(P. \pageref{dfn:family_collection})
			を{\bf 有向点族}\index{ゆうこうてんぞく@有向点族}{\bf (net)}と呼ぶ.
			$(\Lambda,\leq),\ (\Gamma,\preceq)$を有向集合,
			$(x_\lambda)_{\lambda \in \Lambda}$を有向点族とするとき,
			共終かつ序列を保つ写像$f:\Gamma \longrightarrow \Lambda$:
			つまり
			\begin{description}
				\item[(単調性)] $\gamma \preceq \xi \quad \Longrightarrow \quad
					f(\gamma) \leq f(\xi),\quad (\forall \gamma,\xi \in \Gamma)$,
				\item[(共終性)] $f(\Gamma)$が非有界:
					任意の$\lambda \in \Lambda$に対し
					$\lambda \leq f(\gamma)$を満たす$\gamma \in \Gamma$が存在する
			\end{description}
			を満たす写像$f$に対して,$\left(x_{f(\gamma)}\right)_{\gamma \in \Gamma}$を
			$(x_\lambda)$の{\bf 部分有向点族}\index{ぶぶんゆうこうてんぞく@部分有向点族}
			{\bf (subnet)}と呼ぶ:
		\end{dfn}
	\end{screen}
	特に$\N$を有向集合とする有向点族を{\bf 点列}\index{てんれつ@点列}{\bf (sequence)}と呼ぶ.
	また点列$(x_n)_{n \in \N}$に対し
	\begin{align}
		f:\N \ni k \longmapsto n_k \in \N,
		\quad (n_1 < n_2 < n_3 < \cdots)
	\end{align}
	で定まる部分有向点族$\left(x_{n_k}\right)_{k \in \N}$
	を{\bf 部分列}\index{ぶぶんれつ@部分列}{\bf (subsequence)}と呼ぶ.
	一般の部分有向点族ではそれを定める写像$f$に単射性を仮定していないが
	(cf. Tychonoff plank),部分列は$k < j$なら$n_k < n_j$が満たされるものと約束する.
	従って点列の部分有向点族といってもそれが部分列となっているとは限らない.
	
	\begin{screen}
		\begin{dfn}[有向点族の収束\index{ゆうこうてんぞくのしゅうそく@有向点族の収束}]
			$x = (x_\lambda)_{\lambda \in \Lambda}$を位相空間$S$と
			有向集合$(\Lambda,\leq)$で定まる有向点族とする.
			点$a \in S$において,$a$の任意の近傍$U$に対し或る
			$\lambda_0 \in \Lambda$が存在して
			\begin{align}
				\lambda_0 \leq \lambda \quad \Longrightarrow \quad
				x_\lambda \in U
			\end{align}
			となるとき,$(x_\lambda)$は$a$に収束する(converge)といい
			$\lim x_\lambda = a$や$\lim_{\lambda} x_\lambda = a$と書く.
			また$(x_\lambda)_{\lambda \in \Lambda}$が
			部分集合$A$上の有向点族である場合,$(x_\lambda)_{\lambda \in \Lambda}$が
			$A$の点に収束するとき$(x_\lambda)_{\lambda \in \Lambda}$は`$A$で収束する'という.
		\end{dfn}
	\end{screen}
	
	\begin{screen}
		\begin{thm}[有向点族が収束する$\Longleftrightarrow$任意の部分点族が収束する]
		\label{thm:a_net_converges_iff_every_subnet_converges}
			$(x_\lambda)_{\lambda \in \Lambda}$を位相空間$S$
			と有向集合$(\Lambda,\leq)$で定まる有向点族とし,
			また$a$を$S$の任意の点とするとき
			\begin{align}
				\mbox{$(x_\lambda)_{\lambda \in \Lambda}$が$a$に収束する}
				\quad \Longleftrightarrow \quad
				\mbox{$(x_\lambda)_{\lambda \in \Lambda}$
				の任意の部分有向点族が$a$に収束する}
				\label{eq:thm_a_net_converges_iff_every_subnet_converges_2}
			\end{align}
			が成立する.特に$(x_\lambda)_{\lambda \in \Lambda}$が点列であるとき,
			右辺で部分有向点族を部分列に替えても同値関係は成立する.
		\end{thm}
	\end{screen}
	
	\begin{prf}
				$(x_\lambda)_{\lambda \in \Lambda}$が$a$に収束するとき,
				$a$の任意の近傍$U$に対し或る$\lambda_0 \in \Lambda$が存在して
				\begin{align}
					\lambda_0 \leq \lambda
					\quad \Longrightarrow \quad
					x_\lambda \in U
				\end{align}
				を満たす.$(y_\gamma)_{\gamma \in \Gamma}$
				を$(x_\lambda)_{\lambda \in \Lambda}$
				の部分有向点族とするとき,つまりこのとき或る有向集合$(\Gamma,\preceq)$と
				$f:\Gamma \longrightarrow \Lambda$により
				$y_\gamma = x_{f(\gamma)}$と表せるが,$f$の共終性から
				$\lambda_0 \leq f(\gamma_0)$を満たす$\gamma_0 \in \Gamma$が存在し,
				$f$の単調性と$\leq$の推移律より
				\begin{align}
					\gamma_0 \preceq \gamma
					\quad \Longrightarrow \quad
					f(\gamma_0) \leq f(\gamma)
					\quad \Longrightarrow \quad
					\lambda_0 \leq f(\gamma)
					\quad \Longrightarrow \quad
					y_\gamma = x_{f(\gamma)} \in U
				\end{align}
				が従うから$(y_\lambda)_{\gamma \in \Gamma}$は$a$に収束する.
				逆に$(x_\lambda)_{\lambda \in \Lambda}$が$a$に収束しないとき,
				$a$の或る近傍$V$では任意の$\lambda \in \Lambda$に対し
				\begin{align}
					\lambda \leq \mu,
					\quad x_\mu \notin V
					\label{eq:thm_a_net_converges_iff_every_subnet_converges_1}
				\end{align}
				を満たす$\mu \in \Lambda$が取れる.
				ここで
				\begin{align}
					\Gamma \coloneqq \Set{\lambda \in \Lambda}{x_\lambda \notin U}
				\end{align}
				とおけば,任意の有限集合$M \subset \Gamma$に対し
				$\Lambda$における上界$\lambda$が存在するが,
				(\refeq{eq:thm_a_net_converges_iff_every_subnet_converges_1})
				より$\lambda \leq \mu$を満たす
				$\mu \in \Gamma$が取れるから$(\Gamma,\leq)$は有向集合となる.
				恒等写像$\Gamma \longrightarrow \Lambda$は単調性と共終性を満たし,
				この場合の部分有向点族$(x_\gamma)_{\gamma \in \Gamma}$は$a$に収束しないから
				(\refeq{eq:thm_a_net_converges_iff_every_subnet_converges_2})が出る.
				$(x_\lambda)_{\lambda \in \Lambda}$が$a$に収束しない点列であるとき,
				任意の$n \in \N$に対して
				\begin{align}
					\inprod<n> \coloneqq
					\Set{m \in \N}{n < m,\ x_m \notin U}
				\end{align}
				は空ではない.
				$\N$の空でない部分集合の全体を$\mathscr{N}$として
				選択関数$\Phi \in \prod \mathscr{N}$を取り
				\begin{align}
					n_1 &\coloneqq \Phi(\inprod<1>), \\
					n_2 &\coloneqq \Phi(\inprod<n_1>), \\
					n_3 &\coloneqq \Phi(\inprod<n_2>), \\
					&\vdots
				\end{align}
				で$\{n_k\}_{k \in \N}$を定めれば,
				$(x_{n_k})_{k \in \N}$は$a$に収束しない部分列となる.
				\QED
	\end{prf}
	
	\begin{screen}
		\begin{thm}[閉集合は有向点族の極限点集合]
			$A$を位相空間$S$の部分集合とするとき,
			\begin{align}
				a \in \overline{A} \quad \Longleftrightarrow \quad
				\mbox{或る有向集合$(\Lambda,\leq)$と$A$上の有向点族
				$(x_\lambda)_{\lambda \in \Lambda}$が存在して
				$\lim x_\lambda = a$}.
			\end{align}
			特に$S$が第一可算空間であるとき,右辺で
			有向点族を点列に替えて同値関係が成立する.
		\end{thm}
	\end{screen}
	
	\begin{prf}\mbox{}
		\begin{description}
			\item[第一段] $\Longrightarrow$を示す.
				$\mathscr{U}$を$a$の基本近傍系とするとき,二項関係$\preceq$を
				\begin{align}
					U \preceq V \quad \overset{\mathrm{def}}{\Longleftrightarrow} \quad
					U \supset V
				\end{align}
				で定めれば$(\mathscr{U},\preceq)$は有向集合となる.
				定理\ref{thm:belongs_to_closure_iff_clusters}より
				\begin{align}
					a \in \overline{A} \quad \Longleftrightarrow \quad
					\mbox{任意の$U \in \mathscr{U}$に対し$A \cap U \neq \emptyset$}
				\end{align}
				が満たされ,いま$a \in \overline{A}$と仮定しているから
				$x \in \prod_{U \in \mathscr{U}} (A \cap U)$が取れて
				$x = (x_U)_{U \in \mathscr{U}}$は$A$上の有向点族となる.
				このとき$a$の任意の近傍$V$に対し$U_0 \subset V$となる
				$U_0 \in \mathscr{U}$が存在して
				\begin{align}
					\forall U \in \mathscr{U};\quad
					U_0 \preceq U \Longrightarrow x_U \in U \subset U_0 \subset V
				\end{align}
				となり$\lim x_U = a$が従う.$\mathscr{U}$が高々可算集合であるとき,
				つまり$\mathscr{U} = \{U_n\}_{n \in \N}$と表せるとき,
				\begin{align}
					\tilde{U}_n \coloneqq U_1 \cap U_2 \cap \cdots \cap U_n,
					\quad (n=1,2,\cdots) 
				\end{align}
				で減少列$\tilde{\mathscr{U}} \coloneqq \{\tilde{U}_n\}_{n \in \N}$を定めれば
				$\tilde{\mathscr{U}}$も$a$の基本近傍系となり,
				$y \in \prod_{n \in \N}(A \cap \tilde{U}_n)$を取り
				\begin{align}
					y_n \coloneqq y(n),\quad (\forall n \in \N)
				\end{align}
				とおけば,$a$の任意の近傍$V$に対し$\tilde{U}_{n_0} \subset V$
				となる$n_0 \in \N$が存在して
				\begin{align}
					\forall n \in \N;\quad
					n_0 \leq n \Longrightarrow y_n \in \tilde{U}_n \subset \tilde{U}_{n_0}
					\subset V
				\end{align}
				が成り立ち$\lim y_n = a$となる.
				
			\item[第二段] $\Longleftarrow$を示す.$a$に収束する$A$上の有向点族
				$(x_\lambda)_{\lambda \in \Lambda}$が存在するとき,
				\begin{align}
					a \in \overline{\Set{x_\lambda}{\lambda \in \Lambda}} 
					\subset \overline{A}
				\end{align}
				が従う.
				\QED
		\end{description}
	\end{prf}
	
	\begin{screen}
		\begin{dfn}[無限に含まれる]
			$(\Lambda,\leq)$を有向集合,
			$(x_\lambda)_{\lambda \in \Lambda}$を集合$S$上の有向点族,
			$A \subset S$とするとき,任意の$\lambda \in \Lambda$に対し
			$x_\mu \in A$を満たす$\lambda \leq \mu$が取れることを
			`$(x_\lambda)_{\lambda \in \Lambda}$は$A$に
			{\bf 無限に含まれる}\index{むげんにふくまれる@無限に含まれる}
			{\bf (frequently in)}'という.
		\end{dfn}
	\end{screen}
	
	\begin{screen}
		\begin{thm}[有向点族が点$a$の任意の近傍に無限に含まれる$\Longleftrightarrow$
		$a$に収束する部分点族が存在]
		\label{thm:a_net_frequently_in_all_nbhs_iff_some_subnet_converges}\mbox{}
			\begin{description}
				\item[(1)]
					$(x_\lambda)_{\lambda \in \Lambda}$を位相空間$S$
					と有向集合$(\Lambda,\leq)$で定まる有向点族とし,$a$を$S$の点とするとき,
					\begin{align}
						\mbox{$(x_\lambda)_{\lambda \in \Lambda}$が$a$の任意の近傍に無限に含まれる}
						\quad \Longleftrightarrow \quad
						\mbox{$a$に収束する$(x_\lambda)_{\lambda \in \Lambda}$の部分有向点族が存在する}.
					\end{align}
				
				\item[(2)]
					(1)において$\Lambda = \N$かつ$a$が可算な基本近傍系を持つ場合,
					\begin{align}
						\mbox{$(x_n)_{n \in \N}$が$a$の任意の近傍に無限に含まれる}
						\quad \Longleftrightarrow \quad
						\mbox{$a$に収束する$(x_n)_{n \in \N}$の部分列が存在する}.
					\end{align}
			\end{description}
		\end{thm}
	\end{screen}
	
	\begin{prf}\mbox{}
		\begin{description}
			\item[第一段] (1)の$\Longrightarrow$を示す.
				$(x_\lambda)_{\lambda \in \Lambda}$が$a$の任意の近傍に無限に含まれるとき,
				$\mathscr{U}$を$a$の基本近傍系として
				\begin{align}
					\Gamma \coloneqq \Set{(\lambda,U)}{\lambda \in \Lambda,\ U \in \mathscr{U},\ x_\lambda \in U}
				\end{align}
				とおき,$\Gamma$において二項関係$\preceq$を
				\begin{align}
					(\lambda,U) \preceq (\mu,V) 
					\quad \overset{\mathrm{def}}{\Longleftrightarrow} \quad
					\mbox{$\lambda \leq \mu$かつ$U \supset V$}
				\end{align}
				で定めれば$(\Gamma,\preceq)$は有向集合となる.実際
				$\lambda \leq \lambda$かつ$U \supset U$より
				$(\lambda,U) \preceq (\lambda,U),\ (\forall (\lambda,U) \in \Gamma)$となり,
				\begin{align}
					(\lambda_1,U_1) \preceq (\lambda_2,U_2),\
					(\lambda_2,U_2) \preceq (\lambda_3,U_3) 
					&\quad \Longrightarrow \quad
					\lambda_1 \leq \lambda_2,\ \lambda_2 \leq \lambda_3,
					\ U_1 \supset U_2,\ U_2 \supset U_3 \\
					&\quad \Longrightarrow \quad
					\lambda_1 \leq \lambda_3,\ U_1 \supset U_3 \\
					&\quad \Longrightarrow \quad
					(\lambda_1,U_1) \preceq (\lambda_3,U_3)
				\end{align}
				より推移律も出る.また任意に有限個の$(\lambda_i,U_i) \in \Gamma,\ (i=1,2,\cdots,n)$
				を取れば,或る$\lambda \in \Lambda$と$U \in \mathscr{U}$で
				\begin{align}
					\lambda_i \leq \lambda,\ (1 \leq i \leq n);
					\quad \bigcap_{i=1}^n U_i \supset U
				\end{align}
				となるが,このとき$\lambda \leq \mu$かつ$x_\mu \in U$を満たす$\mu \in \Lambda$が存在して
				$(\mu,U)$は$\{(\lambda_i,U_i)\}_{i=1}^n$の上界となる.
				\begin{align}
					f:\Gamma \ni (\lambda,U) \longmapsto \lambda \in \Lambda
				\end{align}
				は単調かつ共終であるから$(x_{f(\gamma)})_{\gamma \in \Gamma}$は部分有向点族となり,
				任意の$(\lambda_0,U_0) \in \Gamma$に対して
				\begin{align}
					(\lambda_0,U_0) \preceq (\lambda,U)
					\quad \Longrightarrow \quad
					x_{f(\lambda,U)} = x_\lambda \in U \subset U_0
				\end{align}
				が成り立つから$(x_{f(\gamma)})_{\gamma \in \Gamma}$は$a$に収束する.
			
			\item[第二段] (2)の$\Longrightarrow$を示す.
				$\mathscr{U} = \{U_k\}_{k \in \N}$を$a$の基本近傍系とすれば
				\begin{align}
					V_k \coloneqq U_1 \cap U_2 \cap \cdots \cap U_k,
					\quad (k=1,2,\cdots)
				\end{align}
				により単調減少な基本近傍系$\{V_k\}_{k \in \N}$が得られる.
				任意の$n,k \in \N$に対し
				\begin{align}
					\inprod<n,k> \coloneqq
					\Set{m \in \N}{n < m,\ x_m \in V_k}
				\end{align}
				とおけば$\inprod<n,k>$は空ではなく,
				選択関数$\Phi \in \prod_{n,k \in \N} \inprod<n,k>$を取り
				\begin{align}
					n_1 &\coloneqq \Phi(1,1), \\
					n_2 &\coloneqq \Phi(n_1,2), \\
					n_3 &\coloneqq \Phi(n_2,3), \\
					&\vdots
				\end{align}
				で$\{n_k\}_{k \in \N}$を定めれば,$(x_{n_k})_{k \in \N}$は
				$(x_n)_{n \in \N}$の部分列となり,任意の$V_{k_0}$に対して
				\begin{align}
					k_0 \leq k \quad \Longrightarrow \quad
					x_{n_k} \in V_k \subset V_{k_0}
				\end{align}
				となるから$(x_{n_k})_{k \in \N}$は$a$に収束する.
		
			\item[第三段] (1)の$\Longleftarrow$を示す.
				或る有向集合$(\Theta,\unlhd)$
				と単調かつ共終な$h:\Theta \longrightarrow \Lambda$が存在して
				$(x_{h(\theta)})_{\theta \in \Theta}$が$a$に収束するとき,
				任意に$\lambda \in \Lambda$と$a$の近傍$U$を取れば,
				或る$\theta_1 \in \Theta$で$\lambda \leq h(\theta_1)$となり,
				また或る$\theta_2 \in \Theta$が存在して
				\begin{align}
					\theta_2 \unlhd \theta \quad \Longrightarrow \quad
					x_{f(\theta)} \in U
				\end{align}
				が成り立つ.$\theta_1,\theta_2 \unlhd \theta$を満たす
				$\theta \in \Theta$が取れるが,このとき$\lambda \leq h(\theta)$かつ
				$x_{h(\theta)} \in U$が従うから$(x_\lambda)_{\lambda \in \Lambda}$
				は$U$に無限に含まれる.(2)においても,部分列は部分有向点族であるから
				$\Longleftarrow$が成立する.
				\QED
		\end{description}
	\end{prf}
	
	\begin{screen}
		\begin{thm}[コンパクト$\Longleftrightarrow$任意の有向点族が収束部分有向点族を持つ]
			位相空間$S$の部分集合$A$に対し,
			\begin{align}
				\mbox{$A$がコンパクト部分集合}
				\quad \Longleftrightarrow \quad
				\mbox{$A$上の任意の有向点族が$A$で収束する部分有向点族を持つ}.
			\end{align}
		\end{thm}
	\end{screen}
	
	\begin{prf}\mbox{}
		\begin{description}
			\item[第一段]
				$\Longrightarrow$を示す.或る有向集合$(\Lambda,\leq)$と$A$上の有向点族
				$(x_\lambda)_{\lambda \in \Lambda}$で
				$A$で収束する部分有向点族を持たないものが存在するとき,
				定理\ref{thm:a_net_frequently_in_all_nbhs_iff_some_subnet_converges}
				より任意の$a \in A$に対し或る$\lambda_a \in \Lambda$と
				$a$の近傍$U_a$が取れて
				\begin{align}
					x_\lambda \notin U_a, \quad (\lambda_a \leq \forall \lambda)
				\end{align}
				が満たされる.このとき$\Set{U_a^{\mathrm{o}}}{a \in A}$は$A$の
				$S$における開被覆となるが,もし有限個の$a_1,\cdots,a_n \in A$で
				\begin{align}
					A \subset U_{a_1}^{\mathrm{o}} \cup U_{a_2}^{\mathrm{o}}
					\cup \cdots \cup U_{a_n}^{\mathrm{o}}
				\end{align}
				が成り立つとすると,$\left\{\lambda_{a_1},\cdots,\lambda_{a_n}\right\}$
				の上界$\lambda \in \Lambda$において
				\begin{align}
					x_\lambda \notin U_{a_1} \cup U_{a_2} \cup \cdots \cup U_{a_n}
				\end{align}
				となり$x_\lambda \in A$に矛盾する.従って
				定理\ref{thm:subset_is_compact_iff_every_original_open_cover_contains_finite_subcover}より
				$A$はコンパクトではない.
				
			\item[第二段]
				$\Longleftarrow$を示す.$\mathscr{F}$を$S$の閉集合族とし,
				$\Set{A \cap F}{F \in \mathscr{F}}$が有限交叉的であるとする.
				\begin{align}
					\mathfrak{M} \coloneqq \Set{\mathscr{M}}{\mbox{$\mathscr{M}$は
					$\mathscr{F}$の空でない有限部分族}}
				\end{align}
				とおいて$\mathfrak{M}$上の二項関係$\preceq$を
				\begin{align}
					\mathscr{M} \preceq \mathscr{N} 
					\quad \overset{\mathrm{def}}{\Longleftrightarrow} \quad
					\mathscr{M} \subset \mathscr{N}
				\end{align}
				で定めれば$(\mathfrak{M},\preceq)$は有向集合となる.
				任意の$\mathscr{M} \in \mathfrak{M}$で
				$A \cap \bigcap \mathscr{M} \neq \emptyset$が満たされるから
				\begin{align}
					x \in \prod_{\mathscr{M} \in \mathfrak{M}} 
					\left(A \cap \bigcap \mathscr{M}\right)
				\end{align}
				が取れて,$x = (x_{\mathscr{M}})_{\mathscr{M} \in \mathfrak{M}}$
				は$A$上の有向点族をなし,仮定より或る$p \in A$が存在して
				$(x_{\mathscr{M}})_{\mathscr{M} \in \mathfrak{M}}$
				の或る部分有向点族が$p$に収束する.このとき任意に$F \in \mathscr{F}$を取れば,
				$(x_{\mathscr{M}})_{\mathscr{M} \in \mathfrak{M}}$は
				$p$の任意の近傍$U$に無限に含まれるから
				\begin{align}
					\{F\} \preceq \mathscr{M},\quad x_{\mathscr{M}} \in U
				\end{align}
				を満たす$\mathscr{M} \in \mathfrak{M}$が存在し,
				\begin{align}
					x_{\mathscr{M}} \in A \cap \bigcap \mathscr{M}
					\subset A \cap F
				\end{align}
				と併せて$U \cap F \neq \emptyset$となる.
				$U$の任意性,定理\ref{thm:belongs_to_closure_iff_clusters}と
				$F$が閉であることから
				\begin{align}
					p \in \overline{F} = F
				\end{align}
				が従い,$F$の任意性から$p \in A \cap \bigcap \mathscr{F}$が成り立つ.
				そして定理\ref{thm:compact_iff_closed_sets_family_finitely_intersect}より
				$A$のコンパクト性が出る.
				\QED
		\end{description}
	\end{prf}
	
	一般の位相空間において部分集合$A$がコンパクトであることの同値条件は
	$A$上の任意の有向点族が$A$で収束する部分有向点族を持つことであったが,
	この条件で有向点族を点列に替えたもの,すなわち
	\begin{itemize}
		\item $A$上の任意の点列が$A$で収束する部分列を持つ
	\end{itemize}
	という性質が成り立つとき,$A$は{\bf 点列コンパクト}
	\index{てんれつこんぱくと@点列コンパクト}{\bf (sequentially compact)}な部分集合であるという.
	
	\begin{screen}
		\begin{thm}[点列コンパクト$\Longrightarrow$可算コンパクト]
			
		\end{thm}
	\end{screen}
\section{Stochastic Processes and $\sigma$-Fields}
\begin{itembox}[l]{Problem 1.5}
	Let $Y$ be a modification of $X$, and suppose that every 
	sample path of both processes are right-continuous sample paths. 
	Then $X$ and $Y$ are indistinguishable.
\end{itembox}

\begin{prf}
	Karatzas-Shreveの問題文には``suppose that both processes have a.s. right-continuous sample paths''
	と書いてあるがこれは誤植である.実際,或る零集合$\emptyset \neq N \in \mathscr{F}$が存在し,
	$\omega \in \Omega \backslash N$に対する$X,Y$のパスが右連続であるとする.このとき
	\begin{align}
		\left\{ X_t = Y_t,\ \forall t \geq 0 \right\}
		&= \left\{ X_t = Y_t,\ \forall t \geq 0 \right\} \cap N
			+ \left\{ X_t = Y_t,\ \forall t \geq 0 \right\} \cap N^c \\
		&= \left\{ X_t = Y_t,\ \forall t \geq 0 \right\} \cap N
			+ \bigcap_{r \in \Q \cap [0,\infty)} \left\{ X_r = Y_r \right\} \cap N^c
	\end{align}
	が成り立つが,右辺第一項は$(\Omega,\mathscr{F},P)$が完備である場合でないと
	可測集合であるという保証がない.従って全てのサンプルパスが
	右連続であると仮定し直す必要がある.
	この場合
	\begin{align}
		\left\{ X_t = Y_t,\ \forall t \geq 0 \right\}
		= \bigcap_{r \in \Q \cap [0,\infty)} \left\{ X_r = Y_r \right\}
	\end{align}
	が成立するから,$P(X_r = Y_r) = 1\   (\forall r \geq 0)$より
	\begin{align}
		P(X_t = Y_t,\ \forall t \geq 0)
		= P \biggl( \bigcap_{r \in \Q \cap [0,\infty)} \left\{ X_r = Y_r \right\} \biggr)
		= 1
	\end{align}
	が従う.
	\QED
\end{prf}

\begin{itembox}[l]{Problem 1.7}
		Let $X$ be a process with every sample path RCLL. 
		Let $A$ be the event that $X$ is continuous on $[0,t_0)$. 
		Show that $A \in \mathscr{F}^X_{t_0}$.
\end{itembox}

\begin{prf}
	$[0,t_0)$に属する有理数の全体を$\Q^* \coloneqq \Q \cap [0,t_0)$と表すとき,
	\begin{align}
		A = \bigcap_{m \geq 1} \bigcup_{n \geq 1} \bigcap_{\substack{p,q \in \Q^* \\ |p-q| < 1/n}}
		\Set{\omega \in \Omega}{\left|X_p(\omega) - X_q(\omega) \right| < \frac{1}{m}}
	\end{align}
	が成立することを示せばよい.実際,$\omega \longmapsto \left(X_p(\omega), X_q(\omega) \right)$の
	$\mathscr{F}^X_{t_0}/\borel{\R^2}$-可測性と
	\begin{align}
		\Phi:\R \times \R \ni (x,y) \longmapsto |x-y| \in \R
	\end{align}
	の$\borel{\R^2}/\borel{\R}$-可測性より
	\begin{align}
		\Set{\omega \in \Omega}{\left|X_p(\omega) - X_q(\omega) \right| < \frac{1}{m}}
		= \Set{\omega \in \Omega}{\left(X_p(\omega), X_q(\omega) \right) \in 
		\Phi^{-1}\left(B_{1/m}(0)\right)}
		\in \mathscr{F}^X_{t_0}
	\end{align}
	が得られ$A \in \mathscr{F}^X_{t_0}$が従う.$\left(B_{1/m}(0) = \Set{x \in \R}{|x| < 1/m}.\right)$
	\begin{description}
		\item[第一段]
			$\omega \in A^c$を任意にとる.このとき
			或る$s \in (0,t_0)$が存在して,$t \longmapsto X_t(\omega)$は
			$t = s$において左側不連続である.従って或る$m \geq 1$については,
			任意の$n \geq 1$に対し$0< s-u < 1/3n$を満たす
			$u$が存在して
			\begin{align}
				\left|X_u(\omega) - X_s(\omega) \right| \geq \frac{1}{m}
			\end{align}
			を満たす.一方でパスの右連続性より
			$0 < p - s,\ q - u < 1/3n$を満たす$p,q \in \Q^*$が存在して
			\begin{align}
				\left|X_p(\omega) - X_s(\omega) \right| < \frac{1}{4m},
				\quad \left|X_q(\omega) - X_u(\omega) \right| < \frac{1}{4m}
			\end{align}
			が成立する.このとき$0 < |p - q| < 1/n$かつ
			\begin{align}
				\left|X_p(\omega) - X_q(\omega) \right|
				\geq \left|X_p(\omega) - X_s(\omega) \right|
					- \left|X_s(\omega) - X_u(\omega) \right|
					- \left|X_q(\omega) - X_u(\omega) \right|
				\geq \frac{1}{2m}
			\end{align}
			が従い
			\begin{align}
				\omega \in \bigcup_{m \geq 1} \bigcap_{n \geq 1} \bigcup_{\substack{p,q \in \Q^* \\ |p-q| < 1/n}}
		\Set{\omega \in \Omega}{\left|X_p(\omega) - X_q(\omega) \right| \geq \frac{1}{m}}
			\end{align}
			を得る.
		
		\item[第二段]
			任意に$\omega \in A$を取る.各点で有限な左極限が存在するという仮定から,
			\begin{align}
				X_{t_0}(\omega) \coloneqq \lim_{t \uparrow t_0}X_t(\omega)
			\end{align}
			と定めることにより
			\footnote{
				実際$X_{t_0}(\omega)$は所与のものであるが,いまは$[0,t_0]$上での連続性を考えればよいから
				便宜上値を取り替える.
			}
			$t \longmapsto X_t(\omega)$は$[0,t_0]$上で一様連続となる.
			従って
			\begin{align}
				\omega \in \bigcap_{m \geq 1} \bigcup_{n \geq 1} \bigcap_{\substack{p,q \in \Q^* \\ |p-q| < 1/n}}
		\Set{\omega \in \Omega}{\left|X_p(\omega) - X_q(\omega) \right| < \frac{1}{m}}
			\end{align}
			を得る.
			\QED
	\end{description}
\end{prf}

\begin{itembox}[l]{Problem 1.10 unsolved}
		Let $X$ be a process with every sample path LCRL, and 
		let A be the event that $X$ is continuous on $[0,x_0]$.
		Let $X$ be adapted to a right-continuous filtration 
		$(\mathscr{F}_t)_{t \geq 0}$. Show that $A \in \mathscr{F}_{t_0}$.
\end{itembox}

\begin{prf}\mbox{}
	\begin{description}
		\item[第一段]
			$\Q^* \coloneqq \Q \cap [0,t_0]$とおく.
			いま,任意の$n \geq 1$と$r \in \Q^*$に対し
			\begin{align}
				B_n (r) \coloneqq
				\bigcup_{m \geq 1} \bigcap_{k \leq m} 
				\Set{\omega \in \Omega}{\left| X_r(\omega)-X_{r+\frac{1}{k}}(\omega) \right| \leq \frac{1}{n}}
			\end{align}
			と定めるとき,
			\begin{align}
				A = \bigcap_{r \in \Q^*} \bigcap_{n \geq 1} B_n(r)
			\end{align}
			が成立することを示す.これが示されれば,
			\begin{align}
				\Set{\omega \in \Omega}{\left| X_r(\omega)-X_{r+\frac{1}{k}}(\omega) \right| \leq \frac{1}{n}}
				\in \mathscr{F}_{r+\frac{1}{k}},
				\quad (\forall r \in \Q^*,\ k \geq 1)
			\end{align}
			とフィルトレーションの右連続性から
			\begin{align}
				B_n (r) \in \bigcap_{k \geq m} \mathscr{F}_{r+\frac{1}{k}} = \mathscr{F}_{r+} = \mathscr{F}_{r}
			\end{align}
			が従い$A \in \mathscr{F}_{t_0}$が出る.
		
		\item[第二段]
			
	\end{description}
\end{prf}

\begin{itembox}[l]{Problem 1.16}
	If the process $X$ is measurable and the random time $T$ is finite, 
	then the function $X_T$ is a random variable.
\end{itembox}

\begin{prf}
	\begin{align}
		\tau:\Omega \ni \omega \longmapsto (T(\omega),\omega) \in [0,\infty) \times \Omega
	\end{align}
	とおけば,
	任意の$A \in \borel{[0,\infty)},\ B \in \mathscr{F}$に対して
	\begin{align}
		\tau^{-1}(A \times B) = \Set{\omega \in \Omega}{(T(\omega),\omega) \in A \times B}
		= T^{-1}(A) \cap B \in \mathscr{F}
	\end{align}
	が満たされる
	\begin{align}
		\Set{A \times B}{A \in \borel{[0,\infty)},\ B \in \mathscr{F}}
		\subset \Set{E \in \borel{[0,\infty)} \otimes \mathscr{F}}{\tau^{-1}(E) \in \mathscr{F}}
	\end{align}
	が従い$\tau$の$\mathscr{F}/\borel{[0,\infty)} \otimes \mathscr{F}$-可測性が出る.
	$X_T = X \circ \tau$より$X_T$は可測$\mathscr{F}/\borel{\R}$である.
	\QED
\end{prf}

\begin{itembox}[l]{Problem 1.17}
	Let $X$ be a measurable process and $T$ a random time. Show that 
	the collection of all sets of the form $\{X_T \in A\}$ and 
	$\{X_T \in A\} \cup \{T = \infty\};A \in \borel{\R}$, forms a 
	sub-$\sigma$-field of $\mathscr{F}$.
\end{itembox}

\begin{prf}
	$X_T$の定義域は$\{T<\infty\}$であるから,
	\begin{align}
		\mathscr{G} \coloneqq \Set{\{T < \infty\} \cap E}{E \in \mathscr{F}}
	\end{align}
	とおけば,前問の結果より$X_T$は可測$\mathscr{G}/\borel{\R}$である.
	$\mathscr{G} \subset \mathscr{F}$より
	\begin{align}
		\mathscr{H} \coloneqq \Set{\{X_T \in A\},\ \{X_T \in A\} \cup \{T = \infty\}}{A \in \borel{\R}}
	\end{align}
	に対して$\mathscr{H} \subset \mathscr{F}$が成立する.
	あとは$\mathscr{H}$が$\sigma$-加法族であることを示せばよい.
	実際,$A = \R$のとき
	\begin{align}
		\{X_T \in A\} \cup \{T = \infty\} = \{T < \infty\} \cup \{T = \infty\} = \Omega
	\end{align}
	となり$\Omega \in \mathscr{H}$が従い,また
	\begin{align}
		&\{X_T \in A\}^c = \{X_T \in A^c\} \cup \{T = \infty\}, \\
		&\left( \{X_T \in A\} \cup \{T = \infty\} \right)^c
		=  \{X_T \in A^c\} \cap \{T < \infty\}
		= \{X_T \in A^c\}
	\end{align}
	より$\mathscr{H}$は補演算で閉じる.更に$B_n \in \mathscr{H}\ (n=1,2,\cdots)$を取れば,
	\begin{align}
		\bigcup_{n=1}^{\infty} B_n = \left\{X_T \in \bigcup_{n=1}^{\infty} A_n \right\}
	\end{align}
	或は
	\begin{align}
		\bigcup_{n=1}^{\infty} B_n = \left\{X_T \in \bigcup_{n=1}^{\infty} A_n \right\} \cup \{T = \infty\}
	\end{align}
	が成立し$\bigcup_{n=1}^\infty B_n \in \mathscr{H}$を得る.
	\QED
\end{prf}
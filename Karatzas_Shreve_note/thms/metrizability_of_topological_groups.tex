\subsection{距離化可能性}
	位相群は一様化可能であるから,定理\ref{thm:Nagata_Smirnov_metrizability}より
	両立する近縁系が可算な基本近縁系を持てば
	擬距離化可能である.しかも,両立する擬距離として
	算法と相性の良いものが取れる.
	
	\begin{screen}
		\begin{dfn}[左不変距離]
			$\left(X,\sigma_X\right)$を群とし,$d$を$X$上の擬距離とする.
			$x$と$y$と$z$を$X$の任意の要素とするとき
			\begin{align}
				d\left(x,y\right) = d\left(\sigma_X\left(z,x\right),\sigma_X\left(z,y\right)\right)
			\end{align}
			が成り立つならば,$d$を{\bf 左不変擬距離}\index{ひだりふへんぎきょり@左不変擬距離}{\bf (left invariant pseudo metric)}と呼ぶ.
			$d$が距離ならばこれを{\bf 左不変距離}\index{ひだりふへんきょり@左不変距離}{\bf (left invariant metric)}と呼ぶ.
		\end{dfn}
	\end{screen}
	
	左不変擬距離とは,$X$の各要素$a$に対して
	\begin{align}
		X \ni x \longmapsto \sigma_X(a,x)
	\end{align}
	なる写像を等距写像たらしめる擬距離である.
\subsection{距離化}
	位相群は一様化可能であるから,定理\ref{thm:Nagata_Smirnov_metrizability}より
	両立する近縁系が可算な基本近縁系を持てば
	擬距離化可能である.しかも,両立する擬距離として
	算法と相性の良いものが取れる.
	
	\begin{screen}
		\begin{thm}[位相群は第一可算ならば可算な基本近縁系が取れる]
		\end{thm}
	\end{screen}
	
	\begin{screen}
		\begin{dfn}[左不変距離]
			$\left(X,\sigma_X\right)$を群とし,$d$を$X$上の擬距離とする.
			$x$と$y$と$a$を$X$の任意の要素とするとき
			\begin{align}
				d\left(x,y\right) = d\left(\sigma_X\left(a,x\right),\sigma_X\left(a,y\right)\right)
			\end{align}
			が成り立つならば,$d$を{\bf 左不変擬距離}\index{ひだりふへんぎきょり@左不変擬距離}{\bf (left invariant pseudo metric)}と呼ぶ.
			%$d$が距離ならばこれを{\bf 左不変距離}\index{ひだりふへんきょり@左不変距離}{\bf (left invariant metric)}と呼ぶ.
		\end{dfn}
	\end{screen}
	
	左不変擬距離とは,$X$の各要素$a$に対して
	\begin{align}
		X \ni x \longmapsto \sigma_X(a,x)
	\end{align}
	なる写像を等距写像たらしめる擬距離である.
	
	\begin{screen}
		\begin{thm}[左不変擬距離位相は平行移動不変である]
			$\left(X,\sigma_X\right)$を群とし,$d$を$X$上の左不変擬距離とし,
			$\mathscr{O}_X$を$d$による擬距離位相とする.このとき,
			$X$の任意の部分集合$u$と$X$の任意の要素$x$に対して
			\begin{align}
				u \in \mathscr{O}_X \Longleftrightarrow \Set{\sigma_X(x,y)}{y \in u} \in \mathscr{O}_X.
			\end{align}
		\end{thm}
	\end{screen}
	
	\begin{sketch}
		任意の$\delta>0$と$a \in X$に対し
		$B_\delta(a) \defeq \Set{x \in X}{d(x,a) < \delta}$と書けば,
		任意の$y \in X$で
		\begin{align}
			z \in y + B_\delta(a)
			\quad \Longleftrightarrow \quad
			d(z-y,a) < \delta
			\quad \Longleftrightarrow \quad
			d(z,y+a) < \delta
			\quad \Longleftrightarrow \quad
			z \in B_\delta(y+a)
		\end{align}
		が成り立つ.従って,部分集合$U$が$U = \bigcup_{a \in U}B_{\delta_a}(a)$と書けるとき
		任意の$x \in X$に対し
		\begin{align}
			x + U = \bigcup_{a \in U} \left(x+B_{\delta_a}(a)\right)
			= \bigcup_{a \in U} B_{\delta_a}(x+a)
		\end{align}
		となるから,$U$が開集合であることと$x + U$が開集合であることは同値になる.
		\QED
	\end{sketch}
	
	位相群においてその位相と両立する距離が左不変であるとは限らない.
	
	\begin{itembox}[l]{両立するが左不変距離でない距離}
		$\left(\left(X,\sigma_X\right),\mathscr{O}_X\right)$を位相群とし,
		左不変距離$d$により距離化可能であるとする.ここで
		\begin{align}
			(x,y) \longmapsto d(x,y) + d(-x,-y)
		\end{align}
		なる写像を$\rho$とおくと,$\rho$は$X$上の距離となり$\mathscr{O}_X$と両立する.
		ただし$\sigma_X$が可換でなければ$\rho$は左不変であるとは限らない.
		\url{https://math.stackexchange.com/questions/3208243/}
	\end{itembox}
	
	ゆえに定理\ref{thm:Nagata_Smirnov_metrizability}の結果だけでは
	両立する距離として左不変であるものが取れるかどうかはわからない.しかし次の定理は
	距離化可能であれば両立する距離で左不変であるものが取れることを保証する.
	
	\begin{screen}
		\begin{thm}[Birkhoff-Kakutani]
			位相群は,第一可算ならば左不変距離により距離化可能である.
		\end{thm}
	\end{screen}
	
	\url{https://math.stackexchange.com/questions/1315338/}
	
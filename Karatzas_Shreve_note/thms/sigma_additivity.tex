	\subsection{完全加法性}
		\begin{screen}
			\begin{dfn}[共通点性]
				$X$を集合とし,$\mathcal{B}$を$X$上の加法族とし,$\mu$を$\mathcal{B}$上の加法的な正値写像とする.
				$\{B_n\}_{n \in \Natural}$を
				\begin{align}
					\mu(B_0) < \infty \wedge
					\bigcap_{n \in \Natural} B_n = \emptyset \wedge
					\forall n \in \Natural\, (\, B_{n+1} \subset B_n\, )
				\end{align}
				を満たす$\mathcal{B}$の部分集合とするとき
				\begin{align}
					\lim_{n \to \infty} \mu(B_n) = 0
				\end{align}
				が成立するなら,$\mu$は{\bf 共通点性}\index{きょうつうてんせい@共通点性}を満たすという.
			\end{dfn}
		\end{screen}
		
		\begin{screen}
			\begin{thm}[有限測度の完全加法性の同値条件]
			\label{thm:equivalent_conditions_of_countable_additivity_for_finite_measure}
				$X$を集合とし,$\mathcal{B}$を$X$上の加法族とし,$\mu$を$\mathcal{B}$上の加法的正値測度として,
				$\mu(X) < \infty$であるとする.このとき,$\mu$が$\mathcal{B}$の上で完全加法的であることと
				$\mu$が共通点性を満たすことは同値である.
			\end{thm}
		\end{screen}
		
		\begin{prf}
			
		\end{prf}
		
		\begin{screen}
			\begin{dfn}[コンパクトクラス]
				$K$を集合とする.$K$の任意の可算部分集合が有限交叉性を満たすとき,つまり
				\begin{align}
					\forall u\, \left[\, u \subset K \wedge
					\Natural \eqp u \Longrightarrow \forall v\,
					\left(\, v \subset u \wedge \exists n \in \Natural\,
					(\, v \eqp n\, ) \Longrightarrow \bigcap v \neq \emptyset\, \right)\, \right]
				\end{align}
				となるとき,$K$を{\bf コンパクトクラス}\index{コンパクトクラス}{\bf (compact class)}と呼ぶ.
			\end{dfn}
		\end{screen}
		
		\begin{screen}
			\begin{thm}[測度有限な集合がコンパクトクラスで近似されるなら共通点性を持つ]
			\label{thm:finite_intersection_property_and_common_point_property}
				$X$を集合とし,$\mathcal{B}$を$X$上の加法族とし,
				$\mu$を$\mathcal{B}$上の加法的正値測度とする.
				また$\mathcal{K}$を$\power{X}$の部分集合でコンパクトクラスであるものとする.
				この下で,任意に正数$\epsilon$と$\mu(B) < \infty$なる$\mathcal{B}$の要素$B$が与えられたとき
				\begin{align}
					A \subset K \subset B \wedge \mu(B \backslash A )< \epsilon
				\end{align}
				を満たす$\mathcal{B}$の要素$A$と$\mathcal{K}$の要素$K$が取れるなら,$\mu$は共通点性を持つ.
			\end{thm}
		\end{screen}
		
		\begin{prf}
			$\{B_n\}_{n \in \Natural}$を
			\begin{align}
				\mu(B_1) < \infty
			\end{align}
			かつ
			\begin{align}
				\bigcap_{n \in \Natural} B_n = \emptyset
			\end{align}
			かつ
			\begin{align}
				\forall n \in \Natural\, (\, B_{n+1} \subset B_n\, )
			\end{align}
			を満たす$\mathcal{B}$の部分集合とする.
			\begin{align}
				\mu(B_N) = 0
			\end{align}
			なる自然数$N$が取れるとき,
			\begin{align}
				\forall n \in \Natural\, \left(\, N \leq n \Longrightarrow \mu(B_n) = 0\, \right)
			\end{align}
			が成り立つから
			\begin{align}
				\lim_{n \to \infty} \mu(B_n) = 0
			\end{align}
			が従う.他方で
			\begin{align}
				\forall n \in \Natural\, \left(\, 0 < \mu(B_n)\, \right)
			\end{align}
			が成り立つとき,$\epsilon$を任意に与えられた正数とすると,
			各自然数$n$で
			\begin{align}
				A_n \subset K_n \subset B_n \wedge \mu(B_n \backslash A_n) < \frac{\epsilon}{2^n}
			\end{align}
			を満たす$\mathcal{B}$の要素$A_n$と$\mathcal{K}$の要素$K_n$が取れる.ここで
			\begin{align}
				\rightharpoondown \{K_n\}_{n \in \Natural} \eqp \Natural
				\label{thm_finite_intersection_property_and_common_point_property_1}
			\end{align}
			の場合は
			\begin{align}
				\{K_n\}_{n \in \Natural} = \{K_n\}_{n \in N}
			\end{align}
			を満たす自然数$N$が取れて
			\begin{align}
				\bigcap_{n \in N} K_n \subset \bigcap_{n\in\Natural} B_n
			\end{align}
			となるので,
			\begin{align}
				\bigcap_{n \in N} A_n = \emptyset
			\end{align}
			が成立する.
			\begin{align}
				\{K_n\}_{n \in \Natural} \eqp \Natural
				\label{thm_finite_intersection_property_and_common_point_property_2}
			\end{align}
			の場合は
			\begin{align}
				\bigcap_{n\in\Natural} K_n \subset \bigcap_{n\in\Natural} B_n
			\end{align}
			が成り立つから,この場合も
			\begin{align}
				\bigcap_{n \in N} K_n = \emptyset
			\end{align}
			を満たす自然数$N$が取れて,
			\begin{align}
				\bigcap_{n \in M} A_n = \emptyset
			\end{align}
			が成立する.ゆえに(\refeq{thm_finite_intersection_property_and_common_point_property_1})
			の場合も(\refeq{thm_finite_intersection_property_and_common_point_property_2})の場合も
			$N$以上の任意の自然数$m$に対して
			\begin{align}
				B_m \subset \bigcup_{n \in N} (B_n \backslash A_n)
			\end{align}
			が成り立ち
			\begin{align}
				\forall m \in \Natural\,
				\left(\, N \leq m \Longrightarrow \mu(B_m) < \epsilon\, \right)
			\end{align}
			が従う.すなわち
			\begin{align}
				\lim_{n \to \infty} \mu(B_n) = 0
			\end{align}
			となる.
			\QED
		\end{prf}
		
		\begin{screen}
			\begin{thm}[全測度が無限である加法的測度の完全加法性の同値条件]
			\label{thm:equivalent_conditions_of_countable_additivity_for_infinite_measures}
				$X$を集合とし,$\mathcal{B}$を$X$上の加法族とし,$\mu$を$\mathcal{B}$上の加法的正値測度として,
				$\mu(X) = \infty$であるとする.
				このとき,$\mu$が$\mathcal{B}$の上で完全加法性であることと,
				$\mu$が共通点性と次の条件を共に満たすことは同値である.
				\begin{itemize}
					\item $\{B_n\}_{n \in \Natural}$を
						\begin{align}
							\forall n \in \Natural\, (\, B_n \subset B_{n+1}\, )
							\wedge \bigcup_{n \in \Natural} B_n \in \mathcal{B}
						\end{align}
						を満たす$\mathcal{B}$の部分集合とするとき,
						\begin{align}
							\mu\left(\bigcup_{n \in \Natural} B_n\right) = \infty 
							\Longrightarrow \lim_{n \to \infty} \mu(B_n) = \infty.
						\end{align}
				\end{itemize}
			\end{thm}
		\end{screen}
		
		\begin{prf}
			
		\end{prf}
		
		\begin{screen}
			\begin{thm}[$\sigma$-有限な加法的測度の完全加法性の同値条件]
			\label{thm:equivalent_conditions_of_countable_additivity_for_sigma_finite_measures}
				$X$を集合とし,$\mathcal{B}$を$X$上の加法族とし,$\mu$を$\mathcal{B}$上の$\sigma$-有限な加法的正値測度として,
				$\mu(X) = \infty$であるとする.また$\{X_n\}_{n \in \Natural}$を
				\begin{align}
					\forall n \in \Natural\, (\, \mu(X_n) < \infty\, ) \wedge
					\forall n \in \Natural\, (\, X_n \subset X_{n+1}\, ) \wedge
					\bigcup_{n \in \Natural} X_n = X
				\end{align}
				を満たす$\mathcal{B}$の部分集合とする.
				このとき,$\mu$が$\mathcal{B}$の上で完全加法性であることと,
				$\mu$が共通点性と次の条件を共に満たすことは同値である.
				\begin{itemize}
					\item $B$を$\mu(B) = \infty$なる$\mathcal{B}$の要素とするとき
						\begin{align}
							\lim_{n \to \infty} \mu(B \cap X_n) = \infty.
						\end{align}
				\end{itemize}
			\end{thm}
		\end{screen}
		
		\begin{prf}
			
		\end{prf}
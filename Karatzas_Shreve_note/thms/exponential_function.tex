\subsection{指数関数}
	
	$z$を複素数とするとき,
	\begin{align}
		\Natural \ni n \longmapsto \sum_{k=0}^n \frac{1}{k!} \cdot z^k
	\end{align}
	なる複素数列は$\C$で収束する.さらに言えば
	\begin{align}
		\sum_{n=0}^\infty \frac{1}{n!} \cdot z^n
	\end{align}
	は絶対収束する.実際,$z \neq 0$であるとき
	\begin{align}
		\left|\frac{n!}{z^n} \cdot \frac{z^{n+1}}{(n+1)!}\right|
		= \frac{|z|}{n+1}
		\longrightarrow 0\quad (n \longrightarrow \infty)
	\end{align}
	が成り立つので,d'Alembertの収束判定法から絶対収束することが従う.
	$z=0$ならば$1$以上の自然数$n$の項は$0$であるから級数は絶対収束する.
	指数関数とは複素数$z$に対して絶対収束級数
	\begin{align}
		\sum_{n = 0}^\infty \frac{1}{n!} \cdot z^n
	\end{align}
	を対応させる写像として定義される.
	
	\begin{screen}
		\begin{dfn}[指数関数]
			複素数$z$に対して
			\begin{align}
				\sum_{n=0}^\infty \frac{1}{n!} \cdot z^n
			\end{align}
			を対応させる$\C$から$\C$への写像を{\bf 指数関数}\index{しすうかんすう@指数関数}{\bf (exponential function)}と呼び,
			\begin{align}
				\exp
			\end{align}
			と書く.
		\end{dfn}
	\end{screen}
	
	複素数$z$に対して
	\begin{align}
		\exp{(z)}
	\end{align}
	の代わりに
	\begin{align}
		e^z
	\end{align}
	とも書く.
	
	\begin{screen}
		\begin{thm}[指数法則]
			$a$と$b$を複素数とするとき
			\begin{align}
				e^a \cdot e^b = e^{a + b}.
			\end{align}
		\end{thm}
	\end{screen}
	
	\begin{sketch}
		定理\ref{thm:convolution_of_absolutely_convergent_series}より
		\begin{align}
			\left(\sum_{n=0}^\infty \frac{1}{n!} \cdot a^n\right) \cdot \left(\sum_{k=0}^\infty \frac{1}{k!} \cdot b^k\right)
			&= \sum_{n=0}^\infty \sum_{k=0}^n \frac{1}{k!} \cdot a^k \cdot \frac{1}{(n-k)!} \cdot b^{n-k} \\
			&= \sum_{n=0}^\infty \frac{1}{n!} \cdot \left( \sum_{k=0}^n \frac{n!}{k! \cdot (n-k)!} \cdot a^k \cdot b^{n-k} \right) \\
			&= \sum_{n=0}^\infty \frac{1}{n!} \cdot (a+b)^n
		\end{align}
		が成り立つ.
		\QED
	\end{sketch}
	
	\begin{screen}
		\begin{thm}[指数関数は$0$を取らない]
			任意の複素数$z$に対して
			\begin{align}
				e^z \neq 0.
			\end{align}
		\end{thm}
	\end{screen}
	
	\begin{sketch}
		$z$を複素数とすれば
		\begin{align}
			e^z \cdot e^{-z} = e^0 = 1
		\end{align}
		が成り立つ.つまり$e^{-z}$という逆元が取れるので
		\begin{align}
			e^z \neq 0
		\end{align}
		である.
		\QED
	\end{sketch}
	
	\begin{screen}
		\begin{thm}[指数関数は実数上で単調増大かつ一対一対応]
			$\exp$を実数上に制限した写像
			\begin{align}
				\R \ni x \longmapsto e^x
			\end{align}
			は単調増大かつ$\R$から$\R_+$への全単射である.
		\end{thm}
	\end{screen}
	
	
	
	\begin{screen}
		\begin{dfn}[三角関数]
			複素数$z$に対して
			\begin{align}
				\frac{e^{\isym \cdot z} + e^{-\isym \cdot z}}{2}
			\end{align}
			を対応させる$\C$から$\C$への写像を{\bf 余弦}\index{よげん@余弦}{\bf (cosine)}と呼び,
			\begin{align}
				\cos
			\end{align}
			と書く.複素数$z$に対して
			\begin{align}
				\frac{e^{\isym \cdot z} - e^{-\isym \cdot z}}{2 \cdot i}
			\end{align}
			を対応させる$\C$から$\C$への写像を{\bf 正弦}\index{せいげん@正弦}{\bf (sine)}と呼び,
			\begin{align}
				\sin
			\end{align}
			と書く.
		\end{dfn}
	\end{screen}
	
	余弦関数の二乗は
	\begin{align}
		(\cos{z})^2
	\end{align}
	ではなく
	\begin{align}
		\cos^2{z}
	\end{align}
	と書く.同様に正弦関数の二乗も
	\begin{align}
		\sin^2{z}
	\end{align}
	と書く.
	
	\begin{screen}
		\begin{thm}[余弦と正弦の二乗和は$1$]
			$z$を複素数とするとき
			\begin{align}
				\cos^2{z} + \sin^2{z} = 1.
			\end{align}
		\end{thm}
	\end{screen}
	
	\begin{sketch}
		$z$を複素数とする.余弦の定義より
		\begin{align}
			\cos^2{z} = \frac{e^{2 \cdot \isym \cdot z} + 2 + e^{-2 \cdot \isym \cdot z}}{4}
		\end{align}
		が成り立ち,正弦の定義より
		\begin{align}
			\sin^2{z} = -\frac{e^{2 \cdot \isym \cdot z} - 2 + e^{-2 \cdot \isym \cdot z}}{4}
		\end{align}
		が成り立つので,
		\begin{align}
			\cos^2{z} + \sin^2{z} = 1
		\end{align}
		が得られる.
		\QED
	\end{sketch}
	
	$\theta$を実数とすれば
	\begin{align}
		e^{\isym \cdot \theta} = \cos{\theta} + \isym \cdot \sin{\theta}
	\end{align}
	が成立するが,この関係を{\bf Eulerの関係式}と呼ぶ.$z$を複素数とすれば
	\begin{align}
		z = x + \isym \cdot y
	\end{align}
	を満たす実数$x$と$y$が取れるが,このとき
	\begin{align}
		e^z = e^x \cdot e^{\isym \cdot y}
	\end{align}
	が成り立ち,
	\begin{align}
		\left|e^z\right| = \left|e^x\right| \cdot \left|e^{\isym \cdot y}\right| = e^x
	\end{align}
	かつ
	\begin{align}
		e^z = e^x \cdot (\cos{y} + \isym \cdot \sin{y}) = \left|e^z\right| \cdot (\cos{y} + \isym \cdot \sin{y})
	\end{align}
	が成り立つ.後述することだが,$w$を任意に与えられた$0$でない複素数とすれば
	\begin{align}
		w = \exp{(z)}
	\end{align}
	を満たす複素数$z$が取れるので,すなわち
	\begin{align}
		w = |w| \cdot (\cos{y} + \isym \cdot \sin{y})
	\end{align}
	を満たす実数$y$が取れる.これを複素数の{\bf 極形式}\index{きょくけいしき@極形式}{\bf (polar form)}と呼び,
	この$y$を$w$の{\bf 偏角}\index{へんかく@偏角}{\bf (argument)}と呼ぶ.ただし$y$は一意に定まるものではない.
	
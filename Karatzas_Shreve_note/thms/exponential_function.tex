\subsection{指数関数}
	
	\begin{screen}
		\begin{dfn}[指数関数]
			複素数$z$に対して
			\begin{align}
				\sum_{n=0}^\infty \frac{1}{n!} \cdot z^n
			\end{align}
			を対応させる$\C$から$\C$への写像を{\bf 指数関数}\index{しすうかんすう@指数関数}{\bf (exponential function)}と呼び,
			\begin{align}
				\exp
			\end{align}
			と書く.
		\end{dfn}
	\end{screen}
	
	複素数$z$に対して
	\begin{align}
		\exp{(z)}
	\end{align}
	の代わりに
	\begin{align}
		e^z
	\end{align}
	とも書く.
	
	\begin{screen}
		\begin{thm}[指数法則]
			$a$と$b$を複素数とするとき
			\begin{align}
				e^{a + b} = e^a \cdot e^b.
			\end{align}
		\end{thm}
	\end{screen}
	
	\begin{screen}
		\begin{dfn}[微分]
			$\Omega$を$\C$の部分集合とし,$a$を$\Omega$の要素とし,$f$を$\Omega$上の複素数値関数とする.このとき
			\begin{align}
				\forall \epsilon \in \R_+\, \exists \delta \in \R_+\, \forall h \in \C\,
				\left[\, |h| < \delta \wedge a + h \in \Omega \Longrightarrow 
				|f(a+h) - f(a) - \alpha \cdot h| < \epsilon \cdot |h|\, \right]
			\end{align}
			を満たす複素数$\alpha$が取れるなら,$f$は$a$で{\bf 微分可能である}\index{びぶんかのう@微分可能}{\bf (differentiable)}という.
		\end{dfn}
	\end{screen}
	
	\begin{screen}
		\begin{thm}[指数関数は微分しても不変]
			$z$を複素数とするとき
			\begin{align}
				\exp' = \exp
			\end{align}
		\end{thm}
	\end{screen}
	
	\begin{screen}
		\begin{thm}[指数関数は$0$を取らない]
			任意の複素数$z$に対して
			\begin{align}
				e^z \neq 0.
			\end{align}
		\end{thm}
	\end{screen}
	
	\begin{sketch}
		$z$を複素数とすれば
		\begin{align}
			e^z \cdot e^{-z} = e^0 = 1
		\end{align}
		が成り立つ.ゆえに
		\begin{align}
			e^z \neq 0
		\end{align}
		である.
		\QED
	\end{sketch}
	
	\begin{screen}
		\begin{thm}[指数関数は実数上で単調増大かつ一対一対応]
			$\exp$を実数上に制限した写像
			\begin{align}
				\R \ni x \longmapsto e^x
			\end{align}
			は単調増大かつ$\R$から$\R_+$への全単射である.
		\end{thm}
	\end{screen}
	
	
	
	\begin{screen}
		\begin{dfn}[三角関数]
			複素数$z$に対して
			\begin{align}
				\frac{e^{\isym \cdot z} + e^{-\isym \cdot z}}{2}
			\end{align}
			を対応させる$\C$から$\C$への写像を{\bf 余弦}\index{よげん@余弦}{\bf (cosine)}と呼び,
			\begin{align}
				\cos
			\end{align}
			と書く.複素数$z$に対して
			\begin{align}
				\frac{e^{\isym \cdot z} - e^{-\isym \cdot z}}{2 \cdot i}
			\end{align}
			を対応させる$\C$から$\C$への写像を{\bf 正弦}\index{せいげん@正弦}{\bf (sine)}と呼び,
			\begin{align}
				\sin
			\end{align}
			と書く.
		\end{dfn}
	\end{screen}
	
	余弦関数の二乗は
	\begin{align}
		(\cos{z})^2
	\end{align}
	ではなく
	\begin{align}
		\cos^2{z}
	\end{align}
	と書く.同様に正弦関数の二乗も
	\begin{align}
		\sin^2{z}
	\end{align}
	と書く.
	
	\begin{screen}
		\begin{thm}[余弦と正弦の二乗和は$1$]
			$z$を複素数とするとき
			\begin{align}
				\cos^2{z} + \sin^2{z} = 1.
			\end{align}
		\end{thm}
	\end{screen}
	
	\begin{sketch}
		$z$を複素数とする.余弦の定義より
		\begin{align}
			\cos^2{z} = \frac{e^{2 \cdot \isym \cdot z} + 2 + e^{-2 \cdot \isym \cdot z}}{4}
		\end{align}
		が成り立ち,正弦の定義より
		\begin{align}
			\sin^2{z} = -\frac{e^{2 \cdot \isym \cdot z} - 2 + e^{-2 \cdot \isym \cdot z}}{4}
		\end{align}
		が成り立つので,
		\begin{align}
			\cos^2{z} + \sin^2{z} = 1
		\end{align}
		が得られる.
		\QED
	\end{sketch}
	
	$\theta$を実数とすれば
	\begin{align}
		e^{\isym \cdot \theta} = \cos{\theta} + \isym \cdot \sin{\theta}
	\end{align}
	が成立するが,この関係を{\bf Eulerの関係式}と呼ぶ.$z$を複素数とすれば
	\begin{align}
		z = x + \isym \cdot y
	\end{align}
	を満たす実数$x$と$y$が取れるが,このとき
	\begin{align}
		e^z = e^x \cdot e^{\isym \cdot y}
	\end{align}
	が成り立ち,
	\begin{align}
		\left|e^z\right| = \left|e^x\right| \cdot \left|e^{\isym \cdot y}\right| = e^x
	\end{align}
	かつ
	\begin{align}
		e^z = e^x \cdot (\cos{y} + \isym \cdot \sin{y}) = \left|e^z\right| \cdot (\cos{y} + \isym \cdot \sin{y})
	\end{align}
	が成り立つ.後述することだが,$w$を任意に与えられた$0$でない複素数とすれば
	\begin{align}
		w = \exp{(z)}
	\end{align}
	を満たす複素数$z$が取れるので,すなわち
	\begin{align}
		w = |w| \cdot (\cos{y} + \isym \cdot \sin{y})
	\end{align}
	を満たす実数$y$が取れる.これを複素数の{\bf 極形式}\index{きょくけいしき@極形式}{\bf (polar form)}と呼び,
	この$y$を$w$の{\bf 偏角}\index{へんかく@偏角}{\bf (argument)}と呼ぶ.ただし$y$は一意に定まるものではない.
	
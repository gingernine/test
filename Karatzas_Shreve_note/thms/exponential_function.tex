\section{指数関数}
	
	$z$を複素数とするとき,
	\begin{align}
		\sum_{n=0}^\infty \frac{1}{n!} \cdot z^n
	\end{align}
	は$\C$で絶対収束する.実際,$z \neq 0$ならば
	\begin{align}
		\left|\frac{n!}{z^n} \cdot \frac{z^{n+1}}{(n+1)!}\right|
		= \frac{|z|}{n+1}
		\longrightarrow 0\quad (n \longrightarrow \infty)
	\end{align}
	が成り立つので,d'Alembertの収束判定法から級数は絶対収束する.
	$z=0$の場合も初項を除く全ての項は$0$であるから級数は絶対収束する.
	指数関数とは複素数$z$に対して
	\begin{align}
		\sum_{n = 0}^\infty \frac{1}{n!} \cdot z^n
	\end{align}
	を対応させる写像として定義される.
	
	\begin{screen}
		\begin{dfn}[指数関数]
			複素数$z$に対して
			\begin{align}
				\sum_{n=0}^\infty \frac{1}{n!} \cdot z^n
			\end{align}
			を対応させる$\C$から$\C$への写像を{\bf 指数関数}\index{しすうかんすう@指数関数}{\bf (exponential function)}と呼び,
			\begin{align}
				\exp
			\end{align}
			と書く.特に
			\begin{align}
				e \defeq \exp{1} = \sum_{n=0}^\infty \frac{1}{n!}
			\end{align}
			で定める実数$e$を{\bf Napier数}\index{Napier数}{\bf (Napier's constant)}と呼ぶ.
		\end{dfn}
	\end{screen}
	
	\begin{screen}
		\begin{thm}[指数法則]\label{thm:law_of_exponentiation_for_exponential_function}
			$a$と$b$を複素数とするとき
			\begin{align}
				\exp{a} \cdot \exp{b} = \exp{(a + b)}.
			\end{align}
		\end{thm}
	\end{screen}
	
	\begin{sketch}
		定理\ref{thm:convolution_of_absolutely_convergent_series}より
		\begin{align}
			\left(\sum_{n=0}^\infty \frac{1}{n!} \cdot a^n\right) \cdot \left(\sum_{n=0}^\infty \frac{1}{n!} \cdot b^n\right)
			&= \sum_{n=0}^\infty \sum_{k=0}^n \frac{1}{k!} \cdot a^k \cdot \frac{1}{(n-k)!} \cdot b^{n-k} \\
			&= \sum_{n=0}^\infty \frac{1}{n!} \cdot \left( \sum_{k=0}^n \frac{n!}{k! \cdot (n-k)!} \cdot a^k \cdot b^{n-k} \right) \\
			&= \sum_{n=0}^\infty \frac{1}{n!} \cdot (a+b)^n
		\end{align}
		が成り立つ.
		\QED
	\end{sketch}
	
	\begin{screen}
		\begin{thm}[有理数に対する指数関数の値は$e$の有理数乗で表せる]
			$r$を有理数とするとき
			\begin{align}
				\exp{r} = e^{r}.
			\end{align}
		\end{thm}
	\end{screen}
	
	\begin{sketch}\mbox{}
		\begin{description}
			\item[第一段]
				まず$r$が整数である場合に示す.
				\begin{align}
					\exp{0} = 1 = e^0
				\end{align}
				が成り立つ.また$n$を自然数とするとき,
				\begin{align}
					\exp{n} = e^n
				\end{align}
				が成り立っているならば
				\begin{align}
					\exp{(n+1)} = \exp{n} \cdot \exp{1} = e^n \cdot e = e^{n+1}
				\end{align}
				も成り立つ.よって数学的帰納法の原理より任意の自然数$r$で
				\begin{align}
					\exp{r} = e^{r}
				\end{align}
				が成り立つ.$r$を負の整数とすると
				\begin{align}
					\exp{r} = (\exp{(-r)})^{-1} = (e^{-r})^{-1} = e^{(-r) \cdot (-1)} = e^{r}
				\end{align}
				が成り立つ.
			
			\item[第二段]
				$z$を複素数とし,$n$を自然数とするとき,
				\begin{align}
					(\exp{z})^{n} = \exp{(n \cdot z)}
				\end{align}
				が成り立つことを示す.まず
				\begin{align}
					(\exp{z})^{0} = 1 = \exp{0} = \exp{(0 \cdot z)}
				\end{align}
				が成り立つ.また$n$を自然数とするとき,
				\begin{align}
					(\exp{z})^{n} = \exp{(n \cdot z)}
				\end{align}
				が成り立っているならば
				\begin{align}
					(\exp{z})^{n+1} = (\exp{z})^{n} \cdot \exp{z} = \exp{(n \cdot z)} \cdot \exp{z}
					= \exp{n \cdot z + z}
					= \exp{(n+1) \cdot z}
				\end{align}
				も成り立つ.よって数学的帰納法の原理より任意の自然数$r$で
				\begin{align}
					(\exp{z})^{n} = \exp{(n \cdot z)}
				\end{align}
				が成り立つ.
				
			\item[第三段]
				$m$を正の自然数とするとき,前段の結果より
				\begin{align}
					(\exp{(m^{-1})})^{m} = \exp{(m \cdot m^{-1})} = \exp{1} = e
				\end{align}
				が成り立つので
				\begin{align}
					\exp{(m^{-1})} = \sqrt[m]{e}
				\end{align}
				が従う.いま$r$を有理数とすると,
				\begin{align}
					r = n \cdot m^{-1}
				\end{align}
				を満たす整数$n$と正の自然数$m$が取れるが,このとき
				\begin{align}
					\exp{r} = \exp{(n \cdot m^{-1})}
					= (\exp{(m^{-1})})^{n}
					= (\sqrt[m]{e})^{n}
					= e^{n \cdot m^{-1}}
					= e^r
				\end{align}
				が成立する.
				\QED
		\end{description}
	\end{sketch}
	
	上の結果に倣って,以降では複素数$z$に対して
	\begin{align}
		\exp{z}
	\end{align}
	を
	\begin{align}
		e^{z}
	\end{align}
	とも書く.
	
	\begin{screen}
		\begin{thm}[$e$のマイナス乗は逆元]\label{thm:inversion_of_exp_z_is_exp_minus_z}
			$z$を複素数とすると,$e^z$の乗法に関する逆元は$e^{-z}$である:
			\begin{align}
				(e^z)^{-1} = e^{-z}.
			\end{align}
			特に$\exp$は$0$を取り得ない.
		\end{thm}
	\end{screen}
	
	\begin{sketch}
		$z$を複素数とすれば,
		\begin{align}
			z + (-z) = (-z) + z = 0
		\end{align}
		かつ
		\begin{align}
			e^0 = 1
		\end{align}
		であるから
		\begin{align}
			e^z \cdot e^{-z} = e^{-z} \cdot e^z = 1
		\end{align}
		が成立する.つまり$e^{z}$の乗法に関する逆元は$e^{-z}$であって,またこのため
		\begin{align}
			e^{z} \neq 0
		\end{align}
		も満たされる.
		\QED
	\end{sketch}
	
	\begin{screen}
		\begin{thm}[指数関数は各点で微分可能]
			$z$を任意に与えられた複素数とすれば,$\exp$は$z$で微分可能であってその微分係数は
			\begin{align}
				\exp{z}
			\end{align}
			である.
		\end{thm}
	\end{screen}
	
	\begin{sketch}
		$z$を任意に与えられた複素数とする.指数法則より,任意の複素数$h$に対して
		\begin{align}
			|\exp{(z+h)} - \exp{z} - h \cdot \exp{z}|
			&= |\exp{z}| \cdot |\exp{h} - 1 - h| \\
			&= |\exp{z}| \cdot \left|\sum_{n=1}^\infty \frac{1}{n!} \cdot h^n - h\right|
		\end{align}
		が成り立つ.ここで
		\begin{align}
			\sum_{n=1}^\infty \frac{1}{n!} \cdot h^{n-1}
		\end{align}
		は絶対収束級数であるから,定理\ref{thm:linearity_of_absolutely_convergent_series}より
		\begin{align}
			\sum_{n=1}^\infty \frac{1}{n!} \cdot h^{n}
			= h \cdot \sum_{n=1}^\infty \frac{1}{n!} \cdot h^{n-1}
			= h \cdot \sum_{n=2}^\infty \frac{1}{n!} \cdot h^{n-1} + h
		\end{align}
		が成り立つ.すなわち
		\begin{align}
			|\exp{(z+h)} - \exp{z} - h \cdot \exp{z}|
			&= |\exp{z}| \cdot \left|\sum_{n=1}^\infty \frac{1}{n!} \cdot h^n - h\right| \\
			&= |\exp{z}| \cdot |h| \cdot \left|\sum_{n=2}^\infty \frac{1}{n!} \cdot h^{n-1}\right|
		\end{align}
		が成り立つ.いま$\epsilon$を任意に与えられた正の実数とすると,
		\begin{align}
			0 < |h| < \min\left\{\epsilon/|\exp{z}|,1/2\right\}
		\end{align}
		ならば
		\begin{align}
			|\exp{(z+h)} - \exp{z} - h \cdot \exp{z}|
			&= |\exp{z}| \cdot |h| \cdot \left|\sum_{n=2}^\infty \frac{1}{n!} \cdot h^{n-1}\right| \\
			&\leq |\exp{z}| \cdot |h| \cdot \sum_{n=2}^\infty \cdot |h|^{n-1} \\
			&\leq |\exp{z}| \cdot |h| \cdot \frac{|h|}{1-|h|} \\
			&< 2 \cdot \epsilon \cdot |h|
		\end{align}
		が成立する.ゆえに$\exp$は$z$で微分可能であって,その微分係数は
		\begin{align}
			\exp{z}
		\end{align}
		である.
		\QED
	\end{sketch}
	
	指数関数は$\C$全体で定義されて,かつ$\C$の任意の要素で微分可能である.
	このように$\C$上で定義された$\C$値関数で,$\C$の各点で微分可能なものを
	{\bf 整関数}\index{せいかんすう@整関数}{\bf (entire function)}と呼ぶ.
	
	いま$y$を実数とすると,任意の自然数$n$で
	\begin{align}
		\overline{(\isym \cdot y)^n} = (-\isym \cdot y)^n
	\end{align}
	が成立する.従って定理\ref{thm:conjugate_of_absolutely_convergent_series_is_series_of_conjugate}より
	\begin{align}
		e^{-\isym \cdot y} = \overline{e^{\isym \cdot y}}
	\end{align}
	が成り立ち,
	\begin{align}
		|e^{\isym \cdot y}|^2 
		= e^{\isym \cdot y} \cdot \overline{e^{\isym \cdot y}}
		= e^{\isym \cdot y} \cdot e^{-\isym \cdot y}
		= e^{0}
		= 1
	\end{align}
	が従う.これで次の主張を得た.
	
	\begin{screen}
		\begin{thm}[$e$の純虚数乗の絶対値は$1$]
			$y$を実数とするとき
			\begin{align}
				e^{-\isym \cdot y} = \overline{e^{\isym \cdot y}}
			\end{align}
			が成立し,特に
			\begin{align}
				|e^{\isym \cdot y}| = 1.
			\end{align}
		\end{thm}
	\end{screen}
	
	\begin{screen}
		\begin{thm}[指数関数は実数上で単調増大かつ一対一対応]\label{thm:real_valued_exponential_function}
			$\exp$を実数上に制限した写像
			\begin{align}
				\R \ni x \longmapsto e^x
			\end{align}
			は単調増大かつ$\R$から$\R_+$への全単射である.
		\end{thm}
	\end{screen}
	
	\begin{sketch}
		指数関数の定義より
		\begin{align}
			e^0 = 1
		\end{align}
		が従う.また$x$を正の実数とすれば
		\begin{align}
			e^{x} \in \R_+
		\end{align}
		が従う.$x$を負の実数とするとき,$-x$は正の実数であるから
		\begin{align}
			e^{-x} \in \R_+
		\end{align}
		が成り立ち,他方で
		\begin{align}
			e^{x} = 1/e^{-x}
		\end{align}
		であるから
		\begin{align}
			e^{x} \in \R_+
		\end{align}
		が従う.ゆえに$\exp$は$\R$から$\R_+$への写像である.また$x$と$y$を
		\begin{align}
			x < y
		\end{align}
		を満たす実数とすれば,定理\ref{thm:linearity_of_absolutely_convergent_series}より
		\begin{align}
			e^{y} - e^{x} = \sum_{n=0}^{\infty} \frac{1}{n!} \cdot (y^n - x^n) 
		\end{align}
		が成り立ち,
		\begin{align}
			0 < y - x < \sum_{n=0}^{\infty} \frac{1}{n!} \cdot (y^n - x^n) 
		\end{align}
		も成り立つから$\exp$は単調増大である.任意の正の実数$x$に対して
		\begin{align}
			x < \sum_{n=0}^{\infty} \frac{1}{n!} \cdot x^n = e^{x}
		\end{align}
		が成り立つので
		\begin{align}
			e^x \longrightarrow \infty \quad (x \longrightarrow \infty)
		\end{align}
		が成り立ち,
		\begin{align}
			e^{-x} = 1/e^{x}
		\end{align}
		であるから
		\begin{align}
			e^x \longrightarrow 0 \quad (x \longrightarrow -\infty)
		\end{align}
		も満たされる.次に実数$x$に対して
		\begin{align}
			e^x = 1 \Longrightarrow x=0
			\label{fom:thm_real_valued_exponential_function}
		\end{align}
		が成り立つことを示す.実際,$0 < x$であれば
		\begin{align}
			1 < 1 + x < \sum_{n=0}^{\infty} \frac{1}{n!} \cdot x^n = e^{x}
		\end{align}
		が成り立ち,$x < 0$であれば
		\begin{align}
			1 < e^{-x}
		\end{align}
		から
		\begin{align}
			e^{x} < 1
		\end{align}
		が従うので(\refeq{fom:thm_real_valued_exponential_function})が成立する.よって
		\begin{align}
			e^x = e^y
		\end{align}
		ならば
		\begin{align}
			e^{x - y} = 1
		\end{align}
		となって
		\begin{align}
			x = y
		\end{align}
		が従う.ゆえに$\exp$は単射である.また$y$を任意に与えられた正の実数とすると,
		$1 < y$ならば
		\begin{align}
			e^{0} < y < e^{y}
		\end{align}
		が成り立つので,$\exp$の連続性と中間値の定理から
		\begin{align}
			y = e^{x}
		\end{align}
		を満たす実数$x$が取れる.$y < 1$ならば
		\begin{align}
			\frac{1}{y} = e^{x}
		\end{align}
		を満たす実数$x$が取れるので
		\begin{align}
			y = e^{-x}
		\end{align}
		が成立する.ゆえに$\exp$は全射である.
		\QED
	\end{sketch}
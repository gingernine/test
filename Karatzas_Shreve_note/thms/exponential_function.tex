\subsection{指数関数}
	
	$z$を複素数とするとき,
	\begin{align}
		\Natural \ni n \longmapsto \sum_{k=0}^n \frac{1}{k!} \cdot z^k
	\end{align}
	なる複素数列は$\C$で収束する.さらに言えば
	\begin{align}
		\sum_{n=0}^\infty \frac{1}{n!} \cdot z^n
	\end{align}
	は絶対収束する.実際,$z \neq 0$であるとき
	\begin{align}
		\left|\frac{n!}{z^n} \cdot \frac{z^{n+1}}{(n+1)!}\right|
		= \frac{|z|}{n+1}
		\longrightarrow 0\quad (n \longrightarrow \infty)
	\end{align}
	が成り立つので,d'Alembertの収束判定法から絶対収束することが従う.
	$z=0$ならば$1$以上の自然数$n$の項は$0$であるから級数は絶対収束する.
	指数関数とは複素数$z$に対して絶対収束級数
	\begin{align}
		\sum_{n = 0}^\infty \frac{1}{n!} \cdot z^n
	\end{align}
	を対応させる写像として定義される.
	
	\begin{screen}
		\begin{dfn}[指数関数]
			複素数$z$に対して
			\begin{align}
				\sum_{n=0}^\infty \frac{1}{n!} \cdot z^n
			\end{align}
			を対応させる$\C$から$\C$への写像を{\bf 指数関数}\index{しすうかんすう@指数関数}{\bf (exponential function)}と呼び,
			\begin{align}
				\exp
			\end{align}
			と書く.
		\end{dfn}
	\end{screen}
	
	複素数$z$に対して
	\begin{align}
		\exp{(z)}
	\end{align}
	の代わりに
	\begin{align}
		e^z
	\end{align}
	とも書く.
	
	\begin{screen}
		\begin{thm}[指数法則]
			$a$と$b$を複素数とするとき
			\begin{align}
				e^a \cdot e^b = e^{a + b}.
			\end{align}
		\end{thm}
	\end{screen}
	
	\begin{sketch}
		定理\ref{thm:convolution_of_absolutely_convergent_series}より
		\begin{align}
			\left(\sum_{n=0}^\infty \frac{1}{n!} \cdot a^n\right) \cdot \left(\sum_{k=0}^\infty \frac{1}{k!} \cdot b^k\right)
			&= \sum_{n=0}^\infty \sum_{k=0}^n \frac{1}{k!} \cdot a^k \cdot \frac{1}{(n-k)!} \cdot b^{n-k} \\
			&= \sum_{n=0}^\infty \frac{1}{n!} \cdot \left( \sum_{k=0}^n \frac{n!}{k! \cdot (n-k)!} \cdot a^k \cdot b^{n-k} \right) \\
			&= \sum_{n=0}^\infty \frac{1}{n!} \cdot (a+b)^n
		\end{align}
		が成り立つ.
		\QED
	\end{sketch}
	
	$z$を複素数とすれば,
	\begin{align}
		z + (-z) = (-z) + z = 0
	\end{align}
	かつ
	\begin{align}
		e^0 = 1
	\end{align}
	であるから
	\begin{align}
		e^z \cdot e^{-z} = e^{-z} \cdot e^z = 1
	\end{align}
	が成立する.従って次の定理が得られる.
	
	\begin{screen}
		\begin{thm}[$e$のマイナス乗は逆元]\label{thm:inversion_of_exp_z_is_exp_minus_z}
			$z$を複素数とすると,$e^z$の乗法に関する逆元は$e^{-z}$である:
			\begin{align}
				(e^z)^{-1} = e^{-z}.
			\end{align}
		\end{thm}
	\end{screen}
	
	\begin{screen}
		\begin{thm}[指数関数は実数上で単調増大かつ一対一対応]\label{thm:real_valued_exponential_function}
			$\exp$を実数上に制限した写像
			\begin{align}
				\R \ni x \longmapsto e^x
			\end{align}
			は単調増大かつ$\R$から$\R_+$への全単射である.そして
			\begin{align}
				e^x \longrightarrow \infty \quad (x \longrightarrow \infty)
			\end{align}
			かつ
			\begin{align}
				e^x \longrightarrow 0 \quad (x \longrightarrow -\infty)
			\end{align}
			も成り立つ.
		\end{thm}
	\end{screen}
	
	\begin{sketch}
		指数関数の定義より,$x = 0$であれば
		\begin{align}
			e^0 = 1
		\end{align}
		が従う.また$x$が正の実数であれば
		\begin{align}
			e^{x} \in \R_+
		\end{align}
		が従う.$x$が負の実数であるとき,$-x$は正の実数であるから
		\begin{align}
			e^{-x} \in \R_+
		\end{align}
		が成り立ち,他方で
		\begin{align}
			e^{x} = 1/e^{-x}
		\end{align}
		であるから
		\begin{align}
			e^{x} \in \R_+
		\end{align}
		が従う.ゆえに$\exp$は$\R$から$\R_+$への写像である.また$x$と$y$を
		\begin{align}
			x < y
		\end{align}
		を満たす実数とすれば,定理\ref{thm:linearity_of_absolutely_convergent_series}より
		\begin{align}
			e^{y} - e^{x} = \sum_{n=0}^{\infty} \frac{1}{n!} \cdot (y^n - x^n) 
		\end{align}
		が成り立ち,
		\begin{align}
			0 < y - x < \sum_{n=0}^{\infty} \frac{1}{n!} \cdot (y^n - x^n) 
		\end{align}
		も成り立つから$\exp$は単調増大である.任意の正の実数$x$に対して
		\begin{align}
			x < \sum_{n=0}^{\infty} \frac{1}{n!} \cdot x^n = e^{x}
		\end{align}
		が成り立つので
		\begin{align}
			e^x \longrightarrow \infty \quad (x \longrightarrow \infty)
		\end{align}
		が成り立ち,
		\begin{align}
			e^{-x} = 1/e^{x}
		\end{align}
		であるから
		\begin{align}
			e^x \longrightarrow 0 \quad (x \longrightarrow -\infty)
		\end{align}
		も満たされる.次に実数$x$に対して
		\begin{align}
			e^x = 1 \Longrightarrow x=0
			\label{fom:thm_real_valued_exponential_function}
		\end{align}
		が成り立つことを示す.実際,$0 < x$であれば
		\begin{align}
			1 < 1 + x < \sum_{n=0}^{\infty} \frac{1}{n!} \cdot x^n = e^{x}
		\end{align}
		が成り立ち,$x < 0$であれば
		\begin{align}
			1 < e^{-x}
		\end{align}
		から
		\begin{align}
			e^{x} < 1
		\end{align}
		が従うので(\refeq{fom:thm_real_valued_exponential_function})が成立する.よって
		\begin{align}
			e^x = e^y
		\end{align}
		ならば
		\begin{align}
			e^{x - y} = 1
		\end{align}
		となって
		\begin{align}
			x = y
		\end{align}
		が従う.ゆえに$\exp$は単射である.また$y$を任意に与えられた正の実数とすると,
		$1 < y$ならば
		\begin{align}
			e^{0} < y < e^{y}
		\end{align}
		が成り立つので,$\exp$の連続性と中間値の定理から
		\begin{align}
			y = e^{x}
		\end{align}
		を満たす実数$x$が取れる.$y < 1$ならば
		\begin{align}
			\frac{1}{y} = e^{x}
		\end{align}
		を満たす実数$x$が取れるので
		\begin{align}
			y = e^{-x}
		\end{align}
		が成立する.ゆえに$\exp$は全射である.
		\QED
	\end{sketch}
	
	\begin{screen}
		\begin{dfn}[三角関数]
			複素数$z$に対して
			\begin{align}
				\frac{e^{\isym \cdot z} + e^{-\isym \cdot z}}{2}
			\end{align}
			を対応させる$\C$から$\C$への写像を{\bf 余弦}\index{よげん@余弦}{\bf (cosine)}と呼び,
			\begin{align}
				\cos
			\end{align}
			と書く.複素数$z$に対して
			\begin{align}
				\frac{e^{\isym \cdot z} - e^{-\isym \cdot z}}{2 \cdot \isym}
			\end{align}
			を対応させる$\C$から$\C$への写像を{\bf 正弦}\index{せいげん@正弦}{\bf (sine)}と呼び,
			\begin{align}
				\sin
			\end{align}
			と書く.
		\end{dfn}
	\end{screen}
	
	余弦関数の二乗は
	\begin{align}
		(\cos{z})^2
	\end{align}
	ではなく
	\begin{align}
		\cos^2{z}
	\end{align}
	と書く.同様に正弦関数の二乗も
	\begin{align}
		\sin^2{z}
	\end{align}
	と書く.
	
	\begin{screen}
		\begin{thm}[余弦と正弦の二乗和は$1$]
			$z$を複素数とするとき
			\begin{align}
				\cos^2{z} + \sin^2{z} = 1.
			\end{align}
		\end{thm}
	\end{screen}
	
	\begin{sketch}
		$z$を複素数とする.余弦の定義より
		\begin{align}
			\cos^2{z} = \frac{e^{2 \cdot \isym \cdot z} + 2 + e^{-2 \cdot \isym \cdot z}}{4}
		\end{align}
		が成り立ち,正弦の定義より
		\begin{align}
			\sin^2{z} = -\frac{e^{2 \cdot \isym \cdot z} - 2 + e^{-2 \cdot \isym \cdot z}}{4}
		\end{align}
		が成り立つので,
		\begin{align}
			\cos^2{z} + \sin^2{z} = 1
		\end{align}
		が得られる.
		\QED
	\end{sketch}
	
	\begin{screen}
		\begin{thm}[正弦の導関数は余弦,余弦の導関数はマイナス正弦]
		\label{thm:derivatives_of_trigonometric_functions}
		\end{thm}
	\end{screen}
	
	$z$を複素数とすれば
	\begin{align}
		e^{\isym \cdot z} = \cos{z} + \isym \cdot \sin{z}
	\end{align}
	が成立するが,この関係を{\bf Eulerの関係式}と呼ぶ.
	
	$\cos$と$\sin$も$\exp$と同様に実数に対しては実数を対応させる写像である.
	\begin{align}
		\cos{0} = 1
	\end{align}
	かつ
	\begin{align}
		\cos{2} < 0
	\end{align}
	であり,かつ
	\begin{align}
		\R \ni t \longmapsto \cos{t}
	\end{align}
	は連続写像であるから,中間値の定理より
	\begin{align}
		\cos{t} = 0
	\end{align}
	を満たす実数$t$が取れる.ゆえに
	\begin{align}
		\Set{t \in \R_+}{\cos{t} = 0}
	\end{align}
	は空ではないので,$\R$においてその下限が存在する.
	
	\begin{screen}
		\begin{dfn}[円周率]
			\begin{align}
				\pi \defeq 2 \cdot \inf{}{\Set{t \in \R_+}{\cos{t} = 0}}
			\end{align}
			により定める実数$\pi$を{\bf 円周率}\index{えんしゅうりつ@円周率}{\bf (pi)}と呼ぶ.
		\end{dfn}
	\end{screen}
	
	円周率の定め方と$\cos$の連続性から
	\begin{align}
		\cos{\frac{\pi}{2}} = 0
	\end{align}
	が成り立ち,また
	\begin{align}
		0 < t < \frac{\pi}{2}
	\end{align}
	を満たす実数$t$に対しては
	\begin{align}
		0 < \cos{t}
	\end{align}
	が成立する.他方でEulerの関係式から
	\begin{align}
		\sin^2{\frac{\pi}{2}} = 1
	\end{align}
	が従う.また平均値の定理より
	\begin{align}
		0 < \xi < \frac{\pi}{2}
	\end{align}
	かつ
	\begin{align}
		\sin{\frac{\pi}{2}}
		= \frac{2}{\pi} \cdot \sin'{\xi}
	\end{align}
	を満たす実数$\xi$が取れて,定理\ref{thm:derivatives_of_trigonometric_functions}から
	\begin{align}
		\sin{\frac{\pi}{2}}
		= \frac{2}{\pi} \cdot \cos{\xi}
	\end{align}
	が成り立つが,$\cos{\xi}$は正であるから
	\begin{align}
		\sin{\frac{\pi}{2}} = 1
	\end{align}
	である.ゆえに
	\begin{align}
		e^{\frac{\pi}{2} \cdot \isym} = \isym
	\end{align}
	が成立する.すなわち
	\begin{align}
		e^{\pi \cdot \isym}
		= e^{\frac{\pi}{2} \cdot \isym} \cdot e^{\frac{\pi}{2} \cdot \isym}
		= -1
	\end{align}
	である.すなわち
	\begin{align}
		e^{2 \cdot \pi \cdot \isym}
		= e^{\pi \cdot \isym} \cdot e^{\pi \cdot \isym}
		= 1
	\end{align}
	である.そして$n$を任意に与えられた正数とすれば
	\begin{align}
		e^{2 \cdot n \cdot \pi \cdot \isym} = 1
	\end{align}
	が成り立つが,これが成り立つのは当然のようであるけれども
	整数の累乗について次の結果を載せておく.
	
	\begin{screen}
		\begin{thm}[指数関数の整数乗]
			$z$を複素数とし,$n$を整数とすると,
			\begin{align}
				e^{n \cdot z} = (e^z)^n.
			\end{align}
		\end{thm}
	\end{screen}
	
	\begin{sketch}
		$z$を複素数とする.まず
		\begin{align}
			e^{0 \cdot z} = e^0 = 1
		\end{align}
		かつ
		\begin{align}
			(e^z)^0 = 1
		\end{align}
		であるから
		\begin{align}
			e^{0 \cdot z} = (e^z)^0
		\end{align}
		が成立する.また$n$を自然数として
		\begin{align}
			e^{n \cdot z} = (e^z)^n
		\end{align}
		が成り立っているとすると,
		\begin{align}
			e^{(n+1) \cdot z}
			= e^{n \cdot z} \cdot e^z
			= (e^z)^n \cdot e^z
			= (e^z)^{n+1}
		\end{align}
		が従う.ゆえに,数学的帰納法の原理より任意の自然数$n$で
		\begin{align}
			e^{n \cdot z} = (e^z)^n
		\end{align}
		が成立する.次に$n$を負の整数とすると,
		\begin{align}
			-n \in \Natural
		\end{align}
		であるから
		\begin{align}
			e^{(-n) \cdot z} = (e^z)^{-n}
		\end{align}
		が成立する.ここで
		\begin{align}
			(-n) \cdot z = -(n \cdot z)
		\end{align}
		が成り立つので,定理\ref{thm:inversion_of_exp_z_is_exp_minus_z}より
		\begin{align}
			e^{(-n) \cdot z} = e^{-(n \cdot z)} = (e^{n \cdot z})^{-1}
		\end{align}
		が成り立つ.一方で
		\begin{align}
			(e^z)^{-n} = ((e^z)^n)^{-1}
		\end{align}
		も成り立つので
		\begin{align}
			(e^{n \cdot z})^{-1} = ((e^z)^n)^{-1}
		\end{align}
		が従い
		\begin{align}
			e^{n \cdot z} = (e^z)^n
		\end{align}
		が得られる.
		\QED
	\end{sketch}
	
	次に考察するのは指数関数の{\bf 周期}\index{しゅうき@周期}{\bf (period)}である.
	結論を言えば複素数$z$と$w$に対して
	\begin{align}
		e^{z} = e^{w} \Longleftrightarrow 
		\exists n \in \Z\, \left(\, z - w = 2 \cdot n \cdot \pi \cdot \isym\, \right)
	\end{align}
	が成り立つので,指数関数は$2 \cdot \pi \cdot \isym$だけずれるごとに同じ値を繰り返す.
	つまり{\bf 指数関数の周期は$2 \cdot \pi \cdot \isym$である.}
	我々はすでに
	\begin{align}
		z - w = 2 \cdot n \cdot \pi \cdot \isym
		\Longrightarrow e^{z} = e^{w + 2 \cdot n \cdot \pi \cdot \isym}
		= e^{w} \cdot e^{2 \cdot n \cdot \pi \cdot \isym}
		= e^{w}
	\end{align}
	が成り立つことを知っているから,以下の目標は逆の導出である.
	
	いま$z$と$w$を複素数として,
	\begin{align}
		e^{z} = e^{w}
	\end{align}
	が成り立っているとする.
	
	$z$を複素数とすれば
	\begin{align}
		z = x + \isym \cdot y
	\end{align}
	を満たす実数$x$と$y$が取れるが,このとき
	\begin{align}
		e^z = e^x \cdot e^{\isym \cdot y}
	\end{align}
	が成り立ち,
	\begin{align}
		\left|e^z\right| = \left|e^x\right| \cdot \left|e^{\isym \cdot y}\right| = e^x
	\end{align}
	かつ
	\begin{align}
		e^z = e^x \cdot (\cos{y} + \isym \cdot \sin{y}) = \left|e^z\right| \cdot (\cos{y} + \isym \cdot \sin{y})
	\end{align}
	が成り立つ.後述することだが,$w$を任意に与えられた$0$でない複素数とすれば
	\begin{align}
		w = \exp{(z)}
	\end{align}
	を満たす複素数$z$が取れるので,すなわち
	\begin{align}
		w = |w| \cdot (\cos{y} + \isym \cdot \sin{y})
	\end{align}
	を満たす実数$y$が取れる.これを複素数の{\bf 極形式}\index{きょくけいしき@極形式}{\bf (polar form)}と呼び,
	この$y$を$w$の{\bf 偏角}\index{へんかく@偏角}{\bf (argument)}と呼ぶ.ただし$y$は一意に定まるものではない.
	
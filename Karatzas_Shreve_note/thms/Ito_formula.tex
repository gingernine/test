\section{伊藤の公式}
	$(\Omega,\mathscr{F},P)$を確率空間とし,$\{\mathscr{F}_{t}\}_{t \in [0,1]}$を
	フィルトレーションとする.
	
	\begin{screen}
		\begin{dfn}[有界変動過程]
			$B$を$[0,1] \times \Omega$上の$\{\mathscr{F}_{t}\}_{t \in [0,1]}$-適合過程する.
			\begin{itemize}
				\item $\Omega$のすべての要素$\omega$で$B_{0}(\omega) = 0$,
				\item $\Omega$のすべての要素$\omega$で$B_{\bullet}(\omega)$は
					$\mathscr{O}_{[0,1]}/\mathscr{O}_{\R}$-連続かつ有界変動,
				\item $[0,1] \times \Omega$の各要素$(t,\omega)$に対し
					\begin{align}
						\sup{}{
							\Set{\sum_{i=0}^{n-1}\left|B_{\tau_{i+1}}(\omega) - B_{\tau_{i}}(\omega)\right|}{
								n \in \Natural \wedge \tau:n+1 \longrightarrow [0,t] \wedge
								\tau_{0} = 0 \wedge \tau_{n} = t \wedge 
								\forall i \in n\, (\, \tau_{i} \leq \tau_{i+1}\, )
							}
						}
					\end{align}
					を対応させる写像を$|B|$と書くとき,
					\begin{align}
						E|B|_{1} < \infty,
					\end{align}
			\end{itemize}
			が満たされるとき,$B$を有界変動過程と呼ぶ.
		\end{dfn}
	\end{screen}
	
	全ての$\omega$で
	\begin{align}
		t \longmapsto |B|_{t}(\omega)
	\end{align}
	は連続で,また$\mathscr{F}_{t}/\borel{\R}$-可測である.
	
	\begin{screen}
		\begin{dfn}[連続セミマルチンゲール]
			$X$を$[0,1] \times \Omega$上の$\R$値$\{\mathscr{F}_{t}\}_{t \in [0,1]}$-適合過程とする.
			$[0,1] \times \Omega$の任意の要素$(t,\omega)$で
			\begin{align}
				X_{t}(\omega) - X_{0}(\omega) = M_{t}(\omega) + B_{t}(\omega)
			\end{align}
			を満たす$\mathscr{M}^{2,c}_{[0,1]}$の要素$M$と有界変動過程$B$が取れるとき,
			$X$を連続セミマルチンゲールと呼ぶ.
		\end{dfn}
	\end{screen}
	
	\begin{screen}
		\begin{thm}[伊藤の公式]
			$X$を$[0,1] \times \Omega$上の連続セミマルチンゲールとし,
			$[0,1] \times \Omega$の任意の要素$(t,\omega)$で
			\begin{align}
				X_{t}(\omega) - X_{0}(\omega) = M_{t}(\omega) + B_{t}(\omega)
			\end{align}
			を満たす$\mathscr{M}^{2,c}_{[0,1]}$の要素$M$と有界変動過程$B$を取る.
			このとき,$C^{2}_{b}(\R)$の任意の要素$f$に対して
			\begin{align}
				f(X_t) = f(X_{0}) + \int_{[0,t]} f'(X_{s})\ dM_{s}
				+ \int_{[0,t]} f'(X_{s})\ dB_{s}
				+ \frac{1}{2} \cdot \int_{[0,t]} f''(X_{s})\ d\inprod<M>_{s}
			\end{align}
			が$[0,1]$の任意の要素$t$で成立する.
		\end{thm}
	\end{screen}
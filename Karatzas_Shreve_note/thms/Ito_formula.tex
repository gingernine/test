\section{伊藤の公式}
	$(\Omega,\mathscr{F},P)$を確率空間とし,$\{\mathscr{F}_{t}\}_{t \in [0,1]}$を
	フィルトレーションとする.
	
	\begin{screen}
		\begin{dfn}[有界変動過程]
			$B$を$[0,1] \times \Omega$上の$\{\mathscr{F}_{t}\}_{t \in [0,1]}$-適合過程する.
			\begin{itemize}
				\item $\Omega$のすべての要素$\omega$で$B_{0}(\omega) = 0$,
				\item $\Omega$のすべての要素$\omega$で$B_{\bullet}(\omega)$は
					$\mathscr{O}_{[0,1]}/\mathscr{O}_{\R}$-連続かつ有界変動,
				\item $[0,1] \times \Omega$の各要素$(t,\omega)$に対し
					\begin{align}
						\sup{}{
							\Set{\sum_{i=0}^{n-1}\left|B_{\tau_{i+1}}(\omega) - B_{\tau_{i}}(\omega)\right|}{
								n \in \Natural \wedge \tau:n+1 \longrightarrow [0,t] \wedge
								\tau_{0} = 0 \wedge \tau_{n} = t \wedge 
								\forall i \in n\, (\, \tau_{i} \leq \tau_{i+1}\, )
							}
						}
					\end{align}
					を対応させる写像を$|B|$と書くとき,
					\begin{align}
						E|B|_{1} < \infty,
					\end{align}
			\end{itemize}
			が満たされるとき,$B$を有界変動過程と呼ぶ.
		\end{dfn}
	\end{screen}
	
	全ての$\omega$で
	\begin{align}
		t \longmapsto |B|_{t}(\omega)
	\end{align}
	は連続で,また$\mathscr{F}_{t}/\borel{\R}$-可測である.
	
	\begin{screen}
		\begin{dfn}[連続セミマルチンゲール]
			$X$を$[0,1] \times \Omega$上の$\R$値$\{\mathscr{F}_{t}\}_{t \in [0,1]}$-適合過程とする.
			$[0,1] \times \Omega$の任意の要素$(t,\omega)$で
			\begin{align}
				X_{t}(\omega) - X_{0}(\omega) = M_{t}(\omega) + B_{t}(\omega)
			\end{align}
			を満たす$\mathscr{M}^{2,c}_{[0,1]}$の要素$M$と有界変動過程$B$が取れるとき,
			$X$を連続セミマルチンゲールと呼ぶ.
		\end{dfn}
	\end{screen}
	
	いま$f$を$[a,b]$上の$3$階連続微分可能関数とすると,微分積分学の基本定理より
	\begin{align}
		f(b) - f(a) = \int_{[a,b]} f'(t)\ dt
	\end{align}
	が成立する.ここで部分積分公式より
	\begin{align}
		\int_{[a,b]} f'(t)\ dt
		= f'(a) \cdot (b-a) + \int_{[a,b]} f''(t) \cdot (b-t)\ dt
	\end{align}
	が成立し,再び部分積分公式より
	\begin{align}
		\int_{[a,b]} f''(t) \cdot (b-t)\ dt
		= \frac{1}{2} \cdot f'(a) \cdot (b-a)^{2} 
		+ \frac{1}{2} \cdot \int_{[a,b]} f'''(t) \cdot (b-t)^{2}\ dt
	\end{align}
	が成立する.ゆえに
	\begin{align}
		f(b) - f(a) = f'(a) \cdot (b-a) + \frac{1}{2} \cdot f''(a) \cdot (b-a)^{2}
		+ \frac{1}{2} \cdot \int_{[a,b]} f'''(x) \cdot (b-x)^{2}\ dx
	\end{align}
	が得られる.
	
	\begin{screen}
		\begin{thm}[伊藤の公式]
			$X$を$[0,1] \times \Omega$上の連続セミマルチンゲールとし,
			$[0,1] \times \Omega$の任意の要素$(t,\omega)$で
			\begin{align}
				X_{t}(\omega) - X_{0}(\omega) = M_{t}(\omega) + B_{t}(\omega)
			\end{align}
			を満たす$\mathscr{M}^{2,c}_{[0,1]}$の要素$M$と有界変動過程$B$を取る.
			このとき,$C^{2}_{b}(\R)$の任意の要素$f$に対して
			\begin{align}
				f(X_t) = f(X_{0}) + \int_{[0,t]} f'(X_{s})\ dM_{s}
				+ \int_{[0,t]} f'(X_{s})\ dB_{s}
				+ \frac{1}{2} \cdot \int_{[0,t]} f''(X_{s})\ d\inprod<M>_{s}
			\end{align}
			が$[0,1]$の任意の要素$t$で成立する.
		\end{thm}
	\end{screen}
	
	\begin{sketch}\mbox{}
		\begin{description}
			\item[第一段]
				$f$を$C^{3}_{b}(\R)$の要素とする.
				また$t$を$[0,1]$の要素とし,$1$以上の任意の自然数$n$に対し
				\begin{align}
					\tau^{n}_{0} = 0
				\end{align}
				及び
				\begin{align}
					\tau^{n}_{i+1} = 
					\min&\{ t, \tau^{n}_{i} + \frac{1}{n}, \\
					&\inf{}{
						\Set{s \in [0,t]}{\tau^{n}_{i} < s \wedge \left[\, 
						\frac{1}{n} \leq \left|M_{\tau^{n}_{i}} - M_{s}\right| \vee
						\frac{1}{n} \leq \left|B_{\tau^{n}_{i}} - B_{s}\right| \vee
						\frac{1}{n} \leq \left|\inprod<M>_{\tau^{n}_{i}} - \inprod<M>_{s}\right|\, \right]}} \}
				\end{align}
				を満たす$\{\mathscr{F}_{t}\}_{t \in [0,1]}$-停止時刻の列を取る.
				\begin{align}
					a < b
				\end{align}
				なる任意の実数$a$と$b$に対して
				\begin{align}
					f(b) - f(a)
					= f'(a) \cdot (b-a) + \frac{1}{2} \cdot f''(a) \cdot (b-a)^{2}
					+ \frac{1}{2} \cdot \int_{[a,b]} f'''(x) \cdot (b-x)^{2}\ dx
				\end{align}
				が成立するので,
				\begin{align}
					f(X_{t}) - f(X_{0})
					&= \sum_{i=0}^{\infty} \left[ f(X_{\tau^{n}_{i+1}}) - f(X_{\tau^{n}_{i}}) \right] \\
					&= \sum_{i=0}^{\infty} f'(X_{\tau^{n}_{i}}) \cdot (X_{\tau^{n}_{i+1}} - X_{\tau^{n}_{i}}) \\
					&\quad + \frac{1}{2} \cdot \sum_{i=0}^{\infty} f''(X_{\tau^{n}_{i}}) \cdot (X_{\tau^{n}_{i+1}} - X_{\tau^{n}_{i}})^{2} \\
					&\quad + \frac{1}{2} \cdot \sum_{i=0}^{\infty} \int_{[\tau^{n}_{i},\tau^{n}_{i+1}]} f'''(x) \cdot (X_{\tau^{n}_{i+1}} - x)^{2}\ dx
				\end{align}
				が成り立つ.ここで
				\begin{align}
					g_{n} \defeq \sum_{i=0}^{\infty} f'(X_{\tau^{n}_{i}}) \cdot (X_{\tau^{n}_{i+1}} - X_{\tau^{n}_{i}})
				\end{align}
				及び
				\begin{align}
					h_{n} \defeq \sum_{i=0}^{\infty} f''(X_{\tau^{n}_{i}}) \cdot (X_{\tau^{n}_{i+1}} - X_{\tau^{n}_{i}})^{2}
				\end{align}
				及び
				\begin{align}
					r_{n} \defeq \sum_{i=0}^{\infty} \int_{[\tau^{n}_{i},\tau^{n}_{i+1}]} f'''(x) \cdot (X_{\tau^{n}_{i+1}} - x)^{2}\ dx
				\end{align}
				とおく.
				
			\item[第二段]
				$g_{n}$は
				\begin{align}
					g_{n} = \sum_{i=0}^{\infty} f'(X_{\tau^{n}_{i}}) \cdot (M_{\tau^{n}_{i+1}} - M_{\tau^{n}_{i}})
					+ \sum_{i=0}^{\infty} f'(X_{\tau^{n}_{i}}) \cdot (B_{\tau^{n}_{i+1}} - B_{\tau^{n}_{i}})
				\end{align}
				と分解できるが,
				\begin{align}
					&\left|\sum_{i=0}^{\infty} f'(X_{\tau^{n}_{i}}) \cdot (B_{\tau^{n}_{i+1}} - B_{\tau^{n}_{i}}) - \int_{[0,t]} f'(X_{s})\ dB_{s}\right| \\
					&= \left|\sum_{i=0}^{\infty} \int_{[\tau^{n}_{i},\tau^{n}_{i+1}]} \left[f'(X_{\tau^{n}_{i}}) - f'(X_{s}) \right]\ dB_{s}\right| \\
					&\leq \sum_{i=0}^{\infty} \int_{[\tau^{n}_{i},\tau^{n}_{i+1}]} \left|f'(X_{\tau^{n}_{i}}) - f'(X_{s}) \right|\ d|B|_{s} \\
					&\leq \left\{\sup{i \in \Natural}{\sup{s \in [\tau^{n}_{i},\tau^{n}_{i+1}]}{
					\left|f'(X_{\tau^{n}_{i}}) - f'(X_{s}) \right|}}\right\} \cdot |B|_{t}
				\end{align}
				が成り立つから,Lebesgueの収束定理より
				\begin{align}
					E\left|\sum_{i=0}^{\infty} f'(X_{\tau^{n}_{i}}) \cdot (B_{\tau^{n}_{i+1}} - B_{\tau^{n}_{i}}) - \int_{[0,t]} f'(X_{s})\ dB_{s}\right|
					\longrightarrow 0
				\end{align}
				が成立する.同様に
				\begin{align}
					&E\left[\sum_{i=0}^{\infty} f'(X_{\tau^{n}_{i}}) \cdot (M_{\tau^{n}_{i+1}} - M_{\tau^{n}_{i}}) - \int_{[0,t]} f'(X_{s})\ dM_{s}\right]^{2} \\
					&= E\left[\sum_{i=0}^{\infty} \int_{[\tau^{n}_{i},\tau^{n}_{i+1}]}\left[f'(X_{\tau^{n}_{i}}) - f'(X_{s}) \right]^{2}\ d\inprod<M>_{s} \right] 
					\label{fom:Ito_formula_1}\\
					&\leq E\left[\left\{\sup{i \in \Natural}{\sup{s \in [\tau^{n}_{i},\tau^{n}_{i+1}]}{
					\left[f'(X_{\tau^{n}_{i}}) - f'(X_{s}) \right]^{2}}}\right\} \cdot \inprod<M>_{t}\right] \longrightarrow 0
				\end{align}
				が成立する.以上で
				\begin{align}
					E\left|g_{n} - \int_{[0,t]} f'(X_{s})\ dM_{s} 
					- \int_{[0,t]} f'(X_{s})\ dB_{s}\right|
					\longrightarrow 0
				\end{align}
				を得る.
				
			\item[第三段]
				$h_{n}$は
				\begin{align}
					h_{n} &= \sum_{i=0}^{\infty} f''(X_{\tau^{n}_{i}}) \cdot (M_{\tau^{n}_{i+1}} - M_{\tau^{n}_{i}})^{2} \\
					&\quad + 2 \cdot \sum_{i=0}^{\infty} f''(X_{\tau^{n}_{i}}) \cdot (M_{\tau^{n}_{i+1}} - M_{\tau^{n}_{i}}) \cdot (B_{\tau^{n}_{i+1}} - B_{\tau^{n}_{i}}) \\
					&\quad + \sum_{i=0}^{\infty} f''(X_{\tau^{n}_{i}}) \cdot (B_{\tau^{n}_{i+1}} - B_{\tau^{n}_{i}})^{2}
				\end{align}
				と分解できる.まず
				\begin{align}
					&\left| \sum_{i=0}^{\infty} f''(X_{\tau^{n}_{i}}) \cdot (B_{\tau^{n}_{i+1}} - B_{\tau^{n}_{i}})^{2} \right| \\
					&\leq \sum_{i=0}^{\infty} \left|f''(X_{\tau^{n}_{i}})\right| \cdot \left|B_{\tau^{n}_{i+1}} - B_{\tau^{n}_{i}}\right| \cdot \left|B_{\tau^{n}_{i+1}} - B_{\tau^{n}_{i}}\right| \\
					&\leq \frac{C}{n} \cdot |B|_{t}
				\end{align}
				及び
				\begin{align}
					\left| \sum_{i=0}^{\infty} f''(X_{\tau^{n}_{i}}) \cdot (M_{\tau^{n}_{i+1}} - M_{\tau^{n}_{i}}) \cdot (B_{\tau^{n}_{i+1}} - B_{\tau^{n}_{i}}) \right|
					\leq \frac{C}{n} \cdot |B|_{t}
				\end{align}
				が成り立つ.また
				\begin{align}
					E\left[\sum_{i=0}^{\infty} f''(X_{\tau^{n}_{i}}) \cdot (M_{\tau^{n}_{i+1}} - M_{\tau^{n}_{i}})^{2}\right]
					= E\left[\sum_{i=0}^{\infty} f''(X_{\tau^{n}_{i}}) \cdot (\inprod<M>_{\tau^{n}_{i+1}} - \inprod<M>_{\tau^{n}_{i}})\right]
					\label{fom:Ito_formula_2}
				\end{align}
				及び
				\begin{align}
					&E\left|\sum_{i=0}^{\infty} f''(X_{\tau^{n}_{i}}) \cdot (\inprod<M>_{\tau^{n}_{i+1}} - \inprod<M>_{\tau^{n}_{i}})
					- \int_{[0,1]} f''(X_{s})\ d\inprod<M>_{s}\right| \\
					&= E\left|\sum_{i=0}^{\infty} \int_{[\tau^{n}_{i},\tau^{n}_{i+1}]} \left[ f''(X_{\tau^{n}_{i}}) - f''(X_{s}) \right]\ d\inprod<M>_{s}\right| \\
					&\leq E\left[\left\{\sup{i \in \Natural}{\sup{s \in [\tau^{n}_{i},\tau^{n}_{i+1}]}{
					\left|f'(X_{\tau^{n}_{i}}) - f'(X_{s}) \right|}}\right\} \cdot \inprod<M>_{t}\right] \\
					&\longrightarrow 0
				\end{align}
				が成り立つので
				\begin{align}
					E\left|h_{n} - \int_{[0,t]} f''(X_{s})\ d\inprod<M>_{s}\right|
					\longrightarrow 0
				\end{align}
				を得る.
				
			\item[第四段]
				$\tau^{n}$の定め方より
				\begin{align}
					\left|X_{\tau^{n}_{i+1}} - X_{\tau^{n}_{i}}\right|
					\leq \left|M_{\tau^{n}_{i+1}} - M_{\tau^{n}_{i}}\right|
					+ \left|B_{\tau^{n}_{i+1}} - B_{\tau^{n}_{i}}\right|
					\leq \frac{2}{n}
				\end{align}
				が成り立つので
				\begin{align}
					|r_{n}|
					&\leq C \cdot \frac{2}{n} \cdot \sum_{i=0}^{\infty} \left|X_{\tau^{n}_{i+1}} - X_{\tau^{n}_{i}}\right|^{2} \\
					&\leq C \cdot \frac{4}{n} \cdot
					\left\{
						\sum_{i=0}^{\infty} \left[M_{\tau^{n}_{i+1}} - M_{\tau^{n}_{i}}\right]^{2}
						+ \sum_{i=0}^{\infty} \left[B_{\tau^{n}_{i+1}} - B_{\tau^{n}_{i}}\right]^{2}
					\right\} \\
					&\leq C \cdot \frac{4}{n} \cdot
					\left\{
						\sum_{i=0}^{\infty} \left[M_{\tau^{n}_{i+1}} - M_{\tau^{n}_{i}}\right]^{2}
						+ \frac{1}{n} \cdot |B|_{t}
					\right\}
				\end{align}
				が成り立つ.ここで
				\begin{align}
					E\left[\sum_{i=0}^{\infty} \left[M_{\tau^{n}_{i+1}} - M_{\tau^{n}_{i}}\right]^{2}\right]
					= E\inprod<M>_{t}
					\label{fom:Ito_formula_3}
				\end{align}
				が成り立つので
				\begin{align}
					E|r_{n}| \longrightarrow 0
				\end{align}
				を得る.
			
			\item[第五段]
				以上より
				\begin{align}
					&E\left|f(X_{t}) - f(X_{0})
					- \left\{\int_{[0,t]} f'(X_{s})\ dM_{s} 
					+ \int_{[0,t]} f'(X_{s})\ dB_{s}
					+ \frac{1}{2} \cdot \int_{[0,t]} f''(X_{s})\ d\inprod<M>_{s}\right\}\right| \\
					&\leq E\left| g_{n} - \int_{[0,t]} f'(X_{s})\ dM_{s} 
					- \int_{[0,t]} f'(X_{s})\ dB_{s} \right|
					+ \frac{1}{2} \cdot E\left| h_{n} - \int_{[0,t]} f''(X_{s})\ d\inprod<M>_{s} \right| 
					+ E|r_{n}| \\
					&\longrightarrow 0 
				\end{align}
				が成り立つので
				\begin{align}
					f(X_{t}) = f(X_{0}) + \int_{[0,t]} f'(X_{s})\ dM_{s} 
					+ \int_{[0,t]} f'(X_{s})\ dB_{s}
					+ \frac{1}{2} \cdot \int_{[0,t]} f''(X_{s})\ d\inprod<M>_{s}
				\end{align}
				が従う.
			
			\item[第六段]
				$f$を$C^{2}_{b}(\R)$の要素とする.このとき
				
			\item[(\refeq{fom:Ito_formula_1})の略証]
				$[0,t]$上の要素$s$に対して,
				\begin{align}
					s \in \left[\tau^{n}_{i},\tau^{n}_{i+1}\right]
				\end{align}
				ならば
				\begin{align}
					f'(X_{\tau^{n}_{i}})
				\end{align}
				を対応させる写像を$F$とすると
				\begin{align}
					&E\left[\sum_{i=0}^{\infty} f'(X_{\tau^{n}_{i}}) \cdot (M_{\tau^{n}_{i+1}} - M_{\tau^{n}_{i}}) - \int_{[0,t]} f'(X_{s})\ dM_{s}\right]^{2} \\
					&= E\left[\int_{[0,t]} F_{s} - f'(X_{s})\ dM_{s}\right]^{2} \\
					&= E\left[\int_{[0,t]} \left[F_{s} - f'(X_{s})\right]^{2}\ d\inprod<M>_{s}\right] \\
					&= E\left[\sum_{i=0}^{\infty} \int_{[\tau^{n}_{i},\tau^{n}_{i+1}]} \left[F_{s} - f'(X_{s})\right]^{2}\ d\inprod<M>_{s}\right] \\
					&= E\left[\sum_{i=0}^{\infty} \int_{[\tau^{n}_{i},\tau^{n}_{i+1}]} \left[f'(X_{\tau^{n}_{i}}) - f'(X_{s})\right]^{2}\ d\inprod<M>_{s}\right]
				\end{align}
				が得られる.
						
			\item[(\refeq{fom:Ito_formula_3})の略証]
				単調収束定理より
				\begin{align}
					E\left[\sum_{i=0}^{\infty} \left[M_{\tau^{n}_{i+1}} - M_{\tau^{n}_{i}}\right]^{2}\right]
					= \sum_{i=0}^{\infty} E\left[M_{\tau^{n}_{i+1}} - M_{\tau^{n}_{i}}\right]^{2}
					= \sum_{i=0}^{\infty} E\left[M_{\tau^{n}_{i+1}}^{2} - M_{\tau^{n}_{i}}^{2}\right]
					\label{fom:Ito_formula_5}
				\end{align}
				が成り立つ.ところで$M^{2} - \inprod<M>$は$\{\mathscr{F}_{t}\}_{t \in [0,1]}$-マルチンゲールなので,
				\begin{align}
					\alpha < \beta
				\end{align}
				なる$[0,1]$の任意の要素$\alpha$と$\beta$に対して
				\begin{align}
					E\left[M_{\beta}^{2} - M_{\alpha}^{2}\right]
					= E\left[\inprod<M>_{\beta} - \inprod<M>_{\alpha}\right]
				\end{align}
				が成立する.ゆえに
				\begin{align}
					(\refeq{fom:Ito_formula_5})
					= \sum_{i=0}^{\infty} E\left[\inprod<M>_{\tau^{n}_{i+1}} - \inprod<M>_{\tau^{n}_{i}}\right]
					= E\left[\sum_{i=0}^{\infty} \left(\inprod<M>_{\tau^{n}_{i+1}} - \inprod<M>_{\tau^{n}_{i}}\right) \right]
					= E\inprod<M>_{t}
				\end{align}
				を得る.
				
			\item[(\refeq{fom:Ito_formula_2})の略証]
				(\refeq{fom:Ito_formula_3})より
				\begin{align}
					\sum_{i=0}^{\infty} \left|f''(X_{\tau^{n}_{i}})\right| \cdot (M_{\tau^{n}_{i+1}} - M_{\tau^{n}_{i}})^{2}
				\end{align}
				は可積分であるから,Lebesgueの収束定理より
				\begin{align}
					E\left[\sum_{i=0}^{\infty} f''(X_{\tau^{n}_{i}}) \cdot (M_{\tau^{n}_{i+1}} - M_{\tau^{n}_{i}})^{2}\right]
					= \sum_{i=0}^{\infty} E\left[f''(X_{\tau^{n}_{i}}) \cdot (M_{\tau^{n}_{i+1}} - M_{\tau^{n}_{i}})^{2}\right]
				\end{align}
				が成り立ち,及び
				\begin{align}
					E\left[f''(X_{\tau^{n}_{i}}) \cdot (M_{\tau^{n}_{i+1}} - M_{\tau^{n}_{i}})^{2}\right]
					&= E\left[\cexp{f''(X_{\tau^{n}_{i}}) \cdot (M_{\tau^{n}_{i+1}} - M_{\tau^{n}_{i}})^{2}}{\mathscr{F}_{\tau^{n}_{i}}}\right] \\
					&= E\left[f''(X_{\tau^{n}_{i}}) \cdot \cexp{(M_{\tau^{n}_{i+1}} - M_{\tau^{n}_{i}})^{2}}{\mathscr{F}_{\tau^{n}_{i}}}\right]
					\label{fom:Ito_formula_4}
				\end{align}
				が成り立つ.ところで$M^{2} - \inprod<M>$は$\{\mathscr{F}_{t}\}_{t \in [0,1]}$-マルチンゲールなので,$\alpha$と$\beta$を
				\begin{align}
					\alpha \leq \beta
				\end{align}
				なる$[0,1]$の要素とすれば,$\mathscr{F}_{\alpha}$の任意の要素$A$に対して
				\begin{align}
					\int_{A} \left(M_{\beta} - M_{\alpha}\right)^{2}\ dP
					= \int_{A} M_{\beta}^{2} - M_{\alpha}^{2}\ dP
					= \int_{A} \inprod<M>_{\beta} - \inprod<M>_{\alpha}\ dP
				\end{align}
				が成立する.つまり
				\begin{align}
					\cexp{\left(M_{\beta} - M_{\alpha}\right)^{2}}{\mathscr{F}_{\alpha}}
					= \cexp{\inprod<M>_{\beta} - \inprod<M>_{\alpha}}{\mathscr{F}_{\alpha}}
				\end{align}
				が成立するので,
				\begin{align}
					(\refeq{fom:Ito_formula_4})
					&= E\left[f''(X_{\tau^{n}_{i}}) \cdot \cexp{\inprod<M>_{\tau^{n}_{i+1}} - \inprod<M>_{\tau^{n}_{i}}}{\mathscr{F}_{\tau^{n}_{i}}}\right] \\
					&= E\left[\cexp{f''(X_{\tau^{n}_{i}}) \cdot \left(\inprod<M>_{\tau^{n}_{i+1}} - \inprod<M>_{\tau^{n}_{i}}\right)}{\mathscr{F}_{\tau^{n}_{i}}}\right] \\
					&= E\left[f''(X_{\tau^{n}_{i}}) \cdot \left(\inprod<M>_{\tau^{n}_{i+1}} - \inprod<M>_{\tau^{n}_{i}}\right)\right]
				\end{align}
				が成り立ち(\refeq{fom:Ito_formula_2})が従う.
		\end{description}
	\end{sketch}
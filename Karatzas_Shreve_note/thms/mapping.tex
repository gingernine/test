\subsection{写像}
	\begin{screen}
		\begin{thm}[全射・単射・像・原像]\label{projective_injective_image_preimage}
			$f$を集合$A$から集合$B$への写像とするとき,
			\begin{description}
				\item[(1)] 任意の$U \subset A$に対し$f^{-1}\left(f(U)\right) \supset U$が成立し,
					特に$f$が単射なら$f^{-1}\left(f(U)\right) = U$となる.
				\item[(2)] 任意の$V \subset B$に対し$f\left(f^{-1}(V)\right) \subset V$が成立し,
					特に$f$が全射なら$f\left(f^{-1}(V)\right) = V$となる.
			\end{description}
		\end{thm}
	\end{screen}
	
	\begin{prf}\mbox{}
		\begin{description}
			\item[(1)] 任意の$x \in U$で$f(x) \in f(U)$となるから
				$x \in f^{-1}\left(f(U)\right)$が成立する.
				$f$が単射であれば,任意の$x \in f^{-1}\left(f(U)\right)$に対し
				$f(x) \in f(U)$となるから或る$x_1 \in U$で$f(x) = f(x_1)$となり,
				単射性より$x = x_1 \in U$が成り立つ.
				
			\item[(2)] 任意に$y \in f\left(f^{-1}(V)\right)$を取れば,
				或る$x \in f^{-1}(V)$で$y = f(x) \in V$となる.$f$が全射であるとき,
				任意の$y \in V$に対し或る$x \in A$が$y = f(x)$を満たすから,
				$x \in f^{-1}(V)$となり$y \in f\left(f^{-1}(V)\right)$が従う.
				\QED
		\end{description}
	\end{prf}
	
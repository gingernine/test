\subsection{写像}
	\begin{screen}
		\begin{dfn}[写像]
		\end{dfn}
	\end{screen}
	
	\begin{screen}
		\begin{dfn}[族・系]\label{dfn:family_collection}
			$x$を集合$A$から集合$B$への写像とするとき,
			$x$を$B$の元の集まりと見做したものを
			``$A$を添数集合\index{てんすうしゅうごう@添数集合}(index set)とする
			$B$の族\index{ぞく@族}(family) (或は系\index{けい@系}(collection))''と呼び,
			$x(a)$の代わりに$x_a$として$(x_a)$や$(x_a)_{a \in A}$,又は
			$A$の元が具体的に書き並べられるときは
			$(x_{a_1},x_{a_2},\cdots)$などとも表記する.
			$B$の元の指す対象によっては
			族を点族\index{てんぞく@点族},
			集合族(系)\index{しゅうごうぞく(けい)@集合族(系)},
			或は関数族(系)\index{かんすうぞく(けい)@関数族(系)}などと呼ぶ.
		\end{dfn}
	\end{screen}
	族$(x_a)_{a \in A}$は写像$x$そのものと同一であるが,
	丸括弧を中括弧に替えた$\{x_a\}_{a \in A}$は$B$の部分集合
	$\Set{x_a}{a \in A}$の別の記法であり,$(x_a)_{a \in A}$とは区別する.
	実際,族と集合の大きな違いは,$(x_a)_{a \in A}$の表記では重複する元も
	別個の存在と認めるのに対し,$\{x_a\}_{a \in A}$の表記では重複する元は区別しないことである.
	例えば$A = \N,\ B = \R$に対して
	\begin{align}
		x_n \coloneqq
		\begin{cases}
			1 & (n:\mbox{奇数}) \\
			-1 & (n:\mbox{偶数})
		\end{cases}
	\end{align}
	と定めるとき,$(x_n) = (1,-1,1,-1,\cdots)$と書ける一方で$\{x_n\} = \{-1,1\}$となる.
	
	\begin{screen}
		\begin{thm}[全射・単射・像・原像]\label{projective_injective_image_preimage}
			$f$を集合$A$から集合$B$への写像とするとき,
			\begin{description}
				\item[(1)] 任意の$U \subset A$に対し$f^{-1}\left(f(U)\right) \supset U$が成立し,
					特に$f$が単射なら$f^{-1}\left(f(U)\right) = U$となる.
				\item[(2)] 任意の$V \subset B$に対し$f\left(f^{-1}(V)\right) \subset V$が成立し,
					特に$f$が全射なら$f\left(f^{-1}(V)\right) = V$となる.
			\end{description}
		\end{thm}
	\end{screen}
	
	\begin{prf}\mbox{}
		\begin{description}
			\item[(1)] 任意の$x \in U$で$f(x) \in f(U)$となるから
				$x \in f^{-1}\left(f(U)\right)$が成立する.
				$f$が単射であれば,任意の$x \in f^{-1}\left(f(U)\right)$に対し
				$f(x) \in f(U)$となるから或る$x_1 \in U$で$f(x) = f(x_1)$となり,
				単射性より$x = x_1 \in U$が成り立つ.
				
			\item[(2)] 任意に$y \in f\left(f^{-1}(V)\right)$を取れば,
				或る$x \in f^{-1}(V)$で$y = f(x) \in V$となる.$f$が全射であるとき,
				任意の$y \in V$に対し或る$x \in A$が$y = f(x)$を満たすから,
				$x \in f^{-1}(V)$となり$y \in f\left(f^{-1}(V)\right)$が従う.
				\QED
		\end{description}
	\end{prf}
	
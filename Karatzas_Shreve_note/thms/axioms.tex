\subsection{順序数}

\subsection{複素数}
	\monologue{
		院生「エジソンは$1+1$が$2$になることを受け入れられず周りの大人を困らせたという逸話が有名ですが,
			彼の疑問はそもそも数とは何かという問題に帰着しますから,
			彼の質問攻めを受けた大人が回答に窮したのも無理はないでしょう.
			しかし数学徒を自認している者ならば,数とは何かと訊ねられたら正確に答える義務があります.
			さて我々が使える道具は集合論のみですが,前節までの集合論の言葉で数を説明するにはどうしたら良いでしょうか?
			小中高と無条件に受け入れ(させられ)てきた四則演算が成り立つ世界を,
			集合の宇宙の中に実現させるにはどうしたら良いのでしょうか?
			ここで本節の大まかな流れを説明いたしましょう.我々は自然数を
			${\bf \omega}$の要素として定義しましたが,今後よく使う書き方として
			\begin{align}
				0 &= \emptyset, \\
				1 &= \{0\} = \{\emptyset\}, \\
				2 &= \{0,1\} = \{\emptyset,\{\emptyset\}\}, \\
				3 &= \{0,1,2\} = \{\emptyset,\{\emptyset\},\{\emptyset,\{\emptyset\}\}\}, \\
				&\vdots
			\end{align}
			と定めます.まさに馴染み深い数字そのものです.
			次に${\bf \omega}$を整数環に拡張しますが,そこでは`半群を群にする操作'を応用します.
			整数環を有理数体に拡張する際には`環から体を作る操作'を応用し,
			有理数体を実数体に拡張する際には`Dedekind切断'を行います.
			実数体が出来たら,次は`'で複素数体の出来上がりです.
			ここでいくつか解消するべき問題(実数体は${\bf \omega}$を部分集合として含んでいるのか,
			複素数体は実数体を部分集合として含んでいるのか,など)がありますが,
			後述に回しましょう.いま述べたように,数の構成には代数学を利用します.
			いまの私の力では代数学に深入りすることはできませんが,
			言い訳がましいですけれども,代数学を代数学として勉強するよりは,
			数の構成を軸に代数の一般論を(ほんの一片ですが)織り交ていく方が
			(私の拙い経験上よく使う)代数の知識を身に付けるのに効率が良いでしょう.」
	}
	
	\begin{screen}
		\begin{dfn}[Peanoシステム]
			次を満たす集合$X,a$と写像$f:X \longrightarrow X$の組$(X,a,f)$を
			{\bf Peanoシステム}\index{Peanoしすてむ@Peanoシステム}と呼ぶ:
			\begin{itemize}
				\item $a \in X$.
				\item $a \notin f(X)$.
				\item $f$は単射である.
				\item 集合$S$が$a$を含み,かつ$S$の任意の要素$x$に対し$f(x) \in S$となるならば,$S = X$.
			\end{itemize}
			四番目の性質は{\bf 数学的帰納法の原理}\index{すうがくてききのうほうのげんり@数学的帰納法の原理}
			{\bf (the principle of mathematical induction)}と呼ばれる.
		\end{dfn}
	\end{screen}
	
	\begin{screen}
		\begin{thm}[$\omega$はPeanoシステム]
			写像$\sigma:\omega \longrightarrow \omega$を
			\begin{align}
				\sigma = \Set{(n, n \cup \{n\})}{n \in \omega}
			\end{align}
			で定めるとき,$(\omega,\emptyset,\sigma)$はPeanoシステムとなる.
			この$\sigma$を{\bf 後継者写像}\index{こうけいしゃしゃぞう@後継者写像}{\bf (successor mapping)}と呼ぶ.
		\end{thm}
	\end{screen}
	
	\begin{prf}
		$\emptyset \in \omega$である.また任意の自然数$n$に対し$\sigma(n) = n \cup \{n\} \neq \emptyset$となる.
		任意の自然数$n,m$に対して
		\begin{align}
			\sigma(x)=\sigma(y) \Longrightarrow &(x \in y \cup \{y\}) \wedge (y \in x \cup \{x\}) \\
					&\Longleftrightarrow \left( x \in y \vee x=y \right) \wedge 
						\left( y \in x \vee y=x \right) \\
					&\Longleftrightarrow  \left( x \in y \wedge y \in x \right) \vee x = y \\
					&\Longleftrightarrow x = y
		\end{align}
		が成立するから$\sigma$は単射である.
	\end{prf}
	
	\begin{screen}
		\begin{thm}[Peanoシステムの写像は全単射]\label{thm:successor_mapping_is_injective}
			Peanoシステム$(X,a,f)$の$f$は$X$から$X \backslash \{a\}$への全単射である.
		\end{thm}
	\end{screen}
	
	\begin{prf}
		$S \coloneqq \{a\} \cup \sigma(X)$とおけば,数学的帰納法の原理より$S = X$が成り立ち$f$の全射性が出る.
		\QED
	\end{prf}
	
	\begin{screen}
		\begin{thm}[再帰定理]\label{thm:Peano_recursion_theorem}
			$Y$を空でない集合,$b$を$Y$の要素,$g$を$Y$から$Y$への写像とし,
			$(X,a,f)$をPeanoシステムとする.
			このとき,次を満たすような写像$u:X \longrightarrow Y$がただ一つ存在する:
			\begin{align}
				u(a) = b,\quad u \circ f = g \circ u.
				\label{eq:thm_Peano_recursion_theorem}
			\end{align}
		\end{thm}
	\end{screen}
	
	\begin{prf}
		$X \times Y$の部分集合で
		\begin{itemize}
			\item $(a,b)$を含む
			\item $(x,y)$を含むなら$(f(x),g(y))$も含む
		\end{itemize}
		を満たすものの全体を$\mathscr{A}$で表し
		\begin{align}
			u \coloneqq \bigcap \mathscr{A}
		\end{align}
		とおく.このとき$u \in \mathscr{A}$であるが,一方で$u$は
		$X$から$Y$への写像になっている.これは
		\begin{align}
			S \coloneqq \Set{x \in X}{\mbox{$(x,y),(x,z) \in u$なら$y=z$}}
		\end{align}
		により定める$S$が$X$に一致することを示せばよい.
		\begin{description}
			\item[第一段] $a \in S$を示す.$b \neq c$となる$Y$の要素$c$に対し,
				$\mathscr{A}$の或る元$A$で$(a,c) \in A$となるとき,
				\begin{align}
					A' \coloneqq A \backslash \{(a,c)\}
				\end{align}
				もまた$\mathscr{A}$に属する.実際$(a,b)$は
				$A$から除かれていないから$(a,b) \in A$,かつ
				定理\ref{thm:successor_mapping_is_injective}より
				\begin{align}
					(x,y) \in A' \quad \Longrightarrow \quad
					(f(x),g(y)) \neq (a,b) \quad \Longrightarrow \quad
					(f(x),g(y)) \in A'
				\end{align}
				が満たされる.従って$b$と異なる$Y$の任意の要素$c$で
				$(a,c) \notin u$が成り立ち$a \in S$が得られる.
				
			\item[第二段] 
				$S$の任意の元$x$に対して或る$Y$の元$y$がただ一つ対応して$(x,y) \in u$となるが,
				このとき
				\begin{align}
					B \coloneqq (X \times Y) \backslash \Set{(x,z)}{z \in Y,\ y \neq z}
				\end{align}
				とおけば$B \in \mathscr{A}$が成り立つ.そして
				$w \neq g(y)$を満たす$Y$の任意の要素$w$に対して
				\begin{align}
					B' \coloneqq B \backslash \{(f(x),w)\}
				\end{align}
				もまた$\mathscr{A}$に属する.実際$a \neq f(x)$かつ
				$(a,b) \in B$より$(a,b) \in B'$となり,また$(s,t) \in B'$に対し
				\begin{itemize}
					\item $s \neq x$ならば$f(s) \neq f(x)$より$(f(s),g(t)) \in B'$,
					\item $s=x$ならば$t = y$より$(f(s),g(t)) = (f(x),g(y)) \in B'$,
				\end{itemize}
				が成立する.よって$w \neq g(y)$ならば$(f(x),w) \notin U$となり$f(x) \in S$が従う.
		\end{description}
		以上と数学的帰納法の原理より$S = X$を得る.すなわち$u$は写像であり,$u$の任意の元$(x,y)$で
		\begin{align}
			u(f(x)) = g(y) = g(u(x))
		\end{align}
		となるから$u \circ f = g \circ u$が成り立つ.また
		写像$v:\omega \longrightarrow X$が$v(a) = b$かつ
		$v \circ f = g \circ v$を満たすとき,
		\begin{itemize}
			\item $u(a) = b = v(a)$,
			\item $u(x) = v(x) \Longrightarrow 
				u(f(x)) = g(u(x)) = g(v(x)) = v(f(x))$
		\end{itemize}
		が成立するから$u = v$となる.よって(\refeq{eq:thm_Peano_recursion_theorem})
		を満たす写像は$u$のみである.
		\QED
	\end{prf}
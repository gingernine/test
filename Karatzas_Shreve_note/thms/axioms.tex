\subsection{公理系}
	数学で云う`集合'とは自動車や動物など`もの'の集まりを指すが,
	勝手なものの集まりが全て集合となるわけではない.
	ものの集まりが集合と呼ばれるためには
	`この日この時間にこの店で買い物をしていた全ての人間'や
	`あの人の机の上に現在置かれている全ての書類'というように
	{\bf 範囲が明確に指定されていなくてはならない.}
	ここで例として整数や実数などと挙げてしまうと,そもそも未だ数とは何かを規定していないため
	現段階では不適切な例となってしまう.しかしこの数というものについては後々に記述する.
	
	ところが,集合の作り方が無制限であると,たとえ範囲を明確に指定していても自己矛盾に陥る集まりが存在する.
	いま,仮に範囲が明確に指定された集まりならば全て集合と見做すことにして,
	次の分出公理と呼ばれるものを敷く.
	\begin{description}
		\item[分出公理]
	\end{description}
	言い換えれば,分出公理とは`集合の元をいくつか取ってきて集めたものもまた集合となる'という公理である.
	以上の規則の下では,次の不合理な主張が出てくる.
	
	\begin{itembox}[l]{Russellのパラドックス}
		自分自身を要素として持たない集まりの全体を
		\begin{align}
			X \coloneqq \Set{x}{x \notin x}
		\end{align}
		とおくとき,$X$は$X \in X$とも$X \notin X$ともなりえない.
	\end{itembox}
	
	注意するのは,今の段階では$\Set{x}{\mbox{$x$は集合である}}$が集合になるということである.
	従って分出公理により$X$もまた集合となる.
	
	\begin{screen}
		ZFC公理系
	\end{screen}
	
	\begin{screen}
		Peanoの自然数の公理
	\end{screen}
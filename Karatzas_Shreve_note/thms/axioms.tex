\subsection{公理系}
	\begin{screen}
		ZFC公理系
	\end{screen}
	
	\begin{screen}
		等号の公理
	\end{screen}
	
\subsection{自然数}
	`数とは何か'という問いは,つまるところ`自然数とは何者か'まで遡る.
	John Von Neumann は空集合$\emptyset$を用いて自然数を構成した:
	\begin{align}
		0 &\coloneqq \emptyset, \\
		1 &\coloneqq \{0\} = \{\emptyset\}, \\
		2 &\coloneqq \{0,1\} = \{\emptyset,\{\emptyset\}\}, \\
		3 &\coloneqq \{0,1,2\} = \{\emptyset,\{\emptyset\},\{\emptyset,\{\emptyset\}\}\}, \\
		&\vdots
	\end{align}
	は$0$を出発点とする自然数の構成例である.以下,自然数の体系を詳しく考察する.
	
	\begin{screen}
		\begin{dfn}[Peanoシステム]
			無限公理により
			\begin{itemize}
				\item $\emptyset$を含み
				\item 任意の要素$n$に対して$n \cup \{n\}$も要素として含む
			\end{itemize}
			を満たす集合が存在するが,このような集合を{\bf 後継者集合}
			\index{こうけいしゃしゅうごう@後継者集合}{\bf (successor set)}と呼ぶ.
			あらゆる後継者集合の共通部分を$\omega$で表し,写像$\sigma:\omega \longrightarrow
			\omega$を
			\begin{align}
				\sigma:n \longmapsto n \cup \{n\}
			\end{align}
			で定めるとき,$\sigma$を{\bf 後継者写像}
			\index{こうけいしゃしゃぞう@後継者写像}{\bf (successor mapping)},
			組$(\omega,\emptyset,\sigma)$を{\bf Peanoシステム}
			\index{Peanoしすてむ@Peanoシステム}{\bf (Peano system)}と呼ぶ.
		\end{dfn}
	\end{screen}
	
	\begin{screen}
		\begin{thm}[数学的帰納法の原理]
		\label{thm:the_principle_of_mathematical_induction}
			$(\omega,\emptyset,\sigma)$をPeanoシステムとするとき,
			任意の後継者集合$S$に対して
			\begin{align}
				S \subset \omega \quad \Longrightarrow \quad S = \omega
			\end{align}
			が成立する.この性質を{\bf 数学的帰納法の原理}
			\index{すうがくてききのうほうのげんり@数学的帰納法の原理}
			{\bf (the principle of mathematical induction)}と呼ぶ.
		\end{thm}
	\end{screen}
	
	\begin{prf}
		$\omega$は最小の後継者集合であるから$\omega \subset S$となり,
		外延性公理より$S = \omega$が出る.
		\QED
	\end{prf}
	
	\begin{screen}
		\begin{thm}[後継者写像は全単射]
		\label{thm:successor_mapping_is_injective}
			$(\omega,\emptyset,\sigma)$をPeanoシステムとするとき,
			後継者写像$\sigma$は次を満たす:
			\begin{description}
				\item[(1)] 任意の$n \in \omega$で$\emptyset \neq \sigma(n)$.
				\item[(2)] 任意の$n \in \omega$で$n \notin n$.
				\item[(4)] $\sigma$は$\omega$から$\omega \backslash \{\emptyset\}$
					への全単射である.
				
				\item[(3)] 任意の$n,m \in \omega$に対し,
					\begin{align}
						n \in m,\quad n = m,\quad m \in n
					\end{align}
					のいずれか一つは成立し,そしてどの二つも両立しない.
				\item[(5)] $E$を$\omega$の空でない部分集合とするとき,或る$k \in E$が存在して,
					任意の$m \in E$に対し$k \in m$又は$k = m$を満たす.
			\end{description}
		\end{thm}
	\end{screen}
	
	\begin{prf}\mbox{}
		\begin{description}
			\item[(1)] 任意の$n \in \omega$に対し$\sigma(n) = n \cup \{n\}$は
				$n$を元として持つから空ではない.
				
			\item[(2)] $S \coloneqq \Set{n \in \omega}{\mbox{任意の$x$に対し$n \subset x \Longrightarrow x \notin n$}}$とおくとき,
				先ず$\emptyset \in S$が成り立つ.また$n \in S$に対し
				\begin{align}
					\sigma(n) \subset x \quad \Longrightarrow \quad
					n \subset x \quad \Longrightarrow x \notin n
				\end{align}
				となり,一方で$n \notin n$より$\sigma(n) \not\subset n$も満たされるから,
				$\sigma(n) \subset x$なら$x \notin n \cup \{n\} = \sigma(n)$が従う.
				これにより$\sigma(n)$は$S$に属するから,
				数学的帰納法の原理より$S = \omega$が成立する.
				
			\item[(3)] $\sigma(n) = \sigma(m)$のとき
				$n \in m \cup \{m\}$かつ$m \in n \cup \{n\}$が満たされるが,このとき論理式では
				\begin{align}
					(n \in m \cup \{m\}) \wedge (m \in n \cup \{n\})
					&\equiv \left( (n \in m) \vee (n=m) \right) \wedge 
						\left( (m \in n) \vee (n=m) \right) \\
					&\equiv \left( (n \in m) \wedge (m \in n) \right) \vee (n = m) \\
					&\equiv (n = m)
				\end{align}
				が成立する((2)より$(n \in m) \wedge (m \in n)$は偽である).
		\end{description}
	\end{prf}
	
	\begin{screen}
		\begin{thm}[再帰定理]\label{thm:Peano_recursion_theorem}
			$X$を空でない集合,$a$を$X$の点,$f$を$X$から$X$への写像とし,
			$(\omega,\emptyset,\sigma)$をPeanoシステムとする.
			このとき,次を満たすような写像$u:\omega \longrightarrow X$がただ一つ存在する:
			\begin{align}
				u(\emptyset) = a,\quad u \circ \sigma = f \circ u.
				\label{eq:thm_Peano_recursion_theorem}
			\end{align}
		\end{thm}
	\end{screen}
	
	\begin{prf}
		$\omega \times X$の部分集合で,
		\begin{itemize}
			\item $(\emptyset,a)$を含む
			\item $(n,x)$を含むなら$(\sigma(n),f(x))$も含む
		\end{itemize}
		を満たすものの全体を$\mathscr{A}$で表し
		\begin{align}
			U \coloneqq \bigcap \mathscr{A}
		\end{align}
		とおく.このとき$U \in \mathscr{A}$であるが,一方で$U$は
		$\omega$から$X$への或る写像のグラフになっている.これは
		\begin{align}
			S \coloneqq \Set{n \in \omega}{\mbox{$(n,x),(n,y) \in U$なら$x=y$}}
		\end{align}
		により定める$S$が$\omega$に一致することを示せばよい.
		\begin{description}
			\item[第一段] $\emptyset \in S$を示す.$a$とは異なる$b \in X$に対し,
				或る$A \in \mathscr{A}$で$(\emptyset,b) \in A$となるとき,
				\begin{align}
					A' \coloneqq A \backslash \{(\emptyset,b)\}
				\end{align}
				もまた$\mathscr{A}$に属する.実際$(\emptyset,a)$は
				$A$から除かれていないから$(\emptyset,a) \in \mathscr{A}$,かつ
				定理\ref{thm:successor_mapping_is_injective}より
				\begin{align}
					(n,x) \in A' \quad \Longrightarrow \quad
					(\sigma(n),f(x)) \neq (\emptyset,b) \quad \Longrightarrow \quad
					(\sigma(n),f(x)) \in A'
				\end{align}
				が満たされる.従って全ての$b \in X,\ (a \neq b)$で
				$(\emptyset,b) \notin U$が成り立ち$\emptyset \in S$が得られる.
				
			\item[第二段] 
				任意に$n \in S$を取れば或る$x \in X$がただ一つ対応して$(n,x) \in U$となるが,
				このとき
				\begin{align}
					B \coloneqq (\omega \times X) \backslash \Set{(n,y)}{y \in X,\ x \neq y}
				\end{align}
				とおけば$B \in \mathscr{A}$が成り立つ.そして
				$z \neq f(x)$を満たす任意の$z \in X$に対して
				\begin{align}
					B' \coloneqq B \backslash \{(\sigma(n),z)\}
				\end{align}
				もまた$\mathscr{A}$に属する.実際$\emptyset \neq \sigma(n)$かつ
				$(\emptyset,a) \in B$より$(\emptyset,a) \in B'$となり,また$(m,t) \in B'$に対し
				\begin{align}
					\begin{cases}
						m \neq n \ \Longrightarrow \ \sigma(m) \neq \sigma(n)
						\ \Longrightarrow\ (\sigma(m),f(t)) \in B', & \\
						m=n\ \Longrightarrow\ t = x\ \Longrightarrow
						\ (\sigma(m),f(t)) = (\sigma(n),f(x)) \in B'
					\end{cases}
				\end{align}
				が成立する.よって$z \neq f(x)$ならば$(\sigma(n),z) \notin U$が満たされ
				$\sigma(n) \in S$が従う.
		\end{description}
		以上と数学的帰納法の原理より$S = \omega$を得る.すなわち$U$をグラフとする写像
		$u:\omega \longrightarrow X$が存在し,任意の$n \in \omega$で
		\begin{align}
			u(\sigma(n)) = f(x) = f(u(n)),
			\quad (\mbox{ただし$x$は$(n,x) \in U$を満たすもの})
		\end{align}
		となるから$u \circ \sigma = f \circ u$が出る.また
		写像$v:\omega \longrightarrow X$が$v(\emptyset) = a$かつ
		$v \circ \sigma = f \circ v$を満たすとき,
		\begin{itemize}
			\item $u(\emptyset) = a = v(\emptyset)$,
			\item $u(n) = v(n) \quad \Longrightarrow \quad 
				u(\sigma(n)) = f(u(n)) = f(v(n)) = v(\sigma(n))$
		\end{itemize}
		が成立するから$u = v$となる.よって(\refeq{eq:thm_Peano_recursion_theorem})
		を満たす写像は$u$のみである.
		\QED
	\end{prf}
\subsection{順序数}
	\monologue{
		院生「$1,2,3,\cdots$で表される数字は,集合論において
			\begin{align}
				0 &= \emptyset, \\
				1 &= \{0\} = \{\emptyset\}, \\
				2 &= \{0,1\} = \{\emptyset,\{\emptyset\}\}, \\
				3 &= \{0,1,2\} = \{\emptyset,\{\emptyset\},\{\emptyset,\{\emptyset\}\}\}, \\
				&\vdots
			\end{align}
			で定められます.上の操作を受け継いで``頑張れば手で書き出せる''類を自然数と呼びます.
			$\emptyset$は集合であり,対集合の公理がありますから$1$もまた集合です.
			そして和集合の公理を使えば$2$が集合であること,更には$3,4,\cdots$と続く自然数が全て集合であることが判るでしょう.
			自然数の冪も自然数同士の集合演算もその結果は全て集合になり,我々は
			そのように素姓の確定しているもののみを集合として扱おうとしていたのです.
			しかし上の操作をいくら続けたところで``要素を数えられる''集合しか作れません.
			上の帰納的な方法では`無限個の要素を持つ集合'は作れないのです.
			というわけで,有限と無限の隔たりを埋めるためには公理が要るでしょう.
			順序数とは自然数の拡張です.
			順序数の要素はまた順序数です.また順序数同士に対しては$\in$と$\subsetneq$の概念が一致します.」
	}
	
	\begin{itemize}
		\item $a \subset \mathrm{OR} \wedge a \neq \emptyset 
			\Longrightarrow \exists x \in a\ \forall y \in a\ (\ x \subset y\ )$.
		
		\item $\forall a\ (\ a \in \mathrm{OR} \Longrightarrow a \subset \mathrm{OR}\ )$
		
		\item $\forall a\ (\ a \subset \mathrm{OR} \Longrightarrow \bigcup a \in \mathrm{OR}\ )$
		
		\item $\forall a,b \in \mathrm{OR}\ (\ a \in b \Longleftrightarrow a \subsetneq b\ )$
		
		\item $R = \Set{x}{\exists a,b \in \mathrm{OR}\ (\ x=(a,b) \wedge a \subset b\ )}$は全順序
		
		\item $\mathrm{OR} \notin V$.
	\end{itemize}
	
	\begin{screen}
		\begin{dfn}[差集合]
			\begin{align}
				a \backslash b = \Set{x}{x \in a \wedge x \notin b}
			\end{align}
		\end{dfn}
	\end{screen}
	
	\begin{screen}
		\begin{thm}[順序数は整列集合]
			\begin{align}
				\forall x,y\ \left(\ Ord(x) \wedge y \subset x \wedge y \neq \emptyset \Longrightarrow 
				\exists z \in y\ \forall t \in y\ (\ z=t \vee z \in t\ )\ )\ \right)
			\end{align}
		\end{thm}
	\end{screen}
	
	\begin{screen}
		\begin{thm}
			\begin{align}
				\forall x,y\ \left(\ Ord(x) \wedge Tran(y) \Longrightarrow 
				(\ y \in x \Longleftrightarrow y \subset x)\ \right)
			\end{align}
		\end{thm}
	\end{screen}
	
	\begin{screen}
		\begin{thm}[超限帰納法]
			$A$を文字$x$のみを自由変項とする式とするとき,次が成立する:
			\begin{align}
				\forall \alpha \in \mathrm{OR}\ 
				\left(\ \forall \beta \in \alpha\ A(\beta)
				\Longrightarrow A(\alpha)\ \right)
				\Longrightarrow \forall \alpha \in \mathrm{OR}\ A(\alpha).
			\end{align}
		\end{thm}
	\end{screen}
	
	\begin{screen}
		\begin{axm}[無限公理]
			\begin{align}
				\exists x \in V\ (\ \emptyset \in x \wedge \forall y\ (\ y \in x
				\Longrightarrow y \cup \{y\} \in x\ )\ ).
			\end{align}
		\end{axm}
	\end{screen}
	
	\begin{screen}
		\begin{thm}[極限数の存在定理]
			極限数は存在する.
		\end{thm}
	\end{screen}
	
	\begin{prf}
		$a$を無限集合として$b = a \cap \mathrm{OR}$とおくとき,
		$\bigcup b$が極限数となることを示す.
		$\alpha \in \bigcup b$なら
		或る集合$x$が$x \in b \wedge \alpha \in x$を満たす.このとき
		$\alpha \cup \{\alpha\} \in x$または$\alpha \cup \{\alpha\} = x$となるが,
		前者の場合は$\alpha \cup \{\alpha\} \in \bigcup b$,
		後者の場合は$\alpha \cup \{\alpha\} \in x \cup \{x\}$および
		$x \cup \{x\} \in a$かつ$x \cup \{x\} \in \mathrm{OR}$より
		$x \cup \{x\} \in b$,ゆえに$\alpha \cup \{\alpha\} \in \bigcup b$,
		従って$\operatorname*{Tran}(\bigcup b)$が成立.
	\end{prf}
	
	\monologue{
		院生「無限公理から極限数の存在が示されましたが,無限公理を仮定せず
			代わりに極限数の存在を公理に採用しても無限公理の主張は導かれます.
			どちらを公理とするかは嗜好に依るでしょうが,本稿の論理の流れとしては
			極限数の存在を公理とした方が自然に感じられますね.しかし
			無限公理の方が主張が簡単ですし,他の書物ではこちらを公理としているようですから
			多数派に合わせるのが良いでしょうか.」
	}
	
	\begin{screen}
		\begin{thm}[${\bf \omega}$は最小の無限集合]
		\label{thm:the_principle_of_mathematical_induction}
			${\bf \omega}$は次の意味で最小の無限集合である:
			\begin{align}
				\forall a\ \left(\ \emptyset \in a \wedge \forall x\ 
				(\ x \in a \Longrightarrow x \cup \{x\} \in a\ ) 
				\Longrightarrow {\bf \omega} \subset a\ \right).
			\end{align}
		\end{thm}
	\end{screen}
	
	\begin{prf}
		$\alpha$を任意に与えられた順序数とし,
		$\alpha$の任意の要素$\beta$に対して
		\begin{align}
			\beta \in {\bf \omega} \Longrightarrow \beta \in a
		\end{align}
		が成立すると仮定する.このとき$\alpha \in {\bf \omega}$なら
		$\alpha = \beta \cup \{\beta\}$を満たす順序数$\beta$に対して
		$\beta \in a$となるから$\alpha \in a$となり,
		\begin{align}
			\alpha \in {\bf \omega} \Longrightarrow \alpha \in a
		\end{align}
		が従う.超限帰納法により
		\begin{align}
			\forall \alpha\ (\ \alpha \in {\bf \omega} \Longrightarrow \alpha \in a\ )
		\end{align}
		となるから$\omega \subset a$が出る.
		\QED
	\end{prf}
	
	\monologue{
		院生「定理\ref{thm:the_principle_of_mathematical_induction}で示された
			${\bf \omega}$の性質は{\bf 数学的帰納法の原理}
			\index{すうがくてききのうほうのげんり@数学的帰納法の原理}
			{\bf (the principle of mathematical induction)}と呼ばれます.
			高校数学だとドミノ倒しに喩えられる数学的帰納法ですが,
			なぜ数学的帰納法による証明が正しいのか簡単に説明いたしましょう.
			」
	}
	
	\begin{screen}
		\begin{thm}[超限帰納法による写像の構成]
			類$G$を$V$上の写像とするとき,
			\begin{align}
				K = \Set{f}{\exists \alpha \in \mathrm{OR}\ \left(\ f:\alpha \longrightarrow V \wedge \forall \beta \in \alpha\ (\ f(\beta) = G(f|_\beta)\ )\ \right)}
			\end{align}
			で$K$を定めて$F = \bigcup K$とおけば,
			\begin{itemize}
				\item $F$は$\mathrm{OR}$上の写像となる.
				\item $\forall \alpha \in \mathrm{OR}\ (\ F(\alpha) = G(F|_\alpha)\ )$が成り立つ.
			\end{itemize}
		\end{thm}
	\end{screen}
	
	\begin{prf}\mbox{}
		\begin{description}
			\item[第一段] $F$が写像であることを示す.
				まず$K$の任意の要素は$V \times V$の部分集合であるから
				\begin{align}
					F \subset V \times V
				\end{align}
				となる.そして$x,y,z$を任意の集合として
				$(x,y) \in F$かつ$(x,z) \in F$であるとすれば,
				$K$の或る要素$f$と$g$が存在して
				\begin{align}
					(x,y) \in f \wedge (x,z) \in g
				\end{align}
				を満たす.前段より$y = f(x) = g(x) = z$となるので
				$F$はsingle-valuedである.
			
			\item[第二段] $\operatorname{dom}(F) \subset \mathrm{OR}$が成り立つことを示す.
				実際
				\begin{align}
					\operatorname{dom}(F) = \bigcup_{f \in K} \operatorname{dom}(f)
				\end{align}
				かつ$\forall f \in K,\ \operatorname{dom}(f) \subset \mathrm{OR}$だから
				$\operatorname{dom}(F) \subset \mathrm{OR}$.
				
			\item[第三段] $\operatorname{Tran}(\operatorname{dom}(F))$であることを示す.
				実際$y \in x \wedge x \in \operatorname{dom}(F)$であるとき,
				或る$f \in K$で$x \in \operatorname{dom}(f)$.
				$\operatorname{dom}(f) \in \mathrm{OR}$より
				$y \in \operatorname{dom}(f)$.
				ゆえに$y \in \operatorname{dom}(F)$.
				
			\item[第四段] $\forall \alpha \in \operatorname{dom}(F)\ (\ F(\alpha) = G(F|_\alpha)\ )$が成り立つことを示す.
				実際,$\alpha \in \operatorname*{dom}(F)$なら
				或る$f$で$\alpha \in \operatorname*{dom}(f)$.
				$f \subset F$より
				\begin{align}
					f(\alpha) = F(\alpha).
				\end{align}
				他方$f|_\alpha = f \cap (\alpha \times V)
				= F \cap (\alpha \times V) = F|_\alpha$より
				\begin{align}
					G(f|_\alpha) = G(F|_\alpha).
				\end{align}
				$f(\alpha) = G(f|_\alpha)$と併せて$F(\alpha) = G(F|_\alpha)$を得る.
			
			\item[第五段] $\forall \beta \in \alpha\ (\ \beta \in \operatorname{dom}(F)\ )
				\Longrightarrow \alpha \in \operatorname{dom}(F)$が成り立つことを示す.
				$\forall \beta \in \alpha\ (\ \beta \in \operatorname{dom}(F)\ )$
				が成り立っているとして
				\begin{align}
					f = F|_\alpha
				\end{align}
				とおけば$f \in K$.このとき$f' = f \cup \{(\alpha,G(f))\}$も
				$K$に属するので
				\begin{align}	
					\alpha \in \operatorname{dom}(f') \subset
					\operatorname{dom}(F)
				\end{align}
				が成立する.前段の結果と超限帰納法より
				\begin{align}
					\mathrm{OR} = \operatorname{dom}(F)
				\end{align}
				を得る.
		\end{description}
	\end{prf}
	
	\begin{screen}
		\begin{thm}
			\begin{align}\label{thm:R_alpha_plus_1_equals_to_power_of_R_alpha}
				\forall \alpha \in \mathrm{OR}\ 
				\left(\ R(\alpha + 1) = \operatorname{P}(R(\alpha))\ \right)
			\end{align}
		\end{thm}
	\end{screen}
	
	\begin{prf}\mbox{}
		\begin{description}
			\item[第一段] $R(\alpha + 1) = R(\alpha) \cup \operatorname{P}(R(\alpha))$
				となることを示す.
				
			\item[第二段] $\alpha$を任意に与えられた空でない順序数とするとき,
				\begin{align}
					\forall \beta \in \alpha\ 
					\left(\ R(\beta + 1) \subset \operatorname{P}(R(\beta))\ \right)
					\Longrightarrow R(\alpha + 1) \subset \operatorname{P}(R(\alpha))
				\end{align}
				が成り立つことを示す.いま
				\begin{align}
					\forall \beta \in \alpha\ 
					\left(\ R(\beta + 1) \subset \operatorname{P}(R(\beta))\ \right)
					\label{eq:thm_R_alpha_plus_1_equals_to_power_of_R_alpha}
				\end{align}
				が成り立つと仮定する.$x$を$R(\alpha + 1)$の任意の要素とすれば,前段の結果より
				\begin{align}
					x \in R(\alpha) \vee x \subset R(\alpha)
				\end{align}
				となる.$x \in R(\alpha)$であるとき,$\alpha$の或る要素$\beta$に対し
				$x \in R(\beta)$となる.前段の結果より$x \in R(\beta + 1)$となり,
				(\refeq{eq:thm_R_alpha_plus_1_equals_to_power_of_R_alpha})より
				$x \subset R(\beta)$となるが,
				\begin{align}
					x \subset R(\beta) &\Longrightarrow x \subset R(\alpha), \\
					x \subset R(\alpha) &\Longrightarrow x \in \operatorname{P}(R(\alpha))
				\end{align}
				と併せて$x \in \operatorname{P}(R(\alpha))$が成り立つ.
				一方で$x \subset R(\alpha)$であるときも$x \in \operatorname{P}(R(\alpha))$
				となるから
				\begin{align}
					R(\alpha + 1) \subset \operatorname{P}(R(\alpha))
				\end{align}
				が従う.超限帰納法より定理の主張が得られる.
		\end{description}
	\end{prf}
	
	\begin{screen}
		\begin{thm}[$V$はwell-founded集合で尽くされる]
			\begin{align}
				V = \bigcup_{\alpha \in \mathrm{OR}} R(\alpha).
			\end{align}
		\end{thm}
	\end{screen}
	
	\begin{prf}
		いま,$S$を$\mathrm{OR}$の空でない部分集合とする.そして
		\begin{align}
			V \neq \bigcup_{\alpha \in S} R(\alpha)
			\Longrightarrow S \neq \mathrm{OR}
		\end{align}
		が成り立つことを示す.$V \neq \bigcup_{\alpha \in S} R(\alpha)$であれば
		正則性公理より或る集合$a$が存在して
		\begin{align}
			a \in V \backslash \bigcup_{\alpha \in S} R(\alpha)
			\wedge a \cap V \backslash \bigcup_{\alpha \in S} R(\alpha) = \emptyset
		\end{align}
		を満たす.このとき
		\begin{align}
			a \in \bigcup_{\alpha \in S} R(\alpha) \wedge a \subset \bigcup_{\alpha \in S} R(\alpha)
		\end{align}
		となる.ここで
		\begin{align}
			f = \Set{x}{\exists s \in a\ (\ x = (s,\mu \alpha (s \in R(\alpha)))\ )}
		\end{align}
		と定めれば$f:a \longrightarrow \mathrm{OR}$が成り立つ.
		$\beta = \bigcup f(a)$とおけば$\beta$は$\mathrm{OR}$に属する.このとき
		\begin{align}
			\forall t\ (\ t \in a \Longrightarrow t \in R(f(t))
			\Longrightarrow t \in R(\beta)\ )
		\end{align}
		となるから$a \subset R(\beta)$,そして定理\ref{thm:R_alpha_plus_1_equals_to_power_of_R_alpha}
		より$a \in R(\beta + 1)$が従う.
		\begin{align}
			\forall \alpha \in S\ (\ a \notin R(\alpha)\ )
		\end{align}
		であったから$\beta + 1 \notin S$であり,ゆえに$S \neq \mathrm{OR}$となる.
		定理の主張は対偶を取れば得られる.
		\QED
	\end{prf}
	
	\begin{screen}
		\begin{dfn}[Peanoシステム]
			次を満たす集合$X,a$と写像$f:X \longrightarrow X$の組$(X,a,f)$を
			{\bf Peanoシステム}\index{Peanoしすてむ@Peanoシステム}と呼ぶ:
			\begin{itemize}
				\item $a \in X$.
				\item $a \notin f(X)$.
				\item $f$は単射である.
				\item 集合$S$が$a$を含み,かつ$S$の任意の要素$x$に対し$f(x) \in S$となるならば,$S = X$.
			\end{itemize}
		\end{dfn}
	\end{screen}
	
	\begin{screen}
		\begin{thm}[$\omega$はPeanoシステム]
			写像$\sigma:\omega \longrightarrow \omega$を
			\begin{align}
				\sigma = \Set{(n, n \cup \{n\})}{n \in \omega}
			\end{align}
			で定めるとき,$(\omega,\emptyset,\sigma)$はPeanoシステムとなる.
			この$\sigma$を{\bf 後継者写像}\index{こうけいしゃしゃぞう@後継者写像}{\bf (successor mapping)}と呼ぶ.
		\end{thm}
	\end{screen}
	
	\begin{prf}
		$\emptyset \in \omega$である.また任意の自然数$n$に対し$\sigma(n) = n \cup \{n\} \neq \emptyset$となる.
		任意の自然数$n,m$に対して
		\begin{align}
			\sigma(x)=\sigma(y) \Longrightarrow &(x \in y \cup \{y\}) \wedge (y \in x \cup \{x\}) \\
					&\Longleftrightarrow \left( x \in y \vee x=y \right) \wedge 
						\left( y \in x \vee y=x \right) \\
					&\Longleftrightarrow  \left( x \in y \wedge y \in x \right) \vee x = y \\
					&\Longleftrightarrow x = y
		\end{align}
		が成立するから$\sigma$は単射である.
	\end{prf}
	
	\begin{screen}
		\begin{thm}[Peanoシステムの写像は全単射]\label{thm:successor_mapping_is_injective}
			Peanoシステム$(X,a,f)$の$f$は$X$から$X \backslash \{a\}$への全単射である.
		\end{thm}
	\end{screen}
	
	\begin{prf}
		$S \coloneqq \{a\} \cup \sigma(X)$とおけば,数学的帰納法の原理より$S = X$が成り立ち$f$の全射性が出る.
		\QED
	\end{prf}
	
	\begin{screen}
		\begin{thm}[再帰定理]\label{thm:Peano_recursion_theorem}
			$Y$を空でない集合,$b$を$Y$の要素,$g$を$Y$から$Y$への写像とし,
			$(X,a,f)$をPeanoシステムとする.
			このとき,次を満たすような写像$u:X \longrightarrow Y$がただ一つ存在する:
			\begin{align}
				u(a) = b,\quad u \circ f = g \circ u.
				\label{eq:thm_Peano_recursion_theorem}
			\end{align}
		\end{thm}
	\end{screen}
	
	\begin{prf}
		$X \times Y$の部分集合で
		\begin{itemize}
			\item $(a,b)$を含む
			\item $(x,y)$を含むなら$(f(x),g(y))$も含む
		\end{itemize}
		を満たすものの全体を$\mathscr{A}$で表し
		\begin{align}
			u \coloneqq \bigcap \mathscr{A}
		\end{align}
		とおく.このとき$u \in \mathscr{A}$であるが,一方で$u$は
		$X$から$Y$への写像になっている.これは
		\begin{align}
			S \coloneqq \Set{x \in X}{\mbox{$(x,y),(x,z) \in u$なら$y=z$}}
		\end{align}
		により定める$S$が$X$に一致することを示せばよい.
		\begin{description}
			\item[第一段] $a \in S$を示す.$b \neq c$となる$Y$の要素$c$に対し,
				$\mathscr{A}$の或る元$A$で$(a,c) \in A$となるとき,
				\begin{align}
					A' \coloneqq A \backslash \{(a,c)\}
				\end{align}
				もまた$\mathscr{A}$に属する.実際$(a,b)$は
				$A$から除かれていないから$(a,b) \in A$,かつ
				定理\ref{thm:successor_mapping_is_injective}より
				\begin{align}
					(x,y) \in A' \quad \Longrightarrow \quad
					(f(x),g(y)) \neq (a,b) \quad \Longrightarrow \quad
					(f(x),g(y)) \in A'
				\end{align}
				が満たされる.従って$b$と異なる$Y$の任意の要素$c$で
				$(a,c) \notin u$が成り立ち$a \in S$が得られる.
				
			\item[第二段] 
				$S$の任意の元$x$に対して或る$Y$の元$y$がただ一つ対応して$(x,y) \in u$となるが,
				このとき
				\begin{align}
					B \coloneqq (X \times Y) \backslash \Set{(x,z)}{z \in Y,\ y \neq z}
				\end{align}
				とおけば$B \in \mathscr{A}$が成り立つ.そして
				$w \neq g(y)$を満たす$Y$の任意の要素$w$に対して
				\begin{align}
					B' \coloneqq B \backslash \{(f(x),w)\}
				\end{align}
				もまた$\mathscr{A}$に属する.実際$a \neq f(x)$かつ
				$(a,b) \in B$より$(a,b) \in B'$となり,また$(s,t) \in B'$に対し
				\begin{itemize}
					\item $s \neq x$ならば$f(s) \neq f(x)$より$(f(s),g(t)) \in B'$,
					\item $s=x$ならば$t = y$より$(f(s),g(t)) = (f(x),g(y)) \in B'$,
				\end{itemize}
				が成立する.よって$w \neq g(y)$ならば$(f(x),w) \notin U$となり$f(x) \in S$が従う.
		\end{description}
		以上と数学的帰納法の原理より$S = X$を得る.すなわち$u$は写像であり,$u$の任意の元$(x,y)$で
		\begin{align}
			u(f(x)) = g(y) = g(u(x))
		\end{align}
		となるから$u \circ f = g \circ u$が成り立つ.また
		写像$v:\omega \longrightarrow X$が$v(a) = b$かつ
		$v \circ f = g \circ v$を満たすとき,
		\begin{itemize}
			\item $u(a) = b = v(a)$,
			\item $u(x) = v(x) \Longrightarrow 
				u(f(x)) = g(u(x)) = g(v(x)) = v(f(x))$
		\end{itemize}
		が成立するから$u = v$となる.よって(\refeq{eq:thm_Peano_recursion_theorem})
		を満たす写像は$u$のみである.
		\QED
	\end{prf}
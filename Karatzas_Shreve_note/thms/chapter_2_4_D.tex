\section{The Invariance Principle and the Wiener Measure}
	\begin{itembox}[l]{P.67 }
		$s = k/n$のとき
		\begin{align}
			\mathscr{F}_s^{X^{(n)}} = \sigma(\xi_1,\cdots,\xi_k).
		\end{align}
	\end{itembox}
	
	\begin{sketch}
		まず
		\begin{align}
			\mathscr{F}_s^{X^{(n)}} \subset \sigma(\xi_1,\cdots,\xi_k)
			\label{The_Invariance_Principle_and_the_Wiener_Measure_1}
		\end{align}
		を示す.$t \geq 0$かつ$[t] < t$なる実数$t$に対し,$Y_t$が
		\begin{align}
			\sigma(\xi_1,\cdots,\xi_{[t]+1})/\borel{\R}
		\end{align}
		可測であることを示す.実際,
		\begin{align}
			\omega \longmapsto \left(\xi_1(\omega),\cdots,\xi_{[t]+1}(\omega)\right)
		\end{align}
		は$\sigma(\xi_1,\cdots,\xi_{[t]+1})/\borel{\R^{[t]+1}}$-可測であって
		\begin{align}
			(x_1,\cdots,x_{[t]+1}) \longmapsto x_1 + \cdots + x_{[t]} + (t-[t])x_{[t] + 1}
		\end{align}
		は$\borel{\R^{[t]+1}}/\borel{\R}$-可測であるから
		$Y_t$は$\sigma(\xi_1,\cdots,\xi_{[t]+1})/\borel{\R}$-可測である.
		$t = [t]$ならば$Y_t$は$\sigma(\xi_1,\cdots,\xi_{[t]})/\borel{\R}$-可測である.
		\begin{align}
			\R \ni x \longmapsto \frac{1}{\sigma \sqrt{n}} x
		\end{align}
		は連続であるから$\borel{\R}/\borel{\R}$-可測であって,従って$X^{(n)}_t$は
		
		$[nt] < nt$ならば$\sigma(\xi_1,\cdots,\xi_{[nt]+1})/\borel{\R}$-可測,
		
		$[nt] = nt$ならば$\sigma(\xi_1,\cdots,\xi_{[nt]})/\borel{\R}$-可測であって,いずれにせよ
		\begin{align}
			\sigma(\xi_1,\cdots,\xi_{[nt]+1})/\borel{\R}
		\end{align}
		可測となる.$r$を$0 \leq r \leq s$なる実数とすれば,
		\begin{align}
			nr < k \Longrightarrow [nr] + 1 \leq k
		\end{align}
		が成り立つので,$X^{(n)}_r$が$\sigma(\xi_1,\cdots,\xi_{k})/\borel{\R}$-可測であるとわかる.
		\begin{align}
			\bigcup_{0 \leq r \leq s} \Set{{X^{(n)}_r}^{-1}(A)}{A \in \borel{\R}}
			\subset \sigma(\xi_1,\cdots,\xi_{k})
		\end{align}
		が成り立つので(\refeq{The_Invariance_Principle_and_the_Wiener_Measure_1})が従う.
		次に
		\begin{align}
			\sigma(\xi_1,\cdots,\xi_k) \subset \mathscr{F}_s^{X^{(n)}} 
			\label{The_Invariance_Principle_and_the_Wiener_Measure_2}
		\end{align}
		を示す.$1 \leq j \leq k$なる整数$j$に対し
		\begin{align}
			\xi_j = \sigma \sqrt{n} \left(X^{(n)}_{j/n} - X^{(n)}_{(j-1)/n} \right)
		\end{align}
		となる.$X^{(n)}_{j/n}$も$X^{(n)}_{(j-1)/n}$も$\mathscr{F}_s^{X^{(n)}}/\borel{\R}$-可測であるから
		$\xi_j$も$\mathscr{F}_s^{X^{(n)}}/\borel{\R}$-可測である.ゆえに
		\begin{align}
			\bigcup_{j=1}^k \Set{{\xi_j}^{-1}(A)}{A \in \borel{\R}}
			\subset \mathscr{F}_s^{X^{(n)}}
		\end{align}
		が成り立つので(\refeq{The_Invariance_Principle_and_the_Wiener_Measure_2})が従う.
		\QED
	\end{sketch}
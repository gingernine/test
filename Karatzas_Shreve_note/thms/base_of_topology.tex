\subsection{開基}
	いま$S$を集合とし,$\mathscr{U}$を$\power{S}$の部分集合とし,
	$\mathscr{U} \cup \{S\}$の空でない有限部分集合の交叉の全体を
	\begin{align}
		\mathscr{B} \defeq \Set{\bigcap u}{
		\forall t\, \left(\, t \in u \Longrightarrow t \in \mathscr{U} \vee t = S\, \right) \wedge u \neq \emptyset \wedge \card{u} < \Natural}
	\end{align}
	とおく.このとき
	\begin{align}
		\mathscr{O} \defeq \Set{\bigcup b}{b \subset \mathscr{B}}
	\end{align}
	で定める$\mathscr{O}$は$S$上の位相構造である.
	
	\begin{description}
		\item[step1]
			$S$と$\emptyset$が$\mathscr{O}$に属することを確認する.まず
			\begin{align}
				S\ (= \bigcap \{S\}) \in \mathscr{B}
			\end{align}
			が成り立つので
			\begin{align}
				S \in \mathscr{O}
			\end{align}
			が従う.また
			\begin{align}
				\emptyset = \bigcup \emptyset		
			\end{align}
			より
			\begin{align}
				\emptyset \in \mathscr{O}
			\end{align}
			も成り立つ.
		
		\item[step2]
			$u$と$v$を$\mathscr{O}$の要素とするとき
			
		\item[step3]
			
	\end{description}
	
	以上で次を得る.
	
	\begin{screen}
		\begin{thm}[位相の生成]
			$S$を集合とし,$\mathscr{U}$を$\power{S}$の部分集合として
			\begin{align}
				\mathscr{B} \defeq \Set{\bigcap u}{
				\forall t\, \left(\, t \in u \Longrightarrow t \in \mathscr{U} \vee t = S\, \right) \wedge u \neq \emptyset \wedge \card{u} < \Natural}
			\end{align}
			とおく.このとき
			\begin{align}
				\mathscr{O} \defeq \Set{\bigcup b}{b \subset \mathscr{B}}
			\end{align}
			で定める$\mathscr{O}$は$S$上の位相構造である.
		\end{thm}
	\end{screen}
	
	$\mathscr{U}$を含む$S$上の位相の中で最小である.
	
	\begin{prf}
		$\mathscr{O}$は定め方より$S$と$\emptyset$を含む.また
		任意の$O_1 = \bigcup \Lambda_1,\ O_2=\bigcup \Lambda_2 \in \mathscr{O},\ 
		(\Lambda_1,\Lambda_2 \subset \mathscr{A})$に対し
		\begin{align}
			\Lambda \coloneqq
			\Set{I \cap J}{I \in \Lambda_1,\ J \in \Lambda_2} \subset \mathscr{A}
		\end{align}
		となるから
		\begin{align}
			O_1 \cap O_2 = \bigcup_{I \in \Lambda_1,\ J \in \Lambda_2} I \cap J
			= \bigcup \Lambda \in \mathscr{O}
		\end{align}
		が成立する.任意に$\emptyset \neq \mathscr{U} \subset \mathscr{O}$を取れば,
		各$U \in \mathscr{U}$に$U = \bigcup \Lambda_U$を満たす
		$\Lambda_U \subset \mathscr{A}$が対応し,このとき
		\begin{align}
			\bigcup_{U \in \mathscr{U}} \Lambda_U \subset \mathscr{A}
		\end{align}
		となるから
		\begin{align}
			\bigcup \mathscr{U} = \bigcup \Biggl(\bigcup_{U \in \mathscr{U}} \Lambda_U\Biggr)
			\in \mathscr{O}
		\end{align}
		が従う.$\mathscr{M}$を含む任意の位相は$\mathscr{A}$を含みかつその任意和で閉じるから$\mathscr{O}$を含む.
		\QED
	\end{prf}
	
	\begin{screen}
		\begin{dfn}[開基]
			位相空間$(S,\mathscr{O})$において,
			$\mathscr{O}$の部分集合$\mathscr{B}$で
			\begin{align}
				\mathscr{O}
				= \Set{\bigcup \mathscr{U}}{\mathscr{U} \subset \mathscr{B}}
			\end{align}
			を満たすものを$\mathscr{O}$の{\bf 開基}\index{かいき@開基}や
			{\bf 基底}\index{きてい@基底},{\bf 基}\index{き@基}{\bf (base)}と呼ぶ.
		\end{dfn}
	\end{screen}
	
	\begin{screen}
		\begin{thm}[Alexanderの定理]
		\end{thm}
	\end{screen}
	
	\begin{screen}
		\begin{dfn}[始位相]
			$f \in \mathscr{F}$を集合$S$から位相空間$(T_f,\mathscr{O}_f)$への写像とするとき,
			全ての$f \in \mathscr{F}$を連続にする最弱の位相を$S$の$\mathscr{F}$-始位相
			(initial topology)と呼ぶ.$\mathscr{F}$-始位相は次が生成する位相である:
			\begin{align}
				\bigcup_{f \in \mathscr{F}} \Set{f^{-1}(O)}{O \in \mathscr{O}_f}.
			\end{align}
		\end{dfn}
	\end{screen}
	
	\begin{screen}
		\begin{dfn}[Cartesian積の位相]
			
		\end{dfn}
	\end{screen}
	
	\begin{screen}
		\begin{dfn}[直積の位相]
			
		\end{dfn}
	\end{screen}
\section{一様可積分性}
	\begin{screen}
	\begin{thm}[一様可積分性の同値条件]\label{thm:appendix_uniform_integrability_equivalence}
		$(X,\mathscr{F},\mu)$を測度空間とし,$\mu(X)<\infty$とする.
		任意の添数集合$\Lambda$に対して$(f_\lambda)_{\lambda \in \Lambda}$を
		$\mathscr{F}/\borel{\C}$-可測関数の族とするとき,次の(1)と(2)は同値である:
		\begin{description}
			\item[(1)] $(f_\lambda)_{\lambda \in \Lambda}$が一様可積分.
			\item[(2)] 
				\begin{align}
					\sup{\lambda \in \Lambda}\int_X|f_\lambda|\ d\mu < \infty
					\label{eq:thm_appendix_uniform_integrability_equivalence_1}
				\end{align}
				かつ,任意の$\epsilon > 0$に対し或る$\delta > 0$が存在して次を満たす:
				\begin{align}
					\mu(B) < \delta
					\Rightarrow \sup{\lambda \in \Lambda}\int_B |f_\lambda|\ d\mu < \epsilon.
					\label{eq:thm_appendix_uniform_integrability_equivalence_2}
				\end{align}
		\end{description}
	\end{thm}
	\end{screen}
	
	\begin{prf}\mbox{}
		\begin{description}
			\item[第一段]
				$(1) \Rightarrow (2)$を示す.任意の$a > 0$に対して
				\begin{align}
					\int_X |f_\lambda|\ d\mu
					= \int_{\{|f_\lambda|>a\}} |f_\lambda|\ d\mu
						+ \int_{\{|f_\lambda| \leq a\}} |f_\lambda|\ d\mu
					\leq \sup{\lambda \in \Lambda}\int_{\{|f_\lambda|>a\}} |f_\lambda|\ d\mu
						+ a\mu(X)
				\end{align}
				が成り立ち,一様可積分性より或る$a > 0$に対して
				\begin{align}
					\sup{\lambda \in \Lambda}\int_{\{|f_\lambda|>a\}} |f_\lambda|\ d\mu < \infty
				\end{align}
				となるから(\refeq{eq:thm_appendix_uniform_integrability_equivalence_1})が従う.
				また任意の$B \in \mathscr{F}$に対して
				\begin{align}
					\int_B |f_\lambda|\ d\mu
					= \int_{\{|f_\lambda|>a\} \cap B} |f_\lambda|\ d\mu
						+ \int_{\{|f_\lambda| \leq a\} \cap B} |f_\lambda|\ d\mu
					\leq \sup{\lambda \in \Lambda}\int_{\{|f_\lambda|>a\}} |f_\lambda|\ d\mu
						+ a\mu(B)
				\end{align}
				が成り立つから,任意の$\epsilon > 0$に対して
				\begin{align}
					\sup{\lambda \in \Lambda}\int_{\{|f_\lambda|>a\}} |f_\lambda|\ d\mu < \frac{\epsilon}{2}
				\end{align}
				を満たす$a > 0$を取り$\delta \coloneqq \epsilon/(2a)$とおけば
				(\refeq{eq:thm_appendix_uniform_integrability_equivalence_2})が成立する.
				
			\item[第二段]
				$(2) \Rightarrow (1)$を示す.任意の$a > 0$に対して
				\begin{align}
					\mu(|f_\lambda| > a) \leq \frac{1}{a} \int_X |f_\lambda|\ d\mu
					\leq \frac{1}{a}\sup{\lambda \in \Lambda}{\int_X|f_\lambda|\ d\mu}
				\end{align}
				が成立するから,(\refeq{eq:thm_appendix_uniform_integrability_equivalence_2})
				を満たす$\delta > 0$に対し或る$a_0 > 0$が存在して
				\begin{align}
					\mu(|f_\lambda| > a) < \delta,
					\quad (\forall \lambda \in \Lambda,\ \forall a > a_0)
				\end{align}
				となり
				\begin{align}
					\int_{\{|f_\lambda|>a\}} |f_\lambda|\ d\mu < \epsilon,
					\quad (\forall \lambda \in \Lambda,\ \forall a > a_0)
				\end{align}
				が従う.
				\QED
		\end{description}
	\end{prf}
	
	\begin{screen}
	\begin{thm}[一様可積分性と平均収束]\label{lem:uniformly_integrable_and_convergence_in_mean}
		$(X,\mathscr{F},\mu)$を$\sigma$-有限測度空間とし,
		$(f_n)_{n=1}^\infty$を$\mathscr{F}/\borel{\C}$-可測関数の族とする.
		$(f_n)_{n=1}^\infty$が$\mu$-a.e.に$\C$で収束するとき,つまり或る零集合$A$が存在して,
		\begin{align}
			f \coloneqq \lim_{n \to \infty} f_n \defunc_A
		\end{align}
		により$\mathscr{F}/\borel{\C}$-可測関数が定まるとき,
		次の(1)と(2)は同値である:
		\begin{description}
			\item[(1)] $(f_n)_{n=1}^\infty$が一様可積分.
			\item[(2)] $f$が可積分で次を満たす:
				\begin{align}
					\int_X |f - f_n|\ d\mu 
					\longrightarrow 0
					\quad (n \longrightarrow \infty).
				\end{align}
		\end{description}
	\end{thm}
	\end{screen}
	
	\begin{screen}
	\begin{thm}[一様可積分性と条件付き期待値]\label{lem:uniformly_integrability_and_conditional_expectations}
		$(X,\mathscr{F},\mu)$を測度空間とする.
		部分$\sigma$-加法族$\mathscr{G} \subset \mathscr{F}$に対し
		$\left. \mu \right|_{\mathscr{G}}$が$\sigma$-有限なら,
		$\mu$-可積分関数$f:X \longrightarrow \R$に対し
		$\left( \cexp{f}{\mathscr{G}} \right)_{\mathscr{G} \subset \mathscr{F}}$は一様可積分である.
	\end{thm}
	\end{screen}
	
	\begin{prf}
		定理\ref{thm:properties_of_conditional_expectations}より
				\begin{align}
					\int_{\left| \cexp{f}{\mathscr{G}} \right| > \lambda} \left| \cexp{f}{\mathscr{G}} \right|\ d\mu
					\leq \int_{\cexp{|f|}{\mathscr{G}} > \lambda} \cexp{|f|}{\mathscr{G}}\ d\mu
					= \int_{\cexp{|f|}{\mathscr{G}} > \lambda} |f|\ d\mu
				\end{align}
				が成り立つ.また$X$の可積分性より,任意の$\epsilon > 0$に対して
				或る$\delta > 0$が存在し
				\begin{align}
					\mu(B) < \delta \Rightarrow \int_B |f|\ d\mu < \epsilon
				\end{align}
				が満たされる.いま,Chebyshevの不等式より
				\begin{align}
					\mu\left( \cexp{|f|}{\mathscr{G}} > \lambda \right)
					\leq \frac{1}{\lambda} \int_X \cexp{|f|}{\mathscr{G}}\ d\mu
					= \frac{1}{\lambda} \int_X |f|\ d\mu
				\end{align}
				となるから,$\epsilon > 0$に対し或る$\lambda_0 > 0$が存在して
				\begin{align}
					\sup{\mathscr{G} \subset \mathscr{F}}{\mu\left( \cexp{|f|}{\mathscr{G}} > \lambda \right)}
					< \delta,
					\quad (\forall \lambda > \lambda_0)
				\end{align}
				が満たされ
				\begin{align}
					\sup{\mathscr{G} \subset \mathscr{F}}{\int_{\cexp{|f|}{\mathscr{G}} > \lambda}|f|\ d\mu}
					< \epsilon,
					\quad (\forall \lambda > \lambda_0)
				\end{align}
				が従う.
		\QED
	\end{prf}
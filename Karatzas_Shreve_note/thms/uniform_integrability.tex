\subsection{一様可積分性}
	\begin{screen}
		\begin{dfn}[一様可積分]
			$(X,\mathscr{F},\mu)$を正値測度空間とし,$\mathscr{U}$を$\mathscr{L}^1(X,\mathscr{F},\mu)$の部分集合とする.
			\begin{itemize}
				\item $\epsilon$を任意に与えられた正数とすると,次を満たす正数$\delta$が取れる:
					\begin{align}
						\forall f \in \mathscr{U}\, \forall B \in \mathscr{F}\, \left(\, \mu(B) < \delta
						\Longrightarrow \int_B |f|\ d\mu < \epsilon\, \right).
					\end{align}
			\end{itemize}
			が満たされているとき,$\mathscr{U}$は{\bf 同程度可積分である}\index{どうていどかせきぶん@同程度可積分}
			{\bf (equi-integrable)}という.同程度可積分性に加えて次の式
			\begin{itemize}
				\item $\mathscr{U}$が$\mathscr{L}^1(X,\mathscr{F},\mu)$で有界である:
					\begin{align}
						\exists b \in \R_+\, \forall f \in \mathscr{U}\, 
						\left(\, \int_X|f|\ d\mu < b\, \right).
					\end{align}
			\end{itemize}
			が満たされているとき,$\mathscr{U}$は{\bf 一様可積分である}\index{いちようかせきぶん@一様可積分}
			{\bf (uniformly integrable)}という.
		\end{dfn}
	\end{screen}
	
	同程度可積分な集合の部分集合もまた同程度可積分であり,一様可積分な集合の部分集合もまた一様可積分である.
	
	\begin{screen}
	\begin{thm}[一様可積分性の同値条件]\label{thm:appendix_uniform_integrability_equivalence}
		$(X,\mathscr{F},\mu)$を測度空間とし,$\mathscr{U}$を$\mathscr{L}^1(X,\mathscr{F},\mu)$の部分集合とする.
		このとき次の(1)と(2)が成り立つ:
		\begin{description}
			\item[(1)] $\mathscr{U}$が一様可積分であるとき,$\epsilon$を任意に与えられた正数とすると,次を満たす正数$a$が取れる:
				\begin{align}
					\forall f \in \mathscr{U}\, \forall \lambda \in \R_+\,
					\left(\, a < \lambda \Longrightarrow \int_{\{|f| > \lambda\}} |f|\ d\mu < \epsilon\, \right).
				\end{align}
			
			\item[(2)] $\mu(X) < \infty$の場合(1)の逆が成立する.つまり,
				\begin{align}
					\forall \epsilon \in \R_+\, \exists a \in \R_+\, 
					\forall f \in \mathscr{U}\, \forall \lambda \in \R_+\,
					\left(\, a < \lambda \Longrightarrow \int_{\{|f| > \lambda\}} |f|\ d\mu < \epsilon\, \right)
				\end{align}
				が成り立つとき$\mathscr{U}$は一様可積分である.
		\end{description}
	\end{thm}
	\end{screen}
	
	\begin{sketch}\mbox{}
		\begin{description}
			\item[(1)]
				$\mathscr{U}$が一様可積分であるとする.
				いま$\epsilon$を任意に与えられた正数とする.このとき
				\begin{align}
					\forall f \in \mathscr{U}\, \forall B \in \mathscr{F}\, \left(\, \mu(B) < \delta
					\Longrightarrow \int_B |f|\ d\mu < \epsilon\, \right)
				\end{align}
				を満たす$\delta$が取れる.ここで
				\begin{align}
					\frac{1}{a}\sup{f \in \mathscr{U}}{\int_X|f|\ d\mu} < \delta
				\end{align}
				を満たす正の実数$a$を取れば,$a < \lambda$なる正数$\lambda$と$\mathscr{U}$の任意の要素$f$に対して
				\begin{align}
					\mu(|f| > \lambda) \leq \frac{1}{\lambda} \int_X |f|\ d\mu < \delta
				\end{align}
				となるので
				\begin{align}
					\forall f \in \mathscr{U}\, \forall \lambda \in \R_+\,
					\left(\, a < \lambda \Longrightarrow \int_{\{|f| > \lambda\}} |f|\ d\mu < \epsilon\, \right)
				\end{align}
				が成立する.
			
			\item[(2)]
				いま$\epsilon$を任意に与えられた正数とする.このとき
				\begin{align}
					\forall f \in \mathscr{U}\, 
					\left(\, \int_{\{|f| > a\}} |f|\ d\mu < \frac{\epsilon}{2}\, \right)
				\end{align}
				を満たす正数$a$が取れる.$f$を$\mathscr{U}$の任意の要素とし,$B$を$\mathscr{F}$の任意の要素とすれば
				\begin{align}
					\int_B |f|\ d\mu
					= \int_{\{|f|>a\} \cap B} |f|\ d\mu
						+ \int_{\{|f| \leq a\} \cap B} |f|\ d\mu
					\leq \frac{\epsilon}{2} + a\mu(B)
				\end{align}				
				が成り立つから,
				\begin{align}
					\sup{f \in \mathscr{U}}\int_X|f|\ d\mu < \infty
				\end{align}
				及び
				\begin{align}
					\forall B \in \mathscr{F}\, \left(\, \mu(B) < \frac{\epsilon}{2a}
					\Longrightarrow \int_B |f|\ d\mu < \epsilon\, \right)
				\end{align}
				が成立する.すなわち$\mathscr{U}$は一様可積分である.
				\QED
		\end{description}
	\end{sketch}
	
	\begin{screen}
	\begin{thm}[一様可積分性と平均収束]\label{lem:uniformly_integrable_and_convergence_in_mean}
		$(X,\mathscr{F},\mu)$を正値測度空間とし,
		\begin{align}
			\mu(X) < \infty
		\end{align}
		とする.また$\{f_n\}_{n=1}^\infty$を$\mathscr{L}^1(X,\mathscr{F},\mu)$の部分集合とし,
		$A$を$\mu$-零集合とし,$X \backslash A$の各点$x$で$(f_n(x))_{n=1}^\infty$が
		$\C$で収束するとする.このとき次の(1)と(2)は同値である:
		\begin{description}
			\item[(1)] $\{f_n\}_{n=1}^\infty$が一様可積分.
			\item[(2)] $f \defeq \lim_{n \to \infty} f_n \defunc_A$と$f$を定めると,$f$は可積分で
				\begin{align}
					\int_X |f - f_n|\ d\mu 
					\longrightarrow 0
					\quad (n \longrightarrow \infty).
				\end{align}
		\end{description}
	\end{thm}
	\end{screen}
	
	\begin{sketch}
		
	\end{sketch}
	
	\begin{screen}
	\begin{thm}[一様可積分性と条件付き期待値]\label{lem:uniformly_integrability_and_conditional_expectations}
		$(X,\mathscr{F},\mu)$を測度空間とし,$\mu(X) < \infty$とし,
		$f$を$\mathscr{L}^1(X,\mathscr{F},\mu)$の要素とする.
		また$\mathscr{S}$を$X$上の$\sigma$-加法族であり$\mathscr{F}$の部分集合であるものの全体とする.
		このとき$\left\{ \cexp{f}{\mathscr{G}} \right\}_{\mathscr{G} \in \mathscr{S}}$は一様可積分である.
	\end{thm}
	\end{screen}
	
	\begin{prf}
		定理\ref{thm:properties_of_conditional_expectations}より
				\begin{align}
					\int_{\left| \cexp{f}{\mathscr{G}} \right| > \lambda} \left| \cexp{f}{\mathscr{G}} \right|\ d\mu
					\leq \int_{\cexp{|f|}{\mathscr{G}} > \lambda} \cexp{|f|}{\mathscr{G}}\ d\mu
					= \int_{\cexp{|f|}{\mathscr{G}} > \lambda} |f|\ d\mu
				\end{align}
				が成り立つ.また$X$の可積分性より,任意の$\epsilon > 0$に対して
				或る$\delta > 0$が存在し
				\begin{align}
					\mu(B) < \delta \Rightarrow \int_B |f|\ d\mu < \epsilon
				\end{align}
				が満たされる.いま,Chebyshevの不等式より
				\begin{align}
					\mu\left( \cexp{|f|}{\mathscr{G}} > \lambda \right)
					\leq \frac{1}{\lambda} \int_X \cexp{|f|}{\mathscr{G}}\ d\mu
					= \frac{1}{\lambda} \int_X |f|\ d\mu
				\end{align}
				となるから,$\epsilon > 0$に対し或る$\lambda_0 > 0$が存在して
				\begin{align}
					\sup{\mathscr{G} \in \mathscr{S}}{\mu\left( \cexp{|f|}{\mathscr{G}} > \lambda \right)}
					< \delta,
					\quad (\forall \lambda > \lambda_0)
				\end{align}
				が満たされ
				\begin{align}
					\sup{\mathscr{G} \in \mathscr{S}}{\int_{\cexp{|f|}{\mathscr{G}} > \lambda}|f|\ d\mu}
					< \epsilon,
					\quad (\forall \lambda > \lambda_0)
				\end{align}
				が従う.
		\QED
	\end{prf}
	
	\begin{screen}
		\begin{thm}[Dunford-Pettis]\label{thm:Dunford_Pettis}
			$(X,\mathcal{B},\mu)$を確率空間とし,$\mathscr{F}$を$\mathscr{L}^1(X,\mathcal{B},\mu)$の
			セミノルムに関して有界な部分集合とする.また$q_1$を$\mathscr{L}^1(X,\mathcal{B},\mu)$から$L^1(X,\mathcal{B},\mu)$への商写像とする.
			このとき,$\mathscr{F}$が同程度可積分であることと,
			\begin{align}
				\mathscr{F}^q \defeq \Set{q_1(f)}{f \in \mathscr{F}}
			\end{align}
			が$L^1(X,\mathcal{B},\mu)$において弱位相に関して相対コンパクトであることは同値である.
		\end{thm}
	\end{screen}
	
	\begin{sketch}
		$q_\infty$を$\mathscr{L}^\infty(X,\mathcal{B},\mu)$から$L^\infty(X,\mathcal{B},\mu)$への商写像とする.
		\begin{align}
			T:L^1(X,\mathcal{B},\mu) \longrightarrow L^\infty(X,\mathcal{B},\mu)^*
		\end{align}
		なる写像$T$を
		\begin{align}
			T(q_1(f)):q_\infty(g) \longmapsto \int_X fg\ d\mu
		\end{align}
		なる関係により定めれば,$T$は等長線型である.従って
		\begin{align}
			T \ast \mathscr{F}^q
		\end{align}
		は$L^\infty(X,\mathcal{B},\mu)^*$において作用素ノルムに関して有界である.
		ゆえにこの集合は$L^\infty(X,\mathcal{B},\mu)^*$の零元を中心とした或る大きさの閉球に含まれるが,
		Banach-Alaogluの定理よりその閉球は汎弱位相でコンパクトである.ゆえに
		\begin{align}
			\mathscr{H}
		\end{align}
		を$T \ast \mathscr{F}^q$の汎弱位相に関する閉包とすれば,$\mathscr{H}$は汎弱コンパクトである.
		なので
		\begin{itemize}
			\item $\mathscr{H}$が$\ran{T}$に含まれていること.
			\item $T^{-1}|_{\mathscr{H}}$が$L^1(X,\mathcal{B},\mu)$の弱位相と
				$L^\infty(X,\mathcal{B},\mu)^*$の半弱位相に関して連続であること.
		\end{itemize}
		が示されれば,
		\begin{align}
			T^{-1} \ast \mathscr{H}
		\end{align}
		が$L^1(X,\mathcal{B},\mu)$の弱位相に関してコンパクトとなり,他方で弱位相はHausdorffなので
		$\mathscr{F}^q$の弱閉包は$T^{-1} \ast \mathscr{H}$に含まれ,
		$\mathscr{F}^q$が弱位相に関して相対コンパクトであることが従う.
	\end{sketch}
	
	\begin{description}
		\item[第一段] $\mathscr{H}$が$\ran{T}$に含まれていることを示す.
			$\varphi$を$\mathscr{H}$の要素とすれば,
			定理\ref{thm:closed_set_is_set_of_limits_of_some_net}より
			有向集合$(\Lambda,\preceq)$と
			\begin{align}
				\{\varphi_\lambda\}_{\lambda \in \Lambda} \subset T \ast \mathscr{F}^q
			\end{align}
			を満たすネット$(\varphi_\lambda)_{\lambda \in \Lambda}$が取れて,
			\begin{align}
				\varphi_\lambda \longrightarrow \varphi
			\end{align}
			が汎弱位相に関して成立する.選択公理より$\Lambda$の各要素$\lambda$に対して
			\begin{align}
				q_1(f_\lambda) = T^{-1}(\varphi_\lambda)
			\end{align}
			となる$f_\lambda$が取れる.仮定より
			\begin{align}
				\{f_\lambda\}_{\lambda \in \Lambda}
			\end{align}
			は同程度可積分であるから,$\epsilon$を任意に与えられた正数とすれば,
			\begin{align}
				\forall A \in \mathcal{B}\, \forall \lambda \in \Lambda\, 
				\left(\, \mu(A) < \delta \Longrightarrow \int_A |f_\lambda|\ d\mu < \epsilon\, \right)
			\end{align}
			を満たす正数$\delta$が取れる.$\varphi_\lambda \longrightarrow \varphi$より
			\begin{align}
				\varphi_\lambda(\defunc_A) \longrightarrow \varphi(\defunc_A)
			\end{align}
			が$\C$の位相で成立するので
			\begin{align}
				\forall A \in \mathcal{B}\, 
				\left(\, \mu(A) < \delta \Longrightarrow |\varphi(\defunc_A)| \leq \epsilon\, \right)
			\end{align}
			が従い,
			\begin{align}
				\mathcal{B} \ni A \longmapsto \varphi(\defunc_A)
			\end{align}
			の可算加法性が従い,定理\ref{thm:dual_space_of_L_p}より
			\begin{align}
				\varphi = T(q_1(f))
			\end{align}
			を満たす$f$が取れる.すなわち
			\begin{align}
				\varphi \in \ran{T}
			\end{align}
			が成り立つ.以上で
			\begin{align}
				\mathscr{H} \subset \ran{T}
			\end{align}
			が示された.
		
		\item[第二段] $\varphi$を$\mathscr{H}$の要素とし,$\varphi$において$T^{-1}|_{\mathscr{H}}$が連続であることを示す.
			$(\varphi_\lambda)_{\lambda \in \Lambda}$を汎弱位相に関して
			\begin{align}
				\varphi_\lambda \longrightarrow \varphi
			\end{align}
			を満たす$\mathscr{H}$上のネットとすれば,
			\begin{align}
				g \in L^\infty(X,\mathcal{B},\mu) \Longrightarrow \varphi_\lambda(g) \longrightarrow \varphi(g)
			\end{align}
			が成り立つので,定理\ref{thm:dual_space_of_L_p}と併せて
			\begin{align}
				\Phi \in L^1(X,\mathcal{B},\mu)^* \Longrightarrow 
				\Phi\left( T^{-1}(\varphi_\lambda) \right) \longrightarrow \Phi\left( T^{-1}(\varphi) \right)
			\end{align}
			が成り立つ.ゆえに,弱位相の意味で
			\begin{align}
				T^{-1}(\varphi_\lambda) \longrightarrow T^{-1}(\varphi)
			\end{align}
			が成り立つ.ゆえに,定理\ref{thm:continuity_and_net}より$T^{-1}|_{\mathscr{H}}$は,$L^1(X,\mathcal{B},\mu)$の弱位相と
			$L^\infty(X,\mathcal{B},\mu)^*$の半弱位相に関して$\varphi$において連続である.
			$\varphi$の任意性より$T^{-1}|_{\mathscr{H}}$は$\mathscr{H}$上で連続である.
			\QED
	\end{description}
\section{導入ーある日の研究室}\mbox{}\\
	院生I「数学とは何だろうか。」\\
	院生U「何だい唐突にそんなこと訊かれても、数学全体なんてとても見渡せるわけないんだから答えようがないさね。
		神のみぞ知るってことでFA。」\\
	院生S「まあまあそう冷たくあしらわんでも。そんなに高尚な答えを求めているんじゃないんだろう。俗っぽく回答してみてはどうかね。」\\
	院生H「数学ってのは、何千年も昔からあまたの男子を魅了してきたんだ。女性にたとえるのが妥当じゃないか。」\\
	院生N「女性でありながら人ならざる者。ああ、つまり数学は女神なのか。司るのは、真理の美ってところかな。」\\
	院生Y「女神とはいえその姿は玉のように美しい少女に違いないさ。論理的緻密さと嘘偽りのない潔白さを鑑みれば、
	その肌はきっと絹地のようにきめ細かくて輝いてるんだ。そして美しさに磨きをかけながら、未来永劫に成長し続けるんだよ!
	繰り返すけど数学は美少女。ここ重要ね。」\\
	院生S「なんだいその理屈は,別にカイコの女の子だって結構じゃないか。」\\
	院生Y「すると僕らはカイコの女の子を研究していることになっちまうぜ。
		もっとこう、僕らにゃ手の届かない高嶺の花みたいな喩えがぴったりくるだろう?」\\
	院生S「いやまあそうだけれど、にしても俗っぽいもので女神を引き出すとは。不敬罪で天誅が下るぜ。」\\
	院生K「女神と聞くとギリシア神話を想起するけど、どうも彼女ら俗っぽいぜ。まあそれもそうかな。
		何せあの神話の世界では、人間や社会の性質を分解して、一つ一つに特化したものが人格もとい神格を持ってるんだものね。」\\
	院生T「性質を抜き出して云々って、ギリシア神話もまるで数学みたいだな。するとあれかい、
		神話で起こる事件ってのは数学の定理に対応しているのかな。」\\
	院生Z「思わぬところで数学とギリシア神話が繋がったね。もう$\mbox{数学}=\mbox{女神っ子}$の図式がこびりついてしまったよ。」\\
	院生I「あゝ愛しの美少女。その美貌が僕を誘惑し夢中にさせてくれる。」\\
	院生U「Iちゃん、そういう趣味が…」\\
	院生I「あれ違うっ、前言撤回!あゝ愛しの数学、その美貌が僕を誘惑し夢中にさせてくれる!」\\
	ポスドク「盛り上がってるところ水を差して申し訳ないが、君らも数学者なんだから、
		数学とは何かを数学的に考察してみたらどうだい?
		こう言うと恰も自己言及的だけれど、ほら、以下の話を読んでごらん。」

\begin{comment}		
\section{共鳴箱定理}
	始めに,次の傲岸不遜な主張をアンサイクロペディアより引用する.
	
	\begin{screen}
		\begin{prp}[Andr$\acute{\mathrm{e}}$ Weilの共鳴箱定理]
			理論をつくるのが一流、あとの学者は(ry.
		\end{prp}
	\end{screen}
	この命題と個人的な事情から次の系を得る:
	`私は共鳴箱にすらなれない腑抜けである'.
	\begin{prf}
		私はヘタレて院を中退するので金輪際学者になりえない.
		\QED
	\end{prf}
\end{comment}
\subsection{級数}
	
	$a$が$\Natural$から$\C$への写像であるとき,つまり$a$が
	\begin{align}
		a: \Natural \longrightarrow \C
	\end{align}
	を満たすとき,$a$を{\bf 複素数列}と呼ぶ.$a$が複素数列であるときは
	\begin{align}
		a(n)
	\end{align}
	の代わりに
	\begin{align}
		a_n
	\end{align}
	と書いて,また$a$を
	\begin{align}
		(a_n)_{n \in \Natural}
	\end{align}
	とも表す.ここで$\C$における総和記号$\sum$の直感的意味を書いておくと,まず$0$に対して
	\begin{align}
		\sum_{k=0}^0 a_k \defeq a_0
	\end{align}
	と定め,次に$1$に対して
	\begin{align}
		\sum_{k=0}^1 a_k \defeq a_0 + a_1
	\end{align}
	と定め,次に$2$に対して
	\begin{align}
		\sum_{k=0}^2 a_k \defeq (a_0 + a_1) + a_2
	\end{align}
	と定め,$3$に対して
	\begin{align}
		\sum_{k=0}^3 a_k \defeq ((a_0 + a_1) + a_2) + a_3 = \left(\sum_{k=0}^2 a_k\right) + a_3
	\end{align}
	と定める.このような再帰的定義によって各自然数$n$に対して
	\begin{align}
		\sum_{k=0}^n a_k
	\end{align}
	なる複素数を対応させることは可能である.厳密な意味付けには超限帰納法を用いるがキリがないので略.
	\begin{align}
		\Natural \ni n \longmapsto \sum_{k=0}^n a_k
	\end{align}
	なる関係により定まる複素数列を$s$とするとき,$s$が$\C$で収束するなら,つまり
	\begin{align}
		|s_n - \alpha| \longrightarrow 0\quad (n \longrightarrow \infty)
	\end{align}
	を満たす複素数$\alpha$が取れるなら
	\begin{align}
		\sum_{n=0}^\infty a_n \defeq \alpha
	\end{align}
	と定め,これを$(a_n)_{n \in \Natural}$の{\bf 級数}\index{きゅうすう@級数}{\bf (series)}と呼ぶ.
	言い換えれば$\sum_{n=0}^\infty a_n$とは$s$が収束する場合の極限のことである.今度は
	\begin{align}
		\Natural \ni n \longmapsto \sum_{k=0}^n |a_k|
	\end{align}
	なる関係により定まる実数列を$t$とする.当然のようだが$t$が$\R$で収束すれば$s$は$\C$で収束する.実際,$t$が収束するなら
	$\epsilon$を任意に与えられた正の実数とすれば
	\begin{align}
		\forall n,m \in \Natural\, \left(\, N < n \wedge N < m \Longrightarrow 
		|t_n - t_m| < \epsilon\, \right)
	\end{align}
	を満たす自然数$N$が取れる.このとき,
	\begin{align}
		n < m
	\end{align}
	かつ
	\begin{align}
		N < n \wedge N < m
	\end{align}
	を満たす任意の自然数$n$と$m$に対して
	\begin{align}
		|s_m - s_n| = \left|\sum_{k=n+1}^m a_k\right| \leq \sum_{k=n+1}^m |a_k| = t_m - t_n < \epsilon
	\end{align}
	が成立するので$s$は$\C$のCauchy列である.$\C$において絶対値に関するCauchy列は収束するので$s$は$\C$で収束する.
	$t$が収束することを
	$\sum_{n=0}^\infty a_n$は$\C$で{\bf 絶対収束する}\index{ぜったいしゅうそく@絶対収束}{\bf (absolutely converge)}という.
	
	\begin{screen}
		\begin{thm}[d'Alembertの収束判定法]
			$a$を複素数列とし,すべての自然数$n$で$a_n \neq 0$であるとする.このとき
			\begin{align}
				\lim_{n \to \infty} \frac{|a_{n+1}|}{|a_n|} < 1
			\end{align}
			ならば$\sum_{n=0}^\infty a_n$は$\C$で絶対収束する.
		\end{thm}
	\end{screen}
	
	式の意味は,極限が存在して,かつその極限が$1$より小さいということである.おざっぱに書き直せば
	\begin{align}
		\exists \alpha \in \R\, \left[\, 
		0 \leq \alpha < 1 \wedge \forall \epsilon \in \R_+\, \exists N \in \N\, \forall n \in \N\,
		\left(\, N < n \Longrightarrow \left|\frac{|a_{n+1}|}{|a_n|} - \alpha\right| < \epsilon\, \right)\, \right]
	\end{align}
	となる.
	
	\begin{sketch}
		
	\end{sketch}
	
	\begin{screen}
		\begin{thm}[Cauchyの冪根判定法]
			$a$を複素数列とする.このとき
			\begin{align}
				\inf{n \in \Natural}{\sup{\substack{k \in \Natural \\ n < k}}{\sqrt[k]{|a_k|}}} < 1
				\label{fom:Cauchy_root_test_1}
			\end{align}
			ならば$\sum_{n=0}^\infty a_n$は$\C$で絶対収束する.
		\end{thm}
	\end{screen}
	
	\begin{sketch}
		
	\end{sketch}
	
	\begin{screen}
		\begin{thm}[絶対収束する数列の和とスケール変換も絶対収束する]
			$a$と$b$を複素数列とし,$\alpha$を複素数とする.$\sum_{n=0}^\infty a_n$と$\sum_{n=0}^\infty b_n$が$\C$で絶対収束するとき,
			\begin{align}
				\sum_{n=0}^\infty (a_n + b_n)
			\end{align}
			も$\C$で絶対収束する.また$\sum_{n=0}^\infty a_n$が$\C$で絶対収束するとき
			\begin{align}
				\sum_{n=0}^\infty \alpha \cdot a_n
			\end{align}
			も$\C$で絶対収束する.
		\end{thm}
	\end{screen}
	
	\begin{sketch}
	\end{sketch}
	
	\begin{screen}
		\begin{thm}[絶対収束級数の畳み込み]\label{thm:convolution_of_absolutely_convergent_series}
			$a$と$b$を複素数列とする.$\sum_{n=0}^\infty a_n$と$\sum_{n=0}^\infty b_n$が$\C$で絶対収束するならば
			$\sum_{n=0}^\infty \sum_{k=0}^n a_k \cdot b_{n-k}$も$\C$で絶対収束して
			\begin{align}
				\left(\sum_{n=0}^\infty a_n\right) \cdot \left(\sum_{n=0}^\infty b_n\right)
				= \sum_{n=0}^\infty \sum_{k=0}^n a_k \cdot b_{n-k}.
			\end{align}
		\end{thm}
	\end{screen}
	
	\begin{sketch}
	\end{sketch}
	
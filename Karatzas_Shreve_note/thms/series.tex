\section{級数}
	
	$a$が$\Natural$から$\C$への写像であるとき,つまり$a$が
	\begin{align}
		a: \Natural \longrightarrow \C
	\end{align}
	を満たすとき,$a$を{\bf 複素数列}と呼ぶ.$a$が複素数列であるときは
	\begin{align}
		a(n)
	\end{align}
	の代わりに
	\begin{align}
		a_n
	\end{align}
	と書いて,また$a$を
	\begin{align}
		(a_n)_{n \in \Natural}
	\end{align}
	とも表す.ここで$\C$における総和記号$\sum$の定義を直感的に書いておくと,まず$0$に対して
	\begin{align}
		\sum_{k=0}^0 a_k \defeq a_0
	\end{align}
	と定め,次に$1$に対して
	\begin{align}
		\sum_{k=0}^1 a_k \defeq a_0 + a_1
	\end{align}
	と定め,次に$2$に対して
	\begin{align}
		\sum_{k=0}^2 a_k \defeq (a_0 + a_1) + a_2
	\end{align}
	と定め,$3$に対して
	\begin{align}
		\sum_{k=0}^3 a_k \defeq ((a_0 + a_1) + a_2) + a_3 = \left(\sum_{k=0}^2 a_k\right) + a_3
	\end{align}
	と定める.このような再帰的定義によって各自然数$n$に対して
	\begin{align}
		\sum_{k=0}^n a_k
	\end{align}
	なる複素数を対応させることは可能である.厳密な意味付けには帰納法による再帰的定義を用いるがキリがないので略.
	\begin{align}
		\Natural \ni n \longmapsto \sum_{k=0}^n a_k
	\end{align}
	なる関係により定まる複素数列を$s$とするとき,$s$が$\C$で収束するなら,つまり
	\begin{align}
		|s_n - \alpha| \longrightarrow 0\quad (n \longrightarrow \infty)
	\end{align}
	を満たす複素数$\alpha$が取れるなら
	\begin{align}
		\sum_{n=0}^\infty a_n \defeq \alpha
	\end{align}
	と定め,これを$(a_n)_{n \in \Natural}$の{\bf 級数}\index{きゅうすう@級数}{\bf (series)}と呼ぶ.
	言い換えれば$\sum_{n=0}^\infty a_n$とは$s$が収束する場合の極限のことである.
	
	\begin{screen}
		\begin{thm}[和の絶対値と絶対値の和]
			$a$を複素数列とするとき,任意の自然数$n$で
			\begin{align}
				\left|\sum_{k=0}^{n} a_{k}\right| \leq \sum_{k=0}^{n} |a_{k}|.
			\end{align}
		\end{thm}
	\end{screen}
	
	\begin{sketch}
		まず
		\begin{align}
			\left|\sum_{k=0}^{0} a_{k}\right| = |a_{0}| = \sum_{k=0}^{0} |a_{k}|
		\end{align}
		が成立する.また$n$を自然数とするとき
		\begin{align}
			\left|\sum_{k=0}^{n} a_{k}\right| \leq \sum_{k=0}^{n} |a_{k}|
		\end{align}
		が成り立っているならば,定理\ref{thm:sub_additivity_of_absolute_value}より
		\begin{align}
			\left|\sum_{k=0}^{n+1} a_{k}\right|
			&= \left|\sum_{k=0}^{n} a_{k} + a_{n+1}\right| \\
			&\leq \left|\sum_{k=0}^{n} a_{k}\right| + |a_{n+1}| \\
			&= \sum_{k=0}^{n+1} |a_{k}|
		\end{align}
		が成立する.ゆえに数学的帰納法の原理から定理の主張を得る.
		\QED
	\end{sketch}
	
	今度は
	\begin{align}
		\Natural \ni n \longmapsto \sum_{k=0}^n |a_k|
	\end{align}
	なる関係により定まる実数列を$t$とする.当然のようだが$t$が$\R$で収束すれば$s$は$\C$で収束する.実際,$t$が収束するならば,
	$\epsilon$を任意に与えられた正の実数とすれば
	\begin{align}
		\forall n,m \in \Natural\, \left(\, N < n \wedge N < m \Longrightarrow 
		|t_n - t_m| < \epsilon\, \right)
	\end{align}
	を満たす自然数$N$が取れる.このとき,
	\begin{align}
		n < m
	\end{align}
	かつ
	\begin{align}
		N < n \wedge N < m
	\end{align}
	を満たす任意の自然数$n$と$m$に対して
	\begin{align}
		|s_m - s_n| = \left|\sum_{k=n+1}^m a_k\right| \leq \sum_{k=n+1}^m |a_k| = t_m - t_n < \epsilon
	\end{align}
	が成立するので$s$は$\C$のCauchy列である.$\C$において絶対値に関するCauchy列は収束するので$s$は$\C$で収束する.
	$t$が収束することを
	$\sum_{n=0}^\infty a_n$は$\C$で{\bf 絶対収束する}\index{ぜったいしゅうそく@絶対収束}{\bf (absolutely converge)}という.
	
	\begin{screen}
		\begin{thm}[級数の絶対値と絶対値の級数]
			$a$を複素数列とするとき,$\sum_{n=0}^\infty a_n$が$\C$で絶対収束していれば
			\begin{align}
				\left|\sum_{k=0}^{\infty} a_{k}\right| \leq \sum_{k=0}^{\infty} |a_{k}|.
			\end{align}
		\end{thm}
	\end{screen}
	
	\begin{sketch}
		$\sum_{n=0}^\infty a_n$が$\C$で絶対収束しているとする.
		まず任意の自然数$n$で
		\begin{align}
			\left|\sum_{k=0}^{n} a_{k}\right| \leq \sum_{k=0}^{n} |a_{k}| \leq \sum_{k=0}^{\infty} |a_{k}|
		\end{align}
		が成立する.また
		\begin{align}
			\Natural \ni n \longmapsto \sum_{k=0}^{n} a_{k}
		\end{align}
		なる数列が収束するので
		\begin{align}
			\lim_{n \to \infty} \left|\sum_{k=0}^{n} a_{k}\right| = \left|\sum_{k=0}^{\infty} a_{k}\right|
		\end{align}
		が成り立ち
		\begin{align}
			\left|\sum_{k=0}^{\infty} a_{k}\right| \leq \sum_{k=0}^{\infty} |a_{k}|
		\end{align}
		が従う.
		\QED
	\end{sketch}
	
	\begin{screen}
		\begin{thm}[d'Alembertの収束判定法]
			$a$を複素数列とし,すべての自然数$n$で$a_n \neq 0$であるとする.このとき
			\begin{align}
				\lim_{n \to \infty} \frac{|a_{n+1}|}{|a_n|} < 1
			\end{align}
			ならば$\sum_{n=0}^\infty a_n$は$\C$で絶対収束する.
		\end{thm}
	\end{screen}
	
	式の意味は,極限が存在して,かつその極限が$1$より小さいということである.大雑把に書き直せば
	\begin{align}
		\exists \alpha \in \R\, \left[\, 
		0 \leq \alpha < 1 \wedge \forall \epsilon \in \R_+\, \exists N \in \N\, \forall n \in \N\,
		\left(\, N < n \Longrightarrow \left|\frac{|a_{n+1}|}{|a_n|} - \alpha\right| < \epsilon\, \right)\, \right]
	\end{align}
	となる.
	
	\begin{sketch}
		$1$より小さい極限が存在するとき,
		\begin{align}
			\alpha \defeq \lim_{n \to \infty} \frac{|a_{n+1}|}{|a_n|}
		\end{align}
		とおけば
		\begin{align}
			\forall n \in \Natural\, 
			\left[\, N \leq n \Longrightarrow \frac{|a_{n+1}|}{|a_n|} < \frac{1+\alpha}{2}\, \right]
		\end{align}
		を満たす自然数$N$が取れる.ここで
		\begin{align}
			\rho \defeq \frac{1+\alpha}{2}
		\end{align}
		とおく.$n$を
		\begin{align}
			N \leq n
		\end{align}
		を満たす任意の自然数とすると,
		\begin{align}
			\frac{|a_{n+1}|}{|a_n|} < \rho = \frac{\rho^{n+1}}{\rho^n}
		\end{align}
		が成り立つので
		\begin{align}
			\frac{|a_{n+1}|}{\rho^{n+1}} < \frac{|a_n|}{\rho^n}
		\end{align}
		が従う.ゆえに
		\begin{align}
			\frac{|a_n|}{\rho^n} \leq \frac{|a_N|}{\rho^N}
		\end{align}
		が成立する.すなわち,
		\begin{align}
			M \defeq \frac{|a_N|}{\rho^N}
		\end{align}
		とおけば
		\begin{align}
			N \leq n
		\end{align}
		を満たす任意の自然数$n$で
		\begin{align}
			|a_n| \leq M \cdot \rho^n
		\end{align}
		が満たされて,
		\begin{align}
			\sum_{n=N}^\infty M \cdot \rho^n = M \cdot \frac{\rho^N}{1-\rho} < \infty
		\end{align}
		であるから
		\begin{align}
			\sum_{n=N}^\infty |a_n| < \infty
		\end{align}
		が従う.よって$\sum_{n=0}^\infty a_n$は$\C$で絶対収束する.
		\QED
	\end{sketch}
	
	\begin{comment}
	\begin{screen}
		\begin{thm}[Cauchyの冪根判定法]
			$a$を複素数列とする.このとき
			\begin{align}
				\inf{n \in \Natural}{\sup{\substack{k \in \Natural \\ n < k}}{\sqrt[k]{|a_k|}}} < 1
				\label{fom:Cauchy_root_test_1}
			\end{align}
			ならば$\sum_{n=0}^\infty a_n$は$\C$で絶対収束する.
		\end{thm}
	\end{screen}
	
	\begin{sketch}
		
	\end{sketch}
	\end{comment}
	
	\begin{screen}
		\begin{thm}[級数の線型性]
		\label{thm:linearity_of_convergent_series}
			$a$と$b$を複素数列とするとき,$\left(\sum_{k=0}^{n} a_{k}\right)_{n \in \Natural}$と
			$\left(\sum_{k=0}^{n} b_{k}\right)_{n \in \Natural}$が$\C$で収束するならば
			$\left(\sum_{k=0}^{n} a_{k}+b_{k}\right)_{n \in \Natural}$も$\C$で収束して
			\begin{align}
				\sum_{n=0}^{\infty} (a_{n} + b_{n}) = \sum_{n=0}^\infty a_n + \sum_{n=0}^\infty b_n
			\end{align}
			が成立する.特に$\sum_{n=0}^\infty a_n$と$\sum_{n=0}^\infty b_n$が絶対収束するならば
			$\sum_{n=0}^\infty (a_n + b_n)$も絶対収束する.
			また$\left(\sum_{k=0}^{n} a_{k}\right)_{n \in \Natural}$が$\C$で収束するならば,
			任意の複素数$\alpha$に対して$\left(\sum_{k=0}^{n} \alpha \cdot a_{k}\right)_{n \in \Natural}$も$\C$で収束して
			\begin{align}
				\alpha \cdot \sum_{n=0}^\infty a_n = \sum_{n=0}^\infty \alpha \cdot a_n
			\end{align}
			が成立する.特に$\sum_{n=0}^\infty a_n$が絶対収束するならば
			$\sum_{n=0}^\infty \alpha \cdot a_n$も絶対収束する.
		\end{thm}
	\end{screen}
	
	\begin{sketch}\mbox{}
		\begin{description}
			\item[第一段] 任意の自然数$n$で
				\begin{align}
					\sum_{k=0}^n a_k + \sum_{k=0}^n b_k = \sum_{k=0}^n (a_k + b_k)
				\end{align}
				が成立するので
				\begin{align}
					\sum_{n=0}^\infty a_n + \sum_{n=0}^\infty b_n = \sum_{n=0}^\infty (a_n + b_n)
				\end{align}
				が成立する.また$\sum_{n=0}^\infty a_n$と$\sum_{n=0}^\infty b_n$が絶対収束するとき,任意の自然数$n$で
				\begin{align}
					\sum_{k=0}^n |a_k + b_k| \leq \sum_{k=0}^n |a_k| + \sum_{k=0}^n |b_k| 
					\leq \sum_{k=0}^\infty |a_k| + \sum_{k=0}^\infty |b_k|
				\end{align}
				が成り立つので$\sum_{n=0}^\infty (a_n + b_n)$は$\C$で絶対収束する.
			
			\item[第二段] 任意の自然数$n$で
				\begin{align}
					\alpha \cdot \sum_{k=0}^n a_k = \sum_{k=0}^n \alpha \cdot a_k
				\end{align}
				が成立するので
				\begin{align}
					\alpha \cdot \sum_{n=0}^\infty a_n = \sum_{n=0}^\infty \alpha \cdot a_n
				\end{align}
				が成立する.また$\sum_{n=0}^\infty a_n$が絶対収束するとき,任意の自然数$n$で
				\begin{align}
					\sum_{k=0}^n |\alpha \cdot a_k| \leq |\alpha| \cdot \sum_{k=0}^n |a_k| 
					\leq |\alpha| \cdot \sum_{k=0}^\infty |a_k| 
				\end{align}
				が成り立つので$\sum_{n=0}^\infty \alpha \cdot a_n$は$\C$で絶対収束する.
				\QED
		\end{description}
	\end{sketch}
	
	\begin{screen}
		\begin{thm}[絶体絶命する級数の実部は実部の級数,虚部は虚部の級数]
		\label{thm:absolutely_convergent_series_is_sum_of_series_of_real_and_imaginary_part}
			$a$を複素数列とする.$\sum_{n=0}^\infty a_n$が$\C$で絶対収束することと,
			$\sum_{n=0}^\infty \Re{a_n}$と$\sum_{n=0}^\infty \Im{a_n}$が共に$\C$で
			絶対収束することは同値である.また$\sum_{n=0}^\infty a_n$が$\C$で絶対収束するならば
			\begin{align}
				\sum_{n=0}^\infty a_n = \sum_{n=0}^\infty \Re{a_n} + \isym \cdot \sum_{n=0}^\infty \Im{a_n}.
			\end{align}
		\end{thm}
	\end{screen}
	
	\begin{sketch}
		$\sum_{n=0}^\infty a_n$が$\C$で絶対収束するならば,任意の自然数$N$で
		\begin{align}
			\sum_{n=0}^N |\Re{a_n}|
			\leq \sum_{n=0}^N |a_n|
			\leq \sum_{n=0}^\infty |a_n|
		\end{align}
		および
		\begin{align}
			\sum_{n=0}^N |\Im{a_n}|
			\leq \sum_{n=0}^N |a_n|
			\leq \sum_{n=0}^\infty |a_n|
		\end{align}
		が成立するので$\sum_{n=0}^\infty \Re{a_n}$と$\sum_{n=0}^\infty \Im{a_n}$は共に$\C$で
		絶対収束する.逆に$\sum_{n=0}^\infty \Re{a_n}$と$\sum_{n=0}^\infty \Im{a_n}$が共に
		$\C$で絶対収束するとき,定理\ref{thm:linearity_of_convergent_series}より
		$\sum_{n=0}^\infty a_n$は$\C$で絶対収束して,かつ
		\begin{align}
			\sum_{n=0}^\infty a_n
			&= \sum_{n=0}^\infty (\Re{a_n} + \isym \cdot \Im{a_n}) \\
			&= \sum_{n=0}^\infty \Re{a_n} + \sum_{n=0}^\infty \isym \cdot \Im{a_n}\\
			&= \sum_{n=0}^\infty \Re{a_n} + \isym \cdot \sum_{n=0}^\infty \Im{a_n}
		\end{align}
		も成立する.
		\QED
	\end{sketch}
	
	\begin{screen}
		\begin{thm}[絶体絶命する級数の共役は共役の級数に一致する]
		\label{thm:conjugate_of_absolutely_convergent_series_is_series_of_conjugate}
			$a$を複素数列とする.$\sum_{n=0}^\infty a_n$が$\C$で絶対収束するならば
			$\sum_{n=0}^\infty \overline{a_n}$も$\C$で絶対収束して
			\begin{align}
				\sum_{n=0}^\infty \overline{a_n} = \overline{\sum_{n=0}^\infty a_n}.
			\end{align}
		\end{thm}
	\end{screen}
	
	\begin{sketch}
		$\sum_{n=0}^\infty a_n$が$\C$で絶対収束するならば,
		定理\ref{thm:absolutely_convergent_series_is_sum_of_series_of_real_and_imaginary_part}より
		$\sum_{n=0}^\infty \Re{a_n}$と$\sum_{n=0}^\infty \Im{a_n}$は共に$\C$で
		絶対収束するので,定理\ref{thm:linearity_of_convergent_series}より
		$\sum_{n=0}^\infty (- \isym) \cdot \Im{a_n}$も$\C$で絶対収束して
		$\sum_{n=0}^\infty \overline{a_n}$も$\C$で絶対収束する.またこのとき
		\begin{align}
			\sum_{n=0}^\infty \overline{a_n}
			&= \sum_{n=0}^\infty (\Re{a_n} - \isym \cdot \Im{a_n}) \\
			&= \sum_{n=0}^\infty \Re{a_n} + \sum_{n=0}^\infty (-\isym) \cdot \Im{a_n} \\
			&= \sum_{n=0}^\infty \Re{a_n} + (-\isym) \cdot \sum_{n=0}^\infty \Im{a_n} \\
			&= \sum_{n=0}^\infty \Re{a_n} - \isym \cdot \sum_{n=0}^\infty \Im{a_n} \\
			&= \overline{\sum_{n=0}^\infty a_n}
		\end{align}
		が成立する.
		\QED
	\end{sketch}
	
	$a$と$b$を複素数列とするとき,
	\begin{align}
		\sum_{n=0}^\infty \sum_{k=0}^n a_k \cdot b_{n-k}
	\end{align}
	なる級数が$\C$で存在しているなら,これを$a$と$b$の{\bf Cauchy積}\index{Cauchy積}{\bf (Cauchy product)}と呼ぶ.Cauchy積が収束するためには$\sum_{n=0}^\infty a_n$と$\sum_{n=0}^\infty b_n$が
	共に$\C$で絶対収束すれば十分であるが,次のMetensの定理はこれより緩い条件の下でCauchy積の収束を保証する.
	
	\begin{screen}
		\begin{thm}[一方が絶対収束していればCauchy積も収束する]\label{thm:convolution_of_absolutely_convergent_series}
			$a$と$b$を複素数列とし,$\alpha$を
			\begin{align}
				\Natural \ni n \longmapsto \sum_{k=0}^n a_k
			\end{align}
			なる複素数列とし,$\beta$を
			\begin{align}
				\Natural \ni n \longmapsto \sum_{k=0}^n b_k
			\end{align}
			なる複素数列とし,$\gamma$を
			\begin{align}
				\Natural \ni n \longmapsto \sum_{k=0}^n \sum_{\ell = 0}^k a_{k-\ell} \cdot b_\ell
			\end{align}
			なる複素数列とする.$\sum_{n=0}^\infty a_n$が$\C$で絶対収束して,
			かつ$\beta$が$\C$で収束するならば,$\gamma$も$\C$で収束して
			\begin{align}
				\left(\sum_{n=0}^\infty a_n\right) \cdot \left(\sum_{n=0}^\infty b_n\right)
				= \sum_{n=0}^\infty \sum_{k=0}^n a_k \cdot b_{n-k}.
			\end{align}
		\end{thm}
	\end{screen}
	
	\begin{sketch}
		任意の自然数$n$で
		\begin{align}
			\gamma_n = \sum_{k=0}^n a_{n-k} \cdot \beta_k
		\end{align}
		が成立する.よって
		\begin{align}
			\alpha^* \defeq \sum_{n=0}^\infty a_n
		\end{align}
		および
		\begin{align}
			\beta^* \defeq \sum_{n=0}^\infty b_n
		\end{align}
		とおけば
		\begin{align}
			\gamma_n - \alpha^* \cdot \beta^*
			= \sum_{k=0}^n a_{n-k} \cdot (\beta_k - \beta^*)
			+ (\alpha_n - \alpha^*) \cdot \beta^*
		\end{align}
		が成立する.いま$\epsilon$を任意に与えられた正の実数とすると,
		\begin{align}
			\forall n \in \Natural\, 
			\left[\, N_1 \leq n \Longrightarrow |\beta_n - \beta^*| < \frac{\epsilon}{3 \cdot \left(\sum_{n=0}^\infty |a_n| + 1\right)}\, \right]
		\end{align}
		を満たす自然数$N_1$と,
		\begin{align}
			\forall n \in \Natural\, 
			\left[\, N_2 \leq n \Longrightarrow |a_n| < \frac{\epsilon}{3 \cdot \left(\sup{k \in N_1}{|\beta_k - \beta^*|} + 1\right)}\, \right]
		\end{align}
		を満たす自然数$N_2$と,
		\begin{align}
			\forall n \in \Natural\, 
			\left[\, N_3 \leq n \Longrightarrow |\alpha_n - \alpha^*| < \frac{\epsilon}{3 \cdot \left(|\beta^*| + 1\right)}\, \right]
		\end{align}
		を満たす自然数$N_3$が取れる.従って任意の自然数$n$に対して
		\begin{align}
			\max{\{N_1 + N_2, N_3\}} \leq n
		\end{align}
		ならば
		\begin{align}
			|\gamma_n - \alpha^* \cdot \beta^*|
			\leq \sum_{k=0}^{N_1-1} |a_{n-k}| \cdot |\beta_k - \beta^*|
			+ \sum_{k=N_1}^{n} |a_{n-k}| \cdot |\beta_k - \beta^*|
			+ |\alpha_n - \alpha^*| \cdot |\beta^*|
			< \epsilon
		\end{align}
		が成立する.ゆえに$\gamma$は$\alpha^* \cdot \beta^*$に収束する.
		\QED
	\end{sketch}
	
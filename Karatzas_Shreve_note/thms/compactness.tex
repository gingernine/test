\subsection{コンパクト性}
	\begin{screen}
		\begin{dfn}[開被覆]
			$(S,\mathscr{O})$を位相空間とし,$b$を$S$の部分集合とするとき,
			$\mathscr{O}$の部分集合$\mathscr{B}$で
			\begin{align}
				b \subset \bigcup \mathscr{B}
			\end{align}
			を満たすものを$b$の$\mathscr{O}$-{\bf 開被覆}\index{かいひふく@開被覆}
			{\bf (open cover)}と呼ぶ.
		\end{dfn}
	\end{screen}
	
	$(S,\mathscr{O})$を位相空間とし,$b$を$S$の部分集合とするとき,
	例えば$\{S\}$と$\mathscr{O}$はどちらも$b$の極端な$\mathscr{O}$-開被覆である.
	また$h$を$S$上の写像で,$S$の要素に対してその$\mathscr{O}$-開近傍を対応させる写像とすれば,
	\begin{align}
		\mathscr{H} \defeq \Set{h(x)}{x \in b}
		\label{fom:dfn_compactness}
	\end{align}
	で定める$\mathscr{H}$は$b$の$\mathscr{O}$-開被覆である.
	
	\begin{screen}
		\begin{dfn}[コンパクト]
			$(S,\mathscr{O})$を位相空間とし,$c$を$S$の部分集合とする.
			このとき$c$の任意の$\mathscr{O}$-開被覆から$c$の有限被覆が取れるならば,言い換えれば,
			$\mathscr{B}$を$c$の任意の$\mathscr{O}$-開被覆とするとき
			\begin{align}
				c \subset \bigcup \mathscr{F} \wedge \card{\mathscr{F}} < \Natural
			\end{align}
			を満たす$\mathscr{B}$の部分集合$\mathscr{F}$が取れるならば,
			$c$を$\mathscr{O}$-{\bf コンパクト集合}\index{こんぱくとしゅうごう@コンパクト集合}
			{\bf (compact set)}と呼ぶ.
		\end{dfn}
	\end{screen}
	
	{\bf コンパクトとは集合の小ささを位相的に表現するための概念}である.
	(\refeq{fom:dfn_compactness})の$b$と$\mathscr{H}$について言えば,
	$b$が$\mathscr{O}$-コンパクトであるならば$\mathscr{H}$から有限個の要素を
	適当に抜き取れば$b$を覆えてしまうのであるから,つまり$\mathscr{O}$-コンパクト集合は
	自身の有限個の要素から近傍を一つずつ適切に徴収すればその中に収まってしまう.
	
	\begin{screen}
		\begin{dfn}[被覆・コンパクト・相対コンパクト・局所コンパクト・$\sigma$-コンパクト]\mbox{}
			\begin{itemize}
				\item 位相空間の部分集合で,その閉包がコンパクトであるものを
					{\bf 相対コンパクト}\index{そうたいこんぱくと@相対コンパクト}な
					{\bf (relatively compact)}部分集合という.
				
				\item 位相空間の任意の点がコンパクトな近傍を持つとき,
					その空間は{\bf 局所コンパクト}である
					\index{きょくしょこんぱくと@局所コンパクト}{\bf (locally compact)}という.
					
				\item 位相空間においてコンパクト集合から成る可算被覆が存在するとき,
					その空間は{\bf $\sigma$-コンパクト}
					\index{しぐまこんぱくと@$\sigma$-コンパクト}であるという.
			\end{itemize}
		\end{dfn}
	\end{screen}
	
	\begin{screen}
		\begin{thm}[部分空間におけるコンパクト性]
		\label{thm:subset_is_compact_iff_every_original_open_cover_contains_finite_subcover}
		\label{thm:compactness_in_subspace}
			$(S,\mathscr{O})$を位相空間とし,$c$を$S$の部分集合とし,
			\begin{align}
				\mathscr{O}_{c} \defeq \Set{c \cap o}{o \in \mathscr{O}}
			\end{align}
			とおく.このとき$c$が$\mathscr{O}$-コンパクトであることと
			$c$が$\mathscr{O}_{c}$-コンパクトであることは同値である.
		\end{thm}
	\end{screen}
	
	\begin{sketch}\mbox{}
		\begin{description}
			\item[第一段]
				$c$が$\mathscr{O}$-コンパクトであるとする.
				$\mathscr{B}$を$c$の$\mathscr{O}_{c}$-開被覆とすれば,
				定理\ref{thm:direct_product_of_non_empty_sets_is_not_empty}より
				$\mathscr{B}$から$\mathscr{O}$への写像$f$で,$\mathscr{B}$の各要素$b$に対して
				\begin{align}
					b = c \cap f(b)
					\label{fom:thm_compactness_in_subspace}
				\end{align}
				を満たすものが取れる.ここで
				\begin{align}
					\mathscr{U} \defeq \Set{f(b)}{b \in \mathscr{B}}
				\end{align}
				とおけば,$\mathscr{U}$は$c$の$\mathscr{O}$-開被覆であるから
				\begin{align}
					c \subset \bigcup \mathscr{V} \wedge \card{\mathscr{V}} < \Natural
				\end{align}
				を満たす$\mathscr{U}$の部分集合$\mathscr{V}$が取れる.
				$v$を$\mathscr{V}$の要素とすれば,
				\begin{align}
					v \in \mathscr{U}
				\end{align}
				なので,(\refeq{fom:thm_compactness_in_subspace})より
				\begin{align}
					c \cap v \in \mathscr{B}
				\end{align}
				が従う.ゆえに
				\begin{align}
					\mathscr{F} \defeq \Set{c \cap v}{v \in \mathscr{V}}
				\end{align}
				とおけば
				\begin{align}
					\mathscr{F} \subset \mathscr{B}
					\label{fom:thm_compactness_in_subspace_2}
				\end{align}
				が成り立つ.さらに
				\begin{align}
					\mathscr{V} \ni v \longmapsto c \cap v
				\end{align}
				なる写像は$\mathscr{F}$への全射であるから,
				定理\ref{thm:if_exists_a_surjection_then_cardinal_of_target_is_bigger}より
				\begin{align}
					\card{\mathscr{F}} \leq \card{\mathscr{V}}
					\label{fom:thm_compactness_in_subspace_3}
				\end{align}
				も成り立つ.さらに$x$を$c$の要素とすれば
				\begin{align}
					x \in v
				\end{align}
				を満たす$\mathscr{V}$の要素$v$が取れるが,このとき
				\begin{align}
					x \in c \cap v \wedge c \cap v \in \mathscr{F}
				\end{align}
				が成り立つので
				\begin{align}
					x \in \bigcup \mathscr{F}
				\end{align}
				が従う.ゆえに
				\begin{align}
					c \subset \bigcup \mathscr{F}
					\label{fom:thm_compactness_in_subspace_4}
				\end{align}
				が従う.(\refeq{fom:thm_compactness_in_subspace_2})
				(\refeq{fom:thm_compactness_in_subspace_3})
				(\refeq{fom:thm_compactness_in_subspace_4})
				より$c$は$\mathscr{O}_{c}$-コンパクトである.
				
			\item[第二段]
				$c$が$\mathscr{O}_{c}$-コンパクトであるとする.$\mathscr{B}$を$c$の
				$\mathscr{O}$-開被覆とすれば,
				\begin{align}
					\mathscr{U} \defeq \Set{c \cap b}{b \in \mathscr{B}}
				\end{align}
				で定める$\mathscr{U}$は$c$の$\mathscr{O}_{c}$-開被覆である.よって
				\begin{align}
					c \subset \bigcup \mathscr{V} \wedge \card{\mathscr{V}} < \Natural
				\end{align}
				を満たす$\mathscr{U}$の部分集合$\mathscr{V}$が取れる.
				ところで,定理\ref{thm:direct_product_of_non_empty_sets_is_not_empty}より
				$\mathscr{V}$から$\mathscr{B}$への写像$h$で,
				$\mathscr{V}$の各要素$v$に対して
				\begin{align}
					v = c \cap h(v)
				\end{align}
				を満たすものが取れる.ここで
				\begin{align}
					\mathscr{F} \defeq \Set{h(v)}{v \in \mathscr{V}}
				\end{align}
				とおけば
				\begin{align}
					\mathscr{F} \subset \mathscr{B}
				\end{align}
				と
				\begin{align}
					\card{\mathscr{F}} \leq \card{\mathscr{V}}
				\end{align}
				(定理\ref{thm:if_exists_a_surjection_then_cardinal_of_target_is_bigger}),
				及び
				\begin{align}
					c \subset \bigcup \mathscr{F}
				\end{align}
				が成り立つ.以上より$c$は$\mathscr{O}$-コンパクトである.
				\QED
		\end{description}
	\end{sketch}
	
	\begin{screen}
		\begin{thm}[コンパクト集合の閉部分集合はコンパクト]
		\label{thm:closed_subset_of_compact_set_is_compact_on_Hausdorff_space}
			$(S,\mathscr{O})$を位相空間とし,$a$を$\mathscr{O}$-閉集合とし,
			$c$を$\mathscr{O}$-コンパクト集合とするとき,
			$a \cap c$は$\mathscr{O}$-コンパクト集合である.
		\end{thm}
	\end{screen}
	
	\begin{sketch}
		$\mathscr{B}$を$a \cap c$の$\mathscr{O}$-開被覆とする.このとき
		\begin{align}
			\mathscr{U} \defeq \Set{u}{u \in \mathscr{B} \vee u = S \backslash a}
		\end{align}
		とおけば,$\mathscr{U}$は$c$の$\mathscr{O}$-開被覆であるから
		\begin{align}
			c \subset \bigcup \mathscr{V} \wedge \card{\mathscr{V}} < \Natural
		\end{align}
		を満たす$\mathscr{U}$の部分集合$\mathscr{V}$が取れる.ここで
		\begin{align}
			\mathscr{F} \defeq \Set{v}{v \in \mathscr{V} \wedge v \neq S \backslash a}
		\end{align}
		とおけば,定理\ref{thm:cardinal_of_bigger_set_is_bigger}より
		\begin{align}
			\card{\mathscr{F}} \leq \card{\mathscr{V}}
		\end{align}
		が成り立つ.また
		\begin{align}
			a \cap c \subset \bigcup \mathscr{F}
			\label{fom:thm_closed_subset_of_compact_set_is_compact_on_Hausdorff_space_1}
		\end{align}
		も成り立つ.実際,$x$を$a \cap c$の要素とすれば
		\begin{align}
			x \in v
		\end{align}
		を満たす$\mathscr{V}$の要素$v$が取れるが,
		\begin{align}
			x \in a
		\end{align}
		より
		\begin{align}
			v \neq S \backslash a
		\end{align}
		が従う.ゆえに
		\begin{align}
			v \in \mathscr{F}
		\end{align}
		が従い(\refeq{fom:thm_closed_subset_of_compact_set_is_compact_on_Hausdorff_space_1})を得る.
		さらに
		\begin{align}
			\mathscr{F} \subset \mathscr{B}
			\label{fom:thm_closed_subset_of_compact_set_is_compact_on_Hausdorff_space_2}
		\end{align}
		も成り立つ.実際,$v$を$\mathscr{F}$の要素とすれば
		\begin{align}
			v \in \mathscr{U}
		\end{align}
		が従うので
		\begin{align}
			v \in \mathscr{B} \vee v = S \backslash a
		\end{align}
		が成り立つ.ところでいま
		\begin{align}
			v \neq S \backslash a
		\end{align}
		であるから,選言三段論法(推論法則\ref{logicalthm:disjunctive_syllogism})より
		\begin{align}
			v \in \mathscr{B}
		\end{align}
		が従い(\refeq{fom:thm_closed_subset_of_compact_set_is_compact_on_Hausdorff_space_2})を得る.
		以上より$a \cap c$は$\mathscr{O}$-コンパクトである.
		\QED
	\end{sketch}
	
	\begin{screen}
		\begin{dfn}[有限交叉性]
			$S$を集合ととする.$S$の任意の有限部分集合の交叉が空でないとき,つまり
			\begin{align}
				\forall b\, \left[\, b \subset S \wedge \card{b} < \Natural
				\Longrightarrow \bigcap b \neq \emptyset\, \right]
			\end{align}
			が成り立つとき,$S$は{\bf 有限交叉性}\index{ゆうげんこうさせい@有限交叉性}
			{\bf (finite intersection property)}を持つという.
		\end{dfn}
	\end{screen}
	
	例えば空集合は有限交叉性を持つ.実際,集合$b$が
	\begin{align}
		b \subset \emptyset \wedge \card{b} < \Natural
	\end{align}
	を満たすとき,
	\begin{align}
		b = \emptyset
	\end{align}
	が従い
	\begin{align}
		\bigcap b = \Univ
	\end{align}
	が成り立つ.ゆえに
	\begin{align}
		\forall b\, \left[\, b \subset \emptyset \wedge \card{b} < \Natural
		\Longrightarrow \bigcap b \neq \emptyset\, \right]
	\end{align}
	が成立する.
	
	\begin{screen}
		\begin{thm}[有限交叉性を用いたコンパクト性の特徴づけ]
		\label{thm:compact_iff_closed_sets_family_finitely_intersect}
			$(S,\mathscr{O})$を位相空間とし,$\mathscr{A}$を$\mathscr{O}$-閉集合の全体とし,
			$b$を$S$の空でない部分集合とする.このとき,$b$が$\mathscr{O}$-コンパクトであることと,
			$\mathscr{A}$の任意の部分集合$u$に対して$\Set{b \cap w}{w \in u}$が有限交叉性を持てば
			\begin{align}
				b \cap \bigcap u \neq \emptyset
			\end{align}
			が成り立つことは同値である.言い換えれば,$b$が$\mathscr{O}$-コンパクトであることと
			\begin{align}
				\forall u\, &\left[\, u \subset \mathscr{A} \wedge
				\forall v\, \left(\, \forall t\, \left(\, t \in v 
				\Longrightarrow \exists w\, (\, w \in u \wedge t = b \cap w\, )\, \right) \wedge \card{v} < \Natural
				\Longrightarrow \bigcap v \neq \emptyset\, \right) \right. \\
				&\quad \left. \Longrightarrow b \cap \bigcap u \neq \emptyset\, \right]
			\end{align}
			が成り立つことは同値である.
		\end{thm}
	\end{screen}
	
	\begin{prf}
		定理\ref{thm:subset_is_compact_iff_every_original_open_cover_contains_finite_subcover}より
		\begin{align}
			&\mbox{$A$がコンパクト部分集合である} \\
			&\Longleftrightarrow \mbox{$A$の$S$における任意の開被覆が($S$における)有限部分被覆を含む} \\
			&\Longleftrightarrow \mbox{$S$の任意にお閉集合族$\mathscr{F}$に対し,
			$A \cap \bigcap \mathscr{F} = \emptyset$なら或る有限集合$\mathscr{M} \subset \mathscr{F}$で
			$A \cap \bigcap \mathscr{M} = \emptyset$} \\
			&\Longleftrightarrow \mbox{$S$の任意の閉集合族$\mathscr{F}$に対し,
			$\Set{F \cap A}{F \in \mathscr{F}}$が有限交叉性を持つなら
			$A \cap \bigcap \mathscr{F} \neq \emptyset$}
		\end{align}
		が従う.
		\QED
	\end{prf}
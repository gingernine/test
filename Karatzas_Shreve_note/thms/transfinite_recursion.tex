	\begin{screen}
		\begin{thm}[超限帰納法]
			$A$を文字$x$のみを自由変項とする式とするとき,次が成立する:
			\begin{align}
				\forall \alpha \in \ON\ 
				\left(\ \forall \beta \in \alpha\ A(\beta)
				\Longrightarrow A(\alpha)\ \right)
				\Longrightarrow \forall \alpha \in \ON\ A(\alpha).
			\end{align}
		\end{thm}
	\end{screen}
	
	\begin{prf}
		正則性公理より
		\begin{align}
			\forall x\ \left(\ \forall y \in x\ A(y)
			\Longrightarrow A(x)\ \right)
			\Longrightarrow \forall x\ A(x)
		\end{align}
		が成り立つが,$A(x)$を$x \in \ON \Longrightarrow A(x)$に置き換えれば
		\begin{align}
			\forall \alpha \in \ON\ 
			\left(\ \forall \beta \in \alpha\ A(\beta)
			\Longrightarrow A(\alpha)\ \right)
			\Longrightarrow \forall \alpha \in \ON\ A(\alpha).
		\end{align}
		が出る.
		\QED
	\end{prf}
	
	\begin{screen}
		\begin{thm}[超限帰納法による写像の構成]\label{thm:transfinite_recursion_theorem}
			類$G$を$V$上の写像とするとき,
			\begin{align}
				\forall \alpha \in \ON\ (\ F(\alpha) = G(F|_\alpha)\ )
			\end{align}
			を満たす写像$F:\ON \longrightarrow V$が唯一つ存在する.
		\end{thm}
	\end{screen}
	
	\begin{prf}\mbox{}
		\begin{description}
			\item[第一段] 与えられた$G:V \longrightarrow V$に対して
				\begin{align}
					K = \Set{f}{\exists \alpha \in \ON\ \left(\ f:\alpha \longrightarrow V \wedge \forall \beta \in \alpha\ (\ f(\beta) = G(f|_\beta)\ )\ \right)}
				\end{align}
				で$K$を定めるとき,$F = \bigcup K$が求める写像である.
				
			\item[第二段] $F$が写像であることを示す.
				まず$K$の任意の要素は$V \times V$の部分集合であるから
				\begin{align}
					F \subset V \times V
				\end{align}
				となる.$x,y,z$を任意の集合とする.
				$(x,y) \in F$かつ$(x,z) \in F$のとき,
				$K$の或る要素$f$と$g$が存在して
				\begin{align}
					(x,y) \in f \wedge (x,z) \in g
				\end{align}
				を満たすが,ここで$f(x) = g(x)$となることを言うために,
				$\alpha = \operatorname{dom}(f),\ 
				\beta = \operatorname{dom}(g)$とおき,
				\begin{align}
					\forall \gamma \in \ON\ (\ \gamma \in \alpha \wedge \gamma \in \beta \Longrightarrow f(\gamma) = g(\gamma)\ )
					\label{eq:thm_transfinite_recursion_theorem_1}
				\end{align}
				が成り立つことを示す.いま$\gamma$を任意の順序数とする.$\gamma = \emptyset$の場合は
				$f|_\gamma = \emptyset$かつ$g|_\gamma = \emptyset$となるから
				\begin{align}
					f(\gamma) = G(\emptyset) = g(\gamma)
				\end{align}
				が成立する.$\gamma \neq \emptyset$の場合は
				\begin{align}
					\forall \xi \in \gamma\ (\ \xi \in \alpha \wedge \xi \in \beta \Longrightarrow f(\xi) = g(\xi)\ )
				\end{align}
				が成り立っていると仮定する.このとき$\gamma \in \alpha \wedge \gamma \in \beta$ならば
				順序数の推移性より$\gamma$の任意の要素$\xi$は$\xi \in \alpha \wedge \xi \in \beta$を満たすから
				\begin{align}
					\forall \xi \in \gamma\ (\ f(\xi) = g(\xi)\ )
				\end{align}
				が成立する.従って
				\begin{align}
					f|_\gamma = g|_\gamma
				\end{align}
				が成立するので$f(\gamma) = g(\gamma)$が得られる.超限帰納法より
				(\refeq{eq:thm_transfinite_recursion_theorem_1})が得られる.
				以上より
				\begin{align}
					y = f(x) = g(x) = z
				\end{align}
				となるので$F$はsingle-valuedである.
			
			\item[第三段] $\operatorname{dom}(F) \subset \ON$が成り立つことを示す.
				実際
				\begin{align}
					\operatorname{dom}(F) = \bigcup_{f \in K} \operatorname{dom}(f)
				\end{align}
				かつ$\forall f \in K\ (\ \operatorname{dom}(f) \subset \ON\ )$だから
				$\operatorname{dom}(F) \subset \ON$となる.
				
			\item[第四段] $\operatorname{Tran}(\operatorname{dom}(F))$であることを示す.
				実際任意の集合$x,y$について
				\begin{align}
					y \in x \wedge x \in \operatorname{dom}(F)
				\end{align}
				が成り立っているとき,或る$f \in K$で$x \in \operatorname{dom}(f)$
				となり,$\operatorname{dom}(f)$は順序数なので,順序数の推移律から
				\begin{align}
					y \in \operatorname{dom}(f)
				\end{align}
				が従う.ゆえに$y \in \operatorname{dom}(F)$となる.
				
			\item[第五段] $\forall \alpha \in \operatorname{dom}(F)\ (\ F(\alpha) = G(F|_\alpha)\ )$が成り立つことを示す.
				実際,$\alpha \in \operatorname*{dom}(F)$なら
				$K$の或る要素$f$に対して$\alpha \in \operatorname*{dom}(f)$となるが,
				$f \subset F$であるから
				\begin{align}
					f(\alpha) = F(\alpha)
				\end{align}
				が成り立つ.これにより$f|_\alpha = f \cap (\alpha \times V)
				= F \cap (\alpha \times V) = F|_\alpha$より
				\begin{align}
					G(f|_\alpha) = G(F|_\alpha)
				\end{align}
				も成り立つ.$f(\alpha) = G(f|_\alpha)$と併せて
				$F(\alpha) = G(F|_\alpha)$を得る.
			
			\item[第六段] 
				$\alpha$を任意の順序数として
				$\forall \beta \in \alpha\ (\ \beta \in \operatorname{dom}(F)\ )
				\Longrightarrow \alpha \in \operatorname{dom}(F)$が成り立つことを示す.
				$\alpha = \emptyset$の場合は
				\begin{align}
					\forall f \in K\ (\ \operatorname{dom}(f) \neq \emptyset
					\Longrightarrow \emptyset \in \operatorname{dom}(f)\ )
				\end{align}
				が満たされるので$\alpha \in \operatorname{dom}(F)$となる
				(定理\ref{thm:properties_of_ordinal_numbers}).
				$\alpha \neq \emptyset$の場合,
				\begin{align}
					\forall \beta \in \alpha\ (\ \beta \in \operatorname{dom}(F)\ )
				\end{align}
				が成り立っているとして$f = F|_\alpha$とおけば,$f$は$\alpha$上の写像であり,
				$\alpha$の任意の要素$\beta$に対して
				\begin{align}
					f(\beta)
					= F|_\alpha(\beta)
					= F(\beta)
					= G(F|_\beta)
					= G(f|_\beta)
				\end{align}
				を満たすから$f \in K$である.このとき$f' = f \cup \{(\alpha,G(f))\}$も
				$K$に属するので
				\begin{align}	
					\alpha \in \operatorname{dom}(f') \subset
					\operatorname{dom}(F)
				\end{align}
				が成立する.超限帰納法より
				\begin{align}
					\forall \alpha \in \ON\ (\ \alpha \in \operatorname{dom}(F)\ )
				\end{align}
				が成立し,前段の結果と併せて
				\begin{align}
					\ON = \operatorname{dom}(F)
				\end{align}
				を得る.
				
			\item[第七段]
				$F$の一意性を示す.類$H$が
				\begin{align}
					H:\ON \longrightarrow V 
					\wedge \forall \alpha \in \ON\ (\ H(\alpha) = G(H|_\alpha)\ )
				\end{align}
				を満たすとき,$F = H$が成り立つことを示す.
				いま,$\alpha$を任意に与えられた順序数とする.$\alpha = \emptyset$の場合は
				\begin{align}
					F|_\emptyset = \emptyset = H|_\emptyset
				\end{align}
				より$F(\emptyset) = H(\emptyset)$となる.$\alpha \neq \emptyset$の場合,
				\begin{align}
					\forall \beta \in \alpha\ (\ F(\beta) = H(\beta)\ )
				\end{align}
				が成り立っていると仮定すれば
				\begin{align}
					F|_\alpha = H|_\alpha
				\end{align}
				が成り立つから$F(\alpha) = H(\alpha)$となる.以上で
				\begin{align}
					\forall \alpha \in \ON\ \left(\ \forall \beta \in \alpha\ 
					(\ F(\beta) = H(\beta)\ ) \Longrightarrow F(\alpha) = H(\alpha)\ \right)
				\end{align}
				が得られた.超限帰納法より
				\begin{align}
					\forall \alpha \in \ON\ (\ F(\alpha) = H(\alpha)\ )
				\end{align}
				が従い$F = H$が出る.
				\QED
		\end{description}
	\end{prf}
	
	\begin{screen}
		\begin{thm}
			\begin{align}\label{thm:R_alpha_plus_1_equals_to_power_of_R_alpha}
				\forall \alpha \in \ON\ 
				\left(\ R(\alpha + 1) = \operatorname{P}(R(\alpha))\ \right)
			\end{align}
		\end{thm}
	\end{screen}
	
	\begin{prf}\mbox{}
		\begin{description}
			\item[第一段] $R(\alpha + 1) = R(\alpha) \cup \operatorname{P}(R(\alpha))$
				となることを示す.
				
			\item[第二段] $\alpha$を任意に与えられた空でない順序数とするとき,
				\begin{align}
					\forall \beta \in \alpha\ 
					\left(\ R(\beta + 1) \subset \operatorname{P}(R(\beta))\ \right)
					\Longrightarrow R(\alpha + 1) \subset \operatorname{P}(R(\alpha))
				\end{align}
				が成り立つことを示す.いま
				\begin{align}
					\forall \beta \in \alpha\ 
					\left(\ R(\beta + 1) \subset \operatorname{P}(R(\beta))\ \right)
					\label{eq:thm_R_alpha_plus_1_equals_to_power_of_R_alpha}
				\end{align}
				が成り立つと仮定する.$x$を$R(\alpha + 1)$の任意の要素とすれば,前段の結果より
				\begin{align}
					x \in R(\alpha) \vee x \subset R(\alpha)
				\end{align}
				となる.$x \in R(\alpha)$であるとき,$\alpha$の或る要素$\beta$に対し
				$x \in R(\beta)$となる.前段の結果より$x \in R(\beta + 1)$となり,
				(\refeq{eq:thm_R_alpha_plus_1_equals_to_power_of_R_alpha})より
				$x \subset R(\beta)$となるが,
				\begin{align}
					x \subset R(\beta) &\Longrightarrow x \subset R(\alpha), \\
					x \subset R(\alpha) &\Longrightarrow x \in \operatorname{P}(R(\alpha))
				\end{align}
				と併せて$x \in \operatorname{P}(R(\alpha))$が成り立つ.
				一方で$x \subset R(\alpha)$であるときも$x \in \operatorname{P}(R(\alpha))$
				となるから
				\begin{align}
					R(\alpha + 1) \subset \operatorname{P}(R(\alpha))
				\end{align}
				が従う.超限帰納法より定理の主張が得られる.
		\end{description}
	\end{prf}
	
	\begin{screen}
		\begin{thm}[$V$はwell-founded集合で尽くされる]
			\begin{align}
				V = \bigcup_{\alpha \in \ON} R(\alpha).
			\end{align}
		\end{thm}
	\end{screen}
	
	\begin{prf}
		いま,$S$を$\ON$の空でない部分集合とする.そして
		\begin{align}
			V \neq \bigcup_{\alpha \in S} R(\alpha)
			\Longrightarrow S \neq \ON
		\end{align}
		が成り立つことを示す.$V \neq \bigcup_{\alpha \in S} R(\alpha)$であれば
		正則性公理より或る集合$a$が存在して
		\begin{align}
			a \in V \backslash \bigcup_{\alpha \in S} R(\alpha)
			\wedge a \cap V \backslash \bigcup_{\alpha \in S} R(\alpha) = \emptyset
		\end{align}
		を満たす.このとき
		\begin{align}
			a \in \bigcup_{\alpha \in S} R(\alpha) \wedge a \subset \bigcup_{\alpha \in S} R(\alpha)
		\end{align}
		となる.ここで
		\begin{align}
			f = \Set{x}{\exists s \in a\ (\ x = (s,\mu \alpha (s \in R(\alpha)))\ )}
		\end{align}
		と定めれば$f:a \longrightarrow \ON$が成り立つ.
		$\beta = \bigcup f(a)$とおけば$\beta$は$\ON$に属する.このとき
		\begin{align}
			\forall t\ (\ t \in a \Longrightarrow t \in R(f(t))
			\Longrightarrow t \in R(\beta)\ )
		\end{align}
		となるから$a \subset R(\beta)$,そして定理\ref{thm:R_alpha_plus_1_equals_to_power_of_R_alpha}
		より$a \in R(\beta + 1)$が従う.
		\begin{align}
			\forall \alpha \in S\ (\ a \notin R(\alpha)\ )
		\end{align}
		であったから$\beta + 1 \notin S$であり,ゆえに$S \neq \ON$となる.
		定理の主張は対偶を取れば得られる.
		\QED
	\end{prf}
	
	\monologue{
		院生「\begin{align}
				V = \bigcup_{\alpha \in \ON} R(\alpha)
			\end{align}
			という美しい式は偶然得られた訳ではありません.John Von Neumann はこの結果を
			予定して正則性公理を導入したのです.さて,超限帰納法による写像の構成を応用して
			次は順序数の足し算と掛け算を定義しましょう.」
	}
	
	\begin{screen}
		\begin{thm}[順序数の加法]\label{thm:the_definition_of_addition_of_ordinal_numbers}
			$\alpha$を$\ON$から任意に選ばれた順序数として,写像$G_\alpha:V \longrightarrow V$を
			\begin{align}
				G_\alpha(x) = 
				\begin{cases}
					\alpha & (\operatorname{dom}(x) = \emptyset) \\
					x(\beta) \cup \{x(\beta)\} & (\operatorname{dom}(x) = \beta \cup \{\beta\}) \\
					\bigcup_{\gamma \in \operatorname{dom}(x)} x(\gamma) & 
					(\operatorname{lim}(\operatorname{dom}(x))) \\
					\emptyset & \mathrm{o.w.}
				\end{cases}
			\end{align}
			で定めるとき,定理\ref{thm:transfinite_recursion_theorem}より
			\begin{align}
				\forall \beta \in \ON\ (\ A_\alpha(\beta) = G_\alpha(A_\alpha|_\beta)\ )
			\end{align}
			を満たす写像$A_\alpha:\ON \longrightarrow V$が唯一つ存在する.ここで
			\begin{align}
				\alpha + \beta = A_\alpha (\beta)
			\end{align}
			と書くと,次が成立する:
			\begin{itemize}
				\item $\forall \beta \in \ON\ (\ \alpha + \beta \in \ON\ )$.
				\item $\alpha \in {\bf \omega}$のとき,$\forall \beta \in {\bf \omega}\ 
				(\ \alpha + \beta \in {\bf \omega}\ )$.
			\end{itemize}
		\end{thm}
	\end{screen}
	
	\begin{prf}
		いま$\beta$を任意に与えられた順序数とする.このとき,
		\begin{align}
			\forall \gamma \in \beta\ (\ \alpha + \gamma \in \ON\ )
		\end{align}
		が成り立っていると仮定すると,$\beta = \gamma + 1$と表せるとき
		\begin{align}
			\alpha + \beta 
			= G_\alpha (F_\alpha|_\beta)
			= F_\alpha(\gamma) + 1
			= (\alpha + \gamma) + 1 \in \ON
		\end{align}
		となり,$\beta$が極限数のときは
		\begin{align}
			\alpha + \beta = \operatorname*{sup}_{\gamma \in \beta} (\alpha + \gamma)
			= \bigcup \Set{\alpha + \gamma}{\gamma \in \beta}
			\in \ON
		\end{align}
		となるので,
		\begin{align}
			\forall \beta \in \ON\ \left(\ \forall \gamma \in \beta\ (\ \alpha + \gamma \in \ON\ ) \Longrightarrow \alpha + \beta \in \ON\ \right)
		\end{align}
		が得られた.超限帰納法により
		\begin{align}
			\forall \beta \in \ON\ (\ \alpha + \beta \in \ON\ )
		\end{align}
		が成立する.また$\alpha \in {\bf \omega}$のとき,
		\begin{align}
			a = \Set{\beta \in {\bf \omega}}{\alpha + \beta \in {\bf \omega}}
		\end{align}
		とおけば
		\begin{align}
			\emptyset \in a \wedge \forall x\ (\ x \in a \Longrightarrow x \cup \{x\} \in a\ )
		\end{align}
		となるので${\bf \omega} \subset a$が従う.よって
		\begin{align}
			\forall \beta \in {\bf \omega}\ 
			(\ \alpha + \beta \in {\bf \omega}\ )
		\end{align}
		も成り立つ.
		\QED
	\end{prf}
	
	\begin{screen}
		\begin{thm}[加法の性質]
			定理\ref{thm:the_definition_of_addition_of_ordinal_numbers}で定めた
			加法は以下の性質を持つ:
			\begin{itemize}
				\item $\forall \alpha \in \ON\ (\ \alpha + 0 = 0 + \alpha = \alpha\ )$,
				
				\item $\forall \alpha \in \ON\ (\ \alpha + 1 = \alpha \cup \{\alpha\}\ )$,
				
				\item $\forall \alpha,\beta,\gamma \in \ON\ (\ (\alpha + \beta) + \gamma = \alpha + (\beta + \gamma)\ )$,
				
				\item $\forall \alpha,\beta \in {\bf \omega}\ (\ \alpha + \beta = \beta + \alpha\ )$,
				
				\item $\forall \alpha,\beta,\gamma \in \ON\ (\ \beta \in \gamma
					\Longrightarrow \alpha + \beta \in \alpha + \gamma\ )$,
				
				\item $\forall \alpha,\beta \in \beta\ (\ \alpha \in \beta
					\Longrightarrow \exists \gamma \in \ON\ (\ \alpha + \gamma = \beta\ )\ )$.
			\end{itemize}
		\end{thm}
	\end{screen}
	
	\begin{screen}
		\begin{thm}[順序数の乗法]
		\label{thm:the_definition_of_multiplication_of_ordinal_numbers}
			$\alpha$を$\ON$から任意に選ばれた順序数として,写像$G_\alpha:V \longrightarrow V$を
			\begin{align}
				G_\alpha(x) = 
				\begin{cases}
					0 & (\operatorname{dom}(x) = \emptyset) \\
					x(\beta) + \alpha & (\operatorname{dom}(x) = \beta + 1) \\
					\bigcup_{\gamma \in \operatorname{dom}(x)} x(\gamma) & 
					(\operatorname{lim}(\operatorname{dom}(x))) \\
					\emptyset & \mathrm{o.w.}
				\end{cases}
			\end{align}
			で定めるとき,定理\ref{thm:transfinite_recursion_theorem}より
			\begin{align}
				\forall \beta \in \ON\ (\ M_\alpha(\beta) = G_\alpha(M_\alpha|_\beta)\ )
			\end{align}
			を満たす写像$M_\alpha:\ON \longrightarrow V$が唯一つ存在する.ここで
			\begin{align}
				\alpha \cdot \beta = M_\alpha (\beta)
			\end{align}
			と書くと,次が成立する:
			\begin{itemize}
				\item $\forall \beta \in \ON\ (\ \alpha \cdot \beta \in \ON\ )$.
				\item $\alpha \in {\bf \omega}$のとき,$\forall \beta \in {\bf \omega}\ 
				(\ \alpha \cdot \beta \in {\bf \omega}\ )$.
			\end{itemize}
		\end{thm}
	\end{screen}
	
	\begin{screen}
		\begin{dfn}[Peanoシステム]
			次を満たす集合$X,a$と写像$f:X \longrightarrow X$の組$(X,a,f)$を
			{\bf Peanoシステム}\index{Peanoしすてむ@Peanoシステム}と呼ぶ:
			\begin{itemize}
				\item $a \in X$.
				\item $a \notin f(X)$.
				\item $f$は単射である.
				\item 集合$S$が$a$を含み,かつ$S$の任意の要素$x$に対し$f(x) \in S$となるならば,$S = X$.
			\end{itemize}
		\end{dfn}
	\end{screen}
	
	\begin{screen}
		\begin{thm}[$\omega$はPeanoシステム]
			写像$\sigma:\omega \longrightarrow \omega$を
			\begin{align}
				\sigma = \Set{(n, n \cup \{n\})}{n \in \omega}
			\end{align}
			で定めるとき,$(\omega,\emptyset,\sigma)$はPeanoシステムとなる.
			この$\sigma$を{\bf 後継者写像}\index{こうけいしゃしゃぞう@後継者写像}{\bf (successor mapping)}と呼ぶ.
		\end{thm}
	\end{screen}
	
	\begin{prf}
		$\emptyset \in \omega$である.また任意の自然数$n$に対し$\sigma(n) = n \cup \{n\} \neq \emptyset$となる.
		任意の自然数$n,m$に対して
		\begin{align}
			\sigma(x)=\sigma(y) \Longrightarrow &(x \in y \cup \{y\}) \wedge (y \in x \cup \{x\}) \\
					&\Longleftrightarrow \left( x \in y \vee x=y \right) \wedge 
						\left( y \in x \vee y=x \right) \\
					&\Longleftrightarrow  \left( x \in y \wedge y \in x \right) \vee x = y \\
					&\Longleftrightarrow x = y
		\end{align}
		が成立するから$\sigma$は単射である.
	\end{prf}
	
	\begin{screen}
		\begin{thm}[Peanoシステムの写像は全単射]\label{thm:successor_mapping_is_injective}
			Peanoシステム$(X,a,f)$の$f$は$X$から$X \backslash \{a\}$への全単射である.
		\end{thm}
	\end{screen}
	
	\begin{prf}
		$S \coloneqq \{a\} \cup \sigma(X)$とおけば,数学的帰納法の原理より$S = X$が成り立ち$f$の全射性が出る.
		\QED
	\end{prf}
	
	\begin{screen}
		\begin{thm}[再帰定理]\label{thm:Peano_recursion_theorem}
			$Y$を空でない集合,$b$を$Y$の要素,$g$を$Y$から$Y$への写像とし,
			$(X,a,f)$をPeanoシステムとする.
			このとき,次を満たすような写像$u:X \longrightarrow Y$がただ一つ存在する:
			\begin{align}
				u(a) = b,\quad u \circ f = g \circ u.
				\label{eq:thm_Peano_recursion_theorem}
			\end{align}
		\end{thm}
	\end{screen}
	
	\begin{prf}
		$X \times Y$の部分集合で
		\begin{itemize}
			\item $(a,b)$を含む
			\item $(x,y)$を含むなら$(f(x),g(y))$も含む
		\end{itemize}
		を満たすものの全体を$\mathscr{A}$で表し
		\begin{align}
			u \coloneqq \bigcap \mathscr{A}
		\end{align}
		とおく.このとき$u \in \mathscr{A}$であるが,一方で$u$は
		$X$から$Y$への写像になっている.これは
		\begin{align}
			S \coloneqq \Set{x \in X}{\mbox{$(x,y),(x,z) \in u$なら$y=z$}}
		\end{align}
		により定める$S$が$X$に一致することを示せばよい.
		\begin{description}
			\item[第一段] $a \in S$を示す.$b \neq c$となる$Y$の要素$c$に対し,
				$\mathscr{A}$の或る元$A$で$(a,c) \in A$となるとき,
				\begin{align}
					A' \coloneqq A \backslash \{(a,c)\}
				\end{align}
				もまた$\mathscr{A}$に属する.実際$(a,b)$は
				$A$から除かれていないから$(a,b) \in A$,かつ
				定理\ref{thm:successor_mapping_is_injective}より
				\begin{align}
					(x,y) \in A' \quad \Longrightarrow \quad
					(f(x),g(y)) \neq (a,b) \quad \Longrightarrow \quad
					(f(x),g(y)) \in A'
				\end{align}
				が満たされる.従って$b$と異なる$Y$の任意の要素$c$で
				$(a,c) \notin u$が成り立ち$a \in S$が得られる.
				
			\item[第二段] 
				$S$の任意の元$x$に対して或る$Y$の元$y$がただ一つ対応して$(x,y) \in u$となるが,
				このとき
				\begin{align}
					B \coloneqq (X \times Y) \backslash \Set{(x,z)}{z \in Y,\ y \neq z}
				\end{align}
				とおけば$B \in \mathscr{A}$が成り立つ.そして
				$w \neq g(y)$を満たす$Y$の任意の要素$w$に対して
				\begin{align}
					B' \coloneqq B \backslash \{(f(x),w)\}
				\end{align}
				もまた$\mathscr{A}$に属する.実際$a \neq f(x)$かつ
				$(a,b) \in B$より$(a,b) \in B'$となり,また$(s,t) \in B'$に対し
				\begin{itemize}
					\item $s \neq x$ならば$f(s) \neq f(x)$より$(f(s),g(t)) \in B'$,
					\item $s=x$ならば$t = y$より$(f(s),g(t)) = (f(x),g(y)) \in B'$,
				\end{itemize}
				が成立する.よって$w \neq g(y)$ならば$(f(x),w) \notin U$となり$f(x) \in S$が従う.
		\end{description}
		以上と数学的帰納法の原理より$S = X$を得る.すなわち$u$は写像であり,$u$の任意の元$(x,y)$で
		\begin{align}
			u(f(x)) = g(y) = g(u(x))
		\end{align}
		となるから$u \circ f = g \circ u$が成り立つ.また
		写像$v:\omega \longrightarrow X$が$v(a) = b$かつ
		$v \circ f = g \circ v$を満たすとき,
		\begin{itemize}
			\item $u(a) = b = v(a)$,
			\item $u(x) = v(x) \Longrightarrow 
				u(f(x)) = g(u(x)) = g(v(x)) = v(f(x))$
		\end{itemize}
		が成立するから$u = v$となる.よって(\refeq{eq:thm_Peano_recursion_theorem})
		を満たす写像は$u$のみである.
		\QED
	\end{prf}
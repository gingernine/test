\subsection{$T_0$空間}
	\begin{screen}
		\begin{dfn}[位相的に識別可能]
			$(S,\mathscr{O})$を位相空間とし,$b$を$S$の部分集合とするときその$\mathscr{O}$-閉包を$\overline{b}$と書く.
			$x$と$y$を$S$の要素とするとき,
			\begin{align}
				x \notin \overline{\{y\}} \vee y \notin \overline{\{x\}}
			\end{align}
			ならば$x$と$y$は$\mathscr{O}$に関して{\bf 位相的に識別可能である}\index{いそうてきにしきべつかのう@位相的に識別可能}
			{\bf (topologically distinguishable)}という.
		\end{dfn}
	\end{screen}
	
	\begin{screen}
		\begin{thm}[位相的に識別可能な二点は相異なる]
			$(S,\mathscr{O})$を位相空間とし,$x$と$y$を$S$の要素とする.このとき
			$x$と$y$が$\mathscr{O}$に関して位相的に識別可能ならば
			\begin{align}
				x \neq y.
			\end{align}
		\end{thm}
	\end{screen}
	
	\begin{prf}
		$x = y$なら
		\begin{align}
			x \in \overline{\{y\}} \wedge y \in \overline{\{x\}}
		\end{align}
		が成り立つので,対偶を取って
		\begin{align}
			x \notin \overline{\{y\}} \vee y \notin \overline{\{x\}} \Longrightarrow x \neq y
		\end{align}
		を得る.
		\QED
	\end{prf}
	
	\begin{screen}
		\begin{dfn}[$T_{0}$]
			$(S,\mathscr{O})$を位相空間とするとき,$S$の任意の2要素が$\mathscr{O}$に関して
			位相的に識別可能であるならば,$(S,\mathscr{O})$を{\bf $T_0$空間}
			\index{$T_0$くうかん@$T_0$空間}或いは{\bf Kolmogorov空間}\index{Kolmogorov空間}と呼ぶ.
		\end{dfn}
	\end{screen}
	
	空虚な真により$(\emptyset,\{\emptyset\})$は$T_{0}$空間である.
	
	
\subsection{一様化可能性}
	本節では,$\left(X,\sigma_X\right)$を群とするとき,$X$の要素$x$の逆元を
	\begin{align}
		-x
	\end{align}
	と書く.
	
	\begin{screen}
		\begin{dfn}[位相群]\label{def:topological_group}
			$\left(X,\sigma_X\right)$を群とし,$\mathscr{O}_X$を$X$上の位相とする.また
			$\mathscr{O}_{X \times X}$を$\mathscr{O}_X$から作られる$X \times X$上の積位相とする.
			\begin{description}
				\item[(tg1)] $\sigma_X$が$\mathscr{O}_{X \times X}/\mathscr{O}_X$-連続である.
				\item[(tg2)] 逆元を対応させる写像,つまり
					\begin{align}
						X \ni x \longmapsto -x
					\end{align}
					なる写像が$\mathscr{O}_X/\mathscr{O}_X$-連続である.
			\end{description}
			が満たされるとき,
			\begin{align}
				\left(\left(X,\sigma_X\right),\mathscr{O}_X\right)
				\label{pair_topological_group}
			\end{align}
			の三つ組を{\bf 位相群}\index{いそうぐん@位相群}{\bf (topological group)}と呼ぶ.また
			$(X,\mathscr{O}_X)$がHausdorffであるとき,(\refeq{pair_topological_group})をHausdorff位相群と呼ぶ.
		\end{dfn}
	\end{screen}
	
	$\left(\left(X,\sigma_X\right),\mathscr{O}_X\right)$を位相群とするとき,
	$\sigma_X$は連続であるから,$a$を$X$の任意の要素として
	\begin{align}
		X \ni x \longmapsto \sigma_X(x,a)
	\end{align}
	なる写像を
	\begin{align}
		\sigma_X^a
	\end{align}
	とおけば,これは$\mathscr{O}_X/\mathscr{O}_X$-連続である.さらにこのとき,
	\begin{align}
		\sigma_X^{-a}
	\end{align}
	なる写像は$\sigma_X^a$の逆写像であって,かつ$\mathscr{O}_X/\mathscr{O}_X$-連続なので,
	$\sigma_X^a$は$\mathscr{O}_X$に関する同相写像である.つまり,{\bf 位相群の左偏写像は同相である.}
	また
	\begin{align}
		X \ni x \longmapsto -x
	\end{align}
	なる写像も,自分自身が逆写像であるので$\mathscr{O}_X$に関して同相写像である.
	
	\begin{screen}
		\begin{dfn}[局所基]
			$\left(\left(X,\sigma_X\right),\mathscr{O}_X\right)$を位相群とするとき,
			$X$の単位元の基本近傍系を{\bf 局所基}\index{きょくしょき@局所基}{\bf (local base)}と呼ぶ.
		\end{dfn}
	\end{screen}
	
	\begin{screen}
		\begin{thm}[すべての要素が逆元で閉じている局所基が取れる]
		\label{thm:there_exists_a_local_base_whose_elements_are_closed_under_inversion}
			$\left(\left(X,\sigma_X\right),\mathscr{O}_X\right)$を位相群とするとき,
			$X$の単位元の基本近傍系を,その任意の要素$b$が
			\begin{align}
				\forall x \in b\, (\, -x \in b\, )
				\label{fom:thm_there_exists_a_local_base_whose_elements_are_closed_under_inversion}
			\end{align}
			を満たすように取れる.
		\end{thm}
	\end{screen}
	
	\begin{sketch}\mbox{}
		\begin{description}
			\item[第一段] 
				$X$の単位元を
				\begin{align}
					0_X
				\end{align}
				と書く.また
				\begin{align}
					X \ni x \longmapsto -x
				\end{align}
				なる写像を$i$とおく.いま$v$を$0_X$の任意の近傍とすると,
				\begin{align}
					u \subset i^{-1} \ast v
				\end{align}
				を満たす$0_X$の開近傍$u$が取れる.ここで
				\begin{align}
					w \defeq (i \ast u) \cap u
				\end{align}
				とおけば,$w$は$0_X$の開近傍であって
				\begin{align}
					w \subset v
				\end{align}
				を満たす.また$x$を$w$の任意の要素とすれば,
				\begin{align}
					x = -y
				\end{align}
				なる$u$の要素$y$が取れるので
				\begin{align}
					-x = y \in u
				\end{align}
				が成り立つ.他方で
				\begin{align}
					x \in u
				\end{align}
				なので
				\begin{align}
					-x \in i \ast u
				\end{align}
				も成り立ち
				\begin{align}
					-x \in (i \ast u) \cap u
				\end{align}
				が従う.ゆえに$w$は
				\begin{align}
					\forall x \in w\, (\, -x \in w\, )
				\end{align}
				を満たす.
				
			\item[第二段]
				$0_X$の近傍の全体を
				\begin{align}
					\mathscr{B}
				\end{align}
				とおき,$\mathscr{B}$の要素$v$に対して,$v$の部分集合で
				(\refeq{fom:thm_there_exists_a_local_base_whose_elements_are_closed_under_inversion})
				を満たす$0_X$の近傍の全体,つまり
				\begin{align}
					\Set{u}{u \in \mathscr{B} \wedge u \subset v \wedge 
					\forall \alpha \in \Phi\, \forall x \in u\, \left(\, |\alpha| \leq 1 \Longrightarrow
					s(\alpha,x) \in u\, \right)}
				\end{align}
				なる集合を対応させる関係を$h$とおくと,上の結果から
		\begin{align}
			\forall v \in \mathscr{B}\, \left(\, h(v) \neq \emptyset\, \right)
		\end{align}
		が成り立つ.ゆえに定理\ref{thm:direct_product_of_non_empty_sets_is_not_empty}より
		\begin{align}
			f \in \prod_{v \in \mathscr{B}} h(v)
		\end{align}
		なる集合$f$が取れる.そして
		\begin{align}
			\left\{f(v)\right\}_{v \in \mathscr{B}}
		\end{align}
		は$0_X$の基本近傍系であり,その全ての要素は均衡している.
		\end{description}
	\end{sketch}
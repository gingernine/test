\subsection{一様化可能性}
	\begin{screen}
		\begin{dfn}[位相群]\label{def:topological_group}
			$\left(X,\sigma_X\right)$を群とし,$\mathscr{O}_X$を$X$上の位相とする.また
			$\mathscr{O}_{X \times X}$を$\mathscr{O}_X$から作られる$X \times X$上の積位相とする.
			\begin{description}
				\item[(tg1)] $\sigma_X$が$\mathscr{O}_{X \times X}/\mathscr{O}_X$-連続である.
				\item[(tg2)] 逆元を対応させる写像,つまり
					\begin{align}
						\Set{(x,-x)}{x \in X}
					\end{align}
					が$\mathscr{O}_X/\mathscr{O}_X$-連続である.
			\end{description}
			が満たされるとき,
			\begin{align}
				\left(\left(X,\sigma_X\right),\mathscr{O}_X\right)
				\label{pair_topological_group}
			\end{align}
			の三つ組を{\bf 位相群}\index{いそうぐん@位相群}{\bf (topological group)}と呼ぶ.また
			$(X,\mathscr{O}_X)$がHausdorffであるとき,(\refeq{pair_topological_group})をHausdorff位相群と呼ぶ.
		\end{dfn}
	\end{screen}
	
	$\left(\left(X,\sigma_X\right),\mathscr{O}_X\right)$を位相群とするとき,
	$\sigma_X$は連続であるから,$a$を$X$の任意の要素として
	\begin{align}
		X \ni x \longmapsto \sigma_X(x,a)
	\end{align}
	なる写像を
	\begin{align}
		\sigma_X^a
	\end{align}
	とおけば,これは$\mathscr{O}_X/\mathscr{O}_X$-連続である.さらにこのとき,
	\begin{align}
		\sigma_X^{-a}
	\end{align}
	なる写像は$\sigma_X^a$の逆写像であって,かつ$\mathscr{O}_X/\mathscr{O}_X$-連続なので,
	$\sigma_X^a$は$\mathscr{O}_X$に関する同相写像である.つまり,{\bf 位相群の左偏写像は同相写像である.}
	また
	\begin{align}
		X \ni x \longmapsto -x
	\end{align}
	なる写像も,自分自身が逆写像であるから,{\bf 逆元を対応させる写像も同相写像である.}
	
	\begin{screen}
		\begin{dfn}[局所基]
			$\left(\left(X,\sigma_X\right),\mathscr{O}_X\right)$を位相群とするとき,
			$X$の単位元の基本近傍系を{\bf 局所基}\index{きょくしょき@局所基}{\bf (local base)}と呼ぶ.
		\end{dfn}
	\end{screen}
	
	\begin{screen}
		\begin{thm}[すべての要素が逆元で閉じている局所基が取れる]
		\label{thm:there_exists_a_local_base_whose_elements_are_closed_under_inversion}
			$\left(\left(X,\sigma_X\right),\mathscr{O}_X\right)$を位相群とするとき,
			$X$の単位元の基本近傍系を,その任意の要素$b$が
			\begin{align}
				\forall x \in b\, (\, -x \in b\, )
				\label{fom:thm_there_exists_a_local_base_whose_elements_are_closed_under_inversion}
			\end{align}
			を満たすように取れる.
		\end{thm}
	\end{screen}
	
	\begin{sketch}\mbox{}
		\begin{description}
			\item[第一段] 
				$X$の単位元を
				\begin{align}
					0_X
				\end{align}
				と書く.また
				\begin{align}
					i \defeq \Set{(x,-x)}{x \in X}
				\end{align}
				とおく.いま$v$を$0_X$の任意の近傍とすると,$i$の$\mathscr{O}_X/\mathscr{O}_X$-連続性から
				\begin{align}
					u \subset i^{-1} \ast v
				\end{align}
				を満たす$0_X$の開近傍$u$が取れる.ここで
				\begin{align}
					w \defeq (i \ast u) \cap u
				\end{align}
				とおけば,$w$は$0_X$の開近傍であって
				\begin{align}
					w \subset v
				\end{align}
				を満たす.また$x$を$w$の任意の要素とすれば,
				\begin{align}
					x = -y
				\end{align}
				なる$u$の要素$y$が取れるので
				\begin{align}
					-x = y \in u
				\end{align}
				が成り立つ.他方で
				\begin{align}
					x \in u
				\end{align}
				なので
				\begin{align}
					-x \in i \ast u
				\end{align}
				も成り立ち
				\begin{align}
					-x \in (i \ast u) \cap u
				\end{align}
				が従う.ゆえに$w$は
				\begin{align}
					\forall x \in w\, (\, -x \in w\, )
				\end{align}
				を満たす.
				
			\item[第二段]
				$0_X$の近傍の全体を
				\begin{align}
					\mathscr{B}
				\end{align}
				とおき,$\mathscr{B}$の要素$v$に対して,$v$の部分集合で
				(\refeq{fom:thm_there_exists_a_local_base_whose_elements_are_closed_under_inversion})
				を満たす$0_X$の近傍の全体,つまり
				\begin{align}
					\Set{u}{u \in \mathscr{B} \wedge u \subset v \wedge 
					\forall x \in u\, \left(\, -x \in u\, \right)}
				\end{align}
				なる集合を対応させる関係を$h$とおくと,上の結果から
				\begin{align}
					\forall v \in \mathscr{B}\, \left(\, h(v) \neq \emptyset\, \right)
				\end{align}
				が成り立つ.ゆえに定理\ref{thm:direct_product_of_non_empty_sets_is_not_empty}より
				\begin{align}
					f \in \prod_{v \in \mathscr{B}} h(v)
				\end{align}
				なる集合$f$が取れる.そして
				\begin{align}
					\left\{f(v)\right\}_{v \in \mathscr{B}}
				\end{align}
				は$0_X$の基本近傍系であり,その全ての要素は
				(\refeq{fom:thm_there_exists_a_local_base_whose_elements_are_closed_under_inversion})を満たす.
				\QED
		\end{description}
	\end{sketch}
	
	次に述べることは{\bf 位相群が一様化可能である}ということである.
	いま$\left(\left(X,\sigma_X\right),\mathscr{O}_X\right)$を位相群とし,
	\begin{align}
		0_X
	\end{align}
	を$\left(X,\sigma_X\right)$の単位元とし,
	\begin{align}
		\mathscr{B}
	\end{align}
	を$0_X$の基本近傍系とし,$\mathscr{B}$のすべての要素は逆元で閉じているとする.つまり
	\begin{align}
		\forall b \in \mathscr{B}\, \left[\, \forall x\, (\, x \in b \Longrightarrow -x \in b\, )\, \right].
	\end{align}
	ここで$\mathscr{B}$の要素$b$に対して
	\begin{align}
		\Set{(x,y) \in X \times X}{\sigma_X(x,-y) \in b}
	\end{align}
	で定められる集合を,$\mathscr{B}$のすべての要素に亘って取ったものの全体を$\mathscr{U}$と定める.つまり
	\begin{align}
		\mathscr{U} \defeq \Set{u}{\exists b \in \mathscr{B}\,
		\left[\, \forall x,y \in X\, \left(\, (x,y) \in u \Longleftrightarrow\sigma_X(x,-y) \in b\, \right)\, \right]}
	\end{align}
	と定める.すると,
	\begin{align}
		\mathscr{V} \defeq \Set{v}{v \subset X \times X \wedge \exists u \in \mathscr{U}\, \left(\, u \subset v\, \right)}
	\end{align}
	で定める$\mathscr{V}$は$X$上の近縁系となる.実際,
	\begin{description}
		\item[(a)] $\mathscr{B}$は空ではないので
			\begin{align}
				\mathscr{U} \neq \emptyset
			\end{align}
			である.ゆえに
			\begin{align}
				\mathscr{V} \neq \emptyset
			\end{align}
			である.また$v$を$\mathscr{V}$の任意の要素とすると
			\begin{align}
				\Set{(x,y)}{x \in X \wedge y \in X \wedge \sigma_X(x,-y) \in b} \subset v
			\end{align}
			を満たす$\mathscr{B}$の要素$b$が取れるが,$X$の任意の要素$x$に対して
			\begin{align}
				\sigma_X(x,-x) = 0_X \in b
			\end{align}
			が成り立つので
			\begin{align}
				\Set{(x,x)}{x \in X} \subset v
			\end{align}
			が成立する.
			
		\item[(b)] $v$を$\mathscr{V}$の任意の要素とする.このとき
			\begin{align}
				\Set{(x,y)}{x \in X \wedge y \in X \wedge \sigma_X(x,-y) \in b} \subset v
			\end{align}
			を満たす$\mathscr{B}$の要素$b$が取れる.
			\begin{align}
				u \defeq \Set{(x,y)}{x \in X \wedge y \in X \wedge \sigma_X(x,-y) \in b}
			\end{align}
			とおけば
			\begin{align}
				u^{-1} = \Set{(y,x)}{x \in X \wedge y \in X \wedge \sigma_X(x,-y) \in b}
			\end{align}
			となるが,
			\begin{align}
				\sigma_X(x,-y) = -\sigma_X(y,-x)
			\end{align}
			かつ$b$は逆元で閉じているので
			\begin{align}
				\forall x,y \in X\, \left[\, \sigma_X(x,-y) \in b \Longleftrightarrow \sigma_X(y,-x) \in b\, \right]
			\end{align}
			が成り立つ.ゆえに
			\begin{align}
				u^{-1} = \Set{(y,x)}{x \in X \wedge y \in X \wedge \sigma_X(y,-x) \in b}
			\end{align}
			が成り立つ.ゆえに
			\begin{align}
				u^{-1} \in \mathscr{U}
			\end{align}
			が成り立つ.
			\begin{align}
				u^{-1} \subset v^{-1}
			\end{align}
			であるから
			\begin{align}
				v^{-1} \in \mathscr{V}
			\end{align}
			が従う.
			
		\item[(c)] $u$と$v$を$\mathscr{V}$の要素とする.このとき
			\begin{align}
				\Set{(x,y)}{x \in X \wedge y \in X \wedge \sigma_X(x,-y) \in a} \subset u
			\end{align}
			を満たす$\mathscr{B}$の要素$a$と
			\begin{align}
				\Set{(x,y)}{x \in X \wedge y \in X \wedge \sigma_X(x,-y) \in b} \subset v
			\end{align}
			を満たす$\mathscr{B}$の要素$b$が取れる.このとき
			\begin{align}
				c \subset a \cap b
			\end{align}
			なる$\mathscr{B}$の要素$c$が取れて
			\begin{align}
				\Set{(x,y)}{x \in X \wedge y \in X \wedge \sigma_X(x,-y) \in c} \subset u \cap v
			\end{align}
			が成り立つので,
			\begin{align}
				u \cap v \subset \mathscr{V}
			\end{align}
			が従う.
			
		\item[(d)] $v$を$\mathscr{V}$の任意の要素とする.すると
			\begin{align}
				\Set{(x,y)}{x \in X \wedge y \in X \wedge \sigma_X(x,-y) \in b} \subset v
			\end{align}
			を満たす$\mathscr{B}$の要素$b$が取れて,$\sigma_X$は連続なので
			\begin{align}
				a \times c \subset \sigma_X^{-1} \ast b
			\end{align}
			を満たす$\mathscr{B}$の要素$a$と$c$が取れる.ここで
			\begin{align}
				d \subset a \cap c
			\end{align}
			を満たす$\mathscr{B}$の要素$d$を取り,
			\begin{align}
				w \defeq \Set{(x,y)}{x \in X \wedge y \in X \wedge \sigma_X(x,-y) \in d}
			\end{align}
			により$\mathscr{V}$の要素$w$を定める.このとき,$x$と$y$と$z$を$X$の任意の要素として
			\begin{align}
				(x,y) \in w \wedge (y,z) \in w
			\end{align}
			であるとすると,
			\begin{align}
				\sigma_X(x,-y) \in d
			\end{align}
			かつ
			\begin{align}
				\sigma_X(y,-z) \in d
			\end{align}
			が成り立つので
			\begin{align}
				\sigma_X\left(\sigma_X(x,-y),\sigma_X(y,-z)\right) \in b
			\end{align}
			が成立する.他方で$\sigma_X$の結合性から
			\begin{align}
				\sigma_X\left(\sigma_X(x,-y),\sigma_X(y,-z)\right)
				&= \sigma_X\left(\sigma_X\left(\sigma_X(x,-y),y\right),-z\right) \\
				&= \sigma_X\left(\sigma_X\left(x,\sigma_X(-y,y)\right),-z\right) \\
				&= \sigma_X(x,-z)
			\end{align}
			が成り立つので,
			\begin{align}
				(x,z) \in v
			\end{align}
			が従う.ゆえに$w$は
			\begin{align}
				w \circ w \subset v
			\end{align}
			を満たす.
			
		\item[(e)] $r$を$X$上の関係とし,
			\begin{align}
				v \subset r
			\end{align}
			を満たす$\mathscr{V}$の要素$v$が取れるとする.このとき
			\begin{align}
				u \subset v
			\end{align}
			なる$\mathscr{U}$の要素が取れて
			\begin{align}
				u \subset r
			\end{align}
			が成り立つので
			\begin{align}
				r \in \mathscr{V}
			\end{align}
			が従う.
			\QED
	\end{description}
	以上より$\mathscr{V}$が$X$上の近縁系であることが示された.後は$\mathscr{V}$により導入する一様位相が
	$\mathscr{O}_X$と両立することを示せば一様化可能性が言えるが,それは次の定理から従う.
	
	\begin{screen}
		\begin{thm}[位相群の位相は局所基で決まる]
			$\left(\left(X,\sigma_X\right),\mathscr{O}_X\right)$を位相群とし,
			$\mathscr{B}$を$\left(X,\sigma_X\right)$の単位元の基本近傍系とする.また$x$を$X$の任意の要素とする.このとき,
			\begin{align}
				\Set{\sigma_X(x,y)}{y \in b}
			\end{align}
			を$\mathscr{B}$の全ての要素に亘って取ったものの全体は$x$の基本近傍系である.
		\end{thm}
	\end{screen}
	
	言い換えれば,
	\begin{align}
		\Set{u}{\exists b \in \mathscr{B}\, 
		\forall z\, \left[\, z \in u \Longleftrightarrow \exists y \in b\, \left(\, z=\sigma_X(x,y)\, \right) \, \right]}
	\end{align}
	が$x$の基本近傍系であるということである.
	
	\begin{sketch}
		$v$を$x$の任意に与えられた近傍とする.また
		\begin{align}
			X \ni y \longmapsto \sigma_X(x,y)
		\end{align}
		なる写像を$\sigma_X^x$とし,
		\begin{align}
			X \ni y \longmapsto \sigma_X(-x,y)
		\end{align}
		なる写像を$\sigma_X^{-x}$とする.$\sigma_X^{-x}$は$\mathscr{O}_X/\mathscr{O}_X$-連続であるから
		\begin{align}
			b \subset {\sigma_X^{-x}}^{-1} \ast v
		\end{align}
		を満たす$\mathscr{B}$の要素$b$が取れる.ここで$y$を$b$の任意の要素とすれば
		\begin{align}
			y = \sigma_X(-x,z)
		\end{align}
		を満たす$v$の要素$z$が取れるが,
		\begin{align}
			\sigma_X(x,y) &= \sigma_X\left(x,\sigma_X(-x,z)\right) \\
			&= \sigma_X\left(\sigma_X(x,-x),z\right) \\
			&= z
		\end{align}
		が成り立つので
		\begin{align}
			\sigma_X^x(y) \in v
		\end{align}
		が成り立つ.すなわち
		\begin{align}
			\sigma_X^x \ast b \subset v
		\end{align}
		が従う.ところで$\sigma_X^x$は$\mathscr{O}_X$に関して同相であるから
		\begin{align}
			\sigma_X^x \ast b \in \mathscr{O}_X
		\end{align}
		である.ゆえに,
		\begin{align}
			\Set{\sigma_X^x \ast b}{b \in \mathscr{B}}
		\end{align}
		は$x$の基本近傍系である.書き直せば
		\begin{align}
			\Set{u}{\exists b \in \mathscr{B}\, 
			\forall z\, \left[\, z \in u \Longleftrightarrow \exists y \in b\, \left(\, z=\sigma_X(x,y)\, \right) \, \right]}
		\end{align}
		は$x$の基本近傍系である.
		\QED
	\end{sketch}
	
	一様化可能性についての言明をまとめるが,式で見づらくなっているところは上で意訳してある.
	\begin{screen}
		\begin{thm}[位相群は一様化可能である]
			$\left(\left(X,\sigma_X\right),\mathscr{O}_X\right)$を位相群とし,
			$\mathscr{B}$を$\left(X,\sigma_X\right)$の単位元の基本近傍系とし,
			\begin{align}
				\forall b \in \mathscr{B}\, \left[\, \forall x\, (\, x \in b \Longrightarrow -x \in b\, )\, \right].
			\end{align}
			であるとする.ここで
			\begin{align}
				\mathscr{U} \defeq \Set{u}{\exists b \in \mathscr{B}\,
				\left[\, \forall x,y \in X\, \left(\, (x,y) \in u \Longleftrightarrow\sigma_X(x,-y) \in b\, \right)\, \right]}
			\end{align}
			とおいて
			\begin{align}
				\mathscr{V} \defeq \Set{v}{v \subset X \times X \wedge \exists u \in \mathscr{U}\, \left(\, u \subset v\, \right)}
			\end{align}
			と定めると,$\mathscr{V}$は$X$上の近縁系となり$\mathscr{O}_X$と両立する.
		\end{thm}
	\end{screen}
	
	\begin{sketch}
		
	\end{sketch}
\section{Continuous, Square-Integrable Martingales}
	\begin{itembox}[l]{Processes of difference of two natural processes}
		Let denote the space of processes represented by difference of two natural processes as
		\begin{align}
			\mathscr{A} \coloneqq \Set{A^{(1)} - A^{(2)}}{A^{(j)} \in NAT,\ j=1,2},
		\end{align}
		and the equivalent class of $A \in \mathscr{A}$ in the meaning of
		(\refeq{eq:equivalence_with_respect_to_path}) in $\mathscr{A}$ as
		$[A]_{\mathscr{A}}$. Similarly define
		\begin{align}
			\mathscr{A}_c \coloneqq \Set{A^{(1)} - A^{(2)}}{A^{(j)} \in NAT,\ \mbox{continuous},\ j=1,2}
		\end{align}
		and the equivalent class of $A \in \mathscr{A}_c$ in the meaning of
		(\refeq{eq:equivalence_with_respect_to_path}) in $\mathscr{A}_c$ as
		$[A]_{\mathscr{A}_c}$.
	\end{itembox}
	
	\begin{itembox}[l]{Definition 5.3 修正}
		For $X \in \mathscr{M}_2$, we define the quadratic variation of $X$ to be the process $\inprod<X>_t \coloneqq A_t$,
		where $A$ is the natural increasing process in the Doob-Meyer decomposition of $x^2$.
		\textcolor{red}{For $X \in \mathscr{M}_2^c$, the quadratic variation $\inprod<X>$ of $X$ 
		to be natural increasing and continuous process.}
	\end{itembox}
	
	\begin{itembox}[l]{Problem 5.7}
		Show that $\inprod<\cdot,\cdot>$ is a bilinear form on $\mathscr{M}_2$, i.e.,
		for any members $X,Y,Z$ of $\mathscr{M}_2$ and real numbers $\alpha,\beta$, we have
		\begin{description}
			\item[(i)] $[\inprod<\alpha X + \beta Y,Z>]_{\mathscr{A}} 
				= [\alpha \inprod<X,Z> + \beta \inprod<Y,Z>]_{\mathscr{A}}$.
			\item[(ii)] $[\inprod<X,Y>]_{\mathscr{A}} = [\inprod<Y,X>]_{\mathscr{A}}$.
			\item[(iii)] $|\inprod<X,Y>|^2 \leq \inprod<X> \inprod<Y>$.
			\item[(iv)] For $P$-a.e. $\omega \in \Omega$,
				\begin{align}
					\check{\xi}_t(\omega) - \check{\xi}_s(\omega)
					\leq \frac{1}{2}[\inprod<X>_t(\omega) - \inprod<X>_s(\omega)
						+ \inprod<Y>_t(\omega) - \inprod<Y>_s(\omega)];
						\quad 0 \leq s < t < \infty,
				\end{align}
				where $\check{\xi}_t$ denotes the total variation of 
				$\check{\xi} \coloneqq \inprod<X,Y>$ on $[0,t]$.
		\end{description}
	\end{itembox}
	
	\begin{prf}\mbox{}
		\begin{description}
			\item[(i)] ナチュラルなプロセス
				$A^{(j)},B^{(j)},C^{(j)},\ (j=1,2)$により
				\begin{align}
					\inprod<\alpha X + \beta Y, Z> = A^{(1)} - A^{(2)},
					\quad \alpha \inprod<X,Z> = B^{(1)} - B^{(2)},
					\quad \beta \inprod<Y,Z> = C^{(1)} - C^{(2)}
				\end{align}
				と表せるから
				\begin{align}
					\inprod<\alpha X + \beta Y, Z> 
					- \left(\alpha \inprod<X,Z> + \beta \inprod<Y,Z>\right)
					= \left(A^{(1)} + B^{(2)} + C^{(2)}\right)
					- \left(A^{(2)} + B^{(1)} + C^{(1)}\right)
				\end{align}
				となり,P. \pageref{lem:uniqueness_of_Doob_Meyer_decomposition}の補題より
				\begin{align}
					\inprod<\alpha X + \beta Y, Z>_t 
					= \alpha \inprod<X,Z>_t + \beta \inprod<Y,Z>_t,
					\quad 0 \leq t < \infty,
					\quad \mbox{a.s. $P$}
				\end{align}
				が従う.
			
			\item[(ii)] 
				$XY - \inprod<X,Y>$も$YX - \inprod<Y,X>$も右連続マルチンゲールであるから
				\begin{align}
					\inprod<X,Y> - \inprod<Y,X>
				\end{align}
				も右連続マルチンゲールであり,P. \pageref{lem:uniqueness_of_Doob_Meyer_decomposition}の補題より
				\begin{align}
					\inprod<X,Y>_t = \inprod<Y,X>_t,
					\quad 0 \leq t < \infty,
					\quad \mbox{a.s. $P$}
				\end{align}
				が従う.

			\item[(iii)] Shwartzの不等式
		\end{description}
	\end{prf}
	
	\begin{itembox}[l]{Lemma 5.9}
		Let $X \in \mathscr{M}_2$ satisfy...
	\end{itembox}
	
	\begin{prf}
		$[0,t]$の分割$\Pi = \{t_0,t_1,\cdots,t_m\}$に対し
		\begin{align}
			
		\end{align}
	\end{prf}
	
	\begin{itembox}[l]{Theorem 5.13 修正}
		Let $X = \Set{X_t,\mathscr{F}_t}{0 \leq t < \infty}$ and $Y = \Set{Y_t,\mathscr{F}_t}{0 \leq t < \infty}$
		be members of $\mathscr{M}_2^c$. \textcolor{red}{There is a unique $[A]_{\mathscr{A}_c}$ such that
		$\Set{X_t Y_t - \tilde{A}_t,\mathscr{F}_t}{0 \leq t < \infty}$ is a continuous martingale
		for every $\tilde{A} \in [A]_{\mathscr{A}_c}$.}
	\end{itembox}
	
	\begin{prf}
		定義より$\inprod<X,Y> \in \mathscr{A}_c$に対して$XY - \inprod<X,Y>$
		は連続マルチンゲールである.また$\inprod<X,Y>$と区別不能な$A \in \mathscr{A}_c$を取れば,任意の$t \geq 0$で
		\begin{align}
			P(X_t Y_t - \inprod<X,Y>_t = X_t Y_t - A_t) = 1
		\end{align}
		となるから$XY-A$もまた連続マルチンゲールとなる.
		$A,B \in \mathscr{A}_c$に対し$XY - A,\ XY-B$が共にマルチンゲールとなるとき,
		$A - B$もマルチンゲールとなり,Theorem 4.14の補題(P. \pageref{lem:uniqueness_of_Doob_Meyer_decomposition})より
		$[A]_{\mathscr{A}_c} = [B]_{\mathscr{A}_c}$が従う.
		\QED
	\end{prf}
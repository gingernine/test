\section{Continuous, Square-Integrable Martingales}
	\begin{itembox}[l]{Processes of difference of two natural processes}
		Let denote the space of processes represented by difference of two natural processes as
		\begin{align}
			\mathscr{A} \coloneqq \Set{A^{(1)} - A^{(2)}}{A^{(j)} \ \mbox{: natural},\ j=1,2},
		\end{align}
		and the equivalent class of $A \in \mathscr{A}$ in the meaning of
		(\refeq{eq:equivalence_with_respect_to_path}) in $\mathscr{A}$ as
		$[A]_{\mathscr{A}}$. Similarly define
		\begin{align}
			\mathscr{A}_c \coloneqq \Set{A^{(1)} - A^{(2)}}{A^{(j)} \ \mbox{: natural, continuous},\ j=1,2}
		\end{align}
		and the equivalent class of $A \in \mathscr{A}_c$ in the meaning of
		(\refeq{eq:equivalence_with_respect_to_path}) in $\mathscr{A}_c$ as
		$[A]_{\mathscr{A}_c}$.
	\end{itembox}
	
	\begin{itembox}[l]{Definition 5.3 修正}
		For $X \in \mathscr{M}_2$, we define the quadratic variation of $X$ to be the process $\inprod<X>_t \coloneqq A_t$,
		where $A$ is the natural increasing process in the Doob-Meyer decomposition of $x^2$.
		\textcolor{red}{For $X \in \mathscr{M}_2^c$, the quadratic variation $\inprod<X>$ of $X$ 
		to be natural increasing and continuous process.}
	\end{itembox}
	
	\begin{itembox}[l]{Problem 5.7 修正}
		Show that $\inprod<\cdot,\cdot>$ is a bilinear form on $\mathscr{M}_2$, i.e.,
		for any members $X,Y,Z$ of $\mathscr{M}_2$ and real numbers $\alpha,\beta$, we have
		\begin{description}
			\item[(i)] $[\inprod<\alpha X + \beta Y,Z>]_{\mathscr{A}} 
				= [\alpha \inprod<X,Z> + \beta \inprod<Y,Z>]_{\mathscr{A}}$.
			\item[(ii)] $[\inprod<X,Y>]_{\mathscr{A}} = [\inprod<Y,X>]_{\mathscr{A}}$.
			\item[(iii)] $|\inprod<X,Y>|^2 \leq \inprod<X> \inprod<Y>$.
			\item[(iv)] For $P$-a.e. $\omega \in \Omega$,
				\begin{align}
					\check{\xi}_t(\omega) - \check{\xi}_s(\omega)
					\leq \frac{1}{2}[\inprod<X>_t(\omega) - \inprod<X>_s(\omega)
						+ \inprod<Y>_t(\omega) - \inprod<Y>_s(\omega)];
						\quad 0 \leq s < t < \infty,
				\end{align}
				where $\check{\xi}_t$ denotes the total variation of 
				$\check{\xi} \coloneqq \inprod<X,Y>$ on $[0,t]$.
				
			\item[(v)] For any stopping time $T$ of $(\mathscr{F}_t)_{t \geq 0}$, we have 
				\begin{align}
					P \left( \inprod<X>_{t \wedge T} = \inprod<X^T>_t,\ \forall 0 \leq t < \infty \right) = 1,
				\end{align}
				where $X^T_t \coloneqq X_{t \wedge T},\ (\forall t \geq 0)$.
		\end{description}
	\end{itembox}
	
	\begin{prf}\mbox{}
		\begin{description}
			\item[(i)] ナチュラルなプロセス
				$A^{(j)},B^{(j)},C^{(j)},\ (j=1,2)$により
				\begin{align}
					\inprod<\alpha X + \beta Y, Z> = A^{(1)} - A^{(2)},
					\quad \alpha \inprod<X,Z> = B^{(1)} - B^{(2)},
					\quad \beta \inprod<Y,Z> = C^{(1)} - C^{(2)}
				\end{align}
				と表せるから
				\begin{align}
					\inprod<\alpha X + \beta Y, Z> 
					- \left(\alpha \inprod<X,Z> + \beta \inprod<Y,Z>\right)
					= \left(A^{(1)} + B^{(2)} + C^{(2)}\right)
					- \left(A^{(2)} + B^{(1)} + C^{(1)}\right)
				\end{align}
				となり,P. \pageref{lem:uniqueness_of_Doob_Meyer_decomposition}の補題より
				\begin{align}
					\inprod<\alpha X + \beta Y, Z>_t 
					= \alpha \inprod<X,Z>_t + \beta \inprod<Y,Z>_t,
					\quad 0 \leq t < \infty,
					\quad \mbox{a.s. $P$}
				\end{align}
				が従う.
			
			\item[(ii)] 
				$XY - \inprod<X,Y>$も$YX - \inprod<Y,X>$も右連続マルチンゲールであるから
				\begin{align}
					\inprod<X,Y> - \inprod<Y,X>
				\end{align}
				も右連続マルチンゲールであり,P. \pageref{lem:uniqueness_of_Doob_Meyer_decomposition}の補題より
				\begin{align}
					\inprod<X,Y>_t = \inprod<Y,X>_t,
					\quad 0 \leq t < \infty,
					\quad \mbox{a.s. $P$}
				\end{align}
				が従う.

			\item[(iii)] Shwartzの不等式
		\end{description}
	\end{prf}
	
	\begin{itembox}[l]{Lemma 5.9}
		Let $X \in \mathscr{M}_2$ satisfy $|X_s| \leq K < \infty$ for all $s \in [0,t]$,
		a.s. $P$. Let $\Pi = \{t_0,t_1,\cdots,t_m\}$, with $0 = t_0 \leq t_1 \leq \cdots \leq
		t_m = t$, be a partition of $[0,t]$. Then $E\left( V_t^{(2)}(\Pi) \right)^2 \leq 6K^4$.
	\end{itembox}
	
	\begin{prf}
		$X$のマルチンゲール性により,任意の$0 \leq s_0 \leq s_1 \leq \cdots \leq s_n < \infty$に対して
		\begin{align}
			E \sum_{k=1}^n \left|X_{s_k} - X_{s_{k-1}}\right|^2
			&= \sum_{k=1}^n E \left\{\cexp{X_{s_k}^2 -2X_{s_k}X_{s_{k-1}} + X_{s_{k-1}}^2}{\mathscr{F}_{s_k}}\right\} \\
			&= \sum_{k=1}^n E \left\{X_{s_k}^2 -2\cexp{X_{s_k}}{\mathscr{F}_{s_k}}X_{s_{k-1}} + X_{s_{k-1}}^2\right\} \\
			&= \sum_{k=1}^n E \left( X_{s_k}^2 - X_{s_{k-1}}^2 \right) \\
			&= E X_{s_n}^2 - E X_{s_0}^2
			\label{eq:chapter_1_lemma_5_9_1}
		\end{align}
		が成立する.いま,
		\begin{align}
			E\left( V_t^{(2)}(\Pi) \right)^2
			= E \left\{ \sum_{k=1}^m \left|X_{t_k} - X_{t_{k-1}}\right|^2 \right\}^2
			= E \sum_{k=1}^m \left|X_{t_k} - X_{t_{k-1}}\right|^4
				+ 2 E \sum_{i=1}^{m-1} \sum_{j=i+1}^m 
				\left|X_{t_i} - X_{t_{i-1}}\right|^2\left|X_{t_j} - X_{t_{j-1}}\right|^2
		\end{align}
		と分解すれば,$\left|X_{t_k} - X_{t_{k-1}}\right|^2 \leq 2\left( X_{t_k}^2 + X_{t_{k-1}}^2 \right)^2 \leq 2K^2$
		と(\refeq{eq:chapter_1_lemma_5_9_1})より右辺第一項は
		\begin{align}
			E \sum_{k=1}^m \left|X_{t_k} - X_{t_{k-1}}\right|^4
			\leq 2K^2 E \sum_{k=1}^m \left|X_{t_k} - X_{t_{k-1}}\right|^2
			= 2K^2 E X_{t_m}^2
			\leq 2K^4
		\end{align}
		となる.また右辺第二項も(\refeq{eq:chapter_1_lemma_5_9_1})より
		\begin{align}
			\sum_{i=1}^{m-1} \sum_{j=i+1}^m 
				E \left|X_{t_i} - X_{t_{i-1}}\right|^2\left|X_{t_j} - X_{t_{j-1}}\right|^2
			&= \sum_{i=1}^{m-1} \sum_{j=i+1}^m 
				E \left[ \cexp{\left|X_{t_i} - X_{t_{i-1}}\right|^2 
				\left|X_{t_j} - X_{t_{j-1}}\right|^2}{\mathscr{F}_{t_j}} \right] \\
			&= \sum_{i=1}^{m-1} \sum_{j=i+1}^m 
				E \left[\left|X_{t_i} - X_{t_{i-1}}\right|^2
				\cexp{\left|X_{t_j} - X_{t_{j-1}}\right|^2}{\mathscr{F}_{t_j}} \right] \\
			&= \sum_{i=1}^{m-1} \sum_{j=i+1}^m 
				E \left|X_{t_i} - X_{t_{i-1}}\right|^2 \left( X_{t_j}^2 - X_{t_{j-1}}^2 \right) \\
			&= \sum_{i=1}^{m-1} E \left|X_{t_i} - X_{t_{i-1}}\right|^2 \left( X_t^2 - X_{t_i}^2 \right) \\
			&\leq 2K^2 E \sum_{i=1}^{m-1} \left|X_{t_i} - X_{t_{i-1}}\right|^2 \\
			&\leq 2K^4
		\end{align}
		となるから$E\left( V_t^{(2)}(\Pi) \right)^2 \leq 6K^4$が出る.
		\QED
	\end{prf}
	
	\begin{itembox}[l]{Lemma 5.10}
		Let $X \in \mathscr{M}_2^c$ satisfy $|X_s| \leq K < \infty$ for all $s \in [0,t]$, a.s. $P$.
		For partitions $\Pi$ of $[0,t]$, we have
		\begin{align}
			\lim_{\Norm{\Pi}{} \to 0} E V_t^{(4)}(\Pi) = 0.
		\end{align}
	\end{itembox}
	
	\begin{prf}\mbox{}
		\begin{description}
			\item[第一段] 
				任意の$\omega \in \Omega$と$\delta > 0$に対し
				\begin{align}
					&\sup{}{\Set{|X_r(\omega) - X_s(\omega|}{s,r \in [0,t],\ |s-r| < \delta}} \\
					&\qquad = \sup{}{\Set{|X_p(\omega) - X_q(\omega)|}{p,q \in Q \cap [0,t],\ |q-p| < \delta}}
					\label{eq:chapter_1_lemma_5_10_1}
				\end{align}
				が成立する.実際,上限を取る範囲の大小関係より$\mbox{(左辺)} \geq \mbox{(右辺)}$が成り立ち,
				一方で任意の$\mbox{(左辺)} > \alpha > 0$に対し
				$|X_r(\omega) - X_s(\omega)| > \alpha$を満たす$s,r \in [0,t],\ (|s-r| < \delta)$を取れば,
				$X$のパスの連続性より
				\begin{align}
					|X_r(\omega) - X_p(\omega)|,\ |X_s(\omega) - X_q(\omega)| < \frac{\beta-\alpha}{2}
				\end{align}
				を満たす$p,q \in \Q \cap [0,t],\ (|p-q| < \delta)$が存在して
				\begin{align}
					|X_p(\omega) - X_q(\omega)| \geq |X_r(\omega) - X_s(\omega) | - |X_r(\omega) - X_p(\omega)| - |X_q(\omega) - X_s(\omega)| > \alpha
				\end{align}
				となり(\refeq{eq:chapter_1_lemma_5_10_1})が出る.
				$\mbox{(左辺)} \leq 2K$より$m_t(X;\delta)$は$\mathscr{F}/\borel{\R}$-可測である.
				また定理\ref{thm:exponentiation_of_supremum_supremum_of_exponentiation}より任意の$a > 0$で
				\begin{align}
					m_t^a(X;\delta) 
					\coloneqq \sup{}{\Set{|X_p(\omega) - X_q(\omega)|^a}{p,q \in Q \cap [0,t],\ |q-p| < \delta}}
				\end{align}
				が満たされる.
				
			\item[第二段]
				H\Ddot{o}lderの不等式より,任意の$\Pi$に対し
				\begin{align}
					E V_t^{(4)}(\Pi) \leq E\left[ V_t^{(2)}(\Pi) \cdot m_t^2(X;\Norm{\Pi}{}) \right]
					\leq \left\{ E \left(V_t^{(2)}(\Pi)\right)^2 \right\}^{1/2}
						\left\{ E m_t^4(X;\Norm{\Pi}{}) \right\}^{1/2}
					\leq \sqrt{6} K^2 \left\{ E m_t^4(X;\Norm{\Pi}{}) \right\}^{1/2}
				\end{align}
				となる.任意に$\Norm{\Pi_n}{} \longrightarrow 0,\ (n \longrightarrow \infty)$
				を満たす分割列$(\Pi_n)_{n=1}^\infty$を取れば
				\begin{align}
					\lim_{n \to \infty} m_t(X;\Norm{\Pi_n}{}) = 0,
					\quad m_t(X;\Norm{\Pi_n}{}) \leq 2K,\ (\forall n \geq 1)
				\end{align}
				が成り立つから,Lebesgueの収束定理より
				\begin{align}
					E m_t^4(X;\Norm{\Pi_n}{}) \longrightarrow 0\quad (n \longrightarrow \infty)
				\end{align}
				が従い$E V_t^{(4)}(\Pi_n) \longrightarrow 0\ (n \longrightarrow \infty)$となる.
				$(\Pi_n)_{n=1}^\infty$の任意性より補題の主張が得られる.
				\QED
		\end{description}
	\end{prf}
	
	\begin{itembox}[l]{Theorem 5.8}
		Let $X$ be in $\mathscr{M}_2^c$. For partitions $\Pi$ of $[0,t]$, we have
		$\lim_{\Norm{\Pi}{} \to 0} V_t^{(2)} = \inprod<X>_t$ (in probability); i.e.,
		for every $\epsilon > 0,\ \eta > 0$ there exists $\delta > 0$ such that $\Norm{\Pi}{} < \delta$ implies
		\begin{align}
			P\left[ \left| V_t^{(2)}(\Pi) - \inprod<X>_t \right| > \epsilon \right] < \eta.
		\end{align}
	\end{itembox}
	
	\begin{prf}\mbox{}
		\begin{description}
			\item[第一段]
				$X^2 - \inprod<X>$のマルチンゲール性より任意の$0 \leq s < t < \infty$に対して
				\begin{align}
					\cexp{(X_t - X_s)^2 - (\inprod<X>_t - \inprod<X>_s)}{\mathscr{F}_s}
					&= \cexp{(X_t - X_s)^2}{\mathscr{F}_s} - \cexp{\inprod<X>_t - \inprod<X>_s}{\mathscr{F}_s} \\
					&= \cexp{X_t^2 - X_s^2}{\mathscr{F}_s} - \cexp{\inprod<X>_t - \inprod<X>_s}{\mathscr{F}_s} \\
					&= 0,
					\quad \mbox{a.s. $P$}
				\end{align}
				となる.従って,任意の$0 \leq u < v \leq s < t < \infty$に対し
				\begin{align}
					E\left|(X_v - X_u)^2 - (\inprod<X>_v - \inprod<X>_u)\right|
					\left|(X_t - X_s)^2 - (\inprod<X>_t - \inprod<X>_s)\right| < \infty
				\end{align}
				であれば
				\begin{align}
					&E \left[\left\{(X_v - X_u)^2 - (\inprod<X>_v - \inprod<X>_u)\right\}
						\left\{(X_t - X_s)^2 - (\inprod<X>_t - \inprod<X>_s)\right\}\right] \\
					&\quad= E \left[\cexp{\left\{(X_v - X_u)^2 - (\inprod<X>_v - \inprod<X>_u)\right\}
						\left\{(X_t - X_s)^2 - (\inprod<X>_t - \inprod<X>_s)\right\}}{\mathscr{F}_s} \right] \\
					&\quad= E \left[\left\{(X_v - X_u)^2 - (\inprod<X>_v - \inprod<X>_u)\right\}
						\cexp{\left\{(X_t - X_s)^2 - (\inprod<X>_t - \inprod<X>_s)\right\}}{\mathscr{F}_s} \right] \\
					&\quad= 0
				\end{align}
				が成立する.
				
			\item[第二段]
				$|X|$及び$\inprod<X>$のパスは全て連続であるから,Problem 2.7より
				\begin{align}
					T_n \coloneqq \inf{}{\Set{t \geq 0}{|X_t| \vee \inprod<X>_t \geq n}}
				\end{align}
				で$(\mathscr{F}_t)$-停止時刻の列$(T_n)_{n=1}^\infty$が定まる.
				このとき任意の$\omega \in \Omega$で
				\begin{align}
					\Set{t \geq 0}{|X_t(\omega)| \vee \inprod<X>_t(\omega) \geq n+1}
					\subset \Set{t \geq 0}{|X_t(\omega)| \vee \inprod<X>_t(\omega) \geq n}
				\end{align}
				となるから
				\begin{align}
					T_n \leq T_{n+1}, \quad (\forall n \geq 1)
				\end{align}
				が成立し,また任意の$K > 0$に対し$\sup{t \in [0,K]}{|X_t(\omega)| \vee \inprod<X>_t(\omega)} < N$を満たす
				$N \in \N$を取れば$T_N(\omega) > K$となり
				\begin{align}
					\lim_{n \to \infty} T_n(\omega) = \infty, \quad (\forall \omega \in \Omega)
					\label{eq:chapter_1_theorem_5_8_1}
				\end{align}
				が従う.
				
			\item[第三段]
				$X^{(n)}$を$X^{(n)}_t \coloneqq X_{t \wedge T_n},\ (\forall t \geq 0)$で定めて,
				$[0,t]$の分割$\Pi = \{t_0,t_1,\cdots,t_m\}$に対し
				\begin{align}
					V^{(2,n)}_t(\Pi) \coloneqq \sum_{k=1}^m \left|X^{(n)}_{t_k} - X^{(n)}_{t_{k-1}}\right|^2
				\end{align}
				とおけば,$\{t \leq T_n\}$の上で$X_t = X^{(n)}_t$となるから
				\begin{align}
					V^{(2,n)}_t(\Pi)(\omega) = V^{(2)}_t(\Pi)(\omega),
					\quad \left(\forall \omega \in \{t \leq T_n\}\right)
					\label{eq:chapter_1_theorem_5_8_2}
				\end{align}
				が成り立つ.$\left|X_{t \wedge T_n}\right| \vee \inprod<X>_{t \wedge T_n} \leq n$であるから,
				Lemma 5.10 と第一段の結果及び$\inprod<X^{(n)}>$の連続性により
				\begin{align}
					E \left| V_t^{(2,n)}(\Pi) - \inprod<X^{(n)}>_t \right|^2
					&= E \left[ \sum_{k=1}^m \left\{\left|X^{(n)}_{t_k} - X^{(n)}_{t_{k-1}}\right|^2 
						- \left(\inprod<X^{(n)}>_{t_k} - \inprod<X^{(n)}>_{t_{k-1}}\right)\right\} \right]^2 \\
					&= E \sum_{k=1}^m \left\{ \left|X^{(n)}_{t_k} - X^{(n)}_{t_{k-1}}\right|^2 
						- \left(\inprod<X^{(n)}>_{t_k} - \inprod<X^{(n)}>_{t_{k-1}}\right) \right\}^2 \\
					&\leq 2 E \sum_{k=1}^m \left|X^{(n)}_{t_k} - X^{(n)}_{t_{k-1}}\right|^4
						+ 2 E \left(\inprod<X^{(n)}>_{t_k} - \inprod<X^{(n)}>_{t_{k-1}}\right)^2 \\
					&\leq 2 E V_t^{(4,n)}(\Pi) 
						+ 2 n E \left[ m_t\left(\inprod<X^{(n)}>;\Norm{\Pi}{}\right) \right] \\
					&\longrightarrow 0,\quad (\Norm{\Pi}{} \longrightarrow 0)
				\end{align}
				が得られる.
			
			\item[第四段]
				任意に$\epsilon > 0$と$\eta > 0$を取る.(\refeq{eq:chapter_1_theorem_5_8_2})より任意の$n \geq 1$で
				\begin{align}
					P\left(\left|V_t^{(2)}(\Pi) - \inprod<X>_t\right| > \epsilon\right)
					&= P\left(\left\{\left|V_t^{(2)}(\Pi) - \inprod<X>_t\right| > \epsilon\right\} \cap \{t > T_n\}\right)
						+ P\left(\left\{\left|V_t^{(2,n)}(\Pi) - \inprod<X^{(n)}>_t\right| > \epsilon\right\} \cap \{t \leq T_n\}\right) \\
					&\leq P(t > T_n) + P\left(\left|V_t^{(2,n)}(\Pi) - \inprod<X^{(n)}>_t\right| > \epsilon\right)
				\end{align}
				が成立し,このとき(\refeq{eq:chapter_1_theorem_5_8_1})より或る$N \geq 1$が存在して
				\begin{align}
					P(t > T_n) < \frac{\eta}{2},\quad (\forall n \geq N)
				\end{align}
				となり,前段の結果より或る$\delta > 0$が存在して$\Norm{\Pi}{} < \delta$なら
				\begin{align}
					P\left(\left|V_t^{(2,N)}(\Pi) - \inprod<X^{(N)}>_t\right| > \epsilon\right)
					\leq \frac{1}{\epsilon} E\left|V_t^{(2,N)}(\Pi) - \inprod<X^{(N)}>_t\right|
					\leq \frac{1}{\epsilon} \left\{E\left|V_t^{(2,N)}(\Pi) - \inprod<X^{(N)}>_t\right|^2 \right\}^{1/2}
					< \frac{\eta}{2}
				\end{align}
				が満たされるから
				\begin{align}
					\Norm{\Pi}{} < \delta
					\quad \Longrightarrow \quad 
					P\left(\left|V_t^{(2)}(\Pi) - \inprod<X>_t\right| > \epsilon\right) < \eta
				\end{align}
				が従う.
				\QED
		\end{description}
	\end{prf}
	
	\begin{itembox}[l]{Theorem 5.13 修正}
		Let $X = \Set{X_t,\mathscr{F}_t}{0 \leq t < \infty}$ and $Y = \Set{Y_t,\mathscr{F}_t}{0 \leq t < \infty}$
		be members of $\mathscr{M}_2^c$. \textcolor{red}{There is a unique $[A]_{\mathscr{A}_c}$ such that
		$\Set{X_t Y_t - \tilde{A}_t,\mathscr{F}_t}{0 \leq t < \infty}$ is a continuous martingale
		for every $\tilde{A} \in [A]_{\mathscr{A}_c}$.}
	\end{itembox}
	
	\begin{prf}
		定義より$\inprod<X,Y> \in \mathscr{A}_c$に対して$XY - \inprod<X,Y>$
		は連続マルチンゲールである.また$\inprod<X,Y>$と区別不能な$A \in \mathscr{A}_c$を取れば,任意の$t \geq 0$で
		\begin{align}
			P(X_t Y_t - \inprod<X,Y>_t = X_t Y_t - A_t) = 1
		\end{align}
		となるから$XY-A$もまた連続マルチンゲールとなる.
		$A,B \in \mathscr{A}_c$に対し$XY - A,\ XY-B$が共にマルチンゲールとなるとき,
		$A - B$もマルチンゲールとなり,Theorem 4.14の補題(P. \pageref{lem:uniqueness_of_Doob_Meyer_decomposition})より
		$[A]_{\mathscr{A}_c} = [B]_{\mathscr{A}_c}$が従う.
		\QED
	\end{prf}
	
	\begin{itembox}[l]{Problem 5.17}
		Let $X$, $Y$ be in $\mathscr{M}^{c,loc}$. Then there is a unique (up to indistinguishablility) adapted,
		continuous process of bounded variation $\inprod<X,Y>$ satisfying $\inprod<X,Y>_0 = 0$,
		such that $XY - \inprod<X,Y> \in \mathscr{M}^{c,loc}$. If $X = Y$, we write $\inprod<X> = \inprod<X,X>$,
		and this process is nondecreasing.
	\end{itembox}
	
	\begin{prf}
		
	\end{prf}
	
	\begin{itembox}[l]{Problem 5.19}
		\begin{description}
			\item[(i)] A local martingale of class $DL$ is a martingale.
			\item[(ii)]
			\item[(iii)]
		\end{description}
	\end{itembox}
	
	\begin{prf}\mbox{}
		\begin{description}
			\item[(i)] $X$を局所マルチンゲールとすれば,
				或る$(\mathscr{F}_t)$-停止時刻の列$(T_n)_{n=1}^\infty$と$P$-零集合$E$が存在して
				\begin{align}
					T_1(\omega) \leq T_2(\omega) \leq \cdots \longrightarrow \infty,
					\quad (\forall \omega \in \Omega \backslash E)
				\end{align}
				かつ全ての$n \geq 1$で$\Set{X_{t \wedge T_n},\mathscr{F}_t}{0 \leq t < \infty}$はマルチンゲールとなる.
				任意に$t \geq 0$を取れば$\left\{t \wedge T_n\right\}_{n=1}^\infty \subset \mathscr{S}_t$となり,
				$X$はクラス$DL$に属しているから$\left(X_{t \wedge T_n}\right)_{n=1}^\infty$は一様可積分である.
				ここで$E \in \mathscr{F}_0$かつ
				\begin{align}
					X_t(\omega) = \lim_{n \to \infty} X_{t \wedge T_n}(\omega),
					\quad (\forall \omega \in \Omega \backslash E)
				\end{align}
				が成り立つから$X_t$は$\mathscr{F}_t/\borel{\R}$-可測であり,
				また定理\ref{lem:uniformly_integrable_and_convergence_in_mean}より
				$X_t$の可積分性及び
				\begin{align}
					E\left| X_t - X_{t \wedge T_n}(\omega) \right| \longrightarrow 0
					\quad (n \longrightarrow \infty)
				\end{align}
				が従う.よって任意に$0 \leq s < t < \infty$を取れば
				\begin{align}
					\int_A X_t\ dP = \lim_{n \to \infty} \int_A X_{t \wedge T_n}\ dP
					= \lim_{n \to \infty} \int_A X_{s \wedge T_n}\ dP
					= \int_A X_s\ dP,
					\quad (\forall A \in \mathscr{F}_s)
				\end{align}
				が満たされ,$\Set{X_t,\mathscr{F}_t}{0 \leq t < \infty}$のマルチンゲール性が得られる.
		\end{description}
	\end{prf}
	
	\begin{itembox}[l]{Definition 5.22 修正}
		\textcolor{red}{$\mathscr{M}_2$ and $\mathscr{M}_2^c$ are vector spaces, 
		where the additions and scalar multiplications are defined by
		\begin{align}
			(X+Y)_t(\omega) \coloneqq X_t(\omega) + Y_t(\omega),
			\quad (\alpha X)_t(\omega) \coloneqq \alpha X_t(\omega),
			\quad (\forall X,Y \in \mathscr{M}_2\ \mbox{(resp. $\mathscr{M}_2^c$)},\ \forall \alpha \in \R).
		\end{align}
		Let denote the quotient space of $\mathscr{M}_2$ and $\mathscr{M}_2^c$ with respect to
		the equivalent relation as in (\refeq{eq:equivalence_with_respect_to_path})
		(P. \pageref{eq:equivalence_with_respect_to_path}) by $\mathfrak{M}_2$ and $\mathfrak{M}_2^c$,
		and denote the elements of each space by $[X]_{\mathfrak{M}_2}$ and  $[X]_{\mathfrak{M}_2^c}$.
		For any $[X]_{\mathfrak{M}_2},[Y]_{\mathfrak{M}_2} \in \mathfrak{M}_2,\ 
		\mbox{(resp. $[X]_{\mathfrak{M}_2^c},[Y]_{\mathfrak{M}_2^c} \in \mathfrak{M}_2^c$)}$
		and $0 \leq t < \infty$, we define a distance by
		\begin{align}
			d\left([X]_{\mathfrak{M}_2},[Y]_{\mathfrak{M}_2}\right)
			&\coloneqq \sum_{n=1}^\infty 2^{-n}\left(\Norm{[X_n] - [Y_n]}{L^2(P)} \wedge 1\right), \\
			d_c\left([X]_{\mathfrak{M}_2^c},[Y]_{\mathfrak{M}_2^c}\right)
			&\coloneqq \sum_{n=1}^\infty 2^{-n}\left(\Norm{[X_n] - [Y_n]}{L^2(P)} \wedge 1\right),
		\end{align}
		where $\Norm{\cdot}{L^2(P)}$ denotes the $L^2$ norm on $L^2(P) = L^2(\Omega,\mathscr{F},P)$.}
	\end{itembox}
	
	\begin{itembox}[l]{Proposition 5.23 修正}
		\begin{description}
			\item[(1)]
				Suppose that the filtration $\{\mathscr{F}_t\}$ satisfies the usual conditions.
				Then $\mathfrak{M}_2$ is a complete metric space under the preceding metric $d$.
			\item[(2)]
				Suppose that for every $t \in [0,\infty)$,
				$\mathscr{F}_t$ contains all the $P$-negligible events in $\mathscr{F}$.
				Then $\mathfrak{M}_2^c$ is a complete metric space under the preceding metric $d_c$.
		\end{description}
	\end{itembox}
	
	\begin{prf} 任意の$0 \leq t < \infty$に対し,$L^2(\Omega,\mathscr{F}_t,P)$における関数類を$[\cdot]_t$と書く.
		\begin{description}
			\item[(1)] $\left([X^{(k)}]_{\mathfrak{M}_2}\right)_{k=1}^\infty$をCauchy列とすれば,
				$|X^{(k)} - X^{(j)}|^2$の劣マルチンゲール性より任意の$0 \leq t \leq n$で
				\begin{align}
					\Norm{[x^{(k)}_t]_t - [X^{(j)}_t]_t}{L^2(\Omega,\mathscr{F}_t,P)} \wedge 1
					&\leq \Norm{[x^{(k)}_n] - [X^{(j)}_n]}{L^2(P)} \wedge 1 \\
					&\leq 2^n d\left( [X^{(k)}]_{\mathfrak{M}_2}, [X^{(j)}]_{\mathfrak{M}_2}\right)
					\longrightarrow 0,
					\quad (k,j \longrightarrow \infty)
				\end{align}
				となるから,定理\ref{thm:Lp_banach}より或る$[X_t]_t \in L^2(\Omega,\mathscr{F}_t,P)$が存在して
				\begin{align}
					E \left|X^{(k)}_t - X_t\right|^2 \longrightarrow 0,
					\quad (k \longrightarrow \infty)
				\end{align}
				を満たす.特に$t = 0$なら$X_t = 0,\ \mbox{a.s. $P$}$が従う.
				H\Ddot{o}lderの不等式より任意の$A \in \mathscr{F}_t$で
				\begin{align}
					\int_A \left|X^{(k)}_t - X_t\right|\ dP
					\leq \left( E \left|X^{(k)}_t - X_t\right|^2 \right)^{1/2}
					\longrightarrow 0,
					\quad (k \longrightarrow \infty)
				\end{align}
				が成り立つから,任意に$0 \leq s < t$を取れば
				\begin{align}
					\int_A X_s\ dP
					= \lim_{k \to \infty} \int_A X^{(k)}_s\ dP
					= \lim_{k \to \infty} \int_A X^{(k)}_t\ dP
					= \int_A X_t\ dP,
					\quad (\forall A \in \mathscr{F}_s)
				\end{align}
				となり$X = \Set{X_t,\mathscr{F}_t}{0 \leq t < \infty}$のマルチンゲール性が出る.
				Theorem 3.13より$X$の$RCLL$な修正$\tilde{X} \in \mathscr{M}_2$が得られ,ここで任意に$\epsilon > 0$及び
				$1/2^N < \epsilon/2$を満たす$N$を取れば,或る$K \geq 1$が存在して
				\begin{align}
					\Norm{[X^{(k)}_n] - [\tilde{X}_n]}{L^2(P)} < \frac{\epsilon}{2},
					\quad (\forall k \geq K)
				\end{align}
				がすべての$n \leq N$で満たされるから
				\begin{align}
					d\left([X^{(k)}]_{\mathfrak{M}_2}, [\tilde{X}]_{\mathfrak{M}_2}\right) < \epsilon,
					\quad (\forall k \geq K)
				\end{align}
				が従う.
		\end{description}
	\end{prf}
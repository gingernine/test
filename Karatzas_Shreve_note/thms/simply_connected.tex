\subsection{単連結}
	ホモトピーと単連結開集合上のCauchyの積分定理について.
	
	$\gamma$と$\eta$を$[0,1]$上の路とし,
	\begin{align}
		0 \notin \ran{\gamma},\quad 0 \notin \ran{\eta}
	\end{align}
	であるとする.また
	\begin{align}
		\begin{gathered}
		H:[0,1] \times [0,1] \longrightarrow \C \backslash \{0\}, \\
		\forall s \in [0,1]\, H(s,0) = \gamma(s), \\
		\forall s \in [0,1]\, H(s,1) = \eta(s), \\
		\forall t \in [0,1]\, H(1,t) = \gamma(1) = \eta(1), \\
		\forall t \in [0,1]\, H(0,t) = \gamma(0) = \eta(0)
		\end{gathered}
	\end{align}
	を満たす$H$が取れるとする.このとき
	\begin{align}
		\Wnd_{\gamma}(0) = \Wnd_{\eta}(0)
	\end{align}
	が成り立つ.実際,
	\begin{align}
		\exp \circ F = H
	\end{align}
	を満たす$[0,1] \times [0,1]$上の$\C$値連続写像$F$が取れて,特に
	\begin{align}
		\forall s \in [0,1]\, \gamma(s) = e^{F(s,0)}
	\end{align}
	を満たす.すなわち
	\begin{align}
		\forall s \in [0,1]\, \gamma(s) = |\gamma(s)| \cdot e^{\isym \cdot \Im{F(s,0)}}
	\end{align}
	が成り立つので
	\begin{align}
		\Ind_{\gamma}(0)
		= \frac{\pvlog{|\gamma(1)|} - \pvlog{|\gamma(0)|}}{2\cdot\pi\cdot\isym}
		+ \frac{\Im{F(1,0)} - \Im{F(0,0)}}{2\cdot\pi}
	\end{align}
	が成り立つ.同様に
	\begin{align}
		\Ind_{\eta}(0)
		= \frac{\pvlog{|\eta(1)|} - \pvlog{|\eta(0)|}}{2\cdot\pi\cdot\isym}
		+ \frac{\Im{F(1,1)} - \Im{F(0,1)}}{2\cdot\pi}
	\end{align}
	が成り立つ.ところで
	\begin{align}
		e^{F(0,t)} = H(0,t) = H(0,0) = e^{F(0,0)}
	\end{align}
	なので
	\begin{align}
		t \longmapsto \frac{\Im{F(0,t)} - \Im{F(0,0)}}{2\cdot\pi}
	\end{align}
	は連続であり整数値であり,特に$t=0$で$0$なのだから
	\begin{align}
		\Im{F(0,1)} = \Im{F(0,0)}
	\end{align}
	が成り立つ.同様に
	\begin{align}
		\Im{F(1,1)} = \Im{F(1,0)}
	\end{align}
	が成り立つ.従って
	\begin{align}
		\Wnd_{\gamma}(0) = \frac{\Im{F(1,0)} - \Im{F(0,0)}}{2\cdot\pi}
		= \frac{\Im{F(1,1)} - \Im{F(0,1)}}{2\cdot\pi}
		= \Wnd_{\eta}(0)
	\end{align}
	が得られる.
	
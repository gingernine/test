\subsection{三角関数}
	\begin{screen}
		\begin{dfn}[三角関数]
			複素数$z$に対して
			\begin{align}
				\frac{e^{\isym \cdot z} + e^{-\isym \cdot z}}{2}
			\end{align}
			を対応させる$\C$から$\C$への写像を{\bf 余弦}\index{よげん@余弦}{\bf (cosine)}と呼び,
			\begin{align}
				\cos
			\end{align}
			と書く.複素数$z$に対して
			\begin{align}
				\frac{e^{\isym \cdot z} - e^{-\isym \cdot z}}{2 \cdot \isym}
			\end{align}
			を対応させる$\C$から$\C$への写像を{\bf 正弦}\index{せいげん@正弦}{\bf (sine)}と呼び,
			\begin{align}
				\sin
			\end{align}
			と書く.
		\end{dfn}
	\end{screen}
	
	余弦関数の二乗は
	\begin{align}
		(\cos{z})^2
	\end{align}
	ではなく
	\begin{align}
		\cos^2{z}
	\end{align}
	と書く.同様に正弦関数の二乗も
	\begin{align}
		\sin^2{z}
	\end{align}
	と書く.
	
	\begin{screen}
		\begin{thm}[余弦と正弦の二乗和は$1$]
			$z$を複素数とするとき
			\begin{align}
				\cos^2{z} + \sin^2{z} = 1.
			\end{align}
		\end{thm}
	\end{screen}
	
	\begin{sketch}
		$z$を複素数とする.余弦の定義より
		\begin{align}
			\cos^2{z} = \frac{e^{2 \cdot \isym \cdot z} + 2 + e^{-2 \cdot \isym \cdot z}}{4}
		\end{align}
		が成り立ち,正弦の定義より
		\begin{align}
			\sin^2{z} = -\frac{e^{2 \cdot \isym \cdot z} - 2 + e^{-2 \cdot \isym \cdot z}}{4}
		\end{align}
		が成り立つので,
		\begin{align}
			\cos^2{z} + \sin^2{z} = 1
		\end{align}
		が得られる.
		\QED
	\end{sketch}
	
	\begin{screen}
		\begin{thm}[正弦の導関数は余弦,余弦の導関数はマイナス正弦]
		\label{thm:derivatives_of_trigonometric_functions}
		\end{thm}
	\end{screen}
	
	$z$を複素数とすれば
	\begin{align}
		e^{\isym \cdot z} = \cos{z} + \isym \cdot \sin{z}
	\end{align}
	が成立するが,この関係を{\bf Eulerの関係式}と呼ぶ.
	
	$\cos$と$\sin$も$\exp$と同様に実数に対しては実数を対応させる写像である.
	\begin{align}
		\cos{0} = 1
	\end{align}
	かつ
	\begin{align}
		\cos{2} < 0
	\end{align}
	であり,かつ
	\begin{align}
		\R \ni t \longmapsto \cos{t}
	\end{align}
	は連続写像であるから,中間値の定理より
	\begin{align}
		\cos{t} = 0
	\end{align}
	を満たす実数$t$が取れる.ゆえに
	\begin{align}
		\Set{t \in \R_+}{\cos{t} = 0}
	\end{align}
	は空ではないので,$\R$においてその下限が存在する.
	
	\begin{screen}
		\begin{dfn}[円周率]
			\begin{align}
				\pi \defeq 2 \cdot \inf{}{\Set{t \in \R_+}{\cos{t} = 0}}
			\end{align}
			により定める実数$\pi$を{\bf 円周率}\index{えんしゅうりつ@円周率}{\bf (pi)}と呼ぶ.
		\end{dfn}
	\end{screen}
	
	円周率の定め方と$\cos$の連続性から
	\begin{align}
		\cos{\frac{\pi}{2}} = 0
	\end{align}
	が成り立ち,また
	\begin{align}
		0 < t < \frac{\pi}{2}
	\end{align}
	を満たす実数$t$に対しては
	\begin{align}
		0 < \cos{t}
	\end{align}
	が成立する.他方でEulerの関係式から
	\begin{align}
		\sin^2{\frac{\pi}{2}} = 1
	\end{align}
	が従う.また平均値の定理より
	\begin{align}
		0 < \xi < \frac{\pi}{2}
	\end{align}
	かつ
	\begin{align}
		\sin{\frac{\pi}{2}}
		= \frac{2}{\pi} \cdot \sin'{\xi}
	\end{align}
	を満たす実数$\xi$が取れて,定理\ref{thm:derivatives_of_trigonometric_functions}から
	\begin{align}
		\sin{\frac{\pi}{2}}
		= \frac{2}{\pi} \cdot \cos{\xi}
	\end{align}
	が成り立つが,$\cos{\xi}$は正であるから
	\begin{align}
		\sin{\frac{\pi}{2}} = 1
	\end{align}
	である.ゆえに
	\begin{align}
		e^{\frac{\pi}{2} \cdot \isym} = \isym
	\end{align}
	が成立する.すなわち
	\begin{align}
		e^{\pi \cdot \isym}
		= e^{\frac{\pi}{2} \cdot \isym} \cdot e^{\frac{\pi}{2} \cdot \isym}
		= -1
	\end{align}
	である.すなわち
	\begin{align}
		e^{2 \cdot \pi \cdot \isym}
		= e^{\pi \cdot \isym} \cdot e^{\pi \cdot \isym}
		= 1
	\end{align}
	である.そして$n$を任意に与えられた正数とすれば
	\begin{align}
		e^{2 \cdot n \cdot \pi \cdot \isym} = 1
	\end{align}
	が成り立つが,これは当然のようであるけれども,
	整数の累乗について次の定理を載せておく.
	
	\begin{screen}
		\begin{thm}[指数関数の整数乗]
		\label{thm:integer_exponentiation_of_exponential_function}
			$z$を複素数とし,$n$を整数とすると,
			\begin{align}
				e^{n \cdot z} = (e^z)^n.
			\end{align}
		\end{thm}
	\end{screen}
	
	\begin{sketch}
		$z$を複素数とする.まず
		\begin{align}
			e^{0 \cdot z} = e^0 = 1
		\end{align}
		かつ
		\begin{align}
			(e^z)^0 = 1
		\end{align}
		であるから
		\begin{align}
			e^{0 \cdot z} = (e^z)^0
		\end{align}
		が成立する.また$n$を自然数として
		\begin{align}
			e^{n \cdot z} = (e^z)^n
		\end{align}
		が成り立っているとすると,
		\begin{align}
			e^{(n+1) \cdot z}
			= e^{n \cdot z} \cdot e^z
			= (e^z)^n \cdot e^z
			= (e^z)^{n+1}
		\end{align}
		が従う.ゆえに,数学的帰納法の原理より任意の自然数$n$で
		\begin{align}
			e^{n \cdot z} = (e^z)^n
		\end{align}
		が成立する.次に$n$を負の整数とすると,
		\begin{align}
			-n \in \Natural
		\end{align}
		であるから
		\begin{align}
			e^{(-n) \cdot z} = (e^z)^{-n}
			\label{fom:thm_integer_exponentiation_of_exponential_function}
		\end{align}
		が成立する.ところで定理\ref{thm:inversions_of_sum_and_product}より
		\begin{align}
			(-n) \cdot z = -(n \cdot z)
		\end{align}
		が成り立つので,定理\ref{thm:inversion_of_exp_z_is_exp_minus_z}より
		\begin{align}
			e^{(-n) \cdot z} = e^{-(n \cdot z)} = (e^{n \cdot z})^{-1}
		\end{align}
		が成り立つ.一方で整数乗の定め方より
		\begin{align}
			(e^z)^{-n} = ((e^z)^n)^{-1}
		\end{align}
		も成り立つので,(\refeq{fom:thm_integer_exponentiation_of_exponential_function})
		と併せて
		\begin{align}
			(e^{n \cdot z})^{-1} = ((e^z)^n)^{-1}
		\end{align}
		が成立し
		\begin{align}
			e^{n \cdot z} = (e^z)^n
		\end{align}
		が従う.
		\QED
	\end{sketch}
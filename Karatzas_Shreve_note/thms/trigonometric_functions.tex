\subsection{三角関数}
	\begin{screen}
		\begin{dfn}[三角関数]
			複素数$z$に対して
			\begin{align}
				\frac{e^{\isym \cdot z} + e^{-\isym \cdot z}}{2}
			\end{align}
			を対応させる$\C$から$\C$への写像を{\bf 余弦}\index{よげん@余弦}{\bf (cosine)}と呼び,
			\begin{align}
				\cos
			\end{align}
			と書く.複素数$z$に対して
			\begin{align}
				\frac{e^{\isym \cdot z} - e^{-\isym \cdot z}}{2 \cdot \isym}
			\end{align}
			を対応させる$\C$から$\C$への写像を{\bf 正弦}\index{せいげん@正弦}{\bf (sine)}と呼び,
			\begin{align}
				\sin
			\end{align}
			と書く.
		\end{dfn}
	\end{screen}
	
	$z$を複素数とすれば
	\begin{align}
		e^{\isym \cdot z} = \cos{z} + \isym \cdot \sin{z}
	\end{align}
	が成立するが,この等式を{\bf Eulerの関係式}と呼ぶ.
	
	$\cos{z}$の二乗は
	\begin{align}
		\cos^2{z}
	\end{align}
	と書く.同様に$\sin{z}$の二乗も
	\begin{align}
		\sin^2{z}
	\end{align}
	と書く.
	
	\begin{screen}
		\begin{thm}[余弦と正弦の二乗和は$1$]
			$z$を複素数とするとき
			\begin{align}
				\cos^2{z} + \sin^2{z} = 1.
			\end{align}
		\end{thm}
	\end{screen}
	
	\begin{sketch}
		$z$を複素数とする.余弦の定義より
		\begin{align}
			\cos^2{z} = \frac{e^{2 \cdot \isym \cdot z} + 2 + e^{-2 \cdot \isym \cdot z}}{4}
		\end{align}
		が成り立ち,正弦の定義より
		\begin{align}
			\sin^2{z} = -\frac{e^{2 \cdot \isym \cdot z} - 2 + e^{-2 \cdot \isym \cdot z}}{4}
		\end{align}
		が成り立つので,
		\begin{align}
			\cos^2{z} + \sin^2{z} = 1
		\end{align}
		が得られる.
		\QED
	\end{sketch}
	
	\begin{screen}
		\begin{thm}[正弦の導関数は余弦,余弦の導関数はマイナス正弦]
		\label{thm:derivatives_of_trigonometric_functions}
			$z$を任意に与えられた複素数とすると,$\cos$も$\sin$も$z$で微分可能であって,
			\begin{align}
				\cos'{z} = - \sin{z}
			\end{align}
			かつ
			\begin{align}
				\sin'{z} = \cos{z}
			\end{align}
			が成り立つ.
		\end{thm}
	\end{screen}
	
	\begin{sketch}
		微分の線型性と連鎖律より
		\begin{align}
			\cos'{z}
			&= \frac{\isym \cdot e^{\isym \cdot z} + (-\isym) \cdot e^{-\isym \cdot z}}{2} \\
			&= \isym \cdot \frac{e^{\isym \cdot z} - e^{-\isym \cdot z}}{2} \\
			&= -\frac{e^{\isym \cdot z} - e^{-\isym \cdot z}}{2 \cdot \isym} \\
			&= -\sin{z}
		\end{align}
		が成り立つ.同様に
		\begin{align}
			\sin'{z}
			&= \frac{\isym \cdot e^{\isym \cdot z} - (-\isym) \cdot e^{-\isym \cdot z}}{2 \cdot \isym} \\
			&= \isym \cdot \frac{e^{\isym \cdot z} + e^{-\isym \cdot z}}{2 \cdot \isym} \\
			&= \frac{e^{\isym \cdot z} + e^{-\isym \cdot z}}{2} \\
			&= \cos{z}
		\end{align}
		も成り立つ.
		\QED
	\end{sketch}
	
	$\cos$と$\sin$も$\exp$と同様に実数に対しては実数を対応させる写像である.
	\begin{screen}
		\begin{thm}[余弦は実数に対して実数を対応させる]
			$t$を実数とすると
			\begin{align}
				\cos{t} = \sum_{n=0}^\infty \frac{(-1)^{n}}{(2 \cdot n)!} \cdot t^{2 \cdot n}
			\end{align}
			が成立する.
		\end{thm}
	\end{screen}
	
	この定理の主張は級数の和($\cos$の分母)を取れば得られる.$\exp$は絶対収束級数で表せているので級数の和は各項の和の総和に一致する.
	
	いま$t$を正の実数とすると,平均値の定理より
	\begin{align}
		0 < \delta < t
	\end{align}
	かつ
	\begin{align}
		\sin{t} = t \cdot \cos{\delta}
	\end{align}
	を満たす実数$\delta$が取れる.ところで
	\begin{align}
		\cos^2{\delta} + \sin^2{\delta} = 1
	\end{align}
	より
	\begin{align}
		-1 \leq \cos{\delta} \leq 1
	\end{align}
	が成り立つから
	\begin{align}
		\sin{t} \leq t
	\end{align}
	が従う.これで
	\begin{align}
		\forall t\, \left(\, t \in \R_{+} \Longrightarrow \sin{t} \leq t\, \right)
		\label{fom:existence_of_pi_1}
	\end{align}
	を得た.再び$t$を正の実数とすると,平均値の定理より
	\begin{align}
		0 < \eta < t
	\end{align}
	かつ
	\begin{align}
		\cos{t} + \frac{t^2}{2} - 1 = t \cdot (-\sin{\eta} + \eta)
	\end{align}
	を満たす実数$\eta$が取れて,(\refeq{fom:existence_of_pi_1})より
	\begin{align}
		0 \leq -\sin{\eta} + \eta
	\end{align}
	が満たされているので
	\begin{align}
		1 - \frac{t^2}{2} \leq \cos{t}
	\end{align}
	が従う.これで
	\begin{align}
		\forall t\, \left(\, t \in \R_{+} 
		\Longrightarrow 1 - \frac{t^2}{2} \leq \cos{t}\, \right)
		\label{fom:existence_of_pi_2}
	\end{align}
	を得た.再び$t$を正の実数とすると,平均値の定理より
	\begin{align}
		0 < \theta < t
	\end{align}
	かつ
	\begin{align}
		\sin{t} - t + \frac{t^3}{3!} = t \cdot \left(\cos{\theta} - 1 + \frac{\theta^2}{2}\right)
	\end{align}
	を満たす実数$\theta$が取れて,(\refeq{fom:existence_of_pi_2})より
	\begin{align}
		0 \leq \cos{\theta} - 1 + \frac{\theta^2}{2}
	\end{align}
	が満たされているので
	\begin{align}
		t - \frac{t^3}{3!} \leq \sin{t}
	\end{align}
	が従う.これで
	\begin{align}
		\forall t\, \left(\, t \in \R_{+} 
		\Longrightarrow t - \frac{t^3}{3!} \leq \sin{t}\, \right)
		\label{fom:existence_of_pi_3}
	\end{align}
	を得た.再び$t$を正の実数とすると,平均値の定理より
	\begin{align}
		0 < \xi < t
	\end{align}
	かつ
	\begin{align}
		\cos{t} + \frac{t^2}{2} - \frac{t^4}{4!} - 1 = t \cdot \left(-\sin{\xi} + \xi - \frac{\xi^3}{3!}\right)
	\end{align}
	を満たす実数$\xi$が取れて,(\refeq{fom:existence_of_pi_3})より
	\begin{align}
		-\sin{\xi} + \xi - \frac{\xi^3}{3!} \leq 0
	\end{align}
	が満たされているので
	\begin{align}
		\cos{t} \leq 1 - \frac{t^2}{2} + \frac{t^4}{4!}
	\end{align}
	が従う.ゆえに
	\begin{align}
		\cos{2} \leq 1 - \frac{2^2}{2} + \frac{2^4}{4!} = -\frac{1}{3} < 0
	\end{align}
	が成立する.また
	\begin{align}
		\R \ni t \longmapsto \cos{t}
	\end{align}
	は実連続写像であるから,中間値の定理より
	\begin{align}
		0 < t < 2 \wedge \cos{t} = 0
	\end{align}
	を満たす実数$t$が取れる.ゆえに
	\begin{align}
		\Set{t \in \R_+}{\cos{t} = 0}
	\end{align}
	は空ではないので,$\R$においてその下限が存在する.
	
	\begin{screen}
		\begin{dfn}[円周率]
			\begin{align}
				\pi \defeq 2 \cdot \inf{}{\Set{t \in \R_+}{\cos{t} = 0}}
			\end{align}
			により定める実数$\pi$を{\bf 円周率}\index{えんしゅうりつ@円周率}{\bf (pi)}と呼ぶ.
		\end{dfn}
	\end{screen}
	
	$\cos$の連続性から
	\begin{align}
		\cos{\frac{\pi}{2}} = 0
	\end{align}
	が成り立ち,また
	\begin{align}
		0 < t < \frac{\pi}{2}
	\end{align}
	を満たす実数$t$に対しては
	\begin{align}
		0 < \cos{t}
	\end{align}
	が成立する.他方でEulerの関係式から
	\begin{align}
		\sin^2{\frac{\pi}{2}} = 1
	\end{align}
	が従う.また平均値の定理より
	\begin{align}
		0 < \xi < \frac{\pi}{2}
	\end{align}
	かつ
	\begin{align}
		\sin{\frac{\pi}{2}}
		= \frac{2}{\pi} \cdot \sin'{\xi}
	\end{align}
	を満たす実数$\xi$が取れて,定理\ref{thm:derivatives_of_trigonometric_functions}から
	\begin{align}
		\sin{\frac{\pi}{2}}
		= \frac{2}{\pi} \cdot \cos{\xi}
	\end{align}
	が成り立つが,$\cos{\xi}$は正であるから
	\begin{align}
		\sin{\frac{\pi}{2}} = 1
	\end{align}
	である.ゆえに
	\begin{align}
		e^{\frac{\pi}{2} \cdot \isym} = \isym
	\end{align}
	が成立する.すなわち
	\begin{align}
		e^{\pi \cdot \isym}
		= e^{\frac{\pi}{2} \cdot \isym} \cdot e^{\frac{\pi}{2} \cdot \isym}
		= -1
	\end{align}
	である.すなわち
	\begin{align}
		e^{2 \cdot \pi \cdot \isym}
		= e^{\pi \cdot \isym} \cdot e^{\pi \cdot \isym}
		= 1
	\end{align}
	である.そして$n$を任意に与えられた正数とすれば
	\begin{align}
		e^{2 \cdot n \cdot \pi \cdot \isym} = 1
	\end{align}
	が成り立つが,これは当然のようであるけれども,
	整数の累乗について次の定理を載せておく.
	
	\begin{screen}
		\begin{thm}[指数関数の整数乗]
		\label{thm:integer_exponentiation_of_exponential_function}
			$z$を複素数とし,$n$を整数とすると,
			\begin{align}
				e^{n \cdot z} = (e^z)^n.
			\end{align}
		\end{thm}
	\end{screen}
	
	\begin{sketch}
		$z$を複素数とする.まず
		\begin{align}
			e^{0 \cdot z} = e^0 = 1
		\end{align}
		かつ
		\begin{align}
			(e^z)^0 = 1
		\end{align}
		であるから
		\begin{align}
			e^{0 \cdot z} = (e^z)^0
		\end{align}
		が成立する.また$n$を自然数として
		\begin{align}
			e^{n \cdot z} = (e^z)^n
		\end{align}
		が成り立っているとすると,
		\begin{align}
			e^{(n+1) \cdot z}
			= e^{n \cdot z} \cdot e^z
			= (e^z)^n \cdot e^z
			= (e^z)^{n+1}
		\end{align}
		が従う.ゆえに,数学的帰納法の原理より任意の自然数$n$で
		\begin{align}
			e^{n \cdot z} = (e^z)^n
		\end{align}
		が成立する.次に$n$を負の整数とすると,
		\begin{align}
			-n \in \Natural
		\end{align}
		であるから
		\begin{align}
			e^{(-n) \cdot z} = (e^z)^{-n}
			\label{fom:thm_integer_exponentiation_of_exponential_function}
		\end{align}
		が成立する.ところで定理\ref{thm:inverse_of_product}より
		\begin{align}
			(-n) \cdot z = -(n \cdot z)
		\end{align}
		が成り立つので,定理\ref{thm:inversion_of_exp_z_is_exp_minus_z}より
		\begin{align}
			e^{(-n) \cdot z} = e^{-(n \cdot z)} = (e^{n \cdot z})^{-1}
		\end{align}
		が成り立つ.一方で整数乗の定め方より
		\begin{align}
			(e^z)^{-n} = ((e^z)^n)^{-1}
		\end{align}
		も成り立つので,(\refeq{fom:thm_integer_exponentiation_of_exponential_function})
		と併せて
		\begin{align}
			(e^{n \cdot z})^{-1} = ((e^z)^n)^{-1}
		\end{align}
		が成立し
		\begin{align}
			e^{n \cdot z} = (e^z)^n
		\end{align}
		が従う.
		\QED
	\end{sketch}
	
	余弦と正弦の振舞いを$[0,2 \cdot \pi]$上に限って考察する.
	まずわかるのは$\sin$の動きである.$x$と$y$を
	\begin{align}
		x < y
	\end{align}
	を満たす$[0,\pi/2]$の要素とすれば,平均値の定理より
	\begin{align}
		x < \delta < y
	\end{align}
	かつ
	\begin{align}
		\frac{\sin{y} - \sin{x}}{y - x} = \cos{\delta}
	\end{align}
	を満たす実数$\delta$が取れて,
	\begin{align}
		0 < \delta < \frac{\pi}{2}
	\end{align}
	なので
	\begin{align}
		0 < \cos{\delta}
	\end{align}
	が成り立つから
	\begin{align}
		\sin{x} < \sin{y}
	\end{align}
	が成立する.つまり$\sin$は$[0,\pi/2]$上で単調に増大する.
	
	\begin{center}
	\begin{tikzpicture}
		\draw[-stealth](-0.5,0)--(3,0) node [anchor=north]{$t$};
		\draw[-stealth](0,-0.5)--(0,2) node [anchor=east]{$\sin{t}$};
		\node[anchor=north west] at (0,0) {$0$};
		\draw[thick, domain = 0:pi/2] plot(\x,{sin(\x r)});
		\draw[dashed](0,1) node [anchor=east]{$1$}--(pi/2,1)--(pi/2,0) node[anchor=north]{$\frac{\pi}{2}$};
	\end{tikzpicture}
	\end{center}
	
	続いて$\cos$の$[0,\pi/2]$上での動きもわかる.$x$と$y$を
	\begin{align}
		x < y
	\end{align}
	を満たす$[0,\pi/2]$の要素とすれば,平均値の定理より
	\begin{align}
		x < \eta < y
	\end{align}
	かつ
	\begin{align}
		\frac{\cos{y} - \cos{x}}{y - x} = -\sin{\eta}
	\end{align}
	を満たす実数$\eta$が取れて,
	\begin{align}
		0 < \eta < \frac{\pi}{2}
	\end{align}
	なので
	\begin{align}
		0 < \sin{\eta}
	\end{align}
	が成り立つから
	\begin{align}
		\cos{y} < \cos{x}
	\end{align}
	が成立する.つまり$\cos$は$[0,\pi/2]$上で単調に減少する.
	
	\begin{center}
	\begin{tikzpicture}
		\draw[-stealth](-0.5,0)--(3,0) node [anchor=north]{$t$};
		\draw[-stealth](0,-0.5)--(0,2) node [anchor=east]{$\cos{t}$};
		\node[anchor=north west] at (0,0) {$0$};
		\draw[thick, domain = 0:pi/2] plot(\x,{cos(\x r)});
		\node[anchor=east] at (0,1) {$1$};
		\node[anchor=north] at (pi/2,0) {$\frac{\pi}{2}$};
	\end{tikzpicture}
	\end{center}
	
	$[\pi/2,\pi]$上の振舞いを記述するには次の主張を仲介する.
	\begin{screen}
		\begin{thm}[余弦と正弦は$\pi/2$ずれて同じ値を取る]
			$z$を任意に与えられた複素数とするとき
			\begin{align}
				\sin{\left(z + \frac{\pi}{2}\right)} = \cos{z}.
			\end{align}
		\end{thm}
	\end{screen}
	
	\begin{sketch}
		実際,
		\begin{align}
			e^{\frac{\pi}{2} \cdot \isym} = \isym
		\end{align}
		かつ
		\begin{align}
			e^{-\frac{\pi}{2} \cdot \isym} = -\isym
		\end{align}
		であるから
		\begin{align}
			\sin{\left(z + \frac{\pi}{2}\right)}
			&= \frac{e^{\isym \cdot \left(z + \frac{\pi}{2}\right)} - e^{-\isym \cdot \left(z + \frac{\pi}{2}\right)}}{2 \cdot \isym} \\
			&= \frac{e^{\isym \cdot z} \cdot e^{\frac{\pi}{2} \cdot \isym} - e^{-\isym \cdot z} \cdot e^{-\frac{\pi}{2} \cdot \isym}}{2 \cdot \isym} \\
			&= \frac{\left(e^{\isym \cdot z} + e^{-\isym \cdot z}\right) \cdot \isym}{2 \cdot \isym} \\
			&= \frac{e^{\isym \cdot z} + e^{-\isym \cdot z}}{2} \\
			&= \cos{z}
		\end{align}
		が成り立つ.
		\QED
	\end{sketch}
	
	すなわち$[\pi/2,\pi]$上の$\sin$の動きは$[0,\pi/2]$上の$\cos$の動きに一致し,
	可視化すれば次の図の赤線を描く.
	
	\begin{center}
	\begin{tikzpicture}
		\draw[-stealth](-0.5,0)--(4,0) node [anchor=north]{$t$};
		\draw[-stealth](0,-0.5)--(0,2) node [anchor=east]{$\sin{t}$};
		\node[anchor=north west] at (0,0) {$0$};
		\draw[domain = 0:pi/2] plot(\x,{sin(\x r)});
		\draw[thick, domain = pi/2:pi, color = red] plot(\x,{sin(\x r)});
		\draw[dashed](0,1) node [anchor=east]{$1$}--(pi/2,1)--(pi/2,0) node[anchor=north]{$\frac{\pi}{2}$};
		\node[anchor=north] at (pi,0) {$\pi$};
	\end{tikzpicture}
	\end{center}
	
	$\cos$は$[\pi/2,\pi]$上で単調に減少し$-1$に達するが,
	これは先と同様に平均値の定理と$\sin$の$[\pi/2,\pi[$での正値性から把握できる.
	可視化すれば次の図の赤線を描く.
	
	\begin{center}
	\begin{tikzpicture}
		\draw[-stealth](-0.5,0)--(4,0) node [anchor=north]{$t$};
		\draw[-stealth](0,-2)--(0,2) node [anchor=east]{$\cos{t}$};
		\node[anchor=north west] at (0,0) {$0$};
		\draw[domain = 0:pi/2] plot(\x,{cos(\x r)});
		\draw[thick, domain = pi/2:pi, color = red] plot(\x,{cos(\x r)});
		\draw[dashed](0,-1) node [anchor=east]{$-1$}--(pi,-1)--(pi,0) node[anchor=south]{$\pi$};
		\node[anchor=south] at (pi/2,0) {$\frac{\pi}{2}$};
	\end{tikzpicture}
	\end{center}
	
	$[\pi,2 \cdot \pi]$上の$\sin$の動きは次の定理から判明する.
	\begin{screen}
		\begin{thm}[正弦は$\pi$ずれると正負が反転する]
			$z$を任意に与えられた複素数とするとき
			\begin{align}
				\sin{(z + \pi)} = -\sin{z}.
			\end{align}
		\end{thm}
	\end{screen}
	
	\begin{sketch}
		実際,
		\begin{align}
			e^{\pi \cdot \isym} = -1
		\end{align}
		かつ
		\begin{align}
			e^{-\pi \cdot \isym} = 1
		\end{align}
		であるから
		\begin{align}
			\sin{(z + \pi)}
			&= \frac{e^{\isym \cdot (z + \pi)} - e^{-\isym \cdot (z + \pi)}}{2 \cdot \isym} \\
			&= \frac{e^{\isym \cdot z} \cdot e^{\pi \cdot \isym} - e^{-\isym \cdot z} \cdot e^{-\pi \cdot \isym}}{2 \cdot \isym} \\
			&= \frac{-e^{\isym \cdot z} + e^{-\isym \cdot z}}{2 \cdot \isym} \\
			&= -\sin{z}
		\end{align}
		が成り立つ.
		\QED
	\end{sketch}
	
	すなわち,$[\pi,2 \cdot \pi]$上の$\sin$の動きを可視化すると次の赤線を描く.
	
	\begin{center}
	\begin{tikzpicture}
		\draw[-stealth](-0.5,0)--(8,0) node [anchor=north]{$t$};
		\draw[-stealth](0,-2)--(0,2) node [anchor=east]{$\sin{t}$};
		\node[anchor=north west] at (0,0) {$0$};
		\draw[domain = 0:pi] plot(\x,{sin(\x r)});
		\draw[thick, domain = pi:2*pi, color = red] plot(\x,{-sin((\x-pi) r)});
		\draw[dashed](0,1) node [anchor=east]{$1$}--(pi/2,1)--(pi/2,0) node[anchor=north]{$\frac{\pi}{2}$};
		\node[anchor=north west] at (pi,0) {$\pi$};
		\draw[dashed](0,-1) node [anchor=east]{$-1$}--(3*pi/2,-1)--(3*pi/2,0) node[anchor=south]{$\frac{3}{2}\cdot\pi$};
		\node[anchor=south] at (2*pi,0) {$2 \cdot \pi$};
	\end{tikzpicture}
	\end{center}
	
	一方で$[\pi,2 \cdot \pi]$上の$\cos$の動きは次の定理により判明する.
	\begin{screen}
		\begin{thm}[余弦は$\pi$を境に対称である]
			$z$を任意に与えられた複素数とするとき
			\begin{align}
				\cos{(\pi - z)} = -\cos{z} = \cos{(\pi + z)}.
			\end{align}
		\end{thm}
	\end{screen}
	
	\begin{sketch}
		まず
		\begin{align}
			\cos{(\pi + z)}
			&= \frac{e^{\isym \cdot (\pi + z)} + e^{-\isym \cdot (\pi + z)}}{2} \\
			&= \frac{e^{\isym \cdot \pi} \cdot e^{\isym \cdot z} + e^{-\isym \cdot \pi} \cdot e^{-\isym \cdot z}}{2} \\
			&= \frac{-e^{\isym \cdot z} - e^{-\isym \cdot z}}{2} \\
			&= -\cos{z}
		\end{align}
		が成り立つ.また
		\begin{align}
			\cos{(-z)} 
			&= \frac{e^{\isym \cdot (-z)} + e^{-\isym \cdot (-z)}}{2} \\
			&= \frac{e^{-\isym \cdot z} + e^{\isym \cdot z}}{2} \\
			&= \cos{z}
		\end{align}
		であるから
		\begin{align}
			\cos{(\pi - z)} = -\cos{(-z)} = -\cos{z}
		\end{align}
		も成立する.
		\QED
	\end{sketch}
	
	すなわち,$[\pi,2 \cdot \pi]$上の$\cos$の動きを可視化すると次の赤線を描く.
	
	\begin{center}
	\begin{tikzpicture}
		\draw[-stealth](-0.5,0)--(8,0) node [anchor=north]{$t$};
		\draw[-stealth](0,-2)--(0,2) node [anchor=east]{$\cos{t}$};
		\node[anchor=north west] at (0,0) {$0$};
		\draw[domain = 0:pi] plot(\x,{cos(\x r)});
		\draw[thick, domain = pi:2*pi, color = red] plot(\x,{cos((2*pi-\x) r)});
		\draw[dashed](0,-1) node [anchor=east]{$-1$}--(pi,-1)--(pi,0) node[anchor=south]{$\pi$};
		\node[anchor=south] at (pi/2,0) {$\frac{\pi}{2}$};
		\node[anchor=south] at (3*pi/2,0) {$\frac{3}{2}\cdot\pi$};
		\draw[dashed](0,1) node [anchor=east]{$1$}--(2*pi,1)--(2*pi,0) node[anchor=north]{$2\cdot\pi$};
	\end{tikzpicture}
	\end{center}
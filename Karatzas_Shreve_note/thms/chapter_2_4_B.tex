\section{Tightness}
	テキスト本文において$m^T(\omega,\delta)$は
	\begin{align}
		m^T(\omega,\delta) \coloneqq \operatorname*{max}_{\substack{|s-t| \leq \delta \\ 0 \leq s,t \leq T}}|\omega(s) - \omega(t)|
	\end{align}
	で定められるが,$\operatorname{max}$と書いて妥当であることを確認しておく.
	まず
	\begin{align}
		D \coloneqq \Set{(s,t) \in \R \times \R}{|s-t| \leq \delta \wedge 0 \leq s,t \leq T}
	\end{align}
	で定められる集合は$\R \times \R$のコンパクト集合である.そして$\omega$は連続写像であるから
	\begin{align}
		\R \times \R \ni (s,t) \longmapsto \omega(s), 
		\quad \R \times \R \ni (s,t) \longmapsto \omega(t)
	\end{align}
	は共に実連続写像である.引き算は連続,絶対値も連続であるから
	\begin{align}
		\R \times \R \ni (s,t) \longmapsto |\omega(s) - \omega(t)|
	\end{align}
	は$\R \times \R$から$\R$への連続写像であり,$D$のコンパクト性から$D$上で最大値を取る.
	
	\begin{itembox}[l]{Problem 4.8}
		Show that $m^T(\omega,\delta)$ is continuous in $\omega \in C[0,\infty)$ under the metric
		$\rho$ of (4.1), is nondecreasing in $\delta$, and 
		$\lim_{\delta \downarrow 0}m^T(\omega,\delta) = 0$ for each $\omega \in C[0,\infty)$.
	\end{itembox}
	
	\begin{sketch}\mbox{}
		\begin{description}
			\item[第一段]
				$m^T(\omega,\delta)$が$\omega$に関して連続であることを示す.まず大雑把に,
				\begin{align}
					\left|\, \operatorname*{max}_{x} |f(x)| - \operatorname*{max}_{x} |g(x)|\, \right|
					\leq \operatorname*{max}_{x} |f(x) - g(x)|
				\end{align}
				が成立する.実際,
				\begin{align}
					\operatorname*{max}_{x} |f(x)| - \operatorname*{max}_{x} |g(x)|
					\leq \operatorname*{max}_{x} |f(x) - g(x)|
				\end{align}
				が成り立つことを確認するには
				\begin{align}
					|f(x_1)| = \operatorname*{max}_{x} |f(x)|
				\end{align}
				なる$x_1$を取り,
				\begin{align}
					\operatorname*{max}_{x} |f(x)| - \operatorname*{max}_{x} |g(x)|
					&= |f(x_1)| - \operatorname*{max}_{x} |g(x)| \\
					&\leq |f(x_1)| - |g(x_1)| \\
					&\leq |f(x_1) - g(x_1)| \\
					&\leq \operatorname*{max}_{x} |f(x) - g(x)|
				\end{align}
				となることを見ればよい.$f,g$を入れ替えれば
				\begin{align}
					\operatorname*{max}_{x} |g(x)| - \operatorname*{max}_{x} |f(x)|
					\leq \operatorname*{max}_{x} |f(x) - g(x)|
				\end{align}
				も成り立つから当初の主張を得る.よって$\omega_1,\omega_2$を$C[0,\infty)$の要素とすれば
				\begin{align}
					\left| m^T(\omega_1,\delta) - m^T(\omega_2,\delta) \right|
					\leq \operatorname*{max}_{\substack{|s-t| \leq \delta \\ 0 \leq s,t \leq T}}
					|(\omega_1(s) - \omega_1(t)) - (\omega_2(s) - \omega_2(t))|
				\end{align}
				が成立する.ところで,いま$\epsilon$を任意に与えられた正数とし,
				\begin{align}
					T \leq n
				\end{align}
				を満たす自然数$n$を取り
				\begin{align}
					\rho(\omega_1,\omega_2) < 2^{-n} \epsilon
				\end{align}
				が満たされていると仮定すれば,
				\begin{align}
					\operatorname*{sup}_{0 \leq t \leq n}|\omega_1(t) - \omega_2(t)|
					< \epsilon
				\end{align}
				となるから
				\begin{align}
					0 \leq t \leq T \Longrightarrow |\omega_1(t) - \omega_2(t)| < \epsilon
				\end{align}
				が満たされる.このとき
				\begin{align}
					0 \leq s,t \leq T \Longrightarrow 
					&|(\omega_1(s) - \omega_1(t)) - (\omega_2(s) - \omega_2(t))| \\
					&\leq |\omega_1(s) - \omega_2(s)| + |\omega_1(t) - \omega_2(t)| \\
					&< 2\epsilon
				\end{align}
				が成り立つので
				\begin{align}
					\left| m^T(\omega_1,\delta) - m^T(\omega_2,\delta) \right| < 2\epsilon
				\end{align}
				が従い,$m^T(\omega,\delta)$の$\omega$に関する連続性が得られた.
			
			\item[第二段]
				$\delta$に関して非減少であることを示す.いま$0 < \delta \leq \delta'$とする.
				\begin{align}
					(s,t) \longmapsto |\omega(s) - \omega(t)|
				\end{align}
				は
				\begin{align}
					\Set{(s,t)}{|s-t| \leq \delta \wedge 0 \leq s,t \leq T}
				\end{align}
				の上で最大値を取るのであるから,
				\begin{align}
					|\tilde{s} - \tilde{t}| \leq \delta \wedge
					0 \leq \tilde{s}, \tilde{t} \leq T
				\end{align}
				かつ
				\begin{align}
					|\omega(\tilde{s}) - \omega(\tilde{t})| = m^T(\omega,\delta)
				\end{align}
				を満たす$\tilde{s},\tilde{t}$を取ることが出来るが,
				\begin{align}
					|\tilde{s} - \tilde{t}| \leq \delta'
				\end{align}
				も満たされるので
				\begin{align}
					|\omega(\tilde{s}) - \omega(\tilde{t})|
					\in \Set{|\omega(s) - \omega(t)|}{|s - t| \leq \delta \wedge 0 \leq s, t \leq T}
				\end{align}
				となり
				\begin{align}
					|\omega(\tilde{s}) - \omega(\tilde{t})| \leq m^T(\omega,\delta')
				\end{align}
				が従う.よって
				\begin{align}
					\delta \leq \delta' \Longrightarrow m^T(\omega,\delta) \leq m^T(\omega,\delta')
				\end{align}
				が示された.
			
			\item[第三段]
				$\lim_{\delta \downarrow 0}m^T(\omega,\delta) = 0$が成り立つことを示す.
				$\epsilon$を任意に与えられた正数とする.$\omega$は$[0,T]$上で一様連続となるので
				\begin{align}
					|s-t| \leq \delta \Longrightarrow |\omega(s) - \omega(t)| < \epsilon
				\end{align}
				を満たす正数$\delta$が取れるが,このとき
				\begin{align}
					\delta' \leq \delta
				\end{align}
				を満たす任意の正数$\delta'$に対しても
				\begin{align}
					|s-t| \leq \delta' \Longrightarrow |\omega(s) - \omega(t)| < \epsilon
				\end{align}
				となるから
				\begin{align}
					\lim_{\delta \downarrow 0}m^T(\omega,\delta) = 0
				\end{align}
				が得られる.
				\QED
		\end{description}
	\end{sketch}
	
	\begin{itembox}[l]{Theorem 4.10}
	\end{itembox}
	
	\begin{sketch}\mbox{}
		\begin{description}
			\item[第一段]
				$\eta$を任意に与えられた正数とする.
				$\{P_n\}_{n=1}^\infty$は緊密なので,$C[0,\infty)$の或るコンパクト部分集合$K$が存在して
				\begin{align}
					\forall n \in \Natural\, (\, 1 - \eta \leq P_n(K)\, )
				\end{align}
				が満たされる.他方で十分大きな正数$\lambda$を取れば
				\begin{align}
					\forall \omega \in K\, (\, |\omega(0)| \leq \lambda\, )
				\end{align}
				となる.これはすなわち
				\begin{align}
					K \subset \Set{\omega}{|\omega(0)| \leq \lambda}
				\end{align}
				を表し,
				\begin{align}
					\forall n \in \Natural\, \left(\, P_n\Set{\omega}{\lambda < |\omega(0)|}
					\leq P_n(C[0,\infty) \backslash K) \leq \eta\, \right)
				\end{align}
				が従う.また$T,\epsilon$を任意に与えられた正数とすれば,或る正数$\delta_0$が存在して
				\begin{align}
					0 < \delta \leq \delta_0
					\Longrightarrow \forall \omega \in K\, \left(\, m^T(\omega,\delta) \leq \epsilon\, \right)
				\end{align}
				が成立する.つまり
				\begin{align}
					0 < \delta \leq \delta_0
					\Longrightarrow K \subset \Set{\omega}{m^T(\omega,\delta) \leq \epsilon}
				\end{align}
				が成り立つので,
				\begin{align}
					0 < \delta \leq \delta_0
					\Longrightarrow \forall n \in \Natural\, \left(\, P_n\Set{\omega}{\epsilon < m^T(\omega,\delta)}
					\leq P_n(C[0,\infty) \backslash K) \leq \eta\, \right)
				\end{align}
				が満たされる.
				
			\item[第二段]
		\end{description}
	\end{sketch}
	
	\begin{itembox}[l]{Problem 4.12}
		Suppose $\{P_n\}_{n=1}^\infty$ is a sequence of probability measures on
		$\left( C[0,\infty),\borel{C[0,\infty)} \right)$ which converges weakly to a probability
		measure $P$. Suppose, in addition, that $\{f_n\}_{n=1}^\infty$ is a uniformly bounded sequence
		of real-valued, continuous functions on $C[0,\infty)$ converging to a continuous function $f$,
		the convergence being uniform on compact subsets of $C[0,\infty)$. Then
		\begin{align}
			\lim_{n \to \infty} \int_{C[0,\infty)} f_n(\omega)\ dP_n(\omega)
			= \int_{C[0,\infty)} f(\omega)\ dP(\omega).
		\end{align}
	\end{itembox}
	
	\begin{sketch}\mbox{}
		\begin{description}
			\item[第一段]
				$\{f_n\}_{n=1}^\infty$は一様有界なので
				\begin{align}
					\forall b \in \Natural\, \forall \omega \in C[0,\infty)\,
					\left(\, |f_n(\omega)| < b\, \right)
				\end{align}
				を満たす正数$b$が存在する.$C[0,\infty)$の各点$\omega$で
				\begin{align}
					f_n(\omega) \longrightarrow f(\omega)\quad (n \longrightarrow \infty)
				\end{align}
				となるから
				\begin{align}
					\forall \omega \in C[0,\infty)\, (\, |f(\omega)| < b\, )
				\end{align}
				が満たされる.すなわち$f$は有界連続であり,$(P_n)_{n=1}^\infty$が$P$に弱収束するので
				\begin{align}
					\lim_{n \to \infty} \int_{C[0,\infty)} f\ dP_n
					= \int_{C[0,\infty)} f\ dP
				\end{align}
				が成立する.
				
			\item[第二段]
				前段の結果より
				\begin{align}
					\left| \int_{C[0,\infty)} f\ dP_n
					- \int_{C[0,\infty)} f\ dP\right|
					\longrightarrow 0 \quad (n \longrightarrow \infty)
				\end{align}
				が成り立つから,
				\begin{align}
					\left|\int_{C[0,\infty)} f_n\ dP_n
					- \int_{C[0,\infty)} f\ dP_n\right|
					\longrightarrow 0 \quad (n \longrightarrow \infty)
					\label{eq:chapter_2_Problem_4_12}
				\end{align}
				が成り立つことを示せば定理の主張が得られる.
				$\{P_n\}_{n=1}^\infty$は相対コンパクトであるからProhorovの定理より緊密である.
				いま$\epsilon$を任意に与えられた正数とすると,$C[0,\infty)$の或るコンパクト部分集合$K$が存在して
				\begin{align}
					\forall n \in \Natural\, \left(\, P_n(C[0,\infty) \backslash K) < \epsilon\, \right)
				\end{align}
				となる.他方で$K$上で$(f_n)_{n=1}^\infty$は$f$に一様収束するので,
				或る自然数$N$を取れば
				\begin{align}
					\forall n \in \Natural\, \left(\, N \leq n
					\Longrightarrow \forall \omega \in K\, (\, |f_n(\omega) - f(\omega)| < \epsilon\, )\, \right)
				\end{align}
				が満たされる.このとき
				\begin{align}
					N \leq n \Longrightarrow
					&\left| \int_{C[0,\infty)} f_n\ dP_n
					- \int_{C[0,\infty)} f\ dP_n\right| \\
					&\leq \int_{C[0,\infty)} |f_n - f|\ dP_n \\
					&\leq \int_K |f_n - f|\ dP_n + \int_{C[0,\infty) \backslash K} |f_n - f|\ dP_n \\ \\
					&< \epsilon P_n(K) + 2b P_n(C[0,\infty) \backslash K) \\
					&< (1+2b) \epsilon
				\end{align}
				が成り立つので(\refeq{eq:chapter_2_Problem_4_12})が示された.
				\QED
		\end{description}
	\end{sketch}
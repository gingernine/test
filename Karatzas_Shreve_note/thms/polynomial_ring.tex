\subsection{多項式環}
	$(R,\sigma,\mu)$を可換環として,その零元と単位元をそれぞれ$\zeta$と$\epsilon$で表す.
	また$\zeta \neq \epsilon$と仮定する.すなわち$(R,\sigma,\mu)$は零環ではない.いま
	\begin{align}
		\tilde{P} \coloneqq \Set{f}{f:\omg \longrightarrow R \wedge 
		\exists n \in \omg\ \forall m \in \omg\ (\ n < m \Longrightarrow f(m) = \zeta\ )}
	\end{align}
	により集合$\tilde{P}$を定める.$\tilde{P}$とは$\omg$から$R$への写像のうち
	或る自然数以降は$\zeta$に張り付いてしまう写像の全体である.$a$を$R$の要素として
	\begin{align}
		\varphi_a \coloneqq \Set{x}{\exists n \in \omg\ (\ n = 0 \Longrightarrow x = (0,a)
		\wedge n \neq 0 \Longrightarrow x = (n,\zeta)\ )}
	\end{align}
	として$\varphi_a$を定めれば,$\varphi_a$は$\omg$から$R$への写像であり
	\begin{align}
		\varphi_a(n) = 
		\begin{cases}
			a, & (n=0), \\
			\zeta, & (n \neq 0)
		\end{cases}
	\end{align}
	を満たすから$\tilde{P}$の要素でもある.ここで
	\begin{align}
		\varphi \coloneqq \Set{x}{\exists a \in R\ \left(\ x=(a,\varphi_a)\ \right)}
	\end{align}
	として$\varphi$を定めれば
	\underline{$\varphi$は$R$から$\tilde{P}$への埋め込み(単射環準同型)となる}.
	
	\begin{prf}
	\end{prf}
	
	\monologue{
		院生[唐突に出てきた$\tilde{P}$ですが,なぜわざわざ波線記号を載せているのか,
		本節の主題である多項式と$\tilde{P}$がどう関係しているかということを説明していきます.
		以降も回りくどい説明が続きますから,多項式環が得られる過程を簡略して述べましょう.いま
		\begin{align}
			X \coloneqq \Set{x}{\exists n \in \omg\ (\ n = 1 \Longrightarrow x=(1,\epsilon) \wedge n \neq 1 \Longrightarrow x=(n,\zeta)\ )}
		\end{align}
		とおくと,$X$は
		\begin{align}
			X(n) =
			\begin{cases}
				\epsilon, & (n = 1), \\
				\zeta, & (n \neq 1)
			\end{cases}
		\end{align}
		を満たす$\omg$から$R$への写像ですから$\tilde{P}$の要素です.ちなみに
		$X$を点列の様式で(不正確な書き方ですが直感的に解釈するには都合が良いでしょう)
		\begin{align}
			(\zeta,\ \epsilon,\ \zeta,\ \zeta,\ \zeta,\ \cdots)
		\end{align}
		と書いてみましょう.すると$X^2$や$X^3$は
		\begin{align}
			&(\zeta,\ \zeta,\ \epsilon,\ \zeta,\ \zeta,\ \cdots), \\
			&(\zeta,\ \zeta,\ \zeta,\ \epsilon,\ \zeta,\ \cdots)
		\end{align}
		と表すことが出来ますし,特に$R$の要素$a$に対して$\varphi(a) \cdot X^n$は
		\begin{align}
			(\zeta, \zeta,\ \cdots,\ \zeta,\ a,\ \zeta,\ \zeta,\ \cdots)
		\end{align}
		と書くことが出来ますね.ここで$f$を$\tilde{P}$の要素とすれば,$f$は$R$の有限個の要素
		$a_0,a_1,\cdots.a_m$を用いて
		\begin{align}
			(a_0,\ a_1,\ \cdots,\ a_m,\ \zeta,\ \zeta,\ \zeta,\ \cdots)
		\end{align}
		と表すことが出来ますから
		\begin{align}
			f = \varphi(a_0) + \varphi(a_1) \cdot X + \varphi(a_2) \cdot X^2 + \cdots + \varphi(a_m) \cdot X^m
		\end{align}
		が成り立つのです.こうして$X$の冪を有限個連ねた式が出来ましたが,これは
		まだ多項式の卵の段階です.$\varphi$が余計ですから少し手を加えて整形しますと,
		多項式環というものが得られるという寸法です.」
	}
	
	上で作った$\tilde{P}$に対して
	\begin{align}
		P \coloneqq \left( \tilde{P} \backslash (\varphi \ast R) \right) \cup R
	\end{align}
	と定める.$P$とは$\tilde{P}$の$R$が埋め込まれた部分を$R$そのものに置き換えた集合である.
	また$\tilde{P}$から$P$への写像を
	\begin{align}
		h \coloneqq \{\, x \mid \quad \exists f \in \tilde{P}\ 
		&(\\
		&\quad \exists a \in R\ (\ f = \varphi(a)\ ) \Longrightarrow x = (f,a) \\
		&\quad \wedge f \notin \varphi \ast R \Longrightarrow x = (f,f)\\
		&)\, \}
	\end{align}
	で定めれば$h$は全単射となる.$h$は$\varphi \ast R$の要素には$\varphi$で対応する$R$の要素を
	当て,$\varphi \ast R$の外側では恒等写像となっている.また
	\begin{align}
		\sigma_P &\coloneqq \Set{x}{\exists f,g \in P\ \left(\ 
			x=((f,g),h(h^{-1}(f)+h^{-1}(g)))\ \right)}, \\
		\mu_P &\coloneqq \Set{x}{\exists f,g \in P\ \left(\ 
			x=((f,g),h(h^{-1}(f) \cdot h^{-1}(g)))\ \right)}
	\end{align}
	と定めれば,
	\underline{$\sigma_P$と$\mu_P$をそれぞれ加法と乗法として$(P,\sigma_P,\mu_P)$は可換環となる}.
	
	\monologue{
		院生「下線部の証明の前に注意しておきます.$\sigma_P$も$\mu_P$も
		定義式に括弧が多くて見づらいですが,見やすいように書けば$P$の要素$f,g$に対して
		\begin{align}
			\sigma_P(f,g) &= h(h^{-1}(f)+h^{-1}(g)), \\
			\mu_P(f,g) &= h(h^{-1}(f) \cdot h^{-1}(g))
		\end{align}
		としているのです.つまり,$P$上の算法は$h$で$\tilde{P}$に引き戻して計算したものを
		再び$h$で移すことにより定めているのですね.ゆえに,$h$が環同型となることは殆ど明らかでしょう.」
	}
	
	\begin{prf}
	\end{prf}
	
	またこのとき\underline{$(\tilde{P},\tilde{\sigma},\tilde{\mu})$と$(P,\sigma_P,\mu_P)$は
	環として$h$によって同型に対応する}.
	
	\begin{prf}	
	\end{prf}
	
	$f$を$P$から任意に選ばれた要素とするとき,
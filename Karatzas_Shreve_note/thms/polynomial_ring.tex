\section{多項式環}
	$(R,\sigma_R,\mu_R)$を可換環とする.また$\zeta_R$を$R$の零元とし,
	$\epsilon_R$を$R$の単位元とする.
	$P$を$\Natural$上の$R$-値写像で有限個の自然数を除いて
	$\zeta_R$に張り付くものの全体とする.つまり$P$は
	\begin{align}
		P \defeq \Set{f}{f:\Natural \longrightarrow R \wedge
		\exists n \in \Natural\, \forall m \in \Natural\,
		(\, n < m \Longrightarrow f(m) = \zeta_R\, )}
	\end{align}
	で定められる.$f$を$P$の要素とするとき
	\begin{align}
		\forall m \in \Natural\, (\, n < m \Longrightarrow f(m) = \zeta_R\, )
	\end{align}
	を満たす最小の自然数$n$が取れるが,その自然数を$f$の次数と呼び
	\begin{align}
		\deg{f}
	\end{align}
	と書く.さらに言えば,$\deg$とは$f$に対して
	\begin{align}
		\mu n\, \left[\, \forall m \in \Natural\, (\, n < m \Longrightarrow f(m) = \zeta_R\, )\, \right]
	\end{align}
	を対応させる写像である.$f$と$g$を$P$の要素とするとき
	\begin{align}
		\Natural \ni n \longmapsto \sigma_R(f(n),g(n))
	\end{align}
	なる写像を対応させる関係を$P$の加法として定め,
	\begin{align}
		\Natural \ni n \longmapsto \sum_{i \in n+1} \mu_R(f(i),g(n-i))
	\end{align}
	なる写像を対応させる関係を$P$の乗法として定める.$P$の加法を
	\begin{align}
		\sigma_P
	\end{align}
	と書き,$P$の乗法を
	\begin{align}
		\mu_P
	\end{align}
	と書く.\underline{このとき$(P,\sigma_P,\mu_P)$は可換環である}.
	
	\begin{sketch}
	\end{sketch}
	
	$P$の特別な要素として,
	\begin{align}
		\Natural \ni n \longmapsto
		\begin{cases}
			\epsilon_R & \mbox{if } n = 1 \\
			\zeta_R & \mbox{if } n \neq 1
		\end{cases}
	\end{align}
	なる写像を$X$と定める.このとき自然数$i$に対して$X^i$は
	\begin{align}
		\Natural \ni n \longmapsto
		\begin{cases}
			\epsilon_R & \mbox{if } n = i \\
			\zeta_R & \mbox{if } n \neq i
		\end{cases}
	\end{align}
	なる写像であって,さらに
	\begin{align}
		n \defeq \deg{f}
	\end{align}
	とおけば
	\begin{align}
		f = \sum_{i \in n} \mu_P \left(\varphi(f(i)),X^i\right)
	\end{align}
	が成立する.
	
	ここで$R$の要素$r$に対して
	\begin{align}
		\Natural \ni n \longmapsto
		\begin{cases}
			r & \mbox{if } n = 0 \\
			\zeta_R & \mbox{if } n \neq 0
		\end{cases}
	\end{align}
	なる写像を対応させる関係を$\varphi$とすると,\underline{$\varphi$は$R$から$P$への埋め込みである.}
	
	\begin{sketch}
	\end{sketch}
	
	そして
	\begin{align}
		R[X] \defeq \left(P \backslash \varphi \ast R\right) \cup R
	\end{align}
	と定める.
	\begin{align}
		P \ni f \longmapsto
		\begin{cases}
			\varphi^{-1}(f) & \mbox{if } f \in \varphi \ast R \\
			f & \mbox{if } f \notin \varphi \ast R
		\end{cases}
	\end{align}
	なる写像を$h$とすると
	\begin{align}
		h:P \bij R[X]
	\end{align}
	が成立して,この$h$によって$P$の算法を$R[X]$に移すことが出来る.つまり,
	$f$と$g$を$R[X]$の要素とするとき
	\begin{align}
		h\left( \sigma_P\left(h^{-1}(f),h^{-1}(g)\right) \right)
	\end{align}
	なる要素を対応させる関係を$R[X]$の加法として定め,
	\begin{align}
		h\left( \mu_P\left(h^{-1}(f),h^{-1}(g)\right) \right)
	\end{align}
	なる要素を対応させる関係を$R[X]$の乗法として定める.$R[X]$の加法を
	\begin{align}
		\sigma_X
	\end{align}
	と書き,$R[X]$の乗法を
	\begin{align}
		\mu_X
	\end{align}
	と書くことにする.すると$h$は$(P,\sigma_P,\mu_P)$から$(R[X],\sigma_X,\mu_X)$への環同型となり,しかも
	\begin{align}
		R \subset R[X]
	\end{align}
	かつ
	\begin{align}
		\sigma_R \subset \sigma_X
	\end{align}
	かつ
	\begin{align}
		\mu_R \subset \mu_X
	\end{align}
	が満たされる.
	
	$f$を$P$の要素とすると,適当な自然数$n$を取って
	\begin{align}
		f = \sum_{i \in n} \mu_P \left(\varphi(f(i)),X^i\right)
	\end{align}
	と書けたわけであるが,このとき
	\begin{align}
		h(f) = \sum_{i \in n} \mu_X \left(f(i),X^i\right)
	\end{align}
	が成立する.特に
	\begin{align}
		f \notin \varphi \ast R
	\end{align}
	ならば
	\begin{align}
		f = \sum_{i \in n} \mu_X \left(f(i),X^i\right)
	\end{align}
	が成立する.
	
	\begin{align}
		(R[X],\sigma_X,\mu_X)
	\end{align}
	のことを$R$上の{\bf 多項式環}\index{たこうしきかん@多項式環}{\bf (polynomial ring)}と呼び,
	\begin{align}
		X
	\end{align}
	のことをその{\bf 不定元}\index{ふていげん@不定元}{\bf (indeterminate)}と呼ぶ.
	
	\begin{screen}
		\begin{thm}[整域上の多項式環は整域]
			整域の上の多項式環は整域である.
		\end{thm}
	\end{screen}
	
	\begin{screen}
		\begin{thm}[体の上の多項式環はEuclid整域]
		\end{thm}
	\end{screen}
	
	\begin{screen}
		\begin{thm}[Euclid整域は主環]
		\end{thm}
	\end{screen}
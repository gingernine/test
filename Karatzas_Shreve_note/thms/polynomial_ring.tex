\section{多項式環}
	$(R,\sigma,\mu)$を可換環として,その零元と単位元をそれぞれ$\zeta$と$\epsilon$で表す.
	また$\zeta \neq \epsilon$と仮定する.すなわち$(R,\sigma,\mu)$は零環ではない.いま
	\begin{align}
		\tilde{P} \coloneqq \Set{f}{f:\omg \longrightarrow R \wedge 
		\exists n \in \omg\ \forall m \in \omg\ (\ n < m \Longrightarrow f(m) = \zeta\ )}
	\end{align}
	により集合$\tilde{P}$を定める.$\tilde{P}$とは$\omg$から$R$への写像のうち
	或る自然数以降は$\zeta$に張り付いてしまう写像の全体である.$a$を$R$の要素として
	\begin{align}
		\varphi_a \coloneqq \Set{x}{\exists n \in \omg\ (\ n = 0 \Longrightarrow x = (0,a)
		\wedge n \neq 0 \Longrightarrow x = (n,\zeta)\ )}
	\end{align}
	として$\varphi_a$を定めれば,$\varphi_a$は$\omg$から$R$への写像であり
	\begin{align}
		\varphi_a(n) = 
		\begin{cases}
			a, & (n=0), \\
			\zeta, & (n \neq 0)
		\end{cases}
	\end{align}
	を満たすから$\tilde{P}$の要素でもある.ここで
	\begin{align}
		\varphi \coloneqq \Set{x}{\exists a \in R\ \left(\ x=(a,\varphi_a)\ \right)}
	\end{align}
	として$\varphi$を定めれば
	\underline{$\varphi$は$R$から$\tilde{P}$への埋め込み(単射環準同型)となる}.
	
	\begin{prf}
	\end{prf}
	
	上で作った$\tilde{P}$に対して
	\begin{align}
		P \coloneqq \left( \tilde{P} \backslash (\varphi \ast R) \right) \cup R
	\end{align}
	と定める.$P$とは$\tilde{P}$の$R$が埋め込まれた部分を$R$そのものに置き換えた集合である.
	また$\tilde{P}$から$P$への写像を
	\begin{align}
		h \coloneqq \{\, x \mid \quad \exists f \in \tilde{P}\ 
		&(\\
		&\quad \exists a \in R\ (\ f = \varphi(a)\ ) \Longrightarrow x = (f,a) \\
		&\quad \wedge f \notin \varphi \ast R \Longrightarrow x = (f,f)\\
		&)\, \}
	\end{align}
	で定めれば$h$は全単射となる.$h$は$\varphi \ast R$の要素には$\varphi$で対応する$R$の要素を
	当て,$\varphi \ast R$の外側では恒等写像となっている.また
	\begin{align}
		\sigma_P &\coloneqq \Set{x}{\exists f,g \in P\ \left(\ 
			x=((f,g),h(h^{-1}(f)+h^{-1}(g)))\ \right)}, \\
		\mu_P &\coloneqq \Set{x}{\exists f,g \in P\ \left(\ 
			x=((f,g),h(h^{-1}(f) \cdot h^{-1}(g)))\ \right)}
	\end{align}
	と定めれば,
	\underline{$\sigma_P$と$\mu_P$をそれぞれ加法と乗法として$(P,\sigma_P,\mu_P)$は可換環となる}.
	
	\begin{prf}
	\end{prf}
	
	またこのとき\underline{$(\tilde{P},\tilde{\sigma},\tilde{\mu})$と$(P,\sigma_P,\mu_P)$は
	環として$h$によって同型に対応する}.
	
	\begin{prf}
	\end{prf}
	
	$f$を$P$から任意に選ばれた要素とするとき,$h^{-1}(f)$は$\tilde{P}$の要素となるから,
	或る自然数$m$及び$m+1$個の$R$の要素$a_0,a_1,\cdots,a_m$が存在して
	\begin{align}
		h^{-1}(f) = \varphi(a_0) + \varphi(a_1) \cdot X + \varphi(a_2) \cdot X^2 + \cdots + \varphi(a_m) \cdot X^m
	\end{align}
	と書ける.$h$は環同型であるから,このとき
	\begin{align}
		f = a_0 + a_1 \cdot X + \cdot a_2 \cdot X^2 + \cdots + a_m \cdot X^m
	\end{align}
	が成り立つ.ゆえに$P$の任意の要素は$R$の要素を係数とする$X$の多項式として書ける.
	この$(P,\sigma_P,\mu_P)$を$R$上の
	{\bf 多項式環}\index{たこうしきかん@多項式環}{\bf (polynomial ring)}と呼び,特に$P$を
	\begin{align}
		R[X]
	\end{align}
	と書く.また$X$のことを{\bf 不定元}\index{ふていげん@不定元}{\bf (indeterminate)}と呼ぶ.
	
	\begin{screen}
		\begin{thm}[整域上の多項式環は整域]
			整域の上の多項式環は整域である.
		\end{thm}
	\end{screen}
	
	\begin{prf}
		$(R,\sigma,\mu)$を整域とし,その多項式環を
		$\left( R[X],\tilde{\sigma},\tilde{\mu} \right)$をその多項式環とする.
		また$(R,\sigma,\mu)$の零元と単位元を$\zeta,\epsilon$で表し,
		$\left( R[X],\tilde{\sigma},\tilde{\mu} \right)$の零元と単位元を
		$\tilde{\zeta},\tilde{\epsilon}$で表す.
		$f,g$を$R[X]$の任意の要素とすれば,
		\begin{align}
			f \neq \tilde{\zeta} \wedge g \neq \tilde{\zeta}
		\end{align}
		が満たされているとき
		
		
	\end{prf}
	
	\begin{screen}
		\begin{thm}[体の上の多項式環はEuclid整域]
		\end{thm}
	\end{screen}
	
	\begin{screen}
		\begin{thm}[Euclid整域は単項イデアル整域]
		\end{thm}
	\end{screen}
\subsection{商位相}
	いま$(S,\mathscr{O})$を位相空間とし,$R$を$S$上の同値関係とし,
	$q$を$S$から$S/R$への商写像とする.このとき
	\begin{align}
		\mathscr{O}_{q} \defeq \Set{o}{o \subset S/R \wedge \inv{q} \ast o \in \mathscr{O}}
	\end{align}
	で定める集合は$S/R$上の位相構造をなす.実際,
	\begin{description}
		\item[step1] $\emptyset$も$S/R$も$S/R$の部分集合であって,また
			\begin{align}
				\inv{q} \ast \emptyset = \emptyset \in \mathscr{O}
			\end{align}
			及び
			\begin{align}
				\inv{q} \ast S/R = S \in \mathscr{O}
			\end{align}
			が成り立つので,$\emptyset$も$S/R$も$\mathscr{O}_{q}$に属する.
				
		\item[step2] $u$と$v$を$\mathscr{O}_{q}$の要素とするとき,
			\begin{align}
				\inv{q} \ast (u \cap v) = \left(\inv{q} \ast u\right) \cap \left(\inv{q} \ast v\right) \in \mathscr{O}
			\end{align}
			が成り立つので
			\begin{align}
				u \cap v \in \mathscr{O}_{q}
			\end{align}
			が従う.
			
		\item[step3] $\mathscr{W}$を$\mathscr{O}_{q}$の部分集合とするとき,
			\begin{align}
				\inv{q} \ast \bigcup \mathscr{W} = \bigcup_{w \in \mathscr{W}} \inv{q} \ast w \in \mathscr{O}
			\end{align}
			が成り立つので
			\begin{align}
				\bigcup \mathscr{W} \in \mathscr{O}
			\end{align}
			が従う.
	\end{description}
	
	また$\mathscr{O}_{q}$は$q$を連続にする最大の位相でもある.実際$\mathscr{U}$を$S/R$上の位相構造とし,
	$q$が$\mathscr{O}/\mathscr{U}$-連続であるとすると,$\mathscr{U}$任意の要素$u$は
	\begin{align}
		u \subset S/R
	\end{align}
	かつ
	\begin{align}
		\inv{q} \ast u \in \mathscr{O}
	\end{align}
	を満たすので
	\begin{align}
		u \in \mathscr{O}_{q}
	\end{align}
	が成立する.すなわち
	\begin{align}
		\mathscr{U} \subset \mathscr{O}_{q}
	\end{align}
	が成立する.
	
	\begin{screen}
		\begin{dfn}[商位相]
			位相空間$(S,\mathscr{O})$とし,$R$を$S$上の同値関係とし,
			$q$を$S$から$S/R$への商写像とする.このとき
			\begin{align}
				\mathscr{O}_{q} \defeq \Set{o}{o \subset S/R \wedge \inv{q} \ast o \in \mathscr{O}}
			\end{align}
			により定める$S/R$上の位相構造を,$(\mathscr{O},R)$-{\bf 商位相}\index{しょういそう@商位相}(quotient topology)と呼ぶ.
		\end{dfn}
	\end{screen}
	
	\begin{screen}
		\begin{thm}[商空間が$T_1 \Longleftrightarrow$同値類が元の空間で閉じている]
		\label{thm:quotient_space_T_1_iff_each_equivalence_class_closed}
			$S$を位相空間,$\sim$を$S$上の同値関係,$\pi:S \longrightarrow S/\sim$を商写像
			とする.このとき次が成り立つ:
			\begin{align}
				\mbox{$S/\sim$が$T_1$空間である}
				\quad \Longleftrightarrow \quad
				\mbox{任意の$x \in S$に対し$\pi(x)$が$S$の閉集合である}.
			\end{align}
		\end{thm}
	\end{screen}
	
	\begin{prf}
		任意の$F \subset S/\sim$に対し
		\begin{align}
			\mbox{$F$が閉} \quad \Longleftrightarrow \quad
			\mbox{$\pi^{-1}(F^c) = \pi^{-1}(F)^c$が開} \quad \Longleftrightarrow \quad
			\mbox{$\pi^{-1}(F)$が閉}
		\end{align}
		となる.いま任意の$x \in S$に対し
		$\pi(x) = \pi^{-1}(\pi(x))$が満たされているから定理の主張を得る.
		\QED
	\end{prf}
	
	\begin{screen}
		\begin{thm}[商写像が開なら,商空間がHausdorff
		$\Longleftrightarrow$対角線集合が閉]
		\label{thm:quotient_space_Hausdorff_iff_diagonal_set_closed}
			$S$を位相空間,$\sim$を$S$上の同値関係,$\pi:S \longrightarrow S/\sim$を商写像
			とする.このとき,$\pi$が開写像であれば次が成立する:
			\begin{align}
				\mbox{$S/\sim$がHausdorff} \quad \Longleftrightarrow \quad
				\mbox{$\Set{(x,y) \in S \times S}{x \sim y}$が閉}.
			\end{align}
		\end{thm}
	\end{screen}
	
	\begin{prf}
		$S/\sim$がHausdorffであるとき,$x \not\sim y$を満たす$(x,y) \in S \times S$に対し
		$\pi(x) \neq \pi(y)$となるから
		\begin{align}
			\pi(x) \in U,\quad \pi(y) \in V,\quad U \cap V = \emptyset
		\end{align}
		を満たす$S/\sim$の開集合$U,V$が取れる.このとき
		$\pi^{-1}(U) \times \pi^{-1}(V)$は$S \times S$の開集合であり
		\begin{align}
			(x,y) \in \pi^{-1}(U) \times \pi^{-1}(V)
			\subset \Set{(s,t) \in S \times S}{s \not\sim t}
		\end{align}
		が成り立つから$\Longrightarrow$が得られる.
		逆に$\Set{(s,t) \in S \times S}{s \not\sim t}$が開集合であるとき,
		$\pi(x) \neq \pi(y)$なら
		\begin{align}
			(x,y) \in U \times V \subset \Set{(s,t) \in S \times S}{s \not\sim t}
		\end{align}
		を満たす$S$の開集合$U,V$が存在し,このとき
		\begin{align}
			\pi(x) \in \pi(U),\quad \pi(y) \in \pi(V),
			\quad \pi(U) \cap \pi(V) = \emptyset
		\end{align}
		となりかつ$\pi$が開写像であるから$\Longleftarrow$が従う.
		\QED
	\end{prf}
	
	\begin{screen}
		\begin{cor}[Hausdorff
		$\Longleftrightarrow$対角線集合が閉]
		\label{cor:quotient_space_Hausdorff_iff_diagonal_set_closed}
			$S$を位相空間とするとき,
			\begin{align}
				\mbox{$S$がHausdorffである}
				\quad \Longleftrightarrow \quad
				\mbox{$\Set{(x,x)}{x \in S}$が$S \times S$で閉じている}.
			\end{align}
		\end{cor}
	\end{screen}
	
	\begin{prf}
		等号$=$を同値関係と見れば$S$と$S/=$は商写像により同相となるから,
		定理\ref{thm:quotient_space_Hausdorff_iff_diagonal_set_closed}より
		\begin{align}
			\mbox{$S$がHausdorff} \quad \Longleftrightarrow \quad
			\mbox{$S/=$がHausdorff} \quad \Longleftrightarrow \quad
			\mbox{$\Set{(x,x)}{x \in S}$が閉}
		\end{align}
		が成立する.
		\QED
	\end{prf}
	
	\begin{screen}
		\begin{dfn}[(位相的)埋め込み写像]
			$S,T$を位相空間とするとき,$S$から$T$への{\bf (位相的)埋め込み}\index{うめこみ@埋め込み}
			{\bf (embedding)}とは,連続な単射$i:S \longrightarrow T$で,$S$と(相対位相を入れた)
			$i(S)$を$i$
			(の終集合を$i(S)$に制限した全単射)により同相とするものである.
		\end{dfn}
	\end{screen}
	
	\begin{screen}
		\begin{dfn}[コンパクト化]
			$S$をコンパクトではない位相空間,$K$をコンパクト位相空間として,
			$S$が$K$に稠密に埋め込まれるとき,言い換えれば,$S$から$K$への
			位相的埋め込み$i$が存在して$i(S)$が$K$で稠密となるとき,$K$を(埋め込み$i$による)
			$S$の{\bf コンパクト化}\index{こんぱくとか@コンパクト化}
			{\bf (compactification)}と呼ぶ.
		\end{dfn}
	\end{screen}
	
	\begin{screen}
		\begin{thm}[一点を追加すればコンパクト空間となる(Alexandroff拡大)]
		\label{thm:Alexandroff_compactification}
			$S$をコンパクトではない位相空間,$x_\infty$を$S$に属さない点とする.
			このとき$K \coloneqq S \cup \{x_\infty\}$とおいて,
			$K$の部分集合$U$で
			\begin{itemize}
				\item $x_\infty \notin U$なら$U$は$S$の開集合
				\item $x_\infty \in U$なら$K \backslash U$は$S$で閉かつコンパクト
			\end{itemize}
			となるものの全体を$\mathscr{O}$と定めれば,$\mathscr{O}$は$K$上の位相となり,
			$K$は$S$の($S$から$K$への恒等写像による)コンパクト化となる.また以下が成立する:
			\begin{description}
				\item[(1)] $\mbox{$S$が$T_1$} \quad 
					\Longleftrightarrow \quad \mbox{$K$が$T_1$}$.
					
				\item[(2)] $\mbox{$S$が局所コンパクトHausdorff} \quad 
					\Longleftrightarrow \quad \mbox{$K$がHausdorff}$.
					
				\item[(3)] $S$がHausdorffであるとき,
					$S$の位相を含み,かつ$K$をコンパクトHausdorff空間とするような
					$K$上の位相は$\mathscr{O}$のみである.
			\end{description}
		\end{thm}
	\end{screen}
	
	\begin{prf}\mbox{}
		\begin{description}
			\item[第一段]
				$\mathscr{O}$が$K$上の位相であることを示す.
				先ず$\emptyset\ (= K \backslash K)$は$S$で開,閉及びコンパクトであるから
				$K,\emptyset \in \mathscr{O}$となる.
				また$U,V \in \mathscr{O}$を取れば,
				\begin{itemize}
					\item $x_\infty \notin U$かつ$x_\infty \notin V$なら
						$U,V$は$S$の開集合であるから$U \cap V \in \mathscr{O}$.
					
					\item $x_\infty \in U$かつ$x_\infty \notin V$のとき,
						$V' \coloneqq V \backslash \{x_\infty\}$とおけば
						\begin{align}
							K \backslash V = S \backslash V'
						\end{align}
						となり,$K \backslash V$は$S$で閉じているから$V'$は$S$の開集合であり
						$U \cap V = U \cap V' \in \mathscr{O}$が従う.
						
					\item $x_\infty \in U$かつ$x_\infty \in V$のとき,
						$K \backslash (U \cap V)= (K \backslash U) \cup (K \backslash V)$
						より$K \backslash (U \cap V)$は$S$で閉かつコンパクトなので
						$U \cap V \in \mathscr{O}$となる.
				\end{itemize}
				従って$\mathscr{O}$は有限交叉で閉じる.
				任意の$\mathscr{U} \subset \mathscr{O}$に対し
				$\mathscr{U}_1 \coloneqq \Set{U \in \mathscr{U}}{x_\infty \in U},
				\mathscr{U}_2 \coloneqq \Set{U \in \mathscr{U}}{x_\infty \notin U}$
				とおけば,$\mathscr{U}_2$の元は$S$の開集合なので
				$x_\infty \notin \bigcup \mathscr{U}$なら
				$\bigcup \mathscr{U} = \bigcup \mathscr{U}_2 \in \mathscr{O}$となる.
				$x_\infty \in \bigcup \mathscr{U}$のとき,
				\begin{align}
					K \left\backslash \bigcup \mathscr{U} \right.
					= \left(K \left\backslash \bigcup \mathscr{U}_1 \right.\right)
					\bigcap \left(S \left\backslash \bigcup \mathscr{U}_2 \right.\right)
					= \Biggl( \bigcap_{U \in \mathscr{U}_1} K \backslash U \Biggr)
					\bigcap \Biggl( \bigcap_{U \in \mathscr{U}_2} S \backslash U \Biggr)
				\end{align}
				より$K\left\backslash \bigcup \mathscr{U}\right.$は$S$で閉じ,
				また定理\ref{thm:closed_subset_of_compact_set_is_compact_on_Hausdorff_space}
				より$S$でコンパクトでもあるから$\bigcup \mathscr{U} \in \mathscr{O}$が従う.
				
			\item[第二段] $S$から$K$への恒等写像$i$が埋め込みであることを示す.
				実際$i$は単射であり,また$\mathscr{O}$が$S$の位相を含むから
				$i$は開写像でもある.任意に$U \in \mathscr{O}$を取れば
				\begin{align}
					\begin{cases}
						x_\infty \in U \Longrightarrow 
						\mbox{$i^{-1}(U) = U$は$S$の開集合,} & \\
						x_\infty \notin U \Longrightarrow
						\mbox{$S \cap U = S \backslash (K \backslash U)$かつ
						$K \backslash U$が$S$の閉集合であるから
						$i^{-1}(U) = S \cap U$は$S$の開集合,}
					\end{cases}
				\end{align}
				が成り立つから$i$の連続性も出る.
				
			\item[第三段] $S$が$K$で稠密であることを示す.
				$S$はコンパクトでないから$\{x_\infty\}$は$K$の開集合ではなく,
				$x_\infty$の任意の近傍は$S$と交わることになる.
				従って定理\ref{thm:belongs_to_closure_iff_clusters}より
				$x_\infty \in \overline{S}$となり$\overline{S} = K$を得る.
				
			\item[第四段] $K$がコンパクトであることを示す.
				$\mathscr{B}$を$K$の開被覆とすれば$x_\infty$を含む
				$B_\infty \in \mathscr{B}$が取れる.$K \backslash B_\infty$は$S$でコンパクトであり,
				$\Set{S \cap B}{B \in \mathscr{B}}$は$K \backslash B_\infty$の
				$S$における開被覆となるから,有限部分集合
				$\mathscr{B}' \subset \mathscr{B}$で
				\begin{align}
					K \backslash B_\infty \subset \bigcup \mathscr{B}'
				\end{align}
				を満たすものが存在する.$K = B_\infty \cup \bigcup \mathscr{B}'$が成り立つから
				$K$はコンパクトである.
				
			\item[第五段] (1)を示す.$S$が$T_1$であるとき,任意の$x \in S$に対し
				$\{x\}$は$S$で閉かつコンパクトであるから$K$で閉じる.また$S \in \mathscr{O}$より
				$\{x_\infty\} = K \backslash S$は$K$で閉となり
				$\Longrightarrow$を得る.逆に$K$が$T_1$であるとき,
				任意の$x \in S$に対し
				\begin{align}
					S \backslash \{x\} = S \cap (K \backslash \{x\})
				\end{align}
				かつ右辺は$S$の開集合であるから$\{x\}$は$S$の閉集合となり$\Longleftarrow$を得る.
			
			\item[第六段] (2)を示す.$S$が局所コンパクトHausdorffであるとして任意に相異なる二点
				$x,y \in K$を取れば,$x,y \in S$なら$x,y$は$S$の開集合で分離されるが,
				$S$の開集合は$K$の開集合となるから$x,y$は$K$の開集合で分離されることになる.
				一方で$x = x_\infty$なら,$y$は$S$でコンパクト(かつ$S$のHausdorff性より閉)な近傍$C$を持ち,
				$K \backslash C \in \mathscr{O}$が従うから$K \backslash C$は
				$x_\infty$の開近傍となる.よって$\Longrightarrow$を得る.
				逆に$K$がHausdorffであるとき,$S$は$K$の部分空間であるから$S$もHausdorffとなる.
				また任意の$x \in S$に対し,$x$と$x_\infty$は$K$の開集合$U,V$で分離されるが,このとき
				$K \backslash V$は$S$で閉かつコンパクトとなり,
				\begin{align}
					x \in (S \cap U) \subset K \backslash V 
				\end{align}
				より$K \backslash V$は$x$の$S$における近傍となるから$S$の局所コンパクト性も出る.
				
			\item[第七段] 
				\QED
		\end{description}
	\end{prf}
	
	\begin{screen}
		\begin{thm}[局所コンパクトなら$T_2 \Longleftrightarrow T_{3{}^1{\mskip -5mu/\mskip -3mu}_2}$]
		\label{thm:T_2_equals_to_T_3_in_locally_compact_spaces}
			局所コンパクトHausdorff空間はTychonoffである.
		\end{thm}
	\end{screen}
	
	\begin{prf}\mbox{}
		\begin{description}
			\item[第一段]
				$S$をコンパクトHausdorff空間とするとき,$S$の閉集合はコンパクトとなるから
				定理\ref{thm:Hausdorff_space_two_disjoint_compact_sets_are_separated_by_nbh}
				より$S$は正則Hausdorffとなり,Urysohnの補題よりTychonoffとなる.
				
			\item[第二段]
				$S$をコンパクトではない局所コンパクトHausdorff空間とすれば,
				定理\ref{thm:Alexandroff_compactification}より
				$S$は或るコンパクトHausdorff空間$K$に埋め込まれる.
				前段より$K$はTychonoffであり,$S$もTychonoffとなる.
				\QED
		\end{description}
	\end{prf}
\subsection{数の構成の一時的なメモ置き場}
	流れを把握していても思うように書けるとは限らない.満足いく体裁で書けるまで
	整理のためにメモだけ置いておく.$\mathcal{L}'$の文法等にはこだわらず大雑把に.
	
\subsubsection{商集合の算法}
	$A$を集合とし,$\sigma$を$A$上の算法とし,$R$を$A$上の同値関係とする.また
	$A$から$A/R$への商写像を$\pi$と書く.このとき
	\begin{align}
		\sigma_\pi \coloneqq \Set{x}{\exists s,t \in A\, 
		\left(\, x=((\pi(s),\pi(t)),\pi(\sigma(s,t)))\, \right)}
	\end{align}
	と定めると,$\sigma_\pi$は$A/R$上の算法となり
	\begin{align}
		\sigma(\pi(s),\pi(t)) = \pi(\sigma(s,t))
	\end{align}
	を満たす.
	
\subsubsection{同型定理}
	$A,A'$を集合,$\sigma,\sigma'$をそれぞれ$A,A'$上の算法とし,
	$f$を$A$から$A'$への写像とする.$f$が
	\begin{align}
		f(\sigma(x,y)) = \sigma'(f(x),f(y))
	\end{align}
	を満たすとき,$f$は$(A,\sigma)$から$(A',\sigma')$への準同型写像であるという.
	ここで$A$上の同値関係を
	\begin{align}
		N \coloneqq \Set{x}{\exists y,z \in A\, 
		\left(\, f(y) = f(z) \wedge x = (y,z)\, \right)}
	\end{align}
	で定める.そして$A$から$A/N$への商写像を$\pi$と書く.このとき
	\begin{align}
		g\left(\pi(x)\right) = f(x)
	\end{align}
	で$g$を定めれば,$g$は$A/N$から$f \ast A$への全単射となる.実際,
	$x,y$を$A/N$の要素とすれば
	\begin{align}
		x = \pi(s) \wedge y = \pi(t)
	\end{align}
	を満たす$A$の要素$s,t$が存在し,
	\begin{align}
		g(x) = g(y) \Longrightarrow f(s) = f(t)
		\Longrightarrow (s,t) \in N
		\Longrightarrow x = \pi(s) = \pi(t) = y
	\end{align}
	が成立するので$g$は単射であり,また$z$を$f \ast A$の要素とすれば
	\begin{align}
		z = f(w)
	\end{align}
	を満たす$A$の要素$w$が存在し,
	\begin{align}
		g(\pi(w)) = f(w) = z
	\end{align}
	が成り立つので$g$は全射である.$A/R$上の算法を
	\begin{align}
		\sigma_\pi(\pi(x),\pi(y)) = \pi(\sigma(x,y))
	\end{align}
	で定めば
	\begin{align}
		g \left( \sigma_\pi(\pi(x),\pi(y)) \right) 
		&= g \left( \pi(\sigma(x,y)) \right) \\
		&= f \left( \sigma(x,y) \right) \\
		&= \sigma'(f(x),f(y)) \\
		&= \sigma' \left( g(\pi(x)),g(\pi(y)) \right) \\
	\end{align}
	が成り立つ.すなわち$g$は同型写像である.
	
	
\subsubsection{整数}
	$(S,o)$を可換半群とするとき,$S \times S$上の同値関係を
	\begin{align}
		R \coloneqq \Set{x}{\exists a,b,c,d \in S\, (\, x=((a,b),(c,d))
		\wedge o(a,d) = o(b,c)\, )}
	\end{align}
	で定め,
	\begin{align}
		G \coloneqq S \times S / R
	\end{align}
	とおく.そして$x,y$をSの要素とするとき,$(x,y)$の同値類を$[x,y]$と書く.
	このとき
	\begin{align}
		\sigma \left([x,y],[x',y'] \right) = \left[o(x,x'),o(y,y')\right]
	\end{align}
	で$\sigma$を定めると,$\sigma$は可換律と結合律を満たす.実際,
	\begin{align}
		\sigma \left( [x,y],[x',y'] \right)
		= \left[ o(x,x'), o(y,y') \right]
		= \left[ o(x',x), o(y',y) \right]
		= \sigma \left( [x',y'],[x,y] \right)
	\end{align}
	と
	\begin{align}
		\sigma \left(\sigma \left([x,y],[x',y']\right),[x'',y''] \right)
		&= \sigma \left(\left[ o(x,x'),o(y,y') \right],[x'',y''] \right) \\
		&= \left[ o(o(x,x'),x''), o(o(y,y'),y'') \right] \\
		&= \left[ o(x,o(x',x''), o(y,o(y',y'')) \right] \\
		&= \sigma \left( [x,y], \left[ o(x',x''),o(y',y'') \right] \right) \\
		&= \sigma \left( [x,y], \sigma \left([x',y'],[x'',y'']\right) \right)
	\end{align}
	が成り立つ.それから,$o$の可換律から
	\begin{align}
		o(x,y) = o(y,x)
	\end{align}
	が成り立つので
	\begin{align}
		((x,x),(y,y)) \in R
	\end{align}
	となり,
	\begin{align}
		[x,x] = [y,y]
	\end{align}
	が成立する.そこで
	\begin{align}
		\zeta \coloneqq [x,x]
	\end{align}
	とおく.このとき
	\begin{align}
		\sigma \left( [x,y],\zeta \right)
		= \sigma \left( \zeta,[x,y] \right)
		= [x,y]
	\end{align}
	が満たされる.実際,$\zeta = [z,z]$より
	$\left( [x,y],\zeta \right) = \left( [x,y],[z,z] \right)$となるから
	\begin{align}
		\sigma \left( [x,y],\zeta \right)
		&= \sigma \left( [x,y],[z,z] \right) \\
		&= \left[ o(x,z), o(y,z) \right]
	\end{align}
	となるが,
	\begin{align}
		o(o(x,z),y) = o(x,o(z,y)) = o(x,o(y,z)) = o(o(y,z),x)
	\end{align}
	より$(o(x,z),o(y,z))$と$(x,y)$は同値となるので
	\begin{align}
		\sigma \left( [x,y],\zeta \right) = [x,y]
	\end{align}
	が成立する.同様にして
	\begin{align}
		\sigma \left( \zeta,[x,y] \right) = [x,y]
	\end{align}
	も成立する.また$[x,y]$に対しては$[y,x]$が
	\begin{align}
		\sigma([x,y],[y,x]) = \sigma([y,x],[x,y]) = \zeta
	\end{align}
	を満たす.そこで$[y,x]$を$[x,y]$の逆元と呼び
	\begin{align}
		-[x,y] \coloneqq [y,x]
	\end{align}
	とおく.$a$を$S$の要素として
	\begin{align}
		\varphi(x) = [o(x,a),a]
	\end{align}
	で$S$から$G$への写像$\varphi$を定めるとき,
	\begin{align}
		\varphi(o(x,y)) = [o(o(x,y),a),a]
		&= \sigma \left( [o(o(x,y),a),a],[a,a] \right) \\
		&= [o(o(o(x,y),a),a),o(a,a)] \\
		&= [o(o(o(x,y),a),a),o(a,a)] \\
		&= [o(o(x,o(y,a)),a),o(a,a)] \\
		&= [o(o(o(y,a),x),a),o(a,a)] \\
		&= [o(o(y,a),o(x,a)),o(a,a)] \\
		&= \sigma \left( [o(y,a),a],[o(x,a),a] \right) \\
		&= \sigma \left( [o(x,a),a],[o(y,a),a] \right) \\
		&= \sigma \left( \varphi(x),\varphi(y) \right)
	\end{align}
	及び
	\begin{align}
		[x,y] &= [o(x,o(a,a)),o(y,o(a,a))] \\
		&= [o(o(x,a),a),o(o(y,a),a)] \\
		&= [o(o(x,a),a),o(o(y,a),a)] \\
		&= [o(o(x,a),a),o(a,o(y,a))] \\
		&= \sigma \left( [o(x,a),a],[a,o(y,a)] \right) \\
		&= \sigma \left( \varphi(x),-\varphi(y) \right)
	\end{align}
	が成立する.特に,$o$が簡約律を満たすなら$\varphi$は単射となる.実際,
	\begin{align}
		\varphi(x) = \varphi(y)
		&\Longrightarrow [o(x,a),a] = [o(y,a),a] \\
		&\Longrightarrow o(o(x,a),a) = o(o(y,a),a) \\
		&\Longrightarrow o(x,o(a,a)) = o(x,o(a,a)) \\
		&\Longrightarrow x = y
	\end{align}
	となる.いま$o$を簡約律と可換律を満たすとして,
	\begin{align}
		\tilde{G} \coloneqq \left( G \backslash (\varphi \ast S) \right) \cup S
	\end{align}
	とおいて
	\begin{align}
		h(x) = 
		\begin{cases}
			x & (x \in G) \\
			\varphi^{-1}(x) & (x \in \varphi \ast S)
		\end{cases}
	\end{align}
	とおくと,$h$は$G$から$\tilde{G}$への全単射となる.そして$\tilde{G}$上の算法を
	\begin{align}
		\tilde{\sigma}(x,y) = h\left(\sigma\left(h^{-1}(x),h^{-1}(y)\right)\right)
	\end{align}
	で定めると$h$は同型写像となる.また
	\begin{align}
		-x \coloneqq h\left( -h^{-1}(x) \right)
	\end{align}
	と書く.このとき,$x,y$を$S$の要素とすれば
	\begin{align}
		h^{-1}(x) = \varphi(x) \wedge h^{-1}(y) = \varphi(y)
	\end{align}
	となるので
	\begin{align}
		\tilde{\sigma}(x,y) = h\left(\sigma\left(\varphi(x),\varphi(y)\right)\right)
		= h\left(\varphi(o(x,y))\right)
		= o(x,y)
	\end{align}
	が満たされる.すなわち{\bf $\tilde{\sigma}$は$o$の拡張となっている}.
	また$\tilde{G}$の任意の要素$x$は,或る$S$の要素$y,z$によって
	\begin{align}
		x = h([y,z])
	\end{align}
	と書けるが,このとき
	\begin{align}
		x &= h\left(\sigma(\varphi(y),-\varphi(z))\right) \\
		&= \tilde{\sigma} \left(h(\varphi(y)),h(-\varphi(z))\right) \\
		&= \tilde{\sigma} \left(h(\varphi(y)),-h(\varphi(z))\right) \\
		&= \tilde{\sigma}(y,-z)
	\end{align}
	が成立するので,{\bf $\tilde{\sigma}$の要素は$S$の要素に対する演算で表せる}.
	この$(\tilde{G},\tilde{\sigma})$を{\bf $S$が生成する群}と呼ぶ.
	$\omg$には加法と乗法が定まっているが,その最小の拡張となる環が整数環である.
	
	整数環の性質:
	\begin{itemize}
		\item 整数環は順序環である.
		\item 整数環はEuclid整域である.
		\item 任意の環に対して整数環からの準同型が存在する.
	\end{itemize}
	
\subsubsection{有理数}
	有理数体は整数環の分数体であるから分数体の構成法をメモしておく.
	$(R,\sigma,\mu)$を整域として,その零元と単位元をそれぞれ$\zeta,\epsilon$で表す.
	また$R \times R \backslash \{\zeta\}$上の同値関係を
	\begin{align}
		\Phi \coloneqq \Set{x}{\exists a,c \in R\, \exists b,d \in R \backslash \{\zeta\}\, \left(\, x=((a,b),(c,d)) \wedge \mu(a,d) = \mu(b,c)\, \right)}
	\end{align}
	で定め,
	\begin{align}
		Q \coloneqq (R \times R \backslash \{\zeta\}) / \Phi
	\end{align}
	とおく.そして$(x,y) \in R \times R \backslash \{\zeta\}$の同値類を$[x,y]$と書く.
	$Q$上の算法を
	\begin{align}
		\sigma_Q &\coloneqq \Set{x}{\exists a,c \in R\, \exists b,d \in R \backslash \{\zeta\}\, \left(\, x=(([a,b],[c,d]),[\sigma(\mu(a,c),\mu(b,c)),\mu(b,d)])\, \right)}, \\
		\mu_Q &\coloneqq \Set{x}{\exists a,c \in R\, \exists b,d \in R \backslash \{\zeta\}\, \left(\, x=(([a,b],[c,d]),[\mu(a,c),\mu(b,d)])\, \right)}
	\end{align}
	で定める.このとき$(Q,\sigma_Q,\mu_Q)$は体となる.
	
	$\sigma_Q,\mu_Q$は可換である.
	\begin{align}
		\sigma_Q([x,y],[x',y']) = [\sigma(\mu(x,y'),\mu(y,x')),\mu(y,y')]
		= 
	\end{align}
	
\newpage
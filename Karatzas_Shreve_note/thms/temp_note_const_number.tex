\subsection{数の構成の一時的なメモ置き場}
	流れを把握していても思うように書けるとは限らない.満足いく体裁で書けるまで
	整理のためにメモだけ置いておく.$\mathcal{L}'$の文法等にはこだわらず大雑把に.
	
\subsubsection{商集合の算法}
	
\subsubsection{同型定理}
	$A,A'$を集合,$\sigma,\sigma'$をそれぞれ$A,A'$上の算法とし,
	$f$を$A$から$A'$への写像とする.$f$が
	\begin{align}
		f(\sigma(x,y)) = \sigma'(f(x),f(y))
	\end{align}
	を満たすとき,$f$は$(A,\sigma)$から$(A',\sigma')$への準同型写像であるという.
	ここで$A$上の同値関係を
	\begin{align}
		N \coloneqq \Set{x}{\exists y,z \in A\, 
		\left(\, f(y) = f(z) \wedge x = (y,z)\, \right)}
	\end{align}
	で定める.そして$A$から$A/N$への商写像を$\pi$と書く.このとき
	\begin{align}
		g\left(\pi(x)\right) = f(x)
	\end{align}
	で$g$を定めれば,$g$は$A/N$から$f \ast A$への全単射となる.実際,
	$x,y$を$A/N$の要素とすれば
	\begin{align}
		x = \pi(s) \wedge y = \pi(t)
	\end{align}
	を満たす$A$の要素$s,t$が存在し,
	\begin{align}
		g(x) = g(y) \Longrightarrow f(s) = f(t)
		\Longrightarrow (s,t) \in N
		\Longrightarrow x = \pi(s) = \pi(t) = y
	\end{align}
	が成立するので$g$は単射であり,また$z$を$f \ast A$の要素とすれば
	\begin{align}
		z = f(w)
	\end{align}
	を満たす$A$の要素$w$が存在し,
	\begin{align}
		g(\pi(w)) = f(w) = z
	\end{align}
	が成り立つので$g$は全射である.$A/R$上の算法を
	\begin{align}
		\sigma_\pi(\pi(x),\pi(y)) = \pi(\sigma(x,y))
	\end{align}
	で定めば
	\begin{align}
		g \left( \sigma_\pi(\pi(x),\pi(y)) \right) 
		&= g \left( \pi(\sigma(x,y)) \right) \\
		&= f \left( \sigma(x,y) \right) \\
		&= \sigma'(f(x),f(y)) \\
		&= \sigma' \left( g(\pi(x)),g(\pi(y)) \right) \\
	\end{align}
	が成り立つ.すなわち$g$は同型写像である.
	
	
\subsubsection{整数}
	$(S,o)$を可換半群とするとき,$S \times S$上の同値関係を
	\begin{align}
		R \coloneqq \Set{x}{\exists a,b,c,d \in S\, (\, x=((a,b),(c,d))
		\wedge o(a,d) = o(b,c)\, )}
	\end{align}
	で定め,
	\begin{align}
		G \coloneqq S \times S / R
	\end{align}
	とおく.そして$x,y$をSの要素とするとき,$(x,y)$の同値類を$[x,y]$と書く.
	このとき
	\begin{align}
		\sigma \left([x,y],[x',y'] \right) = \left[o(x,x'),o(y,y')\right]
	\end{align}
	で$\sigma$を定めると,$\sigma$は可換律と結合律を満たす.実際,
	\begin{align}
		\sigma \left( [x,y],[x',y'] \right)
		= \left[ o(x,x'), o(y,y') \right]
		= \left[ o(x',x), o(y',y) \right]
		= \sigma \left( [x',y'],[x,y] \right)
	\end{align}
	と
	\begin{align}
		\sigma \left(\sigma \left([x,y],[x',y']\right),[x'',y''] \right)
		&= \sigma \left(\left[ o(x,x'),o(y,y') \right],[x'',y''] \right) \\
		&= \left[ o(o(x,x'),x''), o(o(y,y'),y'') \right] \\
		&= \left[ o(x,o(x',x''), o(y,o(y',y'')) \right] \\
		&= \sigma \left( [x,y], \left[ o(x',x''),o(y',y'') \right] \right) \\
		&= \sigma \left( [x,y], \sigma \left([x',y'],[x'',y'']\right) \right)
	\end{align}
	が成り立つ.それから,$o$の可換律から
	\begin{align}
		o(x,y) = o(y,x)
	\end{align}
	が成り立つので
	\begin{align}
		((x,x),(y,y)) \in R
	\end{align}
	となり,
	\begin{align}
		[x,x] = [y,y]
	\end{align}
	が成立する.そこで
	\begin{align}
		\zeta \coloneqq [x,x]
	\end{align}
	とおく.このとき
	\begin{align}
		\sigma \left( [x,y],\zeta \right)
		= \sigma \left( \zeta,[x,y] \right)
		= [x,y]
	\end{align}
	が満たされる.実際,$\zeta = [z,z]$より
	$\left( [x,y],\zeta \right) = \left( [x,y],[z,z] \right)$となるから
	\begin{align}
		\sigma \left( [x,y],\zeta \right)
		&= \sigma \left( [x,y],[z,z] \right) \\
		&= \left[ o(x,z), o(y,z) \right]
	\end{align}
	となるが,
	\begin{align}
		o(o(x,z),y) = o(x,o(z,y)) = o(x,o(y,z)) = o(o(y,z),x)
	\end{align}
	より$(o(x,z),o(y,z))$と$(x,y)$は同値となるので
	\begin{align}
		\sigma \left( [x,y],\zeta \right) = [x,y]
	\end{align}
	が成立する.同様にして
	\begin{align}
		\sigma \left( \zeta,[x,y] \right) = [x,y]
	\end{align}
	も成立する.
\newpage
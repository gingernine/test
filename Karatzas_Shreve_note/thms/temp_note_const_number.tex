\section{数の構成の一時的なメモ置き場}
	流れを把握していても思うように書けるとは限らない.満足いく体裁で書けるまで
	整理のためにメモだけ置いておく.$\mathcal{L}'$の文法等にはこだわらず大雑把に.幾分か雑.
	
\subsection{商集合の算法}
	商集合に対して,割る前の集合上の算法と整合的な算法を定義する.
	$A$を集合とし,$\sigma$を$A$上の算法とし,$R$を$A$上の同値関係とし,
	$A$から$A/R$への商写像を$\pi$と書く.また
	\begin{align}
		\pi(x) = \pi(x') \wedge \pi(y) = \pi(y')
		\Longrightarrow \pi\left(\sigma(x,y)\right) = \pi\left(\sigma(x',y')\right)
	\end{align}
	が満たされているとする.このとき
	\begin{align}
		\sigma_\pi \defeq \Set{x}{\exists s,t \in A\, 
		\left(\, x=((\pi(s),\pi(t)),\pi(\sigma(s,t)))\, \right)}
	\end{align}
	と定めると,$\sigma_\pi$は$A/R$上の算法となり
	\begin{align}
		\sigma_\pi(\pi(s),\pi(t)) = \pi(\sigma(s,t))
	\end{align}
	を満たす.実際,$(x,y) \in \sigma_\pi$かつ$(x,z) \in \sigma_\pi$であれば,
	$A$の或る要素$s,t,s',t'$が存在して
	\begin{align}
		(x,y) = ((\pi(s),\pi(t)),\pi(\sigma(s,t)))
	\end{align}
	と
	\begin{align}
		(x,z) = ((\pi(s'),\pi(t')),\pi(\sigma(s',t')))
	\end{align}
	を満たすが,
	\begin{align}
		(\pi(s),\pi(t)) = (\pi(s'),\pi(t'))
	\end{align}
	から
	\begin{align}
		\pi(s) = \pi(s') \wedge \pi(t) = \pi(t')
	\end{align}
	が従い,
	\begin{align}
		y = \pi(\sigma(s,t))) = \pi(\sigma(s',t'))) = z
	\end{align}
	となるので$\sigma_\pi$は写像である.また$\sigma$が可換なら$\sigma_\pi$も可換となる.実際,
	\begin{align}
		\sigma_\pi(\pi(x),\pi(y)) = \pi(\sigma(x,y))
		= \pi(\sigma(y,x)) = \sigma_\pi(\pi(y),\pi(x))
	\end{align}
	が成り立つ.同様に$\sigma$が結合的なら$\sigma_\pi$も結合的となる.実際,
	\begin{align}
		x = \pi(s),\quad y = \pi(t),\quad z = \pi(u)
	\end{align}
	のとき
	\begin{align}
		\sigma_\pi\left(\sigma_\pi(x,y),z\right)
		&= \sigma_\pi\left(\pi(\sigma(s,t)),\pi(u)\right) \\
		&= \pi\left(\sigma(\sigma(s,t),u)\right) \\
		&= \pi\left(\sigma(s,\sigma(t,u))\right) \\
		&= \sigma_\pi\left(\pi(s),\pi(\sigma(t,u))\right) \\
		&= \sigma_\pi\left(x,\sigma_\pi(y,z)\right)
	\end{align}
	が成り立つ.$a$を$A$の要素として
	\begin{align}
		\forall x \in A\, (\, \sigma(a,x) = \sigma(x,a) = x\, )
	\end{align}
	を満たすとする.$a$を単位元と呼ぶが,このとき
	\begin{align}
		\sigma_\pi(\pi(x),\pi(a)) = \pi(\sigma(x,a)) = \pi(x)
	\end{align}
	かつ
	\begin{align}
		\sigma_\pi(\pi(a),\pi(x)) = \pi(\sigma(a,x)) = \pi(x)
	\end{align}
	が成り立つので$\pi(a)$は$A/R$の単位元となる.また$A$の要素$x,y,z$に対して
	\begin{align}
		\sigma(x,y) = \sigma(y,x) = z
	\end{align}
	となるとき,
	\begin{align}
		\sigma_\pi(\pi(x),\pi(y)) = \pi(\sigma(x,y)) = \pi(z)
	\end{align}
	かつ
	\begin{align}
		\sigma_\pi(\pi(y),\pi(x)) = \pi(\sigma(y,x)) = \pi(z)
	\end{align}
	が成り立つ.この関係は{\bf 逆元は$\pi$で移した先でも逆元となる}ことを示唆している.
	
\subsection{同型定理}
	$A,A'$を集合,$\sigma,\sigma'$をそれぞれ$A,A'$上の算法とし,
	$f$を$A$から$A'$への写像とする.$f$が
	\begin{align}
		f(\sigma(x,y)) = \sigma'(f(x),f(y))
	\end{align}
	を満たすとき,$f$は$(A,\sigma)$から$(A',\sigma')$への準同型写像であるという.
	ここで$A$上の同値関係を
	\begin{align}
		N \defeq \Set{x}{\exists y,z \in A\, 
		\left(\, f(y) = f(z) \wedge x = (y,z)\, \right)}
	\end{align}
	で定める.そして$A$から$A/N$への商写像を$\pi$と書く.このとき
	\begin{align}
		g\left(\pi(x)\right) = f(x)
	\end{align}
	で$g$を定めれば,$g$は$A/N$から$f \ast A$への全単射となる.実際,
	$x,y$を$A/N$の要素とすれば
	\begin{align}
		x = \pi(s) \wedge y = \pi(t)
	\end{align}
	を満たす$A$の要素$s,t$が存在し,
	\begin{align}
		g(x) = g(y) \Longrightarrow f(s) = f(t)
		\Longrightarrow (s,t) \in N
		\Longrightarrow x = \pi(s) = \pi(t) = y
	\end{align}
	が成立するので$g$は単射であり,また$z$を$f \ast A$の要素とすれば
	\begin{align}
		z = f(w)
	\end{align}
	を満たす$A$の要素$w$が存在し,
	\begin{align}
		g(\pi(w)) = f(w) = z
	\end{align}
	が成り立つので$g$は全射である.$A/R$上の算法を
	\begin{align}
		\sigma_\pi(\pi(x),\pi(y)) = \pi(\sigma(x,y))
	\end{align}
	で定めば
	\begin{align}
		g \left( \sigma_\pi(\pi(x),\pi(y)) \right) 
		&= g \left( \pi(\sigma(x,y)) \right) \\
		&= f \left( \sigma(x,y) \right) \\
		&= \sigma'(f(x),f(y)) \\
		&= \sigma' \left( g(\pi(x)),g(\pi(y)) \right) \\
	\end{align}
	が成り立つ.すなわち$g$は同型写像である.
	
\subsection{算法の移し方}
	$A,A'$を集合とし,$\sigma$を$A$上の算法とし,$h$を$A$から$A'$への全単射とする.
	このとき
	\begin{align}
		\sigma'(x,y) \defeq h\left(\sigma(h^{-1}(x),h^{-1}(y))\right)
	\end{align}
	により$A'$上の算法を定めれば,
	\begin{description}
		\item[(1)] $\sigma$が可換なら$\sigma'$も可換となる.
		\item[(2)] $\sigma$が結合的なら$\sigma'$も結合的となる.
		\item[(3)] $a$が$A$の$\sigma$に関する単位元なら
			$h(a)$は$A'$の$\sigma'$に関する単位元となる.
		\item[(4)] $A$の要素$x$の$\sigma$に関する逆元を$-x$と書けば
			$h(-x)$は$h(x)$の$\sigma'$に関する逆元となる.
			ただし$A$の$\sigma$に関する単位元を$a$とする.
	\end{description}
	
	\begin{prf}\mbox{}
		\begin{description}
			\item[(1)] $x,y$を$A'$の要素とすれば
				\begin{align}
					\sigma'(x,y) &= h\left(\sigma(h^{-1}(x),h^{-1}(y))\right) \\
					&= h\left(\sigma(h^{-1}(y),h^{-1}(x))\right) \\
					&= \sigma'(y,x)
				\end{align}
				が成立する.
				
			\item[(2)] $x,y,z$を$A'$の要素とすれば
				\begin{align}
					\sigma'\left( \sigma'(x,y),z \right)
					&= \sigma'\left( h\left(\sigma(h^{-1}(x),h^{-1}(y))\right),z \right) \\
					&= h\left(\sigma\left(h^{-1}\left( h\left(\sigma\left(h^{-1}(x),h^{-1}(y)\right)\right) \right),h^{-1}(z)\right)\right) \\
					&= h\left(\sigma\left(\sigma\left(h^{-1}(x),h^{-1}(y)\right),h^{-1}(z)\right)\right) \\
					&= h\left(\sigma\left(h^{-1}(x),\sigma\left(h^{-1}(y),h^{-1}(z)\right)\right)\right) \\
					&= h\left(\sigma\left(h^{-1}(x),h^{-1}\left(h\left(\sigma\left(h^{-1}(y),h^{-1}(z)\right)\right)\right)\right)\right) \\
					&= \sigma'\left( x,h\left(\sigma(h^{-1}(y),h^{-1}(z))\right) \right) \\
					&= \sigma'\left(x,\sigma'\left(y,z\right)\right)
				\end{align}
				が成立する.
			
			\item[(3)] $x$を$A'$の要素とすれば
				\begin{align}
					\sigma'(x,h(a)) &= h\left(\sigma(h^{-1}(x),a)\right) \\
					&= h\left(h^{-1}(x)\right) \\
					&= x
				\end{align}
				と
				\begin{align}
					\sigma'(h(a),x) &= h\left(\sigma(a,h^{-1}(x))\right) \\
					&= h\left(h^{-1}(x)\right) \\
					&= x
				\end{align}
				が成立する.
				
			\item[(4)] $x$を$A'$の要素として
				\begin{align}
					y \defeq h\left(-h^{-1}(x)\right)
				\end{align}
				とおけば
				\begin{align}
					\sigma'(x,y) = h\left(\sigma(h^{-1}(x),-h^{-1}(x))\right) = h(a)
				\end{align}
				と
				\begin{align}
					\sigma'(y,x) = h\left(\sigma(-h^{-1}(x),h^{-1}(x))\right) = h(a)
				\end{align}
				が成立する.
		\end{description}
	\end{prf}

\subsection{整数}
	$(S,o)$を可換半群とするとき,$S \times S$上の同値関係を
	\begin{align}
		R \defeq \Set{x}{\exists a,b,c,d \in S\, (\, x=((a,b),(c,d))
		\wedge o(a,d) = o(b,c)\, )}
	\end{align}
	で定め,
	\begin{align}
		G \defeq S \times S / R
	\end{align}
	とおく.そして$x,y$をSの要素とするとき,$(x,y)$の同値類を$[x,y]$と書く.
	このとき
	\begin{align}
		\sigma \left([x,y],[x',y'] \right) = \left[o(x,x'),o(y,y')\right]
	\end{align}
	で$\sigma$を定めると,$\sigma$は可換律と結合律を満たす.実際,
	\begin{align}
		\sigma \left( [x,y],[x',y'] \right)
		= \left[ o(x,x'), o(y,y') \right]
		= \left[ o(x',x), o(y',y) \right]
		= \sigma \left( [x',y'],[x,y] \right)
	\end{align}
	と
	\begin{align}
		\sigma \left(\sigma \left([x,y],[x',y']\right),[x'',y''] \right)
		&= \sigma \left(\left[ o(x,x'),o(y,y') \right],[x'',y''] \right) \\
		&= \left[ o(o(x,x'),x''), o(o(y,y'),y'') \right] \\
		&= \left[ o(x,o(x',x''), o(y,o(y',y'')) \right] \\
		&= \sigma \left( [x,y], \left[ o(x',x''),o(y',y'') \right] \right) \\
		&= \sigma \left( [x,y], \sigma \left([x',y'],[x'',y'']\right) \right)
	\end{align}
	が成り立つ.それから,$o$の可換律から
	\begin{align}
		o(x,y) = o(y,x)
	\end{align}
	が成り立つので
	\begin{align}
		((x,x),(y,y)) \in R
	\end{align}
	となり,
	\begin{align}
		[x,x] = [y,y]
	\end{align}
	が成立する.そこで
	\begin{align}
		\zeta \defeq [x,x]
	\end{align}
	とおく.このとき
	\begin{align}
		\sigma \left( [x,y],\zeta \right)
		= \sigma \left( \zeta,[x,y] \right)
		= [x,y]
	\end{align}
	が満たされる.実際,$\zeta = [z,z]$より
	$\left( [x,y],\zeta \right) = \left( [x,y],[z,z] \right)$となるから
	\begin{align}
		\sigma \left( [x,y],\zeta \right)
		&= \sigma \left( [x,y],[z,z] \right) \\
		&= \left[ o(x,z), o(y,z) \right]
	\end{align}
	となるが,
	\begin{align}
		o(o(x,z),y) = o(x,o(z,y)) = o(x,o(y,z)) = o(o(y,z),x)
	\end{align}
	より$(o(x,z),o(y,z))$と$(x,y)$は同値となるので
	\begin{align}
		\sigma \left( [x,y],\zeta \right) = [x,y]
	\end{align}
	が成立する.同様にして
	\begin{align}
		\sigma \left( \zeta,[x,y] \right) = [x,y]
	\end{align}
	も成立する.また$[x,y]$に対しては$[y,x]$が
	\begin{align}
		\sigma([x,y],[y,x]) = \sigma([y,x],[x,y]) = \zeta
	\end{align}
	を満たす.そこで$[y,x]$を$[x,y]$の逆元と呼び
	\begin{align}
		-[x,y] \defeq [y,x]
	\end{align}
	とおく.$a$を$S$の要素として
	\begin{align}
		\varphi(x) = [o(x,a),a]
	\end{align}
	で$S$から$G$への写像$\varphi$を定めるとき,
	\begin{align}
		\varphi(o(x,y)) = [o(o(x,y),a),a]
		&= \sigma \left( [o(o(x,y),a),a],[a,a] \right) \\
		&= [o(o(o(x,y),a),a),o(a,a)] \\
		&= [o(o(o(x,y),a),a),o(a,a)] \\
		&= [o(o(x,o(y,a)),a),o(a,a)] \\
		&= [o(o(o(y,a),x),a),o(a,a)] \\
		&= [o(o(y,a),o(x,a)),o(a,a)] \\
		&= \sigma \left( [o(y,a),a],[o(x,a),a] \right) \\
		&= \sigma \left( [o(x,a),a],[o(y,a),a] \right) \\
		&= \sigma \left( \varphi(x),\varphi(y) \right)
	\end{align}
	及び
	\begin{align}
		[x,y] &= [o(x,o(a,a)),o(y,o(a,a))] \\
		&= [o(o(x,a),a),o(o(y,a),a)] \\
		&= [o(o(x,a),a),o(o(y,a),a)] \\
		&= [o(o(x,a),a),o(a,o(y,a))] \\
		&= \sigma \left( [o(x,a),a],[a,o(y,a)] \right) \\
		&= \sigma \left( \varphi(x),-\varphi(y) \right)
	\end{align}
	が成立する.特に,$o$が簡約律を満たすなら$\varphi$は単射となる.実際,
	\begin{align}
		\varphi(x) = \varphi(y)
		&\Longrightarrow [o(x,a),a] = [o(y,a),a] \\
		&\Longrightarrow o(o(x,a),a) = o(o(y,a),a) \\
		&\Longrightarrow o(x,o(a,a)) = o(x,o(a,a)) \\
		&\Longrightarrow x = y
	\end{align}
	となる.いま$o$を簡約律と可換律を満たすとして,
	\begin{align}
		\tilde{G} \defeq \left( G \backslash (\varphi \ast S) \right) \cup S
	\end{align}
	とおいて
	\begin{align}
		h(x) = 
		\begin{cases}
			x & (x \notin \varphi \ast S) \\
			\varphi^{-1}(x) & (x \in \varphi \ast S)
		\end{cases}
	\end{align}
	とおくと,$h$は$G$から$\tilde{G}$への全単射となる.そして$\tilde{G}$上の算法を
	\begin{align}
		\tilde{\sigma}(x,y) = h\left(\sigma\left(h^{-1}(x),h^{-1}(y)\right)\right)
	\end{align}
	で定めると$h$は同型写像となる.また
	\begin{align}
		-x \defeq h\left( -h^{-1}(x) \right)
	\end{align}
	と書く.このとき,$x,y$を$S$の要素とすれば
	\begin{align}
		h^{-1}(x) = \varphi(x) \wedge h^{-1}(y) = \varphi(y)
	\end{align}
	となるので
	\begin{align}
		\tilde{\sigma}(x,y) = h\left(\sigma\left(\varphi(x),\varphi(y)\right)\right)
		= h\left(\varphi(o(x,y))\right)
		= o(x,y)
	\end{align}
	が満たされる.すなわち{\bf $\tilde{\sigma}$は$o$の拡張となっている}.
	また$\tilde{G}$の任意の要素$x$は,或る$S$の要素$y,z$によって
	\begin{align}
		x = h([y,z])
	\end{align}
	と書けるが,このとき
	\begin{align}
		x &= h\left(\sigma(\varphi(y),-\varphi(z))\right) \\
		&= \tilde{\sigma} \left(h(\varphi(y)),h(-\varphi(z))\right) \\
		&= \tilde{\sigma} \left(h(\varphi(y)),-h(\varphi(z))\right) \\
		&= \tilde{\sigma}(y,-z)
	\end{align}
	が成立するので,{\bf $\tilde{\sigma}$の要素は$S$の要素に対する演算で表せる}.
	この$(\tilde{G},\tilde{\sigma})$を{\bf $S$が生成する群}と呼ぶ.
	$\omg$には加法と乗法が定まっているが,その最小の拡張となる環が整数環である.
	
	整数環の性質:
	\begin{itemize}
		\item 整数環は順序環である.
		\item 整数環はEuclid整域である.
		\item 任意の環に対して整数環からの準同型が存在する.
	\end{itemize}
	
\subsection{有理数}
	有理数体は整数環の分数体であるから分数体の構成法をメモしておく.
	$(R,\sigma,\mu)$を整域として,その零元と単位元をそれぞれ$\zeta,\epsilon$で表す.
	また$R \times R \backslash \{\zeta\}$上の同値関係を
	\begin{align}
		\Phi \defeq \Set{x}{\exists a,c \in R\, \exists b,d \in R \backslash \{\zeta\}\, \left(\, x=((a,b),(c,d)) \wedge \mu(a,d) = \mu(b,c)\, \right)}
	\end{align}
	で定め,
	\begin{align}
		Q \defeq (R \times R \backslash \{\zeta\}) / \Phi
	\end{align}
	とおく.そして$(x,y) \in R \times R \backslash \{\zeta\}$の同値類を$[x,y]$と書く.
	$R \times R \backslash \{\zeta\}$上の算法を
	\begin{align}
		\sigma_P &\defeq \Set{x}{\exists a,c \in R\, \exists b,d \in R \backslash \{\zeta\}\, \left(\, x=(((a,b),(c,d)),(\sigma(\mu(a,c),\mu(b,c)),\mu(b,d)))\, \right)}, \\
		\mu_P &\defeq \Set{x}{\exists a,c \in R\, \exists b,d \in R \backslash \{\zeta\}\, \left(\, x=(((a,b),(c,d)),(\mu(a,c),\mu(b,d)))\, \right)}
	\end{align}
	で定める.煩雑でわかりづらいが,これは分数の計算法則
	\begin{align}
		a/b + c/d = (ad + bd)/(bd),\quad a/b \cdot c/d = (ac)/(bd)
	\end{align}
	を一般形式化したものに過ぎない.
	\begin{align}
		((a,b),(a',b')) \in \Phi \wedge ((c,d),(c',d')) \in \Phi
	\end{align}
	が成り立っているとき
	\begin{align}
		\left(\sigma_P((a,b),(c,d)),\sigma_P((a',b'),(c',d'))\right) \in \Phi
	\end{align}
	が成り立つ.実際
	\begin{align}
		\mu(\mu(a,d),\mu(b',d')) &= \mu(\mu(\mu(a,d),b'),d') \\
		&= \mu(\mu(\mu(d,a),b'),d') \\
		&= \mu(\mu(d,\mu(a,b')),d') \\
		&= \mu(\mu(d,\mu(a',b)),d') \\
		&= \mu(\mu(d,\mu(a',b)),d') \\
		&= \mu(\mu(d,\mu(b,a')),d') \\
		&= \mu(\mu(\mu(d,b),a'),d') \\
		&= \mu(\mu(d,b),\mu(a',d')) \\
		&= \mu(\mu(a',d'),\mu(d,b)) \\
		&= \mu(\mu(a',d'),\mu(b,d)) \\
	\end{align}
	かつ
	\begin{align}
		\mu(\mu(c,b),\mu(b',d')) &= \mu(\mu(b,c),\mu(b',d')) \\
		&= \mu(b,\mu(c,\mu(b',d'))) \\
		&= \mu(b,\mu(c,\mu(d',b'))) \\
		&= \mu(b,\mu(\mu(c,d'),b')) \\
		&= \mu(b,\mu(\mu(c',d),b')) \\
		&= \mu(b,\mu(\mu(d,c'),b')) \\
		&= \mu(b,\mu(d,\mu(c',b'))) \\
		&= \mu(\mu(b,d),\mu(c',b')) \\
		&= \mu(\mu(c',b'),\mu(b,d)) \\
	\end{align}
	が成り立つから
	\begin{align}
		\mu\left(\sigma(\mu(a,d),\mu(c,b)),\mu(b',d')\right)
		&= \sigma\left(\mu(\mu(a,d),\mu(b',d')),\mu(\mu(c,b),\mu(b',d'))\right) \\
		&= \sigma\left(\mu(\mu(a',d'),\mu(b,d)),\mu(\mu(c',b'),\mu(b,d))\right) \\
		&= \mu\left(\sigma(\mu(a',d'),\mu(c',b')),\mu(b,d)\right)
	\end{align}
	が満たされる.同時に
	\begin{align}
		\left(\mu_P((a,b),(c,d)),\mu_P((a',b'),(c',d'))\right) \in \Phi
	\end{align}
	も満たされる.なぜならば,
	\begin{align}
		\mu(\mu(a,c),\mu(b',d')) &= \mu(a,\mu(c,\mu(b',d'))) \\
		&= \mu(a,\mu(c,\mu(d',b'))) \\
		&= \mu(a,\mu(\mu(c,d'),b')) \\
		&= \mu(a,\mu(b',\mu(c,d'))) \\
		&= \mu(\mu(a,b'),\mu(c,d')) \\
		&= \mu(\mu(a',b),\mu(c',d)) \\
		&= \mu(\mu(a',c'),\mu(b,d))
	\end{align}
	が成り立つからである.
	以上より$Q$上の算法は$\sigma_P,\mu_P$から整合的に定められる.それらを
	$\sigma_Q,\mu_Q$と書けば,$\sigma_P,\mu_P$はそれぞれ可換かつ結合的であるから
	$\sigma_Q,\mu_Q$もそれらの性質を持つ.
	\begin{align}
		(\zeta,\epsilon)
	\end{align}
	は$\sigma_P$に関する単位元であり,
	\begin{align}
		(\epsilon,\epsilon)
	\end{align}
	は$\mu_P$に関する単位元である.実際,
	\begin{align}
		&\sigma_P((x,y),(\zeta,\epsilon))
		= (\sigma(\mu(x,\epsilon),\mu(\zeta,y)),\mu(y,\epsilon))
		= (\sigma(x,\zeta),y)
		= (x,y), \\
		&\sigma_P((\zeta,\epsilon),(x,y))
		= (\sigma(\mu(\zeta,y),\mu(x,\epsilon)),\mu(\epsilon,y))
		= (\sigma(\zeta,x),y)
		= (x,y)
	\end{align}
	が成り立ち,また
	\begin{align}
		&\mu_P((x,y),(\epsilon,\epsilon)) = (\mu(x,\epsilon),\mu(y,\epsilon)) = (x,y), \\
		&\mu_P((\epsilon,\epsilon),(x,y)) = (\mu(\epsilon,x),\mu(\epsilon,y)) = (x,y)
	\end{align}
	も成り立つ.よって
	\begin{align}
		\zeta_Q \defeq [\zeta,\epsilon]
	\end{align}
	とおけば$\zeta_Q$は$\sigma_Q$に関する単位元となり,
	\begin{align}
		\epsilon_Q \defeq [\epsilon,\epsilon]
	\end{align}
	とおけば$\epsilon_Q$は$\mu_Q$に関する単位元となる.
	$x$と$y$を$R \times R \backslash \{\zeta\}$の要素とすれば
	\begin{align}
		[x,y] = \zeta_Q \Longleftrightarrow x = \zeta
	\end{align}
	が満たされるので,
	\begin{align}
		[x,y] \neq \zeta_Q
	\end{align}
	ならば
	\begin{align}
		(y,x) \in R \times R \backslash \{\zeta\}
	\end{align}
	となり,
	\begin{align}
		\mu_P([x,y],[y,x]) = [\mu(x,y),\mu(y,x)] = [\mu(x,y),\mu(x,y)] = \epsilon_Q
	\end{align}
	かつ
	\begin{align}
		\mu_P([y,x],[x,y]) = [\mu(y,x),\mu(x,y)] = [\mu(x,y),\mu(x,y)] = \epsilon_Q
	\end{align}
	が成立する.従って{\bf $Q$の非零元は可逆である}.$R$から$Q$への環準同型を
	\begin{align}
		\varphi(x) = [x,\epsilon]
	\end{align}
	で定める.このとき$\varphi$は単射である.実際,
	\begin{align}
		\varphi(x) = \varphi(y) &\Longrightarrow [x,\epsilon] = [y,\epsilon]
		&\Longrightarrow x = \mu(x,\epsilon) = \mu(\epsilon,y) = y
	\end{align}
	となる.
	\begin{align}
		\tilde{Q} \defeq (Q \backslash \varphi \ast R) \cup R
	\end{align}
	とおいて
	\begin{align}
		h(x) = 
		\begin{cases}
			x & (x \notin \varphi \ast R) \\
			\varphi^{-1}(x) & (x \in \varphi \ast R)
		\end{cases}
	\end{align}
	で$Q$から$\tilde{Q}$への全単射を定め,$Q$の算法を$h$により$\tilde{Q}$に移し,それらを
	$\tilde{\sigma},\tilde{\mu}$と書けば,$(\tilde{Q},\tilde{\sigma},\tilde{\mu})$は体となる.
	\begin{align}
		R &\subset \tilde{Q}, \\
		\sigma &\subset \tilde{\sigma}, \\
		\mu &\subset \tilde{\mu}
	\end{align}
	となるので$(\tilde{Q},\tilde{\sigma},\tilde{\mu})$は$(R,\sigma,\mu)$の純粋な拡張であり,
	$x$を$\tilde{Q}$の任意の要素とすれば$R$の或る要素$s,t$が存在して
	\begin{align}
		h^{-1}(x) = [s,t] = \mu_P([s,\epsilon],[\epsilon,t]) = \mu_P(\varphi(s),\varphi(t)^{-1})
	\end{align}
	となり,
	\begin{align}
		t^{-1} \defeq h(\varphi(t)^{-1})
	\end{align}
	とおけば
	\begin{align}
		x = \tilde{\mu}(s,t^{-1})
	\end{align}
	と書ける.すなわち$(\tilde{Q},\tilde{\sigma},\tilde{\mu})$は
	\begin{align}
		R &\subset \tilde{Q}, \\
		\sigma &\subset \tilde{\sigma}, \\
		\mu &\subset \tilde{\mu}
	\end{align}
	を満たす最小の体である.これを{\bf 整域$(R,\sigma,\mu)$の分数体}と呼ぶ.分数体と名付けられたる所以は
	\begin{align}
		s/t \defeq \tilde{\mu}(s,t^{-1})
	\end{align}
	と表記すれば明らかである.

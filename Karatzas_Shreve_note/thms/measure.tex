\section{測度}
	\subsection{Lebesgue拡大}
		\begin{screen}
			\begin{thm}[最小の完備拡大が取れる]
				$(X,\mathscr{A},\mu)$を正値測度空間とする.このとき
				\begin{align}
					\mathscr{B} \defeq
					\Set{B}{B \subset X \wedge \exists A,C \in \mathscr{A}\,
					\left(\, A \subset B \subset C \wedge \mu(C \backslash A)=0\, \right) }
				\end{align}
				により定める$\mathscr{B}$は$X$上の$\sigma$-加法族となる.また
				$B$を$\mathscr{B}$の要素とすると
				\begin{align}
					A \subset B \subset C \wedge \mu(C \backslash A)=0
				\end{align}
				なる$\mathscr{A}$の要素$A$と$C$が取れるが,$B$に対して
				$\mu(A)$を対応させる関係,つまり
				\begin{align}
					\Set{(x,y)}{x \in \mathscr{B} \wedge \exists a,c \in \mathscr{A}\, 
					\left(\, a \subset x \subset c \wedge y = \mu(a)\, \right)}
				\end{align}
				なる関係を$\nu$とすれば,$\nu$は$\mathscr{B}$上の完備な正値測度である.
				そして$(X,\mathscr{C},\lambda)$を
				\begin{align}
					\mathscr{A} \subset \mathscr{C} \wedge \mu \subset \lambda
				\end{align}
				を満たす完備な正値測度空間とすると,
				\begin{align}
					\mathscr{B} \subset \mathscr{C} \wedge \nu \subset \lambda
				\end{align}
				が成立する.つまり$(X,\mathscr{B},\nu)$は$(X,\mathscr{A},\mu)$の完備拡大のうちで最小である.
			\end{thm}
		\end{screen}
		
		\begin{sketch}\mbox{}
			\begin{description}
				\item[第一段] $\mathscr{B}$が$\sigma$-加法族であることを示す.まず
					\begin{align}
						X \in \mathscr{A}
					\end{align}
					と
					\begin{align}
						X \subset X \subset X
					\end{align}
					と
					\begin{align}
						\mu(X \backslash X) = \mu(\emptyset) = 0
					\end{align}
					が成り立つので$X$は$\mathscr{B}$に属する.$B$を$\mathscr{B}$の要素とすれば
					\begin{align}
						A \subset B \subset C \wedge \mu(C \backslash A)=0
					\end{align}
					なる$\mathscr{A}$の要素$A$と$C$が取れて,このとき
					\begin{align}
						(X \backslash C) \subset (X \backslash B) \subset (X \backslash A) 
					\end{align}
					と
					\begin{align}
						\mu\left((X \backslash A)\backslash (X \backslash C)\right) 
						= \mu(C \backslash A) 
						= 0
					\end{align}
					が成り立つから
					\begin{align}
						X \backslash B \in \mathscr{B}
					\end{align}
					が成り立つ.つまり$\mathscr{B}$は補演算で閉じる.
					$\{B_n\}_{n \in \Natural}$を$\mathscr{B}$の部分集合とすれば,各自然数$n$で
					\begin{align}
						A_n \subset B_n \subset C_n \wedge \mu(C_n \backslash A_n)=0
					\end{align}
					を満たす$\mathscr{A}$の要素$A_n$と$C_n$が取れる.このとき
					\begin{align}
						\bigcup_{n \in \Natural} A_n 
						\subset \bigcup_{n \in \Natural} B_n
						\subset \bigcup_{n \in \Natural} C_n
					\end{align}
					が成り立ち,また
					\begin{align}
						\bigcup_{n \in \Natural} C_n \backslash \bigcup_{n \in \Natural} A_n
						\subset \bigcup_{n \in \Natural} (C_n \backslash A_n)
					\end{align}
					より
					\begin{align}
						\mu\left( \bigcup_{n \in \Natural} C_n \backslash \bigcup_{n \in \Natural} A_n \right)
						\leq \sum_{n \in \Natural} \mu(C_n \backslash A_n)
						= 0
					\end{align}
					も成り立つから
					\begin{align}
						\bigcup_{n \in \Natural} B_n \in \mathscr{B}
					\end{align}
					も成立する.ゆえに$\mathscr{B}$は可算合併でも閉じる.ゆえに$\mathscr{B}$は$\sigma$-加法族である.
					
				\item[第二段]
					$\nu$が写像であることを示す.$(x,y)$と$(x,z)$を$\nu$の要素とする.このとき
					\begin{align}
						x = B
					\end{align}
					なる$\mathscr{B}$の要素$B$が取れて,また
					\begin{align}
						A_1 \subset B \subset C_1 \wedge \mu(C_1 \backslash A_1)=0
					\end{align}
					なる$\mathscr{A}$の要素$A_1$と$C_1$と,
					\begin{align}
						A_2 \subset B \subset C_2 \wedge \mu(C_2 \backslash A_2)=0
					\end{align}
					なる$\mathscr{A}$の要素$A_2$と$C_2$も取れる.そして
					\begin{align}
						A_1 \subset B \subset C_2
					\end{align}
					から
					\begin{align}
						\mu(A_1) \leq \mu(C_2) = \mu(A_2)
					\end{align}
					が成り立ち,
					\begin{align}
						A_2 \subset B \subset C_1
					\end{align}
					から
					\begin{align}
						\mu(A_2) \leq \mu(C_1) = \mu(A_1)
					\end{align}
					も成り立つので,
					\begin{align}
						y = \mu(A_1) = \mu(A_2) = z
					\end{align}
					が成立する.ゆえに$\nu$は写像である.
					
				\item[第三段] $\nu$の定義域が$\mathscr{B}$に等しいことを示す.
					$x$を$\dom{\nu}$の要素とすると,
					\begin{align}
						x = B
					\end{align}
					なる$\mathscr{B}$の要素$B$が取れるので
					\begin{align}
						x \in \mathscr{B}
					\end{align}
					が従う.$x$を$\mathscr{B}$の要素とすると,
					\begin{align}
						A \subset x \subset C \wedge \mu(C \backslash A)=0
					\end{align}
					なる$\mathscr{A}$の要素$A$と$C$が取れるので
					\begin{align}
						(x,\mu(A)) \in \nu
					\end{align}
					が成り立ち
					\begin{align}
						x \in \dom{\nu}
					\end{align}
					が従う.$x$の任意性から
					\begin{align}
						\dom{\nu} = \mathscr{B}
					\end{align}
					が得られる.
				
				\item[第四段]	$(X,\mathscr{B},\nu)$が$(X,\mathscr{A},\mu)$の完備拡大のうちで最小であることを示す.
					$B$を$\mathscr{B}$の要素とすると
					\begin{align}
						A \subset x \subset C \wedge \mu(C \backslash A)=0
					\end{align}
					なる$\mathscr{A}$の要素$A$と$C$が取れる.ここで
					\begin{align}
						\mu \subset \lambda
					\end{align}
					より
					\begin{align}
						C \backslash A
					\end{align}
					は$\lambda$-零集合なので,$\lambda$の完備性より
					\begin{align}
						B \backslash A \in \mathscr{C}
					\end{align}
					が成立する.よって
					\begin{align}
						B = B \cup (B \backslash A) \in \mathscr{C}
					\end{align}
					が成立する.よって
					\begin{align}
						\mathscr{B} \subset \mathscr{C}
					\end{align}
					が成立する.また
					\begin{align}
						\nu(B) = \mu(A) = \lambda(A) = \lambda(A) + \lambda(B \backslash A) = \lambda(B)
					\end{align}
					が成立するので
					\begin{align}
						\nu \subset \lambda
					\end{align}
					も成立する.
					\QED
			\end{description}
		\end{sketch}
		
		\begin{screen}
			\begin{dfn}[Lebesgue拡大]
				$(X,\mathscr{A},\mu)$を正値測度空間とするとき,その最小の完備拡大として取れる
				正値測度空間を$(X,\mathscr{A},\mu)$の{\bf Lebesgue拡大}\index{Lebesgueかくだい@Lebesgue拡大}と呼ぶ.
				Lebesgue拡大は往々にして
				\begin{align}
					\left( X,\overline{\mathscr{A}},\overline{\mu} \right)
				\end{align}
				のように上線を載せて書かれる.
			\end{dfn}
		\end{screen}
		
		\begin{screen}
			\begin{thm}[Lebesgue拡大の可測集合族の対称差による表現]
				$(X,\mathscr{A},\mu)$を正値測度空間とし,
				$\left( X,\overline{\mathscr{A}},\overline{\mu} \right)$をそのLebesgue拡大とする.
				このとき
				\begin{align}
					\tilde{\mathscr{A}} \defeq \Set{B}{B \subset X \wedge \exists A,N \in \mathscr{A}\, 
					\left(\, \mu(N) = 0 \wedge B \backslash A \subset N \wedge
					A \backslash B \subset N\, \right)}
				\end{align}
				は$\overline{\mathscr{A}}$に一致する.
			\end{thm}
		\end{screen}
		
		\begin{sketch}
			$B$を$\overline{\mathscr{A}}$の要素とすると
			\begin{align}
				A \subset B \subset C \wedge \mu(C \backslash A)=0
			\end{align}
			なる$\mathscr{A}$の要素$A$と$C$が取れる.
			\begin{align}
				N \defeq C \backslash A
			\end{align}
			とおけば,
			\begin{align}
				\mu(N) = 0
			\end{align}
			かつ
			\begin{align}
				B \backslash A \subset N
			\end{align}
			かつ
			\begin{align}
				A \backslash B = \emptyset \subset N
			\end{align}
			が成り立つので
			\begin{align}
				B \in \tilde{\mathscr{A}}
			\end{align}
			が成立する.次に$B$を$\tilde{\mathscr{A}}$の要素とすると,
			\begin{align}
				\mu(N) = 0 \wedge B \backslash A \subset N \wedge A \backslash B \subset N
			\end{align}
			なる$\mathscr{A}$の要素$A$と$N$が取れる.ここで
			\begin{align}
				\forall x \in A\, (\, x \notin B \Longrightarrow x \in N\, )
			\end{align}
			が成り立つから,
			\begin{align}
				A \backslash N \subset B
			\end{align}
			が成立する.他方で
			\begin{align}
				\forall x \in B\, (\, x \notin A \Longrightarrow x \in N\, )
			\end{align}
			が成り立つから,
			\begin{align}
				B \subset A \cup N
			\end{align}
			が成立する.また
			\begin{align}
				(A \cup N) \backslash (A \backslash N)
				= (A \cup N) \cap ((X \backslash A) \cup N)
			\end{align}
			であるから,$x$を$(A \cup N) \backslash (A \backslash N)$の要素とすると
			\begin{align}
				x \in A
			\end{align}
			ならば
			\begin{align}
				x \notin A \vee x \in N
			\end{align}
			と併せて
			\begin{align}
				x \in N
			\end{align}
			が成り立つ.ゆえに
			\begin{align}
				(A \cup N) \backslash (A \backslash N) \subset N
			\end{align}
			が成り立つ.ゆえに
			\begin{align}
				A \backslash N \subset B \subset A \cup N \wedge
				\mu\left( (A \cup N) \backslash (A \backslash N) \right) = 0
			\end{align}
			が満たされ
			\begin{align}
				B \in \overline{\mathscr{A}}
			\end{align}
			となる.以上より
			\begin{align}
				\overline{\mathscr{A}} = \tilde{\mathscr{A}}
			\end{align}
			が得られた.
			\QED
		\end{sketch}
		
		\begin{screen}
			\begin{lem}[可分値写像による可測写像の一様近似]\label{lem:approximation_of_countably_valued_mappings_on_dist_space}
				$(X,\mathcal{B},\mu)$を正値測度空間とし,$(S,d)$を可分距離空間とし,
				$f$を$\mathcal{B}/\borel{S}$-可測写像とする.このとき
				$S$の可算稠密集合に値を取る$\mathcal{B}/\borel{S}$-可測写像列で,$f$に一様収束するものが取れる.
			\end{lem}
		\end{screen}
		
		\begin{prf}
			$\{a_k\}_{k=1}^\infty$を$S$の可算稠密な部分集合をとする.
			任意の$n \geq 1$に対し
			\begin{align}
				B_n^k \coloneqq \Set{s \in S}{d(s,a_k) < \frac{1}{n}},
				\quad A_n^k \coloneqq f^{-1}\left( B_n^k \right);
				\quad (k=1,2,\cdots)
			\end{align}
			とおけば,
			\begin{align}
				\bigcup_{k=1}^\infty A_n^k 
				= \bigcup_{k=1}^\infty f^{-1}\left( B_n^k \right)
				= f^{-1}(S)
			\end{align}
			より$X = \bigcup_{k=1}^\infty A_n^k$が成り立つ.ここで
			\begin{align}
				\tilde{A}_n^1 \coloneqq A_n^1,
				\quad \tilde{A}_n^k \coloneqq A_n^k \left\backslash \Biggl( \bigcup_{i=1}^{k-1} A_n^i \Biggr)\right.;
				\quad (k=1,2,\cdots)
			\end{align}
			として
			\begin{align}
				 f_n(x) \coloneqq a_k, \quad (x \in \tilde{A}_n^k,\ k=1,2,\cdots)
			\end{align}
			により$\mathcal{B}/\borel{S}$-可測写像列$(f_n)_{n=1}^\infty$を定めれば,
			\begin{align}
				d\left(f_n(x),f(x)\right) < \frac{1}{n},
				\quad (\forall x \in X)
			\end{align}
			が満たされる.
			\QED
		\end{prf}
		
		\begin{screen}
			\begin{thm}[拡大前後の可測性]\label{thm:measurability_before_after_Lebesgue_extension}
				$(X,\mathcal{B},\mu)$を測度空間,そのLebesgue拡大を
				$\left(X,\overline{\mathcal{B}},\overline{\mu}\right)$と書き,
				$(S,d)$を可分距離空間とする.
				このとき,任意の写像$f:X \longrightarrow S$に対し次は同値である:
				\begin{description}
					\item[(a)] 或る$\mathcal{B}/\borel{S}$-可測写像$g$が存在して$\mu$-a.e.に$f = g$となる.
					\item[(b)] $f$は$\overline{\mathcal{B}}/\borel{S}$-可測である.
				\end{description}
			\end{thm}
		\end{screen}
		
		\begin{prf}\mbox{}
			\begin{description}
				\item[第一段]
					$(a)$が成立しているとき,
					$\{f \neq g\} \subset N$を満たす
					$\mu$-零集合$N \in \mathcal{B}$が存在して
					\begin{align}
						f^{-1}(E) \cap \left( g^{-1}(E) \right)^c \subset N,
						\quad g^{-1}(E) \cap \left( f^{-1}(E) \right)^c \subset N,
						\quad (\forall E \in \borel{S})
					\end{align}
					が成り立つから,(\refeq{eq:appendix_Lebesgue_expansion_note_1})より
					$f^{-1}(E) \in \overline{\mathcal{B}}$が従い$(a) \Rightarrow (b)$が出る.
					
				\item[第二段]
					$f$が$\overline{\mathcal{B}}/\borel{S}$-可測のとき,
					$S$の可算稠密な部分集合を$\{a_k\}_{k=1}^\infty$とすれば,
					補題\ref{lem:approximation_of_countably_valued_mappings_on_dist_space}より
					\begin{align}
						f_n(x) = a_k, \ (x \in A_n^k,\ k=1,2,\cdots);
						\quad \sum_{k=1}^\infty A_n^k = X;
						\quad d\left(f_n(x),f(x)\right) < \frac{1}{n},\ (\forall x \in X)
					\end{align}
					を満たす$\overline{\mathcal{B}}/\borel{S}$-可測写像列$(f_n)_{n=1}^\infty$と
					互いに素な集合$\left\{A_n^k\right\}_{k=1}^\infty \subset \overline{\mathcal{B}}$が存在する.
					各$A_n^k$に対し
					\begin{align}
						E_{1,n}^k \subset A_n^k \subset E_{2,n}^k,
						\quad \mu\left(E_{2,n}^k- E_{1,n}^k\right) = 0
					\end{align}
					を満たす$E_{1,n}^k,E_{2,n}^k \in \mathcal{B}$が存在するから,
					一つ$a_0 \in S$を選び
					\begin{align}
						g_n(x) \coloneqq 
						\begin{cases}
							a_k, & (x \in E_{1,n}^k,\ k=1,2,\cdots), \\
							a_0, & (x \in N_n \coloneqq X \backslash \sum_{k=1}^\infty E_{1,n}^k)
						\end{cases}
					\end{align}
					で$\mathcal{B}/\borel{S}$-可測写像列$(g_n)_{n=1}^\infty$を定めて
					$N \coloneqq \bigcup_{n=1}^\infty N_n$とおけば
					\begin{align}
						f_n(x) = g_n(x),
						\quad (\forall x \in X \backslash N,\ \forall n \geq 1)
					\end{align}
					が成り立つ.このとき
					$X \backslash N$上で$\lim_{n \to \infty} g_n(x)$は存在し$\lim_{n \to \infty} f_n(x)=f(x)$に一致するから,
					\begin{align}
						g(x) \coloneqq 
						\begin{cases}
							\displaystyle\lim_{n \to \infty} g_n(x), & (x \in X \backslash N), \\
							a_0, & (x \in N)
						\end{cases}
					\end{align}
					により$\mathcal{B}/\borel{S}$-可測写像$g$を定めれば(a)が満たされる.
					\QED
			\end{description}
		\end{prf}
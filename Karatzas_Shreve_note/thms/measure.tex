\section{測度}
	\subsection{Lebesgue拡大}
		\begin{screen}
			\begin{dfn}[Lebesgue拡大]
				$(X,\mathcal{B},\mu)$を測度空間とするとき,
				\begin{align}
					\overline{\mathcal{B}} &\coloneqq
					\Set{B \subset X}{\exists A_1,A_2 \in \mathcal{B},\ \mbox{s.t.}\quad A_1 \subset B \subset A_2,\ \mu(A_2 - A_1)=0 }, \\
					\overline{\mu}(B) &\coloneqq \mu(A_1) \quad (\forall B \in \overline{\mathcal{B}},\ \mbox{$A_1$ as in above})
				\end{align}
				により得られる完備測度空間$\left( X,\overline{\mathcal{B}},\overline{\mu} \right)$を
				$(X,\mathcal{B},\mu)$のLebesgue拡大と呼ぶ.
			\end{dfn}
		\end{screen}
		$\overline{\mu}$はwell-definedである.実際,$B \subset X$に対し
		$A_1,A_2,B_1,B_2 \in \mathcal{B}$が
		\begin{align}
			&A_1 \subset B \subset A_2, \quad \mu(A_2 - A_1) = 0, \\
			&B_1 \subset B \subset B_2, \quad \mu(B_2 - B_1) = 0,
		\end{align}
		を満たすとき,$A_1 \cup B_1 \subset B \subset A_2 \cap B_2$となるが,
		\begin{align}
			(A_2 \cap B_2) \cap (A_1 \cup B_1)^c
			\subset A_2 \backslash A_1
		\end{align}
		より$\mu(A_1 \cup B_1) = \mu(A_2 \cap B_2)$が従い
		\begin{align}
			\mu(A_2) &= \mu(A_1) \leq \mu(A_1 \cup B_1) = \mu(A_2 \cap B_2) \leq \mu(B_2), \\
			\mu(B_2) &= \mu(B_1) \leq \mu(A_1 \cup B_1) = \mu(A_2 \cap B_2) \leq \mu(A_2)
		\end{align}
		が成り立つから$\mu(A_2) = \mu(B_2)$が出る.
		また,任意の$B \subset X$について
		\begin{align}
			\overline{\mathcal{B}}
			= \Set{B \subset X}{\exists A,N \in \mathcal{B},\ \mbox{s.t.}\quad \mu(N)=0,
			\ B \cap A^c, A \cap B^c \subset N}
			\label{eq:appendix_Lebesgue_expansion_note_1}
		\end{align}
		が成立する.実際,$B \in \overline{\mathcal{B}}$なら
		$A_1 \subset B \subset A_2$かつ$\mu(A_2 - A_1) = 0$を満たす$A_1,A_2 \in \mathcal{B}$が存在するから
		\begin{align}
			A = A_2, \quad N = A_2 - A_1
		\end{align}
		として$(\subset)$を得る.逆に右辺を満たす$A,N$が存在するとき,
		\begin{align}
			A \cap N^c &\subset A \cap B \subset B 
			\subset A \cup (A^c \cap B)
			\subset A \cup N
		\end{align}
		より$A_1 = A\cap N^c,\ A_2 = A \cup N$として$(\supset)$を得る.
	
		\begin{screen}
			\begin{thm}[拡大前後の可測性]\label{thm:measurability_before_after_Lebesgue_extension}
				$(X,\mathcal{B},\mu)$を測度空間,そのLebesgue拡大を
				$\left(X,\overline{\mathcal{B}},\overline{\mu}\right)$と書き,
				$f:X \longrightarrow \R$とする.
				このとき次は同値である:
				\begin{description}
					\item[(a)] 或る$\mathcal{B}/\borel{\R}$-可測関数$g$が存在して
						$f = g\quad \mbox{$\overline{\mu}$-a.e.}$を満たす.
					\item[(b)] 或る$\mathcal{B}/\borel{\R}$-可測関数$g_1,g_2$が存在して
						$g_1 \leq f \leq g_2\quad \mbox{$\overline{\mu}$-a.e.}$
						かつ$g_1 = g_2\quad \mbox{$\overline{\mu}$-a.e.}$を満たす.
					\item[(c)] $f$は$\overline{\mathcal{B}}/\borel{\R}$-可測である.
				\end{description}
			\end{thm}
		\end{screen}
		
		\begin{prf}\mbox{}
			\begin{description}
				\item[第一段]
					$(a)$が成立しているとき,
					$\{f \neq g\} \subset N$かつ$\mu(N) = \overline{\mu}(f \neq g) = 0$を満たす
					$\mu$-零集合$N \in \mathcal{B}$が存在して
					\begin{align}
						f^{-1}(E) \cap \left( g^{-1}(E) \right)^c \subset N,
						\quad g^{-1}(E) \cap \left( f^{-1}(E) \right)^c \subset N,
						\quad (\forall E \in \borel{\R})
					\end{align}
					が成り立つから,(\refeq{eq:appendix_Lebesgue_expansion_note_1})より
					$f^{-1}(E) \in \overline{\mathcal{B}}$が従い$(a) \Rightarrow (c)$が出る.
					以下,$(c) \Rightarrow (b) \Rightarrow (a)$を示す.
					
				\item[第二段]
					$B \subset X$に対して$f = \defunc_B$と表せるとき,
					$f$が$\overline{\mathcal{B}}/\borel{\R}$-可測なら
					$B = f^{-1}(\{1\}) \in \overline{\mathcal{B}}$となるから
					\begin{align}
						A_1 \subset B \subset A_2,\quad \mu(A_2 - A_1)=0
					\end{align}
					を満たす$A_1,A_2 \in \mathcal{B}$が存在し,
					$g_1 \coloneqq \defunc_{A_1},\ g_2 \coloneqq \defunc_{A_2}$とおけば
					$(c) \Rightarrow (b) \Rightarrow (a)$が従う.
				
				\item[第三段]
					$f$が$0 \leq f < \infty$かつ単関数$f = \sum_{n=0}^N \alpha_n \defunc_{B_n}\ 
					(\alpha_0 = 0,\ i \neq j \Rightarrow \alpha_i \neq \alpha_j)$
					として表されるとき,$f$が$\overline{\mathcal{B}}/\borel{\R}$-可測なら
					$B_n = f^{-1}(\{\alpha_n\}) \in \overline{\mathcal{B}},\ (n=0,1,\cdots,n)$となるから
					\begin{align}
						g_{1,n} \leq \defunc_{B_n} \leq g_{2,n},
						\quad \mu(g_{1,n} \neq g_{1,n})=0,
						\quad (n=0,1,\cdots,N)
					\end{align}
					を満たす$\mathcal{B}/\borel{\R}$-可測関数$g_{1,n},g_{2,n}$が存在する.
					\begin{align}
						g_1 \coloneqq \sum_{n=0}^N \alpha_n g_{1,n},
						\quad g_2 \coloneqq \sum_{n=0}^N \alpha_n g_{2,n}
					\end{align}
					とおけば$(c) \Rightarrow (b) \Rightarrow (a)$が従う.
					
				\item[第四段]
					$0 \leq f < \infty$のとき,
					$f$が$\overline{\mathcal{B}}/\borel{\R}$-可測なら
					$f_n(x) \uparrow f(x)\ (\forall x \in X)$を満たす
					非負$\overline{\mathcal{B}}$-可測単関数列$(f_n)_{n=1}^\infty$が存在し,
					前段より各$f_n$に対して
					\begin{align}
						g_{1,n} \leq f_n \leq g_{2,n},
						\quad \mu\left( g_{1,n} \neq g_{2,n} \right)=0
					\end{align}
					を満たす$\mathcal{B}/\borel{\R}$-可測関数$g_{1,n},g_{2,n}$が存在する.
					\begin{align}
						g_1 \coloneqq \liminf_{n \to \infty} g_{1,n},
						\quad \tilde{g}_2 \coloneqq \limsup_{n \to \infty} g_{2,n}
					\end{align}
					とおけば
					\begin{align}
						g_{1,n}(x) = g_{2,n}(x)\ (\forall n \geq 1)
						\quad \Rightarrow \quad g_1(x) = \lim_{n \to \infty} f_n(x) = \tilde{g}_2(x)
					\end{align}
					が成り立ち
					\begin{align}
						\mu(g_1 \neq \tilde{g}_2)
						\leq \mu\Biggl( \bigcup_{n=1}^\infty \left\{g_{1,n} \neq g_{2,n}\right\} \Biggr)=0 \\
					\end{align}
					が従うから,$\{\tilde{g}_2 = \infty\}$は$\mu$-零集合である.
					$g_2 \coloneqq \tilde{g}_2 \defunc_{\{\tilde{g}_2 < \infty\}}$
					として$(c) \Rightarrow (b) \Rightarrow (a)$が従う.
					
				\item[第五段]
					一般の$f:X \longrightarrow \R$に対し
					$f^+ \coloneqq f \defunc_{\{f \geq 0\}},\ f^- \coloneqq -f \defunc_{\{f < 0\}}$とおけば,
					$f$が$\overline{\mathcal{B}}/\borel{\R}$-可測なら
					$f^+,f^-$も$\overline{\mathcal{B}}/\borel{\R}$-可測であり,前段の結果より
					\begin{align}
						g_1^\pm \leq f^\pm \leq g_2^\pm, \quad \mbox{$\overline{\mu}$-a.e.},
						\quad \mu\left( g_1^\pm \neq g_2^\pm \right) = 0,
						\quad \mbox{(複合同順)}
					\end{align}
					を満たす$\mathcal{B}/\borel{\R}$-可測関数$g_1^{\pm},g_2^{\pm}$が存在する.
					ここで
					\begin{align}
						g_1 \coloneqq g_1^+ - g_2^-,
						\quad g_2 \coloneqq g_2^+ - g_1^+
					\end{align}
					とおけば$(c) \Rightarrow (b) \Rightarrow (a)$が従う.
					\QED
			\end{description}
		\end{prf}
		
		\begin{screen}
			\begin{lem}[複素数値関数の可測性]\label{thm:measurability_of_complex_measurable_functions}
				$(X,\mathcal{M})$を可測空間,$f:X \longrightarrow \C$とする.このとき,
				$f$が$\mathcal{M}/\borel{\C}$-可測であることと
				$f$の実部$u$虚部$v$がそれぞれ$\mathcal{M}/\borel{\R}$-可測であることは同値である.
			\end{lem}
		\end{screen}
	
		\begin{prf}
			$z \in \C$に対し$x,y \in \C$の組が唯一つ対応し$z = x + i y$を満たす.この対応関係により定める写像
			\begin{align}
				\varphi:\C \ni z \longmapsto (x,y) \in \R^2
			\end{align}
			は位相同型である.射影を$p_1:\R^2 \ni (x,y) \longmapsto x,
			\ p_2:\R^2 \ni (x,y) \longmapsto y$とすれば
			\begin{align}
				u = p_1 \circ \varphi \circ f,
				\quad v = p_2 \circ \varphi \circ f
			\end{align}
			となるから,$f$が$\mathcal{M}/\borel{\C}$-可測であるなら
			$p_1,p_2,\varphi$の連続性より
			\begin{align}
				u^{-1}(A) = f^{-1} \circ \varphi^{-1} \circ p_1^{-1}(A) \in \mathcal{M},
				\quad v^{-1}(A) = f^{-1} \circ \varphi^{-1} \circ p_2^{-1}(A) \in \mathcal{M},
				\quad (\forall A \in \borel{\R})
			\end{align}
			が成り立ち$u,v$の$\mathcal{M}/\borel{\R}$-可測性が従う.逆に$u,v$が$\mathcal{M}/\borel{\R}$-可測であるとき,
			\begin{align}
				f^{-1}(B) = \Set{x \in X}{(u(x),v(x)) \in \varphi(B)} \in \mathcal{M},
				\quad (\forall B \in \borel{\C})
			\end{align}
			が成り立ち$f$の$\mathcal{M}/\borel{\C}$-可測性が出る.
			\QED
		\end{prf}
	
		\begin{screen}
			\begin{thm}[定理\ref{thm:measurability_before_after_Lebesgue_extension}の系]
				$(X,\mathcal{B},\mu)$を測度空間,そのLebesgue拡大を
				$\left( X,\overline{\mathcal{B}},\overline{\mu} \right)$と書き,
				$f:X \longrightarrow \C$とする.このとき次は同値である:
				\begin{description}
					\item[(a)] 或る$\mathcal{B}/\borel{\C}$-可測関数$g$が存在して
						$f = g\quad \mbox{$\overline{\mu}$-a.e.}$を満たす.
					\item[(b)] $f$は$\overline{\mathcal{B}}/\borel{\C}$-可測である.
				\end{description}
			\end{thm}
		\end{screen}
		
		\begin{prf}
			$u \coloneqq \Re{f},\ v \coloneqq \Im{f}$とおく.このとき,
			定理\ref{thm:measurability_before_after_Lebesgue_extension}と
			補題\ref{thm:measurability_of_complex_measurable_functions}より
			\begin{align}
				\mbox{$f$が$\overline{\mathcal{B}}/\borel{\C}$-可測}
				&\Leftrightarrow \mbox{$u,v$が$\overline{\mathcal{B}}/\borel{\R}$-可測} \\
				&\Leftrightarrow \mbox{$u = h\ (\mbox{$\overline{\mu}$-a.e.}),\ v = k\ (\mbox{$\overline{\mu}$-a.e.})$を満たす
					$\mathcal{B}/\borel{\R}$-可測関数$h,k$が存在する} \\
				&\Leftrightarrow \mbox{$f = g\ (\mbox{$\overline{\mu}$-a.e.})$を満たす
					$\mathcal{B}/\borel{\C}$-可測関数$g$が存在する}
			\end{align}
			が成立する.
			\QED
		\end{prf}
		
	\subsection{有限加法的測度の拡張}
		\begin{screen}
			\begin{thm}[測度の一致の定理]
				$(X,\mathcal{M})$を可測空間,
				$\mathcal{A}$を$\mathcal{M}$を生成する乗法族とし,
				$(X,\mathcal{M})$上の測度$\mu_1,\mu_2$が
				$\mathcal{A}$上で一致しているとする.このとき,
				\begin{align}
					\mu_1(X_n) < \infty,
					\quad \bigcup_{n=1}^\infty X_n = X
				\end{align}
				を満たす増大列$\{X_n\}_{n=1}^\infty \subset \mathcal{A}$が存在すれば
				$\mu_1 = \mu_2$が成り立つ.
			\end{thm}
		\end{screen}
		
		\begin{prf}
			任意の$n = 1,2,\cdots$に対して
			\begin{align}
				\mathscr{D}_n \coloneqq \Set{B \in \mathcal{M}}{\mu_1(B \cap X_n) = \mu_2(B \cap X_n)}
			\end{align}
			とおけば,$\mathscr{D}_n$は$\mathcal{A}$を含むDynkin族であるから,Dynkin族定理より
			\begin{align}
				\mathscr{D}_n = \mathcal{M},\quad (\forall n \geq 1)
			\end{align}
			となり
			\begin{align}
				\mu_1(B) = \lim_{n \to \infty} \mu_1(B \cap X_n)
				= \lim_{n \to \infty} \mu_2(B \cap X_n) = \mu_2(B),
				\quad (\forall B \in \mathcal{M})
			\end{align}
			が従う.
			\QED
		\end{prf}
		
		\begin{screen}
			\begin{thm}[Kolmogorov-Hopf]
				$(X,\mathcal{B},\mu_0)$を有限加法的測度空間($\mathcal{B}$は有限加法族,$\mu_0$は有限加法的)とし,
				\begin{align}
					\mu^*(A) \coloneqq \inf{}{}\Set{\sum_{n=1}^\infty \mu_0(B_n)}{B_n \in \mathcal{B},\ A \subset \bigcup_{n=1}^\infty B_n},
					\quad (\forall A \subset X)
				\end{align}
				により$X$上に外測度を定め,$\mu^*$-可測集合を$\mathcal{B}^*$と書く.このとき,
				\begin{description}
					\item[(1)] $\sigma[\mathcal{B}] \subset \mathcal{B}^*$が成り立つ.
						ここで$\mu' \coloneqq\left.\mu^*\right|_{\mathcal{B}^*},
						\ \mu \coloneqq \left.\mu^*\right|_{\sigma[\mathcal{B}]}$とおく.
					\item[(2)] $\mu_0$が$\mathcal{B}$上で$\sigma$-加法的なら
						$\mu$は$\mu_0$の拡張となっている:
						\begin{align}
							\mu_0(B) = \mu(B),\quad (\forall B \in \mathcal{B}).
							\label{eq:appendix_finite_additive_measure_expansion_1}
						\end{align}
						
					\item[(3)] $\mu_0$が$\mathcal{B}$上で$\sigma$-有限的であるとき,
						(\refeq{eq:appendix_finite_additive_measure_expansion_1})を満たすような
						$\left( X,\sigma[\mathcal{B}] \right)$上の測度は存在しても唯一つである.
					
					\item[(4)] $\mu_0$が$\mathcal{B}$上で$\sigma$-加法的かつ$\sigma$-有限的ならば,
						$\mu$は$\mu_0$の$\left( X,\sigma[\mathcal{B}] \right)$への唯一つの拡張測度であり,
						このとき$\left( X,\mathcal{B}^*,\mu' \right)$は$(X,\sigma[\mathcal{B}],\mu)$の
						Lebesgue拡大に一致する:
						\begin{align}
							\left( X,\mathcal{B}^*,\mu' \right) 
							= \left( X,\overline{\sigma[\mathcal{B}]},\overline{\mu} \right).
							\label{eq:appendix_finite_additive_measure_expansion_5}
						\end{align}
				\end{description}
			\end{thm}
		\end{screen}
		
		\begin{prf}\mbox{}
			\begin{description}
				\item[(1)の証明]
					任意の$B \in \mathcal{B}$が$\mu^*$-可測であること,つまり任意の$A \subset X$に対し
					\begin{align}
						\mu^*(A) \geq \mu^*(A \cap B) + \mu^*(A \cap B^c)
						\label{eq:appendix_finite_additive_measure_expansion_2}
					\end{align}
					となることを示せば,$\mathcal{B} \subset \mathcal{B}^*$すなわち
					$\sigma[\mathcal{B}] \subset \mathcal{B}^*$が従う.
					任意の$A \subset X,\ \epsilon > 0$に対し
					\begin{align}
						A \subset \bigcup_{n=1}^\infty B_n,
						\quad \sum_{n=1}^\infty \mu_0(B_n) < \mu^*(A) + \epsilon
					\end{align}
					を満たす$\{B_n\}_{n=1}^\infty \subset \mathcal{B}$が存在する.
					このとき$A \cap B \subset \bigcup_{n=1}^\infty (B_n \cap B)
					,\ A \cap B^c \subset \bigcup_{n=1}^\infty (B_n \cap B^c)$より
					\begin{align}
						\mu^*(A \cap B) \leq \sum_{n=1}^\infty \mu_0(B_n \cap B),
						\quad \mu^*(A \cap B^c) \leq \sum_{n=1}^\infty \mu_0(B_n \cap B^c)
					\end{align}
					となるから
					\begin{align}
						\mu^*(A) + \epsilon
						&\geq \sum_{n=1}^\infty \mu_0(B_n)
						= \sum_{n=1}^\infty \left\{ \mu_0(B_n \cap B) + \mu_0(B_n \cap B^c) \right\} \\
						&= \sum_{n=1}^\infty \mu_0(B_n \cap B) + \sum_{n=1}^\infty \mu_0(B_n \cap B^c) \\
						&\geq \mu^*(A \cap B) + \mu^*(A \cap B^c)
					\end{align}
					が成り立つ.$\epsilon$の任意性より
					(\refeq{eq:appendix_finite_additive_measure_expansion_2})が出る.
				
				\item[(2)の証明]
					任意に$B \in \mathcal{B}$を取る.まず,
					$B \subset B \cup \emptyset \cup \emptyset \cup \cdots$より
					\begin{align}
						\mu^*(B) \leq \mu_0(B)
					\end{align}
					が成り立つ.一方で
					$B \subset \bigcup_{n=1}^\infty B_n$を満たす$\{B_n\}_{n=1}^\infty \subset \mathcal{B}$に対し
					\begin{align}
						B = \sum_{n=1}^\infty \Biggl( B \cap \Biggl( B_n \backslash \bigcup_{k=1}^{n-1}B_k \Biggr) \Biggr)
					\end{align}
					かつ$B \cap \left( B_n \backslash \bigcup_{k=1}^{n-1}B_k \right) \in \mathcal{B}$が満たされるから,
					$\mu_0$の$\sigma$-加法性より
					\begin{align}
						\mu_0(B) = \sum_{n=1}^\infty \mu_0\Biggl( B \cap \Biggl( B_n \backslash \bigcup_{k=1}^{n-1}B_k \Biggr) \Biggr)
						\leq \sum_{n=1}^\infty \mu_0(B_n)
					\end{align}
					が成り立ち$\mu_0(B) \leq \mu^*(B)$が従う.よって$\mu_0(B) = \mu^*(B) = \mu(B)$が得られる.
				
				\item[(3)の証明]
					$\sigma$-有限の仮定より,或る増大列$X_1 \subset X_2 \subset \cdots
					,\ \{X_n\}_{n=1}^\infty \subset \mathcal{B}$が存在して
					\begin{align}
						\mu_0 (X_n) < \infty \quad \bigcup_{n=1}^\infty X_n = X
						\label{eq:appendix_finite_additive_measure_expansion_3}
					\end{align}
					が成り立つ.一致の定理より,(\refeq{eq:appendix_finite_additive_measure_expansion_1})を満たす
					$\left( X,\sigma[\mathcal{B}] \right)$上の測度は存在しても一つのみである.
					
				\item[(4)の証明]
					(2)と(3)の結果より$\mu$は$\mu_0$の唯一つの拡張測度である.次に
					\begin{align}
						\mathcal{B}^* = \overline{\sigma[\mathcal{B}]}
						\label{eq:appendix_finite_additive_measure_expansion_4}
					\end{align}
					を示す.$E \in \overline{\sigma[\mathcal{B}]}$なら
					或る$B_1,B_2 \in \sigma[\mathcal{B}]$が存在して
					\begin{align}
						B_1 \subset E \subset B_2, \quad \mu(B_2 - B_1) = 0
					\end{align}
					を満たす.このとき(1)より
					$\mu'(B_2 - B_1) = 0$であり,$\left( X,\mathcal{B}^*,\mu' \right)$の完備性より
					$E \backslash B_1 \in \mathcal{B}^*$が満たされ
					\begin{align}
						E = B_1 + E \backslash B_1 \in \mathcal{B}^*
					\end{align}
					が従う.いま,(\refeq{eq:appendix_finite_additive_measure_expansion_3})を満たす
					$\{X_n\}_{n=1}^\infty \subset \mathcal{B}$を取り,
					$E \in \mathcal{B}^*$に対して$E_n \coloneqq E \cap X_n$とおく.このとき
					\begin{align}
						\mu'(E_n) \leq \mu'(X_n) = \mu_0(X_n) < \infty
					\end{align}
					となり,任意の$k = 1,2,\cdots$に対して
					\begin{align}
						E_n \subset \bigcup_{j=1}^\infty B^n_{k,j},
						\quad
						\sum_{j=1}^\infty \mu_0\left( B^n_{k,j} \right)
						< \mu'(E_n) + \frac{1}{k}
					\end{align}
					を満たす$\left\{B^n_{k,j}\right\}_{j=1}^\infty \subset \mathcal{B}$が存在する.
					\begin{align}
						B_{2,n} \coloneqq \bigcap_{k=1}^\infty \bigcup_{j=1}^\infty B^n_{k,j}
					\end{align}
					とおけば$E_n \subset B_{2,n} \in \sigma[\mathcal{B}]$であり,
					任意の$k = 1,2,\cdots$に対して
					\begin{align}
						&\mu'(B_{2,n} - E_n) = \mu'(B_{2,n}) - \mu'(E_n)
						\leq \mu'\Biggl( \bigcup_{j=1}^\infty B^n_{k,j} \Biggr) - \mu'(E_n) \\
						&\qquad \leq \sum_{j=1}^\infty \mu'\left( B^n_{k,j} \right) - \mu'(E_n)
						< \mu'(E_n) + \frac{1}{k} - \mu'(E_n)
						= \frac{1}{k}
					\end{align}
					が成り立つから$\mu'(B_{2,n} - E_n) = 0$となる.
					$E_n$を$B_{2,n} - E_n$に替えれば
					\begin{align}
						B_{2,n} - E_n \subset N_n, \quad \mu(N_n) = 0
					\end{align}
					を満たす$N_n \in \sigma[\mathcal{B}]$が取れる.
					\begin{align}
						B_{1,n} \coloneqq B_{2,n} \cap N_n^c
					\end{align}
					とおけば,$B_{1,n} \subset B_{2,n} \cap (B_{2,n} - E_n)^c = E_n$より
					\begin{align}
						B_{1,n} \subset E_n \subset B_{2,n},
						\quad \mu(B_{2,n} - B_{1,n}) \leq \mu(N_n) = 0
					\end{align}
					が成り立つから,
					\begin{align}
						B_1 \coloneqq \bigcup_{n=1}^\infty B_{1,n},
						\quad B_2 \coloneqq \bigcup_{n=1}^\infty B_{2,n}
					\end{align}
					として
					\begin{align}
						B_1 \subset E \subset B_2,
						\quad \mu(B_2 - B_1) \leq \mu\Biggl( \bigcup_{n=1}^\infty(B_{2,n} - B_{1,n}) \Biggr) = 0
					\end{align}
					が満たされ,$E \in \overline{\sigma[\mathcal{B}]}$が従い
					(\refeq{eq:appendix_finite_additive_measure_expansion_4})が得られる.
					同時に
					\begin{align}
						\overline{\mu}(E) = \mu(B_1) = \mu'(B_1)
						\leq \mu'(E) \leq \mu'(B_2) = \mu(B_2) = \overline{\mu}(E)
					\end{align}
					が成立するから,$\overline{\mu} = \mu'$となり
					(\refeq{eq:appendix_finite_additive_measure_expansion_5})が出る.
					\QED
			\end{description}
		\end{prf}
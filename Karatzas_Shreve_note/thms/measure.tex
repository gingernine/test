\section{測度}
	\subsection{Lebesgue拡大}
		\begin{screen}
			\begin{dfn}[Lebesgue拡大]
				$(X,\mathcal{B},\mu)$を測度空間とするとき,
				\begin{align}
					\overline{\mathcal{B}} &\coloneqq
					\Set{B \subset X}{\exists A_1,A_2 \in \mathcal{B},\ \mbox{s.t.}\quad A_1 \subset B \subset A_2,\ \mu(A_2 \backslash A_1)=0 }, \\
					\overline{\mu}(B) &\coloneqq \mu(A_1) \quad (\forall B \in \overline{\mathcal{B}},\ \mbox{$A_1$ as in above})
				\end{align}
				により得られる完備測度空間$\left( X,\overline{\mathcal{B}},\overline{\mu} \right)$を
				$(X,\mathcal{B},\mu)$のLebesgue拡大と呼ぶ.
			\end{dfn}
		\end{screen}
		$\overline{\mu}$はwell-definedである.実際,$B \subset X$に対し
		$A_1,A_2,B_1,B_2 \in \mathcal{B}$が
		\begin{align}
			&A_1 \subset B \subset A_2, \quad \mu(A_2 \backslash A_1) = 0, \\
			&B_1 \subset B \subset B_2, \quad \mu(B_2 \backslash B_1) = 0,
		\end{align}
		を満たすとき,$A_1 \cup B_1 \subset B \subset A_2 \cap B_2$となるが,
		\begin{align}
			(A_2 \cap B_2) \cap (A_1 \cup B_1)^c
			\subset A_2 \backslash A_1
		\end{align}
		より$\mu(A_1 \cup B_1) = \mu(A_2 \cap B_2)$が従い
		\begin{align}
			\mu(A_2) &= \mu(A_1) \leq \mu(A_1 \cup B_1) = \mu(A_2 \cap B_2) \leq \mu(B_2), \\
			\mu(B_2) &= \mu(B_1) \leq \mu(A_1 \cup B_1) = \mu(A_2 \cap B_2) \leq \mu(A_2)
		\end{align}
		が成り立つから$\mu(A_2) = \mu(B_2)$が出る.
		また,任意の$B \subset X$について
		\begin{align}
			\overline{\mathcal{B}}
			= \Set{B \subset X}{\exists A,N \in \mathcal{B},\ \mbox{s.t.}\quad \mu(N)=0,
			\ B \cap A^c, A \cap B^c \subset N}
			\label{eq:appendix_Lebesgue_expansion_note_1}
		\end{align}
		が成立する.実際,$B \in \overline{\mathcal{B}}$なら
		$A_1 \subset B \subset A_2$かつ$\mu(A_2 \backslash A_1) = 0$を満たす$A_1,A_2 \in \mathcal{B}$が存在するから
		\begin{align}
			A = A_2, \quad N = A_2 \backslash A_1
		\end{align}
		として$(\subset)$を得る.逆に右辺を満たす$A,N$が存在するとき,
		\begin{align}
			A \cap N^c &\subset A \cap B \subset B 
			\subset A \cup (A^c \cap B)
			\subset A \cup N
		\end{align}
		より$A_1 = A\cap N^c,\ A_2 = A \cup N$として$(\supset)$を得る.
		
		\begin{screen}
			\begin{lem}[可分値写像による可測写像の一様近似]\label{lem:approximation_of_countably_valued_mappings_on_dist_space}
				$(X,\mathcal{B},\mu)$を測度空間,$(S,d)$を可分距離空間とする.このとき
				任意の$\mathcal{B}/\borel{S}$-可測写像$f$に対し,
				$S$の可算稠密集合に値を取る$\mathcal{B}/\borel{S}$-可測写像列$(f_n)_{n=1}^\infty$が存在して,
				次の意味で$f$を一様に近似する:
				\begin{align}
					\sup{x \in X}{d\left(f_n(x),f(x)\right)} \longrightarrow 0
					\quad (n \longrightarrow \infty).
					\label{eq:lem_approximation_of_countably_valued_mappings_on_dist_space}
				\end{align}
			\end{lem}
		\end{screen}
		
		\begin{prf}
			$S$の可算稠密な部分集合を$\{a_k\}_{k=1}^\infty$とする.
			任意の$n \geq 1$に対し
			\begin{align}
				B_n^k \coloneqq \Set{s \in S}{d(s,a_k) < \frac{1}{n}},
				\quad A_n^k \coloneqq f^{-1}\left( B_n^k \right);
				\quad (k=1,2,\cdots)
			\end{align}
			とおけば,
			\begin{align}
				\bigcup_{k=1}^\infty A_n^k 
				= \bigcup_{k=1}^\infty f^{-1}\left( B_n^k \right)
				= f^{-1}(S)
			\end{align}
			より$X = \bigcup_{k=1}^\infty A_n^k$が成り立つ.ここで
			\begin{align}
				\tilde{A}_n^1 \coloneqq A_n^1,
				\quad \tilde{A}_n^k \coloneqq A_n^k \left\backslash \Biggl( \bigcup_{i=1}^{k-1} A_n^i \Biggr)\right.;
				\quad (k=1,2,\cdots)
			\end{align}
			として
			\begin{align}
				 f_n(x) \coloneqq a_k, \quad (x \in \tilde{A}_n^k,\ k=1,2,\cdots)
			\end{align}
			により$\mathcal{B}/\borel{S}$-可測写像列$(f_n)_{n=1}^\infty$を定めれば,
			\begin{align}
				d\left(f_n(x),f(x)\right) < \frac{1}{n},
				\quad (\forall x \in X)
			\end{align}
			が満たされ(\refeq{eq:lem_approximation_of_countably_valued_mappings_on_dist_space})が従う.
			\QED
		\end{prf}
		
		\begin{screen}
			\begin{thm}[$T_6$空間に値を取る可測写像列の各点極限は可測]
			\label{lem:measurability_metric_space}
				$S$を$T_6$空間,$(X,\mathcal{B})$を可測空間とする.
				$\mathcal{B}/\borel{S}$-可測写像列$(f_n)_{n=1}^{\infty}$が各点収束すれば,
				$f \coloneqq \lim_{n \to \infty} f_n$で定める$f$もまた可測$\mathcal{B}/\borel{S}$となる.
			\end{thm}
		\end{screen}
	
		\begin{prf}
			$S$の任意の閉集合$A$に対し,
			定理\ref{thm:perfectly_normal_Hausdorff_is_normal_and_closed_is_G_delta}より
			次を満たす閉集合の系$(A_m)_{m=1}^{\infty}$が取れる:
			\begin{align}
				A = \bigcap_{m=1}^\infty A_m 
				= \bigcap_{m=1}^\infty A_m^{\mathrm{o}}.
			\end{align}
			$f(x) \in A$なら任意の$m \in \N$に対し或る$N = N(x,m) \in \N$が存在して
			$f_n(x) \in A_m^{\mathrm{o}},\ (\forall n \geq N)$となるから
			\begin{align}
				f^{-1}(A) \subset \bigcap_{m \geq 1} \bigcup_{N \geq 1} \bigcap_{n \geq N} f_n^{-1}(A_m)
				\label{eq:lem_measurability_metric_space}
			\end{align}
			が従う.$f(x) \notin A$なら
			或る$m \geq 1$で$f(x) \notin A_m$となり,
			或る$N \in \N$に対し
			$f_n(x) \notin A_m,\ (\forall n \geq N)$が成り立ち
			\begin{align}
				f^{-1}(A^c) \subset \bigcup_{m \geq 1} \bigcup_{N \geq 1} \bigcap_{n \geq N} f_n^{-1}(A_m^c)
				\subset \bigcup_{m \geq 1} \bigcap_{N \geq 1} \bigcup_{n \geq N} f_n^{-1}(A_m^c)
			\end{align}
			が従う.(\refeq{eq:lem_measurability_metric_space})と併せれば
			\begin{align}
				f^{-1}(A) = \bigcap_{m \geq 1} \bigcup_{N \geq 1} \bigcap_{n \geq N} f_n^{-1}(A_m)
				\in \mathcal{B}
			\end{align}
			が成立し,$S$の閉集合は$f$により$\mathcal{B}$の元に引き戻されるから$f$の$\mathcal{B}/\borel{S}$-可測性が出る.
			\QED
		\end{prf}
		
		\begin{screen}
			\begin{thm}[拡大前後の可測性]\label{thm:measurability_before_after_Lebesgue_extension}
				$(X,\mathcal{B},\mu)$を測度空間,そのLebesgue拡大を
				$\left(X,\overline{\mathcal{B}},\overline{\mu}\right)$と書き,
				$(S,d)$を可分距離空間とする.
				このとき,任意の写像$f:X \longrightarrow S$に対し次は同値である:
				\begin{description}
					\item[(a)] 或る$\mathcal{B}/\borel{S}$-可測写像$g$が存在して$\mu$-a.e.に$f = g$となる.
					\item[(b)] $f$は$\overline{\mathcal{B}}/\borel{S}$-可測である.
				\end{description}
			\end{thm}
		\end{screen}
		
		\begin{prf}\mbox{}
			\begin{description}
				\item[第一段]
					$(a)$が成立しているとき,
					$\{f \neq g\} \subset N$を満たす
					$\mu$-零集合$N \in \mathcal{B}$が存在して
					\begin{align}
						f^{-1}(E) \cap \left( g^{-1}(E) \right)^c \subset N,
						\quad g^{-1}(E) \cap \left( f^{-1}(E) \right)^c \subset N,
						\quad (\forall E \in \borel{S})
					\end{align}
					が成り立つから,(\refeq{eq:appendix_Lebesgue_expansion_note_1})より
					$f^{-1}(E) \in \overline{\mathcal{B}}$が従い$(a) \Rightarrow (b)$が出る.
					
				\item[第二段]
					$f$が$\overline{\mathcal{B}}/\borel{S}$-可測のとき,
					$S$の可算稠密な部分集合を$\{a_k\}_{k=1}^\infty$とすれば,
					補題\ref{lem:approximation_of_countably_valued_mappings_on_dist_space}より
					\begin{align}
						f_n(x) = a_k, \ (x \in A_n^k,\ k=1,2,\cdots);
						\quad \sum_{k=1}^\infty A_n^k = X;
						\quad d\left(f_n(x),f(x)\right) < \frac{1}{n},\ (\forall x \in X)
					\end{align}
					を満たす$\overline{\mathcal{B}}/\borel{S}$-可測写像列$(f_n)_{n=1}^\infty$と
					互いに素な集合$\left\{A_n^k\right\}_{k=1}^\infty \subset \overline{\mathcal{B}}$が存在する.
					各$A_n^k$に対し
					\begin{align}
						E_{1,n}^k \subset A_n^k \subset E_{2,n}^k,
						\quad \mu\left(E_{2,n}^k- E_{1,n}^k\right) = 0
					\end{align}
					を満たす$E_{1,n}^k,E_{2,n}^k \in \mathcal{B}$が存在するから,
					一つ$a_0 \in S$を選び
					\begin{align}
						g_n(x) \coloneqq 
						\begin{cases}
							a_k, & (x \in E_{1,n}^k,\ k=1,2,\cdots), \\
							a_0, & (x \in N_n \coloneqq X \backslash \sum_{k=1}^\infty E_{1,n}^k)
						\end{cases}
					\end{align}
					で$\mathcal{B}/\borel{S}$-可測写像列$(g_n)_{n=1}^\infty$を定めて
					$N \coloneqq \bigcup_{n=1}^\infty N_n$とおけば
					\begin{align}
						f_n(x) = g_n(x),
						\quad (\forall x \in X \backslash N,\ \forall n \geq 1)
					\end{align}
					が成り立つ.このとき
					$X \backslash N$上で$\lim_{n \to \infty} g_n(x)$は存在し$\lim_{n \to \infty} f_n(x)=f(x)$に一致するから,
					\begin{align}
						g(x) \coloneqq 
						\begin{cases}
							\displaystyle\lim_{n \to \infty} g_n(x), & (x \in X \backslash N), \\
							a_0, & (x \in N)
						\end{cases}
					\end{align}
					により$\mathcal{B}/\borel{S}$-可測写像$g$を定めれば(a)が満たされる.
					\QED
			\end{description}
		\end{prf}
	
	\subsection{コンパクトクラス}
		
		\begin{screen}
			\begin{dfn}[コンパクトクラス]
				$X$を空でない集合,$\mathcal{K}$をその部分集合族とする.
				$\mathcal{K}$の任意の可算部分集合$\{K_n\}_{n \in \Natural}$に対して
				\begin{align}
					\bigcap_{n\in\Natural} K_n = \emptyset
					\Longrightarrow \exists N \in \Natural
					\left[\bigcap_{n=1}^N K_n = \emptyset\right]
				\end{align}
				が成り立つとき,$\mathcal{K}$は$X$上の{\bf コンパクトクラス}
				\index{こんぱくとくらす@コンパクトクラス}{\bf (compact class)}と呼ばれる.
			\end{dfn}
		\end{screen}
		
		\begin{screen}
			\begin{thm}\label{thm:compact_class_Haudorff}
				Hausdorff空間において,コンパクト部分集合から成る任意の族はコンパクトクラスとなる.
			\end{thm}
		\end{screen}
		
		\begin{prf}
			Cantorの共通部分定理(P. \pageref{thm:Cantor_intersection_theorem})より従う.
			\QED
		\end{prf}
		
		\begin{screen}
			\begin{thm}[完全加法性の同値条件]\label{thm:equivalent_conditions_of_countable_additivity}
				$\mathcal{B}$を集合$X$の上の有限加法族,
				$\mu$を$\mathcal{B}$上の有限加法的な正値測度として
				\begin{description}
					\item[(a) (共通点性)] $\{B_n\}_{n=1}^\infty \subset \mathcal{B}$が
						$\mu(B_1) < \infty,\ B_n \supset B_{n+1},\ \bigcap_{n=1}^\infty B_n = \emptyset$
						なら$\lim_{n \to \infty} \mu(B_n) = 0$.
						
					\item[(b)] $\{B_n\}_{n=1}^\infty \subset \mathcal{B}$が
						$B_n \subset B_{n+1},\ \bigcap_{n=1}^\infty B_n \eqqcolon B \in \mathcal{B}$
						かつ$\mu(B) = \infty$なら$\lim_{n \to \infty} \mu(B_n) = \infty$.
					
					\item[(c)] $\mu(X_n) < \infty$かつ$\bigcup_{n=1}^\infty X_n = X$
						を満たす$(X_n)_{n=1}^\infty \subset \mathcal{B}$が存在するとき,
						$\mu(B) = \infty$なら$\lim_{n \to \infty} \mu(B \cap X_n) = \infty$.
				\end{description}
				とおくとき,
				\begin{description}
					\item[(1)] $0 < \mu(X) < \infty$なら$\mu$が$\mathcal{B}$の上で
						完全加法性であることと(a)は同値である.
						
					\item[(2)] $\mu(X) = \infty$なら$\mu$が$\mathcal{B}$の上で
						完全加法性であることと(a)$\wedge$(b)は同値である.
						
					\item[(3)] $\mu(X) = \infty$で$\mu$が$\sigma$-有限的なら,
						$\mu$が$\mathcal{B}$の上で完全加法性であることと(a)$\wedge$(c)は同値である.
				\end{description}
			\end{thm}
		\end{screen}
		
		\begin{prf}
			
		\end{prf}
		
		\begin{screen}
			\begin{thm}[コンパクトクラスと共通点性]\label{thm:compact_class_intersection}
				$\mathcal{B}$を集合$X$の上の有限加法族,
				$\mu$を$\mathcal{B}$上の有限加法的正値測度とする.
				$X$上にコンパクトクラス$\mathcal{K}$が存在して,
				任意の正数$\epsilon$と$0 < \mu(B) < \infty$を満たす$\mathcal{B}$の任意の
				要素$B$に対し,或る$\mathcal{B}$の要素$A$と$\mathcal{K}$の要素$K$が存在して
				\begin{align}
					A \subset K \subset B \wedge \mu(B \backslash A )< \epsilon
				\end{align}
				が成り立つとき,$\mu$は共通点性(
				定理\ref{thm:equivalent_conditions_of_countable_additivity}の
				(a))を持つ.
			\end{thm}
		\end{screen}
		
		\begin{prf}
			$(B_n)_{n \in \N}$を$\mu(B_1) < \infty$かつ$\bigcap_{n\in\N} B_n = \emptyset$
			を満たす減少列とすれば,或る$N$で$\mu(B_N) = 0$となるとき
			\begin{align}
				\forall n \in \N,\quad n \geq N \Longrightarrow \mu(B_n) \leq \mu(B_N) = 0
			\end{align}
			が成り立ち$\lim_{n \to \infty} \mu(B_n) = 0$が従う.
			全ての自然数$n$で$0 < \mu(B_n)$となるなら,任意の正数$\epsilon$に対して
			\begin{align}
				\forall n \in \N,\quad A_n \subset K_n \subset B_n
				\wedge \mu(B_n \backslash A_n) < \frac{\epsilon}{2^n}
			\end{align}
			を満たす$\mathcal{B}$の要素$A_n$と$\mathcal{K}$の要素$K_n$が存在する.このとき
			\begin{align}
				\bigcap_{n\in\N} K_n \subset \bigcap_{n\in\N} B_n = \emptyset
			\end{align}
			が成り立つから,コンパクトクラスの性質より或る1以上の自然数$N$が存在して
			$\bigcap_{n=1}^N A_n \subset \bigcap_{n=1}^N K_n = \emptyset$
			が成立し,$N$以上の任意の自然数$m$で$B_m 
			\subset \left( \bigcup_{n=1}^N B_n \right) \backslash \bigcup_{n=1}^N A_n
			= \bigcup_{n=1}^N \left(B_n \backslash \bigcup_{n=1}^N A_n \right)
			\subset \bigcup_{n=1}^N (B_n \backslash A_n)$となるから
			\begin{align}
				\forall m \in \N,\quad
				m \geq N \Longrightarrow \mu(B_m) 
				\leq \sum_{n=1}^N \mu(B_n \backslash A_n) < \epsilon
			\end{align}
			が従う.
			\QED
		\end{prf}
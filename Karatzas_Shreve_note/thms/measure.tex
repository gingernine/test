\section{測度}
	\subsection{Lebesgue拡大}
		\begin{screen}
			\begin{dfn}[Lebesgue拡大]
				$(X,\mathcal{B},\mu)$を測度空間とするとき,
				\begin{align}
					\overline{\mathcal{B}} &\coloneqq
					\Set{B \subset X}{\exists A_1,A_2 \in \mathcal{B},\ \mbox{s.t.}\quad A_1 \subset B \subset A_2,\ \mu(A_2 \backslash A_1)=0 }, \\
					\overline{\mu}(B) &\coloneqq \mu(A_1) \quad (\forall B \in \overline{\mathcal{B}},\ \mbox{$A_1$ as in above})
				\end{align}
				により得られる完備測度空間$\left( X,\overline{\mathcal{B}},\overline{\mu} \right)$を
				$(X,\mathcal{B},\mu)$のLebesgue拡大と呼ぶ.
			\end{dfn}
		\end{screen}
		$\overline{\mu}$はwell-definedである.実際,$B \subset X$に対し
		$A_1,A_2,B_1,B_2 \in \mathcal{B}$が
		\begin{align}
			&A_1 \subset B \subset A_2, \quad \mu(A_2 \backslash A_1) = 0, \\
			&B_1 \subset B \subset B_2, \quad \mu(B_2 \backslash B_1) = 0,
		\end{align}
		を満たすとき,$A_1 \cup B_1 \subset B \subset A_2 \cap B_2$となるが,
		\begin{align}
			(A_2 \cap B_2) \cap (A_1 \cup B_1)^c
			\subset A_2 \backslash A_1
		\end{align}
		より$\mu(A_1 \cup B_1) = \mu(A_2 \cap B_2)$が従い
		\begin{align}
			\mu(A_2) &= \mu(A_1) \leq \mu(A_1 \cup B_1) = \mu(A_2 \cap B_2) \leq \mu(B_2), \\
			\mu(B_2) &= \mu(B_1) \leq \mu(A_1 \cup B_1) = \mu(A_2 \cap B_2) \leq \mu(A_2)
		\end{align}
		が成り立つから$\mu(A_2) = \mu(B_2)$が出る.
		また,任意の$B \subset X$について
		\begin{align}
			\overline{\mathcal{B}}
			= \Set{B \subset X}{\exists A,N \in \mathcal{B},\ \mbox{s.t.}\quad \mu(N)=0,
			\ B \cap A^c, A \cap B^c \subset N}
			\label{eq:appendix_Lebesgue_expansion_note_1}
		\end{align}
		が成立する.実際,$B \in \overline{\mathcal{B}}$なら
		$A_1 \subset B \subset A_2$かつ$\mu(A_2 \backslash A_1) = 0$を満たす$A_1,A_2 \in \mathcal{B}$が存在するから
		\begin{align}
			A = A_2, \quad N = A_2 \backslash A_1
		\end{align}
		として$(\subset)$を得る.逆に右辺を満たす$A,N$が存在するとき,
		\begin{align}
			A \cap N^c &\subset A \cap B \subset B 
			\subset A \cup (A^c \cap B)
			\subset A \cup N
		\end{align}
		より$A_1 = A\cap N^c,\ A_2 = A \cup N$として$(\supset)$を得る.
		
		\begin{screen}
			\begin{lem}[可分値写像による可測写像の一様近似]\label{lem:approximation_of_countably_valued_mappings_on_dist_space}
				$(X,\mathcal{B},\mu)$を測度空間,$(S,d)$を可分距離空間とする.このとき
				任意の$\mathcal{B}/\borel{S}$-可測写像$f$に対し,
				$S$の可算稠密集合に値を取る$\mathcal{B}/\borel{S}$-可測写像列$(f_n)_{n=1}^\infty$が存在して,
				次の意味で$f$を一様に近似する:
				\begin{align}
					\sup{x \in X}{d\left(f_n(x),f(x)\right)} \longrightarrow 0
					\quad (n \longrightarrow \infty).
					\label{eq:lem_approximation_of_countably_valued_mappings_on_dist_space}
				\end{align}
			\end{lem}
		\end{screen}
		
		\begin{prf}
			$S$の可算稠密な部分集合を$\{a_k\}_{k=1}^\infty$とする.
			任意の$n \geq 1$に対し
			\begin{align}
				B_n^k \coloneqq \Set{s \in S}{d(s,a_k) < \frac{1}{n}},
				\quad A_n^k \coloneqq f^{-1}\left( B_n^k \right);
				\quad (k=1,2,\cdots)
			\end{align}
			とおけば,
			\begin{align}
				\bigcup_{k=1}^\infty A_n^k 
				= \bigcup_{k=1}^\infty f^{-1}\left( B_n^k \right)
				= f^{-1}(S)
			\end{align}
			より$X = \bigcup_{k=1}^\infty A_n^k$が成り立つ.ここで
			\begin{align}
				\tilde{A}_n^1 \coloneqq A_n^1,
				\quad \tilde{A}_n^k \coloneqq A_n^k \left\backslash \Biggl( \bigcup_{i=1}^{k-1} A_n^i \Biggr)\right.;
				\quad (k=1,2,\cdots)
			\end{align}
			として
			\begin{align}
				 f_n(x) \coloneqq a_k, \quad (x \in \tilde{A}_n^k,\ k=1,2,\cdots)
			\end{align}
			により$\mathcal{B}/\borel{S}$-可測写像列$(f_n)_{n=1}^\infty$を定めれば,
			\begin{align}
				d\left(f_n(x),f(x)\right) < \frac{1}{n},
				\quad (\forall x \in X)
			\end{align}
			が満たされ(\refeq{eq:lem_approximation_of_countably_valued_mappings_on_dist_space})が従う.
			\QED
		\end{prf}
		
		\begin{screen}
			\begin{thm}[拡大前後の可測性]\label{thm:measurability_before_after_Lebesgue_extension}
				$(X,\mathcal{B},\mu)$を測度空間,そのLebesgue拡大を
				$\left(X,\overline{\mathcal{B}},\overline{\mu}\right)$と書き,
				$(S,d)$を可分距離空間とする.
				このとき,任意の写像$f:X \longrightarrow S$に対し次は同値である:
				\begin{description}
					\item[(a)] 或る$\mathcal{B}/\borel{S}$-可測写像$g$が存在して$\mu$-a.e.に$f = g$となる.
					\item[(b)] $f$は$\overline{\mathcal{B}}/\borel{S}$-可測である.
				\end{description}
			\end{thm}
		\end{screen}
		
		\begin{prf}\mbox{}
			\begin{description}
				\item[第一段]
					$(a)$が成立しているとき,
					$\{f \neq g\} \subset N$を満たす
					$\mu$-零集合$N \in \mathcal{B}$が存在して
					\begin{align}
						f^{-1}(E) \cap \left( g^{-1}(E) \right)^c \subset N,
						\quad g^{-1}(E) \cap \left( f^{-1}(E) \right)^c \subset N,
						\quad (\forall E \in \borel{S})
					\end{align}
					が成り立つから,(\refeq{eq:appendix_Lebesgue_expansion_note_1})より
					$f^{-1}(E) \in \overline{\mathcal{B}}$が従い$(a) \Rightarrow (b)$が出る.
					
				\item[第二段]
					$f$が$\overline{\mathcal{B}}/\borel{S}$-可測のとき,
					$S$の可算稠密な部分集合を$\{a_k\}_{k=1}^\infty$とすれば,
					補題\ref{lem:approximation_of_countably_valued_mappings_on_dist_space}より
					\begin{align}
						f_n(x) = a_k, \ (x \in A_n^k,\ k=1,2,\cdots);
						\quad \sum_{k=1}^\infty A_n^k = X;
						\quad d\left(f_n(x),f(x)\right) < \frac{1}{n},\ (\forall x \in X)
					\end{align}
					を満たす$\overline{\mathcal{B}}/\borel{S}$-可測写像列$(f_n)_{n=1}^\infty$と
					互いに素な集合$\left\{A_n^k\right\}_{k=1}^\infty \subset \overline{\mathcal{B}}$が存在する.
					各$A_n^k$に対し
					\begin{align}
						E_{1,n}^k \subset A_n^k \subset E_{2,n}^k,
						\quad \mu\left(E_{2,n}^k- E_{1,n}^k\right) = 0
					\end{align}
					を満たす$E_{1,n}^k,E_{2,n}^k \in \mathcal{B}$が存在するから,
					一つ$a_0 \in S$を選び
					\begin{align}
						g_n(x) \coloneqq 
						\begin{cases}
							a_k, & (x \in E_{1,n}^k,\ k=1,2,\cdots), \\
							a_0, & (x \in N_n \coloneqq X \backslash \sum_{k=1}^\infty E_{1,n}^k)
						\end{cases}
					\end{align}
					で$\mathcal{B}/\borel{S}$-可測写像列$(g_n)_{n=1}^\infty$を定めて
					$N \coloneqq \bigcup_{n=1}^\infty N_n$とおけば
					\begin{align}
						f_n(x) = g_n(x),
						\quad (\forall x \in X \backslash N,\ \forall n \geq 1)
					\end{align}
					が成り立つ.このとき
					$X \backslash N$上で$\lim_{n \to \infty} g_n(x)$は存在し$\lim_{n \to \infty} f_n(x)=f(x)$に一致するから,
					\begin{align}
						g(x) \coloneqq 
						\begin{cases}
							\displaystyle\lim_{n \to \infty} g_n(x), & (x \in X \backslash N), \\
							a_0, & (x \in N)
						\end{cases}
					\end{align}
					により$\mathcal{B}/\borel{S}$-可測写像$g$を定めれば(a)が満たされる.
					\QED
			\end{description}
		\end{prf}
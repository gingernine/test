\section{測度}
	\subsection{Lebesgue拡大}
		\begin{screen}
			\begin{dfn}[Lebesgue拡大]
				$(X,\mathcal{B},\mu)$を測度空間とするとき,
				\begin{align}
					\bar{\mathcal{B}} &\coloneqq
					\Set{B \subset X}{\exists A_1,A_2 \in \mathcal{B},\ \mbox{s.t.}\quad A_1 \subset B \subset A_2,\ \mu(A_2 - A_1)=0 }, \\
					\bar{\mu}(B) &\coloneqq \mu(A_1) \quad (\forall B \in \bar{\mathcal{B}},\ \mbox{$A_1$ as in above})
				\end{align}
				により得られる完備測度空間$(X,\bar{\mathcal{B}},\bar{\mu})$を
				$(X,\mathcal{B},\mu)$のLebesgue拡大と呼ぶ.
			\end{dfn}
		\end{screen}
		$(X,\mathcal{B},\mu)$を測度空間とし,そのLebesgue拡大を
		$(X,\bar{\mathcal{B}},\bar{\mu})$と書く.このとき,任意の$B \subset X$について
		\begin{align}
			B \in \bar{\mathcal{B}}
			\quad \Leftrightarrow \quad
			&\exists A,N \in \mathcal{B} \\
			&\quad \mbox{s.t.}\quad \mu(N)=0,
			\ B \cap A^c, A \cap B^c \subset N
			\label{eq:appendix_Lebesgue_expansion_note_1}
		\end{align}
		が成立する.実際,$B \in \mathcal{B}$なら
		$A_1 \subset B \subset A_2$かつ$\mu(A_2 - A_1) = 0$を満たす$A_1,A_2 \in \mathcal{B}$が存在するから
		\begin{align}
			A = A_2, \quad N = A_2 - A_1
		\end{align}
		として$(\Rightarrow)$を得る.逆に右辺を満たす$A,N$が存在するとき,
		\begin{align}
			A \cap N^c &\subset A \cap B \\
			&\subset B \\
			&\subset A \cup (A^c \cap B)
			\subset A \cup N
		\end{align}
		より$A_1 = A\cap N^c,\ A_2 = A \cup N$として$(\Leftarrow)$を得る.
	
		\begin{screen}
			\begin{thm}[完備化前後の可測関数の関係]
				$(X,\mathcal{B},\mu)$を測度空間,そのLebesgue拡大を
				$(X,\bar{\mathcal{B}},\bar{\mu})$と書き,
				$f:X \longrightarrow [-\infty,\infty]$とする.
				このとき次は同値である:
				\begin{description}
					\item[(a)] 或る$\mathcal{B}/\borel{[-\infty,\infty]}$-可測関数$g$が存在して
						$f = g\quad \mbox{$\bar{\mu}$-a.e.}$を満たす.
					\item[(b)] 或る$\mathcal{B}/\borel{[-\infty,\infty]}$-可測関数$g_1,g_2$が存在して
						$g_1(x) \leq f(x) \leq g_2(x)\ (\forall x \in X)$かつ$g_1 = g_2\quad \mbox{$\bar{\mu}$-a.e.}$を満たす.
					\item[(c)] $f$は$\bar{\mathcal{B}}/\borel{[-\infty,\infty]}$-可測である.
				\end{description}
			\end{thm}
		\end{screen}
		
		\begin{prf}\mbox{}
			\begin{description}
				\item[第一段]
					$B \subset X$に対して$f = \defunc_B$と表せるとき,
					\begin{align}
						\mbox{$f$が$\bar{\mathcal{B}}/\borel{[-\infty,\infty]}$-可測}
						&\Leftrightarrow B = f^{-1}(\{1\}) \in \bar{\mathcal{B}} \\
						&\Leftrightarrow \exists A_1,A_2 \in \mathcal{B},\ \mbox{s.t.}\quad A_1 \subset B \subset A_2,\ \mu(A_2 - A_1)=0 \\
						&\Rightarrow \defunc_{A_1} \leq f \leq \defunc_{A_2},\quad \defunc_{A_1}=\defunc_{A_2}\ \mbox{$\bar{\mu}$-a.e.} \\
						&\Rightarrow (b) \\
						&\Rightarrow (c)
					\end{align}
					となる.また$(c)$が満たされているとき,
					\begin{align}
						f(x) = g(x) \quad (\forall x \in X \backslash N),
						\label{eq:appendix_Lebesgue_expansion_note_2}
					\end{align}
					を満たす$\mu$-零集合$N \in \mathcal{B}$が存在して
					\begin{align}
						f^{-1}(E) \cap \left( g^{-1}(E) \right)^c \subset N,
						\quad g^{-1}(E) \cap \left( f^{-1}(E) \right)^c \subset N,
						\quad (\forall E \in \borel{[-\infty,\infty]})
					\end{align}
					が成り立つから,(\refeq{eq:appendix_Lebesgue_expansion_note_1})より
					$f^{-1}(E) \in \bar{\mathcal{B}}$が従い$(c) \Rightarrow (a)$が出る.
				
				\item[第二段]
					$0 \leq f \leq \infty$かつ単関数$f = \sum_{n=0}^N \alpha_n \defunc_{B_n}\ 
					(\alpha_0 = 0,\ i \neq j \Rightarrow \alpha_i \neq \alpha_j)$
					として表されるとき,
					\begin{align}
						\mbox{$f$が$\bar{\mathcal{B}}/\borel{[-\infty,\infty]}$-可測}
						&\Leftrightarrow B_n = f^{-1}(\{\alpha_n\}) \in \bar{\mathcal{B}},\ (n=0,1,\cdots,N) \\
						&\Leftrightarrow \exists g_{1,n},g_{2,n}:\ \mbox{$\mathcal{B}/\borel{[-\infty,\infty]}$-measurable }, \\
							&\qquad \mbox{s.t.}\quad g_{1,n} \leq \defunc_{B_n} \leq g_{2,n},
							\ \mu(g_{1,n} \neq g_{1,n})=0,\ (n=0,1,\cdots,N) \\
						&\Rightarrow g_1 \coloneqq \sum_{n=0}^N \alpha_n g_{1,n},
							\quad g_2 \coloneqq \sum_{n=0}^N \alpha_n g_{2,n}, \\
							&\qquad g_1 \leq f \leq g_2,\quad \mu(g_1 \neq g_2) \leq \mu\Biggl( \bigcup_{n=0}^N \left\{g_{1,n} \neq g_{2,n}\right\} \Biggr)=0 \\
						&\Rightarrow (b) \\
						&\Rightarrow (c)
					\end{align}
					が成り立つ.また前段と同じ理由で$(c) \Rightarrow (a)$が出る.
					
				\item[第三段]
					$0 \leq f \leq \infty$のとき,
					$f$が$\bar{\mathcal{B}}/\borel{[-\infty,\infty]}$-可測なら
					$f_n(x) \uparrow f(x)\ (\forall x \in X)$を満たす
					非負$\bar{\mathcal{B}}$-可測単関数列$(f_n)_{n=1}^\infty$が存在し,
					第二段の結果より各$f_n$に対して
					\begin{align}
						g_{1,n} \leq f_n \leq g_{2,n},
						\quad \mu\left( g_{1,n} \neq g_{2,n} \right)=0
					\end{align}
					を満たす$\mathcal{B}/\borel{[-\infty,\infty]}$-可測写像$g_{1,n},g_{2,n}$が存在する.
					\begin{align}
						g_1 \coloneqq \liminf_{n \to \infty} g_{1,n},
						\quad g_2 \coloneqq \limsup_{n \to \infty} g_{2,n}
					\end{align}
					とおけば
					\begin{align}
						g_{1,n}(x) = g_{2,n}(x)\ (\forall n \geq 1)
						\quad \Rightarrow \quad g_1(x) = \lim_{n \to \infty} f_n(x) = g_2(x)
					\end{align}
					が成り立ち
					\begin{align}
						\mu(g_1 \neq g_2)
						\leq \mu\Biggl( \bigcup_{n=1}^\infty \left\{g_{1,n} \neq g_{2,n}\right\} \Biggr)=0 \\
					\end{align}
					が従うから$(a) \Rightarrow (b)$及び$(b) \Rightarrow (c)$が得られる.
					第一段と同じ理由で$(c) \Rightarrow (a)$も成立する.
					
				\item[第四段]
					一般の$f:X \longrightarrow [-\infty,\infty]$に対し
					$f^+ \coloneqq f \defunc_{\{f \geq 0\}},\ f^- \coloneqq -f \defunc_{\{f < 0\}}$とおけば,
					$f$が$\bar{\mathcal{B}}/\borel{[-\infty,\infty]}$-可測なら
					$f^+,f^-$も$\bar{\mathcal{B}}/\borel{[-\infty,\infty]}$-可測である.従って
					\begin{align}
						g_1^\pm \leq f^\pm \leq g_2^\pm, \quad \mu\left( g_1^\pm \neq g_2^\pm \right) = 0,
						\quad \mbox{(複合同順)}
					\end{align}
					を満たす$\mathcal{B}/\borel{[-\infty,\infty]}$-可測写像$g_1^{\pm},g_2^{\pm}$が存在する.
					ここで
					\begin{align}
						g_1 \coloneqq g_1^+ - g_2^-,
						\quad g_2 \coloneqq g_2^+ - g_1^+
					\end{align}
					とおけば$(a) \Rightarrow (b)$成り立ち,前段と同様に$(b) \Rightarrow (c) \Rightarrow (a)$も得られる.
					\QED
			\end{description}
		\end{prf}
		
		\begin{screen}
			\begin{cor}
				$(X,\mathcal{B},\mu)$を測度空間,そのLebesgue拡大を
				$(X,\bar{\mathcal{B}},\bar{\mu})$と書き,
				$f:X \longrightarrow \C$とする.このとき次は同値である:
				\begin{description}
					\item[(a)] 或る$\mathcal{B}/\borel{\C}$-可測関数$g$が存在して
						$f = g\quad \mbox{$\bar{\mu}$-a.e.}$を満たす.
					\item[(b)] $f$は$\bar{\mathcal{B}}/\borel{\C}$-可測である.
				\end{description}
			\end{cor}
		\end{screen}
	
	\subsection{測度の構成1: 外測度による方法}
	\subsection{測度の構成2: Riesz-Markov-角谷の定理による方法}
	\subsection{測度の構成1と2の関係}
	\subsection{有限加法的測度の拡張}
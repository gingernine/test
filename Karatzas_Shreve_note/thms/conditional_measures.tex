\section{正則条件付測度}
	\begin{screen}
		\begin{dfn}[正則条件付複素測度]
			$(X,\mathscr{F})$を可測空間,$\mathscr{G} \subset \mathscr{F}$を部分$\sigma$-加法族,
			$\mu$を$\mathscr{F}$上の複素測度とするとき,次の(1)(2)(3)を満たす写像
			\begin{align}
				\mu_{\mathscr{G}}(\cdot\, |\, \cdot):\mathscr{F} \times X \longrightarrow \C
			\end{align}
			を$\mathscr{G}$の下での$\mu$の正則条件付複素測度
			(regular conditional complex measure of $\mu$ with respect to $\mathscr{G}$)と呼ぶ:
			\begin{description}
				\item[(1)] 任意の$x \in X$で$\mathscr{F} \ni A \longmapsto \mu_{\mathscr{G}}(A\, |\, x)$は複素測度である.
				\item[(2)] 任意の$A \in \mathscr{F}$で$X \ni x \longmapsto \mu_{\mathscr{G}}(A\, |\, x)$は
					$\mathscr{G}/\borel{\C}$-可測かつ$|\mu|$-可積分である.
				\item[(3)] 任意の$A \in \mathscr{F}$と$B \in \mathscr{G}$に対し次を満たす:
					\begin{align}
						\mu(A \cap B) = \int_B \mu_{\mathscr{G}}(A\, |\, x)\ |\mu|(dx).
					\end{align}
			\end{description}
		\end{dfn}
	\end{screen}
	
	\begin{screen}
		\begin{thm}[正則条件付複素測度の一意性]
			$(X,\mathscr{F})$を可測空間,$\mathscr{G} \subset \mathscr{F}$を部分$\sigma$-加法族,
			$\mu$を$\mathscr{F}$上の複素測度とする.
			$\mathscr{F}$が可算乗法族で生成されるなら,
			$\mu$に対し$\mathscr{G}$の下での正則条件付複素測度
			$\mu_{\mathscr{G}},\nu_{\mathscr{G}}$が存在するとき,
			或る$|\mu|$-零集合$N \in \mathscr{G}$が存在して次が成立する:
			\begin{align}
				\mu_{\mathscr{G}}(A\, |\, x) = \nu_{\mathscr{G}}(A\, |\, x),
				\quad (\forall A \in \mathscr{F},\ \forall x \in X \backslash N).
			\end{align}
		\end{thm}
	\end{screen}
	
	\begin{prf}
		$\mathscr{F}$を生成する可算乗法族を$\{A_n\}_{n=1}^\infty$と書けば,任意の$A_n$に対し
		\begin{align}
			\int_{B} \mu_{\mathscr{G}}(A_n\, |\, x)\ |\mu|(dx)
			= \int_{B} \nu_{\mathscr{G}}(A_n\, |\, x)\ |\mu|(dx),
			\quad (\forall B \in \mathscr{G})
		\end{align}
		が満たされるから$N_n \coloneqq \Set{x \in X}{\mu_{\mathscr{G}}(A_n\, |\, x) \neq \nu_{\mathscr{G}}(A_n\, |\, x)}$
		は$\mathscr{G}$の$|\mu|$-零集合となる.$N \coloneqq \bigcup_{n=1}^\infty N_n$とおけば
		\begin{align}
			\mathscr{D} \coloneqq \Set{A \in \mathscr{F}}{\mu_{\mathscr{G}}(A\, |\, x) = \nu_{\mathscr{G}}(A\, |\, x),\ 
			\forall x \in X \backslash N}
		\end{align}
		によりDynkin族が定まり,$\mathscr{D}$は$\{A_n\}_{n=1}^\infty$を含むから
		Dynkin族定理より定理の主張が従う.
		\QED
	\end{prf}
	
	\begin{screen}
		\begin{thm}[正則条件付測度の存在]
			$(X,\mathscr{F})$を可測空間,$\mathscr{G} \subset \mathscr{F}$を部分$\sigma$-加法族,
			$\mu$を$\mathscr{F}$上の複素測度とする.
			また$\mathscr{F}$が可算族で生成され,かつ或る
			コンパクトクラス$\mathcal{K}$が存在して,
			任意の$\epsilon > 0$と$A \in \mathscr{F}$に対し
			\begin{align}
				A_\epsilon \subset K_\epsilon \subset A,\quad |\mu|(A \backslash A_\epsilon) < \epsilon
			\end{align}
			を満たす$K_\epsilon \in \mathcal{K},\ A_\epsilon \in \mathscr{F}$が取れると仮定する.
			このとき$\mathscr{G}$の下での$\mu$の正則条件付複素測度が存在する.
		\end{thm}
	\end{screen}
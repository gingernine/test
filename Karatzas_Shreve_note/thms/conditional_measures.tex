\section{正則条件付複素測度}
	\begin{screen}
		\begin{dfn}[正則条件付複素測度]
			$(X,\mathscr{F})$を可測空間,$\mathscr{G} \subset \mathscr{F}$を部分$\sigma$-加法族,
			$\mu$を$\mathscr{F}$上の複素測度とするとき,次の(1)(2)(3)を満たす写像
			\begin{align}
				\mu(\cdot\, |\, \mathscr{G})(\cdot):
				\mathscr{F} \times X \ni (A,x) \longmapsto \mu(A\, |\, \mathscr{G})(x) \in \C
			\end{align}
			を$\mathscr{G}$の下での$\mu$の正則条件付複素測度
			(regular conditional complex measure of $\mu$ with respect to $\mathscr{G}$)と呼ぶ:
			\begin{description}
				\item[(1)] 任意の$x \in X$で$\mathscr{F} \ni A \longmapsto \mu(A\, |\, \mathscr{G})(x)$は複素測度である.
				\item[(2)] 任意の$A \in \mathscr{F}$で$X \ni x \longmapsto \mu(A\, |\, \mathscr{G})(x)$は
					$\mathscr{G}/\borel{\C}$-可測かつ$|\mu|$-可積分である.
				\item[(3)] 任意の$A \in \mathscr{F}$と$B \in \mathscr{G}$に対し次を満たす:
					\begin{align}
						\mu(A \cap B) = \int_B \mu(A\, |\, \mathscr{G})\ d|\mu|.
					\end{align}
			\end{description}
		\end{dfn}
	\end{screen}
	
	\begin{screen}
		\begin{thm}[正則条件付複素測度の一意性]
			$(X,\mathscr{F})$を可測空間,$\mathscr{G} \subset \mathscr{F}$を部分$\sigma$-加法族,
			$\mu$を$\mathscr{F}$上の複素測度とし,
			$\mu$に対し$\mathscr{G}$の下での正則条件付複素測度
			$\mu(\cdot\, |\, \mathscr{G})(\cdot)$と$\nu(\cdot\, |\, \mathscr{G})(\cdot)$
			が存在しているとする.このとき,$\mathscr{F}$が可算族で生成されるなら,
			或る$|\mu|$-零集合$N \in \mathscr{G}$が存在して次が成立する:
			\begin{align}
				\mu(A\, |\, \mathscr{G})(x) = \nu(A\, |\, \mathscr{G})(x),
				\quad (\forall A \in \mathscr{F},\ \forall x \in X \backslash N).
			\end{align}
		\end{thm}
	\end{screen}
	
	\begin{prf}
		$\mathscr{F}$を生成する可算族を$\mathscr{A}$とし,
		\begin{align}
			\mathscr{U} \coloneqq \Set{\bigcap_{i=1}^n A_i}{A_i \in \mathscr{A},\ n = 1,2,\cdots}
		\end{align}
		により可算乗法族を定める.$\mathscr{A}$は$\mathscr{F}$を生成するから
		$\sgmalg{\mathscr{U}} = \mathscr{F}$となり,
		任意の$U \in \mathscr{U}$に対し
		\begin{align}
			\int_{B} \mu(U\, |\, \mathscr{G})\ d|\mu|
			= \int_{B} \nu(U\, |\, \mathscr{G})\ d|\mu|,
			\quad (\forall B \in \mathscr{G})
		\end{align}
		が満たされるから$N_U \coloneqq \left\{\mu(U\, |\, \mathscr{G}) \neq \nu(U\, |\, \mathscr{G}) \right\}$
		は$\mathscr{G}$の$|\mu|$-零集合となる.$N \coloneqq \bigcup_{U \in \mathscr{U}} N_U$とおけば
		\begin{align}
			\mathscr{D} \coloneqq \Set{A \in \mathscr{F}}{\mu(A\, |\, \mathscr{G})(x) = \nu(A\, |\, \mathscr{G})(x),\ 
			\forall x \in X \backslash N}
		\end{align}
		によりDynkin族が定まり,$\mathscr{D}$は$\mathscr{U}$を含むから
		Dynkin族定理より定理の主張が従う.
		\QED
	\end{prf}
	
	\begin{screen}
		\begin{thm}[正則条件付複素測度の存在]
			$(X,\mathscr{F})$を可測空間,$\mathscr{G} \subset \mathscr{F}$を部分$\sigma$-加法族,
			$\mu$を$\mathscr{F}$上の複素測度とする.
			また$\mathscr{F}$が可算族で生成され,かつ或る
			コンパクトクラス$\mathcal{K}$が存在して,
			任意の$\epsilon > 0$と$A \in \mathscr{F}$に対し
			\begin{align}
				A_\epsilon \subset K_\epsilon \subset A,\quad |\mu|(A \backslash A_\epsilon) < \epsilon
			\end{align}
			を満たす$K_\epsilon \in \mathcal{K},\ A_\epsilon \in \mathscr{F}$が取れると仮定する.
			このとき$\mathscr{G}$の下での$\mu$の正則条件付複素測度が存在する.
		\end{thm}
	\end{screen}
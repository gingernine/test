\subsection{素元分解}
	\begin{screen}
		\begin{thm}[同伴な要素が生成するイデアルは等しい]
		\label{thm:associate_elements_principle_ideals_are_equivalent}
			$(R,\sigma,\mu)$を環とし,$a,b$を$R$の要素とする.このとき次が成り立つ:
			\begin{align}
				a \sim b \Longrightarrow R\inprod<a> = R\inprod<b>.
			\end{align}
		\end{thm}
	\end{screen}
	
	\begin{prf}
		$a \mid b$ならば$b \in R\inprod<a>$となるから$R\inprod<b> \subset R\inprod<a>$
		が成り立つ.同様に$b \mid a$ならば$R\inprod<a> \subset R\inprod<b>$となるので
		\begin{align}
			a \sim b \Longrightarrow R\inprod<a> = R\inprod<b>
		\end{align}
		が得られる.
		\QED
	\end{prf}
	
	\begin{screen}
		\begin{thm}[単項イデアル整域において素元が生成するイデアルは極大イデアルである]
		\label{thm:principal_ideal_domain_maximum_ideal}
			$(R,\sigma,\mu)$を単項イデアル整域するとき,$p$を$(R,\sigma,\mu)$の素元とすれば,
			$R\inprod<p>$は$(R,\sigma,\mu)$の極大イデアルとなる.
		\end{thm}
	\end{screen}
	
	\begin{prf}
		$I$を$(R,\sigma,\mu)$のイデアルで
		\begin{align}
			R\inprod<p> \subset I
			\label{eq:thm_principal_ideal_domain_maximum_ideal_1}
		\end{align}
		を満たすものとする.$(R,\sigma,\mu)$は単項イデアル整域であるから
		$I$に対して$R$の或る要素$d$が存在し
		\begin{align}
			I = R\inprod<d>
		\end{align}
		となるが,(\refeq{eq:thm_principal_ideal_domain_maximum_ideal_1})より
		$p \in R\inprod<d>$となるので
		\begin{align}
			d \mid p
		\end{align}
		が成り立つ.$p$は素元であるから$d \sim \epsilon \vee d \sim p$となり
		(ただし$\epsilon$は$(R,\sigma,\mu)$の単位元を表す),
		定理\ref{thm:associate_elements_principle_ideals_are_equivalent}より
		\begin{align}
			&d \sim 1 \Longrightarrow R\inprod<d> = R\inprod<\epsilon> = I, \\
			&d \sim p \Longrightarrow R\inprod<d> = R\inprod<p>
		\end{align}
		が成り立つので場合分け法則より
		\begin{align}
			I = R \vee I = R\inprod<p>
		\end{align}
		が成立する.ゆえに$R\inprod<p>$は極大イデアルである.
		\QED
	\end{prf}
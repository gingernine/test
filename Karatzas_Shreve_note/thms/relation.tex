\subsection{関係}
	\begin{screen}
		\begin{dfn}[順序対]
			集合$x,y$に対し,
			\begin{align}
				(x,y) = \{\{x\},\{x,y\}\}
			\end{align}
			で定義される類$(x,y)$を$x$と$y$の{\bf 順序対}\index{じゅんじょつい@順序対}
			{\bf (ordered pair)}と呼ぶ.
		\end{dfn}
	\end{screen}
	
	\begin{screen}
		\begin{thm}[順序対は集合]\mbox{}
			\begin{description}
				\item[(1)] $\forall x,y \in V\ \left(\ (x,y) \in V\ \right)$.
				\item[(2)] $\forall x,y,s,t \in V\ 
					\left(\ (x,y)=(s,t) \Longleftrightarrow x=s \wedge y=t\ \right)$.
			\end{description}
		\end{thm}
	\end{screen}
	
	\begin{screen}
		\begin{dfn}[Cartesian積]
			類$a,b$に対し,$a \times b$を
			\begin{align}
				a \times b = \Set{x}{\exists s \in a\ \exists t \in b\ (\ x=(s,t)\ )}
			\end{align}
			で定め,これを$a$と$b$の{\bf Cartesian 積}\index{Cartesian せき@Cartesian 積}
			{\bf (Cartesian product)}と呼ぶ.
		\end{dfn}
	\end{screen}
	
	\begin{comment}
	\monologue{
		院生「類$a$と類$b$のCartesian 積は
			\begin{align}
				a \times b = \Set{(s,t)}{s \in a \wedge t \in b} 
			\end{align}
			と簡略して書かれることも多いです.ところで他の本やネットなどを見ていると
			Cartesian 積を直積とも呼んでいるそうです.本稿でも後で直積というものを定義いたしますが,
			本稿ではCartesian 積と直積を明確に区別いたします.
			これは巷にあふれる直積の定義の不自然さを解消するためです.
			どういう点が不自然であるか簡単に説明いたしましょう.
			まだ有限とか数だとか定義していませんが,説明の便宜のために使用いたします.
			よく見る直積の定義だと,有限か有限でないかで直積の定め方が変わります.
			\begin{align}
				I_1 \times I_2 \times \cdots \times I_n 
				= \Set{(x_1,x_2,\cdots,x_n)}{x_1 \in I_1 \wedge x_2 \in I_2 \wedge
				\cdots \wedge x_n \in I_n}
			\end{align}
			そして
			\begin{align}
				I_1 \times I_2 \times \cdots \times I_n 
				= \prod_{i=1}^n I_i
			\end{align}
			と書いている.ここで
			$\prod_{i=1}^n I_i$は$\prod_{i\in\{1,2,\cdots,n\}} I_i$の別の記法です.
			他方$I$を$\{1,2,\cdots,n\}$から$V$への写像と見ることもできますから
			\begin{align}
				\prod_{i=1}^n I_i = \Set{f}{f:\{1,2,\cdots,n\} \longrightarrow V \wedge \forall i \in \{1,2,\cdots,n\}\ (\ f(i) \in I_i\ )}
			\end{align}
			となるはずです.食い違います.
			」
	}
	\end{comment}
	
	\begin{screen}
		\begin{dfn}[関係]
			$V \times V$の部分類を{\bf 関係}\index{かんけい@関係}{\bf (relation)}と呼ぶ.
		\end{dfn}
	\end{screen}
	
	いま,関係$E$を
	\begin{align}
		E = \Set{x}{\exists s,t \in V\ (\ x=(s,t) \wedge s = t\ )}
	\end{align}
	と定めてみる.このとき$E$は次の性質を満たす:
	\begin{description}
		\item[(a)] $\forall x \in V\ (\ (x,x) \in E\ )$.
		\item[(b)] $\forall x,y \in V\ (\ (x,y) \in E \Longrightarrow (y,x) \in E\ )$.
		\item[(c)] $\forall x,y,z \in V\ (\ (x,y) \in E \wedge (y,z) \in E \Longrightarrow (x,z) \in E\ )$.
	\end{description}
	性質(a)を反射律と呼ぶ.性質(b)を対称律と呼ぶ.性質(c)を推移律と呼ぶ.
	
	\begin{screen}
		\begin{dfn}[同値関係]
			$a$を集合とするとき,
			\begin{description}
				\item[反射律] $\forall x \in a\ (\ (x,x) \in R\ )$.
				\item[対称律] $\forall x,y \in a\ (\ (x,y) \in R \Longrightarrow (y,x) \in R\ )$.
				\item[推移律] $\forall x,y,z \in a\ (\ (x,y) \in R \wedge (y,z) \in R \Longrightarrow (x,z) \in R\ )$.
			\end{description}
			を満たす関係$R$を$a$上の{\bf 同値関係}\index{どうちかんけい@同値関係}
			{\bf (equivalence relation)}と呼ぶ.
		\end{dfn}
	\end{screen}
	
	\monologue{
		院生「集合$a$に対して$R = E \cap (a \times a)$とおけば$R$は$a$上の同値関係となりますね.」
	}
	
	$E$とは別の関係$O$を
	\begin{align}
		O = \Set{x}{\exists s,t \in V\ (\ x=(s,t) \wedge s \subset t\ )}
	\end{align}
	により定めてみる.このとき$O$は次の性質を満たす:
	\begin{description}
		\item[(a)] $\forall x \in V\ (\ (x,x) \in O\ )$.
		\item[(b')] $\forall x,y \in V\ (\ (x,y) \in O \wedge (y,x) \in O \Longrightarrow x=y\ )$.
		\item[(c)] $\forall x,y,z \in V\ (\ (x,y) \in O \wedge (y,z) \in O \Longrightarrow (x,z) \in O\ )$.
	\end{description}
	性質(b')を反対称律と呼ぶ.
	
	\begin{screen}
		\begin{dfn}[順序関係]
			$a$を集合とするとき,
			\begin{description}
				\item[反射律] $\forall x \in a\ (\ (x,x) \in R\ )$.
				\item[反対称律] $\forall x,y \in a\ (\ (x,y) \in R \wedge (y,x) \in R \Longrightarrow x=y\ )$.
				\item[推移律] $\forall x,y,z \in a\ (\ (x,y) \in R \wedge (y,z) \in R \Longrightarrow (x,z) \in R\ )$.
			\end{description}
			を満たす関係$R$を$a$上の{\bf 順序}\index{じゅんじょ@順序}{\bf (order)}と呼び,
			対$(a,R)$を{\bf 順序集合}\index{じゅんじょしゅうごう@順序集合}{\bf (ordered set)}
			と呼ぶ.特に
			\begin{align}
				\forall x,y \in a\ (\ (x,y) \in R \vee (y,x) \in R\ )
			\end{align}
			が成り立つとき,$R$を$a$上の{\bf 全順序}\index{ぜんじゅんじょ@全順序}
			{\bf (total order)}と呼び$(a,R)$を全順序集合と呼ぶ.			
		\end{dfn}
	\end{screen}
	
	\monologue{
		院生「反射律と推移律のみを満たす関係を{\bf 前順序}\index{ぜんじゅんじょ@前順序}
			{\bf (preorder)}と呼びます.また全順序は{\bf 線型順序}
			\index{せんけいじゅんじょ@線型順序}{\bf (linear order)}とも呼ばれます.
			また表記上の問題ですが,$R$を集合$a$上の順序関係として
			\begin{align}
				x \leq y \Longleftrightarrow (x,y) \in R
			\end{align}
			で記号$\leq$を定めるとき,$(a,\leq)$と順序対の形で表して
			これを順序集合と呼ぶこともあります.」
	}
\section{関係}
	\begin{screen}
		\begin{dfn}[順序対]
			$a,b$を類とするとき,
			\begin{align}
				(a,b) \coloneqq \{\{a\},\{a,b\}\}
			\end{align}
			で定義される類$(a,b)$を$a$と$b$の{\bf 順序対}\index{じゅんじょつい@順序対}
			{\bf (ordered pair)}と呼ぶ.
		\end{dfn}
	\end{screen}
	
	\begin{screen}
		\begin{thm}[集合の順序対は集合]
			$a,b$を類とするとき次が成り立つ:
			\begin{align}
				\set{a} \wedge \set{b} \Longrightarrow \set{(a,b)}.
			\end{align}
		\end{thm}
	\end{screen}
	
	\begin{prf}
		$a,b$を集合とする.このとき定理\ref{thm:pair_of_proper_classes_is_emptyset}より
		$\{a\}$と$\{a,b\}$は共に集合となり,再び定理\ref{thm:pair_of_proper_classes_is_emptyset}より
		$\{\{a\},\{a,b\}\}$は集合となる.
		\QED
	\end{prf}
	
	\begin{screen}
		\begin{thm}[順序対の相等性]\label{thm:equality_of_ordered_pairs}
			$a,b,c,d$を類とするとき次が成り立つ:
			\begin{align}
				\set{a} \wedge \set{b} \wedge \set{c} \wedge \set{d} \Longrightarrow
				\left(\, (a,b) = (c,d) \Longleftrightarrow a=c \wedge b=d\, \right).
			\end{align}
		\end{thm}
	\end{screen}
	
	\begin{screen}
		\begin{dfn}[Cartesian積]
			類$a,b$に対し,$a \times b$を
			\begin{align}
				a \times b \coloneqq \Set{x}{\exists s \in a\ \exists t \in b\ (\ x=(s,t)\ )}
			\end{align}
			で定め,これを$a$と$b$の{\bf Cartesian 積}\index{Cartesian せき@Cartesian 積}
			{\bf (Cartesian product)}と呼ぶ.
		\end{dfn}
	\end{screen}
	
	$a \times b$は
	\begin{align}
		\Set{(s,t)}{s \in a \wedge t \in b} 
	\end{align}
	と簡略して書かれることも多い.
	
	\begin{comment}
	\monologue{
		院生「類$a$と類$b$のCartesian 積は
			\begin{align}
				a \times b = \Set{(s,t)}{s \in a \wedge t \in b} 
			\end{align}
			と簡略して書かれることも多いです.ところで他の本やネットなどを見ていると
			Cartesian 積を直積とも呼んでいるそうです.本稿でも後で直積というものを定義いたしますが,
			本稿ではCartesian 積と直積を明確に区別いたします.
			これは巷にあふれる直積の定義の不自然さを解消するためです.
			どういう点が不自然であるか簡単に説明いたしましょう.
			まだ有限とか数だとか定義していませんが,説明の便宜のために使用いたします.
			よく見る直積の定義だと,有限か有限でないかで直積の定め方が変わります.
			\begin{align}
				I_1 \times I_2 \times \cdots \times I_n 
				= \Set{(x_1,x_2,\cdots,x_n)}{x_1 \in I_1 \wedge x_2 \in I_2 \wedge
				\cdots \wedge x_n \in I_n}
			\end{align}
			そして
			\begin{align}
				I_1 \times I_2 \times \cdots \times I_n 
				= \prod_{i=1}^n I_i
			\end{align}
			と書いている.ここで
			$\prod_{i=1}^n I_i$は$\prod_{i\in\{1,2,\cdots,n\}} I_i$の別の記法です.
			他方$I$を$\{1,2,\cdots,n\}$から$V$への写像と見ることもできますから
			\begin{align}
				\prod_{i=1}^n I_i = \Set{f}{f:\{1,2,\cdots,n\} \longrightarrow V \wedge \forall i \in \{1,2,\cdots,n\}\ (\ f(i) \in I_i\ )}
			\end{align}
			となるはずです.食い違います.
			」
	}
	\end{comment}
	
	二つの類を用いて得られる最大のCartesian積は
	\begin{align}
		\Set{x}{\exists s,t\, (\, x=(s,t)\, )}
	\end{align}
	で与えられ,これは$\Univ \times \Univ$に等しい.
	
	\begin{screen}
		\begin{thm}[$\Univ$のCartesian積]
			次が成り立つ:
			\begin{align}
				\Univ \times \Univ = \Set{x}{\exists s,t\, (\, x=(s,t)\, )}.
			\end{align}
		\end{thm}
	\end{screen}
	
	\begin{prf}
		$\Univ \times \Univ$は形式的には$\Set{x}{\exists s,t \in \Univ\, (\, x=(s,t)\, )}$で定められるが,
		正式には
		\begin{align}
			\Set{x}{\exists s\, \left(\, s = s \wedge \exists t\, (\, t=t \wedge x=(s,t)\, )\, \right)}
		\end{align}
		で定められる.ここで$\chi$を$\mathcal{L}$の任意の対象として
		\begin{align}
			\exists s\, \left(\, s = s \wedge \exists t\, (\, t=t \wedge \chi=(s,t)\, )\, \right)
			\Longleftrightarrow \exists s\, \left(\, \exists t\, (\, \chi=(s,t)\, )\, \right)
			\label{eq:thm_Cartesian_product_of_the_Universe}
		\end{align}
		が成り立つことを示す.いま$\exists s\, \left(\, s = s \wedge \exists t\, (\, t=t \wedge \chi=(s,t)\, )\, \right)$が成り立っていると仮定する.このとき
		\begin{align}
			\sigma \coloneqq \varepsilon s\, \left(\, s = s \wedge \exists t\, (\, t=t \wedge \chi=(s,t)\, )\, \right)
		\end{align}
		とおけば存在記号に関する規則より
		\begin{align}
			\sigma = \sigma \wedge \exists t\, (\, t=t \wedge \chi=(\sigma,t)\, )
		\end{align}
		が成立し,このとき$\wedge$の除去より$\exists t\, (\, t=t \wedge \chi=(\sigma,t)\, )$が成り立つので
		\begin{align}
			\tau \coloneqq \varepsilon t\, (\, t=t \wedge \chi=(\sigma,t)\, )
		\end{align}
		とおけば
		\begin{align}
			\tau = \tau \wedge \chi = (\sigma,\tau)
		\end{align}
		が成立する.$\wedge$の除去より$\chi = (\sigma,\tau)$となり,存在記号に関する規則より
		\begin{align}
			\exists t\, (\, \chi = (\sigma,t)\, )
		\end{align}
		が成立し,再び存在記号に関する規則から
		\begin{align}
			\exists s\, \left(\, \exists t\, (\, \chi = (s,t)\, )\, \right)
		\end{align}
		が成立する.ここで演繹法則を適用すれば
		\begin{align}
			\exists s\, \left(\, s = s \wedge \exists t\, (\, t=t \wedge \chi=(s,t)\, )\, \right)
			\Longrightarrow \exists s\, \left(\, \exists t\, (\, \chi=(s,t)\, )\, \right)
		\end{align}
		が得られる.逆に$\exists s\, \left(\, \exists t\, (\, \chi=(s,t)\, )\, \right)$が成り立っているとすると,
		\begin{align}
			\sigma' \coloneqq \varepsilon s\, \left(\, \exists t\, (\, \chi=(s,t)\, )\, \right)
		\end{align}
		とおけば存在記号に関する規則より
		\begin{align}
			\exists t\, (\, \chi=(\sigma',t)\, )
		\end{align}
		が成立し,
		\begin{align}
			\tau' \coloneqq \varepsilon t\, (\, \chi=(\sigma',t)\, )
		\end{align}
		とおけば
		\begin{align}
			\chi = (\sigma',\tau')
		\end{align}
		が成立する.ここで定理\ref{thm:any_class_equals_to_itself}より$\tau' = \tau'$が満たされるので
		$\wedge$の導入により
		\begin{align}
			\tau' = \tau' \wedge \chi = (\sigma',\tau')
		\end{align}
		が成り立ち,存在記号に関する規則より
		\begin{align}
			\exists t\, (\, t = t \wedge \chi = (\sigma',t)\, )
		\end{align}
		が成り立つ.同じく$\sigma' = \sigma'$も満たされて
		\begin{align}
			\sigma' = \sigma' \wedge \exists t\, (\, t = t \wedge \chi = (\sigma',t)\, )
		\end{align}
		が成り立ち,存在記号に関する規則より
		\begin{align}
			\exists s\, \left(\, s = s \wedge \exists t\, (\, t = t \wedge \chi = (s,t)\, )\, \right)
		\end{align}
		が成立する.ここに演繹法則を適用すれば
		\begin{align}
			\exists s\, \left(\, \exists t\, (\, \chi = (s,t)\, )\, \right)
			\Longrightarrow \exists s\, \left(\, s = s \wedge \exists t\, (\, t=t \wedge \chi=(s,t)\, )\, \right)
		\end{align}
		が得られる.以上より式(\refeq{eq:thm_Cartesian_product_of_the_Universe})が成立する.ところで類の公理より
		\begin{align}
			\begin{gathered}
				\chi \in \Univ \times \Univ \Longleftrightarrow \exists s\, \left(\, s = s \wedge \exists t\, (\, t=t \wedge \chi=(s,t)\, )\, \right), \\
				\chi \in \Set{x}{\exists s,t\, (\, x=(s,t)\, )} \Longleftrightarrow
				\exists s\, \left(\, \exists t\, (\, \chi=(s,t)\, )\, \right)
			\end{gathered}
		\end{align}
		が成り立つので,含意の推移律から
		\begin{align}
			\chi \in \Univ \times \Univ \Longleftrightarrow \chi \in \Set{x}{\exists s,t\, (\, x=(s,t)\, )}
		\end{align}
		が成立する.そして$\chi$の任意性と
		推論法則\ref{metathm:fundamental_law_of_universal_quantifier}から
		\begin{align}
			\forall y\, \left(\, y \in \Univ \times \Univ \Longleftrightarrow y \in \Set{x}{\exists s,t\, (\, x=(s,t)\, )}\, \right)
		\end{align}
		が従い,外延性の公理より定理の主張が得られる.
		\QED
	\end{prf}
	
	\begin{screen}
		\begin{dfn}[関係]
			$\Univ \times \Univ$の部分類を{\bf 関係}\index{かんけい@関係}{\bf (relation)}と呼ぶ.
			また類$a$に対して
			\begin{align}
				\rel{a} \overset{\mathrm{def}}{\Longleftrightarrow} a \subset \Univ \times \Univ
			\end{align}
			と定める.
		\end{dfn}
	\end{screen}
	
	いま,関係$E$を
	\begin{align}
		E = \Set{x}{\exists s,t\ (\ x=(s,t) \wedge s = t\ )}
	\end{align}
	と定めてみる.このとき$E$は次の性質を満たす:
	\begin{description}
		\item[(a)] $\forall x\ (\ (x,x) \in E\ )$.
		\item[(b)] $\forall x,y\ (\ (x,y) \in E \Longrightarrow (y,x) \in E\ )$.
		\item[(c)] $\forall x,y,z\ (\ (x,y) \in E \wedge (y,z) \in E \Longrightarrow (x,z) \in E\ )$.
	\end{description}
	性質(a)を反射律と呼ぶ.性質(b)を対称律と呼ぶ.性質(c)を推移律と呼ぶ.
	
	\begin{screen}
		\begin{dfn}[同値関係]
			$a$を類とし,$R$を関係とする.$R$が$R \subset a \times a$を満たし,さらに
			\begin{description}
				\item[反射律] $\forall x \in a\ (\ (x,x) \in R\ )$.
				\item[対称律] $\forall x,y \in a\ (\ (x,y) \in R \Longrightarrow (y,x) \in R\ )$.
				\item[推移律] $\forall x,y,z \in a\ (\ (x,y) \in R \wedge (y,z) \in R \Longrightarrow (x,z) \in R\ )$.
			\end{description}
			も満たすとき,$R$を$a$上の{\bf 同値関係}\index{どうちかんけい@同値関係}
			{\bf (equivalence relation)}と呼ぶ.
		\end{dfn}
	\end{screen}
	
	\monologue{
		集合$a$に対して$R = E \cap (a \times a)$とおけば$R$は$a$上の同値関係となります.
	}
	
	$E$とは別の関係$O$を
	\begin{align}
		O = \Set{x}{\exists s,t\ (\ x=(s,t) \wedge s \subset t\ )}
	\end{align}
	により定めてみる.このとき$O$は次の性質を満たす:
	\begin{description}
		\item[(a)] $\forall x\ (\ (x,x) \in O\ )$.
		\item[(b')] $\forall x,y\ (\ (x,y) \in O \wedge (y,x) \in O \Longrightarrow x=y\ )$.
		\item[(c)] $\forall x,y,z\ (\ (x,y) \in O \wedge (y,z) \in O \Longrightarrow (x,z) \in O\ )$.
	\end{description}
	性質(b')を反対称律と呼ぶ.
	
	\begin{screen}
		\begin{dfn}[順序関係]
			$a$を類とし,$R$を関係とする.$R$が$R \subset a \times a$を満たし,さらに
			\begin{description}
				\item[反射律] $\forall x \in a\ (\ (x,x) \in R\ )$.
				\item[反対称律] $\forall x,y \in a\ (\ (x,y) \in R \wedge (y,x) \in R \Longrightarrow x=y\ )$.
				\item[推移律] $\forall x,y,z \in a\ (\ (x,y) \in R \wedge (y,z) \in R \Longrightarrow (x,z) \in R\ )$.
			\end{description}
			も満たすとき,$R$を$a$上の{\bf 順序}\index{じゅんじょ@順序}{\bf (order)}と呼ぶ.
			$a$が集合であるときは対$(a,R)$を{\bf 順序集合}\index{じゅんじょしゅうごう@順序集合}
			{\bf (ordered set)}と呼ぶ.特に
			\begin{align}
				\forall x,y \in a\ (\ (x,y) \in R \vee (y,x) \in R\ )
			\end{align}
			が成り立つとき,$R$を$a$上の{\bf 全順序}\index{ぜんじゅんじょ@全順序}
			{\bf (total order)}と呼ぶ.			
		\end{dfn}
	\end{screen}
	
	\monologue{
		反射律と推移律のみを満たす関係を{\bf 前順序}\index{ぜんじゅんじょ@前順序}
		{\bf (preorder)}と呼びます.また全順序は{\bf 線型順序}
		\index{せんけいじゅんじょ@線型順序}{\bf (linear order)}とも呼ばれます.
	}
	
	\begin{screen}
		\begin{dfn}[上限]
		\end{dfn}
	\end{screen}
	
	\begin{screen}
		\begin{dfn}[整列集合]
			$x$が{\bf 整列集合}\index{せいれつしゅうごう@整列集合}{\bf (wellordered set)}
			であるとは,$x$が集合$a$と$a$上の順序$R$の対$(a,R)$に等しく,
			かつ$a$の空でない任意の部分集合が$R$に関する最小元を持つことをいう.
			またこのときの$R$を{\bf 整列順序}\index{せいれつじゅんじょ@整列順序}
			{\bf (wellorder)}と呼ぶ.
		\end{dfn}
	\end{screen}
	
	\begin{screen}
		\begin{thm}[整列順序は全順序]
		\end{thm}
	\end{screen}
	
	\monologue{
		$A(x)$という式を満たすような$x$が`{\bf 唯一つ存在する}\index{ただひとつそんざいする@唯一つ存在する}'
		という概念を定義しましょう.
		当然$A(x)$を満たす$x$が存在していなくてはいけませんから$\exists x A(x)$は満たされるべきですが,
		これに加えて`$y$と$z$に対して$A(y)$と$A(z)$が成り立つなら$y=z$である'という条件を付けるのです.
		しかしこのままでは`唯一つである'ことを表す式は長くなりますから,新しい記号$\exists !$を用意して簡略します.
		その形式的な定義は下に述べます.
		ちなみに,`唯一つである'ことは`{\bf 一意に存在する}'などの言明によっても示唆されます.
	}
	
	$A$を$\mathcal{L}'$の式とし,$x$を$A$に現れる文字とし,$A$に文字$y,z$が現れないとするとき,
	\begin{align}
		\exists!x A(x) \overset{\mathrm{def}}{\Longleftrightarrow}
		\exists x A(x) \wedge \forall y,z\, (\, A(y) \wedge A(z) \Longrightarrow y=z\, )
	\end{align}
	で$\exists !$の意味を定める.
	
	\begin{screen}
		\begin{dfn}[定義域・値・値域]
			$a$を類とするとき,
			\begin{align}
				\dom{a} \coloneqq \Set{x}{\exists y\, (\, (x,y) \in a\, )},
				\quad \ran{a} \coloneqq \Set{y}{\exists x\, (\, (x,y) \in a\, )}				
			\end{align}
			と定めて,$\dom{a}$を$a$の{\bf 定義域}{\bf (domain)}と呼び,
			$\ran{a}$を$a$の{\bf 値域}{\bf (range)}と呼ぶ.
			また
			\begin{align}
				a(t) \coloneqq \Set{x}{\exists y\, (\, x \in y \wedge (t,y) \in a\, )}
			\end{align}
			とおき,これを$t$の$a$による{\bf 値}\index{あたい@値}{\bf (value)}と呼ぶ.
		\end{dfn}
	\end{screen}
	
	\begin{screen}
		\begin{dfn}[single-valued]
			$a$を類とするとき,$a$が\index{single-valued}{\bf single-valuedである}ということを
			\begin{align}
				\sing{a} \overset{\mathrm{def}}{\Longleftrightarrow}
				\forall x,y,z\, (\, (x,y) \in a \wedge (x,z) \in a \Longrightarrow y=z\, )
			\end{align}
			で定める.
		\end{dfn}
	\end{screen}
	
	\begin{screen}
		\begin{thm}[値とは要素となる順序対の片割れである]\label{thm:value_and_ordered_pair}
			$a$を類とするとき
			\begin{align}
				\sing{a} \Longrightarrow \forall t \in \dom{a}\, \left(\, (t,a(t)) \in a\, \right).
			\end{align}
		\end{thm}
	\end{screen}
	
	\begin{sketch}
		$\sing{a}$が成り立っていると仮定する.このとき$t$を$\dom{a}$の任意の要素とすれば,
		\begin{align}
			(t,\eta) \in a
		\end{align}
		を満たす$\eta$が取れる.この$\eta$が$a(t)$に等しいことを示せば良い.
		いま$x$を任意の集合とする.
		\begin{align}
			x \in \eta
		\end{align}
		が成り立っているとすると
		\begin{align}
			\exists y\, (\, x \in y \wedge (t,y) \in a\, )
		\end{align}
		が従うので
		\begin{align}
			x \in a(t)
		\end{align}
		となる.ゆえに先ず
		\begin{align}
			x \in \eta \Longrightarrow x \in a(t)
		\end{align}
		が得られた.逆に
		\begin{align}
			x \in a(t)
		\end{align}
		が成り立っているとき,
		\begin{align}
			\xi \coloneqq \varepsilon y\, (\, x \in y \wedge (t,y) \in a\, )
		\end{align}
		とおけば
		\begin{align}
			x \in \xi \wedge (t,\xi) \in a
		\end{align}
		が満たされるが,$(t,\eta) \in a$と$\sing{a}$より
		\begin{align}
			\xi = \eta
		\end{align}
		となるので,相等性の公理から
		\begin{align}
			x \in \eta
		\end{align}
		も成立する.ゆえに
		\begin{align}
			x \in a(t) \Longrightarrow x \in \eta
		\end{align}
		も得られた.$x$の任意性と外延性の公理から
		\begin{align}
			a(t) = \eta
		\end{align}
		が従う.このとき
		\begin{align}
			(t,\eta) = (t,a(t))
		\end{align}
		となり,$(t,\eta) \in a$と相等性の公理から
		\begin{align}
			(t,a(t)) \in a
		\end{align}
		が満たされる.以上を総合すれば
		\begin{align}
			\sing{a} \Longrightarrow \forall t \in \dom{a}\, \left(\, (t,a(t)) \in a\, \right)
		\end{align}
		が出る.
		\QED
	\end{sketch}
	
	\begin{screen}
		\begin{thm}[single-valuedならば値は一意]
		\label{thm:uniqueness_of_values}
			$a$を類とするとき
			\begin{align}
				\sing{a} \Longrightarrow
				\forall s,t \in \dom{a}\, (\, s=t \Longrightarrow a(s) = a(t)\, ).
			\end{align}
		\end{thm}
	\end{screen}
	
	\begin{sketch}%[\hyperlink{prf:thm_uniqueness_of_values}{証明P. \pageref{prf:thm_uniqueness_of_values}}]
		$\sing{a}$が成り立っていると仮定する.$s,t$を$\dom{a}$の任意の要素とすれば,
		定理\ref{thm:value_and_ordered_pair}より
		\begin{align}
			(s,a(s)) \in a \wedge (t,a(t)) \in a
		\end{align}
		が成立する.このとき
		\begin{align}
			s = t
		\end{align}
		ならば
		\begin{align}
			(s,a(s)) = (t,a(s))
		\end{align}
		となるので
		\begin{align}
			(t,a(s)) \in a
		\end{align}
		が従い,$\sing{a}$と$(t,a(t)) \in a$から
		\begin{align}
			a(s) = a(t)
		\end{align}
		が成立する.ゆえに
		\begin{align}
			s=t \Longrightarrow a(s) = a(t)
		\end{align}
		が示された.
		\QED
	\end{sketch}
	
	\begin{screen}
		\begin{dfn}[写像] $f,a,b$を類とするとき,以下の概念と$\mathcal{L}'$における派生記号を定める.
			\begin{itemize}
				\item $f$が{\bf 写像}\index{しゃぞう@写像}{\bf (mapping)}であるということ:
					\begin{align}
						\fnc{f} \overset{\mathrm{def}}{\Longleftrightarrow}
						\rel{f} \wedge \sing{f}.
					\end{align}
				
				\item $f$が$a$上の写像であるということ:
					\begin{align}
						f \fon a \overset{\mathrm{def}}{\Longleftrightarrow}
						\fnc{f} \wedge \dom{f} = a.
					\end{align}
					
				\item $f$が$a$から$b$への写像であるということ:
					\begin{align}
						f:a \longrightarrow b \overset{\mathrm{def}}{\Longleftrightarrow}
						f \fon a \wedge \ran{f} \subset b.
					\end{align}
				
				\item $f$が$a$から$b$への{\bf 単射}\index{たんしゃ@単射}{\bf (injection)}であるということ:
					\begin{align}
						f:a \inj b \overset{\mathrm{def}}{\Longleftrightarrow}
						f:a \longrightarrow b \wedge \forall x,y,z\, (\, (x,z) \in f \wedge (y,z) \in f
						\Longrightarrow x=y\, ).
					\end{align}
					
				\item $f$が$a$から$b$への{\bf 全射}\index{ぜんしゃ@全射}{\bf (surjection)}であるということ:
					\begin{align}
						f:a \srj b \overset{\mathrm{def}}{\Longleftrightarrow}
						f:a \longrightarrow b \wedge \forall y \in b\, \exists x \in a\, (\, (x,y) \in f\, ).
					\end{align}
					
				\item $f$が$a$から$b$への{\bf 全単射}\index{ぜんたんしゃ@全単射}{\bf (bijection)}であるということ:
					\begin{align}
						f:a \bij b \overset{\mathrm{def}}{\Longleftrightarrow}
						f:a \inj b \wedge f:a \srj b.
					\end{align}
			\end{itemize}
		\end{dfn}
	\end{screen}
	
	\begin{screen}
		\begin{thm}[定義域と値が一致する写像は等しい]
		\label{thm:two_functions_with_same_domain_and_values_coincide}
			$f,g$を類とするとき次が成り立つ:
			\begin{align}
				&\fnc{f} \wedge \fnc{g} \\
				&\Longrightarrow  
				\left(\, \dom{f} = \dom{g} \wedge
				\forall t \in \dom{f}\, \left(\, f(t) = g(t)\, \right) \Longrightarrow f = g\, \right).
			\end{align}
		\end{thm}
	\end{screen}
	
	\begin{prf}
		いま$\left(\, \fnc{f} \wedge \fnc{g}\, \right) \wedge \left(\, \dom{f} = \dom{g}\, \right)$と
		$\forall t\, \left(\, t \in \dom{f} \Longrightarrow f(t) = g(t)\, \right)$が成り立っていると仮定する.
		このとき$\chi$を$\mathcal{L}$の任意の対象として$\chi \in f$が満たされているとすれば,
		$f \subset \Univ \times \Univ$より
		\begin{align}
			\exists s\, \left(\, \exists t\, (\, \chi = (s,t)\, )\, \right)
		\end{align}
		が成立する.ここで
		$\sigma \coloneqq \varepsilon s\, \left(\, \exists t\, (\, \chi = (s,t)\, )\, \right)$とおけば
		存在記号に関する規則より
		\begin{align}
			\exists t\, (\, \chi = (\sigma,t)\, )
		\end{align}
		が成立し,更に$\tau \coloneqq \varepsilon t\, (\, \chi = (\sigma,t)\, )$とおけば
		\begin{align}
			\chi = (\sigma,\tau)
		\end{align}
		が成立する.$\chi \in f$と相等性の公理より
		\begin{align}
			(\sigma,\tau) \in f
		\end{align}
		が従い,存在記号に関する規則より
		\begin{align}
			\exists y\, (\, (\sigma,y) \in f\, )
		\end{align}
		が成立するので
		\begin{align}
			\sigma \in \dom{f}
		\end{align}
		となる.このとき$\fnc{f}$と定理\ref{thm:value_of_a_mapping}より
		\begin{align}
			(\sigma,f(\sigma)) \in f
		\end{align}
		が成立し,$(\sigma,\tau) \in f \wedge (\sigma,f(\sigma)) \in f$と$\sing{f}$が満たされるので
		\begin{align}
			\tau = f(\sigma)
		\end{align}
		が成り立つ.他方で$\forall t\, \left(\, t \in \dom{f} \Longrightarrow f(t) = g(t)\, \right)$と
		推論法則\ref{metathm:fundamental_law_of_universal_quantifier}より
		\begin{align}
			f(\sigma) = g(\sigma)
		\end{align}
		が満たされ,相等性の公理より
		\begin{align}
			\tau = g(\sigma)
		\end{align}
		が成り立つ.また$\sigma \in \dom{f}$と相等性の公理より
		\begin{align}
			\sigma \in \dom{g}
		\end{align}
		が成り立ち,定理\ref{thm:value_of_a_mapping}より
		\begin{align}
			(\sigma,g(\sigma)) \in g
		\end{align}
		となるが,$\tau = g(\sigma)$と定理\ref{thm:equality_of_ordered_pairs}より
		\begin{align}
			(\sigma,g(\sigma)) = (\sigma,\tau)
		\end{align}
		が満たされるので,相等性の公理より
		\begin{align}
			(\sigma,\tau) \in g
		\end{align}
		が成り立ち,再び相等性の公理より$\chi \in g$が成り立つ.ここで演繹法則を適用すれば
		\begin{align}
			\chi \in f \Longrightarrow \chi \in g
		\end{align}
		が得られる.$f$と$g$の立場を替えれば$\chi \in g \Longrightarrow \chi \in f$も得られ,
		$\chi$の任意性と推論法則\ref{metathm:fundamental_law_of_universal_quantifier}より
		\begin{align}
			\forall x\, (\, x \in f \Longleftrightarrow x \in g\, )
		\end{align}
		が従う.そして外延性の公理より
		\begin{align}
			f = g
		\end{align}
		が出てくる.最後に演繹法則を二回適用すれば定理の主張が得られる.
		\QED
	\end{prf}
	
	\begin{screen}
		\begin{dfn}[反転]
			$a$を類とするとき,その{\bf 反転}\index{はんてん@反転}{\bf (inverse)}を
			\begin{align}
				a^{-1} \coloneqq 
				\Set{x}{\exists s,t\, (\, x=(s,t) \wedge (t,s) \in a\, )}
			\end{align}
			で定める.
		\end{dfn}
	\end{screen}
	
	\begin{screen}
		\begin{dfn}[像・原像]
			$a,b$を類とするとき,$b$の$a$による像を
			\begin{align}
				a \ast b \coloneqq \Set{y}{\exists x \in b\, (\, (x,y) \in a\, )} 
			\end{align}
			で定める.また
			\begin{align}
				a^{-1} \ast b
			\end{align}
			を$b$の$a$による原像と呼ぶ.
		\end{dfn}
	\end{screen}
	
	\begin{screen}
		\begin{thm}[原像はそこに写される定義域の要素の全体]
			$a,b$を類とするとき,
			\begin{align}
				a^{-1} \ast b = \Set{x}{\exists y \in b\, (\, (x,y) \in a\, )}.
			\end{align}
		\end{thm}
	\end{screen}
	
	\begin{sketch}
		$x$を$a^{-1} \ast b$の要素とすれば,
		\begin{align}
			(y,x) \in a^{-1}
		\end{align}
		を満たす$b$の要素$y$が取れる.このとき
		\begin{align}
			(x,y) \in a
		\end{align}
		となるので
		\begin{align}
			\exists y \in b\, (\, (x,y) \in a\, )
		\end{align}
		が成立し
		\begin{align}
			x \in \Set{x}{\exists y \in b\, (\, (x,y) \in a\, )}
		\end{align}
		が従う.逆に$x$を$\Set{x}{\exists y \in b\, (\, (x,y) \in a\, )}$の要素とすれば,
		\begin{align}
			(x,y) \in a
		\end{align}
		を満たす$b$の要素$y$が取れる.このとき
		\begin{align}
			(y,x) \in a^{-1}
		\end{align}
		となるので
		\begin{align}
			\exists y \in b\, (\, (y,x) \in a^{-1}\, )
		\end{align}
		が成立し
		\begin{align}
			x \in a^{-1} \ast b
		\end{align}
		が従う.
		\QED
	\end{sketch}
	
	\begin{screen}
		\begin{thm}[single-valuedな類の像は値の全体]
			$a,b$を類とするとき,
			\begin{align}
				\sing{a} \wedge b \subset \dom{a} \Longrightarrow
				a \ast b = \Set{x}{\exists t \in b\, (\, x = a(t)\, )}.
			\end{align}
		\end{thm}
	\end{screen}
	
	\begin{screen}
		\begin{thm}[像は制限写像の値域に等しい]
			$a,b$を類とするとき次が成り立つ:
			\begin{align}
				a \ast b = \ran{a|_b}.
			\end{align}
		\end{thm}
	\end{screen}
	
	\begin{screen}
		\begin{thm}[空集合は写像である]\label{thm:emptyset_is_a_mapping}
			以下が成立する.
			\begin{description}
				\item[(イ)] $\fnc{\emptyset}.$
				\item[(ロ)] $\dom{\emptyset} = \emptyset$.
				\item[(ハ)] $\ran{\emptyset} = \emptyset$.
				\item[(ニ)] $\emptyset$は単射である.
			\end{description}
		\end{thm}
	\end{screen}
	
	\begin{prf}\mbox{}
		\begin{description}
			\item[(イ)]
				定理\ref{thm:emptyset_if_a_subset_of_every_class}より
				\begin{align}
					\emptyset \subset \Univ \times \Univ
				\end{align}
				となるので$\emptyset$は関係である.また$x,y,z$を$\mathcal{L}$の任意の対象とすれば,
				定理\ref{thm:emptyset_does_not_contain_any_class}より
				\begin{align}
					(x,y) \notin \emptyset
				\end{align}
				が成り立つので
				\begin{align}
					\left(\, (x,y) \notin \emptyset \vee (x,z) \notin \emptyset\, \right) \vee y = z
				\end{align}
				が成立する.従って
				\begin{align}
					(x,y) \in \emptyset \wedge (x,z) \in \emptyset \Longrightarrow y = z
				\end{align}
				が成立し,$x,y,z$の任意性より
				\begin{align}
					\forall x,y,z\, \left(\, (x,y) \in \emptyset \wedge (x,z) \in \emptyset
					\Longrightarrow y = z\, \right)
				\end{align}
				が成り立つ.よって$\sing{\emptyset}$も満たされる.
		
			\item[(ロ)] $\chi$を$\mathcal{L}$の任意の対象とすれば
				\begin{align}
					\chi \in \dom{\emptyset} \Longleftrightarrow 
					\exists y\, (\, (\chi,y) \in \emptyset\, )
				\end{align}
				が成り立つので,対偶を取れば
				\begin{align}
					\chi \notin \dom{\emptyset} \Longleftrightarrow 
					\forall y\, (\, (\chi,y) \notin \emptyset\, )
				\end{align}
				が従う.定理\ref{thm:emptyset_does_not_contain_any_class}より
				\begin{align}
					\forall y\, (\, (\chi,y) \notin \emptyset\, )
				\end{align}
				が満たされるので
				\begin{align}
					\chi \notin \dom{\emptyset}
				\end{align}
				が従い,$\chi$の任意性より
				\begin{align}
					\forall x\, (\, x \notin \dom{\emptyset}\, )
				\end{align}
				が成立する.そして定理\ref{thm:uniqueness_of_emptyset}より
				\begin{align}
					\dom{\emptyset} = \emptyset
				\end{align}
				が得られる.
			
			\item[(ニ)] 空虚な真により
				\begin{align}
					\forall x,y,z\, \left(\, (x,z) \in \emptyset \wedge (y,z) \in \emptyset \Longrightarrow x = y\, \right)
				\end{align}
				が成り立つから$\emptyset$は単射である.
				\QED
		\end{description}
	\end{prf}
	
	\begin{screen}
		\begin{dfn}[空写像]
			$\emptyset$を{\bf 空写像}\index{くうしゃぞう@空写像}{\bf (empty mapping)}とも呼ぶ.
		\end{dfn}
	\end{screen}
	
	\begin{screen}
		\begin{dfn}[合成]
			$a,b$を類とするとき,$a$と$b$の{\bf 合成}\index{ごうせい@合成}{\bf (composition)}を
			\begin{align}
				a \circ b \coloneqq 
				\Set{x}{\exists s,t\, \left(\, x=(s,t) \wedge
				\exists u\, \left(\, (s,u) \in a \wedge (u,t) \in b\, \right)\, \right)}
			\end{align}
			で定める.
		\end{dfn}
	\end{screen}
	
	\begin{screen}
		\begin{dfn}[族・系]\label{dfn:family_collection}
			$x$を集合$a$から集合$b$への写像とするとき,
			$x$のことを``$a$を添字集合\index{てんじしゅうごう@添字集合}(index set)とする
			$b$の族\index{ぞく@族}(family) (或は系\index{けい@系}(collection))''とも呼び,
			$x(i)$を$x_i$と書いて
			\begin{align}
				(x_i)_{i \in a} \defeq x
			\end{align}
			とも表記する.
		\end{dfn}
	\end{screen}
	族$(x_i)_{i \in a}$は写像$x$と同じであるが,一方で
	丸括弧を中括弧に替えた
	\begin{align}
		\{x_i\}_{i \in a}
	\end{align}
	は
	\begin{align}
		\Set{x(i)}{i \in a}
	\end{align}
	によって定められる集合であって,$(x_i)_{i \in a}$とは別物である.
	
	\begin{screen}
		\begin{thm}[全射ならば原像の像で元に戻る]\label{projective_injective_image_preimage}
			$f,a,b,v$を類とする.$f$が$a$から$b$への写像であるとき
			\begin{align}
				v \subset b \Longrightarrow f \ast \left(f^{-1} \ast v\right) \subset v
			\end{align}
			が成立し,特に$f$が全射なら
			\begin{align}
				f \ast \left(f^{-1} \ast v\right) = v
			\end{align}
			が成り立つ.
		\end{thm}
	\end{screen}
	
	\begin{sketch}
		$y$を$f \ast \left(f^{-1} \ast v\right)$の要素とすると,
		\begin{align}
			y = f(x)
		\end{align}
		を満たす$f^{-1} \ast v$の要素$x$が取れて
		\begin{align}
			f(x) \in v
		\end{align}
		が成り立つから
		\begin{align}
			y \in v
		\end{align}
		が従う.ゆえに
		\begin{align}
			f \ast \left(f^{-1} \ast v\right) \subset v
		\end{align}
		が成立する.$f$が全射であるとき,$y$を$v$の要素とすれば
		\begin{align}
			y = f(x)
		\end{align}
		を満たす$a$の要素$x$が取れて,
		\begin{align}
			x \in f^{-1} \ast v
		\end{align}
		が成り立つので
		\begin{align}
			y \in f \ast \left(f^{-1} \ast v\right)
		\end{align}
		が従う.ゆえに$f$が全射である場合には
		\begin{align}
			v \subset f \ast \left(f^{-1} \ast v\right)
		\end{align}
		も成立して
		\begin{align}
			v = f \ast \left(f^{-1} \ast v\right)
		\end{align}
		となる.
		\QED
	\end{sketch}
	
	\begin{screen}
		\begin{thm}[単射ならば像の原像で元に戻る]\label{projective_injective_image_preimage_2}
			$f,a,b,u$を類とする.$f$が$a$から$b$への写像であるとき
			\begin{align}
				u \subset a \Longrightarrow u \subset f^{-1} \ast \left(f \ast u\right)
			\end{align}
			が成立し,特に$f$が単射なら
			\begin{align}
				u = f^{-1} \ast \left(f \ast u\right)
			\end{align}
			が成り立つ.
		\end{thm}
	\end{screen}
	
	\begin{sketch}
		$x$を$u$の要素とすると
		\begin{align}
			f(x) \in f \ast u
		\end{align}
		が成り立つから
		\begin{align}
			x \in f^{-1} \ast \left(f \ast u\right)
		\end{align}
		が成立する.ゆえに
		\begin{align}
			u \subset f^{-1} \ast \left(f \ast u\right)
		\end{align}
		が成立する.$y$を$f^{-1} \ast \left(f \ast u\right)$の要素とすれば
		\begin{align}
			f(y) \in f \ast u
		\end{align}
		が成り立って
		\begin{align}
			f(y) = f(z)
		\end{align}
		を満たす$u$の要素$z$が取れる.ゆえに,$f$が単射であるとき
		\begin{align}
			y = z
		\end{align}
		となって
		\begin{align}
			y \in u
		\end{align}
		が従い,
		\begin{align}
			f^{-1} \ast \left(f \ast u\right) \subset u
		\end{align}
		が成立する.ゆえに,$f$が単射であれば
		\begin{align}
			u = f^{-1} \ast \left(f \ast u\right)
		\end{align}
		が成立する.
		\QED
	\end{sketch}
\subsection{関係}
	\begin{screen}
		\begin{dfn}[対]
			$a,b$を類とするとき,
			\begin{align}
				\{a,b\} \coloneqq \Set{x}{x = a \vee x = b}
			\end{align}
			で$\{a,b\}$を定義し,これを$a$と$b$の{\bf 対}\index{つい@対}{\bf (pair)}と呼ぶ.
			特に$\{a,a\}$を$\{a\}$と書く.
		\end{dfn}
	\end{screen}
	
	合併を定義したときと同様に,本来$\{a,b\}$は
	\begin{align}
		\Set{x}{x = \varepsilon t(a=t) \vee x = \varepsilon t(b=t)}
	\end{align}
	で定めるのが本式である.しかしこの場合も
	\begin{align}
		\forall x\ (\ x = \varepsilon t(a=t) \vee x = \varepsilon t(b=t)
		\Longleftrightarrow x = a \vee x = b\ )
	\end{align}
	が成立することを根拠にして
	\begin{align}
		\Set{x}{x = a \vee x = b}
	\end{align}
	という表記を認めているのである.正確さも大切であるが,やはり判りやすい方が良いに決まっている.
	
	\begin{screen}
		\begin{axm}[対の公理]
			集合同士の対は集合である.つまり,$a,b$を類とするとき次が成り立つ:
			\begin{align}
				a,b \in \Univ \Longrightarrow \{a,b\} \in \Univ.
			\end{align}
		\end{axm}
	\end{screen}
	
	\begin{screen}
		\begin{thm}[真類の対は空・集合は自分を構成要素とする対に属する]
			$a,b$を類とするとき次が成り立つ:
			\begin{itemize}
				\item $a,b \notin \Univ \Longrightarrow \{a,b\} = \emptyset.$
				\item $a \in \Univ \Longrightarrow a \in \{a,b\}.$
			\end{itemize}
		\end{thm}
	\end{screen}
	
	\begin{prf}
		$a,b \notin \Univ$ならば
		\begin{align}
			\forall x\ (\ a \neq x\ ) \wedge \forall x\ (\ b \neq x\ )
		\end{align}
		が成り立つ.ゆえに
		\begin{align}
			\forall x\ (\ x \notin \{a,b\}\ )
		\end{align}
		となり$\{a,b\} = \emptyset$が従う.次に$a$が集合であると仮定する.このとき
		\begin{align}
			\tau \coloneqq \varepsilon x(a = x)
		\end{align}
		とおけば$\tau = a \vee \tau = b$が成り立つので
		\begin{align}
			\tau \in \{a,b\}
		\end{align}
		となる.そして相等性の公理より$a \in \{a,b\}$が従う.
		\QED
	\end{prf}
	
	\begin{screen}
		\begin{dfn}[順序対]
			$a,b$を類とするとき,
			\begin{align}
				(a,b) \coloneqq \{\{a\},\{a,b\}\}
			\end{align}
			で定義される類$(a,b)$を$a$と$b$の{\bf 順序対}\index{じゅんじょつい@順序対}
			{\bf (ordered pair)}と呼ぶ.
		\end{dfn}
	\end{screen}
	
	\begin{screen}
		\begin{thm}[順序対は集合]\mbox{}
			\begin{description}
				\item[(1)] $\forall x,y\ \left(\ (x,y) \in \Univ\ \right)$.
				\item[(2)] $\forall x,y,s,t\ 
					\left(\ (x,y)=(s,t) \Longleftrightarrow x=s \wedge y=t\ \right)$.
			\end{description}
		\end{thm}
	\end{screen}
	
	\begin{screen}
		\begin{dfn}[Cartesian積]
			類$a,b$に対し,$a \times b$を
			\begin{align}
				a \times b \coloneqq \Set{x}{\exists s \in a\ \exists t \in b\ (\ x=(s,t)\ )}
			\end{align}
			で定め,これを$a$と$b$の{\bf Cartesian 積}\index{Cartesian せき@Cartesian 積}
			{\bf (Cartesian product)}と呼ぶ.
		\end{dfn}
	\end{screen}
	
	\monologue{
		院生「しつこいですが,本来は$\epsilon$記号を用いて
			\begin{align}
				a \times b \coloneqq \Set{y}{\exists s,t\ 
				(\ \varepsilon a(s) \wedge \varepsilon b(t) \wedge y = \varepsilon x (x=(s,t))\ )}
			\end{align}
			と定めるのが正式です.しかし$s,t$を$\mathcal{L}$の対象とするとき,
			\begin{align}
				\begin{gathered}
					s \in a \Longleftrightarrow \epsilon a(s), \\
					t \in b \Longleftrightarrow \epsilon b(t), \\
					(s,t) = \varepsilon x (x = (s,t))
				\end{gathered}
			\end{align}
			となり,$\mathcal{L}$の任意の対象$y$に対して
			\begin{align}
				&\exists s,t\ 
				(\ \varepsilon a(s) \wedge \varepsilon b(t) \wedge y = \varepsilon x (x=(s,t))\ ) \\ 
				&\Longleftrightarrow \exists s \in a\ \exists t \in b\ (\ y=(s,t)\ )
			\end{align}
			が満たされますから解釈上の不具合は無いのですね.また$a \times b$は
			\begin{align}
				\Set{(s,t)}{s \in a \wedge t \in b} 
			\end{align}
			と簡略して書かれることも多いです.」
	}
	
	\begin{comment}
	\monologue{
		院生「類$a$と類$b$のCartesian 積は
			\begin{align}
				a \times b = \Set{(s,t)}{s \in a \wedge t \in b} 
			\end{align}
			と簡略して書かれることも多いです.ところで他の本やネットなどを見ていると
			Cartesian 積を直積とも呼んでいるそうです.本稿でも後で直積というものを定義いたしますが,
			本稿ではCartesian 積と直積を明確に区別いたします.
			これは巷にあふれる直積の定義の不自然さを解消するためです.
			どういう点が不自然であるか簡単に説明いたしましょう.
			まだ有限とか数だとか定義していませんが,説明の便宜のために使用いたします.
			よく見る直積の定義だと,有限か有限でないかで直積の定め方が変わります.
			\begin{align}
				I_1 \times I_2 \times \cdots \times I_n 
				= \Set{(x_1,x_2,\cdots,x_n)}{x_1 \in I_1 \wedge x_2 \in I_2 \wedge
				\cdots \wedge x_n \in I_n}
			\end{align}
			そして
			\begin{align}
				I_1 \times I_2 \times \cdots \times I_n 
				= \prod_{i=1}^n I_i
			\end{align}
			と書いている.ここで
			$\prod_{i=1}^n I_i$は$\prod_{i\in\{1,2,\cdots,n\}} I_i$の別の記法です.
			他方$I$を$\{1,2,\cdots,n\}$から$V$への写像と見ることもできますから
			\begin{align}
				\prod_{i=1}^n I_i = \Set{f}{f:\{1,2,\cdots,n\} \longrightarrow V \wedge \forall i \in \{1,2,\cdots,n\}\ (\ f(i) \in I_i\ )}
			\end{align}
			となるはずです.食い違います.
			」
	}
	\end{comment}
	
	二つの類を用いて得られる最大のCartesian積は
	\begin{align}
		\Set{x}{\exists s,t\ (\ x=(s,t)\ )}
	\end{align}
	で与えられ,これは$\Univ \times \Univ$に等しい.
	
	\begin{screen}
		\begin{dfn}[関係]
			$\Univ \times \Univ$の部分類を{\bf 関係}\index{かんけい@関係}{\bf (relation)}と呼ぶ.
		\end{dfn}
	\end{screen}
	
	いま,関係$E$を
	\begin{align}
		E = \Set{x}{\exists s,t\ (\ x=(s,t) \wedge s = t\ )}
	\end{align}
	と定めてみる.このとき$E$は次の性質を満たす:
	\begin{description}
		\item[(a)] $\forall x\ (\ (x,x) \in E\ )$.
		\item[(b)] $\forall x,y\ (\ (x,y) \in E \Longrightarrow (y,x) \in E\ )$.
		\item[(c)] $\forall x,y,z\ (\ (x,y) \in E \wedge (y,z) \in E \Longrightarrow (x,z) \in E\ )$.
	\end{description}
	性質(a)を反射律と呼ぶ.性質(b)を対称律と呼ぶ.性質(c)を推移律と呼ぶ.
	
	\begin{screen}
		\begin{dfn}[同値関係]
			$a$を類とし,$R$を関係とする.$R$が$R \subset a \times a$を満たし,この上
			\begin{description}
				\item[反射律] $\forall x \in a\ (\ (x,x) \in R\ )$.
				\item[対称律] $\forall x,y \in a\ (\ (x,y) \in R \Longrightarrow (y,x) \in R\ )$.
				\item[推移律] $\forall x,y,z \in a\ (\ (x,y) \in R \wedge (y,z) \in R \Longrightarrow (x,z) \in R\ )$.
			\end{description}
			も満たすとき,$R$を$a$上の{\bf 同値関係}\index{どうちかんけい@同値関係}
			{\bf (equivalence relation)}と呼ぶ.
		\end{dfn}
	\end{screen}
	
	\monologue{
		院生「集合$a$に対して$R = E \cap (a \times a)$とおけば$R$は$a$上の同値関係となりますね.」
	}
	
	$E$とは別の関係$O$を
	\begin{align}
		O = \Set{x}{\exists s,t\ (\ x=(s,t) \wedge s \subset t\ )}
	\end{align}
	により定めてみる.このとき$O$は次の性質を満たす:
	\begin{description}
		\item[(a)] $\forall x\ (\ (x,x) \in O\ )$.
		\item[(b')] $\forall x,y\ (\ (x,y) \in O \wedge (y,x) \in O \Longrightarrow x=y\ )$.
		\item[(c)] $\forall x,y,z\ (\ (x,y) \in O \wedge (y,z) \in O \Longrightarrow (x,z) \in O\ )$.
	\end{description}
	性質(b')を反対称律と呼ぶ.
	
	\begin{screen}
		\begin{dfn}[順序関係]
			$a$を類とし,$R$を関係とする.$R$が$R \subset a \times a$を満たし,この上
			\begin{description}
				\item[反射律] $\forall x \in a\ (\ (x,x) \in R\ )$.
				\item[反対称律] $\forall x,y \in a\ (\ (x,y) \in R \wedge (y,x) \in R \Longrightarrow x=y\ )$.
				\item[推移律] $\forall x,y,z \in a\ (\ (x,y) \in R \wedge (y,z) \in R \Longrightarrow (x,z) \in R\ )$.
			\end{description}
			も満たすとき,$R$を$a$上の{\bf 順序}\index{じゅんじょ@順序}{\bf (order)}と呼ぶ.
			$a$が集合であるときは対$(a,R)$を{\bf 順序集合}\index{じゅんじょしゅうごう@順序集合}
			{\bf (ordered set)}と呼ぶ.特に
			\begin{align}
				\forall x,y \in a\ (\ (x,y) \in R \vee (y,x) \in R\ )
			\end{align}
			が成り立つとき,$R$を$a$上の{\bf 全順序}\index{ぜんじゅんじょ@全順序}
			{\bf (total order)}と呼ぶ.			
		\end{dfn}
	\end{screen}
	
	\monologue{
		院生「反射律と推移律のみを満たす関係を{\bf 前順序}\index{ぜんじゅんじょ@前順序}
			{\bf (preorder)}と呼びます.また全順序は{\bf 線型順序}
			\index{せんけいじゅんじょ@線型順序}{\bf (linear order)}とも呼ばれます.
			また表記上の問題ですが,集合$R$を集合$a$上の順序関係として
			\begin{align}
				x \leq y \Longleftrightarrow (x,y) \in R
			\end{align}
			で記号$\leq$を定めるとき,$(a,\leq)$と順序対の形で表して
			これを順序集合と呼ぶこともあります.」
	}
	
	\begin{screen}
		\begin{dfn}[整列集合]
			$x$が{\bf 整列集合}\index{せいれつしゅうごう@整列集合}{\bf (wellordered set)}
			であるとは,$x$が集合$a$と$a$上の順序$R$の対$(a,R)$に等しく,
			かつ$a$の空でない任意の部分集合が$R$に関する最小元を持つことをいう.
			またこのときの$R$を{\bf 整列順序}\index{せいれつじゅんじょ@整列順序}
			{\bf (wellorder)}と呼ぶ.
		\end{dfn}
	\end{screen}
	
	\begin{screen}
		\begin{thm}[整列順序は全順序]
		\end{thm}
	\end{screen}
	
	\begin{screen}
		\begin{dfn}[定義域・値・値域]
			$a$を類とするとき,
			\begin{align}
				\dom{a} \coloneqq \Set{x}{\exists y\ (\ (x,y) \in a\ )},
				\quad \ran{a} \coloneqq \Set{y}{\exists x\ (\ (x,y) \in a\ )}				
			\end{align}
			と定めて,$\dom{a}$を$a$の{\bf 定義域}{\bf (domain)}と呼び,
			$\ran{a}$を$a$の{\bf 値域}{\bf (range)}と呼ぶ.
			また$\dom{a}$の要素$t$に対して
			\begin{align}
				a(t) \coloneqq \Set{x}{\exists y\ (\ x \in y \wedge (t,y) \in a\ )}
			\end{align}
			とおき,これを$t$の$a$による{\bf 値}\index{あたい@値}{\bf (value)}と呼ぶ.
		\end{dfn}
	\end{screen}
	
	\begin{screen}
		\begin{dfn}[single-valued・写像]
			類$a$に対して
			\begin{align}
				\sing{a} \overset{\mathrm{def}}{\Longleftrightarrow}
				\forall x,y,z\ (\ (x,y) \in a \wedge (x,z) \in a
				\Longrightarrow y=z\ )
			\end{align}
			で$\sing{a}$を定め,$\sing{a}$が成り立つとき
			$a$は\index{single-valued}{\bf single-valuedである}という.また類$f$に対して
			\begin{align}
				\fnc{f} \overset{\mathrm{def}}{\Longleftrightarrow}
				f \subset \Univ \times \Univ \wedge \sing{f}
			\end{align}
			で$\fnc{f}$を定め,$\fnc{f}$を満たす類$f$を
			{\bf 写像}\index{しゃぞう@写像}{\bf (mapping)}と呼ぶ.
		\end{dfn}
	\end{screen}
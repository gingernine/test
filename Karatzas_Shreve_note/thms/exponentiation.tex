\section{累乗}
	\begin{screen}
		\begin{dfn}[順序数の累乗]
			$\alpha$を順序数とし,$\Univ$上の写像$G_\alpha$を
			\begin{align}
				x \longmapsto 
				\begin{cases}
					1 & \mbox{if } \operatorname{dom}(x) = \emptyset \\
					x(\beta) \cdot \alpha & \mbox{if } \beta \in \ON \wedge \operatorname{dom}(x) = \beta \cup \{\beta\} 
					\wedge x(\beta) \in \ON \\
					\bigcup \ran{x} & \mbox{o.w.}
				\end{cases}
			\end{align}
			なる関係により定めると,
			\begin{align}
				\forall \beta \in \ON\, (\, E_\alpha(\beta) = G_\alpha(E_\alpha|_\beta)\, )
			\end{align}
			を満たす$\ON$上の写像$E_\alpha$が取れる.そして順序数$\beta$に対し
			\begin{align}
				\alpha^\beta \defeq E_\alpha(\beta)
			\end{align}
			と定め,これを$\alpha$の$\beta$乗と呼ぶ.
		\end{dfn}
	\end{screen}
	
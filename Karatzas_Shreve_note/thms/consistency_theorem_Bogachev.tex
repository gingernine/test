\section{The Consistency Theorem}
	Karatzas-Shreve より Bogachev の Measure Theory に載っている
	Kolmogorovの拡張定理の方が洗練された簡潔な証明になっているので
	頭に入りやすい.
	
	\begin{screen}
		\begin{dfn}[$K$-正則]
			$S$を位相空間とし,$P$を$(S,\borel{S})$上の確率測度とする.
			$A \in \borel{S}$が$P$に関して$K$-正則であるとは,任意の$\epsilon > 0$に対し
			或るコンパクト集合$K \subset A$が存在して
			\begin{align}
				P(A - K) < \epsilon
			\end{align}
			が満たされることをいう.任意の$A \in \borel{S}$が$P$に関して$K$-正則であるとき,
			$P$は$K$-正則であるという.
		\end{dfn}
	\end{screen}
	
	\begin{itembox}[l]{完備可分距離空間上のBorel確率測度の正則性}
		$(S,d)$を完備可分距離空間とするとき,$(S,\borel{S})$上の
		任意のBorel確率測度$P$は次の意味で正則である:
		\begin{align}
			P(A) = \inf{}{\Set{P(G)}{A \subset G,\ \mbox{$G$は開集合}}}
			= \sup{}{\Set{P(K)}{K \subset A,\ \mbox{$K$はコンパクト}}},
			\quad (\forall A \in \borel{S}).
		\end{align}
	\end{itembox}
	
	\begin{prf}\mbox{}
		\begin{description}
			\item[第一段]
				$S$が$P$に関して$K$-正則であることを示す.
				$S$の可分性により稠密な部分集合$\{x_n\}_{n=1}^\infty$が存在する.
				\begin{align}
					B_n^k \coloneqq \Set{x \in S}{d(x,x_n) \leq \frac{1}{k}},
					\quad (n,k=1,2,\cdots)
				\end{align}
				とおけば,任意の$k$に対して
				\begin{align}
					P\Biggl( S - \bigcup_{n=1}^N B_n^k \Biggr)
					\longrightarrow 0,
					\quad (N \longrightarrow \infty)
				\end{align}
				が満たされる.いま,任意に$\epsilon > 0$を取れば
				各$k$に対し或る$N_k \in \N$が存在して
				\begin{align}
					P\Biggl( S - \bigcup_{n=1}^{N_k} B_n^k \Biggr)
					< \frac{\epsilon}{2^{k+1}}
				\end{align}
				が成立し,
				\begin{align}
					K \coloneqq \bigcap_{k=1}^\infty \left[ \bigcup_{n=1}^{N_k} B_n^k \right]
				\end{align}
				により$K$を定めれば,$K$は閉集合の積であるから閉,すなわち完備である.
				また
				\begin{align}
					K \subset \bigcup_{n=1}^{N_k} B_n^k,
					\quad (\forall k=1,2,\cdots)
				\end{align}
				より$K$は全有界部分集合である.
				$K$は相対距離に関して完備かつ全有界であるから相対位相に関してコンパクトであり,
				従って$S$のコンパクト部分集合である.そして次が成立する:
				\begin{align}
					P(S - K)
					= P\Biggl( \bigcup_{k=1}^\infty \left[S - \bigcup_{n=1}^{N_k} B_n^k \right] \Biggr)
					\leq \sum_{k=1}^\infty P\Biggl(S - \bigcup_{n=1}^{N_k} B_n^k\Biggr)
					< \epsilon.
				\end{align}
				
				
			\item[第二段]
				任意の$A \in \borel{S}$と$\epsilon > 0$に対して,
				或る閉集合$F$及び開集合$G$が存在して
				\begin{align}
					F \subset A \subset G,
					\quad P(G - F) < \epsilon
				\end{align}
				を満たすことを示す.
				\begin{align}
					\mathscr{B} \coloneqq \Set{A \in \borel{S}}{\mbox{任意の$\epsilon$に対し上式を満たす開集合と閉集合が存在する.}}
				\end{align}
				とおけば,$\mathscr{B}$は$\open{S}$を含む$\sigma$-加法族である.
				実際,任意の開集合$G \neq \emptyset$に対し
				\begin{align}
					F_n \coloneqq \Set{x \in S}{d(x,G^c) \geq \frac{1}{n}},
					\quad (n=1,2,\cdots)
				\end{align}
				により閉集合系$(F_n)_{n=1}^\infty$を定めれば
				$\bigcup_{n=1}^\infty F_n = G$が成り立つから
				\begin{align}
					\open{S} \subset \mathscr{B}
				\end{align}
				が従う.また前段の結果より
				$S \in \mathscr{B}$となり,かつ
				\begin{align}
					F \subset A \subset G \quad \Rightarrow \quad 
					G^c \subset A^c \subset F^c
				\end{align}
				より$\mathscr{B}$は補演算で閉じている.更に$A_n \in \mathscr{B},\ (n=1,2,\cdots)$を取れば,
				任意の$\epsilon > 0$に対して
				\begin{align}
					F_n \subset A_n \subset G_n,
					\quad P(G_n - F_n) < \frac{\epsilon}{2^{n+1}}
				\end{align}
				を満たす閉集合$F_n$と開集合$G_n$が存在し,
				\begin{align}
					P\Biggl( \bigcup_{n=1}^\infty G_n - \bigcup_{n=1}^\infty F_n \Biggr)
					\leq P\Biggl( \bigcup_{n=1}^\infty(G_n - F_n) \Biggr)
					< \epsilon
				\end{align}
				が成り立つから十分大きな$N \in \N$に対して
				\begin{align}
					P\Biggl( \bigcup_{n=1}^\infty G_n - \bigcup_{n=1}^N F_n \Biggr)
					< \epsilon
				\end{align}
				となる.$\bigcup_{n=1}^N F_n$は閉集合であり$\bigcup_{n=1}^\infty G_n$
				は開集合であるから$\bigcup_{n=1}^\infty A_n \in \mathscr{B}$が従う.
				
			\item[第三段]
				任意の$A \in \borel{S}$と$\epsilon > 0$に対し,
				或る閉集合$F$と開集合$G$及びコンパクト集合$K$が存在して
				\begin{align}
					F \subset A \subset G,
					\quad P(G - F) < \frac{\epsilon}{2},
					\quad P(S - K) < \frac{\epsilon}{2}
				\end{align}
				を満たす.特に$F \cap K$はコンパクトであり,このとき
				$F \cap K \subset A \subset G$かつ
				\begin{align}
					P(G - F \cap K)
					\leq P(G - F) + P(G - K)
					\leq P(G - F) + P(S - K)
					< \epsilon
				\end{align}
				が成立する.
				\QED
		\end{description}
	\end{prf}
	
	\begin{screen}
		\begin{dfn}[コンパクトクラス]
			$X$を空でない集合,$\mathcal{K}$をその部分集合族とする.
			任意の$\{K_n\}_{n=1}^\infty \subset \mathcal{K}$について,
			$\bigcap_{n=1}^\infty K_n = \emptyset$なら
			$\bigcap_{n=1}^N K_n = \emptyset$を満たす$N \geq 1$が存在するとき,
			$\mathcal{K}$をコンパクトクラスという.
		\end{dfn}
	\end{screen}
	
	\begin{itembox}[l]{Hausdorff空間のコンパクトクラス}
		$\emptyset \neq S$をHausdorff空間とすれば,$S$のコンパクト部分集合の
		全体はコンパクトクラスとなる.
	\end{itembox}
	
	\begin{prf}
		$(K_n)_{n=1}^\infty$をコンパクト部分集合の族とする.
		$\bigcap_{i=1}^\infty K_i = \emptyset$と仮定するとき,
		$K_1 \subset \bigcup_{n=1}^\infty K_n^c = S$より
		\begin{align}
			K_1 = \bigcup_{n=1}^\infty \left( K_n^c \cap K_1 \right)
		\end{align}
		が成立する.Hausdorff空間においてコンパクト部分集合は閉であるから
		$K_n^c \cap K_1$は$K_1$の開集合であり,
		$K_1$のコンパクト性より或る$N \geq 1$が存在して
		\begin{align}
			K_1 = \bigcup_{n=1}^N \left( K_n^c \cap K_1 \right)
			= K_1 \cap \Biggl( \bigcap_{n=1}^N K_n \Biggr)^c
		\end{align}
		が満たされ$\bigcap_{n=1}^N K_n = \emptyset$が従う.
		\QED
	\end{prf}
	
	$T$を空でない集合とし,任意の$t \in T$に対して可測空間$(\Omega_t,\mathscr{B}_t)$が
	定まっているする.このとき任意の空でない有限部分集合$\Lambda \subset T$に対して
	\begin{align}
		\Omega_\Lambda \coloneqq \prod_{t \in \Lambda} \Omega_t,
		\quad \mathscr{B}_\Lambda \coloneqq \bigotimes_{t \in \Lambda} \mathscr{B}_t
	\end{align}
	により可測空間$(\Omega_\Lambda,\mathscr{B}_\Lambda)$を定める.
	ただし$\Lambda = \{t\}$の場合は$\Omega_{\{t\}} = \Omega_t,\ \mathscr{B}_{\{t\}} = \mathscr{B}_t$
	とする.また
	\begin{align}
		\Omega \coloneqq \prod_{t \in T} \Omega_t,
		\quad \mathscr{B}_\Lambda \coloneqq \bigotimes_{t \in T} \mathscr{B}_t
	\end{align}
	とおく.任意の部分集合
	$\Lambda \subset \Lambda'$に対し,$\Omega_{\Lambda'}$から$\Omega_{\Lambda}$への射影を
	$\pi_{\Lambda',\Lambda}$と書き,特に
	$\pi_{T,\Lambda}$を$\pi_{\Lambda}$と書く.
	
	\begin{itembox}[l]{Kolmogorovの拡張定理}
		任意の有限部分集合$\Lambda \subset T$について,
		$(\Omega_\Lambda,\mathscr{B}_\Lambda)$上に確率測度
		$P_\Lambda$が定まっていて,確率測度の族
		$(P_\Lambda)_{\Lambda \subset T:\mathrm{finite}}$が
		次の整合性条件を満たしていると仮定する:
		\begin{align}
			P_{\Lambda'} \circ \pi_{\Lambda',\Lambda}^{-1}
			= P_{\Lambda},
			\quad (\Lambda \subset \Lambda').
		\end{align}
		このとき,任意の$t \in T$に対し近似的コンパクトクラス$\mathcal{K}_t \subset \mathscr{B}_t$
		が存在するなら,$(\Omega,\mathscr{B})$上に次を満たす確率測度$P$がただ一つ存在する:
		\begin{align}
			P \circ \pi_{\Lambda}^{-1} = P_{\Lambda},
			\quad (\forall \Lambda:\mbox{有限}).
		\end{align}
	\end{itembox}
	
	\begin{prf}\mbox{}
		\begin{description}
			\item[第一段]
				$\mathscr{B}$を生成する加法族を
				\begin{align}
					\mathscr{R} \coloneqq
					\Set{\pi_\Lambda^{-1}(B)}{B \in \mathscr{B}_\Lambda,\ \Lambda \subset T:\mbox{有限集合}}
				\end{align}
				とおき,$\mathscr{R}$上の有限加法的測度$\mu$を
				\begin{align}
					\mu\left( \pi_\Lambda^{-1}(B) \right)
					\coloneqq P_\Lambda(B),
					\quad (\forall \pi_\Lambda^{-1}(B) \in \mathscr{R})
				\end{align}
				により定める.実際この$\mu$はwell-definedであり加法性を持つ.
			
			\item[第二段]
				$\mu$がwell-definedであることを示す.
				\begin{align}
					\pi_\Lambda^{-1}(B) = \pi_{\Lambda'}^{-1}(B')
				\end{align}
				であるとき,$\Lambda'' \coloneqq \Lambda \cup \Lambda'$とおけば
				\begin{align}
					\pi_{\Lambda''}^{-1}\left( \pi_{\Lambda'',\Lambda}^{-1}(B) \right)
					= \pi_\Lambda^{-1}(B)
					= \pi_{\Lambda'}^{-1}(B')
					= \pi_{\Lambda''}^{-1}\left( \pi_{\Lambda'',\Lambda'}^{-1}(B') \right)
				\end{align}
				が成り立つから$\pi_{\Lambda'',\Lambda}^{-1}(B) = \pi_{\Lambda'',\Lambda'}^{-1}(B')$
				が従い(全射の性質),整合性条件より
				\begin{align}
					P_\Lambda(B) 
					= P_{\Lambda''} \circ \pi_{\Lambda'',\Lambda}^{-1}(B)
					= P_{\Lambda''} \circ \pi_{\Lambda'',\Lambda'}^{-1}(B')
					= P_{\Lambda'}(B')
				\end{align}
				が満たされ$\mu(\pi_\Lambda^{-1}(B))$の一意性を得る.
				
			\item[第三段]
				$\mu$の加法性を示す.
				\begin{align}
					\pi_{\Lambda_1}^{-1}(B_1) \cap \pi_{\Lambda_2}^{-1}(B_2) = \emptyset
				\end{align}
				であるとき,$\Lambda_3 \coloneqq \Lambda_1 \cup \Lambda_2$とおけば
				\begin{align}
					\emptyset 
					= \pi_{\Lambda_3}^{-1}\left( \pi_{\Lambda_3,\Lambda_1}^{-1}(B_1) \right)
					\cap \pi_{\Lambda_3}\left( \pi_{\Lambda_3,\Lambda_2}^{-1}(B_2) \right)
					= \pi_{\Lambda_3}^{-1}\left( \pi_{\Lambda_3,\Lambda_1}^{-1}(B_1) \cap \pi_{\Lambda_3,\Lambda_2}^{-1}(B_2) \right)
				\end{align}
				となるから$\pi_{\Lambda_3,\Lambda_1}^{-1}(B_1) \cap \pi_{\Lambda_3,\Lambda_2}^{-1}(B_2)
				= \emptyset$が従い(全射の性質),
				\begin{align}
					\mu\left( \pi_{\Lambda_1}^{-1}(B_1) \cup \pi_{\Lambda_2}^{-1}(B_2) \right)
					&= \mu\left[\pi_{\Lambda_3}^{-1}\left( \pi_{\Lambda_3,\Lambda_1}^{-1}(B_1) \right)
					\cup \pi_{\Lambda_3}\left( \pi_{\Lambda_3,\Lambda_2}^{-1}(B_2) \right) \right] \\
					&= \mu\left[ \pi_{\Lambda_3}^{-1}\left( \pi_{\Lambda_3,\Lambda_1}^{-1}(B_1) \cup \pi_{\Lambda_3,\Lambda_2}^{-1}(B_2) \right) \right] \\
					&= P_{\Lambda_3} \left( \pi_{\Lambda_3,\Lambda_1}^{-1}(B_1) \cup \pi_{\Lambda_3,\Lambda_2}^{-1}(B_2) \right) \\
					&= P_{\Lambda_3} \left( \pi_{\Lambda_3,\Lambda_1}^{-1}(B_1) \right)
						+ P_{\Lambda_3} \left( \pi_{\Lambda_3,\Lambda_2}^{-1}(B_2) \right) \\
					&= \mu\left( \pi_{\Lambda_1}^{-1}(B_1) \right)
						+ \mu\left( \pi_{\Lambda_2}^{-1}(B_2) \right)
				\end{align}
				が成立する.
		\end{description}
	\end{prf}
\subsection{連続双対}
	\begin{screen}
		\begin{dfn}[連続双対]
			$\left((X,\sigma_X),(\Phi,+,\bullet),s,\mathscr{O}_X\right)$を位相線型空間とする.
			このとき,$X$上の$\mathscr{O}_X/\mathscr{O}_\Phi$-連続な線型形式の全体を
			$\left((X,\sigma_X),(\Phi,+,\bullet),s,\mathscr{O}_X\right)$の
			{\bf 連続双対}\index{れんぞくそうつい@連続双対}{\bf (continuous dual)}と呼ぶ.
		\end{dfn}
	\end{screen}
	
	$\left((X,\sigma_X),(\Phi,+,\bullet),s,\mathscr{O}_X\right)$を位相線型空間とし,
	その連続双対を$X^*$と書く.
	$X^*$の要素$f$と$g$に対して
	\begin{align}
		x \longmapsto f(x) + g(x)
	\end{align}
	なる写像を対応させる関係を$X^*$上の加法として,これを
	\begin{align}
		\sigma_{X^*}
	\end{align}
	で定める.また$X^*$の要素$f$と$\Phi$の要素$\alpha$に対して
	\begin{align}
		x \longmapsto \alpha \cdot f(x)
	\end{align}
	なる写像を対応させる関係を$\Phi \times X^*$上のスカラ倍として,これを
	\begin{align}
		s^*
	\end{align}
	で定める.すると
	\begin{align}
		\left(\left(X^*,\sigma_{X^*}\right),(\Phi,+,\bullet),s^*\right)
	\end{align}
	は線型空間である.この線型空間を$\left((X,\sigma_X),(\Phi,+,\bullet),s,\mathscr{O}_X\right)$の
	{\bf 連続双対空間}\index{れんぞくそうついくうかん@連続双対空間}{\bf (continuous dual space)}と呼ぶ.
	
	\begin{screen}
		\begin{dfn}[弱位相]
			$\left((X,\sigma_X),(\Phi,+,\bullet),s,\mathscr{O}_X\right)$を位相線型空間とし,
			$X^*$をその連続双対とする.このとき,$X^*$-始位相を$X$上の{\bf 弱位相}\index{じゃくいそう@弱位相}{\bf (weak topology)}と呼び
			\begin{align}
				\weak{X}{X^*}
			\end{align}
			と書く.
		\end{dfn}
	\end{screen}
	
	\begin{screen}
		\begin{thm}
		\end{thm}
	\end{screen}
	
	\begin{screen}
		\begin{thm}[弱位相は局所凸線型位相]
			$(X,\tau)$:位相線型空間,$X'$:$X$上の連続線型形式の集合,このとき
			$X'$-始位相によって$X$は局所凸位相線型空間となる.
		\end{thm}
	\end{screen}
	
	\begin{sketch}
		$X'$-始位相と$X'$で作る近縁系で導入する一様位相は一致する.
		その近縁系は定理\ref{thm:entourages_introducing_vector_topology}の条件を満たすので
		その一様位相は線型位相であり,また局所凸でもある.
	\end{sketch}
\subsection{位相双対空間}
	\begin{screen}
		\begin{dfn}[位相双対空間・位相第二双対空間]
			$X$を位相線型空間とする.
			\begin{description}
				\item[(1)] $X$上の連続な線型形式,つまり$X$から$\Phi$への連続線型写像
					の全体$X^*$を位相双対空間(topological dual space)と呼ぶ.
					また$X^*$の全ての元を連続にする最弱の位相を$X$の弱位相(weak topology)と呼び
					$\sigma(X,X^*)$と書く.
				
				\item[(2)] 任意の$x \in X$に対し$\varphi_x:X^* \ni x^* \longmapsto \inprod<x,x^*>_{X,X^*}$
					により$X^*$上の線型形式$\varphi_x$が定まる.このとき$\Set{\varphi_x}{x \in X}$の元を全て連続にする最弱の位相を$X$の汎弱位相
					(weak$\ast$ topology)と呼び$\sigma(X^*,X)$と書く.
			\end{description}
		\end{dfn}
	\end{screen}
	
	\begin{screen}
		\begin{thm}
		\end{thm}
	\end{screen}
	
	\begin{screen}
		\begin{thm}[弱位相は局所凸線型位相]
			$(X,\tau)$:位相線型空間,$X'$:$X$上の連続線型形式の集合,このとき
			$X'$-始位相によって$X$は局所凸位相線型空間となる.
		\end{thm}
	\end{screen}
	
	\begin{sketch}
		$X'$-始位相と$X'$で作る近縁系で導入する一様位相は一致する.
		その近縁系は定理\ref{thm:entourages_introducing_vector_topology}の条件を満たすので
		その一様位相は線型位相であり,また局所凸でもある.
	\end{sketch}
	\begin{screen}
		\begin{thm}[選択公理と直積]
			次は同値である.
			\begin{description}
				\item[(イ)] $\forall a\, \exists f\, \left(\, f \fon a \wedge \forall x \in a\, (\, x \neq \emptyset \Longrightarrow f(x) \in x\, )\, \right)$
				\item[(ロ)] $\forall a\, \forall h\, \left[\, h \fon a \wedge \forall x \in a\, (\, h(x) \neq \emptyset\, )
				\Longrightarrow \exists f\, \left(\, f \fon a \wedge \forall x \in a\, (\, f(x) \in h(x)\, )\, \right)\, \right]$
			\end{description}
		\end{thm}
	\end{screen}
	
	\begin{prf}\mbox{}
		\begin{description}
			\item[第一段] $a$を集合,$h$を$a$上の恒等写像とする.このとき
				\begin{align}
					a' &\coloneqq a \backslash \{\emptyset\}, \\
					h' &\coloneqq h|_{a'}
				\end{align}
				とおけば
				\begin{align}
					\exists f\, \left(\, f \fon a' \wedge \forall x \in a'\, (\, f(x) \in h'(x)\, )\, \right)
				\end{align}
				が成立する.
				\begin{align}
					f'' \coloneqq \varepsilon f\, \left(\, f \fon a' \wedge \forall x \in a'\, (\, f(x) \in h'(x)\, )\, \right)
				\end{align}
				とおいて
				\begin{align}
					f' &\coloneqq f'' \cup \{(\emptyset,\emptyset)\}, \\
					f &\coloneqq f'|_a
				\end{align}
				とおけば
				\begin{align}
					\exists f\, \left(\, f \fon a \wedge \forall x \in a\, (\, x \neq \emptyset \Longrightarrow f(x) \in x\, )\, \right)
				\end{align}
				が成立する.
			
			\item[第二段] $a$を集合とし,$h$を
				\begin{align}
					h \fon a \wedge \forall x \in a\, (\, h(x) \neq \emptyset\, )
				\end{align}
				を満たす集合とする.
				\begin{align}
					b \coloneqq h \ast a
				\end{align}
				とおけば
				\begin{align}
					\exists f\, \left(\, f \fon b \wedge \forall x \in b\, (\, x \neq \emptyset \Longrightarrow f(x) \in x\, )\, \right)
				\end{align}
				が成り立つので,
				\begin{align}
					\tilde{f} \coloneqq \varepsilon f\, \left(\, f \fon b \wedge \forall x \in b\, (\, x \neq \emptyset \Longrightarrow f(x) \in x\, )\, \right)
				\end{align}
				とおいて
				\begin{align}
					f \coloneqq \Set{x}{\exists s \in a\, \left(\, x=(s,\tilde{f}(h(s)))\, \right)}
				\end{align}
				とおけば
				\begin{align}
					f \fon a \wedge \forall x \in a\, (\, f(x) \in h(x)\, )
				\end{align}
				が成立する.
		\end{description}
	\end{prf}
	
	\begin{screen}
		\begin{dfn}[選択関数]
			$a$を集合とするとき,
			\begin{align}
				f \fon a \wedge \forall x \in a\, (\, x \neq \emptyset \Longrightarrow f(x) \in x\, )
			\end{align}
			を満たす写像$f$を$a$上の{\bf 選択関数}\index{せんたくかんすう@選択関数}{\bf (choice function)}と呼ぶ.
		\end{dfn}
	\end{screen}
	
	\monologue{
		院生「$a = \emptyset$ならば空写像が$a$上の選択関数となりますね.空集合だけでなく,
			どの集合の上にも選択関数が存在することを保証するのが選択公理です.」
	}
	
	\begin{screen}
		\begin{axm}[選択公理]
			いかなる集合の上にも選択関数が存在する:
			\begin{align}
				\forall a\, \exists f\ \left(\ 
				f \fon a \wedge \forall x \in a\ 
				(\ x \neq \emptyset \Longrightarrow f(x) \in x\ )\ \right). 
			\end{align}
		\end{axm}
	\end{screen}
	
	
	\begin{screen}
		\begin{thm}[整列可能定理]
			任意の集合は,或る順序数と全単射で結ばれる:
			\begin{align}
				\forall x\ \exists \alpha \in \ON\ 
				\exists f\ \left( f:\alpha \bij x \right).
			\end{align}
		\end{thm}
	\end{screen}
	
	\begin{prf} $\chi$を任意に与えられた$\mathcal{L}$の対象とする.
		\begin{description}
			\item[第一段]
				$\chi = \emptyset$の場合,
				空写像$\emptyset$が$\emptyset$から$\chi$への全単射となる.従って
				\begin{align}
					\chi = \emptyset \Longrightarrow \exists \alpha \in \ON\ 
				\exists f\ \left( f:\alpha \bij \chi \right)
				\end{align}
				が成立する.
				
			\item[第二段]
				$\chi \neq \emptyset$の場合,
				\begin{align}
					P \coloneqq \dirpro{\chi} \backslash \{\chi\}
				\end{align}
				とおけば
				\begin{align}
					\forall p \in P\, (\, \chi \backslash p \neq \emptyset\, )
				\end{align}
				が満たされるので,選択公理より
				\begin{align}
					g \fon P \wedge \forall p \in P\, (\, g(p) \in \chi \backslash p\, ) 
				\end{align}
				を満たす写像$g$が存在する.
				\begin{align}
					G \coloneqq \Set{z}{\exists s\, \left(\, 
						\left(\, \ran{s} \subset \chi \Longrightarrow z=(s,g(\ran{s}))\, \right) 
						\wedge \left(\, \ran{s} \not\subset \chi \Longrightarrow z=(s,\emptyset)\, \right)\, \right)}
				\end{align}
				で$\Univ$上の写像$G$を定めれば
				\begin{align}
					\forall \alpha \in \ON\, \left(\, F(\alpha) = G(F|_\alpha)\, \right)
				\end{align}
				を満たす類$F$が存在して,$G$の定め方より
				\begin{align}
					\alpha \in \ON \Longrightarrow F(\alpha) = 
					\begin{cases}
						g(F \ast \alpha) & (F \ast \alpha \subsetneq \chi) \\
						\emptyset & (F \ast \alpha = a \vee F \ast \alpha \not\subset \chi)
					\end{cases}
				\end{align}
				が成立する.
				\begin{align}
					\forall \alpha \in \ON\, \left(\, 
					F \ast \alpha \subsetneq \chi \Longrightarrow g(F \ast \alpha) \in \chi\, \right)
				\end{align}
				が満たさるので
				\begin{align}
					F:\ON \longrightarrow \chi \cup \{\emptyset\}
				\end{align}
				が成立することに注意しておく.
			
			\item[第三段]
				$S$を類とするとき
				\begin{align}
					\ord{S} \wedge \forall \alpha \in S\, \left(\, F \ast \alpha \neq a\, \right)
					\Longrightarrow F \ast S \subset a \wedge
					F|_S:S \bij F \ast S \wedge \set{S}
				\end{align}
				が成り立つことを示す.
		\end{description}
	\end{prf}
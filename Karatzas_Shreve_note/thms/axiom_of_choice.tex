	\begin{screen}
		\begin{thm}[選択公理と直積]
			次は同値である.
			\begin{description}
				\item[(イ)] $\forall a\, \exists f\, \left(\, f \fon a \wedge \forall x \in a\, (\, x \neq \emptyset \Longrightarrow f(x) \in x\, )\, \right)$
				\item[(ロ)] $\forall a\, \forall h\, \left[\, h \fon a \wedge \forall x \in a\, (\, h(x) \neq \emptyset\, )
				\Longrightarrow \exists f\, \left(\, f \fon a \wedge \forall x \in a\, (\, f(x) \in h(x)\, )\, \right)\, \right]$
			\end{description}
		\end{thm}
	\end{screen}
	
	\begin{prf}\mbox{}
		\begin{description}
			\item[第一段] $a$を集合,$h$を$a$上の恒等写像とする.このとき
				\begin{align}
					a' &\coloneqq a \backslash \{\emptyset\}, \\
					h' &\coloneqq h|_{a'}
				\end{align}
				とおけば
				\begin{align}
					\exists f\, \left(\, f \fon a' \wedge \forall x \in a'\, (\, f(x) \in h'(x)\, )\, \right)
				\end{align}
				が成立する.
				\begin{align}
					f'' \coloneqq \varepsilon f\, \left(\, f \fon a' \wedge \forall x \in a'\, (\, f(x) \in h'(x)\, )\, \right)
				\end{align}
				とおいて
				\begin{align}
					f' &\coloneqq f'' \cup \{(\emptyset,\emptyset)\}, \\
					f &\coloneqq f'|_a
				\end{align}
				とおけば
				\begin{align}
					\exists f\, \left(\, f \fon a \wedge \forall x \in a\, (\, x \neq \emptyset \Longrightarrow f(x) \in x\, )\, \right)
				\end{align}
				が成立する.
			
			\item[第二段] $a$を集合とし,$h$を
				\begin{align}
					h \fon a \wedge \forall x \in a\, (\, h(x) \neq \emptyset\, )
				\end{align}
				を満たす集合とする.
				\begin{align}
					b \coloneqq h \ast a
				\end{align}
				とおけば
				\begin{align}
					\exists f\, \left(\, f \fon b \wedge \forall x \in b\, (\, x \neq \emptyset \Longrightarrow f(x) \in x\, )\, \right)
				\end{align}
				が成り立つので,
				\begin{align}
					\tilde{f} \coloneqq \varepsilon f\, \left(\, f \fon b \wedge \forall x \in b\, (\, x \neq \emptyset \Longrightarrow f(x) \in x\, )\, \right)
				\end{align}
				とおいて
				\begin{align}
					f \coloneqq \Set{x}{\exists s \in a\, \left(\, x=(s,\tilde{f}(h(s)))\, \right)}
				\end{align}
				とおけば
				\begin{align}
					f \fon a \wedge \forall x \in a\, (\, f(x) \in h(x)\, )
				\end{align}
				が成立する.
		\end{description}
	\end{prf}
	
	\begin{screen}
		\begin{thm}[整列可能定理]
			任意の集合は,或る順序数と全単射で結ばれる:
			\begin{align}
				\forall a\ \exists \alpha \in \ON\ 
				\exists f\ \left( f:\alpha \bij a \right).
			\end{align}
		\end{thm}
	\end{screen}
	
	\begin{prf}
		$S$を類とするとき
		\begin{align}
			\ord{S} \wedge \forall \alpha \in S\, \left(\, F \ast \alpha \neq a\, \right)
			\Longrightarrow F \ast S \subset a \wedge
			F|_S:S \bij F \ast S \wedge \set{S}
		\end{align}
		が成り立つ.
	\end{prf}
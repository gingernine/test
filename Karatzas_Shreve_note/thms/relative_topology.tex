\subsection{相対位相}
	\begin{screen}
		\begin{dfn}[相対位相]
			$(S,\mathscr{O})$を位相空間,$M \subset S$を部分集合,
			$i:M \longrightarrow S$を恒等写像とするとき,
			\begin{align}
				\mathscr{O}_M \coloneqq 
				\Set{i^{-1}(O) = O \cap M}{O \in \mathscr{O}}
			\end{align}
			で定める$\mathscr{O}_M$を$M$の{\bf 相対位相}
			\index{そうたいいそう@相対位相}{\bf (relative topology)}と呼ぶ.
			また相対位相が定まった部分集合をもとの空間に対し{\bf 部分位相空間}
			\index{ぶぶんいそうくうかん@部分位相空間}{\bf (topological subspace)}と呼び,
			紛れが無ければ単に{\bf 部分空間}\index{ぶぶんくうかん@部分空間}とも呼ぶ.
		\end{dfn}
	\end{screen}
	
	\begin{screen}
		\begin{dfn}[$\R$上の位相]
			$\R$上の位相は$\C$上の位相の相対位相として定める:
			\begin{align}
				\mathscr{O}_\R \defeq \Set{O \cap \R}{O \in \mathscr{O}_\C}.
			\end{align}
		\end{dfn}
	\end{screen}
	
	\begin{screen}
		\begin{thm}[$\R$の開集合はボールから成る]
			$O$を$\R$の部分集合とするとき,
			\begin{align}
				O \in \mathscr{O}_\R \Longleftrightarrow
				\forall x \in O\, \exists r \in \R_+\, \left(\, \Set{y \in \R}{|x-y| < r} \subset O\, \right).
			\end{align}
		\end{thm}
	\end{screen}
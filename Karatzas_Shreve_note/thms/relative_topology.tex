\subsection{相対位相}
	$(S,\mathscr{O})$を位相空間とし,$b$を$S$の部分集合とするとき,
	$b$に対して$\mathscr{O}$と整合的な位相構造を導入することが出来る.実際
	\begin{align}
		\mathscr{O}_{b} \defeq \Set{o \cap b}{o \in \mathscr{O}}
	\end{align}
	と定めてみれば,$\mathscr{O}_{b}$は$b$上の位相構造をなしている.確認してみれば,
	\begin{itemize}
		\item $S$と$\emptyset$は$\mathscr{O}$の要素であって
			\begin{align}
				b = S \cap b
			\end{align}
			および
			\begin{align}
				\emptyset = \emptyset \cap b
			\end{align}
			が成り立つので,$b$も$\emptyset$も$\mathscr{O}_{b}$に属する.
			
		\item $p$と$q$を$\mathscr{O}_{b}$の要素とすれば
			\begin{align}
				p = u \cap b
			\end{align}
			を満たす$\mathscr{O}$の要素$u$と
			\begin{align}
				q = v \cap b
			\end{align}
			を満たす$\mathscr{O}$の要素$v$が取れるが,
			\begin{align}
				u \cap v \in \mathscr{O}
			\end{align}
			かつ
			\begin{align}
				p \cap q = (u \cap v) \cap b
			\end{align}
			が成り立つので
			\begin{align}
				p \cap q \in \mathscr{O}_{b}
			\end{align}
			が従う.
			
		\item $\mathscr{P}$を$\mathscr{O}_{b}$の部分集合とするとき,
			\begin{align}
				\mathscr{Q} \defeq \Set{o \in \mathscr{O}}{\exists p \in \mathscr{P}\,
				(\, p = o \cap b\, )}
			\end{align}
			とおけば
			\begin{align}
				\bigcup \mathscr{P} = b \cap \bigcup \mathscr{Q}
			\end{align}
			が成立する.実際,$x$を任意の集合とするとき,
			\begin{align}
				x \in \bigcup \mathscr{P}
			\end{align}
			ならば
			\begin{align}
				x \in p
			\end{align}
			を満たす$\mathscr{P}$の要素$p$が取れるが,このとき
			\begin{align}
				p = o \cap b
			\end{align}
			なる$\mathscr{O}$の要素$o$が取れて,
			\begin{align}
				o \in \mathscr{Q}
			\end{align}
			及び
			\begin{align}
				x \in b \wedge x \in \bigcup \mathscr{Q}
			\end{align}
			が従う.逆に
			\begin{align}
				x \in b \cap \bigcup \mathscr{Q}
			\end{align}
			ならば
			\begin{align}
				x \in o
			\end{align}
			を満たす$\mathscr{Q}$の要素$o$が取れるが,このとき
			\begin{align}
				p = o \cap b
			\end{align}
			なる$\mathscr{P}$の要素$p$が取れて,しかも
			\begin{align}
				x \in o \cap b = p
			\end{align}
			が成り立つので
			\begin{align}
				x \in \bigcup \mathscr{P}
			\end{align}
			が従う.以上より
			\begin{align}
				\bigcup \mathscr{P} \in \mathscr{O} 
			\end{align}
			が成立する.
	\end{itemize}
	
	\begin{screen}
		\begin{dfn}[相対位相]
			$(S,\mathscr{O})$を位相空間,$M \subset S$を部分集合,
			$i:M \longrightarrow S$を恒等写像とするとき,
			\begin{align}
				\mathscr{O}_M \coloneqq 
				\Set{i^{-1}(O) = O \cap M}{O \in \mathscr{O}}
			\end{align}
			で定める$\mathscr{O}_M$を$M$の{\bf 相対位相}
			\index{そうたいいそう@相対位相}{\bf (relative topology)}と呼ぶ.
			また相対位相が定まった部分集合をもとの空間に対し{\bf 部分位相空間}
			\index{ぶぶんいそうくうかん@部分位相空間}{\bf (topological subspace)}と呼び,
			紛れが無ければ単に{\bf 部分空間}\index{ぶぶんくうかん@部分空間}とも呼ぶ.
		\end{dfn}
	\end{screen}
	
	\begin{screen}
		\begin{dfn}[$\R$上の位相]
			$\R$上の位相は$\C$上の位相の相対位相として定める:
			\begin{align}
				\mathscr{O}_\R \defeq \Set{O \cap \R}{O \in \mathscr{O}_\C}.
			\end{align}
		\end{dfn}
	\end{screen}
	
	\begin{screen}
		\begin{thm}[$\R$の開集合はボールから成る]
			$O$を$\R$の部分集合とするとき,
			\begin{align}
				O \in \mathscr{O}_\R \Longleftrightarrow
				\forall x \in O\, \exists r \in \R_+\, \left(\, \Set{y \in \R}{|x-y| < r} \subset O\, \right).
			\end{align}
		\end{thm}
	\end{screen}
\section{Gauss過程}
	$(\Omega,\mathscr{F},P)$を確率空間とし,$X$をこの上の実数値確率過程とする.
	$X$が連続な確率過程であって,かつ
	\begin{itemize}
		\item $s,t$を$s < t$なる$\mathbf{T}$の要素とするとき
			\begin{align}
				A \in \borel{\R} \Longrightarrow
				P\left((X_t-X_s)^{-1}(A)\right)
				= \int_A \frac{1}{\sqrt{2\pi(t-s)}} \exp\left(-\frac{x^2}{2(t-s)}\right)\ dx
			\end{align}
			が成り立つ.
		
		\item $s,t$を$s < t$なる$\mathbf{T}$の要素とするとき$X_t - X_s$が$\mathscr{F}^X_s$と独立.
			
		\item $P$-a.sに$X_0 = 0$.
	\end{itemize}
	が満たされるとき,$X$を$(\Omega,\mathscr{F},P)$上のGauss過程と呼ぶ.
	$X$が$(\Omega,\mathscr{F},P)$上のGauss過程であって,さらに
	\begin{align}
		s,t \in \mathbf{T} \Longrightarrow \int_\Omega X_s \cdot X_t\ dP = \min{\{s,t\}}
	\end{align}
	を満たすとき,$X$を$(\Omega,\mathscr{F},P)$上のWiener過程と呼ぶ.
	
	{\bf 以降は所与の確率空間$(\Omega,\mathscr{F},P)$上にWiener過程が存在するとして考察を進め,
	それを基にしてWiener空間の構成法を述べたのち,然るべき確率空間$(\Omega,\mathscr{F},P)$及びその上のWiener過程の構成法を述べる予定.}
	
	\begin{screen}
		\begin{dfn}[座標過程]
			$F$を$\mathbf{T}$上の$\R$値写像の集合とする.このとき
			\begin{align}
				X \defeq \Set{((t,w),w(t))}{t \in \mathbf{T} \wedge w \in F}
			\end{align}
			で定める$X$を$(\mathbf{T},F)$-座標過程と呼ぶ.つまり,$X$の$w$に対する標本路は$w$そのものである.
		\end{dfn}
	\end{screen}
	
	\begin{screen}
		\begin{dfn}[筒集合]
			$F$を$\mathbf{T}$上の$\R$値写像の集合とするとき,
			\begin{align}
				\cyl{F} \defeq 
				\Set{x}{\exists t \in \mathbf{T}\, \exists A \in \borel{\R}\, 
				\left(\, x = \Set{w \in F}{w(t) \in A}\, \right)}
			\end{align}
			で定める集合を$F$の筒集合族と呼ぶ.
		\end{dfn}
	\end{screen}
	
	\begin{screen}
		\begin{thm}[座標過程は確率過程]\label{thm:coordinate_process_is_stochastic}
			$F$を$\mathbf{T}$上の$\R$値写像の集合とし,$X$を$(\mathbf{T},F)$-座標過程とする.このとき
			各$t \in \mathbf{T}$で$X_t$は$\sigma(\cyl{F})/\borel{\R}$-可測である.
		\end{thm}
	\end{screen}
	
	\begin{sketch}
		$t$を$\mathbf{T}$の要素とし,$A$を$\borel{\R}$の要素とするとき,
		\begin{align}
			\Set{w \in F}{X_t(w) \in A}
			= \Set{w \in F}{w(t) \in A}
			\in \cyl{F}
		\end{align}
		が成り立つ.
		\QED
	\end{sketch}
	
	$(\Omega,\mathscr{F},P)$を確率空間とし,$X$をこの上の確率過程とするとき,
	$\Omega$の要素$\omega$に対してその標本路
	\begin{align}
		\mathbf{T} \ni t \longmapsto X_t(\omega)
	\end{align}
	を対応させる写像,つまり
	\begin{align}
		\Omega \ni \omega \longmapsto \Set{(t,X_t(\omega))}{t \in \mathbf{T}}
	\end{align}
	なる写像を$X_\bullet$と書く.
	
	\begin{screen}
		\begin{thm}[$X_\bullet$は可測]
			$F$を$\mathbf{T}$上の$\R$値写像の集合とし,$(\Omega,\mathscr{F},P)$を確率空間とし,$X$をこの上の確率過程とする.このとき
			\begin{align}
				\forall \omega \in \Omega\, \left(\, X_\bullet(\omega) \in F\, \right)
			\end{align}
			ならば$X_\bullet$は$\mathscr{F}/\sigma(\cyl{F})$-可測である.
		\end{thm}
	\end{screen}
	
	\begin{sketch}
		$E$を$\cyl{F}$の要素として
		\begin{align}
			X_\bullet^{-1}(E) \in \mathscr{F}
		\end{align}
		が成り立つことを示す.$A$は筒集合であるから,
		\begin{align}
			E = \Set{w \in F}{w(t) \in A}
		\end{align}
		を満たす$\mathbf{T}$の要素$t$および$\borel{\R}$の要素$A$が取れる.
		このとき
		\begin{align}
			X_\bullet^{-1}(E) = \Set{\omega \in \Omega}{X_\bullet(\omega) \in E}
			= \Set{\omega \in \Omega}{X_t(\omega) \in A}
		\end{align}
		が成り立ち,$X$は確率過程であるので右辺は$\mathscr{F}$に属する.いま
		\begin{align}
			\cyl{F} \subset \Set{E \in \sigma(\cyl{F})}{X_\bullet^{-1}(E) \in \mathscr{F}}
		\end{align}
		が満たされることが分かったので$X_\bullet$は$\mathscr{F}/\sigma(\cyl{F})$-可測である.
		\QED
	\end{sketch}
	
	\begin{screen}
		\begin{thm}[連続写像の全体の筒集合族はBorel集合族を生成する]
			$C$を$\mathbf{T}$上の実連続写像の全体とするとき
			\begin{align}
				\sigma(\cyl{C}) = \borel{C}.
			\end{align}
		\end{thm}
	\end{screen}
	
	\begin{sketch}
	\end{sketch}
	
	\begin{screen}
		\begin{thm}\label{thm:Wiener_process_on_continuous_functions}
			$(\Omega,\mathscr{F},P)$を確率空間とし,$X$をこの上のGauss過程とし,
			$C$を$\mathbf{T}$上の実連続写像の全体とし,
			$B$を$(\mathbf{T},C)$-座標過程とする.また
			\begin{align}
				\mu_X \defeq P X_\bullet^{-1}
			\end{align}
			と定める.このとき$B$は$(C,\borel{C},\mu_X)$上のGauss過程である.
			さらに$X$が$(\Omega,\mathscr{F},P)$上のWiener過程ならば$B$も$(C,\borel{C},\mu_X)$上のWiener過程である.
		\end{thm}
	\end{screen}
	
	\begin{sketch}\mbox{}
		\begin{description}
			\item[第一段]
				定理\ref{thm:coordinate_process_is_stochastic}より$B$は連続な確率過程である.
				また
				\begin{align}
					\Set{w \in C}{B_0(w)=0} = \Set{w \in C}{w(0) = 0}
				\end{align}
				なので
				\begin{align}
					X_\bullet^{-1}\left(\Set{w \in C}{B_0(w)=0}\right)
					= \Set{\omega \in \Omega}{X_0(\omega) = 0}
				\end{align}
				が成り立つ.ゆえに
				\begin{align}
					\mu_X\left(\Set{w \in C}{B_0(w) \neq 0}\right) = 0
				\end{align}
				が成り立つ.
				
			\item[第二段]
				$s,t$を$s < t$なる$\mathbf{T}$の要素とするとき,
				\begin{align}
					(B_t - B_s) \circ X_\bullet = X_t - X_s
				\end{align}
				が成り立つので,
				\begin{align}
					\mu_X (B_t - B_s)^{-1}
					= P \left((B_t - B_s) \circ X_\bullet\right)^{-1}
					= P (X_t - X_s)^{-1}
				\end{align}
				が成り立つ.すなわち
				\begin{align}
					A \in \borel{\R} \Longrightarrow
					\mu_X\left((B_t-B_s)^{-1}(A)\right)
					= \int_A \frac{1}{\sqrt{2\pi(t-s)}} \exp\left(-\frac{x^2}{2(t-s)}\right)\ dx
				\end{align}
				も満たされる.
			
			\item[第三段]
				$B_t - B_s$と$\mathscr{F}_s^B$が独立であることを示す.$t \in \mathbf{T}$なる$r$と$\borel{\R}$の要素$E$に対して
				\begin{align}
					X_\bullet^{-1}(B_t^{-1}(E)) = X_t^{-1}(E)
				\end{align}
				が成り立つので
				\begin{align}
					\Set{x}{\exists r \in [0,s]\, \exists E \in \borel{\R}\, \left(\, x=B_r^{-1}(E)\, \right)}
					\subset \Set{A \in \mathscr{F}_s^B}{X_\bullet^{-1}(A) \in \mathscr{F}^X_s}
				\end{align}
				が成り立つ.左辺は$\mathscr{F}_s^B$を生成するので
				\begin{align}
					A \in \mathscr{F}_s^B \Longrightarrow X_\bullet^{-1}(A) \in \mathscr{F}^X_s
				\end{align}
				が成り立つ.$X_t-X_s$と$\mathscr{F}^X_s$の独立性を使えば,
				\begin{align}
					U \in \borel{\R} \wedge A \in \mathscr{F}_s^B \Longrightarrow
					\mu_X\left((B_t-B_s)^{-1}(U) \cap A\right)
					&= P X_\bullet^{-1}\left((B_t-B_s)^{-1}(U) \cap A\right) \\
					&= P \left( X_\bullet^{-1}\left((B_t-B_s)^{-1}(U)\right) \cap X_\bullet^{-1}(A)\right) \\
					&= P \left( (X_t-X_s)^{-1}(U) \cap X_\bullet^{-1}(A)\right) \\
					&= P \left( (X_t-X_s)^{-1}(U) \right) P X_\bullet^{-1}(A) \\
					&= \mu_X\left((B_t-B_s)^{-1}(U) \right) \mu_X(A)
				\end{align}
				が成立し,$B_t-B_s$と$\mathscr{F}^B_s$との独立性が従う.
				
			\item[第四段]
				$f:\R^2 \ni (x,y) \longmapsto x \cdot y$なる$f$に対して
				\begin{align}
					\int_C f \circ (B_t,B_s)\ d\mu_X = 
					\int_{\R^2} f\ d\mu_X(B_t,B_s)^{-1}
				\end{align}
				が成り立つ.ただし$\mu_X(B_t,B_s)^{-1}$の意味は
				\begin{align}
					A \in \borel{\R^2} \Longrightarrow
					\mu_X(B_t,B_s)^{-1}(A) = \mu_X\left(\Set{w \in C}{(B_t(w),B_s(w)) \in A} \right)
				\end{align}
				である.他方で
				\begin{align}
					\mu_X(B_t,B_s)^{-1} = P (X_t,X_s)^{-1}
				\end{align}
				が成り立つので
				\begin{align}
					\int_C f \circ (B_t,B_s)\ d\mu_X = 
					\int_{\R^2} f\ dP(X_t,X_s)^{-1}
					= \int_\Omega f \circ (X_t,X_s)\ dP
				\end{align}
				が従う.ゆえに,$W$がWiener過程であるときは$B$もまたWiener過程である.
				\QED
		\end{description}
	\end{sketch}
	
	\begin{screen}
		\begin{dfn}[座標過程をGauss過程とする確率測度の一意性]\label{thm:uniqueness_of_gauss_measure}
			$C$を$\mathbf{T}$上の実連続写像の全体とし,$B$を$(\mathbf{T},C)$-座標過程とし,
			$\mu_1,\mu_2$を$\borel{C}$上の確率測度とする.このとき$B$が
			$(C,\borel{C},\mu_1)$上のGauss過程であり,かつ
			$(C,\borel{C},\mu_2)$上のGauss過程でもあるならば,
			$\mu_1 = \mu_2$.
		\end{dfn}
	\end{screen}
	
	\begin{prf}\mbox{}
		\begin{description}
			\item[第一段]
				$\mu_1$と$\mu_2$が$\cyl{C}$上で一致することを示す.$E$を$\cyl{C}$の要素とすれば,
				\begin{align}
					E = \Set{w \in C}{w(t) \in A}
				\end{align}
				なる$\mathbf{T}$の要素$t$と$\borel{\R}$の要素$A$が取れる.すなわち
				\begin{align}
					E = B_t^{-1}(A)
				\end{align}
				が成り立つ.いま,仮定より
				\begin{align}
					\mu_1(B_t - B_0 \in A) = \int_A \frac{1}{\sqrt{2\pi t}} \exp\left(-\frac{x^2}{2t}\right)\ dx
					= \mu_2(B_t - B_0 \in A)
				\end{align}
				が満たされているが,同時に
				\begin{align}
					&\mu_1(N_1) = 0 \wedge \forall w \in C \backslash N_1\, \left(\, B_0(w) = 0\, \right) \\
					&\wedge \\
					&\mu_2(N_2) = 0 \wedge \forall w \in C \backslash N_2\, \left(\, B_0(w) = 0\, \right)
				\end{align}
				なる$N_1,N_2$が取れるので,
				\begin{align}
					\mu_1(B_t \in A) &= \mu_1\left(\{B_t \in A\} \cap (C \backslash N_1)\right) \\
					&= \mu_1\left(\{B_t - B_0 \in A\} \cap (C \backslash N_1)\right) \\
					&= \mu_1(B_t - B_0 \in A) \\
					&= \mu_2(B_t - B_0 \in A) \\
					&= \mu_2\left(\{B_t - B_0 \in A\} \cap (C \backslash N_2)\right) \\
					&= \mu_2\left(\{B_t \in A\} \cap (C \backslash N_2)\right) \\
					&= \mu_2(B_t \in A)
				\end{align}
				が成立する.ゆえに$\mu_1$と$\mu_2$は$\cyl{C}$上で一致する
				
			\item[第二段]
				いま$n$を$2$以上の自然数とし,$t_1,t_2,\cdots,t_n$を
				\begin{align}
					t_1 < t_2 < \cdots < t_n
				\end{align}
				なる$\mathbf{T}$の要素とする.このとき
				\begin{align}
					E \in \borel{\R^n} \Longrightarrow
					&\mu_1\left(\left\{\left(B_{t_1},B_{t_2}-B_{t_1},\cdots,B_{t_n}-B_{t_{n-1}}\right) \in E\right\}\right) \\
					&= \mu_2\left(\left\{\left(B_{t_1},B_{t_2}-B_{t_1},\cdots,B_{t_n}-B_{t_{n-1}}\right) \in E\right\}\right)
				\end{align}
				が成り立つことを示す.ところで
				\begin{align}
					E_1 \times E_2 \times \cdots \times E_n \quad (E_1, E_2,\cdots ,E_n \in \borel{\R})
				\end{align}
				の形の集合の全体は$\borel{\R^n}$を生成する乗法族をなすから,
				これに対して
				\begin{align}
					&\mu_1\left(\left\{\left(B_{t_1},B_{t_2}-B_{t_1},\cdots,B_{t_n}-B_{t_{n-1}}\right) \in E_1 \times E_2 \times \cdots \times E_n\right\}\right) \\
					&= \mu_2\left(\left\{\left(B_{t_1},B_{t_2}-B_{t_1},\cdots,B_{t_n}-B_{t_{n-1}}\right) \in E_1 \times E_2 \times \cdots \times E_n\right\}\right)
				\end{align}
				が成り立つことを示せば良い.実際,仮定より
				\begin{align}
					\mu_1\left(B_{t_i}-B_{t_{i-1}} \in E_i\right)
					&= \int_{E_i} \frac{1}{\sqrt{2\pi(t_i-t_{i-1})}} \exp\left(-\frac{x^2}{2(t_i-t_{i-1})}\right)\ dx \\
					&= \mu_2(B_{t_i}-B_{t_{i-1}} \in E_i)
				\end{align}
				が成り立ち,かつ$B_{t_1},B_{t_2}-B_{t_1},\cdots,B_{t_n} - B_{t_{n-1}}$は
				$\mu_1$に関しても$\mu_2$に関しても独立なので,
				\begin{align}
					&\mu_1\left(\left\{\left(B_{t_1},B_{t_2}-B_{t_1},\cdots,B_{t_n}-B_{t_{n-1}}\right) \in E_1 \times E_2 \times \cdots \times E_n\right\}\right) \\
					&= \mu_1\left(B_{t_1} \in E_1\right) \cdot \mu_1\left(B_{t_2}-B_{t_{1}} \in E_2\right) 
					\cdot \cdots \cdot \mu_1\left(B_{t_n}-B_{t_{n-1}} \in E_n\right) \\
					&= \mu_2\left(B_{t_1} \in E_1\right) \cdot \mu_2\left(B_{t_2}-B_{t_{1}} \in E_2\right) 
					\cdot \cdots \cdot \mu_2\left(B_{t_n}-B_{t_{n-1}} \in E_n\right) \\
					&= \mu_2\left(\left\{\left(B_{t_1},B_{t_2}-B_{t_1},\cdots,B_{t_n}-B_{t_{n-1}}\right) \in E_1 \times E_2 \times \cdots \times E_n\right\}\right)
				\end{align}
				が得られる.
				
			\item[第三段]
				$\borel{C} = \sigma(\cyl{C})$であるから,
				\begin{align}
					\mathscr{A} \defeq \Set{x}{\exists n \in \Natural\, \exists A\, 
					\left(\, A:n \longrightarrow \cyl{C} \wedge x = \bigcap_{i \in n}A(i)\, \right)}
				\end{align}
				により$\cyl{C}$の有限交叉の全体を定めれば
				\begin{align}
					\borel{C} = \sigma(\mathscr{A})
				\end{align}
				が成立する.$\mathscr{A}$は乗法族であるので,
				\begin{align}
					\mathscr{A} \subset \Set{E \in \borel{C}}{\mu_1(E) = \mu_2(E)}
				\end{align}
				が成り立つことを示せばDynkin族定理より$\mu_1 = \mu_2$が従う.いま$E$を$\mathscr{A}$の要素とする.
				このとき$E$は空か$\cyl{C}$の要素か,或いは
				\begin{align}
					t_1 < t_2 < \cdots < t_n
				\end{align}
				なる$\mathbf{T}$の要素$t_1,\cdots,t_n$と$\borel{\R}$の要素$A_1,\cdots,A_n$により
				\begin{align}
					E = B_{t_1}^{-1}(A_1) \cap B_{t_2}^{-1}(A_2) \cap \cdots \cap B_{t_n}^{-1}(A_n)
				\end{align}
				なる形で表される.最後の場合,$E$は
				\begin{align}
					E = \Set{w \in C}{\left(B_{t_1}(w),B_{t_2}(w),\cdots,B_{t_n}(w)\right) \in A_1 \times A_2 \times \cdots \times A_n}
				\end{align}
				と書き直される.ここで$\varphi$を
				\begin{align}
					\varphi:\R^n \ni (x_1,x_2,\cdots,x_n) \longmapsto (x_1,x_2-x_1,\cdots,x_n - x_{n-1})
				\end{align}
				なる写像とすれば,$\varphi$は$\R^n$から$\R^n$への同相写像であるから
				\begin{align}
					\varphi \ast \left(A_1 \times A_2 \times \cdots \times A_n\right) \in \borel{\R^n}
				\end{align}
				が満たされる.そして第二段の結果より
				\begin{align}
					\mu_1(E) &= \mu_1\left(\left\{\left(B_{t_1},B_{t_2}-B_{t_1},\cdots,B_{t_n}-B_{t_{n-1}}\right)
					\in \varphi \ast \left(A_1 \times A_2 \times \cdots \times A_n\right)\right\}\right) \\
					&= \mu_2\left(\left\{\left(B_{t_1},B_{t_2}-B_{t_1},\cdots,B_{t_n}-B_{t_{n-1}}\right)
					\in \varphi \ast \left(A_1 \times A_2 \times \cdots \times A_n\right)\right\}\right) \\
					&= \mu_2(E)
				\end{align}
				が成立する.
				\QED
		\end{description}
	\end{prf}
	
	\begin{screen}
		\begin{dfn}[Wiener測度]
			$C$を$\mathbf{T}$上の実連続写像の全体とし,$B$を$(\mathbf{T},C)$-座標過程とし,$\mu_w$を$\borel{C}$上の確率測度とする.
			$\mu_w$が$B$を$(C,\borel{C},\mu_w)$上のWiener過程たらしめるとき,
			$\mu_w$を$\borel{C}$上の{\bf Wiener測度}\index{Wienerそくど@Wiener測度}{\bf (Wiener measure)}と呼び,
			$(C,\borel{C},\mu_w)$を{\bf Wiener空間}\index{Wienerくうかん@Wiener空間}{\bf (Wiener space)}と呼ぶ.
		\end{dfn}
	\end{screen}
	
	Wiener測度の存在は定理\ref{thm:Wiener_process_on_continuous_functions}より,
	Wiener測度の一意性は定理\ref{thm:uniqueness_of_gauss_measure}より従う.
	次に然るべき確率空間$(\Omega,\mathscr{F},P)$及びその上のWiener過程の構成法を述べる.
\section{Wiener過程}
\label{sec:Wiener_process}
	$\mathbf{T}$を$[0,\infty[$か$[0,T]$とする.
	$\mathbf{T}$の各要素$t$に対し,$(X_t,\mathscr{B}_t)$は全て$(\R,\borel{\R})$であるとする.
	$\mathbf{T}$の部分集合$\Lambda$に対して
	\begin{align}
		\R^\Lambda \defeq \Set{x}{x:\Lambda \longrightarrow \R}
	\end{align}
	とおく.
	\begin{align}
		\bigotimes_{t \in \Lambda} \mathscr{B}_t
	\end{align}
	は
	\begin{align}
		\bigcup_{t \in \Lambda} \Set{\Set{x}{x:\Lambda \longrightarrow \R \wedge x(t) \in A}}{A \in \borel{\R}}
	\end{align}
	が生成する$\sigma$-加法族である.特に
	\begin{align}
		t_1 < t_2 < \cdots < t_n
	\end{align}
	なる$\mathbf{T}$の要素$t_1,t_2,\cdots,t_n$によって
	\begin{align}
		\Lambda = \{t_1,t_2,\cdots,t_n\}
	\end{align}
	が成り立っているときは
	\begin{align}
		\bigotimes_{t \in \Lambda} \mathscr{B}_t
		= \Set{\Set{x}{x:\Lambda \longrightarrow \R \wedge (x(t_1),x(t_2),\cdots,x(t_n)) \in A}}{A \in \borel{\R^n}}
	\end{align}
	が成立する.このとき,$(\R^\Lambda,\mathscr{B}_\Lambda)$上の確率測度$P_\Lambda$を
	\begin{align}
		&P_\Lambda(\Set{x}{x:\Lambda \longrightarrow \R \wedge (x(t_1),x(t_2),\cdots,x(t_n)) \in A}) \\
		&= \int_A p(t_1,0,y_1)p(t_2-t_1,y_1,y_2) \cdots p(t_n-t_{n-1},y_{n-1},y_n)\ dy_1 dy_2\cdots dy_n
	\end{align}
	で定める.
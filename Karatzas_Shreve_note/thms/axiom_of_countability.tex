\section{可算公理}
	\begin{screen}
		\begin{thm}[可算コンパクト性の同値条件]
		\end{thm}
	\end{screen}
	
	\begin{screen}
		\begin{dfn}[開基]
			位相空間$(S,\mathscr{O})$において,
			$\mathscr{O}$の部分集合$\mathscr{B}$で
			\begin{align}
				\mathscr{O}
				= \Set{\bigcup \mathscr{U}}{\mathscr{U} \subset \mathscr{B}}
			\end{align}
			を満たすもの,ただし$\bigcup \emptyset = \emptyset$,
			を$\mathscr{O}$の{\bf 開基}\index{かいき@開基}や
			{\bf 基底}\index{きてい@基底},{\bf 基}\index{き@基}{\bf (base)}と呼ぶ.
			基底は一意に定まるものではない.
		\end{dfn}
	\end{screen}
	
	\begin{screen}
		\begin{dfn}[可算公理]
			位相空間$S$において,任意の点が高々可算な基本近傍系を持つとき
			$S$は{\bf 第一可算公理}
			\index{だいいちかさんこうり@第一可算公理}
			{\bf (the first axiom of countability)}を満たす,或は
			$S$は第一可算であるといい,
			$S$が高々可算な基底を持つとき
			$S$は{\bf 第二可算公理}
			\index{だいにかさんこうり@第二可算公理}
			{\bf (the second axiom of countability)}を満たす,或は
			$S$は第二可算であるという.
		\end{dfn}
	\end{screen}
	空集合(要素数0)を含む任意の有限位相空間は,その冪集合が有限集合であるから
	第二可算公理を満たす.
	
	\begin{screen}
		\begin{thm}[第二可算なら第一可算]
			空でない第二可算空間は第一可算である.
		\end{thm}
	\end{screen}
	
	\begin{prf}
		$\mathscr{B}$を空でない第二可算空間$S$の可算基とするとき,任意の$x \in S$に対して
		\begin{align}
			\mathscr{U}(x) \coloneqq
			\Set{B \in \mathscr{B}}{x \in B}
		\end{align}
		で可算な基本近傍系が定まる.実際
		$x$の任意の近傍$U$に対し或る$B \in \mathscr{B}$で
		\begin{align}
			x \in B \subset U^{\mathrm{o}}
		\end{align}
		が成立し,定義より$B \in \mathscr{U}(x)$が満たされる.
		\QED
	\end{prf}
	
	\begin{screen}
		\begin{dfn}[稠密・可分]
			位相空間$S$において,$\overline{M} = S$を満たすような部分集合$M$を
			$S$で{\bf 稠密}\index{ちゅうみつ@稠密}な{\bf (dense)}部分集合と呼ぶ.
			また高々可算かつ稠密な部分集合$M$が存在するとき$S$は{\bf 可分}
			\index{かぶん@可分}である{\bf (separable)}という.
		\end{dfn}
	\end{screen}
	
	\begin{screen}
		\begin{thm}[第二可算なら可分]\label{thm:second_countable_then_separable}
			第二可算位相空間は可分である.
		\end{thm}
	\end{screen}
	
	\begin{prf}
		$\mathscr{B}$を第二可算空間$S$の可算基とするとき,
		$S = \emptyset$なら$\emptyset$は$S$の唯一の部分集合であり,
		要素数$0$かつ$\overline{\emptyset} = \emptyset = S$を満たすから
		$S$は可分である.$S \neq \emptyset$のとき,
		選択関数$\Phi \in \prod \mathscr{B} = \prod_{B \in \mathscr{B}} B$を取り
		\begin{align}
			M \coloneqq \Set{\Phi(B)}{B \in \mathscr{B}}
		\end{align}
		で可算集合を定めれば,任意の$x \in S$及び$x$の任意の近傍$U$に対し
		$x \in B \subset U^{\mathrm{o}}$を満たす
		$B \in \mathscr{B}$が存在して
		\begin{align}
			\Phi(B) \in B \cap M \subset U \cap M
		\end{align}
		となるから,定理\ref{thm:belongs_to_closure_iff_clusters}より
		$S = \overline{M}$が成立する.
		\QED
	\end{prf}
	
	\begin{screen}
		\begin{dfn}[局所有限]
			$\mathscr{F}$を位相空間$S$の部分集合族とする.
			任意の$x \in S$が$\mathscr{F}$の高々有限個の元としか交叉しない近傍を持つとき,
			$\mathscr{F}$は{\bf 局所有限}\index{きょくしょゆうげん@局所有限}
			{\bf (locally finite)}であるという.
			つまり,$\mathscr{F}$が局所有限であることの論理式で表現すると
			\begin{align}
				\forall x \in S \exists V \in \mathscr{V}(x) \exists \mathscr{G} \in
				\mathcal{P}(\mathscr{F})
				\left(\exists i \in \omega(\mathscr{G} \simeq i) \wedge
				\forall G \in \mathscr{G}(V \cap G \neq \emptyset) \wedge
				\forall F \in \mathscr{F} \backslash \mathscr{G}(V \cap F = \emptyset)\right).
			\end{align}
			また
			$\mathscr{F}$が局所有限な部分集合族の高々可算個の合併で表されるとき,
			$\mathscr{F}$は{\bf $\sigma$-局所有限}
			\index{しぐまきょくしょゆうげん@$\sigma$-局所有限}であるという.
		\end{dfn}
	\end{screen}
	
	後述の一様位相空間(距離空間や位相線型空間に共通する構造が定義された空間)の或るクラスは
	$\sigma$-局所有限な基底を持つ(定理\ref{thm:if_uniformity_has_countable_base_then_has_topology_has_sigma_locally_finite_base}).従って以下のいくつかの定理はそのまま
	距離空間や第一可算位相線型空間に適用される.
	
	\begin{screen}
		\begin{thm}[$\sigma$-局所有限な基底が存在すれば第一可算]
			$\sigma$-局所有限な基底が存在する空でない位相空間は第一可算である.
		\end{thm}
	\end{screen}
	
	\begin{prf}
		$S$を空でない位相空間,$\mathscr{B} = \bigcup_{n=1}^\infty \mathscr{B}_n$を
		$\sigma$-局所有限な基底とする(各$\mathscr{B}_n$は局所有限).
		任意の$x \in S$で
		\begin{align}
			\mathscr{U}_n(x) \coloneqq \Set{B \in \mathscr{B}_n}{x \in B},
			\quad \mathscr{U}(x) \coloneqq \bigcup_{n=1}^\infty \mathscr{U}_n(x)
		\end{align}
		と定めれば,局所有限性より$\mathscr{U}_n(x)$は有限であるから
		$\mathscr{U}(x)$は高々可算である.また$x$の任意の近傍$U$に対し
		\begin{align}
			x \in B \subset U^{\mathrm{o}}
		\end{align}
		を満たす$B \in \mathscr{B}$が存在し,定義より$B \in \mathscr{U}(x)$
		が成り立つから$\mathscr{U}(x)$は$x$の高々可算な基本近傍系をなす.
		\QED
	\end{prf}
	
	\begin{screen}
		\begin{thm}[可分空間の局所有限な開集合族は高々可算集合]
		\label{thm:locally_finite_family_of_open_sets_is_countable_in_separable_space}
			$S$を空でない可分位相空間,
			$M$を$S$で稠密な高々可算集合,$\mathscr{B}$を
			$S$の空でない開集合から成る族とするとき,
			\begin{align}
				\mathscr{B} = \bigcup_{m \in M} \Set{B \in \mathscr{B}}{m \in B}
				\label{eq:thm_locally_finite_family_of_open_sets_is_countable_in_separable_space}
			\end{align}
			が成立する.特に$\mathscr{B}$が局所有限なら$\mathscr{B}$は高々可算集合である.
		\end{thm}
	\end{screen}
	
	\begin{prf}
		稠密性より任意の$E \in \mathscr{B}$は
		$E \cap M \neq \emptyset$を満たすから,$m \in E \cap M$で$
		E \in \Set{B \in \mathscr{B}}{m \in B}$となり
		(\refeq{eq:thm_locally_finite_family_of_open_sets_is_countable_in_separable_space})が出る.
		$\mathscr{B}$が局所有限なら$\Set{B \in \mathscr{B}}{m \in B}$は全て有限集合となり
		$\mathscr{B}$は高々可算集合となる.
		\QED	
	\end{prf}
	
	\begin{screen}
		\begin{thm}[$\sigma$-局所有限な基底が存在すれば,可分$\Longleftrightarrow$第二可算]
			$\sigma$-局所有限な基底が存在する空でない位相空間において,
			可分であることと第二可算であることは同値になる.
		\end{thm}
	\end{screen}
	
	\begin{prf}
		空でない可分位相空間において$\sigma$-局所有限な基底が存在するとき,
		定理\ref{thm:locally_finite_family_of_open_sets_is_countable_in_separable_space}
		よりその基底は高々可算集合であるから第二可算性が満たされる.
		逆に第二可算なら可分であるから定理の主張を得る.
		\QED
	\end{prf}
	
	\begin{screen}
		\begin{thm}[正則かつ$\sigma$-局所有限な基底を持つ$\Longrightarrow$完全正規]
		\end{thm}
	\end{screen}
	
	\begin{screen}
		\begin{dfn}[細分・パラコンパクト]\mbox{}
			\begin{itemize}
				\item $\mathscr{A}$と$\mathscr{B}$を或る集合の被覆とする.
					任意の$B \in \mathscr{B}$に対し$B \subset A$を満たす
					$A \in \mathscr{A}$が存在するとき,
					$\mathscr{B}$を$\mathscr{A}$の{\bf 細分}
					\index{さいぶん@細分}{\bf (refinement)}と呼ぶ.
					位相空間において,被覆の細分で元が全て開(閉)集合であるものを
					{\bf 開(閉)細分}\index{かいさいぶん@開細分}
					{\bf (open(closed) refinement)}と呼ぶ.
					
				\item 任意の開被覆が局所有限な開細分を持つ
					位相空間は{\bf パラコンパクト}\index{ぱらこんぱくと@パラコンパクト}
					{\bf (paracompact)}であるという.
			\end{itemize}
		\end{dfn}
	\end{screen}
	
	\begin{screen}
		\begin{thm}[正則空間の開被覆に対し,$\sigma$-局所有限な開細分が存在する
		$\Longleftrightarrow$局所有限な開細分が存在する]
			$S$を正則空間,$\mathscr{S}$を$S$の開被覆とするとき,以下は全て同値になる:
			\begin{description}
				\item[(a)] $\mathscr{S}$が$\sigma$-局所有限な開細分を持つ.
				\item[(b)] $\mathscr{S}$が局所有限な細分を持つ.
				\item[(c)] $\mathscr{S}$が局所有限な閉細分を持つ.
				\item[(d)] $\mathscr{S}$が局所有限な開細分を持つ.
			\end{description}
		\end{thm}	
	\end{screen}
	
	\begin{screen}
		\begin{thm}[第二可算空間の任意の基底は可算基を内包する]\label{thm:countable_base_of_second_countable_space}
			$\mathscr{B}$を第二可算空間$S$の任意の基底とするとき,或る可算部分集合
			$\mathscr{B}_0 \subset \mathscr{B}$もまた$S$の基底となる.
		\end{thm}
	\end{screen}
	
	\begin{prf}
		$\mathscr{D}$を$S$の可算基とする.
		任意の開集合$U$に対し或る$\mathscr{B}_U \subset \mathscr{B}$が存在して
		$U = \bigcup_{V \in \mathscr{B}_U}V$を満たすから,
		\begin{align}
			\mathscr{D}_U \coloneqq
			\Set{W \in \mathscr{D}}{W \subset V,\ V \in \mathscr{B}_U}
			\label{eq:thm_countable_base_of_second_countable_space_1}
		\end{align}
		とおけば$U = \bigcup_{V \in \mathscr{B}_U} V
			= \bigcup_{V \in \mathscr{B}_U} \bigcup_{\substack{W \in \mathscr{D}_U \\ W \subset V}} W
			\subset \bigcup_{W \in \mathscr{D}_U} W
			\subset U$より
		\begin{align}
			U = \bigcup_{W \in \mathscr{D}_U} W
			\label{eq:thm_countable_base_of_second_countable_space_2}
		\end{align}
		が成り立つ.ここで(\refeq{eq:thm_countable_base_of_second_countable_space_1})より
		任意の$W \in \mathscr{D}_U$に対して
		$\Set{V \in \mathscr{B}}{W \subset V} \neq \emptyset$であるから
		\begin{align}
			\Phi_U \in \prod_{W \in \mathscr{D}_U} \Set{V \in \mathscr{B}}{W \subset V}
		\end{align}
		が取れる.$\mathscr{B}_U' \coloneqq \Set{\Phi_U(W)}{W \in \mathscr{D}_U}$とすれば
		$U = \bigcup_{W \in \mathscr{D}_U} W \subset \bigcup_{W \in \mathscr{D}_U} \Phi(W)
		\subset \bigcup_{V \in \mathscr{B}_U'} V \subset U$より
		\begin{align}
			U = \bigcup_{V \in \mathscr{B}_U'} V
			\label{eq:thm_countable_base_of_second_countable_space_3}
		\end{align}
		が満たされ,
		\begin{align}
			\mathscr{B}_0 \coloneqq \bigcup_{W \in \mathscr{D}} \mathscr{B}_W'
		\end{align}
		と定めれば$\mathscr{B}_0$は求める$S$の可算基となる.実際,任意の開集合$U$に対し
		(\refeq{eq:thm_countable_base_of_second_countable_space_2})と
		(\refeq{eq:thm_countable_base_of_second_countable_space_3})より
		\begin{align}
			U = \bigcup_{W \in \mathscr{D}_U} W
			= \bigcup_{W \in \mathscr{D}_U} \bigcup_{V \in \mathscr{B}_W'} V
		\end{align}
		となる.
		\QED
	\end{prf}
	
	\begin{screen}
		\begin{thm}[局所コンパクトHausdorff空間が第二可算なら$\sigma$-コンパクト]\label{thm:second_countable_Hausdorff_sigma_compact}
			$S$が第二可算性をもつ局所コンパクトHausdorff空間なら,
			次を満たすコンパクト部分集合の列$(K_n)_{n=1}^\infty$が存在する:
			\begin{align}
				K_n \subset K_{n+1}^{\mathrm{o}},
				\quad S = \bigcup_{n=1}^\infty K_n.
			\end{align}
		\end{thm}
	\end{screen}
	
	\begin{prf}
		任意の$x \in S$に対して閉包がコンパクトな開近傍$U_x$を取っておく.
		$\mathscr{O}$を$S$の開集合系として
		\begin{align}
			\mathscr{B} \coloneqq
			\Set{U \in \mathscr{O}}{\mbox{$\overline{U}$がコンパクト}}
		\end{align}
		とおけば,$\mathscr{B}$は$\mathscr{O}$の基底となる.実際,
		任意の$O \in \mathscr{O}$に対し$O \cap U_x \in \mathscr{B}$かつ
		\begin{align}
			O = \bigcup_{x \in O} O \cap U_x
		\end{align}
		となる.従って定理\ref{thm:countable_base_of_second_countable_space}より
		或る可算部分集合$\{U_n\}_{n=1}^\infty \subset \mathscr{B}$が
		$\mathscr{O}$の基底となる.いま,$K_1 \coloneqq \overline{U_1}$として,
		またコンパクト集合$K_n$が選ばれたとして,
		$K_n$の有限被覆$\mathscr{U}_n \subset \mathscr{B}_0$を取り
		\begin{align}
			K_{n+1} \coloneqq \overline{U_{n+1}} \cup \bigcup_{V \in \mathscr{U}_n} \overline{V}
		\end{align}
		とすれば,$K_{n+1}$はコンパクトであり$K_n \subset K_{n+1}^{\mathrm{o}}$を満たす.
		この操作で$(K_n)_{n=1}^\infty$を構成すれば
		\begin{align}
			S = \bigcup_{n=1}^\infty U_n \subset \bigcup_{n=1}^\infty K_n \subset S
		\end{align}
		が成立する.
		\QED
	\end{prf}
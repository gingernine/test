\subsection{強位相}
	\begin{screen}
		\begin{thm}[連続線型写像は有界集合を有界集合に写す]
		\end{thm}
	\end{screen}
	
	$\left((X,\sigma_X),(\Phi,+,\bullet),s,\mathscr{O}_X\right)$を位相線型空間とし,
	\begin{align}
		\mathscr{B}
	\end{align}
	をその局所基とする.また
	\begin{align}
		\left(\left(X^*,\sigma_{X^*}\right),(\Phi,+,\bullet),s^*\right)
	\end{align}
	を連続双対空間とする.
	\begin{align}
		B
	\end{align}
	を$\left((X,\sigma_X),(\Phi,+,\bullet),s,\mathscr{O}_X\right)$の有界集合とするとき,
	\begin{align}
		X^* \ni f \longmapsto \sup{x \in B}{|f(x)|}
	\end{align}
	なる関係を$p_B$と定めると$p_B$は$\left(\left(X^*,\sigma_{X^*}\right),(\Phi,+,\bullet),s^*\right)$上のセミノルムとなる.
	\begin{align}
		\mathscr{P} \defeq \Set{p_B}{\mbox{$B$は$\left((X,\sigma_X),(\Phi,+,\bullet),s,\mathscr{O}_X\right)$の有界集合}}
	\end{align}
	でセミノルムの集合$\mathscr{P}$を定め,$\mathscr{P}$から作る近縁系で導入する$X^*$上の位相を
	\begin{align}
		\strong{X^*}{X}
	\end{align}
	で表し,これを$X^*$上の{\bf 強位相}\index{きょういそう@強位相}{\bf (strong topology)}と呼ぶ.いま
	$X$上にノルム
	\begin{align}
		\Norm{\cdot}{X}
	\end{align}
	が定まっていて,このノルムが$\mathscr{O}_X$と両立しているとする.このとき
	\begin{align}
		B \defeq \Set{x \in X}{\Norm{x}{X} \leq 1}
	\end{align}
	は$\left((X,\sigma_X),(\Phi,+,\bullet),s,\mathscr{O}_X\right)$の有界集合となるから
	\begin{align}
		\Set{f \in X^*}{p_B(f) < 1}
	\end{align}
	も有界な零元の近傍となる(\cite{key4-1}thm1.37).そして
	\begin{align}
		p_B
	\end{align}
	は$X^*$上のノルムとなり,$\strong{X^*}{X}$と両立する.
	この$p_B$を{\bf 作用素ノルム}\index{さようそのるむ@作用素ノルム}{\bf (operator norm)}と呼ぶが,まだ準備が足りていないので
	厳密な証明は後回し(2019/5/9).つまり作用素ノルムとは,
	$X$がノルム空間であるときに
	\begin{align}
		f \longmapsto \sup{\Norm{x}{}\leq 1}{|f(x)|}
	\end{align}
	なる関係で定められる連続双対$X^*$上のノルムのことである.
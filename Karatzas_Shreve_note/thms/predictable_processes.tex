\section{可予測過程}
	$\mathbf{T}$は$[0,\infty[$か$[0,T]$を表すものとする.
	$(\Omega, \mathscr{F}, P)$を確率空間とし,$\{\mathscr{F}_t\}_{t \in \mathbf{T}}$をフィルトレーションとする.
	
	\begin{itemize}
		\item $f:\mathbf{T} \times \Omega \longrightarrow \R$.
		
		\item 各$\omega \in \Omega$で,
			\begin{align}
				\mathbf{T} \ni s \longmapsto f(s,\omega)
			\end{align}
			が各$t \in \mathbf{T} \backslash \{0\}$において左連続.
			
		\item 各$t \in \mathbf{T}$で
			\begin{align}
				\Omega \ni \omega \longmapsto f(t,\omega)
			\end{align}
			は$\mathscr{F}_t/\borel{\R}$-可測.
	\end{itemize}
	
	なる$f$の全体を$\mathscr{L}_{\mathbf{T}}$で表す.また$\mathscr{L}_{\mathbf{T}}$の全ての要素を可測にする
	$\mathbf{T} \times \Omega$上の最小の$\sigma$-加法族を$\mathscr{P}_{\mathbf{T}}$で表す:
	\begin{align}
		\mathscr{P}_{\mathbf{T}} \defeq 
		\sigma\left(\Set{f^{-1}(A)}{f \in \mathscr{L}_{\mathbf{T}} \wedge A \in \borel{\R}}\right).
	\end{align}
	
	$\mathbf{T} \times \Omega$上の実数値写像で$\mathscr{P}_{\mathbf{T}}/\borel{\R}$-可測なるものを
	$\{\mathscr{F}_t\}_{t \in \mathbf{T}}$-{\bf 可予測過程}\index{かよそくかてい@可予測過程}{\bf (predictable process)}と呼ぶ.
	
	\begin{screen}
		\begin{thm}[$\mathscr{P}_{\mathbf{T}}$を生成する乗法族]\label{thm:pi_system_generating_predictable_sets}
			$\{0\} \times B\ (B \in \mathscr{F}_0)$または$(u,v] \times B\ ((u,v] \subset \mathbf{T},\ B \in \mathscr{F}_u)$
			なる形の集合の全体を$\mathscr{U}_{\mathbf{T}}$とおくとき,
			$\mathscr{U}_{\mathbf{T}}$は乗法族であって
			\begin{align}
				\sigma\left(\mathscr{U}_{\mathbf{T}}\right) = \mathscr{P}_{\mathbf{T}}.
			\end{align}
		\end{thm}
	\end{screen}
	
	\begin{sketch}\mbox{}
		\begin{description}
			\item[第一段] $\mathscr{U}_{\mathbf{T}}$が乗法族であることを示す.
				$A,B$を$\mathscr{U}_{\mathbf{T}}$から任意に選ばれた要素とする.$A$が
				\begin{align}
					A = \{0\} \times A'
				\end{align}
				なる形の場合は,$B$が
				\begin{align}
					B = (a,b] \times B'
				\end{align}
				なる形なら
				\begin{align}
					A \cap B = \emptyset
				\end{align}
				となり,$B$が
				\begin{align}
					B = \{0\} \times B'
				\end{align}
				なる形なら
				\begin{align}
					A \cap B = \{0\} \times (A' \cap B')
				\end{align}
				となるので,いずれにせよ
				\begin{align}
					A \cap B \in \mathscr{U}_{\mathbf{T}}
				\end{align}
				となる.$A$が
				\begin{align}
					A = (u,v] \times A'
				\end{align}
				なる形の場合は,$B$が
				\begin{align}
					B = (a,b] \times B'
				\end{align}
				なる形なら,$(u,v] \cap (a,b]$は重なれば区間,重ならないなら空となり$A \cap B$も空となる.重なる場合
				\begin{align}
					(u,v] \cap (a,b] = (\max{\{u,a\}}, \min{\{v,b\}}]
				\end{align}
				となるが,このとき
				\begin{align}
					A' \in \mathscr{F}_{\max{\{u,a\}}} \wedge B' \in \mathscr{F}_{\max{\{u,a\}}}
				\end{align}
				となるので
				\begin{align}
					A \cap B = (\max{\{u,a\}}, \min{\{v,b\}}] \times (A' \cap B') \in \mathscr{U}_{\mathbf{T}}
				\end{align}
				が成り立つ.以上より$\mathscr{U}_{\mathbf{T}}$が乗法族であることが示された.
			
			\item[第二段]
				$\sigma\left(\mathscr{U}_{\mathbf{T}}\right) = \mathscr{P}_{\mathbf{T}}$を示す.
				$A$を$\mathscr{U}_{\mathbf{T}}$から任意に選ばれた要素とすると,
				\begin{align}
					\defunc_A \in \mathscr{L}_{\mathbf{T}}
				\end{align}
				が成り立つ.ゆえに
				\begin{align}
					A = \defunc_A^{-1}(\{1\}) \in \mathscr{P}_{\mathbf{T}}
				\end{align}
				が成り立つので
				\begin{align}
					\mathscr{U}_{\mathbf{T}} \subset \mathscr{P}_{\mathbf{T}}
				\end{align}
				が従い
				\begin{align}
					\sigma\left(\mathscr{U}_{\mathbf{T}}\right) \subset \mathscr{P}_{\mathbf{T}}
				\end{align}
				が得られる.次に$\phi$を$\mathscr{L}_{\mathbf{T}}$から任意に選ばれた要素とする.
				ここで$\mathbf{T} = [0,\infty)$の場合と$\mathbf{T} = [0,T]$の場合に分ける.
				\begin{description}
					\item[(a)] $\mathbf{T} = [0,\infty)$の場合,自然数$n$に対して
						\begin{align}
							\phi^n(t,\omega)
							= \phi(0,\omega)\defunc_{\{0\}}(t) + 
							\sum_{j=0}^{n2^n-1}\phi(j/2^n,\omega) \defunc_{(j/2^n,(j+1)/2^n]}(t)
						\end{align}
						で$\phi^n$を定めると,
						\begin{align}
							\forall t \in \mathbf{T}\, \forall \omega \in \Omega\,
							\phi^n(t,\omega) \longrightarrow \phi(t,\omega)
							\quad (n \longrightarrow \infty)
						\end{align}
						が成り立つ.他方で$\borel{\R}$の任意の要素$E$に対して
						\begin{align}
							{\phi^n}^{-1}(E)
							= \{0\} \times \phi_0^{-1}(E) \cup 
							\bigcup_{j=0}^{n2^n-1} (j/2^n,(j+1)/2^n] \times \phi_{j/2^n}^{-1}(E)
						\end{align}
						が成り立つので,$\phi^n$は$\sigma\left(\mathscr{U}_{\mathbf{T}}\right)/\borel{\R}$-可測である.
						ゆえに$\phi$も$\sigma\left(\mathscr{U}_{\mathbf{T}}\right)/\borel{\R}$-可測である.
						$\phi$の任意性より
						\begin{align}
							\Set{f^{-1}(A)}{f \in \mathscr{L}_{\mathbf{T}} \wedge A \in \borel{\R}}
							\subset \sigma\left(\mathscr{U}_{\mathbf{T}}\right)
						\end{align}
						が従うので
						\begin{align}
							\mathscr{P}_{\mathbf{T}} \subset \sigma\left(\mathscr{U}_{\mathbf{T}}\right)
						\end{align}
						が得られる.
						
					\item[(b)] $\mathbf{T} = [0,T]$の場合,自然数$n$に対して
						\begin{align}
							\phi^n(t,\omega)
							= \phi(0,\omega)\defunc_{\{0\}}(t) + 
							\sum_{j=0}^{n2^n-1}\phi(jT/2^n,\omega) \defunc_{(jT/2^n,(j+1)T/2^n]}(t)
						\end{align}
						で$\phi^n$を定めると,$(\phi^n)_{n\in\Natural}$は$\phi$に各点収束し,
						$\phi^n$は$\sigma\left(\mathscr{U}_{\mathbf{T}}\right)/\borel{\R}$-可測で,
						$\phi$の$\sigma\left(\mathscr{U}_{\mathbf{T}}\right)/\borel{\R}$-可測性が従い
						\begin{align}
							\mathscr{P}_{\mathbf{T}} \subset \sigma\left(\mathscr{U}_{\mathbf{T}}\right)
						\end{align}
						が得られる.\QED
				\end{description}
		\end{description}
	\end{sketch}
	
	\begin{screen}
		\begin{thm}[右連続化したフィルトレーションに関する可予測過程は元のフィルトレーションに適合する]
		\label{thm:predictable_process_adapted_to_right_continuous_filtration}
			$\{\mathscr{G}_t\}_{t \in \mathbf{T}}$をフィルトレーションとし,各$t \in \mathbf{T}$で
			\begin{align}
				\mathscr{F}_t \defeq \mathscr{G}_{t+},
			\end{align}
			ただし$\mathbf{T} = [0,T]$の場合は
			\begin{align}
				\mathscr{F}_T \defeq \mathscr{G}_{T}
			\end{align}
			として右連続なフィルトレーション$\{\mathscr{F}_t\}_{t \in \mathbf{T}}$を定める.
			このとき$f$を$\{\mathscr{F}_t\}_{t \in \mathbf{T}}$-可予測過程とすると,
			各$t \in \mathbf{T} \backslash \{0\}$で
			\begin{align}
				\Omega \ni \omega \longmapsto f(t,\omega)
			\end{align}
			なる写像は$\mathscr{G}_t/\borel{\R}$-可測である.
		\end{thm}
	\end{screen}
	
	\begin{sketch}
		$\mathbf{T} \times \Omega$上のDynkin族を
		\begin{align}
			\mathscr{D} \defeq \Set{A \in \mathscr{P}_{\mathbf{T}}}{
			\mbox{各$t \in \mathbf{T} \backslash \{0\}$で$\Omega \ni \omega \longmapsto \defunc_A(t,\omega)$が
			$\mathscr{G}_t/\borel{\R}$-可測}}
		\end{align}
		で定める.このとき
		\begin{align}
			\mathscr{U}_{\mathbf{T}} \subset \mathscr{D}
			\label{fom:thm_predictable_process_adapted_to_right_continuous_filtration}
		\end{align}
		が成り立てばDynkin族定理と定理\ref{thm:pi_system_generating_predictable_sets}より
		\begin{align}
			\mathscr{P}_{\mathbf{T}} = \sigma\left(\mathscr{U}_{\mathbf{T}}\right) = \mathscr{D}
		\end{align}
		が成立する.すると$\mathscr{P}_{\mathbf{T}}$-可測単関数に対して定理の主張が満たされ,
		その各点収束極限で表せる$\mathscr{P}_{\mathbf{T}}$-可測関数に対しても定理の主張が満たされる.
		いま,$A$を$\mathscr{U}_{\mathbf{T}}$から任意に選ばれた要素とする.
		\begin{description}
			\item[(a)]
				$A = \{0\} \times B$なる形の場合,$t \in \mathbf{T} \backslash \{0\}$なる$t$を取ると
				\begin{align}
					\Omega \ni \omega \longmapsto \defunc_A(t,\omega)
				\end{align}
				は恒等的に$0$となる.ゆえに$\mathscr{G}_t/\borel{\R}$-可測である.
				
			\item[(b)]
				$A = (u,v] \times B$なる形の場合,$t \in \mathbf{T} \backslash \{0\}$なる$t$を取ると,
				\begin{align}
					t \leq u \vee v < t
				\end{align}
				なら
				\begin{align}
					\Omega \ni \omega \longmapsto \defunc_A(t,\omega)
				\end{align}
				は恒等的に$0$となるから$\mathscr{G}_t/\borel{\R}$-可測である.
				\begin{align}
					u < t \leq v
				\end{align}
				なら
				\begin{align}
					\Omega \ni \omega \longmapsto \defunc_A(t,\omega)
				\end{align}
				は$\defunc_B$に一致し,$\defunc_B$は$\mathscr{G}_u/\borel{\R}$-可測なので
				$\mathscr{G}_t/\borel{\R}$-可測である.
		\end{description}
		以上より
		\begin{align}
			A \in \mathscr{D}
		\end{align}
		が成立するので,(\refeq{fom:thm_predictable_process_adapted_to_right_continuous_filtration})が従う.
		\QED
	\end{sketch}
	
	\begin{screen}
		\begin{thm}[可予測過程は発展的可測]\label{thm:predictable_process_is_progressively_measurable}
			$\{\mathscr{F}_t\}_{t \in \mathbf{T}}$-可予測過程は$\{\mathscr{F}_t\}_{t \in \mathbf{T}}$-発展的可測である.
		\end{thm}
	\end{screen}
	
	\begin{sketch}
		いま
		\begin{align}
			\mathscr{D} \defeq \Set{A \in \mathscr{P}_{\mathbf{T}}}{
			\mbox{$\defunc_A$が$\{\mathscr{F}_t\}_{t \in \mathbf{T}}$-発展的可測}}
		\end{align}
		としてDynkin族を定める.
		\begin{align}
			\mathscr{U}_{\mathbf{T}} \subset \mathscr{D}
			\label{fom:thm_composition_of_predictable_process_and_stopping_time}
		\end{align}
		が成り立てば,Dynkin族定理と定理\ref{thm:pi_system_generating_predictable_sets}より
		\begin{align}
			\mathscr{P}_{\mathbf{T}} = \sigma\left(\mathscr{U}_{\mathbf{T}}\right) = \mathscr{D}
		\end{align}
		が成立する.いま,$A$を$\mathscr{U}_{\mathbf{T}}$から任意に選ばれた要素とし,
		$t$を$\mathbf{T}$から任意に選ばれた要素とする.
		\begin{description}
			\item[(a)]
				$A = \{0\} \times B$なる形の場合,
				\begin{align}
					\defunc_A|_{[0,t] \times \Omega} = \defunc_A
				\end{align}
				が成り立ち,また$\{0\} \in \borel{[0,t]}$かつ$B \in \mathscr{F}_0 \subset \mathscr{F}_t$であるから
				$\defunc_A|_{[0,t] \times \Omega}$は$\borel{[0,t]} \otimes \mathscr{F}_t/\borel{\R}$-可測である.
				
			\item[(b)]
				$A = (u,v] \times B$なる形の場合,
				\begin{align}
					t \leq u
				\end{align}
				なら$\defunc_A|_{[0,t] \times \Omega}$は恒等的に$0$となり,
				\begin{align}
					u < t
				\end{align}
				なら
				\begin{align}
					\defunc_A|_{[0,t] \times \Omega} = \defunc_{(u,\min{\{v,t\}}] \times B}
				\end{align}
				となり,$(u,\min{v,t}] \in \borel{[0,t]}$かつ$B \in \mathscr{F}_u \subset \mathscr{F}_t$となるので
				$\defunc_A|_{[0,t] \times \Omega}$は$\borel{[0,t]} \otimes \mathscr{F}_t/\borel{\R}$-可測である.
		\end{description}
		以上より
		\begin{align}
			A \in \mathscr{D}
		\end{align}
		が成立するので,(\refeq{fom:thm_composition_of_predictable_process_and_stopping_time})が従う.
		\QED
	\end{sketch}
\section{確率過程}
	$(\Omega,\mathscr{F},P)$を確率空間とし,$\mathbf{T}$を$[0,\infty[,[0,\infty],[0,T]$のいずれかとする.
	また$(S,d)$を距離空間とする.$X$を$\mathbf{T} \times \Omega$上の$S$値写像とするとき,
	$\mathbf{T}$の要素$t$に対して
	\begin{align}
		X_t \defeq \Set{(\omega,X(t,\omega))}{\omega \in \Omega}
	\end{align}
	により$X_t$を定める.つまり$X_t$は
	\begin{align}
		\omega \longmapsto X(t,\omega)
	\end{align}
	なる写像である.
	
	\begin{screen}
		\begin{dfn}[確率過程]
			集合$X$が
			\begin{itemize}
				\item $X:\mathbf{T} \times \Omega \longrightarrow S$.
				\item $t$を$\mathbf{T}$から任意に選ばれた要素とするとき$X_t$が$\mathscr{F}/\borel{S}$-可測である.
			\end{itemize}
			を満たすとき,$X$を$(\Omega,\mathscr{F},P)$上の$S$値{\bf 確率過程}{\bf (stochastic process)}と呼ぶ.
		\end{dfn}
	\end{screen}
	
	$\omega$を$\Omega$の要素とするとき,
	\begin{align}
		\mathbf{T} \ni t \longmapsto X_t(\omega)
	\end{align}
	なる写像を$\omega$の{\bf 標本路}{\bf (sample path)}と呼ぶ.標本路の性質に対して
	確率過程$X$の呼び名が分かれる:
	\begin{itemize}
		\item $\Omega$の全ての要素の標本路が$RCLL$である場合,$X$を$RCLL$な$S$値確率過程と呼ぶ.
		\item $\Omega$の全ての要素の標本路が連続である場合,$X$を連続な$S$値確率過程と呼ぶ.
		\item 標本路が$P$-a.s.に$RCLL$である場合,$X$を$P$-a.s.に$RCLL$な$S$値確率過程と呼ぶ.
		\item 標本路が$P$-a.s.に連続である場合,$X$を$P$-a.s.に連続な$S$値確率過程と呼ぶ.
	\end{itemize}
	
	$X$を$(\Omega,\mathscr{F},P)$上の$S$値確率過程とし,$\{\mathscr{F}_t\}_{t \in \mathbf{T}}$を$\mathscr{F}$に付随するフィルトレーションとする.このとき
	\begin{itemize}
		\item $X$が$\borel{\mathbf{T}} \otimes \mathscr{F}/\borel{S}$-可測であるとき,
			$X$は{\bf 可測な確率過程}{\bf (measurable stochastic process)}であるという.
		\item $\mathbf{T}$の任意の$t$要素に対して,$X$の$[0,t] \times \Omega$への制限写像
			$X|_{[0,t] \times \Omega}$が$\borel{[0,t]} \otimes \mathscr{F}_t/\borel{S}$-可測であるとき,
			$X$は{\bf 発展的可測な確率過程}{\bf (progressively measurable stochastic process)}であるという.
	\end{itemize}
	
	
	\begin{screen}
		\begin{dfn}[適合]
			$X$を$(\Omega,\mathscr{F},P)$上の$S$値確率過程とし,
			$\{\mathscr{F}_t\}_{t \in \mathbf{T}}$を$\mathscr{F}$に付随するフィルトレーションとするとき,
			\begin{itemize}
				\item $t$を$\mathbf{T}$から任意に選ばれた要素とするとき$X_t$が$\mathscr{F}_t/\borel{S}$-可測
			\end{itemize}
			であるなら$X$は$\{\mathscr{F}_t\}_{t \in \mathbf{T}}$に{\bf 適合している}
			{\bf (adapted to $\{\mathscr{F}_t\}_{t \in \mathbf{T}}$)}という.または
			$X$を$\{\mathscr{F}_t\}_{t \in \mathbf{T}}$-適合な$(\Omega,\mathscr{F},P)$上の$S$値確率過程と呼ぶ.
		\end{dfn}
	\end{screen}
	
	実数値確率過程$X$が$\{\mathscr{F}_t\}_{t \in \mathbf{T}}$-優マルチンゲールであるとは,
	\begin{itemize}
		\item $X$は$\{\mathscr{F}_t\}_{t \in \mathbf{T}}$に適合していて,かつ$\mathbf{T}$の各要素$t$で$X_t$は可積分である.
		\item $s,t$を$\mathbf{T}$の要素とするとき,$s \leq t$ならば$P$-a.s.に$\cexp{X_t}{\mathscr{F}_s} \leq X_s$である.
			つまり
			\begin{align}
				s \leq t \Longrightarrow 
				\forall E \in \mathscr{F}_s \left[ \int_E \cexp{X_t}{\mathscr{F}_s}\ dP \leq \int_E X_s\ dP\right].
			\end{align}
	\end{itemize}
	が満たされていることを指す.二つ目の条件が
	\begin{itemize}
		\item $s,t$を$\mathbf{T}$の要素とするとき,$s \leq t$ならば$P$-a.s.に$X_s \leq \cexp{X_t}{\mathscr{F}_s}$である.
	\end{itemize}
	に置き換わった場合,$X$は$\{\mathscr{F}_t\}_{t \in \mathbf{T}}$-劣マルチンゲールと呼ばれ,
	$X$が$\{\mathscr{F}_t\}_{t \in \mathbf{T}}$-優マルチンゲールであり$\{\mathscr{F}_t\}_{t \in \mathbf{T}}$-劣マルチンゲールでもある場合,
	つまり二つ目の条件が
	\begin{itemize}
		\item $s,t$を$\mathbf{T}$の要素とするとき,$s \leq t$ならば$P$-a.s.に$X_s = \cexp{X_t}{\mathscr{F}_s}$である.
	\end{itemize}
	に置き換わった場合,$X$は$\{\mathscr{F}_t\}_{t \in \mathbf{T}}$-マルチンゲールと呼ばれる.
	
	以下性質を列挙する:
	\begin{itembox}[l]{優マルチンゲールの正負を反転したものは劣マルチンゲール}
		$X$を$\{\mathscr{F}_t\}_{t \in \mathbf{T}}$-優マルチンゲールとするとき,
		$-X$は$\{\mathscr{F}_t\}_{t \in \mathbf{T}}$-劣マルチンゲールである.逆も然り.
	\end{itembox}
	
	\begin{prf}
		$-X$の適合性及び$-X_t$可積分性は$X$に対する仮定より従う.
		$s,t$を$s \leq t$なる$\mathbf{T}$の要素とし,$E$を$\mathscr{F}_s$の要素とするとき,
		\begin{align}
			\int_E X_t\ dP \leq \int_E X_s\ dP
		\end{align}
		が成立する.よって
		\begin{align}
			\int_E -X_s\ dP \leq \int_E -X_t\ dP
		\end{align}
		が成立する.よって
		\begin{align}
			\int_E -X_s\ dP \leq \int_E \cexp{-X_t}{\mathscr{F}_s}\ dP
		\end{align}
		が成立する.
		\QED
	\end{prf}
	
	\begin{itembox}[l]{マルチンゲールは期待値が一定}
		$X$を$\{\mathscr{F}_t\}_{t \in \mathbf{T}}$-マルチンゲールとするとき,
		\begin{align}
			\forall t \in \mathbf{T}\, (\, E(X_t) = E(X_0)\, ).
		\end{align}
	\end{itembox}
	
	\begin{prf}
		$t$を$\mathbf{T}$の要素とすれば$P$-a.s.に$\cexp{X_t}{\mathscr{F}_0} = X_0$となるので
		\begin{align}
			E(X_t) = E(\cexp{X_t}{\mathscr{F}_0}) = E(X_0)
		\end{align}
		が成り立つ.
		\QED
	\end{prf}
	
	\begin{itembox}[l]{期待値が一定な優マルチンゲールはマルチンゲール}
		$X$を$\{\mathscr{F}_t\}_{t \in \mathbf{T}}$-優マルチンゲールとするとき,
		\begin{align}
			\forall t \in \mathbf{T}\, (\, E(X_t) = E(X_0)\, )
		\end{align}
		ならば$X$は$\{\mathscr{F}_t\}_{t \in \mathbf{T}}$-マルチンゲールである.
	\end{itembox}
	
	\begin{prf}
		$s,t$を$\mathbf{T}$の要素とし,$s \leq t$とする.このとき$P$-a.s.に
		\begin{align}
			0 \leq X_s - \cexp{X_t}{\mathscr{F}_s}
		\end{align}
		が成り立つので,$P$-a.s.に
		\begin{align}
			X_s - \cexp{X_t}{\mathscr{F}_s} = \left| X_s - \cexp{X_t}{\mathscr{F}_s} \right|
		\end{align}
		となる.ゆえに
		\begin{align}
			0 = E\left(X_s - \cexp{X_t}{\mathscr{F}_s}\right)
			= E\left| X_s - \cexp{X_t}{\mathscr{F}_s} \right|
		\end{align}
		が成り立つので,$P$-a.s.に
		\begin{align}
			X_s = \cexp{X_t}{\mathscr{F}_s}
		\end{align}
		が成り立つ.
		\QED
	\end{prf}
	
	\begin{itembox}[l]{Jensenの不等式の応用:マルチンゲール}
		$X$を$\{\mathscr{F}_t\}_{t \in \mathbf{T}}$-マルチンゲールとし,$f$を
		$f:\R \longrightarrow \R$なる凸関数(resp. 凹関数)とする.このとき
		\begin{align}
			\forall t \in \mathbf{T}\, (\, E|f \circ X_t| < \infty\, )
		\end{align}
		ならば$f \circ X$は$\{\mathscr{F}_t\}_{t \in \mathbf{T}}$-劣マルチンゲール(resp. 優マルチンゲール)である.
	\end{itembox}
	
	\begin{prf}
		$s,t$を$s \leq t$なる$\mathbf{T}$の要素とする.
		Jensenの不等式(定理\ref{thm:Jensen_inequality_for_convex_functions})より,$P$-a.s.の$\omega( \in \Omega)$に対し
		\begin{align}
			f(X_s(\omega)) = f(\cexp{X_t}{\mathscr{F}_s}(\omega)) \leq \cexp{f \circ X_t}{\mathscr{F}_s}(\omega)
		\end{align}
		が成り立つ.$f$が凹関数である場合,$-f$は凸関数であるから$P$-a.s.の$\omega( \in \Omega)$に対し
		\begin{align}
			-f(X_s(\omega)) = -f(\cexp{X_t}{\mathscr{F}_s}(\omega)) \leq \cexp{-f \circ X_t}{\mathscr{F}_s}(\omega)
		\end{align}
		が成り立つ.ゆえに
		\begin{align}
			\cexp{f \circ X_t}{\mathscr{F}_s}(\omega) \leq f(X_s(\omega))
		\end{align}
		が成り立つ.
		\QED
	\end{prf}
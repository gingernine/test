\section{距離空間上の連続写像}
\subsection{広義一様収束を定める距離}
	$(X,d_X),(Y,d_Y)$を距離空間とし,距離位相を導入して
	\begin{align}
		C(X,Y) \coloneqq \Set{f:X \longrightarrow Y}{\mbox{$f$は連続写像}}
	\end{align}
	とおく.このとき$K \subset X$をコンパクト集合とすれば,
	\begin{align}
		\rho_K(f,g) \coloneqq \sup{x \in K}{d_Y(f(x),g(x))},
		\quad (f,g \in C(X,Y))
	\end{align}
	により定める$\rho_K$は$C(X,Y)$の擬距離となる.実際,$f(K),g(K)$は$Y$のコンパクト部分集合であるから
	\begin{align}
		\operatorname{diam}(f(K)) = \sup{y,y' \in f(K)}{d_Y(y,y')} < \infty,
	\end{align}
	及び$\operatorname{diam}(g(K)) < \infty$が成り立ち,任意に$x_0 \in K$を取れば
	\begin{align}
		\sup{x \in K}{d_Y(f(x),g(x))} 
		&\leq \sup{x \in K}{d_Y(f(x),f(x_0))} + d_Y(f(x_0),g(x_0)) + \sup{x \in K}{d_Y(g(x_0),g(x))} \\
		&\leq \operatorname{diam}(f(K)) + d_Y(f(x_0),g(x_0)) + \operatorname{diam}(g(K))
		< \infty
	\end{align}
	となるから$\rho_K$は$[0,\infty)$値である.
	また$d_Y$が対称性と三角不等式を満たすから$\rho$も対称性を持ち三角不等式を満たす.
	いま,$X$が$\sigma$-コンパクトであると仮定する.つまり
	\begin{align}
		K_1 \subset K_2 \subset K_3 \subset \cdots,
		\quad \bigcup_{n=1}^\infty K_n = X
		\label{eq:completeness_and_separability_of_space_of_continuous_functions_1}
	\end{align}
	を満たすコンパクト部分集合の増大列$(K_n)_{n=1}^\infty$が存在するとき,$\rho_n = \rho_{K_n}$とすれば
	\begin{align}
		\rho_n(f,g) = 0\ (\forall n \geq 1) \quad \Longrightarrow \quad f = g
	\end{align}
	が成り立つから,
	\begin{align}
		\rho(f,g) \coloneqq \sum_{n=1}^\infty 2^{-n} \left( 1 \wedge \rho_n(f,g) \right),
		\quad (f,g \in C(X,Y))
		\label{eq:distance_in_spaces_of_continuous_functions}
	\end{align}
	により$C(X,Y)$上に距離$\rho$が定まる.特に,
	定理\ref{thm:second_countable_Hausdorff_sigma_compact}より
	$X$が可分かつ局所コンパクトなら
	\begin{align}
		K_n \subset K_{n+1}^{\mathrm{o}},
		\quad X = \bigcup_{n=1}^\infty K_n
		\label{eq:distance_in_spaces_of_continuous_functions_2}
	\end{align}
	を満たすコンパクト部分集合の列$(K_n)_{n=1}^\infty$が存在するから$\rho$が定義される.
	
	\begin{screen}
		\begin{thm}[広義一様収束を定める距離]
			$(X,d_X)$を可分な局所コンパクト距離空間,$(Y,d_Y)$を距離空間とし,
			(\refeq{eq:distance_in_spaces_of_continuous_functions_2})
			を満たす$(K_n)_{n=1}^\infty$で$\rho$を定めるとき,$f,f_n \in C(X,Y)$に対して
			次が成り立つ.
			\begin{align}
				\mbox{$(f_n)_{n=1}^\infty$が$f$に広義一様収束する}
				\quad \Longleftrightarrow \quad
				\rho(f,f_n) \longrightarrow 0
				\quad (n \longrightarrow \infty).
			\end{align}
		\end{thm}
	\end{screen}
	
	\begin{screen}
		\begin{thm}[$C(X,Y)$の可分性]\label{thm:separability_of_spaces_of_continuous_functions}
			$(X,d_X)$を$\sigma$-コンパクト距離空間,$(Y,d_Y)$を可分距離空間とするとき,$C(X,Y)$は$\rho$により可分距離空間となる.
		\end{thm}
	\end{screen}
	
	\begin{prf}\mbox{}
		\begin{description}
			\item[第一段]
				三段にわたり,コンパクト集合$K \subset X$に対して或る高々可算集合$D(K) \subset C(X,Y)$があり,
				任意の$\epsilon > 0$と$f \in C(X,Y)$に対して
				次を満たす$g \in D(K)$が存在することを示す:
				\begin{align}
					d_Y(f(x),g(x)) < \epsilon,
					\quad (\forall x \in K).
					\label{eq:completeness_and_separability_of_space_of_continuous_functions_5}
				\end{align}
				$x \in X$の半径$\delta > 0$の開球を$B_\delta(x)$と書けば,$K$のコンパクト性より任意の$m \in \N$に対し
				\begin{align}
					K \subset \bigcup_{i=1}^{k(m)} B_{1/m}(x^m_i)
				\end{align}
				を満たす$\left\{ x^m_1, \cdots, x^m_{k(m)} \right\} \subset K$が存在する.
				また$Y$はLindel\Ddot{o}f性を持つから,任意の$\ell \geq 1$に対し
				\begin{align}
					\mathscr{U}_\ell
					\coloneqq \Set{U_j^\ell}{U_j^\ell:open,\ \operatorname{diam}\left( U_j^\ell \right) < \frac{1}{\ell};\ j=1,2,\cdots},
					\quad Y = \bigcup_{j=1}^\infty U^\ell_j
				\end{align}
				を満たす開被覆$\mathscr{U}_\ell$が存在する.
				一方で,$f \in C(X,Y)$は$K$上で一様連続であるから
				\begin{align}
					C_{m,n} \coloneqq
					\Set{f \in C(X,Y)}{\mbox{任意の$x,x' \in K$に対し}d_X(x,x') < \frac{1}{m} \Rightarrow d_Y(f(x),f(x')) < \frac{1}{n}}
				\end{align}
				とすれば
				\begin{align}
					C(X,Y) = \bigcap_{n=1}^\infty \bigcup_{m=1}^\infty C_{m,n}
					\label{eq:completeness_and_separability_of_space_of_continuous_functions_6}
				\end{align}
				が成り立つ.いま,任意に$m,n,\ell$及び$i=(i_1,\cdots,i_{k(m)}) \in \N^{k(m)}$を取り
				\begin{align}
					D^i_{m,n,\ell} \coloneqq
					\Set{g \in C_{m,n}}{g(x^m_j) \in U^\ell_{i_j},\ (\forall j = 1,\cdots,k(m))}
				\end{align}
				とおけば,例えば$i=(1,\cdots,1) \in \N^{k(m)}$と$y \in U^\ell_1$に対して恒等写像$g:X\longrightarrow \{y\}$は
				$g \in D^i_{m,n,\ell}$となるから
				\begin{align}
					\Phi_{m,n,\ell} \in \prod_{\substack{i \in \N^{k(m)} \\ D^i_{m,n,\ell} \neq \emptyset}} D^i_{m,n,\ell}
				\end{align}
				が存在する.ここで
				\begin{align}
					D_{m,n,\ell} \coloneqq \Set{\Phi_{m,n,\ell}(i)}{i \in \N^{k(m)}}
				\end{align}
				により$D_{m,n,\ell}$を定めて
				\begin{align}
					D_{m,n} \coloneqq \bigcup_{\ell=1}^\infty D_{m,n,\ell}, \quad
					D(K) \coloneqq \bigcup_{m,n=1}^\infty D_{m,n}
				\end{align}
				とおく.
			
			\item[第二段]
				任意の$f \in C_{m,n}$と$\epsilon > 0$に対し或る$g \in D_{m,n}$が存在して
				\begin{align}
					d_Y\left(f(x^m_j),g(x^m_j)\right) < \epsilon,
					\quad (\forall j=1,\cdots,k(m))
				\end{align}
				を満たすことを示す.実際,$1/\ell < \epsilon$となる$\ell$に対し
				$\mathscr{U}_\ell$は$Y$の被覆であるから,
				\begin{align}
					f(x^m_j) \in U^\ell_{i_j},
					\quad (\forall j=1,\cdots,k(m))
				\end{align}
				となる$i=(i_1,\cdots,i_{k(m)}) \in \N^{k(m)}$が取れる.
				従って$D^i_{m,n,\ell} \neq \emptyset$であり,
				\begin{align}
					g \coloneqq \Phi_{m,n,\ell}(i)
				\end{align}
				に対して
				\begin{align}
					d_Y\left(f(x^m_j),g(x^m_j)\right) < \frac{1}{\ell} < \epsilon,
					\quad (\forall j=1,\cdots,k(m))
				\end{align}
				が成立する.
				
			\item[第三段]
				$D(K)$が(\refeq{eq:completeness_and_separability_of_space_of_continuous_functions_5})
				を満たすことを示す.任意に$f \in C(X,Y)$と$\epsilon > 0$を取れば,
				(\refeq{eq:completeness_and_separability_of_space_of_continuous_functions_6})より
				$1/n < \epsilon/3$を満たす$n$及び或る$m$に対して$f \in C_{m,n}$となる.
				このとき,前段の結果より或る$g \in D_{m,n} \subset D(K)$が存在して
				\begin{align}
					d_Y\left(f(x^m_j),g(x^m_j)\right) < \frac{\epsilon}{3},
					\quad (\forall j=1,\cdots,k(m))
				\end{align}
				を満たす.$f,g \in C_{m,n}$より
				任意の$x \in B_{1/m}(x^m_j)$に対して
				\begin{align}
					d_Y\left(f(x),f(x^m_j)\right),\ d_Y\left(g(x),g(x^m_j)\right) < \frac{1}{n} < \frac{\epsilon}{3}
				\end{align}
				が成り立ち,任意の$x \in K$は或る$B_{1/m}(x^m_j)$に含まれるから,
				\begin{align}
					d_Y\left(f(x),g(x)\right)
					&\leq d_Y\left(f(x),f(x^m_j)\right) + d_Y\left(f(x^m_j),g(x^m_j)\right) + d_Y\left(g(x),g(x^m_j)\right) \\
					&< \frac{\epsilon}{3}+\frac{\epsilon}{3}+\frac{\epsilon}{3} \\
					&= \epsilon
				\end{align}
				が従い(\refeq{eq:completeness_and_separability_of_space_of_continuous_functions_5})が出る.
				
			\item[第四段]				
				$(K_n)_{n=1}^\infty$を(\refeq{eq:completeness_and_separability_of_space_of_continuous_functions_1})を満たす
				コンパクト集合列とすれば,各$K_n$に対し$D(K_n)$が存在し,
				\begin{align}
					D \coloneqq \bigcup_{n=1}^\infty D(K_n)
				\end{align}
				と定めれば$D$は$C(X,Y)$で高々可算かつ稠密となる.実際,任意の$\epsilon > 0$と$f \in C(X,Y)$に対して,
				\begin{align}
					\sum_{n=N+1}^\infty 2^{-n} < \frac{\epsilon}{2}
				\end{align}
				を満たす$N \geq 1$を取れば,
				\begin{align}
					\rho_N(f,g) < \frac{\epsilon}{2}
				\end{align}
				を満たす$g \in D(K_N) \subset D$が存在するから
				\begin{align}
					\rho(f,g) &= \sum_{n=1}^N 2^{-n} \left( 1 \wedge \rho_n(f,g) \right)
						+ \sum_{n=N+1}^\infty 2^{-n} \left( 1 \wedge \rho_n(f,g) \right) \\
					&< \frac{\epsilon}{2} + \frac{\epsilon}{2} \\
					&< \epsilon
				\end{align}
				が成り立つ.
				\QED
		\end{description}
	\end{prf}
	
	\begin{screen}
		\begin{thm}[$C(X,Y)$の完備性]\label{thm:completeness_of_spaces_of_continuous_functions}
			$(X,d_X)$を可分な局所コンパクト距離空間,$(Y,d_Y)$を距離空間,$(f_n)_{n=1}^\infty$を$C(X,Y)$の列とし,
			(\refeq{eq:distance_in_spaces_of_continuous_functions_2})
			を満たす$(K_n)_{n=1}^\infty$で$\rho$を定める.このとき
			各点$x \in X$で$\lim_{n \to \infty} f_n(x)$が存在すれば
			\begin{align}
				f \coloneqq \lim_{n \to \infty} f_n \in C(X,Y),
				\quad \rho(f,f_n) \longrightarrow 0 \quad (n \longrightarrow \infty)
				\label{eq:completeness_and_separability_of_space_of_continuous_functions_3}
			\end{align}
			が成立する.特に$(Y,d_Y)$が完備なら$C(X,Y)$は$\rho$により完備距離空間となる.
		\end{thm}
	\end{screen}
	
	\begin{prf}\mbox{}
		\begin{description}
			\item[第一段]
				任意の$j \geq 1$に対し
				\begin{align}
					\rho_j(f_n,f) \longrightarrow 0
					\quad (n \longrightarrow \infty)
					\label{eq:completeness_and_separability_of_space_of_continuous_functions_4}
				\end{align}
				が成り立つことを示す.実際,任意の$\epsilon > 0$に対し或る$N \geq 1$が存在して
				\begin{align}
					\rho_j(f_n,f_m) < \frac{\epsilon}{2}
					\quad (\forall n,m \geq N)
				\end{align}
				が満たされ,また$f$の定め方より任意の$x \in K_j$に対し
				\begin{align}
					d_Y(f_m(x),f(x)) < \frac{\epsilon}{2}
				\end{align}
				を満たす$m \geq N$が存在するから,
				\begin{align}
					d_Y(f_n(x),f(x)) \leq d_Y(f_n(x),f_m(x)) + d_Y(f_m(x),f(x)) 
					< \frac{\epsilon}{2} + \frac{\epsilon}{2}
					= \epsilon,
					\quad (\forall n \geq N)
				\end{align}
				が従い
				\begin{align}
					\rho_j(f_n,f) \leq \epsilon,
					\quad (\forall n \geq N)
				\end{align}
				が成立する.
				
			\item[第二段]
				$f$の連続性を示す.任意に$\epsilon > 0$と$x \in X$及び
				$x \in \interior{K_j}$を満たす$K_j$を取れば,
				(\refeq{eq:completeness_and_separability_of_space_of_continuous_functions_4})より
				\begin{align}
					\rho_j(f_n,f) < \frac{\epsilon}{3}
				\end{align}
				を満たす$n \geq 1$が存在する.
				また$f_n$の連続性より$x$の或る開近傍$W$が存在して
				\begin{align}
					d_Y(f_n(x),f_n(x')) < \frac{\epsilon}{3},
					\quad (\forall x' \in W)
				\end{align}
				となるから,
				\begin{align}
					d_Y(f(x),f(x'))
					\leq d_Y(f(x),f_n(x)) + d_Y(f_n(x),f_n(x')) + d_Y(f_n(x'),f(x'))
					< \epsilon,
					\quad (\forall x' \in W \cap \interior{K_j})
				\end{align}
				が従い$f$の$x$における連続性が出る.
			
			\item[第三段]
				(\refeq{eq:completeness_and_separability_of_space_of_continuous_functions_3})を示す.
				任意の$\epsilon > 0$に対し,
				\begin{align}
					\sum_{k=k_0+1}^\infty 2^{-k} < \frac{\epsilon}{2}
				\end{align}
				を満たす$k_0 \geq 1$が存在する.
				また(\refeq{eq:completeness_and_separability_of_space_of_continuous_functions_4})より
				或る$n_0 \geq 1$が存在して
				\begin{align}
					\rho_{k_0}(f_n,f) < \frac{\epsilon}{2},
					\quad (\forall n \geq n_0)
				\end{align}
				となるから
				\begin{align}
					\rho(f_n,f) < \epsilon, \quad (\forall n \geq n_0)
				\end{align}
				が成立する.
				\QED
		\end{description}
	\end{prf}
	
	\begin{screen}
		\begin{thm}[$C(X,Y)$の完備可分性]\label{thm:appendix_complete_separability_of_spaces_of_continuous_functions}
			$(X,d_X)$を可分な局所コンパクト距離空間,$(Y,d_Y)$を距離空間とし,
			(\refeq{eq:distance_in_spaces_of_continuous_functions_2})
			を満たす$(K_n)_{n=1}^\infty$で$\rho$を定めるとき,
			$C(X,Y)$は$\rho$により完備可分距離空間となる.
		\end{thm}
	\end{screen}
	
	\begin{prf}
		定理\ref{thm:separability_of_spaces_of_continuous_functions}と
		定理\ref{thm:completeness_of_spaces_of_continuous_functions}より従う.
		\QED
	\end{prf}

\subsection{正規族}
	\begin{screen}
		\begin{dfn}[正規族]
			$(X,d_X),(Y,d_Y)$を距離空間,
			$\mathscr{F} \subset C(X,Y)$とする.
			$\mathscr{F}$の任意の列$(f_n)_{n=1}^\infty$が
			$X$で広義一様収束する(収束先が連続写像である必要はない)部分列を含むとき,
			$\mathscr{F}$を正規族(normal family)という.
		\end{dfn}
	\end{screen}
	
	\begin{screen}
		\begin{thm}[正規族の相対コンパクト性]
			$(X,d_X)$を可分な局所コンパクト距離空間,$(Y,d_Y)$を距離空間とし,
			(\refeq{eq:distance_in_spaces_of_continuous_functions_2})
			を満たす$(K_n)_{n=1}^\infty$で$\rho$を定め$C(X,Y)$に距離位相を導入する.
			このとき,正規族$\mathscr{F}$に対して
			\begin{align}
				\overline{\mathscr{F}} \subset C(X,Y).
			\end{align}
			が成立し,また$\overline{\mathscr{F}}$はコンパクトとなる.
		\end{thm}
	\end{screen}
	
	\begin{screen}
		\begin{dfn}[同程度連続]
			$(X,d_X),(Y,d_Y)$を距離空間,$\mathscr{F} \subset C(X,Y)$とする.
			任意の$\epsilon > 0$に対し
			\begin{align}
				d_X(p,q) < \delta \quad \Longrightarrow \quad
				d_Y(f(p),f(q)) < \epsilon,\ (\forall f \in \mathscr{F})
			\end{align}
			を満たす$\delta > 0$が存在するとき,$\mathscr{F}$は同程度連続である(equicontinuous)という.
		\end{dfn}
	\end{screen}
	
	\begin{screen}
		\begin{thm}[Ascoli-Arzela]
			$(X,d_X)$を可分な局所コンパクト距離空間,$(Y,d_Y)$を距離空間とするとき
			\begin{align}
				\mbox{$\mathscr{F} \subset C(X,Y)$が正規族} \quad \Longleftrightarrow \quad
				\begin{cases}
					\mbox{$\mathscr{F}$が任意のコンパクト集合$K \subset X$で同程度連続,} & \\
					\mbox{各点$x \in X$で$\overline{\Set{f(x)}{f \in \mathscr{F}}}$がコンパクトである.} & 
				\end{cases}
				\label{eq:thm_Ascoli_Arzela}
			\end{align}
		\end{thm}
	\end{screen}
	
	\begin{prf}\mbox{}
		\begin{description}
			\item[第一段]
				$E = \{x_n\}_{n=1}^\infty$を$X$で可算稠密な集合とし,
				$\mathscr{F}$が(\refeq{eq:thm_Ascoli_Arzela})右辺の仮定を満たしているとする.
				$K \subset X$を任意のコンパクト集合とすれば,
				同程度連続性より任意の$\epsilon > 0$に対し或る$\delta > 0$が存在して
				\begin{align}
					p,q \in K,\ d_X(p,q) < \delta \quad \Longrightarrow \quad
					\sup{f \in \mathscr{F}}{d_Y(f(p),f(q))} < \frac{\epsilon}{3}
				\end{align}
				が成立する.また半径$\delta/2$の$K$の開被覆$B_1,\cdots,B_M$が存在し,
				$E$は稠密であるから
				\begin{align}
					p_j \in B_j \cap E, \quad j=1,\cdots,M
				\end{align}
				を選んでおく.任意に$\{f_n\}_{n=1}^\infty \subset \mathscr{F}$を取れば,
				\begin{align}
					\overline{\Set{f_n(x_1)}{n=1,2,\cdots}}
					\subset \overline{\Set{f(x)}{f \in \mathscr{F}}}
				\end{align}
				より$\overline{\left\{f_{n}(x_1)\right\}_{n=1}^\infty}$は
				コンパクトであるから収束部分列$\left\{f_{n(k,1)}(x_1)\right\}_{k=1}^\infty$が存在する.
				同様に$\left\{f_{n(k,1)}(x_2)\right\}_{k=1}^\infty$の
				或る部分列$\left\{f_{n(k,2)}(x_2)\right\}_{k=1}^\infty$は
				$Y$で収束し,繰り返せば部分添数系
				\begin{align}
					\{n(k,1)\}_{k=1}^\infty \supset
					\{n(k,2)\}_{k=1}^\infty \supset
					\{n(k,3)\}_{k=1}^\infty \supset
					\cdots
				\end{align}
				が構成される.$n(k) \coloneqq n(k,k),\ (\forall k \geq 1)$
				とおけば任意の$x_i \in E$に対して
				$\left\{f_{n(k)}(x_i)\right\}_{k=i}^\infty$は収束列
				$\left\{f_{n(k,i)}(x_i)\right\}_{k=1}^\infty$の部分列となるから収束し,
				従って或る$N \geq 1$が存在して
				\begin{align}
					u,v > N \quad \Longrightarrow \quad
					d_Y\left(f_{n(u)}(p_j),f_{n(v)}(p_j)\right) < \frac{\epsilon}{3},
					\ (\forall j=1,2,\cdots,M)
				\end{align}
				を満たす.任意に$x \in K$を取れば或る$j$で$x \in B_j$かつ
				$d_X(x,p_j) < \delta$となるから,$u,v > N$なら
				\begin{align}
					d_Y(f_{n(u)}(x),f_{n(v)}(x)) 
					&\leq d_Y\left(f_{n(u)}(x),f_{n(u)}(p_j)\right) 
					+ d_Y\left(f_{n(u)}(p_j),f_{n(v)}(p_j)\right)
					+ d_Y\left(f_{n(v)}(p_j),f_{n(v)}(x)\right) \\
					&< \frac{\epsilon}{3} + \frac{\epsilon}{3} + \frac{\epsilon}{3} < \epsilon
					\label{eq:thm_Ascoli_Arzela_2}
				\end{align}
				が成り立つ.$\{f_{n(k)}(x)\}_{k=1}^\infty$は相対コンパクトであるから
				$\lim_{k \to \infty} f_{n(k)}(x) \eqqcolon f(x) \in Y$が存在し,
				(\refeq{eq:thm_Ascoli_Arzela_2})より$\left(f_{n(k)}\right)_{k=1}^\infty$は$f$に$K$で一様収束するから
				$\mathscr{F}$は正規族である.
			
			\item[第二段]
				$X$の可分性と定理\ref{thm:second_countable_Hausdorff_sigma_compact}より
				\begin{align}
					K_n \subset K_{n+1}^{\mathrm{o}},
					\quad X = \bigcup_{n=1}^\infty K_n
				\end{align}
				を満たすコンパクト部分集合の列$(K_n)_{n=1}^\infty$が存在するから,これに対し
				(\refeq{eq:distance_in_spaces_of_continuous_functions})の距離$\rho$を定める.
				$\mathscr{F}$が正規族であるなら$\mathscr{F}$は$\rho$に関して全有界となるから,
				任意の$\epsilon > 0$に対して
				\begin{align}
					\mathscr{F}
					= \bigcup_{i=1}^N \Set{f \in \mathscr{F}}{\rho(f,f_i) < \epsilon}
				\end{align}
				を満たす$\{f_i\}_{i=1}^N \subset \mathscr{F}$が存在する.
				$K \subset X$がコンパクトなら或る$n$で$K \subset K_n$となり,或る$\delta > 0$が存在して
				\begin{align}
					p,q \in K,\ d_X(p,q) < \delta
					\quad \Longrightarrow \quad
					d_Y(f_i(p),f_i(q)) < \epsilon,\ (\forall i = 1,\cdots,N)
				\end{align}
				が成り立つから,任意の$f \in \mathscr{F}$に対し$\rho(f,f_i) < \epsilon$
				を満たす$f_i$を取れば
				\begin{align}
					p,q \in K,\ d_X(p,q) < \delta
					\quad \Longrightarrow \quad
					d_Y(f(p),f(q)) 
					&\leq d_Y(f(p),f_i(p)) + d_Y(f_i(p),f_i(q)) + d_Y(f_i(q),f(q)) \\
					&\leq 2^n \epsilon + \epsilon + 2^n \epsilon \\
					&= (2^{n+1} + 1)\epsilon
				\end{align}
				となる.すなわち$\mathscr{F}$は任意のコンパクト部分集合上で同程度連続である.
				また任意の$x \in X$に対し
				\begin{align}
					\Gamma(x) \coloneqq \Set{f(x)}{f \in \mathscr{F}}
				\end{align}
				とおくとき,任意の$\{w_n\}_{n=1}^\infty \subset \overline{\Gamma(x)}$に対して
				\begin{align}
					f_n(x) \in \Gamma(x),
					\quad d_Y(f_n(x),w_n) < \frac{1}{n}, \quad n=1,2,\cdots
				\end{align}
				を満たす$\left(f_n(x)\right)_{n=1}^\infty$を取れば,
				$\mathscr{F}$が正規族であるから
				収束部分列$\left(f_{n_k}(x)\right)_{k=1}^\infty$が存在して
				\begin{align}
					\lim_{k \to \infty} f_{n_k}(x)
					= \lim_{k \to \infty} w_{n_k}
					\in \overline{\Gamma(x)}
				\end{align}
				が成り立つから$\overline{\Gamma(x)}$はコンパクトである.
				\QED
		\end{description}
	\end{prf}
\section{連続写像の空間上の完備可分距離}
	$(X,d_X),(Y,d_Y)$を距離空間とし,
	\begin{align}
		C(X,Y) \coloneqq \Set{f:X \longrightarrow Y}{\mbox{$f$は連続写像}}
	\end{align}
	とおく.このとき$K \subset X$をコンパクト集合とすれば,
	\begin{align}
		\rho_K(f,g) \coloneqq \sup{x \in K}{d_Y(f(x),g(x))},
		\quad (f,g \in C(X,Y))
	\end{align}
	により定める$\rho_K$は$C(X,Y)$の擬距離となる.実際,$f(K),g(K)$は$Y$のコンパクト部分集合であるから
	\begin{align}
		\operatorname{diam}(f(K)) = \sup{y,y' \in f(K)}{d_Y(y,y')} < \infty,
	\end{align}
	及び$\operatorname{diam}(g(K)) < \infty$が成り立ち,任意に$x_0 \in K$を取れば
	\begin{align}
		\sup{x \in K}{d_Y(f(x),g(x))} 
		&\leq \sup{x \in K}{d_Y(f(x),f(x_0))} + d_Y(f(x_0),g(x_0)) + \sup{x \in K}{d_Y(g(x_0),g(x))} \\
		&\leq \operatorname{diam}(f(K)) + d_Y(f(x_0),g(x_0)) + \operatorname{diam}(g(K))
		< \infty
	\end{align}
	となるから$\rho_K$は$[0,\infty)$値である.
	また$d_Y$が対称性と三角不等式を満たすから$\rho$も対称性を持ち三角不等式を満たす.
	いま,$X$が$\sigma$-コンパクトであると仮定する.つまり
	\begin{align}
		K_1 \subset K_2 \subset K_3 \subset \cdots,
		\quad \bigcup_{n=1}^\infty K_n = X
		\label{eq:completeness_and_separability_of_space_of_continuous_functions_1}
	\end{align}
	を満たすコンパクト部分集合の増大列$(K_n)_{n=1}^\infty$が存在するとき,$\rho_n = \rho_{K_n}$とすれば
	\begin{align}
		\rho_n(f,g) = 0\ (\forall n \geq 1) \quad \Rightarrow \quad f = g
	\end{align}
	が成り立つから,
	\begin{align}
		\rho(f,g) \coloneqq \sum_{n=1}^\infty 2^{-n} \left( 1 \wedge \rho_n(f,g) \right),
		\quad (f,g \in C(X,Y))
	\end{align}
	により$C(X,Y)$上に距離$\rho$が定まる.
	
	\begin{screen}
		\begin{thm}[$C(X,Y)$の可分性]
			$(X,d_X)$を$\sigma$-コンパクト距離空間,$(Y,d_Y)$を可分距離空間とするとき,$C(X,Y)$は$\rho$により可分距離空間となる.
		\end{thm}
	\end{screen}
	
	\begin{prf}\mbox{}
		\begin{description}
			\item[第一段]
				三段にわたり,コンパクト集合$K \subset X$に対して或る高々可算集合$D(K) \subset C(X,Y)$があり,
				任意の$\epsilon > 0$と$f \in C(X,Y)$に対して
				次を満たす$g \in D(K)$が存在することを示す:
				\begin{align}
					d_Y(f(x),g(x)) < \epsilon,
					\quad (\forall x \in K).
					\label{eq:completeness_and_separability_of_space_of_continuous_functions_5}
				\end{align}
				$x \in X$の半径$\delta > 0$の開球を$B_\delta(x)$と書けば,$K$のコンパクト性より任意の$m \in \N$に対し
				\begin{align}
					K \subset \bigcup_{i=1}^{k(m)} B_{1/m}(x^m_i)
				\end{align}
				を満たす$\left\{ x^m_1, \cdots, x^m_{k(m)} \right\} \subset K$が存在する.
				また$Y$はLindel\Ddot{o}f性を持つから,任意の$\ell \geq 1$に対し
				\begin{align}
					\mathscr{U}_\ell
					\coloneqq \Set{U_j^\ell}{U_j^\ell:open,\ \operatorname{diam}\left( U_j^\ell \right) < \frac{1}{\ell};\ j=1,2,\cdots},
					\quad Y = \bigcup_{j=1}^\infty U^\ell_j
				\end{align}
				を満たす開被覆$\mathscr{U}_\ell$が存在する.
				一方で,$f \in C(X,Y)$は$K$上で一様連続であるから
				\begin{align}
					C_{m,n} \coloneqq
					\Set{f \in C(X,Y)}{\mbox{任意の$x,x' \in K$に対し}d_X(x,x') < \frac{1}{m} \Rightarrow d_Y(f(x),f(x')) < \frac{1}{n}}
				\end{align}
				とすれば
				\begin{align}
					C(X,Y) = \bigcap_{n=1}^\infty \bigcup_{m=1}^\infty C_{m,n}
					\label{eq:completeness_and_separability_of_space_of_continuous_functions_6}
				\end{align}
				が成り立つ.いま,任意に$m,n,\ell$及び$i=(i_1,\cdots,i_{k(m)}) \in \N^{k(m)}$を取り
				\begin{align}
					D^i_{m,n,\ell} \coloneqq
					\Set{g \in C_{m,n}}{g(x^m_j) \in U^\ell_{i_j},\ (\forall j = 1,\cdots,k(m))}
				\end{align}
				とおき,$D^i_{m,n,\ell} = \emptyset$の場合は$D^i_{m,n,\ell} = \{\emptyset\}$と定めれば,
				$D^i_{m,n,\ell}$は全ての$i \in \N^{k(m)}$について空ではないから
				\begin{align}
					\Phi_{m,n,\ell} \in \prod_{i \in \N^{k(m)}} D^i_{m,n,\ell}
				\end{align}
				が存在する.ここで
				\begin{align}
					D_{m,n,\ell} \coloneqq \Set{\Phi_{m,n,\ell}(i)}{i \in \N^{k(m)}}
				\end{align}
				により$D_{m,n,\ell}$を定めて
				\begin{align}
					D_{m,n} \coloneqq \bigcup_{\ell=1}^\infty D_{m,n,\ell}, \quad
					D(K) \coloneqq \bigcup_{m,n=1}^\infty D_{m,n}
				\end{align}
				とおく.
			
			\item[第二段]
				任意の$f \in C_{m,n}$と$\epsilon > 0$に対し或る$g \in D_{m,n}$が存在して
				\begin{align}
					d_Y\left(f(x^m_j),g(x^m_j)\right) < \epsilon,
					\quad (\forall j=1,\cdots,k(m))
				\end{align}
				を満たすことを示す.実際,$1/\ell < \epsilon$となる$\ell$に対し
				$\mathscr{U}_\ell$は$Y$の被覆であるから,
				\begin{align}
					f(x^m_j) \in U^\ell_{i_j},
					\quad (\forall j=1,\cdots,k(m))
				\end{align}
				となる$i=(i_1,\cdots,i_{k(m)}) \in \N^{k(m)}$が取れる.
				従って$D^i_{m,n,\ell} \neq \emptyset$であり,
				\begin{align}
					g \coloneqq \Phi_{m,n,\ell}(i)
				\end{align}
				に対して
				\begin{align}
					d_Y\left(f(x^m_j),g(x^m_j)\right) < \frac{1}{\ell} < \epsilon,
					\quad (\forall j=1,\cdots,k(m))
				\end{align}
				が成立する.
				
			\item[第三段]
				$D(K)$が(\refeq{eq:completeness_and_separability_of_space_of_continuous_functions_5})
				を満たすことを示す.任意に$f \in C(X,Y)$と$\epsilon > 0$を取れば,
				(\refeq{eq:completeness_and_separability_of_space_of_continuous_functions_6})より
				$1/n < \epsilon/3$を満たす$n$及び或る$m$に対して$f \in C_{m,n}$となる.
				このとき,前段の結果より或る$g \in D_{m,n} \subset D(K)$が存在して
				\begin{align}
					d_Y\left(f(x^m_j),g(x^m_j)\right) < \frac{\epsilon}{3},
					\quad (\forall j=1,\cdots,k(m))
				\end{align}
				を満たす.$f,g \in C_{m,n}$より
				任意の$x \in B_{1/m}(x^m_j)$に対して
				\begin{align}
					d_Y\left(f(x),f(x^m_j)\right),\ d_Y\left(g(x),g(x^m_j)\right) < \frac{1}{n} < \frac{\epsilon}{3}
				\end{align}
				が成り立ち,任意の$x \in K$は或る$B_{1/m}(x^m_j)$に含まれるから,
				\begin{align}
					d_Y\left(f(x),g(x)\right)
					&\leq d_Y\left(f(x),f(x^m_j)\right) + d_Y\left(f(x^m_j),g(x^m_j)\right) + d_Y\left(g(x),g(x^m_j)\right) \\
					&< \frac{\epsilon}{3}+\frac{\epsilon}{3}+\frac{\epsilon}{3} \\
					&= \epsilon
				\end{align}
				が従い(\refeq{eq:completeness_and_separability_of_space_of_continuous_functions_5})が出る.
				
			\item[第四段]				
				$(K_n)_{n=1}^\infty$を(\refeq{eq:completeness_and_separability_of_space_of_continuous_functions_1})を満たす
				コンパクト集合列とすれば,各$K_n$に対し(\refeq{eq:completeness_and_separability_of_space_of_continuous_functions_5})
				を満たす$D(K_n)$が存在し,
				\begin{align}
					D \coloneqq \bigcup_{n=1}^\infty D(K_n)
				\end{align}
				と定めれば$D$は$C(X,Y)$で高々可算かつ稠密となる.実際,任意の$\epsilon > 0$と$f \in C(X,Y)$に対して,
				\begin{align}
					\sum_{n=N+1}^\infty 2^{-n} < \frac{\epsilon}{2}
				\end{align}
				を満たす$N \geq 1$を取れば,
				\begin{align}
					\rho_N(f,g) < \frac{\epsilon}{2}
				\end{align}
				を満たす$g \in D(K_N) \subset D$が存在するから
				\begin{align}
					\rho(f,g) &= \sum_{n=1}^N 2^{-n} \left( 1 \wedge \rho_n(f,g) \right)
						+ \sum_{n=N+1}^\infty 2^{-n} \left( 1 \wedge \rho_n(f,g) \right) \\
					&< \frac{\epsilon}{2} + \frac{\epsilon}{2} \\
					&< \epsilon
				\end{align}
				が成り立つ.
				\QED
		\end{description}
	\end{prf}
	
	\begin{screen}
		\begin{thm}[$C(X,Y)$の完備性]
			$(X,d_X)$を距離空間,$(Y,d_Y)$を完備距離空間とする.$X$において
			\begin{align}
				K_1 \subset K_2 \subset K_3 \subset \cdots,
				\quad \bigcup_{n=1}^\infty \interior{K_n} = X
				\label{eq:completeness_and_separability_of_space_of_continuous_functions_2}
			\end{align}
			を満たすコンパクト部分集合の列$(K_n)_{n=1}^\infty$が存在するとき,$C(X,Y)$は$\rho$により完備距離空間となる.
		\end{thm}
	\end{screen}
	
	\begin{prf}\mbox{}
		\begin{description}
			\item[第一段]
				$(f_n)_{n=1}^\infty$を$C(X,Y)$のCauchy列とする.
				任意の$x \in X$に対し$x$を含む$K_j$を取れば
				\begin{align}
					1 \wedge d_Y(f_n(x),f_m(x)) 
					\leq 1 \wedge \rho_j(f_n,f_m)
					\leq 2^j \rho(f_n,f_m) \longrightarrow 0
					\quad (n,m \longrightarrow \infty)
				\end{align}
				となるから,$Y$の完備性より$\lim_{n \to \infty} f_n(x)$が存在する.ここで
				\begin{align}
					f(x) \coloneqq \lim_{n \to \infty} f_n(x), \quad (x \in X)
				\end{align}
				により写像$f:X \longrightarrow Y$を定めれば,$f$は連続であり
				\begin{align}
					\rho(f_n,f) \longrightarrow 0 \quad (n \longrightarrow \infty)
					\label{eq:completeness_and_separability_of_space_of_continuous_functions_3}
				\end{align}
				を満たす.
			
			\item[第二段]
				任意の$K_j$に対し
				\begin{align}
					\rho_j(f_n,f) \longrightarrow 0
					\quad (n \longrightarrow \infty)
					\label{eq:completeness_and_separability_of_space_of_continuous_functions_4}
				\end{align}
				が成り立つことを示す.実際,任意の$\epsilon > 0$に対し或る$N \geq 1$が存在して
				\begin{align}
					\rho_j(f_n,f_m) < \frac{\epsilon}{2}
					\quad (\forall n,m \geq N)
				\end{align}
				が満たされ,また$f$の定め方より任意の$x \in K_j$に対し
				\begin{align}
					d_Y(f_m(x),f(x)) < \frac{\epsilon}{2}
				\end{align}
				を満たす$m \geq N$が存在するから,
				\begin{align}
					d_Y(f_n(x),f(x)) \leq d_Y(f_n(x),f_m(x)) + d_Y(f_m(x),f(x)) 
					< \frac{\epsilon}{2} + \frac{\epsilon}{2}
					= \epsilon,
					\quad (\forall n \geq N)
				\end{align}
				が従い
				\begin{align}
					\rho_j(f_n,f) \leq \epsilon,
					\quad (\forall n \geq N)
				\end{align}
				が成立する.
				
			\item[第三段]
				$f$の連続性を示す.任意に$\epsilon > 0$と$x \in X$及び
				$x \in \interior{K_j}$を満たす$K_j$を取れば,
				(\refeq{eq:completeness_and_separability_of_space_of_continuous_functions_4})より
				\begin{align}
					\rho_j(f_n,f) < \frac{\epsilon}{3}
				\end{align}
				を満たす$n \geq 1$が存在する.
				また$f_n$の連続性より$x$の或る開近傍$W$が存在して
				\begin{align}
					d_Y(f_n(x),f_n(x')) < \frac{\epsilon}{3},
					\quad (\forall x' \in W)
				\end{align}
				となるから,
				\begin{align}
					d_Y(f(x),f(x'))
					\leq d_Y(f(x),f_n(x)) + d_Y(f_n(x),f_n(x')) + d_Y(f_n(x'),f(x'))
					< \epsilon,
					\quad (\forall x' \in W \cap \interior{K_j})
				\end{align}
				が従い$f$の$x$における連続性が出る.
			
			\item[第四段]
				(\refeq{eq:completeness_and_separability_of_space_of_continuous_functions_3})を示す.
				任意の$\epsilon > 0$に対し,
				\begin{align}
					\sum_{k=k_0+1}^\infty 2^{-k} < \frac{\epsilon}{2}
				\end{align}
				を満たす$k_0 \geq 1$が存在する.
				また(\refeq{eq:completeness_and_separability_of_space_of_continuous_functions_4})より
				或る$n_0 \geq 1$が存在して
				\begin{align}
					\rho_{k_0}(f_n,f) < \frac{\epsilon}{2},
					\quad (\forall n \geq n_0)
				\end{align}
				となるから
				\begin{align}
					\rho(f_n,f) < \epsilon, \quad (\forall n \geq n_0)
				\end{align}
				が成立する.
				\QED
		\end{description}
	\end{prf}
	
	以上二つの定理より次の主張が得られる.
	\begin{screen}
		\begin{thm}[$C(X,Y)$の完備可分性]\label{thm:appendix_complete_separability_of_spaces_of_continuous_functions}
			$(X,d_X)$を距離空間,$(Y,d_Y)$を完備可分距離空間とする.$X$において
			\begin{align}
				K_1 \subset K_2 \subset K_3 \subset \cdots,
				\quad \bigcup_{n=1}^\infty \interior{K_n} = X
			\end{align}
			を満たすコンパクト部分集合の列$(K_n)_{n=1}^\infty$が存在するとき,$C(X,Y)$は$\rho$により完備可分距離空間となる.
		\end{thm}
	\end{screen}
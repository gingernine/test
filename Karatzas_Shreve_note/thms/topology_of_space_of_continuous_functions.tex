\section{連続写像の空間上の完備可分距離}
	$(X,d_X),(Y,d_Y)$を距離空間とし,
	\begin{align}
		C(X,Y) \coloneqq \Set{f:X \longrightarrow Y}{\mbox{$f$は連続写像}}
	\end{align}
	とおく.このとき$K \subset X$をコンパクト集合とすれば,任意の$f,g \in C(X,Y)$に対し
	\begin{align}
		\sup{x \in K}{d_Y(f(x),g(x))} < \infty
	\end{align}
	が満たされる,実際,$f(K),g(K)$は$Y$のコンパクト部分集合であるから
	\begin{align}
		\operatorname{diam}(f(K)) = \sup{y,y' \in f(K)}{d_Y(y,y')} < \infty,
	\end{align}
	及び$\operatorname{diam}(g(K)) < \infty$が成り立ち,任意に$x_0 \in K$を取れば
	\begin{align}
		\sup{x \in K}{d_Y(f(x),g(x))} 
		&\leq \sup{x \in K}{d_Y(f(x),f(x_0))} + d_Y(f(x_0),g(x_0)) + \sup{x \in K}{d_Y(g(x_0),g(x))} \\
		&\leq \operatorname{diam}(f(K)) + d_Y(f(x_0),g(x_0)) + \operatorname{diam}(g(K))
		< \infty
	\end{align}
	となる.いま,$X$が$\sigma$-コンパクトであると仮定する.つまり
	\begin{align}
		K_1 \subset K_2 \subset K_3 \subset \cdots,
		\quad \bigcup_{n=1}^\infty K_n = X
		\label{eq:completeness_and_separability_of_space_of_continuous_functions_1}
	\end{align}
	を満たすコンパクト部分集合の増大列$(K_n)_{n=1}^\infty$が存在するとき,
	\begin{align}
		\rho(f,g) \coloneqq \sum_{n=1}^\infty 2^{-n} \left( 1 \wedge \sup{x \in K_n}{d_Y(f(x),g(x))} \right),
		\quad (f,g \in C(X,Y))
	\end{align}
	で定める$\rho$は$C(X,Y)$上の距離関数となる.実際,$f=g$なら$\sup{x \in K_n}{d_Y(f(x),g(x))}=0,\ (\forall n \geq 1)$より
	$\rho(f,g)=0$となり,逆に$\rho(f,g)=0$なら,任意の$x \in X$に対し$x \in K_n$を満たす$K_n$があるから
	\begin{align}
		d_Y(f(x),g(x)) \leq \sup{x \in K_n}{d_Y(f(x),g(x))} = 0
	\end{align}
	が成り立ち$f = g$が従う.また$d_Y$の対称性より$\rho$の対称性が従い,
	\begin{align}
		\sup{x \in K_n}{d_Y(f(x),g(x))}
		\leq \sup{x \in K_n}{d_Y(f(x),h(x))} + \sup{x \in K_n}{d_Y(h(x),g(x))},
		\quad (\forall n=1,2,\cdots)
	\end{align}
	より$\rho$は三角不等式を満たす.
	
	\begin{screen}
		\begin{thm}[$C(X,Y)$の可分性]
			$(X,d_X)$を$\sigma$-コンパクト距離空間,$(Y,d_Y)$を可分距離空間とするとき,$C(X,Y)$は$\rho$により可分距離空間となる.
		\end{thm}
	\end{screen}
	
	\begin{prf}\mbox{}
		\begin{description}
			\item[第一段]
				
		\end{description}
	\end{prf}
	
	\begin{screen}
		\begin{thm}[$C(X,Y)$の完備性]
			$(X,d_X)$を距離空間,$(Y,d_Y)$を完備距離空間とする.$X$において
			\begin{align}
				K_1 \subset K_2 \subset K_3 \subset \cdots,
				\quad \bigcup_{n=1}^\infty \interior{K_n} = X
				\label{eq:completeness_and_separability_of_space_of_continuous_functions_2}
			\end{align}
			を満たすコンパクト部分集合の列$(K_n)_{n=1}^\infty$が存在するとき,$C(X,Y)$は$\rho$により完備距離空間となる.
		\end{thm}
	\end{screen}
	
	\begin{prf}\mbox{}
		\begin{description}
			\item[第一段]
				$(f_n)_{n=1}^\infty$を$C(X,Y)$のCauchy列とする.
				任意の$x \in X$に対し$x$を含む$K_j$を取れば
				\begin{align}
					1 \wedge d_Y(f_n(x),f_m(x)) 
					\leq 1 \wedge \sup{t \in K_j}{d_Y(f_n(t),f(t))}
					\leq 2^j \rho(f_n,f_m) \longrightarrow 0
					\quad (n,m \longrightarrow \infty)
				\end{align}
				となるから,$Y$の完備性より$\lim_{n \to \infty} f_n(x)$が存在する.ここで
				\begin{align}
					f(x) \coloneqq \lim_{n \to \infty} f_n(x), \quad (x \in X)
				\end{align}
				により写像$f:X \longrightarrow Y$を定めれば,$f$は連続であり
				\begin{align}
					\rho(f_n,f) \longrightarrow 0 \quad (n \longrightarrow \infty)
					\label{eq:completeness_and_separability_of_space_of_continuous_functions_3}
				\end{align}
				を満たす.
			
			\item[第二段]
				任意の$K_j$に対し
				\begin{align}
					\sup{x \in K_j}{d_Y(f_n(x),f(x))} \longrightarrow 0
					\quad (n \longrightarrow \infty)
					\label{eq:completeness_and_separability_of_space_of_continuous_functions_4}
				\end{align}
				が成り立つことを示す.実際,任意の$\epsilon > 0$に対し或る$N \geq 1$が存在して
				\begin{align}
					\sup{x \in K_j}{d_Y(f_n(x),f_m(x))} < \frac{\epsilon}{2}
					\quad (\forall n,m \geq N)
				\end{align}
				が満たされ,また$f$の定め方より任意の$x \in K_j$に対し
				\begin{align}
					d_Y(f_m(x),f(x)) < \frac{\epsilon}{2}
				\end{align}
				を満たす$m \geq N$が存在するから,
				\begin{align}
					d_Y(f_n(x),f(x)) \leq d_Y(f_n(x),f_m(x)) + d_Y(f_m(x),f(x)) 
					< \frac{\epsilon}{2} + \frac{\epsilon}{2}
					= \epsilon,
					\quad (\forall n \geq N)
				\end{align}
				が従い
				\begin{align}
					\sup{x \in K_j}{d_Y(f_n(x),f(x))} \leq \epsilon,
					\quad (\forall n \geq N)
				\end{align}
				が成立する.
				
			\item[第三段]
				$f$の連続性を示す.任意に$\epsilon > 0$と$x \in X$及び
				$x \in \interior{K_j}$を満たす$K_j$を取れば,
				(\refeq{eq:completeness_and_separability_of_space_of_continuous_functions_4})より
				\begin{align}
					d_Y(f_n(t),f(t)) < \frac{\epsilon}{3},
					\quad (\forall t \in K_j)
				\end{align}
				を満たす$n \geq 1$が存在する.
				また$f_n$の連続性より$x$の或る開近傍$W$が存在して
				\begin{align}
					d_Y(f_n(x),f_n(x')) < \frac{\epsilon}{3},
					\quad (\forall x' \in W)
				\end{align}
				となるから,
				\begin{align}
					d_Y(f(x),f(x'))
					\leq d_Y(f(x),f_n(x)) + d_Y(f_n(x),f_n(x')) + d_Y(f_n(x'),f(x'))
					< \epsilon,
					\quad (\forall x' \in W \cap \interior{K_j})
				\end{align}
				が従い$f$の$x$における連続性が出る.
			
			\item[第四段]
				(\refeq{eq:completeness_and_separability_of_space_of_continuous_functions_3})を示す.
				任意の$\epsilon > 0$に対し,
				\begin{align}
					\sum_{k=k_0+1}^\infty 2^{-k} < \frac{\epsilon}{2}
				\end{align}
				を満たす$k_0 \geq 1$が存在する.
				また(\refeq{eq:completeness_and_separability_of_space_of_continuous_functions_4})より
				或る$n_0 \geq 1$が存在して
				\begin{align}
					\sup{x \in K_{k_0}}{d_Y(f_n(x),f(x))} < \frac{\epsilon}{2},
					\quad (\forall n \geq n_0)
				\end{align}
				となるから
				\begin{align}
					\rho(f_n,f) < \epsilon, \quad (\forall n \geq n_0)
				\end{align}
				が成立する.
				\QED
		\end{description}
	\end{prf}
	
	\begin{screen}
		\begin{thm}[$C(X,Y)$の完備可分性]
			$(X,d_X)$を距離空間,$(Y,d_Y)$を完備可分距離空間とする.$X$において
			\begin{align}
				K_1 \subset K_2 \subset K_3 \subset \cdots,
				\quad \bigcup_{n=1}^\infty \interior{K_n} = X
				\label{eq:completeness_and_separability_of_space_of_continuous_functions_2}
			\end{align}
			を満たすコンパクト部分集合の列$(K_n)_{n=1}^\infty$が存在するとき,$C(X,Y)$は$\rho$により完備可分距離空間となる.
		\end{thm}
	\end{screen}
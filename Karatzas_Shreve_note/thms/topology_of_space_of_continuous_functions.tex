\section{連続写像の空間の位相}
	$(X,d_X),(Y,d_Y)$を距離空間とし,
	\begin{align}
		C(X,Y) \coloneqq \Set{f:X \longrightarrow Y}{\mbox{$f$は連続写像}}
	\end{align}
	とおく.$X$が$\sigma$-コンパクトであるとき,つまり
	\begin{align}
		K_1 \subset K_2 \subset K_3 \subset \cdots,
		\quad \bigcup_{n=1}^\infty K_n = X 
	\end{align}
	を満たすコンパクト部分集合の列$(K_n)_{n=1}^\infty$が存在するとき,
	\begin{align}
		\rho(f,g) \coloneqq \sum_{n=1}^\infty 2^{-n} \left( 1 \wedge \sup{x \in K_n}{d_Y(f(x),g(x))} \right),
		\quad (f,g \in C(X,Y))
	\end{align}
	により定める$\rho$は$C(X,Y)$上の距離関数となる.
	実際,$f \in C(X,Y)$に対し$f(K_n)$はコンパクトであるから
	$\operatorname{diam}(f(K_n)) < \infty$
	\bddddegin{align}
		d_Y(f(x),g(x)) \leq d_Y(f(x),f(x_0)) + d_Y(f(x_0),g(x_0)) + d_Y(g(x_0),g(x))
		\leq \operatorname{diam}(f(K_n)) + d_Y(f(x_0),g(x_0)) + \operatorname{diam}(g(K_n))
	\end{align}
	
	\begin{screen}
		\begin{thm}
			$X$を$\sigma$-コンパクトな距離空間,$Y$を距離空間とするとき$C(X,Y)$は可分距離空間である.
		\end{thm}
	\end{screen}
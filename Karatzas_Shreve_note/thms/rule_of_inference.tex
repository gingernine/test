\subsection{推論規則メモ}
	\begin{screen}
		\begin{itemize}
			\item $(\rightharpoondown A) \vee A$
			\item $\rightharpoondown A$と$A$が共に定理なら$\bot$は定理となる.
			\item $\bot \Longrightarrow A$
			\item $(A \vee B) \Longrightarrow ((A \vee C) \Longrightarrow (B \vee C))$
			\item $A \vee B,\ A \Longrightarrow C,\ B \Longrightarrow C$が全て定理なら
				$C$は定理である.
			\item 
		\end{itemize}
	\end{screen}
	
	\begin{itembox}[l]{ならばとまたは}
		$B \Longrightarrow (A \Longrightarrow B)$は定理である.
	\end{itembox}
	
	\begin{prf}
		$B$を公理に追加した場合,$A$を公理に追加しても$B$は真であるから
		$A \Longrightarrow B$は<$B$を公理に追加した場合の>定理となる.
		従って$B \Longrightarrow (A \Longrightarrow B)$は定理である.
	\end{prf}
	
	\begin{itembox}[l]{ならばとまたは}
		$A \Longrightarrow B$と$\rightharpoondown A \vee B$は同値.
	\end{itembox}
	
	\begin{prf}
		$A \Longrightarrow B$が真であると仮定する.
		$(A \Longrightarrow B) \Longrightarrow (\ (A \vee \rightharpoondown A) \Longrightarrow (B \vee \rightharpoondown A)\ )$は公理であるから
		$(A \vee \rightharpoondown A) \Longrightarrow (B \vee \rightharpoondown A)$
		は定理となり,排中律より$A \vee \rightharpoondown A$は公理なので
		$B \vee \rightharpoondown A$は定理,よって$\rightharpoondown A \vee B$は定理である.
		以上で
		\begin{align}
			(A \Longrightarrow B) \Longrightarrow (\rightharpoondown A \vee B)
		\end{align}
		は定理である.逆に$\rightharpoondown A \vee B$が公理であると仮定する.
		このとき$A$を公理に追加すれば,$\bot$が定理となり$B$も定理となる.従って
		\begin{align}
			(\rightharpoondown A) \Longrightarrow (A \Longrightarrow B)
		\end{align}
		は定理となる.$B \Longrightarrow (A \Longrightarrow B)$も定理であるから,
		場合分けの法則より$A \Longrightarrow B$は定理となる.以上で
		\begin{align}
			(\rightharpoondown A \vee B) \Longrightarrow (A \Longrightarrow B)
		\end{align}
		は定理である.
	\end{prf}
	
	\begin{screen}
		$A$を式,$A$において$x$のみ自由変項,このとき
		\begin{align}
			\exists x\ A \Longleftrightarrow \left( \tau_x(A) \mid x \right) A
		\end{align}
		は定理である.
	\end{screen}
	
	\begin{screen}
		\begin{align}
			\left( \tau_x(A) \mid x \right) A
		\end{align}
		は定理である.
	\end{screen}
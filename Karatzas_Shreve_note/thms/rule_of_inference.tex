\subsection{推論規則メモ}
	
	\begin{screen}
		\begin{metaaxm}[基本的な推論規則]
			$A,B,C$を$\mathcal{L}'$の閉式とするとき,次の規則を認める:
			\begin{description}
				\item[三段論法] $A$ならびに$A \Longrightarrow B$が定理なら$B$は定理である.
				\item[演繹法則] $A$を公理に追加した下で$B$が定理であるなら,
					元の公理系で$A \Longrightarrow B$は定理である.
				\item[$\vee$の導入1] $A \Longrightarrow (A \vee B)$は定理である.
				\item[$\vee$の導入2] $A \Longrightarrow (B \vee A)$は定理である.
				\item[$\wedge$の導入] $A,B$が共に定理なら$A \wedge B$は定理である.
				\item[$\wedge$の除去1] $A \wedge B \Longrightarrow A$は定理である.
				\item[$\wedge$の除去2] $A \wedge B \Longrightarrow B$は定理である.
				\item[場合分け法則] $A \Longrightarrow C$と$B \Longrightarrow C$が共に定理であるとき
					$(A \vee B) \Longrightarrow C$は定理である.
			\end{description}	
		\end{metaaxm}
	\end{screen}
	
	\begin{screen}
		\begin{metathm}[推移律]
			$A,B,C$を$\mathcal{L}'$の閉式とするとき,
			$A \Longrightarrow B$と$B \Longrightarrow C$が共に定理ならば
			$A \Longrightarrow C$は定理である.
		\end{metathm}
	\end{screen}
	
	\begin{prf}
		$A$を公理に追加する.このとき三段論法より$B$が定理となり,
		再び三段論法より$C$が定理となる.ゆえに$A \Longrightarrow C$は定理である.
		\QED
	\end{prf}
	
	\begin{screen}
		\begin{metathm}
			$A,B,C$を$\mathcal{L}'$の閉式とするとき,
			$(A \Longrightarrow B) \Longrightarrow 
			(\ (A \vee C) \Longrightarrow (B \vee C)\ )$は定理である.
		\end{metathm}
	\end{screen}
	
	\begin{prf}
		いま$A \Longrightarrow B$を公理に追加する.
		このとき$A$を公理に追加すれば$B$は定理となり,
		$B \Longrightarrow (B \vee C)$が定理であるから$B \vee C$も定理となる.
		これより$A \Longrightarrow B$を公理とした下では
		\begin{align}
			A \Longrightarrow (B \vee C)
		\end{align}
		が定理となる.他方で$C \Longrightarrow (B \vee C)$は定理であるから,
		場合分け法則より$A \Longrightarrow B$を公理とした下では
		\begin{align}
			(A \vee C) \Longrightarrow (B \vee C)
		\end{align}
		が定理となる.従って演繹法則より
		\begin{align}
			(A \Longrightarrow B) \Longrightarrow 
			(\ (A \vee C) \Longrightarrow (B \vee C)\ )
		\end{align}
		は定理となる.
		\QED
	\end{prf}
	
	\begin{screen}
		\begin{metathm}
			$A,B$を$\mathcal{L}'$の閉式とするとき,
			$B \Longrightarrow (A \Longrightarrow B)$は定理である.
		\end{metathm}
	\end{screen}
	
	\begin{prf}
		$B$を公理に追加した場合,$A$を公理に追加しても$B$は真であるから,このとき
		\begin{align}
			A \Longrightarrow B
		\end{align}
		は定理となる.従って演繹法則より$B \Longrightarrow (A \Longrightarrow B)$は定理である.
		\QED
	\end{prf}
	
	\monologue{
		院生「$A$を$\mathcal{L}'$の閉式とするとき,``$A$は定理である''という言明は
		``$A$が成り立つ''や``$A$は正しい''や``$A$となる''などとも書かれます.」
	}
	
	\begin{screen}
		\begin{metaaxm}[矛盾]
			$A$を$\mathcal{L}'$の閉式とするとき,次の規則を認める:
			\begin{description}
				\item[矛盾の発生] $A$と$\rightharpoondown A$が共に定理なら$\bot$は定理となる.
				\item[否定の導出] $A \Longrightarrow \bot$が定理なら$\rightharpoondown A$は定理となる.
				\item[矛盾からはあらゆる式が導かれる] $\bot \Longrightarrow A$は定理である.
			\end{description}
		\end{metaaxm}
	\end{screen}
	
	\begin{screen}
		\begin{metaaxm}[排中律]
			$A$を$\mathcal{L}'$の閉式とするとき,次の規則を認める:
			\begin{itemize}
				\item $A \vee \rightharpoondown A$は定理である.
			\end{itemize}
		\end{metaaxm}
	\end{screen}
	
	\begin{screen}
		\begin{metathm}[二重否定の法則]
			$A$を$\mathcal{L}'$の閉式とするとき,
			$A \Longleftrightarrow\ \rightharpoondown \rightharpoondown A$は定理である.
		\end{metathm}
	\end{screen}
	
	\begin{prf}
		$A$が成り立っていると仮定する.このとき$\rightharpoondown A$が成り立っていれば
		$\bot$が定理となるので
		\begin{align}
			\rightharpoondown A \Longrightarrow \bot
		\end{align}
		が成り立つ.よって$\rightharpoondown \rightharpoondown A$が定理となり,
		\begin{align}
			A \Longrightarrow\ \rightharpoondown \rightharpoondown A
		\end{align}
		が得られる.逆に$\rightharpoondown \rightharpoondown A$が成り立っていると仮定すると,
		$\rightharpoondown A$が成り立っているなら$\bot$が定理となるから
		$A$も定理となる.すなわちこのとき
		\begin{align}
			\rightharpoondown A \Longrightarrow A
		\end{align}
		が成り立つ.$A \vee \rightharpoondown A$と$A \Longrightarrow A$も定理であるから
		$A$が定理となり,
		\begin{align}
			\rightharpoondown A \Longrightarrow\ \rightharpoondown A
		\end{align}
		も得られる.
	\end{prf}
	
	\begin{screen}
		\begin{metathm}[背理法の原理]
			$A$を$\mathcal{L}'$の閉式とするとき,
			$\rightharpoondown A \Longrightarrow \bot$が成り立つならば$A$は定理である.
		\end{metathm}
	\end{screen}
	
	\begin{prf}
		$\rightharpoondown A \Longrightarrow \bot$が成り立つとき,否定の導出より
		$\rightharpoondown \rightharpoondown A$が成り立つが,二重否定の法則より
		$A$も成立する.
		\QED
	\end{prf}
	
	\begin{itembox}[l]{ならばとまたは}
		$A \Longrightarrow B$と$\rightharpoondown A \vee B$は同値.
	\end{itembox}
	
	\begin{prf}
		$A \Longrightarrow B$が真であると仮定する.
		$(A \Longrightarrow B) \Longrightarrow (\ (A \vee \rightharpoondown A) \Longrightarrow (B \vee \rightharpoondown A)\ )$は定理であるから
		$(A \vee \rightharpoondown A) \Longrightarrow (B \vee \rightharpoondown A)$
		は定理となり,排中律より$A \vee \rightharpoondown A$は公理なので
		$B \vee \rightharpoondown A$は定理,よって$\rightharpoondown A \vee B$は定理である.
		以上で
		\begin{align}
			(A \Longrightarrow B) \Longrightarrow (\rightharpoondown A \vee B)
		\end{align}
		は定理である.逆に$\rightharpoondown A \vee B$が公理であると仮定する.
		このとき$A$を公理に追加すれば,$\bot$が定理となり$B$も定理となる.従って
		\begin{align}
			(\rightharpoondown A) \Longrightarrow (A \Longrightarrow B)
		\end{align}
		は定理となる.$B \Longrightarrow (A \Longrightarrow B)$も定理であるから,
		場合分けの法則より$A \Longrightarrow B$は定理となる.以上で
		\begin{align}
			(\rightharpoondown A \vee B) \Longrightarrow (A \Longrightarrow B)
		\end{align}
		は定理である.
	\end{prf}
	
	\begin{itembox}[l]{対偶命題は同値}
		$A \Longrightarrow B$と$\rightharpoondown B \Longrightarrow\ \rightharpoondown A$は同値.
	\end{itembox}
	
	\begin{prf}
		\begin{align}
			(A \Longrightarrow B) &\Longleftrightarrow (\rightharpoondown A \vee B) \\
			&\Longleftrightarrow (B \vee \rightharpoondown A) \\
			&\Longleftrightarrow (\rightharpoondown \rightharpoondown B \vee \rightharpoondown A) \\
			&\Longleftrightarrow (\rightharpoondown B \Longrightarrow\ \rightharpoondown A)
		\end{align}
	\end{prf}
	
	\begin{itembox}[l]{De Morganの法則}
		\begin{itemize}
			\item $\rightharpoondown (A \vee B) \Longleftrightarrow\ \rightharpoondown A \wedge \rightharpoondown B$.
			
			\item $\rightharpoondown (A \wedge B) \Longleftrightarrow\ \rightharpoondown A \vee \rightharpoondown B$.
		\end{itemize}
	\end{itembox}
	
	\begin{prf}
		$A \Longrightarrow A \vee B$より
		\begin{align}
			\rightharpoondown (A \vee B) \Longrightarrow\ \rightharpoondown A
		\end{align}
		同様に$\rightharpoondown (A \vee B) \Longrightarrow\ \rightharpoondown B$より
		\begin{align}
			\rightharpoondown (A \vee B) \Longrightarrow\ \rightharpoondown A \wedge \rightharpoondown B
		\end{align}
		また$A$かつ$\rightharpoondown A \wedge \rightharpoondown B$が成り立っていれば
		$\bot$より
		\begin{align}
			\rightharpoondown A \wedge \rightharpoondown B \Longrightarrow \bot
		\end{align}
		すなわち
		\begin{align}
			A \Longrightarrow\ \rightharpoondown(\rightharpoondown A \wedge \rightharpoondown B)
		\end{align}
		同様に
		\begin{align}
			B \Longrightarrow\ \rightharpoondown(\rightharpoondown A \wedge \rightharpoondown B)
		\end{align}
		なので
		\begin{align}
			(A \vee B) \Longrightarrow\ \rightharpoondown(\rightharpoondown A \wedge \rightharpoondown B)
		\end{align}
		対偶を取れば
		\begin{align}
			\rightharpoondown A \wedge \rightharpoondown B
			\Longrightarrow\ \rightharpoondown (A \vee B)
		\end{align}
	\end{prf}
	
	\begin{screen}
		$A$を式とし,$A$に現れる文字のうち$x$のみが自由変項であるとする.このとき以下は公理である:
		\begin{description}
			\item[$\varepsilon$記号の公理] $\varepsilon x A(x)$は対象である.
			\item[存在記号の公理] $A (\varepsilon x A(x)) \Longleftrightarrow \exists x A(x)$.
			\item[全称記号の公理] $A (\varepsilon x \rightharpoondown A(x)) \Longleftrightarrow \forall x A(x)$.
			\item[存在記号の公理] 対象$\tau$に対して
				\begin{align}
					A(\tau) \Longrightarrow \exists x A(x).
				\end{align}
		\end{description}
	\end{screen}
	
	\begin{screen}
		\begin{metathm}
			いかなる対象$\tau$に対しても
			\begin{align}
				\forall x A(x) \Longrightarrow A(\tau)
			\end{align}
			が成り立つ.逆に,いかなる対象$\tau$に対しても$A(\tau)$が成り立てば$\forall x A(x)$が成り立つ.
		\end{metathm}
	\end{screen}
	
	\begin{prf}
		$\tau$を$\mathcal{L}$の任意の対象とすれば公理より
		\begin{align}
			\rightharpoondown A(\tau) \Longrightarrow\ \rightharpoondown A
			\left( \varepsilon x \rightharpoondown A(x) \right)
		\end{align}
		が成り立つから,対偶を取って
		\begin{align}
			A \left( \varepsilon x \rightharpoondown A(x) \right)
			\Longrightarrow A(\tau)
		\end{align}
		を得る.公理より$A \left( \varepsilon x \rightharpoondown A(x) \right)$と
		$\forall x A(x)$は同値であるから
		\begin{align}
			\forall x A(x) \Longrightarrow A(\tau)
		\end{align}
		が出る.逆にいかなる対象$\tau$に対しても$A(\tau)$が成り立つとき,
		特に
		\begin{align}
			A \left( \varepsilon x \rightharpoondown A(x) \right)
		\end{align}
		が成り立つので$\forall x A(x)$も成り立つ.
		\QED
	\end{prf}
	
	\begin{screen}
		\begin{metathm}
			$A$と$B$を式とし,$A$,$B$に現れる自由変項は一つのみ.このとき以下は定理である:
			\begin{itemize}
				\item $\mathcal{L}$の任意の対象$\tau$に対して
					$A(\tau) \Longleftrightarrow B(\tau)$となるとき次が成り立つ:
					\begin{align}
						\exists x A(x) \Longleftrightarrow \exists x B(x).
					\end{align}
				
				\item $\mathcal{L}$の任意の対象$\tau$に対して
					$A(\tau) \Longleftrightarrow B(\tau)$となるとき次が成り立つ:
					\begin{align}
						\forall x A(x) \Longleftrightarrow \forall x B(x).
					\end{align}
					
				\item $\exists x \rightharpoondown A(x) \Longleftrightarrow\ \rightharpoondown \forall x A(x)$.
				
				\item $\forall x \rightharpoondown A(x) \Longleftrightarrow\ \rightharpoondown \exists x A(x)$.
				
				\item $\exists x ( A(x) \vee B(x) ) \Longleftrightarrow \exists x A(x) \vee \exists x B(x)$.
				
				\item $\forall x ( A(x) \wedge B(x) ) \Longleftrightarrow \forall x A(x) \wedge \forall x B(x)$.
			\end{itemize}
		\end{metathm}
	\end{screen}
	
	\begin{prf}\mbox{}
		\begin{description}
			\item[第一段]
				$\mathcal{L}$の任意の対象$\tau$に対して
				$A(\tau) \Longleftrightarrow B(\tau)$が満たされているとする.
				このとき$\exists x A(x)$が成り立っていると仮定すると,
				\begin{align}
					\tau \coloneqq \varepsilon x A(x)
				\end{align}
				とおけば公理より$A(\tau)$が成立するから$B(\tau)$も成立し,
				再び公理より$\exists x \rightharpoondown \rightharpoondown A(x)$が成り立つので
				\begin{align}
					\exists x A(x) \Longrightarrow \exists x B(x)
				\end{align}
				が得られる.同様にして$\exists x B(x) \Longrightarrow \exists x A(x)$も得られる.
				
			\item[第二段]
				$\mathcal{L}$の任意の対象$\tau$に対して
				$A(\tau) \Longleftrightarrow B(\tau)$が満たされているとする.
				$\forall x A(x)$が成り立っていると仮定するとき,
				$\tau$を$\mathcal{L}$の任意の対象とすれば$A(\tau)$が成り立つから$B(\tau)$も成立し,
				$\tau$の任意性より$\forall x B(x)$が成り立つ.よって
				\begin{align}
					\forall x A(x) \Longrightarrow \forall x B(x)
				\end{align}
				が得られる.同様にして$\forall x B(x) \Longrightarrow \forall x A(x)$も得られる.
				
			\item[第三段]
				公理より
				\begin{align}
					\exists x \rightharpoondown A(x) \Longleftrightarrow\ 
					\rightharpoondown A(\varepsilon x \rightharpoondown A(x))
				\end{align}
				は定理である.同様に公理より
				\begin{align}
					A(\varepsilon x \rightharpoondown A(x)) \Longleftrightarrow \forall x A(x) 
				\end{align}
				もまた定理であり,
				\begin{align}
					\rightharpoondown A(\varepsilon x \rightharpoondown A(x)) \Longleftrightarrow\ 
					\rightharpoondown \forall x A(x)
				\end{align}
				が定理となるので$\exists x \rightharpoondown A(x) \Longleftrightarrow\ \rightharpoondown \forall x A(x)$を得る.
			
			\item[第四段]
				前段の結果より
				\begin{align}
					\forall x \rightharpoondown A(x) \Longleftrightarrow
					\rightharpoondown \exists x \rightharpoondown \rightharpoondown A(x)
				\end{align}
				が成り立ち,また第一段の結果より
				\begin{align}
					\exists x \rightharpoondown \rightharpoondown A(x)
					\Longleftrightarrow \exists x A(x)
				\end{align}
				も成り立つから
				\begin{align}
					\forall x \rightharpoondown A(x) \Longleftrightarrow
					\rightharpoondown \exists x A(x)
				\end{align}
				が得られる.
			
			\item[第五段]
				いま$c(x) \overset{\mathrm{def}}{\Longleftrightarrow} A(x) \vee B(x)$とおけば,
				$\exists x ( A(x) \vee B(x) )$と$\exists x ( C(x) )$は同じ記号列であるから
		\begin{align}
			\exists x ( A(x) \vee B(x) ) \Longrightarrow \exists x C(x).
		\end{align}
		また公理より
		\begin{align}
			\exists x C(x) \Longrightarrow C(\varepsilon x C(x)).
		\end{align}
		$C(\varepsilon x C(x))$と$A(\varepsilon x C(x)) \vee B(\varepsilon x C(x))$は同じ記号列であるから
		\begin{align}
			C(\varepsilon x C(x)) \Longrightarrow A(\varepsilon x C(x)) \vee B(\varepsilon x C(x)).
		\end{align}
		ここで
		\begin{align}
			A(\varepsilon x C(x)) &\Longrightarrow A(\varepsilon x A(x)), \\
			A(\varepsilon x A(x)) &\Longrightarrow \exists x A(x), \\
			\exists x A(x) &\Longrightarrow \exists x A(x) \vee \exists x B(x)
		\end{align}
		かつ
		\begin{align}
			B(\varepsilon x C(x)) &\Longrightarrow B(\varepsilon x B(x)), \\
			B(\varepsilon x B(x)) &\Longrightarrow \exists x B(x), \\
			\exists x B(x) &\Longrightarrow \exists x A(x) \vee \exists x B(x).
		\end{align}
		以上と場合分けの原理を総合して
		\begin{align}
			\exists x ( A(x) \vee B(x) ) \Longrightarrow \exists x A(x) \vee \exists x B(x)
		\end{align}
		を得る.他方,公理より
		\begin{align}
			\exists x A(x) &\Longrightarrow A(\varepsilon x A(x)), \\
			A(\varepsilon x A(x)) &\Longrightarrow A(\varepsilon x A(x)) \vee B(\varepsilon x A(x)), \\
			A(\varepsilon x A(x)) \vee B(\varepsilon x A(x)) &\Longrightarrow C(\varepsilon x A(x)), \\
			C(\varepsilon x A(x)) &\Longrightarrow C(\varepsilon x C(x)), \\
			C(\varepsilon x C(x)) &\Longrightarrow \exists x C(x), \\
			\exists x C(x) &\Longrightarrow \exists x (A(x) \vee B(x))
		\end{align}
		同様に$\exists x A(x) \Longrightarrow \exists x (A(x) \vee B(x))$も成り立つので
		\begin{align}
			\exists x A(x) \vee \exists x B(x) \Longrightarrow \exists x (A(x) \vee B(x))
		\end{align}
		が成立する.
			
			\item[第六段]
				簡略して説明すれば
				\begin{align}
					\forall x \left( A(x) \wedge B(x) \right)
					&\Longleftrightarrow\ \rightharpoondown \exists x \rightharpoondown \left( A(x) \wedge B(x) \right) & (\mbox{第三段の結果の対偶}) \\
					&\Longleftrightarrow\ \rightharpoondown \exists x \left( \rightharpoondown A(x) \vee \rightharpoondown B(x) \right) & (\mbox{De Morganの法則}) \\
					&\Longleftrightarrow\ \rightharpoondown \left( \exists x \rightharpoondown A(x) \vee \exists x \rightharpoondown B(x) \right) & (\mbox{前段の結果}) \\
					&\Longleftrightarrow\ \rightharpoondown \left( \rightharpoondown \forall x A(x) \vee \rightharpoondown \forall x B(x) \right) & (\mbox{第三段の結果}) \\
					&\Longleftrightarrow\ \rightharpoondown \rightharpoondown \forall x A(x) \wedge \rightharpoondown \rightharpoondown \forall x B(x) & (\mbox{De Morganの法則}) \\
					&\Longleftrightarrow \forall x A(x) \wedge \forall x B(x) &(\mbox{二重否定の法則})
				\end{align}
				となる.
				\QED
		\end{description}
	\end{prf}
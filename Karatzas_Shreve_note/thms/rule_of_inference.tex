\subsection{推論規則メモ}
	$\mathscr{S}$は公理の全体.
	\begin{screen}
		\begin{schema}[公理図式群A]
			何が定理となるかを認める規則:
			\begin{itemize}
				\item $A$ならびに$A \Longrightarrow B$が定理なら$B$は定理である.
				\item $A$を$\mathscr{S}$に追加した下で$B$が定理であるなら,
					$\mathscr{S}$を公理とした下で$A \Longrightarrow B$は定理である.
				\item $A \Longrightarrow (A \vee B)$は定理である.
				\item $A \Longrightarrow (B \vee A)$は定理である.
				\item $A,B$が共に定理なら$A \wedge B$は定理である.
				\item $A \wedge B \Longrightarrow A$と
					$A \wedge B \Longrightarrow B$は定理である.
				\item $A \vee B,\ A \Longrightarrow C,\ B \Longrightarrow C$が全て定理であるとき$C$は定理である.
				\item 
			\end{itemize}	
		\end{schema}
	\end{screen}
	
	\begin{itembox}[l]{推移律}
		$A \Longrightarrow B$と$B \Longrightarrow C$が共に定理ならば
		$A \Longrightarrow C$は定理である.
	\end{itembox}
	
	\begin{prf}
		$A$を公理に追加する.このとき三段論法より$B$が定理となり,
		再び三段論法より$C$が定理となる.ゆえに$A \Longrightarrow C$は定理である.
	\end{prf}
	
	\begin{itembox}[l]{}
		$(A \Longrightarrow B) \Longrightarrow 
		(\ (A \vee C) \Longrightarrow (B \vee C)\ )$は定理である.
	\end{itembox}
	
	\begin{prf}
		いま$A \Longrightarrow B$を公理に追加する.
		このとき$A$を公理に追加すれば$B$は定理となり,
		$B \Longrightarrow (B \vee C)$が定理であるから$B \vee C$も定理となる.
		これより<<$A \Longrightarrow B$を公理とした下では>>
		\begin{align}
			A \Longrightarrow (B \vee C)
		\end{align}
		が定理となる.無条件で$C \Longrightarrow (B \vee C)$は定理であるから,
		よって<<$A \Longrightarrow B$と$A \vee C$を公理とした下では>>
		$B \vee C$が定理となる.以上より
		<<$A \Longrightarrow B$を公理とした下では>>
		\begin{align}
			(A \vee C) \Longrightarrow (B \vee C)
		\end{align}
		が定理となり
		\begin{align}
			(A \Longrightarrow B) \Longrightarrow 
			(\ (A \vee C) \Longrightarrow (B \vee C)\ )
		\end{align}
		が定理となる.
	\end{prf}
	
	\begin{itembox}[l]{ならばとまたは}
		$B \Longrightarrow (A \Longrightarrow B)$は定理である.
	\end{itembox}
	
	\begin{prf}
		$B$を公理に追加した場合,$A$を公理に追加しても$B$は真であるから
		$A \Longrightarrow B$は<$B$を公理に追加した場合の>定理となる.
		従って$B \Longrightarrow (A \Longrightarrow B)$は定理である.
	\end{prf}
	
	\begin{screen}
		\begin{schema}[公理図式群B]
			$\bot$に関する規則:
			\begin{itemize}
				\item $A$と$\rightharpoondown A$が共に定理なら$\bot$は定理となる.
				\item $\bot \Longrightarrow A$は定理である.
			\end{itemize}
		\end{schema}
	\end{screen}
	
	\begin{screen}
		\begin{schema}
			排中律と呼ばれる規則:
			\begin{itemize}
				\item $A \vee \rightharpoondown A$は定理である.
			\end{itemize}
		\end{schema}
	\end{screen}
	
	\begin{itembox}[l]{ならばとまたは}
		$A \Longrightarrow B$と$\rightharpoondown A \vee B$は同値.
	\end{itembox}
	
	\begin{prf}
		$A \Longrightarrow B$が真であると仮定する.
		$(A \Longrightarrow B) \Longrightarrow (\ (A \vee \rightharpoondown A) \Longrightarrow (B \vee \rightharpoondown A)\ )$は定理であるから
		$(A \vee \rightharpoondown A) \Longrightarrow (B \vee \rightharpoondown A)$
		は定理となり,排中律より$A \vee \rightharpoondown A$は公理なので
		$B \vee \rightharpoondown A$は定理,よって$\rightharpoondown A \vee B$は定理である.
		以上で
		\begin{align}
			(A \Longrightarrow B) \Longrightarrow (\rightharpoondown A \vee B)
		\end{align}
		は定理である.逆に$\rightharpoondown A \vee B$が公理であると仮定する.
		このとき$A$を公理に追加すれば,$\bot$が定理となり$B$も定理となる.従って
		\begin{align}
			(\rightharpoondown A) \Longrightarrow (A \Longrightarrow B)
		\end{align}
		は定理となる.$B \Longrightarrow (A \Longrightarrow B)$も定理であるから,
		場合分けの法則より$A \Longrightarrow B$は定理となる.以上で
		\begin{align}
			(\rightharpoondown A \vee B) \Longrightarrow (A \Longrightarrow B)
		\end{align}
		は定理である.
	\end{prf}
	
	\begin{screen}
		$A$を式,$A$において$x$のみ自由変項,このとき以下は公理である:
		\begin{description}
			\item[存在記号の公理] $A (\varepsilon x A(x)) \Longleftrightarrow \exists x A(x)$.
			\item[全称記号の公理] $A (\varepsilon x \rightharpoondown A(x)) \Longleftrightarrow \forall x A(x)$.
		\end{description}
	\end{screen}
\section{Ascoli-Arzelaの定理}
	\begin{screen}
		\begin{dfn}[正規族]
			$(X,d_X),(Y,d_Y)$を距離空間,
			$\mathscr{F} \subset C(X,Y)$とする.
			$\mathscr{F}$の任意の列$\{f_n\}_{n=1}^\infty$と
			$X$の任意のコンパクト部分集合$K$,及び任意の$\epsilon > 0$に対し或る$N \geq 1$が存在して
			\begin{align}
				n,m > N \quad \Longrightarrow \quad
				\sup{x \in K}{d_Y(f_n(x),f_m(x))} < \epsilon
			\end{align}
			となるとき,$\mathscr{F}$を正規族(normal family)という.
		\end{dfn}
	\end{screen}
	
	\begin{screen}
		\begin{dfn}[同程度連続]
			$(X,d_X),(Y,d_Y)$を距離空間,$\mathscr{F} \subset C(X,Y)$とする.
			任意の$\epsilon > 0$に対し或る$\delta > 0$が存在して
			\begin{align}
				d_X(p,q) < \delta \quad \Longrightarrow \quad
				\sup{f \in \mathscr{F}}{d_Y(f(p),f(q))} < \epsilon
			\end{align}
			となるとき,$\mathscr{F}$は同程度連続である(equicontinuous)という.
		\end{dfn}
	\end{screen}
	
	\begin{screen}
		\begin{thm}[Ascoli-Arzela]
			$(X,d_X)$を可分距離空間,$(Y,d_Y)$を距離空間,$E$を$X$で可算稠密な部分集合とする.
			\begin{description}
				\item[(1)] $\mathscr{F} \subset C(X,Y)$に対して次が成り立つ:
					\begin{align}
						\mbox{$\mathscr{F}$が正規族} \quad \Longleftarrow \quad
						\begin{cases}
							\mbox{$\mathscr{F}$が同程度連続,} & \\
							\mbox{各点$x \in E$で$\overline{\Set{f(x)}{f \in \mathscr{F}}}$がコンパクトである.} & 
						\end{cases}
					\end{align}
					
				\item[(2)] 
					$X$において
					\begin{align}
						K_1 \subset K_2 \subset K_3 \subset \cdots,
						\quad \bigcup_{n=1}^\infty \interior{K_n} = X
					\end{align}
					を満たすコンパクト部分集合の列$(K_n)_{n=1}^\infty$が存在するなら
					(1)は必要十分条件で成り立つ:
			\end{description}
		\end{thm}
	\end{screen}
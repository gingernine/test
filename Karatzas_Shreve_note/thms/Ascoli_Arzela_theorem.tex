\section{Ascoli-Arzelaの定理}
	\begin{screen}
		\begin{dfn}[正規族]
		\end{dfn}
	\end{screen}
	
	\begin{screen}
		\begin{thm}[Ascoli-Arzela]
			$(X,d)$を可分距離空間とし,$E$を$X$で可算稠密な部分集合とする.
			また$(S,\rho)$を距離空間,$\mathscr{F}$を$X$から$S$への連続写像の集合とする.
			このとき次の関係が成立する:
			\begin{align}
				\mbox{$\mathscr{F}$が正規族} \Longleftrightarrow
				\begin{cases}
					\mbox{$\mathscr{F}$が同程度連続,} & \\
					\mbox{各点$x \in E$で$\closure{\Set{f(x)}{f \in \mathscr{F}}}$がコンパクトである.} & 
				\end{cases}
			\end{align}
		\end{thm}
	\end{screen}
\section{半群}
	\begin{screen}
		\begin{dfn}[算法]
			$a$を類とするとき,$a \times a$から$a$への写像を$a$上の
			{\bf 算法}\index{さんぽう@算法}{\bf (operation)}と呼ぶ.
		\end{dfn}
	\end{screen}
	
	いま,$a$を類とし,$o$を$a$上の算法とする.
	\begin{description}
		\item[可換律\index{かかんりつ@可換律} (commutative law)] $\forall x,y \in a\, \left(\, o(x,y) = o(y,x)\, \right)$.
		\item[結合律\index{けつごうりつ@結合律} (associative law)] $\forall x,y,z \in a\, \left(\, o(o(x,y),z) = o(x,o(y,z))\, \right)$.
		\item[簡約律\index{かんやくりつ@簡約律} (cancellation law)] $\forall x,y,z \in a\, \left(\, o(x,z) = o(y,z) \Longrightarrow x = y\, \right)$.
	\end{description}
	
	$\ON$上の加法と乗法は$\ON$上の算法である.
	それらは結合律を満たす一方で可換律と簡約律は満たさないが,
	定義域を$\Natural \times \Natural$上に制限すれば全てを満たすようになる.ここで
	\begin{align}
		+_\Natural \defeq +|_{\Natural \times \Natural}
	\end{align}
	及び
	\begin{align}
		\cdot_\Natural \defeq \cdot|_{\Natural \times \Natural}
	\end{align}
	として$\Natural$上の加法と乗法を定義する.
	
	\begin{screen}
		\begin{dfn}[半群]
			$a$を集合とし,$o$を$a$上の算法とする.$o$が結合律を満たしているとき対$(a,o)$を
			{\bf 半群}\index{はんぐん@半群}{\bf (semi-group)}と呼ぶ.
			また$o$が結合律と可換律を満たすとき$(a,o)$を
			{\bf 可換半群}\index{かかんはんぐん@可換半群}{\bf (commutative semi-group)}と呼び,
			$o$が結合律と簡約律を満たすとき$(a,o)$を
			{\bf 簡約的半群}\index{かんやくてきはんぐん@簡約的半群}{\bf (cancellable semi-group)}と呼ぶ.
		\end{dfn}
	\end{screen}
	
	\begin{screen}
		\begin{thm}[$\omg$は加法に関して半群となる]
			$(\Natural,+_\Natural)$は簡約的可換半群である.
		\end{thm}
	\end{screen}
	
	\begin{sketch}
		
	\end{sketch}
\section{複素解析}

\subsection{正則関数}
	$\alpha$を複素数とするとき,$f$が$\alpha$で微分可能であるということを
	\begin{align}
		f \diffble \alpha \defarrow
		\exists a \in \C\, \forall \epsilon \in \R_+\, \exists \delta \in \R_+\,
		\forall z \in \dom{f}\, 
		\left(\, 0 < |z - \alpha| < \delta \Longrightarrow 
		\left| \frac{f(z) - f(\alpha)}{z-\alpha} - a\right| < \epsilon\, \right)
	\end{align}
	で定め,$f$が$\Omega$上の{\bf 正則関数}\index{せいそくかんすう@正則関数}{\bf (holomorphic function)}であるということを
	\begin{align}
		\hol_\Omega(f) \defarrow f:\Omega \longrightarrow \C 
		\wedge \forall \alpha \in \Omega\, \left(\, f \diffble \alpha\, \right)
	\end{align}
	で定める.また$\Omega$上の正則関数の全体を
	\begin{align}
		\Holomorphic{\Omega} \defeq \Set{f}{\hol_\Omega(f)}
	\end{align}
	と表す.ここで
	\begin{align}
		\operatorname{difl}_{f,\alpha}(a)
		\defarrow
		\forall \epsilon \in \R_+\, \exists \delta \in \R_+\,
		\forall z \in \dom{f}\, 
		\left(\, 0 < |z - \alpha| < \delta \Longrightarrow 
		\left| \frac{f(z) - f(\alpha)}{z-\alpha} - a\right| < \epsilon\, \right)
	\end{align}
	と略記しておく.
	
	\begin{screen}
		\begin{thm}[微係数の一意性]
			$f$を$\Holomorphic{\Omega}$の要素とし,$\alpha,\beta$を$\Omega$の要素とする.このとき
			\begin{align}
				\alpha = \beta \Longrightarrow 
				\varepsilon a \operatorname{difl}_{f,\alpha}(a)
				= \varepsilon a \operatorname{difl}_{f,\beta}(a)
			\end{align}
			が成り立つ.
		\end{thm}
	\end{screen}
	
	\begin{screen}
		\begin{dfn}[導関数]
			$f$を$\Holomorphic{\Omega}$の要素とするとき,
			\begin{align}
				f' \defeq \Set{x}{\exists \alpha \in \Omega\, 
				\left(\, x=(\alpha,\varepsilon a \operatorname{difl}_{f,\alpha}(a))\, \right)}
			\end{align}
			で定める写像を$f$の{\bf 導関数}\index{どうかんすう@導関数}{\bf (derivative function)}と呼ぶ.
		\end{dfn}
	\end{screen}
	
	\begin{screen}
		\begin{thm}[連鎖律]
			$f$を$\Holomorphic{\Omega}$の要素とし,$\Omega'$を
			\begin{align}
				f \ast \Omega \subset \Omega'
			\end{align}
			を満たす開集合とするとき,$g$を$H(\Omega')$の要素とすれば
			\begin{align}
				g \circ f \in \Holomorphic{\Omega}
			\end{align}
			が成立する.特に
			\begin{align}
				h \coloneqq g \circ f
			\end{align}
			とおけば
			\begin{align}
				\alpha \in \Omega \Longrightarrow h'(\alpha) = g'(f(\alpha)) \cdot f'(\alpha).
			\end{align}
		\end{thm}
	\end{screen}
	
	\begin{sketch}
		$\alpha$を$\Omega$の要素とし,$\epsilon$を正数とする.ここで
		\begin{align}
			\eta^2 + \left(|f'(\alpha)| + |g'(f(\alpha))| \right) \eta = \epsilon
		\end{align}
		を満たす正数$\eta$を取る.$\eta$に対し,
		\begin{align}
			|z-\alpha| < \delta_1 \Longrightarrow
			\left| (f(z) - f(\alpha)) - f'(\alpha)(z-\alpha) \right| < \eta |z-\alpha|
		\end{align}
		を満たす正数$\delta_1$と,
		\begin{align}
			|w-f(\alpha)| < \delta_2 \Longrightarrow
			\left| (g(w) - g(f(\alpha))) - g'(f(\alpha))(w-f(\alpha)) \right| < \eta |w-f(\alpha)|
		\end{align}
		を満たす正数$\delta_2$を取る.また$f$は$\alpha$で連続であるから
		\begin{align}
			|z-\alpha| < \delta_3 \Longrightarrow \left| f(z) - f(\alpha) \right| < \delta_2
		\end{align}
		を満たす正数$\delta_3$が取れる.このとき
		\begin{align}
			\delta \defeq \operatorname{min}\{\delta_1, \delta_3\}
		\end{align}
		とおけば,
		\begin{align}
			|z-\alpha| < \delta
		\end{align}
		なる$\Omega$の任意の要素$z$に対して
		\begin{align}
			\left| \left(g(f(z)) - g(f(\alpha))\right) - g'(f(\alpha))(f(z)-f(\alpha)) \right| 
			&< \eta |f(z)-f(\alpha)| \\
			&< \eta \left( \eta|z-\alpha| + |f'(\alpha)||z-\alpha| \right),
		\end{align}
		及び
		\begin{align}
			&\left| \left(g(f(z)) - g(f(\alpha))\right) - g'(f(\alpha))(f(z)-f(\alpha)) \right| \\
			&= \left| \left(g(f(z)) - g(f(\alpha))\right) - g'(f(\alpha))f'(\alpha)(z-\alpha)
			- g'(f(\alpha)) \left( (f(z) - f(\alpha)) - f'(\alpha)(z-\alpha) \right) \right|
		\end{align}
		から
		\begin{align}
			&\left| \left(g(f(z)) - g(f(\alpha))\right) - g'(f(\alpha))f'(\alpha)(z-\alpha) \right| \\
			&\leq \left| \left(g(f(z)) - g(f(\alpha))\right) - g'(f(\alpha))(f(z)-f(\alpha)) \right|
			+ \left| g'(f(\alpha)) \right| \left| (f(z) - f(\alpha)) - f'(\alpha)(z-\alpha) \right|
		\end{align}
		が成り立つので,
		\begin{align}
			\left| \left(g(f(z)) - g(f(\alpha))\right) - g'(f(\alpha))f'(\alpha)(z-\alpha) \right|
			< \left[ \eta^2 + \left(|f'(\alpha)| + |g'(f(\alpha))| \right) \eta \right] |z-\alpha|
		\end{align}
		が従う.ゆえに
		\begin{align}
			0 < |z-\alpha| < \delta
			\Longrightarrow \left| \frac{h(z) - h(\alpha)}{z-\alpha} - g'(f(\alpha)) \cdot f'(\alpha) \right| < \epsilon
		\end{align}
		が成り立つ.$\alpha$の任意性から
		\begin{align}
			h \in \Holomorphic{\Omega}
		\end{align}
		が従い,また
		\begin{align}
			\alpha \in \Omega \Longrightarrow h'(\alpha) = g'(f(\alpha)) \cdot f'(\alpha)
		\end{align}
		も示された.
		\QED
	\end{sketch}
	
\subsection{級数展開}
	$\alpha$を複素数とし,$r$を正数とするとき,中心$\alpha$半径$r$の円板を
	\begin{align}
		D(\alpha;r) \defeq \Set{z}{z \in \C \wedge |z - \alpha| < r}
	\end{align}
	で定める.また中心を抜いた円板を
	\begin{align}
		D'(\alpha;r) \defeq \Set{z}{z \in \C \wedge 0 < |z - \alpha| < r}
	\end{align}
	と定め,$D(\alpha;r)$の閉包は
	\begin{align}
		\overline{D}(\alpha;r)
	\end{align}
	と書く.
	
	\begin{screen}
		\begin{dfn}[解析関数]
			$f$を$\Omega$上の$\C$値関数とするとき,
			\begin{align}
				D(\alpha;r) \subset \Omega
			\end{align}
			なる円板の上で$f$が{\bf 収束級数展開可能である}{\bf (representable by a convergent series)}ということを
			\begin{align}
				&f \representable D(\alpha;r) \defarrow \\
				&\exists c\, \left(\, c:\Natural \longrightarrow \C \wedge
				\limsup_{n \to \infty} \sqrt[n]{c(n)} < \frac{1}{r} \wedge
				\forall z \in D(\alpha;r)\, \left(\, f(z) = \sum_{n=0}^\infty C(n)(z-\alpha)^n\, \right)\, \right)
			\end{align}
			で定める.また$f$が$\Omega$で{\bf 解析的である}\index{かいせきてき@解析的}{\bf (analytic)}ということを
			\begin{align}
				f \analytic \Omega \defarrow
				\forall \alpha \in \C\, \forall r \in \R_+\,
				\left(\, D(\alpha;r) \subset \Omega \Longrightarrow
				f \representable D(\alpha;r)\, \right)
			\end{align}
			で定義する.つまり,解析的とは局所的に収束級数展開可能であるということである.
			$\Omega$上の解析関数の全体を
			\begin{align}
				\Analytic{\Omega} \defarrow
				\Set{f}{f:\Omega \longrightarrow \C \wedge f \analytic \Omega}
			\end{align}
			で定める.
		\end{dfn}
	\end{screen}
	
	後述することであるが
	\begin{align}
		\Holomorphic{\Omega} = \Analytic{\Omega}
	\end{align}
	が成立する.
	
\subsection{複素積分}
	$\gamma$を$[\alpha,\beta]$から$\C$への区分的$C^1$関数,
	$f$を$X \defeq \ran{\gamma}$から$\C$への
	$\borel{X}/\borel{\C}$可測関数,
	$\lambda$を一次元Lebesgue測度とするとき,
	\begin{align}
		\mu(E) \defeq \int_E \gamma'\ d\lambda,
		\quad (E \in \borel{[\alpha,\beta]})
	\end{align}
	により$([\alpha,\beta],\borel{[\alpha,\beta]})$上に複素測度が定まる.このとき
	\begin{align}
		\mu \gamma^{-1}(A) \defeq \mu\left( \gamma^{-1}(A) \right),
		\quad (A \in \borel{X})
	\end{align}
	は$(X,\borel{X})$上の複素測度となり
	\begin{align}
		\int_X f\ d\mu \gamma^{-1} = \int_{[\alpha,\beta]} f(\gamma) \gamma'\ d\lambda
	\end{align}
	が成立する.$\gamma$に関する$f$の複素(線)積分を
	\begin{align}
		\int_\gamma f \defeq \int_{\ran{\gamma}} f\ d\mu \gamma^{-1}
	\end{align}
	で定義する.
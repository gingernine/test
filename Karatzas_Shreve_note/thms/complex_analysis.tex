\section{複素解析}

\subsection{正則関数}
	$\alpha$を複素数とするとき,$f$が$\alpha$で微分可能であるということを
	\begin{align}
		f \diffble \alpha \defarrow
		\exists a \in \C\, \forall \epsilon \in \R_+\, \exists \delta \in \R_+\,
		\forall z \in \dom{f}\, 
		\left(\, 0 < |z - \alpha| < \delta \Longrightarrow 
		\left| \frac{f(z) - f(\alpha)}{z-\alpha} - a\right| < \epsilon\, \right)
	\end{align}
	で定め,$f$が$\Omega$上の{\bf 正則関数}\index{せいそくかんすう@正則関数}{\bf (holomorphic function)}であるということを
	\begin{align}
		\hol_\Omega(f) \defarrow f:\Omega \longrightarrow \C 
		\wedge \forall \alpha \in \Omega\, \left(\, f \diffble \alpha\, \right)
	\end{align}
	で定める.また$\Omega$上の正則関数の全体を
	\begin{align}
		\Holomorphic{\Omega} \defeq \Set{f}{\hol_\Omega(f)}
	\end{align}
	と表す.ここで
	\begin{align}
		\operatorname{deriv}_{f,\alpha}(a)
		\defarrow
		\forall \epsilon \in \R_+\, \exists \delta \in \R_+\,
		\forall z \in \dom{f}\, 
		\left(\, 0 < |z - \alpha| < \delta \Longrightarrow 
		\left| \frac{f(z) - f(\alpha)}{z-\alpha} - a\right| < \epsilon\, \right)
	\end{align}
	と略記しておく.
	
	\begin{screen}
		\begin{thm}[微係数の一意性]
			$f$を$\Holomorphic{\Omega}$の要素とし,$\alpha$を$\Omega$の要素とする.このとき
			\begin{align}
				\forall a,b \in \C\, 
				\left(\, \operatorname{deriv}_{f,\alpha}(a) \wedge \operatorname{deriv}_{f,\alpha}(b)
				\Longrightarrow a = b\, \right)
			\end{align}
			が成り立つ.
		\end{thm}
	\end{screen}
	
	\begin{screen}
		\begin{dfn}[導関数]
			$f$を$\Holomorphic{\Omega}$の要素とするとき,
			\begin{align}
				f' \defeq \Set{x}{\exists \alpha \in \Omega\, 
				\exists a \in \C\, \left(\, x=(\alpha,a) \wedge \operatorname{deriv}_{f,\alpha}(a)\, \right)}
			\end{align}
			で定める写像を$f$の{\bf 導関数}\index{どうかんすう@導関数}{\bf (derivative function)}と呼ぶ.
		\end{dfn}
	\end{screen}
	
	\begin{screen}
		\begin{thm}[連鎖律]
			$f$を$\Holomorphic{\Omega}$の要素とし,$\Omega'$を
			\begin{align}
				f \ast \Omega \subset \Omega'
			\end{align}
			を満たす開集合とするとき,$g$を$H(\Omega')$の要素とすれば
			\begin{align}
				g \circ f \in \Holomorphic{\Omega}
			\end{align}
			が成立する.特に
			\begin{align}
				h \coloneqq g \circ f
			\end{align}
			とおけば
			\begin{align}
				\alpha \in \Omega \Longrightarrow h'(\alpha) = g'(f(\alpha)) \cdot f'(\alpha).
			\end{align}
		\end{thm}
	\end{screen}
	
	\begin{sketch}
		$\alpha$を$\Omega$の要素とし,$\epsilon$を正数とする.ここで
		\begin{align}
			\eta^2 + \left(|f'(\alpha)| + |g'(f(\alpha))| \right) \eta = \epsilon
		\end{align}
		を満たす正数$\eta$を取る.$\eta$に対し,
		\begin{align}
			|z-\alpha| < \delta_1 \Longrightarrow
			\left| (f(z) - f(\alpha)) - f'(\alpha)(z-\alpha) \right| < \eta |z-\alpha|
		\end{align}
		を満たす正数$\delta_1$と,
		\begin{align}
			|w-f(\alpha)| < \delta_2 \Longrightarrow
			\left| (g(w) - g(f(\alpha))) - g'(f(\alpha))(w-f(\alpha)) \right| < \eta |w-f(\alpha)|
		\end{align}
		を満たす正数$\delta_2$を取る.また$f$は$\alpha$で連続であるから
		\begin{align}
			|z-\alpha| < \delta_3 \Longrightarrow \left| f(z) - f(\alpha) \right| < \delta_2
		\end{align}
		を満たす正数$\delta_3$が取れる.このとき
		\begin{align}
			\delta \defeq \operatorname{min}\{\delta_1, \delta_3\}
		\end{align}
		とおけば,
		\begin{align}
			|z-\alpha| < \delta
		\end{align}
		なる$\Omega$の任意の要素$z$に対して
		\begin{align}
			\left| \left(g(f(z)) - g(f(\alpha))\right) - g'(f(\alpha))(f(z)-f(\alpha)) \right| 
			&< \eta |f(z)-f(\alpha)| \\
			&< \eta \left( \eta|z-\alpha| + |f'(\alpha)||z-\alpha| \right),
		\end{align}
		及び
		\begin{align}
			&\left| \left(g(f(z)) - g(f(\alpha))\right) - g'(f(\alpha))(f(z)-f(\alpha)) \right| \\
			&= \left| \left(g(f(z)) - g(f(\alpha))\right) - g'(f(\alpha))f'(\alpha)(z-\alpha)
			- g'(f(\alpha)) \left( (f(z) - f(\alpha)) - f'(\alpha)(z-\alpha) \right) \right|
		\end{align}
		から
		\begin{align}
			&\left| \left(g(f(z)) - g(f(\alpha))\right) - g'(f(\alpha))f'(\alpha)(z-\alpha) \right| \\
			&\leq \left| \left(g(f(z)) - g(f(\alpha))\right) - g'(f(\alpha))(f(z)-f(\alpha)) \right|
			+ \left| g'(f(\alpha)) \right| \left| (f(z) - f(\alpha)) - f'(\alpha)(z-\alpha) \right|
		\end{align}
		が成り立つので,
		\begin{align}
			\left| \left(g(f(z)) - g(f(\alpha))\right) - g'(f(\alpha))f'(\alpha)(z-\alpha) \right|
			< \left[ \eta^2 + \left(|f'(\alpha)| + |g'(f(\alpha))| \right) \eta \right] |z-\alpha|
		\end{align}
		が従う.ゆえに
		\begin{align}
			0 < |z-\alpha| < \delta
			\Longrightarrow \left| \frac{h(z) - h(\alpha)}{z-\alpha} - g'(f(\alpha)) \cdot f'(\alpha) \right| < \epsilon
		\end{align}
		が成り立つ.$\alpha$の任意性から
		\begin{align}
			h \in \Holomorphic{\Omega}
		\end{align}
		が従い,また
		\begin{align}
			\alpha \in \Omega \Longrightarrow h'(\alpha) = g'(f(\alpha)) \cdot f'(\alpha)
		\end{align}
		も示された.
		\QED
	\end{sketch}
	
\subsection{解析関数}
	\begin{screen}
		\begin{dfn}[計数測度]
			可測空間$(\Natural,\dirpro{\Natural})$において,
			\begin{align}
				\mu_c \defeq \Set{x}{\exists a\, \left(\, a \subset \Natural \wedge x = (a,\card{a})\, \right)}
			\end{align}
			により定める測度$\mu_c$を{\bf 計数測度}\index{けいすうそくど@計数測度}{\bf (counting measure)}と呼ぶ.
		\end{dfn}
	\end{screen}
	
	\begin{screen}
		\begin{dfn}[級数]
			$f$を$\Natural$上の$\C$値関数とする.
			\begin{align}
				\exists \alpha \in \C\, \left(\, 
				\sum_{k=0}^n f(k) \longrightarrow \alpha\, \right)
			\end{align}
			が成り立つとき,
			\begin{align}
				\sum_{n=0}^\infty f(n) \defeq \lim_{n \to \infty} \sum_{k=0}^n f(k)
			\end{align}
			と書いてこれを$f$の{\bf 級数}\index{きゅうすう@級数}{\bf (series)}と呼ぶ.また
			\begin{align}
				\exists \alpha \in \C\, \left(\, 
				\sum_{k=0}^n |f(k)| \longrightarrow \alpha\, \right)
			\end{align}
			が成り立つとき$f$の級数は{\bf 絶対収束する}\index{ぜったいしゅうそく@絶対収束}{\bf (absolutely converge)}という.
		\end{dfn}
	\end{screen}
	
	\begin{screen}
		\begin{thm}[級数は積分]
			$\Natural$上の$\C$値関数$f$に対して
			\begin{align}
				\sum_{n=0}^\infty |f(n)| < \infty
				\Longleftrightarrow \int_\Natural |f|\ d\mu_c < \infty
			\end{align}
			が成り立ち,かつこのとき
			\begin{align}
				\sum_{n=0}^\infty f(n) = \int_\Natural f\ d\mu_c
			\end{align}
			が成立する.
		\end{thm}
	\end{screen}
	
	$\alpha$を複素数とし,$r$を正数とするとき,中心$\alpha$半径$r$の円板を
	\begin{align}
		D(\alpha;r) \defeq \Set{z}{z \in \C \wedge |z - \alpha| < r}
	\end{align}
	で定める.また中心を抜いた円板を
	\begin{align}
		D'(\alpha;r) \defeq \Set{z}{z \in \C \wedge 0 < |z - \alpha| < r}
	\end{align}
	と定め,$D(\alpha;r)$の閉包は
	\begin{align}
		\overline{D}(\alpha;r)
	\end{align}
	と書く.
	
	\begin{screen}
		\begin{dfn}[解析関数]
			$f$を$\Omega$上の$\C$値関数とするとき,
			\begin{align}
				D(\alpha;r) \subset \Omega
			\end{align}
			なる円板の上で$f$が{\bf 収束級数展開可能である}{\bf (representable by a convergent series)}ということを
			\begin{align}
				&f \representable D(\alpha;r) \defarrow \\
				&\exists c\, \left(\, c:\Natural \longrightarrow \C \wedge
				\limsup_{n \to \infty} \sqrt[n]{c(n)} < \frac{1}{r} \wedge
				\forall z \in D(\alpha;r)\, \left(\, f(z) = \sum_{n=0}^\infty C(n)(z-\alpha)^n\, \right)\, \right)
			\end{align}
			で定める.また$f$が$\Omega$で{\bf 解析的である}\index{かいせきてき@解析的}{\bf (analytic)}ということを
			\begin{align}
				f \analytic \Omega \defarrow
				\forall \alpha \in \C\, \forall r \in \R_+\,
				\left(\, D(\alpha;r) \subset \Omega \Longrightarrow
				f \representable D(\alpha;r)\, \right)
			\end{align}
			で定義する.つまり,解析的とは局所的に収束級数展開可能であるということである.
			$\Omega$上の解析関数の全体を
			\begin{align}
				\Analytic{\Omega} \defeq
				\Set{f}{f:\Omega \longrightarrow \C \wedge f \analytic \Omega}
			\end{align}
			で定める.
		\end{dfn}
	\end{screen}
	
	後述することであるが
	\begin{align}
		\Holomorphic{\Omega} = \Analytic{\Omega}
	\end{align}
	が成立する.解析関数は,後述できないかもしれないWeierstrass解析関数とは別物である.
	
\subsection{複素積分}
	$\gamma$を$[\alpha,\beta]$から$\C$への関数で,右連続かつ有界変動であるとする.
	このとき$\gamma$を$[\alpha,\beta]$上の{\bf 路}\index{ろ@路}
	{\bf (contour)}と呼ぶ.$\gamma$から作る複素Stieltjes測度を
	\begin{align}
		\mu_\gamma
	\end{align}
	とする.$\gamma$は$[\alpha,\beta]$から$\C$への可測関数であるから
	\begin{align}
		\tilde{\mu}_\gamma(A) \defeq \mu_\gamma\left( \gamma^{-1}(A) \right),
		\quad (A \in \borel{\C})
	\end{align}
	で定める$\tilde{\mu}_\gamma$は$(\C,\borel{\C})$上の複素測度となり,このとき
	$\Omega$上の関数$f$を$|\tilde{\mu}_\gamma|$に関する可積分関数とすれば
	\begin{align}
		\int_\Omega f\ d\tilde{\mu}_\gamma = \int_{[\alpha,\beta]} f(\gamma)\ d\mu_\gamma
	\end{align}
	が成立する.特に$\gamma$が$[\alpha,\beta]$上で絶対連続なら,
	$\lambda$を一次元Lebesgue測度とすれば
	\begin{align}
		\mu_\gamma(E) = \int_E \gamma'\ d\lambda,
		\quad (\forall E \in \borel{[\alpha,\beta]})
	\end{align}
	となるので
	\begin{align}
		\int_{[\alpha,\beta]} f(\gamma)\ d\mu_\gamma = \int_{[\alpha,\beta]} f(\gamma) \cdot \gamma'\ d\lambda
	\end{align}
	が成立する.
	$\gamma$に関する$f$の複素(線)積分を
	\begin{align}
		\int_\gamma f \defeq \int_\Omega f\ d\tilde{\mu}_\gamma
	\end{align}
	で定義する.
	
	$\gamma$は$[\alpha,\beta]$上の写像であったが,$\gamma$に関する線積分はパラメータ変項で不変であることを示す.
	\begin{align}
		\varphi:[\alpha_1,\beta_1] \longrightarrow [\alpha,\beta]
	\end{align}
	で
	\begin{align}
		x \in [\alpha_1,\beta_1] \Longrightarrow \varphi(x) = \alpha + \frac{\beta-\alpha}{\beta_1-\alpha_1}(x-\alpha_1)
	\end{align}
	なる$\varphi$により
	\begin{align}
		\gamma_1 \coloneqq \gamma \circ \varphi
	\end{align}
	と定めれば,
	\begin{align}
		\int_\gamma f = \int_{\gamma_1} f
	\end{align}
	が成立する.これは
	\begin{align}
		\tilde{\mu}_\gamma = \tilde{\mu}_{\gamma_1}
	\end{align}
	が成り立つことを示せば良いが,
	\begin{align}
		\tilde{\mu}_\gamma = \mu_\gamma \circ \gamma^{-1},
		\quad \tilde{\mu}_{\gamma_1} = \mu_{\gamma_1} \circ \varphi^{-1} \circ \gamma^{-1} 
	\end{align}
	であり,$\gamma$は$[\alpha,\beta]$上のBorel可測関数であるから,
	\begin{align}
		\forall E \in \borel{[\alpha,\beta]}\, 
		\left(\, \mu_{\gamma_1} \circ \varphi^{-1}(E) = \mu_\gamma(E)\, \right)
	\end{align}
	を示せば良い.$E$が$(s,t]$なる形の場合は
	\begin{align}
		\mu_{\gamma_1} \circ \varphi^{-1}((s,t])
		&= \mu_{\gamma_1} \left(\left(\varphi^{-1}(s),\varphi^{-1}(t)\right]\right) \\
		&= \gamma_1\left( \varphi^{-1}(t) \right) - \gamma_1\left( \varphi^{-1}(s) \right) \\
		&= \gamma(t) - \gamma(s) \\
		&= \mu_\gamma((s,t])
	\end{align}
	が成り立つ.$E = \{\alpha\}$なら
	\begin{align}
		\mu_{\gamma_1} \circ \varphi^{-1}(\{\alpha\})
		= \mu_{\gamma_1}(\{\alpha_1\})
		= 0
		= \mu_\gamma(\{\alpha\})
	\end{align}
	が成り立つ.一致の定理より
	\begin{align}
		\mu_{\gamma_1} \circ \varphi^{-1} = \mu_\gamma
	\end{align}
	となる.
	
	次に逆向きの路に関する積分を考える.$\gamma$が連続であるとき,
	\begin{align}
		\gamma_2 : [\alpha,\beta] \longrightarrow \C
	\end{align}
	で
	\begin{align}
		t \in [\alpha, \beta] \Longrightarrow \gamma_2(t) = \gamma(\alpha + \beta - t) 
	\end{align}
	を満たす$\gamma_2$は有界変動かつ連続となるので,複素Stieltjes測度を構成できる.
	$\gamma_2$を$\gamma$の{\bf 逆路}\index{ぎゃくろ@逆路}{\bf (inverse contour)}と呼ぶ.
	このとき
	\begin{align}
		\int_{\gamma_2} f = - \int_\gamma f
	\end{align}
	が成立する.これは
	\begin{align}
		\tilde{\mu}_{\gamma_2} = - \tilde{\mu}_\gamma 
	\end{align}
	を示せば良いが,
	\begin{align}
		\psi(t) = \alpha + \beta - t
	\end{align}
	なる$\psi:[\alpha,\beta] \longrightarrow [\alpha,\beta]$を定めれば
	\begin{align}
		\mu_{\gamma_2} \gamma_2^{-1} = \mu_{\gamma_2} \circ \psi^{-1} \circ \gamma^{-1}
	\end{align}
	となるので
	\begin{align}
		\forall E \in \borel{[\alpha,\beta]}\,
		\left(\, \mu_{\gamma_2}(\psi^{-1} \ast E) = -\mu_\gamma(E)\, \right)
	\end{align}
	を示せば十分である.$E$が$(s,t)$なる形の場合
	\begin{align}
		\mu_{\gamma_2}\left(\psi^{-1} \ast (s,t)\right)
		&= \mu_{\gamma_2}\left( \left( \psi^{-1}(t),\psi^{-1}(s) \right) \right) \\
		&= \gamma_2\left(\psi^{-1}(s)\right) - \gamma_2\left(\psi^{-1}(t)\right) \\
		&= \gamma(s) - \gamma(t) \\
		&= -\mu_\gamma((s,t))
	\end{align}
	が成立する.$\mu_\gamma$も$\mu_{\gamma_2}$も一点の測度は$0$であるから
	\begin{align}
		\mu_{\gamma_2} \circ \psi^{-1} = -\mu_\gamma
	\end{align}
	が従う.
	
	\begin{screen}
		\begin{thm}[正則関数に対する微積分学の基本定理]
			$f$を$H(\Omega)$の要素とし,$f'$が連続であるとし,$\gamma$を$[\alpha,\beta]$から$\Omega$への絶対連続な路とする.このとき
			\begin{align}
				\int_{\gamma} f' = f(\gamma(\beta)) - f(\gamma(\alpha))
			\end{align}
			が成立する.特に$\gamma$が閉路なら積分値は$0$である.
		\end{thm}
	\end{screen}
	
	\begin{prf}
		微積分学の基本定理より
		\begin{align}
			\int_{\gamma} f'
			= \int_{[\alpha,\beta]} f'(\gamma(t)) \cdot \gamma'(t)\ dt
			= f(\gamma(\beta)) - f(\gamma(\alpha))
		\end{align}
		となる.
		\QED
	\end{prf}
\section{複素測度に関する積分}
	\begin{screen}
		\begin{thm}[複素測度の極分解]\label{thm:polar_decomposition_of_complex_measures}
			可測空間$(X,\mathscr{F})$上の任意の複素測度$\mu$に対し,次の意味での極分解
			\begin{align}
				\quad \mu(E) = \int_E e^{i\theta}\ d|\mu|,
				\quad (\forall E \in \mathscr{F})
			\end{align}
			を満たす$\mathscr{F}/\borel{\C}$-可測関数$\theta$が存在する.
			$\lambda \not\equiv 0$なら$e^{i \theta}$は
			$L^1(|\mu|)$の元として唯一つに決まる.
		\end{thm}
	\end{screen}
	
	\begin{prf} $\mu \equiv 0$なら$|\mu| \equiv 0$より$\theta \equiv \pi$でよい.
		$\mu \not\equiv 0$の場合,
		Lebesgue-Radon-Nikodymの定理より
		\begin{align}
			\mu(E) = \int_E h\ d|\mu|,
			\quad (\forall E \in \mathscr{F})
		\end{align}
		を満たす$[h] \in L^1(|\mu|)$が唯一つ存在する.このとき$|\mu|(E) > 0$なら
		\begin{align}
			\frac{1}{|\mu|(E)} \left| \int_E h\ d|\mu| \right|
			= \frac{|\mu(E)|}{|\mu|(E)} \leq 1
		\end{align}
		となるから,定理\refeq{thm:mean_value_of_integral_and_closed_set}より
		$|\mu|$-a.e.に$|h| \leq 1$となる.また
		\begin{align}
			E_r \coloneqq \{|h| \leq r\}
		\end{align}
		とおき$\{A_n\}_{n=1}^\infty \subset \mathscr{F}$を$E_r$の任意の分割とすれば,
		\begin{align}
			\sum_{n=1}^\infty |\mu(A_n)|
			= \sum_{n=1}^\infty \left|\int_{A_n} h\ d|\mu|\right|
			\leq \sum_{n=1}^\infty \int_{A_n} |h|\ d|\mu|
			\leq r \sum_{n=1}^\infty |\mu|(A_n)
			= r |\mu|(E_r)
		\end{align}
		が成り立つから$r < 1$なら$|\mu|(E_r) = 0$となり
		\begin{align}
			|\mu|\left(|h|< 1 \right)
			= |\mu| \Biggl(\bigcap_{n=1}^\infty E_{1-1/n} \Biggr)
			= 0
		\end{align}
		が従う.よって$|\mu|$-a.e.に$|h|=1$となる.ここで
		\begin{align}
			\theta(x) \coloneqq
			\begin{cases}
				0, & h(x) = 1, \\
				\pi, & h(x) \neq 1
			\end{cases}
		\end{align}
		と定めれば$[h] = [e^{i \theta}]$が成立する.
		\QED
	\end{prf}
	
	\begin{screen}
		\begin{dfn}[複素測度に関する積分]
			$(X,\mathscr{F})$を可測空間,$\mu$を$(X,\mathscr{F})$上の複素測度,
			$f$を$\mathscr{F}/\borel{\C}$-可測関数とする.
			$f$が$|\mu|$-可積分であるとき,極分解$d\mu = e^{i\theta}\ d|\mu|$を用いて
			\begin{align}
				\int_X f\ d\mu \coloneqq \int_X f e^{i \theta}\ d|\mu|
			\end{align}
			により$f$の$\mu$に関する積分を定める.
		\end{dfn}
	\end{screen}
	
	$\mu \not\equiv 0$なら極分解は定理\ref{thm:polar_decomposition_of_complex_measures}
	の意味で一意であるから$\mu$に関する積分はwell-definedである.
	$\mu \equiv 0$なら$|\mu| \equiv 0$であるから任意の可測写像は$|\mu|$について可積分となり,
	$\mu$に関する積分値は0で確定する(well-defined).
	
	\begin{screen}
		\begin{thm}[総変動測度の積分表現]
			$(X,\mathscr{F},\mu)$を正値測度空間,
			$f$を$\mathscr{F}/\borel{\C}$-可測な$\mu$-可積分関数とするとき,
			\begin{align}
				\lambda(E) \coloneqq \int_E f\ d\mu, \quad (\forall E \in \mathscr{F})
			\end{align}
			で複素測度$\lambda$を定めれば次が成り立つ:
			\begin{align}
				|\lambda|(E) = \int_E |f|\ d\mu, \quad (\forall E \in \mathscr{F}).
			\end{align}
		\end{thm}
	\end{screen}
	
	\begin{screen}
		\begin{thm}[積分の測度に関する線型性]\label{thm:linearity_of_integral_respect_to_complex_measure}
			$(X,\mathscr{F})$を可測空間,$\mu,\nu$をこの上の複素測度とする.$f:X \rightarrow \C$が$|\mu|$と$|\nu|$について可積分であるなら,
			$\alpha,\beta \in \C$に対し$|\alpha \mu + \beta \nu|$についても可積分であり,更に次が成り立つ:
			\begin{align}
				\int_X f\ d(\alpha\mu + \beta\nu) = \alpha \int_X f\ d\mu + \beta \int_X f\ d\nu.
			\end{align}
		\end{thm}
	\end{screen}
	
	\begin{prf}
		\begin{description}
			\item[第一段]
				$f$が可測単関数の場合について証明する.
				$a_i \in \C,\ A_i \in \mathcal{M}\ (i=1,\cdots,n,\ \sum_{i=1}^{n} A_i = X)$を用いて
				\begin{align}
					f = \sum_{i=1}^{n} a_i \defunc_{A_i}
				\end{align}
				と表されている場合,
				\begin{align}
					&\int_X f(x)\ (\alpha\mu + \beta\nu)(dx)
					= \sum_{i=1}^{n} a_i (\alpha\mu + \beta\nu)(A_i) \\
					&\qquad = \alpha \sum_{i=1}^{n} a_i \mu(A_i) + \beta \sum_{i=1}^{n} a_i \nu(A_i)
					= \alpha \int_X f(x)\ \mu(dx) + \beta \int_X f(x)\ \nu(dx)
				\end{align}
				が成り立つ.
				
			\item[第二段]
			$f$が一般の可測関数の場合について証明する.任意の$A \in \mathcal{M}$に対して
			\begin{align}
				\left| (\alpha \mu + \beta \nu)(A) \right| \leq |\alpha||\mu(A)| + |\beta||\nu(A)| \leq |\alpha||\mu|(A) + |\beta||\nu|(A)
 			\end{align}
 			が成り立つから,左辺で$A$を任意に分割しても右辺との大小関係は変わらず
 			\begin{align}
 				|\alpha \mu + \beta \nu|(A) \leq |\alpha||\mu|(A) + |\beta||\nu|(A)
 			\end{align}
 			となる.従って$f$が$|\mu|$と$|\nu|$について可積分であるなら
 			\begin{align}
 				\int_X |f(x)|\ |\alpha \mu + \beta \nu|(dx) \leq |\alpha| \int_X |f(x)|\ |\mu|(dx) + |\beta| \int_X |f(x)|\ |\nu|(dx) < \infty
 			\end{align}
 			が成り立ち前半の主張を得る.$f$の単関数近似列$(f_n)_{n=1}^{\infty}$を取れば,前段の結果と積分の定義より
 			\begin{align}
 				&\left| \int_X f(x)\ (\alpha\mu + \beta\nu)(dx) - \alpha \int_X f(x)\ \mu(dx) - \beta \int_X f(x)\ \nu(dx) \right| \\
 					&\qquad \leq \left| \int_X f(x)\ (\alpha\mu + \beta\nu)(dx) - \int_X f_n(x)\ (\alpha\mu + \beta\nu)(dx) \right| \\
 					&\qquad \quad + |\alpha| \left| \int_X f(x)\ \mu(dx) - \int_X f_n(x)\ \mu(dx) \right|
 					+ |\beta| \left| \int_X f(x)\ \nu(dx) - \int_X f_n(x)\ \nu(dx) \right| \\
 				&\qquad \longrightarrow 0 \quad (n \longrightarrow \infty)
 			\end{align}
 			が成り立ち後半の主張が従う.
 			\QED
		\end{description}
	\end{prf}
	
	\begin{screen}
		\begin{thm}[積分の複素共役]
			$(X,\mathscr{F})$を可測空間,$\mu$を複素測度,
			$f:X \rightarrow \C$を$|\mu|$について可積分な$\mathscr{F}/\borel{\C}$-可測関数とするとき
			次が成り立つ:
			\begin{align}
				\int_X f\ d\overline{\mu}
				= \overline{\int_X \overline{f}\ d\mu}.
			\end{align}
		\end{thm}
	\end{screen}
	
	\begin{prf}
		$u = \Re{f},\ v = \Im{f},\ \gamma = \Re{\mu},\ \theta = \Im{\mu}$とすれば,
		定理\refeq{thm:linearity_of_integral_respect_to_complex_measure}より
		\begin{align}
			\int_X f\ d\overline{\mu} &= \int_X f\ d\gamma - i \int_X f\ d\theta \\
			&= \int_X u\ d\gamma + i \int_X v\ d\gamma - i \int_X u\ d\theta + \int_X v\ d\theta \\
			&= \overline{\int_X u\ d\gamma - i \int_X v\ d\gamma + i \int_X u\ d\theta + \int_X v\ d\theta} \\
			&= \overline{\int_X \overline{f}\ d\gamma + i \int_X \overline{f}\ d\theta} \\
			&= \overline{\int_X \overline{f}\ d\mu}
		\end{align}
		が成立する.
		\QED
	\end{prf}
	
	\begin{screen}
		\begin{thm}[Rieszの表現定理(複素測度)]
		\end{thm}
	\end{screen}
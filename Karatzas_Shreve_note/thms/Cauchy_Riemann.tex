\subsection{Cauchy-Riemann方程式}
	\begin{screen}
		\begin{dfn}[全微分]\label{def:totally_differentiability}
			$D$を$\R^2$の開集合とし,$(a,b)$を$D$の要素とし,$u$を$D$上の実数値関数とする.このとき
			\begin{align}
				&\forall \epsilon \in \R_+\, \exists \delta \in \R_+\, \forall h,k \in \R\, \\
				&\sqrt{h^2 + k^2} < \delta \Longrightarrow
				\left|u(a+h,b+k) - u(a,b) - \alpha \cdot h - \beta \cdot k\right| 
				< \epsilon \cdot \sqrt{h^2 + k^2}
				\label{fom:def_totally_differentiability}
			\end{align}
			を満たす実数$\alpha$と$\beta$が取れるなら,
			$u$は$(a,b)$において{\bf 全微分可能である}\index{ぜんびぶんかのう@全微分可能}{\bf (totally differentiable)}という.
		\end{dfn}
	\end{screen}
	
	(\refeq{fom:def_totally_differentiability})の$\alpha$と$\beta$が一意に定まることを示す.
	定義\ref{def:totally_differentiability}の設定の下で
	\begin{align}
		&\forall \epsilon \in \R_+\, \exists \delta \in \R_+\, \forall h,k \in \R\, \\
		&\sqrt{h^2 + k^2} < \delta \Longrightarrow
		\left|u(a+h,b+k) - u(a,b) - \gamma \cdot h - \zeta \cdot k\right| 
		< \epsilon \cdot \sqrt{h^2 + k^2}
	\end{align}
	を満たす実数$\gamma$と$\zeta$が取れる場合,$\epsilon$を任意に与えられた正の実数とすると
	\begin{align}
		\forall h,k \in \R\,
		\left[\, \sqrt{h^2 + k^2} < \delta \Longrightarrow
		\left|u(a+h,b+k) - u(a,b) - \alpha \cdot h - \beta \cdot k\right| 
		< \epsilon \cdot \sqrt{h^2 + k^2}\, \right]
	\end{align}
	かつ
	\begin{align}
		\forall h,k \in \R\,
		\left[\, \sqrt{h^2 + k^2} < \delta \Longrightarrow
		\left|u(a+h,b+k) - u(a,b) - \gamma \cdot h - \zeta \cdot k\right| 
		< \epsilon \cdot \sqrt{h^2 + k^2}\, \right]
	\end{align}
	を満たす正の実数$\delta$が取れるが,このとき
	\begin{align}
		|h| < \delta
	\end{align}
	ならば
	\begin{align}
		\left|u(a+h,b) - u(a,b) - \alpha \cdot h\right| < \epsilon \cdot |h|
		\label{fom:totally_differentiable_then_partially_differentiable}
	\end{align}
	かつ
	\begin{align}
		\left|u(a+h,b) - u(a,b) - \gamma \cdot h\right| < \epsilon \cdot |h|
	\end{align}
	が成り立つので
	\begin{align}
		\left|\alpha \cdot h - \gamma \cdot h\right| < \epsilon \cdot |h|
	\end{align}
	が従う.すなわち
	\begin{align}
		\left|\alpha - \gamma\right| < \epsilon
	\end{align}
	が従う.そして$\epsilon$の任意性より
	\begin{align}
		\alpha = \gamma
	\end{align}
	が出る.同様にして
	\begin{align}
		\beta = \zeta
	\end{align}
	も得られる.また(\refeq{fom:totally_differentiable_then_partially_differentiable})は
	$u$が第一変数に関して微分可能であることを示している.これについて
	
	\begin{screen}
		\begin{dfn}[偏微分]\label{def:partially_differentiability}
			$D$を$\R^2$の開集合とし,$(a,b)$を$D$の要素とし,$u$を$D$上の実数値関数とする.このとき
			\begin{align}
				&\forall \epsilon \in \R_+\, \exists \delta \in \R_+\, \forall h \in \R\, \\
				&|h| < \delta \Longrightarrow
				\left|u(a+h,b) - u(a,b) - \alpha \cdot h\right| < \epsilon \cdot |h|
			\end{align}
			を満たす実数$\alpha$が取れるなら,$u$は$(a,b)$において第一変数に関して
			{\bf 偏微分可能である}\index{へんびぶんかのう@偏微分可能}{\bf (partially differentiable)}という.
			同様に
			\begin{align}
				&\forall \epsilon \in \R_+\, \exists \delta \in \R_+\, \forall k \in \R\, \\
				&|k| < \delta \Longrightarrow
				\left|u(a,b+k) - u(a,b) - \beta \cdot k\right| < \epsilon \cdot |k|
			\end{align}
			を満たす実数$\beta$が取れるなら,$u$は$(a,b)$において第二変数に関して偏微分可能であるという.
		\end{dfn}
	\end{screen}
	
	以上をまとめると
	
	\begin{screen}
		\begin{thm}[全微分可能なら偏微分可能]
			$D$を$\R^2$の開集合とし,$(a,b)$を$D$の要素とし,
			$u$を$D$上の実数値関数とする.$u$が$(a,b)$で全微分可能であるとき,
			$u$は第一変数と第二変数に関して$(a,b)$で偏微分可能であって,そして
			\begin{align}
				\partial_1 u(a,b) \defeq \lim_{h \to 0} \frac{u(a+h,b) - u(a,b)}{h}
			\end{align}
			および
			\begin{align}
				\partial_2 u(a,b) \defeq \lim_{k \to 0} \frac{u(a,b+k) - u(a,b)}{k}
			\end{align}
			とおけば
			\begin{align}
				&\forall \epsilon \in \R_+\, \exists \delta \in \R_+\, \forall h,k \in \R\, \\
				&\sqrt{h^2 + k^2} < \delta \Longrightarrow
				\left|u(a+h,b+k) - u(a,b) - \partial_1 u(a,b) \cdot h - \partial_2 u(a,b) \cdot k\right| 
				< \epsilon \cdot \sqrt{h^2 + k^2}
			\end{align}
			が成立する.
		\end{thm}
	\end{screen}
	
	いま,$\Omega$を$\C$の空でない開集合とし,$f$を$\Omega$上の複素数値関数とし,$\zeta$を$\Omega$の要素とし,$a$と$b$を
	\begin{align}
		\zeta = a + \isym \cdot b
	\end{align}
	を満たす実数とする.ここで
	\begin{align}
		\R^2 \ni (x,y) \longmapsto x + \isym \cdot y
	\end{align}
	なる関係により定める写像を$\varphi$とすると,$\varphi$は$\R^2$から$\C$への同相写像であるから
	\begin{align}
		D \defeq \varphi^{-1} \ast \Omega
	\end{align}
	で定める$D$は$\R^2$の開集合である.また
	\begin{align}
		D \ni (x,y) \longmapsto \Re{f(x + \isym \cdot y)}
	\end{align}
	なる写像を$u$とし,
	\begin{align}
		D \ni (x,y) \longmapsto \Im{f(x + \isym \cdot y)}
	\end{align}
	なる写像を$v$とすると,その定め方より$\Omega$の任意の要素$z$に対して
	\begin{align}
		f(z) = u(\varphi^{-1}(z)) + \isym \cdot v(\varphi^{-1}(z))
	\end{align}
	が成立する.同じことであるが,$D$の任意の要素$(x,y)$に対して
	\begin{align}
		f(x + \isym \cdot y) = u(x,y) + \isym \cdot v(x,y)
	\end{align}
	も成立する.
	
	この設定の下で,$u$と$v$が$(a,b)$で全微分可能であるとき
	\begin{align}
		\partial_1 u(a,b) \defeq \lim_{h \to 0} \frac{u(a+h,b) - u(a,b)}{h}
	\end{align}
	および
	\begin{align}
		\partial_2 u(a,b) \defeq \lim_{k \to 0} \frac{u(a,b+k) - u(a,b)}{k}
	\end{align}
	と定めて,同様に$\partial_1 v(a,b)$と$\partial_2 v(a,b)$を定める.このとき
	\begin{align}
		\partial_1 u(a,b) + \isym \cdot \partial_1 v(a,b)
		= \partial_2 v(a,b) - \isym \cdot \partial_2 u(a,b)
	\end{align}
	を{\bf Cauchy-Riemann方程式}と呼ぶ.
	
	\begin{screen}
		\begin{thm}[微分可能であることとCauchy-Riemann方程式]
			記号は全て上で設定したものとする.
			$f$が$\zeta$で微分可能であることと,
			$u$と$v$が共に$(a,b)$で全微分可能であって
			かつCauchy-Riemann方程式が成り立つことは同値である.また$f$が$\zeta$で微分可能であるとき,
			その微分係数を$f'(\zeta)$と書けば
			\begin{align}
				f'(\zeta) &= \partial_1 u(a,b) + \isym \cdot \partial_1 v(a,b) \\
				&= \partial_2 v(a,b) - \isym \cdot \partial_2 u(a,b).
			\end{align}
		\end{thm}
	\end{screen}
	
	\begin{sketch}\mbox{}
		\begin{description}
			\item[第一段]
				$f$が$\zeta$で微分可能であるとする.すると
				\begin{align}
					&\forall \epsilon \in \R_+\, \exists \delta \in \R_+\, \forall h \in \C\, \\
					&\left[\, |h| < \delta \Longrightarrow 
					\left|f(\zeta+h) - f(\zeta) - (\alpha + \isym \cdot \beta) \cdot h\right| < \epsilon \cdot |h|\, \right]
				\end{align}
				を満たす実数$\alpha$と$\beta$が取れる.また,
				$h$と$k$を実数とすれば
				\begin{align}
					&f(\zeta + (h + \isym \cdot k)) - f(\zeta) - (\alpha + \isym \cdot \beta) \cdot (h + \isym \cdot k) \\
					&= \left\{u(a+h,b+k) - u(a,b)
					- \alpha \cdot h + \beta \cdot k\right\} \\
					&\quad + i \cdot \left\{v(a+h,b+k) - v(a,b)
					- \beta \cdot h - \alpha \cdot k\right\}
					\label{fom:Cauchy_Riemann_1}
				\end{align}
				が成り立つ.いま$\epsilon$を任意に与えられた正の実数とする.
				\begin{align}
					\forall h \in \C\, 
					\left[\, |h| < \delta \Longrightarrow 
					\left|f(\zeta+h) - f(\zeta) - (\alpha + \isym \cdot \beta) \cdot h\right| < \epsilon \cdot |h|\, \right]
				\end{align}
				を満たす正の実数$\delta$を取ると,任意の実数$h$と$k$に対して
				\begin{align}
					\sqrt{h^2 + k^2} < \delta
				\end{align}
				ならば
				\begin{align}
					&\left|u(a+h,b+k) - u(a,b) - \alpha \cdot h + \beta \cdot k\right| \\
					&\leq \left|f(\zeta + (h + \isym \cdot k))- f(\zeta)
					- (\alpha + \isym \cdot \beta) \cdot (h + \isym \cdot k)\right| &\mbox{(\refeq{fom:Cauchy_Riemann_1})より} \\
					&< \epsilon \cdot \sqrt{h^2 + k^2}
				\end{align}
				が成立する.ゆえに$u$は$(a,b)$において全微分可能である.
				同様に任意の実数$h$と$k$に対して
				\begin{align}
					\sqrt{h^2 + k^2} < \delta \Longrightarrow
					&\left|v(a+h,b+k) - v(a,b) - \beta \cdot h - \alpha \cdot k\right| \\
					&< \epsilon \cdot \sqrt{h^2 + k^2}
				\end{align}
				も成り立つので,$v$もまた$(a,b)$において全微分可能である.そして
				\begin{align}
					\partial_1 u(a,b) = \alpha = \partial_2 v(a,b)
				\end{align}
				かつ
				\begin{align}
					\partial_2 u(a,b) = - \beta = -\partial_1 v(a,b)
				\end{align}
				が成り立つ.
			
			\item[第二段]
				$u$と$v$が$(a,b)$で全微分可能であって,かつ
				\begin{align}
					\partial_1 u(a,b) = \partial_2 v(a,b)
				\end{align}
				と
				\begin{align}
					\partial_2 u(a,b) = -\partial_1 v(a,b)
				\end{align}
				が成り立っているとき,$f$が$\zeta$で微分可能であることを示す.
				いま$\epsilon$を任意に与えられた正の実数とする.
				すると,任意の複素数$h$に対して
				\begin{align}
					&|h| < \delta_1  \\
					&\Longrightarrow
					\left|u(\varphi^{-1}(\zeta+h)) - u(\varphi^{-1}(\zeta)) 
					- \partial_1 u(a,b) \cdot \Re{h} - \partial_2 u(a,b) \cdot \Im{h}\right| < \epsilon \cdot |h|
				\end{align}
				を満たす正の実数$\delta_1$と,
				\begin{align}
					&|h| < \delta_2 \\
					&\Longrightarrow
					\left|v(\varphi^{-1}(\zeta+h)) - v(\varphi^{-1}(\zeta)) 
					- \partial_1 v(a,b) \cdot \Re{h} - \partial_2 v(a,b) \cdot \Im{h}\right| < \epsilon \cdot |h|
				\end{align}
				を満たす正の実数$\delta_2$が取れる.ここでCauchy-Riemannの関係式から
				\begin{align}
					&|h| < \delta_1 \\
					&\Longrightarrow
					\left|u(\varphi^{-1}(\zeta+h)) - u(\varphi^{-1}(\zeta)) 
					- \partial_1 u(a,b) \cdot \Re{h} + \partial_1 v(a,b) \cdot \Im{h}\right| < \epsilon \cdot |h|
				\end{align}
				かつ
				\begin{align}
					&|h| < \delta_2 \\
					&\Longrightarrow
					\left|v(\varphi^{-1}(\zeta+h)) - v(\varphi^{-1}(\zeta)) 
					- \partial_1 v(a,b) \cdot \Re{h} - \partial_1 u(a,b) \cdot \Im{h}\right| < \epsilon \cdot |h|
				\end{align}
				が成り立つので,任意の複素数$h$に対して
				\begin{align}
					|h| < \min{\{\delta_1,\delta_2\}}
				\end{align}
				ならば
				\begin{align}
					&\left|f(\zeta+h) - f(\zeta) - (\partial_1 u(a,b) + \isym \cdot \partial_1 v(a,b)) \cdot h\right| \\
					&\leq \left|u(\varphi^{-1}(\zeta+h)) - u(\varphi^{-1}(\zeta)) 
					- \partial_1 u(a,b) \cdot \Re{h} + \partial_1 v(a,b) \cdot \Im{h}\right| \\
					&\quad + \left|v(\varphi^{-1}(\zeta+h)) - v(\varphi^{-1}(\zeta)) 
					- \partial_1 v(a,b) \cdot \Re{h} - \partial_1 u(a,b) \cdot \Im{h}\right| \\
					&< 2 \cdot \epsilon \cdot |h|
				\end{align}
				が成立する.ゆえに$f$は$\zeta$で微分可能である.
				\QED
		\end{description}
	\end{sketch}
	
	無断で偏微分の順序を入れ替えている答案が多かったので念のため次の結果を載せておきます.
	
	\begin{screen}
		\begin{thm}[偏導関数が全微分可能なら偏微分は順序を替えても等しい]
			$D$を$\R^2$の開集合とし,$(a,b)$を$D$の要素とし,$u$を$D$上の実数値関数とする.
			また$u$は$D$の各要素で第一変数と第二変数それぞれに関して偏微分可能であるとし,
			$u$の第一変数に関する導関数を$\partial_1 u$とし,第二変数に関する導関数を$\partial_2 u$とする.
			このとき,$\partial_1 u$と$\partial_2 u$が共に$(a,b)$で全微分可能であるならば
			\begin{align}
				\partial_1 \partial_2 u(a,b) = \partial_2 \partial_1 u(a,b)
			\end{align}
			が成り立つ.ただし$\partial_1 \partial_2 u(a,b)$と$\partial_2 \partial_1 u(a,b)$は
			\begin{align}
				\partial_1 \partial_2 u(a,b) \defeq \lim_{h \to 0} \frac{\partial_2 u(a+h,b) - \partial_2 u(a,b)}{h}
			\end{align}
			および
			\begin{align}
				\partial_2 \partial_1 u(a,b) \defeq \lim_{k \to 0} \frac{\partial_1 u(a,b+k) - \partial_1 u(a,b)}{k}
			\end{align}
			で定められた実数である.
		\end{thm}
	\end{screen}
	
	\begin{sketch}
		いま$\epsilon$を任意に与えられた正の実数とする.$D$は開集合であるから,任意の実数$h$と$k$に対して
		\begin{align}
			\sqrt{h^2 + k^2} < r \Longrightarrow (a+h,b+k) \in D
		\end{align}
		を満たす正の実数$r$が取れる.ここで
		\begin{align}
			\sqrt{h^2 + k^2} < r
		\end{align}
		を満たす実数$h$と$k$に対して
		\begin{align}
			\Delta(h,k) \defeq \left\{u(a+h,b+k) - u(a+h,b)\right\} - \left\{u(a,b+k) - u(a,b)\right\}
		\end{align}
		とおく.$\Delta(h,k)$とは
		\begin{align}
			(a,a+h) \ni x \longmapsto u(x,b+k) - u(x,b)
		\end{align}
		なる写像の差であるから,平均値の定理より
		\begin{align}
			0 < \theta < 1
		\end{align}
		かつ
		\begin{align}
			\Delta(h,k) = h \cdot \left[\left\{\partial_1 u(a+\theta \cdot h,b+k) - \partial_1 u(a+ \theta \cdot h,b)\right\} 
			- \left\{\partial_1 u(a,b+k) - \partial_1 u(a,b)\right\}\right]
		\end{align}
		を満たす実数$\theta$が取れる.また$\partial_1 u$の全微分可能性より
		\begin{align}
			&\sqrt{h^2 + k^2} < \delta_1 \\
			&\Longrightarrow \left|\partial_1 u(a+\theta \cdot h,b+k) - \partial_1 u(a,b) 
			- \partial_1 \partial_1 u(a,b) \cdot (\theta \cdot h) - \partial_2 \partial_1 u(a,b) \cdot k\right|
			< \epsilon \cdot \sqrt{h^2 + k^2}
		\end{align}
		を満たす正の実数$\delta_1(< r)$と,
		\begin{align}
			&|h| < \delta_2 \\
			&\Longrightarrow \left|\partial_1 u(a+\theta \cdot h,b) - \partial_1 u(a,b) 
			- \partial_1 \partial_1 u(a,b) \cdot (\theta \cdot h)\right|
			< \epsilon \cdot |h|
		\end{align}
		を満たす正の実数$\delta_2(< r)$が取れるので,
		\begin{align}
			\delta \defeq \min\{\delta_1,\delta_2\}
		\end{align}
		とおけば
		\begin{align}
			\sqrt{h^2 + k^2} < \delta
		\end{align}
		を満たす任意の実数$h$と$k$に対して
		\begin{align}
			&\left|\frac{\Delta(h,k)}{h} - \partial_2 \partial_1 u(a,b) \cdot k\right| \\
			&\leq \left|\partial_1 u(a+\theta \cdot h,b+k) - \partial_1 u(a,b) 
			- \partial_1 \partial_1 u(a,b) \cdot (\theta \cdot h) - \partial_2 \partial_1 u(a,b) \cdot k\right| \\
			&\quad + \left|\partial_1 u(a+\theta \cdot h,b) - \partial_1 u(a,b) 
			- \partial_1 \partial_1 u(a,b) \cdot (\theta \cdot h)\right| \\
			&< 2 \cdot \epsilon \cdot \sqrt{h^2 + k^2}
		\end{align}
		が成立する.他方で
		\begin{align}
			\Delta(h,k) = \left\{u(a+h,b+k) - u(a,b+k)\right\} - \left\{u(a+h,b) - u(a,b)\right\}
		\end{align}
		も成り立つので,上の内容で第一変数と第二変数の立場を入れ替えれば
		\begin{align}
			\sqrt{h^2 + k^2} < \eta \Longrightarrow 
			\left|\frac{\Delta(h,k)}{k} - \partial_1 \partial_2 u(a,b) \cdot h\right| 
			< 2 \cdot \epsilon \cdot \sqrt{h^2 + k^2}
		\end{align}
		を満たす正の実数$\eta$が取れる.従って,
		\begin{align}
			\sqrt{2} \cdot |h| < \min\{\delta,\eta\}
		\end{align}
		を満たす任意の実数$h$に対して
		\begin{align}
			\left|\partial_2 \partial_1 u(a,b) \cdot h - \partial_1 \partial_2 u(a,b) \cdot h\right|
			&\leq \left|\frac{\Delta(h,k)}{h} - \partial_2 \partial_1 u(a,b) \cdot h\right| 
			+ \left|\frac{\Delta(h,k)}{h} - \partial_1 \partial_2 u(a,b) \cdot h\right| \\
			&< 4 \cdot \sqrt{2} \cdot \epsilon \cdot |h|
		\end{align}
		が成立する.よって
		\begin{align}
			\left|\partial_2 \partial_1 u(a,b) - \partial_1 \partial_2 u(a,b)\right| < 4 \cdot \sqrt{2} \cdot \epsilon
		\end{align}
		が従い,$\epsilon$は任意に与えられていたので
		\begin{align}
			\partial_2 \partial_1 u(a,b) = \partial_1 \partial_2 u(a,b)
		\end{align}
		が出る.
		\QED
	\end{sketch}
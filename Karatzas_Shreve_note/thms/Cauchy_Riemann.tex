\subsection{Cauchy-Riemann方程式}
	\begin{screen}
		\begin{dfn}[全微分可能]
			$D$を$\R^2$の開集合とし,$(x,y)$を$D$の要素とし,
			$u$を$D$上の実数値関数とする.このとき
			\begin{align}
				&\forall \epsilon \in \R_+\, \exists \delta \in \R_+\, \forall h,k \in \R\, \\
				&\sqrt{h^2 + k^2} < \delta \Longrightarrow
				\left|u(x+h,y+k) - u(x,y) - \alpha \cdot h - \beta \cdot k\right| 
				< \epsilon \cdot \sqrt{h^2 + k^2}
				\label{fom:def_totally_differentiability}
			\end{align}
			を満たす実数$\alpha$と$\beta$が取れるなら,
			$u$は$(x,y)$において{\bf 全微分可能である}\index{ぜんびぶんかのう@全微分可能}{\bf (totally differentiable)}という.
		\end{dfn}
	\end{screen}
	
	$D$を$\R^2$の開集合とし,$(x,y)$を$D$の要素とし,$u$を$D$上の実数値関数とし,
	$u$が$(x,y)$で全微分可能であるとする.このとき
	\begin{align}
		&\forall \epsilon \in \R_+\, \exists \delta \in \R_+\, \forall h,k \in \R\, \\
		&\sqrt{h^2 + k^2} < \delta \Longrightarrow
		\left|u(x+h,y+k) - u(x,y) - \alpha \cdot h - \beta \cdot k\right| 
		< \epsilon \cdot \sqrt{h^2 + k^2}
	\end{align}
	を満たす実数$\alpha$と$\beta$が取れるが,同時に
	\begin{align}
		&\forall \epsilon \in \R_+\, \exists \delta \in \R_+\, \forall h,k \in \R\, \\
		&\sqrt{h^2 + k^2} < \delta \Longrightarrow
		\left|u(x+h,y+k) - u(x,y) - \gamma \cdot h - \zeta \cdot k\right| 
		< \epsilon \cdot \sqrt{h^2 + k^2}
	\end{align}
	を満たす実数$\gamma$と$\zeta$が取れる場合
	\begin{align}
		\alpha = \gamma
	\end{align}
	かつ
	\begin{align}
		\beta = \zeta
	\end{align}
	が成り立つ.実際,$\epsilon$を任意に与えられた正の実数とすると
	\begin{align}
		\forall h,k \in \R\,
		\left[\, \sqrt{h^2 + k^2} < \delta \Longrightarrow
		\left|u(x+h,y+k) - u(x,y) - \alpha \cdot h - \beta \cdot k\right| 
		< \epsilon \cdot \sqrt{h^2 + k^2}\, \right]
	\end{align}
	かつ
	\begin{align}
		\forall h,k \in \R\,
		\left[\, \sqrt{h^2 + k^2} < \delta \Longrightarrow
		\left|u(x+h,y+k) - u(x,y) - \gamma \cdot h - \zeta \cdot k\right| 
		< \epsilon \cdot \sqrt{h^2 + k^2}\, \right]
	\end{align}
	を満たす正の実数$\delta$が取れるが,このとき
	\begin{align}
		|h| < \delta
	\end{align}
	ならば
	\begin{align}
		\left|u(x+h,y) - u(x,y) - \alpha \cdot h\right| < \epsilon \cdot |h|
	\end{align}
	かつ
	
	\begin{screen}
		\begin{thm}[全微分可能なら偏微分可能]
		\end{thm}
	\end{screen}
	
	\begin{screen}
		\begin{thm}[偏導関数が連続なら全微分可能]
		\end{thm}
	\end{screen}

	$D$を$\C$の開集合とし,$f$を$D$上の複素数値関数とし,
	$z$を$D$の要素とし,$x$と$y$を
	\begin{align}
		z = x + i \cdot y
	\end{align}
	なる実数とする.また
	\begin{align}
		D \ni z \longmapsto \Re{f(z)}
	\end{align}
	なる写像を$u$とし,
	\begin{align}
		D \ni z \longmapsto \Im{f(z)}
	\end{align}
	なる写像を$v$とし,$\varphi$を
	\begin{align}
		\R^2 \ni (a,b) \longmapsto a + i \cdot b
	\end{align}
	なる関係で定める.そして
	\begin{align}
		\R_+ \defeq \Set{x \in \R}{0 < x}
	\end{align}
	とおく.
	
	\begin{itembox}[l]{Cauchy-Riemann方程式}
		上の設定の下で$u \circ \varphi$が$(x,y)$で全微分可能であるとき,
		(\refeq{fom:def_totally_differentiability})の$\alpha$に当たる実数を
		\begin{align}
			\partial_1 u
		\end{align}
		と書き,$\beta$に当たる実数を
		\begin{align}
			\partial_2 u
		\end{align}
		と書く.同様に$\partial_1 v$と$\partial_2 v$を定める.
		$u \circ \varphi$と$v \circ \varphi$が共に$(x,y)$で全微分可能であるとき,
		\begin{align}
			\partial_1 u = \partial_2 v
		\end{align}
		と
		\begin{align}
			\partial_2 u = -\partial_1 v
		\end{align}
		の二つの式を{\bf Cauchy-Riemann方程式}と呼ぶ.
	\end{itembox}
	
	\begin{screen}
		\begin{thm}[複素微分可能であることとCauchy-Riemann方程式]
			記号は全て上で設定したものとする.
			$f$が$z$で複素微分可能であることと,
			$u \circ \varphi$と$v \circ \varphi$が共に$(x,y)$で全微分可能であって
			かつCauchy-Riemann方程式が成り立つことは同値である.
		\end{thm}
	\end{screen}
	
	\begin{sketch}\mbox{}
		\begin{description}
			\item[第一段]
				いま$f$が$z$で複素微分可能であるとする.すると
				\begin{align}
					&\forall \epsilon \in \R_+\, \exists \delta \in \R_+\, \forall h \in \C\, \\
					&\left[\, |h| < \delta \Longrightarrow 
					\left|f(z+h) - f(z) - (\alpha + i \cdot \beta) \cdot h\right| < \epsilon \cdot |h|\, \right]
				\end{align}
				を満たす実数$\alpha$と$\beta$が取れる.このとき,
				$h$と$k$を実数とすれば
				\begin{align}
					&f(z + (h + i \cdot k)) - f(z) - (\alpha + i \cdot \beta) \cdot (h + i \cdot k) \\
					&= \left\{u(\varphi(x+h,y+k)) - u(\varphi(x,y))
					- \alpha \cdot h + \beta \cdot k\right\} \\
					&\quad + i \cdot \left\{v(\varphi(x+h,y+k)) - v(\varphi(x,y))
					- \beta \cdot h - \alpha \cdot k\right\}
					\label{fom:Cauchy_Riemann_1}
				\end{align}
				が成り立つ.ここで$\epsilon$を任意に与えられた正の実数とする.
				\begin{align}
					\forall h \in \C\, 
					\left[\, |h| < \delta \Longrightarrow 
					\left|f(z+h) - f(z) - (\alpha + i \cdot \beta) \cdot h\right| < \epsilon \cdot |h|\, \right]
				\end{align}
				を満たす正の実数$\delta$を取ると,任意の実数$h$と$k$に対して
				\begin{align}
					\sqrt{h^2 + k^2} < \delta
				\end{align}
				であれば
				\begin{align}
					&\left|u(\varphi(x+h,y+k)) - u(\varphi(x,y))
					- \alpha \cdot h + \beta \cdot k\right| \\
					&\leq \left|f(z + (h + i \cdot k))- f(z)
					- (\alpha + i \cdot \beta) \cdot (h + i \cdot k)\right| &\mbox{(\refeq{fom:Cauchy_Riemann_1})より} \\
					&< \epsilon \cdot \sqrt{h^2 + k^2}
				\end{align}
				が成立する.ゆえに$u \circ \varphi$は$(x,y)$において全微分可能である.
				同様に任意の実数$h$と$k$に対して
				\begin{align}
					\sqrt{h^2 + k^2} < \delta \Longrightarrow
					&\left|v(\varphi(x+h,y+k)) - v(\varphi(x,y))
					- \beta \cdot h - \alpha \cdot k\right| \\
					&< \epsilon \cdot \sqrt{h^2 + k^2}
				\end{align}
				も成り立つので,$v \circ \varphi$もまた$(x,y)$において全微分可能である.
				そして
				\begin{align}
					\partial_1 u = \alpha = \partial_2 v
				\end{align}
				かつ
				\begin{align}
					\partial_2 u = - \beta = -\partial_1 v
				\end{align}
				が成り立つ.
			
			\item[第二段]
				逆に$u \circ \varphi$と$v \circ \varphi$が$(x,y)$で全微分可能であって,かつ
				\begin{align}
					\partial_1 u = \partial_2 v
				\end{align}
				と
				\begin{align}
					\partial_2 u = -\partial_1 v
				\end{align}
				が成り立っているとき,$f$が$z$で複素微分可能であることを示す.
				いま$\epsilon$を任意に与えられた正の実数とする.
				すると,任意の複素数$h$に対して
				\begin{align}
					|h| < \delta_1 \Longrightarrow
					\left|u(z+h) - u(z) - \partial_1 u \cdot \Re{h} - \partial_2 u \cdot \Im{h}\right| < \epsilon \cdot |h|
				\end{align}
				を満たす正の実数$\delta_1$と,
				\begin{align}
					|h| < \delta_2 \Longrightarrow
					\left|v(z+h) - v(z) - \partial_1 v \cdot \Re{h} - \partial_2 v \cdot \Im{h}\right| < \epsilon \cdot |h|
				\end{align}
				を満たす正の実数$\delta_2$が取れる.ここで
				Cauchy-Riemannの関係式から
				\begin{align}
					|h| < \delta_1 \Longrightarrow
					\left|u(z+h) - u(z) - \partial_1 u \cdot \Re{h} + \partial_1 v \cdot \Im{h}\right| < \epsilon \cdot |h|
				\end{align}
				かつ
				\begin{align}
					|h| < \delta_2 \Longrightarrow
					\left|v(z+h) - v(z) - \partial_1 v \cdot \Re{h} - \partial_1 u \cdot \Im{h}\right| < \epsilon \cdot |h|
				\end{align}
				が成り立つので,任意の複素数$h$に対して
				\begin{align}
					|h| < \min{\{\delta_1,\delta_2\}}
				\end{align}
				であれば
				\begin{align}
					&\left|f(z+h) - f(z) - (\partial_1 u + i \cdot \partial_1 v) \cdot h\right| \\
					&\leq \left|u(z+h) - u(z) - \partial_1 u \cdot \Re{h} + \partial_1 v \cdot \Im{h}\right| \\
					&\quad + \left|v(z+h) - v(z) - \partial_1 v \cdot \Re{h} - \partial_1 u \cdot \Im{h}\right| \\
					&< 2 \cdot \epsilon \cdot |h|
				\end{align}
				が成立する.ゆえに$f$は$z$で複素微分可能である.
				\QED
		\end{description}
	\end{sketch}
\begin{itembox}[l]{Lemma for Dynkin system theorem}
		集合$\Omega$の部分集合族$\mathscr{D}$が
		教科書本文中のDynkin族の定義(i)と(ii)を満たしているとする.このとき,
		Dynkin族の定義(iii)は,
		$\mathscr{D}$が可算直和で閉じていることと同値である.
\end{itembox}

\begin{prf}
	$\mathscr{D}$が可算直和について閉じているとする.このとき
	単調増大列$A_1 \subset A_2 \subset \cdots$を取り
	\begin{align}
		B_1 \coloneqq A_1,
		\quad B_n \coloneqq A_n \backslash A_{n-1},
		\quad (n \geq 2)
	\end{align}
	とおけば,Dynkin族の定義(ii)より$B_n \in \mathscr{D}\ (n \geq 1)$が満たされ
	\begin{align}
		\bigcup_{n=1}^{\infty} A_n = \sum_{n=1}^{\infty} B_n \in \mathscr{D} 
	\end{align}
	が成立する.逆に$\mathscr{D}$が(iii)を満たしているとして,互いに素な
	集合列$(B_n)_{n=1}^{\infty} \subset \mathscr{D}$を取る.
	$A^c = \Omega \backslash A$
	とDynkin族の定義(i)(ii)より,
	$A,B \in \mathscr{D}$が$A \cap B = \emptyset$を満たしていれば
	$A^c \cap B^c = A^c - B\in \mathscr{D}$が成り立ち
	\begin{align}
		B_1^c \cap B_2^c \cap \cdots \cap B_n^c
		= \left( \cdots \left( \left( B_1^c \cap B_2^c \right) \cap B_3^c \right) \cap \cdots \cap B_{n-1}^c \right) \cap B_n^c
		\in \mathscr{D},
		\quad (n=1,2,\cdots)
	\end{align}
	が得られる.よって
	\begin{align}
		D_n \coloneqq \bigcup_{i=1}^n B_i = \Omega \backslash \Biggl( \bigcap_{i=1}^n B_i^c \Biggr),
		\quad (n=1,2,\cdots)
	\end{align}
	により定める単調増大列$(D_n)_{n=1}^{\infty}$は$\mathscr{D}$に含まれ
	\begin{align}
		\sum_{n=1}^{\infty} B_n = \bigcup_{n=1}^{\infty} D_n \in \mathscr{D}
	\end{align}
	が成立する.
	\QED
\end{prf}

\begin{itembox}[l]{Dynkin system theorem}
		Let $\mathscr{C}$ be a collection of subsets of $\Omega$ 
		which is closed under pairwise intersection. If $\mathscr{D}$ is 
		a Dynkin system containing $\mathscr{C}$, then $\mathscr{D}$ also 
		contains the $\sigma$-field $\sigma(\mathscr{C})$ generated by $\mathscr{C}$.
\end{itembox}

\begin{prf}$\mathscr{C}$を含む最小のDynkin族を$\delta(\mathscr{C})$と書き,
	$\delta(\mathscr{C}) = \sigma(\mathscr{C})$が成り立つことを示す.
	\begin{description}
		\item[第一段]
			$\delta(\mathscr{C})$が交演算について閉であれば
			$\delta(\mathscr{C})$は$\sigma$-加法族である.実際,
			\begin{align}
				A^c = \Omega \backslash A
			\end{align}
			より$\delta(\mathscr{C})$は補演算で閉じるから,
			交演算で閉じていれば,
			$A_n \in \delta(\mathscr{C})\ (n=1,2,\cdots)$に対し
			\begin{align}
				\bigcup_{n=1}^{\infty} A_n
				= \sum_{n=1}^{\infty} A_1^c \cap A_2^c \cap \cdots \cap A_{n-1}^c \cap A_n
				\in \delta(\mathscr{C})
			\end{align}
			が従い$\sigma(\mathscr{C}) \subset \delta(\mathscr{C}) \subset \mathscr{D}$が得られる
			\footnote{
				$\sigma$-加法族はDynkin族であるから,$\delta(\mathscr{C}) \subset \sigma(\mathscr{C})$
				が成り立ち$\sigma(\mathscr{C}) = \delta(\mathscr{C})$となる.
			}
			.
			
		\item[第二段]
			$\delta(\mathscr{C})$が交演算について閉じていることを示す.いま,
			\begin{align}
				\mathscr{D}_1 \coloneqq
				\Set{B \in \delta(\mathscr{C})}{ A \cap B \in \delta(\mathscr{C}),\ 
				\forall A \in \mathscr{C}}
			\end{align}
			により定める$\mathscr{D}_1$はDynkin族であり$\mathscr{C}$を含むから
			\begin{align}
				\delta(\mathscr{C}) \subset \mathscr{D}_1
			\end{align}
			が成立する.従って
			\begin{align}
				\mathscr{D}_2 \coloneqq
				\Set{B \in \delta(\mathscr{C})}{ A \cap B \in \delta(\mathscr{C}),\ 
				\forall A \in \delta(\mathscr{C})}
			\end{align}
			によりDynkin族$\mathscr{D}_2$を定めれば,$\mathscr{C} \subset \mathscr{D}_2$が満たされ
			\begin{align}
				\delta(\mathscr{C}) \subset \mathscr{D}_2
			\end{align}
			が得られる.よって$\delta(\mathscr{C})$は交演算について閉じている.
			\QED
	\end{description}
\end{prf}

\begin{itembox}[l]{Problem 1.4}
\label{thm:application_dynkin_system_theorem_to_independence}
		Let $X = \Set{X_t}{0 \leq t < \infty}$ be a stochastic process 
		for which $X_0,X_{t_1} - X_{t_0}, \cdots, X_{t_n} - X_{t_{n-1}}$ are 
		independent random variables, for every integer $n \geq 1$ and indices 
		$0 = t_0 < t_1 < \cdots < t_n < \infty$. Then for any fixed $0 \leq s < t < \infty$, 
		the increment $X_t - X_s$ is independent of $\mathscr{F}^X_s$.
\end{itembox}
この主張の逆も成立する:
\begin{prf}
	先ず任意の$s \leq t \leq r$に対し$\sigma(X_t - X_s) \subset \mathscr{F}^X_r$が成り立つ.実際,
	\begin{align}
		\Phi:\R^d \times \R^d \ni (x,y) \longmapsto x - y
	\end{align}
	の連続性と$\borel{\R^d \times \R^d} = \borel{\R^d} \otimes \borel{\R^d}$より,
	任意の$E \in \borel{\R^d}$に対して
	\begin{align}
		(X_t - X_s)^{-1}(E) 
		= \left\{ \left( X_t,X_s \right) \in \Phi^{-1}(E) \right\}
		\in \sigma(X_s,X_t) \subset \mathscr{F}^X_r
		\label{eq:thm_application_dynkin_system_theorem_to_independence_1}
	\end{align}
	が満たされる.よって任意に$A_0 \in \sigma(X_0),\ A_i \in \sigma(X_{t_i} - X_{t_{i-1}})$を取れば,
	$X_{t_n} - X_{t_{n-1}}$が$\mathscr{F}^X_{t_{n-1}}$と独立であるから
	\begin{align}
		\prob{A_0 \cap A_1 \cap \cdots \cap A_n}
		= \prob{A_0 \cap A_1 \cap \cdots \cap A_{n-1}} \prob{A_n}
	\end{align}
	が成立する.帰納的に
	\begin{align}
		\prob{A_0 \cap A_1 \cap \cdots \cap A_n}
		= \prob{A_0} \prob{A_1} \cdots \prob{A_n}
	\end{align}
	が従い$X_0,X_{t_1} - X_{t_0}, \cdots, X_{t_n} - X_{t_{n-1}}$の独立性を得る.
	\QED
\end{prf}

\begin{prf}[Problem 1.4]\mbox{}
	\begin{description}
		\item[第一段]
			Dynkin族を次で定める:
			\begin{align}
				\mathscr{D} \coloneqq
				\Set{A \in \mathscr{F}}{\prob{A \cap B} = \prob{A}\prob{B},\ \forall B \in \sigma(X_t - X_s)}.
			\end{align}
			いま,任意に$0 = s_0 < \cdots < s_n = s$を取り固定し
			\begin{align}
				\mathscr{A}_{s_0, \cdots, s_n} \coloneqq
				\Set{\bigcap_{i=0}^n A_i}{A_0 \in \sigma(X_0),\ A_i \in \sigma(X_{s_i} - X_{s_j}),\ i=1,\cdots,n}
			\end{align}
			により乗法族を定めれば,仮定より$\sigma(X_{s_i} - X_{s_{i-1}})$と$\sigma(X_t - X_s)$が独立であるから
			\begin{align}
				\mathscr{A}_{s_0, \cdots, s_n}
				\subset \mathscr{D}
			\end{align}
			が成立し,Dynkin族定理により
			\begin{align}
				\sigma(X_{s_0},X_{s_1}-X_{s_0},\cdots,X_{s_n} - X_{s_{n-1}})
				= \sgmalg{\mathscr{A}_{s_0, \cdots, s_n}}
				\subset \mathscr{D}
				\label{eq:thm_application_dynkin_system_theorem_to_independence_2}
			\end{align}
			が従う.
		
		\item[第二段]
			$\sigma(X_{s_0},X_{s_1}-X_{s_0},\cdots,X_{s_n} - X_{s_{n-1}})$の全体が
			$\mathscr{F}^X_s$を生成することを示す.先ず,
			(\refeq{eq:thm_application_dynkin_system_theorem_to_independence_1})より
			\begin{align}
				\bigcup_{\substack{n \geq 1 \\ s_0 < \cdots < s_n}} 
				\sigma(X_{s_0},X_{s_1}-X_{s_0},\cdots,X_{s_n} - X_{s_{n-1}})
				\subset \mathscr{F}^X_s
				\label{eq:thm_application_dynkin_system_theorem_to_independence_3}
			\end{align}
			が成立する.一方で,任意の
			$X_r^{-1}(E)\ (\forall E \in \borel{\R^d},\ 0 < r \leq s)$について,
			\begin{align}
				\Psi:\R^d \times \R^d \ni (x,y) \longmapsto x + y
			\end{align}
			で定める連続写像を用いれば
			\begin{align}
				X_r^{-1}(E)
				= \left( X_r - X_0 + X_0 \right)^{-1}(E)
				= \left\{\left( X_r - X_0, X_0\right) \in \Psi^{-1}(E) \right\}
			\end{align}
			となり,$X_r^{-1}(E) \in \sigma(X_0, X_r - X_0)$が満たされ
			\begin{align}
				\sigma(X_r) \subset \sigma(X_0, X_r - X_0)
				\subset \sigma(X_0, X_r - X_0,X_s - X_r)
				\label{eq:thm_application_dynkin_system_theorem_to_independence_4}
			\end{align}
			が出る.
			$\sigma(X_0) \subset \sigma(X_0,X_s - X_0)$
			も成り立ち
			\begin{align}
				\bigcup_{0 \leq r \leq s} \sigma(X_r) \subset 
				\bigcup_{\substack{n \geq 1 \\ s_0 < \cdots < s_n}} \sigma(X_{s_0},X_{s_1}-X_{s_0},\cdots,X_{s_n} - X_{s_{n-1}})
			\end{align}
			が従うから,(\refeq{eq:thm_application_dynkin_system_theorem_to_independence_3})
			と併せて
			\begin{align}
				\mathscr{F}^X_s
				= \sgmalg{\bigcup_{\substack{n \geq 1 \\ s_0 < \cdots < s_n}} \sigma(X_{s_0},X_{s_1}-X_{s_0},\cdots,X_{s_n} - X_{s_{n-1}})}
				\label{eq:thm_application_dynkin_system_theorem_to_independence_5}
			\end{align}
			が得られる.
		
		\item[第三段]
			任意の$0 = s_0 < s_1 < \cdots < s_n = s$に対し,
			(\refeq{eq:thm_application_dynkin_system_theorem_to_independence_1})と
			(\refeq{eq:thm_application_dynkin_system_theorem_to_independence_4})より
			\begin{align}
				\sigma(X_{s_0},X_{s_1}-X_{s_0},\cdots,X_{s_n} - X_{s_{n-1}})
				= \sigma(X_{s_0},X_{s_1},\cdots,X_{s_n})
				\label{eq:thm_application_dynkin_system_theorem_to_independence_6}
			\end{align}
			が成り立つ.
		
		\item[第四段]
			二つの節点$0 = s_0 < \cdots < s_n = s$と$0 = r_0 < \cdots < r_m = s$
			の合併を$0 = u_0 < \cdots < u_k = s$と書けば
			\begin{align}
				\sigma(X_{s_0},\cdots,X_{s_n})
				\cup \sigma(X_{r_0},\cdots,X_{r_m})
				\subset \sigma(X_{u_0},\cdots,X_{u_k})
			\end{align}
			が成り立つから
			\begin{align}
				\bigcup_{\substack{n \geq 1 \\ s_0 < \cdots < s_n}} \sigma(X_{s_0},X_{s_1},\cdots,X_{s_n})
			\end{align}
			は交演算で閉じている.従って
			(\refeq{eq:thm_application_dynkin_system_theorem_to_independence_2}),
			(\refeq{eq:thm_application_dynkin_system_theorem_to_independence_5}),
			(\refeq{eq:thm_application_dynkin_system_theorem_to_independence_6})及び
			Dynkin族定理により
			\begin{align}
				\mathscr{F}^X_s 
				= \sgmalg{\bigcup_{\substack{n \geq 1 \\ s_0 < \cdots < s_n}} \sigma(X_{s_0},X_{s_1}-X_{s_0},\cdots,X_{s_n} - X_{s_{n-1}})}
				= \sgmalg{\bigcup_{\substack{n \geq 1 \\ s_0 < \cdots < s_n}} \sigma(X_{s_0},X_{s_1},\cdots,X_{s_n})}
				\subset \mathscr{D}
			\end{align}
			が従い定理の主張を得る.
			\QED
	\end{description}
\end{prf}
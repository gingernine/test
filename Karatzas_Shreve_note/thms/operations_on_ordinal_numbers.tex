本稿では,自然数の全体$\Natural$,整数の全体$\Z$,有理数の全体$\Q$,実数の全体$\R$,複素数の全体$\C$を
\begin{align}
	\Natural \subset \Z \subset \Q \subset \R \subset \C
\end{align}
となるように構成する.当然,加減乗除も純粋な延長として,つまり,それぞれの集合に定める加法を
\begin{align}
	+_\Natural,\quad +_\Z,\quad +_\Q.\quad +_\R,\quad +_\C
\end{align}
と書けば
\begin{align}
	+_\Natural \subset +_\Z \subset +_\Q \subset +_\R \subset +_\C
\end{align}
が満たされ,それぞれの集合に定める乗法を
\begin{align}
	\cdot_\Natural,\quad \cdot_\Z,\quad \cdot_\Q.\quad \cdot_\R,\quad \cdot_\C
\end{align}
と書けば
\begin{align}
	\cdot_\Natural \subset \cdot_\Z \subset \cdot_\Q \subset \cdot_\R \subset \cdot_\C
\end{align}
が満たされるように構成する.これらは$\ON$に定められる加法と乗法が大元となっていて,$\ON$に定める加法を$+$とし,乗法を$\cdot$とすれば,
$\Natural$上の加法と乗法とは
\begin{align}
	+_\Natural \defeq +|_{\Natural \times \Natural}
\end{align}
と
\begin{align}
	\cdot_\Natural \defeq \cdot|_{\Natural \times \Natural}
\end{align}
によって定義されるものである.
順序についても,$\Natural$の順序は$\ON$の順序の制限で,つまり
\begin{align}
	O_\Natural \defeq \Set{(n,m)}{n,m \in \Natural \wedge (\, n \in m \vee n = m\, )}
\end{align}
で定められるが,残りの$\Z,\Q,\R$に定める順序を
\begin{align}
	O_\Z,\quad O_\Q,\quad O_\R
\end{align}
と書けば
\begin{align}
	O_\Natural \subset O_\Z \subset O_\Q \subset O_\R
\end{align}
が満たされるようにする.`埋め込めば拡張となる'ように数を構成している文献もあるが,それでは詰めが甘くもどかしい.
このように構成すると,例えば測度論の本では
\begin{align}
	0 \cdot \infty = \infty \cdot 0 = 0
\end{align}
を`ローカルなルール'としている場合があるが,本稿では定理として導かれる.
($\infty$の定義はP. \pageref{def:infinity})

\section{加法}
	$\alpha$と$\beta$を順序数とするとき,$\alpha$に$\beta$を``足す''という操作を導入したい.つまり,
	足し算の記号
	\begin{align}
		+
	\end{align}
	を何らかの意味で定めて
	\begin{align}
		\alpha + \beta
	\end{align}
	を実行したいのである.先ずは簡単に,$\beta$が$0$の場合は
	\begin{align}
		\alpha + 0 \defeq \alpha
	\end{align}
	と定めてしまう.$\beta$が$1$の場合は,ちょうど
	\begin{align}
		\alpha \cup \{\alpha\}
	\end{align}
	が$\alpha$の直後の元であったからこれを$\alpha + 1$を定めることにする.
	$\alpha + 2$も$\alpha + 1$の直後の元として
	\begin{align}
		\alpha + 2 \defeq (\alpha + 1) + 1
	\end{align}
	で定めることにして,この調子で
	\begin{align}
		&\alpha + 3 \defeq (\alpha + 2) + 1 \\
		&\alpha + 4 \defeq (\alpha + 3) + 1 \\
		&\alpha + 5 \defeq (\alpha + 4) + 1
	\end{align}
	としていくわけであるが,例えば$\beta$が$\Natural$である場合,$\Natural$は極限数であるから上の操作をいくら続けても
	\begin{align}
		\alpha + \Natural
	\end{align}
	に到達することは不可能である.仕方が無いから一番自然な方法として
	\begin{align}
		\Set{\alpha + k}{k \in \Natural}
	\end{align}
	の上限を$\alpha + \Natural$と定める.以上の操作をヒントにして,$\alpha + \beta$は
	\begin{itemize}
		\item $\beta$に対して$\beta = \gamma + 1$を満たす順序数$\gamma$が取れるなら
			\begin{align}
				\alpha + \beta \defeq (\alpha + \gamma) + 1,
			\end{align}
		
		\item $\beta$が極限数なら
			\begin{align}
				\alpha + \beta \defeq \bigcup \Set{\alpha + \gamma}{\gamma \in \beta},
			\end{align}
	\end{itemize}
	で定められる.意味的には再帰操作を繰り返しているのだから,それを$\mathcal{L}'$のことばで表すためには
	\ref{sec:recursive_definition}節の方法を応用すれば良い.
	また次節で述べることであるが``足し算''と同様にすれば$\ON$上に``掛け算''も定めることが出来る.
	それぞれ主に``加法''と``乗法''と呼ばれ,これらの延長が複素数に対する通常の四則演算に発展していく.
	
	\begin{screen}
		\begin{dfn}[順序数の加法]
			$\alpha$を順序数とし,$\Univ$上の写像$G_\alpha$を
			\begin{align}
				x \longmapsto 
				\begin{cases}
					\alpha & \mbox{if } \operatorname{dom}(x) = \emptyset \\
					x(\beta) \cup \{x(\beta)\} & \mbox{if } \beta \in \ON \wedge \operatorname{dom}(x) = \beta \cup \{\beta\} \\
					\bigcup \ran{x} & \mbox{o.w.}
				\end{cases}
			\end{align}
			なる関係により定めると,
			\begin{align}
				\forall \beta \in \ON\, (\, A_\alpha(\beta) = G_\alpha(A_\alpha|_\beta)\, )
			\end{align}
			を満たす$\ON$上の写像$A_\alpha$が取れる.ここで
			\begin{align}
				+ \defeq \Set{((\alpha,\beta),y)}{\alpha \in \ON \wedge \beta \in \ON \wedge y = A_\alpha (\beta)}
			\end{align}
			により$+$を定め,これを$\ON$上の{\bf 加法}\index{かほう@加法}{\bf (summation)}と呼ぶ.
		\end{dfn}
	\end{screen}
	
	$\alpha$を順序数とすれば,$G_\alpha$とは正式には
	\begin{align}
		\{\, (x,y) \mid \quad &\left(\, \dom{x} = \emptyset \Longrightarrow y = \alpha\, \right) \\
		&\wedge \forall \beta \in \ON\, \left(\, \dom{x} = \beta \cup \{\beta\} \Longrightarrow y = x(\beta) \cup \{x(\beta)\}\, \right) \\
		&\wedge \left[\, \dom{x} \neq \emptyset \wedge \forall \beta \in \ON\, \left(\, \dom{x} \neq \beta \cup \{\beta\}\, \right)
		\Longrightarrow y = \bigcup \ran{x}\, \right]\, \}
	\end{align}
	によって定められた写像である.そして$A_\alpha$とは
	\begin{align}
		\ON \ni \beta \longmapsto
		\begin{cases}
			\alpha & \mbox{if } \beta = \emptyset \\
			(\alpha + \gamma) \cup \{\alpha + \gamma\} & \mbox{if } \gamma \in \ON \wedge \beta = \gamma \cup \{\gamma\} \\
			\bigcup \Set{\alpha + \gamma}{\gamma \in \beta} & \mbox{if } \limo{\beta}
		\end{cases}
	\end{align}
	を満たす写像である.
	
	\begin{screen}
		\begin{thm}[$+$は$\ON$への全射である]\label{thm:summation_on_ordinal_numbers_is_a_mapping}
			$+$は$\ON \times \ON$から$\ON$への全射である:
			\begin{align}
				+:\ON \times \ON \srj \ON.
			\end{align}
		\end{thm}
	\end{screen}
	
	\begin{sketch}\mbox{}
		\begin{description}
			\item[第一段] $\dom{+} = \ON \times \ON$が成り立つことを示す.
				$x$を$\dom{+}$の要素とすれば
				\begin{align}
					(x,y) \in +
				\end{align}
				を満たす集合$y$が取れる.このとき
				\begin{align}
					x = (\alpha,\beta)
				\end{align}
				を満たす順序数$\alpha$と$\beta$が取れて,
				\begin{align}
					(\alpha, \beta) \in \ON \times \ON
				\end{align}
				なので
				\begin{align}
					x \in \ON \times \ON
				\end{align}
				が成り立つ.次に$x$を$\ON \times \ON$の要素とする.このとき
				\begin{align}
					x = (\alpha,\beta)
				\end{align}
				を満たす順序数$\alpha$と$\beta$が取れて,
				\begin{align}
					((\alpha,\beta),A_\alpha(\beta)) \in +
				\end{align}
				が成り立つから
				\begin{align}
					(\alpha,\beta) \in \dom{+}
				\end{align}
				が成り立つ.ゆえに
				\begin{align}
					x \in \dom{+}
				\end{align}
				が従う.$x$の任意性から
				\begin{align}
					\dom{+} = \ON \times \ON
				\end{align}
				が成り立つ.
				
			\item[第二段] $+$がsingle-valuedであることを示す.
				$\alpha,\alpha',\beta,\beta'$を順序数とするとき
				\begin{align}
					(\alpha,\beta) = (\alpha',\beta') \Longrightarrow A_\alpha(\beta) = A_{\alpha'}(\beta')
				\end{align}
				が成り立つことを示せば良いが,これは
				\begin{align}
					\alpha  = \alpha' \Longrightarrow A_\alpha = A_{\alpha'}
					\label{form:thm_summation_on_ordinal_numbers_is_a_mapping_1}
				\end{align}
				が成り立つことを示せば良い.ところで$A_\alpha$と$A_{\alpha'}$はどちらも$\ON$を定義域とする写像なので,
				\begin{align}
					\alpha  = \alpha' \Longrightarrow 
					\forall \beta \in \ON\, \left(\, A_\alpha(\beta) = A_{\alpha'}(\beta)\, \right)
				\end{align}
				を示せば定理\ref{thm:two_functions_with_same_domain_and_values_coincide}より
				(\refeq{form:thm_summation_on_ordinal_numbers_is_a_mapping_1})が従う.いま
				\begin{align}
					\alpha = \alpha'
				\end{align}
				であるとする.そして
				\begin{align}
					\beta \in \ON\, \left(\, A_\alpha(\beta) = A_{\alpha'}(\beta)\, \right)
					\label{form:thm_summation_on_ordinal_numbers_is_a_mapping_2}
				\end{align}
				なることは超限帰納法により証明する.まず
				\begin{align}
					A_\alpha(\emptyset) = \alpha
				\end{align}
				と
				\begin{align}
					A_{\alpha'}(\emptyset) = \alpha'
				\end{align}
				から
				\begin{align}
					A_\alpha(\emptyset) = A_{\alpha'}(\emptyset)
				\end{align}
				が従う.次に$\beta$を任意に与えられた$0$でない順序数として,
				\begin{align}
					\forall \gamma \in \beta\, \left(\, A_\alpha(\gamma) = A_{\alpha'}(\gamma)\, \right)
				\end{align}
				が成り立っているとする.このとき
				\begin{align}
					\beta = \gamma \cup \{\gamma\}
				\end{align}
				なる順序数$\gamma$が取れるなら
				\begin{align}
					A_\alpha(\beta) = A_\alpha(\gamma) \cup \left\{A_\alpha(\gamma)\right\}
				\end{align}
				かつ
				\begin{align}
					A_{\alpha'}(\beta) = A_{\alpha'}(\gamma) \cup \left\{A_{\alpha'}(\gamma)\right\}
				\end{align}
				が成立し,いま
				\begin{align}
					A_\alpha(\gamma) = A_{\alpha'}(\gamma)
				\end{align}
				なので
				\begin{align}
					A_\alpha(\beta) = A_{\alpha'}(\beta)
				\end{align}
				が従う.
				\begin{align}
					\limo{\beta}
				\end{align}
				であるときは
				\begin{align}
					A_\alpha(\beta) = \bigcup \Set{A_\alpha(\gamma)}{\gamma \in \beta}
				\end{align}
				かつ
				\begin{align}
					A_{\alpha'}(\beta) = \bigcup \Set{A_{\alpha'}(\gamma)}{\gamma \in \beta}
				\end{align}
				が成立して
				\begin{align}
					A_\alpha(\beta) = A_{\alpha'}(\beta)
				\end{align}
				が従う.ゆえに超限帰納法より(\refeq{form:thm_summation_on_ordinal_numbers_is_a_mapping_2})が従う.
				
			\item[第三段] $\ran{+} \subset \ON$であることを示す.
				$y$を$\ran{+}$の要素とする.このとき
				\begin{align}
					(x,y) \in +
				\end{align}
				を満たす集合$x$が取れて,
				\begin{align}
					x = (\alpha,\beta)
				\end{align}
				を満たす順序数$\alpha$と$\beta$が取れて
				\begin{align}
					y = A_\alpha(\beta)
				\end{align}
				が成り立つ.ゆえに,
				\begin{align}
					\forall \gamma \in \ON\, \left(\, A_\alpha(\gamma) \in \ON\, \right)
					\label{form:thm_summation_on_ordinal_numbers_is_a_mapping_3}
				\end{align}
				なることを示せば良い.これは超限帰納法により示す.まず
				\begin{align}
					A_\alpha(\emptyset) = \alpha
				\end{align}
				より
				\begin{align}
					A_\alpha(\emptyset) \in \ON
				\end{align}
				が成り立つ.次に$\gamma$を任意に与えられた$0$でない順序数として,
				\begin{align}
					\forall \delta \in \gamma\, \left(\, A_\alpha(\delta) \in \ON\, \right)
				\end{align}
				が成り立っているとする.このとき
				\begin{align}
					\gamma = \delta \cup \{\delta\}
				\end{align}
				なる順序数$\delta$が取れるなら
				\begin{align}
					A_\alpha(\gamma) = A_\alpha(\delta) \cup \left\{A_\alpha(\delta)\right\}
				\end{align}
				が成立し,定理\ref{thm:latter_element_is_ordinal}より
				\begin{align}
					A_\alpha(\gamma) \in \ON
				\end{align}
				が従う.
				\begin{align}
					\limo{\gamma}
				\end{align}
				であるときは
				\begin{align}
					A_\alpha(\gamma) = \bigcup \Set{A_\alpha(\delta)}{\delta \in \gamma}
				\end{align}
				が成立して,定理\ref{thm:union_of_set_of_ordinal_numbers_is_ordinal}より
				\begin{align}
					A_\alpha(\gamma) \in \ON
				\end{align}
				が従う.ゆえに超限帰納法より(\refeq{form:thm_summation_on_ordinal_numbers_is_a_mapping_3})が従う.
				ゆえに
				\begin{align}
					y \in \ON
				\end{align}
				が成り立ち
				\begin{align}
					\ran{+} \subset \ON
				\end{align}
				が得られた.
				
			\item[第四段] $\ran{+} = \ON$を示す.実際,$\alpha$を順序数とすれば
				\begin{align}
					\alpha = +(\alpha,0)
				\end{align}
				が成り立つから
				\begin{align}
					\alpha \in \ran{+}
				\end{align}
				が成り立ち
				\begin{align}
					\ON \subset \ran{+}
				\end{align}
				が従う.前段の結果と併せて
				\begin{align}
					\ran{+} = \ON
				\end{align}
				を得る.
				\QED
		\end{description}
	\end{sketch}
	
	\begin{itembox}[l]{中置記法}
		$\alpha$と$\beta$を順序数とするとき
		\begin{align}
			+(\alpha,\beta)
		\end{align}
		は
		\begin{align}
			\alpha + \beta
		\end{align}
		とも表記される.このような書き方を{\bf 中置記法}\index{ちゅうちきほう@中置記法}{\bf (infix notation)}と呼ぶ.
		また$\alpha + \beta$なる順序数を$\alpha$と$\beta$の{\bf 和}\index{わ@和}{\bf (sum)}と呼ぶ.
	\end{itembox}
	
	\begin{screen}
		\begin{thm}[順序数は$0$を足しても$0$に足しても変わらない]
			\begin{align}
				\forall \alpha \in \ON\, \left(\, \alpha + 0 = 0 + \alpha = \alpha\, \right).
			\end{align}
		\end{thm}
	\end{screen}
	
	\begin{screen}
		\begin{thm}[後者は$1$を足したもの]
			\begin{align}
				\forall \alpha \in \ON\, \left(\, \alpha + 1 = \alpha \cup \{\alpha\}\, \right).
			\end{align}
		\end{thm}
	\end{screen}
	
	\begin{screen}
		\begin{thm}[加法は結合的]
			\begin{align}
				\forall \alpha,\beta,\gamma \in \ON\, \left[\, (\alpha + \beta) + \gamma = \alpha + (\beta + \gamma)\, \right].
			\end{align}
		\end{thm}
	\end{screen}
	
	\begin{screen}
		\begin{thm}[和は順序を保存する]
			\begin{align}
				\forall \alpha,\beta,\gamma \in \ON\, (\, \beta < \gamma
				\Longrightarrow \alpha + \beta < \alpha + \gamma\, ).
			\end{align}
		\end{thm}
	\end{screen}
	
	\begin{screen}
		\begin{thm}[大小関係を埋め合わせる順序数が取れる]
			\begin{align}
				\forall \alpha,\beta \in \ON\, \left[\, \alpha < \beta
				\Longrightarrow \exists \gamma \in \ON\, (\, \gamma < \beta \wedge \alpha + \gamma = \beta\, )\, \right].
			\end{align}
		\end{thm}
	\end{screen}
	
	\begin{screen}
		\begin{thm}[自然数の和は自然数]\label{thm:sum_of_natural_numbers_is_a_natural_number}
			\begin{align}
				\forall n, m \in \Natural\, (\, n + m \in \Natural\, ).
			\end{align}
		\end{thm}
	\end{screen}
	
	\begin{sketch}
		$n$を自然数とする.そして
		\begin{align}
			a \defeq \Set{x}{x \in \Natural \wedge n + x \in \Natural}
		\end{align}
		とおく.このとき,
		\begin{align}
			n + 0 = n
		\end{align}
		より
		\begin{align}
			0 \in a
		\end{align}
		が成り立ち,また$m$を$a$の要素とするとき,
		\begin{align}
			n + m \in \Natural
		\end{align}
		が成り立つので
		\begin{align}
			(n + m) + 1 \in \Natural
		\end{align}
		が成り立ち,また
		\begin{align}
			n + (m + 1) = (n + m) + 1
		\end{align}
		と
		\begin{align}
			m + 1 \in \Natural
		\end{align}
		も成り立つから
		\begin{align}
			m + 1 \in a
		\end{align}
		が成立する.ゆえに
		\begin{align}
			\emptyset \in a \wedge \forall m\, \left(\, m \in a \Longrightarrow m \cup \{m\} \in a\, \right)
		\end{align}
		が成り立つので,定理\ref{thm:the_principle_of_mathematical_induction}より
		\begin{align}
			\Natural \subset a
		\end{align}
		が従う.ゆえに,$m$を自然数とすれば
		\begin{align}
			n + m \in \Natural
		\end{align}
		となる.
		\QED
	\end{sketch}
	
	\begin{screen}
		\begin{thm}[自然数の和は可換]
			\begin{align}
				\forall n,m \in \Natural\ (\, n + m = m + n\, ).
			\end{align}
		\end{thm}
	\end{screen}
本稿では
\begin{align}
	\omg \subset \Z \subset \Q \subset \R \subset \C
\end{align}
となるように構成する.当然,加減乗除も$\ON$に定めた加法と乗法の純粋な延長として定めていく.
`埋め込めば拡張となる'ように数を構成している文献もあるが,それでは詰めが甘くもどかしい.

\section{加法}
	$\alpha$と$\beta$を順序数とするとき,$\alpha$に$\beta$を``足す''という操作を導入したい.つまり,
	足し算の記号
	\begin{align}
		+
	\end{align}
	を何らかの意味で定めて
	\begin{align}
		\alpha + \beta
	\end{align}
	を実行したいのである.先ずは簡単に,$\beta$が$0$の場合は
	\begin{align}
		\alpha + 0 \defeq \alpha
	\end{align}
	と定めてしまう.$\beta$が$1$の場合は,ちょうど
	\begin{align}
		\alpha \cup \{\alpha\}
	\end{align}
	が$\alpha$の直後の元であったからこれを$\alpha + 1$を定めることにする.
	$\alpha + 2$も$\alpha + 1$の直後の元として
	\begin{align}
		\alpha + 2 \defeq (\alpha + 1) + 1
	\end{align}
	で定めることにして,この調子で
	\begin{align}
		&\alpha + 3 \defeq (\alpha + 2) + 1 \\
		&\alpha + 4 \defeq (\alpha + 3) + 1 \\
		&\alpha + 5 \defeq (\alpha + 4) + 1
	\end{align}
	としていくわけであるが,例えば$\beta$が$\Natural$である場合,$\Natural$が極限数であるから上の操作をいくら続けても
	\begin{align}
		\alpha + \Natural
	\end{align}
	に到達することは不可能であるが,仕方が無いから一番自然な方法として
	\begin{align}
		\Set{\alpha + k}{k \in \Natural}
	\end{align}
	の上限を$\alpha + \Natural$と定める.以上の操作をヒントにして,$\alpha + \beta$は
	\begin{itemize}
		\item $\beta$に対して$\beta = \gamma + 1$を満たす順序数$\gamma$が取れるなら
			\begin{align}
				\alpha + \beta \defeq (\alpha + \gamma) + 1,
			\end{align}
		
		\item $\beta$が極限数なら
			\begin{align}
				\alpha + \beta \defeq \bigcup \Set{\alpha + \gamma}{\gamma \in \beta},
			\end{align}
	\end{itemize}
	で定められる.意味的には再帰操作を繰り返しているのだから,それを$\mathcal{L}'$のことばで表すためには
	\ref{sec:recursive_definition}節の方法を応用すれば良い.
	また次節で述べることであるが``足し算''と同様にすれば$\ON$上に``掛け算''も定めることが出来る.
	それぞれ主に``加法''と``乗法''と呼ばれ,これらの延長が複素数に対する通常の四則演算に発展していく.
	
	\begin{screen}
		\begin{dfn}[順序数の加法]
			$\alpha$を順序数とし,$\Univ$上の写像$G_\alpha$を
			\begin{align}
				x \longmapsto 
				\begin{cases}
					\alpha & \mbox{if } \operatorname{dom}(x) = \emptyset \\
					x(\beta) \cup \{x(\beta)\} & \mbox{if } \beta \in \ON \wedge \operatorname{dom}(x) = \beta \cup \{\beta\} \\
					\bigcup \operatorname{ran}(x) & \mbox{o.w.}
				\end{cases}
			\end{align}
			なる関係により定めると,
			\begin{align}
				\forall \beta \in \ON\, (\, A_\alpha(\beta) = G_\alpha(A_\alpha|_\beta)\, )
			\end{align}
			を満たす$\ON$上の写像$A_\alpha$が取れる.ここで
			\begin{align}
				+ \defeq \Set{((\alpha,\beta),y)}{\alpha \in \ON \wedge \beta \in \ON \wedge y = A_\alpha (\beta)}
			\end{align}
			により$+$を定め,これを$\ON$上の{\bf 加法}\index{かほう@加法}{\bf (summation)}と呼ぶ.
		\end{dfn}
	\end{screen}
	
	$\alpha$を順序数とすれば,$G_\alpha$とは正式には
	\begin{align}
		\{\, (x,y) \mid \quad &\left(\, \dom{x} = \emptyset \Longrightarrow y = \alpha\, \right) \\
		&\wedge \forall \beta \in \ON\, \left(\, \dom{x} = \beta \cup \{\beta\} \Longrightarrow y = x(\beta) \cup \{x(\beta)\}\, \right) \\
		&\wedge \left[\, \dom{x} \neq \emptyset \wedge \forall \beta \in \ON\, \left(\, \dom{x} \neq \beta \cup \{\beta\}\, \right)
		\Longrightarrow y = \ran{x}\, \right]\, \}
	\end{align}
	によって定められた写像である.そして$A_\alpha$とは
	\begin{align}
		\ON \ni \beta \longmapsto
		\begin{cases}
			\alpha & \mbox{if } \beta = \emptyset \\
			(\alpha + \gamma) \cup \{\alpha + \gamma\} & \mbox{if } \gamma \in \ON \wedge \beta = \gamma \cup \{\gamma\} \\
			\bigcup \Set{\alpha + \gamma}{\gamma \in \beta} & \mbox{if } \limo{\beta}
		\end{cases}
	\end{align}
	を満たす写像である.
	
	\begin{screen}
		\begin{thm}[$+$は写像である]
			次が成り立つ.
			\begin{align}
				+:\ON \times \ON \longrightarrow \ON.
			\end{align}
		\end{thm}
	\end{screen}
	
	\begin{sketch}
		
	\end{sketch}
	
	$\alpha$と$\beta$を順序数とするとき
	\begin{align}
		+(\alpha,\beta)
	\end{align}
	は
	\begin{align}
		\alpha + \beta
	\end{align}
	とも表記される.このような書き方を{\bf 中置記法}\index{ちゅうちきほう@中置記法}{\bf (infix notation)}と呼ぶ.
	また$\alpha + \beta$なる順序数を$\alpha$と$\beta$の{\bf 和}\index{わ@和}{\bf (sum)}とも呼ぶ.
	
	\begin{screen}
		\begin{thm}[自然数の和は自然数]\label{thm:the_definition_of_addition_of_ordinal_numbers}
			次が成立する:
			\begin{itemize}
				\item $\forall \alpha,\alpha' \in \ON\, \left(\, \alpha = \alpha' \Longrightarrow A_\alpha = A_{\alpha'}\, \right)$.
				\item $\forall \beta \in \ON\, (\, \alpha + \beta \in \ON\, )$.
				\item $\alpha \in {\bf \omega}$のとき,$\forall \beta \in {\bf \omega}\, (\, \alpha + \beta \in {\bf \omega}\, )$.
			\end{itemize}
		\end{thm}
	\end{screen}
	
	\begin{prf}
		いま$\beta$を任意に与えられた順序数とする.このとき,
		\begin{align}
			\forall \gamma \in \beta\ (\ \alpha + \gamma \in \ON\ )
		\end{align}
		が成り立っていると仮定すると,$\beta = \gamma + 1$と表せるとき
		\begin{align}
			\alpha + \beta 
			= G_\alpha (F_\alpha|_\beta)
			= F_\alpha(\gamma) + 1
			= (\alpha + \gamma) + 1 \in \ON
		\end{align}
		となり,$\beta$が極限数のときは
		\begin{align}
			\alpha + \beta = \operatorname*{sup}_{\gamma \in \beta} (\alpha + \gamma)
			= \bigcup \Set{\alpha + \gamma}{\gamma \in \beta}
			\in \ON
		\end{align}
		となるので,
		\begin{align}
			\forall \beta \in \ON\ \left(\ \forall \gamma \in \beta\ (\ \alpha + \gamma \in \ON\ ) \Longrightarrow \alpha + \beta \in \ON\ \right)
		\end{align}
		が得られた.超限帰納法により
		\begin{align}
			\forall \beta \in \ON\ (\ \alpha + \beta \in \ON\ )
		\end{align}
		が成立する.また$\alpha \in {\bf \omega}$のとき,
		\begin{align}
			a = \Set{\beta \in {\bf \omega}}{\alpha + \beta \in {\bf \omega}}
		\end{align}
		とおけば
		\begin{align}
			\emptyset \in a \wedge \forall x\ (\ x \in a \Longrightarrow x \cup \{x\} \in a\ )
		\end{align}
		となるので${\bf \omega} \subset a$が従う.よって
		\begin{align}
			\forall \beta \in {\bf \omega}\ 
			(\ \alpha + \beta \in {\bf \omega}\ )
		\end{align}
		も成り立つ.
		\QED
	\end{prf}
	
	\begin{screen}
		\begin{thm}[加法の性質]
		\label{thm:properties_of_addition_of_ordinal_numbers}
			定理\ref{thm:the_definition_of_addition_of_ordinal_numbers}で定めた
			加法は以下の性質を持つ:
			\begin{itemize}
				\item $\forall \alpha \in \ON\ (\ \alpha + 0 = 0 + \alpha = \alpha\ )$,
				
				\item $\forall \alpha \in \ON\ (\ \alpha + 1 = \alpha \cup \{\alpha\}\ )$,
				
				\item $\forall \alpha,\beta,\gamma \in \ON\ (\ (\alpha + \beta) + \gamma = \alpha + (\beta + \gamma)\ )$,
				
				\item $\forall \alpha,\beta \in {\bf \omega}\ (\ \alpha + \beta = \beta + \alpha\ )$,
				
				\item $\forall \alpha,\beta,\gamma \in \ON\ (\ \beta \in \gamma
					\Longrightarrow \alpha + \beta \in \alpha + \gamma\ )$,
				
				\item $\forall \alpha,\beta \in \beta\ (\ \alpha \in \beta
					\Longrightarrow \exists \gamma \in \ON\ (\ \alpha + \gamma = \beta\ )\ )$.
			\end{itemize}
		\end{thm}
	\end{screen}

\section{乗法}
	\begin{screen}
		\begin{thm}[順序数の乗法]
		\label{thm:the_definition_of_multiplication_of_ordinal_numbers}
			$\alpha$を$\ON$から任意に選ばれた順序数として,$\Univ$上の写像$G_\alpha$を
			\begin{align}
				G_\alpha(x) = 
				\begin{cases}
					0 & (\operatorname{dom}(x) = \emptyset) \\
					x(\beta) + \alpha & (
					\exists \beta \in \ON\ (\ \operatorname{dom}(x) = \beta \cup \{\beta\}\ )) \\
					\bigcup \operatorname{ran}(x) & \mathrm{o.w.}
				\end{cases}
			\end{align}
			で定めるとき,定理\ref{thm:transfinite_recursion_theorem}より
			\begin{align}
				\forall \beta \in \ON\ (\ M_\alpha(\beta) = G_\alpha(M_\alpha|_\beta)\ )
			\end{align}
			を満たす$\ON$上の写像$M_\alpha$が唯一つ存在する.ここで
			\begin{align}
				\alpha \cdot \beta = M_\alpha (\beta)
			\end{align}
			と書くと,次が成立する:
			\begin{itemize}
				\item $\forall \beta \in \ON\ (\ \alpha \cdot \beta \in \ON\ )$.
				\item $\alpha \in {\bf \omega}$のとき,$\forall \beta \in {\bf \omega}\ 
				(\ \alpha \cdot \beta \in {\bf \omega}\ )$.
			\end{itemize}
		\end{thm}
	\end{screen}
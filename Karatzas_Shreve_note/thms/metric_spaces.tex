\subsection{距離空間}
	\begin{screen}
		\begin{dfn}[(擬)距離関数・距離位相]
			空でない\footnotemark
			集合$S$において,
			\begin{description}
				\item[(PM1)] $d(x,x) = 0,\quad (\forall x \in S)$
				\item[(PM2)] $d(x,y) = d(y,x),\quad (\forall x,y \in S)$
				\item[(PM3)] $d(x,y) \leq d(x,z) + d(z,y),\quad (\forall x,y,z \in S)$
			\end{description}
			を満たす関数$d:S \times S \longrightarrow [0,\infty)$を
			擬距離\index{ぎきょり@擬距離}(pseudometric)と呼ぶ.これらに加えて
			\begin{itemize}
				\item $d(x,y) = 0 \Longrightarrow x=y,
				\quad (\forall x,y \in S)$
			\end{itemize}
			が満たされるとき$d$を距離\index{きょり@距離}(metric)と呼び,
			$S$と(擬)距離$d$との対$(S,d)$を(擬)距離空間と呼ぶ.また
			\begin{align}
				&\mbox{$O \subset S$が開集合である}
				\quad \overset{\mathrm{def}}{\Longleftrightarrow} \quad \\
				&\quad\mbox{$O \neq \emptyset$,或は任意の$x \in O$に対し或る$r_x > 0$が存在して
					$\Set{y \in S}{d(x,y) < r_x} \subset O$となる}
			\end{align}
			で定める開集合系を(擬)距離位相と呼ぶ.
			$d$で入れる(擬)距離位相を$d$-位相とも書く.
		\end{dfn}
	\end{screen}
	
	\begin{screen}
		距離が一様同値であることと距離から作られる近縁系が一致することは同値.
	\end{screen}
	
	\begin{itembox}[l]{同値であるが一様同値でない距離の例のコピー}
		$X=\R$, $d_1(x,y)=|x−y|$, $d_2(x,y)=|x^3−y^3|$
		because $x \longmapsto x^3$ isn't uniformly continuous on $\R$,
		these aren't uniformly equivalent; take $x=n$, $y=n+1/n$, 
		then $d_1(x,y)=1/n$, $d_2(x,y) \geq 3$, so for $\epsilon=3$, 
		every $\delta$ fails by taking $1/n<\delta$. 
		However, essentially because $x \longmapsto x^3$ is continuous, 
		they do generate the same topology.
		
		???
		https://math.stackexchange.com/questions/793816/example-of-metrics-that-generating-the-same-topology-but-not-uniformly-equivalen
	\end{itembox}
	
	
	\footnotetext{$S$が空集合である場合,$S \times S$の上で定義し得る写像は
	空写像のみである.空写像は距離の定義を満たす.}
	
	\begin{screen}
		\begin{dfn}[球]
			$(S,d)$を擬距離空間とするとき,$x \in S$と$r > 0$により
			\begin{align}
				\Set{y \in S}{d(x,y) < r}
			\end{align}
			で表される集合を(中心$x$,半径$r$の)開球\index{かいきゅう@開球}(open ball)と呼ぶ.
			$<$を$\leq$に替えたものは閉球\index{へいきゅう@閉球}(closed ball)と呼ぶ.
		\end{dfn}
	\end{screen}
	
	\begin{screen}
		\begin{thm}[開球・閉球はそれぞれ開集合・閉集合]
		\label{thm:open_ball_is_open}
			擬距離位相空間の開球は開集合,閉球は閉集合である.
		\end{thm}
	\end{screen}
	
	\begin{prf}
		$(S,d)$を擬距離空間とすれば
		任意の$x \in S$で$d_x:S \ni y \longmapsto d(x,y)$は連続であり,
		半径$r$の開球は$d_x^{-1}([0,r))$,半径$r$の閉球は$d_x^{-1}([0,r])$
		と書けるからそれぞれ$S$の開集合,閉集合である.
		\QED
	\end{prf}
	
	\begin{screen}
		\begin{thm}[擬距離空間において開球全体は基底をなす]
		\label{thm:base_on_pseudometric_space}
			$(S,d)$を擬距離空間として擬距離位相を入れるとき,
			中心$x$半径$r$の開球を$B_r(x)$と書けば
			\begin{align}
				\mathscr{B} \coloneqq \Set{B_{1/n}(x)}{x \in S,\ n \in \N}
				\label{eq:thm_base_on_pseudometric_space_1}
			\end{align}
			は$S$の基底をなす.
			また$S$が可分であるとき,つまり$S$で稠密な高々可算集合$M$が存在するとき,
			\begin{align}
				\mathscr{B}_0 \coloneqq \Set{B_{1/n}(x)}{x \in M,\ n \in \N}
			\end{align}
			は$S$の高々可算な基底となる.すなわち可分な擬距離空間は第二可算である.
		\end{thm}
	\end{screen}
	
	\begin{prf}
		定理\ref{thm:open_ball_is_open}より任意の
		$\mathscr{U} \subset \mathscr{B}$に対し
		$\bigcup \mathscr{U}$
		は開集合となる.一方で$O$が開集合なら,任意の$x \in O$に対し
		\begin{align}
			B_{1/n_x}(x) \subset O
		\end{align}
		を満たす$n_x \in \N$が存在して
		\begin{align}
			O = \bigcup_{x \in O} B_{1/n_x}(x) \in \mathscr{B}
		\end{align}
		が従うから,$\mathscr{B}$は$S$の基底をなす.
		$S$で稠密な高々可算集合$M$が存在するとき,
		任意の$x \in S$と$n \in \N$に対し
		\begin{align}
			B_{1/(3n)}(x) \cap M \neq \emptyset
		\end{align}
		となるから,或る$m \in B_{1/(3n)}(x) \cap M$と
		$1/(3n) < 1/N < 2/(3n)$を満たす$N \in \N$で
		\begin{align}
			x \in B_{1/N}(m) \subset B_{1/n}(x),\quad B_{1/N}(m) \in \mathscr{B}_0
		\end{align}
		が成り立つ.すなわち任意の開集合は$\mathscr{B}_0$の元の合併で表せるから
		$\mathscr{B}_0$は$S$の基底となる.
		\QED
	\end{prf}
	
	\begin{screen}
		\begin{thm}[擬距離位相は第一可算]
			$(S,d)$を擬距離空間として擬距離位相を導入すれば,
			任意の$x \in S$に対して
			\begin{align}
				\left\{\Set{y \in S}{d(x,y) < \frac{1}{n}}\right\}_{n=1}^\infty
			\end{align}
			は$x$の基本近傍系となる.すなわち擬距離位相は第一可算空間を定める.
		\end{thm}
	\end{screen}
	
	\begin{prf}
		$U$を$x$を近傍とすれば或る$r > 0$で
		$\Set{y \in S}{d(x,y) < r} \subset U$となる.
		このとき$1/n < r$なら
		\begin{align}
			\Set{y \in S}{d(x,y) < \frac{1}{n}}
			\subset \Set{y \in S}{d(x,y) < r} \subset U
		\end{align}
		が成り立つ.
		\QED
	\end{prf}
	
	\begin{screen}
		\begin{thm}[擬距離関数の連続性]\label{thm:continuity_of_pseudometrics}
			$(S,d)$を擬距離空間として擬距離位相を導入するとき,以下が成り立つ:
			\begin{description}
				\item[(1)] $S \times S \ni (x,y) \longmapsto d(x,y)$は直積位相に関し連続である.
				
				\item[(2)] 任意の空でない部分集合$A$に対し
					$S \ni x \longmapsto d(x,A)$は連続である.特に$A$が閉なら
					\begin{align}
						x \in A \quad \Longleftrightarrow \quad
						d(x,A) = 0. 
					\end{align} 
			\end{description}
		\end{thm}
	\end{screen}
	
	\begin{screen}
		\begin{thm}[擬距離空間は完全正規]
			任意の擬距離位相空間は完全正規である.特に以下は全て同値である:
			\begin{description}
				\item[(a)] 擬距離が距離である.
				\item[(b)] 擬距離位相が$T_0$である.
				\item[(c)] 擬距離位相が$T_6$である.
			\end{description}
		\end{thm}
	\end{screen}
	
	\begin{prf}\mbox{}
		\begin{description}
			\item[第一段]
				$(S,d)$を擬距離空間として擬距離位相を入れるとき,$A,B$を交わらない$S$の閉集合として
				\begin{align}
					f(x) \coloneqq \frac{d(x,A)}{d(x,A) + d(x,B)},
					\quad (\forall x \in S)
				\end{align}
				により$f:S \longrightarrow [0,1]$を定めれば,
				定理\ref{thm:continuity_of_pseudometrics}より$f$は連続であり
				\begin{align}
					A = f^{-1}(\{0\}),\quad B = f^{-1}(\{1\})
				\end{align}
				が満たされるから$S$は完全正規である.
				
			\item[第二段] 
				$d$が距離なら$S$はHausdorffである.
				実際,相異なる二点$x,y \in S$に対し
				\begin{align}
					B_\epsilon(x) \coloneqq \Set{s \in S}{d(s,x) < \frac{\epsilon}{2}},
					\quad B_\epsilon(y) \coloneqq \Set{s \in S}{d(s,y) < \frac{\epsilon}{2}},
					\quad (\epsilon \coloneqq d(x,y))
				\end{align}
				で交わらない開球を定めれば,$x$と$y$は
				$B_\epsilon(x)$と$B_\epsilon(y)$で分離される.
				従って距離位相は完全正規Hausdorffとなり$(a) \Longrightarrow (c)$を得る.
				
			\item[第三段]
				位相空間が$T_6$なら$T_0$であるから$(c) \Longrightarrow (b)$が従う.
				また$S$が$T_0$であるとき,相異なる二点$x,y$に対し
				$x \notin \overline{\{y\}}$又は$y \notin \overline{\{x\}}$となるが,
				$x \notin \overline{\{y\}}$とすれば
				$\overline{\{y\}} \subset S \backslash B_r(x)$
				を満たす$r > 0$が存在し,$d(x,y) \geq r > 0$が成り立つから
				$d$は距離となる.これにより$(b) \Longrightarrow (a)$が出る.
				\QED
		\end{description}
	\end{prf}
	
	\begin{screen}
		\begin{thm}[距離空間の部分空間の距離]
			$(S,d)$を距離空間,$M$を$S$の空でない部分集合とし,
			$S$に距離位相を入れる.このとき$M$の相対位相$\mathscr{O}_M$は
			\begin{align}
				d_M(x,y) \coloneqq d(x,y),
				\quad (\forall x,y \in M)
			\end{align}
			で定める相対距離により導入する距離位相$\mathscr{O}_{d_M}$と一致する.
		\end{thm}
	\end{screen}
	
	\begin{prf} 任意の$x \in M$と$r > 0$に対し
		\begin{align}
			\Set{y \in M}{d_M(x,y) < r}
			= M \cap \Set{y \in S}{d(x,y) < r}
		\end{align}
		が成り立つから,相対開集合は$d_M$-開球の合併で表され,
		逆に$d_M$-開集合は$M$と$d$-開集合の交叉で表せる.
		\QED
	\end{prf}
	
	\begin{screen}
		\begin{thm}[距離空間のCartesian積の距離]
			$(S_1,d_1)$と$(S_2,d_2)$を距離空間としてそれぞれに距離位相を導入し,
			$(S,\mathscr{O}_S)$をその直積位相空間とする.このとき
			\begin{align}
				S \times S \ni ((x_1,x_2),(y_1,y_2)) \longmapsto
				\sqrt{d_1(x_1,y_1)^2 + d_2(x_2,y_2)^2} 
			\end{align}
			なる関係で定める写像を$d$とすれば,$d$は$S$上の距離となり,
			$S$は$d$で距離化可能である.特に$(S_1,d_1)$と$(S_2,d_2)$が共に完備(resp. 可分)なら
			$(S,d)$も完備(resp. 可分)となる.
		\end{thm}
	\end{screen}
	
	\begin{screen}
		\begin{thm}[距離空間の可算直積の距離]
			$\{(S_n,d_n)\}_{n \in \Natural}$を距離空間の族として距離位相を導入し,
			$(S,\mathscr{O}_S)$をその直積位相空間とする.このとき
			\begin{align}
				S \times S \ni (x,y) \longmapsto \sum_{n=0}^\infty 2^{-n}\left(d_n(x_n,y_n) \wedge 1\right)
			\end{align}
			なる関係で定める写像を$d$とすれば,$d$は$S$上の距離となり,
			$S$は$d$で距離化可能である.特に全ての$n$について$(S_n,d_n)$が完備(resp. 可分)なら
			$(S,d)$も完備(resp. 可分)となる.
		\end{thm}
	\end{screen}
	
	\begin{screen}
		\begin{thm}[擬距離の距離化]
			$(S,d)$を擬距離空間とするとき,
			$x \sim y \overset{\mathrm{def}}{\Longleftrightarrow} d(x,y) = 0$
			で$S$に同値関係が定まる.また商写像を$\pi:S \longrightarrow S/\sim$と書けば
			\begin{align}
				\rho(\pi(x),\pi(y)) \coloneqq d(x,y),
				\quad (\forall \pi(x),\pi(y) \in S/\sim)
			\end{align}
			により$S/\sim$に距離$\rho$が定まり,$S$の$d$-位相の商位相は
			$S/\sim$の$\rho$-位相に一致する.
		\end{thm}
	\end{screen}
	
	\begin{prf}\mbox{}
		\begin{description}
			\item[第一段] $\rho$がwell-definedであることを示す.実際,
				$\pi(x) = \pi(x'),\ \pi(y) = \pi(y')$ならば
				\begin{align}
					d(x',y') &\leq d(x',x) + d(x,y) + d(y,y') = d(x,y)
					\leq d(x,x') + d(x',y') + d(y',y) = d(x',y')
				\end{align}
				より$d(x,y) = d(x',y')$が成り立つから
				$\rho(\pi(x),\pi(y)) = \rho(\pi(x'),\pi(y'))$が満たされる.
				
			\item[第二段] $d$-開球と$\rho$-開球をそれぞれ$B_d(x;r),\ B_\rho(\pi(x);r),
				\ (x \in S,\ r>0)$と書けば
				\begin{align}
					\pi\left(B_d(x;r)\right) = B_\rho(\pi(x);r),
					\quad B_d(x;r) = \pi^{-1}\left(B_\rho(\pi(x);r)\right)
				\end{align}
				が成り立つ.$U$を商位相の空でない開集合とすれば,
				定理\ref{thm:base_on_pseudometric_space}と
				定理\ref{projective_injective_image_preimage}より
				\begin{align}
					U = \pi\left(\pi^{-1}(U)\right)
					= \pi\Biggl(\bigcup_{x \in \pi^{-1}(U)}B_d(x;r_x)\Biggr)
					= \bigcup_{x \in \pi^{-1}(U)} B_\rho(\pi(x);r_x)
				\end{align}
				となるから$U$は$\rho$-開集合でもある.
				同様に$V$を空でない$\rho$-開集合とすれば
				\begin{align}
					\pi^{-1}(V)
					= \pi^{-1}\Biggl(\bigcup_{\pi(x) \in V} B_\rho\left(\pi(x);\epsilon_{\pi(x)}\right)\Biggr)
					= \bigcup_{\pi(x) \in V} B_d\left(x;\epsilon_{\pi(x)}\right)
				\end{align}
				が成り立つから$V$は商空間の開集合でもある.
				従って商位相と$\rho$-位相は一致する.
				\QED
		\end{description}
	\end{prf}
	
	\begin{screen}
		\begin{dfn}[距離化可能]
			位相空間において,その位相と一致する距離位相を定める距離が存在するとき,
			その空間は距離化可能(metrizable)であるという.
		\end{dfn}
	\end{screen}
	
	\begin{screen}
		\begin{thm}[連続単射な開写像による距離化可能性の遺伝]\label{thm:heredity_of_metrizability}
			$X,Y$を位相空間,$f:X \longrightarrow Y$を連続単射な開写像とする.
			$X$が距離$d_X$で距離化可能なら,$f(X)$の相対位相は次で定める$d_Y$により距離化可能である:
			\begin{align}
				d_Y\left(f(x),f(y)\right) = d_X(x,y),
				\quad (\forall x,y \in X).
				\label{eq:thm_heredity_of_metrizability}
			\end{align}
			逆に$f(X)$の相対位相が或る距離$d_Y$で距離化可能であるとき,
			(\refeq{eq:thm_heredity_of_metrizability})で定める$d_X$により$X$は距離化可能である.
		\end{thm}
	\end{screen}
	
	\begin{prf}
		$X$に距離$d_X$が定まっているとき,或は$f(X)$に距離$d_Y$が定まっているとき,
		(\refeq{eq:thm_heredity_of_metrizability})で$d_Y$或は$d_X$を定めれば
		いずれも距離となる.このとき任意の$f(x_0) = y_0$と$r > 0$に対し
		\begin{align}
			B^X_r(x_0) \coloneqq \Set{x \in X}{d_X(x_0,x) < r},
			\quad B^Y_r(y_0) \coloneqq \Set{y \in f(X)}{d_Y(y_0,y) < r}
		\end{align}
		とおけば
		\begin{align}
			f\left(B^X_r(x_0)\right) = B^Y_r(y_0),
			\quad B^X_r(x_0) = f^{-1}\left(B^Y_r(y_0)\right)
			\label{eq:thm_heredity_of_metrizability_2}
		\end{align}
		が成立する.$X$が距離化可能であるとき,$U$を$f(X)$の相対開集合とすれば
		$f^{-1}(U)$は$X$の開集合であるから
		\begin{align}
			f^{-1}(U) = \bigcup_{x \in f^{-1}(U)}B^X_{r_x}(x)
		\end{align}
		と表され,定理\ref{projective_injective_image_preimage}と
		(\refeq{eq:thm_heredity_of_metrizability_2})より
		\begin{align}
			U = f\Biggl(\bigcup_{x \in f^{-1}(U)}B^X_{r_x}(x)\Biggr)
			= \bigcup_{x \in f^{-1}(U)}B^Y_{r_x}(f(x))
		\end{align}
		となるから$U$は$d_Y$による距離位相の開集合である.逆に$V$を$d_Y$による距離位相の開集合とすれば,
		(\refeq{eq:thm_heredity_of_metrizability_2})より$f^{-1}(V)$は
		$d_X$による開球の和で書けるから$X$の開集合であり,$f$が全射開写像であるから
		$V = f\left(f^{-1}(V)\right)$は$f(X)$の相対開集合である.
		後半の主張は$f$の逆写像$f^{-1}:f(X) \longrightarrow X$に対し前半の結果を当てはめて得られる.
		\QED
	\end{prf}
	
	\begin{screen}
		\begin{thm}[距離化可能性の同値条件]\label{thm:Nagata_Smirnov_metrizability}
			任意の位相空間について以下は同値である:
			\begin{description}
				\item[(a)] 擬距離化可能である.
				\item[(b)] 正則かつ$\sigma$-局所有限な基底を持つ.
				\item[(c)] 一様化可能であり,両立する近縁系は可算な基本近縁系を持つ.
			\end{description}
			特に,次もまた同値となる:
			\begin{description}
				\item[(a')] 距離化可能である.
				\item[(b')] 正則Hausdorff
					($T_3$)かつ$\sigma$-局所有限な基底を持つ.
				\item[(c')] Hausdorff一様化可能であり,両立する近縁系は可算な基本近縁系を持つ.
			\end{description}
		\end{thm}
	\end{screen}
	
	\begin{prf} $X$を位相空間とする.\mbox{}
		\begin{description}
			\item[第一段] $X$が$T_3$かつ$\sigma$-局所有限な基底を持てば
				$X$は$T_4$かつ全ての閉集合は$G_\delta$となる.
				$X$は局所有限な部分集合族$\mathscr{B}_n$の合併
				$\bigcup_{n=1}^\infty \mathscr{B}_n$で表せる基底を有する.
				任意の$n \in \Z_+$及び$B \in \mathscr{B}_n$に対し
				\begin{align}
					\begin{cases}
						f_{n,B}(x) > 0, & (x \in B), \\
						f_{n,B}(x) = 0, & (x \in X \backslash B)
					\end{cases}
				\end{align}
		\end{description}
		を満たす連続写像$f_{n,B}:X \longrightarrow [0,1/n]$が存在する.
		\begin{align}
			J \coloneqq \Set{(n,B)}{n \in \Z_+,\ B \in \mathscr{B}_n}
		\end{align}
		とおいて
		\begin{align}
			F(x) \coloneqq \left(f_{n,B}(x)\right)_{(n,B) \in J}
		\end{align}
		により$F$を定めれば,$F$は単射となる.また
		\begin{align}
			d\left((x_j)_{j \in J},(y_j)_{j \in J}\right)
			\coloneqq \sup{j \in J}{|x_j - y_j|}
		\end{align}
		により$[0,1]^J$に距離を定めれば,
		$F$は$X$から$[0,1]^J$への開写像かつ連続写像となる.
		任意の$x_0 \in X$と$\epsilon > 0$に対して
		\begin{align}
			x \in W \quad \Longrightarrow \quad d(F(x),F(x_0)) < \epsilon
			\label{eq:thm_Nagata_Smirnov_metrizability_1}
		\end{align}
		を満たす$x_0$の開近傍$W$が存在する.$n$を固定すれば
		或る$x_0$の近傍$U_n$は$\mathscr{B}_n$の高々有限個の元としか交わらない.
		従って或る開近傍$V_n \subset U_n$が存在し,すべての$B \in \mathscr{B}_n$で
		\begin{align}
			x \in V_n \quad \Longrightarrow \quad d(F(x),F(x_0)) < \epsilon
			\label{eq:thm_Nagata_Smirnov_metrizability_2}
		\end{align}
		が満たされる,$1/N < \epsilon/2$を満たす$N \in \Z_+$を取り
		\begin{align}
			W \coloneqq V_1 \cap \cdots \cap U_N
		\end{align}
		とおけば,$n \leq N$の場合任意の$B \in \mathscr{B}_n$で
		(\refeq{eq:thm_Nagata_Smirnov_metrizability_2})が成立し,
		$n > N$の場合任意の$B \in \mathscr{B}_n$で
		\begin{align}
			d(F(x),F(x_0)) \leq \frac{2}{n} < \epsilon, \quad (\forall x \in X)
		\end{align}
		となるから(\refeq{eq:thm_Nagata_Smirnov_metrizability_1})が成り立つ.
	\end{prf}
	
\subsection{範疇定理}
	\begin{screen}
		\begin{thm}[Cantorの共通部分定理]\label{thm:Cantor_intersection_theorem}
			$(S,\mathscr{O}_S)$をHausdorff空間とし,
			$\mathcal{K}$を$S$のコンパクト部分集合から成る集合とする.
			このとき,$\mathcal{K}$の空でない任意の有限部分集合$U$
			に対して$\bigcap U \neq \emptyset$が成り立つなら
			$\bigcap \mathcal{K} \neq \emptyset$が成り立つ.つまり
			\begin{align}
				\forall U\, \left(\, U \subset \mathcal{K} \wedge \exists n \in \Natural\, (\, U \eqp n\, )
				\Longrightarrow \bigcap U \neq \emptyset\, \right)
				\Longrightarrow \bigcap \mathcal{K} \neq \emptyset.
			\end{align}
		\end{thm}
	\end{screen}
	
	\begin{prf}
		いま
		\begin{align}
			\bigcap \mathcal{K} = \emptyset
		\end{align}
		が成り立っているとする.このとき$K$を$\mathcal{K}$の要素とすれば
		\begin{align}
			K \subset \bigcup_{C \in \mathcal{K}} S \backslash C
		\end{align}
		となる.Hausdorff性より
		\begin{align}
			C \in \mathcal{K} \Longrightarrow S \backslash C \in \mathscr{O}_S
		\end{align}
		が成り立つので,$K$のコンパクト性より
		\begin{align}
			K \subset \bigcup_{C \in \mathcal{K}'} S \backslash C
		\end{align}
		を満たす$\mathcal{K}$の或る有限部分集合$\mathcal{K}'$が取れる.
		\begin{align}
			\mathcal{K}'' \defeq \mathcal{K}' \cup \{K\}
		\end{align}
		とおけば$\mathcal{K}''$は$\mathcal{K}$の有限部分集合であり,また
		\begin{align}
			\bigcap \mathcal{K}'' = K \cap \bigcap \mathcal{K}' = \emptyset
		\end{align}
		を満たす.
		\QED
	\end{prf}
	
	\begin{screen}
		\begin{dfn}[疎集合・第一類集合・第二類集合]
			位相空間$S$の部分集合$A$が疎である(nowhere dense)とは
			$A$の閉包の内核が$\overline{A}^{\mathrm{o}} = \emptyset$を満たすことをいう.
			$S$が可算個の疎集合の合併で表せるとき$S$を第一類集合(a set of the first category)と呼び,
			そうでない場合はこれを第二類集合と呼ぶ.
		\end{dfn}
	\end{screen}
	
	\begin{screen}
		\begin{thm}[Baireの範疇定理]\label{thm:Baire_category_theorem}
			空でない完備距離空間と局所コンパクトHausdorff空間は第二類集合である.
		\end{thm}
	\end{screen}
	
	\begin{prf} $S \neq \emptyset$を完備距離空間,或は局所コンパクトHausdorff空間とする.\mbox{}
		\begin{description}
			\item[第一段]
				$(V_n)_{n=1}^\infty$を$S$で稠密な開集合系とするとき
				\begin{align}
					\overline{\bigcap_{n=1}^\infty V_n} = S,
					\label{eq:thm_Baire_category_theorem_1}
				\end{align}
				となることを示す.実際(\refeq{eq:thm_Baire_category_theorem_1})が満たされていれば,
				任意の疎集合系$(E_n)_{n=1}^\infty$に対して
				\begin{align}
					V_n \coloneqq \overline{E_n}^c,
					\quad n=1,2,\cdots
				\end{align}
				で開集合系$(V_n)$を定めると定理\ref{thm:topology_note_closure_interior}より
				\begin{align}
					\overline{V_n} = \overline{E_n}^{ca} = \overline{E_n}^{ic} = \emptyset^c = S
				\end{align}
				となるから,$\bigcap_{n=1}^\infty V_n \neq \emptyset$が従い
				$S \neq \bigcup_{n=1}^\infty \overline{E_n} \supset \bigcup_{n=1}^\infty E_n$
				が成り立つ.従って$S$は第二類である.
				
			\item[第二段]
				任意の空でない開集合$B_0$に対し$B_0 \cap \left( \bigcap_{n=1}^\infty V_n \right) \neq \emptyset$
				となることを示せば(\refeq{eq:thm_Baire_category_theorem_1})が従う.
				$V_1$は稠密であるから$B_0 \cap V_1 \neq \emptyset$となり,
				点$x_1 \in B_0 \cap V_1$を取れば,
				$S$が距離空間なら或る半径$<1$の開球$B_1$が存在して
				\begin{align}
					x_1 \in B_1 \subset \overline{B_1} \subset B_0 \cap V_1
					\label{eq:thm_Baire_category_theorem_2}
				\end{align}
				を満たす.$S$が局所コンパクトHausdorffの場合も,
				定理\ref{thm:each_point_in_regular_space_has_closesd_local_base}と
				定理\ref{thm:T_2_equals_to_T_3_in_locally_compact_spaces}より
				(\refeq{eq:thm_Baire_category_theorem_2})を満たす
				相対コンパクトな開集合$B_1$が取れる.
				同様に半径$<1/n$の開球,或は相対コンパクトな開集合$B_n$と$x_n \in S$で
				\begin{align}
					x_n \in B_n \subset \overline{B_n} \subset B_{n-1} \cap V_n
				\end{align}
				を満たすものが存在する.このとき$S$が完備距離空間なら$(x_n)_{n=1}^\infty$は
				Cauchy列をなし,その極限点$x_\infty$は
				\begin{align}
					x_\infty \in \bigcap_{n=1}^\infty \overline{B_n}
				\end{align}
				を満たす.$S$が局所コンパクトHausdorff空間なら定理\ref{thm:Cantor_intersection_theorem}より
				\begin{align}
					\bigcap_{n=1}^\infty \overline{B_n} \neq \emptyset
				\end{align}
				となるから,いずれの場合も
				\begin{align}
					\emptyset \neq \bigcap_{n=1}^\infty \overline{B_n} 
					\subset B_0 \cap \Biggl( \bigcap_{n=1}^\infty V_n \Biggr)
				\end{align}
				が従い定理の主張が得られる.
				\QED
		\end{description}
	\end{prf}
	
	\begin{screen}
		\begin{lem}[同相写像に関して閉包(内部)の像は像の閉包(内部)に一致する]
		\label{lem:image_of_closure_is_closure_of_image}
			$A$を位相空間$S$の部分集合,$h:S \longrightarrow S$を同相写像とするとき
			次が成り立つ:
			\begin{description}
				\item[(1)] $h(A^a) = h(A)^a$.
				\item[(2)] $h(A^i) = h(A)^i$.
			\end{description}
		\end{lem}
	\end{screen}
	
	\begin{prf}\mbox{}
		\begin{description}
			\item[(1)]
				$h(A) \subset h(A^a)$かつ$h(A^a)$は閉であるから$h(A)^a \subset h(A^a)$が従う.一方で
				任意の$x \in h(A^a)$に対し$x = h(y)$を満たす
				$y \in A^a$と$x$の任意の近傍$V$を取れば,
				$h^{-1}(V) \cap A \neq \emptyset$より
				$V \cap h(A) \neq \emptyset$が成り立ち
				$x \in h(A)^a$となる.
				
			\item[(2)]
				$h(A^i) \subset h(A)$かつ$h(A^i)$は開であるから
				$h(A^i) \subset h(A)^i$が従う.一方で
				任意の開集合$O \subset h(A)$に対し
				$h^{-1}(O) \subset A$より
				$h^{-1}(O) \subset A^i$となり,
				$O \subset h(A^i)$が成り立つから
				$h(A)^i \subset h(A^i)$が得られる.
				\QED
		\end{description}
	\end{prf}
	
	\begin{screen}
		\begin{thm}[第一類集合の性質]
			$S$を位相空間とする.
			\begin{description}
				\item[(a)] $A \subset B \subset S$に対し$B$が第一類なら$A$も第一類である.
				\item[(b)] 第一類集合の可算和も第一類である.
				\item[(c)] 内核が空である閉集合は第一類である.
				\item[(d)] $S$から$S$への位相同型$h$と$E \subset S$に対し次が成り立つ:
					\begin{align}
						\mbox{$E$が第一類} \quad \Longleftrightarrow \quad
						\mbox{$h(E)$が第一類}.
					\end{align}
			\end{description}
		\end{thm}
	\end{screen}
	
	\begin{prf}\mbox{}
		\begin{description}
			\item[(a)] $B = \bigcup_{n=1}^\infty E_n$
				を満たす疎集合系$(E_n)_{n=1}^\infty$に対し
				$A \cap E_n$は疎であり$A = \bigcup_{n=1}^\infty (A \cap E_n)$となる.
			\item[(b)] $A_n \subset S,\ (n=1,2,\cdots)$が第一類集合とし
				$(E_{n,i})_{i=1}^\infty$を$A_n = \bigcup_{i=1}^\infty E_{n,i}$
				を満たす疎集合系とすれば
				\begin{align}
					\bigcup_{n=1}^\infty A_n
					= \bigcup_{n,i=1}^\infty E_{n,i}
				\end{align}
				が成り立つ.
				
			\item[(c)] 内核が空である閉集合はそれ自身が疎であり,自身の可算和に一致する.
			\item[(d)] $E$が第一類のとき,$E = \bigcup_{i=1}^\infty E_i$を満たす
				疎集合系$(E_i)_{i=1}^\infty$に対し
				定理\ref{thm:topology_note_closure_interior}と
				補題\ref{lem:image_of_closure_is_closure_of_image}より
				\begin{align}
					\emptyset = h(E_i^{ai})
					= h(E_i^a)^i
					= h(E_i)^{ai}
				\end{align}
				が成り立つから$h(E_i)$は疎であり,
				\begin{align}
					h(E) = \bigcup_{i=1}^\infty h(E_i)
				\end{align}
				となるから$h(E)$も第一類である.$h(E)$が第一類なら$E = h^{-1}(h(E))$も第一類である.
				\QED
		\end{description}
	\end{prf}
\subsection{距離空間}
	\begin{screen}
		\begin{dfn}[(擬)距離関数・距離位相]
			空でない集合$S$において,
			\begin{description}
				\item[(PM1)] $d(x,x) = 0,\quad (\forall x \in S)$
				\item[(PM2)] $d(x,y) = d(y,x),\quad (\forall x,y \in S)$
				\item[(PM3)] $d(x,y) \leq d(x,z) + d(z,y),\quad (\forall x,y,z \in S)$
			\end{description}
			を満たす関数$d:S \times S \longrightarrow [0,\infty)$を
			擬距離(pseudometric)と呼ぶ.これらに加えて
			\begin{description}
				\item[(PM4)] $d(x,y) = 0 \quad \Longleftrightarrow \quad x=y,
				\quad (\forall x,y \in S)$
			\end{description}
			が満たされるとき$d$を距離(metric)と呼び,
			$S$と(擬)距離$d$との対$(S,d)$を(擬)距離空間と呼ぶ.また
			\begin{align}
				&\mbox{$O \subset S$が開集合である}
				\quad \overset{\mathrm{def}}{\Longleftrightarrow} \quad \\
				&\quad\mbox{$O \neq \emptyset$,或は任意の$x \in O$に対し或る$r_x > 0$が存在して
					$\Set{y \in S}{d(x,y) < r_x} \subset O$となる}
			\end{align}
			で定める開集合系を(擬)距離位相と呼ぶ.
		\end{dfn}
	\end{screen}
	
	\begin{screen}
		\begin{thm}[擬距離位相は第一可算]
			$(S,d)$を擬距離空間として擬距離位相を導入すれば,
			任意の$x \in S$に対して
			\begin{align}
				\left\{\Set{y \in S}{d(x,y) < \frac{1}{n}}\right\}_{n=1}^\infty
			\end{align}
			は$x$の基本近傍系となる.すなわち擬距離位相は第一可算空間を定める.
		\end{thm}
	\end{screen}
	
	\begin{prf}
		$U$を$x$を近傍とすれば或る$r > 0$で
		$\Set{y \in S}{d(x,y) < r} \subset U$となる.
		このとき$1/n < r$なら
		\begin{align}
			\Set{y \in S}{d(x,y) < \frac{1}{n}}
			\subset \Set{y \in S}{d(x,y) < r} \subset U
		\end{align}
		が成り立つ.
		\QED
	\end{prf}
	
	\begin{screen}
		\begin{thm}
		\end{thm}
	\end{screen}
	
	\begin{screen}
		\begin{thm}[距離でないと分離性が成り立たない]
		\label{thm:pseudometric_is_metric_iff_T_0}
			$(S,d)$を擬距離空間として擬距離位相を入れるとき,
			\begin{align}
				\mbox{$d$が距離である} \quad \Longleftrightarrow \quad
				\mbox{$S$が$T_0$である} \quad \Longleftrightarrow \quad
				\mbox{$S$が$T_2$である}.
				\label{eq:thm_pseudometric_is_metric_iff_T_0}
			\end{align}
		\end{thm}
	\end{screen}
	
	\begin{prf}\mbox{}
		\begin{description}
			\item[第一段] 	$d$が距離なら$S$はHausdorffである.
				実際,相異なる二点$x,y \in S$に対し
				\begin{align}
					B_\epsilon(x) \coloneqq \Set{s \in S}{d(s,x) < \frac{\epsilon}{2}},
					\quad B_\epsilon(y) \coloneqq \Set{s \in S}{d(s,y) < \frac{\epsilon}{2}},
					\quad (\epsilon \coloneqq d(x,y))
				\end{align}
				で交わらない開球を定めれば,$x$と$y$は
				$B_\epsilon(x)$と$B_\epsilon(y)$で分離される.
			
			\item[第二段]
				$S$が$T_0$であるとき,相異なる二点$x,y$に対し
				$x \notin \overline{\{y\}}$或は$y \notin \overline{\{x\}}$となる.
				$x \notin \overline{\{y\}}$とすれば
				$\overline{\{y\}} \subset S \backslash B_r(x)$
				を満たす$r > 0$が存在し,$d(x,y) \geq r > 0$が成り立つから
				$d$は距離となる.
				$T_2 \Longrightarrow T_0$より
				(\refeq{eq:thm_pseudometric_is_metric_iff_T_0})を得る.
				\QED
		\end{description}
	\end{prf}
	
	\begin{screen}
		\begin{thm}[擬距離関数の連続性]\label{thm:continuity_of_pseudometrics}
			$(S,d)$を擬距離空間として擬距離位相を導入するとき,以下が成り立つ:
			\begin{description}
				\item[(1)] $S \times S \ni (x,y) \longmapsto d(x,y)$は直積位相に関し連続である.
				
				\item[(2)] 任意の空でない部分集合$A$に対し
					$S \ni x \longmapsto d(x,A)$は連続である.特に$A$が閉なら
					\begin{align}
						x \in A \quad \Longleftrightarrow \quad
						d(x,A) = 0. 
					\end{align} 
			\end{description}
		\end{thm}
	\end{screen}
	
	\begin{screen}
		\begin{thm}[擬距離空間は完全正規]
			任意の擬距離位相空間は完全正規である.特に
			\begin{align}
				\mbox{擬距離が距離である} \quad \Longleftrightarrow \quad
				\mbox{擬距離位相が$T_6$である}.
			\end{align}
		\end{thm}
	\end{screen}
	
	\begin{prf}
		$(S,d)$を擬距離空間として擬距離位相を入れるとき,$A,B$を交わらない$S$の閉集合として
		\begin{align}
			f(x) \coloneqq \frac{d(x,A)}{d(x,A) + d(x,B)},
			\quad (\forall x \in S)
		\end{align}
		により$f:S \longrightarrow \R$を定めれば,
		定理\ref{thm:continuity_of_pseudometrics}より$f$は連続であり
		\begin{align}
			A = f^{-1}(\{0\}),\quad B = f^{-1}(\{1\})
		\end{align}
		が満たされるから$S$は完全正規である.また$d$が距離であるとき,
		定理\ref{thm:pseudometric_is_metric_iff_T_0}より
		$S$はHausdorffかつ完全正規となるから$T_6$となる.
		逆に$S$が$T_6$なら$T_0$であるから$d$は距離となる.
		\QED
	\end{prf}
	
	\begin{screen}
		\begin{thm}[距離空間の部分空間の距離]
			$(S,d)$を距離空間,$M$を$S$の空でない部分集合とし,
			$S$に距離位相を入れる.このとき$M$の相対位相$\mathscr{O}_M$は
			\begin{align}
				d_M(x,y) \coloneqq d(x,y),
				\quad (\forall x,y \in M)
			\end{align}
			で定める相対距離により導入する距離位相$\mathscr{O}_{d_M}$と一致する.
		\end{thm}
	\end{screen}
	
	\begin{prf} 任意の$x \in M$と$r > 0$に対し
		\begin{align}
			\Set{y \in M}{d_M(x,y) < r}
			= M \cap \Set{y \in S}{d(x,y) < r}
		\end{align}
		が成り立つから,相対開集合は$d_M$-開球の合併で表され,
		逆に$d_M$-開集合は$M$と$d$-開集合の交叉で表せる.
		\QED
	\end{prf}
	
	\begin{screen}
		\begin{thm}[距離空間の高々可算直積の距離]
			$((S_n,d_n))_{n=1}^N$を距離空間の族として距離位相を導入し,
			$S$をその直積位相空間とする.また$x \in S$に対し
			$x(n)$を$x_n$と書く.このとき$N < \infty$なら
			\begin{align}
				d(x,y)
				\coloneqq \left\{\sum_{n=1}^N d_n(x_n,y_n)^2\right\}^{1/2},
				\quad (\forall x,y \in S)
			\end{align}
			により,$N = \infty$なら
			\begin{align}
				d(x,y) \coloneqq
				\sum_{n=1}^\infty 2^{-n}\left(d_n(x_n,y_n) \wedge 1\right),
				\quad (\forall x,y \in S)
			\end{align}
			により,$S$は距離化可能である.特に$(S_n,d_n)$が全て完備(resp. 可分)なら
			$(S,d)$も完備(resp. 可分)となる.
		\end{thm}
	\end{screen}
	
	\begin{screen}
		\begin{thm}[距離空間において可分$\Longrightarrow$第二可算]
		\end{thm}
	\end{screen}
	
	\begin{screen}
		\begin{thm}[擬距離の距離化]
			$(S,d)$を擬距離空間とするとき,
			$x \sim y \overset{\mathrm{def}}{\Longleftrightarrow} d(x,y) = 0$
			で$S$に同値関係が定まる.また商写像を$\pi:S \longrightarrow S/\sim$と書けば
			\begin{align}
				\rho(\pi(x),\pi(y)) \coloneqq d(x,y),
				\quad (\forall \pi(x),\pi(y) \in S/\sim)
			\end{align}
			により$S/\sim$に距離$\rho$が定まり,$\rho$-位相は
			$S$の$d$-位相の商位相に一致する.
		\end{thm}
	\end{screen}
	
\subsection{範疇定理}
	\begin{screen}
		\begin{thm}[Cantorの共通部分定理]\label{thm:Cantor_intersection_theorem}
			$S$をHausdorff空間とし,
			$(K_n)_{n=1}^\infty$をコンパクト部分集合の列とする.
			このとき,任意の$n \geq 1$に対して$\bigcap_{i=1}^n K_i \neq \emptyset$なら
			$\bigcap_{i=1}^\infty K_i \neq \emptyset$が成り立つ.
		\end{thm}
	\end{screen}
	
	\begin{prf}
		$\bigcap_{i=1}^\infty K_i = \emptyset$と仮定すれば,
		$K_1 \subset \bigcup_{n=1}^\infty K_n^c = S$と$K_1$のコンパクト性より
		\begin{align}
			K_1 \subset \bigcup_{n=1}^N K_n^c = \Biggl( \bigcap_{n=1}^N K_n \Biggr)^c
		\end{align}
		を満たす$N \geq 1$が存在し,$\bigcap_{n=1}^N K_n \subset K_1$より$\bigcap_{n=1}^N K_n = \emptyset$が従う.
		\QED
	\end{prf}
	
	\begin{screen}
		\begin{dfn}[疎集合・第一類集合・第二類集合]
			位相空間$S$の部分集合$A$が疎である(nowhere dense)とは
			$A$の閉包の内核が$\overline{A}^{\mathrm{o}} = \emptyset$を満たすことをいう.
			$S$が可算個の疎集合の合併で表せるとき$S$を第一類集合(the set of the first category)と呼び,
			そうでない場合はこれを第二類集合と呼ぶ.
		\end{dfn}
	\end{screen}
	
	\begin{screen}
		\begin{thm}[Baireの範疇定理]\label{thm:Baire_category_theorem}
			空でない完備距離空間と局所コンパクトHausdorff空間は第二類集合である.
		\end{thm}
	\end{screen}
	
	\begin{prf} $S \neq \emptyset$を完備距離空間,或は局所コンパクトHausdorff空間とする.\mbox{}
		\begin{description}
			\item[第一段]
				$(V_n)_{n=1}^\infty$を$S$で稠密な開集合系とするとき
				\begin{align}
					\overline{\bigcap_{n=1}^\infty V_n} = S,
					\label{eq:thm_Baire_category_theorem_1}
				\end{align}
				となることを示す.実際(\refeq{eq:thm_Baire_category_theorem_1})が満たされていれば,
				任意の疎集合系$(E_n)_{n=1}^\infty$に対して
				\begin{align}
					V_n \coloneqq \overline{E_n}^c,
					\quad n=1,2,\cdots
				\end{align}
				で開集合系$(V_n)$を定めると定理\ref{thm:topology_note_closure_interior}より
				\begin{align}
					\overline{V_n} = \overline{E_n}^{ca} = \overline{E_n}^{ic} = \emptyset^c = S
				\end{align}
				となるから,$\bigcap_{n=1}^\infty V_n \neq \emptyset$が従い
				$S \neq \bigcup_{n=1}^\infty \overline{E_n} \supset \bigcup_{n=1}^\infty E_n$
				が成り立つ.従って$S$は第二類である.
				
			\item[第二段]
				任意の空でない開集合$B_0$に対し$B_0 \cap \left( \bigcap_{n=1}^\infty V_n \right) \neq \emptyset$
				となることを示せば(\refeq{eq:thm_Baire_category_theorem_1})が従う.
				$V_1$は稠密であるから$B_0 \cap V_1 \neq \emptyset$となり,
				点$x_1 \in B_0 \cap V_1$を取れば,
				$S$が距離空間なら或る半径$<1$の開球$B_1$が存在して
				\begin{align}
					x_1 \in B_1 \subset \overline{B_1} \subset B_0 \cap V_1
					\label{eq:thm_Baire_category_theorem_2}
				\end{align}
				を満たす.$S$が局所コンパクトHausdorffの場合も,
				定理\ref{thm:each_point_in_regular_space_has_closesd_local_base}と
				定理\ref{thm:T_2_equals_to_T_3_in_locally_compact_spaces}より
				(\refeq{eq:thm_Baire_category_theorem_2})を満たす
				相対コンパクトな開集合$B_1$が取れる.
				同様に半径$<1/n$の開球,或は相対コンパクトな開集合$B_n$と$x_n \in S$で
				\begin{align}
					x_n \in B_n \subset \overline{B_n} \subset B_{n-1} \cap V_n
				\end{align}
				を満たすものが存在する.このとき$S$が完備距離空間なら$(x_n)_{n=1}^\infty$は
				Cauchy列をなし,その極限点$x_\infty$は
				\begin{align}
					x_\infty \in \bigcap_{n=1}^\infty \overline{B_n}
				\end{align}
				を満たす.$S$が局所コンパクトHausdorff空間なら定理\ref{thm:Cantor_intersection_theorem}より
				\begin{align}
					\bigcap_{n=1}^\infty \overline{B_n} \neq \emptyset
				\end{align}
				となるから,いずれの場合も
				\begin{align}
					\emptyset \neq \bigcap_{n=1}^\infty \overline{B_n} 
					\subset B_0 \cap \Biggl( \bigcap_{n=1}^\infty V_n \Biggr)
				\end{align}
				が従い定理の主張が得られる.
				\QED
		\end{description}
	\end{prf}
	
	\begin{screen}
		\begin{lem}[同相写像に関して閉包(内部)の像は像の閉包(内部)に一致する]
		\label{lem:image_of_closure_is_closure_of_image}
			$A$を位相空間$S$の部分集合,$h:S \longrightarrow S$を同相写像とするとき
			次が成り立つ:
			\begin{description}
				\item[(1)] $h(A^a) = h(A)^a$.
				\item[(2)] $h(A^i) = h(A)^i$.
			\end{description}
		\end{lem}
	\end{screen}
	
	\begin{prf}\mbox{}
		\begin{description}
			\item[(1)]
				$h(A) \subset h(A^a)$かつ$h(A^a)$は閉であるから$h(A)^a \subset h(A^a)$が従う.一方で
				任意の$x \in h(A^a)$に対し$x = h(y)$を満たす
				$y \in A^a$と$x$の任意の近傍$V$を取れば,
				$h^{-1}(V) \cap A \neq \emptyset$より
				$V \cap h(A) \neq \emptyset$が成り立ち
				$x \in h(A)^a$となる.
				
			\item[(2)]
				$h(A^i) \subset h(A)$かつ$h(A^i)$は開であるから
				$h(A^i) \subset h(A)^i$が従う.一方で
				任意の開集合$O \subset h(A)$に対し
				$h^{-1}(O) \subset A$より
				$h^{-1}(O) \subset A^i$となり,
				$O \subset h(A^i)$が成り立つから
				$h(A)^i \subset h(A^i)$が得られる.
				\QED
		\end{description}
	\end{prf}
	
	\begin{screen}
		\begin{thm}[第一類集合の性質]
			$S$を位相空間とする.
			\begin{description}
				\item[(a)] $A \subset B \subset S$に対し$B$が第一類なら$A$も第一類である.
				\item[(b)] 第一類集合の可算和も第一類である.
				\item[(c)] 内核が空である閉集合は第一類である.
				\item[(d)] $S$から$S$への位相同型$h$と$E \subset S$に対し次が成り立つ:
					\begin{align}
						\mbox{$E$が第一類} \quad \Longleftrightarrow \quad
						\mbox{$h(E)$が第一類}.
					\end{align}
			\end{description}
		\end{thm}
	\end{screen}
	
	\begin{prf}\mbox{}
		\begin{description}
			\item[(a)] $B = \bigcup_{n=1}^\infty E_n$
				を満たす疎集合系$(E_n)_{n=1}^\infty$に対し
				$A \cap E_n$は疎であり$A = \bigcup_{n=1}^\infty (A \cap E_n)$となる.
			\item[(b)] $A_n \subset S,\ (n=1,2,\cdots)$が第一類集合とし
				$(E_{n,i})_{i=1}^\infty$を$A_n = \bigcup_{i=1}^\infty E_{n,i}$
				を満たす疎集合系とすれば
				\begin{align}
					\bigcup_{n=1}^\infty A_n
					= \bigcup_{n,i=1}^\infty E_{n,i}
				\end{align}
				が成り立つ.
				
			\item[(c)] 内核が空である閉集合はそれ自身が疎であり,自身の可算和に一致する.
			\item[(d)] $E$が第一類のとき,$E = \bigcup_{i=1}^\infty E_i$を満たす
				疎集合系$(E_i)_{i=1}^\infty$に対し
				定理\ref{thm:topology_note_closure_interior}と
				補題\ref{lem:image_of_closure_is_closure_of_image}より
				\begin{align}
					\emptyset = h(E_i^{ai})
					= h(E_i^a)^i
					= h(E_i)^{ai}
				\end{align}
				が成り立つから$h(E_i)$は疎であり,
				\begin{align}
					h(E) = \bigcup_{i=1}^\infty h(E_i)
				\end{align}
				となるから$h(E)$も第一類である.$h(E)$が第一類なら$E = h^{-1}(h(E))$も第一類である.
				\QED
		\end{description}
	\end{prf}
\section{実数}
	\begin{screen}
		\begin{dfn}[Dedekind切断]
			$\Q$の任意の部分集合$A$に対して,順序対$(\Q \backslash A,A)$が
			{\bf Dedekind切断}\index{Dedekindせつだん@Dedekind切断}{\bf (Dedekind cut)}であるということを
			\begin{align}
				\mbox{順序対$(\Q \backslash A,A)$がDedekind切断である} \Longleftrightarrow\ 
				&A \neq \emptyset \wedge A \neq \Q\ \wedge \\
				&\forall x \in \Q \backslash A\ \forall y \in A\ (\ x < y\ )\ \wedge \\
				&\forall x \in A\ \exists y \in A\ (\ y < x\ )
			\end{align}
			で定義する.
		\end{dfn}
	\end{screen}
	
	\monologue{
		Dedekind切断とは数直線を左右に分割する操作をイメージしますね.例えば
		\begin{align}
			A = \Set{q \in \Q}{0 < q}
		\end{align}
		に対して$(\Q \backslash A,A)$はDedekind切断となります.
		実数の構成においてこの集合$A$は重要ですから,これを$\Q_+$と表して後で使いましょう.
		上の定義では$(\Q \backslash A,A)$がDedekind切断であるというとき
		$A$が最小元をもたないことを条件に入れましたが,ここは
		`$\Q \backslash A$が最大元を持たない'という条件に取り替えても構いません.
	}
	
	いま$R = \Set{x}{\mbox{$(\Q \backslash x,x)$はDedekind切断である}}$として$R$を定め,
	\begin{align}
		T = \Set{x}{\exists a,b \in R\ (\ x = (a,b) \wedge b \subset a\ )}
	\end{align}
	と定める.この$T$は$R$上の全順序となる.
	任意の$a,b \in R$に対して,$a \not\subset b$ならば
	或る有理数$x$が$x \in a$かつ$x \notin b$を満たす.
	このとき$b$の任意の要素$y$に対して$x < y$となり,
	$x \in a$かつ$x < y$より$y \in a$となるので$b \subset a$が成り立つ.ゆえに
	\begin{align}
		\rightharpoondown (a \subset b) \Longrightarrow b \subset a
	\end{align}
	が得られた.これは$a \subset b \vee b \subset a$と同値であるから$T$は全順序である.
	
	$X$を$R$の部分集合で,$X \neq \emptyset$かつ$X$は$R$において上に有界であるとする.
	このとき$\bigcup X$は$X$の上限となる.
	
	\begin{screen}
		\begin{thm}
			$\Q$の部分集合$A$に対して$(\Q \backslash A,A)$をDedekind切断とするとき,
			次が成り立つ:
			\begin{description}
				\item[(1)] $\forall q \in \Q\ (\ \exists a \in A\ (\ a < q\ )\Longleftrightarrow q \in A\ )$.
				\item[(2)] $\forall q \in \Q\ (\ \exists a \in \Q \backslash A\ (\ q < a\ )\Longrightarrow q \in \Q \backslash A\ )$.
			\end{description}
		\end{thm}
	\end{screen}
	
	\begin{prf}
		$q$を任意の有理数とすれば,$A$は最小元を持たないので
		\begin{align}
			q \in A \Longrightarrow \exists a \in A\ (\ a < q\ )
		\end{align}
		となる.逆に$q \notin A$ならば$A$の任意の要素$a$に対して$q < a$となるから,対偶を取って
		\begin{align}
			\exists a \in A\ (\ a < q\ ) \Longrightarrow q \in A
		\end{align}
		を得る.$q \notin \Q \backslash A$ならば$A$の任意の要素$a$に対して$a < q$となるから,
		対偶を取って(2)を得る.
		\QED
	\end{prf}
\section{微分積分学の基本定理}
	
	\begin{screen}
		\begin{thm}[$\R$の開集合は交わらない開区間の高々可算和で書ける]
		\label{thm:any_open_subset_of_R_is_at_most_countable_union_of_disjoint_open_intervals}
			$\mathscr{I}$を$\R$の開区間の全体とする.$u$を$\R$の開集合とすると,
			$\mathscr{I}$の部分集合$\mathscr{S}$で,
			\begin{align}
				\forall i,j \in \mathscr{S}\, \left(\, i \neq j \Longrightarrow i \cap j = \emptyset\, \right)
			\end{align}
			かつ
			\begin{align}
				\card{\mathscr{S}} \leq \Natural
			\end{align}
			かつ
			\begin{align}
				u = \bigcup \mathscr{S}
			\end{align}
			を満たすものが取れる.
		\end{thm}
	\end{screen}
	
	\begin{sketch}
		$u$が空であるときは
		\begin{align}
			\mathscr{S} \defeq \emptyset 
		\end{align}
		とおけばよい.以下では
		\begin{align}
			u \neq \emptyset
		\end{align}
		であるとする.$u$上の同値関係を
		\begin{align}
			\sim\ \defeq \Set{(x,y)}{x \in u \wedge y \in u \wedge \left[\min\{x,y\},\max\{x,y\}\right] \subset u}
		\end{align}
		により定め,$q$を$u$から$\mathscr{S}$への商写像とする.このとき,$x$を$u$の要素とすれば
		\begin{align}
			q(x) \in \mathscr{I}
		\end{align}
		が成り立つ.実際,$s$と$t$を
		\begin{align}
			s < t
		\end{align}
		なる$q(x)$の要素とすれば,
		\begin{align}
			[s,t] \subset u
		\end{align}
		なので
		\begin{align}
			[s,t] \subset q(x)
		\end{align}
		が従う.すなわち$q(x)$は弧状連結である.ゆえに$q(x)$は連結集合である.また$y$を$q(x)$の要素とすれば,
		$u$は開集合なので
		\begin{align}
			[y-r,y+r] \subset u
		\end{align}
		を満たす正の実数$r$が取れるが,このとき
		\begin{align}
			[y-r,y+r] \subset q(x)
		\end{align}
		が成り立つので$q(x)$は$\R$の開集合である.以上より$q(x)$は$\R$の開区間である.
		\begin{align}
			\mathscr{S} \defeq u/\sim
		\end{align}
		とおけば
		\begin{align}
			\mathscr{S} \subset \mathscr{I}
		\end{align}
		であって,定理\ref{thm:equivalence_classes_of_not_equivalent_elements_are_disjoint}より
		\begin{align}
			\forall i,j \in \mathscr{S}\, \left(\, i \neq j \Longrightarrow i \cap j = \emptyset\, \right)
		\end{align}
		が成立し,定理\ref{thm:union_of_quotient_set_is_the_original_set}より
		\begin{align}
			u = \bigcup \mathscr{S}
		\end{align}
		が成立する.また
		\begin{align}
			Q \defeq \Set{r \in \Q}{\exists x \in u\, \left(\, r \in q(x)\, \right)}
		\end{align}
		とおいて
		\begin{align}
			f \defeq \Set{a}{\exists r \in Q \, \exists x \in u\, \left[\, a = (r,q(x)) \wedge r \in q(x)\, \right]}
		\end{align}
		とおけば,
		\begin{align}
			f: Q \srj \mathscr{S}
		\end{align}
		が成り立つので,定理\ref{thm:if_exists_a_surjection_then_cardinal_of_target_is_bigger}と
		定理\ref{thm:cardinal_of_bigger_set_is_bigger}より
		\begin{align}
			\card{\mathscr{S}} \leq \card{Q} \leq \Natural
		\end{align}
		が得られる.
		\QED
	\end{sketch}
	
	\begin{screen}
		\begin{dfn}[絶対連続関数]
			$a$と$b$を$a < b$なる実数とし,$f$を$[a,b]$上の$\C$値関数とし,
			$\mathscr{I}$を$\R$上の開区間の全体とし,
			\begin{align}
				\mathscr{I}_{[a,b]} \defeq \Set{I \cap [a,b]}{I \in \mathscr{I}}
			\end{align}
			とおく.また$\lambda$を一次元Lebesgue測度とする.いま
			$\epsilon$を任意に与えられた正の実数とするとき,正の実数$\delta$が取れて,
			\begin{align}
				&\emptyset \neq \mathscr{S} \wedge \\
				&\emptyset \notin \mathscr{S} \wedge \\
				&\mathscr{S} \subset \mathscr{I}_{[a,b]} \wedge \\
				&\card{\mathscr{S}} \leq \Natural \wedge \\
				&\forall i,j \in \mathscr{S}\, (\, i \neq j \Longrightarrow i \cap j = \emptyset\, ) \wedge \\
				&\lambda\left(\bigcup \mathscr{S}\right) < \delta
			\end{align}
			を満たす任意の集合$\mathscr{S}$に対して
			\begin{align}
				\sum_{s \in \mathscr{S}} |f(\sup{}{s}) - f(\inf{}{s})| < \epsilon
			\end{align}
			が成り立つとする.このとき$f$を$[a,b]$上の{\bf 絶対連続関数}
			\index{ぜったいれんぞくかんすう@絶対連続関数}{\bf (absolutely continuous function)}と呼ぶ.
		\end{dfn}
	\end{screen}
	
	\begin{screen}
		\begin{thm}[$AC$は線型空間である]
		\end{thm}
	\end{screen}
	
	\begin{screen}
		\begin{thm}[絶対連続関数は有界変動である]
		\end{thm}
	\end{screen}
	
	\begin{screen}
		\begin{thm}[絶対連続な非減少関数は可測集合を可測集合に写す]
		\label{thm:absolutely_continuous_functions_map_measurable_set_to_measurable_set}
			$a$と$b$を$a<b$なる実数とし,$f$を$[a,b]$上の非減少$\R$値絶対連続関数とし,
			$\lambda$を一次元Lebesgue測度とし,$\mathfrak{M}$をLebesgue可測集合の全体とする.
			また
			\begin{align}
				\mathfrak{M}_{[a,b]} \defeq \Set{E \cap [a,b]}{E \in \mathfrak{M}}
			\end{align}
			とおく.このとき$\mathfrak{M}_{[a,b]}$の任意の要素$E$に対して
			\begin{align}
				f \ast E \in \mathfrak{M}
			\end{align}
			が成り立ち,特に
			\begin{align}
				\lambda(E) = 0 \Longrightarrow \lambda(f \ast E) = 0.
			\end{align}
		\end{thm}
	\end{screen}
	
	\begin{sketch}
		いま$\mathscr{I}$を$\R$上の開区間の全体とし,
		\begin{align}
			\mathscr{I}_{[a,b]} \defeq \Set{I \cap [a,b]}{I \in \mathscr{I}}
		\end{align}
		とおく.また$\epsilon$を任意に与えられた正の実数とする.$f$は絶対連続であるから,
		このとき正の実数$\delta$が取れて,
		\begin{align}
			&\emptyset \neq \mathscr{S} \wedge \\
			&\emptyset \notin \mathscr{S} \wedge \\
			&\mathscr{S} \subset \mathscr{I}_{[a,b]} \wedge \\
			&\card{\mathscr{S}} \leq \Natural \wedge \\
			&\forall i,j \in \mathscr{S}\, (\, i \neq j \Longrightarrow i \cap j = \emptyset\, ) \wedge \\
			&\lambda\left(\bigcup \mathscr{S}\right) < \delta
		\end{align}
		を満たす任意の集合$\mathscr{S}$に対して
		\begin{align}
			\sum_{s \in \mathscr{S}} |f(\sup{}{s}) - f(\inf{}{s})| < \epsilon
		\end{align}
		が成り立つ.いま$E$を$\mathfrak{M}_{[a,b]}$の要素として
		\begin{align}
			\lambda(E) = 0
		\end{align}
		であるとする.
		\begin{align}
			E = \emptyset
		\end{align}
		ならば
		\begin{align}
			f \ast E = \emptyset \in \mathfrak{M}
		\end{align}
		が成り立つ.
		\begin{align}
			E \neq \emptyset
		\end{align}
		であるとき,$\lambda$の正則性より
		\begin{align}
			E \subset V
		\end{align}
		かつ
		\begin{align}
			\lambda(V) < \delta
		\end{align}
		を満たす$\R$の開集合$V$が取れて,定理\ref{thm:any_open_subset_of_R_is_at_most_countable_union_of_disjoint_open_intervals}より
		\begin{align}
			V = \bigcup \mathscr{T}
		\end{align}
		かつ
		\begin{align}
			\card{\mathscr{T}} \leq \Natural
		\end{align}
		かつ
		\begin{align}
			\forall i,j \in \mathscr{T}\, (\, i \neq j \Longrightarrow i \cap j = \emptyset\, )
		\end{align}
		を満たす$\mathscr{I}$の部分集合$\mathscr{T}$が取れる.
		\begin{align}
			\mathscr{S} \defeq \Set{I \cap [a,b]}{I \in \mathscr{T} \wedge I \cap [a,b] \neq \emptyset}
		\end{align}
		とおけば
		\begin{align}
			\lambda\left(\bigcup \mathscr{S}\right)
			= \lambda\left([a,b] \cap V\right) \leq \lambda(V) < \delta
		\end{align}
		が成り立つ.ところで$f$は連続であるから,$\mathscr{S}$の要素$s$に対して
		$f \ast s$は連結である.ゆえにこれは$\R$の区間であり,すなわちLebesgue可測集合である.よって
		\begin{align}
			f \ast \bigcup \mathscr{S} = \bigcup_{s \in \mathscr{S}} f \ast s \in \mathfrak{M}
		\end{align}
		である.$f$は非減少なので
		\begin{align}
			\lambda\left(f \ast \bigcup \mathscr{S}\right)
			= \lambda\left(\bigcup_{s \in \mathscr{S}} f \ast s\right)
			\leq \sum_{s \in \mathscr{S}} \left(f(\sup{}{s}) - f(\inf{}{s})\right)
			< \epsilon
		\end{align}
		が満たされる.以上でこの$E$に対して
		\begin{align}
			\forall \epsilon \in \R_+\, \exists F \in \mathfrak{M}\,
			\left(\, f \ast E \subset F \wedge \lambda(F) < \epsilon\, \right)
		\end{align}
		が成り立つことが示された.すなわち各自然数$n$に対して
		\begin{align}
			\Set{F \in \mathfrak{M}}{f \ast E \subset F \wedge \lambda(F) < \frac{1}{n}}
		\end{align}
		は空でないから,定理\ref{thm:direct_product_of_non_empty_sets_is_not_empty}より
		\begin{align}
			\prod_{n \in \Natural} \Set{F \in \mathfrak{M}}{f \ast E \subset F \wedge \lambda(F) < \frac{1}{n}}
		\end{align}
		の要素$h$が取れる.
		\begin{align}
			F \defeq \bigcap_{n \in \Natural} h(n)
		\end{align}
		とおけば
		\begin{align}
			f \ast E \subset F
		\end{align}
		かつ
		\begin{align}
			\lambda(F) = 0
		\end{align}
		が成り立つので,Lebesgue測度の完備性より
		\begin{align}
			f \ast E \in \mathfrak{M}
		\end{align}
		及び
		\begin{align}
			\lambda(f \ast E) = 0
		\end{align}
		が従う.つまり,$f$は零集合を零集合に写す.次に$E$を$\mathfrak{M}_{[a,b]}$の一般の要素とする.このとき
		\begin{align}
			E = F \cup N
		\end{align}
		を満たす$\R$の$F_\sigma$集合$F$とLebesgue零集合$N$が取れるが,
		\begin{align}
			F \subset [a,b]
		\end{align}
		なので$F$は$\sigma$-コンパクトであり
		\begin{align}
			f \ast F \in \mathfrak{M}
		\end{align}
		が満たされる.また$N$はLebesgue零集合なので
		\begin{align}
			f \ast N \in \mathfrak{M}
		\end{align}
		も満たされる.ゆえに
		\begin{align}
			f \ast E = f \ast F \cup f \ast N \in \mathfrak{M}
		\end{align}
		が成立する.
		\QED
	\end{sketch}
	
	\begin{screen}
		\begin{thm}[絶対連続関数の総変動関数も絶対連続である]
			
		\end{thm}
	\end{screen}
	
	\begin{screen}
		\begin{thm}[微分積分学の基本定理]\label{thm:the_fundamental_theorem_of_calculus}
		\end{thm}
	\end{screen}
	
	\begin{sketch}\mbox{}
		\begin{description}
			\item[第一段]
				$f$が$\R$値の非減少関数であるとして考察する.
				\begin{align}
					[a,b] \ni x \longmapsto x + f(x)
				\end{align}
				なる写像を$g$とすれば,$g$は単調増大であるから
				定理\ref{thm:absolutely_continuous_functions_map_measurable_set_to_measurable_set}より
				\begin{align}
					\forall E\, \left[\, E \in \mathfrak{M}_{[a,b]} \Longrightarrow g \ast E \in \mathfrak{M}\, \right]
				\end{align}
				を満たす.また$\mathfrak{M}_{[a,b]}$上の写像$\mu$を
				\begin{align}
					\mathfrak{M}_{[a,b]} \ni E \longmapsto \lambda(g \ast E)
				\end{align}
				なる関係により定めれば,$g$が単射であるから$\mu$は$\mathfrak{M}_{[a,b]}$上の複素測度である.
				また定理\ref{thm:absolutely_continuous_functions_map_measurable_set_to_measurable_set}より
				\begin{align}
					\lambda(E) = 0 \Longrightarrow \lambda(g \ast E) = 0
				\end{align}
				が成り立つので
				\begin{align}
					\mu \ll \lambda
				\end{align}
				が成立し,Radon-Nikodymの定理から$\mathfrak{M}_{[a,b]}$の任意の要素$E$に対して
				\begin{align}
					\mu(E) = \int_E h\ d\lambda
				\end{align}
				を満たす$\mathscr{L}^1([a,b],\mathfrak{M}_{[a,b]},\lambda)$の要素$h$が取れる.
				$x$を
				\begin{align}
					a \leq x \leq b
				\end{align}
				を満たす$[a,b]$の任意の要素とすれば
				\begin{align}
					g(x) - g(a) = \lambda(g \ast [a,x]) = \mu([a,x]) = \int_{[a,x]} h\ d\lambda
				\end{align}
				が成り立つので,
				\begin{align}
					f(x) - f(a) = \int_{[a,x]} h-1\ d\lambda
				\end{align}
				が従う.ゆえに微分定理よりLebesgue点$x$において$f$は微分可能であって,その微分係数は
				\begin{align}
					h(x) - 1
				\end{align}
				に一致する.
				\begin{align}
					[a,b] \ni x \longmapsto
					\begin{cases}
						h(x) - 1 & \mbox{if $x$がLebesgue点} \\
						0 & \mbox{if $x$がLebesgue点でない} \\
					\end{cases}
				\end{align}
				なる写像を$f'$と定めれば
				\begin{align}
					\forall x\, \left[\, a \leq x \leq b \Longrightarrow f(x) - f(a) = \int_{[a,x]} f'\ d\lambda\, \right]
					\label{fom:the_fundamental_theorem_of_calculus}
				\end{align}
				が得られる.逆に$f$がa.e.に微分可能で(\refeq{fom:the_fundamental_theorem_of_calculus})が成立していれば
				$f$は絶対連続である.
		\end{description}
	\end{sketch}
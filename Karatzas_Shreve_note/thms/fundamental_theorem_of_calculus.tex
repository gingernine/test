\section{基本定理}
	
	\begin{screen}
		\begin{thm}[$\R$の開集合は交わらない開区間の高々可算和で書ける]
			$\mathscr{I}$を$\R$の開区間の全体とする.$u$を$\R$の開集合とすると,
			$\mathscr{I}$の部分集合$\mathscr{S}$で,
			\begin{align}
				\forall i,j \in \mathscr{S}\, \left(\, i \neq j \Longrightarrow i \cap j = \emptyset\, \right)
			\end{align}
			かつ
			\begin{align}
				\card{\mathscr{S}} \leq \Natural
			\end{align}
			かつ
			\begin{align}
				u = \bigcup \mathscr{S}
			\end{align}
			を満たすものが取れる.
		\end{thm}
	\end{screen}
	
	\begin{sketch}
		$u$上の同値関係を
		\begin{align}
			\sim\ \defeq \Set{(x,y)}{x \in u \wedge y \in u \wedge \left[\min\{x,y\},\max\{x,y\}\right] \subset u}
		\end{align}
		により定め,$q$を$u$から$\mathscr{S}$への商写像とする.このとき,$x$を$u$の要素とすれば
		\begin{align}
			q(x) \in \mathscr{I}
		\end{align}
		が成り立つ.実際,$s$と$t$を
		\begin{align}
			s < t
		\end{align}
		なる$q(x)$の要素とすれば,
		\begin{align}
			[s,t] \subset u
		\end{align}
		なので
		\begin{align}
			[s,t] \subset q(x)
		\end{align}
		が従う.すなわち$q(x)$は弧状連結である.ゆえに$q(x)$は連結集合である.また$y$を$q(x)$の要素とすれば,
		$u$は開集合なので
		\begin{align}
			[y-r,y+r] \subset u
		\end{align}
		を満たす正の実数$r$が取れるが,このとき
		\begin{align}
			[y-r,y+r] \subset q(x)
		\end{align}
		が成り立つので$q(x)$は$\R$の開集合である.以上より$q(x)$は$\R$の開区間である.
		\begin{align}
			\mathscr{S} \defeq u/\sim
		\end{align}
		とおけば
		\begin{align}
			\mathscr{S} \subset \mathscr{I}
		\end{align}
		であって,定理\ref{thm:equivalence_classes_of_not_equivalent_elements_are_disjoint}より
		\begin{align}
			\forall i,j \in \mathscr{S}\, \left(\, i \neq j \Longrightarrow i \cap j = \emptyset\, \right)
		\end{align}
		が成立し,定理\ref{thm:union_of_quotient_set_is_the_original_set}より
		\begin{align}
			u = \bigcup \mathscr{S}
		\end{align}
		が成立する.また
		\begin{align}
			Q \defeq \Set{r \in \Q}{\exists x \in u\, \left(\, r \in q(x)\, \right)}
		\end{align}
		とおいて
		\begin{align}
			f \defeq \Set{a}{\exists r \in Q \, \exists x \in u\, \left[\, a = (r,q(x)) \wedge r \in q(x)\, \right]}
		\end{align}
		とおけば,
		\begin{align}
			f: Q \srj \mathscr{S}
		\end{align}
		が成り立つので,定理\ref{thm:if_exists_a_surjection_then_cardinal_of_target_is_bigger}と
		定理\ref{thm:cardinal_of_bigger_set_is_bigger}より
		\begin{align}
			\card{\mathscr{S}} \leq \card{Q} \leq \Natural
		\end{align}
		が得られる.
		\QED
	\end{sketch}
	
	\begin{screen}
		\begin{dfn}[絶対連続関数]
			$a$と$b$を$a < b$なる実数とし,$f$を$[a,b]$上の$\C$値関数とし,
			$\mathscr{I}$を$\R$上の開区間の全体とし,
			\begin{align}
				\mathscr{I}[a,b] \defeq \Set{I \cap [a,b]}{I \in \mathscr{I}}
			\end{align}
			とおく.また$\lambda$を一次元Lebesgue測度とする.いま
			$\epsilon$を任意に与えられた正の実数とするとき,正の実数$\delta$が取れて,
			\begin{align}
				&\emptyset \neq \mathscr{S} \wedge \\
				&\emptyset \notin \mathscr{S} \wedge \\
				&\mathscr{S} \subset \mathscr{I}[a,b] \wedge \\
				&\card{\mathscr{S}} < \Natural \wedge \\
				&\forall i,j \in \mathscr{S}\, (\, i \neq j \Longrightarrow i \cap j = \emptyset\, ) \wedge \\
				&\lambda\left(\bigcup \mathscr{S}\right) < \delta
			\end{align}
			を満たす任意の集合$\mathscr{S}$に対して
			\begin{align}
				N \defeq \card{\mathscr{S}}
			\end{align}
			とおいて
			\begin{align}
				h:N \bij \mathscr{S}
			\end{align}
			なる写像を取れば
			\begin{align}
				\sum_{n \in N} |f(\sup{}{h(n)}) - f(\inf{}{h(n)})| < \epsilon
			\end{align}
			が成り立つとする.このとき$f$を$[a,b]$上の{\bf 絶対連続関数}
			\index{ぜったいれんぞくかんすう@絶対連続関数}{\bf (absolutely continuous function)}と呼ぶ.
		\end{dfn}
	\end{screen}
	
	\begin{screen}
		\begin{thm}[絶対連続関数は有界変動である]
		\end{thm}
	\end{screen}
	
	\begin{screen}
		\begin{thm}[絶対連続な非減少関数は可測集合を可測集合に写す]
			
		\end{thm}
	\end{screen}
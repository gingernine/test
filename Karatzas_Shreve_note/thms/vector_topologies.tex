	本節で扱う線型空間はすべて複素数体か実数体をスカラーとして考える.また$\Phi$は
	\begin{align}
		\Phi \defeq \C
	\end{align}
	か
	\begin{align}
		\Phi \defeq \R
	\end{align}
	で定められたものと考える.すなわち,$\mathscr{O}_\Phi$は$\mathscr{O}_\C$または$\mathscr{O}_\R$を指す.
	
\subsection{線型位相}
	\begin{screen}
		\begin{dfn}[位相線型空間]\label{def:topological_vector_space}
			$\left(\left(X,\sigma_X\right),(\Phi,+,\bullet),s\right)$を線型空間とし,
			$\mathscr{O}_X$を$X$上の位相とする.また
			$\mathscr{O}_{X \times X}$を$\mathscr{O}_X$から作られる$X \times X$上の積位相とし,
			$\mathscr{O}_{\Phi \times X}$を$\mathscr{O}_\Phi$と$\mathscr{O}_X$から作られる$\Phi \times X$上の積位相とする.
			\begin{description}
				\item[(tvs1)] $\sigma_X$が$\mathscr{O}_{X \times X}/\mathscr{O}_X$-連続である.
				\item[(tvs2)] $s$が$\mathscr{O}_{\Phi \times X}/\mathscr{O}_X$-連続である.
			\end{description}
			が満たされるとき,$\mathscr{O}_X$を$X$上の{\bf 線型位相}\index{せんけいいそう@線型位相}{\bf (vector topology)}と呼び,
			\begin{align}
				\left(\left(X,\sigma_X\right),(\Phi,+,\bullet),s,\mathscr{O}_X\right)
				\label{pair_topological_vector_space}
			\end{align}
			を{\bf 位相線型空間}\index{いそうせんけいくうかん@位相線型空間}
			{\bf (topological vector space)}と呼ぶ.また
			$(X,\mathscr{O}_X)$がHausdorffであるとき,(\refeq{pair_topological_vector_space})を
			Hausdorff位相線型空間と呼ぶ.
		\end{dfn}
	\end{screen}
	
	$\left(\left(X,\sigma_X\right),(\Phi,+,\bullet),s,\mathscr{O}_X\right)$を位相線型空間とするとき,
	$\sigma_X$は連続であるから,$a$を$X$の任意の要素として
	\begin{align}
		X \ni x \longmapsto \sigma_X(x,a)
	\end{align}
	なる写像を
	\begin{align}
		\sigma_X^a
	\end{align}
	とおけば,これは$\mathscr{O}_X/\mathscr{O}_X$-連続である.さらにこのとき,
	\begin{align}
		\sigma_X^{-a}
	\end{align}
	なる写像は$\sigma_X^a$の逆写像であって,かつ$\mathscr{O}_X/\mathscr{O}_X$-連続なので,
	$\sigma_X^a$は$\mathscr{O}_X$に関する同相写像である.つまり,{\bf 位相線型空間の平行移動は同相である.}
	また,$\sigma_X$は可換であるから
	\begin{align}
		X \ni x \longmapsto \sigma_X(a,x)
	\end{align}
	なる写像は$\sigma_X^a$に一致する.ゆえにこれも同相である.
	
	同様に$x$を$X$の任意の要素とすると
	\begin{align}
		\Phi \ni \alpha \longmapsto s(\alpha,x)
	\end{align}
	なる写像は$\mathscr{O}_\Phi/\mathscr{O}_X$-連続であり,$\alpha$を$\Phi$の任意の要素とすると
	\begin{align}
		X \ni x \longmapsto s(\alpha,x)
		\label{partial_continuity_of_summation_of_topological_vector_spaces_1}
	\end{align}
	なる写像は$\mathscr{O}_X/\mathscr{O}_X$-連続である.とくに
	\begin{align}
		\alpha \neq 0
	\end{align}
	ならば
	\begin{align}
		X \ni x \longmapsto s(\alpha^{-1},x)
	\end{align}
	なる写像は(\refeq{partial_continuity_of_summation_of_topological_vector_spaces_1})の逆写像であり,
	かつ$\mathscr{O}_X/\mathscr{O}_X$-連続であるから,(\refeq{partial_continuity_of_summation_of_topological_vector_spaces_1})
	は$\mathscr{O}_X$に関する同相写像である.
	つまり,{\bf 位相線型空間のスケール変換は,$0$倍でない限り同相である.}
	
	\begin{screen}
		\begin{thm}[位相線型空間は位相群]
			$\left(\left(X,\sigma_X\right),(\Phi,+,\bullet),s,\mathscr{O}_X\right)$を位相線型空間とするとき,
			$\left(\left(X,\sigma_X\right),\mathscr{O}_X\right)$は位相群である.
		\end{thm}
	\end{screen}
	
	\begin{sketch}
	\end{sketch}
	
	$\left(\left(X,\sigma_X\right),(\Phi,+,\bullet),s\right)$を線型空間とし,
	$A$を$X$の部分集合とする.このとき,
	\begin{align}
		|\alpha| \leq 1
	\end{align}
	なる$\Phi$の任意の要素$\alpha$に対して
	\begin{align}
		\Set{s(\alpha,x)}{x \in A} \subset A
	\end{align}
	が成り立つならば,$A$を$\left(\left(X,\sigma_X\right),(\Phi,+,\bullet),s\right)$の
	{\bf 均衡集合}\index{きんこうしゅうごう@均衡集合}{\bf (balanced set)}と呼ぶ.
	$A$が均衡であることを直感的に書けば
	\begin{align}
		\forall \alpha \in \Phi\,
		\left(\, |\alpha| \leq 1 \Longrightarrow \alpha A \subset A\, \right)
	\end{align}
	ということになる.例えば$X$が$\R^2$ならば均衡集合とは円のことを指し,$X$が$\R^3$ならば球のことを指す.
	\begin{align}
		-x = s(-1,x)
	\end{align}
	が成り立つので,均衡とは逆元で閉じていることの拡張概念である.
	定理\ref{thm:there_exists_a_local_base_whose_elements_are_closed_under_inversion}の拡張として次を得る.
	
	\begin{screen}
		\begin{thm}[均衡な局所基が取れる]
			$\left(\left(X,\sigma_X\right),(\Phi,+,\bullet),s,\mathscr{O}_X\right)$を位相線型空間とする.
			このとき,$X$の零元の基本近傍系で,そのすべての要素が均衡集合であるものが取れる.
		\end{thm}
	\end{screen}
	
	\begin{sketch}
		$0_X$を$X$の零元とし,$v$を$0_X$の近傍とする.$s$は$\mathscr{O}_{\Phi \times X}/\mathscr{O}_X$-連続であるから,
		\begin{align}
			\Set{\alpha \in \Phi}{|\alpha| < \delta} \times w \subset s^{-1} \ast v
		\end{align}
		を満たす正の実数$\delta$と$0_X$の開近傍$w$が取れる.ここで
		\begin{align}
			u \defeq \bigcup_{\substack{\alpha \in \Phi \\ 0 < |\alpha| < \delta}} \Set{s(\alpha,x)}{x \in w}
		\end{align}
		とおくと,$u$は$0_X$の均衡な開近傍であって,
		\begin{align}
			u \subset v
		\end{align}
		を満たす.
		\begin{description}
			\item[step1] $u$は開集合である.実際
				\begin{align}
					0 < |\alpha| < \delta
				\end{align}
				なる$\alpha$に対して
				\begin{align}
					X \ni x \longmapsto s(\alpha,x)
				\end{align}
				なる写像を$s^\alpha$とおけば,$s^\alpha$は$\mathscr{O}_X$に関して同相であって,かつ
				\begin{align}
					s^\alpha \ast w = \Set{s(\alpha,x)}{x \in w}
				\end{align}
				であるから,
				\begin{align}
					\Set{s(\alpha,x)}{x \in w} \in \mathscr{O}_X
				\end{align}
				が成り立つ.また
				\begin{align}
					0_X \in w
				\end{align}
				より
				\begin{align}
					0_X \in \Set{s(\alpha,x)}{x \in w}
				\end{align}
				も成り立つ.ゆえに$u$は$0_X$の開近傍である.
				
			\item[step2] $u$が$\left(\left(X,\sigma_X\right),(\Phi,+,\bullet),s\right)$の均衡集合であることを示す.
				いま$\beta$を
				\begin{align}
					|\beta| \leq 1
				\end{align}
				なる$\Phi$の要素とし,$x$を$u$の要素とする.このとき
				\begin{align}
					x = s(\alpha,y)
				\end{align}
				かつ
				\begin{align}
					0 < |\alpha| < \delta
				\end{align}
				かつ
				\begin{align}
					y \in w
				\end{align}
				を満たす$\alpha$と$y$が取れて
				\begin{align}
					s(\beta,x) = s(\beta,s(\alpha,y) = s(\beta \cdot \alpha,y)
				\end{align}
				が成り立つ.
				\begin{align}
					\beta = 0
				\end{align}
				ならば
				\begin{align}
					s(\beta \cdot \alpha,y) = 0_X \in u
				\end{align}
				が成り立ち,
				\begin{align}
					\beta \neq 0
				\end{align}
				ならば
				\begin{align}
					0 < |\beta \cdot \alpha| < \delta
				\end{align}
				が成り立つので
				\begin{align}
					s(\beta \cdot \alpha,y) \in s^{\beta \cdot \alpha} \ast w \subset u
				\end{align}
				が従う.ゆえに
				\begin{align}
					|\beta| \leq 1 \Longrightarrow \Set{s(\beta,x)}{x \in u} \subset u
				\end{align}
				が成り立つ.ゆえに$u$は均衡している.
				
			\item[step3] 定理\ref{thm:union_of_subsets_is_subclass}より
				\begin{align}
					u \subset v
				\end{align}
				が成立する.
		\end{description}
		$0_X$の近傍の全体を
		\begin{align}
			\mathscr{B}
		\end{align}
		とおき,$\mathscr{B}$の要素$v$に対して,$v$に含まれる均衡な$0_X$の近傍の全体,つまり
		\begin{align}
			\Set{u}{u \in \mathscr{B} \wedge u \subset v \wedge 
			\forall \alpha \in \Phi\, \forall x \in u\, \left(\, |\alpha| \leq 1 \Longrightarrow
			s(\alpha,x) \in u\, \right)}
		\end{align}
		なる集合を対応させる関係を$h$とおくと,上の結果から
		\begin{align}
			\forall v \in \mathscr{B}\, \left(\, h(v) \neq \emptyset\, \right)
		\end{align}
		が成り立つ.ゆえに定理\ref{thm:direct_product_of_non_empty_sets_is_not_empty}より
		\begin{align}
			f \in \prod_{v \in \mathscr{B}} h(v)
		\end{align}
		なる集合$f$が取れる.そして
		\begin{align}
			\left\{f(v)\right\}_{v \in \mathscr{B}}
		\end{align}
		は$0_X$の基本近傍系であり,その全ての要素は均衡している.
		\QED
	\end{sketch}
	
	次に述べることは{\bf 位相線型空間が一様化可能である}ということである.
	いま$\left(\left(X,\sigma_X\right),(\Phi,+,\bullet),s,\mathscr{O}_X\right)$を位相線型空間とし,
	\begin{align}
		0_X
	\end{align}
	を$X$の零元とし,
	\begin{align}
		\mathscr{B}
	\end{align}
	を$0_X$の基本近傍系とし,$\mathscr{B}$の要素は全て均衡集合であるとする.
	ここで$\mathscr{B}$の要素$b$に対して
	\begin{align}
		\Set{(x,y) \in X \times X}{\sigma_X(x,-y) \in b}
	\end{align}
	で定められる集合を,$\mathscr{B}$のすべての要素に亘って取ったものの全体を$\mathscr{U}$と定める.つまり
	\begin{align}
		\mathscr{U} \defeq \Set{u}{\exists b \in \mathscr{B}\,
		\left[\, \forall x,y \in X\, \left(\, (x,y) \in u \Longleftrightarrow\sigma_X(x,-y) \in b\, \right)\, \right]}
	\end{align}
	と定める.すると,
	\begin{align}
		\mathscr{V} \defeq \Set{v}{v \subset X \times X \wedge \exists u \in \mathscr{U}\,
		\left(\, u \subset v\, \right)}
	\end{align}
	で定める$\mathscr{V}$は$X$上の近縁系となる.実際,
	\begin{description}
		\item[(a)] $\mathscr{B}$は空ではないので
			\begin{align}
				\mathscr{U} \neq \emptyset
			\end{align}
			である.ゆえに
			\begin{align}
				\mathscr{V} \neq \emptyset
			\end{align}
			である.また$v$を$\mathscr{V}$の任意の要素とすると
			\begin{align}
				\Set{(x,y)}{x \in X \wedge y \in X \wedge \sigma_X(x,-y) \in b} \subset v
			\end{align}
			を満たす$\mathscr{B}$の要素$b$が取れるが,$X$の任意の要素$x$に対して
			\begin{align}
				\sigma_X(x,-x) = 0_X \in b
			\end{align}
			が成り立つので
			\begin{align}
				\Set{(x,x)}{x \in X} \subset v
			\end{align}
			が成立する.
			
		\item[(b)] $v$を$\mathscr{V}$の任意の要素とする.このとき
			\begin{align}
				\Set{(x,y)}{x \in X \wedge y \in X \wedge \sigma_X(x,-y) \in b} \subset v
			\end{align}
			を満たす$\mathscr{B}$の要素$b$が取れる.
			\begin{align}
				u \defeq \Set{(x,y)}{x \in X \wedge y \in X \wedge \sigma_X(x,-y) \in b}
			\end{align}
			とおけば
			\begin{align}
				u^{-1} = \Set{(y,x)}{x \in X \wedge y \in X \wedge \sigma_X(x,-y) \in b}
			\end{align}
			となるが,$b$は均衡なので
			\begin{align}
				\forall x,y \in X\, \left[\, \sigma_X(x,-y) \in b \Longleftrightarrow \sigma_X(y,-x) \in b\, \right]
			\end{align}
			が成り立つ.ゆえに
			\begin{align}
				u^{-1} = \Set{(y,x)}{x \in X \wedge y \in X \wedge \sigma_X(y,-x) \in b}
			\end{align}
			が成り立つ.ゆえに
			\begin{align}
				u^{-1} \in \mathscr{U}
			\end{align}
			が成り立つ.
			\begin{align}
				u^{-1} \subset v^{-1}
			\end{align}
			であるから
			\begin{align}
				v^{-1} \in \mathscr{V}
			\end{align}
			が従う.
			
		\item[(c)] $u$と$v$を$\mathscr{V}$の要素とする.このとき
			\begin{align}
				\Set{(x,y)}{x \in X \wedge y \in X \wedge \sigma_X(x,-y) \in a} \subset u
			\end{align}
			を満たす$\mathscr{B}$の要素$a$と
			\begin{align}
				\Set{(x,y)}{x \in X \wedge y \in X \wedge \sigma_X(x,-y) \in b} \subset v
			\end{align}
			を満たす$\mathscr{B}$の要素$b$が取れる.このとき
			\begin{align}
				c \subset a \cap b
			\end{align}
			なる$\mathscr{B}$の要素$c$が取れて
			\begin{align}
				\Set{(x,y)}{x \in X \wedge y \in X \wedge \sigma_X(x,-y) \in c} \subset u \cap v
			\end{align}
			が成り立つので,
			\begin{align}
				u \cap v \subset \mathscr{V}
			\end{align}
			が従う.
			
		\item[(d)] $v$を$\mathscr{V}$の任意の要素とする.すると
			\begin{align}
				\Set{(x,y)}{x \in X \wedge y \in X \wedge \sigma_X(x,-y) \in b} \subset v
			\end{align}
			を満たす$\mathscr{B}$の要素$b$が取れて,$\sigma_X$は連続なので
			\begin{align}
				a \times c \subset \sigma_X^{-1} \ast b
			\end{align}
			を満たす$\mathscr{B}$の要素$a$と$c$が取れる.ここで
			\begin{align}
				d \subset a \cap c
			\end{align}
			を満たす$\mathscr{B}$の要素$d$を取り,
			\begin{align}
				w \defeq \Set{(x,y)}{x \in X \wedge y \in X \wedge \sigma_X(x,-y) \in d}
			\end{align}
			により$\mathscr{V}$の要素$w$を定める.このとき,$x$と$y$と$z$を$X$の任意の要素として
			\begin{align}
				(x,y) \in w \wedge (y,z) \in w
			\end{align}
			であるとすると,
			\begin{align}
				\sigma_X(x,-y) \in d
			\end{align}
			かつ
			\begin{align}
				\sigma_X(y,-z) \in d
			\end{align}
			が成り立つので
			\begin{align}
				\sigma_X\left(\sigma_X(x,-y),\sigma_X(y,-z)\right) \in b
			\end{align}
			が成立する.他方で$\sigma_X$の結合性から
			\begin{align}
				\sigma_X\left(\sigma_X(x,-y),\sigma_X(y,-z)\right)
				&= \sigma_X\left(\sigma_X\left(\sigma_X(x,-y),y\right),-z\right) \\
				&= \sigma_X\left(\sigma_X\left(x,\sigma_X(-y,y)\right),-z\right) \\
				&= \sigma_X(x,-z)
			\end{align}
			が成り立つので,
			\begin{align}
				(x,z) \in v
			\end{align}
			が従う.ゆえに$w$は
			\begin{align}
				w \circ w \subset v
			\end{align}
			を満たす.
			
		\item[(e)] $r$を$X$上の関係とし,
			\begin{align}
				v \subset r
			\end{align}
			を満たす$\mathscr{V}$の要素$v$が取れるとする.このとき
			\begin{align}
				u \subset v
			\end{align}
			なる$\mathscr{U}$の要素が取れて
			\begin{align}
				u \subset r
			\end{align}
			が成り立つので
			\begin{align}
				r \in \mathscr{V}
			\end{align}
			が従う.
			\QED
	\end{description}
	以上より$\mathscr{V}$が$X$上の近縁系であることが示された.後は$\mathscr{V}$により導入する一様位相が
	$\mathscr{O}_X$と両立することを示せば一様化可能性が言えるが,その前に次の定理を証明しておく.
	
	位相線型空間は一様化可能であるから,$(X,\mathscr{O}_X)$は$T_0$ならTychonoffである.
	
	\begin{screen}
		\begin{thm}[平行移動不変位相]
			$\tau$を線型空間$X$の位相とする.
			任意の$V \subset X$と$x \in X$に対して
			\begin{align}
				V \in \tau \quad \Longleftrightarrow \quad
				x + V \in \tau
			\end{align}
			が満たされるとき,$\tau$は平行移動不変である(translation invariant)という.
			定理\ref{thm:continuity_of_translations_multiples}より
			位相線型空間において平行移動は同相写像となるから線型位相は平行移動不変である.
		\end{thm}
	\end{screen}
	
	\begin{screen}
		\begin{thm}[平行移動不変位相は0の基本近傍系で決まる]
			$\tau$を線型空間$X$の平行移動不変位相,
			$\mathscr{U}$を$0 \in X$の基本近傍系とするとき,
			任意の$x \in X$に対して
			\begin{align}
				\mathscr{U}(x) \defeq
				\Set{x + U}{U \in \mathscr{U}}
			\end{align}
			は$x$の基本近傍系となる.すなわち次が成立する:
			\begin{align}
				\tau = 
				\Set{O \subset X}{\mbox{$O = \emptyset$,或は任意の$x \in O$に対し
				$x+U_x \subset O$を満たす$U_x \in \mathscr{U}$が存在する}}.
			\end{align}
		\end{thm}
	\end{screen}
	
	\begin{prf}
		$V$を$x$の任意の近傍とすれば
		定理\ref{thm:continuity_of_translations_multiples}より
		$-x + V^{\mathrm{o}}$は$0$の開近傍となる.このとき或る
		$U \in \mathscr{U}$が
		\begin{align}
			U \subset -x + V^{\mathrm{o}} \subset -x + V
		\end{align}
		を満たし$x + U \subset V$が従うから
		$\mathscr{U}(x)$は$x$の基本近傍系をなしている.
		このとき定理\ref{thm:a_local_base_restores_the_topology}より
		\begin{align}
			O \in \tau &\quad \Longleftrightarrow \quad
			\mbox{$O = \emptyset$,或は任意の$x \in O$に対し
				$U \subset O$を満たす$U \in \mathscr{U}(x)$が存在する} \\
			&\quad \Longleftrightarrow \quad
			\mbox{$O = \emptyset$,或は任意の$x \in O$に対し
				$x+U_x\subset O$を満たす$U_x \in \mathscr{U}$が存在する} \\
		\end{align}
		が成立する.
		\QED
	\end{prf}
	
	\begin{screen}
		\begin{dfn}[平行移動不変距離・絶対斉次距離]
			$d$を線型空間$X$上に定まる距離とする.
			\begin{description}
				\item[(1)] 次が満たされるとき$d$は平行移動不変
				\index{へいこういどうふへん@平行移動不変}である(invariant)という:
					\begin{align}
						d(x+z, y+z) = d(x,y),\quad (\forall x,y,z \in X).
					\end{align}
					
				\item[(2)]  次が満たされるとき$d$は
					絶対斉次的\index{ぜったいせいじてき@絶対斉次的}
					である(absolutely homogeneous)と呼ぶことにする:
					\begin{align}
						d(\alpha x, \alpha y) = |\alpha| d(x,y),
						\quad (\forall \alpha \in \Phi,\ x,y \in X).
					\end{align}
			\end{description}
		\end{dfn}
	\end{screen}
	
	\begin{screen}
		\begin{thm}[平行移動不変距離による距離位相は平行移動不変]
			線型空間$X$に平行移動不変距離$d$が定まっているとき,
			$d$による距離位相は平行移動不変となる.
		\end{thm}
	\end{screen}
	
	\begin{prf}
		任意の$\delta>0$と$a \in X$に対し
		$B_\delta(a) \defeq \Set{x \in X}{d(x,a) < \delta}$と書けば,
		任意の$y \in X$で
		\begin{align}
			z \in y + B_\delta(a)
			\quad \Longleftrightarrow \quad
			d(z-y,a) < \delta
			\quad \Longleftrightarrow \quad
			d(z,y+a) < \delta
			\quad \Longleftrightarrow \quad
			z \in B_\delta(y+a)
		\end{align}
		が成り立つ.従って,部分集合$U$が$U = \bigcup_{a \in U}B_{\delta_a}(a)$と書けるとき
		任意の$x \in X$に対し
		\begin{align}
			x + U = \bigcup_{a \in U} \left(x+B_{\delta_a}(a)\right)
			= \bigcup_{a \in U} B_{\delta_a}(x+a)
		\end{align}
		となるから,$U$が開集合であることと$x + U$が開集合であることは同値になる.
		\QED
	\end{prf}
	
	\begin{screen}
		\begin{thm}[絶対斉次的かつ平行移動不変な距離はノルムで導入する距離に限られる]
			ノルムで導入する距離は絶対斉次的かつ平行移動不変であり,
			かつそのような距離はノルムで導入する距離に限られる.
		\end{thm}
	\end{screen}
	
	\begin{prf}
		$\Norm{\cdot}{}$を線型空間$X$のノルムとするとき,
		\begin{align}
			d(x,y) \defeq \Norm{x-y}{}, \quad (\forall x,y \in X)
		\end{align}
		で距離を定めれば
		\begin{align}
			d(x+z,y+z) = \Norm{x+z-(y+z)}{} = \Norm{x-y}{} = d(x,y),
			\quad d(\alpha x, \alpha y)
			= \Norm{\alpha (x-y)}{} = |\alpha|\Norm{x-y}{} = |\alpha|d(x,y)
		\end{align}
		が成立する.逆に$X$上の距離$d$が絶対斉次的かつ平行移動不変であるとき,
		\begin{align}
			\Norm{x}{} \defeq d(x,0),\quad (\forall x \in X)
		\end{align}
		でノルムが定まる.実際$\Norm{\alpha x}{} = d(\alpha x,0) 
		= |\alpha|d(x,0) = |\alpha|\Norm{x}{}$かつ
		\begin{align}
			\quad \Norm{x+y}{} = d(x+y,0) = d(x,-y) 
			\leq d(x,0) + d(0,-y) = d(x,0) + d(y,0) = \Norm{x}{} + \Norm{y}{}
		\end{align}
		が成立する.
		\QED
	\end{prf}
	
	\begin{screen}
		\begin{thm}[ノルムで導入する距離位相は線型位相]
			$(X,\Norm{\cdot}{})$をノルム空間とするとき,
			$d(x,y) \defeq \Norm{x-y}{}$で定める距離$d$による距離位相は線型位相となる.
		\end{thm}
	\end{screen}
	
	\begin{prf}
		距離位相は$T_6$位相空間を定めるから$X$は定義\ref{def:topological_vector_space}の(tvs2)を満たす.また
		\begin{align}
			d(x+y,x'+y') \leq d(x+y,x'+y) + d(x'+y,x'+y') = d(x,x') + d(y,y')
		\end{align}
		より加法の連続性が得られ,
		\begin{align}
			d(\alpha x, \alpha'x') &\leq d(\alpha x, \alpha'x) + d(\alpha'x,\alpha'x') \\
			&= d((\alpha - \alpha') x, 0) + |\alpha'|d(x,x')
			= |\alpha-\alpha'|d(x,0) + |\alpha'|d(x,x')
		\end{align}
		よりスカラ倍の連続性も出る.
		\QED
	\end{prf}
	
	\begin{screen}
		\begin{thm}[位相線型空間の連結性]\label{thm:topological_vector_spaces_connected}
			位相線型空間は連結である.
		\end{thm}
	\end{screen}
	
	\begin{prf}
		零元のみの空間は密着空間であるから連結である.
		$X \neq \{0\}$を位相線型空間とするとき,任意に$a,b \in X$を取り
		\begin{align}
			f:[0,1] \ni t \longmapsto a + t(b - a) \in X
		\end{align}
		と定めれば$f$は$[0,1]$から$X$への連続写像である.実際,
		定理\ref{thm:continuity_of_translations_multiples}より
		$\Phi \ni t \longmapsto t(b-a)$が連続であるから
		\begin{align}
			g:[0,1] \ni t \longmapsto t(b-a)
		\end{align}
		は$[0,1]$の相対位相に関して連続であり,かつ$h:X \ni x \longmapsto a + x$もまた連続であるから
		$f = h \circ g$の連続性が従う.
		よって$X$は弧状連結であるから定理\ref{thm:connected_path_connected}より連結である.
		\QED
	\end{prf}
	
	\begin{screen}
		\begin{dfn}[位相線形空間の有界集合]\label{def:boundedness_in_tvs}
			$X$を位相線型空間,$E$を$X$の部分集合とする.0の任意の近傍$V$に対し
			或る$s = s(V) > 0$が存在して
			\begin{align}
				E \subset t V, \quad (\forall t > s)
			\end{align}
			となるとき,$E$は有界であるという.
		\end{dfn}
	\end{screen}
	
	\begin{screen}
		\begin{thm}
		\end{thm}
	\end{screen}
	
	\begin{screen}
		\begin{dfn}[局所基・局所凸・局所コンパクト・局所有界]
			$(X,\tau)$を位相線型空間とする.
			\begin{description}
				\item[(1)] $0 \in X$の基本近傍系を$X$の局所基(local base)と呼ぶ.
				\item[(2)] すべての元が凸集合であるような局所基が取れるとき,$X$は局所凸(locally convex)であるという.
				\item[(3)] $0 \in X$がコンパクトな近傍を持つとき,$X$は局所コンパクト(locally compact)であるという.
				\item[(4)] $0 \in X$が有界な近傍を持つとき,$X$は局所有界(locally bounded)であるという.
			\end{description}
		\end{dfn}
	\end{screen}
	
	\begin{screen}
		\begin{dfn}[$F$-空間・Frechet空間・ノルム空間]
			$(X,\tau)$を位相線型空間とする.
			平行移動不変距離$d$により$X$が距離化可能でかつ完備距離空間となるとき,
			$X$を$F$-空間と呼ぶ.局所凸な$F$-空間をFrechet空間と呼び
		\end{dfn}
	\end{screen}
	
	\begin{screen}
		\begin{dfn}[集合の線型演算]
			$X$を体$\Phi$上の位相線型空間,$A,B$を$X$の部分集合,$\alpha,\beta \in \Phi$とする.
			このとき
			\begin{align}
				\alpha A + \beta B \defeq \Set{\alpha a + \beta b}{a \in A,\ b \in B}
			\end{align}
			と書く.
		\end{dfn}
	\end{screen}
	
	\begin{screen}
		\begin{thm}
			$X$を位相線型空間,$A,B$を部分集合とする.
			\begin{description}
				\item[(1)] $\alpha \overline{A} = \overline{\alpha A}$
				\item[(2)] $\alpha (A^{\mathrm{o}}) = (\alpha A)^{\mathrm{o}}$
			\end{description}
		\end{thm}
	\end{screen}
	
	\begin{prf}\mbox{}
		\begin{description}
			\item[(1)] $\alpha = 0$或は$A = \emptyset$の場合は両辺が
				$\{0\}$或は$\emptyset$となって等号が成立する.
				$\alpha \neq 0$かつ$A \neq \emptyset$の場合,
				\begin{align}
					x \in \alpha \overline{A}
					\quad &\Longleftrightarrow \quad
					\alpha^{-1}x \in \overline{A} \\
					\quad &\Longleftrightarrow \quad
					\left(\alpha^{-1}x + V\right) \cap A \neq \emptyset, \quad 
						(\mbox{$\forall V$: neighborhood of 0}) \\
					\quad &\Longleftrightarrow \quad
					\left(x + V\right) \cap \alpha A \neq \emptyset, \quad 
						(\mbox{$\forall V$: neighborhood of 0}) \\
					\quad &\Longleftrightarrow \quad
					x \in \overline{\alpha A}
				\end{align}
				が成り立つ.
				
			\item[(2)] $\alpha = 0$或は$A = \emptyset$の場合は両辺が
				$\{0\}$或は$\emptyset$となって等号が成立する.
				$\alpha \neq 0$かつ$A \neq \emptyset$の場合,
				\begin{align}
					x \in \alpha (A^{\mathrm{o}})
					\quad &\Longleftrightarrow \quad
					\alpha^{-1}x \in A^{\mathrm{o}} \\
					\quad &\Longleftrightarrow \quad
					\mbox{$\exists V$: neighborhood of 0},\quad \alpha^{-1}x + V \subset A \\
					\quad &\Longleftrightarrow \quad
					\mbox{$\exists V$: neighborhood of 0},\quad x + V \subset \alpha A \\
					\quad &\Longleftrightarrow \quad
					x \in (\alpha A)^{\mathrm{o}}
				\end{align}
				が成り立つ.
				
		\end{description}
	\end{prf}
	
	\begin{screen}
		\begin{thm}[斉次距離で距離化可能なら距離と位相の有界性は一致する]
			位相線型空間$(X,\tau)$が斉次的な距離$d$で距離化可能である場合,
			$X$の部分集合の$d$-有界性と$\tau$-有界性は一致する.
		\end{thm}
	\end{screen}
	
	\begin{prf}
		任意の$\alpha>0,\ \delta>0$に対し,
		$B_{\delta}(0) \defeq \Set{x \in X}{d(x,0) < \delta}$とおけば斉次性より
		\begin{align}
			x \in \alpha B_{\delta}(0) 
			\quad \Longleftrightarrow \quad d\left( \alpha^{-1}x,0 \right) < \delta
			\quad \Longleftrightarrow \quad \alpha^{-1}d(x,0) < \delta
			\quad \Longleftrightarrow \quad x \in B_{\alpha\delta}(0)
		\end{align}
		が成立する.$\{B_r(0)\}_{r > 0}$は$X$の局所基となるから,
		$E \subset X$が$d$-有界のときも$\tau$-有界のときも
		$E \subset B_R(0)$を満たす$R > 0$が存在する.
		$E$が$d$-有界集合である場合,任意に0の近傍$V$を取れば
		或る$r > 0$で$B_r(0) \subset V$となり
		\begin{align}
			E \subset B_R(0) \subset B_t(0) = \frac{t}{r} B_r(0) \subset \frac{t}{r}V,
			\quad (\forall t > R)
		\end{align}
		が成立するから$E$は$\tau$-有界集合である.
		逆に$E$が$\tau$-有界集合であるとき,任意に$x \in X$を取れば
		\begin{align}
			E \subset B_R(0) \subset B_{d(x,0) + R}(x)
		\end{align}
		が成立するから$E$は$d$-有界集合である.
		\QED
	\end{prf}
	
	\begin{screen}
		\begin{thm}[位相線型空間上の同程度連続性]
		\label{thm:equicontinuity_on_topological_vector_spaces}
			$X,Y$を位相線型空間とし,$\zeta_X,\zeta_Y$をそれぞれ$X,Y$の零元とし,$\mathscr{F}$を$X$から$Y$への線型写像の族とする.
			また$\mathscr{B}_X,\mathscr{B}_Y$をそれぞれ$X,Y$の局所基とする.そして
			\begin{align}
				\mathscr{V}_X &\defeq \Set{V}{\exists B \in \mathscr{B}_X\, \left(\, 
					\Set{(x,y)}{x,y \in X \wedge y-x \in B} \subset V\, \right)}, \\
				\mathscr{V}_Y &\defeq \Set{V}{\exists B \in \mathscr{B}_Y\, \left(\, 
					\Set{(x,y)}{x,y \in Y \wedge y-x \in B} \subset V\, \right)}
			\end{align}
			で$X,Y$上の近縁系を定める.このとき
			\begin{description}
				\item[(a)] $\forall x \in X\, \forall V \in \mathscr{V}_Y\, \exists U \in \mathscr{V}_X\, \forall f \in \mathscr{F}\,
					\forall y \in X\, \left(\, (x,y) \in U \Longrightarrow (f(x),f(y)) \in V\, \right)$
					
				\item[(b)] $\forall V \in \mathscr{V}_Y\, \exists U \in \mathscr{V}_X\, \forall f \in \mathscr{F}\,
					\forall y \in X\, \left(\, (\zeta_X,y) \in U \Longrightarrow (\zeta_Y,f(y)) \in V\, \right)$
					
				\item[(c)] $\forall B \in \mathscr{B}_Y\, \exists C \in \mathscr{B}_X\, \forall f \in \mathscr{F}\, 
					\left(\, f \ast C \subset B\, \right)$
				
				\item[(d)] $\forall V \in \mathscr{V}_Y\, \exists U \in \mathscr{V}_X\, \forall f \in \mathscr{F}\,
					\forall x,y \in X\, \left(\, (x,y) \in U \Longrightarrow (f(x),f(y)) \in V\, \right)$
			\end{description}
			は全て同値である.
		\end{thm}
	\end{screen}
	
	式(a)は$\mathscr{F}$が同程度連続であるということを表す.
	
	式(b)は$\mathscr{F}$が零元において同程度連続であるということを表す.
	
	式(c)は$\mathscr{F}$の要素の像が一様に抑えられることを表す.
	
	式(d)は$\mathscr{F}$が一様同程度連続であるということを表す.
	
	この定理は,位相線型空間上の線型写像の集合については零元における同程度連続性から一様同程度連続性が導かれることを主張しているが,
	同じ主張は位相群で成立する.その場合$\mathscr{F}$は群準同型写像の集合とすればよい.
	
	\begin{sketch}
		(a)から(b)は直ちに従う.
		(b)が成立しているとする.$B$を$\mathscr{B}_Y$の要素として取り,
		\begin{align}
			V_B \defeq \Set{(x,y)}{x,y \in Y \wedge y - x \in B}
		\end{align}
		とおくと,$\mathscr{V}_X$の或る要素$U$が取れて
		\begin{align}
			\forall f \in \mathscr{F}\, \forall y \in X\, (\, (\zeta_X,y) \in U
			&\Longrightarrow (\zeta_Y,f(y)) \in V_B \\
			&\Longrightarrow f(y) \in B\, )
		\end{align}
		が成立する.ゆえに
		\begin{align}
			C \subset \Set{y}{y \in X \wedge (\zeta_X,y) \in U}
		\end{align}
		なる$\mathscr{B}_X$の要素$C$を取れば
		\begin{align}
			\forall f \in \mathscr{F}\, \left(\, f \ast C \subset B\, \right)
		\end{align}
		が従う.
		
		次に(c)が成立しているとする.$V$を$\mathscr{V}_Y$の要素とすると
		\begin{align}
			\Set{(x,y)}{x,y \in Y \wedge y - x \in B} \subset V
		\end{align}
		を満たす$\mathscr{B}_Y$の要素$B$が取れる.$B$に対し
		\begin{align}
			\forall f \in \mathscr{F}\, \left(\, f \ast C \subset B\, \right)
		\end{align}
		を満たす$\mathscr{B}_X$の要素$C$が取れるが,
		\begin{align}
			U \defeq \Set{(x,y)}{x,y \in X \wedge y - x \in C}
		\end{align}
		とおくと
		\begin{align}
			\forall f \in \mathscr{F}\, \forall x,y \in X\, (\, (x,y) \in U 
			&\Longrightarrow y - x \in C \\
			&\Longrightarrow f(y) - f(x) \in B \\
			&\Longrightarrow (f(x),f(y)) \in V\, )
		\end{align}
		が従う.一様同程度連続ならば同程度連続であるから定理の主張が得られる.
		\QED
	\end{sketch}
	
	\begin{screen}
		\begin{thm}[同程度連続な写像族の有界性]
			$X,Y$を位相線形空間,$\mathscr{F}$を$X$から$Y$への連続線型写像の族とする.
			$\mathscr{F}$が同程度連続であるとき,
		\end{thm}
	\end{screen}
	
	\begin{screen}
		\begin{thm}[Banach-Steinhaus]
			
		\end{thm}
	\end{screen}
	
	\begin{screen}
		\begin{thm}[開写像原理]
			$X$
		\end{thm}
	\end{screen}
	この節では,$\left(X,\sigma_X\right)$を群とするとき,$X$の要素$x$の逆元を
	\begin{align}
		-x
	\end{align}
	と書く.また扱う線型空間はすべて複素数体か実数体をスカラーとして考える.以降現れる$\Phi$は
	\begin{align}
		\Phi \defeq \C
	\end{align}
	か
	\begin{align}
		\Phi \defeq \R
	\end{align}
	で定められたものと考える.すなわち,$\mathscr{O}_\Phi$は$\mathscr{O}_\C$または$\mathscr{O}_\R$を指す.
	
\subsection{線型位相}
	\begin{screen}
		\begin{dfn}[位相線型空間]\label{def:topological_vector_space}
			$\left(\left(X,\sigma_X\right),(\Phi,+,\bullet),s\right)$を線型空間とし,
			$\mathscr{O}_X$を$X$上の位相とする.また
			$\mathscr{O}_{X \times X}$を$\mathscr{O}_X$から作られる$X \times X$上の積位相とし,
			$\mathscr{O}_{\Phi \times X}$を$\mathscr{O}_\Phi$と$\mathscr{O}_X$から作られる$\Phi \times X$上の積位相とする.
			\begin{description}
				\item[(tvs1)] $\sigma_X$が$\mathscr{O}_{X \times X}/\mathscr{O}_X$-連続である.
				\item[(tvs2)] $s$が$\mathscr{O}_{\Phi \times X}/\mathscr{O}_X$-連続である.
			\end{description}
			が満たされるとき,$\mathscr{O}_X$を$X$上の{\bf 線型位相}\index{せんけいいそう@線型位相}{\bf (vector topology)}と呼び,
			\begin{align}
				\left(\left(X,\sigma_X\right),(\Phi,+,\bullet),s,\mathscr{O}_X\right)
				\label{pair_topological_vector_space}
			\end{align}
			を{\bf 位相線型空間}\index{いそうせんけいくうかん@位相線型空間}
			{\bf (topological vector space)}と呼ぶ.
		\end{dfn}
	\end{screen}
	
	$\left(\left(X,\sigma_X\right),(\Phi,+,\bullet),s,\mathscr{O}_X\right)$を位相線型空間とするとき,
	$\sigma_X$は連続であるから,$a$を$X$の任意の要素として
	\begin{align}
		X \ni x \longmapsto \sigma_X(x,a)
	\end{align}
	なる写像を
	\begin{align}
		\sigma_X^a
	\end{align}
	とおけば,これは$\mathscr{O}_X/\mathscr{O}_X$-連続である.さらにこのとき,
	\begin{align}
		\sigma_X^{-a}
	\end{align}
	なる写像は$\sigma_X^a$の逆写像であって,かつ$\mathscr{O}_X/\mathscr{O}_X$-連続なので,
	$\sigma_X^a$は$\mathscr{O}_X$に関する同相写像である.つまり,{\bf 位相線型空間の平行移動は同相である.}
	また,$\sigma_X$は可換であるから
	\begin{align}
		X \ni x \longmapsto \sigma_X(a,x)
	\end{align}
	なる写像は$\sigma_X^a$に一致する.ゆえにこれも同相である.
	
	同様に$x$を$X$の任意の要素とすると
	\begin{align}
		\Phi \ni \alpha \longmapsto s(\alpha,x)
	\end{align}
	なる写像は$\mathscr{O}_\Phi/\mathscr{O}_X$-連続であり,$\alpha$を$\Phi$の任意の要素とすると
	\begin{align}
		X \ni x \longmapsto s(\alpha,x)
		\label{partial_continuity_of_summation_of_topological_vector_spaces_1}
	\end{align}
	なる写像は$\mathscr{O}_X/\mathscr{O}_X$-連続である.とくに
	\begin{align}
		\alpha \neq 0
	\end{align}
	ならば
	\begin{align}
		X \ni x \longmapsto s(\alpha^{-1},x)
	\end{align}
	なる写像は(\refeq{partial_continuity_of_summation_of_topological_vector_spaces_1})の逆写像であり,
	かつ$\mathscr{O}_X/\mathscr{O}_X$-連続であるから,(\refeq{partial_continuity_of_summation_of_topological_vector_spaces_1})
	は$\mathscr{O}_X$に関する同相写像である.
	つまり,{\bf 位相線型空間のスケール変換は,$0$倍でない限り同相である.}
	
	\begin{screen}
		\begin{thm}[位相線型空間は位相群]\label{thm:topological_vector_space_is_topological_group}
			$\left(\left(X,\sigma_X\right),(\Phi,+,\bullet),s,\mathscr{O}_X\right)$を位相線型空間とするとき,
			$\left(\left(X,\sigma_X\right),\mathscr{O}_X\right)$は位相群である.
		\end{thm}
	\end{screen}
	
	\begin{sketch}
		定義より$\sigma_X$は$\mathscr{O}_{X \times X}/\mathscr{O}_X$-連続である.また
		定理\ref{thm:inverse_element_equals_to_its_minus}より
		\begin{align}
			X \ni x \longmapsto -x
		\end{align}
		なる写像は
		\begin{align}
			X \ni x \longmapsto s(-1,x)
		\end{align}
		に一致するので$\mathscr{O}_X/\mathscr{O}_X$-連続である.ゆえに
		$\left(\left(X,\sigma_X\right),\mathscr{O}_X\right)$は位相群である.
		\QED
	\end{sketch}
	
	\begin{screen}
		\begin{dfn}[均衡集合]
			$\left(\left(X,\sigma_X\right),(\Phi,+,\bullet),s\right)$を線型空間とし,
			$A$を$X$の部分集合とする.このとき,$\Phi$の任意の要素$\alpha$に対して
			\begin{align}
				|\alpha| \leq 1 \Longrightarrow \Set{s(\alpha,x)}{x \in A} \subset A
			\end{align}
			が成り立つならば,$A$を$\left(\left(X,\sigma_X\right),(\Phi,+,\bullet),s\right)$の
			{\bf 均衡集合}\index{きんこうしゅうごう@均衡集合}{\bf (balanced set)}と呼ぶ.
		\end{dfn}
	\end{screen}
	
	$A$が均衡であることを直感的に書けば
	\begin{align}
		\forall \alpha \in \Phi\,
		\left(\, |\alpha| \leq 1 \Longrightarrow \alpha A \subset A\, \right)
	\end{align}
	ということになる.例えば$X$が$\R^2$ならば均衡集合とは円のことを指し,$X$が$\R^3$ならば球のことを指す.
	$X$の任意の要素$x$に対して
	\begin{align}
		-x = s(-1,x)
	\end{align}
	が成り立つので,均衡とは逆元で閉じていることの拡張概念である.
	定理\ref{thm:there_exists_a_local_base_whose_elements_are_closed_under_inversion}の拡張として次を得る.
	
	\begin{screen}
		\begin{thm}[均衡な局所基が取れる]
			$\left(\left(X,\sigma_X\right),(\Phi,+,\bullet),s,\mathscr{O}_X\right)$を位相線型空間とする.
			このとき,$X$の零元の基本近傍系で,そのすべての要素が均衡集合であるものが取れる.
		\end{thm}
	\end{screen}
	
	\begin{sketch}
		$0_X$を$X$の零元とし,$v$を$0_X$の近傍とする.$s$は$\mathscr{O}_{\Phi \times X}/\mathscr{O}_X$-連続であるから,
		\begin{align}
			\Set{\alpha \in \Phi}{|\alpha| < \delta} \times w \subset s^{-1} \ast v
		\end{align}
		を満たす正の実数$\delta$と$0_X$の開近傍$w$が取れる.ここで
		\begin{align}
			u \defeq \bigcup_{\substack{\alpha \in \Phi \\ 0 < |\alpha| < \delta}} \Set{s(\alpha,x)}{x \in w}
		\end{align}
		とおくと,$u$は$0_X$の均衡な開近傍であって,
		\begin{align}
			u \subset v
		\end{align}
		を満たす.
		\begin{description}
			\item[step1] $u$は開集合である.実際
				\begin{align}
					0 < |\alpha| < \delta
				\end{align}
				なる$\alpha$に対して
				\begin{align}
					X \ni x \longmapsto s(\alpha,x)
				\end{align}
				なる写像を$s^\alpha$とおけば,$s^\alpha$は$\mathscr{O}_X$に関して同相であって,かつ
				\begin{align}
					s^\alpha \ast w = \Set{s(\alpha,x)}{x \in w}
				\end{align}
				であるから,
				\begin{align}
					\Set{s(\alpha,x)}{x \in w} \in \mathscr{O}_X
				\end{align}
				が成り立つ.また
				\begin{align}
					0_X \in w
				\end{align}
				より
				\begin{align}
					0_X \in \Set{s(\alpha,x)}{x \in w}
				\end{align}
				も成り立つ.ゆえに$u$は$0_X$の開近傍である.
				
			\item[step2] $u$が$\left(\left(X,\sigma_X\right),(\Phi,+,\bullet),s\right)$の均衡集合であることを示す.
				いま$\beta$を
				\begin{align}
					|\beta| \leq 1
				\end{align}
				なる$\Phi$の要素とし,$x$を$u$の要素とする.このとき
				\begin{align}
					x = s(\alpha,y)
				\end{align}
				かつ
				\begin{align}
					0 < |\alpha| < \delta
				\end{align}
				かつ
				\begin{align}
					y \in w
				\end{align}
				を満たす$\alpha$と$y$が取れて
				\begin{align}
					s(\beta,x) = s(\beta,s(\alpha,y) = s(\beta \cdot \alpha,y)
				\end{align}
				が成り立つ.
				\begin{align}
					\beta = 0
				\end{align}
				ならば
				\begin{align}
					s(\beta \cdot \alpha,y) = 0_X \in u
				\end{align}
				が成り立ち,
				\begin{align}
					\beta \neq 0
				\end{align}
				ならば
				\begin{align}
					0 < |\beta \cdot \alpha| < \delta
				\end{align}
				が成り立つので
				\begin{align}
					s(\beta \cdot \alpha,y) \in s^{\beta \cdot \alpha} \ast w \subset u
				\end{align}
				が従う.ゆえに
				\begin{align}
					|\beta| \leq 1 \Longrightarrow \Set{s(\beta,x)}{x \in u} \subset u
				\end{align}
				が成り立つ.ゆえに$u$は均衡している.
				
			\item[step3] 定理\ref{thm:union_of_subsets_is_subclass}より
				\begin{align}
					u \subset v
				\end{align}
				が成立する.
		\end{description}
		$0_X$の近傍の全体を
		\begin{align}
			\mathscr{B}
		\end{align}
		とおき,$\mathscr{B}$の要素$v$に対して,$v$に含まれる均衡な$0_X$の近傍の全体,つまり
		\begin{align}
			\Set{u}{u \in \mathscr{B} \wedge u \subset v \wedge 
			\forall \alpha \in \Phi\, \forall x \in u\, \left(\, |\alpha| \leq 1 \Longrightarrow
			s(\alpha,x) \in u\, \right)}
		\end{align}
		なる集合を対応させる関係を$h$とおくと,上の結果から
		\begin{align}
			\forall v \in \mathscr{B}\, \left(\, h(v) \neq \emptyset\, \right)
		\end{align}
		が成り立つ.ゆえに定理\ref{thm:direct_product_of_non_empty_sets_is_not_empty}より
		\begin{align}
			f \in \prod_{v \in \mathscr{B}} h(v)
		\end{align}
		なる集合$f$が取れる.そして
		\begin{align}
			\left\{f(v)\right\}_{v \in \mathscr{B}}
		\end{align}
		は$0_X$の基本近傍系であり,その全ての要素は均衡している.
		\QED
	\end{sketch}
	
	位相群に対して示した一様位相的性質を位相線型空間に対しても述べ直しておく.
	
	\begin{screen}
		\begin{thm}[位相線型空間の位相は局所基で決まる]
		\label{thm:topologies_on_topological_vector_spaces_are_invariant}
			$\left(\left(X,\sigma_X\right),(\Phi,+,\bullet),s,\mathscr{O}_X\right)$を位相線型空間とし,
			$\mathscr{B}$を$\left(X,\sigma_X\right)$の単位元の基本近傍系とする.また$x$を$X$の任意の要素とする.このとき,
			\begin{align}
				\Set{\sigma_X(x,y)}{y \in b}
			\end{align}
			を$\mathscr{B}$の全ての要素に亘って取ったものの全体は$x$の基本近傍系である.
		\end{thm}
	\end{screen}
	
	言い換えれば,
	\begin{align}
		\Set{u}{\exists b \in \mathscr{B}\, 
		\forall z\, \left[\, z \in u \Longleftrightarrow \exists y \in b\, \left(\, z=\sigma_X(x,y)\, \right) \, \right]}
		\label{fom:thm_topologies_on_topological_vector_spaces_are_invariant}
	\end{align}
	が$x$の基本近傍系であるということである.
	
	\begin{sketch}
		定理\ref{thm:topological_vector_space_is_topological_group}より
		$\left(\left(X,\sigma_X\right),\mathscr{O}_X\right)$は位相群であるから,
		定理\ref{thm:topologies_on_topological_groups_are_invariant}より
		(\refeq{fom:thm_topologies_on_topological_vector_spaces_are_invariant})は$x$の基本近傍系である.
		\QED
	\end{sketch}
	
	\begin{screen}
		\begin{thm}[位相線型空間は一様化可能である]\label{thm:topological_vector_spaces_are_uniformazable}
			$\left(\left(X,\sigma_X\right),(\Phi,+,\bullet),s,\mathscr{O}_X\right)$を位相線型空間とし,
			$\mathscr{B}$を$\left(X,\sigma_X\right)$の単位元の基本近傍系とし,そのすべての要素が均衡しているとする.ここで
			\begin{align}
				\mathscr{U} \defeq \Set{u}{\exists b \in \mathscr{B}\,
				\left[\, \forall x,y \in X\, \left(\, (x,y) \in u \Longleftrightarrow\sigma_X(x,-y) \in b\, \right)\, \right]}
			\end{align}
			とおいて
			\begin{align}
				\mathscr{V} \defeq \Set{v}{v \subset X \times X \wedge \exists u \in \mathscr{U}\, \left(\, u \subset v\, \right)}
			\end{align}
			と定めると,$\mathscr{V}$は$X$上の近縁系であって$\mathscr{O}_X$と両立する.
		\end{thm}
	\end{screen}
	
	\begin{sketch}
		$\mathscr{B}$の要素$b$は均衡しているので
		\begin{align}
			\forall x \in b\, (\, -x \in b\, )
		\end{align}
		を満たす.また$\left(\left(X,\sigma_X\right),\mathscr{O}_X\right)$は位相群であるから,
		定理\ref{thm:topological_groups_are_uniformazable}より
		$\mathscr{V}$は$X$上の近縁系であって$\mathscr{O}_X$と両立する.
		\QED
	\end{sketch}
	
	\begin{screen}
		\begin{thm}[位相線型空間は$T_0$ならばTychonoffである]
			$\left(\left(X,\sigma_X\right),(\Phi,+,\bullet),s,\mathscr{O}_X\right)$を位相線型空間とするとき,
			$\left(X,\mathscr{O}_X\right)$は$T_0$ならばTychonoffである.
		\end{thm}
	\end{screen}
	
	\begin{sketch}
		定理\ref{thm:topological_vector_spaces_are_uniformazable}と
		定理\ref{thm:T_0_iff_T_2_on_uniform_topological_space}から従う
		\QED
	\end{sketch}
	
	\begin{screen}
		\begin{thm}[位相線型空間の連結性]\label{thm:topological_vector_spaces_connected}
			位相線型空間は連結である.
		\end{thm}
	\end{screen}
	
	\begin{prf}
		零元のみの空間は密着空間であるから連結である.
		$X \neq \{0\}$を位相線型空間とするとき,任意に$a,b \in X$を取り
		\begin{align}
			f:[0,1] \ni t \longmapsto a + t(b - a) \in X
		\end{align}
		と定めれば$f$は$[0,1]$から$X$への連続写像である.実際,
		定理\ref{thm:continuity_of_translations_multiples}より
		$\Phi \ni t \longmapsto t(b-a)$が連続であるから
		\begin{align}
			g:[0,1] \ni t \longmapsto t(b-a)
		\end{align}
		は$[0,1]$の相対位相に関して連続であり,かつ$h:X \ni x \longmapsto a + x$もまた連続であるから
		$f = h \circ g$の連続性が従う.
		よって$X$は弧状連結であるから定理\ref{thm:connected_path_connected}より連結である.
		\QED
	\end{prf}
	
	\begin{screen}
		\begin{dfn}[位相線形空間の有界集合]\label{def:boundedness_in_tvs}
			$X$を位相線型空間,$E$を$X$の部分集合とする.$0$の任意の近傍$V$に対し
			\begin{align}
				E \subset t V
			\end{align}
			を満たす$t > 0$が取れるとき,$E$は有界であるという.
		\end{dfn}
	\end{screen}
	
	\begin{screen}
		\begin{thm}
		\end{thm}
	\end{screen}
	
	\begin{screen}
		\begin{dfn}[局所基・局所凸・局所コンパクト・局所有界]
			$(X,\tau)$を位相線型空間とする.
			\begin{description}
				\item[(1)] $0 \in X$の基本近傍系を$X$の局所基(local base)と呼ぶ.
				\item[(2)] すべての元が凸集合であるような局所基が取れるとき,$X$は局所凸(locally convex)であるという.
				\item[(3)] $0 \in X$がコンパクトな近傍を持つとき,$X$は局所コンパクト(locally compact)であるという.
				\item[(4)] $0 \in X$が有界な近傍を持つとき,$X$は局所有界(locally bounded)であるという.
			\end{description}
		\end{dfn}
	\end{screen}
	
	\begin{screen}
		\begin{dfn}[$F$-空間・Frechet空間・ノルム空間]
			$(X,\tau)$を位相線型空間とする.
			平行移動不変距離$d$により$X$が距離化可能でかつ完備距離空間となるとき,
			$X$を$F$-空間と呼ぶ.局所凸な$F$-空間をFrechet空間と呼び
		\end{dfn}
	\end{screen}
	
	\begin{screen}
		\begin{thm}
			$X$を位相線型空間,$A,B$を部分集合とする.
			\begin{description}
				\item[(1)] $\alpha \overline{A} = \overline{\alpha A}$
				\item[(2)] $\alpha (A^{\mathrm{o}}) = (\alpha A)^{\mathrm{o}}$
			\end{description}
		\end{thm}
	\end{screen}
	
	\begin{prf}\mbox{}
		\begin{description}
			\item[(1)] $\alpha = 0$或は$A = \emptyset$の場合は両辺が
				$\{0\}$或は$\emptyset$となって等号が成立する.
				$\alpha \neq 0$かつ$A \neq \emptyset$の場合,
				\begin{align}
					x \in \alpha \overline{A}
					\quad &\Longleftrightarrow \quad
					\alpha^{-1}x \in \overline{A} \\
					\quad &\Longleftrightarrow \quad
					\left(\alpha^{-1}x + V\right) \cap A \neq \emptyset, \quad 
						(\mbox{$\forall V$: neighborhood of 0}) \\
					\quad &\Longleftrightarrow \quad
					\left(x + V\right) \cap \alpha A \neq \emptyset, \quad 
						(\mbox{$\forall V$: neighborhood of 0}) \\
					\quad &\Longleftrightarrow \quad
					x \in \overline{\alpha A}
				\end{align}
				が成り立つ.
				
			\item[(2)] $\alpha = 0$或は$A = \emptyset$の場合は両辺が
				$\{0\}$或は$\emptyset$となって等号が成立する.
				$\alpha \neq 0$かつ$A \neq \emptyset$の場合,
				\begin{align}
					x \in \alpha (A^{\mathrm{o}})
					\quad &\Longleftrightarrow \quad
					\alpha^{-1}x \in A^{\mathrm{o}} \\
					\quad &\Longleftrightarrow \quad
					\mbox{$\exists V$: neighborhood of 0},\quad \alpha^{-1}x + V \subset A \\
					\quad &\Longleftrightarrow \quad
					\mbox{$\exists V$: neighborhood of 0},\quad x + V \subset \alpha A \\
					\quad &\Longleftrightarrow \quad
					x \in (\alpha A)^{\mathrm{o}}
				\end{align}
				が成り立つ.
				
		\end{description}
	\end{prf}
	
	\begin{screen}
		\begin{thm}[位相線型空間上の同程度連続性]
		\label{thm:equicontinuity_on_topological_vector_spaces}
			$X,Y$を位相線型空間とし,$\zeta_X,\zeta_Y$をそれぞれ$X,Y$の零元とし,$\mathscr{F}$を$X$から$Y$への線型写像の族とする.
			また$\mathscr{B}_X,\mathscr{B}_Y$をそれぞれ$X,Y$の局所基とする.そして
			\begin{align}
				\mathscr{V}_X &\defeq \Set{V}{\exists B \in \mathscr{B}_X\, \left(\, 
					\Set{(x,y)}{x,y \in X \wedge y-x \in B} \subset V\, \right)}, \\
				\mathscr{V}_Y &\defeq \Set{V}{\exists B \in \mathscr{B}_Y\, \left(\, 
					\Set{(x,y)}{x,y \in Y \wedge y-x \in B} \subset V\, \right)}
			\end{align}
			で$X,Y$上の近縁系を定める.このとき
			\begin{description}
				\item[(a)] $\forall x \in X\, \forall V \in \mathscr{V}_Y\, \exists U \in \mathscr{V}_X\, \forall f \in \mathscr{F}\,
					\forall y \in X\, \left(\, (x,y) \in U \Longrightarrow (f(x),f(y)) \in V\, \right)$
					
				\item[(b)] $\forall V \in \mathscr{V}_Y\, \exists U \in \mathscr{V}_X\, \forall f \in \mathscr{F}\,
					\forall y \in X\, \left(\, (\zeta_X,y) \in U \Longrightarrow (\zeta_Y,f(y)) \in V\, \right)$
					
				\item[(c)] $\forall B \in \mathscr{B}_Y\, \exists C \in \mathscr{B}_X\, \forall f \in \mathscr{F}\, 
					\left(\, f \ast C \subset B\, \right)$
				
				\item[(d)] $\forall V \in \mathscr{V}_Y\, \exists U \in \mathscr{V}_X\, \forall f \in \mathscr{F}\,
					\forall x,y \in X\, \left(\, (x,y) \in U \Longrightarrow (f(x),f(y)) \in V\, \right)$
			\end{description}
			は全て同値である.
		\end{thm}
	\end{screen}
	
	式(a)は$\mathscr{F}$が同程度連続であるということを表す.
	
	式(b)は$\mathscr{F}$が零元において同程度連続であるということを表す.
	
	式(c)は$\mathscr{F}$の要素の像が一様に抑えられることを表す.
	
	式(d)は$\mathscr{F}$が一様同程度連続であるということを表す.
	
	この定理は,位相線型空間上の線型写像の集合については零元における同程度連続性から一様同程度連続性が導かれることを主張しているが,
	同じ主張は位相群で成立する.その場合$\mathscr{F}$は群準同型写像の集合とすればよい.
	
	\begin{sketch}
		(a)から(b)は直ちに従う.
		(b)が成立しているとする.$B$を$\mathscr{B}_Y$の要素として取り,
		\begin{align}
			V_B \defeq \Set{(x,y)}{x,y \in Y \wedge y - x \in B}
		\end{align}
		とおくと,$\mathscr{V}_X$の或る要素$U$が取れて
		\begin{align}
			\forall f \in \mathscr{F}\, \forall y \in X\, (\, (\zeta_X,y) \in U
			&\Longrightarrow (\zeta_Y,f(y)) \in V_B \\
			&\Longrightarrow f(y) \in B\, )
		\end{align}
		が成立する.ゆえに
		\begin{align}
			C \subset \Set{y}{y \in X \wedge (\zeta_X,y) \in U}
		\end{align}
		なる$\mathscr{B}_X$の要素$C$を取れば
		\begin{align}
			\forall f \in \mathscr{F}\, \left(\, f \ast C \subset B\, \right)
		\end{align}
		が従う.
		
		次に(c)が成立しているとする.$V$を$\mathscr{V}_Y$の要素とすると
		\begin{align}
			\Set{(x,y)}{x,y \in Y \wedge y - x \in B} \subset V
		\end{align}
		を満たす$\mathscr{B}_Y$の要素$B$が取れる.$B$に対し
		\begin{align}
			\forall f \in \mathscr{F}\, \left(\, f \ast C \subset B\, \right)
		\end{align}
		を満たす$\mathscr{B}_X$の要素$C$が取れるが,
		\begin{align}
			U \defeq \Set{(x,y)}{x,y \in X \wedge y - x \in C}
		\end{align}
		とおくと
		\begin{align}
			\forall f \in \mathscr{F}\, \forall x,y \in X\, (\, (x,y) \in U 
			&\Longrightarrow y - x \in C \\
			&\Longrightarrow f(y) - f(x) \in B \\
			&\Longrightarrow (f(x),f(y)) \in V\, )
		\end{align}
		が従う.一様同程度連続ならば同程度連続であるから定理の主張が得られる.
		\QED
	\end{sketch}
	
	\begin{screen}
		\begin{thm}[同程度連続な写像族の有界性]
			$X,Y$を位相線形空間,$\mathscr{F}$を$X$から$Y$への連続線型写像の族とする.
			$\mathscr{F}$が同程度連続であるとき,
		\end{thm}
	\end{screen}
	
	\begin{screen}
		\begin{thm}[Banach-Steinhaus]
			
		\end{thm}
	\end{screen}
	
	\begin{screen}
		\begin{thm}[開写像原理]
			$X$
		\end{thm}
	\end{screen}
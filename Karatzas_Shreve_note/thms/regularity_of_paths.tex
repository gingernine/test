\section{$RCLL$修正}
	本節の主題は劣マルチンゲールが或る条件下で$RCLL$な修正を持つということである.
	いつも通り$(\Omega,\mathscr{F},P)$を確率空間とし,$\mathbf{T} \defeq [0,\infty[$とし,
	$\{\mathscr{F}_t\}_{t \in \mathbf{T}}$を$\mathscr{F}$に付随するフィルトレーションとし,
	$X$を$(\Omega,\mathscr{F},P)$上の$\{\mathscr{F}_t\}_{t \in \mathbf{T}}$-劣マルチンゲールとする.
	
	いま$\lambda$を正の実数とし,$N$を$0$でない自然数とする.また$[0,N]$の稠密な部分集合を
	\begin{align}
		D^N \defeq \bigcup_{n \in \Natural} \Set{\frac{k}{2^n}}{k \in \{0,1,\cdots,N \cdot 2^n\}}
	\end{align}
	により定める.このとき
	\begin{align}
		\left\{\lambda < \sup{t \in D^N}{X_t}\right\}
		= \bigcup_{n \in \Natural} \left\{\lambda < \max_{k \in \left\{0,1,2,\cdots,N \cdot 2^n\right\}} X_{\frac{k}{2^n}T}\right\}
	\end{align}
	が成り立つ.ちなみにこの式から
	\begin{align}
		\Omega \ni \omega \longmapsto \sup{t \in D^N}X_t(\omega)
	\end{align}
	が$\mathscr{F}_N/\borel{[-\infty,\infty]}$-可測であることが従う.$n$を自然数として
	\begin{align}
		E_n \defeq  \left\{\lambda < \max_{k \in \left\{0,1,2,\cdots,N \cdot 2^n\right\}} X_{\frac{k}{2^n}T}\right\}
	\end{align}
	とおき,$E_n$をさらに分解して
	\begin{align}
		E_n^0 \defeq \left\{\lambda < X_0\right\}
	\end{align}
	及び,$\left\{1,2,\cdots,N \cdot 2^n\right\}$の各要素$k$に対して
	\begin{align}
		E_n^k \defeq \left\{ \max_{m \in \left\{0,1,2,\cdots,k-1\right\}} X_{\frac{m}{2^n}} \leq \lambda \right\} \cap \left\{\lambda < X_{\frac{k}{2^n}}\right\}
	\end{align}
	とおく.$k$を$\left\{0,1,\cdots,N \cdot 2^n\right\}$の要素とすれば
	\begin{align}
		E_n^k \in \mathscr{F}_{\frac{k}{2^n}}
	\end{align}
	が成立するので,劣マルチンゲール性から
	\begin{align}
		\int_{E_n^k} X_{\frac{k}{2^n}}\ dP \leq \int_{E_n^k} X_N\ dP
	\end{align}
	が成り立つ.他方で
	\begin{align}
		\forall \omega \in E_n^k\, \left(\, \lambda < X_{\frac{k}{2^n}}(\omega)\, \right)
	\end{align}
	が成り立つから
	\begin{align}
		P(E_n^k) \leq \frac{1}{\lambda} \cdot \int_{E_n^k} X_{\frac{k}{2^n}}\ dP
	\end{align}
	が成立する.これにより
	\begin{align}
		P(E_n) = \sum_{k=0}^{2^n} P(E_n^k)
		\leq \frac{1}{\lambda} \cdot \sum_{k=0}^{N \cdot 2^n} \int_{E_n^k} X_{\frac{m}{2^n}}\ dP
		\leq \frac{1}{\lambda} \cdot \sum_{k=0}^{N \cdot 2^n} \int_{E_n^k} X_N\ dP
		= \frac{1}{\lambda} \cdot \int_{E_n} X_N\ dP
		\label{fom:Doob_upper_bound_inequality_1}
	\end{align}
	が成立する.$\{E_n\}_{n \in \Natural}$は単調に増大して
	\begin{align}
		\left\{\lambda < \sup{t \in D^N}{X_t}\right\}
	\end{align}
	に一致するので,(\refeq{fom:Doob_upper_bound_inequality_1})で$n \longrightarrow \infty$として
	\begin{align}
		P\left(\lambda < \sup{t \in D^N}{X_t}\right)
		\leq \frac{1}{\lambda} \cdot \int_{\left\{\lambda < \sup{t \in D^N}{X_t}\right\}} X_N\ dP
	\end{align}
	が成立する.さらに右辺は
	\begin{align}
		\frac{1}{\lambda} \cdot E\left(X_N^+\right)
	\end{align}
	で抑えられるので,以上で次を得た.
	
	\begin{screen}
		\begin{thm}[Doobの上限不等式]\label{thm:Doob_sup_bounded_inequality}
			$(\Omega,\mathscr{F},P)$を確率空間とし,$\mathbf{T} \defeq [0,\infty[$とし,
			$\{\mathscr{F}_t\}_{t \in \mathbf{T}}$を$\mathscr{F}$に付随するフィルトレーションとし,
			$X$を$(\Omega,\mathscr{F},P)$上の$\{\mathscr{F}_t\}_{t \in \mathbf{T}}$-劣マルチンゲールとする.
			このとき,$\lambda$と正の実数とし,$N$を$0$でない自然数として
			\begin{align}
				D^N \defeq \bigcup_{n \in \Natural} \Set{\frac{k}{2^n}}{k \in \{0,1,\cdots,N \cdot 2^n\}}
			\end{align}
			とおけば,
			\begin{align}
				P\left(\lambda < \sup{t \in D^N}{X_t}\right)
				\leq \frac{1}{\lambda} \cdot E\left(X_N^+\right).
			\end{align}
		\end{thm}
	\end{screen}
	
	上の設定をそのままにして,次は
	\begin{align}
		P\left(\inf{t \in D^N}{X_t} < -\lambda\right)
		\leq \frac{1}{\lambda} \cdot \left[E\left(X_N^+\right) - E(X_0)\right]
	\end{align}
	が成り立つことを示す.今度は
	\begin{align}
		E_n \defeq  \left\{\min_{k \in \left\{0,1,2,\cdots,N \cdot 2^n\right\}} X_{\frac{k}{2^n}} < -\lambda\right\}
	\end{align}
	とおいて,また
	\begin{align}
		E_n^0 \defeq \left\{X_0 < -\lambda\right\}
	\end{align}
	及び,$\left\{1,2,\cdots,N \cdot 2^n\right\}$の各要素$k$に対して
	\begin{align}
		E_n^k \defeq \left\{ -\lambda \leq \min_{m \in \left\{0,1,2,\cdots,k-1\right\}} X_{\frac{m}{2^n}} \right\} \cap \left\{ X_{\frac{k}{2^n}} < -\lambda \right\}
	\end{align}
	とおく.ここで
	\begin{align}
		\Omega \ni \omega \longmapsto
		\begin{cases}
			0 & \mbox{if } \omega \in E_n^0 \\
			\frac{1}{2^n} & \mbox{if } \omega \in E_n^1 \\
			\frac{2}{2^n} & \mbox{if } \omega \in E_n^2 \\
			\vdots & \\
			\frac{N \cdot 2^n - 1}{2^n} & \mbox{if } \omega \in E_n^{N \cdot 2^n - 1} \\
			\frac{N \cdot 2^n}{2^n} & \mbox{if } \omega \in E_n^{N \cdot 2^n} \\
			N & \mbox{if } \omega \in \Omega \backslash E_n
		\end{cases}
	\end{align}
	なる関係を$\tau$とすれば,$\tau$は$\{\mathscr{F}_t\}_{t \in \mathbf{T}}$-停止時刻である.
	任意抽出定理より
	\begin{align}
		E(X_0) \leq E(X_\tau)
	\end{align}
	が成立し,$\tau$の定め方より
	\begin{align}
		E(X_\tau) = \sum_{k=0}^{N \cdot 2^n} \int_{E_n^k} X_{\frac{k}{2^n}}\ dP + \int_{\Omega \backslash E_n} X_N\ dP
		\leq \sum_{k=0}^{N \cdot 2^n} \int_{E_n^k} X_{\frac{k}{2^n}}\ dP + E\left(X_N^+\right)
	\end{align}
	が成立する.他方で
	\begin{align}
		\forall \omega \in E_n^k\, \left(\, X_{\frac{k}{2^n}}(\omega) < -\lambda\, \right)
	\end{align}
	から
	\begin{align}
		\int_{E_n^k} X_{\frac{k}{2^n}}\ dP \leq -\lambda \cdot P(E_n^k)
	\end{align}
	が成り立つので
	\begin{align}
		E(X_0) \leq \sum_{k=0}^{N \cdot 2^n} \int_{E_n^k} X_{\frac{k}{2^n}}\ dP + E\left(X_N^+\right)
		\leq -\lambda \cdot P(E_n) + E\left(X_N^+\right)
	\end{align}
	が従う.移項すれば
	\begin{align}
		P\left(E_n\right)
		\leq \frac{1}{\lambda} \cdot \left[E\left(X_N^+\right) - E(X_0)\right]
	\end{align}
	が成立し,$n \longrightarrow \infty$として
	\begin{align}
		P\left(\inf{t \in D^N}{X_t} < -\lambda\right)
		\leq \frac{1}{\lambda} \cdot \left[E\left(X_N^+\right) - E(X_0)\right]
	\end{align}
	が得られる.以上をまとめると,
	
	\begin{screen}
		\begin{thm}[Doobの下限不等式]\label{thm:Doob_inf_bounded_inequality}
			設定と記号は定理\ref{thm:Doob_sup_bounded_inequality}の物を継承する.このとき
			\begin{align}
				P\left(\inf{t \in D^N}{X_t} < -\lambda\right)
				\leq \frac{1}{\lambda} \cdot \left[E\left(X_N^+\right) - E(X_0)\right].
			\end{align}
		\end{thm}
	\end{screen}
	
	この二つの不等式から次を得る.
	
	\begin{screen}
		\begin{thm}[劣マルチンゲールのパスは有界区間上で有界]\label{thm:path_of_submartingale_is_bounded_on_bounded_interval}
			$(\Omega,\mathscr{F},P)$を確率空間とし,$\mathbf{T} \defeq [0,\infty[$とし,
			$\{\mathscr{F}_t\}_{t \in \mathbf{T}}$を$\mathscr{F}$に付随するフィルトレーションとし,
			$X$を$(\Omega,\mathscr{F},P)$上の$\{\mathscr{F}_t\}_{t \in \mathbf{T}}$-劣マルチンゲールとする.
			また
			\begin{align}
				D \defeq \bigcup_{n \in \Natural} \Set{\frac{k}{2^n}}{k \in \Natural}
			\end{align}
			とおく.このとき$P$-零集合$A$が取れて,$\omega$を$\Omega \backslash A$の任意の要素とし,$N$を任意の自然数とすれば
			\begin{align}
				- \infty < \inf{t \in D \cap [0,N]}X_t(\omega) \wedge \sup{t \in D \cap [0,N]}X_t(\omega) < \infty
			\end{align}
			が成り立つ.
		\end{thm}
	\end{screen}
	
	\begin{sketch}
		$N$を$0$でない自然数として
		\begin{align}
			D^N \defeq \bigcup_{n \in \Natural} \Set{\frac{k}{2^n}}{k \in \{0,1,\cdots,N \cdot 2^n\}}
		\end{align}
		とおくと,定理\ref{thm:Doob_sup_bounded_inequality}より任意の自然数$n$に対して
		\begin{align}
			P\left(n < \sup{t \in D^N}{X_t}\right)
			\leq \frac{1}{n} \cdot E\left(X_N^+\right).
		\end{align}
		が成り立つから
		\begin{align}
			P\left(\sup{t \in D^N}{X_t} = \infty\right) = 0
		\end{align}
		が従う.定理\ref{thm:Doob_inf_bounded_inequality}からも
		\begin{align}
			P\left(\inf{t \in D^N}{X_t} = -\infty\right) = 0
		\end{align}
		が導かれるので,
		\begin{align}
			A_N \defeq \left\{\sup{t \in D^N}{X_t} = \infty\right\} \cup \left\{\inf{t \in D^N}{X_t} = -\infty\right\}
		\end{align}
		とおいて
		\begin{align}
			A \defeq \bigcup_{N=1}^\infty A_N
		\end{align}
		とおけば,$A$は$P$-零集合である.また$\omega$を$\Omega \backslash A$の要素とし$N$を正の実数とすれば,
		\begin{align}
			D^N = D \cap [0,N]
		\end{align}
		であるから
		\begin{align}
			- \infty < \inf{t \in D \cap [0,N]}X_t(\omega) \wedge \sup{t \in D \cap [0,N]}X_t(\omega) < \infty
		\end{align}
		が成立する.
		\QED
	\end{sketch}
	
	次は劣マルチンゲールの殆ど全てのパスが各点で左極限と右極限を持つことを示す.
	
	いま$N$と$n$を$0$でない自然数とし,$t_0,t_1,\cdots,t_n$を
	\begin{align}
		0 = t_0 < t_1 < \cdots < t_n = N
	\end{align}
	なる実数列とする.そして
	\begin{align}
		F \defeq \{t_0,t_1,\cdots,t_n\}
	\end{align}
	とおく.また$X$を$(\Omega,\mathscr{F},P)$上の$\{\mathscr{F}_t\}_{t \in \mathbf{T}}$-劣マルチンゲールとする.
	$\alpha$と$\beta$を
	\begin{align}
		\alpha < \beta
	\end{align}
	なる実数とし,
	\begin{align}
		\Omega \ni \omega \longmapsto \min{}{\Set{t \in F}{X_t(\omega) < \alpha}}
	\end{align}
	なる関係を$\tau_1$とし,
	\begin{align}
		\Omega \ni \omega \longmapsto \min{}{\Set{t \in F}{\tau_1(\omega) < t \wedge \beta < X_t(\omega)}}
	\end{align}
	なる関係を$\sigma_1$とし,
	\begin{align}
		\Omega \ni \omega \longmapsto \min{}{\Set{t \in F}{\sigma_1(\omega) < t \wedge X_t(\omega) < \alpha}}
	\end{align}
	なる関係を$\tau_2$とし,
	\begin{align}
		\Omega \ni \omega \longmapsto \min{}{\Set{t \in F}{\tau_2(\omega) < t \wedge \beta < X_t(\omega)}}
	\end{align}
	なる関係を$\sigma_2$とし,繰り返して
	\begin{align}
		\Omega \ni \omega \longmapsto \min{}{\Set{t \in F}{\sigma_{i-1}(\omega) < t \wedge X_t(\omega) < \alpha}}
	\end{align}
	なる関係を$\tau_i$とし,
	\begin{align}
		\Omega \ni \omega \longmapsto \min{}{\Set{t \in F}{\tau_i(\omega) < t \wedge \beta < X_t(\omega)}}
	\end{align}
	なる関係を$\sigma_i$とし,$\tau_{n+1}$及び$\sigma_n$まで定める.ただし
	\begin{align}
		\min{}{\emptyset} = \infty
	\end{align}
	と定める.また$\tau_0$及び$\sigma_0$を
	\begin{align}
		\Omega \ni \omega \longmapsto 0
	\end{align}
	なる関係とする.$\Omega$の要素$\omega$に
	\begin{align}
		\sigma_i(\omega) < \infty
	\end{align}
	を満たす最大の自然数$i$を対応させる関係を
	\begin{align}
		U_F^{\alpha,\beta;X}
	\end{align}
	と書く.$U_F^{\alpha,\beta;X}$とは,{\bf $X$のパスが$F$の時点を動いた時に$\alpha$から$\beta$へ上向きに渡った回数を示す写像}である.
	ここで
	\begin{align}
		2 \cdot i - 1 \leq n
	\end{align}
	なる自然数$i$に対し$\tau_i$と$\sigma_i$が$\{\mathscr{F}_t\}_{t \in \mathbf{T}}$-停止時刻であることを示す.
	まず
	\begin{align}
		\left\{\tau_1 = t_0\right\} = \left\{ X_{t_0} < \alpha \right\},
	\end{align}
	及び
	\begin{align}
		k \in \{1,2,\cdots,n\}
	\end{align}
	に対して
	\begin{align}
		\left\{\tau_1 = t_k\right\} = \left(\bigcap_{j=0}^{k-1} \left\{\alpha \leq X_{t_j}\right\}\right) 
		\cap \left\{ X_{t_j}< \alpha \right\}
	\end{align}
	が成り立ち
	\begin{align}
		k \in \{0,1,2,\cdots,n\} \Longrightarrow \left\{\tau_1 = t_k\right\} \in \mathscr{F}_{t_k}
 	\end{align}
 	が成立するので,$\tau_1$は$\{\mathscr{F}_t\}_{t \in \mathbf{T}}$-停止時刻である.$\sigma_1$に関しては
 	\begin{align}
		\left\{\sigma_1 = t_1\right\} = \left\{\tau_1 = t_0\right\} \cap \left\{\beta < X_{t_1}\right\},
	\end{align}
	及び
	\begin{align}
		k \in \{2,\cdots,n\}
	\end{align}
	に対して
	\begin{align}
		\left\{\sigma_1 = t_k\right\} = 
		\bigcup_{r=0}^{k-1}\left( \left\{\tau_1=t_r\right\} \cap \bigcap_{j=r}^{k-1} \left\{X_{t_j} \leq \beta\right\} \cap \left\{\beta < X_{t_k}\right\} \right)
	\end{align}
	が成り立ち
	\begin{align}
		k \in \{1,2,\cdots,n\} \Longrightarrow \left\{\sigma_1 = t_k\right\} \in \mathscr{F}_{t_k}
 	\end{align}
 	が成立するので,$\sigma_1$も$\{\mathscr{F}_t\}_{t \in \mathbf{T}}$-停止時刻である.
 	$\sigma_{i-1}$まで$\{\mathscr{F}_t\}_{t \in \mathbf{T}}$-停止時刻であるとわかったときに,$\tau_i$に関しては
 	\begin{align}
		\left\{\tau_i = t_{2 \cdot i-2}\right\} = \left\{\sigma_{i-1} = t_{2 \cdot i-3}\right\} \cap \left\{X_{t_{2 \cdot i-2}} < \alpha\right\},
	\end{align}
	及び
	\begin{align}
		k \in \{2 \cdot i - 1,2 \cdot i,\cdots,n\}
	\end{align}
	に対して
	\begin{align}
		\left\{\tau_i = t_k\right\} = 
		\bigcup_{r=2 \cdot i-3}^{k-1}\left( \left\{\sigma_{i-1}=t_r\right\} \cap \bigcap_{j=r}^{k-1} \left\{\alpha \leq X_{t_j}\right\} \cap \left\{X_{t_k} < \alpha\right\} \right)
	\end{align}
	が成り立ち
	\begin{align}
		k \in \{2 \cdot i-2,2 \cdot i-1,\cdots,n\} \Longrightarrow \left\{\tau_i = t_k\right\} \in \mathscr{F}_{t_k}
 	\end{align}
 	が成立するので,$\tau_i$は$\{\mathscr{F}_t\}_{t \in \mathbf{T}}$-停止時刻である.また
 	\begin{align}
		\left\{\sigma_i = t_{2 \cdot i-1}\right\} = \left\{\tau_i = t_{2 \cdot i-2}\right\} \cap \left\{\beta < X_{t_{2 \cdot i-1}}\right\},
	\end{align}
	及び
	\begin{align}
		k \in \{2 \cdot i,2 \cdot i+1,\cdots,n\}
	\end{align}
	に対して
	\begin{align}
		\left\{\sigma_i = t_k\right\} = 
		\bigcup_{r=2 \cdot i-2}^{k-1}\left( \left\{\tau_i=t_r\right\} \cap \bigcap_{j=r}^{k-1} \left\{X_{t_j} \leq \beta\right\} \cap \left\{\beta < X_{t_k}\right\} \right)
	\end{align}
	が成り立ち
	\begin{align}
		k \in \{2 \cdot i - 1,2 \cdot i,\cdots,n\} \Longrightarrow \left\{\sigma_i = t_k\right\} \in \mathscr{F}_{t_k}
 	\end{align}
 	が成立するので,$\sigma_i$もまた$\{\mathscr{F}_t\}_{t \in \mathbf{T}}$-停止時刻である.
 	
 	また$j$を自然数とすれば
 	\begin{align}
 		\left\{U_F^{\alpha,\beta;X} = j\right\} = \{\sigma_j < \infty\} \cap \{\sigma_{j+1} = \infty\}
 	\end{align}
 	が成り立つので$U_F^{\alpha,\beta;X}$は$\mathscr{F}/\borel{\R}$-可測である.
 	
 	\begin{itembox}[l]{Doobの上渡回数定理}
 		$U_F^{\alpha,\beta;X}$は次を満たす.
 		\begin{align}
 			E\left( U_F^{\alpha,\beta;X} \right) \leq \frac{E(X_N^+) + |\alpha|}{\beta - \alpha}.
 		\end{align}
 	\end{itembox}
 	
 	\begin{sketch}
 		$\omega$を$\Omega$の要素として
 		\begin{align}
 			j \defeq U_F^{\alpha,\beta;X}(\omega)
 		\end{align}
 		とおく.
 		\begin{description}
 			\item[第一段]
 				$j = 0$のとき,
 				\begin{align}
 					\sum_{i=1}^n \left( X_{\min{}{\{\tau_{i+1}(\omega),N\}}}(\omega) - X_{\min{}{\{\sigma_i(\omega),N\}}}(\omega) \right)
	 				&= 0 \\
	 				&\leq (\alpha - \beta) \cdot U_F^{\alpha,\beta;X}(\omega) + X_N^+(\omega) + |\alpha|
	 			\end{align}
	 			が満たされる.
	 		
	 		\item[第二段]
	 			$j = 1$のとき,
	 			\begin{align}
	 				\sigma_1(\omega) \leq N < \tau_2(\omega)
	 			\end{align}
	 			ならば
	 			\begin{align}
	 				\sum_{i=1}^n \left( X_{\min{}{\{\tau_{i+1}(\omega),N\}}}(\omega) - X_{\min{}{\{\sigma_i(\omega),N\}}}(\omega) \right)
	 				&= X_N(\omega) - X_{\sigma_1(\omega)}(\omega) \\
	 				&= \alpha - X_{\sigma_1(\omega)}(\omega) + X_N(\omega) - \alpha \\
	 				&\leq (\alpha - \beta) \cdot U_F^{\alpha,\beta;X}(\omega) + X_N^+(\omega) + |\alpha|
	 			\end{align}
	 			が成立し,
	 			\begin{align}
	 				\tau_2(\omega) \leq N < \sigma_2(\omega)
	 			\end{align}
	 			ならば
	 			\begin{align}
	 				\sum_{i=1}^n \left( X_{\min{}{\{\tau_{i+1}(\omega),N\}}}(\omega) - X_{\min{}{\{\sigma_i(\omega),N\}}}(\omega) \right)
 					&= X_{\tau_2(\omega)}(\omega) - X_{\sigma_1(\omega)}(\omega) \\
 					&\leq (\alpha - \beta) \cdot U_F^{\alpha,\beta;X}(\omega) \\
 					&\leq (\alpha - \beta) \cdot U_F^{\alpha,\beta;X}(\omega) + X_N^+(\omega) + |\alpha|
 				\end{align}
 				が成立する.
 			
 			\item[第三段]
 				$2 \leq j$のとき,
 				\begin{align}
 					\sigma_j(\omega) \leq N < \tau_{j+1}(\omega)
 				\end{align}
 				ならば
 				\begin{align}
 					&\sum_{i=1}^n \left( X_{\min{}{\{\tau_{i+1}(\omega),N\}}}(\omega) - X_{\min{}{\{\sigma_i(\omega),N\}}}(\omega) \right) \\
 					&= \sum_{i=1}^{j-1} \left( X_{\tau_{i+1}(\omega)}(\omega) - X_{\sigma_i(\omega)}(\omega) \right)
 					+ X_N(\omega) - X_{\sigma_j(\omega)}(\omega) \\
 					&= \sum_{i=1}^{j-1} \left( X_{\tau_{i+1}(\omega)}(\omega) - X_{\sigma_i(\omega)}(\omega) \right)
 					+ \alpha - X_{\sigma_j(\omega)}(\omega) + X_N(\omega) - \alpha \\
 					&\leq (\alpha-\beta) \cdot U_F^{\alpha,\beta;X}(\omega) + X_N^+(\omega) + |\alpha|
 				\end{align}
 				が成立し, 	
 				\begin{align}
 					\tau_{j+1}(\omega) \leq N < \sigma_{j+1}(\omega)
 				\end{align}
 				ならば
 				\begin{align}
 					&\sum_{i=1}^n \left( X_{\min{}{\{\tau_{i+1}(\omega),N\}}}(\omega) - X_{\min{}{\{\sigma_i(\omega),N\}}}(\omega) \right) \\
 					&= \sum_{i=1}^{j} \left( X_{\tau_{i+1}(\omega)}(\omega) - X_{\sigma_i(\omega)}(\omega) \right) \\
 					&\leq (\alpha-\beta) \cdot U_F^{\alpha,\beta;X}(\omega) + X_N^+(\omega) + |\alpha|
 				\end{align}
 				が成立する.
 		\end{description}
 			
 		ゆえに$\Omega$の任意の要素$\omega$において
 		\begin{align}
 			\sum_{i=1}^n \left( X_{\min{}{\{\tau_{i+1}(\omega),N\}}}(\omega) - X_{\min{}{\{\sigma_i(\omega),N\}}}(\omega) \right)
 			\leq (\alpha-\beta) \cdot U_F^{\alpha,\beta;X}(\omega) + X_N^+(\omega) + |\alpha|
 		\end{align}
 		が成立する.左辺の各項については任意抽出定理より
 		\begin{align}
 			0 \leq E\left(X_{\min{}{\{\tau_{i+1},N\}}} - X_{\min{}{\{\sigma_i,N\}}}\right)
 		\end{align}
 		が成り立つため,
 		\begin{align}
 			0 \leq (\alpha-\beta) \cdot E\left(U_F^{\alpha,\beta;X}\right) + E(X_N^+) + |\alpha|
 		\end{align}
	 	が成立する.移項して
	 	\begin{align}
	 		E\left( U_F^{\alpha,\beta;X} \right) \leq \frac{E(X_N^+) + |\alpha|}{\beta - \alpha}
	 	\end{align}
	 	を得る.
	 	\QED
	 \end{sketch}
 	
 	$n$を自然数として
 	\begin{align}
 		D_n^N \defeq \Set{\frac{k}{2^n}}{k \in \{0,1,2,\cdots,N \cdot 2^n\}}
 	\end{align}
 	と定めれば
 	\begin{align}
 		E\left( U_{D_n^N}^{\alpha,\beta;X} \right) \leq \frac{E(X_N^+) + |\alpha|}{\beta - \alpha}
 	\end{align}
 	が成立する.ところで
 	\begin{align}
 		\left\{U_{D_n^N}^{\alpha,\beta;X}\right\}_{n \in \Natural}
 	\end{align}
 	は単調増大列であるから,
 	\begin{align}
 		U_{D^N}^{\alpha,\beta;X} \defeq \lim_{n \to \infty} U_{D_n^N}^{\alpha,\beta;X}
 	\end{align}
 	と定めれば
 	\begin{align}
 		E\left( U_{D^N}^{\alpha,\beta;X} \right) \leq \frac{E(X_N^+) + |\alpha|}{\beta - \alpha}
 	\end{align}
 	が成立する.すなわち
 	\begin{align}
 		P\left(U_{D^N}^{\alpha,\beta;X} = \infty\right) = 0
 	\end{align}
 	である.ここで
 	\begin{align}
 		D^N \defeq \bigcup_{n \in \Natural} D_n^N
 	\end{align}
 	及び
 	\begin{align}
 		B_{\alpha,\beta}^{(N)} \defeq \left\{ U_{D^N}^{\alpha,\beta;X} = \infty \right\}
 	\end{align}
 	とおけば
 	
 	\begin{itembox}[l]{$[0,N]$上では殆ど全てのパスが各点で左極限を持つ}
 		\begin{align}
 			\Set{\omega \in \Omega}{\exists t \in (0,N]\, \left(\, 
 			\sup{\substack{s \in D^N \\ s < t}}{\inf{\substack{u \in D^N \\ s < u < t}}{X_u(\omega)}} 
 			< \inf{\substack{s \in D^N \\ s < t}}{\sup{\substack{u \in D^N \\ s < u < t}}{X_u(\omega)}}\, \right)}
 			\subset \bigcup_{\substack{\alpha,\beta \in \Q \\ \alpha < \beta}} B_{\alpha,\beta}^{(N)}.
 		\end{align}
 	\end{itembox}
 	
 	\begin{sketch}
 		$\omega$を左辺の集合の要素とする.つまり,
 		\begin{align}
 			t \in (0,N]
 		\end{align}
 		なる$t$で,そこにおいて
 		\begin{align}
 			\sup{\substack{s \in D^N \\ s < t}}{\inf{\substack{u \in D^N \\ s < u < t}}{X_u(\omega)}} 
 			< \inf{\substack{s \in D^N \\ s < t}}{\sup{\substack{u \in D^N \\ s < u < t}}{X_u(\omega)}}
 		\end{align}
 		となるものが取れる.
 		\begin{align}
 			\sup{\substack{s \in D^N \\ s < t}}{\inf{\substack{u \in D^N \\ s < u < t}}{X_u(\omega)}} 
 			< \alpha < \beta < \inf{\substack{s \in D^N \\ s < t}}{\sup{\substack{u \in D^N \\ s < u < t}}{X_u(\omega)}}
 		\end{align}
 		を満たす有理数$\alpha$と$\beta$を取る.また$M$を任意に選ばれた$0$でない自然数とする.このとき
 		\begin{align}
 			s_1 < t_1 < s_2 < t_2 < \cdots < s_M < t_M
 		\end{align}
 		なる$D^N$の点列を,$\{1,2,\cdots,M\}$の各要素$i$で
 		\begin{align}
 			X_{s_i}(\omega) < \alpha \wedge \beta < X_{t_i}(\omega)
 		\end{align}
 		を満たすように取れる.ゆえに,
 		\begin{align}
 			\{s_1,t_1,\cdots,s_M,t_M\} \subset D^N_n
 		\end{align}
 		なる自然数$n$を取れば
 		\begin{align}
 			M \leq U_{D_n^N}^{\alpha,\beta;X}(\omega)
 		\end{align}
 		が成立する.ゆえに
 		\begin{align}
 			M \leq U_{D^N}^{\alpha,\beta;X}(\omega)
 		\end{align}
 		が成立する.$M$の任意性ゆえに
 		\begin{align}
 			U_{D^N}^{\alpha,\beta;X}(\omega) = \infty
 		\end{align}
 		が成立する.
 		\QED
 	\end{sketch}
 	
 	同様の証明で次も得られる.
 	
 	\begin{itembox}[l]{$[0,N]$上では殆ど全てのパスが各点で右極限を持つ}
 		\begin{align}
 			\Set{\omega \in \Omega}{\exists t \in [0,N)\, \left(\, 
 			\sup{\substack{s \in D^N \\ t < s}}{\inf{\substack{u \in D^N \\ t < u < s}}{X_u(\omega)}} 
 			< \inf{\substack{s \in D^N \\ t < s}}{\sup{\substack{u \in D^N \\ t < u < s}}{X_u(\omega)}}\, \right)}
 			\subset \bigcup_{\substack{\alpha,\beta \in \Q \\ \alpha < \beta}} B_{\alpha,\beta}^{(N)}.
 		\end{align}
 	\end{itembox}
 	
 	ここで
 	\begin{align}
 		D \defeq \bigcup_{n \in \Natural} \Set{\frac{k}{2^n}}{k \in \Natural}
 	\end{align}
 	及び
 	\begin{align}
 		B \defeq \bigcup_{N=1}^\infty \bigcup_{\substack{\alpha,\beta \in \Q \\ \alpha < \beta}} B_{\alpha,\beta}^{(N)}
 	\end{align}
 	とおけば,$B$は$P$-零集合であって,また上で示したことと
 	\begin{align}
 		D^N = D \cap [0,N]
 	\end{align}
 	が成り立つことにより
 	\begin{align}
 		\Set{\omega \in \Omega}{\exists t \in ]0,\infty[\, \left(\, 
 		\sup{\substack{s \in D \\ s < t}}{\inf{\substack{u \in D \\ s < u < t}}{X_u(\omega)}} 
 		< \inf{\substack{s \in D \\ s < t}}{\sup{\substack{u \in D \\ s < u < t}}{X_u(\omega)}}\, \right)}
 		\subset B
 	\end{align}
 	と
 	\begin{align}
 		\Set{\omega \in \Omega}{\exists t \in [0,\infty[\, \left(\, 
 		\sup{\substack{s \in D \\ t < s}}{\inf{\substack{u \in D \\ t < u < s}}{X_u(\omega)}} 
 		< \inf{\substack{s \in D \\ t < s}}{\sup{\substack{u \in D \\ t < u < s}}{X_u(\omega)}}\, \right)}
 		\subset B
 	\end{align}
 	が成立する.また定理\ref{thm:path_of_submartingale_is_bounded_on_bounded_interval}より
 	$P$-零集合$A$で,$\omega$を$\Omega \backslash A$の要素とし$N$を正の自然数とすれば
	\begin{align}
		- \infty < \inf{t \in D \cap [0,N]}X_t(\omega) \wedge \sup{t \in D \cap [0,N]}X_t(\omega) < \infty
	\end{align}
	が成り立つようにできるものが取れる.つまり,
	\begin{align}
		\omega \in \Omega \backslash (A \cup B)
	\end{align}
	なる$\omega$に対しては,$]0,\infty[$の各要素$t$において
	\begin{align}
		\sup{\substack{s \in D \\ s < t}}{\inf{\substack{u \in D \\ s < u < t}}{X_u(\omega)}} \in \R
	\end{align}
	かつ
	\begin{align}
		\sup{\substack{s \in D \\ s < t}}{\inf{\substack{u \in D \\ s < u < t}}{X_u(\omega)}} 
		= \inf{\substack{s \in D \\ t < s}}{\sup{\substack{u \in D \\ t < u < s}}{X_u(\omega)}}
	\end{align}
	が成立し,$[0,\infty[$の各要素$t$において
	\begin{align}
		\sup{\substack{s \in D \\ t < s}}{\inf{\substack{u \in D \\ t < u < s}}{X_u(\omega)}} \in \R
	\end{align}
	かつ
	\begin{align}
		\sup{\substack{s \in D \\ t < s}}{\inf{\substack{u \in D \\ t < u < s}}{X_u(\omega)}} 
		= \inf{\substack{s \in D \\ t < s}}{\sup{\substack{u \in D \\ t < u < s}}{X_u(\omega)}}
	\end{align}
	が成立する.以上をまとめると,
	
	\begin{screen}
		\begin{thm}[劣マルチンゲールの殆ど全てのパスが各点で実数値の左極限及び右極限を持つ]
		\label{thm:almost_all_paths_of_submartingale_have_left_and_right_limits}
			定理\ref{thm:path_of_submartingale_is_bounded_on_bounded_interval}の設定と記号を継承する.
			このとき$P$-零集合$B$で次を満たすものが取れる.$\omega$を$\Omega \backslash B$の任意の要素とすれば
			\begin{description}
				\item[左極限] $]0,\infty[$の各要素$t$において
					\begin{align}
						\sup{\substack{s \in D \\ s < t}}{\inf{\substack{u \in D \\ s < u < t}}{X_u(\omega)}} \in \R
					\end{align}
					かつ
					\begin{align}
						\sup{\substack{s \in D \\ s < t}}{\inf{\substack{u \in D \\ s < u < t}}{X_u(\omega)}} 
						= \inf{\substack{s \in D \\ t < s}}{\sup{\substack{u \in D \\ t < u < s}}{X_u(\omega)}}
					\end{align}
					が成り立つ.
				
				\item[右極限] $[0,\infty[$の各要素$t$において
					\begin{align}
						\sup{\substack{s \in D \\ t < s}}{\inf{\substack{u \in D \\ t < u < s}}{X_u(\omega)}} \in \R
					\end{align}
					かつ
					\begin{align}
						\sup{\substack{s \in D \\ t < s}}{\inf{\substack{u \in D \\ t < u < s}}{X_u(\omega)}} 
						= \inf{\substack{s \in D \\ t < s}}{\sup{\substack{u \in D \\ t < u < s}}{X_u(\omega)}}
					\end{align}
					が成り立つ.
			\end{description}
		\end{thm}
	\end{screen}
	
	本節の主題は次の定理である.
	
	\begin{screen}
		\begin{thm}[劣マルチンゲールの右極限を取った写像は$RCLL$な劣マルチンゲールで修正となりうる]
		\label{thm:right_limit_process_is_a_RCLL_submartingale}
			設定と記号は定理\ref{thm:almost_all_paths_of_submartingale_have_left_and_right_limits}のものを継承する.
			また$\{\mathscr{F}_t\}_{t \in \mathbf{T}}$は完備であるとする.このとき$\mathbf{T} \times \Omega$上の写像$Y$を
			\begin{align}
				(t,\omega) \longmapsto 
				\begin{cases}
					\displaystyle \sup{\substack{s \in D \\ t < s}}{\inf{\substack{u \in D \\ t < u < s}}{X_u(\omega)}} 
					& \mbox{if } \omega \in \Omega \backslash B \\
					0 & \mbox{if } \omega \in B
				\end{cases}
			\end{align}
			なる関係により定めると,$Y$は$(\Omega,\mathscr{F},P)$上の$RCLL$な
			$\{\mathscr{F}_{t+}\}_{t \in \mathbf{T}}$-劣マルチンゲールである.
			また$\{\mathscr{F}_t\}_{t \in \mathbf{T}}$が右連続であって,かつ
			\begin{align}
				\mathbf{T} \ni t \longmapsto EX_t
			\end{align}
			も右連続であるとき,$Y$は$(\Omega,\mathscr{F},P)$上の$RCLL$な
			$\{\mathscr{F}_t\}_{t \in \mathbf{T}}$-劣マルチンゲールで,$X$の修正である.
		\end{thm}
	\end{screen}
	
	\begin{sketch}\mbox{}
		\begin{description}
			\item[第一段]
				$Y$が$\{\mathscr{F}_{t+}\}_{t \in \mathbf{T}}$-適合であることを示す.$t$を$\mathbf{T}$の要素とする.
				また$\{t_n\}_{n \in \Natural}$を前段の物とする.
				\begin{align}
					t < u
				\end{align}
				なる実数$u$を任意に取れば,
				\begin{align}
					\forall n \in \Natural\, (\, N < n \Longrightarrow t_n < u\, )
				\end{align}
				なる自然数$N$が取れる.このとき
				\begin{align}
					N < n
				\end{align}
				なる自然数$n$に対して$X_{t_n}$は$\mathscr{F}_u/\borel{\R}$-可測であり,仮定より
				\begin{align}
					B \in \mathscr{F}_u
				\end{align}
				となるから
				\begin{align}
					X_{t_n} \defunc_{\Omega \backslash B}
				\end{align}
				もまた$\mathscr{F}_u/\borel{\R}$-可測である.そして
				\begin{align}
					\left\{X_{t_n} \defunc_{\Omega \backslash B}\right\}_{N < n}
				\end{align}
				は$Y_t$に各点収束するので,定理\ref{lem:measurability_metric_space}より$Y_t$も$\mathscr{F}_u/\borel{\R}$-可測である.
				\begin{align}
					t < u
				\end{align}
				なる実数$u$の任意性より$Y_t$の$\mathscr{F}_{t+}/\borel{\R}$-可測性が従う.
			
			\item[第二段] $t$を$\mathbf{T}$の要素とするとき,
				\begin{align}
					A \in \mathscr{F}_t \Longrightarrow \int_A X_t\ dP \leq \int_A Y_t\ dP
					\label{fom:thm_right_limit_process_is_a_RCLL_submartingale_1}
				\end{align}
				が成り立つことを示す.
				\begin{align}
					\forall n \in \Natural\, \left(\, t < t_n\, \right)
				\end{align}
				かつ
				\begin{align}
					\forall n \in \Natural\, \left(\, t_{n+1} \leq t_n\, \right)
				\end{align}
				かつ
				\begin{align}
					\lim_{n \to \infty} t_n = t
				\end{align}
				を満たす$D$の部分集合$\{t_n\}_{n \in \Natural}$を取ると,
				\begin{align}
					\left\{X_{t_n}\right\}_{n \in \Natural}
				\end{align}
				は一様可積分であって,かつ$\Omega \backslash B$上で
				\begin{align}
					Y_t(\omega) = \lim_{n \to \infty} X_{t_n}(\omega)
				\end{align}
				が成立するので,定理\ref{lem:uniformly_integrable_and_convergence_in_mean}より
				\begin{align}
					E|Y_t| < \infty
				\end{align}
				かつ
				\begin{align}
					E\left|Y_t - X_{t_n}\right| \longrightarrow 0 \quad (n \longrightarrow \infty)
					\label{fom:thm_right_limit_process_is_a_RCLL_submartingale_2}
				\end{align}
				が成立する.$A$を$\mathscr{F}_t$の任意の要素とすると,$X$の劣マルチンゲール性より全ての自然数$n$で
				\begin{align}
					\int_A X_t\ dP \leq \int_A X_{t_n}\ dP
				\end{align}
				が成立するので
				\begin{align}
					\int_A X_t\ dP \leq \int_A Y_t\ dP
				\end{align}
				が従う.ゆえに(\refeq{fom:thm_right_limit_process_is_a_RCLL_submartingale_1})が得られた.
				
			\item[第三段]
				$Y$が$\{\mathscr{F}_{t+}\}_{t \in \mathbf{T}}$-劣マルチンゲールであることを示す.
				$s$と$t$を
				\begin{align}
					s < t
				\end{align}
				なる$\mathbf{T}$の要素として,
				\begin{align}
					\forall n \in \Natural\, \left(\, s < s_n < t\, \right)
				\end{align}
				かつ
				\begin{align}
					\forall n \in \Natural\, \left(\, s_{n+1} \leq s_n\, \right)
				\end{align}
				かつ
				\begin{align}
					\lim_{n \to \infty} s_n = s
				\end{align}
				を満たす$D$の部分集合$\{s_n\}_{n \in \Natural}$を取る.$A$を$\mathscr{F}_{s+}$の要素とすると,
				任意の自然数$n$に対して
				\begin{align}
					A \in \mathscr{F}_{s_n}
				\end{align}
				が成り立つので,$X$の劣マルチンゲール性と(\refeq{fom:thm_right_limit_process_is_a_RCLL_submartingale_1})から
				\begin{align}
					\int_A X_{s_n}\ dP \leq \int_A X_t\ dP \leq \int_A Y_t\ dP
				\end{align}
				が成立する.他方で
				\begin{align}
					E\left|Y_s - X_{s_n}\right| \longrightarrow 0 \quad (n \longrightarrow \infty)
				\end{align}
				が成立するので
				\begin{align}
					\int_A Y_s\ dP \leq \int_A Y_t\ dP
				\end{align}
				が従う.ゆえに$Y$は$\{\mathscr{F}_{t+}\}_{t \in \mathbf{T}}$-劣マルチンゲールである.
				
			\item[第四段]
				$Y$が$RCLL$であることを示す.$\omega$を$\Omega \backslash B$の要素とし,$t$を$\mathbf{T}$の要素とする.
				$\epsilon$を任意に与えられた正の実数とすると,
				\begin{align}
					r \in D \wedge t < r < t + \delta \Longrightarrow
					\left|Y_t(\omega) - X_r(\omega)\right| < \epsilon
				\end{align}
				を満たす正の実数$\delta$が取れる.このとき$s$を
				\begin{align}
					t < s < t+\delta
				\end{align}
				なる任意の実数とすると,
				\begin{align}
					r \in D \wedge s < r < t + \delta \wedge \left|Y_s(\omega) - X_r(\omega)\right| < \epsilon
				\end{align}
				なる実数$r$を取れば
				\begin{align}
					\left|Y_t(\omega) - Y_s(\omega)\right|
					\leq \left|Y_t(\omega) - X_r(\omega)\right| + \left|X_r(\omega) - Y_s(\omega)\right|
					< 2 \cdot \epsilon
				\end{align}
				が成立する.ゆえに$Y$は右連続である.次に左極限の存在を示すが,ここで
				\begin{align}
					Z_t(\omega) \defeq \sup{\substack{s \in D \\ s < t}}{\inf{\substack{u \in D \\ s < u < t}}{X_u(\omega)}} 
				\end{align}
				とおき,また
				\begin{align}
					0 < t
				\end{align}
				であるとする.$\epsilon$を任意に与えられた正の実数とすると,
				\begin{align}
					r \in D \wedge t - \delta < r < t \Longrightarrow
					\left|Z_t(\omega) - X_r(\omega)\right| < \epsilon
				\end{align}
				を満たす正の実数$\delta$が取れる.このとき$s$を
				\begin{align}
					t - \delta < s < t
				\end{align}
				なる任意の実数とすると,
				\begin{align}
					r \in D \wedge s < r < t \wedge \left|Y_s(\omega) - X_r(\omega)\right| < \epsilon
				\end{align}
				なる実数$r$を取れば
				\begin{align}
					\left|Z_t(\omega) - Y_s(\omega)\right|
					\leq \left|Z_t(\omega) - X_r(\omega)\right| + \left|X_r(\omega) - Y_s(\omega)\right|
					< 2 \cdot \epsilon
				\end{align}
				が成立する.ゆえに$Y$は$t$で左極限を持ち,それは$Z_t$に一致する.
				
			\item[第五段]
				$\{\mathscr{F}_t\}_{t \in \mathbf{T}}$が右連続であって,かつ
				\begin{align}
					\mathbf{T} \ni t \longmapsto EX_t
				\end{align}
				も右連続であるとき,$Y$が$X$の修正であることを示す.$t$を$\mathbf{T}$の要素として,
				\begin{align}
					\forall n \in \Natural\, \left(\, t < t_n\, \right)
				\end{align}
				かつ
				\begin{align}
					\forall n \in \Natural\, \left(\, t_{n+1} \leq t_n\, \right)
				\end{align}
				かつ
				\begin{align}
					\lim_{n \to \infty} t_n = t
				\end{align}
				を満たす$D$の部分集合$\{t_n\}_{n \in \Natural}$を取る.このとき
				\begin{align}
					EX_t = \lim_{n \to \infty} EX_{t_n}
				\end{align}
				が成立し,一方で(\refeq{fom:thm_right_limit_process_is_a_RCLL_submartingale_2})から
				\begin{align}
					EY_t = \lim_{n \to \infty} EX_{t_n}
				\end{align}
				も成り立つから
				\begin{align}
					EX_t = EY_t
				\end{align}
				が成り立つ.またいまは$Y_t$が$\mathscr{F}_t/\borel{\R}$-可測であるから,
				(\refeq{fom:thm_right_limit_process_is_a_RCLL_submartingale_1})より
				\begin{align}
					X_t \leq Y_t
				\end{align}
				が$P$-a.s.に成立する.ゆえに
				\begin{align}
					X_t = Y_t
				\end{align}
				が$P$-a.s.に成立する.ゆえに$Y$は$X$の修正である.
				\QED
		\end{description}
	\end{sketch}
	
	最後に,{\bf 右連続な劣マルチンゲールは殆ど全てのパスが$RCLL$である}ことを証明して本節を終える.
	
	\begin{screen}
		\begin{thm}[パスの正則性]
			$(\Omega,\mathscr{F},P)$を確率空間とし,$\mathbf{T} \defeq [0,\infty[$とし,
			$\{\mathscr{F}_t\}_{t \in \mathbf{T}}$を$\mathscr{F}$に付随するフィルトレーションとし,
			$X$を$(\Omega,\mathscr{F},P)$上の右連続な$\{\mathscr{F}_t\}_{t \in \mathbf{T}}$-劣マルチンゲールとする.
			このとき,$\Omega \backslash B$のすべての要素のパスが$RCLL$であるように$P$-零集合$B$が取れる.
		\end{thm}
	\end{screen}
	
	\begin{sketch}
		定理\ref{thm:almost_all_paths_of_submartingale_have_left_and_right_limits}より,
		$P$-零集合$B$が取れて,$\omega$を$\Omega \backslash B$の要素とすれば
		$\mathbf{T}$の各要素$t$において
		\begin{align}
			\sup{\substack{s \in D \\ s < t}}{\inf{\substack{u \in D \\ s < u < t}}{X_u(\omega)}} \in \R
		\end{align}
		かつ
		\begin{align}
			\sup{\substack{s \in D \\ s < t}}{\inf{\substack{u \in D \\ s < u < t}}{X_u(\omega)}} 
			= \inf{\substack{s \in D \\ s < t}}{\sup{\substack{u \in D \\ s < u < t}}{X_u(\omega)}}
		\end{align}
		が成り立つ.いま$\omega$を$\Omega \backslash B$から任意に選ばれた要素とし,
		$t$を$\mathbf{T}$から任意に選ばれた要素とし,
		\begin{align}
			\sup{\substack{s \in D \\ s < t}}{\inf{\substack{u \in D \\ s < u < t}}{X_u(\omega)}}
			= \sup{s < t}{\inf{s < u < t}{X_u(\omega)}}
		\end{align}
		かつ
		\begin{align}
			\inf{\substack{s \in D \\ s < t}}{\sup{\substack{u \in D \\ s < u < t}}{X_u(\omega)}}
			= \inf{s < t}{\sup{s < u < t}{X_u(\omega)}}
		\end{align}
		が成り立つことを示す.
		
	\end{sketch}
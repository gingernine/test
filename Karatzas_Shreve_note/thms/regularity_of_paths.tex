\section{$RCLL$修正}
	$(\Omega,\mathscr{F},P)$を確率空間とし,$\mathbf{T} \defeq [0,\infty[$とし,
	$\{\mathscr{F}_t\}_{t \in \mathbf{T}}$を$\mathscr{F}$に付随するフィルトレーションとし,
	$X$を$(\Omega,\mathscr{F},P)$上の$\{\mathscr{F}_t\}_{t \in \mathbf{T}}$-劣マルチンゲールとする.
	
	いま$\lambda$と正の実数とし,$N$を$0$でない自然数とする.また$[0,N]$の稠密な部分集合を
	\begin{align}
		D^N \defeq \bigcup_{n \in \Natural} \Set{\frac{k}{2^n}}{k \in \{0,1,\cdots,N \cdot 2^n\}}
	\end{align}
	により定める.このとき
	\begin{align}
		\left\{\lambda < \sup{t \in D^N}{X_t}\right\}
		= \bigcup_{n \in \Natural} \left\{\lambda < \max_{k \in \left\{0,1,2,\cdots,N \cdot 2^n\right\}} X_{\frac{k}{2^n}T}\right\}
	\end{align}
	が成り立つ.ちなみにこの式から
	\begin{align}
		\Omega \ni \omega \longmapsto \sup{t \in D^N}X_t(\omega)
	\end{align}
	が$\mathscr{F}_N/\borel{[-\infty,\infty]}$-可測であることが従う.$n$を自然数として
	\begin{align}
		E_n \defeq  \left\{\lambda < \max_{k \in \left\{0,1,2,\cdots,N \cdot 2^n\right\}} X_{\frac{k}{2^n}T}\right\}
	\end{align}
	とおき,$E_n$をさらに分解して
	\begin{align}
		E_n^0 \defeq \left\{\lambda < X_0\right\}
	\end{align}
	及び,$\left\{1,2,\cdots,N \cdot 2^n\right\}$の各要素$k$に対して
	\begin{align}
		E_n^k \defeq \left\{ \max_{m \in \left\{0,1,2,\cdots,k-1\right\}} X_{\frac{m}{2^n}} \leq \lambda \right\} \cap \left\{\lambda < X_{\frac{k}{2^n}}\right\}
	\end{align}
	とおく.$k$を$\left\{0,1,\cdots,N \cdot 2^n\right\}$の要素とすれば
	\begin{align}
		E_n^k \in \mathscr{F}_{\frac{k}{2^n}}
	\end{align}
	が成立するので,劣マルチンゲール性から
	\begin{align}
		\int_{E_n^k} X_{\frac{k}{2^n}}\ dP \leq \int_{E_n^k} X_N\ dP
	\end{align}
	が成り立つ.他方で
	\begin{align}
		\forall \omega \in E_n^k\, \left(\, \lambda < X_{\frac{k}{2^n}}(\omega)\, \right)
	\end{align}
	が成り立つから
	\begin{align}
		P(E_n^k) \leq \frac{1}{\lambda} \cdot \int_{E_n^k} X_{\frac{k}{2^n}}\ dP
	\end{align}
	が成立する.これにより
	\begin{align}
		P(E_n) = \sum_{k=0}^{2^n} P(E_n^k)
		\leq \frac{1}{\lambda} \cdot \sum_{k=0}^{N \cdot 2^n} \int_{E_n^k} X_{\frac{m}{2^n}}\ dP
		\leq \frac{1}{\lambda} \cdot \sum_{k=0}^{N \cdot 2^n} \int_{E_n^k} X_N\ dP
		= \frac{1}{\lambda} \cdot \int_{E_n} X_N\ dP
		\label{fom:Doob_upper_bound_inequality_1}
	\end{align}
	が成立する.$\{E_n\}_{n \in \Natural}$は単調に増大して
	\begin{align}
		\left\{\lambda < \sup{t \in D^N}{X_t}\right\}
	\end{align}
	に一致するので,(\refeq{fom:Doob_upper_bound_inequality_1})で$n \longrightarrow \infty$として
	\begin{align}
		P\left(\lambda < \sup{t \in D^N}{X_t}\right)
		\leq \frac{1}{\lambda} \cdot \int_{\left\{\lambda < \sup{t \in D^N}{X_t}\right\}} X_N\ dP
	\end{align}
	が成立する.さらに右辺は
	\begin{align}
		\frac{1}{\lambda} \cdot E\left(X_N^+\right)
	\end{align}
	で抑えられるので,以上で次を得た.
	
	\begin{screen}
		\begin{thm}[Doobの上限不等式]\label{thm:Doob_sup_bounded_inequality}
			$(\Omega,\mathscr{F},P)$を確率空間とし,$\mathbf{T} \defeq [0,\infty[$とし,
			$\{\mathscr{F}_t\}_{t \in \mathbf{T}}$を$\mathscr{F}$に付随するフィルトレーションとし,
			$X$を$(\Omega,\mathscr{F},P)$上の$\{\mathscr{F}_t\}_{t \in \mathbf{T}}$-劣マルチンゲールとする.
			このとき,$\lambda$と正の実数とし,$N$を$0$でない自然数として
			\begin{align}
				D^N \defeq \bigcup_{n \in \Natural} \Set{\frac{k}{2^n}}{k \in \{0,1,\cdots,N \cdot 2^n\}}
			\end{align}
			とおけば,
			\begin{align}
				P\left(\lambda < \sup{t \in D^N}{X_t}\right)
				\leq \frac{1}{\lambda} \cdot E\left(X_N^+\right).
			\end{align}
		\end{thm}
	\end{screen}
	
	上の設定をそのままにして,次は
	\begin{align}
		P\left(\inf{t \in D^N}{X_t} < -\lambda\right)
		\leq \frac{1}{\lambda} \cdot \left[E\left(X_N^+\right) - E(X_0)\right]
	\end{align}
	が成り立つことを示す.今度は
	\begin{align}
		E_n \defeq  \left\{\min_{k \in \left\{0,1,2,\cdots,N \cdot 2^n\right\}} X_{\frac{k}{2^n}} < -\lambda\right\}
	\end{align}
	とおいて,また
	\begin{align}
		E_n^0 \defeq \left\{X_0 < -\lambda\right\}
	\end{align}
	及び,$\left\{1,2,\cdots,N \cdot 2^n\right\}$の各要素$k$に対して
	\begin{align}
		E_n^k \defeq \left\{ -\lambda \leq \min_{m \in \left\{0,1,2,\cdots,k-1\right\}} X_{\frac{m}{2^n}} \right\} \cap \left\{ X_{\frac{k}{2^n}} < -\lambda \right\}
	\end{align}
	とおく.ここで
	\begin{align}
		\Omega \ni \omega \longmapsto
		\begin{cases}
			0 & \mbox{if } \omega \in E_n^0 \\
			\frac{1}{2^n} & \mbox{if } \omega \in E_n^1 \\
			\frac{2}{2^n} & \mbox{if } \omega \in E_n^2 \\
			\vdots & \\
			\frac{N \cdot 2^n - 1}{2^n} & \mbox{if } \omega \in E_n^{N \cdot 2^n - 1} \\
			\frac{N \cdot 2^n}{2^n} & \mbox{if } \omega \in E_n^{N \cdot 2^n} \\
			N & \mbox{if } \omega \in \Omega \backslash E_n
		\end{cases}
	\end{align}
	なる関係を$\tau$とすれば,$\tau$は$\{\mathscr{F}_t\}_{t \in \mathbf{T}}$-停止時刻である.
	任意抽出定理より
	\begin{align}
		E(X_0) \leq E(X_\tau)
	\end{align}
	が成立し,$\tau$の定め方より
	\begin{align}
		E(X_\tau) = \sum_{k=0}^{N \cdot 2^n} \int_{E_n^k} X_{\frac{k}{2^n}}\ dP + \int_{\Omega \backslash E_n} X_N\ dP
		\leq \sum_{k=0}^{N \cdot 2^n} \int_{E_n^k} X_{\frac{k}{2^n}}\ dP + E\left(X_N^+\right)
	\end{align}
	が成立する.他方で
	\begin{align}
		\forall \omega \in E_n^k\, \left(\, X_{\frac{k}{2^n}}(\omega) < -\lambda\, \right)
	\end{align}
	から
	\begin{align}
		\int_{E_n^k} X_{\frac{k}{2^n}}\ dP \leq -\lambda \cdot P(E_n^k)
	\end{align}
	が成り立つので
	\begin{align}
		E(X_0) \leq \sum_{k=0}^{N \cdot 2^n} \int_{E_n^k} X_{\frac{k}{2^n}}\ dP + E\left(X_N^+\right)
		\leq -\lambda \cdot P(E_n) + E\left(X_N^+\right)
	\end{align}
	が従う.移項すれば
	\begin{align}
		P\left(E_n\right)
		\leq \frac{1}{\lambda} \cdot \left[E\left(X_N^+\right) - E(X_0)\right]
	\end{align}
	が成立し,$n \longrightarrow \infty$として
	\begin{align}
		P\left(\inf{t \in D^N}{X_t} < -\lambda\right)
		\leq \frac{1}{\lambda} \cdot \left[E\left(X_N^+\right) - E(X_0)\right]
	\end{align}
	が得られる.以上をまとめると,
	
	\begin{screen}
		\begin{thm}[Doobの下限不等式]\label{thm:Doob_inf_bounded_inequality}
			$(\Omega,\mathscr{F},P)$を確率空間とし,$\mathbf{T} \defeq [0,\infty[$とし,
			$\{\mathscr{F}_t\}_{t \in \mathbf{T}}$を$\mathscr{F}$に付随するフィルトレーションとし,
			$X$を$(\Omega,\mathscr{F},P)$上の$\{\mathscr{F}_t\}_{t \in \mathbf{T}}$-劣マルチンゲールとする.
			このとき,$\lambda$と正の実数とし,$N$を$0$でない自然数として
			\begin{align}
				D^N \defeq \bigcup_{n \in \Natural} \Set{\frac{k}{2^n}}{k \in \{0,1,\cdots,N \cdot 2^n\}}
			\end{align}
			とおけば,
			\begin{align}
				P\left(\inf{t \in D^N}{X_t} < -\lambda\right)
				\leq \frac{1}{\lambda} \cdot \left[E\left(X_N^+\right) - E(X_0)\right].
			\end{align}
		\end{thm}
	\end{screen}
	
	この二つの不等式から次を得る.
	
	\begin{screen}
		\begin{thm}[劣マルチンゲールのパスは有界区間上で有界]
			$(\Omega,\mathscr{F},P)$を確率空間とし,$\mathbf{T} \defeq [0,\infty[$とし,
			$\{\mathscr{F}_t\}_{t \in \mathbf{T}}$を$\mathscr{F}$に付随するフィルトレーションとし,
			$X$を$(\Omega,\mathscr{F},P)$上の$\{\mathscr{F}_t\}_{t \in \mathbf{T}}$-劣マルチンゲールとする.
			また
			\begin{align}
				D \defeq \bigcup_{n \in \Natural} \Set{\frac{k}{2^n}}{k \in \Natural}
			\end{align}
			とおく.このとき$P$-零集合$A$で,$\omega$を$\Omega \backslash A$の要素とし$N$を正の自然数とすれば
			\begin{align}
				- \infty < \inf{t \in D \cap [0,N]}X_t(\omega) \wedge \sup{t \in D \cap [0,N]}X_t(\omega) < \infty
			\end{align}
			が成り立つようにできるものが取れる.
		\end{thm}
	\end{screen}
	
	\begin{sketch}
		$N$を$0$でない自然数として
		\begin{align}
			D^N \defeq \bigcup_{n \in \Natural} \Set{\frac{k}{2^n}}{k \in \{0,1,\cdots,N \cdot 2^n\}}
		\end{align}
		とおくと,定理\ref{thm:Doob_sup_bounded_inequality}より任意の自然数$n$に対して
		\begin{align}
			P\left(n < \sup{t \in D^N}{X_t}\right)
			\leq \frac{1}{n} \cdot E\left(X_N^+\right).
		\end{align}
		が成り立つから
		\begin{align}
			P\left(\sup{t \in D^N}{X_t} = \infty\right) = 0
		\end{align}
		が従う.定理\ref{thm:Doob_inf_bounded_inequality}からも
		\begin{align}
			P\left(\inf{t \in D^N}{X_t} = -\infty\right) = 0
		\end{align}
		が導かれるので,
		\begin{align}
			A_N \defeq \left\{\sup{t \in D^N}{X_t} = \infty\right\} \cup \left\{\inf{t \in D^N}{X_t} = -\infty\right\}
		\end{align}
		とおいて
		\begin{align}
			A \defeq \bigcup_{N=1}^\infty A_N
		\end{align}
		とおけば,$A$は$P$-零集合である.また$\omega$を$\Omega \backslash A$の要素とし$N$を正の実数とすれば,
		\begin{align}
			D^N = D \cap [0,N]
		\end{align}
		であるから
		\begin{align}
			- \infty < \inf{t \in D \cap [0,N]}X_t(\omega) \wedge \sup{t \in D \cap [0,N]}X_t(\omega) < \infty
		\end{align}
		が成立する.
		\QED
	\end{sketch}
	
	いま$N$と$n$を$0$でない自然数とし,$t_0,t_1,\cdots,t_n$を
	\begin{align}
		0 = t_0 < t_1 < \cdots < t_n = N
	\end{align}
	なる実数列とする.そして
	\begin{align}
		F \defeq \{t_0,t_1,\cdots,t_n\}
	\end{align}
	とおく.また$X$を$(\Omega,\mathscr{F},P)$上の$\{\mathscr{F}_t\}_{t \in \mathbf{T}}$-劣マルチンゲールとする.
	$\alpha$と$\beta$を
	\begin{align}
		\alpha < \beta
	\end{align}
	なる実数とし,
	\begin{align}
		\Omega \ni \omega \longmapsto \min{}{\Set{t \in F}{X_t(\omega) < \alpha}}
	\end{align}
	なる関係を$\tau_1$とし,
	\begin{align}
		\Omega \ni \omega \longmapsto \min{}{\Set{t \in F}{\tau_1(\omega) < t \wedge \beta < X_t(\omega)}}
	\end{align}
	なる関係を$\sigma_1$とし,
	\begin{align}
		\Omega \ni \omega \longmapsto \min{}{\Set{t \in F}{\sigma_1(\omega) < t \wedge X_t(\omega) < \alpha}}
	\end{align}
	なる関係を$\tau_2$とし,
	\begin{align}
		\Omega \ni \omega \longmapsto \min{}{\Set{t \in F}{\tau_2(\omega) < t \wedge \beta < X_t(\omega)}}
	\end{align}
	なる関係を$\sigma_2$とし,繰り返して
	\begin{align}
		\Omega \ni \omega \longmapsto \min{}{\Set{t \in F}{\sigma_{i-1}(\omega) < t \wedge X_t(\omega) < \alpha}}
	\end{align}
	なる関係を$\tau_i$とし,
	\begin{align}
		\Omega \ni \omega \longmapsto \min{}{\Set{t \in F}{\tau_i(\omega) < t \wedge \beta < X_t(\omega)}}
	\end{align}
	なる関係を$\sigma_i$と定める.ただし
	\begin{align}
		\min{}{\emptyset} = \infty
	\end{align}
	と定める.また$\tau_0$及び$\sigma_0$を
	\begin{align}
		\Omega \ni \omega \longmapsto 0
	\end{align}
	なる関係とする.$\Omega$の要素$\omega$に
	\begin{align}
		\sigma_i(\omega) < \infty
	\end{align}
	を満たす最大の自然数$i$を対応させる関係を
	\begin{align}
		U_F^{\alpha,\beta;X}
	\end{align}
	と書く.$U_F^{\alpha,\beta;X}$とは,{\bf $X$のパスが$F$の時点を動いた時に$\alpha$から$\beta$へ上向きに渡った回数を示す写像}である.
	ここで
	\begin{align}
		2 \cdot i - 1 \leq n
	\end{align}
	なる自然数$i$に対し$\tau_i$と$\sigma_i$が$\{\mathscr{F}_t\}_{t \in \mathbf{T}}$-停止時刻であることを示す.
	まず
	\begin{align}
		\left\{\tau_1 = t_0\right\} = \left\{ X_{t_0} < \alpha \right\},
	\end{align}
	及び
	\begin{align}
		k \in \{1,2,\cdots,n\}
	\end{align}
	のとき
	\begin{align}
		\left\{\tau_1 = t_k\right\} = \left(\bigcap_{j=0}^{k-1} \left\{\alpha \leq X_{t_j}\right\}\right) 
		\cap \left\{ X_{t_j}< \alpha \right\}
	\end{align}
	が成り立ち
	\begin{align}
		k \in \{0,1,2,\cdots,n\} \Longrightarrow \left\{\tau_1 = t_k\right\} \in \mathscr{F}_{t_k}
 	\end{align}
 	が成立するので,$\tau_1$は$\{\mathscr{F}_t\}_{t \in \mathbf{T}}$-停止時刻である.$\sigma_1$に関しては
 	\begin{align}
		\left\{\sigma_1 = t_1\right\} = \left\{\tau_1 = t_0\right\} \cap \left\{\beta < X_{t_1}\right\},
	\end{align}
	及び
	\begin{align}
		k \in \{2,\cdots,n\}
	\end{align}
	のとき
	\begin{align}
		\left\{\sigma_1 = t_k\right\} = 
		\bigcup_{r=0}^{k-1}\left( \left\{\tau_1=t_r\right\} \cap \bigcap_{j=r}^{k-1} \left\{X_{t_j} \leq \beta\right\} \cap \left\{\beta < X_{t_k}\right\} \right)
	\end{align}
	が成り立ち
	\begin{align}
		k \in \{1,2,\cdots,n\} \Longrightarrow \left\{\sigma_1 = t_k\right\} \in \mathscr{F}_{t_k}
 	\end{align}
 	が成立するので,$\sigma_1$も$\{\mathscr{F}_t\}_{t \in \mathbf{T}}$-停止時刻である.
 	$\sigma_{i-1}$まで$\{\mathscr{F}_t\}_{t \in \mathbf{T}}$-停止時刻であるとわかったときに,$\tau_i$に関しては
 	\begin{align}
		\left\{\tau_i = t_{2 \cdot i-2}\right\} = \left\{\sigma_{i-1} = t_{2 \cdot i-3}\right\} \cap \left\{X_{t_{2 \cdot i-2}} < \alpha\right\},
	\end{align}
	及び
	\begin{align}
		k \in \{2 \cdot i - 1,2 \cdot i,\cdots,n\}
	\end{align}
	のとき
	\begin{align}
		\left\{\tau_i = t_k\right\} = 
		\bigcup_{r=2 \cdot i-3}^{k-1}\left( \left\{\sigma_{i-1}=t_r\right\} \cap \bigcap_{j=r}^{k-1} \left\{\alpha \leq X_{t_j}\right\} \cap \left\{X_{t_k} < \alpha\right\} \right)
	\end{align}
	が成り立ち
	\begin{align}
		k \in \{2 \cdot i-2,2 \cdot i-1,\cdots,n\} \Longrightarrow \left\{\tau_i = t_k\right\} \in \mathscr{F}_{t_k}
 	\end{align}
 	が成立するので,$\tau_i$は$\{\mathscr{F}_t\}_{t \in \mathbf{T}}$-停止時刻である.また
 	\begin{align}
		\left\{\sigma_i = t_{2 \cdot i-1}\right\} = \left\{\tau_i = t_{2 \cdot i-2}\right\} \cap \left\{\beta < X_{t_{2 \cdot i-1}}\right\},
	\end{align}
	及び
	\begin{align}
		k \in \{2 \cdot i,2 \cdot i+1,\cdots,n\}
	\end{align}
	のとき
	\begin{align}
		\left\{\sigma_i = t_k\right\} = 
		\bigcup_{r=2 \cdot i-2}^{k-1}\left( \left\{\tau_i=t_r\right\} \cap \bigcap_{j=r}^{k-1} \left\{X_{t_j} \leq \beta\right\} \cap \left\{\beta < X_{t_k}\right\} \right)
	\end{align}
	が成り立ち
	\begin{align}
		k \in \{2 \cdot i - 1,2 \cdot i,\cdots,n\} \Longrightarrow \left\{\sigma_i = t_k\right\} \in \mathscr{F}_{t_k}
 	\end{align}
 	が成立するので,$\sigma_i$もまた$\{\mathscr{F}_t\}_{t \in \mathbf{T}}$-停止時刻である.
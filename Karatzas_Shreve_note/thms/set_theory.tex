	\begin{screen}
		\begin{axm}[外延性の公理]
			$a,b$を類とするとき,次が成り立つ:
			\begin{align}
				\forall x\ (\ x \in a  \Longleftrightarrow x \in b\ )
				\Longrightarrow a=b.
			\end{align}
		\end{axm}
	\end{screen}
	
	\begin{screen}
		\begin{thm}[任意の類は自分自身と等しい]\label{thm:any_class_equals_to_itself}
			$a$を類とするとき次が成り立つ:
			\begin{align}
				a = a.
			\end{align}
		\end{thm}
	\end{screen}
	
	\begin{prf}
		$\mathcal{L}$の任意の対象$\tau$に対して,推論法則\ref{metathm:reflective_law_of_implication}より
		\begin{align}
			\tau \in a \Longleftrightarrow \tau \in a
		\end{align}
		が成り立つから,$\tau$の任意性と推論法則\ref{metathm:fundamental_law_of_universal_quantifier}より
		\begin{align}
			\forall x\ (\ x \in a  \Longleftrightarrow x \in b\ )
		\end{align}
		が成り立つ.従って外延性の公理より$a = a$が得られる.
		\QED
	\end{prf}
	
	\begin{screen}
		\begin{axm}[類の公理] 以下を公理とする:
			\begin{description}
				\item[(i)] 要素となりうる類は集合である.つまり,$a,b$を類とするとき次が成り立つ.
					\begin{align}
						a \in b \Longrightarrow \set{a}.
					\end{align}
					
					
				\item[(ii)] $A$を$\mathcal{L}'$の式とし,$x$を$A$に現れる文字とし,$t$を$A(x)$に現れない文字とし,
					$x$のみが$A$で量化されていないとする.このとき次が成り立つ.
					\begin{align}
						\forall t\ (\ t \in \Set{x}{A(x)} \Longleftrightarrow A(t)\ ).
					\end{align}
			\end{description}
		\end{axm}
	\end{screen}
	
	\begin{screen}
		\begin{thm}[$\mathcal{L}$の対象も$\Set{x}{A(x)}$の形で表せる]
			次が成り立つ:
			\begin{align}
				\forall y\ \left(\ y = \Set{x}{x \in y}\ \right).
			\end{align}
		\end{thm}
	\end{screen}
	
	\begin{prf}
		$\tau$を$\mathcal{L}$の任意の対象とすると類の公理より
		\begin{align}
			\forall s\ (\ s \in \tau \Longleftrightarrow s \in \Set{x}{x \in \tau}\ )
		\end{align}
		が成り立つから,外延性の公理より
		\begin{align}
			\tau = \Set{x}{x \in \tau}
		\end{align}
		が従う.$\tau$の任意性と推論法則\ref{metathm:fundamental_law_of_universal_quantifier}より
		\begin{align}
			\forall y\ \left(\ y = \Set{x}{x \in y}\ \right)
		\end{align}
		を得る.
		\QED
	\end{prf}
	
	\monologue{
		院生「集合は$\mathcal{L}$の対象であるとは限りません.例えば$\tau$を$\mathcal{L}$の対象とすれば
			\begin{align}
				\tau = \Set{x}{x \in \tau}
			\end{align}
			が成り立つので,推論規則\ref{metaaxm:rules_of_quantifiers}と集合の定義より
			$\Set{x}{x \in \tau}$は集合であるとわかりますが,
			これは言語を$\mathcal{L}'$に拡張した際に導入された表記なので$\mathcal{L}$の対象ではありませんね.
			またこの例は`等しい'とはどういうことかについて或る示唆を与えています.
			数学において`等しい'とは`同一物である'わけではないということです.
			$\tau$と$\Set{x}{x \in \tau}$は等しいですが明確な違いがありますね.
			数学において`等しい'とは`同一視する'という意味です.それを形式的に述べたものが次の相等性の公理です.」
	}
	
	\begin{screen}
		\begin{axm}[相等性の公理]
			$A$を$\mathcal{L}'$の式とし,$x$を$A$に現れる文字とし,
			$x$のみが$A$で量化されていないとする.このとき$a,b$を類とすれば次が成り立つ:
			\begin{align}
				a = b \Longrightarrow (\, A(a) \Longleftrightarrow A(b)\, ).
			\end{align}
		\end{axm}
	\end{screen}
	
	\begin{screen}
		\begin{thm}[外延性の公理は同値関係で成立する]\label{thm:axiom_of_extensionality_equivalent}
			$a,b$を類とするとき,次が成り立つ:
			\begin{align}
				\forall x\ (\ x \in a  \Longleftrightarrow x \in b\ )
				\Longleftrightarrow a=b.
			\end{align}
		\end{thm}
	\end{screen}
	
	\begin{prf}
		$a = b$が成り立っていると仮定すれば,相等性の公理より$\mathcal{L}$の任意の対象$\tau$に対して
		\begin{align}
			\tau \in a \Longleftrightarrow \tau \in b
		\end{align}
		が満たされるから,推論法則\ref{metathm:fundamental_law_of_universal_quantifier}より
		\begin{align}
			\forall x\ (\ x \in a  \Longleftrightarrow x \in b\ )
		\end{align}
		が成立する.よって演繹法則より
		\begin{align}
			a = b \Longrightarrow \forall t\ (\ t \in a  \Longleftrightarrow t \in b\ )
		\end{align}
		が成り立つ.外延性の公理と併せれば定理の主張が得られる.
		\QED
	\end{prf}
	
	\monologue{
		院生「等しい類同士は同じ$\mathcal{L}$の対象を要素に持つと示されましたが,
			このとき要素に持つ集合まで一致します.これは相等性の公理から明らかでしょうが,
			詳しくは部分類の箇所で説明いたしましょう.」
	}
	
	\begin{screen}
		\begin{thm}[$\Univ$は集合の全体である]
		\label{thm:V_is_the_whole_of_sets}
		\label{thm:every_set_is_equivalent_to_some_individual_in_L}
			$a$を類とするとき次が成り立つ:
			\begin{align}
				\set{a} \Longleftrightarrow a \in \Univ.
			\end{align}
		\end{thm}
	\end{screen}
	
	\begin{prf}
		$a$を類とするとき,まず類の公理より
		\begin{align}
			a \in \Univ \Longrightarrow \set{a}
		\end{align}
		が得られる.逆に$\exists x\ (\ a = x\ )$が成り立っていると仮定する.このとき
		\begin{align}
			\tau \coloneqq \varepsilon x (\ a = x\ )
		\end{align}
		とおけば,定理\ref{thm:any_class_equals_to_itself}より$\tau = \tau$となるので,類の公理より
		\begin{align}
			\tau \in \Univ
		\end{align}
		が成り立つ.そして相等性の公理より
		\begin{align}
			a \in \Univ
		\end{align}
		が従うから$\set{a} \Longrightarrow a \in \Univ$も得られる.
		\QED
	\end{prf}
	
	\begin{screen}
		\begin{dfn}[空集合]
			$\emptyset$を{\bf 空集合}\index{くうしゅうごう@空集合}{\bf (empty set)}と呼ぶ.
		\end{dfn}
	\end{screen}
	
	\begin{screen}
		\begin{axm}[空集合の公理]
			$\emptyset$は$\mathcal{L}$のいかなる対象も要素に持たない:
			\begin{align}
				\forall x\ (\ x \notin \emptyset\ ).
			\end{align}
		\end{axm}
	\end{screen}
	
	\monologue{
		院生「空集合は集合の系譜の起点となります.聖書物語でいうところのアダムです.」
	}
	
	\begin{screen}
		\begin{thm}[$\mathcal{L}$のいかなる対象も要素に持たない類は空集合に等しい]
		\label{thm:uniqueness_of_emptyset}
			$a$を類とするとき次が成り立つ:
			\begin{align}
				\forall x\, (\, x \notin a\, ) \Longleftrightarrow a = \emptyset.
			\end{align}
		\end{thm}
	\end{screen}
	
	\begin{prf}
		$a$を類として$\forall x\, (\, x \notin a\, )$が成り立っていると仮定する.このとき
		$\tau$を$\mathcal{L}$の任意の対象とすれば$\tau \notin a$と
		$\tau \notin \emptyset$が共に満たされ,推論規則\ref{metaaxm:fundamental_rules_of_inference}より
		$\tau \notin a \vee \tau \in \emptyset$と$\tau \notin a \vee \tau \in \emptyset$
		が成り立ち,推論法則\ref{metathm:rule_of_inference_3}より
		\begin{align}
			\tau \in a \Longrightarrow \tau \in \emptyset, \quad
			\tau \in \emptyset \Longrightarrow \tau \in a
		\end{align}
		が成り立つ.よって$\tau \in a \Longleftrightarrow \tau \in \emptyset$が成立し,
		$\tau$の任意性と推論法則\ref{metathm:fundamental_law_of_universal_quantifier}を適用して
		\begin{align}
			\forall x\, (\, x \in a \Longleftrightarrow x \in \emptyset\, )
		\end{align}
		が得られる.ゆえに外延性の公理より$a = \emptyset$が成立し,演繹法則より
		\begin{align}
			\forall x\, (\, x \notin a\, ) \Longrightarrow a = \emptyset
		\end{align}
		が従う.逆に$a = \emptyset$が成り立っていると仮定する.ここで$\chi$を$\mathcal{L}$の任意の対象とすれば
		相等性の公理より
		\begin{align}
			\chi \in a \Longrightarrow \chi \in \emptyset
		\end{align}
		が成立し,対偶を取れば$\chi \notin \emptyset \Longrightarrow \chi \notin a$が成り立つ.
		空集合の公理より$\chi \notin \emptyset$は満たされているので三段論法より$\chi \notin a$が成立し,
		$\chi$の任意性と推論法則\ref{metathm:fundamental_law_of_universal_quantifier}より
		$\forall x\, (\, x \notin a\, )$が成立する.ここに演繹法則を適用して
		\begin{align}
			a = \emptyset \Longrightarrow \forall x\, (\, x \notin a\, )
		\end{align}
		も得られる.
		\QED
	\end{prf}
	
	\begin{screen}
		\begin{thm}[空集合はいかなる類も要素に持たない]
		\label{thm:emptyset_does_not_contain_any_class}
			$a,b$を類とするとき次が成り立つ:
			\begin{align}
				b = \emptyset \Longrightarrow a \notin b.
			\end{align}
		\end{thm}
	\end{screen}
	
	\begin{prf}
		$a,b$を類とするとき,類の公理より
		\begin{align}
			a \in b \Longrightarrow \set{a}
		\end{align}
		が成立する.いま$\set{a}$が成り立っていると仮定する.
		このとき$\tau \coloneqq \varepsilon x (\ a = x\ )$とおけば
		存在記号に関する規則から$a = \tau$が成り立つので,
		相等性の公理より$\tau \in b$が従い,存在記号に関する規則より
		\begin{align}
			\exists x\, (\, x \in b\, )
		\end{align}
		が成り立つ.よって演繹法則から
		\begin{align}
			\set{a} \Longrightarrow \exists x\, (\, x \in b\, )
		\end{align}
		が成り立ち,含意の推移律から
		\begin{align}
			a \in b \Longrightarrow \exists x\, (\, x \in b\, )
		\end{align}
		が従う.この対偶を取り推論法則\ref{metathm:properties_of_quantifiers}を適用すれば
		\begin{align}
			\forall x\, (\, x \notin b\, ) \Longrightarrow a \notin b
		\end{align}
		が得られる.定理\ref{thm:uniqueness_of_emptyset}より
		$b = \emptyset \Longrightarrow \forall x\, (\, x \notin b\, )$も成り立ち,含意の推移律から
		\begin{align}
			b = \emptyset \Longrightarrow a \notin b
		\end{align}
		が従う.
		\QED
	\end{prf}
	
	\begin{screen}
		\begin{dfn}[部分類]
			$a,b$を類とするとき,
			\begin{align}
				a \subset b \overset{\mathrm{def}}{\Longleftrightarrow}
				\forall x\ (\ x \in a \Longrightarrow x \in b\ )
			\end{align}
			と定める.式$a \subset b$を``$a$は$b$の{\bf 部分類}\index{ぶぶんるい@部分類}{\bf (subclass)}である''
			と翻訳し,特に$a$が集合である場合は``$a$は$b$の{\bf 部分集合}\index{ぶぶんしゅうごう@部分集合}{\bf (subset)}である''と翻訳する.
			また類$a,b$に対して次を定める:
			\begin{align}
				a \subsetneq b \overset{\mathrm{def}}{\Longleftrightarrow}
				a \subset b \wedge a \neq b.
			\end{align}
		\end{dfn}
	\end{screen}
	
	空虚な真の一例として次の結果を得る.
	
	\begin{screen}
		\begin{thm}[空集合は全ての類に含まれる]
			$a$を類とするとき次が成り立つ:
			\begin{align}
				\emptyset \subset a.
			\end{align}
		\end{thm}
	\end{screen}
	
	\begin{prf}
		$a$を類とする.$\tau$を$\mathcal{L}$の任意の対象とすれば$\tau \notin \emptyset$が成り立つから,
		推論規則\ref{metaaxm:fundamental_rules_of_inference}を適用して
		\begin{align}
			\tau \notin \emptyset \vee \tau \in a
		\end{align}
		が成り立つ.これは$\tau \in \emptyset \Longrightarrow \tau \in a$が成り立つことと同値であり,
		$\tau$の任意性と推論法則\ref{metathm:fundamental_law_of_universal_quantifier}より
		\begin{align}
			\forall x\ (\ x \in \emptyset \Longrightarrow x \in a\ )
		\end{align}
		が成立する.以上から$\emptyset \subset a$が得られる.
		\QED
	\end{prf}
	
	\begin{screen}
		\begin{thm}[類はその部分類に属する全ての類を要素に持つ]\label{thm:subclass_contains_all_elements}
			$a,b,c$を類とすれば次が成り立つ:
			\begin{align}
				a \subset b \Longrightarrow (\ c \in a \Longrightarrow c \in b\ ).
			\end{align}
		\end{thm}
	\end{screen}
	
	\begin{prf}	
		いま$a \subset b$が成り立っているとする.このとき$c \in a$が成り立っていると仮定すれば
		類の公理より$\set{c}$が成り立つ.ここで
		\begin{align}
			\tau \coloneqq \varepsilon x(\ c=x\ )
		\end{align}
		とおくと相等性の公理より$\tau \in a$が成り立ち,
		$a \subset b$と推論法則\ref{metathm:fundamental_law_of_universal_quantifier}から
		$\tau \in b$が従う.再び相等性の公理を適用すれば$c \in b$が成り立つので,
		演繹法則より$a \subset b$が成り立っている下で
		\begin{align}
			c \in a \Longrightarrow c \in b
		\end{align}
		が成立する.再び演繹法則を適用すれば定理の主張が得られる.
		\QED
	\end{prf}
	
	\begin{screen}
		\begin{thm}[$\Univ$は最大の類である]
			$a$を類とするとき次が成り立つ:
			\begin{align}
				a \subset \Univ.
			\end{align}
		\end{thm}
	\end{screen}
	
	\begin{prf}
		$\tau$を$\mathcal{L}$の任意の対象とすれば,定理\ref{thm:any_class_equals_to_itself}と類の公理より
		$\tau \in \Univ$が成立するので,推論規則\ref{metaaxm:fundamental_rules_of_inference}より
		\begin{align}
			\tau \notin a \vee \tau \in \Univ
		\end{align}
		が成立する.このとき推論法則\ref{metathm:rule_of_inference_3}より
		$\tau \in a \Longrightarrow \tau \in \Univ$が成立し,$\tau$の任意性と
		推論法則\ref{metathm:fundamental_law_of_universal_quantifier}から
		\begin{align}
			\forall x\ (\ x \in a \Longrightarrow x \in \Univ\ )
		\end{align}
		が従う.
		\QED
	\end{prf}
	
	\begin{screen}
		\begin{thm}[互いに互いの部分類となる類同士は等しい]\label{thm:mutually_sub_classes_are_equivalent}
			$a,b$を類とするとき次が成り立つ:
			\begin{align}
				a \subset b \wedge b \subset a \Longleftrightarrow a = b.
			\end{align}
		\end{thm}
	\end{screen}
	
	\begin{prf}
		$a \subset b \wedge b \subset a$が成り立っていると仮定する.
		このとき$\tau$を$\mathcal{L}$の任意の対象とすれば,
		$a \subset b$と推論法則\ref{metathm:fundamental_law_of_universal_quantifier}より
		$\tau \in a \Longrightarrow \tau \in b$が成立し,他方で
		$b \subset a$と推論法則\ref{metathm:fundamental_law_of_universal_quantifier}より
		$\tau \in b \Longrightarrow \tau \in a$が成立するので
		\begin{align}
			\tau \in a \Longleftrightarrow \tau \in b
		\end{align}
		が成り立つ.$\tau$の任意性と推論法則\ref{metathm:fundamental_law_of_universal_quantifier}および
		外延性の公理より$a = b$が出るので,演繹法則より
		\begin{align}
			a \subset b \wedge b \subset a \Longrightarrow a = b
		\end{align}
		が得られる.逆に$a = b$が満たされていれば,
		$\tau$を$\mathcal{L}$の任意の対象とすれば$\tau \in a \Longrightarrow \tau \in b$と
		$\tau \in b \Longrightarrow \tau \in a$が共に成り立つので,
		推論法則\ref{metathm:fundamental_law_of_universal_quantifier}より
		$a \subset b$と$b \subset a$が共に従う.よって演繹法則より
		\begin{align}
			a = b \Longrightarrow a \subset b \wedge b \subset a
		\end{align}
		も得られる.
		\QED
	\end{prf}
	
	\monologue{
		院生「定理\ref{thm:subclass_contains_all_elements}と定理\ref{thm:mutually_sub_classes_are_equivalent}より,
			類$a,b$が$a = b$を満たすならば,$a$と$b$は要素に持つ$\mathcal{L}$の対象のみならず,
			要素に持つ類までも一致するのですね.これは
			\begin{itemize}
				\item 要素となりうる類は集合である
				\item 集合は$\mathcal{L}$の或る対象に等しい
			\end{itemize}
			からの帰結です.」
	}
	
	次の話題に進む前に新しい推論法則を提示しておく.
	
	\begin{screen}
		\begin{metathm}[量化記号の性質(ロ)]\label{metathm:properties_of_quantifiers_2}
			$A,B$を$\mathcal{L}'$の式とし,$x$を$A,B$に現れる文字とするとき,$x$のみが$A,B$で量化されていないならば以下は定理である:
			\begin{description}
				\item[(a)] $\exists x ( A(x) \vee B(x) ) \Longleftrightarrow \exists x A(x) \vee \exists x B(x)$.
				
				\item[(b)] $\forall x ( A(x) \wedge B(x) ) \Longleftrightarrow \forall x A(x) \wedge \forall x B(x)$.
			\end{description}
		\end{metathm}
	\end{screen}
	
	\begin{prf}\mbox{}
		\begin{description}
			\item[(a)]
				いま$c(x) \overset{\mathrm{def}}{\Longleftrightarrow} A(x) \vee B(x)$とおけば,
				$\exists x ( A(x) \vee B(x) )$と$\exists x ( C(x) )$は同じ記号列であるから
				\begin{align}
					\exists x ( A(x) \vee B(x) ) \Longrightarrow \exists x C(x)
					\label{eq:metathm_properties_of_quantifiers_1}
				\end{align}
				が成立する.また推論法則\ref{metathm:transitive_law_of_implication}より
				\begin{align}
					\exists x C(x) \Longrightarrow C(\varepsilon x C(x))
					\label{eq:metathm_properties_of_quantifiers_2}
				\end{align}
				が成立する.$C(\varepsilon x C(x))$と$A(\varepsilon x C(x)) \vee B(\varepsilon x C(x))$
				は同じ記号列であるから
				\begin{align}
					C(\varepsilon x C(x)) \Longrightarrow A(\varepsilon x C(x)) \vee B(\varepsilon x C(x))
					\label{eq:metathm_properties_of_quantifiers_3}
				\end{align}
				が成立する.ここで推論法則\ref{metathm:transitive_law_of_implication}と
				推論規則\ref{metaaxm:fundamental_rules_of_inference}より
				\begin{align}
					A(\varepsilon x C(x)) &\Longrightarrow \exists x A(x) \\
						&\Longrightarrow \exists x A(x) \vee \exists x B(x), \\
					B(\varepsilon x C(x)) &\Longrightarrow \exists x B(x) \\
						&\Longrightarrow \exists x A(x) \vee \exists x B(x)
				\end{align}
				が成立するので,場合分け法則より
				\begin{align}
					A(\varepsilon x C(x)) \vee B(\varepsilon x C(x))
					\Longrightarrow \exists x A(x) \vee \exists x B(x)
					\label{eq:metathm_properties_of_quantifiers_4}
				\end{align}
				が成り立つ.(\refeq{eq:metathm_properties_of_quantifiers_1})
				(\refeq{eq:metathm_properties_of_quantifiers_2})
				(\refeq{eq:metathm_properties_of_quantifiers_3})
				(\refeq{eq:metathm_properties_of_quantifiers_4})
				に推論法則\ref{metathm:transitive_law_of_implication}を順次適用すれば
				\begin{align}
					\exists x ( A(x) \vee B(x) ) \Longrightarrow \exists x A(x) \vee \exists x B(x)
				\end{align}
				が得られる.他方,推論規則\ref{metaaxm:rules_of_quantifiers}より
				\begin{align}
					\exists x A(x) &\Longrightarrow A(\varepsilon x A(x)) \\
						&\Longrightarrow A(\varepsilon x A(x)) \vee B(\varepsilon x A(x)) \\
						&\Longrightarrow C(\varepsilon x A(x)) \\
						&\Longrightarrow C(\varepsilon x C(x)) \\
						&\Longrightarrow \exists x C(x) \\
						&\Longrightarrow \exists x (A(x) \vee B(x))
				\end{align}
				が成立し,$A$を$B$に置き換えれば
				$\exists x B(x) \Longrightarrow \exists x (A(x) \vee B(x))$も成り立つので,
				場合分け法則より
				\begin{align}
					\exists x A(x) \vee \exists x B(x) \Longrightarrow \exists x (A(x) \vee B(x))
				\end{align}
				も得られる.
			
			\item[(b)]
				簡略して説明すれば
				\begin{align}
					\forall x \left( A(x) \wedge B(x) \right)
					&\Longleftrightarrow\ \rightharpoondown \exists x \rightharpoondown \left( A(x) \wedge B(x) \right) & (\mbox{推論法則\ref{metathm:properties_of_quantifiers}(c)の対偶}) \\
					&\Longleftrightarrow\ \rightharpoondown \exists x \left( \rightharpoondown A(x) \vee \rightharpoondown B(x) \right) & (\mbox{De Morganの法則}) \\
					&\Longleftrightarrow\ \rightharpoondown \left( \exists x \rightharpoondown A(x) \vee \exists x \rightharpoondown B(x) \right) & (\mbox{前段の対偶}) \\
					&\Longleftrightarrow\ \rightharpoondown \left( \rightharpoondown \forall x A(x) \vee \rightharpoondown \forall x B(x) \right) & (\mbox{推論法則\ref{metathm:properties_of_quantifiers}(c)}) \\
					&\Longleftrightarrow\ \rightharpoondown \rightharpoondown \forall x A(x) \wedge \rightharpoondown \rightharpoondown \forall x B(x) & (\mbox{De Morganの法則}) \\
					&\Longleftrightarrow \forall x A(x) \wedge \forall x B(x) &(\mbox{二重否定の法則})
				\end{align}
				となる.
				\QED
		\end{description}
	\end{prf}
	
	\monologue{
		院生「我々は$\mathcal{L}$の式$A$を用いて$\Set{x}{A(x)}$の記法を導入しましたが,
			今後は$\mathcal{L}'$の式$B$に対しても$\Set{x}{B(x)}$の形で書けると便利なことが多いです.
			当然ながらそれは$B$が同値な$\mathcal{L}$の式で表現できることが条件です.
			そこで真価を発揮するのは$\varepsilon$記号です.後述する通り,
			$\varepsilon$記号は量化記号の意味を表現するのに便利であるだけでなく,
			$\mathcal{L}'$の式を同値な$\mathcal{L}$の式に書き直す際にも有能なのです.」
	}
	
	$a$を類とするとき,$a$は$\mathcal{L}$の対象であるか$\Set{x}{A(x)}$の形をしている.そこで,文字$x$に対し
	\begin{itemize}
		\item $a$が$\mathcal{L}$の対象ならば$\varepsilon a(x) \overset{\mathrm{def}}{\Longleftrightarrow} x \in a$,
		\item $a$が$\Set{x}{A(x)}$の形をしていれば$\varepsilon a(x) \overset{\mathrm{def}}{\Longleftrightarrow} A(x)$,
	\end{itemize}
	として記号列$\varepsilon a(x)$を定める.この記法は
	\begin{align}
		\forall x\, (\, \varepsilon a(x) \Longleftrightarrow x \in a\, )
		\label{eq:a_meaning_of_epsilon_notation}
	\end{align}
	を満たすことを意図している.$\varepsilon$記号を用いているのは,
	量化記号に関する推論規則で$\varepsilon$記号を定めたときと導入の動機が似ているためである.
	
	\begin{screen}
		\begin{dfn}[対]
			$a,b$を類とするとき,
			\begin{align}
				\{a,b\} \coloneqq \Set{x}{
					\forall t\, \left(\, t \in x \Longleftrightarrow \varepsilon a(t)\, \right) \vee 
					\forall t\, \left(\, t \in x \Longleftrightarrow \varepsilon b(t)\, \right)}
			\end{align}
			で$\{a,b\}$を定義し,これを$a$と$b$の{\bf 対}\index{つい@対}{\bf (pair)}と呼ぶ.
			特に$\{a,a\}$を$\{a\}$と書く.
		\end{dfn}
	\end{screen}
	
	$a,b$を類とするとき
	\begin{align}
		\forall x\, \left(\, x = a \vee x = b \Longleftrightarrow 
			\forall t\, \left(\, t \in x \Longleftrightarrow \varepsilon a(t)\, \right) \vee 
			\forall t\, \left(\, t \in x \Longleftrightarrow \varepsilon b(t)\, \right)\, 
		\right)
	\end{align}
	が成立する.従って
	\begin{align}
		\forall x\, (\, x=a \vee x=b \Longleftrightarrow x \in \{a,b\}\, )
		\label{eq:definition_of_a_pair_of_classes}
	\end{align}
	が成立する.これを根拠にして,$\{a,b\}$を
	\begin{align}
		\Set{x}{x = a \vee x = b}
	\end{align}
	と表しても良いことにする.もとより,対の定義はこちらの表記を正当化することを予期したものである.
	
	\monologue{
		院生「一般の類$a,b$に対して,本来
			\begin{align}
				\Set{x}{x = a \vee x = b}
			\end{align}
			は類として失格です.なぜならば,$a,b$の一方でも$\mathcal{L}$の対象ではない場合はそもそも
			\begin{align}
				x = a \vee x = b
			\end{align}
			が$\mathcal{L}$の式でないからです.しかしいちいち$\varepsilon$記号を使っていては見た目が煩雑になりますから,
			表記上は
			\begin{align}
				\Set{x}{x = a \vee x = b}
			\end{align}
			も認めるのです.以後もこのように妥協する場面に直面するでしょうが,
			しかし$\varepsilon$記号を用いれば正式な形に書き直せるのですから解釈上の不具合は無いのです.」
	}
	
	\begin{screen}
		\begin{axm}[対の公理]
			集合同士の対は集合である.つまり,$a,b$を類とするとき次が成り立つ:
			\begin{align}
				\set{a} \wedge \set{b} \Longrightarrow 
				\set{\{a,b\}}.
			\end{align}
		\end{axm}
	\end{screen}
	
	\begin{screen}
		\begin{thm}[真類の対は空・集合は自分を構成要素とする対に属する]
		\label{thm:pair_of_proper_classes_is_emptyset}
			$a,b$を類とするとき次が成り立つ:
			\begin{description}
				\item[(i)] $\set{a} \Longrightarrow a \in \{a,b\}.$
				
				\item[(ii)] $\rightharpoondown \set{a} \wedge \rightharpoondown \set{b} \Longleftrightarrow \{a,b\} = \emptyset.$
			\end{description}
		\end{thm}
	\end{screen}
	
	\begin{prf}\mbox{}
		\begin{description}
			\item[(i)]
				まず存在記号に関する規則より
				\begin{align}
					\set{a} \Longrightarrow a = \varepsilon x(\, a = x\, )
				\end{align}
				も成り立つ.ここで$\set{a}$が成り立っていると仮定して$\tau \coloneqq \varepsilon x(\, a = x\, )$とおけば,
				三段論法より$\tau = a$が成立し,$\vee$の導入より$\tau = a \vee \tau = b$が成り立つ.
				(\refeq{eq:definition_of_a_pair_of_classes})と
				推論法則\ref{metathm:fundamental_law_of_universal_quantifier}より
				$\mathcal{L}$の対象である$\tau$に対しては
				\begin{align}
					\tau = a \vee \tau = b \Longleftrightarrow \tau \in \{a,b\}
				\end{align}
				が満たされるので,三段論法より$\tau \in \{a,b\}$が成り立ち,相等性の公理より
				\begin{align}
					a \in \{a,b\}
				\end{align}
				が従う.ここに演繹法則を適用すれば(i)が得られる.
			
			\item[(ii)]
				いま$\rightharpoondown \set{a} \wedge \rightharpoondown \set{b}$が成り立っているとする.
				このとき推論法則\ref{metathm:properties_of_quantifiers}より
				\begin{align}
					\forall x\, (\, a \neq x\, ) \wedge \forall x\, (\, b \neq x\, )
				\end{align}
				が成り立ち,推論法則\ref{metathm:properties_of_quantifiers_2}より
				\begin{align}
					\forall x\, (\, a \neq x \wedge b \neq x\, )
				\end{align}
				が成立する.ここで$\chi$を$\mathcal{L}$の任意の対象とすれば,
				(\refeq{eq:definition_of_a_pair_of_classes})と
				推論法則\ref{metathm:fundamental_law_of_universal_quantifier}より
				\begin{align}
					a = \chi \vee b = \chi \Longleftrightarrow \chi \in \{a,b\}
				\end{align}
				が成立し,$\wedge$の除去と対偶命題の同値性から
				\begin{align}
					a \neq \chi \wedge b \neq \chi \Longrightarrow \chi \notin \{a,b\}
				\end{align}
				が成り立つ.いま$a \neq \chi \wedge b \neq \chi$が満たされているので三段論法より
				$\chi \notin \{a,b\}$が成立し,$\chi$の任意性と
				推論法則\ref{metathm:fundamental_law_of_universal_quantifier}より
				\begin{align}
					\forall x\, (\, x \notin \{a,b\}\, )
				\end{align}
				が成立する.このとき定理\ref{thm:uniqueness_of_emptyset}より$\{a,b\} = \emptyset$が従うので,演繹法則を適用して
				\begin{align}
					\rightharpoondown \set{a} \wedge \rightharpoondown \set{b} \Longrightarrow \{a,b\} = \emptyset
				\end{align}
				が得られる.一方で(i)の結果と定理\ref{thm:emptyset_does_not_contain_any_class}より
				\begin{align}
					\set{a} \Longrightarrow a \in \{a,b\} \Longrightarrow \{a,b\} \neq \emptyset
				\end{align}
				が成り立ち,同様に$\set{b} \Longrightarrow \{a,b\} \neq \emptyset$も成り立つので
				場合分け法則より
				\begin{align}
					\set{a} \vee \set{b} \Longrightarrow \{a,b\} \neq \emptyset
				\end{align}
				が成立する.この対偶を取りDe Morganの法則を適用すれば
				\begin{align}
					\{a,b\} = \emptyset \Longrightarrow\, \rightharpoondown \set{a} \wedge \rightharpoondown \set{b}
				\end{align}
				も得られる.
				\QED
		\end{description}
	\end{prf}
	
	\monologue{
		院生「上の定理から{\bf 集合は或る類の要素である}という真な言明が得られます.
			実際,$a$を集合とすれば$\{a\}$も集合となり,そして$a \in \{a\}$が成り立ちますね.」
	}
	
	\begin{screen}
		\begin{dfn}[合併]
			$a$を類とするとき
			\begin{align}
				\bigcup a \coloneqq \Set{x}{\exists t \in a\, (\, x \in t\, )}
				\label{eq:definition_of_union_1}
			\end{align}
			で$\bigcup a$を定め,これを$a$の{\bf 合併}\index{がっぺい@合併}{\bf (union)}と呼ぶ.
		\end{dfn}
	\end{screen}
	
	類$a$の合併も
	\begin{align}
		\bigcup a \coloneqq \Set{x}{\exists t\, (\, \varepsilon a(t) \wedge x \in t\, )}
	\end{align}
	と定めるのが本式である.しかし
	\begin{align}
		\forall x\, \left(\, \exists t\, (\, \varepsilon a(t) \wedge x \in t\, )
		\Longleftrightarrow \exists t \in a\, (\, x \in t\, )\, \right)
		\label{eq:definition_of_union_2}
	\end{align}
	が成立するので式(\refeq{eq:definition_of_union_1})を受け入れているのである.
	実際,式(\refeq{eq:a_meaning_of_epsilon_notation})より
	\begin{align}
		\forall t\, \left(\, \varepsilon a(t) \Longleftrightarrow t \in a\, \right)
	\end{align}
	が満たされるので,$\chi$と$\tau$を$\mathcal{L}$の任意の対象とすれば同値関係の遺伝性質より
	\begin{align}
		\varepsilon a(\tau) \wedge \chi \in \tau \Longleftrightarrow \tau \in a \wedge \chi \in \tau
	\end{align}
	が成立する.このとき$\tau$の任意性と推論法則\ref{metathm:properties_of_quantifiers}より
	\begin{align}
		\exists t\, (\, \varepsilon a(t) \wedge \chi \in t\, )
		\Longleftrightarrow \exists t\, (\, t \in a \wedge \chi \in t\, )
	\end{align}
	が成立し,$\chi$の任意性と推論法則\ref{metathm:fundamental_law_of_universal_quantifier}より
	(\refeq{eq:definition_of_union_2})が得られる.正確さも大切だが,やはり判りやすい方が良い.
	
	\begin{screen}
		\begin{axm}[合併の公理]
			集合の合併は集合である.つまり,$a$を類とするとき次が成り立つ:
			\begin{align}
				\set{a} \Longrightarrow \set{\bigcup a}.
			\end{align}
		\end{axm}
	\end{screen}
	
	$a,b$を類とするとき,その対の合併を
	\begin{align}
		a \cup b \overset{\mathrm{def}}{\Longleftrightarrow} \bigcup \{a,b\}
	\end{align}
	と書く.
	
	\begin{screen}
		\begin{thm}[合併の性質]\mbox{}
			\begin{description}
				\item[(i)] $\bigcup \emptyset = \emptyset$
				\item[(ii)] $\set{a} \wedge \set{b} \Longrightarrow 
					\forall x\, (\, x \in a \cup b \Longleftrightarrow x \in a \vee x \in b\, )$
				\item[(iii)] $\Set{x}{A(x)} \cup \Set{x}{\rightharpoondown A(x)} = \Univ$.
			\end{description}
		\end{thm}
	\end{screen}
	
	\begin{prf}\mbox{}
		\begin{description}
			\item[(i)]
				$\chi$と$\tau$を$\mathcal{L}$の任意の対象とすれば,
				空集合の公理と推論法則\ref{metathm:fundamental_law_of_universal_quantifier}より
				$\chi \notin \emptyset$が成立し,さらに$\vee$の導入より
				\begin{align}
					\chi \notin \emptyset \vee \tau \notin \chi
				\end{align}
				が成立する.
				ここで$\chi$の任意性と推論法則\ref{metathm:fundamental_law_of_universal_quantifier}より
				\begin{align}
					\forall x\, (\, x \notin \emptyset \vee \tau \notin \emptyset\, )
				\end{align}
				が成り立つ.ここで推論法則\ref{metathm:properties_of_quantifiers}とDe Morganの法則より
				\begin{align}
					\forall x\, (\, x \notin \emptyset \vee \tau \notin \emptyset\, )
					&\Longleftrightarrow \forall x\, \rightharpoondown (\, x \in \emptyset \wedge \tau \in \emptyset\, ) \\
					&\Longleftrightarrow\, \rightharpoondown \exists x\, (\, x \in \emptyset \wedge \tau \in \emptyset\, )
				\end{align}
				が成立するので,三段論法より$\rightharpoondown \exists x\, (\, x \in \emptyset \wedge \tau \in \emptyset\, )$が成立する.
				他方で合併の定義の対偶を取れば
				\begin{align}
					\rightharpoondown \exists x\, (\, x \in \emptyset \wedge \tau \in \emptyset\, )
					\Longleftrightarrow \tau \notin \bigcup \emptyset
				\end{align}
				が満たされるので,再び三段論法より$\tau \notin \bigcup \emptyset$が成立する.
				$\tau$の任意性と推論法則\ref{metathm:fundamental_law_of_universal_quantifier}より
				\begin{align}
					\forall t\, (\, t \notin \bigcup \emptyset\, )
				\end{align}
				が成立し,定理\ref{thm:uniqueness_of_emptyset}より
				\begin{align}
					\bigcup \emptyset = \emptyset
				\end{align}
				が従う.
		\end{description}
	\end{prf}
	
	\begin{screen}
		\begin{dfn}[交叉]
			$a$を類とするとき,
			\begin{align}
				\bigcap a \coloneqq \Set{x}{\forall t \in a\ (\ x \in t\ )}
			\end{align}
			で$\bigcap a$を定め,これを$a$の{\bf 交叉}\index{こうさ@交叉}{\bf (intersection)}と呼ぶ.
		\end{dfn}
	\end{screen}
	
	$a,b$を類とするとき,その対の合併を
	\begin{align}
		a \cap b \overset{\mathrm{def}}{\Longleftrightarrow} \bigcap \{a,b\}
	\end{align}
	と書く.
	
	\begin{screen}
		\begin{thm}[交叉の性質]\mbox{}
			\begin{description}
				\item[(1)] $\bigcap \emptyset = \Univ$
				\item[(2)] $\forall x\ (\ x \in a \cap b \Longleftrightarrow x \in a \wedge x \in b\ )$
				\item[(3)] $\Set{x}{A(x)} \cap \Set{x}{\rightharpoondown A(x)} = \emptyset$.
			\end{description}
		\end{thm}
	\end{screen}
	
	\begin{prf}
		$x$を$\mathcal{L}$の任意の対象とするとき,空虚な真より
		\begin{align}
			t \in \emptyset \Longrightarrow x \in t
		\end{align}
		は$\mathcal{L}$のいかなる対象$t$に対してもに真となる.ゆえに$\forall t \in \emptyset\ (\ x \in t\ )$が成立し
		\begin{align}
			\forall x\ (\ x \in \bigcap \emptyset\ )
		\end{align}
		が従う.$\forall x\ (\ x \in \Univ\ )$と併せて$\bigcap \emptyset = \Univ$を得る.
		\QED
	\end{prf}
	
	\monologue{
		院生「$\bigcup \emptyset$が$\emptyset$に等しいのは受け容れられますが,
			$\bigcap \emptyset$が$\Univ$に等しいというのは直感に合いません.空虚な真おそるべしです.」
	}
	
	\begin{screen}
		\begin{thm}
			
		\end{thm}
	\end{screen}
	
	\begin{prf}\mbox{}
		\begin{description}
			\item[(1)] $a^{-1}$の任意の要素$t$に対し或る$V$の要素$x,y$が存在して
				\begin{align}
					(x,y) \in a \wedge t = (y,x)
				\end{align}
				を満たす.$((x,y),(y,x)) \in f$より$((x,y),t) \in f$が成り立つから
				$t \in f \ast a$となる.逆に$f \ast a$の任意の要素$t$に対して
				$a$の或る要素$x$が存在して
				\begin{align}
					x \in a \wedge (x,t) \in f
				\end{align}
				となる.$x$に対し$V$の或る要素$a,b$が存在して$x=(a,b)$となるので
				\begin{align}
					((a,b),t) \in f
				\end{align}
				となり,$V$の或る要素$c,d$が存在して
				\begin{align}
					((a,b),t) = ((c,d),(d,c))
				\end{align}
				となる.$(a,b) = (c,d)$より$a=c$かつ$b=d$となり,
				$t = (d,c)$かつ$(d,c)=(b,a)$より$t=(b,a)$,従って
				$t \in a^{-1}$が成り立つ.
		\end{description}
	\end{prf}
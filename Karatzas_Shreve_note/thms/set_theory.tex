\subsection{Dynkin族定理}
	\begin{screen}
		\begin{dfn}[乗法族・Dynkin族]\label{def:Dynkin_system_theorem}
			集合$X$の部分集合の族$\mathscr{A}$が
			任意の$A,B \in \mathscr{A}$に対し$A \cap B \in \mathscr{A}$を満たすとき
			$\mathscr{A}$を$X$上の乗法族($\pi$-system)という.
			$X$の部分集合の族$\mathscr{D}$が
			\begin{description}
				\item[(D1)] $X \in \mathscr{D}$,
				\item[(D2)] $A,B \in \mathscr{D},
					\ A \subset B \quad \Longrightarrow \quad B \backslash A \in \mathscr{D}$,
				\item[(D3)] $\{A_n\}_{n=1}^\infty \subset \mathscr{D},
					\ A_n \cap A_m = \emptyset\ (n \neq m)
					\quad \Longrightarrow \quad \bigcup_{n=1}^\infty A_n \in \mathscr{D}$,
			\end{description}
			を満たすとき,$\mathscr{D}$を$X$上のDynkin族(Dynkin system)という.
		\end{dfn}
	\end{screen}
	
	\begin{screen}
		\begin{dfn}[Dynkin族定理]\label{thm:Dynkin_system_theorem}
			集合$X$上の乗法族$\mathscr{A}$に対し,
			$\mathscr{A}$を含む最小のDynkin族を$\delta(\mathscr{A})$と書くとき,
			\begin{align}
				\delta(\mathscr{A}) = \sigma(\mathscr{A}).
			\end{align}
		\end{dfn}
	\end{screen}
	
	\begin{prf}\mbox{}
		\begin{description}
			\item[第一段]
				$\delta(\mathscr{C})$が交演算で閉じていれば
				$\delta(\mathscr{C})$は$\sigma$-加法族となる.実際任意の$A \in \delta(\mathscr{A})$に対し
				\begin{align}
					A^c = X \backslash A \in \delta(\mathscr{A})
				\end{align}
				となるから,$\delta(\mathscr{C})$が交演算で閉じていれば任意の
				$A_n \in \delta(\mathscr{C})\ (n=1,2,\cdots)$に対し
				\begin{align}
					\bigcup_{n=1}^{\infty} A_n
					= \bigcup_{n=1}^{\infty} A_1^c \cap A_2^c \cap \cdots \cap A_{n-1}^c \cap A_n
					\in \delta(\mathscr{C})
				\end{align}
				が従う.$\sigma$-加法族はDynkin族であるから
				$\sigma(\mathscr{C}) \subset \delta(\mathscr{C})$も成り立ち
				$\sigma(\mathscr{C}) = \delta(\mathscr{C})$が得られる.
			
			\item[第二段]
				$\delta(\mathscr{C})$が交演算について閉じていることを示す.いま,
				\begin{align}
					\mathscr{D}_1 \coloneqq
					\Set{B \in \delta(\mathscr{C})}{ A \cap B \in \delta(\mathscr{C}),\ 
					\forall A \in \mathscr{C}}
				\end{align}
				により定める$\mathscr{D}_1$はDynkin族であり$\mathscr{C}$を含むから
				\begin{align}
					\delta(\mathscr{C}) \subset \mathscr{D}_1
				\end{align}
				が成立する.従って
				\begin{align}
					\mathscr{D}_2 \coloneqq
					\Set{B \in \delta(\mathscr{C})}{ A \cap B \in \delta(\mathscr{C}),\ 
					\forall A \in \delta(\mathscr{C})}
				\end{align}
				によりDynkin族$\mathscr{D}_2$を定めれば,$\mathscr{C} \subset \mathscr{D}_2$が満たされ
				\begin{align}
					\delta(\mathscr{C}) \subset \mathscr{D}_2
				\end{align}
				が得られる.よって$\delta(\mathscr{C})$は交演算について閉じている.
				\QED
		\end{description}
	\end{prf}
	
	\begin{screen}
		\begin{thm}
			集合$X$の部分集合族$\mathscr{D}$が
			の定義\ref{def:Dynkin_system_theorem}の(D1),(D2)を満たしているとき,
			$\mathscr{D}$が(D3)を満たすことと
			$\mathscr{D}$が増大列の可算和で閉じることは同値である.
		\end{thm}
	\end{screen}
	
	\begin{prf}
		$\mathscr{D}$が可算直和について閉じているとする.このとき
		単調増大列$A_1 \subset A_2 \subset \cdots$を取り
		\begin{align}
			B_1 \coloneqq A_1,
			\quad B_n \coloneqq A_n \backslash A_{n-1},
			\quad (n \geq 2)
		\end{align}
		とおけば(D2)より$B_n \in \mathscr{D},\ (\forall n \geq 1)$が満たされ,
		$n \neq m$なら$B_n \cap B_m = \emptyset$となるから
		\begin{align}
			\bigcup_{n=1}^{\infty} A_n = \bigcup_{n=1}^{\infty} B_n \in \mathscr{D} 
		\end{align}
		が成立する.逆に$\mathscr{D}$が増大列の可算和で閉じているとする.
		(D1)(D2)より互いに素な$A,B \in \mathscr{D}$に対し
		$A^c \in \mathscr{D}$及び$A^c \cap B^c = A^c \backslash B\in \mathscr{D}$が成り立つから,
		$\mathscr{D}$の互いに素な集合列$(B_n)_{n=1}^{\infty}$を取れば
		\begin{align}
			B_1^c \cap B_2^c \cap \cdots \cap B_n^c
			= \left( \cdots \left( \left( B_1^c \cap B_2^c \right) \cap B_3^c \right) \cap \cdots \cap B_{n-1}^c \right) \cap B_n^c
			\in \mathscr{D},
			\quad (n=1,2,\cdots)
		\end{align}
		が得られる.よって
		\begin{align}
			D_n \coloneqq \bigcup_{i=1}^n B_i = X \backslash \Biggl( \bigcap_{i=1}^n B_i^c \Biggr),
			\quad (n=1,2,\cdots)
		\end{align}
		により$\mathscr{D}$の単調増大列$(D_n)_{n=1}^{\infty}$を定めれば
		\begin{align}
			\bigcup_{n=1}^{\infty} B_n = \bigcup_{n=1}^{\infty} D_n \in \mathscr{D}
		\end{align}
		が成立する.
		\QED
	\end{prf}

\subsection{上限下限}
	\begin{screen}
		\begin{thm}[上限の冪と冪の上限]\label{thm:exponentiation_of_supremum_supremum_of_exponentiation}
			任意の空でない$S \subset [0,\infty)$と$t > 0$に対し次が成立する:
			\begin{align}
				(\sup{}{S})^t = \sup{}{\Set{s^t}{s \in S}}.
			\end{align}
		\end{thm}
	\end{screen}
	
	\begin{prf}
		$S=\{0\}$なら両辺0で一致するので,$S$は$\{0\}$より真に大きいとする.このとき
		任意の$s \in S$に対し$s^t \leq (\sup{}{S})^t$となるから$\sup{}{\Set{s^t}{s \in S}} \leq (\sup{}{S})^t$が従う.
		また任意の$(\sup{}{S})^t > \alpha > 0$に対し$s > \alpha^{1/t}$を満たす$s \in S$が存在し
		$(\sup{}{S})^t \geq s^t > \alpha$となるから$\sup{}{\Set{s^t}{s \in S}} = (\sup{}{S})^t$が得られる.
		\QED
	\end{prf}

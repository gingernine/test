\section{乗法}
	$\alpha$と$\beta$を順序数とするとき,次は$\alpha$に$\beta$を``掛ける''という操作を施したい.若干見づらいが
	\begin{align}
		\cdot
	\end{align}
	を掛け算の記号として,先ずは簡単に,$\beta$が$0$の場合は
	\begin{align}
		\alpha \cdot 0 \defeq 0
	\end{align}
	と定めてしまう.$\beta$が$1$の場合は,$1$を掛けるという操作を
	$\alpha$が$1$個だけあるとの意味で
	\begin{align}
		\alpha \cdot 1 \defeq \alpha
	\end{align}
	と定める.$2$を掛けるという操作は$\alpha$が$2$個あるという意味で
	\begin{align}
		\alpha \cdot 2 \defeq \alpha + \alpha
	\end{align}
	と定める.$3$を掛けるという操作も$\alpha$が$3$個あるという意味で
	\begin{align}
		\alpha \cdot 3 \defeq (\alpha + \alpha) + \alpha
	\end{align}
	と定めるが,ところでこれは
	\begin{align}
		(\alpha \cdot 2) + \alpha
	\end{align}
	に等しいので,これに倣って
	\begin{align}
		&\alpha \cdot 4 \defeq (\alpha \cdot 3) + \alpha \\
		&\alpha \cdot 5 \defeq (\alpha \cdot 4) + \alpha \\
		&\alpha \cdot 6 \defeq (\alpha \cdot 5) + \alpha
	\end{align}
	と以降も定義していく.また,例えば$\beta$が$\Natural$である場合,加法の時と同様に
	\begin{align}
		\alpha \cdot \Natural \defeq \bigcup \Set{\alpha \cdot k}{k \in \Natural}
	\end{align}
	と定める.以上の操作をヒントにして,$\alpha \cdot \beta$は
	\begin{itemize}
		\item $\beta$に対して$\beta = \gamma + 1$を満たす順序数$\gamma$が取れるなら
			\begin{align}
				\alpha \cdot \beta \defeq (\alpha \cdot \gamma) + \alpha,
			\end{align}
		
		\item $\beta$が極限数なら
			\begin{align}
				\alpha \cdot \beta \defeq \bigcup \Set{\alpha \cdot \gamma}{\gamma \in \beta},
			\end{align}
	\end{itemize}
	で定められる.
	
	\begin{screen}
		\begin{dfn}[順序数の乗法]
			$\alpha$を順序数とし,$\Univ$上の写像$G_\alpha$を
			\begin{align}
				x \longmapsto 
				\begin{cases}
					0 & \mbox{if } \operatorname{dom}(x) = \emptyset \\
					x(\beta) + \alpha & \mbox{if } \beta \in \ON \wedge \operatorname{dom}(x) = \beta \cup \{\beta\} \\
					\bigcup \ran{x} & \mbox{o.w.}
				\end{cases}
			\end{align}
			なる関係により定めると,
			\begin{align}
				\forall \beta \in \ON\, (\, M_\alpha(\beta) = G_\alpha(M_\alpha|_\beta)\, )
			\end{align}
			を満たす$\ON$上の写像$M_\alpha$が取れる.ここで
			\begin{align}
				\cdot \defeq \Set{((\alpha,\beta),y)}{\alpha \in \ON \wedge \beta \in \ON \wedge y = M_\alpha (\beta)}
			\end{align}
			により$\cdot$を定め,これを$\ON$上の{\bf 乗法}\index{じょうほう@乗法}{\bf (multiplication)}と呼ぶ.
		\end{dfn}
	\end{screen}
	
	\begin{screen}
		\begin{thm}[$\cdot$は$\ON$への全射である]\label{thm:multiplication_on_ordinal_numbers_is_a_mapping}
			$\cdot$は$\ON \times \ON$から$\ON$への全射である:
			\begin{align}
				\cdot:\ON \times \ON \srj \ON.
			\end{align}
		\end{thm}
	\end{screen}
	
	$\alpha$と$\beta$を順序数とするとき,
	\begin{align}
		\cdot(\alpha,\beta)
	\end{align}
	もまた中置記法により
	\begin{align}
		\alpha \cdot \beta
	\end{align}
	と表記する.また$\alpha \cdot \beta$なる順序数を$\alpha$と$\beta$の{\bf 積}\index{せき@積}{\bf (product)}と呼ぶ.
	
	\begin{screen}
		\begin{thm}[$0$を掛けたら$0$]
			\begin{align}
				\forall \alpha \in \ON\, \left(\, \alpha \cdot 0 = 0 \cdot \alpha = 0\, \right).
			\end{align}
		\end{thm}
	\end{screen}
	
	\begin{screen}
		\begin{thm}[$1$を掛けても変わらない]
			\begin{align}
				\forall \alpha \in \ON\, \left(\, \alpha \cdot 1 = 1 \cdot \alpha = \alpha\, \right).
			\end{align}
		\end{thm}
	\end{screen}
	
	\begin{screen}
		\begin{thm}[乗法は結合的]
			\begin{align}
				\forall \alpha,\beta,\gamma \in \ON\, 
				\left[\, (\alpha \cdot \beta) \cdot \gamma = \alpha \cdot (\beta \cdot \gamma)\, \right].
			\end{align}
		\end{thm}
	\end{screen}
	
	\begin{screen}
		\begin{thm}[自然数の積は自然数]\label{thm:product_of_natural_numbers_is_a_natural_number}
			\begin{align}
				\forall n, m \in \Natural\, (\, n \cdot m \in \Natural\, ).
			\end{align}
		\end{thm}
	\end{screen}
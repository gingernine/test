\subsection{密集点}
	\begin{screen}
		\begin{dfn}[集積点・密集点]
			$(S,\mathscr{O})$を位相空間とし,$S$は空でないとする.
			また$\mathcal{V}$を$S$上の写像で,$S$の要素に対して
			その$\mathscr{O}$-近傍系を対応させるものとする.
			$x$を$S$の要素として,$b$を$S$の部分集合とするとき,
			\begin{align}
				\forall v \in \mathcal{V}_{x}\, 
				\left[\, v \cap (b \backslash \{x\}) \neq \emptyset\, \right]
			\end{align}
			が成り立つならば$x$は$b$の$\mathscr{O}$-{\bf 集積点}\index{しゅうせきてん@集積点}
			{\bf (accumulation point)}であるという.同様に,
			\begin{align}
				\forall v \in \mathcal{V}_{x}\, \left[\, v \cap b \neq \emptyset\, \right]
			\end{align}
			が成り立つならば$x$は$b$の$\mathscr{O}$-{\bf 密集点}\index{みっしゅうてん@密集点}
			{\bf (cluster point)}であるという.
		\end{dfn}
	\end{screen}
	
	集積点と密集点の明確な違いは$T_1$空間(後述)において現れる.
	まず,$x$は$\{x\}$の自明な$\mathscr{O}$-密集点である.
	実際$v$を$x$の$\mathscr{O}$-近傍とすれば
	\begin{align}
		x \in v
	\end{align}
	が成り立つので
	\begin{align}
		v \cap \{x\} \neq \emptyset
	\end{align}
	が従う.しかし$\{x\}$が$\mathscr{O}$-閉集合であるときは
	$x$は$\{x\}$の$\mathscr{O}$-集積点ではない.さらに言えば,
	この場合$S$のいかなる要素も$\{x\}$の$\mathscr{O}$-集積点ではない.
	
	\begin{screen}
		\begin{thm}[閉である一点集合は集積点を持たない]
		\label{thm:closed_singleton_has_no_accumulation_point}
			$(S,\mathscr{O})$を位相空間とし,$S$は空でないとし,$x$を$S$の要素とする.
			このとき,$\{x\}$が$\mathscr{O}$-閉集合ならば$S$のいかなる要素も$\{x\}$の$\mathscr{O}$-集積点ではない.つまり,$\mathcal{V}$を$S$上の写像で,$S$の要素に対して
			その$\mathscr{O}$-近傍系を対応させるものとすれば
			\begin{align}
				S \backslash \{x\} \in \mathscr{O} \Longrightarrow
				\forall y \in S\, \exists v \in \mathcal{V}_{y}\, 
				\left[\, x \notin v \vee x = y\, \right].
			\end{align}
		\end{thm}
	\end{screen}
	
	$x,y,v$を集合とするとき,
	\begin{align}
		x \notin v \vee x = y
	\end{align}
	は
	\begin{align}
		\forall t\, \left[\, t = x \wedge t \neq y \Longrightarrow t \notin v\, \right]
	\end{align}
	と同値である.すなわち
	\begin{align}
		v \cap (\{x\} \backslash \{y\}) = \emptyset
	\end{align}
	と同値である.
	
	\begin{sketch}
		$\{x\}$が$\mathscr{O}$-閉集合であるとし,$y$を$S$の要素とする.
		\begin{align}
			x = y
		\end{align}
		ならば
		\begin{align}
			x \notin S \vee x = y
		\end{align}
		が成り立つので
		\begin{align}
			\exists v \in \mathcal{V}_{y}\, \left[\, x \notin v \vee x = y\, \right]
		\end{align}
		が満たされる.
		\begin{align}
			x \neq y
		\end{align}
		であるとき,
		\begin{align}
			v \defeq S \backslash \{x\}
		\end{align}
		とおけば$v$は$y$を要素に持つ$\mathscr{O}$-開集合であるから
		\begin{align}
			v \in \mathcal{V}_{y}
		\end{align}
		が成り立つ.そして
		\begin{align}
			x \notin v \vee x = y
		\end{align}
		が成り立つので,この場合も
		\begin{align}
			\exists v \in \mathcal{V}_{y}\, \left[\, x \notin v \vee x = y\, \right]
		\end{align}
		が満たされる.
		\QED
	\end{sketch}
	
	\begin{screen}
		\begin{thm}[閉集合は密集点集合]
		\label{thm:belongs_to_closure_iff_clusters}
			$(S,\mathscr{O})$を位相空間とし,$S$は空でないとする.
			また$x$を$S$の要素とし,$b$を$S$の部分集合とする.このとき,
			$x$が$b$の$\mathscr{O}$-閉包に属していることと
			$x$が$b$の$\mathscr{O}$-密集点であることは同値である.つまり,
			\begin{align}
				x \in \overline{b} \Longleftrightarrow
				\forall v \in \mathcal{V}_{x}\, (\, v \cap b \neq \emptyset\, )
				\label{thm_belongs_to_closure_iff_clusters}
			\end{align}
			が成り立つ.ただし$\overline{b}$は$b$の$\mathscr{O}$-閉包であり,
			$\mathcal{V}_{x}$とは$x$の$\mathscr{O}$-近傍系である.
		\end{thm}
	\end{screen}
	
	特に,{\bf $b$が$\mathscr{O}$-閉集合であることと$b$の$\mathscr{O}$-密集点全体が
	$b$に一致することは同値である.}つまり,$\mathcal{V}$を$S$上の写像で,$S$の要素に対してその
	$\mathscr{O}$-近傍系を対応させるものとすれば,
	\begin{align}
		S \backslash b \in \mathscr{O} \Longleftrightarrow
		b = \Set{x \in S}{\forall v \in \mathcal{V}_{x}\, (\, v \cap b \neq \emptyset\, )}
	\end{align}
	が成り立つ.実際,定理\ref{thm:closed_set_coincides_with_its_closure}より
	$b$が$\mathscr{O}$-閉集合であることと
	\begin{align}
		b = \overline{b}
	\end{align}
	が成り立つことは同値であり,(\refeq{thm_belongs_to_closure_iff_clusters})より
	\begin{align}
		\forall x\, \left[\, x \in \overline{b} \Longleftrightarrow
		x \in S \wedge \forall v \in \mathcal{V}_{x}\, (\, v \cap b \neq \emptyset\, )\, \right]
	\end{align}
	が成り立つので,
	\begin{align}
		S \backslash b \in \mathscr{O} \Longleftrightarrow
		\forall x\, \left[\, x \in b \Longleftrightarrow
		x \in S \wedge \forall v \in \mathcal{V}_{x}\, (\, v \cap b \neq \emptyset\, )\, \right]
	\end{align}
	が得られる.
	
	\begin{sketch}
		$x$の$\mathscr{O}$-近傍$v$で
		\begin{align}
			v \cap b = \emptyset
		\end{align}
		を満たすものが取れるとき,
		\begin{align}
			x \subset o \wedge o \subset v
		\end{align}
		を満たす$\mathscr{O}$-開集合$o$が取れるが,このとき
		\begin{align}
			b \subset S \backslash o
		\end{align}
		が成り立つので
		\begin{align}
			\overline{b} \subset S \backslash o
		\end{align}
		が従う.よってこのとき
		\begin{align}
			x \notin \overline{b}
		\end{align}
		である.逆に$x$が$\overline{b}$に属さないとき,
		\begin{align}
			v \defeq S \backslash \overline{b}
		\end{align}
		とおけば,$v$は$x$を要素に持つ$\mathscr{O}$-開集合であるから
		\begin{align}
			v \in \mathcal{V}_{x}
		\end{align}
		が成り立つ.よってこのとき
		\begin{align}
			\exists v \in \mathcal{V}_{x}\, (\, v \cap b = \emptyset\, )
		\end{align}
		が成立する.
		\QED
	\end{sketch}
	
	定理\ref{thm:belongs_to_closure_iff_clusters}をそのまま適用することにより,
	閉包によって集積点を特徴づけることが出来る.
	\begin{screen}
		\begin{thm}[集積点は閉包に捕らえられる]
			$(S,\mathscr{O})$を位相空間とし,$S$は空でないとする.
			$x$を$S$の要素とし,$b$を$S$の部分集合とするとき,
			\begin{align}
				x \in \overline{b \backslash \{x\}} \Longleftrightarrow
				\forall v \in \mathcal{V}_{x}\, \left[\, v \cap (b \backslash \{x\}) \neq \emptyset\, \right]
			\end{align}
		\end{thm}
	\end{screen}
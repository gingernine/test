\subsection{密集点}
	\begin{screen}
		\begin{dfn}[集積点・密集点]
			位相空間$S$の点$x$と部分集合$A$について,
			$x$の任意の近傍$U$に対し
			\begin{align}
				(U \backslash \{x\}) \cap A \neq \emptyset
			\end{align}
			となるとき,$x$は$A$の{\bf 集積点}\index{しゅうせきてん@集積点}
			{\bf (accumulation point)}であるという.
			同様に$x$の任意の近傍$U$に対し
			\begin{align}
				U \cap A \neq \emptyset
			\end{align}
			となるとき,$x$は$A$の{\bf 密集点}\index{みっしゅうてん@密集点}
			{\bf (cluster point)}であるという.
		\end{dfn}
	\end{screen}
	
	集積点と密集点の明確な違いは$T_1$空間(後述)において現れる.
	\begin{screen}
		\begin{thm}[閉である一点集合は集積点を持たない]
		\label{thm:closed_singleton_has_no_accumulation_point}
			位相空間において,閉じている一点集合は集積点を持たない.特に
			$\{x\}$が閉であるとき,$x$は$\{x\}$の密集点ではあるが集積点ではない.
		\end{thm}
	\end{screen}
	
	\begin{prf}
		一点集合$\{x\}$が閉であるとする.このとき$y \neq x$なら
		$U \coloneqq \{x\}^c$は$y$の開近傍となり
		\begin{align}
			(U \backslash \{y\}) \cap \{x\} = \emptyset
		\end{align}
		を満たすから$y$は$\{x\}$の集積点ではない.
		$x$は$\{x\}$の集積点となりえないから$\{x\}$は集積点を持たない.
		\QED
	\end{prf}
	
	\begin{screen}
		\begin{thm}[閉集合は密集点集合]
		\label{thm:belongs_to_closure_iff_clusters}
			位相空間$S$の点$x$と部分集合$A$について次が成り立つ:
			\begin{align}
				x \in \overline{A} \quad \Longleftrightarrow \quad
				\mbox{$x$は$A$の密集点である}.
				\label{eq:thm_belongs_to_closure_iff_clusters}
			\end{align}
			特に,$A$が閉であることと$A$の密集点全体が$A$に一致することは同値になる.
		\end{thm}
	\end{screen}
	
	\begin{prf}
		$x$の或る近傍$U$が$U \cap A = \emptyset$を満たすとき,
		定理\ref{thm:topology_note_closure_interior}より
		\begin{align}
			x \in U^i \subset A^{ci} = A^{ac}
		\end{align}
		となり$x \notin \overline{A}$が従う.逆に
		$x \notin \overline{A}$なら
		$\overline{A}^c$は$A$と交わらない$x$の開近傍となるから
		(\refeq{eq:thm_belongs_to_closure_iff_clusters})が出る.
		また
		\begin{align}
			\mbox{$A$が閉} \quad \Longleftrightarrow \quad A = \overline{A}
			\quad \Longleftrightarrow \quad
			\mbox{$A$の密集点全体が$A$に一致}
		\end{align}
		が成立する.
		\QED
	\end{prf}
	
	\begin{screen}
		\begin{thm}[$x \in \overline{A \backslash \{x\}}$$\Longleftrightarrow$$x$が$A$の集積点]
			位相空間$S$の点$x$と部分集合$A$について次が成り立つ:
			\begin{align}
				x \in \overline{A \backslash \{x\}} \quad \Longleftrightarrow \quad
				\mbox{$x$は$A$の集積点である}.
			\end{align}
		\end{thm}
	\end{screen}
	
	\begin{prf}
		$x$の任意の近傍$U$に対し
		$U \cap (A \backslash \{x\}) = (U \backslash \{x\}) \cap A$となるから,
		定理\ref{thm:belongs_to_closure_iff_clusters}と併せて
		\begin{align}
			x \in \overline{A \backslash \{x\}} 
			&\quad \Longleftrightarrow \quad
			\mbox{$x$の任意の近傍$U$に対し$U \cap (A \backslash \{x\}) \neq \emptyset$} \\
			&\quad \Longleftrightarrow \quad
			\mbox{$x$の任意の近傍$U$に対し$(U \backslash \{x\}) \cap A \neq \emptyset$}
			\quad \Longleftrightarrow \quad
			\mbox{$x$は$A$の集積点}
		\end{align}
		が成立する.
		\QED
	\end{prf}
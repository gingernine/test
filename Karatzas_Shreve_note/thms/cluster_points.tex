\subsection{密集点}
	\begin{screen}
		\begin{dfn}[集積点・密集点]
			$(S,\mathscr{O})$を位相空間とし,$S$は空でないとする.
			また$\mathcal{V}$を$S$上の写像で,$S$の要素に対して
			その$\mathscr{O}$-近傍系を対応させるものとする.
			$x$を$S$の要素として,$a$を$S$の部分集合とするとき,
			\begin{align}
				\forall v \in \mathcal{V}_{x}\, \left[\, (v \backslash \{x\}) \cap a \neq \emptyset\, \right]
			\end{align}
			が成り立つならば$x$は$a$の$\mathscr{O}$-{\bf 集積点}\index{しゅうせきてん@集積点}
			{\bf (accumulation point)}であるという.同様に,
			\begin{align}
				\forall v \in \mathcal{V}_{x}\, \left[\, v \cap a \neq \emptyset\, \right]
			\end{align}
			が成り立つならば$x$は$a$の$\mathscr{O}$-{\bf 密集点}\index{みっしゅうてん@密集点}
			{\bf (cluster point)}であるという.
		\end{dfn}
	\end{screen}
	
	集積点と密集点の明確な違いは$T_1$空間(後述)において現れる.
	まず,$x$は$\{x\}$の自明な$\mathscr{O}$-密集点である.
	実際$v$を$x$の$\mathscr{O}$-近傍とすれば
	\begin{align}
		x \in v
	\end{align}
	が成り立つので
	\begin{align}
		v \cap \{x\} \neq \emptyset
	\end{align}
	が従う.しかし$\{x\}$が$\mathscr{O}$-閉集合であるときは
	$x$は$\{x\}$の$\mathscr{O}$-集積点ではない.さらに言えば,
	この場合$S$のいかなる要素も$\{x\}$の$\mathscr{O}$-集積点ではない.
	
	\begin{screen}
		\begin{thm}[閉である一点集合は集積点を持たない]
		\label{thm:closed_singleton_has_no_accumulation_point}
			$(S,\mathscr{O})$を位相空間とし,$S$は空でないとし,$x$を$S$の要素とする.
			このとき,$\{x\}$が$\mathscr{O}$-閉集合ならば$S$のいかなる要素も$\{x\}$の$\mathscr{O}$-集積点ではない.つまり,$\mathcal{V}$を$S$上の写像で,$S$の要素に対して
			その$\mathscr{O}$-近傍系を対応させるものとすれば
			\begin{align}
				S \backslash \{x\} \in \mathscr{O} \Longrightarrow
				\forall y \in S\, \exists v \in \mathcal{V}_{y}\, 
				\left[\, x \notin v \vee x = y\, \right].
			\end{align}
		\end{thm}
	\end{screen}
	
	$x \notin v \vee x = y$とは$x \in v \backslash \{y\}$の否定である.すなわち
	\begin{align}
		(v \backslash \{y\}) \cap \{x\} \neq \emptyset
	\end{align}
	の否定である.
	
	\begin{sketch}
		$\{x\}$が$\mathscr{O}$-閉集合であるとし,$y$を$S$の要素とする.
		\begin{align}
			x = y
		\end{align}
		ならば
		\begin{align}
			x \notin S \vee x = y
		\end{align}
		が成り立つので
		\begin{align}
			\exists v \in \mathcal{V}_{y}\, \left[\, x \notin v \vee x = y\, \right]
		\end{align}
		が満たされる.
		\begin{align}
			x \neq y
		\end{align}
		であるとき,
		\begin{align}
			v \defeq S \backslash \{x\}
		\end{align}
		とおけば$v$は$y$を要素に持つ$\mathscr{O}$-開集合であるから
		\begin{align}
			v \in \mathcal{V}_{y}
		\end{align}
		が成り立つ.そして
		\begin{align}
			x \notin v \vee x = y
		\end{align}
		が成り立つので,この場合も
		\begin{align}
			\exists v \in \mathcal{V}_{y}\, \left[\, x \notin v \vee x = y\, \right]
		\end{align}
		が満たされる.
		\QED
	\end{sketch}
	
	\begin{screen}
		\begin{thm}[閉集合は密集点集合]
		\label{thm:belongs_to_closure_iff_clusters}
			$(S,\mathscr{O})$を位相空間とし,$S$は空でないとする.
			また$x$を$S$の要素とし,$a$を$S$の部分集合とする.このとき,
			$x$が$a$の$\mathscr{O}$-閉包に属していることと
			$x$が$a$の$\mathscr{O}$-密集点であることは同値である.つまり,
			\begin{align}
				x \in \overline{a} \Longleftrightarrow
				\forall v \in \mathcal{V}_{x}\, (\, v \cap a \neq \emptyset\, )
			\end{align}
			が成り立つ.ただし$\overline{a}$は$a$の$\mathscr{O}$-閉包であり,
			$\mathcal{V}_{x}$とは$x$の$\mathscr{O}$-近傍系である.
		\end{thm}
	\end{screen}
	
	特に,$a$が$\mathscr{O}$-閉集合であることと$a$の$\mathscr{O}$-密集点全体が
	$a$に一致することは同値である.つまり,$\mathcal{V}$を$S$上の写像で,$S$の要素に対してその
	$\mathscr{O}$-近傍系を対応させるものとすれば,
	\begin{align}
		S \backslash a \in \mathscr{O} \Longleftrightarrow
		a = \Set{x \in S}{\forall v \in \mathcal{V}_{x}\, (\, v \cap a \neq \emptyset\, )}
	\end{align}
	が成り立つ.実際,定理\ref{thm:closed_set_coincides_with_its_closure}より
	$a$が$\mathscr{O}$-閉集合であることと
	\begin{align}
		a = \overline{a}
	\end{align}
	が成り立つことは同値であり,
	
	\begin{sketch}
		$x$の$\mathscr{O}$-近傍$v$で
		\begin{align}
			v \cap a = \emptyset
		\end{align}
		を満たすものが取れるとき,
		\begin{align}
			x \subset o \wedge o \subset v
		\end{align}
		を満たす$\mathscr{O}$-開集合$o$が取れるが,このとき
		\begin{align}
			a \subset S \backslash o
		\end{align}
		が成り立つので
		\begin{align}
			\overline{a} \subset S \backslash o
		\end{align}
		が従う.よってこのとき
		\begin{align}
			x \notin \overline{a}
		\end{align}
		である.逆に$x$が$\overline{a}$に属さないとき,
		\begin{align}
			v \defeq S \backslash \overline{a}
		\end{align}
		とおけば,$v$は$x$を要素に持つ$\mathscr{O}$-開集合であるから
		\begin{align}
			v \in \mathcal{V}_{x}
		\end{align}
		が成り立つ.よってこのとき
		\begin{align}
			\exists v \in \mathcal{V}_{x}\, (\, v \cap a = \emptyset\, )
		\end{align}
		が成立する.
		\QED
		また
		\begin{align}
			\mbox{$A$が閉} \quad \Longleftrightarrow \quad A = \overline{A}
			\quad \Longleftrightarrow \quad
			\mbox{$A$の密集点全体が$A$に一致}
		\end{align}
		が成立する.
		\QED
	\end{sketch}
	
	\begin{screen}
		\begin{thm}[$x \in \overline{A \backslash \{x\}}$$\Longleftrightarrow$$x$が$A$の集積点]
			位相空間$S$の点$x$と部分集合$A$について次が成り立つ:
			\begin{align}
				x \in \overline{A \backslash \{x\}} \quad \Longleftrightarrow \quad
				\mbox{$x$は$A$の集積点である}.
			\end{align}
		\end{thm}
	\end{screen}
	
	\begin{prf}
		$x$の任意の近傍$U$に対し
		$U \cap (A \backslash \{x\}) = (U \backslash \{x\}) \cap A$となるから,
		定理\ref{thm:belongs_to_closure_iff_clusters}と併せて
		\begin{align}
			x \in \overline{A \backslash \{x\}} 
			&\quad \Longleftrightarrow \quad
			\mbox{$x$の任意の近傍$U$に対し$U \cap (A \backslash \{x\}) \neq \emptyset$} \\
			&\quad \Longleftrightarrow \quad
			\mbox{$x$の任意の近傍$U$に対し$(U \backslash \{x\}) \cap A \neq \emptyset$}
			\quad \Longleftrightarrow \quad
			\mbox{$x$は$A$の集積点}
		\end{align}
		が成立する.
		\QED
	\end{prf}
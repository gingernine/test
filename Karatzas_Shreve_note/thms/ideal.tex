\section{イデアル}
	\monologue{
		まず和の記号$\sum$を定めましょう.例えば,いま実数の列
		\begin{align}
			a_0,\ a_1,\ a_2,\ a_3,\ \cdots
		\end{align}
		が与えられたとすれば,その$n$個の和
		\begin{align}
			a_0 + a_1 + \cdots + a_{n-1}
		\end{align}
		を$\sum$を用いて
		\begin{align}
			\sum_{i=0}^{n-1} a_i
		\end{align}
		と書くように定めれば便利です.$n$個の和とは直感的には添え字を順に辿って$n$個の要素を
		合計すれば良いだけですが,その操作を$\mathcal{L}'$の言葉で表現しなくては数学ではありません.
		我々が使える道具の中で,順番に足すという再帰的な操作を表現するには写像の概念が最適でしょう.
	}
	
	\begin{screen}
		\begin{dfn}[イデアル]
			$(R,\sigma,\mu)$を環とするとき,$R$の部分集合$J$が
			\begin{itemize}
				\item $\forall a,b \in J\ (\ \sigma(a,b) \in J\ )$
				\item $\forall a \in J\ \forall r \in R\ (\ \mu(r,a) \in J\ )$
			\end{itemize}
			を満たすとき,$J$を$R$の{\bf 左イデアル}\index{ひたりいである@左イデアル}{\bf (left ideal)}と呼ぶ.
			また二つ目の条件を
			\begin{itemize}
				\item $\forall a \in J\ \forall r \in R\ (\ \mu(a,r) \in J\ )$
			\end{itemize}
			に取り替えた場合,$J$を$R$の{\bf 右イデアル}\index{みぎいである@右イデアル}{\bf (right ideal)}と呼ぶ.
			左イデアルであり右イデアルでもある部分集合を{\bf イデアル}\index{いである@イデアル}{\bf (ideal)}と呼ぶ.
		\end{dfn}
	\end{screen}
	
	考察対象は主に左イデアルである.左右を反転させれば左イデアルに関する結果は右イデアルにも当てはまる.
	
	\begin{screen}
		\begin{thm}[左イデアルは加法に関して群をなす]
			$(R,\sigma,\mu)$を環とし,$J$をこの環の左イデアルとするとき,
			\begin{align}
				\sigma_J \coloneqq \sigma|_{J \times J}
			\end{align}
			とおけば$(J,\sigma_J)$は可換群となる.
		\end{thm}
	\end{screen}
	
	\monologue{
		つまり,左イデアルとは左側からの掛け算で閉じている加法部分群であると言えます.
	}
	
	
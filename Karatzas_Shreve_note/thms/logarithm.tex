\subsection{対数関数}
	
	$z$を複素数とするとき,$z$の{\bf 対数}\index{たいすう@対数}{\bf (logarithm)}とは
	\begin{align}
		z = \exp{w}
	\end{align}
	を満たす複素数$w$のことを指すが,$\exp$は周期関数であるからそのような$w$は整数の数だけ,
	つまり可算無限個存在する.対数関数とは指数関数の逆写像にあたるもので,
	複素数$z$に対して対数の全体を対応させる写像である.つまり$z$に対し
	\begin{align}
		\Set{w \in \C}{z = \exp{w}}
	\end{align}
	なる$\C$の部分集合を対応させる写像であるが,正確には``関数''ではない.出端からアイデンティティが崩壊しているが,
	ちなみに関数と写像の違いは値が数であるか否かである.対数関数は写像ではあるが関数ではなく,
	値の中に対数が無数に存在している.これが理由で対数関数は{\bf 多価関数}\index{たかかんすう@多価関数}
	{\bf (multivalued function)}と呼ばれている.
	値の中から偏角に関する条件によって対数を抜き取れば``関数''となり,その抜き取る操作を{\bf 対数の枝を取る}という.
	
	まずは$0$でない複素数$z$に対して
	\begin{align}
		\Set{w \in \C}{z = \exp{w}} \neq \emptyset
	\end{align}
	であることを示す.ちなみに定理\ref{thm:inversion_of_exp_z_is_exp_minus_z}より
	\begin{align}
		\Set{w \in \C}{0 = \exp{w}} = \emptyset
	\end{align}
	が成り立つ.
	
	\begin{screen}
		\begin{thm}[絶対値が$1$の複素数は$e$の純虚数乗で表せる]
		\label{thm:complex_number_with_absolute_value_1_is_exp_pure_imaginary}
			$z$を複素数とすると,
			\begin{align}
				|z| = 1
			\end{align}
			ならば
			\begin{align}
				z = e^{\isym \cdot y}
			\end{align}
			を満たす実数$y$が取れる.
		\end{thm}
	\end{screen}
	
	\begin{sketch}
		$z$を
		\begin{align}
			|z| = 1
		\end{align}
		を満たす複素数とする.
		\begin{align}
			z = u + \isym \cdot v
		\end{align}
		を満たす実数$u$と$v$を取ると,
		\begin{align}
			u^2 + v^2 = 1
		\end{align}
		であるから
		\begin{align}
			-1 \leq u \leq 1
		\end{align}
		が成立する.ところで
		\begin{align}
			\cos{0} = 1
		\end{align}
		かつ
		\begin{align}
			\cos{\pi} = -1
		\end{align}
		かつ
		\begin{align}
			[0,\pi] \ni t \longmapsto \cos{t}
		\end{align}
		は連続であるから,中間値の定理より
		\begin{align}
			u = \cos{\theta}
		\end{align}
		を満たす実数$\theta$が取れる.
		\begin{align}
			\sin^2{\theta} = 1 - \cos^2{\theta} = 1 - u^2 = v^2
		\end{align}
		が成り立つので
		\begin{align}
			v = \sin{\theta} \vee v = -\sin{\theta}
		\end{align}
		が従うが,
		\begin{align}
			v = \sin{\theta}
		\end{align}
		の場合は
		\begin{align}
			z = \cos{\theta} + \isym \cdot \sin{\theta} = e^{\isym \cdot \theta}
		\end{align}
		が成り立ち,
		\begin{align}
			v = -\sin{\theta}
		\end{align}
		の場合は
		\begin{align}
			z = \cos{\theta} + \isym \cdot (-\sin{\theta})
			= \cos{(-\theta)} + \isym \cdot \sin{(-\theta)}
			= e^{\isym \cdot (-\theta)}
		\end{align}
		が成り立つので,いずれの場合も
		\begin{align}
			\exists y \in \R\, \left(\, z = e^{\isym \cdot y}\, \right)
		\end{align}
		が満たされる.
		\QED
	\end{sketch}
	
	\begin{screen}
		\begin{thm}[$0$でない複素数には対数が存在する]
			$z$を$0$でない複素数とすると,
			\begin{align}
				z = \exp{w}
			\end{align}
			を満たす複素数$w$が取れる.
		\end{thm}
	\end{screen}
	
	\begin{sketch}
		$z$を$0$でない複素数とすると,
		定理\ref{thm:complex_number_with_absolute_value_1_is_exp_pure_imaginary}より
		\begin{align}
			\frac{z}{|z|} = e^{\isym \cdot y}
		\end{align}
		を満たす実数$y$が取れる.また
		定理\ref{thm:real_valued_exponential_function}より
		\begin{align}
			|z| = e^{x}
		\end{align}
		を満たす実数$x$が取れるので
		\begin{align}
			z = |z| \cdot e^{\isym \cdot y} = e^{x + \isym \cdot y}
		\end{align}
		が成立する.
		\QED
	\end{sketch}
	
	\begin{screen}
		\begin{dfn}[対数関数]
			複素数$z$に対して
			\begin{align}
				\Set{w \in \C}{z = \exp{w}}
			\end{align}
			を対応させる$\C$上の写像を{\bf 対数関数}\index{たいすうかんすう@対数関数}{\bf (logarithmic function)}と呼び,
			\begin{align}
				\log
			\end{align}
			と書く.
		\end{dfn}
	\end{screen}
	
	\begin{screen}
		\begin{thm}
			$z^\alpha$は$z \neq 0$で正則で$(z^\alpha)' = \alpha z^{\alpha-1}$
		\end{thm}
	\end{screen}
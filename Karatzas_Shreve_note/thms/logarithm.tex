\subsection{対数関数}
	
	$z$を複素数とするとき,$z$の{\bf 対数}\index{たいすう@対数}{\bf (logarithm)}とは
	\begin{align}
		z = \exp{w}
	\end{align}
	を満たす複素数$w$のことを指すが,$\exp$は周期関数であるからそのような$w$は整数の数だけ,
	つまり可算無限個存在する.対数関数とは指数関数の逆写像にあたるもので,
	複素数$z$に対して対数の全体を対応させる写像である.つまり$z$に対し
	\begin{align}
		\Set{w \in \C}{z = \exp{w}}
	\end{align}
	なる$\C$の部分集合を対応させる写像であるが,正確には``関数''ではない.
	出端から名前と実態が食い違っているが,
	ちなみに関数と写像の違いは値が数であるか否かである.対数関数は写像ではあるが関数ではなく,
	値の中に対数が無数に存在している.これが理由で対数関数は{\bf 多価関数}\index{たかかんすう@多価関数}
	{\bf (multivalued function)}と呼ばれている.
	値の中から偏角に関する条件によって対数を抜き取れば``関数''となり,その抜き取る操作を{\bf 対数の枝を取る}という.
	
	まずは$0$でない複素数$z$に対して
	\begin{align}
		\Set{w \in \C}{z = \exp{w}} \neq \emptyset
	\end{align}
	であることを示す.ちなみに定理\ref{thm:inversion_of_exp_z_is_exp_minus_z}より
	\begin{align}
		\Set{w \in \C}{0 = \exp{w}} = \emptyset
	\end{align}
	が成り立つ.
	
	\begin{screen}
		\begin{thm}[絶対値が$1$の複素数は$e$の純虚数乗で表せる]
		\label{thm:complex_number_with_absolute_value_1_is_exp_pure_imaginary}
			$z$を複素数とすると,
			\begin{align}
				|z| = 1
			\end{align}
			ならば
			\begin{align}
				z = e^{\isym \cdot y}
			\end{align}
			を満たす実数$y$が取れる.
		\end{thm}
	\end{screen}
	
	\begin{sketch}
		$z$を
		\begin{align}
			|z| = 1
		\end{align}
		を満たす複素数とする.
		\begin{align}
			z = u + \isym \cdot v
		\end{align}
		を満たす実数$u$と$v$を取ると,
		\begin{align}
			u^2 + v^2 = 1
		\end{align}
		であるから
		\begin{align}
			-1 \leq u \leq 1
		\end{align}
		が成立する.ところで
		\begin{align}
			\cos{0} = 1
		\end{align}
		かつ
		\begin{align}
			\cos{\pi} = -1
		\end{align}
		かつ
		\begin{align}
			[0,\pi] \ni t \longmapsto \cos{t}
		\end{align}
		は連続であるから,中間値の定理より
		\begin{align}
			u = \cos{\theta}
		\end{align}
		を満たす実数$\theta$が取れる.
		\begin{align}
			\sin^2{\theta} = 1 - \cos^2{\theta} = 1 - u^2 = v^2
		\end{align}
		が成り立つので
		\begin{align}
			v = \sin{\theta} \vee v = -\sin{\theta}
		\end{align}
		が従うが,
		\begin{align}
			v = \sin{\theta}
		\end{align}
		の場合は
		\begin{align}
			z = \cos{\theta} + \isym \cdot \sin{\theta} = e^{\isym \cdot \theta}
		\end{align}
		が成り立ち,
		\begin{align}
			v = -\sin{\theta}
		\end{align}
		の場合は
		\begin{align}
			z = \cos{\theta} + \isym \cdot (-\sin{\theta})
			= \cos{(-\theta)} + \isym \cdot \sin{(-\theta)}
			= e^{\isym \cdot (-\theta)}
		\end{align}
		が成り立つので,いずれの場合も
		\begin{align}
			\exists y \in \R\, \left(\, z = e^{\isym \cdot y}\, \right)
		\end{align}
		が満たされる.
		\QED
	\end{sketch}
	
	\begin{screen}
		\begin{thm}[$0$でない複素数には対数が存在する]
			$z$を$0$でない複素数とすると,
			\begin{align}
				z = \exp{w}
			\end{align}
			を満たす複素数$w$が取れる.
		\end{thm}
	\end{screen}
	
	\begin{sketch}
		$z$を$0$でない複素数とすると,
		定理\ref{thm:complex_number_with_absolute_value_1_is_exp_pure_imaginary}より
		\begin{align}
			\frac{z}{|z|} = e^{\isym \cdot y}
		\end{align}
		を満たす実数$y$が取れる.また
		定理\ref{thm:real_valued_exponential_function}より
		\begin{align}
			|z| = e^{x}
		\end{align}
		を満たす実数$x$が取れるので
		\begin{align}
			z = |z| \cdot e^{\isym \cdot y} = e^{x + \isym \cdot y}
		\end{align}
		が成立する.
		\QED
	\end{sketch}
	
	\begin{screen}
		\begin{dfn}[対数関数]
			複素数$z$に対して
			\begin{align}
				\Set{w \in \C}{z = \exp{w}}
			\end{align}
			を対応させる$\C$上の写像を{\bf 対数関数}\index{たいすうかんすう@対数関数}{\bf (logarithmic function)}と呼び,
			\begin{align}
				\log
			\end{align}
			と書く.
		\end{dfn}
	\end{screen}
	
	例えば$\pvlog$は$-1$で不連続.
	
	\begin{screen}
		\begin{thm}[$\pvlog$の正則性]\label{thm:pv_log_is_holomorphic}
			$\pvlog$は$\C^*$上で正則であって,$z$を$\C^*$の要素とすると
			\begin{align}
				\pvlog'{z} = \frac{1}{z}.
			\end{align}
		\end{thm}
	\end{screen}
	
	\begin{sketch}\mbox{}
		\begin{description}
			\item[第一段]
				$z$を$\C^*$の要素として,$\pvlog$が$z$で連続であることを示す.
				定理\ref{thm:real_valued_exponential_function}より
				\begin{align}
					e^{\theta} = \frac{|z|}{2}
				\end{align}
				を満たす実数$\theta$と
				\begin{align}
					e^{\eta} = \frac{3}{2} \cdot |z|
				\end{align}
				を満たす実数$\eta$が取れる.$\exp$は$\R$上で単調増大であって,かつ
				\begin{align}
					|z| = |e^{\pvlog{z}}| = e^{\Re{\pvlog{z}}}
				\end{align}
				が成り立つので,
				\begin{align}
					\theta < \Re{\pvlog{z}} < \eta
				\end{align}
				が満たされる.また$z$は$\C^*$の要素なので
				\begin{align}
					-\pi < \Im{z} < \pi
				\end{align}
				も満たされる.従って
				\begin{align}
					\forall w \in \C\,
					\left[\, |\pvlog{z} - w| < \epsilon_0
					\Longrightarrow \left(\, \theta < \Re{w} < \eta
					\wedge -\pi < \Im{w} < \pi\, \right)\, \right]
				\end{align}
				を満たす正の実数$\epsilon_0$が取れる.いま$\epsilon$を
				\begin{align}
					\epsilon \leq \epsilon_0
				\end{align}
				を満たす正の実数とする.ここで
				\begin{align}
					A \defeq \Set{w \in \C}{-\pi \leq \Im{w} \leq \pi}
				\end{align}
				及び
				\begin{align}
					B \defeq \Set{w \in \C}{\theta \leq \Re{w} \leq \eta}
				\end{align}
				及び
				\begin{align}
					C \defeq \Set{w \in \C}{\epsilon \leq |w - \pvlog{z}|}
				\end{align}
				とおけば,$A$と$B$と$C$は共に$\C$の閉集合であって,
				\begin{align}
					K \defeq A \cap B \cap C
				\end{align}
				で定める$K$は$\C$のコンパクト部分集合である.
				\begin{align}
					\C \ni w \longmapsto |e^{w} - z|
				\end{align}
				は連続なので,$K$上で最小値を取る.その最小値を$\rho$とおけば
				\begin{align}
					|e^{\xi} - z| = \rho
				\end{align}
				を満たす$K$の要素$\xi$が取れるが,
				\begin{align}
					e^{w} = z
				\end{align}
				を満たす$K$の要素$w$は$\pvlog{z}$に限られるので
				\begin{align}
					0 < |e^{\xi} - z| = \rho
				\end{align}
				が成り立つ.他方で
				\begin{align}
					\Re{w} < \theta
				\end{align}
				を満たす複素数$w$に対しては
				\begin{align}
					\frac{|z|}{2} \leq |z| - |e^{w}| \leq |e^{w} - z|
				\end{align}
				が成り立ち,
				\begin{align}
					\eta < \Re{w}
				\end{align}
				を満たす複素数$w$に対しては
				\begin{align}
					\frac{|z|}{2} \leq |e^{w}| - |z| \leq |e^{w} - z|
				\end{align}
				が成り立つ.ゆえに
				\begin{align}
					\forall w\,
					\left[\, w \in A \cap C \Longrightarrow 
					\min\left\{\rho,|z|/2\right\} \leq |e^{w} - z|\, \right]
					\label{fom:thm_pv_log_is_holomorphic}
				\end{align}
				を得る.ここで
				\begin{align}
					\delta \defeq \min\left\{\rho,|z|/2\right\}
				\end{align}
				とおく.$\zeta$を任意に与えられた複素数とすると,
				\begin{align}
					\epsilon \leq |\pvlog{\zeta} - \pvlog{z}|
				\end{align}
				であるとき
				\begin{align}
					\pvlog{\zeta} \in A \cap C
				\end{align}
				が成り立つので,このとき(\refeq{fom:thm_pv_log_is_holomorphic})より
				\begin{align}
					\delta \leq |e^{\pvlog{\zeta}} - z| = |\zeta - z|
				\end{align}
				が成立する.すなわち
				\begin{align}
					\forall \zeta \in \C\, \left(\,
					\epsilon \leq |\pvlog{\zeta} - \pvlog{z}| 
					\Longrightarrow \delta \leq |\zeta - z|\, \right)
				\end{align}
				が成立する.この対偶を取れば
				\begin{align}
					\forall \zeta \in \C\, \left(\,
					|\zeta - z| < \delta \Longrightarrow |\pvlog{\zeta} - \pvlog{z}| < \epsilon\, \right)
				\end{align}
				が従う.ゆえに$\pvlog$は$z$において連続である.
				
			\item[第二段]
				$z$を$\C^*$の要素として,$\pvlog$が$z$で微分可能であって,その微分係数が
				\begin{align}
					\frac{1}{z}
				\end{align}
				であることを示す.いま$\epsilon$を任意に与えられた正の実数とする.
				\begin{align}
					\zeta \defeq \pvlog{z}
				\end{align}
				とおくと,$\exp$の$\zeta$での微分可能性より
				\begin{align}
					\forall w \in \C\, 
					\left[\, 0 < |w - \zeta| < \delta
					\Longrightarrow \left|\frac{e^{w} - e^{\zeta}}{w-\zeta} - e^{\zeta}\right| < \frac{|z|}{2}\, \right]
				\end{align}
				を満たす正の実数$\delta$が取れる.すなわち,
				\begin{align}
					0 < |w - \zeta| < \delta
				\end{align}
				を満たす複素数$w$に対して
				\begin{align}
					\frac{|z|}{2} = |e^{\zeta}| - \frac{|z|}{2} 
					< \left|\frac{e^{w} - e^{\zeta}}{w- \zeta}\right|
				\end{align}
				が成立する.
		\end{description}
	\end{sketch}
	
	\begin{screen}
		\begin{thm}[冪関数の主値は$\C^*$上で正則]
			$\alpha$を複素数とする.このとき
			\begin{align}
				\C^* \ni z \longmapsto e^{\alpha \cdot \pvlog{z}}
			\end{align}
			なる写像は$\C^*$上で正則であって,$\C^*$の要素$z$における微分係数は
			\begin{align}
				\alpha \cdot e^{(\alpha-1) \cdot \pvlog{z}}
			\end{align}
			である.
		\end{thm}
	\end{screen}
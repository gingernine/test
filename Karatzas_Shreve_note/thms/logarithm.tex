\subsection{対数関数}
	
	$z$を複素数とするとき,$z$の{\bf 対数}\index{たいすう@対数}{\bf (logarithm)}とは
	\begin{align}
		z = \exp{(w)}
	\end{align}
	を満たす複素数$w$のことを指すが,$\exp$は周期関数であるからそのような$w$は無限個存在する.
	対数関数とは指数関数の逆写像にあたるもので,複素数$z$に対して対数の全体を対応させる写像である.つまり$z$に対し
	\begin{align}
		\Set{w \in \C}{z = \exp{(w)}}
	\end{align}
	なる$\C$の部分集合を対応させる写像であるが,正確には``関数''ではない.出端からアイデンティティが崩壊しているが,
	ちなみに関数と写像の違いは値が数であるか否かである.対数関数は写像ではあるが関数ではなく,
	値の中に対数が無数に存在している.これが理由で対数関数は{\bf 多価関数}\index{たかかんすう@多価関数}
	{\bf (multivalued function)}と呼ばれている.
	値の中から偏角に関する条件によって対数を抜き取れば``関数''となり,その抜き取る操作を{\bf 対数の枝を取る}という.
	
	まずは$0$でない複素数$z$に対して
	\begin{align}
		\Set{w \in \C}{z = \exp{(w)}} \neq \emptyset
	\end{align}
	であることを示す.
	
	\begin{screen}
		\begin{thm}[$0$でない複素数には対数が存在する]
			$z$を$0$でない複素数とすると,
			\begin{align}
				z = \exp{(w)}
			\end{align}
			を満たす複素数$w$が取れる.
		\end{thm}
	\end{screen}
	
	\begin{sketch}
	\end{sketch}
	
	\begin{screen}
		\begin{dfn}[対数関数]
			複素数$z$に対して
			\begin{align}
				\Set{w \in \C}{z = \exp{(w)}}
			\end{align}
			を対応させる$\C$上の写像を{\bf 対数関数}\index{たいすうかんすう@対数関数}{\bf (logarithmic function)}と呼び,
			\begin{align}
				\log
			\end{align}
			と書く.
		\end{dfn}
	\end{screen}
	
	\begin{screen}
		\begin{thm}
			$z^\alpha$は$z \neq 0$で正則で$(z^\alpha)' = \alpha z^{\alpha-1}$
		\end{thm}
	\end{screen}
	
	$e^{-z} = (e^z)^{-1}$を示せ.
	
	$e^{kz} = (e^z)^k$を示せ.
	
	$e^{\frac{2\pi}{3}\isym}$と$e^{\frac{4\pi}{3}\isym}$が$z^3 = 1$の根であることを確かめよ.
	
	
	
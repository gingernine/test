\section{Weak Convergence}
	\begin{itembox}[l]{Definition 4.3}
		It follows, in particular, that the weak limit $P$ is a probability measure, 
		and that it is unique.
	\end{itembox}
	
	\begin{prf}
		$f \equiv 1$として
		\begin{align}
			P(S) = \lim_{n \to \infty} P_n(S) = 1
		\end{align}
		が従うから$P$は確率測度である.また任意の有界連続関数$f:S \longrightarrow \R$に対し
		\begin{align}
			\int_S f\ dP = \int_S f\ dQ
		\end{align}
		が成り立つとき,任意の閉集合$A \subset S$に対して
		\begin{align}
			f_k(s) \coloneqq \frac{1}{1 + k d(s,A)},
			\quad (k=1,2,\cdots)
		\end{align}
		と定めれば$\lim_{k \to \infty} f_k = \defunc_A$(各点収束)が満たされるから,Lebesgueの収束定理より
		\begin{align}
			P(A) = \lim_{k \to \infty} \int_S f_k\ dP
			= \lim_{k \to \infty} \int_S f_k\ dQ
			= Q(A)
		\end{align}
		となり,測度の一致の定理より$P = Q$が得られる.すなわち弱極限は一意である.
		\QED
	\end{prf}
	
	\begin{itembox}[l]{lemma: change of variables for expectation}
		$(\Omega,\mathscr{F},P)$を確率空間,
		$(S,\mathscr{S})$を可測空間とする.
		このとき任意の
		有界$\mathscr{S}/\borel{\R}$-可測関数$f$
		と$\mathscr{F}/\mathscr{S}$-可測写像$X$に対して
		\begin{align}
			\int_\Omega f(X)\ dP = \int_S f\ dPX^{-1}
		\end{align}
		が成立する.
	\end{itembox}
	
	\begin{prf}
		任意の$A \in \mathscr{S}$に対して
		\begin{align}
			\int_S \defunc_A dPX^{-1}
			= P(X^{-1}(A))
			= \int_\Omega \defunc_{X^{-1}(A)}\ dP
			= \int_\Omega \defunc_{A}(X)\ dP
		\end{align}
		が成り立つから,任意の$\mathscr{S}/\borel{\R}$-可測単関数$g$に対し
		\begin{align}
			\int_\Omega g(X)\ dP = \int_S g\ dPX^{-1}
		\end{align}
		となる.$f$が有界なら一様有界な単関数で近似できるので,Lebesgueの収束定理より
		\begin{align}
			\int_\Omega f(X)\ dP = \int_S f\ dPX^{-1}
		\end{align}
		が出る.
		\QED
	\end{prf}
	
	\begin{itembox}[l]{Definition 4.4}
		Equivalently, $X_n \overset{\mathscr{D}}{\longrightarrow} X$ if and only if
		\begin{align}
			\lim_{n \to \infty} E_n f(X_n) = E f(X)
		\end{align} 
		for every bounded, continuous real-valued function $f$ on $S$, 
		where $E_n$ and $E$ denote expectations with respect to $P_n$ and $P$, respectively.
	\end{itembox}
	
	\begin{prf}
		任意の有界実連続関数$f:S \longrightarrow \R$に対し
		\begin{align}
			\int_\Omega f(X_n)\ dP_n = \int_S f\ dP_nX_n^{-1},
			\quad \int_\Omega f(X)\ dP = \int_S f\ dPX^{-1},
		\end{align}
		が成り立つから,$P_nX_n^{-1}$が$PX^{-1}$に弱収束することと
		$\lim_{n \to \infty} E_n f(X_n) = E f(X)$は同値である.
		\QED
	\end{prf}
	
	\begin{itembox}[l]{Problem 4.5}
		Suppose $\{X_n\}_{n=1}^\infty$ is a sequence of random variables taking values 
		in a metric space $(S_1,\rho_1)$ and converging in distribution to $X$. Suppose 
		$(S_2,\rho_2)$ is another metric space, and $\varphi:S_1 \longrightarrow S_2$ 
		is continuous. Show that $Y_n \coloneqq \varphi(X_n)$ converges in distribution 
		to $Y \coloneqq \varphi(X)$.
	\end{itembox}
	
	\begin{prf}
		任意の有界実連続関数$f:S_2 \longrightarrow \R$に対し
		$f \circ \varphi$は$S_1$上の有界実連続関数であるから
		\begin{align}
			\int_{S_2} f\ dPY_n^{-1}
			&= \int_{\Omega} f(Y_n)\ dP
			= \int_{\Omega} f(\varphi(X_n))\ dP
			= \int_{S_1} f \circ \varphi\ dPX_n^{-1} \\
			& \longrightarrow 
			\int_{S_1} f \circ \varphi\ dPX^{-1}
			= \int_{S_2} f\ dPY^{-1}
			\quad (n \longrightarrow \infty)
		\end{align}
		が成立する.
		\QED
	\end{prf}
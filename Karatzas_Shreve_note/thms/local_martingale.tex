\section{局所マルチンゲール}
	
	$\{\mathscr{F}_{t}\}_{t \in [0,1]}$-停止時刻の全体を
	\begin{align}
		\mathscr{T}
	\end{align}
	とおく.また$\Natural$から$\mathscr{T}$への写像$\tau$で,
	\begin{itemize}
		\item $\Omega$の任意の要素$\omega$に対して
			\begin{align}
				\tau_{0}(\omega) = 0,
			\end{align}
			
		\item 任意の自然数$n$及び$\Omega$の任意の要素$\omega$に対して
			\begin{align}
				\tau_{n}(\omega) \leq \tau_{n+1}(\omega),
			\end{align}
			
		\item $\omega$を$\Omega$の任意の要素とするとき
			\begin{align}
				\tau_{n}(\omega) = 1
			\end{align}
			を満たす自然数$n$が取れる.
	\end{itemize}
	
	を満たすものを,便宜上$\{\mathscr{F}_{t}\}_{t \in [0,1]}$-{\bf 増大停止時刻列}\index{ぞうだいていしじこくれつ@増大停止時刻列}と呼ぶことにする.
	
	\begin{screen}
		\begin{dfn}[局所マルチンゲール]
			$X$を$(\Omega,\mathscr{F},P)$上の確率過程とするとき,$\{\mathscr{F}_{t}\}_{t \in [0,1]}$-増大停止時刻列$\tau$が取れて,
			任意の自然数$n$に対して
			\begin{align}
				X^{\tau_{n}}
			\end{align}
			が$\{\mathscr{F}_{t}\}_{t \in [0,1]}$-マルチンゲールとなるならば,言い換えれば
			$X$を局所的にマルチンゲール化する$\{\mathscr{F}_{t}\}_{t \in [0,1]}$-増大停止時刻列が取れるならば,
			$X$を$\{\mathscr{F}_{t}\}_{t \in [0,1]}$-{\bf 局所マルチンゲール}\index{きょくしょまるちんげーる@局所マルチンゲール}
			{\bf (local martingale)}と呼ぶ.$(\Omega,\mathscr{F},P)$上の連続な
			$\{\mathscr{F}_{t}\}_{t \in [0,1]}$-局所マルチンゲール$X$で,$\Omega$の任意の要素$\omega$に対して
			\begin{align}
				X_{0}(\omega) = 0
			\end{align}
			を満たすものの全体を
			\begin{align}
				\mathscr{M}_{c,loc}
			\end{align}
			とおく.
		\end{dfn}
	\end{screen}
	
	\begin{screen}
		\begin{thm}[マルチンゲールは局所マルチンゲール]
			\begin{align}
				\mathscr{M}_{\mathbf{T}} \subset \mathscr{M}^{loc}_{\mathbf{T}}.
			\end{align}
		\end{thm}
	\end{screen}
	
	\begin{sketch}
		
	\end{sketch}
	
	いま$X$を有界な$\{\mathscr{F}_{t}\}_{t \in [0,1]}$-局所マルチンゲールとし,
	$\tau$を$X$を局所的にマルチンゲール化する$\{\mathscr{F}_{t}\}_{t \in [0,1]}$-増大停止時刻列とする.
	$s$と$t$を
	\begin{align}
		s < t
	\end{align}
	なる$[0,1]$の要素とし,$A$を$\mathscr{F}_{s}$の要素とすれば,任意の自然数$n$で
	\begin{align}
		\int_{A} X^{\tau_{n}}_{t}\ dP = \int_{A} X^{\tau_{n}}_{s}\ dP
	\end{align}
	が成り立つ.ここで$\Omega$の任意の要素$\omega$に対し
	\begin{align}
		\lim_{n \to \infty} X^{\tau_{n}}_{t}(\omega) = X_{t}(\omega)
	\end{align}
	及び
	\begin{align}
		\lim_{n \to \infty} X^{\tau_{n}}_{s}(\omega) = X_{s}(\omega)
	\end{align}
	が成り立つので,Lebesgueの収束定理より
	\begin{align}
		\int_{A} X_{t}\ dP = \int_{A} X_{s}\ dP
	\end{align}
	が従う.つまり{\bf 有界な$\{\mathscr{F}_{t}\}_{t \in [0,1]}$-局所マルチンゲールは
	$\{\mathscr{F}_{t}\}_{t \in [0,1]}$-マルチンゲールである.}
	
	\begin{screen}
		\begin{thm}[$\mathscr{M}_{c,loc}$の要素は局所的に有界マルチンゲール化できる]
			$X$を$\mathscr{M}_{c,loc}$の要素とするとき,$\{\mathscr{F}_{t}\}_{t \in [0,1]}$-増大停止時刻列$\tau$で,
			任意の自然数$n$に対して
			\begin{align}
				X^{\tau_{n}}
			\end{align}
			が有界な連続$\{\mathscr{F}_{t}\}_{t \in [0,1]}$-マルチンゲールとなるものが取れる.
		\end{thm}
	\end{screen}
	
	\begin{sketch}
		自然数$n$に対して
		\begin{align}
			\Omega \ni \omega \longmapsto
			\begin{cases}
				\inf{}{\Set{t \in [0,1]}{n \leq |X_{t}(\omega)|}} 
				& \mbox{if } \Set{t \in [0,1]}{n \leq |X_{t}(\omega)|} \neq \emptyset \\
				1 & \mbox{if } \Set{t \in [0,1]}{n \leq |X_{t}(\omega)|} = \emptyset
			\end{cases}
		\end{align}
		なる写像を対応させる写像を$\sigma$とおくと,定理\ref{thm:increasing_stopping_times_made_from_continuous_martingales}より
		任意の自然数$n$に対して$\sigma_{n}$は$\{\mathscr{F}_{t}\}_{t \in [0,1]}$-停止時刻であり,また
		$[0,1]$の任意の要素$t$及び$\Omega$の任意の要素$\omega$に対し
		\begin{align}
			\left| X^{\sigma_{n}}_{t}(\omega) \right| \leq n
		\end{align}
		が成立する.$\upsilon$を$X$を局所的にマルチンゲール化する$\{\mathscr{F}_{t}\}_{t \in [0,1]}$-増大停止時刻列として,
		自然数$n$に対して
		\begin{align}
			\Omega \ni \omega \longmapsto \min\left\{ \sigma_{n}(\omega),\upsilon_{n}(\omega) \right\}
		\end{align}
		なる写像を対応させる写像を$\tau$とすれば,$\tau$は定理の主張を満たす$\{\mathscr{F}_{t}\}_{t \in [0,1]}$-増大停止時刻列である.
		\QED
	\end{sketch}
	
	\begin{screen}
		\begin{dfn}[総変動過程]
			$X$を$[0,1] \times \Omega$上の$\R$値$\mathscr{F}$-過程とし,
			$\Omega$の任意の要素$\omega$に対する
			標本路$X_{\bullet}(\omega)$は$[0,1]$上で有界変動であるとする.このとき
			$[0,1] \times \Omega$の各要素$(t,\omega)$に対し
			\begin{align}
				\sup{}{
					\Set{\sum_{i=0}^{n-1}\left|X_{\tau_{i+1}}(\omega) - X_{\tau_{i}}(\omega)\right|}{
						n \in \Natural \wedge \tau:n+1 \longrightarrow [0,t] \wedge
						\tau_{0} = 0 \wedge \tau_{n} = t \wedge 
						\forall i \in n\, (\, \tau_{i} \leq \tau_{i+1}\, )
					}
				}
			\end{align}
			を対応させる写像を$X$の{\bf 総変動過程}\index{そうへんどうかてい@総変動過程}と呼ぶ.
		\end{dfn}
	\end{screen}
	
	\begin{screen}
		\begin{thm}[適合過程の総変動過程は適合]
			$X$を$[0,1] \times \Omega$上の連続な$\R$値$\{\mathscr{F}_{t}\}_{t \in [0,1]}$-適合過程とし,
			$|X|$を$X$の総変動過程とする.このとき$|X|$は$[0,1] \times \Omega$上の連続な$\{\mathscr{F}_{t}\}_{t \in [0,1]}$-適合過程である.
		\end{thm}
	\end{screen}
	
	\begin{sketch}
	\end{sketch}
	
	\begin{screen}
		\begin{thm}[有界変動な局所マルチンゲールは定数]
			$X$を$\mathscr{M}_{c,loc}$の要素とし,$\Omega$の任意の要素$\omega$に対する
			標本路$X_{\bullet}(\omega)$は$[0,1]$上で有界変動であるとする.
			このとき$P$-零集合$F$が取れて,$[0,1]$の任意の要素$t$及び
			$\Omega \backslash E$の任意の要素$\omega$に対して
			\begin{align}
				X_{t}(\omega) = 0.
			\end{align}
		\end{thm}
	\end{screen}
	
	\begin{sketch}\mbox{}
		\begin{description}
			\item[第一段]
				$X$の総変動過程$|X|$が有界であるとする.つまり,
				$[0,1]$の任意の要素$t$及び$\Omega$の任意の要素$\omega$に対して
				\begin{align}
					|X|_{t}(\omega) \leq b
				\end{align}
				を満たす実数$b$が取れる.このとき$X$も有界であるから,
				\begin{align}
					X \in \mathscr{M}_{c}^{2}
				\end{align}
				が成り立つ.特に
				\begin{align}
					E\left(X_{1}^{2}\right)
					&= E\left[\sum_{k=0}^{n-1}\left(X_{\frac{k+1}{n}}^{2} - X_{\frac{k}{n}}^{2}\right)\right] \\
					&= E\left[\sum_{k=0}^{n-1}\left(X_{\frac{k+1}{n}} - X_{\frac{k}{n}}\right)^{2}\right] \\
					&\leq E\left[|X|_{1} \cdot \sup{k \in n}{\left|X_{\frac{k+1}{n}} - X_{\frac{k}{n}}\right|}\right] \\
					&\leq b \cdot E\left[\sup{k \in n}{\left|X_{\frac{k+1}{n}} - X_{\frac{k}{n}}\right|}\right]
				\end{align}
				が成り立ち,Lebesgueの収束定理より
				\begin{align}
					E\left[\sup{k \in n}{\left|X_{\frac{k+1}{n}} - X_{\frac{k}{n}}\right|}\right] \longrightarrow 0 \quad (n \longrightarrow \infty)
				\end{align}
				となるから
				\begin{align}
					E\left(X_{1}^{2}\right) = 0
				\end{align}
				を得る.そして$X^{2}$は$\{\mathscr{F}_{t}\}_{t \in [0,1]}$-劣マルチンゲール性であるから,$[0,1]$の任意の要素$t$で
				\begin{align}
					E\left(X_{t}^{2}\right) = 0
				\end{align}
				が成立する.
				
			\item[第二段] $\{\mathscr{F}_{t}\}_{t \in [0,1]}$-増大停止時刻列$\sigma$で,
				任意の自然数$n$及び$\Omega$の任意の要素$\omega$に対して
				\begin{align}
					\sigma_{n}(\omega) = 
					\begin{cases}
						\inf{}{\Set{t \in [0,1]}{n \leq |X|_{t}(\omega)}} 
						& \mbox{if } \Set{t \in [0,1]}{n \leq |X|_{t}(\omega)} \neq \emptyset \\
						1 & \mbox{if } \Set{t \in [0,1]}{n \leq |X|_{t}(\omega)} = \emptyset
					\end{cases}
				\end{align}
				を満たすものを取る.このとき任意の自然数$n$,$[0,1]$の任意の要素$t$
				及び$\Omega$の任意の要素$\omega$に対して
				\begin{align}
					\left|X^{\sigma_{n}}\right|_{t}(\omega) = 
					\begin{cases}
						\left|X\right|_{t}(\omega) &\mbox{if } t \leq \sigma_{n}(\omega) \\
						\left|X\right|_{\sigma_{n}(\omega)}(\omega) &\mbox{if } \sigma_{n}(\omega) < t
					\end{cases}
				\end{align}
				が成り立つので,つまり
				\begin{align}
					\left|X^{\sigma_{n}}\right| = |X|^{\sigma_{n}}
				\end{align}
				である.ゆえに前段の結果より,任意の自然数$n$及び
				$[0,1]$の任意の要素$t$に対して
				\begin{align}
					E\left({X^{\sigma_{n}}_{t}}^{2}\right) = 0
				\end{align}
				が成立する.ここで
				\begin{align}
					F \defeq \bigcup_{n \in \Natural} \bigcup_{t \in [0,1]}
					\Set{\omega \in \Omega}{X^{\sigma_{n}}_{t}(\omega) \neq 0}
				\end{align}
				とおくと,$\omega$を$\Omega \backslash F$の任意の要素とし,
				$t$を$[0,1]$の任意の要素とすれば,任意の自然数$n$に対して
				\begin{align}
					X^{\sigma_{n}}_{t}(\omega) = 0
				\end{align}
				が成り立つので
				\begin{align}
					X_{t}(\omega) = 0
				\end{align}
				が従う.
				\QED
		\end{description}
	\end{sketch}
	
	\begin{screen}
	\end{screen}
	
	\begin{screen}
		\begin{thm}[二次関数が非負であるための条件]
		\label{thm:nonnegative_condition_of_quadratic_function}
			$a,b,c$を実数とし,$a$は非負であるとする.このとき$\R$の任意の要素$x$で
			\begin{align}
				0 \leq a \cdot x^{2} + b \cdot x + c
				\label{fom:nonnegative_condition_of_quadratic_function}
			\end{align}
			が成り立つならば
			\begin{align}
				b^{2} \leq 4 \cdot a \cdot c.
			\end{align}
		\end{thm}
	\end{screen}
	
	\begin{sketch}
		いま(\refeq{fom:nonnegative_condition_of_quadratic_function})が満たされているとする.
		\begin{align}
			a=0
		\end{align}
		のとき,
		\begin{align}
			b \neq 0
		\end{align}
		ならば
		\begin{align}
			b \cdot -\frac{c+1}{b} + c = -1
		\end{align}
		が成り立つので
		\begin{align}
			\exists x \in \R\, (\, b \cdot x + c < 0\, )
		\end{align}
		が従う.ゆえに$a=0$のときは
		\begin{align}
			b = 0
		\end{align}
		及び
		\begin{align}
			b^{2} \leq 4 \cdot a \cdot c
		\end{align}
		が満たされる.
		\begin{align}
			0 < a
		\end{align}
		であるとすると,
		\begin{align}
			a \cdot x^{2} + b \cdot x + c
			= a \cdot \left(x+\frac{b}{2 \cdot a}\right)^{2} - \frac{b^{2} - 4 \cdot a \cdot c}{4 \cdot a}
		\end{align}
		から
		\begin{align}
			\frac{b^{2} - 4 \cdot a \cdot c}{4 \cdot a}
			\leq a \cdot \left(x+\frac{b}{2 \cdot a}\right)^{2}
		\end{align}
		が従う.特に
		\begin{align}
			x = -\frac{b}{2 \cdot a}
		\end{align}
		のとき
		\begin{align}
			\frac{b^{2} - 4 \cdot a \cdot c}{4 \cdot a} \leq 0
		\end{align}
		が成り立つ.すなわちこのとき
		\begin{align}
			b^{2} \leq 4 \cdot a \cdot c
		\end{align}
		が成り立つ.
		\QED
	\end{sketch}
	
	\begin{screen}
		\begin{thm}[二次変分に対するSchwartzの不等式]
		\label{thm:Schwartz_inequality_for_quadratic_variations}
			$X$と$Y$を$\mathscr{M}_{c,loc}$の要素とするとき,
			$P$-零集合$F$が取れて,$\Omega \backslash F$の任意の要素$\omega$と
			$s \leq t$なる$[0,1]$の任意の要素$s$と$t$に対して
			\begin{align}
				\left[\inprod<X,Y>_{t}(\omega) - \inprod<X,Y>_{s}(\omega)\right]^{2}
				\leq \left[\inprod<X>_{t}(\omega) - \inprod<X>_{s}(\omega)\right] 
				\cdot \left[\inprod<Y>_{t}(\omega) - \inprod<Y>_{s}(\omega)\right].
			\end{align}
		\end{thm}
	\end{screen}
	
	\begin{sketch}
		$\R$上の写像$E$を
		\begin{align}
			x \longmapsto \Set{\omega \in \Omega}{\exists t \in [0,1]\, \left[\, 
			\inprod<X+x \cdot Y>_{t}(\omega) \neq
			\inprod<X>_{t}(\omega) + 2 \cdot x \cdot \inprod<X,Y>_{t}(\omega)
			+ x^{2} \cdot \inprod<Y>_{t}(\omega)\, \right]}
		\end{align}
		なる関係で定めて
		\begin{align}
			F \defeq \bigcup_{x \in \Q} E_{x}
		\end{align}
		とおく.このとき$\omega$を$\Omega \backslash F$の任意の要素とし,
		$s$と$t$を
		\begin{align}
			s \leq t
		\end{align}
		なる$[0,1]$の任意の要素とすれば,任意の有理数$x$に対して
		\begin{align}
			\inprod<X>_{s}(\omega) + 2 \cdot x \cdot \inprod<X,Y>_{s}(\omega)
			+ x^{2} \cdot \inprod<Y>_{s}(\omega)
			&= \inprod<X+x \cdot Y>_{s}(\omega) \\
			&\leq \inprod<X+x \cdot Y>_{t}(\omega) \\
			&= \inprod<X>_{t}(\omega) + 2 \cdot x \cdot \inprod<X,Y>_{t}(\omega)
			+ x^{2} \cdot \inprod<Y>_{t}(\omega)
		\end{align}
		が成り立つ.ゆえに任意の実数$x$に対して
		\begin{align}
			0 &\leq x^{2} \cdot \left[\inprod<Y>_{t}(\omega) - \inprod<Y>_{s}(\omega) \right] \\
			&\quad + 2 \cdot x \cdot \left[\inprod<X,Y>_{t}(\omega) - \inprod<X,Y>_{s}(\omega) \right] \\
			&\quad + \left[\inprod<X>_{t}(\omega) - \inprod<X>_{s}(\omega) \right]
		\end{align}
		が成り立つ.ゆえに定理\ref{thm:nonnegative_condition_of_quadratic_function}より
		\begin{align}
			\left[\inprod<X,Y>_{t}(\omega) - \inprod<X,Y>_{s}(\omega)\right]^{2}
			\leq \left[\inprod<X>_{t}(\omega) - \inprod<X>_{s}(\omega)\right] \cdot
			\left[\inprod<Y>_{t}(\omega) - \inprod<Y>_{s}(\omega)\right]
		\end{align}
		が成り立つ.
		\QED
	\end{sketch}
	
\subsection{局所マルチンゲール}
	
	$\{\mathscr{F}_{t}\}_{t \in \mathbf{T}}$-停止時刻の全体を
	\begin{align}
		\mathscr{T}_{\mathbf{T}}
	\end{align}
	とおく.また$\Natural$から$\mathscr{T}_{\mathbf{T}}$への写像$\tau$で,
	\begin{itemize}
		\item $\Omega$の任意の要素$\omega$に対して
			\begin{align}
				\tau_{0}(\omega) = 0,
			\end{align}
			
		\item 任意の自然数$n$及び$\Omega$の任意の要素$\omega$に対して
			\begin{align}
				\tau_{n}(\omega) \leq \tau_{n+1}(\omega),
			\end{align}
			
		\item $\mathbf{T} = [0,T]$ならば,$\omega$を$\Omega$の任意の要素とするとき
			\begin{align}
				\tau_{n}(\omega) = T
			\end{align}
			を満たす自然数$n$が取れる.
	\end{itemize}
	
	を満たすものを,便宜上$\{\mathscr{F}_{t}\}_{t \in \mathbf{T}}$-{\bf 増大停止時刻列}
	\index{ぞうだいていしじこくれつ@増大停止時刻列}と呼ぶことにする.
	
	\begin{screen}
		\begin{thm}[連続適合過程で作る増大停止時刻列]
		\label{thm:increasing_stopping_times_made_from_continuous_adapted_process}
			$X$を$[0,1] \times \Omega$上の連続な$\{\mathscr{F}_{t}\}_{t \in [0,1]}$-
			適合過程とする.自然数$n$に対して
			\begin{align}
				\Omega \ni \omega \longmapsto 
				\begin{cases}
					\inf{}{\Set{t \in [0,1]}{n \leq |X_t(\omega)|}} & \mbox{if } \Set{t \in [0,1]}{n \leq |X_t(\omega)|} \neq \emptyset \\
					\infty & \mbox{if } \Set{t \in [0,1]}{n \leq |X_t(\omega)|} = \emptyset
				\end{cases}
			\end{align}
			なる写像を対応させる$\Natural$上の写像を$\tau$とすると,$\tau$は
			$\{\mathscr{F}_t\}_{t \in [0,1]}$-増大停止時刻列である.
			また$\Omega$のすべての要素$\omega$及び$[0,1]$の任意の要素$t$に対して
			\begin{align}
				\left|X^{\tau_n}_t(\omega)\right| \leq n
			\end{align}
			が成り立つ.
		\end{thm}
	\end{screen}
	
	\begin{sketch}
		$[0,1]$の任意の要素$t$に対して
		\begin{align}
			\left\{\tau_n \leq t\right\} = \Set{\omega \in \Omega}{n \leq |X_t(\omega)|}
		\end{align}
		が成り立つので$\tau_n$は$\{\mathscr{F}_t\}_{t \in \mathbf{T}}$-停止時刻である.
		また$RLCC$なパスは有界区間上で有界であるから,パスが$RCLL$である$\omega$に対しては
		\begin{align}
			\sup{n \in \Natural}{\tau_n(\omega)} = \infty
		\end{align}
		が成り立つ.
		\QED
	\end{sketch}
	
	右連続な劣マルチンゲールは殆ど全てのパスが$RCLL$なので,
	上の様に構成する停止時刻の列$\left\{\tau_n\right\}_{n \in \Natural}$は殆どすべての$\omega$に対し
	\begin{align}
		0 = \tau_0(\omega) \leq \tau_1(\omega) \leq \tau_2(\omega) \leq \longrightarrow \infty
	\end{align}
	を満たす.
	
	\begin{screen}
		\begin{dfn}[局所マルチンゲール]
			$X$を$\mathbf{T} \times \Omega$上の$\R$値$\mathscr{F}$-過程とするとき,
			$\{\mathscr{F}_{t}\}_{t \in \mathbf{T}}$-増大停止時刻列$\tau$が取れて,
			任意の自然数$n$に対して
			\begin{align}
				X^{\tau_{n}}
			\end{align}
			が$\{\mathscr{F}_{t}\}_{t \in \mathbf{T}}$-マルチンゲールとなるならば,言い換えれば
			$X$を局所的にマルチンゲール化する$\{\mathscr{F}_{t}\}_{t \in \mathbf{T}}$-増大停止時刻列が取れるならば,
			$X$を$\{\mathscr{F}_{t}\}_{t \in \mathbf{T}}$-{\bf 局所マルチンゲール}\index{きょくしょまるちんげーる@局所マルチンゲール}
			{\bf (local martingale)}と呼ぶ.
		\end{dfn}
	\end{screen}
	
	$\{\mathscr{F}_{t}\}_{t \in [0,1]}$-局所連続マルチンゲール$X$で,$\Omega$の任意の要素$\omega$に対して
	\begin{align}
		X_{0}(\omega) = 0
	\end{align}
	を満たすものの全体を
	\begin{align}
		\mathscr{M}_{c,loc}
	\end{align}
	とおく.
	
	\begin{screen}
		\begin{thm}[有界な局所マルチンゲールはマルチンゲール]
		\label{thm:bounded_local_martingale_is_martingale}
			$X$を$\{\mathscr{F}_{t}\}_{t \in [0,1]}$-局所マルチンゲールとするとき,$X$が有界であるならば
			$X$は$\{\mathscr{F}_{t}\}_{t \in [0,1]}$-マルチンゲールである.
		\end{thm}
	\end{screen}
	
	\begin{sketch}
		$\tau$を$X$を局所的にマルチンゲール化する$\{\mathscr{F}_{t}\}_{t \in [0,1]}$-増大停止時刻列とする.
		$s$と$t$を
		\begin{align}
			s < t
		\end{align}
		なる$[0,1]$の要素とし,$A$を$\mathscr{F}_{s}$の要素とすれば,任意の自然数$n$で
		\begin{align}
			\int_{A} X^{\tau_{n}}_{t}\ dP = \int_{A} X^{\tau_{n}}_{s}\ dP
		\end{align}
		が成り立つ.ここで$\Omega$の任意の要素$\omega$に対し
		\begin{align}
			\lim_{n \to \infty} X^{\tau_{n}}_{t}(\omega) = X_{t}(\omega)
		\end{align}
		及び
		\begin{align}
			\lim_{n \to \infty} X^{\tau_{n}}_{s}(\omega) = X_{s}(\omega)
		\end{align}
		が成り立つので,Lebesgueの収束定理より
		\begin{align}
			\int_{A} X_{t}\ dP = \int_{A} X_{s}\ dP
		\end{align}
		が従う.
		\QED
	\end{sketch}
	
	\begin{screen}
		\begin{thm}[$\mathscr{M}_{c,loc}$の要素は局所的に有界マルチンゲール化できる]
		\label{thm:increasing_stopping_times_which_locally_bound_martingale}
			$X$を$\mathscr{M}_{c,loc}$の要素とするとき,$\{\mathscr{F}_{t}\}_{t \in [0,1]}$-増大停止時刻列$\tau$で,
			任意の自然数$n$に対して
			\begin{align}
				X^{\tau_{n}}
			\end{align}
			が有界な連続$\{\mathscr{F}_{t}\}_{t \in [0,1]}$-マルチンゲールとなるものが取れる.
		\end{thm}
	\end{screen}
	
	\begin{sketch}
		自然数$n$に対して
		\begin{align}
			\Omega \ni \omega \longmapsto
			\begin{cases}
				\inf{}{\Set{t \in [0,1]}{n \leq |X_{t}(\omega)|}} 
				& \mbox{if } \Set{t \in [0,1]}{n \leq |X_{t}(\omega)|} \neq \emptyset \\
				1 & \mbox{if } \Set{t \in [0,1]}{n \leq |X_{t}(\omega)|} = \emptyset
			\end{cases}
		\end{align}
		なる写像を対応させる写像を$\sigma$とおくと,定理\ref{thm:increasing_stopping_times_made_from_continuous_adapted_process}より
		任意の自然数$n$に対して$\sigma_{n}$は$\{\mathscr{F}_{t}\}_{t \in [0,1]}$-停止時刻であり,また
		$[0,1]$の任意の要素$t$及び$\Omega$の任意の要素$\omega$に対し
		\begin{align}
			\left| X^{\sigma_{n}}_{t}(\omega) \right| \leq n
		\end{align}
		が成立する.$\upsilon$を$X$を局所的にマルチンゲール化する$\{\mathscr{F}_{t}\}_{t \in [0,1]}$-増大停止時刻列として,
		自然数$n$に対して
		\begin{align}
			\Omega \ni \omega \longmapsto \min\left\{ \sigma_{n}(\omega),\upsilon_{n}(\omega) \right\}
		\end{align}
		なる写像を対応させる写像を$\tau$とすれば,$\tau$は定理の主張を満たす$\{\mathscr{F}_{t}\}_{t \in [0,1]}$-増大停止時刻列である.
		\QED
	\end{sketch}
	
	\begin{screen}
		\begin{dfn}[総変動過程]
			$X$を$\mathbf{T} \times \Omega$上の$\R$値$\mathscr{F}$-過程とし,
			$\mathbf{T}$の任意の要素$t$に対して,$\Omega$の任意の要素$\omega$の標本路$X_{\bullet}(\omega)$は
			$[0,t]$上で有界変動であるとする.このとき$\mathbf{T} \times \Omega$の各要素$(t,\omega)$に対し
			\begin{align}
				\sup{}{
					\Set{\sum_{i=0}^{n-1}\left|X_{\tau_{i+1}}(\omega) - X_{\tau_{i}}(\omega)\right|}{
						n \in \Natural \wedge \tau:n+1 \longrightarrow [0,t] \wedge
						\tau_{0} = 0 \wedge \tau_{n} = t \wedge 
						\forall i \in n\, (\, \tau_{i} \leq \tau_{i+1}\, )
					}
				}
			\end{align}
			を対応させる$\mathbf{T} \times \Omega$上の写像を$X$の{\bf 総変動過程}\index{そうへんどうかてい@総変動過程}と呼ぶことにする.
		\end{dfn}
	\end{screen}
	
	\begin{screen}
		\begin{thm}[適合過程の総変動過程は適合]
			$X$を$[0,1] \times \Omega$上の連続な$\R$値$\{\mathscr{F}_{t}\}_{t \in [0,1]}$-適合過程とし,
			$|X|$を$X$の総変動過程とする.このとき$|X|$は$[0,1] \times \Omega$上の連続な$\{\mathscr{F}_{t}\}_{t \in [0,1]}$-適合過程である.
		\end{thm}
	\end{screen}
	
	\begin{sketch}
	
	\end{sketch}
	
	\begin{screen}
		\begin{thm}[マルチンゲールは局所マルチンゲール]
			$X$を$[0,1] \times \Omega$上の$\{\mathscr{F}_{t}\}_{t \in [0,1]}$-連続マルチンゲールとするとき
			\begin{align}
				X \in \mathscr{M}_{c,loc}.
			\end{align}
		\end{thm}
	\end{screen}
	
	\begin{sketch}
		
	\end{sketch}
	
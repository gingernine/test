\section{局所マルチンゲール}
	
	$\{\mathscr{F}_{t}\}_{t \in [0,1]}$-停止時刻の全体を
	\begin{align}
		\mathscr{T}
	\end{align}
	とおく.また$\Natural$から$\mathscr{T}$への写像$\tau$で,
	\begin{itemize}
		\item $\Omega$の任意の要素$\omega$に対して
			\begin{align}
				\tau_{0}(\omega) = 0,
			\end{align}
			
		\item 任意の自然数$n$及び$\Omega$の任意の要素$\omega$に対して
			\begin{align}
				\tau_{n}(\omega) \leq \tau_{n+1}(\omega),
			\end{align}
			
		\item $\omega$を$\Omega$の任意の要素とするとき
			\begin{align}
				\tau_{n}(\omega) = 1
			\end{align}
			を満たす自然数$n$が取れる.
	\end{itemize}
	
	を満たすものを,便宜上$\{\mathscr{F}_{t}\}_{t \in [0,1]}$-{\bf 増大停止時刻列}\index{ぞうだいていしじこくれつ@増大停止時刻列}と呼ぶことにする.
	
	\begin{screen}
		\begin{dfn}[局所マルチンゲール]
			$X$を$(\Omega,\mathscr{F},P)$上の確率過程とするとき,$\{\mathscr{F}_{t}\}_{t \in [0,1]}$-増大停止時刻$\tau$が取れて,
			任意の自然数$n$に対して
			\begin{align}
				X^{\tau_{n}}
			\end{align}
			が$\{\mathscr{F}_{t}\}_{t \in [0,1]}$-マルチンゲールとなるならば,言い換えれば
			$X$を局所的にマルチンゲール化する$\{\mathscr{F}_{t}\}_{t \in [0,1]}$-増大停止時刻列が取れるならば,
			$X$を$\{\mathscr{F}_{t}\}_{t \in [0,1]}$-{\bf 局所マルチンゲール}\index{きょくしょまるちんげーる@局所マルチンゲール}
			{\bf (local martingale)}と呼ぶ.$(\Omega,\mathscr{F},P)$上の連続な
			$\{\mathscr{F}_{t}\}_{t \in [0,1]}$-局所マルチンゲール$X$で,$\Omega$の任意の要素$\omega$に対して
			\begin{align}
				X_{0}(\omega) = 0
			\end{align}
			を満たすものの全体を
			\begin{align}
				\mathscr{M}_{c,loc}
			\end{align}
			とおく.
		\end{dfn}
	\end{screen}
	
	いま$X$を有界な$\{\mathscr{F}_{t}\}_{t \in [0,1]}$-局所マルチンゲールとし,
	$\tau$を$X$を局所的にマルチンゲール化する$\{\mathscr{F}_{t}\}_{t \in [0,1]}$-増大停止時刻列とする.
	$s$と$t$を
	\begin{align}
		s < t
	\end{align}
	なる$[0,1]$の要素とし,$A$を$\mathscr{F}_{s}$の要素とすれば,任意の自然数$n$で
	\begin{align}
		\int_{A} X^{\tau_{n}}_{t}\ dP = \int_{A} X^{\tau_{n}}_{s}\ dP
	\end{align}
	が成り立つ.ここで$\Omega$の任意の要素$\omega$に対し
	\begin{align}
		\lim_{n \to \infty} X^{\tau_{n}}_{t}(\omega) = X_{t}(\omega)
	\end{align}
	及び
	\begin{align}
		\lim_{n \to \infty} X^{\tau_{n}}_{s}(\omega) = X_{s}(\omega)
	\end{align}
	が成り立つので,Lebesgueの収束定理より
	\begin{align}
		\int_{A} X_{t}\ dP = \int_{A} X_{s}\ dP
	\end{align}
	が従う.つまり{\bf 有界な$\{\mathscr{F}_{t}\}_{t \in [0,1]}$-局所マルチンゲールは
	$\{\mathscr{F}_{t}\}_{t \in [0,1]}$-マルチンゲールである.}
	
	\begin{screen}
		\begin{thm}[$\mathscr{M}_{c,loc}$の要素は局所的に有界マルチンゲール化できる]
			$X$を$\mathscr{M}_{c,loc}$の要素とするとき,$\{\mathscr{F}_{t}\}_{t \in [0,1]}$-増大停止時刻列$\tau$で,
			任意の自然数$n$に対して
			\begin{align}
				X^{\tau_{n}}
			\end{align}
			が有界な連続$\{\mathscr{F}_{t}\}_{t \in [0,1]}$-マルチンゲールとなるものが取れる.
		\end{thm}
	\end{screen}
	
	\begin{sketch}
		自然数$n$に対して
		\begin{align}
			\Omega \ni \omega \longmapsto
			\begin{cases}
				\inf{}{\Set{t \in [0,1]}{n \leq |X_{t}(\omega)|}} 
				& \mbox{if } \Set{t \in [0,1]}{n \leq |X_{t}(\omega)|} \neq \emptyset \\
				1 & \mbox{if } \Set{t \in [0,1]}{n \leq |X_{t}(\omega)|} = \emptyset
			\end{cases}
		\end{align}
		なる写像を対応させる写像を$\sigma$とおくと,定理\ref{thm:increasing_stopping_times_made_from_continuous_martingales}より
		任意の自然数$n$に対して$\sigma_{n}$は$\{\mathscr{F}_{t}\}_{t \in [0,1]}$-停止時刻であり,また
		$[0,1]$の任意の要素$t$及び$\Omega$の任意の要素$\omega$に対し
		\begin{align}
			\left| X^{\sigma_{n}}_{t}(\omega) \right| \leq n
		\end{align}
		が成立する.$\upsilon$を$X$を局所的にマルチンゲール化する$\{\mathscr{F}_{t}\}_{t \in [0,1]}$-増大停止時刻列として,
		自然数$n$に対して
		\begin{align}
			\Omega \ni \omega \longmapsto \min\left\{ \sigma_{n}(\omega),\upsilon_{n}(\omega) \right\}
		\end{align}
		なる写像を対応させる写像を$\tau$とすれば,$\tau$は定理の主張を満たす$\{\mathscr{F}_{t}\}_{t \in [0,1]}$-増大停止時刻列である.
		\QED
	\end{sketch}
	
	\begin{screen}
		\begin{thm}[マルチンゲールは局所マルチンゲール]
			\begin{align}
				\mathscr{M}_{\mathbf{T}} \subset \mathscr{M}^{loc}_{\mathbf{T}}.
			\end{align}
		\end{thm}
	\end{screen}
	
	\begin{sketch}
		
	\end{sketch}
	
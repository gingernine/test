\section{位相線型空間}
	本節で扱う線型空間はすべて複素数体か実数体をスカラーとして考える.また$\Phi$は
	\begin{align}
		\Phi \defeq \C
	\end{align}
	か
	\begin{align}
		\Phi \defeq \R
	\end{align}
	で定められたものと考える.すなわち,$\mathscr{O}_\Phi$は$\mathscr{O}_\C$または$\mathscr{O}_\R$を指す.
	
\subsection{線型位相}
	\begin{screen}
		\begin{dfn}[位相線型空間]\label{def:topological_vector_space}
			$\left((X,\sigma_X),(\Phi,+,\bullet),s\right)$を線型空間とし,
			$\tau$を$X$上の位相とする.また
			$\mathscr{O}_{X \times X}$を$\tau$から作られる$X \times X$上の直積位相とし,
			$\mathscr{O}_{\Phi \times X}$を$\mathscr{O}_\Phi$と$\tau$から作られる$\Phi \times X$上の直積位相とする.
			\begin{description}
				\item[(tvs1)] $\sigma_X$が$\mathscr{O}_{X \times X}/\tau$-連続である.
				\item[(tvs2)] $s$が$\mathscr{O}_{\Phi \times X}/\tau$-連続である.
			\end{description}
			が満たされるとき,$\tau$を$X$上の{\bf 線型位相}\index{せんけいいそう@線型位相}{\bf (vector topology)}と呼び,
			\begin{align}
				\left((X,\sigma_X),(\Phi,+,\bullet),s,\tau\right)
			\end{align}
			を{\bf 位相線型空間}\index{いそうせんけいくうかん@位相線型空間}
			{\bf (topological vector space)}と呼ぶ.
		\end{dfn}
	\end{screen}
	
	\begin{screen}
		\begin{thm}[位相線型空間は$T_3$]\label{thm:tvs_T_3_space}
			$X$を位相線型空間とするとき,コンパクト部分集合$K \subset X$と
			閉集合$C \subset X$に対し,$K \cap C = \emptyset$なら或る0の開近傍$V$が存在して
			次を満たす:
			\begin{align}
				(K + V) \cap (C + V) = \emptyset.
				\label{eq:thm_tvs_T_3_space}
			\end{align}
		\end{thm}
	\end{screen}
	
	実際は位相線型空間は一様化可能であるから,$T_0$ならTychonoffである.
	
	\begin{prf}
		$K = \emptyset$なら$K + V = \emptyset$となり(\refeq{eq:thm_tvs_T_3_space})が成立する.
		$K \neq \emptyset$の場合,任意に$x \in K$を取れば
	\end{prf}
	
	\begin{screen}
		\begin{thm}[平行移動・スカラ倍は連続]\label{thm:continuity_of_translations_multiples}
			$(X,\tau)$を位相線型空間とするとき以下が成立する.
			\begin{description}
				\item[(1)] 任意の$a \in X$に対し$X \ni x \longmapsto a + x \in X$は
					同相写像である.
					
				\item[(2)] 任意の$\alpha \in \Phi$に対し
					$X \ni x \longmapsto \alpha x \in X$は連続であり,
					特に$\alpha \neq 0$のとき同相写像となる.
					
				\item[(3)] 任意の$x \in X$に対し
					$\Phi \ni \alpha \longmapsto \alpha x \in X$は連続である.
			\end{description}
		\end{thm}
	\end{screen}
	
	\begin{prf}
		連続性は定理\ref{thm:multivariable_continuous_mapping_is_one_variable_continuous}より従う.
		また,(1)(2)において$x \longmapsto -a+x,\ x \longmapsto \alpha^{-1}x$
		が逆写像となる.
		\QED
	\end{prf}
	
	\begin{screen}
		\begin{thm}[平行移動不変位相]
			$\tau$を線型空間$X$の位相とする.
			任意の$V \subset X$と$x \in X$に対して
			\begin{align}
				V \in \tau \quad \Longleftrightarrow \quad
				x + V \in \tau
			\end{align}
			が満たされるとき,$\tau$は平行移動不変である(translation invariant)という.
			定理\ref{thm:continuity_of_translations_multiples}より
			位相線型空間において平行移動は同相写像となるから線型位相は平行移動不変である.
		\end{thm}
	\end{screen}
	
	\begin{screen}
		\begin{thm}[位相線型空間は一様空間]
			$\mathscr{B}$を位相線型空間$(X,\tau)$の0ベクトルの均衡な基本近傍系とするとき,
			\begin{align}
				\Phi \defeq
				\Set{\Set{(x,y)}{x,y \in X,\ x-y \in B}}{B \in \mathscr{B}}
			\end{align}
			は近縁系となり,この近縁系が定める位相は$\tau$に一致する.
		\end{thm}
	\end{screen}
	
	\begin{screen}
		\begin{thm}[平行移動不変位相は0の基本近傍系で決まる]
			$\tau$を線型空間$X$の平行移動不変位相,
			$\mathscr{U}$を$0 \in X$の基本近傍系とするとき,
			任意の$x \in X$に対して
			\begin{align}
				\mathscr{U}(x) \defeq
				\Set{x + U}{U \in \mathscr{U}}
			\end{align}
			は$x$の基本近傍系となる.すなわち次が成立する:
			\begin{align}
				\tau = 
				\Set{O \subset X}{\mbox{$O = \emptyset$,或は任意の$x \in O$に対し
				$x+U_x \subset O$を満たす$U_x \in \mathscr{U}$が存在する}}.
			\end{align}
		\end{thm}
	\end{screen}
	
	\begin{prf}
		$V$を$x$の任意の近傍とすれば
		定理\ref{thm:continuity_of_translations_multiples}より
		$-x + V^{\mathrm{o}}$は$0$の開近傍となる.このとき或る
		$U \in \mathscr{U}$が
		\begin{align}
			U \subset -x + V^{\mathrm{o}} \subset -x + V
		\end{align}
		を満たし$x + U \subset V$が従うから
		$\mathscr{U}(x)$は$x$の基本近傍系をなしている.
		このとき定理\ref{thm:a_local_base_restores_the_topology}より
		\begin{align}
			O \in \tau &\quad \Longleftrightarrow \quad
			\mbox{$O = \emptyset$,或は任意の$x \in O$に対し
				$U \subset O$を満たす$U \in \mathscr{U}(x)$が存在する} \\
			&\quad \Longleftrightarrow \quad
			\mbox{$O = \emptyset$,或は任意の$x \in O$に対し
				$x+U_x\subset O$を満たす$U_x \in \mathscr{U}$が存在する} \\
		\end{align}
		が成立する.
		\QED
	\end{prf}
	
	\begin{screen}
		\begin{dfn}[平行移動不変距離・絶対斉次距離]
			$d$を線型空間$X$上に定まる距離とする.
			\begin{description}
				\item[(1)] 次が満たされるとき$d$は平行移動不変
				\index{へいこういどうふへん@平行移動不変}である(invariant)という:
					\begin{align}
						d(x+z, y+z) = d(x,y),\quad (\forall x,y,z \in X).
					\end{align}
					
				\item[(2)]  次が満たされるとき$d$は
					絶対斉次的\index{ぜったいせいじてき@絶対斉次的}
					である(absolutely homogeneous)と呼ぶことにする:
					\begin{align}
						d(\alpha x, \alpha y) = |\alpha| d(x,y),
						\quad (\forall \alpha \in \Phi,\ x,y \in X).
					\end{align}
			\end{description}
		\end{dfn}
	\end{screen}
	
	\begin{screen}
		\begin{thm}[平行移動不変距離による距離位相は平行移動不変]
			線型空間$X$に平行移動不変距離$d$が定まっているとき,
			$d$による距離位相は平行移動不変となる.
		\end{thm}
	\end{screen}
	
	\begin{prf}
		任意の$\delta>0$と$a \in X$に対し
		$B_\delta(a) \defeq \Set{x \in X}{d(x,a) < \delta}$と書けば,
		任意の$y \in X$で
		\begin{align}
			z \in y + B_\delta(a)
			\quad \Longleftrightarrow \quad
			d(z-y,a) < \delta
			\quad \Longleftrightarrow \quad
			d(z,y+a) < \delta
			\quad \Longleftrightarrow \quad
			z \in B_\delta(y+a)
		\end{align}
		が成り立つ.従って,部分集合$U$が$U = \bigcup_{a \in U}B_{\delta_a}(a)$と書けるとき
		任意の$x \in X$に対し
		\begin{align}
			x + U = \bigcup_{a \in U} \left(x+B_{\delta_a}(a)\right)
			= \bigcup_{a \in U} B_{\delta_a}(x+a)
		\end{align}
		となるから,$U$が開集合であることと$x + U$が開集合であることは同値になる.
		\QED
	\end{prf}
	
	\begin{screen}
		\begin{thm}[絶対斉次的かつ平行移動不変な距離はノルムで導入する距離に限られる]
			ノルムで導入する距離は絶対斉次的かつ平行移動不変であり,
			かつそのような距離はノルムで導入する距離に限られる.
		\end{thm}
	\end{screen}
	
	\begin{prf}
		$\Norm{\cdot}{}$を線型空間$X$のノルムとするとき,
		\begin{align}
			d(x,y) \defeq \Norm{x-y}{}, \quad (\forall x,y \in X)
		\end{align}
		で距離を定めれば
		\begin{align}
			d(x+z,y+z) = \Norm{x+z-(y+z)}{} = \Norm{x-y}{} = d(x,y),
			\quad d(\alpha x, \alpha y)
			= \Norm{\alpha (x-y)}{} = |\alpha|\Norm{x-y}{} = |\alpha|d(x,y)
		\end{align}
		が成立する.逆に$X$上の距離$d$が絶対斉次的かつ平行移動不変であるとき,
		\begin{align}
			\Norm{x}{} \defeq d(x,0),\quad (\forall x \in X)
		\end{align}
		でノルムが定まる.実際$\Norm{\alpha x}{} = d(\alpha x,0) 
		= |\alpha|d(x,0) = |\alpha|\Norm{x}{}$かつ
		\begin{align}
			\quad \Norm{x+y}{} = d(x+y,0) = d(x,-y) 
			\leq d(x,0) + d(0,-y) = d(x,0) + d(y,0) = \Norm{x}{} + \Norm{y}{}
		\end{align}
		が成立する.
		\QED
	\end{prf}
	
	\begin{screen}
		\begin{thm}[ノルムで導入する距離位相は線型位相]
			$(X,\Norm{\cdot}{})$をノルム空間とするとき,
			$d(x,y) \defeq \Norm{x-y}{}$で定める距離$d$による距離位相は線型位相となる.
		\end{thm}
	\end{screen}
	
	\begin{prf}
		距離位相は$T_6$位相空間を定めるから$X$は定義\ref{def:topological_vector_space}の(tvs2)を満たす.また
		\begin{align}
			d(x+y,x'+y') \leq d(x+y,x'+y) + d(x'+y,x'+y') = d(x,x') + d(y,y')
		\end{align}
		より加法の連続性が得られ,
		\begin{align}
			d(\alpha x, \alpha'x') &\leq d(\alpha x, \alpha'x) + d(\alpha'x,\alpha'x') \\
			&= d((\alpha - \alpha') x, 0) + |\alpha'|d(x,x')
			= |\alpha-\alpha'|d(x,0) + |\alpha'|d(x,x')
		\end{align}
		よりスカラ倍の連続性も出る.
		\QED
	\end{prf}
	
	\begin{screen}
		\begin{thm}[位相線型空間の連結性]\label{thm:topological_vector_spaces_connected}
			位相線型空間は連結である.
		\end{thm}
	\end{screen}
	
	\begin{prf}
		零元のみの空間は密着空間であるから連結である.
		$X \neq \{0\}$を位相線型空間とするとき,任意に$a,b \in X$を取り
		\begin{align}
			f:[0,1] \ni t \longmapsto a + t(b - a) \in X
		\end{align}
		と定めれば$f$は$[0,1]$から$X$への連続写像である.実際,
		定理\ref{thm:continuity_of_translations_multiples}より
		$\Phi \ni t \longmapsto t(b-a)$が連続であるから
		\begin{align}
			g:[0,1] \ni t \longmapsto t(b-a)
		\end{align}
		は$[0,1]$の相対位相に関して連続であり,かつ$h:X \ni x \longmapsto a + x$もまた連続であるから
		$f = h \circ g$の連続性が従う.
		よって$X$は弧状連結であるから定理\ref{thm:connected_path_connected}より連結である.
		\QED
	\end{prf}
	
	\begin{screen}
		\begin{dfn}[位相線形空間の有界集合]\label{def:boundedness_in_tvs}
			$X$を位相線型空間,$E$を$X$の部分集合とする.0の任意の近傍$V$に対し
			或る$s = s(V) > 0$が存在して
			\begin{align}
				E \subset t V, \quad (\forall t > s)
			\end{align}
			となるとき,$E$は有界であるという.
		\end{dfn}
	\end{screen}
	
	\begin{screen}
		\begin{thm}
		\end{thm}
	\end{screen}
	
	\begin{screen}
		\begin{dfn}[局所基・局所凸・局所コンパクト・局所有界]
			$(X,\tau)$を位相線型空間とする.
			\begin{description}
				\item[(1)] $0 \in X$の基本近傍系を$X$の局所基(local base)と呼ぶ.
				\item[(2)] すべての元が凸集合であるような局所基が取れるとき,$X$は局所凸(locally convex)であるという.
				\item[(3)] $0 \in X$がコンパクトな近傍を持つとき,$X$は局所コンパクト(locally compact)であるという.
				\item[(4)] $0 \in X$が有界な近傍を持つとき,$X$は局所有界(locally bounded)であるという.
			\end{description}
		\end{dfn}
	\end{screen}
	
	\begin{screen}
		\begin{dfn}[$F$-空間・Frechet空間・ノルム空間]
			$(X,\tau)$を位相線型空間とする.
			平行移動不変距離$d$により$X$が距離化可能でかつ完備距離空間となるとき,
			$X$を$F$-空間と呼ぶ.局所凸な$F$-空間をFrechet空間と呼び
		\end{dfn}
	\end{screen}
	
	\begin{screen}
		\begin{thm}[部分空間が$F$-空間なら閉]
			$(X,\tau)$を位相線形空間,$Y \subset X$を部分空間とする.
			このとき$Y$が$F$-空間なら$Y$は$\tau$-閉である.
		\end{thm}
	\end{screen}
	
	\begin{prf}
	\end{prf}
	
	\begin{screen}
		\begin{dfn}[集合の線型演算]
			$X$を体$\Phi$上の位相線型空間,$A,B$を$X$の部分集合,$\alpha,\beta \in \Phi$とする.
			このとき
			\begin{align}
				\alpha A + \beta B \defeq \Set{\alpha a + \beta b}{a \in A,\ b \in B}
			\end{align}
			と書く.
		\end{dfn}
	\end{screen}
	
	\begin{screen}
		\begin{thm}
			$X$を位相線型空間,$A,B$を部分集合とする.
			\begin{description}
				\item[(1)] $\alpha \overline{A} = \overline{\alpha A}$
				\item[(2)] $\alpha (A^{\mathrm{o}}) = (\alpha A)^{\mathrm{o}}$
			\end{description}
		\end{thm}
	\end{screen}
	
	\begin{prf}\mbox{}
		\begin{description}
			\item[(1)] $\alpha = 0$或は$A = \emptyset$の場合は両辺が
				$\{0\}$或は$\emptyset$となって等号が成立する.
				$\alpha \neq 0$かつ$A \neq \emptyset$の場合,
				\begin{align}
					x \in \alpha \overline{A}
					\quad &\Longleftrightarrow \quad
					\alpha^{-1}x \in \overline{A} \\
					\quad &\Longleftrightarrow \quad
					\left(\alpha^{-1}x + V\right) \cap A \neq \emptyset, \quad 
						(\mbox{$\forall V$: neighborhood of 0}) \\
					\quad &\Longleftrightarrow \quad
					\left(x + V\right) \cap \alpha A \neq \emptyset, \quad 
						(\mbox{$\forall V$: neighborhood of 0}) \\
					\quad &\Longleftrightarrow \quad
					x \in \overline{\alpha A}
				\end{align}
				が成り立つ.
				
			\item[(2)] $\alpha = 0$或は$A = \emptyset$の場合は両辺が
				$\{0\}$或は$\emptyset$となって等号が成立する.
				$\alpha \neq 0$かつ$A \neq \emptyset$の場合,
				\begin{align}
					x \in \alpha (A^{\mathrm{o}})
					\quad &\Longleftrightarrow \quad
					\alpha^{-1}x \in A^{\mathrm{o}} \\
					\quad &\Longleftrightarrow \quad
					\mbox{$\exists V$: neighborhood of 0},\quad \alpha^{-1}x + V \subset A \\
					\quad &\Longleftrightarrow \quad
					\mbox{$\exists V$: neighborhood of 0},\quad x + V \subset \alpha A \\
					\quad &\Longleftrightarrow \quad
					x \in (\alpha A)^{\mathrm{o}}
				\end{align}
				が成り立つ.
				
		\end{description}
	\end{prf}
	
	\begin{screen}
		\begin{thm}[斉次距離で距離化可能なら距離と位相の有界性は一致する]
			位相線型空間$(X,\tau)$が斉次的な距離$d$で距離化可能である場合,
			$X$の部分集合の$d$-有界性と$\tau$-有界性は一致する.
		\end{thm}
	\end{screen}
	
	\begin{prf}
		任意の$\alpha>0,\ \delta>0$に対し,
		$B_{\delta}(0) \defeq \Set{x \in X}{d(x,0) < \delta}$とおけば斉次性より
		\begin{align}
			x \in \alpha B_{\delta}(0) 
			\quad \Longleftrightarrow \quad d\left( \alpha^{-1}x,0 \right) < \delta
			\quad \Longleftrightarrow \quad \alpha^{-1}d(x,0) < \delta
			\quad \Longleftrightarrow \quad x \in B_{\alpha\delta}(0)
		\end{align}
		が成立する.$\{B_r(0)\}_{r > 0}$は$X$の局所基となるから,
		$E \subset X$が$d$-有界のときも$\tau$-有界のときも
		$E \subset B_R(0)$を満たす$R > 0$が存在する.
		$E$が$d$-有界集合である場合,任意に0の近傍$V$を取れば
		或る$r > 0$で$B_r(0) \subset V$となり
		\begin{align}
			E \subset B_R(0) \subset B_t(0) = \frac{t}{r} B_r(0) \subset \frac{t}{r}V,
			\quad (\forall t > R)
		\end{align}
		が成立するから$E$は$\tau$-有界集合である.
		逆に$E$が$\tau$-有界集合であるとき,任意に$x \in X$を取れば
		\begin{align}
			E \subset B_R(0) \subset B_{d(x,0) + R}(x)
		\end{align}
		が成立するから$E$は$d$-有界集合である.
		\QED
	\end{prf}
	
	\begin{screen}
		\begin{thm}[位相線型空間上の同程度連続性]
		\label{thm:equicontinuity_on_topological_vector_spaces}
			$X,Y$を位相線型空間とし,$\zeta_X,\zeta_Y$をそれぞれ$X,Y$の零元とし,$\mathscr{F}$を$X$から$Y$への線型写像の族とする.
			また$\mathscr{B}_X,\mathscr{B}_Y$をそれぞれ$X,Y$の局所基とする.そして
			\begin{align}
				\mathscr{V}_X &\defeq \Set{V}{\exists B \in \mathscr{B}_X\, \left(\, 
					\Set{(x,y)}{x,y \in X \wedge y-x \in B} \subset V\, \right)}, \\
				\mathscr{V}_Y &\defeq \Set{V}{\exists B \in \mathscr{B}_Y\, \left(\, 
					\Set{(x,y)}{x,y \in Y \wedge y-x \in B} \subset V\, \right)}
			\end{align}
			で$X,Y$上の近縁系を定める.このとき
			\begin{description}
				\item[(a)] $\forall x \in X\, \forall V \in \mathscr{V}_Y\, \exists U \in \mathscr{V}_X\, \forall f \in \mathscr{F}\,
					\forall y \in X\, \left(\, (x,y) \in U \Longrightarrow (f(x),f(y)) \in V\, \right)$
					
				\item[(b)] $\forall V \in \mathscr{V}_Y\, \exists U \in \mathscr{V}_X\, \forall f \in \mathscr{F}\,
					\forall y \in X\, \left(\, (\zeta_X,y) \in U \Longrightarrow (\zeta_Y,f(y)) \in V\, \right)$
					
				\item[(c)] $\forall B \in \mathscr{B}_Y\, \exists C \in \mathscr{B}_X\, \forall f \in \mathscr{F}\, 
					\left(\, f \ast C \subset B\, \right)$
				
				\item[(d)] $\forall V \in \mathscr{V}_Y\, \exists U \in \mathscr{V}_X\, \forall f \in \mathscr{F}\,
					\forall x,y \in X\, \left(\, (x,y) \in U \Longrightarrow (f(x),f(y)) \in V\, \right)$
			\end{description}
			は全て同値である.
		\end{thm}
	\end{screen}
	
	式(a)は$\mathscr{F}$が同程度連続であるということを表す.
	
	式(b)は$\mathscr{F}$が零元において同程度連続であるということを表す.
	
	式(c)は$\mathscr{F}$の要素の像が一様に抑えられることを表す.
	
	式(d)は$\mathscr{F}$が一様同程度連続であるということを表す.
	
	この定理は,位相線型空間上の線型写像の集合については零元における同程度連続性から一様同程度連続性が導かれることを主張しているが,
	同じ主張は位相群で成立する.その場合$\mathscr{F}$は群準同型写像の集合とすればよい.
	
	\begin{sketch}
		(a)から(b)は直ちに従う.
		(b)が成立しているとする.$B$を$\mathscr{B}_Y$の要素として取り,
		\begin{align}
			V_B \defeq \Set{(x,y)}{x,y \in Y \wedge y - x \in B}
		\end{align}
		とおくと,$\mathscr{V}_X$の或る要素$U$が取れて
		\begin{align}
			\forall f \in \mathscr{F}\, \forall y \in X\, (\, (\zeta_X,y) \in U
			&\Longrightarrow (\zeta_Y,f(y)) \in V_B \\
			&\Longrightarrow f(y) \in B\, )
		\end{align}
		が成立する.ゆえに
		\begin{align}
			C \subset \Set{y}{y \in X \wedge (\zeta_X,y) \in U}
		\end{align}
		なる$\mathscr{B}_X$の要素$C$を取れば
		\begin{align}
			\forall f \in \mathscr{F}\, \left(\, f \ast C \subset B\, \right)
		\end{align}
		が従う.
		
		次に(c)が成立しているとする.$V$を$\mathscr{V}_Y$の要素とすると
		\begin{align}
			\Set{(x,y)}{x,y \in Y \wedge y - x \in B} \subset V
		\end{align}
		を満たす$\mathscr{B}_Y$の要素$B$が取れる.$B$に対し
		\begin{align}
			\forall f \in \mathscr{F}\, \left(\, f \ast C \subset B\, \right)
		\end{align}
		を満たす$\mathscr{B}_X$の要素$C$が取れるが,
		\begin{align}
			U \defeq \Set{(x,y)}{x,y \in X \wedge y - x \in C}
		\end{align}
		とおくと
		\begin{align}
			\forall f \in \mathscr{F}\, \forall x,y \in X\, (\, (x,y) \in U 
			&\Longrightarrow y - x \in C \\
			&\Longrightarrow f(y) - f(x) \in B \\
			&\Longrightarrow (f(x),f(y)) \in V\, )
		\end{align}
		が従う.一様同程度連続ならば同程度連続であるから定理の主張が得られる.
		\QED
	\end{sketch}
	
	\begin{screen}
		\begin{thm}[同程度連続な写像族の有界性]
			$X,Y$を位相線形空間,$\mathscr{F}$を$X$から$Y$への連続線型写像の族とする.
			$\mathscr{F}$が同程度連続であるとき,
		\end{thm}
	\end{screen}
	
	\begin{screen}
		\begin{thm}[Banach-Steinhaus]
			
		\end{thm}
	\end{screen}
	
	\begin{screen}
		\begin{thm}[開写像原理]
			$X$
		\end{thm}
	\end{screen}

\subsection{局所凸空間}
	\begin{screen}
		\begin{thm}[近縁系で導入する一様位相が線型位相となるとき]
		\label{thm:entourages_introducing_vector_topology}
			$(X,\Phi,\sigma,\mu)$を線型空間とし,
			$\zeta$を$X$の$\sigma$に関する単位元とし,
			$\mathscr{V}$を$X$上の近縁系とする.
			$\mathscr{V}$の基本近縁系$\mathscr{U}$で
			\begin{itemize}
				\item $\bigcap \mathscr{U} = \Set{(x,x)}{x \in X}$,
				\item $\forall U \in \mathscr{U}\, \left(\, x + U_\zeta \subset U_x\, \right)$,
				\item $\forall U \in \mathscr{U}\, \forall \alpha \in \Phi\,
					\exists V \in \mathscr{U} \left(\, \alpha \neq 0 \Longrightarrow V \subset \alpha U_\zeta\, \right)$,
				\item $\forall U \in \mathscr{U}\, \forall \alpha \in \Phi\,
					\left(\, |\alpha| \leq 1 \Longrightarrow \alpha U_\zeta \subset U_\zeta\, \right)$,
				\item $\forall U \in \mathscr{U}\, \left(\, \forall x \in X\, 
				\exists s \in ]0,\infty[\, (\, x \in s U_\zeta\, )\, \right)$
			\end{itemize}
			を満たすものが取れるとき,$\mathscr{V}$により導入する$X$上の一様位相は線型位相となる.
			ただし
			\begin{align}
				U_x \defeq \Set{y}{(x,y) \in U}.
			\end{align}
		\end{thm}
	\end{screen}
	
	\begin{sketch}\mbox{}
		\begin{description}
			\item[第一段] 一様位相が不変位相であることを示す.
				$x$を$X$の要素とする.$x$において
				\begin{align}
					\Set{U_x}{U \in \mathscr{U}}
				\end{align}
				は基本近傍系となるが,
				\begin{align}
					\forall U \in \mathscr{U}\, \left(\, x + U_\zeta \subset U_x\, \right)
				\end{align}
				が満たされているので
				\begin{align}
					\Set{x + U_\zeta}{U \in \mathscr{U}}
				\end{align}
				も$x$の基本近傍系となる.従って一様位相は不変位相である.
				
			\item[第二段]
				加法$\sigma$が$(\zeta,\zeta)$において連続となることを示す.
				$B$を$\zeta$の近傍とすれば,
				\begin{align}
					U_\zeta \subset B
				\end{align}
				なる$\mathscr{U}$の要素$U$が取れる.また
				\begin{align}
					W \circ W \subset U
				\end{align}
				なる$\mathscr{U}$の要素$W$も取れる.このとき
				\begin{align}
					W_\zeta \times W_\zeta \subset \sigma^{-1} \ast B
				\end{align}
				が成立する.実際,
				\begin{align}
					(x,y) \in W_\zeta \times W_\zeta
					\label{eq:thm_entourages_introducing_vector_topology}
				\end{align}
				なる$x,y$に対し,
				\begin{align}
					y \in W_\zeta
				\end{align}
				から
				\begin{align}
					x + y \in x + W_\zeta
				\end{align}
				となり
				\begin{align}
					x + y \in W_x
				\end{align}
				が成り立つので,
				\begin{align}
					(\zeta,x) \in W \wedge (x,x+y) \in W
				\end{align}
				となり
				\begin{align}
					(\zeta,x+y) \in W
				\end{align}
				が従う.よって
				\begin{align}
					x+y \in U_\zeta
				\end{align}
				が成り立ち,(\refeq{eq:thm_entourages_introducing_vector_topology})
				が示された.
				
			\item[第三段]
				$x$と$\alpha$をそれぞれ$X$と$\Phi$の要素として,
				スカラ倍$\mu$が$(\alpha,x)$で連続となることを示す.
				$B$を$\alpha x$の近傍とする.このとき
				\begin{align}
					-\alpha x + B
				\end{align}
				は$\zeta$の近傍となるので,
				\begin{align}
					U_\zeta \subset -\alpha x + B 
				\end{align}
				を満たす$\mathscr{U}$の要素$U$が取れる.$U$に対し
				\begin{align}
					W \circ W \subset U
				\end{align}
				なる$\mathscr{U}$の要素$W$を取り,
				\begin{align}
					x \in s W_\zeta
				\end{align}
				なる正数$s$を取り
				\begin{align}
					t \defeq s/(1+|\alpha|s)
				\end{align}
				とおく.このとき
				\begin{align}
					y \in x+t W_\zeta \wedge |\beta - \alpha| < 1/s
					\Longrightarrow \beta y - \alpha x
					&= \beta (y-x) + (\beta - \alpha)x \\
					&\in \beta t W_\zeta + (\beta - \alpha) s W_\zeta \\
					&\subset W_\zeta + W_\zeta \\
					&\subset U_\zeta \\
					&\subset -\alpha x + B
				\end{align}
				が成立する.ゆえに
				\begin{align}
					y \in x+t W_\zeta \wedge |\beta - \alpha| < 1/s
					\Longrightarrow \beta y \in B
				\end{align}
				が成り立ち,
				\begin{align}
					\mu^{-1} \ast B
				\end{align}
				が$(\alpha,x)$の近傍であることが示された.
				
		\end{description}
	\end{sketch}
	
	\begin{screen}
		\begin{dfn}[局所凸・Frechet空間]
			位相線型空間の零元の基本近傍系で,全ての要素が凸であるものが取れるとき,
			その空間は局所凸\index{きょくしょとつ@局所凸}である(locally convex)と呼ばれる.
			また局所凸なF-空間をFrechet空間\index{Frechetくうかん@Frechet空間}と呼ぶ.
		\end{dfn}
	\end{screen}
	
	\begin{screen}
		\begin{lem}[局所凸空間の直積は局所凸]
			$Z$を線型空間,$(X_\lambda)_{\lambda \in \Lambda}$を
			局所凸位相線型空間の族とし,$0 \in X_\lambda$の
			基本近傍系(全ての元は凸)を$\mathscr{U}_\lambda$
			と書く.また各$\lambda \in \Lambda$に対し
			写像$f_\lambda:Z \longrightarrow X_\lambda$が定まっているとする.
			このとき次が成り立つ:
			\begin{description}
				\item[(1)] 
					$0 \in Z$を含み,かつ定理のを満たすような集合系$\mathscr{U}$を
					\begin{align}
						\mathscr{U} \defeq
						\Set{\bigcap_{\lambda \in H} f_\lambda^{-1}(V_\lambda)}{
						\mbox{$H$は$\Lambda$の空でない有限部分集合},\ 
						V_\lambda \in \mathscr{U}_\lambda}
					\end{align}
					で定める.また任意の$V \in \mathscr{U}_\lambda$に対し
					$f_\lambda^{-1}(V)$が凸であるとき($\forall \lambda \in \Lambda$),
					\begin{align}
						\mathscr{U}(x) \defeq
						\Set{x+U}{U\in\mathscr{U}},
						\quad(\forall x \in Z)
					\end{align}
					とおけば,$\{\mathscr{U}(x)\}_{x \in Z}$を基本近傍系とする
					$Z$の位相がただ一つに定まり,$Z$の和を連続にする.
			\end{description}
		\end{lem}
	\end{screen}
	
	\begin{screen}
		\begin{thm}[局所凸空間とはセミノルムの族で生成される空間]
			
		\end{thm}
	\end{screen}
	
\subsection{商空間の位相}

\subsection{位相双対空間}
	\begin{screen}
		\begin{dfn}[位相双対空間・位相第二双対空間]
			$X$を位相線型空間とする.
			\begin{description}
				\item[(1)] $X$上の連続な線型形式,つまり$X$から$\Phi$への連続線型写像
					の全体$X^*$を位相双対空間(topological dual space)と呼ぶ.
					また$X^*$の全ての元を連続にする最弱の位相を$X$の弱位相(weak topology)と呼び
					$\sigma(X,X^*)$と書く.
				
				\item[(2)] 任意の$x \in X$に対し$\varphi_x:X^* \ni x^* \longmapsto \inprod<x,x^*>_{X,X^*}$
					により$X^*$上の線型形式$\varphi_x$が定まる.このとき$\Set{\varphi_x}{x \in X}$の元を全て連続にする最弱の位相を$X$の汎弱位相
					(weak$\ast$ topology)と呼び$\sigma(X^*,X)$と書く.
			\end{description}
		\end{dfn}
	\end{screen}
	
	\begin{screen}
		\begin{thm}
		\end{thm}
	\end{screen}
	
	\begin{screen}
		\begin{thm}[弱位相は局所凸線型位相]
			$(X,\tau)$:位相線型空間,$X'$:$X$上の連続線型形式の集合,このとき
			$X'$-始位相によって$X$は局所凸位相線型空間となる.
		\end{thm}
	\end{screen}
	
	\begin{sketch}
		$X'$-始位相と$X'$で作る近縁系で導入する一様位相は一致する.
		その近縁系は定理\ref{thm:entourages_introducing_vector_topology}の条件を満たすので
		その一様位相は線型位相であり,また局所凸でもある.
	\end{sketch}

\subsection{双対の強位相}
\subsection{位相線型空間}
	
	位相線形空間$(X,\tau)$に対し,その部分集合$Y$上の相対位相を$\tau_Y$と書き,
	また$X$が或る距離$d$で距離付け可能なとき,
	$d$により導入する位相を$\tau_d$と書く.位相$\tau$に関する開集合,閉集合,近傍,
	Cauchy列は$\tau$-開集合(resp. 閉集合,近傍,Cauchy列)と書く.
	
	\begin{screen}
		\begin{thm}[部分空間が$F$-空間なら閉]
			$(X,\tau)$を位相線形空間,$Y \subset X$を部分空間とする.
			このとき$Y$が$F$-空間なら$Y$は$\tau$-閉である.
		\end{thm}
	\end{screen}
	
	\begin{prf}
		$Y$に対し或る平行移動不変な距離$d$が存在して$\tau_Y = \tau_d$を満たす.
		このとき
		\begin{align}
			B_{1/n} \coloneqq \Set{y \in Y}{d(y,0) < \frac{1}{n}},
			\quad n=1,2,\cdots
		\end{align}
		で$\tau_Y$-開集合を定めれば,$B_{1/n}$は$0$を含むから
		或る0の$\tau$-近傍$U_n$が存在して
		\begin{align}
			B_{1/n} = Y \cap U_n, \quad n=1,2,\cdots
		\end{align}
		を満たす.
	\end{prf}
	
	\begin{screen}
		\begin{dfn}[同程度連続]
			$X,Y$を位相空間とし,$\mathscr{F}$を$X$から$Y$への連続写像からなる集合とする.
			任意の$x \in X$及び$f(x)$の近傍$V$に対し或る$x$の近傍$U \subset X$が存在して
			\begin{align}
				\quad U \subset f^{-1}(V),\quad (\forall f \in \mathscr{F})
			\end{align}
			が満たされるとき,$\mathscr{F}$は$x$で同程度連続(equicontinuous at $x$)であるという.
		\end{dfn}
	\end{screen}
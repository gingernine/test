\subsection{位相線型空間}
	
	位相線形空間$(X,\tau)$に対し,その部分集合$Y$上の相対位相を$\tau_Y$と書き,
	また$X$が或る距離$d$で距離付け可能なとき,
	$d$により導入する位相を$\tau_d$と書く.位相$\tau$に関する開集合,閉集合,近傍,
	Cauchy列は$\tau$-開集合(resp. 閉集合,近傍,Cauchy列)と書く.
	
	\begin{screen}
		\begin{thm}[部分空間が$F$-空間なら閉]
			$(X,\tau)$を位相線形空間,$Y \subset X$を部分空間とする.
			このとき$Y$が$F$-空間なら$Y$は$\tau$-閉である.
		\end{thm}
	\end{screen}
	
	\begin{prf}
		$Y$に対し或る平行移動不変な距離$d$が存在して$\tau_Y = \tau_d$を満たす.
		このとき
		\begin{align}
			B_{1/n} \coloneqq \Set{y \in Y}{d(y,0) < \frac{1}{n}},
			\quad n=1,2,\cdots
		\end{align}
		で$\tau_Y$-開集合を定めれば,$B_{1/n}$は$0$を含むから
		或る0の$\tau$-近傍$U_n$が存在して
		\begin{align}
			B_{1/n} = Y \cap U_n, \quad n=1,2,\cdots
		\end{align}
		を満たす.
	\end{prf}
	
	\begin{screen}
		\begin{dfn}[集合の線型演算]
			$X$を$\Phi$上の位相線型空間,$A,B$を$X$の部分集合,$\alpha,\beta \in \Phi$とする.
			このとき
			\begin{align}
				\alpha A + \beta B \coloneqq \Set{\alpha a + \beta b}{a \in A,\ b \in B}
			\end{align}
			と書く.
		\end{dfn}
	\end{screen}
	
	\begin{screen}
		\begin{dfn}[位相線形空間の有界集合]
			$X$を位相線型空間,$E$を$X$の部分集合とする.0の任意の近傍$V$に対し
			或る$s = s(V) > 0$が存在して
			\begin{align}
				E \subset t V, \quad (\forall t > s)
			\end{align}
			となるとき,$E$は有界であるという.
		\end{dfn}
	\end{screen}
	
	\begin{screen}
		\begin{dfn}[位相線型空間における同程度連続性]
			$X,Y$を位相線形空間,$\mathscr{F}$を$X$から$Y$への連続線型写像の族とする.
			このとき,$\mathscr{F}$が同程度連続であるとは,$0 \in Y$の任意の近傍$V$に対し
			\begin{align}
				f(U) \subset V,\quad (\forall f \in \mathscr{F})
			\end{align}
			を満たす$0 \in X$の近傍$U$が存在することである.
		\end{dfn}
	\end{screen}
	
	\begin{screen}
		\begin{thm}[同程度連続な写像族の有界性]
			$X,Y$を位相線形空間,$\mathscr{F}$を$X$から$Y$への連続線型写像の族とする.
			$\mathscr{F}$が同程度連続であるとき,
		\end{thm}
	\end{screen}
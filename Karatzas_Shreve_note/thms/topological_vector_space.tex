\subsection{位相線型空間 (Rudin note)}
	\begin{screen}
		\begin{thm}[多変数連続写像は一変数写像として連続]
		\label{thm:multivariable_continuous_mapping_is_one_variable_continuous}
			$\Lambda$を任意濃度の空でない集合とし,
			$\left( (S_\lambda,\tau_\lambda) \right)_{\lambda \in \Lambda}$を位相空間の族とする.
			
		\end{thm}
	\end{screen}
	
	以降扱う線型空間はすべて体$\Phi (=\C,\R)$をスカラーとして考え,位相はEuclid距離による距離位相を導入する.
	
	\begin{screen}
		\begin{dfn}[位相線型空間]\label{def:topological_vector_space}
			$\Phi$上の線型空間$X$で定められる位相$\tau$が
			\begin{description}
				\item[(tvs1)] $X \times X \ni (x,y) \longmapsto x+y \in X$
					及び$\Phi \times X \ni (\alpha,x) \longmapsto \alpha x \in X$
					が$\tau$及びその直積位相に関し連続である.
				\item[(tvs2)]
					$(X,\tau)$は$T_1$位相空間である.
			\end{description}
			を満たすとき線型位相(vector topology)と呼び,
			$(X,\tau)$を位相線型空間(topological vector space)と呼ぶ.
		\end{dfn}
	\end{screen}
	
	\begin{screen}
		\begin{thm}[位相線型空間は$T_3$]
		\end{thm}
	\end{screen}
	
	\begin{screen}
		\begin{dfn}[平行移動不変距離]
			線型空間$X$上に定まる距離$d$が
			\begin{align}
				d(x+z, y+z) = d(x,y),\quad (\forall x,y,z \in X)
			\end{align}
			を満たすとき,$d$を平行移動不変距離(invariant metric)と呼ぶ.
			平行移動不変距離$d$がさらに
			\begin{align}
				d(\alpha x, \alpha y) = |\alpha| d(x,y),
				\quad (\forall \alpha \in \Phi,\ x,y \in X)
			\end{align}
			を満たすとき$d$は斉次的である(homogeneous)という.
			例えば$X$にノルム$\Norm{\cdot}{}$が定まっている場合,
			$d(x,y) \coloneqq \Norm{x-y}{}$により定まる距離$d$は斉次的かつ平行移動不変である.
		\end{dfn}
	\end{screen}
	
	\begin{screen}
		\begin{thm}[斉次的な平行移動不変距離による距離位相は線型位相]
			$X$を線型空間とする.$X$において斉次的な平行移動不変距離$d$が存在するとき,
			$d$で導入する距離位相は線型位相となる.
		\end{thm}
	\end{screen}
	
	\begin{prf}
		距離位相は$T_4$位相空間を定めるから$X$は定義\ref{def:topological_vector_space}の(tvs2)を満たす.また
		\begin{align}
			d(x+y,x'+y') \leq d(x+y,x'+y) + d(x'+y,x'+y') = d(x,x') + d(y,y')
		\end{align}
		より加法の連続性が得られ,
		\begin{align}
			d(\alpha x, \alpha'x') &\leq d(\alpha x, \alpha'x) + d(\alpha'x,\alpha'x') \\
			&= d((\alpha - \alpha') x, 0) + |\alpha'|d(x,x')
			= |\alpha-\alpha'|d(x,0) + |\alpha'|d(x,x')
		\end{align}
		よりスカラ倍の連続性も出る.
		\QED
	\end{prf}
	
	\begin{screen}
		\begin{thm}[平行移動・スカラ倍の連続性]\label{thm:continuity_of_translations_multiples}
			$(X,\tau)$を位相線型空間とするとき,任意の$a \in X$に対し
			\begin{align}
				X \ni x \longmapsto a + x \in X,
				\quad \Phi \ni \alpha \longmapsto \alpha a \in X
			\end{align}
			はいずれも連続である.同様に任意の$\beta \in \Phi$に対し
			$X \ni x \longmapsto \beta x$もまた連続である.
		\end{thm}
	\end{screen}
	
	\begin{prf}
		定理\ref{thm:multivariable_continuous_mapping_is_one_variable_continuous}より従う.
		\QED
	\end{prf}
	
	\begin{screen}
		\begin{thm}[位相線型空間の連結性]\label{thm:topological_vector_spaces_connected}
			位相線型空間は連結である.
		\end{thm}
	\end{screen}
	
	\begin{prf}
		零元のみの空間は密着空間であるから連結である.
		$X \neq \{0\}$を位相線型空間とするとき,任意に$a,b \in X$を取り
		\begin{align}
			f:[0,1] \ni t \longmapsto a + t(b - a) \in X
		\end{align}
		と定めれば$f$は$[0,1]$から$X$への連続写像である.実際,
		定理\ref{thm:continuity_of_translations_multiples}より
		$\Phi \ni t \longmapsto t(b-a)$が連続であるから
		\begin{align}
			g:[0,1] \ni t \longmapsto t(b-a)
		\end{align}
		は$[0,1]$の相対位相に関して連続であり,かつ$h:X \ni x \longmapsto a + x$もまた連続であるから
		$f = h \circ g$の連続性が従う.
		よって$X$は弧状連結であるから定理\ref{thm:connected_path_connected}より連結である.
		\QED
	\end{prf}
	
	\begin{screen}
		\begin{dfn}[位相線形空間の有界集合]
			$X$を位相線型空間,$E$を$X$の部分集合とする.0の任意の近傍$V$に対し
			或る$s = s(V) > 0$が存在して
			\begin{align}
				E \subset t V, \quad (\forall t > s)
			\end{align}
			となるとき,$E$は有界であるという.
		\end{dfn}
	\end{screen}
	
	\begin{screen}
		\begin{thm}
		\end{thm}
	\end{screen}
	
	位相線形空間$(X,\tau)$に対し,その部分集合$Y$上の相対位相を$\tau_Y$と書き,
	また$X$が或る距離$d$で距離付け可能なとき,
	$d$により導入する位相を$\tau_d$と書く.位相$\tau$に関する開集合,閉集合,近傍,
	Cauchy列は$\tau$-開集合(resp. 閉集合,近傍,Cauchy列)と書く.
	
	\begin{screen}
		\begin{dfn}[局所基・局所凸・局所コンパクト・局所有界]
			$(X,\tau)$を位相線型空間とする.
			\begin{description}
				\item[(1)] $0 \in X$の基本近傍系を$X$の局所基(local base)と呼ぶ.
				\item[(2)] すべての元が凸集合であるような局所基が取れるとき,$X$は局所凸(locally convex)であるという.
				\item[(3)] $0 \in X$がコンパクトな近傍を持つとき,$X$は局所コンパクト(locally compact)であるという.
				\item[(4)] $0 \in X$が有界な近傍を持つとき,$X$は局所有界(locally bounded)であるという.
			\end{description}
		\end{dfn}
	\end{screen}
	
	\begin{screen}
		\begin{thm}[局所基は平行移動により任意の点の近傍系となる]
			$X$を位相線型空間とするとき以下が成り立つ:
			\begin{description}
				\item[(1)] 任意の$x \in X$に対し$\Set{x + V}{V \in \mathscr{B}}$は
					$x$の近傍系となる.
				\item[(2)] $x$の近傍系$\mathbb{V}(x)$に対し$\mathscr{B} = \Set{-x + V}{V \in \mathbb{V}(x)}$
					となる.
				\item[(3)] $X$が局所コンパクト空間であるとき,任意の点はコンパクトな近傍を持つ. 
 			\end{description}
		\end{thm}
	\end{screen}
	
	\begin{screen}
		\begin{dfn}[$F$-空間・Frechet空間・ノルム空間]
			$(X,\tau)$を位相線型空間とする.
			$d$により$X$が距離化可能でかつ完備距離空間となるとき,
			$X$を$F$-空間と呼ぶ.局所凸な$F$-空間をFrechet空間と呼び
		\end{dfn}
	\end{screen}
	
	\begin{screen}
		\begin{thm}[部分空間が$F$-空間なら閉]
			$(X,\tau)$を位相線形空間,$Y \subset X$を部分空間とする.
			このとき$Y$が$F$-空間なら$Y$は$\tau$-閉である.
		\end{thm}
	\end{screen}
	
	\begin{prf}
		$Y$に対し或る平行移動不変な距離$d$が存在して$\tau_Y = \tau_d$を満たす.
		このとき
		\begin{align}
			B_{1/n} \coloneqq \Set{y \in Y}{d(y,0) < \frac{1}{n}},
			\quad n=1,2,\cdots
		\end{align}
		で$\tau_Y$-開集合を定めれば,$B_{1/n}$は$0$を含むから
		或る0の$\tau$-近傍$U_n$が存在して
		\begin{align}
			B_{1/n} = Y \cap U_n, \quad n=1,2,\cdots
		\end{align}
		を満たす.
	\end{prf}
	
	\begin{screen}
		\begin{dfn}[集合の線型演算]
			$X$を体$\Phi$上の位相線型空間,$A,B$を$X$の部分集合,$\alpha,\beta \in \Phi$とする.
			このとき
			\begin{align}
				\alpha A + \beta B \coloneqq \Set{\alpha a + \beta b}{a \in A,\ b \in B}
			\end{align}
			と書く.
		\end{dfn}
	\end{screen}
	
	\begin{screen}
		\begin{thm}
			$X$を位相線型空間,$A,B$を部分集合とする.
			\begin{description}
				\item[(1)] $\alpha \overline{A} = \overline{\alpha A}$
				\item[(2)] $\alpha (A^{\mathrm{o}}) = (\alpha A)^{\mathrm{o}}$
			\end{description}
		\end{thm}
	\end{screen}
	
	\begin{prf}\mbox{}
		\begin{description}
			\item[(1)] $\alpha = 0$或は$A = \emptyset$の場合は両辺が
				$\{0\}$或は$\emptyset$となって等号が成立する.
				$\alpha \neq 0$かつ$A \neq \emptyset$の場合,
				\begin{align}
					x \in \alpha \overline{A}
					\quad &\Longleftrightarrow \quad
					\alpha^{-1}x \in \overline{A} \\
					\quad &\Longleftrightarrow \quad
					\left(\alpha^{-1}x + V\right) \cap A \neq \emptyset, \quad 
						(\mbox{$\forall V$: neighborhood of 0}) \\
					\quad &\Longleftrightarrow \quad
					\left(x + V\right) \cap \alpha A \neq \emptyset, \quad 
						(\mbox{$\forall V$: neighborhood of 0}) \\
					\quad &\Longleftrightarrow \quad
					x \in \overline{\alpha A}
				\end{align}
				が成り立つ.
				
			\item[(2)] $\alpha = 0$或は$A = \emptyset$の場合は両辺が
				$\{0\}$或は$\emptyset$となって等号が成立する.
				$\alpha \neq 0$かつ$A \neq \emptyset$の場合,
				\begin{align}
					x \in \alpha (A^{\mathrm{o}})
					\quad &\Longleftrightarrow \quad
					\alpha^{-1}x \in A^{\mathrm{o}} \\
					\quad &\Longleftrightarrow \quad
					\mbox{$\exists V$: neighborhood of 0},\quad \alpha^{-1}x + V \subset A \\
					\quad &\Longleftrightarrow \quad
					\mbox{$\exists V$: neighborhood of 0},\quad x + V \subset \alpha A \\
					\quad &\Longleftrightarrow \quad
					x \in (\alpha A)^{\mathrm{o}}
				\end{align}
				が成り立つ.
				
		\end{description}
	\end{prf}
	
	\begin{screen}
		\begin{dfn}[位相線型空間における同程度連続性]
			$X,Y$を位相線形空間,$\mathscr{F}$を$X$から$Y$への連続線型写像の族とする.
			このとき,$\mathscr{F}$が同程度連続であるとは,$0 \in Y$の任意の近傍$V$に対し
			\begin{align}
				f(U) \subset V,\quad (\forall f \in \mathscr{F})
			\end{align}
			を満たす$0 \in X$の近傍$U$が存在することである.
		\end{dfn}
	\end{screen}
	
	\begin{screen}
		\begin{thm}[同程度連続な写像族の有界性]
			$X,Y$を位相線形空間,$\mathscr{F}$を$X$から$Y$への連続線型写像の族とする.
			$\mathscr{F}$が同程度連続であるとき,
		\end{thm}
	\end{screen}
	
	\begin{screen}
		\begin{thm}[Banach-Steinhaus]
			
		\end{thm}
	\end{screen}
	
	\begin{screen}
		\begin{thm}[開写像原理]
			$X$
		\end{thm}
	\end{screen}
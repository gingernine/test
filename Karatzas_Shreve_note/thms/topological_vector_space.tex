\subsection{位相線型空間 (Rudin note)}

	\begin{screen}
		\begin{thm}[全射・単射・像・原像]
			$f$を集合$A$から集合$B$への写像とするとき,
			\begin{description}
				\item[(1)] 任意の$U \subset A$に対し$f^{-1}\left(f(U)\right) \supset U$が成立し,
					特に$f$が単射なら$f^{-1}\left(f(U)\right) = U$となる.
				\item[(2)] 任意の$V \subset B$に対し$f\left(f^{-1}(V)\right) \subset V$が成立し,
					特に$f$が全射なら$f\left(f^{-1}(V)\right) = V$となる.
			\end{description}
		\end{thm}
	\end{screen}
	
	\begin{prf}\mbox{}
		\begin{description}
			\item[(1)] 任意の$x \in U$で$f(x) \in f(U)$となるから
				$x \in f^{-1}\left(f(U)\right)$が成立する.
				$f$が単射であれば,任意の$x \in f^{-1}\left(f(U)\right)$に対し
				$f(x) \in f(U)$となるから或る$x_1 \in U$で$f(x) = f(x_1)$となり,
				単射性より$x = x_1 \in U$が成り立つ.
				
			\item[(2)] 任意に$y \in f\left(f^{-1}(V)\right)$を取れば,
				或る$x \in f^{-1}(V)$で$y = f(x) \in V$となる.$f$が全射であるとき,
				任意の$y \in V$に対し或る$x \in A$が$y = f(x)$を満たすから,
				$x \in f^{-1}(V)$となり$y \in f\left(f^{-1}(V)\right)$が従う.
				\QED
		\end{description}
	\end{prf}
	
	\begin{screen}
		\begin{thm}[距離化可能性の遺伝]
			$(X,\tau_X),(Y,\tau_Y)$を位相空間,$f:X \longrightarrow Y$を同相写像とする.
			$X$が距離$d_X$で距離化可能なら,
			\begin{align}
				d_Y\left(f(x),f(y)\right) \coloneqq d_X(x,y),
				\quad (\forall x,y \in X)
			\end{align}
			により$Y$は距離化可能である.
		\end{thm}
	\end{screen}
	
	\begin{prf}
		$\tau_{d_Y}$で$d_Y$により導入する距離位相を表す.
		任意に$U \in \tau_Y$を取れば$f^{-1}(U) \in \tau_X$となるから,
		任意の$y_0 \in U$及び$x_0 = f^{-1}(y_0)$に対し
		\begin{align}
			B^X_r(x_0) \coloneqq \Set{x \in X}{d_X(x_0,x) < r} \subset f^{-1}(U)
		\end{align}
		を満たす$r > 0$が存在して
		\begin{align}
			f\left(B^X_r(x_0)\right) = \Set{y \in Y}{d_Y(y_0,y) < r} \subset U
		\end{align}
		となり$U \in \tau_{d_Y}$が従う.一方で任意の点$y_1 = f(x_1)$を中心とする$d_Y$の球
		\begin{align}
			B^Y_\epsilon (y_1) \coloneqq \Set{y \in Y}{d_Y(y_1,y) < \epsilon}
		\end{align}
		に対し
		\begin{align}
			f^{-1}\left(B^Y_\epsilon (y_1)\right)
			= \Set{x \in X}{d_X(x_1,x) < \epsilon} \in \tau_X
		\end{align}
		が成り立つから,$B^Y_\epsilon (y_1) \in \tau_Y$が従い
		$\tau_{d_Y} \subset \tau_Y$が得られる.
		\QED
	\end{prf}
	
	\begin{screen}
		\begin{thm}[位相線型空間の連結性]\label{thm:topological_vector_spaces_connected}
			位相線型空間は連結である.
		\end{thm}
	\end{screen}
	
	\begin{prf}
		位相線型空間は弧状連結であるから定理\ref{thm:connected_path_connected}より連結である.
		\QED
	\end{prf}
	
	位相線形空間$(X,\tau)$に対し,その部分集合$Y$上の相対位相を$\tau_Y$と書き,
	また$X$が或る距離$d$で距離付け可能なとき,
	$d$により導入する位相を$\tau_d$と書く.位相$\tau$に関する開集合,閉集合,近傍,
	Cauchy列は$\tau$-開集合(resp. 閉集合,近傍,Cauchy列)と書く.
	
	\begin{screen}
		\begin{thm}[部分空間が$F$-空間なら閉]
			$(X,\tau)$を位相線形空間,$Y \subset X$を部分空間とする.
			このとき$Y$が$F$-空間なら$Y$は$\tau$-閉である.
		\end{thm}
	\end{screen}
	
	\begin{prf}
		$Y$に対し或る平行移動不変な距離$d$が存在して$\tau_Y = \tau_d$を満たす.
		このとき
		\begin{align}
			B_{1/n} \coloneqq \Set{y \in Y}{d(y,0) < \frac{1}{n}},
			\quad n=1,2,\cdots
		\end{align}
		で$\tau_Y$-開集合を定めれば,$B_{1/n}$は$0$を含むから
		或る0の$\tau$-近傍$U_n$が存在して
		\begin{align}
			B_{1/n} = Y \cap U_n, \quad n=1,2,\cdots
		\end{align}
		を満たす.
	\end{prf}
	
	\begin{screen}
		\begin{dfn}[集合の線型演算]
			$X$を体$\Phi$上の位相線型空間,$A,B$を$X$の部分集合,$\alpha,\beta \in \Phi$とする.
			このとき
			\begin{align}
				\alpha A + \beta B \coloneqq \Set{\alpha a + \beta b}{a \in A,\ b \in B}
			\end{align}
			と書く.
		\end{dfn}
	\end{screen}
	
	\begin{screen}
		\begin{thm}
			$X$を位相線型空間,$A,B$を部分集合とする.
			\begin{description}
				\item[(1)] $\alpha \overline{A} = \overline{\alpha A}$
				\item[(2)] $\alpha (A^{\mathrm{o}}) = (\alpha A)^{\mathrm{o}}$
			\end{description}
		\end{thm}
	\end{screen}
	
	\begin{prf}\mbox{}
		\begin{description}
			\item[(1)] $\alpha = 0$或は$A = \emptyset$の場合は両辺が
				$\{0\}$或は$\emptyset$となって等号が成立する.
				$\alpha \neq 0$かつ$A \neq \emptyset$の場合,
				\begin{align}
					x \in \alpha \overline{A}
					\quad &\Longleftrightarrow \quad
					\alpha^{-1}x \in \overline{A} \\
					\quad &\Longleftrightarrow \quad
					\left(\alpha^{-1}x + V\right) \cap A \neq \emptyset, \quad (\forall V) \\
					\quad &\Longleftrightarrow \quad
					\left(x + V\right) \cap \alpha A \neq \emptyset, \quad (\forall V) \\
					\quad &\Longleftrightarrow \quad
					x \in \overline{\alpha A}
				\end{align}
				が成り立つ.
		\end{description}
	\end{prf}
	
	\begin{screen}
		\begin{dfn}[位相線形空間の有界集合]
			$X$を位相線型空間,$E$を$X$の部分集合とする.0の任意の近傍$V$に対し
			或る$s = s(V) > 0$が存在して
			\begin{align}
				E \subset t V, \quad (\forall t > s)
			\end{align}
			となるとき,$E$は有界であるという.
		\end{dfn}
	\end{screen}
	
	\begin{screen}
		\begin{dfn}[位相線型空間における同程度連続性]
			$X,Y$を位相線形空間,$\mathscr{F}$を$X$から$Y$への連続線型写像の族とする.
			このとき,$\mathscr{F}$が同程度連続であるとは,$0 \in Y$の任意の近傍$V$に対し
			\begin{align}
				f(U) \subset V,\quad (\forall f \in \mathscr{F})
			\end{align}
			を満たす$0 \in X$の近傍$U$が存在することである.
		\end{dfn}
	\end{screen}
	
	\begin{screen}
		\begin{thm}[同程度連続な写像族の有界性]
			$X,Y$を位相線形空間,$\mathscr{F}$を$X$から$Y$への連続線型写像の族とする.
			$\mathscr{F}$が同程度連続であるとき,
		\end{thm}
	\end{screen}
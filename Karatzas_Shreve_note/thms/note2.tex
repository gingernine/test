\section{note}
	関数論の積分定理までの流れはRudinのReal and Complex Analysisの10章を参考にしたが,
	Rudinとの違いは,本稿では積分路の微分可能性を外して,つまり連続性と有界変動性のみを仮定した下で,
	関数論の特に和書にはあまり見られない回転数と指数の繋がりを精密に分析し,
	AhlforsやRudinの流儀の``スマートな論理展開''を実現し(かけ)ているところである.
	回転数については,偏角の連続選択定理はBeardonのComplex Analysis7章を,
	回転数と指数が一致することの証明はMathStackExchangeに助けてもらった.
	Cauchy-Riemann方程式については磯祐介複素関数論入門を主に参考にした.
\section{周期性}
	次に考察するのは指数関数の{\bf 周期}\index{しゅうき@周期}{\bf (period)}である.
	結論を言えば複素数$z$と$w$に対して
	\begin{align}
		e^{z} = e^{w} \Longleftrightarrow 
		\exists n \in \Z\, \left(\, z - w = 2 \cdot n \cdot \pi \cdot \isym\, \right)
	\end{align}
	が成り立つので,指数関数は$2 \cdot \pi \cdot \isym$だけずれるごとに同じ値を繰り返す.
	つまり{\bf 指数関数の周期は$2 \cdot \pi \cdot \isym$である.}
	我々はすでに
	\begin{align}
		z - w = 2 \cdot n \cdot \pi \cdot \isym
		\Longrightarrow e^{z} = e^{w + 2 \cdot n \cdot \pi \cdot \isym}
		= e^{w} \cdot e^{2 \cdot n \cdot \pi \cdot \isym}
		= e^{w}
	\end{align}
	が成り立つことを知っているから,以下の目標は逆の導出である.
	
	いま$z$と$w$を複素数として,
	\begin{align}
		e^{z} = e^{w}
	\end{align}
	が成り立っているとする.指数法則より
	\begin{align}
		e^{w} = e^{z} \cdot e^{w-z}
	\end{align}
	が成り立つので
	\begin{align}
		e^{z} = e^{z} \cdot e^{w-z}
	\end{align}
	が成立し,両辺に$e^{-z}$を掛けて
	\begin{align}
		1 = e^{w-z}
	\end{align}
	を得る.ここで
	\begin{align}
		w - z = x + \isym \cdot y
	\end{align}
	を満たす実数$x$と$y$を取れば
	\begin{align}
		1 = e^{x} \cdot e^{\isym \cdot y}
	\end{align}
	となるが,
	\begin{align}
		0 < e^{x}
	\end{align}
	かつ
	\begin{align}
		|e^{\isym \cdot y}| = 1
	\end{align}
	なので
	\begin{align}
		1 = |e^{x} \cdot e^{\isym \cdot y}| = e^{x}
	\end{align}
	が成り立ち
	\begin{align}
		x = 0
	\end{align}
	が従う.よって
	\begin{align}
		w - z = \isym \cdot y
	\end{align}
	が従う.ゆえに,
	\begin{align}
		1 = e^{\isym \cdot y} \Longrightarrow 
		\exists n \in \Z\, \left(\, y = 2 \cdot n \cdot \pi\, \right)
	\end{align}
	を示せば
	\begin{align}
		e^{z} = e^{w} \Longrightarrow 
		\exists n \in \Z\, \left(\, z - w = 2 \cdot n \cdot \pi \cdot \isym\, \right)
	\end{align}
	が得られる.
	
	\begin{screen}
		\begin{thm}[$1 = e^{\isym \cdot y}$を満たす実数$y$は$2 \cdot \pi$の整数倍に限られる]
			$y$を実数とするとき,
			\begin{align}
				1 = e^{\isym \cdot y}
			\end{align}
			ならば
			\begin{align}
				y = 2 \cdot n \cdot \pi
			\end{align}
			を満たす整数$n$が取れる.
		\end{thm}
	\end{screen}
	
	\begin{sketch}\mbox{}
		\begin{description}
			\item[第一段]
				いま$y$を
				\begin{align}
					0 < y < 2 \cdot \pi
				\end{align}
				を満たす実数として,
				\begin{align}
					e^{\isym \cdot y} \neq 1
				\end{align}
				が成り立つことを示す.実際,
				\begin{align}
					e^{\isym \cdot \frac{y}{4}} = u + \isym \cdot v
				\end{align}
				を満たす実数$u$と$v$を取ると,
				\begin{align}
					0 < \frac{y}{4} < \frac{\pi}{2}
				\end{align}
				であるから
				\begin{align}
					0 < u \wedge 0 < v
				\end{align}
				である.そして
				\begin{align}
					e^{\isym \cdot y} = (u + \isym \cdot v)^4
					= (u^4 - 6 \cdot u^2 \cdot v^2 + v^4) 
					+ 4 \cdot \isym \cdot u \cdot v \cdot (u^2 - v^2)
				\end{align}
				が成り立つ.
				\begin{align}
					u^2 \neq v^2
				\end{align}
				ならば
				\begin{align}
					0 \neq u \cdot v \cdot (u^2 - v^2)
				\end{align}
				なので$e^{\isym \cdot y}$は実数ではない.
				\begin{align}
					u^2 = v^2
				\end{align}
				ならば
				\begin{align}
					u^2 = v^2 = \frac{1}{2}
				\end{align}
				が従い
				\begin{align}
					e^{\isym \cdot y} = -1
				\end{align}
				が成り立つ.よっていずれの場合も
				\begin{align}
					e^{\isym \cdot y} \neq 1
				\end{align}
				である.
				
			\item[第二段]
				$y$を任意に与えられた実数とすれば,
				\begin{align}
					2 \cdot n \cdot \pi \leq y < 2 \cdot (n+1) \cdot \pi
				\end{align}
				を満たす整数$n$が取れる.このとき
				\begin{align}
					0 \leq y - 2 \cdot n \cdot \pi < 2 \cdot \pi
				\end{align}
				かつ
				\begin{align}
					e^{\isym \cdot y}
					= e^{\isym \cdot y} \cdot e^{\isym \cdot (- 2 \cdot n \cdot \pi)}
					= e^{\isym \cdot (y - 2 \cdot n \cdot \pi)}
				\end{align}
				であるから,前段の結果より
				\begin{align}
					e^{\isym \cdot y} = 1
				\end{align}
				ならば
				\begin{align}
					y = 2 \cdot n \cdot \pi
				\end{align}
				が成り立つ.
				\QED
		\end{description}
	\end{sketch}
	
	以上で次の主張が得られた.
	\begin{screen}
		\begin{thm}[指数関数の周期は$2 \cdot \pi \cdot \isym$]
		\label{thm:period_of_exponential_function_is_2_pi_i}
			任意に与えられた複素数$z$と$w$に対して
			\begin{align}
				e^{z} = e^{w} \Longleftrightarrow 
				\exists n \in \Z\, \left(\, z - w = 2 \cdot n \cdot \pi \cdot \isym\, \right).
			\end{align}
		\end{thm}
	\end{screen}
	
	余弦と正弦は指数関数によって定められているので,
	これらも$2 \cdot \pi \cdot \isym$ずれるごとに同じ値を繰り返す.
	また定理\ref{thm:period_of_exponential_function_is_2_pi_i}から
	余弦と正弦の零点の全体を把握することができる.
	
	実際,$z$を複素数とすれば
	\begin{align}
		\cos{z} = 0 
		&\Longleftrightarrow e^{\isym \cdot z} + e^{-\isym \cdot z} = 0 \\
		&\Longleftrightarrow e^{2 \cdot \isym \cdot z} = -1 \\
		&\Longleftrightarrow e^{2 \cdot \isym \cdot z} = e^{\pi \cdot \isym} \\
		&\Longleftrightarrow \exists n \in \Z\,
		\left(\, 2 \cdot z - \pi = 2 \cdot n \cdot \pi\, \right) \\
		&\Longleftrightarrow \exists n \in \Z\,
		\left(\, z = \frac{\pi}{2} + n \cdot \pi\, \right)
	\end{align}
	が成り立ち,同様に
	\begin{align}
		\sin{z} = 0
		\Longleftrightarrow e^{2 \cdot \isym \cdot z} = e^{0}
		\Longleftrightarrow \exists n \in \Z\, \left(\, z = n \cdot \pi\, \right)
	\end{align}
	が成り立つ.以上を次の主張としてまとめておく.
	
	\begin{screen}
		\begin{thm}[余弦と正弦の零点]
			$\cos$の零点は$\pi/2$を$\pi$の整数倍だけずらした実数の全体である:
			\begin{align}
				\Set{z \in \C}{\cos{z} = 0}
				= \Set{z}{\exists n \in \Z\, \left(\, z = \frac{\pi}{2} + n \cdot \pi\, \right)}.
			\end{align}
			$\sin$の零点は$\pi$を整数倍した実数の全体である:
			\begin{align}
				\Set{z \in \C}{\sin{z} = 0}
				= \Set{z}{\exists n \in \Z\, \left(\, z = n \cdot \pi\, \right)}.
			\end{align}
		\end{thm}
	\end{screen}
	
	$z$を複素数とすれば
	\begin{align}
		z = x + \isym \cdot y
	\end{align}
	を満たす実数$x$と$y$が取れるが,このとき
	\begin{align}
		e^z = e^x \cdot e^{\isym \cdot y}
	\end{align}
	が成り立ち,また
	\begin{align}
		\left|e^z\right| = \left|e^x\right| \cdot \left|e^{\isym \cdot y}\right| = e^x
	\end{align}
	が成り立つので
	\begin{align}
		e^z = e^x \cdot (\cos{y} + \isym \cdot \sin{y}) = \left|e^z\right| \cdot (\cos{y} + \isym \cdot \sin{y})
	\end{align}
	が従う.後述することだが,$w$を$0$でない複素数とすれば
	\begin{align}
		w = \exp{z}
	\end{align}
	を満たす複素数$z$が取れるので,
	\begin{align}
		w = |w| \cdot (\cos{y} + \isym \cdot \sin{y})
	\end{align}
	を満たす実数$y$が取れる.これを複素数の{\bf 極形式}\index{きょくけいしき@極形式}{\bf (polar form)}と呼び,
	この$y$を$w$の{\bf 偏角}\index{へんかく@偏角}{\bf (argument)}と呼ぶ.先に示したように$w$の偏角たる実数は整数の個数だけ存在する.
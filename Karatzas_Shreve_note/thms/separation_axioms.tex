\subsection{分離公理}
	\begin{screen}
		\begin{dfn}[位相的に識別可能・分離]
			$S$を位相空間とする.
			\begin{itemize}
				\item $x,y \in S$に対し$x \notin \overline{\{y\}}$
					或は$y \notin \overline{\{x\}}$が満たされるとき,
					$x$と$y$は{\bf 位相的に識別可能}
					\index{いそうてきにしきべつかのう@位相的に識別可能}である
					{\bf (topologically distinguishable)}という.
				\item $A,B \subset S$に対し$\overline{A} \cap B = \emptyset$
					かつ$A \cap \overline{B} = \emptyset$が満たされるとき,
					$A$と$B$は{\bf 分離される}
					\index{ぶんりされる@(集合が)分離される}
					{\bf (separeted)}という.点と点,点と集合の分離は一点集合を考える.
				\item $A,B \subset S$が{\bf 近傍で分離される}
					\index{きんぼうでぶんりされる@近傍で分離される}
					{\bf (separated by neighborhoods)}とは,
					$A,B$が互いに交わらない近傍を持つことをいう.
				\item 閉集合$A,B \subset S$が
					{\bf 関数で分離される}
					\index{かんすうでぶんりされる@関数で分離される}
					{\bf (separated by a function)}とは,
					或る連続関数$f:S \longrightarrow [0,1]$によって$f(A) = \{0\},\ f(B) = \{1\}$
					が満たされることをいう.
				\item 閉集合$A,B \subset S$が
					{\bf 関数でちょうど分離される}
					\index{かんすうでちょうどぶんりされる@関数でちょうど分離される}
					{\bf (precisely separated by a function)}とは,
					或る連続関数$f:S \longrightarrow [0,1]$によって
					$A = f^{-1}(\{0\}),\ B = f^{-1}(\{1\})$が満たされることをいう.
			\end{itemize}
		\end{dfn}
	\end{screen}
	
	\begin{screen}
		\begin{thm}[位相的に識別可能な二点は相異なる]
			$S$を位相空間とするとき,任意の$x,y \in S$に対し
			\begin{align}
				\mbox{$x$と$y$が位相的に識別可能} \quad \Longrightarrow \quad
				x \neq y .
			\end{align}
		\end{thm}
	\end{screen}
	
	\begin{prf}
		$x = y$なら$y \in \overline{\{x\}}$かつ$x \in \overline{\{y\}}$となる.
		後述の$T_0$空間とは,この逆が満たされる位相空間である.
		\QED
	\end{prf}
	
	\begin{screen}
		\begin{thm}[分離される集合は他方を含まない近傍を持つ]
		\label{thm:the_equivalent_condition_of_separatedness}
			位相空間$S$において,$A,B \subset S$が分離されることと
			\begin{align}
				A \subset U,\quad B \subset V,\quad 
				A \cap V = \emptyset,
				\quad B \cap U = \emptyset
				\label{eq:thm_the_equivalent_condition_of_separatedness}
			\end{align}
			を満たす開集合$U,V$が存在することは同値である.
		\end{thm}
	\end{screen}
	
	\begin{prf}
		$A,B \subset S$が分離されるとき,$U \coloneqq \overline{B}^c,\ V \coloneqq \overline{A}^c$
		とおけば(\refeq{eq:thm_the_equivalent_condition_of_separatedness})が成立する.
		逆に$A,B$に対し(\refeq{eq:thm_the_equivalent_condition_of_separatedness})を満たす
		開集合$U,V$が存在するとき,$\closure{A} \subset V^c \subset B^c$及び
		$\closure{B} \subset U^c \subset A^c$となるから$A,B$は分離される.
		\QED
	\end{prf}
	
	\begin{screen}
		\begin{thm}[部分空間の互いに素な閉集合はもとの空間で分離される]
		\label{thm:disjoint_relative_closed_sets_are_separated}
			$S$を位相空間,$T$を$S$の部分集合とする.
			このとき$T$上の相対閉集合$A,B$に対し,
			$A \cap B = \emptyset$ならば$A \cap \overline{B} = \emptyset$
			かつ$\overline{A} \cap B = \emptyset$が成り立つ.ただし上線は$S$における閉包を表す.
		\end{thm}
	\end{screen}
	
	\begin{prf}
		$A,B$は一方が空なら分離される.そうでない場合は対偶を示す.
		$A \cap \overline{B} \neq \emptyset$のとき,
		$x \in A \cap \overline{B}$を取り,
		$U$を$x$の$T$における近傍とすれば,
		$x$の$S$における近傍$V$で$U = T \cap V$を満たすものが存在する.
		このとき
		\begin{align}
			U \cap B = (T \cap V) \cap B = T \cap (V \cap B) = V \cap B
		\end{align}
		となるが,一方で$x \in \overline{B}$と定理\ref{thm:belongs_to_closure_iff_clusters}より
		\begin{align}
			V \cap B \neq \emptyset
		\end{align}
		が成り立ち,$B$は$T$で閉じているから$x \in B$が従う.
		対称的に$\overline{A} \cap B \neq \emptyset$の場合も
		$A \cap B \neq \emptyset$が成立する.
		\QED
	\end{prf}
	
	\begin{screen}
		\begin{dfn}[分離公理]\mbox{}
			\begin{itemize}
				\item 任意の二点が位相的に識別可能である位相空間を{\bf $T_0$空間}
					\index{$T_0$くうかん@$T_0$空間},
					或は{\bf Kolmogorov空間}という.
				\item 任意の二点が分離される位相空間を{\bf $T_1$空間}
					\index{$T_1$くうかん@$T_1$空間}という.
				\item 任意の二点が近傍で分離される位相空間を{\bf $T_2$空間}
					\index{$T_2$くうかん@$T_2$空間},
					或は{\bf Hausdorff空間}\index{Hausdorffくうかん@Hausdorff空間}という.
				\item 任意の交わらない点と閉集合が近傍で分離される位相空間を
					{\bf 正則(regular)空間}\index{せいそくくうかん@正則空間}という.
				\item $T_0$かつ正則な位相空間を{\bf $T_3$空間}
					\index{$T_3$くうかん@$T_3$空間},
					或は{\bf 正則Hausdorff空間}
					\index{せいそくHausdorffくうかん@正則Hausdorff空間}という.
				\item 任意の交わらない点と閉集合が関数で分離される位相空間を
					{\bf 完全正則(completely regular)空間}
					\index{かんぜんせいそくくうかん@完全正則空間}という.
				\item $T_0$かつ完全正則な位相空間を{\bf $T_{3{}^1{\mskip -5mu/\mskip -3mu}_2}$空間}
					\index{$T_{3{}^1{\mskip -5mu/\mskip -3mu}_2}$くうかん@$T_{3{}^1{\mskip -5mu/\mskip -3mu}_2}$空間}や
					{\bf 完全正則Hausdorff空間}
					\index{かんぜんせいそくHausdorffくうかん@完全正則Hausdorff空間},
					或は{\bf Tychonoff空間}\index{Tychonoffくうかん@Tychonoff空間}という.
				\item 任意の交わらない二つの閉集合が近傍で分離される位相空間を
					{\bf 正規(normal)空間}\index{せいきくうかん@正規空間}という.
				\item $T_1$かつ正規な位相空間を{\bf $T_4$空間}
					\index{$T_4$くうかん@$T_4$空間},
					或は{\bf 正規Hausdorff空間}
					\index{せいきHausdorffくうかん@正規Hausdorff空間}という.
				\item 任意の部分位相空間が正規である位相空間を
					{\bf 全部分正規(completely normal)空間}
					\index{ぜんぶぶんせいきくうかん@全部分正規空間}という.
				\item $T_1$かつ全部分正規な位相空間を{\bf $T_5$空間}
					\index{$T_5$くうかん@$T_5$空間},
					或は{\bf 全部分正規Hausdorff空間}
					\index{ぜんぶぶんせいきHausdorffくうかん@全部分正規Hausdorff空間}という.
				\item 任意の交わらない二つの閉集合が関数でちょうど分離される位相空間を
					{\bf 完全正規(perfectly normal)空間}
					\index{かんぜんせいきくうかん@完全正規空間}という.
				\item $T_1$かつ完全正規な位相空間を{\bf $T_6$空間}
					\index{$T_6$くうかん@$T_6$空間},
					或は{\bf 完全正規Hausdorff空間}
					\index{かんぜんせいきHausdorffくうかん@完全正規Hausdorff空間}という.
			\end{itemize}
		\end{dfn}
	\end{screen}
	
	\begin{screen}
		\begin{thm}[$T_1$空間とは一点集合が閉である空間]
			位相空間$S$に対し,以下は全て同値になる:
			\begin{description}
				\item[(a)] $S$が$T_1$である.
				\item[(b)] $S$が$T_0$であり,位相的に識別可能な任意の二点が分離される.
				\item[(c)] $S$の任意の一点集合が閉である.
				\item[(d)] $x \in S$が$A \subset S$の集積点であることと$x$の任意の開近傍が$A$と交わることは同値になる.
			\end{description}
		\end{thm}
	\end{screen}
	
	\begin{prf}
		$x$が$A$の集積点であるとき,任意に$x$の近傍$U$を取る.
		いま,$x$の或る開近傍$U_{n-1}$と$x_{n-1} \in U_{n-1},\ (x \neq x_{n-1})$
		が取れたとして,
		\begin{align}
			U_n \coloneqq U_{n-1} \cap (S \backslash \{x_{n-1}\})
		\end{align}
		は$x$の開近傍となり或る$x_n \in (U_{n-1} \backslash \{x\}) \cap A$が取れる.
		$U_0 \coloneqq U^{\mathrm{o}},\ 
		x_0 \in (U^{\mathrm{o}} \backslash \{x\}) \cap A$を出発点とすれば
		$A$は$U$の無限集合$\{x_n\}_{n=1}^\infty$を含む.
	\end{prf}
	
	$T_1$空間でもHausdorffであるとは限らない.実際,$\N$において
	\begin{align}
		\Set{O \subset \N}{\mbox{$O = \emptyset$,又は$\N \backslash O$が有限集合}}
	\end{align}
	で位相を定めるとき,一点集合は常に閉となるが,
	任意の空でない二つの開集合は必ず交叉する(そうでないと有限集合が無限集合を包含することになる)
	のでHausdorff空間とはならない.一方でHausdorff空間は常に$T_1$である.
	
	\begin{screen}
		\begin{thm}[$T_2 \Longrightarrow T_1$]
			Hausdorff空間は$T_1$である.
		\end{thm}
	\end{screen}
	
	\begin{prf}
		$x$をHausdorff空間の点とする.$x$と異なる任意の点$y$に対して
		\begin{align}
			x \in U_y,\quad y \in V_y,\quad U_y \cap V_y = \emptyset
		\end{align}
		を満たす開集合$U_y,V_y$が存在し,このとき
		\begin{align}
			\{x\} = \bigcap_{y\, :\, x \neq y} V_y^c
		\end{align}
		となるから$\{x\}$は閉である.つまりHausdorff空間は$T_1$である.
		\QED
	\end{prf}
	
	\begin{screen}
		\begin{thm}[Hausdorff空間のコンパクト部分集合は閉]
			Hausdorff空間のコンパクト部分集合は閉である.
		\end{thm}
	\end{screen}
	
	\begin{prf}
		$S$をHausdorff空間,$K \subset S$をコンパクト部分集合とするとき,
		任意に$x \in S \backslash K,\ y \in K$を取れば
		\begin{align}
			x \in U_y,\quad y \in V_y, \quad U_y \cap V_y = \emptyset
		\end{align}
		を満たす開集合$U_y,V_y$が取れる.或る$\{y_i\}_{i=1}^n \subset K$に対し
		$K \subset \bigcup_{i=1}^n V_{y_i}$となるから,
		$U \coloneqq \bigcap_{i=1}^n U_{y_i}$とおけば
		\begin{align}
			x \in U,\quad U \subset \bigcap_{i=1}^n \left(S\backslash V_{y_i}\right)
			\subset S \backslash K
		\end{align}
		が成立する.従って$S \backslash K$は開集合であり,$K$は閉集合である.
		\QED
	\end{prf}
	
	\begin{screen}
		\begin{thm}[Hausdorff空間とは交わらない二つのコンパクト集合が近傍で分離される空間]
		\label{thm:Hausdorff_space_two_disjoint_compact_sets_are_separated_by_nbh}
			位相空間において,Hausdorffであることと,
			交わらない二つのコンパクト部分集合が近傍で分離されることは同値である.
		\end{thm}
	\end{screen}
	
	\begin{prf}
		$A,B$をHausdorff空間の交わらないコンパクト集合とするとき,
		任意の$p \in A$に対し
		\begin{align}
			p \in U_p,\quad B \subset V_p,\quad U_p \cap V_p = \emptyset
			\label{eq:thm_Hausdorff_space_two_disjoint_compact_sets_are_separated_by_nbh_1}
		\end{align}
		を満たす開集合$U_p,V_p$が存在する.実際
		任意の$q \in B$に対し
		\begin{align}
			p \in U_p(q),\quad q \in V_p(q),\quad U_p(q) \cap U_p(q) = \emptyset
		\end{align}
		を満たす開集合$U_p(q), U_p(q)$が取れ,$B$のコンパクト性より
		或る$\{q_i\}_{i=1}^n \subset B$で$B \subset \bigcup_{i=1}^n U_p(q_i)$となるから,
		\begin{align}
			U_p \coloneqq \bigcap_{i=1}^n U_p(q_i),
			\quad V_p \coloneqq \bigcup_{i=1}^n V_p(q_i)
		\end{align}
		とおけば(\refeq{eq:thm_Hausdorff_space_two_disjoint_compact_sets_are_separated_by_nbh_1})
		が成立する.$A$のコンパクト性より或る$\{p_j\}_{j=1}^m \subset A$で
		$A \subset \bigcup_{j=1}^m U_{p_j}$となるから,
		\begin{align}
			U \coloneqq \bigcup_{j=1}^m U_{p_j},
			\quad V \coloneqq \bigcap_{j=1}^m V_{p_j}
		\end{align}
		とおけば$A$と$B$は$U,V$により分離される.
		逆の主張は一点集合がコンパクトであることより従う.
		\QED
	\end{prf}
	
	\begin{screen}
		\begin{thm}[Hausdorff空間値連続写像の等価域は閉]
		\label{thm:equivalence_set_of_two_mappings_into_Hausdorff_space_is_closed}
			$S$を位相空間,$T$をHausdorff空間,$f,g$を
			$S$から$T$への連続写像とするとき,$E \coloneqq \Set{x \in S}{f(x) = g(x)}$は$S$で閉じている.
			特に$\overline{E}=X$なら$f=g$となる.
		\end{thm}
	\end{screen}
	
	\begin{prf}
		任意に$x \in \Set{x \in S}{f(x) \neq g(x)}$を取れば,Hausdorff性より
		\begin{align}
			f(x) \in A,\quad g(x) \in B,\quad A \cap B = \emptyset
		\end{align}
		を満たす$T$の開集合$A,B$が存在する.
		$f^{-1}(A) \cap g^{-1}(B)$は$x$の開近傍であり,
		\begin{align}
			f^{-1}(A) \cap g^{-1}(B) \subset \Set{x \in S}{f(x) \neq g(x)}
		\end{align}
		となるから$\Set{x \in S}{f(x) \neq g(x)}$は$S$の開集合である.
		従って$E$は閉である.
		\QED
	\end{prf}
	
	\begin{screen}
		\begin{thm}[$T_3 \Longrightarrow T_2$]
			$T_3$空間はHausdorffである.
		\end{thm}
	\end{screen}
	
	\begin{prf}
		$T_3$空間は$T_0$であるから,相異なる二点$x,y$に対して
		$x \in \overline{\{y\}}$或は$y \in \overline{\{x\}}$が成り立つ.正則性より
		\begin{align}
			f(x) = 0,\quad f(y) = 1
		\end{align}
		を満たす連続写像が存在し,$x,y$は$f^{-1}([0,1/2))$と$f^{-1}((1/2,1])$で分離される.
		\QED
	\end{prf}
	
	\begin{screen}
		\begin{thm}[正則空間とは交わらないコンパクト集合と閉集合が近傍で分離される空間]
		\label{thm:each_point_in_regular_space_has_closesd_local_base}\mbox{}
			\begin{description}
				\item[(1)] 位相空間において,正則性と,交わらないコンパクト集合と閉集合が近傍で分離されることは同値である.
					
				\item[(2)]
					$K,W,\ (K \subset W)$をそれぞれ局所コンパクトな$T_3$空間のコンパクト集合,
					開集合とするとき,相対コンパクトな開集合$U$が存在して次を満たす:
					\begin{align}
						K \subset U \subset \overline{U} \subset W.
						\label{eq:thm_each_point_in_regular_space_has_closesd_local_base}
					\end{align}
			\end{description}
		\end{thm}
	\end{screen}
	
	\begin{prf}\mbox{}
		\begin{description}
			\item[(1)]
				$K,F$を正則空間のコンパクト集合,閉集合とするとき,
				$K \cap F = \emptyset$なら任意の点$x \in K$に対して
				\begin{align}
					x \in U_x,\ \quad F \subset V_x,
					\quad U_x \cap V_x = \emptyset
				\end{align}
				を満たす開集合$U_x,V_x$が取れる.
				$K$はコンパクトであるから或る$\{x_i\}_{i=1}^n \subset K$で
				$K \subset \bigcup_{i=1}^n U_{x_i}$となり
				\begin{align}
					K \subset U \coloneqq \bigcup_{i=1}^n U_{x_i},
					\quad F \subset V \coloneqq \bigcap_{i=1}^n V_{x_i},
					\quad U \cap V = \emptyset
				\end{align}
				が成立する.逆の主張は一点集合がコンパクトであることにより従う.
			\item[(2)]
				任意の$x \in K$に対し,$F_x \subset W$
				を満たす閉近傍$F_x$とコンパクトな近傍$C_x$が存在する.
				或る$\{y_i\}_{i=1}^m \subset K$で
				\begin{align}
					K \subset 
					\left(C_{y_1}^{\mathrm{o}} \cap F_{y_1}^{\mathrm{o}}\right) 
					\cup \cdots \cup 
					\left(C_{y_m}^{\mathrm{o}} \cap F_{y_m}^{\mathrm{o}}\right)
				\end{align}
				となるが,ここで
				$U \coloneqq 
				\bigcup_{i=1}^m C_{y_i}^{\mathrm{o}} \cap F_{y_i}^{\mathrm{o}}$
				とおけば,Hausdorff空間において$C_{y_i}$は閉じているから
				\begin{align}
					\overline{U} \subset \bigcup_{i=1}^m C_{y_i}
				\end{align}
				が成り立つ.
				定理\ref{thm:closed_subset_of_compact_set_is_compact_on_Hausdorff_space}
				より$\overline{U}$のコンパクト性が得られ,かつこのとき
				\begin{align}
					K \subset U \subset \overline{U} \subset \bigcup_{i=1}^m F_{y_i}
					\subset W
				\end{align}
				も満たされる.
				\QED
		\end{description}
	\end{prf}
	
	\begin{screen}
		\begin{thm}[完全正則なら正則]
		\end{thm}
	\end{screen}
	
	\begin{screen}
		\begin{thm}[完全正則空間とは交わらないコンパクト集合と閉集合が関数で分離される空間]
			位相空間において,完全正則であることと,交わらないコンパクト集合と閉集合が
			関数で分離されることは同値である.
		\end{thm}
	\end{screen}
	
	\begin{prf}
		$K,C$をそれぞれ完全正則空間$S$のコンパクト部分集合と閉集合とする.
		任意の$x \in K$に対し
		\begin{align}
			f_x(y) = 
			\begin{cases}
				0, & (y=x), \\
				1, & (y \in C)
			\end{cases} 
		\end{align}
		を満たす連続写像$f_x:S \longrightarrow [0,1]$が存在し,
		$K$のコンパクト性より或る$x_1,x_2,\cdots,x_n \in K$で
		\begin{align}
			K \subset \bigcup_{i=1}^n \Set{x \in K}{f_{x_i}(x) < \frac{1}{2}}
		\end{align}
		が成り立つ.$x \in K$なら$\prod_{i=1}^n f_{x_i}(x) < 1/2$,
		$x \in C$なら$\prod_{i=1}^n f_{x_i}(x) = 1$となるから,
		$f \coloneqq \prod_{i=1}^n f_{x_i}$として
		\begin{align}
			g(x) \coloneqq 2 \operatorname{max}\left\{f(x),\frac{1}{2}\right\} - 1
		\end{align}
		により連続写像$g:S \longrightarrow [0,1]$を定めれば
		\begin{align}
			g(x) = 
			\begin{cases}
				0, & (x \in K), \\
				1, & (x \in C)
			\end{cases}
		\end{align}
		が従う.すなわち$K,C$は$g$で分離される.
		一点はコンパクトであるから逆の主張も得られる.
		\QED
	\end{prf}
	
	\begin{screen}
		\begin{thm}[実数値関数の族が生成する始位相は完全正則]
		\label{thm:initial_topology_of_continuous_functions_is_completely_regular}
			$S$を集合とし,$\mathscr{C}$を$S$から$\R$への実数値関数の集合とする.このとき
			$S$は$\mathscr{C}$-始位相により完全正則空間となる.
		\end{thm}
	\end{screen}
	
	\begin{prf}
		$S$に$\mathscr{C}$-始位相を入れるとき,任意の$x \in S$と$x$を含まない(空でない)始位相の閉集合$F$に対して
		\begin{align}
			x \in \bigcap_{i=1}^n f_i^{-1}(O_i) \subset S \backslash F
		\end{align}
		を満たす$f_i \in \mathscr{C}$と$\R$の開集合$O_i,\ (i=1,\cdots,n)$が取れる.
		$\R$は完全正則であるから,各$i$で
		\begin{align}
			g_i:\R \longrightarrow [0,1]
		\end{align}
		かつ
		\begin{align}
			g_i(f_i(x)) = 1
		\end{align}
		かつ
		\begin{align}
			r \in \R \backslash O_i \Longrightarrow g_i(r) = 0
		\end{align}
		を満たす連続写像$g_i$が存在して,
		\begin{align}
			y \in S \backslash f_i^{-1}(O_i)
			\Longrightarrow g_i(f_i(y)) = 0
		\end{align}
		が成立する.
		\begin{align}
			x \in S \Longrightarrow h(x) = \operatorname{min}\left\{g_1(f_1(x)),\, g_2(f_2(x)),\cdots,g_n(f_n(x))\right\}
		\end{align}
		なる写像$h$を定めれば,$h$は
		\begin{align}
			h:S \longrightarrow [0,1]
		\end{align}
		を満たし,$\mathscr{C}$-始位相に関して連続であり,
		$h(x)=1$かつ$F$上で$0$となる.
		\QED
	\end{prf}
	
	\begin{screen}
		\begin{thm}[完全正則空間の位相は実連続写像全体の始位相に一致する]
			$(S,\mathscr{O})$を位相空間とし,$C(S)$を実連続写像の全体とし,
			\begin{align}
				\mathscr{Z} \coloneqq \Set{\bigcap_{f \in \mathscr{F}} f^{-1}\ast\{0\}}{\mathscr{F} \subset C(S) \wedge \mathscr{F} \neq \emptyset}
			\end{align}
			とおくとき,以下は同値となる:
			\begin{description}
				\item[(a)] $S$が完全正則である.
				\item[(b)] $S$の$C(S)$-始位相が$\mathscr{O}$に一致する.
				\item[(c)] $S$の閉集合全体と$\mathscr{Z}$が一致する.
			\end{description}
		\end{thm}
	\end{screen}
		
	\begin{prf}\mbox{}
		\begin{description}
			\item[$(a) \Longrightarrow (c)$]
				$S$が完全正則であるとき,$C=\emptyset$なら
				\begin{align}
					f:S \longrightarrow \{1\}
				\end{align}
				なる$f$により,$C = S$なら
				\begin{align}
					f:S \longrightarrow \{0\}
				\end{align}
				なる$f$により
				\begin{align}
					C = f^{-1} \ast \{0\}
				\end{align}
				となる.$C$が$\emptyset$でも$S$でもない閉集合であるとき,
				任意の$x \in S \backslash C$に対し或る$f_x \in C(S)$で
				\begin{align}
					f_x(y) = \begin{cases}
						1, & (y=x),\\
						0, & (y \in C)
					\end{cases}
				\end{align}
				を満たすものが存在する.このとき
				\begin{align}
					C \subset \bigcap_{x \in S \backslash C} f_x^{-1} \ast \{0\}
				\end{align}
				となるが,一方で
				\begin{align}
					x \notin C \Longrightarrow x \notin f_x^{-1} \ast \{0\}
				\end{align}
				も成り立つので
				\begin{align}
					C = \bigcap_{x \in S \backslash C} f_x^{-1} \ast \{0\}
				\end{align}
				が成り立つ.従って
				\begin{align}
					C \in \mathscr{Z}
				\end{align}
				となる.一方で$f \in C(S)$に対し$f^{-1} \ast \{0\}$は閉であるから
				$\mathscr{Z}$は$S$の閉集合の族であり$(c)$が満たされる.
				
			\item[$(c) \Longrightarrow (b)$]
				$C(S)$の要素は$\mathscr{O}$に関して連続であり,
				$C(S)$-始位相は$C(S)$のすべての要素を連続にする最弱の位相であるから,
				\begin{align}
					\mbox{$C(S)$-始位相} \subset \mathscr{O}
				\end{align}
				が成り立つ.一方で$(c)$が満たされているとき,
				$O$を$\mathscr{O}$の要素とすると
				\begin{align}
					S \backslash O = \bigcap_{f \in \mathscr{F}} f^{-1}\ast\{0\}
				\end{align}
				を満たす$\subset \Set{f^{-1}(\{0\})}{f \in C(S)}$の部分集合$\mathscr{F}$が存在して,
				\begin{align}
					O = \bigcup_{f \in \mathscr{F}} f^{-1} \ast (\R \backslash \{0\})
				\end{align}
				となる.各$f$で$f^{-1} \ast (\R \backslash \{0\})$は
				$C(S)$-始位相の要素であるから,その合併である$O$も
				$C(S)$-始位相の要素である.ゆえに$(b)$が従う.
			
			\item[$(b) \Longrightarrow (a)$] 
				定理\ref{thm:initial_topology_of_continuous_functions_is_completely_regular}
				より従う.
				\QED
		\end{description}
	\end{prf}
	
	\begin{screen}
		\begin{thm}[正規空間とは交わらない二つの閉集合が関数で分離される空間(Urysohnの補題)]
		\label{thm:Urysohn_lemma}
			位相空間において,正規性と,任意の交わらない二つの閉集合が関数で分離されることは同値である.
		\end{thm}
	\end{screen}
	
	\begin{screen}
		\begin{thm}[正則かつ正規なら完全正則]
		\label{thm:if_regular_and_normal_then_completely_normal}
			正則かつ正規な(空でない)位相空間は完全正則である.
		\end{thm}
	\end{screen}
	
	\begin{prf}
		点$x$と空でない閉集合$F,\ (x \notin F)$に対し,
		正則なら$x$の閉近傍$E$で$E \cap F = \emptyset$を満たすものが取れる.
		加えて正規なら,Urysohnの補題より$E$と$F$は関数で分離されるから$x$と$F$も関数で分離される.
		\QED
	\end{prf}
	
	\begin{screen}
		\begin{thm}[$T_4 \Longrightarrow T_3$]
		\end{thm}
	\end{screen}
	
	\begin{screen}
		\begin{thm}[$T_6 \Longrightarrow T_5 \Longrightarrow T_4$]
			完全正規空間は全部分正規である.
			特に,全部分正規なら正規であるから
			$T_6 \Longrightarrow T_5 \Longrightarrow T_4$となる.
		\end{thm}
	\end{screen}
	
	\begin{prf}
		$S$を$T_6$空間,$T$を$S$の部分位相空間,$A,B$を$T$の空でない閉集合とするとき,
		定理\ref{thm:disjoint_relative_closed_sets_are_separated}より
		\begin{align}
			A \cap \overline{B} = \emptyset,\quad \overline{A} \cap B = \emptyset
		\end{align}
		となる.ただし上線は$S$における閉包を表す.完全正規性より
		\begin{align}
			\overline{A} = f^{-1}(\{0\}),
			\quad \overline{B} = g^{-1}(\{0\}),
			\quad \left( f^{-1}(\{1\}) = \emptyset = g^{-1}(\{1\}) \right)
		\end{align}
		を満たす連続写像$f,g:S \longrightarrow [0,1]$が取れるから,ここで
		$h:S \longrightarrow \R$を$h \coloneqq f - g$で定めれば
		\begin{align}
			\begin{cases}
				h(x) < 0, & (x \in A), \\
				h(x) > 0, & (x \in B)
			\end{cases}
		\end{align}
		が成り立ち,$A \subset T \cap h^{-1}((-\infty,0))$かつ
		$B \subset T \cap h^{-1}((0,\infty))$より$A,B$は$T$における開近傍で分離される.
		\QED
	\end{prf}
	
	\begin{screen}
		\begin{dfn}[$G_\delta$集合・$F_\sigma$集合]
			位相空間の部分集合で,開集合の可算交叉で表されるものを$G_\delta$集合,
			閉集合の可算和で表されるものを$F_\sigma$集合と呼ぶ.
			特に,任意の閉集合が$G_\delta$である空間では任意の開集合が$F_\sigma$となる.
		\end{dfn}
	\end{screen}
	
	\begin{screen}
		\begin{thm}[完全正規空間とは正規かつ閉集合が全て$G_\delta$である空間]
		\label{thm:perfectly_normal_Hausdorff_is_normal_and_closed_is_G_delta}\mbox{}
			\begin{description}
				\item[(1)]
					$F$を完全正規空間の閉集合とすれば,次を満たす閉集合系$(F_n)_{n=1}^\infty$が存在する:
					\begin{align}
						F = \bigcap_{n=1}^\infty F_n,
						\quad F_n^{\mathrm{o}} \supset F_{n+1}. 
					\end{align}
					
				\item[(2)]
					位相空間において,完全正規であることと,正規かつ任意の閉集合が$G_\delta$であることは同値である.
			\end{description}
		\end{thm}
	\end{screen}
	
	\begin{prf}
		$S$を完全正規空間,$A,B$を互いに交わらない$S$の閉集合とすれば,
		$A=f^{-1}(\{0\}),\ B = f^{-1}(\{1\})$を満たす連続関数
		$f:S \longrightarrow \R$が存在する.このとき
		$U \coloneqq f^{-1}([0,1/2)),\ V \coloneqq f^{-1}((1/2,1])$
		で開集合$U,V$を定めれば
		\begin{align}
			A \subset U,\quad B \subset V,\quad U \cap V = \emptyset
		\end{align}
		となるから$S$は正規である.また$F$を閉集合とすれば
		或る連続関数$g:S \longrightarrow \R,\ (\emptyset = g^{-1}(\{1\}))$により
		\begin{align}
			F = g^{-1}(\{0\}) 
			= g^{-1}\Biggl(\bigcap_{n=1}^\infty\left[0,n^{-1}\right)\Biggr)
			= \bigcap_{n=1}^\infty g^{-1}\left(\left[0,n^{-1}\right)\right)
		\end{align}
		が成立するから$F$は$G_\delta$である.特に,このとき
		$F_n \coloneqq g^{-1}\left(\left[0,n^{-1}\right]\right)$とおけば
		\begin{align}
			F = \bigcap_{n=1}^\infty g^{-1}\left(\left[0,n^{-1}\right]\right)
			= \bigcap_{n=1}^\infty F_n,
			\quad F_n^{\mathrm{o}} \supset g^{-1}\left(\left[0,n^{-1}\right)\right)
			\supset g^{-1}\left(\left[0,(n+1)^{-1}\right]\right)
			= F_{n+1}
		\end{align}
		となり(1)の主張が得られる.逆に$S$が正規かつ
		閉集合が全て$G_\delta$であるとき,任意の交わらない閉集合$A,B$に対し
		$A = \bigcap_{n=1}^\infty U_n,\ B = \bigcap_{n=1}^\infty V_n$
		を満たす開集合系$(U_n)_{n=1}^\infty,\ (V_n)_{n=1}^\infty$が取れて,
		定理\ref{thm:Urysohn_lemma}より各$n \geq 1$で
		\begin{align}
			f_n(A) = \{0\},\quad f_n(S \backslash U_n) = \{1\},
			\quad g_n(B) = \{0\},\quad g_n(S \backslash V_n) = \{1\}
		\end{align}
		を満たす連続写像$f_n,g_n:S \longrightarrow [0,1]$が存在する.
		ここで連続写像を$f \coloneqq \sum_{n=1}^\infty 2^{-n} f_n,\ 
		g \coloneqq \sum_{n=1}^\infty 2^{-n} g_n$で定めれば
		\begin{align}
			\begin{cases}
				f(x) = 0, & (x \in A), \\
				f(x) > 0, & (x \notin A),
			\end{cases}
			\quad \begin{cases}
				g(x) = 0, & (x \in B), \\
				g(x) > 0, & (x \notin B),
			\end{cases}
		\end{align}
		となり,$h \coloneqq f/(f+g)$とおけば$A = h^{-1}(\{0\}),\ B = h^{-1}(\{1\})$が成立する.
		従って$S$は完全正規である.
		\QED
	\end{prf}
	
	\begin{screen}
		\begin{thm}[連続な単射の引き戻しによる分離性の遺伝]
			$S,T$を位相空間とする.$S$から$T$への連続単射が存在するとき,
			$T$が$T_k$-空間$(k=0,1,\cdots,6)$なら
			$S$もまた$T_k$-空間となる.
		\end{thm}
	\end{screen}
	
	\begin{prf}
		任意に異なる二点$s_1,s_2 \in S$を取れば単射性より$f(s_1) \neq f(s_2)$となる.
		$T$の分離性より
	\end{prf}
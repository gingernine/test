\section{Stieltjes積分}
\subsection{Stieltjes測度}
	$\R$の左半開区間とは,$a < b$なる実数$a$と$b$によって
	\begin{align}
		]a,b]
	\end{align}
	で表される区間か,もしくは
	\begin{align}
		\begin{gathered}
			]-\infty,b],\\ 
			]a,\infty[,\\ 
			\R
		\end{gathered}
	\end{align}
	のいずれかを指す.いま,$\R$の左半開区間の全体を含む乗法族を
	\begin{align}
		\mathscr{A} \defeq \{\, x \mid \quad &\exists a,b \in \R\, (\, a < b \wedge x=]a,b] \, ) \\
		&\vee \exists b \in \R\, (\, x = ]-\infty,b]\, ) \\
		&\vee \exists a \in \R\, (\, x = ]a,\infty[\, ) \\
		&\vee x = \R\, \\
		&\vee x = \emptyset \}
	\end{align}
	とおき,$\mathscr{A}$の有限非交和の全体から成る集合を
	\begin{align}
		\mathfrak{F} \defeq \Set{\bigcup u}{u \subset \mathscr{A} \wedge \Fin{u} 
		\wedge \forall s,t \in u\, (\, s \neq t \Longrightarrow s \cap t = \emptyset\, )}
	\end{align}
	とおけば,$\mathfrak{F}$は$\borel{\R}$を生成し,
	また定理\ref{thm:forming_finitely_additive_class}より$\R$の上の加法族となる.
	$f$を
	\begin{align}
		f:\R \longrightarrow \R
	\end{align}
	なる単調非減少関数とする.$\R$の空でない左半開区間$I$に対し
	\begin{align}
		\sup{}{\Set{f(b) - f(a)}{a,b \in \R \wedge a < b \wedge ]a,b] \subset I}}
	\end{align}
	を対応させ,$\emptyset$に対して$0$を対応させる写像を$m$とおく.このとき
	\begin{align}
		\mu_0:\mathfrak{F} \longrightarrow [0,\infty]
	\end{align}
	なる写像$\mu_0$を
	\begin{align}
		\bigcup u \longmapsto \sum_{s \in u}m(s)
	\end{align}
	なる関係により定めれば,定理\ref{thm:forming_finitely_additive_class}より
	$\mu_0$は$\mathfrak{F}$上の有限加法的測度をなす.また,$n \geq 1$なる任意の自然数$n$に対して
	\begin{align}
		\mu_0((-n,n]) = f(n) - f(-n) < \infty
	\end{align}
	となるから,$\mu_0$は$\mathfrak{F}$上で$\sigma$-有限である.
	
	\begin{screen}
		\begin{thm}[右連続性と完全加法性]
			単調非減少関数$f$を用いて定めた$\mu_0$について,
			$f$が$\R$の各点で右連続であることと$\mu_0$が$\mathfrak{F}$の上で完全加法的であることは同値である.
		\end{thm}
	\end{screen}
	
	\begin{prf}\mbox{}
		\begin{description}
			\item[第一段]
				
			\item[第二段]
				$f$が右連続であるとし,
				\begin{align}
					I = (a_1,b_1] \times \cdots \times (a_d,b_d],
					\quad (-\infty \leq a_\lambda \leq b_\lambda \leq \infty,\ 
					\lambda = 1,\cdots,d)
				\end{align}
				を取る.$0 < \mu_0(I) < \infty$のとき,任意の$\epsilon > 0$に対し
				\begin{align}
					I_\epsilon \coloneqq 
					\left(\alpha_{1,\epsilon},\beta_{1,\epsilon}\right]
					\times \cdots \times
					\left(\alpha_{d,\epsilon},\beta_{d,\epsilon}\right],
					\quad (-\infty < \alpha_{\lambda,\epsilon} < \beta_{\lambda,\epsilon} < \infty),
					\quad I_\epsilon \subset I,
					\quad \mu(I \backslash I_\epsilon) < \epsilon
				\end{align}
				を満たす左半開区間$I_\epsilon$が存在し,
				\begin{align}
					I_\epsilon \subset 
					K_\epsilon \coloneqq \left[\alpha_{1,\epsilon},\beta_{1,\epsilon}\right]
					\times \cdots \times
					\left[\alpha_{d,\epsilon},\beta_{d,\epsilon}\right]
					\subset I
				\end{align}
				かつ$K_\epsilon$はコンパクト集合である.
				
				定理\ref{thm:compact_class_Haudorff}より
				$\R^d$のコンパクト集合全体はコンパクトクラスとなるから,
				定理\ref{thm:compact_class_intersection}より
				定理\ref{thm:equivalent_conditions_of_countable_additivity}の(a)が満たされる.
		\end{description}
	\end{prf}
	
	$f$が右連続であれば,定理\ref{thm:appendix_Kolmogorov_Hopf}より$\mu_0$は
	$\borel{\R}$の上の$\sigma$-有限測度$\mu$に一意に拡張され,このとき
	\begin{align}
		\forall a,b \in \R\, \left(\, a < b \Longrightarrow \mu(]a,b]) = f(b) - f(a)\, \right)
	\end{align}
	が成立する.また,測度の一致の定理よりこの関係を満たす$\borel{\R}$上の測度は$\mu$に限られる.
	
	\begin{screen}
		\begin{dfn}[Lebesgue-Stieltjes測度]
			$f$を$\R$上の右連続単調非減少な$\R$値関数とするとき,
			\begin{align}
				\forall a,b \in \R\, \left(\, a < b \Longrightarrow \mu(]a,b]) = f(b) - f(a)\, \right)
			\end{align}
			を満たす$\borel{\R}$の上の測度$\mu$が唯一つ存在する.
			このとき$(\R,\borel{\R},\mu)$のLebesgue拡大で得られる完備測度を
			$f$による$1$次元Lebesgue-Stieltjes測度と呼び,
			特に$f$が$\R$上の恒等写像の場合はそれを$f$による$1$次元Lebesgue測度と呼ぶ.
		\end{dfn}
	\end{screen}
	
	次に任意の区間上のStieltjes測度を構成する.
	$I$を$\R$の区間とする.つまり$I$は,$a < b$なる実数$a,b$によって
	\begin{align}
		\begin{gathered}
			]a,b[, \\
			]a,b],\\ 
			[a,b[,\\ 
			[a,b]
		\end{gathered}
	\end{align}
	のいずれかで表されるか,もしくは
	\begin{align}
		\begin{gathered}
			]-\infty,a[, \\
			]-\infty,a], \\
			[a,\infty[, \\
			]a,\infty[, \\
			\R
		\end{gathered}
	\end{align}
	のいずれかで表される.
	いま,$f$を$I$上で定義された右連続単調非減少とし,$I$が有界であるときは$f$は$I$上で有界な関数とする.
	\begin{align}
		\inf{}{I} \in \R
	\end{align}
	ならば
	\begin{align}
		\alpha \defeq \inf{}{\Set{f(x)}{\inf{}{I} < x < \sup{}{I}}}
	\end{align}
	とおき,同様に
	\begin{align}
		\sup{}{I} \in \R
	\end{align}
	ならば
	\begin{align}
		\beta \defeq \sup{}{\Set{f(x)}{\inf{}{I} < x < \sup{}{I}}}
	\end{align}
	とおく.そして$\R$上の右連続単調非減少関数$\hat{f}$を
	\begin{align}
		x \longmapsto 
		\begin{cases}
			\alpha & \mbox{if }-\infty < x \leq \inf{}{I_\lambda} \\
			f(x) & \mbox{if }\inf{}{I_\lambda} < x < \sup{}{I_\lambda} \\
			\beta & \mbox{if }\sup{}{I_\lambda} \leq x < \infty
		\end{cases}
	\end{align}
	なる関係で定める.すると前節の結果より$\borel{\R}$上の正値測度$\hat{\mu}$で
	\begin{align}
		\forall a,b \in \R\, \left(\, a < b \Longrightarrow \hat{\mu}(]a,b]) = \hat{f}(b) - \hat{f}(a)\, \right)
	\end{align}
	を満たすものが取れる.定理\ref{thm:Borel_algebra_of_relative_topology}より
	\begin{align}
		\borel{I} = \Set{I \cap E}{E \in \borel{\R^d}} \subset \borel{\R^d}
	\end{align}
	が成り立つから,
	\begin{align}
		\borel{I} \ni E \longmapsto \hat{\mu}(E)
	\end{align}
	なる写像を$\mu$とおけば$\mu$は$\borel{I}$上の測度となる.
	\begin{align}
		\inf{}{I} \in I
	\end{align}
	のとき
	\begin{align}
		a \defeq \inf{}{I}
	\end{align}
	とおくと
	\begin{align}
		\mu(\{a\}) = \hat{\mu}(\{a\}) = 0
	\end{align}
	が成り立ち,また
	\begin{align}
		]u,v] \subset I
	\end{align}
	なる任意の有界左半開区間に対しては
	\begin{align}
		\mu(]u,v]) = f(v) - f(u)
	\end{align}
	が成り立つ.そして$]u,v]$の形の$I$の部分区間の全体,$\inf{}{I} \in I$の場合は
	それに$\{a\}$を加えたもの,は$\borel{I}$を生成する乗法族をなすため,
	$f$に対して上の関係を満たす(加えて$\inf{}{I} \in I$の場合は
	$\{a\}$の測度が$0$である)ような$\borel{I}$上の測度はただ一つである.

\subsection{Stieltjes積分}
	$I$を$\R$の区間とし,$A$を$I$上で右連続かつ単調非減少関数な関数とし,
	$\mu_A$を$\borel{I}$上の$A$のStieltjes測度とする.また
	$f$を$I$上の$\borel{I}/\borel{\C}$-可測写像とする.
	このとき,$f$が$\mu_A$に関して可積分なら
	\begin{align}
		\int_I f(s)\ dA_s \defeq \int_I f\ d\mu_A
	\end{align}
	と定め,これを$f$の$A$によるStieltjes積分と呼ぶ.
	
	\begin{screen}
		\begin{thm}[Riemann-Stieltjes積分との関係]
			$F:I \longrightarrow \C$が右連続或は左連続なら
		\end{thm}
	\end{screen}
	
	\begin{screen}
		\begin{thm}[時間変更]
			$u$を$[a,b]$から$\R$への非減少連続関数とし,
			$A$を$[u(a),u(b)]$から$\R$への非減少右連続関数とし,
			$f$を$[u(a),u(b)]$から$[0,\infty[$への$\borel{[u(a),u(b)]}/\borel{[0,\infty[}$-可測関数とする.
			このとき
			\begin{align}
				\int_{[a,b]} f(u(s))\ dA_{u(s)} = \int_{[u(a),u(b)]} f(t)\ dA_t
			\end{align}
			が成立する.ただし右辺は$A$のBorel-Stieltjes測度による積分を表し,
			ここで左辺は$A \circ u$のBorel-Stieltjes測度による積分を表す.
		\end{thm}
	\end{screen}
	
	\begin{sketch}
		$\mu_A$を$A$のBorel-Stieltjes測度とし,$\mu_{A_u}$を$A \circ u$のBorel-Stieltjes測度とするとき,
		左辺の積分は
		\begin{align}
			\int_{[a,b]} f \circ u\ d\mu_{A_u} = \int_{[a,b]} f\ d\mu_{A_u}u^{-1}
		\end{align}
		と書けるので,
		\begin{align}
			\mu_A = \mu_{A_u} u^{-1}
		\end{align}
		が成り立つことを示せばよい.これは$\{u(a)\}$に対する測度と$]s,t]$なる形の部分区間の測度が一致することを見ればよい.
		まずStieltjes測度の構成法より
		\begin{align}
			\mu_A(\{u(a)\}) = 0
		\end{align}
		が満たされる.他方で
		\begin{align}
			\alpha \defeq \sup{}{\left( u^{-1} \ast \{u(a)\} \right)}
		\end{align}
		とおけば
		\begin{align}
			[a,\alpha] = u^{-1} \ast \{u(a)\}
		\end{align}
		が成り立ち,
		\begin{align}
			\mu_{A_u} \left( u^{-1} \ast \{u(a)\} \right)
			= \mu_{A_u}([a,\alpha])
			= A(u(\alpha)) - A(u(a)) + \mu_{A_u}(\{a\})
			= 0
		\end{align}
		が成り立つ.以上で
		\begin{align}
			\mu_A(\{u(a)\}) = \mu_{A_u} \left( u^{-1} \ast \{u(a)\} \right)
		\end{align}
		が示された.次に$]s,t]$を$[u(a),u(b)]$の部分区間とする.
		\begin{align}
			\mu_A(]s,t]) = A(t) - A(s)
		\end{align}
		となる.他方で
		\begin{align}
			\alpha &\defeq \sup{}{\left( u^{-1} \ast \{s\} \right)}, \\
			\beta &\defeq \sup{}{\left( u^{-1} \ast \{t\} \right)}
		\end{align}
		とおけば
		\begin{align}
			]\alpha,\beta] = u^{-1} \ast ]s,t]
		\end{align}
		となり,
		\begin{align}
			\mu_{A_u} \left( u^{-1} \ast ]s,t] \right) = \mu_{A_u}(]\alpha,\beta])
			= A(u(\beta)) - A(u(\alpha))
			= A(t) - A(s)
		\end{align}
		が成り立つ.以上で
		\begin{align}
			\mu_A(]s,t]) = \mu_{A_u} \left( u^{-1} \ast ]s,t] \right)
		\end{align}
		が示された.
		\QED
	\end{sketch}
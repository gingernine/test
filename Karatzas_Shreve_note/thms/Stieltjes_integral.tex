\section{Stieltjes積分}
	$\R$の区間$I$とは,
	$(a,b),(a,b],[a,b),[a,b],\ (-\infty \leq a \leq b \leq \infty)$のいずれかを指す.このとき
	\begin{align}
		\mathfrak{F} \coloneqq \Set{\sum_{i=1}^n I_i}{I_i \subset \R:\mbox{区間},\ n=1,2,\cdots}
	\end{align}
	は$\R$上の加法族をなす.$f:\R \longrightarrow \R$を単調非減少関数として,任意の区間$I$に対し
	\begin{align}
		m_0(I) \coloneqq \sup{}{\Set{f(\beta) - f(\alpha)}{(\alpha,\beta] \subset I}}
	\end{align}
	とおき,
	\begin{align}
		\mu_0(F) \coloneqq \sum_{i=1}^n m(I_i),
		\quad (\forall F = I_1 + I_2 + \cdots + I_n \in \mathfrak{F})
	\end{align}
	により有限加法的関数$\mu_0$を定める.
	
	\begin{screen}
		\begin{thm}[集合関数の完全加法性]
			単調非減少関数$f$を用いて定める$\mu_0$について,
			$f$が右連続であることと$\mu_0$が$\mathfrak{F}$で完全加法的であることは同値である.
		\end{thm}
	\end{screen}
	
	任意の区間$I$と$I$上右連続単調非減少な関数$f_I$に対し,
	\begin{align}
		\mathfrak{F}_I \coloneqq \Set{I \cap F}{F \in \mathfrak{F}}
	\end{align}
	は加法族をなし
	\begin{align}
		\mu_0^I(I \cap F) \coloneqq \mu_0(I \cap F)
	\end{align}
	は$\mathfrak{F}_I$上で完全加法的となる.
	$\mu_0^I$の拡張測度を$\mu_I$と書き,これを$f_I$のStieltjes測度と呼ぶ.
	
	\begin{screen}
		\begin{thm}[Riemann-Stieltjes積分との関係]
			$F:I \longrightarrow \C$が右連続或は左連続なら
			\begin{align}
				\int_I F\ d\mu_I = \int_I F\ df_i.
			\end{align}
		\end{thm}
	\end{screen}
	
	\begin{screen}
		\begin{thm}[時間変更]
			
		\end{thm}
	\end{screen}
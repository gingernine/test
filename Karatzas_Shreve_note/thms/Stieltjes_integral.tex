\section{Stieltjes積分}
	$\R$の区間$I$とは,
	$(a,b),(a,b],[a,b),[a,b],\ (-\infty \leq \inf{}{I} \leq \sup{}{I} \leq \infty)$のいずれかを指す.ここで
	\begin{align}
		\mathfrak{F} \coloneqq \Set{\sum_{i=1}^n I_i}{I_i \subset \R:\mbox{区間},\ n=1,2,\cdots}
	\end{align}
	とおけば,定理\ref{thm:forming_finitely_additive_class}より$\mathfrak{F}$は$\R$上の加法族をなす.
	$f:\R \longrightarrow \R$を単調非減少関数として,任意の区間$I$に対し
	\begin{align}
		m_0(I) \coloneqq \sup{}{\Set{f(\beta) - f(\alpha)}{(\alpha,\beta] \subset I}}
	\end{align}
	とおき,また$I = \emptyset$なら$m_0(I) \coloneqq 0$として
	\begin{align}
		\mu_0(F) \coloneqq \sum_{i=1}^n m_0(I_i),
		\quad (\forall F = I_1 + I_2 + \cdots + I_n \in \mathfrak{F})
	\end{align}
	により$\mu_0$を定める.この$\mu_0$はwell-definedであり有限加法的な$m_0$の拡張である.実際,
	\begin{align}
		I_1 + I_2 + \cdots + I_n = J_1 + J_2 + \cdots + J_m
	\end{align}
	に対し,$\sum_{i=1}^n I_i = \sum_{i=1}^n \sum_{j=1}^m I_i \cap J_j = \sum_{j=1}^m J_i$かつ$I_i \cap J_j$は区間であるから
	\begin{align}
		\mu_0\Biggl(\sum_{i=1}^n I_i\Biggr)
		= \sum_{i=1}^n \sum_{j=1}^m m_0(I_i \cap J_j)
		= \mu_0\Biggl(\sum_{j=1}^m J_j\Biggr)
	\end{align}
	が成り立ち,また有限加法性は$\mu_0$の定義より従う.
	
	\begin{screen}
		\begin{thm}[右連続性と完全加法性]
			単調非減少関数$f:\R \longrightarrow \R$を用いて定める$\mu_0$について,
			$f$が右連続であることと$\mu_0$が$\mathfrak{F}$で完全加法的であることは同値である.
		\end{thm}
	\end{screen}
	
	$\mathfrak{F}$が加法族であるから,空でない任意の区間$I \neq \emptyset$に対して
	\begin{align}
		\mathfrak{F}_I \coloneqq \Set{I \cap F}{F \in \mathfrak{F}}
	\end{align}
	は加法族をなす.$I$上右連続単調非減少な,
	ただし$I$が有界なら$I$上で有界な関数$f_I$に対し
	\begin{align}
		a_0 \coloneqq \inf{}{\Set{f(x)}{\inf{}{I} < x < \sup{}{I}}},
		\quad b_0 \coloneqq \sup{}{\Set{f(x)}{\inf{}{I} < x < \sup{}{I}}}
	\end{align}
	とすれば,$\inf{}{I} \in I$なら$a_0 = f(\inf{}{I})$,
	$\sup{}{I} \in I$なら$b_0 = f(\sup{}{I})$であり,
	\begin{align}
		f(x) \coloneqq 
		\begin{cases}
			a_0 & -\infty < x \leq \inf{}{I} \\
			f_I(x) & \inf{}{I} < x < \sup{}{I} \\
			b_0 & \sup{}{I} \leq x < \infty
		\end{cases}
	\end{align}
	により$f_I$を$f$に拡張して$\mu_0$を定めるとき,
	\begin{align}
		\mu_{0,I}(I \cap F) \coloneqq \mu_0(I \cap F)
	\end{align}
	は$\mathfrak{F}_I$上で完全加法的となる.
	$\mu_{0,I}$の拡張測度を$\mu_I$と書き,これを$f_I$のStieltjes測度と呼ぶ.
	$\mu_I$のLebesgue拡大をLebesgue-Stieltjes測度と呼び,
	特に$I = \R,\ f(x) = x$のときLebesgue測度と呼ぶ.
	
	\begin{screen}
		\begin{thm}[左半開区間のStiletjes測度]
			$(\alpha,\beta] \subset I,\ (-\infty < \alpha < \beta < \infty)$に対して
			\begin{align}
				\mu((\alpha,\beta]) = f(\beta) - f(\alpha).
			\end{align}
		\end{thm}
	\end{screen}
	
	\begin{screen}
		\begin{thm}[Riemann-Stieltjes積分との関係]
			$F:I \longrightarrow \C$が右連続或は左連続なら
		\end{thm}
	\end{screen}
	
	\begin{screen}
		\begin{thm}[時間変更]
			
		\end{thm}
	\end{screen}
\section{Stieltjes積分}
\subsection{$\R^d$上のStieltjes測度}
	$\R$の左半開区間とは$(a,b],\ (-\infty \leq a \leq b \leq \infty)$を指す.ただし
	\begin{align}
		(a,b] =
		\begin{cases}
			\emptyset, & a=b, \\
			(-\infty,b], & a=-\infty,\ b < \infty, \\
			(a,\infty), & -\infty < a,\ b = \infty, \\
			(-\infty,\infty), & a=-\infty,\ b = \infty, \\
		\end{cases}
	\end{align}
	と考える.ここで$d \geq 1$に対し$\left(a_1,b_1\right] \times \left(a_2,b_2\right] \times
	\cdots \times \left(a_d,b_d\right]$の形の集合を$\R^d$の左半開区間として
	\begin{align}
		\mathfrak{F} \coloneqq \Set{\sum_{i=1}^n I_i}{I_i \subset \R^d:\mbox{左半開区間},\ n=1,2,\cdots}
	\end{align}
	とおけば,$\mathfrak{F}$は$\borel{\R^d}$を生成し,
	また定理\ref{thm:forming_finitely_additive_class}より$\R^d$の上の加法族となる.
	$f_\lambda:\R \longrightarrow \R,\ (\lambda = 1,\cdots,d)$を単調非減少関数として,
	任意の左半開区間$I = I^1 \times \cdots \times I^d \subset \R^d$($I^\lambda$は$\R$の左半開区間)に対し
	\begin{align}
		m_0(I) \coloneqq \prod_{\lambda=1}^d 
		\sup{}{\Set{f_\lambda(\beta_\lambda) - f_\lambda(\alpha_\lambda)}{\left(\alpha_\lambda,\beta_\lambda\right] \subset I^\lambda}}
	\end{align}
	とおき,また$I = \emptyset$なら$m_0(I) \coloneqq 0$とすれば,定理\ref{thm:forming_finitely_additive_class}より
	\begin{align}
		\mu_0(F) \coloneqq \sum_{i=1}^n m_0(I_i),
		\quad (\forall F = I_1 + I_2 + \cdots + I_n \in \mathfrak{F})
	\end{align}
	は$\mathfrak{F}$で有限加法的となる.そして
	Caratheodoryの拡張定理より$(\R^d,\mathfrak{F},\mu_0)$は
	完備測度空間$(\R^d,\mathfrak{M},\mu^*)$に拡張される.
	
	\begin{screen}
		\begin{dfn}[Lebesgue-Stieltjes測度]
			単調非減少関数の族$(f_\lambda)_{\lambda=1}^d$で構成する
			完備測度空間$(\R^d,\mathfrak{M},\mu^*)$を$d$次元Lebesgue-Stieltjes測度空間と呼ぶ.
			特に$f_\lambda$が全て恒等写像の場合,$d$次元Lebesgue測度空間と呼ぶ.
		\end{dfn}
	\end{screen}
	
	\begin{screen}
		\begin{thm}[右連続性と完全加法性]
			単調非減少関数$f_\lambda:\R \longrightarrow \R,\ (\lambda=1,\cdots,d)$を用いて定める$\mu_0$について,
			全ての$f_\lambda$が右連続であることと$\mu_0$が$\mathfrak{F}$の上で完全加法的であることは同値である.
		\end{thm}
	\end{screen}
	
	任意の$n \geq 1$に対して
	\begin{align}
		\mu_0((-n,n] \times \cdots \times (-n,n]) 
		= \prod_{\lambda=1}^d \left\{f_\lambda(n) - f_\lambda(-n)\right\} < \infty
	\end{align}
	となるから$\mu_0$は$\mathfrak{F}$上で$\sigma$-有限的である.従って$f_\lambda$が全て右連続であれば
	定理\ref{thm:appendix_Kolmogorov_Hopf}より$\mu_0$は
	$(\R^d,\borel{\R^d})$の上の$\sigma$-有限測度$\mu$に一意に拡張され,
	このとき$\left(\R^d,\overline{\borel{\R^d}},\overline{\mu}\right) = \left(\R^d,\mathfrak{M},\mu^*\right)$
	が成立する.$\mu$をBorel-Stieltjes測度と呼ぶ.
	
\subsection{任意の区間上のStieltjes測度}
	$I \subset \R$を区間,つまり
	$(a,b),(a,b],[a,b),[a,b],\ (-\infty \leq a \leq b \leq \infty)$のいずれかとし,
	また$f$を$I$上で定義された右連続単調非減少な,
	ただし$I$が有界なら$I$上で有界な関数とする.
	\begin{align}
		x_0 \coloneqq \inf{}{\Set{f(x)}{\inf{}{I} < x < \sup{}{I}}},
		\quad x_1 \coloneqq \sup{}{\Set{f(x)}{\inf{}{I} < x < \sup{}{I}}}
	\end{align}
	とおけば,$\inf{}{I} \in I$なら$x_0 = f(\inf{}{I})$,
	$\sup{}{I} \in I$なら$x_1 = f(\sup{}{I})$であり,
	\begin{align}
		\hat{f}(x) \coloneqq 
		\begin{cases}
			x_0 & -\infty < x \leq \inf{}{I} \\
			f(x) & \inf{}{I} < x < \sup{}{I} \\
			x_1 & \sup{}{I} \leq x < \infty
		\end{cases}
	\end{align}
	により$f$を$\hat{f}$に拡張すればBorel-Stieltjes測度空間$(\R,\borel{\R},\mu)$が定まる.
	定理\ref{thm:Borel_algebra_of_relative_topology}より
	\begin{align}
		\borel{I} = \Set{I \cap E}{E \in \borel{\R}} \subset \borel{\R}
	\end{align}
	が成り立つから,
	\begin{align}
		\mu_I(I \cap E) \coloneqq \mu(I \cap E),
		\quad (\forall E \in \borel{\R})
	\end{align}
	とおけば$(I,\borel{I},\mu_I)$は測度空間となる.この$\mu_I$を$f$のBorel-Stieltjes測度と呼ぶ.
	$\mu_I$のLebesgue拡大をLebesgue-Stieltjes測度と呼び,
	
\subsection{Stieltjes積分}
	\begin{screen}
		\begin{thm}[左半開区間のStiletjes測度]
			$(\alpha,\beta] \subset I,\ (-\infty < \alpha < \beta < \infty)$に対して
			\begin{align}
				\mu((\alpha,\beta]) = f(\beta) - f(\alpha).
			\end{align}
		\end{thm}
	\end{screen}
	
	\begin{screen}
		\begin{thm}[Riemann-Stieltjes積分との関係]
			$F:I \longrightarrow \C$が右連続或は左連続なら
		\end{thm}
	\end{screen}
	
	\begin{screen}
		\begin{thm}[時間変更]
			
		\end{thm}
	\end{screen}
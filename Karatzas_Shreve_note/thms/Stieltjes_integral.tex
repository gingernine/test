\section{Stieltjes積分}
\subsection{$\R^d$上のStieltjes測度}
	$\R$の左半開区間とは$(a,b],\ (-\infty \leq a \leq b \leq \infty)$を指す.ただし
	\begin{align}
		(a,b] =
		\begin{cases}
			\emptyset, & a=b, \\
			(-\infty,b], & a=-\infty,\ b < \infty, \\
			(a,\infty), & -\infty < a,\ b = \infty, \\
			(-\infty,\infty), & a=-\infty,\ b = \infty, \\
		\end{cases}
	\end{align}
	と考える.ここで$d \geq 1$に対し$\left(a_1,b_1\right] \times \left(a_2,b_2\right] \times
	\cdots \times \left(a_d,b_d\right]$の形の集合を$\R^d$の左半開区間として
	\begin{align}
		\mathfrak{F} \coloneqq \Set{\sum_{i=1}^n I_i}{I_i \subset \R^d:\mbox{左半開区間},\ n=1,2,\cdots}
	\end{align}
	とおけば,$\mathfrak{F}$は$\borel{\R^d}$を生成し,
	また定理\ref{thm:forming_finitely_additive_class}より$\R^d$の上の加法族となる.
	$f_\lambda:\R \longrightarrow \R,\ (\lambda = 1,\cdots,d)$を単調非減少関数として,
	任意の空でない左半開区間$I = I^1 \times \cdots \times I^d \subset \R^d$($I^\lambda$は$\R$の左半開区間)に対し
	\begin{align}
		m_0(I) \coloneqq \prod_{\lambda=1}^d 
		\sup{}{\Set{f_\lambda(\beta_\lambda) - f_\lambda(\alpha_\lambda)}{
			\left(\alpha_\lambda,\beta_\lambda\right] \subset I^\lambda,
			\ -\infty < \alpha_\lambda < \beta_\lambda < \infty}}
	\end{align}
	とおき,$I = \emptyset$なら$m_0(I) \coloneqq 0$とすれば,定理\ref{thm:forming_finitely_additive_class}より
	\begin{align}
		\mu_0(F) \coloneqq \sum_{i=1}^n m_0(I_i),
		\quad (\forall F = I_1 + I_2 + \cdots + I_n \in \mathfrak{F})
		\label{eq:Lebesgue_Stieltjes_measure_on_Rd}
	\end{align}
	により$\mathfrak{F}$上の有限加法的測度が定まる.また,任意の$n \geq 1$に対して
	\begin{align}
		\mu_0((-n,n] \times \cdots \times (-n,n]) 
		= \prod_{\lambda=1}^d \left\{f_\lambda(n) - f_\lambda(-n)\right\} < \infty
	\end{align}
	となるから$\mu_0$は$\mathfrak{F}$上で$\sigma$-有限的である.
	
	\begin{screen}
		\begin{thm}[右連続性と完全加法性]
			単調非減少関数$f_\lambda:\R \longrightarrow \R,\ (\lambda=1,\cdots,d)$を用いて定める$\mu_0$について,
			全ての$f_\lambda$が右連続であることと$\mu_0$が$\mathfrak{F}$の上で完全加法的であることは同値である.
		\end{thm}
	\end{screen}
	
	\begin{prf}\mbox{}
		\begin{description}
			\item[第一段]
				
			\item[第二段]
				全ての$f_\lambda$が右連続であるとし,
				\begin{align}
					I = (a_1,b_1] \times \cdots \times (a_d,b_d],
					\quad (-\infty \leq a_\lambda \leq b_\lambda \leq \infty,\ 
					\lambda = 1,\cdots,d)
				\end{align}
				を取る.$0 < \mu_0(I) < \infty$のとき,任意の$\epsilon > 0$に対し
				\begin{align}
					I_\epsilon \coloneqq 
					\left(\alpha_{1,\epsilon},\beta_{1,\epsilon}\right]
					\times \cdots \times
					\left(\alpha_{d,\epsilon},\beta_{d,\epsilon}\right],
					\quad (-\infty < \alpha_{\lambda,\epsilon} < \beta_{\lambda,\epsilon} < \infty),
					\quad I_\epsilon \subset I,
					\quad \mu(I \backslash I_\epsilon) < \epsilon
				\end{align}
				を満たす左半開区間$I_\epsilon$が存在し,
				\begin{align}
					I_\epsilon \subset 
					K_\epsilon \coloneqq \left[\alpha_{1,\epsilon},\beta_{1,\epsilon}\right]
					\times \cdots \times
					\left[\alpha_{d,\epsilon},\beta_{d,\epsilon}\right]
					\subset I
				\end{align}
				かつ$K_\epsilon$はコンパクト集合である.
				
				定理\ref{thm:compact_class_Haudorff}より
				$\R^d$のコンパクト集合全体はコンパクトクラスとなるから,
				定理\ref{thm:compact_class_intersection}より
				定理\ref{thm:equivalent_conditions_of_countable_additivity}の(a)が満たされる.
		\end{description}
	\end{prf}
	
	\begin{screen}
		\begin{dfn}[Lebesgue-Stieltjes測度]
			単調非減少関数の族$(f_\lambda)_{\lambda=1}^d$が全て右連続であれば
			(\refeq{eq:Lebesgue_Stieltjes_measure_on_Rd})の$\mu_0$は
			$\mathfrak{F}$の上で完全加法的となるから,定理\ref{thm:appendix_Kolmogorov_Hopf}より
			$(\R^d,\mathfrak{F},\mu_0)は$完備測度空間$(\R^d,\mathfrak{M},\mu^*)$に拡張される.
			この$\mu^*$を$(f_\lambda)_{\lambda=1}^d$の$d$次元Lebesgue-Stieltjes測度と呼び,
			特に$f_\lambda$が全て恒等写像の場合$d$次元Lebesgue測度と呼ぶ.
		\end{dfn}
	\end{screen}
	
	$f_\lambda$が全て右連続であれば
	定理\ref{thm:appendix_Kolmogorov_Hopf}より$\mu_0$は
	$(\R^d,\borel{\R^d})$の上の$\sigma$-有限測度$\mu$に一意に拡張され,このとき
	\begin{align}
		\left(\R^d,\overline{\borel{\R^d}},\overline{\mu}\right) 
		= \left(\R^d,\mathfrak{M},\mu^*\right)
	\end{align}
	が成立する.この拡張測度$\mu$を$(f_\lambda)_{\lambda=1}^d$のBorel-Stieltjes測度と呼ぶ.
	
\subsection{任意の区間上のStieltjes測度}
	$I_\lambda,\ (\lambda=1,\cdots,d)$を$\R$の区間,つまり
	$(a,b),(a,b],[a,b),[a,b],\ (-\infty \leq a \leq b \leq \infty)$のいずれかとするとき,
	\begin{align}
		I \coloneqq I_1 \times \cdots \times I_d
	\end{align}
	の形の集合$I$を$\R^d$の区間と呼ぶ.
	いま,各$\lambda=1,\cdots,d$に対し,$f_\lambda$を$I_\lambda$上で定義された右連続単調非減少な,
	ただし$I_\lambda$が有界なら$I_\lambda$上で有界な関数として
	\begin{align}
		a_\lambda \coloneqq \inf{}{\Set{f_\lambda(x)}{\inf{}{I_\lambda} < x < \sup{}{I_\lambda}}},
		\quad b_\lambda \coloneqq \sup{}{\Set{f_\lambda(x)}{\inf{}{I_\lambda} < x < \sup{}{I_\lambda}}}
	\end{align}
	とおけば,$\inf{}{I_\lambda} \in I_\lambda$なら$a_\lambda = f_\lambda(\inf{}{I_\lambda})$,
	$\sup{}{I} \in I$なら$b_\lambda = f(\sup{}{I_\lambda})$であるから
	\begin{align}
		\hat{f}_\lambda(x) \coloneqq 
		\begin{cases}
			a_\lambda & -\infty < x \leq \inf{}{I_\lambda} \\
			f_\lambda(x) & \inf{}{I_\lambda} < x < \sup{}{I_\lambda} \\
			b_\lambda & \sup{}{I_\lambda} \leq x < \infty
		\end{cases}
	\end{align}
	は$f_\lambda$の拡張となり,$\left( \hat{f}_\lambda \right)_{\lambda=1}^d$に対して
	Borel-Stieltjes測度空間$(\R^d,\borel{\R^d},\mu)$が定まる.
	定理\ref{thm:Borel_algebra_of_relative_topology}より
	\begin{align}
		\borel{I} = \Set{I \cap E}{E \in \borel{\R^d}} \subset \borel{\R^d}
	\end{align}
	が成り立つから,
	\begin{align}
		\mu_I(I \cap E) \coloneqq \mu(I \cap E),
		\quad (\forall E \in \borel{\R^d})
	\end{align}
	とおけば$(I,\borel{I},\mu_I)$は測度空間となる.この$\mu_I$もまた
	$(f_\lambda)_{\lambda=1}^d$のBorel-Stieltjes測度と呼ぶ.
	
	\begin{screen}
		\begin{thm}[Borel-Stiletjes測度の一意性]
			$f_\lambda$を区間$I_\lambda \subset \R$で定義された右連続な単調非減少関数,
			$\mu$を$(f_\lambda)_{\lambda=1}^d$のBorel-Stieltjes測度とするとき,
			任意の$(\alpha_1,\beta_1] \times \cdots \times (\alpha_d,\beta_d]
			,\ (-\infty < \alpha_\lambda < \beta_\lambda < \infty,\ (\alpha_\lambda,\beta_\lambda] \subset I_\lambda)$に対して
			\begin{align}
				\mu\left((\alpha_1,\beta_1] \times \cdots \times (\alpha_d,\beta_d]\right) 
				= \prod_{\lambda=1}^d \left\{ f_\lambda(\beta_\lambda) - f_\lambda(\alpha_\lambda) \right\}
				\label{eq:thm_uniqueness_of_Borel_Stieltjes_measure}
			\end{align}
			が満たされる.また$(f_\lambda)_{\lambda=1}^d$に対し(\refeq{eq:thm_uniqueness_of_Borel_Stieltjes_measure})を満たす
			$(I,\borel{I})$上の測度は唯一つである.
		\end{thm}
	\end{screen}
	
\subsection{Stieltjes積分}
	\begin{screen}
		\begin{thm}[Riemann-Stieltjes積分との関係]
			$F:I \longrightarrow \C$が右連続或は左連続なら
		\end{thm}
	\end{screen}
	
	\begin{screen}
		\begin{thm}[時間変更]
			
		\end{thm}
	\end{screen}
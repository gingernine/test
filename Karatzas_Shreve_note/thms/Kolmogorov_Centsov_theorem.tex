\section{Kolmogorov-\v{C}entsovの定理}
	\begin{screen}
		\begin{thm}
			The only $\borel{\R^{d,[0,\infty)}}$-measurable set contained 
			in $C[0,\infty)$ is the empty set.
		\end{thm}
	\end{screen}
	
	\begin{prf}
				点列$(t_i)_{i=1}^{\infty}\ (t_i \in [0,\infty))$と
				$A \in \borel{\R^d \times \R^d \times \cdots}$を用いて
				表される$\R^{d,[0,\infty)}$の部分集合
				\begin{align}
					E = \Set{\omega \in \R^{d,[0,\infty)}}{(\omega(t_1),\omega(t_2),\cdots) \in A}
					\label{eq:Kolmogorov_Centsov_1}
				\end{align}
				の全体を次でおく:
				\begin{align}
					\mathcal{E} \coloneqq
					\Set{E \subset \R^{d,[0,\infty)}}{
					\mbox{$E$ is in the form of (\refeq{eq:Kolmogorov_Centsov_1})}}.
				\end{align}
				このとき,$\mathcal{E}$は$\sigma$-加法族であり
				$\mathscr{C}$を含む.実際,任意の$C \in \mathscr{C}$は
				\begin{align}
					C = \Set{\omega \in \R^{d,[0,\infty)}}{(\omega(t_1),\cdots,\omega(t_n)) \in B},
					\quad (t_i \in [0,\infty),\ B \in \borel{(\R^d)^n})
				\end{align}
				と表されるから,$t_n = t_{n+1} = t_{n+2} = \cdots$として$C \in \mathcal{E}$が従う.
				また$E_i \in \mathcal{E}\ (i=1,2,\cdots)$に対して
				\begin{align}
					d
				\end{align}
				\QED
	\end{prf}
	
	
\subsection{Kolmogorovの連続変形定理}
	\begin{screen}
		\begin{thm}[Kolmogorovの連続変形定理]
			$(X,\mathscr{B})$を空でない可測空間,
			$\mu$をこの上の有限測度で$\mu(X) > 0$,
			$(S,d)$を完備距離空間,$T$を正数として,
			$0 \leq t \leq T$を満たす任意の実数$t$に対して
			$\borel{S}/\mathscr{B}$-可測写像$f_t:X \longrightarrow S$
			が定まっているとする.また$\alpha,\beta,C$を正数とする.このとき,
			\begin{align}
				\forall s,t \in [0,T],\quad
				\int_X d(f_s(x),f_t(x))^\alpha\ \mu(dx)
				\leq C |t-s|^{1+\beta}
			\end{align}
			が成り立っているならば,$0 < \gamma < \beta/\alpha$を満たす任意の実数$\gamma$
			に対して或る$\mu$-零集合$N$が取れて,$0 \leq t \leq T$を満たす任意の実数$t$に対して
			$\borel{S}/\mathscr{B}$-可測写像$g_t$が存在し,
			また或る$\mathscr{B}/\borel{\R}$-可測写像$h$も存在して,
			\begin{align}
				\forall t \in [0,T],\quad \mu(f_t \neq g_t) = 0
			\end{align}
			及び
			\begin{align}
				\forall x \in X \backslash N\ \forall s,t \in [0,T],\quad
				|t-s| < h(x) \Longrightarrow
				d(g_s(x),g_t(x)) < \frac{2}{1-2^{-\gamma}}|t-s|^\gamma
			\end{align}
			が成立する.
		\end{thm}
	\end{screen}
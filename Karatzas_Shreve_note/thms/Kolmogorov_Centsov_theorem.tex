\section{Kolmogorov-\v{C}entsovの定理}
	\begin{itembox}[l]{Exercise 2.7}
		The only $\borel{(\R^d)^{[0,\infty)}}$-measurable set contained 
		in $C[0,\infty)^d$ is the empty set.
	\end{itembox}
	
	\begin{prf}\mbox{}
		\begin{description}
			\item[第一段]
				$\borel{(\R^d)^{[0,\infty)}} = \sigma(B_t;\ 0 \leq t < \infty)$
				が成り立つことを示す.先ず,任意の$C \in \mathscr{C}$は
				\begin{align}
					C &= \Set{\omega \in (\R^d)^{[0,\infty)}}{(\omega(t_1),\cdots,\omega(t_n)) \in A} \\
					&=  \Set{\omega \in (\R^d)^{[0,\infty)}}{(B_{t_1}(\omega),\cdots,B_{t_n}(\omega)) \in A},
					\quad (A \in \borel{(\R^d)^n})
				\end{align}
				の形で表されるから$\mathscr{C} \subset \sigma(B_t;\ 0 \leq t < \infty)$
				が従い$\borel{(\R^d)^{[0,\infty)}} \subset \sigma(B_t;\ 0 \leq t < \infty)$
				を得る.逆に
				\begin{align}
					\sigma(B_t) \subset \mathscr{C},
					\quad (\forall t \geq 0)
				\end{align}
				より$\borel{(\R^d)^{[0,\infty)}} \supset \sigma(B_t;\ 0 \leq t < \infty)$
				も成立し$\borel{(\R^d)^{[0,\infty)}} = \sigma(B_t;\ 0 \leq t < \infty)$
				が出る.
				
			\item[第二段]
				高々可算集合$S = \{t_1,t_2,\cdots\} \subset [0,\infty)$に対して
				\begin{align}
					\mathcal{E}_S \coloneqq \Set{\Set{\omega \in (\R^d)^{[0,\infty)}}{(\omega(t_1),\omega(t_2),\cdots) \in A}}{A \in \borel{(\R^d)^{\# S}}}
				\end{align}
				とおけば
				\footnote{
					$S$が可算無限なら$(\R^d)^{\# S} = \R^\infty$.
				},座標過程$B$は
				$(\omega(t_1),\omega(t_2),\cdots) = (B_{t_1}(\omega),B_{t_2}(\omega),\cdots)$
				を満たすから
				\begin{align}
					\mathcal{E}_S = \Set{\left\{(B_{t_1},B_{t_2},\cdots) \in A\right\}}{A \in \borel{(\R^d)^{\# S}}} \eqqcolon \mathcal{F}^B_S
				\end{align}
				が成立する.従って第一章のLemma3 for Exercise 1.8と前段の結果より
				\begin{align}
					\borel{(\R^d)^{[0,\infty)}}
					&= \sigma(B_t;\ 0 \leq t < \infty)
					= \mathcal{F}^B_{[0,\infty)}
					= \bigcup_{S \subset [0,\infty):at\ most\ countable} \mathcal{F}^B_S\\
					&= \bigcup_{S \subset [0,\infty):at\ most\ countable} \mathcal{E}_S
				\end{align}
				を得る.すなわち,$\borel{(\R^d)^{[0,\infty)}}$の任意の元は
				$\Set{\omega \in (\R^d)^{[0,\infty)}}{(\omega(t_1),\omega(t_2),\cdots) \in A}$
				の形で表現され,$A \neq \emptyset$ならば
				$\Set{\omega \in (\R^d)^{[0,\infty)}}{(\omega(t_1),\omega(t_2),\cdots) \in A} \not\subset C[0,\infty)^d$となり主張が従う.
				\QED
		\end{description}
	\end{prf}
	
	\begin{itembox}[l]{Theorem 2.8 の主張は次のように変更するべきである:}
		Suppose that a process $X = \Set{X_t}{0 \leq t \leq T}$ on a probability space 
		$(\Omega,\mathscr{F},P)$ satisfies the condition
		\begin{align}
			E|X_t - X_s|^\alpha \leq C|t-s|^{1 + \beta},
			\quad 0 \leq s,t \leq T,
		\end{align}
		for some positive constants $\alpha,\beta$, and $C$. Then there exists a 
		continuous modification $\tilde{X} = \Set{\tilde{X}_t}{0 \leq t \leq T}$ of $X$, 
		which is locally H\Ddot{o}lder-continuous with exponent $\gamma$ for every 
		$\gamma \in (0,\beta/\alpha)$. More precisely, for every $\gamma \in (0,\beta/\alpha)$,
		\begin{align}
			\sup{\substack{0 < |t-s| < h(\omega) \\ s,t \in [0,T]}}{\frac{\left| \tilde{X}_t(\omega) - \tilde{X}_s(\omega) \right|}{|t-s|^\gamma}} \leq \frac{2}{1-2^{-\gamma}},
			\quad \forall \omega \in \Omega^*,
		\end{align}
		for some $\Omega^* \in \mathscr{F}$ with $P(\Omega^*)=1$ and 
		positive random variable $h$, where $\Omega^*$ and $h$ depend on $\gamma$.
	\end{itembox}
	
	なぜならば,式(2.8)において$P$の中身が$\Omega^*$に一致するかどうかわからないためである.
	可測集合でなければ$P$で測ることはできない.ただし
	今の場合は$(\Omega,\mathscr{F},P)$が完備確率空間ならば式(2.8)の表記で問題ない.
	
	
	\begin{itembox}[l]{Theorem 2.8 memo}
		証明中の式(2.10)直後の
		``where $n^*(\omega)$ is a positive, integer-valued random variable''
		について.
	\end{itembox}
	
	\begin{prf}
		$\N \coloneqq \{1,2,\cdots\}$とおき,
		$\N$の冪集合を$2^\N$で表せば,$(\N,2^\N)$は可測空間となる.
		示せばよいのは$n^*$の$\mathscr{F}/2^\N$-可測性である.
		ただし,$n^*$は証明文中においてwell-definedでないため,明確な意味を持たせる必要がある.
		\begin{align}
			A_0 \coloneqq \Omega,
			\quad A_n \coloneqq \Set{\omega \in \Omega}{\max{1 \leq k \leq 2^n}{\left| X_{k/2^n}(\omega) - X_{(k-1)/2^n}(\omega) \right|} \geq 2^{-\gamma n}},
			\quad (n=1,2,\cdots)
		\end{align}
		とおくとき,$\Omega^*$は
		\begin{align}
			\Omega^* \coloneqq \bigcup_{\ell =1}^{\infty} \bigcap_{n = \ell}^\infty A_n^c
		\end{align}
		により定まる集合である.
		任意の$\omega \in \Omega^*$に対して或る$\ell \geq 1$が存在し,
		\begin{align}
			\max{1 \leq k \leq 2^n}{\left| X_{k/2^n}(\omega) - X_{(k-1)/2^n}(\omega) \right|} < 2^{-\gamma n},
			\quad (\forall n \geq \ell)
		\end{align}
		を満たす.このような$\ell$のうち最小なものを$n^*(\omega)$と定めれば
		\begin{align}
			{n^*}^{-1}(\ell) = \left\{ \bigcap_{n = \ell}^\infty A_n^c \right\} \cap \left\{ \bigcap_{n = \ell-1}^\infty A_n^c \right\}^c,
			\quad (\ell =1,2,\cdots)
		\end{align}
		が成立し$n^*$の$\mathscr{F}/2^\N$-可測性が従う.
		\QED
	\end{prf}
	
	確率変数$h$について,厳密には
	\begin{align}
		h(\omega) \coloneqq 
		\begin{cases}
			2^{-n^*(\omega)}, & (\omega \in \Omega^*), \\
			0, & (\omega \in \Omega \backslash \Omega^*)
		\end{cases}
	\end{align}
	とおけばよい.
	
	\begin{itembox}[l]{Theorem 2.8 memo}
	\end{itembox}
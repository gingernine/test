\subsection{Kolmogorovの連続変形定理}
	\begin{screen}
		\begin{thm}[Kolmogorovの連続変形定理]
			$(X,\mathscr{F})$を空でない可測空間,
			$\mu$をこの上の有限測度で$\mu(X) > 0$,
			$(S,d)$を完備距離空間,
			$(T,\rho)$を全有界距離空間として,以下を仮定する:
			\begin{itemize}
				\item $T$の任意の要素$t$に対して$\borel{S}/\mathscr{F}$-可測写像
					$f_t:X \longrightarrow S$が定まっている.
					
				\item 正の実数$\alpha,\beta,C$に対して
					\begin{align}
						\forall s,t \in T,\quad
						\int_X d(f_s,f_t)^\alpha\ d\mu
						\leq C \rho(s,t)^{1+\beta}
					\end{align}
					が成立している.ただし$d(f_s,f_t)$とは写像$X \ni x \longmapsto d(f_s(x),f_t(x))$を表す.
				
				\item $1$以上の任意の自然数$n$に対して,$T$は半径$1/n$の球を
					適切に$n$個取れば覆われる.
			\end{itemize}
			このとき,$0 < \gamma < \beta/\alpha$を満たす任意の実数$\gamma$
			に対して或る$\mu$-零集合$N$が取れて,$T$の任意の要素$t$に対して
			$\borel{S}/\mathscr{F}$-可測写像$g_t$が存在し,$X$の任意の要素$x$で
			写像$T \ni t \longmapsto g_t(x)$は連続となり,かつ
			\begin{align}
				\forall t \in T,\quad \mu(f_t \neq g_t) = 0
			\end{align}
			が成立する.加えて或る$\mathscr{F}/\borel{\R}$-可測写像$h$も存在して,
			\begin{align}
				\forall x \in X \backslash N,\ \forall s,t \in [0,T],\quad
				\rho(s,t) < h(x) \Longrightarrow
				d(g_s(x),g_t(x)) < \frac{2}{1-2^{-\gamma}}\rho(s,t)^\gamma
			\end{align}
			が成立する.
		\end{thm}
	\end{screen}
	
	\begin{prf}
		考察中
	\end{prf}
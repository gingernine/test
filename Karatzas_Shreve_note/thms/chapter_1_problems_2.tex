\section{Stopping Times}
	\begin{itembox}[l]{$[0,\infty]$の位相}
		$[0,\infty]$の位相は拡張実数$[-\infty,\infty]$の相対位相である.
		$O \subset [-\infty,\infty]$が開集合であるとは,
		任意の$x \in O$に対し,
		\begin{description}
			\item[(O1)] $x \in \R$なら或る$\epsilon > 0$が存在して
				$B_\epsilon(x) \subset O$が満たされる,
			
			\item[(O2)] $x = \infty$なら或る$a \in \R$が存在して
				$(a,\infty] \subset O$が満たされる,
			
			\item[(O3)] $x = -\infty$なら或る$a \in \R$が存在して
				$[-\infty,a) \subset O$が満たされる,
		\end{description}
		で定義される.この性質を満たす$O$の全体に$\emptyset$を加えたものが
		$[-\infty,\infty]$の位相であり,
		\begin{align}
			[-\infty,r),\quad (r,r'), \quad (r,\infty],
			\quad (r,r' \in \Q)
		\end{align}
		の全体が可算開基となる.従って$[0,\infty]$の位相の可算開基は
		\begin{align}
			[0,r),\quad (r,r'), \quad (r,\infty],
			\quad (r,r' \in \Q \cap [0,\infty])
		\end{align}
		の全体であり,写像$\tau:\Omega \longrightarrow [0,\infty]$が
		$\mathscr{F}/\borel{[0,\infty]}$-可測性を持つかどうかを調べるには
		\begin{align}
			\{\tau < a\} = \tau^{-1}([0,a)) \in \mathscr{F},
			\quad (\forall a \in (0,\infty))
		\end{align}
		が満たされているかどうかを確認すれば十分である.
	\end{itembox}
	
	\begin{itembox}[l]{Problem 2.2}
		Let $X$ be a stochastic process and $T$ a stopping time of 
		$\left\{ \mathscr{F}^X_t \right\}$. Suppose that for some pair $\omega,\omega' \in \Omega$, 
		we have $X_t(\omega) = X_t(\omega')$ for all $t \in [0,T(\omega)] \cap [0,\infty)$. 
		Show that $T(\omega) = T(\omega')$. 
	\end{itembox}
	
	\begin{prf}[参照元:\cite{key3}]
		$\omega,\omega'$を分離しない集合族$\mathscr{H}$を
		\begin{align}
			\mathscr{H} \coloneqq \Set{A \subset \Omega}{\{\omega,\omega'\} \subset A,\ or\ \{\omega,\omega'\} \subset \Omega \backslash A}
		\end{align}
		により定めれば,$\mathscr{H}$は$\sigma$-加法族である.このとき,
		$\{T = T(\omega)\} \in \mathscr{H}$を示せばよい.
		\begin{description}
			\item[case1]
				$T(\omega) = \infty$の場合,
				任意の$A \in \borel{\R^d}$及び$0 \leq t < \infty$に対して,
				仮定より
				\begin{align}
					\omega \in X_t^{-1}(A) \quad \Leftrightarrow \quad
					\omega' \in X_t^{-1}(A)
				\end{align}
				が成り立ち
				\begin{align}
					\sigma(X_t;\ 0 \leq t < \infty) \subset \mathscr{H}
				\end{align}
				となる.任意の$t \geq 0$に対し$\{T \leq t\} \in \mathscr{F}^X_t \subset 
				\sigma(X_t;\ 0 \leq t < \infty)$が満たされるから
				\begin{align}
					\{T = \infty\} = \bigcap_{n=1}^\infty \{T \leq n\}^c
					\in \sigma(X_t;\ 0 \leq t < \infty) \subset \mathscr{H}
				\end{align}
				が成立し,$\omega \in \{T = \infty\}$より$\omega' \in \{T = \infty\}$が従い
				$T(\omega) = T(\omega')$を得る.
				
			\item[case2]
				$T(\omega) < \infty$の場合,
				case1と同様に任意の$0 \leq t \leq T(\omega)$に対し
				$\sigma(X_t) \subset \mathscr{H}$が満たされるから
				\begin{align}
					\mathscr{F}^X_{T(\omega)} \subset \mathscr{H}
				\end{align}
				が成り立つ.$\{T = T(\omega)\} \in \mathscr{F}^X_{T(\omega)}$より
				$\omega' \in \{T = T(\omega)\}$が従い$T(\omega) = T(\omega')$を得る.
				\QED
		\end{description}
	\end{prf}
	
	\begin{itembox}[l]{Lemma for Proposition 2.3}
		$(\mathscr{F}_t)_{t \geq 0}$を可測空間
		$(\Omega,\mathscr{F})$のフィルトレーションとするとき,
		任意の$t \geq 0$及び任意の点列$s_1  > s_2 > \cdots > t, (s_n \downarrow t)$
		に対して次が成立する:
		\begin{align}
			\bigcap_{s>t} \mathscr{F}_s = \bigcap_{n=1}^\infty \mathscr{F}_{s_n}.
		\end{align}
	\end{itembox}
	
	\begin{prf}
		先ず任意の$n \geq 1$に対して
		\begin{align}
			\bigcap_{s > t} \mathscr{F}_s \subset \mathscr{F}_{s_n}
		\end{align}
		が成り立つから
		\begin{align}
			\bigcap_{s > t} \mathscr{F}_s \subset \bigcap_{n=1}^\infty \mathscr{F}_{s_n}
		\end{align}
		を得る.一方で,任意の$s > t$に対し$s \geq s_n$を満たす$n$が存在するから,
		\begin{align}
			\mathscr{F}_s \supset  \mathscr{F}_{s_n}
			\supset \bigcap_{n=1}^\infty \mathscr{F}_{s_n}
		\end{align}
		が成立し
		\begin{align}
			\bigcap_{s > t} \mathscr{F}_s \supset \bigcap_{n=1}^\infty \mathscr{F}_{s_n}
		\end{align}
		が従う.
		\QED
	\end{prf}
	
	$(\mathscr{F}_{t+})_{t \geq 0}$は右連続である.実際,任意の$t \geq 0$で
	\begin{align}
		\bigcap_{s > t} \mathscr{F}_{s+} = \bigcap_{s > t} \bigcap_{u > s} \mathscr{F}_u
		= \bigcap_{s > t} \mathscr{F}_s
		= \mathscr{F}_{t+}
	\end{align}
	が成立する.
	
	\begin{itembox}[l]{Corollary 2.4\footnotemark}
		$T$ is an optional time of the filtration $\{\mathscr{F}_t\}$ if and only if 
		it is a stopping time of the (right-continuous!) filtration $\{\mathscr{F}_{t+}\}$.
	\end{itembox}
	\footnotetext{
		optional time の訳語がわからないので弱停止時刻と呼ぶ.
	}
	言い換えれば,確率時刻$T$に対し
	\begin{align}
		\{T < t\} \in \mathscr{F}_t,\ \forall t \geq 0
		\quad \Leftrightarrow \quad
		\{T \leq t\} \in \mathscr{F}_{t+},\ \forall t \geq 0
	\end{align}
	が成り立つことを主張している.
	\begin{prf}
		$T$が$(\mathscr{F}_{t+})$-停止時刻であるとき,
		任意の$n \geq 1$に対して$\{T \leq t - 1/n\} \in \mathscr{F}_{(t-1/n)+} \subset \mathscr{F}_t$
		が満たされるから
		\begin{align}
			\{T < t\} = \bigcup_{n=1}^\infty \left\{T \leq t - \frac{1}{n}\right\} \in \mathscr{F}_t
		\end{align}
		が従う.逆に$T$が$(\mathscr{F}_t)$-弱停止時刻のとき,任意の$m \geq 1$に対し
		\begin{align}
			\{T \leq t\} = \bigcap_{n=m}^\infty \left\{T < t+\frac{1}{n} \right\}
			\in \mathscr{F}_{t + 1/m}
		\end{align}
		が成立するから
		\begin{align}
			\{T \leq t\} \in \bigcap_{n=1}^\infty \mathscr{F}_{t + 1/n} = \mathscr{F}_{t+}
		\end{align}
		を得る.
		\QED
	\end{prf}
	
	\begin{itembox}[l]{Problem 2.6}
		If the set $\Gamma$ in Example 2.5 is open, show that $H_\Gamma$ is 
		an optional time.
	\end{itembox}
	
	\begin{prf}
		$\{H_\Gamma < 0\}=\emptyset$であるから,以下$t > 0$とする.
		$H_\Gamma(\omega) < t \Leftrightarrow \exists s < t,\ X_s(\omega) \in \Gamma$より
		\begin{align}
			\{H_\Gamma < t\} = \bigcup_{0 \leq s < t} \{X_s \in \Gamma\}
		\end{align}
		となる.また全てのパスが右連続であることと$\Gamma$が開集合であることにより
		\begin{align}
			\bigcup_{0 \leq s < t} \{X_s \in \Gamma\}
			= \bigcup_{\substack{0 \leq r < t \\ r \in \Q}} \{X_r \in \Gamma\}
		\end{align}
		が成り立ち$\{H_\Gamma < t\} \in \mathscr{F}_t$が従う.
		\QED
	\end{prf}
	
	\begin{itembox}[l]{Problem 2.7}
		If the set $\Gamma$ in Example 2.5 is closed and the sample paths of the 
		process $X$ are continuous, then $H_\Gamma$ is a stopping time.
	\end{itembox}
	
	\begin{prf}\mbox{}
		\begin{description}
			\item[第一段]
				$\R^d$上のEuclid距離を$\rho$で表し,
				\begin{align}
					\rho(x,\Gamma) \coloneqq \inf{y \in \Gamma}{\rho(x, y)},
					\quad \Gamma_n \coloneqq \Set{x \in \R^d}{\rho(x,\Gamma) < \frac{1}{n}},
					\quad (x \in \R^d,\ n=1,2,\cdots)
				\end{align}
				とおく.$\R^d \ni x \longmapsto \rho(x,\Gamma)$の連続性より$\Gamma_n$は開集合であるから,
				Problem 2.6の結果より$T_n \coloneqq H_{\Gamma_n}$で定める$T_n,\ n=1,2,\cdots$は
				$(\mathscr{F}_t)$-弱停止時刻であり,
				また$H \coloneqq H_\Gamma$とおけば次の(1)と(2)が成立する:
				\begin{description}
					\setlength{\leftskip}{3.0cm}
					\item[(1)] $\{H = 0\} = \{X_0 \in \Gamma\}$,
					
					\setlength{\leftskip}{3.0cm}
					\item[(2)] $H(\omega) \leq t 
					\quad \Leftrightarrow \quad 
					T_n(\omega) < t,\ \forall n=1,2,\cdots,
					\quad (\forall \omega \in \{H>0\},\ \forall t>0)$.
				\end{description}
				(1)と(2)及び$T_n,\ n=1,2,\cdots$が$(\mathscr{F}_t)$-弱停止時刻であることにより
				\begin{align}
					\{H \leq t\}
					= \{H \leq t\} \cap \{H > 0\} + \{H = 0\}
					= \left\{ \bigcap_{n=1}^\infty \{T_n < t\} \right\} \cap \{H > 0\} + \{H = 0\}
					\in \mathscr{F}_t,
					\quad (\forall t \geq 0)
				\end{align}
				が成立するから$H$は$(\mathscr{F}_t)$-停止時刻である.
			
			\item[第二段]
				(1)を示す.実際,
				$X_0(\omega) \in \Gamma$なら$H(\omega) = 0$であり,
				$X_0(\omega) \notin \Gamma$なら,$\Gamma$が閉であることと
				パスの連続性より
				\begin{align}
					X_t(\omega) \notin \Gamma,
					\quad (0 \leq t \leq h)
				\end{align}
				を満たす$h > 0$が存在して$H(\omega) \geq h > 0$となる.
		
			\item[第三段]
				$\omega \in \{H>0\},\ t > 0$として(2)を示す.まずパスの連続性より
				\begin{align}
					T_n(\omega) < t \quad \Leftrightarrow \quad
					\exists s \leq t, \quad X_s(\omega) \in \Gamma_n
				\end{align}
				が成り立つ.$H(\omega) \leq t$の場合,
				$\beta \coloneqq H(\omega)$とおけば,$\Gamma$が閉であることと
				パスの連続性より
				\begin{align}
					X_\beta(\omega) \in \Gamma \subset \Gamma_n,
					\quad (\forall n=1,2,\cdots)
				\end{align}
				が満たされ$T_n(\omega) < t\ (\forall n \geq 1)$が従う.
				逆に,$H(\omega) > t$のとき
				\begin{align}
					X_s(\omega) \notin \Gamma,
					\quad (\forall s \in [0,t])
				\end{align}
				が満たされ,パスの連続性と$\rho$の連続性より
				$[0,t] \ni s \longmapsto \rho(X_s(\omega),\Gamma)$
				は連続であるから,
				\begin{align}
					d \coloneqq \min{s \in [0,t]}{\rho(X_s(\omega),\Gamma)} > 0
				\end{align}
				が定まる.このとき$1/n < d/2$を満たす$n \geq 1$を一つ取れば
				\begin{align}
					X_s(\omega) \notin \Gamma_n,
					\quad (\forall s \in [0,t])
				\end{align}
				が成立する.実際,任意の$s \in [0,t],\ x \in \Gamma_n$に対し
				\begin{align}
					\rho(X_s(\omega),x)
					\geq \rho(X_s(\omega),\Gamma) - \rho(x,\Gamma)
					\geq d - \frac{d}{2}
					= \frac{d}{2}
					> \frac{1}{n}
				\end{align}
				が満たされる.従って$T_n(\omega) \geq t$となる.
				\QED
		\end{description}
	\end{prf}
	
	\begin{itembox}[l]{Problem 2.17 修正}
		Let $T,S$ be stopping times and $Z$ an $\mathscr{F}/\borel{\R}$-measurable, 
		integrable random variable. We have
		\begin{description}
			\item[(i)] $\defunc_{\{T \leq S\}} \cexp{Z}{\mathscr{F}_T} = \defunc_{\{T \leq S\}} \cexp{Z}{\mathscr{F}_{S \wedge T}},\ \mbox{$P$-a.s.}$
			\item[(ii)] $\defunc_{\{T < S\}} \cexp{Z}{\mathscr{F}_T} = \defunc_{\{T < S\}} \cexp{Z}{\mathscr{F}_{S \wedge T}},\ \mbox{$P$-a.s.}$
			\item[(iii)] $\cexp{\cexp{Z}{\mathscr{F}_T}}{\mathscr{F}_S} = \cexp{Z}{\mathscr{F}_{S \wedge T}},\ \mbox{$P$-a.s.}$
		\end{description}
	\end{itembox}
	
	\begin{prf}\mbox{}
		\begin{description}
			\item[第一段]
				任意の$A \in \mathscr{F}_T$に対し$A \cap \{T \leq S\} \in \mathscr{F}_{S \wedge T}$
				が成り立つ.実際,
				\begin{align}
					A \cap \{T \leq S\} \cap \{S \wedge T \leq t\}
					= \biggl[ A \cap \{T \leq t\} \biggr] \cap \{T \leq S\} \cap \{S \wedge T \leq t\}
					\in \mathscr{F}_t,
					\quad (\forall t \geq 0)
				\end{align}
				が成立する.同様に$A \cap \{T < S\} \in \mathscr{F}_{S \wedge T}$も得られる.
				
			\item[第二段]
				任意の$A \in \mathscr{F}_T$に対し,前段の結果より
				\begin{align}
					\int_A \defunc_{\{T \leq S\}} \cexp{Z}{\mathscr{F}_T}\ dP
					= \int_{A \cap \{T \leq S\}} Z\ dP
					= \int_{A \cap \{T \leq S\}} \cexp{Z}{\mathscr{F}_{S \wedge T}}\ dP
					= \int_A \defunc_{\{T \leq S\}} \cexp{Z}{\mathscr{F}_{S \wedge T}}\ dP
				\end{align}
				が従う.$\defunc_{\{T \leq S\}} \cexp{Z}{\mathscr{F}_{S \wedge T}}$
				も$\mathscr{F}_T/\borel{\R}$-可測であるから(i)が得られ,同様に(ii)も出る.
			
			\item[第三段]
				任意の$B \in \mathscr{F}_S$に対し,第一段と第二段の結果により
				\begin{align}
					\int_B \cexp{\cexp{Z}{\mathscr{F}_T}}{\mathscr{F}_S}\ dP
					&= \int_B \cexp{Z}{\mathscr{F}_T}\ dP
					= \int_B \defunc_{\{S < T\}} \cexp{Z}{\mathscr{F}_T}\ dP
						+ \int_B \defunc_{\{T \leq S\}} \cexp{Z}{\mathscr{F}_T}\ dP \\
					&= \int_{B \cap \{S < T\}} Z\ dP
						+ \int_B \defunc_{\{T \leq S\}} \cexp{Z}{\mathscr{F}_{S \wedge T}}\ dP \\
					&= \int_{B \cap \{S < T\}} \cexp{Z}{\mathscr{F}_{S \wedge T}}\ dP
						+ \int_B \defunc_{\{T \leq S\}} \cexp{Z}{\mathscr{F}_{S \wedge T}}\ dP \\
					&= \int_B \cexp{Z}{\mathscr{F}_{S \wedge T}}\ dP
				\end{align}
				が成り立つ.$\cexp{Z}{\mathscr{F}_{S \wedge T}}$も$\mathscr{F}_S/\borel{\R}$-可測
				であるから(iii)を得る.
				\QED
		\end{description}
	\end{prf}
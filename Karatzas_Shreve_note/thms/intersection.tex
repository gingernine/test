\section{交叉}
	\begin{screen}
		\begin{dfn}[交叉]
			$a$を類とするとき,
			\begin{align}
				\bigcap a \coloneqq \Set{x}{\forall t \in a\, (\, x \in t\, )}
			\end{align}
			で$\bigcap a$を定め,これを$a$の{\bf 交叉}\index{こうさ@交叉}{\bf (intersection)}と呼ぶ.
		\end{dfn}
	\end{screen}
	
	\begin{screen}
		\begin{thm}[空集合の交叉は宇宙となる]\label{thm:union_of_the_emptyset_is_the_Universe}
			次が成立する:
			\begin{align}
				\bigcap \emptyset = \Univ.
			\end{align}
		\end{thm}
	\end{screen}
	
	\begin{prf}
		$x$を$\mathcal{L}$の任意の対象とするとき,空虚な真より
		\begin{align}
			t \in \emptyset \Longrightarrow x \in t
		\end{align}
		は$\mathcal{L}$のいかなる対象$t$に対してもに真となる.ゆえに$\forall t \in \emptyset\ (\ x \in t\ )$が成立し
		\begin{align}
			\forall x\ (\ x \in \bigcap \emptyset\ )
		\end{align}
		が従う.$\forall x\ (\ x \in \Univ\ )$と併せて$\bigcap \emptyset = \Univ$を得る.
		\QED
	\end{prf}
	
	\monologue{
		院生「$\bigcup \emptyset$が$\emptyset$に等しいのは受け容れられますが,
			$\bigcap \emptyset$が$\Univ$に等しいというのは不思議に感じられます.」
	}
	
	\begin{screen}
		\begin{thm}[交叉は任意の元に含まれる]
			$a$を類とするとき次が成立する:
			\begin{align}
				\forall x\, (\, x \in a \Longrightarrow \bigcap a \subset x\, ).
			\end{align}
		\end{thm}
	\end{screen}
	
	$a,b$を類とするとき,その対の合併を
	\begin{align}
		a \cap b \coloneqq \bigcap \{a,b\}
	\end{align}
	と書く.
	
	\begin{screen}
		\begin{thm}
			\begin{align}
				\forall x\, (\, x \in a \cap b \Longleftrightarrow x \in a \wedge x \in b\, ).
			\end{align}
		\end{thm}
	\end{screen}
	
	\begin{screen}
		\begin{thm}[交叉の可換律]
			\begin{align}
				a \cap b = b \cap a.
			\end{align}
		\end{thm}
	\end{screen}
	
	\begin{screen}
		\begin{thm}[対の交叉が空ならばその構成要素は共通元を持たない]
		\label{thm:if_pair_is_empty_then_its_members_do_not_intersect}
			$a,b$を類とするとき次が成立する:
			\begin{align}
				a \cap b = \emptyset \Longrightarrow \forall x\, (\, x \in a \Longrightarrow x \notin b\, ).
			\end{align}
		\end{thm}
	\end{screen}
	
	\begin{prf}
		いま$\exists x\, (\, x \in a \wedge x \in b\, )$が成り立っているとする.
		\begin{align}
			\chi \coloneqq \varepsilon x\, (\, x \in a \wedge x \in b\, )
		\end{align}
		とおけば,存在記号に関する規則より
		\begin{align}
			\chi \in a \wedge \chi \in b
		\end{align}
		が成立し,$\wedge$の除去により$\chi \in a$と$\chi \in b$が共に成立する.
		ここで$\tau$を$\mathcal{L}$の任意の対象とすれば
		定理\ref{thm:pair_members_are_exactly_the_given_two}より
		\begin{align}
			\tau \in \{a,b\} \Longrightarrow \tau = a \vee \tau = b
		\end{align}
		が成立し,他方で相等性の公理より$\tau = a \Longrightarrow \chi \in \tau$と
		$\tau = b \Longrightarrow \chi \in \tau$が成り立つので場合分け法則から
		\begin{align}
			\tau = a \vee \tau = b \Longrightarrow \chi \in \tau
		\end{align}
		も満たされ,含意の推移律より
		\begin{align}
			\tau \in \{a,b\} \Longrightarrow \chi \in \tau
		\end{align}
		が従う.そして$\tau$の任意性と推論法則\ref{metathm:fundamental_law_of_universal_quantifier}から
		\begin{align}
			\forall t\, \left(\, t \in \{a,b\} \Longrightarrow \chi \in t\, \right)
		\end{align}
		が成立する.これにより
		\begin{align}
			\chi \in a \cap b
		\end{align}
		が従い,存在記号に関する規則より
		\begin{align}
			\exists x\, (\, x \in a \cap b\, )
		\end{align}
		となるから定理\ref{thm:uniqueness_of_emptyset}より
		\begin{align}
			a \cap b \neq \emptyset
		\end{align}
		が従う.ここに再び演繹法則を適用すれば
		\begin{align}
			\exists x\, (\, x \in a \wedge x \in b\, ) \Longrightarrow a \cap b \neq \emptyset
		\end{align}
		が得られる.ところで$\sigma$を$\mathcal{L}$の任意の対象とすれば,De Morganの法則と
		推論法則\ref{metathm:rule_of_inference_3}より
		\begin{align}
			\sigma \in a \wedge \sigma \in b 
			&\Longleftrightarrow\ \rightharpoondown (\, \sigma \notin a \vee \sigma \notin b\, ) \\
			&\Longleftrightarrow\ \rightharpoondown (\, \sigma \in a \Longrightarrow \sigma \notin b\, )
		\end{align}
		が成り立つので含意の推移律から
		\begin{align}
			\sigma \in a \wedge \sigma \in b \Longleftrightarrow\
			\rightharpoondown (\, \sigma \in a \Longrightarrow \sigma \notin b\, )
		\end{align}
		となり,推論法則\ref{metathm:properties_of_quantifiers}より
		\begin{align}
			\exists x\, (\, x \in a \wedge x \in b\, ) \Longleftrightarrow
			\exists x\, \rightharpoondown (\, x \in a \Longrightarrow x \notin b\, )
		\end{align}
		が成立する.よって含意の推移律より
		\begin{align}
			\exists x\, \rightharpoondown (\, x \in a \Longrightarrow x \notin b\, )
			\Longrightarrow a \cap b \neq \emptyset
		\end{align}
		が成立し,この対偶を取り推論法則\ref{metathm:properties_of_quantifiers}を適用すれば
		\begin{align}
			a \cap b = \emptyset \Longrightarrow \forall x\, (\, x \in a \Longrightarrow x \notin b\, )
		\end{align}
		が出てくる.
		\QED
	\end{prf}
	
	\begin{screen}
		\begin{thm}
			
		\end{thm}
	\end{screen}
	
	\begin{prf}\mbox{}
		\begin{description}
			\item[(1)] $a^{-1}$の任意の要素$t$に対し或る$V$の要素$x,y$が存在して
				\begin{align}
					(x,y) \in a \wedge t = (y,x)
				\end{align}
				を満たす.$((x,y),(y,x)) \in f$より$((x,y),t) \in f$が成り立つから
				$t \in f \ast a$となる.逆に$f \ast a$の任意の要素$t$に対して
				$a$の或る要素$x$が存在して
				\begin{align}
					x \in a \wedge (x,t) \in f
				\end{align}
				となる.$x$に対し$V$の或る要素$a,b$が存在して$x=(a,b)$となるので
				\begin{align}
					((a,b),t) \in f
				\end{align}
				となり,$V$の或る要素$c,d$が存在して
				\begin{align}
					((a,b),t) = ((c,d),(d,c))
				\end{align}
				となる.$(a,b) = (c,d)$より$a=c$かつ$b=d$となり,
				$t = (d,c)$かつ$(d,c)=(b,a)$より$t=(b,a)$,従って
				$t \in a^{-1}$が成り立つ.
		\end{description}
	\end{prf}
\section{交叉}
	交叉とは合併の対となる概念である.$a$を類とするとき,$a$の全ての要素が共通して持つ集合の全体を$a$の交叉と呼び,
	合併の記号を上下に反転させて
	\begin{align}
		\bigcap a
	\end{align}
	と書く.またいささか奇妙な結果であるが,空虚な真の為せる業により空の交叉は宇宙に一致する.
	
	\begin{screen}
		\begin{dfn}[交叉]
			$a$を類とするとき,$a$の{\bf 交叉}\index{こうさ@交叉}{\bf (intersection)}を
			\begin{align}
				\bigcap a \defeq \Set{x}{\forall t \in a\, (\, x \in t\, )}
			\end{align}
			で定める.
		\end{dfn}
	\end{screen}
	
	上の定義に現れた
	\begin{align}
		\forall t \in a\, (\, x \in t\, )
	\end{align}
	とは
	\begin{align}
		\forall t\, (\, t \in a \Longrightarrow x \in t\, )
	\end{align}
	を略記した式である.
	
	\begin{screen}
		\begin{thm}[空集合の交叉は宇宙となる]\label{thm:union_of_the_emptyset_is_the_Universe}
			次が成立する:
			\begin{align}
				\bigcap \emptyset = \Univ.
			\end{align}
		\end{thm}
	\end{screen}
	
	\begin{prf}
		$x$を$\mathcal{L}$の任意の対象とするとき,空虚な真より
		\begin{align}
			t \in \emptyset \Longrightarrow x \in t
		\end{align}
		は$\mathcal{L}$のいかなる対象$t$に対してもに真となる.ゆえに
		\begin{align}
			\forall t \in \emptyset\, (\, x \in t\, )
		\end{align}
		が成立し
		\begin{align}
			\forall x\, (\, x \in \bigcap \emptyset\, )
		\end{align}
		が従う.
		\begin{align}
			\forall x\, (\, x \in \Univ\, )
		\end{align}
		も成り立つから
		\begin{align}
			\forall x\, (\, x \in \Univ \Longleftrightarrow x \in \bigcap \emptyset\, )
		\end{align}
		が成立して,外延性の公理より
		\begin{align}
			\bigcap \emptyset = \Univ
		\end{align}
		が従う.
		\QED
	\end{prf}
	
	\begin{screen}
		\begin{thm}[交叉は全ての要素に含まれる]
		\label{thm:intersection_is_obtained_by_all_elements}
			$a$を類とするとき
			\begin{align}
				\forall x\, (\, x \in a \Longrightarrow \bigcap a \subset x\, ).
			\end{align}
		\end{thm}
	\end{screen}
	
	\begin{screen}
		\begin{thm}[全ての要素に共通して含まれる類は交叉にも含まれる]
		\label{thm:if_obtained_by_all_elements_then_obtained_by_intersection}
			$a$と$b$を類とするとき
			\begin{align}
				\forall x \in a\, (\, b \subset x\, ) \Longrightarrow b \subset \bigcap a.
			\end{align}
		\end{thm}
	\end{screen}
	
	\begin{screen}
		\begin{thm}[等しい類の交叉は等しい]\label{thm:intersections_of_equal_classes_are_equal}
			$a$と$b$を類とするとき
			\begin{align}
				a = b \Longrightarrow \bigcap a = \bigcap b.
			\end{align}
		\end{thm}
	\end{screen}
	
	\begin{itembox}[l]{対の交叉}
		$a$と$b$を類とするとき,その対の交叉を
		\begin{align}
			a \cap b \defeq \bigcap \{a,b\}
		\end{align}
		と書く.
	\end{itembox}
	
	\begin{screen}
		\begin{thm}
			\begin{align}
				\forall x\, (\, x \in a \cap b \Longleftrightarrow x \in a \wedge x \in b\, ).
			\end{align}
		\end{thm}
	\end{screen}
	
	\begin{screen}
		\begin{thm}[交叉の可換律]
			\begin{align}
				a \cap b = b \cap a.
			\end{align}
		\end{thm}
	\end{screen}
	
	\begin{screen}
		\begin{thm}[対の交叉が空ならばその構成要素は共通元を持たない]
		\label{thm:if_pair_is_empty_then_its_members_do_not_intersect}
			$a,b$を類とするとき次が成立する:
			\begin{align}
				a \cap b = \emptyset \Longleftrightarrow \forall x\, (\, x \in a \Longrightarrow x \notin b\, ).
			\end{align}
		\end{thm}
	\end{screen}
	
	\begin{sketch}
		定理\ref{thm:uniqueness_of_emptyset}より
		\begin{align}
			a \cap b = \emptyset \Longleftrightarrow \forall x\, \left(\, x \notin a \cap b\, \right)
		\end{align}
		が成立する.また
		\begin{align}
			\forall x\, \left(\, x \notin a \cap b \Longleftrightarrow x \notin a \vee x \notin b\, \right)
		\end{align}
		かつ
		\begin{align}
			\forall x\, \left(\, (\, x \notin a \vee x \notin b\, ) \Longleftrightarrow (\, x \in a \Longrightarrow x \notin b\, )\, \right)
		\end{align}
		が成り立つので
		\begin{align}
			\forall x\, \left(\, x \notin a \cap b \Longleftrightarrow (\, x \in a \Longrightarrow x \notin b\, )\, \right)
		\end{align}
		が成立し,
		\begin{align}
			\forall x\, \left(\, x \notin a \cap b\, \right) \Longleftrightarrow 	
			\forall x\, (\, x \in a \Longrightarrow x \notin b\, )
		\end{align}
		が従う.ゆえに
		\begin{align}
			a \cap b = \emptyset \Longleftrightarrow \forall x\, (\, x \in a \Longrightarrow x \notin b\, ).
		\end{align}
		が得られる.
		\QED
	\end{sketch}
	
	\begin{screen}
		\begin{dfn}[差類]
			$a,b$を類するとき,$a$に属するが$b$には属さない集合の全体を
			$a$から$b$を引いた{\bf 差類}\index{さるい@差類}
			{\bf (class difference)}と呼び,記号は
			\begin{align}
				a \backslash b \defeq \Set{x}{x \in a \wedge x \notin b}
			\end{align}
			で定める.特に$a \backslash b$が集合であるときこれを
			{\bf 差集合}\index{さしゅうごう@差集合}{\bf (set difference)}と呼ぶ.
			また
			\begin{align}
				b \subset a
			\end{align}
			である場合,$a \backslash b$を$a$における$b$の{\bf 補類}\index{ほるい@補類}{\bf (complement)}或いは
			$a \backslash b$が集合であるとき{\bf 補集合}\index{ほしゅうごう@補集合}と呼ぶ.
		\end{dfn}
	\end{screen}
	
	$\set{a} \Longrightarrow \set{a \backslash b}$
	
	\begin{screen}
		\begin{thm}
			$a$と$b$を類とするとき,
			\begin{align}
				b \subset a
			\end{align}
			であれば
			\begin{align}
				\set{a \backslash b} \wedge \set{b} \Longrightarrow \set{a}.
			\end{align}
		\end{thm}
	\end{screen}
	
	\begin{sketch}
		対の公理から
		\begin{align}
			\{a \backslash b,b\}
		\end{align}
		は集合であり,合併の公理と
		\begin{align}
			a = (a \backslash b) \cup b
		\end{align}
		より
		\begin{align}
			\set{a}
		\end{align}
		が従う.
		\QED
	\end{sketch}
	
	\begin{screen}
		\begin{thm}[合併を引いた類は要素の差の交叉で書ける]
		\label{thm:difference_of_union_is_intersection_of_differences_of_elements}
			$a$と$b$を類とするとき,$a$が集合であれば
			\begin{align}
				a \backslash \bigcup b = \bigcap \Set{a \backslash t}{t \in b}.
			\end{align}
		\end{thm}
	\end{screen}
	
	\monologue{
		上の定理の式で
		\begin{align}
			\Set{a \backslash t}{t \in b}
		\end{align}
		と書いていますが,これは
		\begin{align}
			\Set{x}{\exists t \in b\, (\, x=a \backslash t\, )}
		\end{align}
		の略記です.ところがこれもまだ略記されたもので,正しく書くと
		\begin{align}
			\Set{x}{\exists t \in b\, 
			\forall s\, (\, s \in x \Longleftrightarrow s \in a \wedge s \notin t\, )}
		\end{align}
		となります.以降も煩雑さを避けるためにこのように略記します.
	}
	
	\begin{screen}
		\begin{thm}[二つの類の合併の差類は差類同士の交叉]
		\label{thm:difference_of_union_of_two_classes_is_intersection_of_two_differences}
			$a$と$b$と$c$を類とするとき
			\begin{align}
				a \backslash (b \cup c) = (a \backslash b) \cap (a \backslash c).
			\end{align}
		\end{thm}
	\end{screen}
	
	\begin{screen}
		\begin{thm}[交叉を引いた類は要素の差の合併で書ける]
		\label{thm:difference_of_intersection_is_union_of_differences_of_elements}
			$a$と$b$を類とするとき
			\begin{align}
				a \backslash \bigcap b = \bigcup \Set{a \backslash t}{t \in b}.
			\end{align}
		\end{thm}
	\end{screen}
	
	\begin{screen}
		\begin{thm}[二つの類の交叉の差類は差類同士の合併]
		\label{thm:difference_of_intersection_of_two_classes_is_union_of_two_differences}
			$a$と$b$と$c$を類とするとき
			\begin{align}
				a \backslash (b \cap c) = (a \backslash b) \cup (a \backslash c).
			\end{align}
		\end{thm}
	\end{screen}
	
	\begin{screen}
		\begin{thm}
			
		\end{thm}
	\end{screen}
	
	\begin{prf}\mbox{}
		\begin{description}
			\item[(1)] $a^{-1}$の任意の要素$t$に対し或る$V$の要素$x,y$が存在して
				\begin{align}
					(x,y) \in a \wedge t = (y,x)
				\end{align}
				を満たす.$((x,y),(y,x)) \in f$より$((x,y),t) \in f$が成り立つから
				$t \in f \ast a$となる.逆に$f \ast a$の任意の要素$t$に対して
				$a$の或る要素$x$が存在して
				\begin{align}
					x \in a \wedge (x,t) \in f
				\end{align}
				となる.$x$に対し$V$の或る要素$a,b$が存在して$x=(a,b)$となるので
				\begin{align}
					((a,b),t) \in f
				\end{align}
				となり,$V$の或る要素$c,d$が存在して
				\begin{align}
					((a,b),t) = ((c,d),(d,c))
				\end{align}
				となる.$(a,b) = (c,d)$より$a=c$かつ$b=d$となり,
				$t = (d,c)$かつ$(d,c)=(b,a)$より$t=(b,a)$,従って
				$t \in a^{-1}$が成り立つ.
		\end{description}
	\end{prf}
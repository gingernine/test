\section{有理数}
	\begin{screen}
		\begin{thm}[分数体]\label{thm:field_of_fractions}
			環$R$に対し,$R$が整域であるということと$R$が或る体の部分環であるということは同値である.
			$R$を整域とするとき,$R$を部分環として含む最小の体は$R$の{\bf 分数体}
			\index{ぶんすうたい@分数体}{\bf (field of fractions)}と呼ばれる.
		\end{thm}
	\end{screen}
	
	$\Z$は整域であるから,定理\ref{thm:field_of_fractions}より$\Z$を部分環として含む
	体$F$が存在する.$\Z$の任意の要素$n$に対し,$n$が$0$でなければ$F$の中に$n^{-1}$が存在するが,
	この乗法に関する逆元を用いれば$\Z$を部分環として含む最小の体は
	\begin{align}
		\Set{x}{\exists n,m \in \Z\ (\ x = n \cdot m^{-1} \wedge m \neq 0\ )}
	\end{align}
	と書ける.この集合を$\Q$で表し,{\bf 有理数体}\index{ゆうりすうたい@有理数体}{\bf (field of rationals)}と呼ぶ.

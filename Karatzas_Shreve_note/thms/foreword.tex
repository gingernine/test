\chapter*{}
書き始めはKaratzas Shreveの補助ノートにするつもりでしたが,紆余曲折を経て我流の数学大河をつらつら綴りだすに至りました.
その源流は公理的集合論との出会いと自己不信にあります.
一度この身に浴びた形式主義の洗礼は,今や呪縛として脳の活動を不能ならしめ,
焦燥感を喚起して精神を曇らせています.
日々鬱々と,大学院生にあるまじき恐るべき後進性を晒していますが,それでも自身をこの呪縛から解き放つためには,
途中で溺れることなくその果てまで泳ぎ切るほかは無いのでしょう.

私は,数学を知る前の人でも理解できるくらいの緻密な論理を展開することを最大のモットーとしています.
それは自身の後学のためでもありますし,見落としや誤謬を少なくするための方便でもあります.
修士課程でつまづいた私がプロの真似っこのように数学のノートを執筆することが,
滑稽で身の程を知らないだとか,或いは親不孝の道楽だとか小憎らしく思われることは先刻承知です.
ただ,勉強したことを理解したと,腹の底から自信をもって宣言できるようになるには,
自分の頭で悉く辿りつくせる世界の中に落とし込まないと気が済まないだけなのです.
知の感触を素手で確かめるには,他に術が無いのです.
まあ全てを自分で調達しようというのは無謀のように見えて,それでも実際挑戦してみると,
それは決して無駄な放蕩ではなく,むしろ一生ものの高尚な趣味を得たような気分になってきます.

ところで電子テキストは書き直しや書き足しに便利です.
この利点を生かせば,勉強や復習をするたびに,或いは何かを発見するたびに,或いは誰か別の人でも,
都度修正したり書き加えたりしていくことで,本稿は生き物のようにいくらでも洗練されて,
いくらでも膨張していくことができるでしょう.
本稿がいつか,現代版のブルバキとして,悩める厳密な頭脳の持ち主の一助となってくれることを夢に見ています.

本稿の構成について,本編はKaratzas Shreveの補助ノートのつもりで書いたものですが,
自身の力不足が原因で半端に投げ出したままになっています.
付録の方は異様に肥大化して,目次だけ追うと集合論理から始まり一般位相空間論や積分論を経て確率解析に繋がっているように見えますが,
一番新しく書き始めた章は初っ端の集合論理です.
全体を通して,特に位相や積分論の章で顕著ですが,書き方が統一されていなかったり,
定義されていない記号が出てきたり,キソ概念を誤解していたり,
(場合によっては怪しい)主張だけ書いておいて証明が付いていなかったり,
見るに堪えないお粗末な代物となってしまっています.
いずれ全面を書き直すつもりですが,それが何年後になるかは神のみぞ知るところでしょう.
以上書いてみて読み直してみると,どの段落もまとまりがなく一貫性を持ちません.
統一感の無さと読みにくさは数学ノートの方にも無事継承されていますから
研究室内で公開するのは恥ずかしい限りなのですが,
駄学生の分際で僭越ながら,知識や知恵の共有に役に立てるならば幸いと思います.
\\
\\
\\
\rightline{2019年4月}
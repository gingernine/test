\chapter*{}
書き始めはKaratzas Shreveの補助ノートにするつもりでしたが,紆余曲折を経て我流の数学大河をつらつら綴りだすに至りました.
その源流は公理的集合論との出会いと自己不信にあります.一度この身に浴びた形式主義の洗礼は,
今や呪縛として脳の活動を不能ならしめ,焦燥感を喚起して精神を食い破っています.
日々鬱々と,大学院生にあるまじき恐るべき後進性を晒していますが,それでも自身をこの呪縛から解き放つためには,
諦観に屈従して途中に溺れることなく,その果てまで泳ぎ切るほかは無いのでしょう.

私は,数学を知る前の人でも理解できるくらいの緻密な論理を展開することを最大のモットーとしています.
それは自身の後学のためでもありますし,見落としや誤謬を少なくするための方便でもあります.
\begin{comment}
巷にあふれる数学書はどういうわけか難解である.
それは主に,厳密性に欠けていたり,論理のギャップが大きかったり,或いは高い水準からスタートしていることに起因している.
他者に向けて書いている筈が著者の知識の整理にしか役に立たないのでは,わざわざ出版する意味は無い.
それから公理を明記していない物が多い.公理の選び方によっていくらでも世界は変わってしまうのだから,
暗黙の了解で済ませるのは横着に思われる.
\end{comment}
数学者でも優等生でもない一介の学生である私が数学のノートを執筆することを,
滑稽だとか身の程を知らないだとか小憎らしく思われることは先刻承知です.
別に御門違いな自負でもって執筆しているわけではありませんし,気に入った本の真似っこで遊んでいるわけでもありません.
勉強したことを理解したと,腹の底から自信をもって宣言できるようになるには,
自分の頭で悉く辿りつくせる世界の中に落とし込まないと気が済まないだけなのです.
知の感触を素手で確かめるには,他に術が無いのです.

電子テキストは書き直しや書き足しに便利です.
この利点を生かせば,勉強や復習をするたびに,或いは何かを発見するたびに,或いは誰か別の人でも,
都度書き加えていくことで,本稿は生き物のようにいくらでも洗練されて,いくらでも膨張していくことができるでしょう.
本稿がいつか,現代版のブルバキとして,悩める厳密な頭脳の持ち主の一助となってくれることを夢に見ています.

本稿の構成について,本編はKaratzas Shreveの補助ノートを途中で投げ出したままになっています.
付録の方が異様に肥大化して,章だけ見ると集合論理から始まり一般位相空間論や積分論を経て確率解析に繋がっているように見えますが,
一番新しく書き始めた章は初っ端の集合論理です.
以降の位相や積分論などの章は,書き方に統一感が無かったり,定義されていない記号が出てきたり,キソ概念を誤解していたり,
(場合によっては怪しい)主張だけ書いておいて証明が付いていなかったり,見るに堪えないお粗末な代物となってしまっています.
いずれ全面を書き直すつもりですが,それが何年後になるかは神のみぞ知るところでしょう.
\\
\\
\\
\rightline{2019年4月}
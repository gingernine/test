\chapter*{このノートについて}

このノートが誰かに読まれるのかどうかはわかりませんが,万が一読まれた場合のために
ノートの構成と動機ともろもろの弁解を書いておきます.

始めはKaratzas Shreveの補助ノートにするつもりでした.
しかしながら紆余曲折を経て我流の数学大河をつらつら綴りだすに至り,
本業をほったらかして,いや別に丸っきりほったらかしているわけでもないのですが,寝食を忘れて執筆に没頭しています.

その源流は公理的集合論との出会いにありますが,それは全く穏やかなものではなく,
当時修論の方で心が折れていたところへ追い打ちをかけるように激しいパラダイムシフトを引き起こし,
僕はコテンパンに打ちのめされて数学に対して全くの盲目になりました.
何もできなくなったときに何もしないままでいると本当に死んでしまいますから,
起死回生の策として数学を根本から勉強し直し,その全貌の再構成を試みたのです.
おかげで日々鬱々と,大学院生にあるまじき恐るべき後進性を晒していますが.

殆どの人が問題なく通過していく修士課程でつまづいた僕がプロの猿真似のように数学のノートを執筆すると,
鼻で笑われたり,或いは難題から逃げているだとか,滑稽で身の程を知らないだとか,
親不孝の道楽者だとか小憎らしく思われるだろうことは先刻承知です.
けれどもかの大革命は僕の脳の活動に重い枷をかけて,
自分で悉く辿りつくせる世界の中に落とし込まないと何も理解できなくなってしまったのです.
知の感触を素手で確かめるには,僕にはこうする他に術が無いのです.
まあ全てを自分で調達しようというのは愚かで無謀に見えますが,それでも実際挑戦してみると決して無駄な放蕩ではなくて,
むしろ挫けた心が立て直されて本業にも加速して打ち込めるような気分になってきます.

ポジティブな話に変わりますが,本稿はTexで書いていますから修正や追加が簡単であるだけでなく,
元のテキストファイルをGithubなどで共有すれば共同開発することもできるのです.
すなわち,もの好き同士で徒党を組んで,知識を仕入れるたびに,復習をするたびに,或いは何かを発見するたびに,
その都度書き直したり書き加えたりしていくことで,本稿は生き物のようにいくらでも洗練されて,いくらでも膨張していくことができるでしょう.
僕はブルバキの数学原論を殆ど読んだことがないのですけれども,この執筆作業をおこがましくもブルバキプロジェクトと呼んでいて,
さらにおこがましく,多少拙くてもある程度書き始めておけば優秀な人が引き継いでくれて日本でもブルバキが生まれるのではないかと思っています.
内容はまだ大して充実していませんが,本稿がいつか,現代の数学原論として,悩める厳密な頭脳の持ち主の一助となってくれることを夢に見ています.

本稿の構成について,本編はKaratzas Shreveの補助のつもりで書いたものですが,
上で述べたアホみたいなことが原因で途中で投げ出したままになっています.
付録はもともと補助の補助のつもりで付けていたものですが今では異様に肥大化していて,
目次だけ追うと集合論理から始まり一般位相空間論や積分論を経て確率解析に繋がっているように見えますが,
書いた時期は全くバラバラで,特に一番新しく書き始めた章は初っ端の集合論理です.
全体を通して,特に位相や位相解析の章で顕著ですが,単に書き方が変だったり,
間違っていることを堂々と書いていたり,記号が統一されていなかったり定義されていなかったり,キソ概念を誤解していたり,
(場合によっては怪しい)主張だけ書いておいて証明をほったらかしていたり,
自分で決めたルールに違約していたり,などなど今はまだ見るに堪えない\sout{インチキな}お粗末な代物です.
少しずつ書き直してはいるのですが,いつになったらマトモなものに仕上がるのかは神のみぞ知るところでしょう.

自作の研究資料を研究室内で公開するというのは僕の憧れの先輩が始められた密かな伝統行事でして,
落ちこぼれの分際で僭越ながら,本稿が知識や知恵の共有に役に立てるならば幸いと思います.
\\
\\
\rightline{2019年4月}
\\
\\
読み返すたびに間違いや気に食わない記述が見つかるので,どうもこのノートには間違いや気に食わない記述が可算無限個存在しているようです.
となると,このノートの文字数はどう見ても\sout{有限}有界ですからどこかに間違いや気に食わない記述が集積している筈で...ああ,位相の章でした.

右連続な二乗可積分マルチンゲールの二次変分の存在はDoob-Meyer分解により示すのが一般論ですが,
その延長で連続マルチンゲールの連続な二次変分の存在を示すとなるとフィルトレーションはusualのままです.
しかし実際は,連続の場合ならフィルトレーションの右連続性は課さずに,
つまりDoob-Meyer分解とは別のルートで連続な二次変分が得られます.
右連続な場合も連続版の証明の真似でいけそうだと踏んで挑戦してみたら,
ナチュラル性に関して厄介な問題が浮上してうまくいきませんでした.それにしても右連続な場合と連続な場合とで証明が大きく異なるのは疲れます.
なんとか似た証明が出来ないものか考えていたり探していたりするので,気が向いたら情報お願いします.
\\
\\
\rightline{2019年5月}
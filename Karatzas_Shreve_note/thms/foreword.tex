
書き始めはKaratzas Shreveの補助ノートにするつもりでしたが,紆余曲折を経て我流の数学大河をつらつら綴りだすに至りました.
その源流は公理的集合論との出会いにありますが,それは全く穏やかなものではなく,
当時修論の方で心が折れていたところに追い打ちをかけるように激しいパラダイムシフトを引き起こし,
僕はコテンパンに打ちのめされて数学に対して盲目になりました.いや,むしろそれ以前の僕の方が盲であって,
実はあのめぐり逢いはプラスに働いていてこのダークバカな脳みそを根本から叩き直してくれているのかもしれませんが.
でもしかしながら,一度この身に浴びてしまった形式主義の洗礼は今や呪いの如きになって,脳の活動を不能にせんばかりに重い枷をかけています.
日々鬱々と,大学院生にあるまじき恐るべき後進性を晒していますが,
それでもこの呪縛から解放されるためには,途中で溺れることなくその果てまで泳ぎ切るほかは無いのでしょう.

本稿は,数学を知る前の人でも理解できるくらいの緻密な論理を展開することを最大のモットーとしています.
それは自身の後学のためでもありますし,見落としや誤謬を少なくするための方便でもあります.
殆どの人が問題なく通過していく修士課程でつまづいた僕がプロの猿真似のように数学のノートを執筆することで,
鼻で笑われたり,或いは難題から逃げているだとか,滑稽で身の程を知らないだとか,
親不孝の道楽者だとか小憎らしく思われることは先刻承知です.
けれども,ただ,勉強したことを理解したと腹の底から自信をもって宣言できるようになるには,
自分の頭で悉く辿りつくせる世界の中に落とし込まないと気が済まないだけなのです.
知の感触を素手で確かめるには,他に術が無いのです.
まあ全てを自分で調達しようというのは愚かで無謀で本業を疎かにしているように見えますが,それでも実際挑戦してみると,
それは決して無駄な放蕩ではなくて,むしろ挫けた心が立て直されて本業にも加速して打ち込めるような気分になってきます.

ところで電子テキストは書き直しや書き足しが簡単であるだけでなく,Githubなどで共有すれば共同開発することもできるのです.
すなわち,知識を仕入れるたびに,復習をするたびに,或いは何かを発見するたびに,或いは誰か別の人でも,
都度修正したり書き加えたりしていくことで,本稿は生き物のようにいくらでも洗練されて,いくらでも膨張していくことができるでしょう.
僕はこの執筆作業をおこがましくもブルバキプロジェクトと呼んでいますが,あのような大叙事詩は誰かがある程度書き始めておけば
日本からでもすぐに生まれてくれるのではないかと思えるのです.内容はまだ大して充実していませんが,
本稿がいつか,現代版のブルバキとして,悩める厳密な頭脳の持ち主の一助となってくれることを夢に見ています.

本稿の構成について,本編はKaratzas Shreveの補助ノートのつもりで書いたものですが,
自身の力不足が原因で途中で投げ出したままになっています.
付録の方は異様に肥大化していて,目次だけ追うと集合論理から始まり一般位相空間論や積分論を経て確率解析に繋がっているように見えますが,
書いた時期は全くバラバラで,特に一番新しく書き始めた章は初っ端の集合論理です.
全体を通して,特に位相や積分論の章で顕著ですが,間違ったことを堂々と書いていたり,
書き方が統一されていなかったり,定義されていない記号が出てきたり,キソ概念を誤解していたり,
(場合によっては怪しい)主張だけ書いておいて証明をほったらかしていたり,
自分で決めたルールに違約していたり,などなど今はまだ見るに堪えない\sout{インチキな}お粗末な代物です.
少しずつ書き直してはいるのですが,いつになったらマトモなものに仕上がるのかは神のみぞ知るところでしょう.

以上を読み直してみると,たったこれだけしか書いていないのにどの段落もまとまりがなくて,
しかも半分自分の稚さの弁解であることが輪をかけて体裁を悪くしています.
数学ノートの方はさらに悲劇的な読みにくさのオンパレードになっていますが,
たとえ誰にも読まれないとしても,自作の研究資料を研究室内で公開するというのは僕の憧れの先輩が始められた密かな伝統行事です.
落ちこぼれの分際で僭越ながら,本稿が知識や知恵の共有に役に立てるならば幸いと思います.

\rightline{2019年4月}
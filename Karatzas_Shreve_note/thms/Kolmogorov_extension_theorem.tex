\section{拡張定理}
	いま,$T$を空でない集合とし,$T$の任意の要素$t$に対して可測空間$(X_t,\mathscr{B}_t)$が
	定まっていて,また
	\begin{align}
		\forall t \in T\, (\, X_t \neq \emptyset\, )
	\end{align}
	が満たされているとする.$\mathscr{F}$を$T$の空でない任意の有限部分集合の全体として,$\mathscr{F}$の任意の要素$\Lambda$に対して
	\begin{align}
		X_\Lambda \defeq \prod_{t \in \Lambda} X_t,
		\quad \mathscr{B}_\Lambda \defeq \bigotimes_{t \in \Lambda} \mathscr{B}_t
	\end{align}
	により可測空間$(X_\Lambda,\mathscr{B}_\Lambda)$を定める.また
	\begin{align}
		X \defeq \prod_{t \in T} X_t,
		\quad \mathscr{B} \defeq \bigotimes_{t \in T} \mathscr{B}_t
	\end{align}
	とおく.$\mathscr{F}$の任意の要素$\Lambda,\Lambda'$に対し,
	$\Lambda \subset \Lambda'$であるとき$X_{\Lambda'}$から
	$X_{\Lambda}$への射影を$\pi_{\Lambda',\Lambda}$と書き,
	また$X$から$X_\Lambda$への射影を$\pi_{\Lambda}$と書く.以上が準備となる.
	
	定理の首脳部に入る前に次を証明しておく.
	\begin{screen}
		\begin{lem}[射影の可測性]\label{lem:Kolmogorov_extension_theorem}
			$\Lambda$と$\Lambda'$を$\Lambda \subset \Lambda'$なる$T$の空でない部分集合とするとき,
			射影$\pi_{\Lambda',\Lambda}$は$\mathscr{B}_{\Lambda'}/\mathscr{B}_\Lambda$-可測である.
			また射影$\pi_\Lambda$は$\mathscr{B}/\mathscr{B}_\Lambda$-可測である.
		\end{lem}
	\end{screen}
	
	\begin{prf}
		$t$を$\Lambda$の要素とすれば$\pi_{\Lambda,\{t\}}$は
		$X_\Lambda$から$X_t$への射影であるから,直積$\sigma$-加法族の定義より$\pi_{\Lambda,\{t\}}$は
		$\mathscr{B}_\Lambda/\mathscr{B}_t$-可測である.
		特に$\pi_\Lambda$は$\mathscr{B}/\mathscr{B}_\Lambda$-可測である.
		また$\Lambda \subset \Lambda'$であるとき,
		$t$を$\Lambda$の要素として$B$を$\mathscr{B}_t$の要素とすれば
		\begin{align}
			\pi_{\Lambda',\Lambda}^{-1} \ast \left(\pi_{\Lambda,\{t\}}^{-1} \ast B\right)
			= \pi_{\Lambda',\{t\}}^{-1} \ast B
		\end{align}
		が成立するので
		\begin{align}
			\bigcup_{t\in\Lambda} \Set{\pi_{\Lambda,\{t\}}^{-1} \ast B}{B \in \mathscr{B}_t}
			\subset \Set{B \in \mathscr{B}_\Lambda}{\pi_{\Lambda',\Lambda}^{-1} \ast B \in \mathscr{B}_{\Lambda'}}
		\end{align}
		が成り立つ.左辺は$\mathscr{B}_\Lambda$を生成し右辺は$\sigma$-加法族であるから
		$\pi_{\Lambda',\Lambda}$の$\mathscr{B}_{\Lambda'}/\mathscr{B}_\Lambda$-可測性が従う.
		\QED
	\end{prf}
	
	本節の主題は次である.いま,$\mathscr{F}$の任意の要素$\Lambda$に対し,
	$(X_\Lambda,\mathscr{B}_\Lambda)$上に確率測度$\mu_\Lambda$が定まっていて
	\begin{align}
		\forall \Lambda,\Lambda' \in \mathscr{F},\quad
		\Lambda \subset \Lambda' \Longrightarrow
		\mu_{\Lambda'} \pi_{\Lambda',\Lambda}^{-1}
		= \mu_\Lambda
	\end{align}
	が成り立っているとする.この式を{\bf 両立条件}\index{りょうりつじょうけん@両立条件}
	{\bf (consistency condition)}と呼ぶ.
	加えて,$T$の任意の要素$t$に対して$\mathscr{B}_t$に含まれるコンパクトクラス$\mathcal{K}_t$が取れて,
	任意の正数$\epsilon$と$\mathscr{B}_t$の任意の要素$B$に対して
	\begin{align}
		K \subset B \wedge \mu_{\{t\}}(B \backslash K) < \epsilon
	\end{align}
	なる$\mathcal{K}_t$の要素$K$が取れるとする.このとき,
	\begin{align}
		\forall \Lambda \in \mathscr{F},\quad 
		\mu \pi_{\Lambda}^{-1} = \mu_\Lambda.
	\end{align}
	を満たす$(X,\mathscr{B})$上の確率測度$\mu$が唯一つだけ取れる.
	
	\begin{prf}\mbox{}
		\begin{description}
			\item[第一段]
				集合$\mathscr{R}$を
				\begin{align}
					\mathscr{R} \defeq \bigcup_{\Lambda\in\mathscr{F}} \Set{\pi_\Lambda^{-1} \ast B}{B \in \mathscr{B}_\Lambda}
				\end{align}
				により定めれば,$\mathscr{R}$は$X$上の加法族となり$\mathscr{B}$を生成する.まず
				\begin{align}
					X = \pi_\Lambda^{-1} \ast X_\Lambda
				\end{align}
				より$X$は$\mathscr{R}$の要素である.また$A$を$\mathscr{R}$の要素とすれば
				\begin{align}
					A = \pi_\Lambda^{-1} \ast B
				\end{align}
				を満たす$\mathscr{F}$の要素$\Lambda$と$\mathscr{B}_\Lambda$の要素$B$が取れて,このとき
				\begin{align}
					X \backslash A = \pi_\Lambda^{-1} \ast (X_\Lambda \backslash B)
				\end{align}
				が成り立つので
				\begin{align}
					X \backslash A \in \mathscr{R}
				\end{align}
				も成り立つ.$A$と$A'$を$\mathscr{R}$の要素とすれば,
				\begin{align}
					A = \pi_\Lambda^{-1} \ast B
				\end{align}
				を満たす$\mathscr{F}$の要素$\Lambda$と$\mathscr{B}_\Lambda$の要素$B$が取れて,かつ
				\begin{align}
					A' = \pi_{\Lambda'}^{-1} \ast B'
				\end{align}
				を満たする$\mathscr{F}$の要素$\Lambda'$と$\mathscr{B}_{\Lambda'}$の要素$B'$も取れる.
				ここで
				\begin{align}
					\Lambda'' \defeq \Lambda \cup \Lambda'
				\end{align}
				とおけば
				\begin{align}
					A = \pi_{\Lambda''}^{-1} \ast \left(\pi_{\Lambda'',\Lambda}^{-1} \ast B\right)
				\end{align}
				と
				\begin{align}
					A' = \pi_{\Lambda''}^{-1} \ast \left(\pi_{\Lambda'',\Lambda'}^{-1} \ast B'\right)
				\end{align}
				が成り立つ.よって
				\begin{align}
					A \cup A' =  \pi_{\Lambda''}^{-1}\left(\pi_{\Lambda'',\Lambda}^{-1}(B) \cup 
					\pi_{\Lambda'',\Lambda'}^{-1}(B')\right)
				\end{align}
				となるが,補題\ref{lem:Kolmogorov_extension_theorem}より
				$\pi_{\Lambda'',\Lambda}^{-1}(B)$と$\pi_{\Lambda'',\Lambda'}^{-1}(B')$は共に$\mathscr{B}_{\Lambda''}$に属するので
				\begin{align}
					A \cup A' \in \mathscr{R}
				\end{align}
				が成立する.以上より$\mathscr{R}$は$X$上の加法族である.また
				\begin{align}
					\bigcup_{t \in T}\Set{\pi_{\{t\}}^{-1}(B)}{B \in \mathscr{B}_t} \subset \mathscr{R}
				\end{align}
				が成り立つが,左辺は$\mathscr{B}$を生成するので
				\begin{align}
					\mathscr{B} \subset \sigma(\mathscr{R})
				\end{align}
				を得る.一方で$\mathscr{F}$の任意の要素$\Lambda$に対し$\pi_\Lambda$は
				$\mathscr{B}/\mathscr{B}_\Lambda$-可測であるから
				\begin{align}
					\mathscr{R} \subset \mathscr{B}
				\end{align}
				も成り立ち
				\begin{align}
					\sigma(\mathscr{R}) = \mathscr{B}
				\end{align}
				が出る.
				
			\item[第二段]
				$\mu$を
				\begin{align}
					\mathscr{R} \ni \pi_\Lambda^{-1} \ast B \longmapsto P_\Lambda(B)
				\end{align}
				なる関係により定めると,$\mu$は写像であって,また$\mathscr{R}$の上の加法的である.
				まず$\mu$が写像であることを示す.
				\begin{align}
					\pi_\Lambda^{-1} \ast B = \pi_{\Lambda'}^{-1} \ast B'
				\end{align}
				が成り立っているとき,
				\begin{align}
					\Lambda'' \defeq \Lambda \cup \Lambda'
				\end{align}
				とおけば
				\begin{align}
					\pi_{\Lambda''}^{-1} \ast \left( \pi_{\Lambda'',\Lambda}^{-1} \ast B \right)
					= \pi_\Lambda^{-1} \ast B
					= \pi_{\Lambda'}^{-1} \ast B'
					= \pi_{\Lambda''}^{-1} \ast \left( \pi_{\Lambda'',\Lambda'}^{-1} \ast B' \right)
				\end{align}
				が成り立つ.ここで$\pi_{\Lambda''}$は全射なので定理\ref{projective_injective_image_preimage}より
				\begin{align}
					\pi_{\Lambda'',\Lambda}^{-1} \ast B = \pi_{\Lambda'',\Lambda'}^{-1} \ast B'
				\end{align}
				が従い,両立条件から
				\begin{align}
					\mu_\Lambda(B) 
					= \mu_{\Lambda''} \pi_{\Lambda'',\Lambda}^{-1}(B)
					= \mu_{\Lambda''} \pi_{\Lambda'',\Lambda'}^{-1}(B')
					= \mu_{\Lambda'}(B')
				\end{align}
				が成り立つ.ゆえに$\mu$は写像である.次に$\mu$の加法性を示す.
				\begin{align}
					\pi_{\Lambda_1}^{-1} \ast B_1 \cap \pi_{\Lambda_2}^{-1} \ast B_2 = \emptyset
				\end{align}
				であるとき,
				\begin{align}
					\Lambda_3 \defeq \Lambda_1 \cup \Lambda_2
				\end{align}
				とおけば
				\begin{align}
					\emptyset 
					= \pi_{\Lambda_3}^{-1} \ast \left( \pi_{\Lambda_3,\Lambda_1}^{-1} \ast B_1 \right)
					\cap \pi_{\Lambda_3} \ast \left( \pi_{\Lambda_3,\Lambda_2}^{-1} \ast B_2 \right)
					= \pi_{\Lambda_3}^{-1} \ast \left( \pi_{\Lambda_3,\Lambda_1}^{-1} \ast B_1 \cap \pi_{\Lambda_3,\Lambda_2}^{-1} \ast B_2 \right)
				\end{align}
				が成り立って,$\pi_{\Lambda_3}$が全射であるので
				\begin{align}
					\pi_{\Lambda_3,\Lambda_1}^{-1} \ast B_1 \cap \pi_{\Lambda_3,\Lambda_2}^{-1} \ast B_2 = \emptyset
				\end{align}
				が従う.よって両立条件と併せて
				\begin{align}
					\mu\left( \pi_{\Lambda_1}^{-1} \ast B_1 \cup \pi_{\Lambda_2}^{-1} \ast B_2 \right)
					&= \mu\left[\pi_{\Lambda_3}^{-1} \ast \left( \pi_{\Lambda_3,\Lambda_1}^{-1} \ast B_1 \right)
					\cup \pi_{\Lambda_3} \ast \left( \pi_{\Lambda_3,\Lambda_2}^{-1} \ast B_2 \right) \right] \\
					&= \mu\left[ \pi_{\Lambda_3}^{-1} \ast \left( \pi_{\Lambda_3,\Lambda_1}^{-1} \ast B_1 \cup \pi_{\Lambda_3,\Lambda_2}^{-1} \ast B_2 \right) \right] \\
					&= \mu_{\Lambda_3} \left( \pi_{\Lambda_3,\Lambda_1}^{-1} \ast B_1 \cup \pi_{\Lambda_3,\Lambda_2}^{-1} \ast B_2 \right) \\
					&= \mu_{\Lambda_3} \left( \pi_{\Lambda_3,\Lambda_1}^{-1} \ast B_1 \right)
						+ \mu_{\Lambda_3} \left( \pi_{\Lambda_3,\Lambda_2}^{-1} \ast B_2 \right) \\
					&= \mu\left( \pi_{\Lambda_1}^{-1} \ast B_1 \right)
						+ \mu\left( \pi_{\Lambda_2}^{-1} \ast B_2 \right)
				\end{align}
				が成立する.
			
			\item[第三段]
				$\mu$が$\mathscr{R}$の上で完全加法的であることを示す.
				\begin{align}
					\bigcup_{t \in T} \Set{\pi_t^{-1} \ast K}{K \in \mathcal{K}_t}
				\end{align}
				の有限和の全体を$\mathcal{K}$とおけば$\mathcal{K}$はコンパクトクラスである(未証明).また
				\begin{align}
					\mathscr{D} \defeq 
					\Set{B \in \mathscr{B}}{\forall \epsilon \in \R_+\, \exists K \in \mathcal{K}\, 
					\left(\, \mu(B \backslash K) < \epsilon\, \right)}
				\end{align}
				は$X$上のDynkin族である.ここで
				\begin{align}
					\bigcup_{t \in T} \Set{\pi_t^{-1} \ast B}{B \in \mathscr{B}_t}
				\end{align}
				の有限交叉の全体は$\mathscr{B}$を生成する乗法族となるので,
				$\bigcap_{i=1}^n \pi_{t_i}^{-1} \ast B_i$をその要素とするとき
				\begin{align}
					\bigcap_{i=1}^n \pi_{t_i}^{-1} \ast B_i \in \mathscr{D}
					\label{Kolmogorov_extension_theorem_1}
				\end{align}
				が成り立つことを示せば
				\begin{align}
					\mathscr{D} = \mathscr{B}
				\end{align}
				が成り立ち,定理\ref{thm:finite_intersection_property_and_common_point_property}と
				定理\ref{thm:equivalent_conditions_of_countable_additivity_for_finite_measure}より
				$\mu$の完全加法性が従う.$\epsilon$を任意に与えられた正数とすれば,各$B_i$に対して
				\begin{align}
					K_i \subset B_i \wedge \mu_{t_i}(B_i \backslash K_i) < \epsilon
				\end{align}
				を満たす$K_i$が取れて,
				\begin{align}
					\bigcup_{i=1}^n \pi_{t_i}^{-1} \ast K_i \in \mathcal{K}
				\end{align}
				かつ
				\begin{align}
					\mu\left( \bigcap_{i=1}^n \pi_{t_i}^{-1} \ast B_i \backslash 
					\bigcup_{i=1}^n \pi_{t_i}^{-1} \ast K_i \right)
					= \mu\left( \bigcap_{i=1}^n \pi_{t_i}^{-1} \ast (B_i \backslash K_i) \right)
					< \epsilon
				\end{align}
				が成り立つ.すなわち(\refeq{Kolmogorov_extension_theorem_1})が成り立つ.
				\QED
		\end{description}
	\end{prf}
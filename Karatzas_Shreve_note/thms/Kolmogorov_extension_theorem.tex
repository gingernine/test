\section{拡張定理}
	確率空間を実数空間の上に作ってしまうと
	Kolmogorovの拡張定理の証明は論理的にも見た目の上でも煩雑になるようだが,
	Bogachevが説明する通り,コンパクトクラスの概念を使って
	一般化されたKolmogorovの拡張定理は主張のみならず証明も洗練されたものとなる.
	一見長い証明となるが,内容は抽象的で捉えやすい.
	
	いま,$T$を空でない集合とし,$T$の任意の要素$t$に対して可測空間$(X_t,\mathscr{B}_t)$が
	定まっていて,また
	\begin{align}
		\forall t \in T\, (\, X_t \neq \emptyset\, )
	\end{align}
	が満たされているとする.$\mathscr{F}$を$T$の空でない任意の有限部分集合の全体として,$\mathscr{F}$の任意の要素$\Lambda$に対して
	\begin{align}
		X_\Lambda \defeq \prod_{t \in \Lambda} X_t,
		\quad \mathscr{B}_\Lambda \defeq \bigotimes_{t \in \Lambda} \mathscr{B}_t
	\end{align}
	により可測空間$(X_\Lambda,\mathscr{B}_\Lambda)$を定める.また
	\begin{align}
		X \defeq \prod_{t \in T} X_t,
		\quad \mathscr{B} \defeq \bigotimes_{t \in T} \mathscr{B}_t
	\end{align}
	とおく.$\mathscr{F}$の任意の要素$\Lambda,\Lambda'$に対し,
	$\Lambda \subset \Lambda'$であるとき$X_{\Lambda'}$から
	$X_{\Lambda}$への射影を$\pi_{\Lambda',\Lambda}$と書き,
	また$X$から$X_\Lambda$への射影を$\pi_{\Lambda}$と書く.以上が準備となる.
	
	定理の首脳部に入る前に次の補題を証明する.
	\begin{screen}
		\begin{lem}[射影の可測性]\label{lem:Kolmogorov_extension_theorem}
			$T$の任意の空でない部分集合$\Lambda,\Lambda'$に対し(有限性は要らない),$\Lambda \subset \Lambda'$であるとき
			射影$\pi_{\Lambda',\Lambda}$は$\mathscr{B}_{\Lambda'}/\mathscr{B}_\Lambda$-可測である.
			また射影$\pi_\Lambda$は$\mathscr{B}/\mathscr{B}_\Lambda$-可測である.
		\end{lem}
	\end{screen}
	
	\begin{prf}
		$\Lambda,\Lambda'$を$T$の空でない部分集合とする.このとき,まず$\Lambda$の任意の要素$t$に対し$\pi_{\Lambda,\{t\}}$は
		$X_\Lambda$から$X_t$への射影であるから,直積$\sigma$-加法族の定義より$\pi_{\Lambda,\{t\}}$は
		$\mathscr{B}_\Lambda/\mathscr{B}_t$-可測性を持つ.特に$\Lambda=T$の場合
		\begin{align}
			\pi_{\Lambda,\{t\}} = \pi_\Lambda,\quad \mathscr{B}_\Lambda = \mathscr{B}
		\end{align}
		であるから$\pi_\Lambda$の$\mathscr{B}/\mathscr{B}_\Lambda$-可測性が得られる.
		また$\Lambda \subset \Lambda'$であるとき,$\Lambda$の任意の要素$t$と$\mathscr{B}_t$の任意の要素$B$に対し
		\begin{align}
			\pi_{\Lambda',\Lambda}^{-1}\left(\pi_{\Lambda,\{t\}}^{-1}(B)\right)
			= \pi_{\Lambda',\{t\}}^{-1}(B) \in \mathscr{B}_{\Lambda'}
		\end{align}
		が成立する.従って
		\begin{align}
			\bigcup_{t\in\Lambda} \Set{\pi_{\Lambda,\{t\}}^{-1}(B)}{B \in \mathscr{B}_t}
			\subset \Set{B \in \mathscr{B}_\Lambda}{\pi_{\Lambda',\Lambda}^{-1}(B) \in \mathscr{B}_{\Lambda'}}
		\end{align}
		となり,左辺は$\mathscr{B}_\Lambda$を生成し右辺は$\sigma$-加法族であるから
		$\pi_{\Lambda',\Lambda}$の$\mathscr{B}_{\Lambda'}/\mathscr{B}_\Lambda$-可測性が従う.
		\QED
	\end{prf}
	
	本節の主題は次である.いま,$\mathscr{F}$の任意の要素$\Lambda$に対し,
	$(X_\Lambda,\mathscr{B}_\Lambda)$上に確率測度$\mu_\Lambda$が定まっていて
	\begin{align}
		\forall \Lambda,\Lambda' \in \mathscr{F},\quad
		\Lambda \subset \Lambda' \Longrightarrow
		\mu_{\Lambda'} \pi_{\Lambda',\Lambda}^{-1}
		= \mu_\Lambda
	\end{align}
	が成り立っているとする.この式を{\bf 両立条件}\index{りょうりつじょうけん@両立条件}
	{\bf (consistency condition)}と呼ぶ.
	加えて,$T$の任意の要素$t$に対して$\mathscr{B}_t$に含まれるコンパクトクラス$\mathcal{K}_t$が取れて,
	任意の正数$\epsilon$と$\mathscr{B}_t$の任意の要素$B$に対して
	\begin{align}
		K \subset B \wedge \mu_{\{t\}}(B \backslash K) < \epsilon
	\end{align}
	なる$\mathcal{K}_t$の要素$K$が取れるとする.このとき,
	\begin{align}
		\forall \Lambda \in \mathscr{F},\quad 
		\mu \pi_{\Lambda}^{-1} = \mu_\Lambda.
	\end{align}
	を満たす$(X,\mathscr{B})$上の確率測度$\mu$が唯一つだけ取れる.
	
	\begin{prf}\mbox{}
		\begin{description}
			\item[第一段]
				$\mathscr{R} = \bigcup_{\Lambda\in\mathscr{F}} \Set{\pi_\Lambda^{-1}(B)}{B \in \mathscr{B}_\Lambda}$
				で$\mathscr{R}$を定めれば,$\mathscr{R}$は$X$上の有限加法族となり$\mathscr{B}$を生成する.
				実際
				\begin{align}
					X = \pi_\Lambda^{-1}(X_\Lambda) \in \mathscr{R}
				\end{align}
				が成立し,また任意に$\mathscr{R}$の要素$A$を取れば,
				$A$は或る$\Lambda \in \mathscr{F}$と$B \in \mathscr{B}_\Lambda$を用いて
				\begin{align}
					A = \pi_\Lambda^{-1}(B)
				\end{align}
				と表され,$X_\Lambda \backslash B \in \mathscr{B}_\Lambda$であるから
				\begin{align}
					X \backslash A = \pi_\Lambda^{-1}(X_\Lambda \backslash B) \in \mathscr{R}
				\end{align}
				が成り立つ.また任意に$\mathscr{R}$の要素$A,A'$を取れば,
				或る$\mathscr{F}$の要素$\Lambda$と$\mathscr{B}_\Lambda$の要素$B$及び
				或る$\mathscr{F}$の要素$\Lambda'$と$\mathscr{B}_{\Lambda'}$の要素$B'$を用いて
				\begin{align}
					A = \pi_\Lambda^{-1}(B),\quad A' = \pi_\Lambda^{-1}(B')
				\end{align}
				と表され,このとき$\Lambda'' = \Lambda \cup \Lambda'$とおけば
				\begin{align}
					A = \pi_{\Lambda''}^{-1}\left(\pi_{\Lambda'',\Lambda}^{-1}(B)\right),
					\quad A' = \pi_{\Lambda''}^{-1}\left(\pi_{\Lambda'',\Lambda'}^{-1}(B')\right)
				\end{align}
				が成り立つ.補題\ref{lem:Kolmogorov_extension_theorem}より
				$\pi_{\Lambda'',\Lambda}^{-1}(B)$と$\pi_{\Lambda'',\Lambda'}^{-1}(B')$は共に$\mathscr{B}_{\Lambda''}$に属するので
				\begin{align}
					A \cup A' =  \pi_{\Lambda''}^{-1}\left(\pi_{\Lambda'',\Lambda}^{-1}(B) \cup 
					\pi_{\Lambda'',\Lambda'}^{-1}(B')\right)
				\end{align}
				となり$A \cup A' \in \mathscr{R}$が成立する.以上より$\mathscr{R}$は$X$上の有限加法族をなしている.
				また$T$の任意の要素$t$に対して$\{t\}$は$\mathscr{F}$に属するから
				\begin{align}
					\bigcup_{t \in T}\Set{\pi_{\{t\}}^{-1}(B)}{B \in \mathscr{B}_t} \subset \mathscr{R}
				\end{align}
				が成り立ち,左辺は$\mathscr{B}$を生成するので$\mathscr{B} \subset \sigma(\mathscr{R})$を得る.
				一方で$\mathscr{F}$の任意の要素$\Lambda$に対し$\pi_\Lambda$は
				$\mathscr{B}/\mathscr{B}_\Lambda$-可測である(補題\ref{lem:Kolmogorov_extension_theorem})
				から$\mathscr{R} \subset \mathscr{B}$も得られ,以上で
				\begin{align}
					\sigma(\mathscr{R}) = \mathscr{B}
				\end{align}
				が出る.
				
			\item[第二段]
				$\mu$を
				\begin{align}
					\mu = \Set{(x,y)}{\exists \Lambda \in \mathscr{F}
					\exists B \in \mathscr{B}_\Lambda
					\left(x=\pi_\Lambda^{-1}(B) \wedge y = P_\Lambda(B)\right)}
				\end{align}
				により定める.このとき$\mu$はsigle-valuedであり,
				$\mathscr{R}$上の有限加法的測度となる.まず$\mu$がsingle-valuedであることを示す.
				\begin{align}
					\pi_\Lambda^{-1}(B) = \pi_{\Lambda'}^{-1}(B')
				\end{align}
				であるとき,$\Lambda'' \coloneqq \Lambda \cup \Lambda'$とおけば
				\begin{align}
					\pi_{\Lambda''}^{-1}\left( \pi_{\Lambda'',\Lambda}^{-1}(B) \right)
					= \pi_\Lambda^{-1}(B)
					= \pi_{\Lambda'}^{-1}(B')
					= \pi_{\Lambda''}^{-1}\left( \pi_{\Lambda'',\Lambda'}^{-1}(B') \right)
				\end{align}
				が成り立つから$\pi_{\Lambda'',\Lambda}^{-1}(B) = \pi_{\Lambda'',\Lambda'}^{-1}(B')$
				が従い(定理\ref{projective_injective_image_preimage}),整合性条件より
				\begin{align}
					\mu_\Lambda(B) 
					= \mu_{\Lambda''} \circ \pi_{\Lambda'',\Lambda}^{-1}(B)
					= \mu_{\Lambda''} \circ \pi_{\Lambda'',\Lambda'}^{-1}(B')
					= \mu_{\Lambda'}(B')
				\end{align}
				が満たされ$\mu$の一意性を得る.次に$\mu$の加法性を示す.
				\begin{align}
					\pi_{\Lambda_1}^{-1}(B_1) \cap \pi_{\Lambda_2}^{-1}(B_2) = \emptyset
				\end{align}
				であるとき,$\Lambda_3 \coloneqq \Lambda_1 \cup \Lambda_2$とおけば
				\begin{align}
					\emptyset 
					= \pi_{\Lambda_3}^{-1}\left( \pi_{\Lambda_3,\Lambda_1}^{-1}(B_1) \right)
					\cap \pi_{\Lambda_3}\left( \pi_{\Lambda_3,\Lambda_2}^{-1}(B_2) \right)
					= \pi_{\Lambda_3}^{-1}\left( \pi_{\Lambda_3,\Lambda_1}^{-1}(B_1) \cap \pi_{\Lambda_3,\Lambda_2}^{-1}(B_2) \right)
				\end{align}
				となるから$\pi_{\Lambda_3,\Lambda_1}^{-1}(B_1) \cap \pi_{\Lambda_3,\Lambda_2}^{-1}(B_2)
				= \emptyset$が従い(全射の性質),
				\begin{align}
					\mu\left( \pi_{\Lambda_1}^{-1}(B_1) \cup \pi_{\Lambda_2}^{-1}(B_2) \right)
					&= \mu\left[\pi_{\Lambda_3}^{-1}\left( \pi_{\Lambda_3,\Lambda_1}^{-1}(B_1) \right)
					\cup \pi_{\Lambda_3}\left( \pi_{\Lambda_3,\Lambda_2}^{-1}(B_2) \right) \right] \\
					&= \mu\left[ \pi_{\Lambda_3}^{-1}\left( \pi_{\Lambda_3,\Lambda_1}^{-1}(B_1) \cup \pi_{\Lambda_3,\Lambda_2}^{-1}(B_2) \right) \right] \\
					&= \mu_{\Lambda_3} \left( \pi_{\Lambda_3,\Lambda_1}^{-1}(B_1) \cup \pi_{\Lambda_3,\Lambda_2}^{-1}(B_2) \right) \\
					&= \mu_{\Lambda_3} \left( \pi_{\Lambda_3,\Lambda_1}^{-1}(B_1) \right)
						+ \mu_{\Lambda_3} \left( \pi_{\Lambda_3,\Lambda_2}^{-1}(B_2) \right) \\
					&= \mu\left( \pi_{\Lambda_1}^{-1}(B_1) \right)
						+ \mu\left( \pi_{\Lambda_2}^{-1}(B_2) \right)
				\end{align}
				が成立する.
			
			\item[第三段]
				$\mu$が$\mathscr{R}$の上で完全加法的であることを定理\ref{thm:compact_class_intersection}と併せて示す.
		\end{description}
	\end{prf}
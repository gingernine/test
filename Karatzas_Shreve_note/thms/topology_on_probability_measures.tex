\section{確率測度の族の位相}
	$(X,\mathscr{O}_X)$を位相空間とし,$\mathscr{P}$を$(X,\borel{X})$上の確率測度の集合とし,
	\begin{align}
		C_b(X,\R) \defeq \Set{f}{f:X \longrightarrow \R \wedge \mbox{$f$は$\mathscr{O}_X/\mathscr{O}_\R$-連続}
		\wedge \exists b \in \R_+\, \forall x \in X\, (\, |f(x)| \leq b\, )}
	\end{align}
	とおく.$f$を$C_b(X,\R)$の要素とすれば
	\begin{align}
		\mathscr{P} \ni \mu \longmapsto \int_X f\ d\mu
	\end{align}
	なる関係で$\mathscr{P}$上の$\R$値写像が定まる.この写像を$\varphi_f$と書けば,
	\begin{align}
		\Set{\varphi_f}{f \in C_b(X,\R)}
	\end{align}
	によって$\mathscr{P}$上に始位相が定められる.その位相を
	\begin{align}
		\mathscr{O}_{\mathscr{P}}
	\end{align}
	とおく.$\mu$を$\mathscr{P}$の要素とし,$(\mu_\lambda)_{\lambda \in \Lambda}$を$\mathscr{P}$のネットとするとき,
	$\mathscr{O}_{\mathscr{P}}$に関して
	\begin{align}
		\mu_\lambda \longrightarrow \mu
	\end{align}
	となるとき$(\mu_\lambda)_{\lambda \in \Lambda}$は$\mu$に{\bf 弱収束する}という.
	また$(\mu_\lambda)_{\lambda \in \Lambda}$が$\mu$に弱収束することと
	\begin{align}
		\forall f \in C_b(X,\R)\,
		\left(\, \int_X f\ d\mu_\lambda \longrightarrow \int_X f\ d\mu\, \right)
	\end{align}
	は同値である.
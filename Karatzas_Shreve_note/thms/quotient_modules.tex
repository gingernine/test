\subsection{商加群}
	\begin{screen}
		\begin{thm}[商加群]\label{thm:quotient_module}
			$\left(\left(X,\sigma_X\right),\left(R,\sigma_R,\mu_R\right),s\right)$を加群とし,
			$\Psi$を$X$上の同値関係とし,
			\begin{align}
				X_q \defeq X/\Psi
			\end{align}
			とおく.また$q$を$X$から$X_q$への商写像とする.
			同値関係が$\sigma_X$と$s$によって不変であるとき,つまり
			\begin{align}
				\forall x,y,a,b \in X\,
				\left[\, (x,a) \in \Psi \wedge (y,b) \in \Psi \Longrightarrow 
				\left(\sigma_X(x,y), \sigma_X(a,b)\right) \in \Psi\, \right]
				\label{fom:thm_quotient_module_1}
			\end{align}
			と
			\begin{align}
				\forall r \in R\, \forall x,y \in X\,
				\left[\, (x,y) \in \Psi \Longrightarrow 
				\left(s(r,x), s(r,y)\right) \in \Psi\, \right]
				\label{fom:thm_quotient_module_2}
			\end{align}
			が満たされているとき,
			\begin{align}
				X_q \times X_q \ni \left(q(x),q(y)\right) \longmapsto q\left(\sigma_X(x,y)\right)
			\end{align}
			なる関係を$\sigma_q$とし,
			\begin{align}
				R \times X_q \ni \left(\alpha,q(x)\right) \longmapsto q\left(s(\alpha,x)\right)
			\end{align}
			なる関係を$s_q$とすると,
			\begin{align}
				\left(\left(X_q,\sigma_q\right),\left(R,\sigma_R,\mu_R\right),s_q\right)
			\end{align}
			は加群である.
		\end{thm}
	\end{screen}
	
	$\sigma_q$とは
	\begin{align}
		\sigma_q \defeq \Set{z}{\exists x,y \in X\, 
		\left[\, z=\left(\left(q(x),q(y)\right),q\left(\sigma_X(x,y)\right)\right)\, \right]}
	\end{align}
	により定められた集合であり,$s_q$とは
	\begin{align}
		s_q \defeq \Set{z}{\exists \alpha \in R\, \exists x \in X\, 
		\left[\, z=\left(\left(\alpha,q(x)\right),q\left(s(\alpha,x)\right)\right)\, \right]}
	\end{align}
	により定められた集合である.
	
	\begin{sketch}\mbox{}
		\begin{description}
			\item[第一段] $\left(X_q,\sigma_q\right)$がAbel群であることを示す.
			
			\item[第二段] $s_q$が 
				\begin{align}
					s_q:R \times X_q \longrightarrow X_q
				\end{align}
				を満たすことを示す.$x$と$y$と$z$を任意に与えられた集合とし,
				\begin{align}
					(x,y) \in s_q \wedge (x,z) \in s_q
				\end{align}
				とする.このとき$R$の要素$r$と$t$,および$X$の要素$a$と$b$で
				\begin{align}
					(x,y) = \left((r,q(a)),q(s(r,a))\right)
				\end{align}
				かつ
				\begin{align}
					(x,z) = \left((t,q(b)),q(s(t,b))\right)
				\end{align}
				を満たすものが取れる.ここで
				\begin{align}
					(r,q(a)) = x = (t,q(b))
				\end{align}
				より
				\begin{align}
					(a,b) \in \Phi
				\end{align}
				と
				\begin{align}
					\left(s(r,a), s(t,b)\right) = \left(s(r,a), s(r,b)\right) 
				\end{align}
				が成り立つので,(\refeq{fom:thm_quotient_module_2})より
				\begin{align}
					\left(s(r,a), s(t,b)\right) \in \Psi
				\end{align}
				が従う.ゆえに
				\begin{align}
					y = q(s(r,a)) = q(s(t,b)) = z
				\end{align}
				が従う.ゆえに$s_q$は写像である.また$x$を$\dom{s_q}$の要素とすれば
				\begin{align}
					(x,y) \in s_q
				\end{align}
				なる集合$y$と
				\begin{align}
					(x,y) = \left(\left(\alpha,q(z)\right),q\left(s(\alpha,z)\right)\right)
				\end{align}
				を満たす$R$の要素$\alpha$及び$X$の要素$z$が取れて
				\begin{align}
					x \in R \times X_q
				\end{align}
				となる.ゆえに
				\begin{align}
					\dom{s_q} \subset R \times X_q
				\end{align}
				である.逆に$x$を$R \times X_q$の要素とすれば
				\begin{align}
					x = \left(\alpha,q(z)\right)
				\end{align}
				を満たす$R$の要素$\alpha$及び$X$の要素$z$が取れて,このとき
				\begin{align}
					\left(\left(\alpha,q(z)\right),q\left(s(\alpha,z)\right)\right) \in s_q
				\end{align}
				が成り立つので
				\begin{align}
					x \in \dom{s_q}
				\end{align}
				となる.以上で
				\begin{align}
					\dom{s_q} = R \times X_q
				\end{align}
				が得られた.ゆえに$s_q$は$R \times X_q$から$X_q$への写像である.
				
			\item[第三段] $s_q$がスカラ倍であることを示す.
				いま$\alpha$と$\beta$を$R$の要素とし,$x$と$y$を$X_q$の要素とする.ここで
				\begin{align}
					x = q(\eta)
				\end{align}
				と
				\begin{align}
					y = q(\xi)
				\end{align}
				を満たす$X$の要素$\eta$と$\xi$を取る.まず
				\begin{align}
					s_q\left(\sigma_R(\alpha,\beta),x\right)
					&= q\left(s\left(\sigma_R(\alpha,\beta),\eta\right)\right) \\
					&= q\left(\sigma_X\left(s(\alpha,\eta),s(\beta,\eta)\right)\right) \\
					&= \sigma_q\left(q\left(s(\alpha,\eta)\right),q\left(s(\beta,\eta)\right)\right) \\
					&= \sigma_q\left(s_q(\alpha,x),s_q(\beta,x)\right)
				\end{align}
				が成立する.また
				\begin{align}
					s_q\left(\alpha,\sigma_q(x,y)\right)
					&= s_q\left(\alpha,q\left(\sigma_X(\eta,\xi)\right)\right) \\
					&= q\left(s\left(\alpha,\sigma_X(\eta,\xi)\right)\right) \\
					&= q\left(\sigma_X(s(\alpha,\eta),s(\alpha,\xi))\right) \\
					&= \sigma_q\left(q\left(s(\alpha,\eta)\right),q\left(s(\alpha,\xi)\right)\right) \\
					&= \sigma_q\left(s_q(\alpha,x),s_q(\alpha,y)\right)
				\end{align}
				も成立する.また
				\begin{align}
					s_q\left(\mu_R(\alpha,\beta),x\right)
					&= q\left(s\left(\mu_R(\alpha,\beta),\eta\right)\right) \\
					&= q\left(s\left(\alpha,s(\beta,\eta)\right)\right) \\
					&= s_q\left(\alpha,q\left(s(\beta,\eta)\right)\right) \\
					&= s_q\left(\alpha,s_q(\beta,x)\right)
				\end{align}
				も成立する.最後に
				\begin{align}
					s_q(1_R,x) = q\left(s(1_R,\eta)\right) = q(\eta) = x
				\end{align}
				も成立する.
				\QED
		\end{description}
	\end{sketch}
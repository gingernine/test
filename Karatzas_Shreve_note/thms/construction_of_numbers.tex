\section{半群}
	\begin{screen}
		\begin{dfn}[算法]
			$a$を類とするとき,$a \times a$から$a$への写像を$a$上の
			{\bf 算法}\index{さんぽう@算法}{\bf (operation)}と呼ぶ.
		\end{dfn}
	\end{screen}
	
	いま,$a$を類とし,$o$を$a$上の算法とする.
	\begin{description}
		\item[可換律\index{かかんりつ@可換律} (commutative law)] $\forall x,y \in a\, \left(\, o(x,y) = o(y,x)\, \right)$.
		\item[結合律\index{けつごうりつ@結合律} (associative law)] $\forall x,y,z \in a\, \left(\, o(o(x,y),z) = o(x,o(y,z))\, \right)$.
		\item[簡約律\index{かんやくりつ@簡約律} (cancellation law)] $\forall x,y,z \in a\, \left(\, o(x,z) = o(y,z) \Longrightarrow x = y\, \right)$.
	\end{description}
	
	$\ON$上の加法と乗法は$\ON$上の算法である.
	それらは結合律を満たす一方で可換律と簡約律は満たさないが,
	定義域を$\Natural \times \Natural$上に制限すれば全てを満たすようになる.ここで
	\begin{align}
		+_\Natural \defeq +|_{\Natural \times \Natural}
	\end{align}
	及び
	\begin{align}
		\cdot_\Natural \defeq \cdot|_{\Natural \times \Natural}
	\end{align}
	として$\Natural$上の加法と乗法を定義する.
	
	\begin{screen}
		\begin{dfn}[半群]
			$a$を集合とし,$o$を$a$上の算法とする.$o$が結合律を満たしているとき対$(a,o)$を
			{\bf 半群}\index{はんぐん@半群}{\bf (semi-group)}と呼ぶ.
			また$o$が結合律と可換律を満たすとき$(a,o)$を
			{\bf 可換半群}\index{かかんはんぐん@可換半群}{\bf (commutative semi-group)}と呼び,
			$o$が結合律と簡約律を満たすとき$(a,o)$を
			{\bf 簡約的半群}\index{かんやくてきはんぐん@簡約的半群}{\bf (cancellable semi-group)}と呼ぶ.
		\end{dfn}
	\end{screen}
	
	\begin{screen}
		\begin{thm}[$\omg$は加法に関して半群となる]
			$(\Natural,+_\Natural)$は簡約的可換半群である.
		\end{thm}
	\end{screen}
	
	\begin{sketch}
		
	\end{sketch}
	
	\begin{screen}
		\begin{dfn}[一般結合法則]
			空でない集合$S$に次を満たす二項演算$\ast:S \times S \longrightarrow S$
			が定義されているとき,$a_1,a_2,a_3,a_4$を$S$の元として,
			$a_1,a_2,a_3,a_4$の並びを替えずに$\ast$で評価していくと
			\begin{align}
				(a_1 \ast (a_2 \ast a_3)) \ast a_4,
				\quad ((a_1 \ast &a_2) \ast a_3) \ast a_4,
				\quad (a_1 \ast a_2) \ast (a_3 \ast a_4), \\
				\quad a_1 \ast (a_2 \ast (a_3 \ast a_4)),
				&\quad a_1 \ast ((a_2 \ast a_3) \ast a_4)
			\end{align}
			の5通りの評価法が考えうるが(括弧の中を優先して評価する),これは
			\begin{align}
				a_1 \ast a_2 \ast a_3 \ast a_4
			\end{align}
			の3つの$\ast$に演算の順番を付けることに対応している.
			特に,この場合は$\ast$が結合律を満たしていれば5通りの評価は全て同値になる.
			一般に$n$個の$a_1,a_2,\cdots,a_n \in S$を取りこれらに対して$n-1$回の評価を行うとき,
			$a_1,a_2,\cdots,a_n$の並びを替えない限り演算の順番をどう設定しても
			得られる結果に影響しない(最終的な評価がただ一つに確定する)ならば,
			$\ast$は{\bf 一般結合法則}\index{いっぱんけつごうほうそく@一般結合法則}
			{\bf (generalized associative law)}を満たすという.またその結果を
			\begin{align}
				a_1 \ast a_2 \ast \cdots \ast a_n
			\end{align}
			と書く.
		\end{dfn}
	\end{screen}
	
	\begin{screen}
		\begin{thm}[結合法則から一般結合法則が従う]
		\label{thm:generalized_associative_law_on_semigroup}
			$(S,\ast)$を半群とするとき$\ast$は一般結合法則を満たす.
		\end{thm}
	\end{screen}
	
	\begin{prf}
		$n > 3$を選ぶとき,
		任意の$k$個$(3 \leq k < n)$の元に対する演算の結果が評価順に依存しないと仮定すると
		$n$個の元に対する演算の結果も評価順に依存せず確定することを示す.
		$a_1,a_2,\cdots,a_n \in S$に対し,並びを替えずに$n-1$回評価するとき,
		$n-1$回目の演算は
		\begin{align}
			(\mbox{$a_1,a_2,\cdots,a_k$に対する評価}) \ast
			(\mbox{$a_{k+1},a_{k+2},\cdots,a_n$に対する評価})
			\label{eq:thm_generalized_associative_law_on_semigroup}
		\end{align}
		となる.ただし$k$は$1 \leq k \leq n-1$を満たす.仮定より第一項と第二項について
		\begin{align}
			\mbox{(第一項)} &= (\cdots((a_1 \ast a_2) \ast a_3)\cdots) \ast a_k, \\
			\mbox{(第二項)} &= a_{k+1} \ast (\cdots(a_{n-2} \ast (a_{n-1} \ast a_n))\cdots)
		\end{align}
		が成り立つから,ここで$\ast$の結合律を繰り返し用いることにより
		\begin{align}
			(\refeq{eq:thm_generalized_associative_law_on_semigroup}) 
			&= ((\cdots((a_1 \ast a_2) \ast a_3)\cdots) \ast a_k) \ast (a_{k+1} \ast (\cdots(a_{n-2} \ast (a_{n-1} \ast a_n))\cdots)) \\
			&= (((\cdots((a_1 \ast a_2) \ast a_3)\cdots) \ast a_k) \ast a_{k+1}) \ast (a_{k+2} \ast (\cdots(a_{n-2} \ast (a_{n-1} \ast a_n))\cdots)) \\
			&\vdots \\
			&= ((\cdots((a_1 \ast a_2) \ast a_3)\cdots) \ast a_{n-2}) \ast (a_{n-1} \ast a_n) \\
			&= (((\cdots((a_1 \ast a_2) \ast a_3)\cdots) \ast a_{n-2}) \ast a_{n-1}) \ast a_n
		\end{align}
		が得られる.$3$個の元の演算は評価順に依らないから,数学的帰納法より$\ast$は一般結合法則を満たす.
		\QED
	\end{prf}

\section{整数}
	\begin{screen}
		\begin{dfn}[商集合]
			$a$を集合とし,$R$を$a$上の同値関係とする.$x$を$a$の要素とするとき
			\begin{align}
				\Set{y}{(y,x) \in R}
			\end{align}
			を$x$の$R$に関する{\bf 同値類}\index{どうちるい@同値類}{\bf (equivalence class)}と呼び,
			$[x]$などで表す.また
			\begin{align}
				a / R \coloneqq \Set{x}{\exists y \in a\ \forall z\ (\ (y,z) \in R \Longleftrightarrow z \in x\ )}
			\end{align}
			で定められる類$a/R$を,$a$を$R$で割った
			{\bf 商集合}\index{しょうしゅうごう@商集合}{\bf (quotient set)}と呼ぶ.
		\end{dfn}
	\end{screen}
	
	$a$が空であれば$R$も$a/R$も空となる.
	
	\monologue{
		院生「商``集合''と名前を付けましたが,集合であることは後で示します.また上の定義の設定の下では
			\begin{align}
				a/R = \Set{x}{\exists y \in a\ (\ x = [y]\ )}
			\end{align}
			が成り立ちます.これも後で証明しますが,商集合とは同値類の集まりであるということが判るでしょう.」
	}
	
	\begin{screen}
		\begin{thm}[同値類の性質]
			$a$を集合とし,$R$を$a$上の同値関係として,$y$を$a$の要素とするとき$y$の$R$に関する同値類を
			$[y]$で表す.このとき次が成り立つ:
			\begin{description}
				\item[(1)] $\forall y \in a\ \left(\ [y] \subset a\ \right)$
				\item[(2)] $\forall y \in a\ \left(\ y \in [y]\ \right)$
				\item[(3)] $\forall y,z \in a\ \left(\ (y,z) \in R \Longleftrightarrow [y] = [z]\ \right)$
				\item[(4)] $\forall y,z \in a\ \left(\ (y,z) \notin R \Longleftrightarrow [y] \cap [z] = \emptyset\ \right)$
			\end{description}
		\end{thm}
	\end{screen}
	
	\begin{prf} $a$が空であれば空虚な真より(1)(2)(3)(4)は全て成立する.以下では$a \neq \emptyset$として証明する.
		\begin{description}
			\item[(1)] $s,t$を$\mathcal{L}$の任意の対象とするとき,$s \in a$であれば
				\begin{align}
					t \in [s] \Longrightarrow (s,t) \in R
				\end{align}
				が成り立つ.$R \subset a \times a$より
				\begin{align}
					(s,t) \in R \Longrightarrow t \in a
				\end{align}
				が従い
				\begin{align}
					t \in [s] \Longrightarrow t \in a
				\end{align}
				が得られる.$t$の任意性より
				\begin{align}
					[s] \subset a
				\end{align}
				となり,$s$の任意性より(1)が出る.
				
			\item[(2)] $t$を$\mathcal{L}$の任意の対象とするとき,$t \in a$であれば
				\begin{align}
					(t,t) \in R
				\end{align}
				となるから$t \in [t]$が成立する.$t$の任意性より
				\begin{align}
					\forall y \in a\ \left(\ y \in [y]\ \right)
				\end{align}
				が得られる.
				
			\item[(3)] $s,t$を$\mathcal{L}$の任意の対象として,$s,t \in a$であると仮定する.
				\begin{align}
					(s,t) \in R
				\end{align}
				が成り立っているとき,$\tau$を$\mathcal{L}$の任意の対象とすれば
				\begin{align}
					\tau \in [s] \Longleftrightarrow (\tau,s) \in R
				\end{align}
				となり,$R$の推移律より$(\tau,s) \in R$ならば$(\tau,t) \in R$となるから
				\begin{align}
					\tau \in [s] \Longrightarrow \tau \in [t]
				\end{align}
				が従う.同様に$\tau \in [t] \Longrightarrow \tau \in [s]$も成り立つので
				$[s] = [t]$となり
				\begin{align}
					(s,t) \in R \Longrightarrow [s] = [t]
				\end{align}
				が得られる.逆に$[s] = [t]$が成り立っているとき,$s \in [s]$より$s \in [t]$が従い
				\begin{align}
					[s] = [t] \Longrightarrow (s,t) \in R
				\end{align}
				も得られる.$s,t$の任意性より(2)が出る.
			
			\item[(4)] $s,t$を$\mathcal{L}$の任意の対象として,$s,t \in a$であると仮定する.
				\begin{align}
					[s] \cap [t] \neq \emptyset
				\end{align}
				が成り立っているとき,$[s] \cap [t]$の要素を$u$とすれば
				\begin{align}
					(s,u) \in R \wedge (u,t) \in R
				\end{align}
				となるので$(s,t) \in R$が従う.ゆえに
				\begin{align}
					(s,t) \notin R \Longrightarrow [s] \cap [t] = \emptyset
				\end{align}
				が得られる.逆に$(s,t) \in R$が成り立っているとき,(2)より$[s] = [t]$となるから
				\begin{align}
					[s] \cap [t] \neq \emptyset \Longrightarrow (s,t) \in R
				\end{align}
				も得られる.$s,t$の任意性より(3)が出る.
				\QED
		\end{description}
	\end{prf}
	
	\monologue{
		院生「(1)の主張は{\bf 同値類は空でない}ということですね.
			(2)の主張は{\bf 同値な要素の同値類は一致する}ということで,
			(2)と(3)を併せれば{\bf 同値類同士は一致していなければ交わらない}と言えます.」
	}
	
	\begin{screen}
		\begin{dfn}[商写像]
			$a$を集合とし,$R$を$a$上の同値関係として,$y$を$a$の要素とするとき$y$の$R$に関する同値類を
			$[y]$で表す.このとき
			\begin{align}
				f \coloneqq \Set{x}{\exists t \in a\ \left(\ x=(t,[t])\ \right)}
			\end{align}
			で定められる$f$を{\bf 商写像}\index{しょうしゃぞう@商写像}{\bf (quotient mapping)}と呼ぶ.
		\end{dfn}
	\end{screen}
	
	\monologue{
		院生「$f$が写像であることを述べる前に商写像と名前を付けましたが,以下に示す通り
			$f$は$a$から$a/R$への全射となっています.また商写像は
			{\bf 自然な全射}\index{しぜんなぜんしゃ@自然な全射}{\bf (natural surjection)}や
			{\bf 標準的全射}\index{ひょうじゅんてきぜんしゃ@標準的全射}{\bf (canonical surjection)}
			とも呼ばれます.」
	}
	
	\begin{screen}
		\begin{thm}[商写像は全射である]\label{thm:quotient_mapping_is_a_surjection}
			$a$を集合とし,$R$を$a$上の同値関係として,$y$を$a$の要素とするとき$y$の$R$に関する同値類を
			$[y]$で表す.このとき次が成り立つ:
			\begin{align}
				\forall x\ \left(\ x \in a/R \Longleftrightarrow \exists y \in a\ (\ x=[y]\ )\ \right).
				\label{eq:thm_quotient_mapping_is_a_surjection}
			\end{align}
			特に$a$から$a/R$への商写像は写像であり,さらに言えば全射である.
		\end{thm}
	\end{screen}
	
	\begin{prf}
		$a$が空である場合は$a/R$が空となるので,空虚な真より(\refeq{eq:thm_quotient_mapping_is_a_surjection})
		が成り立つ.また商写像も空となり,空写像は空集合から空集合への全単射であるから
		主張は全て従う.以下では$a$が空でない場合で証明する.
	\end{prf}
	
	\begin{screen}
		\begin{thm}[商集合の性質]
			$a$を集合とし,$R$を$a$上の同値関係として,$y$を$a$の要素とするとき$y$の$R$に関する同値類を
			$[y]$で表す.このとき次が成り立つ:
			\begin{description}
				\item[(1)] $a/R \in \Univ$
				\item[(2)] $a = \bigcup (a/R)$
			\end{description}
		\end{thm}
	\end{screen}
	
	\begin{prf} $a$が空であれば$a/R$は空となり(1)が成立する.また$\emptyset = \bigcup \emptyset$より
		(2)も成立する.以下では$a$が空でない場合で証明する.
		\begin{description}
			\item[(1)] $a/R$は$a$から$a/R$への商写像の値域であるから,置換公理より$a/R \in \Univ$が従う.
			\item[(2)] $\tau$を$\mathcal{L}$の任意の対象とすれば,$\tau \in a$ならば
				\begin{align}
					\tau \in [\tau]
				\end{align}
				となるから$\tau \in \bigcup (a/R)$が成立する.ゆえに
				\begin{align}
					\tau \in a \Longrightarrow \tau \in \bigcup (a/R)
				\end{align}
				が得られる.逆に$\tau \in \bigcup (a/R)$が成り立っているとすれば
				$\tau$に対して$a$の或る要素$y$が取れて$\tau \in [y]$となるが,
				$[y] \subset a$より$\tau \in a$が従うので
				\begin{align}
					\tau \in \bigcup (a/R) \Longrightarrow \tau \in a
				\end{align}
				も得られる.$\tau$の任意性より$a = \bigcup (a/R)$が出る.
				\QED
		\end{description}
	\end{prf}
	
	\begin{screen}
		\begin{thm}[半群の群への拡張]\label{thm:extension_of_semigroup}
			$(a,o)$を簡約的可換半群とするとき,次を満たす群$(G,\Gamma)$が存在する:
			\begin{itemize}
				\item $a \subset G$.
				\item $\forall x,y \in a\ (\ o(x,y) = \Gamma(x,y)\ )$.
			\end{itemize}
		\end{thm}
	\end{screen}
	
	\begin{screen}
		\begin{dfn}[整数]
			$(\omg,\sigma)$が生成する群を$(\Z,+)$で表す.そして$\Z$の要素を
			{\bf 整数}\index{せいすう@整数}{\bf (integer)}と呼ぶ.
		\end{dfn}
	\end{screen}
	
\section{有理数}
	\begin{screen}
		\begin{thm}[分数体]\label{thm:field_of_fractions}
			環$R$に対し,$R$が整域であるということと$R$が或る体の部分環であるということは同値である.
			$R$を整域とするとき,$R$を部分環として含む最小の体は$R$の{\bf 分数体}
			\index{ぶんすうたい@分数体}{\bf (field of fractions)}と呼ばれる.
		\end{thm}
	\end{screen}
	
	$\Z$は整域であるから,定理\ref{thm:field_of_fractions}より$\Z$を部分環として含む
	体$F$が存在する.$\Z$の任意の要素$n$に対し,$n$が$0$でなければ$F$の中に$n^{-1}$が存在するが,
	この乗法に関する逆元を用いれば$\Z$を部分環として含む最小の体は
	\begin{align}
		\Set{x}{\exists n,m \in \Z\ (\ x = n \cdot m^{-1} \wedge m \neq 0\ )}
	\end{align}
	と書ける.この集合を$\Q$で表し,{\bf 有理数体}\index{ゆうりすうたい@有理数体}{\bf (field of rationals)}と呼ぶ.

\section{実数}
	\begin{screen}
		\begin{dfn}[Dedekind切断]
			$\Q$の任意の部分集合$A$に対して,順序対$(\Q \backslash A,A)$が
			{\bf Dedekind切断}\index{Dedekindせつだん@Dedekind切断}{\bf (Dedekind cut)}であるということを
			\begin{align}
				\mbox{順序対$(\Q \backslash A,A)$がDedekind切断である} \Longleftrightarrow\ 
				&A \neq \emptyset \wedge A \neq \Q\ \wedge \\
				&\forall x \in \Q \backslash A\ \forall y \in A\ (\ x < y\ )\ \wedge \\
				&\forall x \in A\ \exists y \in A\ (\ y < x\ )
			\end{align}
			で定義する.
		\end{dfn}
	\end{screen}
	
	\monologue{
		院生「Dedekind切断とは数直線を左右に分割する操作をイメージしますね.例えば
			\begin{align}
				A = \Set{q \in \Q}{0 < q}
			\end{align}
			に対して$(\Q \backslash A,A)$はDedekind切断となります.
			実数の構成においてこの集合$A$は重要ですから,これを$\Q_+$と表して後で使いましょう.
			上の定義では$(\Q \backslash A,A)$がDedekind切断であるというとき
			$A$が最小元をもたないことを条件に入れましたが,ここは
			`$\Q \backslash A$が最大元を持たない'という条件に取り替えても構いません.」
	}
	
	いま$R = \Set{x}{\mbox{$(\Q \backslash x,x)$はDedekind切断である}}$として$R$を定め,
	\begin{align}
		T = \Set{x}{\exists a,b \in R\ (\ x = (a,b) \wedge b \subset a\ )}
	\end{align}
	と定める.この$T$は$R$上の全順序となる.
	任意の$a,b \in R$に対して,$a \not\subset b$ならば
	或る有理数$x$が$x \in a$かつ$x \notin b$を満たす.
	このとき$b$の任意の要素$y$に対して$x < y$となり,
	$x \in a$かつ$x < y$より$y \in a$となるので$b \subset a$が成り立つ.ゆえに
	\begin{align}
		\rightharpoondown (a \subset b) \Longrightarrow b \subset a
	\end{align}
	が得られた.これは$a \subset b \vee b \subset a$と同値であるから$T$は全順序である.
	
	\monologue{
		さて,高校まで扱ってきた数は`切れ目'がありませんでした.
		つまり,まるで時間の流れのように数直線は`連続'していたのです.
		集合論のことばで`数の連続性'を規定するとどうなるでしょう.
		それには同値な条件がいくつかありますが,今回述べるものは
		`上に有界な部分集合は上限を有する'という性質です.
	}
	
	$X$を$R$の部分集合で,$X \neq \emptyset$かつ$X$は$R$において上に有界であるとする.
	このとき$\bigcup X$は$X$の上限となる.
	
	\begin{screen}
		\begin{thm}
			$\Q$の部分集合$A$に対して$(\Q \backslash A,A)$をDedekind切断とするとき,
			次が成り立つ:
			\begin{description}
				\item[(1)] $\forall q \in \Q\ (\ \exists a \in A\ (\ a < q\ )\Longleftrightarrow q \in A\ )$.
				\item[(2)] $\forall q \in \Q\ (\ \exists a \in \Q \backslash A\ (\ q < a\ )\Longrightarrow q \in \Q \backslash A\ )$.
			\end{description}
		\end{thm}
	\end{screen}
	
	\begin{prf}
		$q$を任意の有理数とすれば,$A$は最小元を持たないので
		\begin{align}
			q \in A \Longrightarrow \exists a \in A\ (\ a < q\ )
		\end{align}
		となる.逆に$q \notin A$ならば$A$の任意の要素$a$に対して$q < a$となるから,対偶を取って
		\begin{align}
			\exists a \in A\ (\ a < q\ ) \Longrightarrow q \in A
		\end{align}
		を得る.$q \notin \Q \backslash A$ならば$A$の任意の要素$a$に対して$a < q$となるから,
		対偶を取って(2)を得る.
		\QED
	\end{prf}
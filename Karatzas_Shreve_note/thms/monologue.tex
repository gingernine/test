%院生の捨て独白集
%
	\monologue{
		院生「現代的な数学では,数や関数など数学に関するあらゆるものは集合で構成されます.
			そして集合そのものは述語論理を基礎にして公理的に規定されます.
			この意味で集合論の勉強には論理学の知識が必要であると聞きますけれども,
			真に受けて論理学の本を眺めてみれば,はじめから集合そのものが出てきたり,
			変数に数で添え字をつけたり,述語関数などといったものを取り扱っていたりしているものばかりで残念です.
			論理学を基に集合論を展開しようというのですから,集合論の諸概念を予定して論理学を説明するのは本末転倒です.
			とはいえ集合論と論理学とは切っても切り離せないのですから,いっそ同時並行でそつなく理解してやりましょう.
			(いわゆるメタ数学についてはいまのところ手を出すつもりはありません.)」
	}
	
	\begin{quote}
		初めに言(ことば)があった。言は神と共にあった。言は神であった。\\
		この言は、初めに神と共にあった。\\
		万物は言によって成った。成ったもので、言によらずに成ったものは何一つなかった。
	\end{quote}
	ヨハネによる福音書の冒頭である.本稿の世界もまた数学のことば,言い換えれば論理のみによって創られる(予定).
	現代的な数学では,数や関数など数学に関するあらゆるものは集合で構成される.
	
	\monologue{
		院生「私の指導教官に``新約聖書がはじめにギリシア語で書かれたとき,`ことば'にはlogosが充てられた.
			logosは`言語'の意味を持つと同時に`論理'の意味も持つ''と教わりました.
			つまり,ギリシア語版の福音書では``初めに論理があった''とも解釈できるのですね.
			一方で日本語訳では言葉ではなく言と書かれています.なぜ``言葉''ではなく``言''と書くのでしょうか.
			一説によれば言葉の葉の字の由来は万葉古今集仮名序にあり,
			現代的に説明すれば,見聞きしたり感動したりしたところを種にして生じる語彙のことを木の葉に喩えているらしいです.
			言葉は人が発するものであり,たいていの場合食い違いなく通用するのですから,すなわち
			葉が付かない``言''とは,人為の介入する前から世界を認識し,人が自覚する前から人の心に通底している
			コードとでも解釈されるでしょうか.聖書の引用文の通り%は森羅万象はことばによって成り,ことばによって尽くされるという意味であるから,
			キリスト教においてことばとは神であり森羅万象を超越しているのですから,言の字に神性を伴わせても良いですよね.
			本稿の世界もまた数学のことばによって創られますが,``はじめにことばありき''の名句が国籍や文化を問わず
			現代まで受け入れられてきたという事実を鑑みれば,ことばから始めようというのは人が生来持っている直観に対して自然な起こりなのでしょう.」
			%しかしながら,神なることばが世界の悉くを尽くせる一方で,人が創造する数学の世界は論理のみによっては完結し得ないという事実もあります.
	}
	
	\monologue{
		院生「集合論の言語の設定は思いの外厄介ですね.いや,私にとって厄介というだけですが.
			一旦言語を設定してはみるものの,行き詰れば設定をやり直すことになりますから,
			以下記述する内容はあくまで仮の形です.それから,私自身集合論も論理学も
			ド素人ですから,言語に対する認識が専門家とズレていることも十分あり得ます.
			勉強を進める中で自分の誤解に気付けばその時点で全てやり直しです.
			見る人が見れば滑稽千万な破綻が見つかるかもしれませんが,
			しかし私としてはHilbertの形式主義,つまり文字と特殊記号を一定の法則で並べただけの
			無意味な記号列に対して推論規則や公理により形式上の意味を付けるという姿勢を
			貫いているつもりです.予防的な言い訳はこの程度にして,本論に入りましょう.」
	}
	
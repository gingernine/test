\section{Continuous Time Martingales}
\subsection{Fundamental Inequalities}
	
	\begin{itembox}[l]{Lemma: 凸関数の片側微係数の存在}
		任意の凸関数$\varphi:\R \longrightarrow \R$は
		各点で左右の微係数が存在する.特に,凸関数は連続である.
	\end{itembox}
	
	\begin{prf}
		凸性より任意の$x < y < z$に対して
		\begin{align}
			\frac{\varphi(y) - \varphi(x)}{y - x} 
			\leq \frac{\varphi(z) - \varphi(x)}{z - x}
			\leq \frac{\varphi(z) - \varphi(y)}{z - y}
			\label{ineq:lem:convex_function_measurability_1}
		\end{align}
		が満たされる.従って,$x$を固定すれば,$x$に単調減少に近づく任意の点列$(x_n)_{n=1}^{\infty}$に対し
		 \begin{align}
		 	\left(\frac{f(x_n)-f(x)}{x_n-x}\right)_{n=1}^{\infty} 
		 	\label{seq:lem:convex_function_measurability_2}
		 \end{align}
		 は下に有界な単調減少列となり下限が存在する.$x$に単調減少に近づく別の点列$(y_k)_{k=1}^{\infty}$を取れば
		 \begin{align}
		 	\inf{k \in \N}{\frac{f(y_k)-f(x)}{y_k-x}} \leq \frac{f(x_n)-f(x)}{x_n-x} \quad (n=1,2,\cdots)
		 \end{align}
		 より
		 \begin{align}
		 	\inf{k \in \N}{\frac{f(y_k)-f(x)}{y_k-x}} \leq \inf{n \in \N}{\frac{f(x_n)-f(x)}{x_n-x}}
		 \end{align}
		 が成立し,$(x_n),(y_k)$の立場を変えれば逆向きの不等号も得られる.
		 すなわち極限は点列に依らず確定し,$\varphi$は$x$で右側微係数を持つ.
		 同様に左側微係数も存在し,特に$\varphi$の連続性及びBorel可測性が従う.
		 \QED
	\end{prf}
	
	\begin{itembox}[l]{Lemma: Jensenの不等式}
		$(\Omega,\mathscr{F},P)$を確率空間,
		$X:\Omega \longrightarrow \R$を可測$\mathscr{F}/\borel{\R}$とする.
		このとき,任意の部分$\sigma$-加法族$\mathscr{G} \subset \mathscr{F}$及び
		凸関数$\varphi:\R \longrightarrow \R$に対し,
		$X,\varphi(X)$が$\Omega$上$P$に関して可積分であるなら次が成立する:
		\begin{align}
			\varphi\left(\cexp{X}{\mathscr{G}} \right)
			\leq \cexp{\varphi(X)}{\mathscr{G}},
			\quad \mbox{$P$-a.s.}
		\end{align}
	\end{itembox}
		\begin{prf}
			$\varphi$は各点$x \in \R$で右側接線を持つから,
			それを$t \longmapsto a_x t + b_x$と表せば,
			\begin{align}
				\varphi(t) = \sup{x \in \R}{\left\{ a_x t + b_x \right\}} \quad (\forall t \in \R)
				\label{eq:prp_properties_of_expanded_conditional_expectation_1}
			\end{align}
			が成立する.
			よって任意の$x \in \R$に対して
			\begin{align}
				\varphi(X(\omega)) \geq a_x X(\omega) + b_x
			\end{align}
			が満たされるから
			\begin{align}
				\cexp{\varphi(X)}{\mathcal{G}}
				\geq a_x \cexp{X}{\mathcal{G}} + b_x \quad (\forall x \in \R),
				\quad \mbox{$P$-a.s.}
			\end{align}
			が従い,$x \in \R$の任意性と
			(\refeq{eq:prp_properties_of_expanded_conditional_expectation_1})より
			\begin{align}
				\cexp{\varphi(X)}{\mathcal{G}} \geq \varphi\left( \cexp{X}{\mathcal{G}} \right),
				\quad \mbox{$P$-a.s.}
			\end{align}
			が得られる.
			\QED
		\end{prf}
		
	\begin{itembox}[l]{Proposition 3.6}
		Let $\Set{X_t,\mathscr{F}_t}{0 \leq t < \infty}$ be a martingale (respectively, 
		submartingale), and $\varphi:\R \longrightarrow \R$ a convex (respectively, convex 
		nondecreasing) function, such that $E\left|\varphi(X_t)\right| < \infty$ holds 
		for every $t \geq 0$. Then $\Set{\varphi(X_t),\mathscr{F}_t}{0 \leq t < \infty}$ 
		is a submartingale.
	\end{itembox}
	
	\begin{prf}
		$(X_t)_{t \geq 0}$がマルチンゲールであり
		$\varphi$が凸であるとき,Jensenの不等式より
		$P$-a.s.の$\omega \in \Omega$に対し
		\begin{align}
			\varphi(X_s(\omega))
			= \varphi(\cexp{X_t}{\mathscr{F}_s}(\omega))
			\leq \cexp{\varphi(X_t)}{\mathscr{F}_s}(\omega)
		\end{align}
		が成り立つ.$(X_t)_{t \geq 0}$が劣マルチンゲールであり
		$\varphi$が凸かつ単調増大であるとき,$P$-a.s.の$\omega \in \Omega$に対し
		\begin{align}
			\varphi(X_s(\omega))
			\leq \varphi(\cexp{X_t}{\mathscr{F}_s}(\omega))
			\leq \cexp{\varphi(X_t)}{\mathscr{F}_s}(\omega)
		\end{align}
		が成り立つ.
		\QED
	\end{prf}
	
	\begin{itembox}[l]{離散時間の任意抽出定理}
	\end{itembox}
	
	\begin{itembox}[l]{Theorem 3.8 (i)}
		Let $\Set{X_t,\mathscr{F}_t}{0 \leq t < \infty}$ be a submartingale 
		whose every path is right-continuous, let $[\sigma, \tau]$ be a subinterval of 
		$[0,\infty)$, and let $\alpha < \beta,\ \lambda > 0$ be real numbers. We have the 
		following results:
		\begin{description}
			\item[(i)] First submartingale inequality:
				\begin{align}
					\lambda \cdot P\left[ \sup{\sigma \leq t \leq \tau}{X_t} \geq \lambda \right]
					\leq E(X^+_\tau).
				\end{align}
		\end{description}
	\end{itembox}
	
	\begin{prf}
		$n \geq 1$に対し$[\sigma,\tau]$を$2^n$等分に分割し
		\begin{align}
			E_n &\coloneqq \left\{ \max{k=0,1,\cdots,2^n}{X_{\sigma + \frac{k}{2^n}(\tau - \sigma)} > \lambda} \right\}, \\
			E_n^0 &\coloneqq \left\{ X_\sigma > \lambda \right\}, \quad
			E_n^m \coloneqq \left\{ \max{k=0,1,\cdots,m-1}{X_{\sigma + \frac{k}{2^n}(\tau - \sigma)}} \leq \lambda,\ X_{\sigma + \frac{m}{2^n}(\tau - \sigma)} > \lambda \right\},
			\quad (1 \leq m \leq 2^n)
		\end{align}
		とおけば,
		\begin{align}
			E_n^m \in \mathscr{F}_{\sigma + \frac{m}{2^n}(\tau - \sigma)} \subset \mathscr{F}_\tau,
			\quad E_n = \sum_{m=0}^{2^n} E_n^m,
			\quad (n=1,2,\cdots)	
		\end{align}
		かつ,$E_1 \subset E_2 \subset E_3 \subset \cdots$と$X$のパスの右連続性より
		\begin{align}
			\left\{ \sup{\sigma \leq t \leq \tau}{X_t} > \lambda \right\}
			= \bigcup_{n=1}^\infty \left\{ \max{k=0,1,\cdots,2^n}{X_{\sigma + \frac{k}{2^n}(\tau - \sigma)} > \lambda} \right\}
			= \lim_{n \to \infty} E_n
		\end{align}
		が満たされる.いま,任意の$n \geq 1$についてChebyshevの不等式と劣マルチンゲール性より
		\begin{align}
			P(E_n) = \sum_{m=0}^{2^n} P(E_n^m)
			\leq \frac{1}{\lambda} \sum_{m=0}^{2^n} \int_{E_n^m} X_{\sigma + \frac{m}{2^n}(\tau - \sigma)}\ dP
			\leq \frac{1}{\lambda} \sum_{m=0}^{2^n} \int_{E_n^m} X_{\tau}\ dP
			= \frac{1}{\lambda} \int_{E_n} X_\tau\ dP
			\leq \frac{1}{\lambda} E(X^+_\tau)
		\end{align}
		となるから,$n \longrightarrow \infty$として
		\begin{align}
			P\left[ \sup{\sigma \leq t \leq \tau}{X_t} > \lambda \right]
			\leq \frac{1}{\lambda} E(X^+_\tau)
		\end{align}
		を得る.特に,任意の$m \in \N$に対して
		\begin{align}
			P\left[ \sup{\sigma \leq t \leq \tau}{X_t} > \lambda - \frac{1}{m} \right]
			\leq \frac{1}{\lambda - 1/m} E(X^+_\tau)
		\end{align}
		が成り立ち,$m \longrightarrow \infty$として
		\begin{align}
			P\left[ \sup{\sigma \leq t \leq \tau}{X_t} \geq \lambda \right]
			\leq \frac{1}{\lambda} E(X^+_\tau)
		\end{align}
		が従う.
		\QED
	\end{prf}
	
	\begin{itembox}[l]{Theorem 3.8 (ii)}
		Second submartingale inequality:
		\begin{align}
			\lambda \cdot P\left[ \sup{\sigma \leq t \leq \tau}{X_t} \leq -\lambda \right]
					\leq E(X^+_\tau) - E(X_\sigma).
		\end{align}
	\end{itembox}
	
	\begin{prf}
		$n \geq 1$に対し$[\sigma,\tau]$を$2^n$等分に分割し
		\begin{align}
			E_n &\coloneqq \left\{ \min{k=0,1,\cdots,2^n}{X_{\sigma + \frac{k}{2^n}(\tau - \sigma)} < -\lambda} \right\}, \\
			E_n^0 &\coloneqq \left\{ X_\sigma < -\lambda \right\}, \quad
			E_n^m \coloneqq \left\{ \min{k=0,1,\cdots,m-1}{X_{\sigma + \frac{k}{2^n}(\tau - \sigma)}} \geq -\lambda,\ X_{\sigma + \frac{m}{2^n}(\tau - \sigma)} < -\lambda \right\},
			\quad (1 \leq m \leq 2^n)
		\end{align}
		として,また
		\begin{align}
			T(\omega) \coloneqq
			\begin{cases}
				\sigma + \frac{m}{2^n} (\tau - \sigma), & (\omega \in E_n^m,\ m=0,1,\cdots,2^n),\\
				\tau, & (\omega \in \Omega \backslash E_n),
			\end{cases}
			\quad (\forall \omega \in \Omega)
		\end{align}
		により$(\mathscr{F}_t)$-停止時刻を定めれば,任意抽出定理より
		\begin{align}
			E(X_\sigma) 
			&\leq E(X_T)
			= \sum_{m=0}^{2^n} \int_{E_n^m} X_{\sigma+\frac{m}{2^n}(\tau-\sigma)}\ dP + \int_{\Omega \backslash E_n} X_\tau\ dP
			\leq \sum_{m=0}^{2^n} (-\lambda) P(E_n^m) + E(X_\tau^+) \\
			&= -\lambda P(E_n) + E(X_\tau^+)
		\end{align}
		が成立する.移項して$n \longrightarrow \infty$とすれば
		\begin{align}
			P\left[ \inf{\sigma \leq t \leq \tau}{X_t} < -\lambda \right]
			\leq \frac{1}{\lambda} \left\{ E(X^+_\tau) - E(X_\sigma) \right\} 
		\end{align}
		が得られ,(i)の証明と同様にして
		\begin{align}
			P\left[ \inf{\sigma \leq t \leq \tau}{X_t} \leq -\lambda \right]
			\leq \frac{1}{\lambda} \left\{ E(X^+_\tau) - E(X_\sigma) \right\}
		\end{align}
		が従う.
		\QED
	\end{prf}
	
	\begin{itembox}[l]{Lemma: Theorem 3.8 (iii)}
		確率過程$X = \Set{X_t}{0 \leq t < \infty}$のすべてのパスが右連続であるとき,
		$[\sigma,\tau]$の$2^n$等分点を
		\begin{align}
			F_n \coloneqq \Set{\tau_i^n}{\tau_i^n = \sigma + \frac{i}{2^n}(\tau - \sigma),\ i=0,1,\cdots,2^n},
			\quad n=1,2,\cdots
		\end{align}
		とおけば次が成立する:
		\begin{align}
			U_{[\sigma,\tau]}(\alpha,\beta;X)
			= \sup{n \in \N}{U_{F_n}(\alpha,\beta;X)},
			\quad D_{[\sigma,\tau]}(\alpha,\beta;X)
			= \sup{n \in \N}{D_{F_n}(\alpha,\beta;X)}.
		\end{align}
	\end{itembox}
	Karatzas-Shreve本文中では
	\begin{align}
		\tau_1(\omega) = \min{}{\Set{t \in F}{X_t(\omega) \leq \alpha}}
	\end{align}
	と定めているが,
	\begin{align}
		\tau_1(\omega) = \min{}{\Set{t \in F}{X_t(\omega) < \alpha}}
	\end{align}
	と定める方がよい.実際,こうでないと今の補題が従わない.また$\sigma_0 \equiv 0,\ \tau_0 \equiv 0$と考える.
	\begin{prf}
		$U_{[\sigma,\tau]}(\alpha,\beta;X) \leq \sup{n \in \N}{U_{F_n}(\alpha,\beta;X)}$
		が成立すれば主張を得る.いま,任意に有限部分集合$F \subset [\sigma,\tau]$を取り
		\begin{align}
			\tau_1(\omega) \coloneqq \min{}{\Set{t \in F}{X_t(\omega) < \alpha}},
			\quad \sigma_1(\omega) \coloneqq \min{}{\Set{t \in F}{t \geq \tau_1(\omega),\ X_t(\omega) > \beta}},
			\cdots
		\end{align}
		を定め,$\omega \in \Omega$を任意に取り
		$U_F(\alpha,\beta;X(\omega)) = j \geq 1$と仮定する.このとき
		\begin{align}
			X_{\tau_i(\omega)}(\omega) < \alpha,
			\quad X_{\sigma_i(\omega)}(\omega) > \beta,
			\quad (i=1,\cdots,j)
		\end{align}
		が満たされ,$t \longrightarrow X_t(\omega)$の右連続性より,
		十分大きい$n \in \N$に対して或る$t_i,s_i \in F_n,\ (1 \leq i \leq j)$が
		\begin{align}
			\tau_1(\omega) \leq t_1 < \sigma_1(\omega) \leq s_1 < 
			\cdots < \tau_j(\omega) \leq t_j < \sigma_j(\omega) \leq s_j
		\end{align}
		かつ
		\begin{align}
			X_{t_i}(\omega) < \alpha,
			\quad X_{s_i}(\omega) > \beta,
			\quad (\forall i=1,\cdots,j)
		\end{align}
		を満たす.これにより
		\begin{align}
			U_F(\alpha,\beta;X(\omega)) = j \leq U_{F_n}(\alpha,\beta;X(\omega))
		\end{align}
		が従い,$\omega$の任意性より$U_F(\alpha,\beta;X) \leq \sup{n \in \N}{U_{F_n}(\alpha,\beta;X)}$が出る.
		\QED
	\end{prf}
	
	\begin{itembox}[l]{Theorem 3.8 (iii)}
		Upcrossing and downcrossing inequalities:
		\begin{align}
			E U_{[\sigma,\tau]}(\alpha,\beta;X) \leq \frac{E(X_\tau^+) + |\alpha|}{\beta-\alpha},
			\quad E D_{[\sigma,\tau]}(\alpha,\beta;X) \leq \frac{E(X_\tau-\alpha)^+}{\beta-\alpha}.
		\end{align}
	\end{itembox}
	
	\begin{prf}\mbox{}
		\begin{description}
			\item[第一段]
				有限部分集合$F = \{t_1,\cdots,t_n\} \subset [\sigma,\tau]$に対し
				\begin{align}
					\tau_1(\omega) \coloneqq \min{}{\Set{t \in F}{X_t(\omega) < \alpha}},
					\quad \sigma_1(\omega) \coloneqq \min{}{\Set{t \in F}{t \geq \tau_1(\omega),\ X_t(\omega) > \beta}},
					\cdots
				\end{align}
				で定める$\tau_i,\sigma_i,\ (i=1,2,\cdots)$が$(\mathscr{F}_t)$-停止時刻であることを示す.
				実際,任意の$t_j \in F$に対して
				\begin{align}
					\left\{\tau_1 = t_j\right\} &= \left[\bigcap_{k=1}^{j-1} \left\{ X_{t_k} \geq \alpha \right\}\right] \cap \left\{ X_{t_j} < \alpha \right\} \in \mathscr{F}_{t_j}, \\
					&\vdots \\
					\left\{\tau_i = t_j\right\} &= \bigcup_{r=1}^{j-1} \left[ \{\sigma_{i-1} = t_r\} \cap \bigcap_{k=r}^{j-1} \left\{ X_{t_k} \geq \alpha \right\}\right] \cap \left\{ X_{t_j} < \alpha \right\}   \in \mathscr{F}_{t_j}, \\
					\left\{\sigma_i = t_j\right\} &= \bigcup_{r=1}^{j-1} \left[ \{\tau_i = t_r\} \cap \bigcap_{k=r}^{j-1} \left\{ X_{t_k} \leq \beta \right\}\right] \cap \left\{ X_{t_j} > \beta \right\}   \in \mathscr{F}_{t_j}
				\end{align}
				が成立するから$\left\{ \tau_i \leq t \right\} \in \mathscr{F}_t\ (\forall t \geq 0)$
				が満たされる.
				\begin{align}
					\tau_1(\omega) \coloneqq \min{}{\Set{t \in F}{X_t(\omega) > \beta}},
					\quad \sigma_1(\omega) \coloneqq \min{}{\Set{t \in F}{t \geq \tau_1(\omega),\ X_t(\omega) < \alpha}},
					\cdots
				\end{align}
				により$\tau_i,\sigma_i,\ (i=1,2,\cdots)$を定めてもこれらは$(\mathscr{F}_t)$-停止時刻となる.
				
			\item[第二段]
				補題の有限部分集合$F_n \subset [\sigma,\tau]$に対し
				\begin{align}
					E U_{F_n}(\alpha,\beta;X) \leq \frac{E(X_\tau^+) + |\alpha|}{\beta-\alpha}
					\label{eq:chapter_1_Theorem_3_8_1}
				\end{align}
				が成立することを示せば,単調収束定理より
				\begin{align}
					E U_{[\sigma,\tau]}(\alpha,\beta;X)
					= E \left( \sup{n \in \N}{U_{F_n}(\alpha,\beta;X)} \right)
					\leq \frac{E(X_\tau^+) + |\alpha|}{\beta-\alpha}
				\end{align}
				が従う.実際,$j = U_{F_n}(\alpha,\beta;X(\omega))$ならば
				$\sigma_j(\omega) \leq \tau < \tau_{j+1}(\omega)$或は
				$\tau_{j+1}(\omega) \leq \tau < \sigma_{j+1}(\omega)$であるから
				\begin{align}
					Z(\omega) \coloneqq
					\sum_{i=1}^\infty \left( X_{\tau_{i+1}(\omega)\wedge\tau}(\omega) - X_{\sigma_i(\omega)\wedge\tau}(\omega) \right)
					= \begin{cases}
						\displaystyle\sum_{i=1}^j \left( X_{\tau_{i+1}(\omega)}(\omega) - X_{\sigma_i(\omega)}(\omega) \right), & (1), \\
						\displaystyle\sum_{i=1}^{j-1} \left( X_{\tau_{i+1}(\omega)}(\omega) - X_{\sigma_i(\omega)}(\omega) \right) + \left( X_{\tau}(\omega) - X_{\sigma_j(\omega)}(\omega) \right), & (2)
					\end{cases}
				\end{align}
				となり,$X_{\tau_i} < \alpha,\ X_{\sigma_i} > \beta$より
				\begin{align}
					(2)
					&= \sum_{i=1}^{j-1} \left( X_{\tau_{i+1}(\omega)}(\omega) - X_{\sigma_i(\omega)}(\omega) \right) + \left( X_{\tau}(\omega) - \alpha \right) + 
					\left( \alpha - X_{\sigma_j(\omega)}(\omega) \right)
					\leq j(\alpha - \beta) + X^+_{\tau}(\omega) + |\alpha| \\
					&= U_{F_n}(\alpha,\beta;X(\omega))(\alpha - \beta) + X^+_{\tau}(\omega) + |\alpha|
				\end{align}
				及び
				\begin{align}
					(1)
					\leq j (\alpha - \beta)
					\leq j (\alpha - \beta) + X^+_{\tau}(\omega) + |\alpha|
					= U_{F_n}(\alpha,\beta;X(\omega))(\alpha - \beta) + X^+_{\tau}(\omega) + |\alpha|
				\end{align}
				が満たされ
				\begin{align}
					E Z \leq (\alpha - \beta) EU_{F_n}(\alpha,\beta;X) + E(X_\tau^+) + |\alpha|
				\end{align}
				が従う.$\Set{X_t,\mathscr{F}_t}{0 \leq t < \infty}$の劣マルチンゲール性と任意抽出定理より
				\begin{align}
					E\left( X_{\tau_{i+1} \wedge \tau} - X_{\sigma_i \wedge \tau} \right) \geq 0,
					\quad (i=1,\cdots,j)
				\end{align}
				が成り立つから,$E Z \geq 0$となり(\refeq{eq:chapter_1_Theorem_3_8_1})が得られる.
				
			\item[第三段]
				
		\end{description}
	\end{prf}
	
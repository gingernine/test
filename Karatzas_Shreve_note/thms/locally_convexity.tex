\subsection{局所凸}
	\begin{screen}
		\begin{thm}[ゼロの近傍はスケール変換によって任意の点を併呑する]
		\label{thm:neighbor_of_zero_is_absorbing}
			$\left(\left(X,\sigma_X\right),(\Phi,+,\bullet),s,\mathscr{O}_X\right)$を位相線型空間とし,
			$0_X$を$\left(X,\sigma_X\right)$の単位元とし,
			$u$を$0_X$の近傍とし,$x$を$X$の要素とする.このとき
			\begin{align}
				\forall t \in \R_+\,
				\left[\, r < t \Longrightarrow x \in \Set{s(t,z)}{z \in u}\, \right]
			\end{align}
			を満たす正の実数$r$が取れる.
		\end{thm}
	\end{screen}
	
	直感的に書き直せば,$x$に対して
	\begin{align}
		r < t \Longrightarrow x \in t \cdot u
	\end{align}
	を満たす実数$r$が取れるということである.
	
	\begin{sketch}
		スケール変換は連続であるから,
		\begin{align}
			\Phi \ni \alpha \longmapsto s(\alpha,x)
		\end{align}
		なる写像を$\varphi$とおけば
		\begin{align}
			\Set{\alpha \in \Phi}{|\alpha| < \delta} \subset \varphi^{-1} \ast u
		\end{align}
		を満たす正の実数$\delta$が取れる.そして$t$を
		\begin{align}
			\frac{1}{\delta} < t
		\end{align}
		なる任意の実数とすれば
		\begin{align}
			\frac{1}{t} < \delta
		\end{align}
		が成り立つから
		\begin{align}
			s(1/t,x) \in u
		\end{align}
		が成り立つ.そして
		\begin{align}
			x = s\left(1,x\right) = s\left(t \cdot (1/t),x\right) = s\left(t,s\left(1/t,x\right)\right)
		\end{align}
		であるから
		\begin{align}
			x \in \Set{s(t,z)}{z \in u}
		\end{align}
		が成立する.
		\QED
	\end{sketch}
	
	位相線型空間は一様化可能であるが,本節では逆に{\bf 線型位相を導入しうる近縁系とはどのようなものであるか}を考察する.
	そのような近縁系が具えているべき条件を見つけるには,位相線型空間を一様化する近縁系が持つ性質を洗い出せば良い.
	
	いま$\left(\left(X,\sigma_X\right),(\Phi,+,\bullet),s,\mathscr{O}_X\right)$を位相線型空間とし,
	$0_X$を$\left(X,\sigma_X\right)$の単位元とし,
	\begin{align}
		\mathscr{B}
	\end{align}
	を$\left(\left(X,\sigma_X\right),\mathscr{O}_X\right)$の局所基とし,
	その任意の要素$b$が均衡しているとする.つまり
	\begin{align}
		\forall \alpha \in \Phi\,
		\left[\, |\alpha| \leq 1 \Longrightarrow \forall x \in b\, \left(\, s(\alpha,x) \in b\, \right)\, \right].
	\end{align}
	また$\mathscr{U}$と$\mathscr{V}$を定理\ref{thm:topological_vector_spaces_are_uniformazable}の要領で構成する集合とする.
	その定め方より$\mathscr{U}$は$\mathscr{V}$の基本近縁系である.
	$\mathscr{V}$よりは$\mathscr{U}$の方が具体的に定められているから,
	$\mathscr{U}$が具えている性質を見る方が容易い.
	ちなみに$u$を$\mathscr{U}$の要素とすると
	\begin{align}
		u = \Set{(p,q)}{p \in X \wedge q \in X \wedge \sigma_X(-p,q) \in b}
	\end{align}
	を満たす$\mathscr{B}$の要素$b$が取れるが,このとき
	\begin{align}
		b = \Set{x}{\left(0_X,x\right) \in u}
	\end{align}
	が成り立つ.これは
	\begin{align}
		u[x] \defeq \Set{y}{(x,y) \in u}
	\end{align}
	なる表記を用いて直感的に書き直せば
	\begin{align}
		b = u[0_X]
	\end{align}
	が成り立つということである.つまり,当然のようだが,$\mathscr{U}$によって$\mathscr{B}$を復元できるのである.
	
	\begin{description}
		\item[(a)] $u$を$\mathscr{U}$の要素とし,$x$を$X$の要素とすると,
			\begin{align}
				\Set{y}{(x,y) \in u} = \Set{\sigma_X(x,z)}{(0_X,z) \in u}
				\label{fom:pre_thm_entourages_introducing_vector_topology_1}
			\end{align}
			が成立する.これを直感的に書き直せば
			\begin{align}
				u[x] = x + u[0_X]
			\end{align}
			となるが,意味としては{\bf $x$の基本近傍系は$0_X$の基本近傍系を$x$だけ平行移動すれば得られるということである.}
			実際,
			\begin{align}
				u = \Set{(p,q)}{p \in X \wedge q \in X \wedge \sigma_X(-p,q) \in b}
			\end{align}
			を満たす$\mathscr{B}$の要素$b$が取れて,$y$を$X$の任意の要素とすると
			\begin{align}
				(x,y) \in u &\Longleftrightarrow \sigma_X\left(-x,y\right) \in b \\
				&\Longleftrightarrow \sigma_X\left(-0_X,\sigma_X\left(-x,y\right)\right) \in b \\
				&\Longleftrightarrow \left(0_X,\sigma_X\left(-x,y\right)\right) \in u
			\end{align}
			が成り立ち,さらに
			\begin{align}
				y = \sigma_X\left(x,\sigma_X\left(-x,y\right)\right)
			\end{align}
			であるから
			\begin{align}
				\left(0_X,\sigma_X\left(-x,y\right)\right) \in u
				\Longleftrightarrow \exists z \in X\, \left(\, y=\sigma_X\left(x,z\right) 
				\wedge \left(0_X,z\right) \in u\, \right)
			\end{align}
			も成り立つ.ゆえに(\refeq{fom:pre_thm_entourages_introducing_vector_topology_1})が得られた.
		
		\item[(b)] $u$を$\mathscr{U}$の要素とし,$\alpha$を
			\begin{align}
				\alpha \neq 0
			\end{align}
			なる任意のスカラーとすれば,スケール変換は連続であるから
			\begin{align}
				\Set{s(\alpha,z)}{(0_X,z) \in u}
			\end{align}
			は$0_X$の近傍である.この集合は直感的には
			\begin{align}
				\alpha \cdot u[0_X]
			\end{align}
			と書けるが,近傍なのだから
			\begin{align}
				b \subset \alpha \cdot u[0_X]
			\end{align}
			を満たす$\mathscr{B}$の要素$b$が取れて,
			\begin{align}
				v \defeq \Set{(p,q)}{p \in X \wedge q \in X \wedge \sigma_X(-p,q) \in b}
			\end{align}
			により$\mathscr{U}$の要素$v$を定めれば
			\begin{align}
				v[0_X] \subset \alpha \cdot u[0_X]
			\end{align}
			が成り立つ.
			
		\item[(c)] $u$を$\mathscr{U}$の要素とすれば
			\begin{align}
				u[0_X]
			\end{align}
			は均衡集合である.
			
		\item[(d)] 定理\ref{thm:neighbor_of_zero_is_absorbing}より,
			$\mathscr{U}$の要素$u$と$X$の要素$x$が任意に与えられれば
			\begin{align}
				x \in r \cdot u[0_X]
			\end{align}
			を満たす正の実数$r$が取れる.
	\end{description}
	
	以上で$\mathscr{U}$の性質を四つ抜き出したが,逆にこれらを全て揃えている基本近縁系が取れるなら,
	近縁系によって導入する一様位相は線型位相である.
	
	\begin{screen}
		\begin{thm}[線型位相を導入する近縁系]
		\label{thm:entourages_introducing_vector_topology}
			$\left(\left(X,\sigma_X\right),(\Phi,+,\bullet),s\right)$を線型空間とし,
			$0_X$を$\left(X,\sigma_X\right)$の単位元とし,$\mathscr{V}$を$X$上の近縁系とし,
			$\mathscr{O}_X$を$\mathscr{V}$で導入する$X$上の一様位相とする.
			$\mathscr{V}$の基本近縁系$\mathscr{U}$で
			\begin{description}
				\item[(a)] $\mathscr{U}$の要素$u$と$X$の要素$x$が任意に与えられたときに
					\begin{align}
						\Set{y}{(x,y) \in u} = \Set{\sigma_X(x,z)}{(0_X,z) \in u}
					\end{align}
					が成り立つ.
						
				\item[(b)] $\mathscr{U}$の要素$u$と正の実数$t$が任意に与えられたときに
					\begin{align}
						\Set{z}{(0_X,z) \in v} \subset \Set{s(t,z)}{(0_X,z) \in u}
					\end{align}
					を満たす$\mathscr{U}$の要素$v$が取れる.
					
				\item[(c)] $\mathscr{U}$の要素$u$と$\Phi$の要素$\alpha$が任意に与えられたときに
					\begin{align}
						|\alpha| \leq 1 \Longrightarrow \Set{s(\alpha,z)}{(0_X,z) \in u} \subset \Set{z}{(0_X,z) \in u}
					\end{align}
					が成り立つ.
					
				\item[(d)] $u$を$\mathscr{U}$の要素とすると,$X$の要素$x$が任意に与えられたときに
					\begin{align}
						s\left(r^{-1},x\right) \in u[0_X]
					\end{align}
					を満たす正の実数$r$が取れる.
			\end{description}
			を満たすものが取れるとき,$\left(\left(X,\sigma_X\right),(\Phi,+,\bullet),s,\mathscr{O}_X\right)$
			は位相線型空間である.
		\end{thm}
	\end{screen}
	
	ちなみに$\sigma_X$が連続であるためには(a)が満たされていれば十分であり,
	(b)と(c)と(d)は$s$が連続であるための十分条件である.
	
	\begin{sketch}\mbox{}
		\begin{description}
			\item[第一段] $\mathscr{U}$の要素$u$と$X$の要素$a$に対し
				\begin{align}
					u[a] \defeq \Set{x}{(a,x) \in u}
				\end{align}
				と定める.また
				\begin{align}
					X \ni x \longmapsto \sigma_X(a,x)
				\end{align}
				なる写像を
				\begin{align}
					\sigma_X^a
				\end{align}
				と書く.
				
			\item[第二段]
				$\sigma_X$が$\left(0_X,0_X\right)$において連続であることを示す.
				$b$を$0_X$の近傍とすれば
				\begin{align}
					u[0_X] \subset b
				\end{align}
				を満たす$\mathscr{U}$の要素$u$と
				\begin{align}
					w \circ w \subset u
				\end{align}
				を満たす$\mathscr{U}$の要素$w$が取れる.このとき
				\begin{align}
					w[0_X] \times w[0_X] \subset \sigma_X^{-1} \ast b
					\label{fom:thm_entourages_introducing_vector_topology_1}
				\end{align}
				が成立する.実際,$x$と$y$を$w[0_X]$の要素とすると
				\begin{align}
					\sigma_X\left(x,y\right) \in \Set{\sigma_X(x,z)}{(0_X,z) \in w}
				\end{align}
				が成り立ち,(a)より
				\begin{align}
					\Set{\sigma_X(x,z)}{(0_X,z) \in w} = w[x]
				\end{align}
				であるから
				\begin{align}
					\left(x,\sigma_X\left(x,y\right)\right) \in w
				\end{align}
				が従う.ゆえにいま
				\begin{align}
					(0_X,x) \in w \wedge \left(x,\sigma_X\left(x,y\right)\right) \in w
				\end{align}
				が成り立っているので
				\begin{align}
					\left(0_X,\sigma_X\left(x,y\right)\right) \in u
				\end{align}
				が従う.ゆえに
				\begin{align}
					\left(x,y\right) \in w[0_X] \times w[0_X] \Longrightarrow
					\sigma_X\left(x,y\right) \in u[0_X]
				\end{align}
				が成り立つ.ゆえに(\refeq{fom:thm_entourages_introducing_vector_topology_1})が得られた.
				ゆえに$\sigma_X$は$\left(0_X,0_X\right)$で連続である.
				
			\item[第三段] $x$と$y$を$X$の要素として,$\sigma_X$が$(x,y)$において連続であることを示す.
				いま$b$を$\sigma_X\left(x,y\right)$の近傍とすると
				\begin{align}
					u\left[\sigma_X(x,y)\right] \subset b
				\end{align}
				を満たす$\mathscr{U}$の要素$u$が取れて,また
				\begin{align}
					w \circ w \subset u
				\end{align}
				を満たす$\mathscr{U}$の要素$w$も取れる.そして
				\begin{align}
					w[x]
				\end{align}
				は$x$の近傍であり,
				\begin{align}
					w[y]
				\end{align}
				は$y$の近傍である.そして(a)より
				\begin{align}
					w[x] = \Set{\sigma_X(x,p)}{(0_X,p) \in w}
				\end{align}
				であるから,$r$を$w[x]$の要素とすれば
				\begin{align}
					r = \sigma_X(x,p)
				\end{align}
				なる$X$の要素$p$が取れて,同様に$t$を$w[y]$の要素とすれば
				\begin{align}
					t = \sigma_X(y,q)
				\end{align}
				なる$X$の要素$q$が取れる.このとき前段の結果より
				\begin{align}
					\left(0_X,\sigma_X\left(p,q\right)\right) \in u
				\end{align}
				が成り立ち,かつ
				\begin{align}
					\sigma_X(r,t) &= \sigma_X\left(\sigma_X\left(x,p\right),\sigma_X\left(y,q\right)\right) \\
					&= \sigma_X\left(\sigma_X\left(\sigma_X\left(x,p\right),y\right),q\right) \\
					&= \sigma_X\left(\sigma_X\left(x,\sigma_X\left(p,y\right)\right),q\right) \\
					&= \sigma_X\left(\sigma_X\left(x,\sigma_X\left(y,p\right)\right),q\right) \\
					&= \sigma_X\left(\sigma_X\left(\sigma_X\left(x,y\right),p\right),q\right) \\
					&= \sigma_X\left(\sigma_X\left(x,y\right),\sigma_X\left(p,q\right)\right) \\
				\end{align}
				が成り立つので
				\begin{align}
					\sigma_X(r,t) \in \Set{\sigma_X\left(\sigma_X\left(x,y\right),z\right)}{(0_X,z) \in u}
				\end{align}
				が従う.ゆえに
				\begin{align}
					\sigma_X(r,t) \in u\left[\sigma_X(x,y)\right]
				\end{align}
				が従う.ゆえに
				\begin{align}
					w[x] \times w[y] \subset \sigma_X^{-1} \ast b
				\end{align}
				が従う.ゆえに$\sigma_X$は$(x,y)$において連続である.
				
			\item[第四段]
				$x$を$X$の要素とし,$\alpha$を$\Phi$の要素として,$s$が$(\alpha,x)$において連続であることを示す.
				いま$b$を$s(\alpha,x)$の近傍とすると
				\begin{align}
					u[s(\alpha,x)] \subset b
				\end{align}
				を満たす$\mathscr{U}$の要素$u$が取れて,また
				\begin{align}
					w \circ w \subset u
				\end{align}
				を満たす$\mathscr{U}$の要素$w$も取れる.(d)より
				\begin{align}
					s\left(r^{-1},x\right) \in w[0_X]
				\end{align}
				を満たす正の実数$r$が取れるので,ここで
				\begin{align}
					t \defeq \frac{r}{1+|\alpha| \cdot r}
				\end{align}
				とおく.このとき
				\begin{align}
					|\beta - \alpha| < \frac{1}{r}
				\end{align}
				なる$\Phi$の要素$\beta$に対して
				\begin{align}
					\left|\beta - \alpha\right| \cdot r < 1
				\end{align}
				が成り立ち,(c)より$w[0_X]$は均衡しているので
				\begin{align}
					s\left(\beta - \alpha,x\right)
					= s\left(\left(\beta - \alpha\right) \cdot r,s\left(r^{-1},x\right)\right)
					\in w[0_X]
					\label{fom:thm_entourages_introducing_vector_topology_2}
				\end{align}
				が従う.一方で$y$を
				\begin{align}
					s\left(t^{-1},\sigma_X\left(-x,y\right)\right) \in w[0_X]
				\end{align}
				を満たす$X$の要素とすれば,
				\begin{align}
					\left|\beta \cdot t\right| < \frac{1 + |\alpha| \cdot r}{r} \cdot \frac{r}{1+|\alpha| \cdot r} = 1
				\end{align}
				と$w[0_X]$が均衡していることから
				\begin{align}
					s\left(\beta,\sigma_X\left(-x,y\right)\right) 
					= s\left(\beta \cdot t,s\left(t^{-1},\sigma_X\left(-x,y\right)\right)\right) \in w[0_X]
					\label{fom:thm_entourages_introducing_vector_topology_3}
				\end{align}
				が従う.ところでこの$\beta$と$y$に対して
				\begin{align}
					\sigma_X\left(-s\left(\alpha,x\right), s\left(\beta,y\right)\right)
					= \sigma_X\left(s\left(\beta - \alpha,x\right),s\left(\beta,\sigma_X\left(-x,y\right)\right)\right)
				\end{align}
				が成り立つので,(\refeq{fom:thm_entourages_introducing_vector_topology_2})と
				(\refeq{fom:thm_entourages_introducing_vector_topology_3})及び
				(\refeq{fom:thm_entourages_introducing_vector_topology_1})から
				\begin{align}
					\sigma_X\left(-s\left(\alpha,x\right), s\left(\beta,y\right)\right) \in u[0_X]
				\end{align}
				が従う.ゆえに
				\begin{align}
					s\left(\beta,y\right) \in u[s(\alpha,x)]
				\end{align}
				が成立する.以上で
				\begin{align}
					\Set{\beta \in \Phi}{|\beta - \alpha| < \frac{1}{r}} \times
					\Set{\sigma_X\left(x,\zeta\right)}{\exists z \in X\, 
					\left[\, (0_X,z) \in w \wedge \zeta = s(t,z)\, \right]}
					\subset s^{-1} \ast b
				\end{align}
				が示されたが,まだ
				\begin{align}
					\Set{\sigma_X\left(x,\zeta\right)}{\exists z \in X\, 
					\left[\, (0_X,z) \in w \wedge \zeta = s(t,z)\, \right]}
				\end{align}
				が$x$の近傍であることについて言及していない.実際(b)より
				\begin{align}
					v[0_X] \subset \Set{s(t,z)}{(0_X,z) \in w}
				\end{align}
				を満たす$\mathscr{U}$の要素$v$が取れて,(a)より
				\begin{align}
					v[x] = \Set{\sigma_X(x,z)}{(0_X,z) \in v}
				\end{align}
				であるから
				\begin{align}
					v[x] \subset \Set{\sigma_X\left(x,\zeta\right)}{\exists z \in X\, 
					\left[\, (0_X,z) \in w \wedge \zeta = s(t,z)\, \right]}
				\end{align}
				が成り立つ.ゆえに
				\begin{align}
					\Set{\sigma_X\left(x,\zeta\right)}{\exists z \in X\, 
					\left[\, (0_X,z) \in w \wedge \zeta = s(t,z)\, \right]}
				\end{align}
				は$x$の近傍である.ゆえに$s$は$(\alpha,x)$で連続である.
				\QED
		\end{description}
	\end{sketch}
	
	\begin{screen}
		\begin{dfn}[局所凸空間]
			全ての要素が凸である局所基が取れる位相線型空間を
			{\bf 局所凸空間}\index{きょくしょとつ@局所凸}{\bf (locally convex space)}と呼ぶ.
		\end{dfn}
	\end{screen}
	
	\begin{screen}
		\begin{thm}[ノルム空間は局所凸]
		\end{thm}
	\end{screen}
	
	\begin{screen}
		\begin{dfn}[Frechet空間]
			局所凸な$F$-空間を{\bf Frechet空間}\index{Frechetくうかん@Frechet空間}と呼ぶ.
		\end{dfn}
	\end{screen}
	
	Banach空間はFrechet空間である.
	
	\begin{screen}
		\begin{thm}[局所凸空間とはセミノルムの族で生成される空間]
			
		\end{thm}
	\end{screen}
\subsection{局所凸}
	\begin{screen}
		\begin{thm}[線型位相を導入する近縁系]
		\label{thm:entourages_introducing_vector_topology}
			$\left(\left(X,\sigma_X\right),(\Phi,+,\bullet),s\right)$を線型空間とし,
			$0_X$を$\left(X,\sigma_X\right)$の単位元とし,$\mathscr{V}$を$X$上の近縁系とし,
			$\mathscr{O}_X$を$\mathscr{V}$で導入する$X$上の一様位相とする.
			$\mathscr{V}$の基本近縁系$\mathscr{U}$で
			\begin{itemize}
				\item $\forall U \in \mathscr{U}\, \left(\, x + U_\zeta \subset U_x\, \right)$,
				\item $\forall U \in \mathscr{U}\, \forall \alpha \in \Phi\,
					\exists V \in \mathscr{U} \left(\, \alpha \neq 0 \Longrightarrow V \subset \alpha U_\zeta\, \right)$,
				\item $\forall U \in \mathscr{U}\, \forall \alpha \in \Phi\,
					\left(\, |\alpha| \leq 1 \Longrightarrow \alpha U_\zeta \subset U_\zeta\, \right)$,
				\item $\forall U \in \mathscr{U}\, \left(\, \forall x \in X\, 
				\exists s \in ]0,\infty[\, (\, x \in s U_\zeta\, )\, \right)$
			\end{itemize}
			を満たすものが取れるとき,$\left(\left(X,\sigma_X\right),(\Phi,+,\bullet),s,\mathscr{O}_X\right)$
			は位相線型空間である.
		\end{thm}
	\end{screen}
	
	\begin{sketch}\mbox{}
		\begin{description}
			\item[第一段] 一様位相が不変位相であることを示す.
				$x$を$X$の要素とする.$x$において
				\begin{align}
					\Set{U_x}{U \in \mathscr{U}}
				\end{align}
				は基本近傍系となるが,
				\begin{align}
					\forall U \in \mathscr{U}\, \left(\, x + U_\zeta \subset U_x\, \right)
				\end{align}
				が満たされているので
				\begin{align}
					\Set{x + U_\zeta}{U \in \mathscr{U}}
				\end{align}
				も$x$の基本近傍系となる.従って一様位相は不変位相である.
				
			\item[第二段]
				加法$\sigma$が$(\zeta,\zeta)$において連続となることを示す.
				$B$を$\zeta$の近傍とすれば,
				\begin{align}
					U_\zeta \subset B
				\end{align}
				なる$\mathscr{U}$の要素$U$が取れる.また
				\begin{align}
					W \circ W \subset U
				\end{align}
				なる$\mathscr{U}$の要素$W$も取れる.このとき
				\begin{align}
					W_\zeta \times W_\zeta \subset \sigma^{-1} \ast B
				\end{align}
				が成立する.実際,
				\begin{align}
					(x,y) \in W_\zeta \times W_\zeta
					\label{eq:thm_entourages_introducing_vector_topology}
				\end{align}
				なる$x,y$に対し,
				\begin{align}
					y \in W_\zeta
				\end{align}
				から
				\begin{align}
					x + y \in x + W_\zeta
				\end{align}
				となり
				\begin{align}
					x + y \in W_x
				\end{align}
				が成り立つので,
				\begin{align}
					(\zeta,x) \in W \wedge (x,x+y) \in W
				\end{align}
				となり
				\begin{align}
					(\zeta,x+y) \in W
				\end{align}
				が従う.よって
				\begin{align}
					x+y \in U_\zeta
				\end{align}
				が成り立ち,(\refeq{eq:thm_entourages_introducing_vector_topology})
				が示された.
				
			\item[第三段]
				$x$と$\alpha$をそれぞれ$X$と$\Phi$の要素として,
				スカラ倍$\mu$が$(\alpha,x)$で連続となることを示す.
				$B$を$\alpha x$の近傍とする.このとき
				\begin{align}
					-\alpha x + B
				\end{align}
				は$\zeta$の近傍となるので,
				\begin{align}
					U_\zeta \subset -\alpha x + B 
				\end{align}
				を満たす$\mathscr{U}$の要素$U$が取れる.$U$に対し
				\begin{align}
					W \circ W \subset U
				\end{align}
				なる$\mathscr{U}$の要素$W$を取り,
				\begin{align}
					x \in s W_\zeta
				\end{align}
				なる正数$s$を取り
				\begin{align}
					t \defeq s/(1+|\alpha|s)
				\end{align}
				とおく.このとき
				\begin{align}
					y \in x+t W_\zeta \wedge |\beta - \alpha| < 1/s
					\Longrightarrow \beta y - \alpha x
					&= \beta (y-x) + (\beta - \alpha)x \\
					&\in \beta t W_\zeta + (\beta - \alpha) s W_\zeta \\
					&\subset W_\zeta + W_\zeta \\
					&\subset U_\zeta \\
					&\subset -\alpha x + B
				\end{align}
				が成立する.ゆえに
				\begin{align}
					y \in x+t W_\zeta \wedge |\beta - \alpha| < 1/s
					\Longrightarrow \beta y \in B
				\end{align}
				が成り立ち,
				\begin{align}
					\mu^{-1} \ast B
				\end{align}
				が$(\alpha,x)$の近傍であることが示された.
				
		\end{description}
	\end{sketch}
	
	\begin{screen}
		\begin{dfn}[局所凸・Frechet空間]
			位相線型空間の零元の基本近傍系で,全ての要素が凸であるものが取れるとき,
			その空間は局所凸\index{きょくしょとつ@局所凸}である(locally convex)と呼ばれる.
			また局所凸なF-空間をFrechet空間\index{Frechetくうかん@Frechet空間}と呼ぶ.
		\end{dfn}
	\end{screen}
	
	\begin{screen}
		\begin{lem}[局所凸空間の直積は局所凸]
			$Z$を線型空間,$(X_\lambda)_{\lambda \in \Lambda}$を
			局所凸位相線型空間の族とし,$0 \in X_\lambda$の
			基本近傍系(全ての元は凸)を$\mathscr{U}_\lambda$
			と書く.また各$\lambda \in \Lambda$に対し
			写像$f_\lambda:Z \longrightarrow X_\lambda$が定まっているとする.
			このとき次が成り立つ:
			\begin{description}
				\item[(1)] 
					$0 \in Z$を含み,かつ定理のを満たすような集合系$\mathscr{U}$を
					\begin{align}
						\mathscr{U} \defeq
						\Set{\bigcap_{\lambda \in H} f_\lambda^{-1}(V_\lambda)}{
						\mbox{$H$は$\Lambda$の空でない有限部分集合},\ 
						V_\lambda \in \mathscr{U}_\lambda}
					\end{align}
					で定める.また任意の$V \in \mathscr{U}_\lambda$に対し
					$f_\lambda^{-1}(V)$が凸であるとき($\forall \lambda \in \Lambda$),
					\begin{align}
						\mathscr{U}(x) \defeq
						\Set{x+U}{U\in\mathscr{U}},
						\quad(\forall x \in Z)
					\end{align}
					とおけば,$\{\mathscr{U}(x)\}_{x \in Z}$を基本近傍系とする
					$Z$の位相がただ一つに定まり,$Z$の和を連続にする.
			\end{description}
		\end{lem}
	\end{screen}
	
	\begin{screen}
		\begin{thm}[局所凸空間とはセミノルムの族で生成される空間]
			
		\end{thm}
	\end{screen}
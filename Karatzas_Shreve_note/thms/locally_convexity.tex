\subsection{局所凸}
	\begin{screen}
		\begin{thm}[ゼロの近傍をスケール変換すれば空間全体を覆うことが出来る]
		\label{thm:neighbor_of_zero_is_absorbing}
			$\left(\left(X,\sigma_X\right),(\Phi,+,\bullet),s,\mathscr{O}_X\right)$を線型位相空間とし,
			$0_X$を$\left(X,\sigma_X\right)$の単位元とし,
			$u$を$0_X$の近傍とし,$x$を$X$の要素とする.このとき
			\begin{align}
				\forall t \in \R_+\,
				\left[\, r < t \Longrightarrow x \in \Set{s(t,z)}{z \in u}\, \right]
			\end{align}
			を満たす正の実数$r$が取れる.
		\end{thm}
	\end{screen}
	
	\begin{sketch}
		スケール変換は連続であるから,
		\begin{align}
			\Phi \ni \alpha \longmapsto s(\alpha,x)
		\end{align}
		なる写像を$\varphi$とおけば
		\begin{align}
			\Set{\alpha \in \Phi}{|\alpha| < \delta} \subset \varphi^{-1} \ast u
		\end{align}
		を満たす正の実数$\delta$が取れる.そして$t$を
		\begin{align}
			\frac{1}{\delta} < t
		\end{align}
		なる任意の実数とすれば
		\begin{align}
			\frac{1}{t} < \delta
		\end{align}
		が成り立つから
		\begin{align}
			s(1/t,x) \in u
		\end{align}
		が成り立つ.そして
		\begin{align}
			x = s\left(1,x\right) = s\left(t \cdot (1/t),x\right) = s\left(t,s\left(1/t,x\right)\right)
		\end{align}
		であるから
		\begin{align}
			x \in \Set{s(t,z)}{z \in u}
		\end{align}
		が成立する.
		\QED
	\end{sketch}
	
	位相線型空間は一様化可能であるが,本節では逆に線型空間に対して線型位相を導入しうる近縁系とはどのようなものであるかを考察する.
	そのような近縁系が具えているべき条件を見つけるには,位相線型空間を一様化する近縁系が持つ性質を洗い出せば良い.
	
	いま$\left(\left(X,\sigma_X\right),(\Phi,+,\bullet),s,\mathscr{O}_X\right)$を線型位相空間とし,
	$0_X$を$\left(X,\sigma_X\right)$の単位元とし,
	\begin{align}
		\mathscr{B}
	\end{align}
	を$\left(\left(X,\sigma_X\right),\mathscr{O}_X\right)$の局所基とし,
	その任意の要素$b$が均衡しているとする.つまり
	\begin{align}
		\forall \alpha \in \Phi\,
		\left[\, |\alpha| \leq 1 \Longrightarrow \forall x \in b\, \left(\, s(\alpha,x) \in b\, \right)\, \right].
	\end{align}
	また$\mathscr{U}$と$\mathscr{V}$を定理\ref{thm:topological_vector_spaces_are_uniformazable}の要領で構成する集合とする.
	その定め方より$\mathscr{U}$は$\mathscr{V}$の基本近縁系である.
	$\mathscr{V}$よりは$\mathscr{U}$の方が具体的に定められているから,
	$\mathscr{U}$が具えている性質を見る方が容易い.
	ちなみに$u$を$\mathscr{U}$の要素とすると
	\begin{align}
		u = \Set{(s,t)}{s \in X \wedge t \in X \wedge \sigma_X(-s,t) \in b}
	\end{align}
	を満たす$\mathscr{B}$の要素$b$が取れるが,このとき
	\begin{align}
		b = \Set{x \in X}{\left(0_X,x\right) \in u}
	\end{align}
	が成り立つ.これは
	\begin{align}
		u[x] \defeq \Set{y}{(x,y) \in u}
	\end{align}
	なる表記を用いて直感的に書き直せば
	\begin{align}
		b = u[0_X]
	\end{align}
	が成り立つということである.当然のようだが,$\mathscr{U}$によって$\mathscr{B}$を復元できる.
	
	\begin{description}
		\item[(a)] $u$を$\mathscr{U}$の要素とし,$x$を$X$の要素とすると,
			\begin{align}
				\Set{y}{(x,y) \in u} = \Set{\sigma_X(x,z)}{(0_X,z) \in u}
				\label{fom:pre_thm_entourages_introducing_vector_topology_1}
			\end{align}
			が成立する.これを直感的に書き直せば
			\begin{align}
				u[x] = x + u[0_X]
			\end{align}
			となるが,意味としては{\bf $x$の基本近傍系は$0_X$の基本近傍系を$x$だけ平行移動すれば得られるということである.}
			実際,
			\begin{align}
				u = \Set{(s,t)}{s \in X \wedge t \in X \wedge \sigma_X(-s,t) \in b}
			\end{align}
			を満たす$\mathscr{B}$の要素$b$が取れて,$y$を$X$の任意の要素とすると
			\begin{align}
				(x,y) \in u &\Longleftrightarrow \sigma_X\left(-x,y\right) \in b \\
				&\Longleftrightarrow \sigma_X\left(-0_X,\sigma_X\left(-x,y\right)\right) \in b \\
				&\Longleftrightarrow \left(0_X,\sigma_X\left(-x,y\right)\right) \in u
			\end{align}
			が成り立ち,さらに
			\begin{align}
				y = \sigma_X\left(x,\sigma_X\left(-x,y\right)\right)
			\end{align}
			であるから
			\begin{align}
				\left(0_X,\sigma_X\left(-x,y\right)\right) \in u
				\Longleftrightarrow \exists z \in X\, \left(\, y=\sigma_X\left(x,z\right) 
				\wedge \left(0_X,z\right) \in u\, \right)
			\end{align}
			も成り立つ.ゆえに(\refeq{fom:pre_thm_entourages_introducing_vector_topology_1})が得られた.
		
		\item[(b)] $u$を$\mathscr{U}$の要素とし,$\alpha$を
			\begin{align}
				\alpha \neq 0
			\end{align}
			なる任意のスカラーとすれば,スケール変換は連続であるから
			\begin{align}
				\Set{s(\alpha,z)}{(0_X,z) \in u}
			\end{align}
			は$0_X$の近傍である.この集合は直感的には
			\begin{align}
				\alpha \cdot u[0_X]
			\end{align}
			と書けるが,近傍なのだから
			\begin{align}
				b \subset \alpha \cdot u[0_X]
			\end{align}
			を満たす$\mathscr{B}$の要素$b$が取れて,
			\begin{align}
				v \defeq \Set{(s,t)}{s \in X \wedge t \in X \wedge \sigma_X(-s,t) \in b}
			\end{align}
			により$\mathscr{U}$の要素$v$を定めれば
			\begin{align}
				v[0_X] \subset \alpha \cdot u[0_X]
			\end{align}
			が成り立つ.
			
		\item[(c)] $u$を$\mathscr{U}$の要素とすれば
			\begin{align}
				u[0_X]
			\end{align}
			は均衡集合である.
			
		\item[(d)] 定理\ref{thm:neighbor_of_zero_is_absorbing}より,
			$\mathscr{U}$の要素$u$と$X$の要素$x$が任意に与えられれば
			\begin{align}
				x \in r \cdot u[0_X]
			\end{align}
			を満たす正の実数$r$が取れる.
	\end{description}
	
	以上で$\mathscr{U}$の性質を四つ抜き出したが,逆にこれらを全て揃えている基本近縁系が取れるなら
	近縁系は線型位相を作り出すのである.
	
	\begin{screen}
		\begin{thm}[線型位相を導入する近縁系]
		\label{thm:entourages_introducing_vector_topology}
			$\left(\left(X,\sigma_X\right),(\Phi,+,\bullet),s\right)$を線型空間とし,
			$0_X$を$\left(X,\sigma_X\right)$の単位元とし,$\mathscr{V}$を$X$上の近縁系とし,
			$\mathscr{O}_X$を$\mathscr{V}$で導入する$X$上の一様位相とする.
			$\mathscr{V}$の基本近縁系$\mathscr{U}$で
			\begin{description}
				\item[(a)] $\mathscr{U}$の要素$u$と$X$の要素$x$が任意に与えられたときに
					\begin{align}
						\Set{y}{(x,y) \in u} = \Set{\sigma_X(x,z)}{(0_X,z) \in u}
					\end{align}
					が成り立つ.
						
				\item[(b)] $\mathscr{U}$の要素$u$と$\Phi$の要素$\alpha$が任意に与えられたときに
					\begin{align}
						\alpha \neq 0 \Longrightarrow \Set{z}{(0_X,z) \in v} \subset \Set{s(\alpha,z)}{(0_X,z) \in u}
					\end{align}
					を満たす$\mathscr{U}$の要素$v$が取れる.
					
				\item[(c)] $\mathscr{U}$の要素$u$と$\Phi$の要素$\alpha$が任意に与えられたときに
					\begin{align}
						|\alpha| \leq 1 \Longrightarrow \Set{s(\alpha,z)}{(0_X,z) \in u} \subset \Set{z}{(0_X,z) \in u}
					\end{align}
					が成り立つ.
					
				\item[(d)] $\mathscr{U}$の要素$u$と$X$の要素$x$が任意に与えられたときに
					\begin{align}
						x \in \Set{s(r,z)}{(0_X,z) \in u}
					\end{align}
					を満たす正の実数$r$が取れる.
			\end{description}
			を満たすものが取れるとき,$\left(\left(X,\sigma_X\right),(\Phi,+,\bullet),s,\mathscr{O}_X\right)$
			は位相線型空間である.
		\end{thm}
	\end{screen}
	
	\begin{sketch}\mbox{}
		\begin{description}
			\item[第一段] $\mathscr{U}$の要素$u$と$X$の要素$a$に対し
				\begin{align}
					u[a] \defeq \Set{x \in X}{(a,x) \in u}
				\end{align}
				と定める.また
				\begin{align}
					X \ni x \longmapsto \sigma_X(a,x)
				\end{align}
				なる写像を
				\begin{align}
					\sigma_X^a
				\end{align}
				と書く.
				
			\item[第二段]
				$\sigma_X$が$\left(0_X,0_X\right)$において連続であることを示す.
				$b$を$0_X$の近傍とすれば
				\begin{align}
					u[0_X] \subset b
				\end{align}
				を満たす$\mathscr{U}$の要素$u$と
				\begin{align}
					w \circ w \subset u
				\end{align}
				を満たす$\mathscr{U}$の要素$w$が取れる.このとき
				\begin{align}
					w[0_X] \times w[0_X] \subset \sigma_X^{-1} \ast b
					\label{fom:thm_entourages_introducing_vector_topology_1}
				\end{align}
				が成立する.実際,$x$と$y$を$w[0_X]$の要素とすると
				\begin{align}
					\sigma_X\left(x,y\right) \in \Set{\sigma_X(x,z)}{(0_X,z) \in w}
				\end{align}
				が成り立ち,(a)より
				\begin{align}
					\Set{\sigma_X(x,z)}{(0_X,z) \in w} = w[x]
				\end{align}
				であるから
				\begin{align}
					\left(x,\sigma_X\left(x,y\right)\right) \in w
				\end{align}
				が従う.ゆえにいま
				\begin{align}
					(0_X,x) \in w \wedge \left(x,\sigma_X\left(x,y\right)\right) \in w
				\end{align}
				が成り立っているので
				\begin{align}
					\left(0_X,\sigma_X\left(x,y\right)\right) \in u
				\end{align}
				が従う.ゆえに
				\begin{align}
					\left(x,y\right) \in w[0_X] \times w[0_X] \Longrightarrow
					\sigma_X\left(x,y\right) \in u[0_X]
				\end{align}
				が成り立つ.ゆえに(\refeq{fom:thm_entourages_introducing_vector_topology_1})が得られた.
				ゆえに$\sigma_X$は$\left(0_X,0_X\right)$で連続である.
				
			\item[第三段] $x$と$y$を$X$の要素として,$\sigma_X$が$\left(x,y\right)$において連続であることを示す.
				
			\item[第一段] 一様位相が不変位相であることを示す.
				$x$を$X$の要素とする.$x$において
				\begin{align}
					\Set{U_x}{U \in \mathscr{U}}
				\end{align}
				は基本近傍系となるが,
				\begin{align}
					\forall U \in \mathscr{U}\, \left(\, x + U_\zeta \subset U_x\, \right)
				\end{align}
				が満たされているので
				\begin{align}
					\Set{x + U_\zeta}{U \in \mathscr{U}}
				\end{align}
				も$x$の基本近傍系となる.従って一様位相は不変位相である.
				
			\item[第三段]
				$x$と$\alpha$をそれぞれ$X$と$\Phi$の要素として,
				スカラ倍$\mu$が$(\alpha,x)$で連続となることを示す.
				$B$を$\alpha x$の近傍とする.このとき
				\begin{align}
					-\alpha x + B
				\end{align}
				は$\zeta$の近傍となるので,
				\begin{align}
					U_\zeta \subset -\alpha x + B 
				\end{align}
				を満たす$\mathscr{U}$の要素$U$が取れる.$U$に対し
				\begin{align}
					W \circ W \subset U
				\end{align}
				なる$\mathscr{U}$の要素$W$を取り,
				\begin{align}
					x \in s W_\zeta
				\end{align}
				なる正数$s$を取り
				\begin{align}
					t \defeq s/(1+|\alpha|s)
				\end{align}
				とおく.このとき
				\begin{align}
					y \in x+t W_\zeta \wedge |\beta - \alpha| < 1/s
					\Longrightarrow \beta y - \alpha x
					&= \beta (y-x) + (\beta - \alpha)x \\
					&\in \beta t W_\zeta + (\beta - \alpha) s W_\zeta \\
					&\subset W_\zeta + W_\zeta \\
					&\subset U_\zeta \\
					&\subset -\alpha x + B
				\end{align}
				が成立する.ゆえに
				\begin{align}
					y \in x+t W_\zeta \wedge |\beta - \alpha| < 1/s
					\Longrightarrow \beta y \in B
				\end{align}
				が成り立ち,
				\begin{align}
					\mu^{-1} \ast B
				\end{align}
				が$(\alpha,x)$の近傍であることが示された.
				
		\end{description}
	\end{sketch}
	
	\begin{screen}
		\begin{dfn}[局所凸・Frechet空間]
			位相線型空間の零元の基本近傍系で,全ての要素が凸であるものが取れるとき,
			その空間は局所凸\index{きょくしょとつ@局所凸}である(locally convex)と呼ばれる.
			また局所凸なF-空間をFrechet空間\index{Frechetくうかん@Frechet空間}と呼ぶ.
		\end{dfn}
	\end{screen}
	
	\begin{screen}
		\begin{thm}[局所凸空間とはセミノルムの族で生成される空間]
			
		\end{thm}
	\end{screen}
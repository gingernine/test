\subsection{連続性}
	近傍概念を用いることにより,写像の連続性を精密に定義することができる.
	
	\begin{screen}
		\begin{dfn}[連続]
			$(S,\mathscr{O}_{T})$と$(T,\mathscr{O}_{T})$を位相空間とし,
			$S$と$T$は空でないとし,$f$を$S$から$T$への写像とする.$x$を$S$の要素とするとき,
			$f(x)$の任意の$\mathscr{O}_{T}$-近傍を
			$f$で引き戻したものが$x$の$\mathscr{O}_{S}$-近傍であるならば,つまり
			$\mathcal{U}_{x}$を$x$の$\mathscr{O}_{S}$-近傍系とし,
			$\mathcal{V}_{f(x)}$を$f(x)$の$\mathscr{O}_{T}$-近傍系としたとき
			\begin{align}
				\forall v \in \mathcal{V}_{f(x)}\,
				\left(\, f^{-1} \ast v \in \mathcal{U}_{x}\, \right)
			\end{align}
			が成り立つならば,$f$は$x$において$\mathscr{O}_{S}/\mathscr{O}_{T}$-{\bf 連続}
			\index{れんぞく@連続}である{\bf (continuous)}という.
			$f$が$S$のすべての要素において$\mathscr{O}_{S}/\mathscr{O}_{T}$-連続であるとき
			$f$は$\mathscr{O}_{S}/\mathscr{O}_{T}$-連続であるといい,
			$f$を$\mathscr{O}_{S}/\mathscr{O}_{T}$-{\bf 連続写像}
			\index{れんぞくしゃぞう@連続写像}{\bf (continuous mapping)}と呼ぶ.
		\end{dfn}
	\end{screen}
	
	\begin{screen}
		\begin{thm}[連続写像は開集合を開集合に引き戻す]
		\label{thm:continuous_iff_preimage_of_open_sets_are_open}
		\label{thm:continuous_on_every_point_iff_continuous}
			$(S,\mathscr{O}_{T})$と$(T,\mathscr{O}_{T})$を位相空間とし,
			$S$と$T$は空でないとし,$f$を$S$から$T$への写像とする.
			このとき,$f$が$\mathscr{O}_{S}/\mathscr{O}_{T}$-連続であることと
			\begin{align}
				\forall o \in \mathscr{O}_{T}\, 
				\left(\, f^{-1} \ast o \in \mathscr{O}_{S}\, \right)
			\end{align}
			が成り立つことは同値である.
		\end{thm}
	\end{screen}
	
	\begin{prf}
		$f$が$\mathscr{O}_{S}/\mathscr{O}_{T}$-連続であるとし,
		$o$を$\mathscr{O}_{T}$-開集合とする.このとき$x$を$f^{-1} \ast o$の要素とすれば
		\begin{align}
			f(x) \in o
		\end{align}
		が成り立つが,$o$は$\mathscr{O}_{T}$-開集合であるから
		$f(x)$の$\mathscr{O}_{T}$-近傍である.よって
		$f^{-1} \ast o$は$x$の$\mathscr{O}_{S}$-近傍である.
		よって定理\ref{thm:local_base_defines_open_sets}より
		\begin{align}
			f^{-1} \ast o \in \mathscr{O}_{S}
		\end{align}
		が従う.次は逆に
		\begin{align}
			\forall o \in \mathscr{O}_{T}\, 
			\left(\, f^{-1} \ast o \in \mathscr{O}_{S}\, \right)
		\end{align}
		が成り立っているとする.ここで$x$を$S$の要素とし,$v$を$f(x)$の$\mathscr{O}_{T}$-近傍とすると,
		\begin{align}
			f(x) \in o \wedge o \subset v
		\end{align}
		なる$\mathscr{O}_{T}$-開集合$o$が取れるが,このとき
		\begin{align}
			f^{-1} \ast o \subset f^{-1} \ast v
		\end{align}
		が成り立ち,かつ$f^{-1} \ast o$は$x$を要素に持つ$\mathscr{O}_{S}$-開集合であるから,
		$f^{-1} \ast v$は$x$の$\mathscr{O}_{S}$-近傍である.
		ゆえに$f$は$\mathscr{O}_{S}/\mathscr{O}_{T}$-連続である.
		\QED
	\end{prf}
	
	\begin{screen}
		\begin{dfn}[同相]
			$f$に逆写像$f^{-1}$が存在し,$f,f^{-1}$が共に連続であるとき,
			$f$を{\bf 同相写像}\index{どうそうしゃぞう@同相写像}{\bf (homeomorphism)}
			や{\bf 位相同型写像}\index{いそうどうけいしゃぞう@位相同型写像},
			或は単に{\bf 同相}や{\bf 位相同型}と呼ぶ.
			また$S,T$間に同相写像が存在するとき$S$と$T$は
			{\bf 同相}\index{どうそう@同相}である{\bf (homeomorphic)},
			或は{\bf 位相同型}であるという.
		\end{dfn}
	\end{screen}
	
	\begin{screen}
		\begin{dfn}[開写像]
			$S$の任意の開集合の$f$による像が$T$の開集合となるとき,
			$f$を{\bf 開写像}\index{かいしゃぞう@開写像}{\bf (open mapping)}と呼ぶ.
		\end{dfn}
	\end{screen}
	
	\begin{screen}
		\begin{thm}[コンパクト集合の連続写像による像はコンパクト]
		\end{thm}
	\end{screen}
	
	
	
	\begin{screen}
		\begin{thm}[部分空間と制限写像の連続性]
			$S,T$を位相空間,$f$を$S$から$T$への写像とする.
			また$g:S \longrightarrow f(S)$を
			$f$の終集合を$f(S)$へ制限した写像とする.このとき次が成り立つ:
			\begin{align}
				\mbox{$f:S \longrightarrow T$が連続である} 
				\quad \Longleftrightarrow \quad
				\mbox{$g:S \longrightarrow f(S)$が($f(S)$の相対位相に関して)連続である}.
			\end{align}
		\end{thm}
	\end{screen}
	
	\begin{prf}
		$U \coloneqq f(S)$とおけば$T$の任意の開集合$O$に対し
		\begin{align}
			g^{-1}(U \cap O) = f^{-1}(U \cap O) = f^{-1}(O)
		\end{align}
		が成り立つから,$f$と$g$の連続性は一致する.
		\QED
	\end{prf}
	
	\begin{screen}
		\begin{thm}[位相の生成]
			$S$を集合,$\mathscr{M}$を$S$の部分集合の族として
			\begin{align}
				\mathscr{A} \coloneqq
				\Set{\bigcap \mathscr{F}}{\mbox{$\mathscr{F}$は$\mathscr{M}$の有限部分集合}}
			\end{align}
			とおくとき,$\mathscr{M}$を含む最小の位相は
			\begin{align}
				\mathscr{O} \coloneqq
				\Set{\bigcup \Lambda}{\Lambda \subset \mathscr{A}}
				\cup \{S\}
			\end{align}
			で与えられる.この$\mathscr{O}$を$\mathscr{M}$が生成する$S$の位相と呼ぶ.
		\end{thm}
	\end{screen}
	
	\begin{prf}
		$\mathscr{O}$は定め方より$S$と$\emptyset$を含む.また
		任意の$O_1 = \bigcup \Lambda_1,\ O_2=\bigcup \Lambda_2 \in \mathscr{O},\ 
		(\Lambda_1,\Lambda_2 \subset \mathscr{A})$に対し
		\begin{align}
			\Lambda \coloneqq
			\Set{I \cap J}{I \in \Lambda_1,\ J \in \Lambda_2} \subset \mathscr{A}
		\end{align}
		となるから
		\begin{align}
			O_1 \cap O_2 = \bigcup_{I \in \Lambda_1,\ J \in \Lambda_2} I \cap J
			= \bigcup \Lambda \in \mathscr{O}
		\end{align}
		が成立する.任意に$\emptyset \neq \mathscr{U} \subset \mathscr{O}$を取れば,
		各$U \in \mathscr{U}$に$U = \bigcup \Lambda_U$を満たす
		$\Lambda_U \subset \mathscr{A}$が対応し,このとき
		\begin{align}
			\bigcup_{U \in \mathscr{U}} \Lambda_U \subset \mathscr{A}
		\end{align}
		となるから
		\begin{align}
			\bigcup \mathscr{U} = \bigcup \Biggl(\bigcup_{U \in \mathscr{U}} \Lambda_U\Biggr)
			\in \mathscr{O}
		\end{align}
		が従う.$\mathscr{M}$を含む任意の位相は$\mathscr{A}$を含みかつその任意和で閉じるから$\mathscr{O}$を含む.
		\QED
	\end{prf}
	
	\begin{screen}
		\begin{thm}[Alexanderの定理]
		\end{thm}
	\end{screen}
	
	\begin{screen}
		\begin{dfn}[始位相]
			$f \in \mathscr{F}$を集合$S$から位相空間$(T_f,\mathscr{O}_f)$への写像とするとき,
			全ての$f \in \mathscr{F}$を連続にする最弱の位相を$S$の$\mathscr{F}$-始位相
			(initial topology)と呼ぶ.$\mathscr{F}$-始位相は次が生成する位相である:
			\begin{align}
				\bigcup_{f \in \mathscr{F}} \Set{f^{-1}(O)}{O \in \mathscr{O}_f}.
			\end{align}
		\end{dfn}
	\end{screen}
	
	\begin{screen}
		\begin{dfn}[Cartesian積の位相]
			
		\end{dfn}
	\end{screen}
	
	\begin{screen}
		\begin{dfn}[直積の位相]
			
		\end{dfn}
	\end{screen}
\subsection{連続性}
	近傍概念を用いることにより,写像の連続性を精密に定義することができる.
	
	\begin{screen}
		\begin{dfn}[連続]
			$(S,\mathscr{O}_{S})$と$(T,\mathscr{O}_{T})$を位相空間とし,
			$f$を$S$から$T$への写像とする.$x$を$S$の要素とするとき,
			$f(x)$の任意の$\mathscr{O}_{T}$-近傍を
			$f$で引き戻したものが$x$の$\mathscr{O}_{S}$-近傍であるならば,つまり
			$\mathcal{U}_{x}$を$x$の$\mathscr{O}_{S}$-近傍系とし,
			$\mathcal{V}_{f(x)}$を$f(x)$の$\mathscr{O}_{T}$-近傍系としたとき
			\begin{align}
				\forall v \in \mathcal{V}_{f(x)}\,
				\left(\, f^{-1} \ast v \in \mathcal{U}_{x}\, \right)
			\end{align}
			が成り立つならば,$f$は$x$において$\mathscr{O}_{S}/\mathscr{O}_{T}$-{\bf 連続}
			\index{れんぞく@連続}である{\bf (continuous)}という.
			$f$が$S$のすべての要素において$\mathscr{O}_{S}/\mathscr{O}_{T}$-連続であるとき
			$f$は$\mathscr{O}_{S}/\mathscr{O}_{T}$-連続であるといい,
			$f$を$\mathscr{O}_{S}/\mathscr{O}_{T}$-{\bf 連続写像}
			\index{れんぞくしゃぞう@連続写像}{\bf (continuous mapping)}と呼ぶ.
		\end{dfn}
	\end{screen}
	
	$S$上の写像で$S$の要素に対してその$\mathscr{O}_{S}$-近傍系を対応させるものを$\mathcal{U}$とし,
	$T$上の写像で$T$の要素に対してその$\mathscr{O}_{T}$-近傍系を対応させるものを$\mathcal{V}$とすれば,
	$f$が$\mathscr{O}_{S}/\mathscr{O}_{T}$-連続であるとは
	\begin{align}
		\forall x \in S\, \forall v \in \mathcal{V}_{f(x)}\,
		\left(\, f^{-1} \ast v \in \mathcal{U}_{x}\, \right)
	\end{align}
	が成り立つことである.特に
	\begin{align}
		S = \emptyset
	\end{align}
	なるときは$S$上の写像は空写像の他に無いわけだが,空写像は上の式を満たすので$\mathscr{O}_{S}/\mathscr{O}_{T}$-連続である.
	
	\begin{screen}
		\begin{thm}[連続写像は開集合を開集合に引き戻す]
		\label{thm:continuous_iff_preimage_of_open_sets_are_open}
		\label{thm:continuous_on_every_point_iff_continuous}
			$(S,\mathscr{O}_{S})$と$(T,\mathscr{O}_{T})$を位相空間とし,$f$を$S$から$T$への写像とする.
			このとき,$f$が$\mathscr{O}_{S}/\mathscr{O}_{T}$-連続であることと
			\begin{align}
				\forall o \in \mathscr{O}_{T}\, 
				\left(\, f^{-1} \ast o \in \mathscr{O}_{S}\, \right)
			\end{align}
			が成り立つことは同値である.
		\end{thm}
	\end{screen}
	
	\begin{prf}
		$f$が$\mathscr{O}_{S}/\mathscr{O}_{T}$-連続であるとし,
		$o$を$\mathscr{O}_{T}$-開集合とする.このとき$x$を$f^{-1} \ast o$の要素とすれば
		\begin{align}
			f(x) \in o
		\end{align}
		が成り立つが,$o$は$\mathscr{O}_{T}$-開集合であるから
		$f(x)$の$\mathscr{O}_{T}$-近傍である.よって
		$f^{-1} \ast o$は$x$の$\mathscr{O}_{S}$-近傍である.
		よって定理\ref{thm:local_base_defines_open_sets}より
		\begin{align}
			f^{-1} \ast o \in \mathscr{O}_{S}
		\end{align}
		が従う.次は逆に
		\begin{align}
			\forall o \in \mathscr{O}_{T}\, 
			\left(\, f^{-1} \ast o \in \mathscr{O}_{S}\, \right)
		\end{align}
		が成り立っているとする.ここで$x$を$S$の要素とし,$v$を$f(x)$の$\mathscr{O}_{T}$-近傍とすると,
		\begin{align}
			f(x) \in o \wedge o \subset v
		\end{align}
		なる$\mathscr{O}_{T}$-開集合$o$が取れるが,このとき
		\begin{align}
			f^{-1} \ast o \subset f^{-1} \ast v
		\end{align}
		かつ
		\begin{align}
			x \in f^{-1} \ast o
		\end{align}
		かつ
		\begin{align}
			f^{-1} \ast o \in \mathscr{O}_{S}
		\end{align}
		が成り立つので$f^{-1} \ast v$は$x$の$\mathscr{O}_{S}$-近傍である.
		ゆえに$f$は$\mathscr{O}_{S}/\mathscr{O}_{T}$-連続である.
		\QED
	\end{prf}
	
	\begin{screen}
		\begin{dfn}[同相]
			$f$に逆写像$f^{-1}$が存在し,$f,f^{-1}$が共に連続であるとき,
			$f$を{\bf 同相写像}\index{どうそうしゃぞう@同相写像}{\bf (homeomorphism)}
			や{\bf 位相同型写像}\index{いそうどうけいしゃぞう@位相同型写像},
			或は単に{\bf 同相}や{\bf 位相同型}と呼ぶ.
			また$S,T$間に同相写像が存在するとき$S$と$T$は
			{\bf 同相}\index{どうそう@同相}である{\bf (homeomorphic)},
			或は{\bf 位相同型}であるという.
		\end{dfn}
	\end{screen}
	
	\begin{screen}
		\begin{dfn}[開写像]
			$S$の任意の開集合の$f$による像が$T$の開集合となるとき,
			$f$を{\bf 開写像}\index{かいしゃぞう@開写像}{\bf (open mapping)}と呼ぶ.
		\end{dfn}
	\end{screen}
	
	\begin{screen}
		\begin{thm}[コンパクト集合の連続写像による像はコンパクト]
			$(S,\mathscr{O}_{S})$と$(T,\mathscr{O}_{T})$を位相空間とし,
			$f$を$S$から$T$への写像とする.
			$S$が$\mathscr{O}_{S}$-コンパクトであるとき,
			$f \ast S$は$\mathscr{O}_{T}$-コンパクトである.
		\end{thm}
	\end{screen}
	
	\begin{sketch}
		$\mathscr{P}$を$\mathscr{O}_{T}$の部分集合で
		\begin{align}
			f \ast S \subset \bigcup \mathscr{P}
		\end{align}
		を満たすものとする.このとき
		\begin{align}
			\mathscr{Q} \defeq \Set{f^{-1} \ast p}{p \in \mathscr{P}}
		\end{align}
		とおけば
		\begin{align}
			\mathscr{Q} \subset \mathscr{O}_{S}
		\end{align}
		(定理\ref{thm:continuous_iff_preimage_of_open_sets_are_open})と
		\begin{align}
			S = \bigcup \mathscr{Q}
		\end{align}
		が成り立つので,$\mathscr{Q}$の有限部分集合$\mathscr{R}$で
		\begin{align}
			S = \bigcup \mathscr{R}
		\end{align}
		を満たすものが取れる.ここで
		\begin{align}
			\mathscr{R} \ni r \longmapsto \Set{p \in \mathscr{P}}{r = f^{-1} \ast p}
		\end{align}
		なる写像を$h$とおけば,$\mathscr{R}$の任意の要素$r$に対して
		\begin{align}
			h(r) \neq \emptyset
		\end{align}
		が満たされるので,$\mathscr{R}$から$\mathscr{P}$への写像$g$で,
		$\mathscr{R}$の任意の要素$r$に対して
		\begin{align}
			r = f^{-1} \ast g(r)
		\end{align}
		を満たすものが取れる(定理\ref{thm:direct_product_of_non_empty_sets_is_not_empty}).
		\begin{align}
			\mathscr{S} \defeq \Set{g(r)}{r \in \mathscr{R}}
		\end{align}
		とおけば$\mathscr{S}$は$\mathscr{P}$の部分集合であって,
		\begin{align}
			\card{\mathscr{S}} \leq \card{\mathscr{R}}
		\end{align}
		(定理\ref{thm:if_exists_a_surjection_then_cardinal_of_target_is_bigger})及び
		\begin{align}
			f \ast S \subset \bigcup \mathscr{S}
		\end{align}
		が成り立つので,$f \ast S$は$\mathscr{O}_{T}$-コンパクトである.
		\QED
	\end{sketch}
	
	\begin{screen}
		\begin{thm}[定義域を制限しても連続]
			$(S,\mathscr{O}_{S})$と$(T,\mathscr{O}_{T})$を位相空間とし,$f$を$S$から$T$への
			$\mathscr{O}_{S}/\mathscr{O}_{T}$-連続写像とする.
			このとき,$S$の任意の部分集合$b$に対して
			\begin{align}
				\mathscr{O}_{b} \defeq \Set{o \cap b}{o \in \mathscr{O}_{S}}
			\end{align}
			とおけば,$\rest{f}{b}$は$\mathscr{O}_{b}/\mathscr{O}_{T}$-連続である.
		\end{thm}
	\end{screen}
	
	\begin{sketch}
		$o$を$\mathscr{O}_{T}$-開集合とすれば
		\begin{align}
			{\rest{f}{b}}^{-1} \ast o = b \cap \left( f^{-1} \ast o \right)
		\end{align}
		が成立するが,
		\begin{align}
			f^{-1} \ast o \in \mathscr{O}_{S}
		\end{align}
		なので
		\begin{align}
			{\rest{f}{b}}^{-1} \ast o \in \mathscr{O}_{b}
		\end{align}
		が従う.よって定理\ref{thm:continuous_iff_preimage_of_open_sets_are_open}より
		$\rest{f}{b}$は$\mathscr{O}_{b}/\mathscr{O}_{T}$-連続である.
		\QED
	\end{sketch}
	
	いま$(S,\mathscr{O}_{S})$を位相空間とし,$f$を$S$上の$\R$-値写像とする.値域が$\R$の部分集合なので
	\begin{align}
		f:S \longrightarrow \C
	\end{align}
	も成り立つが,このとき$f$が$\mathscr{O}_{S}/\mathscr{O}_{\R}$-連続であることと
	$f$が$\mathscr{O}_{S}/\mathscr{O}_{\C}$-連続であることは同値である.実際,
	$\mathscr{O}_{\C}$の任意の要素$o$に対して
	\begin{align}
		o \cap \R \in \mathscr{O}_{S}
	\end{align}
	かつ
	\begin{align}
		f^{-1} \ast o = f^{-1} \ast (o \cap \R)
	\end{align}
	が成立するので,$f$が$\mathscr{O}_{S}/\mathscr{O}_{\R}$-連続であるとすれば
	\begin{align}
		f^{-1} \ast o \in \mathscr{O}_{S}
	\end{align}
	が従う.逆に$o$を$\mathscr{O}_{\R}$の要素とすれば
	\begin{align}
		o = u \cap \R
	\end{align}
	を満たす$\mathscr{O}_{\C}$-開集合$u$が取れるので,$f$が$\mathscr{O}_{S}/\mathscr{O}_{\C}$-連続であるならば
	\begin{align}
		f^{-1} \ast o = f^{-1} \ast u \in \mathscr{O}_{S}
	\end{align}
	が従う.以上のことは一般化出来て次の主張が得られる.
	
	\begin{screen}
		\begin{thm}[終域が縮まっても連続]
			$(S,\mathscr{O}_{S})$と$(T,\mathscr{O}_{T})$を位相空間とし,$f$を$S$から$T$への写像とし,$b$を
			\begin{align}
				f \ast S \subset b
			\end{align}
			なる$T$の部分集合とし,
			\begin{align}
				\mathscr{O}_{b} \defeq \Set{o \cap b}{o \in \mathscr{O}_{T}}
			\end{align}
			とおく.このとき$f$が$\mathscr{O}_{S}/\mathscr{O}_{b}$-連続であることと
			$f$が$\mathscr{O}_{S}/\mathscr{O}_{T}$-連続であることは同値である.
		\end{thm}
	\end{screen}
	
	\begin{sketch}
		上述の$\C$を$T$に,$\R$を$b$に置き換えればよい.
		\QED
	\end{sketch}
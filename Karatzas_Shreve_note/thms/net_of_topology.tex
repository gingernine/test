\section{集合か位相的な}
\subsection{位相}
	位相空間$S$の部分集合$A$について,$A$の内核を$A^i$或は$A^{\mathrm{o}}$と書き,
	$A$の閉包を$A^a$或は$\overline{A}$と書く.
	また$A^{ca} = (A^c)^a,\ A^{ic} = (A^i)^c$と略記する.
	\begin{screen}
		\begin{thm}[閉包・内核]\label{thm:topology_note_closure_interior}
			$S$を位相空間,$h:S \longrightarrow S$を同相,$A$を部分集合とするとき次が成り立つ.
			\begin{description}
				\item[(1)] $A^{ic} = A^{ca}$.
				\item[(2)] $h(E^a) = h(E)^a$.
				\item[(3)] $h(E^i) = h(E)^i$.
			\end{description}
		\end{thm}
	\end{screen}
	
	\begin{prf}\mbox{}
		\begin{description}
			\item[(1)]
				$A^i \subset A$より$A^{ic} \supset A^c$が従い,
				$A^{ic}$が閉であるから$A^{ic} \supset A^{ca}$となる.
				一方で$A^c \subset A^{ca}$より$A \supset A^{cac}$が従い,
				$A^{cac}$は開であるから$A^i \supset A^{cac}$すなわち
				$A^{ic} \subset A^{ca}$となる.
			
			\item[(2)]
				$h(E) \subset h(E^a)$かつ$h(E^a)$は閉であるから$h(E)^a \subset h(E^a)$が従う.一方で
				任意の$x \in h(E^a)$に対し$x = h(y)$を満たす
				$y \in E^a$と$x$の任意の近傍$V$を取れば,
				$h^{-1}(V) \cap E \neq \emptyset$より
				$V \cap h(E) \neq \emptyset$が成り立ち
				$x \in h(E)^a$となる.
				
			\item[(3)]
				$h(E^i) \subset h(E)$かつ$h(E^i)$は開であるから
				$h(E^i) \subset h(E)^i$が従う.一方で
				任意の開集合$O \subset h(E)$に対し
				$h^{-1}(O) \subset E$より
				$h^{-1}(O) \subset E^i$となり,
				$O \subset h(E^i)$が成り立つから
				$h(E)^i \subset h(E^i)$が得られる.
				\QED
		\end{description}
	\end{prf}
	
	\begin{screen}
		\begin{thm}[有限交叉性]\label{thm:finite_intersection_property}
			位相空間$S$がコンパクトであることと,
			任意の閉集合系$(U_\lambda)_{\lambda \in \Lambda}$に対して
			\begin{align} 
				\bigcap_{\lambda \in F} U_\lambda \neq \emptyset
				\mbox{, for every finite subset $F \subset \Lambda$}
				\quad \Longrightarrow \quad \bigcap_{\lambda \in \Lambda} U_\lambda \neq \emptyset
				\label{eq:finite_intersection_property}
			\end{align}
			となることは同値である.
		\end{thm}
	\end{screen}
	
	\begin{prf}
		任意の閉集合系$(U_\lambda)_{\lambda \in \Lambda}$に対して
		$\bigcap_{\lambda \in \Lambda} U_\lambda = \emptyset$なら
		$(U_\lambda^c)_{\lambda \in \Lambda}$は$S$の開被覆となるから,
		$S$がコンパクトであることと(\refeq{eq:finite_intersection_property})は同値である.
		\QED
	\end{prf}
	
	\begin{screen}
		\begin{thm}
			$X$を局所コンパクトHausdorff空間,
			$K,U \subset X$をそれぞれコンパクト集合,開集合とする.このとき
			或る開集合$V$が存在して,$\overline{V}$はコンパクトであり,かつ次を満たす:
			\begin{align}
				K \subset V \subset \overline{V} \subset U.
			\end{align}
		\end{thm}
	\end{screen}
	
	\begin{screen}
		\begin{thm}[可算コンパクト性の同値条件]
		\end{thm}
	\end{screen}
	
\subsection{範疇定理}
	\begin{screen}
		\begin{dfn}[疎集合・第一類集合・第二類集合]
			位相空間$S$の部分集合$A$が疎である(nowhere dense)とは
			$A$の閉包の内核が$\overline{A}^{\mathrm{o}} = \emptyset$を満たすことをいう.
			$S$が可算個の疎集合の合併で表せるとき$S$を第一類集合(the set of the first category)と呼び,
			そうでない場合はこれを第二類集合と呼ぶ.
		\end{dfn}
	\end{screen}
	
	\begin{screen}
		\begin{thm}[Baire]\label{thm:Baire_category_theorem}
			$S \neq \emptyset$が完備距離空間,或は局所コンパクトHausdorff空間なら
			$S$は第二類集合である.
		\end{thm}
	\end{screen}
	
	\begin{prf}\mbox{}
		\begin{description}
			\item[第一段]
				$(V_n)_{n=1}^\infty$を$S$で稠密な開集合系とするとき
				\begin{align}
					\overline{\bigcap_{n=1}^\infty V_n} = S,
					\label{eq:thm_Baire_category_theorem_1}
				\end{align}
				となることを示す.実際(\refeq{eq:thm_Baire_category_theorem_1})が満たされていれば,
				任意の疎集合系$(E_n)_{n=1}^\infty$に対して
				\begin{align}
					V_n \coloneqq \overline{E_n}^c,
					\quad n=1,2,\cdots
				\end{align}
				で開集合系$(V_n)$を定めると定理\ref{thm:topology_note_closure_interior}より
				\begin{align}
					\overline{V_n} = \overline{E_n}^{ca} = \overline{E_n}^{ic} = \emptyset^c = S
				\end{align}
				となるから,$\bigcap_{n=1}^\infty V_n \neq \emptyset$が従い
				$S \neq \bigcup_{n=1}^\infty \overline{E_n} \supset \bigcup_{n=1}^\infty E_n$
				が成り立つ.従って$S$は第二類である.
				
			\item[第二段]
				任意の空でない開集合$B_0$に対し$B_0 \cap \left( \bigcap_{n=1}^\infty V_n \right) \neq \emptyset$
				となることを示せば(\refeq{eq:thm_Baire_category_theorem_1})が従う.
				$V_1$の稠密性より或る点$x_1 \in B_0 \cap V_1$が存在し,
				次を満たす開集合$B_1$が取れる:
				\begin{align}
					x_1 \in \overline{B_1} \subset B_0 \cap V_1.
					\label{eq:thm_Baire_category_theorem_2}
				\end{align}
				このとき,$S$が距離空間なら$B_1$は半径$1$以下の開球,
				局所コンパクトHausdorff空間なら$\overline{B_1}$がコンパクトであるようにできる.
				繰り返して,半径$1/n$以下の開球,
				或は閉包がコンパクトな開集合$B_n$と$x_n \in S$が存在して
				\begin{align}
					x_n \in \overline{B_n} \subset B_{n-1} \cap V_n
				\end{align}
				を満たす.このとき$S$が完備距離空間なら$(x_n)_{n=1}^\infty$は
				Cauchy列をなすから,その極限点$x_\infty$は
				\begin{align}
					x_\infty \in \bigcap_{n=1}^\infty \overline{B_n}
				\end{align}
				を満たす.$S$が局所コンパクトHausdorff空間なら有限交叉性より
				\begin{align}
					\bigcap_{n=1}^\infty \overline{B_n} \neq \emptyset
				\end{align}
				となるから,いずれの場合も
				\begin{align}
					\emptyset \neq \bigcap_{n=1}^\infty \overline{B_n} 
					\subset B_0 \cap \Biggl( \bigcap_{n=1}^\infty V_n \Biggr)
				\end{align}
				が従い定理の主張が得られる.
				\QED
		\end{description}
	\end{prf}
	
	\begin{screen}
		\begin{thm}[第一類集合の性質]
			$S$を位相空間とする.
			\begin{description}
				\item[(a)] $A \subset B \subset S$に対し$B$が第一類なら$A$も第一類である.
				\item[(b)] 第一類集合の可算和も第一類である.
				\item[(c)] 内核が空である閉集合は第一類である.
				\item[(d)] $S$から$S$への位相同型$h$と$E \subset S$に対し次が成り立つ:
					\begin{align}
						\mbox{$E$が第一類} \quad \Longleftrightarrow \quad
						\mbox{$h(E)$が第一類}.
					\end{align}
			\end{description}
		\end{thm}
	\end{screen}
	
	\begin{prf}\mbox{}
		\begin{description}
			\item[(a)] $B = \bigcup_{n=1}^\infty E_n$
				を満たす疎集合系$(E_n)_{n=1}^\infty$に対し
				$A \cap E_n$は疎であり$A = \bigcup_{n=1}^\infty (A \cap E_n)$となる.
			\item[(b)] $A_n \subset S,\ (n=1,2,\cdots)$が第一類集合とし
				$(E_{n,i})_{i=1}^\infty$を$A_n = \bigcup_{i=1}^\infty E_{n,i}$
				を満たす疎集合系とすれば
				\begin{align}
					\bigcup_{n=1}^\infty A_n
					= \bigcup_{n,i=1}^\infty E_{n,i}
				\end{align}
				が成り立つ.
				
			\item[(c)] 内核が空である閉集合はそれ自身が疎であり,自身の可算和に一致する.
			\item[(d)] $E$が第一類のとき,$E = \bigcup_{i=1}^\infty E_i$を満たす
				疎集合系$(E_i)_{i=1}^\infty$に対し定理\ref{thm:topology_note_closure_interior}より
				\begin{align}
					\emptyset = h(E_i^{ai})
					= h(E_i^a)^i
					= h(E_i)^{ai}
				\end{align}
				が成り立つから$h(E_i)$は疎であり,
				\begin{align}
					h(E) = \bigcup_{i=1}^\infty h(E_i)
				\end{align}
				となるから$h(E)$も第一類である.$h(E)$が第一類なら$E = h^{-1}(h(E))$も第一類である.
				\QED
		\end{description}
	\end{prf}
	
\subsection{有向点族}
\subsection{位相線型空間}
	
	位相線形空間$(X,\tau)$に対し,その部分集合$Y$上の相対位相を$\tau_Y$と書き,
	また$X$が或る距離$d$で距離付け可能なとき,
	$d$により導入する位相を$\tau_d$と書く.位相$\tau$に関する開集合,閉集合,近傍,
	Cauchy列は$\tau$-開集合(resp. 閉集合,近傍,Cauchy列)と書く.
	
	\begin{screen}
		\begin{thm}[部分空間が$F$-空間なら閉]
			$(X,\tau)$を位相線形空間,$Y \subset X$を部分空間とする.
			このとき$Y$が$F$-空間なら$Y$は$\tau$-閉である.
		\end{thm}
	\end{screen}
	
	\begin{prf}
		$Y$に対し或る平行移動不変な距離$d$が存在して$\tau_Y = \tau_d$を満たす.
		このとき
		\begin{align}
			B_{1/n} \coloneqq \Set{y \in Y}{d(y,0) < \frac{1}{n}},
			\quad n=1,2,\cdots
		\end{align}
		で$\tau_Y$-開集合を定めれば,$B_{1/n}$は$0$を含むから
		或る0の$\tau$-近傍$U_n$が存在して
		\begin{align}
			B_{1/n} = Y \cap U_n, \quad n=1,2,\cdots
		\end{align}
		を満たす.
	\end{prf}
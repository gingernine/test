\section{位相的}
\subsection{位相}
	位相空間$S$の部分集合$A$について,$A$の内核を$A^i$或は$A^{\mathrm{o}}$と書き,
	$A$の閉包を$A^a$或は$\overline{A}$と書く.
	\begin{screen}
		\begin{thm}[閉包・内核]\label{thm:topology_note_closure_interior}
			$S$を位相空間,$A$を部分集合とする.$A^{ic} = (A^i)^c,\ A^{ca} = (A^c)^a$と書けば次が成り立つ:
			\begin{align}
				A^{ic} = A^{ca}.
			\end{align}
		\end{thm}
	\end{screen}
	
	\begin{prf}
		$A^i \subset A$より$A^{ic} \supset A^c$が従い,
		$A^{ic}$が閉であるから$A^{ic} \supset A^{ca}$となる.
		一方で$A^c \subset A^{ca}$より$A \supset A^{cac}$が従い,
		$A^{cac}$は開であるから$A^i \supset A^{cac}$すなわち
		$A^{ic} \subset A^{ca}$となる.
		\QED
	\end{prf}
	
\subsection{範疇定理}
	\begin{screen}
		\begin{dfn}[疎集合・第一類]
			位相空間$S$の部分集合$A$が疎である(nowhere dense)とは
			$A$の閉包の内核が$\overline{A}^{\mathrm{o}} = \emptyset$を満たすことをいう.
			$S$が可算個の疎集合の合併で表せるとき$S$を第一類集合(the set of the first category)と呼び,
			そうでない場合はこれを第二類集合と呼ぶ.
		\end{dfn}
	\end{screen}
	
	\begin{screen}
		\begin{thm}[Baire]
			$S \neq \emptyset$が完備距離空間,或は局所コンパクトHausdorff空間なら
			$S$は第二類集合である.
		\end{thm}
	\end{screen}
	
	\begin{prf}\mbox{}
		\begin{description}
			\item[第一段]
				$(V_n)_{n=1}^\infty$を$S$で稠密な開集合系とするとき
				$\overline{\bigcap_{n=1}^\infty V_n} = S$となることを示す.
				実際これが成立すれば,任意の疎集合系$(E_n)_{n=1}^\infty$に対して
				\begin{align}
					V_n \coloneqq \overline{E_n}^c,
					\quad n=1,2,\cdots
				\end{align}
				で定める開集合系$(V_n)$は定理\ref{thm:topology_note_closure_interior}より
				\begin{align}
					\overline{V_n} = \overline{E_n}^{ca} = \overline{E_n}^{ic} = S
				\end{align}
				を満たすから,$\bigcap_{n=1}^\infty V_n \neq \emptyset$が従い
				$S \neq \bigcup_{n=1}^\infty \overline{E_n} \supset \bigcup_{n=1}^\infty E_n$
				が成り立つ.従って$S$は第二類である.
				
			\item[第二段]
				
		\end{description}
	\end{prf}
	
\subsection{有向点族}
\subsection{位相線型空間}
	
\section{集合か位相}
\subsection{Dynkin族定理}
	\begin{screen}
		\begin{dfn}[乗法族・Dynkin族]\label{def:Dynkin_system_theorem}
			集合$X$の部分集合の族$\mathscr{A}$が
			任意の$A,B \in \mathscr{A}$に対し$A \cap B \in \mathscr{A}$を満たすとき
			$\mathscr{A}$を$X$上の乗法族($\pi$-system)という.
			$X$の部分集合の族$\mathscr{D}$が
			\begin{description}
				\item[(D1)] $X \in \mathscr{D}$,
				\item[(D2)] $A,B \in \mathscr{D},
					\ A \subset B \quad \Longrightarrow \quad B \backslash A \in \mathscr{D}$,
				\item[(D3)] $\{A_n\}_{n=1}^\infty \subset \mathscr{D},
					\ A_n \cap A_m = \emptyset\ (n \neq m)
					\quad \Longrightarrow \quad \sum_{n=1}^\infty A_n \in \mathscr{D}$,
			\end{description}
			を満たすとき,$\mathscr{D}$を$X$上のDynkin族(Dynkin system)という.
		\end{dfn}
	\end{screen}
	
	\begin{screen}
		\begin{dfn}[Dynkin族定理]\label{thm:Dynkin_system_theorem}
			集合$X$上の乗法族$\mathscr{A}$に対し,
			$\mathscr{A}$を含む最小のDynkin族を$\delta(\mathscr{A})$と書くとき,
			\begin{align}
				\delta(\mathscr{A}) = \sigma(\mathscr{A}).
			\end{align}
		\end{dfn}
	\end{screen}
	
	\begin{prf}\mbox{}
		\begin{description}
			\item[第一段]
				$\delta(\mathscr{C})$が交演算で閉じていれば
				$\delta(\mathscr{C})$は$\sigma$-加法族となる.実際任意の$A \in \delta(\mathscr{A})$に対し
				\begin{align}
					A^c = X \backslash A \in \delta(\mathscr{A})
				\end{align}
				となるから,$\delta(\mathscr{C})$が交演算で閉じていれば任意の
				$A_n \in \delta(\mathscr{C})\ (n=1,2,\cdots)$に対し
				\begin{align}
					\bigcup_{n=1}^{\infty} A_n
					= \sum_{n=1}^{\infty} A_1^c \cap A_2^c \cap \cdots \cap A_{n-1}^c \cap A_n
					\in \delta(\mathscr{C})
				\end{align}
				が従う.$\sigma$-加法族はDynkin族であるから
				$\sigma(\mathscr{C}) \subset \delta(\mathscr{C})$も成り立ち
				$\sigma(\mathscr{C}) = \delta(\mathscr{C})$が得られる.
			
			\item[第二段]
				$\delta(\mathscr{C})$が交演算について閉じていることを示す.いま,
				\begin{align}
					\mathscr{D}_1 \coloneqq
					\Set{B \in \delta(\mathscr{C})}{ A \cap B \in \delta(\mathscr{C}),\ 
					\forall A \in \mathscr{C}}
				\end{align}
				により定める$\mathscr{D}_1$はDynkin族であり$\mathscr{C}$を含むから
				\begin{align}
					\delta(\mathscr{C}) \subset \mathscr{D}_1
				\end{align}
				が成立する.従って
				\begin{align}
					\mathscr{D}_2 \coloneqq
					\Set{B \in \delta(\mathscr{C})}{ A \cap B \in \delta(\mathscr{C}),\ 
					\forall A \in \delta(\mathscr{C})}
				\end{align}
				によりDynkin族$\mathscr{D}_2$を定めれば,$\mathscr{C} \subset \mathscr{D}_2$が満たされ
				\begin{align}
					\delta(\mathscr{C}) \subset \mathscr{D}_2
				\end{align}
				が得られる.よって$\delta(\mathscr{C})$は交演算について閉じている.
				\QED
		\end{description}
	\end{prf}
	
	\begin{screen}
		\begin{thm}
			集合$X$の部分集合族$\mathscr{D}$が
			の定義\ref{def:Dynkin_system_theorem}の(D1),(D2)を満たしているとき,
			$\mathscr{D}$が(D3)を満たすことと
			$\mathscr{D}$が増大列の可算和で閉じることは同値である.
		\end{thm}
	\end{screen}
	
	\begin{prf}
		$\mathscr{D}$が可算直和について閉じているとする.このとき
		単調増大列$A_1 \subset A_2 \subset \cdots$を取り
		\begin{align}
			B_1 \coloneqq A_1,
			\quad B_n \coloneqq A_n \backslash A_{n-1},
			\quad (n \geq 2)
		\end{align}
		とおけば(D2)より$B_n \in \mathscr{D},\ (\forall n \geq 1)$が満たされ
		\begin{align}
			\bigcup_{n=1}^{\infty} A_n = \sum_{n=1}^{\infty} B_n \in \mathscr{D} 
		\end{align}
		が成立する.逆に$\mathscr{D}$が増大列の可算和で閉じているとする.
		(D1)(D2)より互いに素な$A,B \in \mathscr{D}$に対し
		$A^c \in \mathscr{D}$及び$A^c \cap B^c = A^c \backslash B\in \mathscr{D}$が成り立つから,
		$\mathscr{D}$の互いに素な集合列$(B_n)_{n=1}^{\infty}$を取れば
		\begin{align}
			B_1^c \cap B_2^c \cap \cdots \cap B_n^c
			= \left( \cdots \left( \left( B_1^c \cap B_2^c \right) \cap B_3^c \right) \cap \cdots \cap B_{n-1}^c \right) \cap B_n^c
			\in \mathscr{D},
			\quad (n=1,2,\cdots)
		\end{align}
		が得られる.よって
		\begin{align}
			D_n \coloneqq \bigcup_{i=1}^n B_i = X \backslash \Biggl( \bigcap_{i=1}^n B_i^c \Biggr),
			\quad (n=1,2,\cdots)
		\end{align}
		により$\mathscr{D}$の単調増大列$(D_n)_{n=1}^{\infty}$を定めれば
		\begin{align}
			\sum_{n=1}^{\infty} B_n = \bigcup_{n=1}^{\infty} D_n \in \mathscr{D}
		\end{align}
		が成立する.
		\QED
	\end{prf}

\subsection{上限下限}
	\begin{screen}
		\begin{thm}[上限の冪と冪の上限]\label{thm:exponentiation_of_supremum_supremum_of_exponentiation}
			任意の空でない$S \subset [0,\infty)$と$t > 0$に対し次が成立する:
			\begin{align}
				(\sup{}{S})^t = \sup{}{\Set{s^t}{s \in S}}.
			\end{align}
		\end{thm}
	\end{screen}
	
	\begin{prf}
		$S=\{0\}$なら両辺0で一致するので,$S$は$\{0\}$より真に大きいとする.このとき
		任意の$s \in S$に対し$s^t \leq (\sup{}{S})^t$となるから$\sup{}{\Set{s^t}{s \in S}} \leq (\sup{}{S})^t$が従う.
		また任意の$(\sup{}{S})^t > \alpha > 0$に対し$s > \alpha^{1/t}$を満たす$s \in S$が存在し
		$(\sup{}{S})^t \geq s^t > \alpha$となるから$\sup{}{\Set{s^t}{s \in S}} = (\sup{}{S})^t$が得られる.
		\QED
	\end{prf}

\subsection{位相}
	\begin{screen}
		\begin{dfn}[位相]
			集合$S$の部分集合の族$\mathscr{O}$が
			以下を満たすとき,$\mathscr{O}$を$S$の位相(topology)と呼ぶ:
			\begin{description}
				\item[(O1)] $\emptyset, S \in \mathscr{O}$,
				\item[(O2)] $O_1,O_2 \in \mathscr{O} 
					\quad \Longrightarrow \quad O_1 \cap O_2 \in \mathscr{O}$,
				\item[(O3)] $\displaystyle\mathscr{U} \subset \mathscr{O}
					\quad \Longrightarrow \quad \bigcup \mathscr{U} = 
					\bigcup_{U \in \mathscr{U}} U \in \mathscr{O}$.
			\end{description}
		\end{dfn}
	\end{screen}
	\begin{screen}
		\begin{dfn}[近傍・基本近傍系]
			空でない位相空間$S$において,$x \in S$と$U \subset S$に対し
			\begin{align}
				x \in U^{\mathrm{o}}
			\end{align}
			が満たされるとき$U$は$x$の近傍(neighborhood)であるという.
			同様に$A \subset S$と$V \subset S$に対し
			\begin{align}
				A \subset V^{\mathrm{o}}
			\end{align}
			が満たされるとき,$V$は$A$の近傍であるという.
			点$x$の近傍全体を$\mathscr{V}(x)$と書くとき,
			$S$は$x$の最大の近傍であるから$\mathscr{V}(x)$は空ではない.
			また$\mathscr{V}(x)$の空でない部分集合$\mathscr{U}(x)$が
			\begin{align}
				\forall V \in \mathscr{V}(x),
				\quad \exists U \in \mathscr{U}(x),
				\quad U \subset V
			\end{align}
			を満たすとき,$\mathscr{U}(x)$を$x$の基本近傍系(local base of a point $x$)と呼ぶ.
		\end{dfn}
	\end{screen}
	
	\begin{screen}
		\begin{thm}[基本近傍系は開集合を決定する]\label{thm:local_base_defines_open_sets}
			$S$を空でない位相空間,
			$\mathscr{U}(x)$を点$x$の基本近傍系とすれば
			\begin{align}
				\mbox{$O$が$S$の開集合} \quad \Longleftrightarrow \quad 
				\mbox{$O = \emptyset$,或は任意の$x \in O$に対し
				$U \subset O$を満たす$U \in \mathscr{U}(x)$が存在する}
			\end{align}
			が成立する.すなわち,$\{\mathscr{U}(x)\}_{x \in S}$を基本近傍系とする$S$の位相は唯一つである.
		\end{thm}
	\end{screen}
	
	\begin{prf}
		$O$が開集合なら任意の$x \in O$に対し$O$は$x$の近傍となるから,
		或る$U \in \mathscr{U}(x)$が存在して$U \subset O$を満たす.
		逆に任意の$x \in O$に対し$U \subset O$を満たす$U \in \mathscr{U}(x)$が存在するとき,
		\begin{align}
			x \in U^{\mathrm{o}} \subset O^{\mathrm{o}}
		\end{align}
		となり$O = O^{\mathrm{o}}$が成立するから$O$は開集合である.
		\QED
	\end{prf}
	
	\begin{screen}
		\begin{thm}[基本近傍系は位相を復元する]\mbox{}
			\begin{description}
				\item[(1)] 
					$(S,\mathscr{O})$を空でない位相空間とし,各点
					$x \in S$に対し$\mathscr{U}(x)$を基本近傍系とすれば以下が成り立つ:
					\begin{description}
						\item[(LB1)] $\mathscr{U}(x)$は空ではなく,また任意の$U \in \mathscr{U}(x)$は$x \in U$を満たす.
						\item[(LB2)] 任意の$U,V \in \mathscr{U}(x)$に対し或る$W \in \mathscr{U}(x)$
							が存在して$W \subset U \cap V$を満たす.
						\item[(LB3)] 任意の$U \in \mathscr{U}(x)$に対し或る$V \in \mathscr{U}(x)$が存在し,
							任意の$y \in V$に対し$W_y \subset V$を満たす$W_y \in \mathscr{U}(y)$が取れる.
					\end{description}
				\item[(2)]
					空でない集合$S$の各点$x$に対し(LB1)(LB2)(LB3)を満たす部分集合族$\mathscr{U}(x)$が与えられれば,
					\begin{align}
						\mathscr{O} \coloneqq
						\Set{O \subset S}{\mbox{$O = \emptyset$,或は任意の$x \in O$に対し
						$U \subset O$を満たす$U \in \mathscr{U}(x)$が存在する}}
					\end{align}
					により$S$に位相が定まり,$\{\mathscr{U}(x)\}_{x \in S}$は
					$(S,\mathscr{O})$において基本近傍系となる.
				\item[(3)] 空でない位相空間$(S,\mathscr{O})$から基本近傍系
					$\{\mathscr{U}(x)\}_{x \in S}$を得れば,
					$\{\mathscr{U}(x)\}_{x \in S}$を基本近傍系とする位相
					を(2)の手続きで構成することにより$\mathscr{O}$を復元できる.
			\end{description}
		\end{thm}
	\end{screen}
	
	\begin{prf}\mbox{}
		\begin{description}
			\item[(1)] 任意の$U \in \mathscr{U}(x)$は$x$の近傍であるから
				$(LB1)$が満たされる.また$U,V \in \mathscr{U}(x)$に対し
				\begin{align}
					x \in U^{\mathrm{o}} \cap V^{\mathrm{o}} = (U \cap V)^{\mathrm{o}}
				\end{align}
				となるから$U \cap V$は$x$の近傍であり(LB2)も従う.
				任意の$U \in \mathscr{U}(x)$に対し$V \coloneqq U^{\mathrm{o}}$とおけば,
				$V$は任意の$y \in V$の開近傍となるから(LB3)も得られる.
			
			\item[(2)] 
				$\mathscr{U}(x)$は空ではないから$S \in \mathscr{O}$となる.
				また$O_1,O_2 \in \mathscr{O}$を取れば,
				任意の$x \in O_1 \cap O_2$に対し
				\begin{align}
					x \in U_1 \subset O_1,
					\quad x \in U_2 \subset O_2
				\end{align}
				を満たす$U_1,U_2 \in \mathscr{U}(x)$が存在し,
				(LB2)より或る$U_3 \in \mathscr{U}(x)$に対して
				\begin{align}
					U_3 \subset U_1 \cap U_2 \subset O_1 \cap O_2
				\end{align}
				が成り立つから$O_1 \cap O_2 \in \mathscr{O}$となる.
				任意に$\mathscr{G} \subset \mathscr{O}$を取れば
				任意の$x \in \bigcup \mathscr{G}$は或る$G \in \mathscr{G}$の点であるから,
				\begin{align}
					U \subset G \subset \bigcup \mathscr{G}
				\end{align}
				を満たす$U \in \mathscr{U}(x)$が存在し$\bigcup \mathscr{G} \in \mathscr{O}$が従う.
				よって$\mathscr{O}$は位相である.
				また(LB3)の$V$は$\mathscr{O}$の元であり
				\begin{align}
					x \in V \subset U^{\mathrm{o}}
				\end{align}
				が成り立つから任意の$U \in \mathscr{U}(x)$は$x$の近傍である.
				そして$W$を$x$の任意の近傍とすれば,$\mathscr{O}$の定め方より或る$U \in \mathscr{U}(x)$が
				$U \subset W^{\mathrm{o}}$を満たすから$\mathscr{U}(x)$は$x$の基本近傍系である.
			
			\item[(3)] 
				定理\ref{thm:local_base_defines_open_sets}より
				$\{\mathscr{U}(x)\}_{x \in S}$を基本近傍系とする位相は唯一つであるから
				主張が従う.
				\QED
		\end{description}
	\end{prf}
	
	\begin{screen}
		\begin{dfn}[相対位相]
			$(S,\mathscr{O})$を位相空間,$M \subset S$を部分集合,
			$i:M \longrightarrow S$を恒等写像とするとき,
			\begin{align}
				\mathscr{O}_M \coloneqq 
				\Set{i^{-1}(O) = O \cap M}{O \in \mathscr{O}}
			\end{align}
			で定まる$i$による$\mathscr{O}$の引き戻しを$M$の相対位相(relative topology)と呼ぶ.
		\end{dfn}
	\end{screen}
	
	\begin{screen}
		\begin{thm}[位相の生成]
			$S$を集合,$\mathcal{P}(S)$を冪集合として
			任意に$M \subset \mathcal{P}(S)$を取り
			\begin{align}
				\mathscr{A} \coloneqq
				\Set{\bigcap_{i=1}^n I_i}{I_i \in M,\ n = 1,2,\cdots}
			\end{align}
			とおくとき,$M$を含む最小の位相は
			\begin{align}
				\mathscr{O} \coloneqq
				\Set{\bigcup \Lambda}{\Lambda \subset \mathscr{A}}
				\cup \{S\}
			\end{align}
			で与えられる.この$\mathscr{O}$を$M$が生成する$S$の位相と呼ぶ.
		\end{thm}
	\end{screen}
	
	\begin{prf}
		$\mathscr{O}$は定め方より$S$と$\emptyset$を含む.また
		任意の$O_1 = \bigcup \Lambda_1,\ O_2=\bigcup \Lambda_2 \in \mathscr{O},\ 
		(\Lambda_1,\Lambda_2 \subset \mathscr{A})$に対し
		\begin{align}
			\Set{I \cap J}{I \in \Lambda_1,\ J \in \Lambda_2} \subset \mathscr{A}
		\end{align}
		となるから
		\begin{align}
			O_1 \cap O_2 = \bigcup_{I \in \Lambda_1,\ J \in \Lambda_2} I \cap J \in \mathscr{O}
		\end{align}
		が成立する.任意に$\emptyset \neq \mathscr{U} \subset \mathscr{O}$を取れば,
		各$U \in \mathscr{U}$に$U = \bigcup \Lambda_U$を満たす
		$\Lambda_U \subset \mathscr{A}$が対応し,このとき
		\begin{align}
			\bigcup_{U \in \mathscr{U}} \Lambda_U \subset \mathscr{A}
		\end{align}
		となるから
		\begin{align}
			\bigcup \mathscr{U} = \bigcup \Biggl(\bigcup_{U \in \mathscr{U}} \Lambda_U\Biggr)
			\in \mathscr{O}
		\end{align}
		が従う.$M$を含む任意の位相は$\mathscr{A}$を含みかつその任意和で閉じるから$\mathscr{O}$を含む.
		\QED
	\end{prf}
	
	\begin{screen}
		\begin{dfn}[始位相]
			$f \in \mathscr{F}$を集合$S$から位相空間$(T_f,\mathscr{O}_f)$への写像とするとき,
			全ての$f \in \mathscr{F}$を連続にする最弱の位相を$S$の$\mathscr{F}$-始位相
			(initial topology)と呼ぶ.$\mathscr{F}$-始位相は次が生成する位相である:
			\begin{align}
				\bigcup_{f \in \mathscr{F}} \Set{f^{-1}(O)}{O \in \mathscr{O}_f}.
			\end{align}
		\end{dfn}
	\end{screen}
	
\subsection{分離公理}
	\begin{screen}
		\begin{dfn}[位相的に識別可能・分離]
			$S$を位相空間とする.
			\begin{itemize}
				\item $x,y \in S$に対し$x \notin \overline{\{y\}}$
					或は$y \notin \overline{\{x\}}$が満たされるとき,
					$x$と$y$は位相的に識別可能(topologically distinguishable)であるという.
				\item $A,B \subset S$に対し$\overline{A} \cap B = \emptyset$
					或は$A \cap \overline{B} = \emptyset$が満たされるとき,
					$A$と$B$は分離される(separeted)という.点と点,点と集合の分離は一点集合を考える.
				\item $A,B \subset S$が近傍で分離される(separated by neighborhoods)とは,
					$A,B$が互いに交わらない近傍を持つことをいう.
				\item 閉集合$A,B \subset S$が関数で分離される(separated by a function)とは,
					或る連続関数$f:S \longrightarrow [0,1]$によって$f(A) = \{0\},\ f(B) = \{1\}$
					が満たされることをいう.
				\item 閉集合$A,B \subset S$が関数でちょうど分離される
					(precisely separated by a function)とは,
					或る連続関数$f:S \longrightarrow [0,1]$によって
					$A = f^{-1}(\{0\}),\ B = f^{-1}(\{1\})$が満たされることをいう.
			\end{itemize}
		\end{dfn}
	\end{screen}
	
	\begin{screen}
		\begin{thm}[位相的に識別可能な二点は相異なる]
			$S$を位相空間とするとき,任意の$x,y \in S$に対し
			\begin{align}
				\mbox{$x$と$y$が位相的に識別可能} \quad \Longrightarrow \quad
				x \neq y .
			\end{align}
		\end{thm}
	\end{screen}
	
	\begin{prf}
		$x = y$なら$\overline{\{x\}} = \overline{\{y\}}$となる.
		後述の$T_0$空間とは,この逆が満たされる位相空間である.
		\QED
	\end{prf}
	
	\begin{screen}
		\begin{thm}[分離される集合は他方を含まない近傍を持つ]
		\label{thm:the_equivalent_condition_of_separatedness}
			位相空間$S$において,$A,B \subset S$が分離されることと
			\begin{align}
				A \subset U,\quad B \subset V,\quad 
				A \cap V = \emptyset,
				\quad B \cap U = \emptyset
				\label{eq:thm_the_equivalent_condition_of_separatedness}
			\end{align}
			を満たす開集合$U,V$が存在することは同値である.
		\end{thm}
	\end{screen}
	
	\begin{prf}
		$A,B \subset S$が分離されるとき,$U \coloneqq \overline{B}^c,\ V \coloneqq \overline{A}^c$
		とおけば(\refeq{eq:thm_the_equivalent_condition_of_separatedness})が成立する.
		逆に$A,B$に対し(\refeq{eq:thm_the_equivalent_condition_of_separatedness})を満たす
		開集合$U,V$が存在するとき,$\closure{A} \subset V^c \subset B^c$及び
		$\closure{B} \subset U^c \subset A^c$となるから$A,B$は分離される.
		\QED
	\end{prf}
	
	\begin{screen}
		\begin{dfn}[分離公理]\mbox{}
			\begin{itemize}
				\item 任意の二点が位相的に識別可能である位相空間を$T_0$空間,或はKolmogorov空間という.
				\item 任意の二点が分離される位相空間を$T_1$空間という.
				\item 任意の二点が互いに交わらない近傍を持つ位相区間を$T_2$空間,或はHausdorff空間という.
				\item 任意の交わらない点(一点集合)と閉集合が近傍で分離される位相空間を
					正則(regular)空間という.
				\item $T_0$かつ正則な位相空間を$T_3$空間,或は正則Hausdorff空間という.
				\item 任意の交わらない二つの閉集合が近傍で分離される位相空間を正規(normal)空間という.
				\item $T_1$かつ正規な位相空間を$T_4$空間,或は正規Hausdorff空間という.
				\item 任意の部分位相空間が正規である位相空間は全部分正規(completely normal)であるという.
				\item $T_1$かつ全部分正規な位相空間を$T_5$空間,或は全部分正規Hausdorff空間という.
				\item 任意の交わらない二つの閉集合が関数でちょうど分離される位相空間は完全正規(perfectly normal)であるという.
				\item $T_1$かつ完全正規な位相空間を$T_6$空間,或は完全正規Hausdorff空間という.
			\end{itemize}
		\end{dfn}
	\end{screen}
	
	\begin{screen}
		\begin{thm}[$T_1$空間とは一点集合が閉である空間]
			位相空間$S$に対し,
			\begin{align}
				\mbox{$S$が$T_1$}
				&\quad \Longleftrightarrow \quad \mbox{$S$は$T_0$かつ位相的に識別可能な任意の二点が分離される} \\
				&\quad \Longleftrightarrow \quad \mbox{$S$の任意の一点集合は閉} \\
				&\quad \Longleftrightarrow \quad \mbox{$x \in S$が$A \subset S$の集積点であることと$x$の任意の近傍が$A$と交わることは同値}.
			\end{align}
		\end{thm}
	\end{screen}
	
	\begin{screen}
		\begin{thm}[Hausdorff空間のコンパクト部分集合は閉]
			Hausdorff空間のコンパクト部分集合は閉である.
		\end{thm}
	\end{screen}
	
	\begin{prf}
		$S$をHausdorff空間,$K \subset S$をコンパクト部分集合とするとき,
		任意に$x \in S \backslash K,\ y \in K$を取れば
		\begin{align}
			x \in U_y,\quad y \in V_y, \quad U_y \cap V_y = \emptyset
		\end{align}
		を満たす開集合$U_y,V_y$が取れる.或る$\{y_i\}_{i=1}^n \subset K$に対し
		$K \subset \bigcup_{i=1}^n V_{y_i}$となるから,
		$U \coloneqq \bigcap_{i=1}^n U_{y_i}$とおけば
		\begin{align}
			x \in U,\quad U \subset \bigcap_{i=1}^n \left(S\backslash V_{y_i}\right)
			\subset S \backslash K
		\end{align}
		が成立する.従って$S \backslash K$は開集合であり,$K$は閉集合である.
		\QED
	\end{prf}
	
	\begin{screen}
		\begin{thm}[Hausdorff空間においてコンパクト集合の閉部分集合はコンパクト]
			$S$をHausdorff空間,$K \subset S$をコンパクト部分集合,$F \subset S$を閉集合とするとき,
			$K \cap F$はコンパクトである.
		\end{thm}
	\end{screen}
	
	\begin{prf}
		$K \cap F$の任意の開被覆に$S \backslash F$を加えれば
		$K$の開被覆となるから,そのうち$K$の有限被覆を取ることができる.
		$S \backslash F$を除けば$K \cap F$の有限被覆が残り
		$K \cap F$のコンパクト性が出る.
		\QED
	\end{prf}
	
	\begin{screen}
		\begin{thm}[Hausdorff空間とは交わらない二つのコンパクト集合が近傍で分離される空間]
		\label{thm:Hausdorff_space_two_disjoint_compact_sets_are_separated_by_nbh}
			位相空間において,Hausdorff性と,交わらない二つのコンパクト集合が近傍で分離されることは同値である.
		\end{thm}
	\end{screen}
	
	\begin{prf}
		$A,B$をHausdorff空間の交わらないコンパクト集合とするとき,
		任意の$p \in A$に対し
		\begin{align}
			p \in U_p,\quad B \subset V_p,\quad U_p \cap V_p = \emptyset
			\label{eq:thm_Hausdorff_space_two_disjoint_compact_sets_are_separated_by_nbh_1}
		\end{align}
		を満たす開集合$U_p,V_p$が存在する.実際
		任意の$q \in B$に対し
		\begin{align}
			p \in U_p(q),\quad q \in V_p(q),\quad U_p(q) \cap U_p(q) = \emptyset
		\end{align}
		を満たす開集合$U_p(q), U_p(q)$が取れ,$B$のコンパクト性より
		或る$\{q_i\}_{i=1}^n \subset B$で$B \subset \bigcup_{i=1}^n U_p(q_i)$となるから,
		\begin{align}
			U_p \coloneqq \bigcap_{i=1}^n U_p(q_i),
			\quad V_p \coloneqq \bigcup_{i=1}^n V_p(q_i)
		\end{align}
		とおけば(\refeq{eq:thm_Hausdorff_space_two_disjoint_compact_sets_are_separated_by_nbh_1})
		が成立する.$A$のコンパクト性より或る$\{p_j\}_{j=1}^m \subset A$で
		$A \subset \bigcup_{j=1}^m U_{p_j}$となるから,
		\begin{align}
			U \coloneqq \bigcup_{j=1}^m U_{p_j},
			\quad V \coloneqq \bigcap_{j=1}^m V_{p_j}
		\end{align}
		とおけば$A$と$B$は$U,V$により分離される.
		逆の主張は一点集合がコンパクトであることより従う.
		\QED
	\end{prf}
	
	\begin{screen}
		\begin{thm}[Hausdorff空間値連続写像の等価域は閉]
			$S$を位相空間,$T$をHausdorff空間,$f,g$を
			$S$から$T$への連続写像とするとき,$E \coloneqq \Set{x \in S}{f(x) = g(x)}$は$S$で閉じている.
			特に,$E$が$X$で稠密なら$f=g$となる.
		\end{thm}
	\end{screen}
	
	\begin{prf}
		任意に$x \in \Set{x \in S}{f(x) \neq g(x)}$を取れば,Hausdorff性より
		\begin{align}
			f(x) \in A,\quad g(x) \in B,\quad A \cap B = \emptyset
		\end{align}
		を満たす$T$の開集合$A,B$が存在する.
		$f^{-1}(A) \cap g^{-1}(B)$は$x$の開近傍であり,
		\begin{align}
			f^{-1}(A) \cap g^{-1}(B) \subset \Set{x \in S}{f(x) \neq g(x)}
		\end{align}
		となるから$\Set{x \in S}{f(x) \neq g(x)}$は$S$の開集合である.
		従って$E$は閉である.
		\QED
	\end{prf}
	
	\begin{screen}
		\begin{thm}[正則空間とは交わらないコンパクト集合と閉集合が近傍で分離できる空間]
		\label{thm:each_point_in_regular_space_has_closesd_local_base}\mbox{}
			\begin{description}
				\item[(1)] 位相空間において,正則性と,交わらないコンパクト集合と閉集合が近傍で分離されることは同値である.
					
				\item[(2)]
					$K,W$をそれぞれ局所コンパクトな正則空間のコンパクト集合,開集合とするとき,
					閉包がコンパクトな開集合$U$が存在して次を満たす:
					\begin{align}
						K \subset U \subset \overline{U} \subset W.
						\label{eq:thm_each_point_in_regular_space_has_closesd_local_base}
					\end{align}
			\end{description}
		\end{thm}
	\end{screen}
	
	\begin{prf}\mbox{}
		\begin{description}
			\item[(1)]
				$K,F$を正則空間のコンパクト集合,閉集合とするとき,
				$K \cap F = \emptyset$なら任意の点$x \in K$に対して
				\begin{align}
					x \in U_x,\ \quad F \subset V_x,
					\quad U_x \cap V_x = \emptyset
				\end{align}
				を満たす開集合$U_x,V_x$が取れる.
				$K$はコンパクトであるから或る$\{x_i\}_{i=1}^n \subset K$で
				$K \subset \bigcup_{i=1}^n U_{x_i}$となり
				\begin{align}
					K \subset U \coloneqq \bigcup_{i=1}^n U_{x_i},
					\quad F \subset V \coloneqq \bigcap_{i=1}^n V_{x_i},
					\quad U \cap V = \emptyset
				\end{align}
				が成立する.逆の主張は一点集合がコンパクトであることにより従う.
			\item[(2)]
				任意の$x \in K$に対し,$\overline{U_x} \subset W$
				となる開近傍$U_x$と閉包がコンパクトな開近傍$C_x$が存在するから,
				\begin{align}
					K \subset (C_{y_1} \cap U_{y_1}) \cup \cdots \cup (C_{y_m} \cap U_{y_m})
				\end{align}
				を満たす$\{y_i\}_{i=1}^m \subset K$に対し
				$U \coloneqq \bigcup_{i=1}^m C_{y_i} \cap U_{y_i}$
				とおけば,$\overline{U}$はコンパクトであり
				(\refeq{eq:thm_each_point_in_regular_space_has_closesd_local_base})を満たす.
				\QED
		\end{description}
	\end{prf}
	
	\begin{screen}
		\begin{thm}[局所コンパクトなら$T_2$と$T_3$は同値]
		\label{thm:T_2_equals_to_T_3_in_locally_compact_spaces}
			局所コンパクト位相空間において,$T_2 \Longleftrightarrow T_3$である.
		\end{thm}
	\end{screen}
	
	\begin{prf}
		$T_3$ならば$T_2$であるから$\Longleftarrow$を得る.
		逆に$S$を局所コンパクトHausdorff空間とし,点$x$と閉集合$F$が$x \notin F$を満たしているとする.
		$x$のコンパクトな近傍$K$を取れば,Hausdorff性より$K \cap F$はコンパクトであるから
		\begin{align}
			U_0 \cap V_0 = \emptyset, \quad x \in U_0,  \quad K \cap F \subset V_0
		\end{align}
		を満たす開集合$U_0,V_0$が存在する.このとき,
		\begin{align}
			U \coloneqq U_0 \cap K^{\mathrm{o}},
			\quad V \coloneqq V_0 \cup (S \backslash K)
		\end{align}
		により開集合$U,V$を定めれば
		\begin{align}
			U \cap V = \emptyset,
			\quad x \in U,
			\quad F \subset V
		\end{align}
		が成立し,$S$の正則性が出る.$S$は$T_0$空間でもあるから$T_3$である.
		\QED
	\end{prf}
	
	\begin{screen}
		\begin{thm}[正規空間とは交わらない二つの閉集合が関数で分離される空間(Urysohnの補題)]
		\label{thm:Urysohn_lemma}
			位相空間において,正規性と,任意の交わらない二つの閉集合が関数で分離されることは同値である.
		\end{thm}
	\end{screen}
	
	\begin{screen}
		\begin{dfn}[$G_\delta$集合・$F_\sigma$集合]
			位相空間の部分集合で,開集合の可算交叉で表されるものを$G_\delta$集合,
			閉集合の可算和で表されるものを$F_\sigma$集合と呼ぶ.
			特に,任意の閉集合が$G_\delta$である空間では任意の開集合が$F_\sigma$となる.
		\end{dfn}
	\end{screen}
	
	\begin{screen}
		\begin{thm}[完全正規空間とは正規かつ閉集合が全て$G_\delta$である空間]
		\label{thm:perfectly_normal_Hausdorff_is_normal_and_closed_is_G_delta}\mbox{}
			\begin{description}
				\item[(1)]
					$F$を完全正規空間の閉集合とすれば,次を満たす閉集合系$(F_n)_{n=1}^\infty$が存在する:
					\begin{align}
						F = \bigcap_{n=1}^\infty F_n,
						\quad F_n^{\mathrm{o}} \supset F_{n+1}. 
					\end{align}
					
				\item[(2)]
					位相空間において,完全正規であることと,正規かつ任意の閉集合が$G_\delta$であることは同値である.
			\end{description}
		\end{thm}
	\end{screen}
	
	\begin{prf}
		$S$を完全正規空間,$A,B$を互いに交わらない$S$の閉集合とすれば,
		$A=f^{-1}(\{0\}),\ B = f^{-1}(\{1\})$を満たす連続関数
		$f:S \longrightarrow \R$が存在する.このとき
		$U \coloneqq f^{-1}([0,1/2)),\ V \coloneqq f^{-1}((1/2,1])$
		で開集合$U,V$を定めれば
		\begin{align}
			A \subset U,\quad B \subset V,\quad U \cap V = \emptyset
		\end{align}
		となるから$S$は正規である.また$F$を閉集合とすれば
		或る連続関数$g:S \longrightarrow \R,\ (\emptyset = g^{-1}(\{1\}))$により
		\begin{align}
			F = g^{-1}(\{0\}) 
			= g^{-1}\Biggl(\bigcap_{n=1}^\infty\left[0,n^{-1}\right)\Biggr)
			= \bigcap_{n=1}^\infty g^{-1}\left(\left[0,n^{-1}\right)\right)
		\end{align}
		が成立するから$F$は$G_\delta$である.特に,このとき
		$F_n \coloneqq g^{-1}\left(\left[0,n^{-1}\right]\right)$とおけば
		\begin{align}
			F = \bigcap_{n=1}^\infty g^{-1}\left(\left[0,n^{-1}\right]\right)
			= \bigcap_{n=1}^\infty F_n,
			\quad F_n^{\mathrm{o}} \supset g^{-1}\left(\left[0,n^{-1}\right)\right)
			\supset g^{-1}\left(\left[0,(n+1)^{-1}\right]\right)
			= F_{n+1}
		\end{align}
		となり(1)の主張が得られる.逆に$S$が正規かつ
		閉集合が全て$G_\delta$であるとき,任意の交わらない閉集合$A,B$に対し
		$A = \bigcap_{n=1}^\infty U_n,\ B = \bigcap_{n=1}^\infty V_n$
		を満たす開集合系$(U_n)_{n=1}^\infty,\ (V_n)_{n=1}^\infty$が取れて,
		定理\ref{thm:Urysohn_lemma}より各$n \geq 1$で
		\begin{align}
			f_n(A) = \{0\},\quad f_n(S \backslash U_n) = \{1\},
			\quad g_n(B) = \{0\},\quad g_n(S \backslash V_n) = \{1\}
		\end{align}
		を満たす連続写像$f_n,g_n:S \longrightarrow [0,1]$が存在する.
		ここで連続写像を$f \coloneqq \sum_{n=1}^\infty 2^{-n} f_n,\ 
		g \coloneqq \sum_{n=1}^\infty 2^{-n} g_n$で定めれば
		\begin{align}
			\begin{cases}
				f(x) = 0, & (x \in A), \\
				f(x) > 0, & (x \notin A),
			\end{cases}
			\quad \begin{cases}
				g(x) = 0, & (x \in B), \\
				g(x) > 0, & (x \notin B),
			\end{cases}
		\end{align}
		となり,$h \coloneqq f/(f+g)$とおけば$A = h^{-1}(\{0\}),\ B = h^{-1}(\{1\})$が成立する.
		従って$S$は完全正規である.
		\QED
	\end{prf}
	
	\begin{screen}
		\begin{thm}[連続な単射の引き戻しによる分離性の遺伝]
			$S,T$を位相空間とする.$S$から$T$への連続単射が存在するとき,
			$T$が$T_k$-空間$(k=0,1,\cdots,6)$なら
			$S$もまた$T_k$-空間となる.
		\end{thm}
	\end{screen}
	
	\begin{prf}
		任意に異なる二点$s_1,s_2 \in S$を取れば単射性より$f(s_1) \neq f(s_2)$となる.
		$T$の分離性より
	\end{prf}
	
\subsection{可算公理}
	\begin{screen}
		\begin{thm}[可算コンパクト性の同値条件]
		\end{thm}
	\end{screen}
	
	\begin{screen}
		\begin{thm}[第二可算空間の任意の基底は可算基を内包する]\label{thm:countable_base_of_second_countable_space}
			$\mathscr{B}$を第二可算空間$S$の任意の基底とするとき,或る可算部分集合
			$\mathscr{B}_0 \subset \mathscr{B}$もまた$S$の基底となる.
			すなわち第二可算空間はLindel\Ddot{o}f性を持つ.
		\end{thm}
	\end{screen}
	
	\begin{prf}
		$\mathscr{D}$を$S$の可算基とする.
		任意の開集合$U$に対し或る$\mathscr{B}_U \subset \mathscr{B}$が存在して
		$U = \bigcup_{V \in \mathscr{B}_U}V$を満たすから,
		\begin{align}
			\mathscr{D}_U \coloneqq
			\Set{W \in \mathscr{D}}{W \subset V,\ V \in \mathscr{B}_U}
			\label{eq:thm_countable_base_of_second_countable_space_1}
		\end{align}
		とおけば$U = \bigcup_{V \in \mathscr{B}_U} V
			= \bigcup_{V \in \mathscr{B}_U} \bigcup_{\substack{W \in \mathscr{D}_U \\ W \subset V}} W
			\subset \bigcup_{W \in \mathscr{D}_U} W
			\subset U$より
		\begin{align}
			U = \bigcup_{W \in \mathscr{D}_U} W
			\label{eq:thm_countable_base_of_second_countable_space_2}
		\end{align}
		が成り立つ.ここで(\refeq{eq:thm_countable_base_of_second_countable_space_1})より
		任意の$W \in \mathscr{D}_U$に対して
		$\Set{V \in \mathscr{B}}{W \subset V} \neq \emptyset$であるから
		\begin{align}
			\Phi_U \in \prod_{W \in \mathscr{D}_U} \Set{V \in \mathscr{B}}{W \subset V}
		\end{align}
		が取れる.$\mathscr{B}_U' \coloneqq \Set{\Phi_U(W)}{W \in \mathscr{D}_U}$とすれば
		$U = \bigcup_{W \in \mathscr{D}_U} W \subset \bigcup_{W \in \mathscr{D}_U} \Phi(W)
		\subset \bigcup_{V \in \mathscr{B}_U'} V \subset U$より
		\begin{align}
			U = \bigcup_{V \in \mathscr{B}_U'} V
			\label{eq:thm_countable_base_of_second_countable_space_3}
		\end{align}
		が満たされ,
		\begin{align}
			\mathscr{B}_0 \coloneqq \bigcup_{W \in \mathscr{D}} \mathscr{B}_W'
		\end{align}
		と定めれば$\mathscr{B}_0$は求める$S$の可算基となる.実際,任意の開集合$U$に対し
		(\refeq{eq:thm_countable_base_of_second_countable_space_2})と
		(\refeq{eq:thm_countable_base_of_second_countable_space_3})より
		\begin{align}
			U = \bigcup_{W \in \mathscr{D}_U} W
			= \bigcup_{W \in \mathscr{D}_U} \bigcup_{V \in \mathscr{B}_W'} V
		\end{align}
		となる.
		\QED
	\end{prf}
	
	\begin{screen}
		\begin{thm}[局所コンパクトHausdorff空間が第二可算なら$\sigma$-コンパクト]\label{thm:second_countable_Hausdorff_sigma_compact}
			$S$が第二可算性をもつ局所コンパクトHausdorff空間なら,
			次を満たすコンパクト部分集合の列$(K_n)_{n=1}^\infty$が存在する:
			\begin{align}
				K_n \subset K_{n+1}^{\mathrm{o}},
				\quad S = \bigcup_{n=1}^\infty K_n.
			\end{align}
		\end{thm}
	\end{screen}
	
	\begin{prf}
		任意の$x \in S$に対して閉包がコンパクトな開近傍$U_x$を取っておく.
		$\mathscr{O}$を$S$の開集合系として
		\begin{align}
			\mathscr{B} \coloneqq
			\Set{U \in \mathscr{O}}{\mbox{$\overline{U}$がコンパクト}}
		\end{align}
		とおけば,$\mathscr{B}$は$\mathscr{O}$の基底となる.実際,
		任意の$O \in \mathscr{O}$に対し$O \cap U_x \in \mathscr{B}$かつ
		\begin{align}
			O = \bigcup_{x \in O} O \cap U_x
		\end{align}
		となる.従って定理\ref{thm:countable_base_of_second_countable_space}より
		或る可算部分集合$\{U_n\}_{n=1}^\infty \subset \mathscr{B}$が
		$\mathscr{O}$の基底となる.いま,$K_1 \coloneqq \overline{U_1}$として,
		またコンパクト集合$K_n$が選ばれたとして,
		$K_n$の有限被覆$\mathscr{U}_n \subset \mathscr{B}_0$を取り
		\begin{align}
			K_{n+1} \coloneqq \overline{U_{n+1}} \cup \bigcup_{V \in \mathscr{U}_n} \overline{V}
		\end{align}
		とすれば,$K_{n+1}$はコンパクトであり$K_n \subset K_{n+1}^{\mathrm{o}}$を満たす.
		この操作で$(K_n)_{n=1}^\infty$を構成すれば
		\begin{align}
			S = \bigcup_{n=1}^\infty U_n \subset \bigcup_{n=1}^\infty K_n \subset S
		\end{align}
		が成立する.
		\QED
	\end{prf}
	
\subsection{距離空間}
	\begin{screen}
		\begin{thm}[距離関数の連続性]
			$(x,y) \longmapsto d(x,y)$は直積位相に関し連続である.
			$x \longmapsto d(x,A)$は連続である.
		\end{thm}
	\end{screen}
	
	\begin{screen}
		\begin{thm}[距離空間の完全正規性]
			任意の距離空間は,その距離で導入する位相により$T_6$空間となる.
		\end{thm}
	\end{screen}
	
	\begin{prf}
		$(S,d)$を距離空間とし距離位相を導入すれば,$S$はHausdorffとなる.
		実際相異なる二点$x,y$に対し
		\begin{align}
			B_\epsilon(x) \coloneqq \Set{s \in S}{d(s,x) < \frac{\epsilon}{2}},
			\quad B_\epsilon(y) \coloneqq \Set{s \in S}{d(s,y) < \frac{\epsilon}{2}},
			\quad (\epsilon \coloneqq d(x,y))
		\end{align}
		で交わらない開球を定めれば,$x,y$は
		これらで分離される.また$A,B$を交わらない閉集合として
		\begin{align}
			f(x) \coloneqq \frac{d(x,A)}{d(x,A) + d(x,B)},
			\quad (\forall x \in S)
		\end{align}
		により連続写像$f:S \longrightarrow \R$を定めれば,
		$A = f^{-1}(\{0\}),\ B = f^{-1}(\{1\})$となるから$S$は完全正規である.
		\QED
	\end{prf}
	
\subsection{範疇定理}
	\begin{screen}
		\begin{thm}[Cantorの共通部分定理]\label{thm:Cantor_intersection_theorem}
			$S$をHausdorff空間とし,
			$(K_n)_{n=1}^\infty$をコンパクト部分集合の列とする.
			このとき,任意の$n \geq 1$に対して$\bigcap_{i=1}^n K_i \neq \emptyset$なら
			$\bigcap_{i=1}^\infty K_i \neq \emptyset$が成り立つ.
		\end{thm}
	\end{screen}
	
	\begin{prf}
		$\bigcap_{i=1}^\infty K_i = \emptyset$と仮定すれば,
		$K_1 \subset \bigcup_{n=1}^\infty K_n^c = S$と$K_1$のコンパクト性より
		\begin{align}
			K_1 \subset \bigcup_{n=1}^N K_n^c = \Biggl( \bigcap_{n=1}^N K_n \Biggr)^c
		\end{align}
		を満たす$N \geq 1$が存在し,$\bigcap_{n=1}^N K_n \subset K_1$より$\bigcap_{n=1}^N K_n = \emptyset$が従う.
		\QED
	\end{prf}
	
	\begin{screen}
		\begin{dfn}[疎集合・第一類集合・第二類集合]
			位相空間$S$の部分集合$A$が疎である(nowhere dense)とは
			$A$の閉包の内核が$\overline{A}^{\mathrm{o}} = \emptyset$を満たすことをいう.
			$S$が可算個の疎集合の合併で表せるとき$S$を第一類集合(the set of the first category)と呼び,
			そうでない場合はこれを第二類集合と呼ぶ.
		\end{dfn}
	\end{screen}
	
	\begin{screen}
		\begin{thm}[Baireの範疇定理]\label{thm:Baire_category_theorem}
			空でない完備距離空間と局所コンパクトHausdorff空間は第二類集合である.
		\end{thm}
	\end{screen}
	
	\begin{prf} $S \neq \emptyset$を完備距離空間,或は局所コンパクトHausdorff空間とする.\mbox{}
		\begin{description}
			\item[第一段]
				$(V_n)_{n=1}^\infty$を$S$で稠密な開集合系とするとき
				\begin{align}
					\overline{\bigcap_{n=1}^\infty V_n} = S,
					\label{eq:thm_Baire_category_theorem_1}
				\end{align}
				となることを示す.実際(\refeq{eq:thm_Baire_category_theorem_1})が満たされていれば,
				任意の疎集合系$(E_n)_{n=1}^\infty$に対して
				\begin{align}
					V_n \coloneqq \overline{E_n}^c,
					\quad n=1,2,\cdots
				\end{align}
				で開集合系$(V_n)$を定めると定理\ref{thm:topology_note_closure_interior}より
				\begin{align}
					\overline{V_n} = \overline{E_n}^{ca} = \overline{E_n}^{ic} = \emptyset^c = S
				\end{align}
				となるから,$\bigcap_{n=1}^\infty V_n \neq \emptyset$が従い
				$S \neq \bigcup_{n=1}^\infty \overline{E_n} \supset \bigcup_{n=1}^\infty E_n$
				が成り立つ.従って$S$は第二類である.
				
			\item[第二段]
				任意の空でない開集合$B_0$に対し$B_0 \cap \left( \bigcap_{n=1}^\infty V_n \right) \neq \emptyset$
				となることを示せば(\refeq{eq:thm_Baire_category_theorem_1})が従う.
				$V_1$は稠密であるから$B_0 \cap V_1 \neq \emptyset$となり,
				点$x_1 \in B_0 \cap V_1$を取れば,
				$S$が距離空間なら或る半径$<1$の開球$B_1$が存在して
				\begin{align}
					x_1 \in B_1 \subset \overline{B_1} \subset B_0 \cap V_1
					\label{eq:thm_Baire_category_theorem_2}
				\end{align}
				を満たす.$S$が局所コンパクトHausdorffの場合も,
				定理\ref{thm:each_point_in_regular_space_has_closesd_local_base}と
				定理\ref{thm:T_2_equals_to_T_3_in_locally_compact_spaces}より
				(\refeq{eq:thm_Baire_category_theorem_2})を満たす
				相対コンパクトな開集合$B_1$が取れる.
				同様に半径$<1/n$の開球,或は相対コンパクトな開集合$B_n$と$x_n \in S$で
				\begin{align}
					x_n \in B_n \subset \overline{B_n} \subset B_{n-1} \cap V_n
				\end{align}
				を満たすものが存在する.このとき$S$が完備距離空間なら$(x_n)_{n=1}^\infty$は
				Cauchy列をなし,その極限点$x_\infty$は
				\begin{align}
					x_\infty \in \bigcap_{n=1}^\infty \overline{B_n}
				\end{align}
				を満たす.$S$が局所コンパクトHausdorff空間なら定理\ref{thm:Cantor_intersection_theorem}より
				\begin{align}
					\bigcap_{n=1}^\infty \overline{B_n} \neq \emptyset
				\end{align}
				となるから,いずれの場合も
				\begin{align}
					\emptyset \neq \bigcap_{n=1}^\infty \overline{B_n} 
					\subset B_0 \cap \Biggl( \bigcap_{n=1}^\infty V_n \Biggr)
				\end{align}
				が従い定理の主張が得られる.
				\QED
		\end{description}
	\end{prf}
	
	\begin{screen}
		\begin{thm}[閉包・内核]\label{thm:topology_note_closure_interior}
			$S$を位相空間,$h:S \longrightarrow S$を同相,$A$を$S$の部分集合とするとき次が成り立つ.
			\begin{description}
				\item[(1)] $A^{ic} = A^{ca}$.
				\item[(2)] $h(A^a) = h(A)^a$.
				\item[(3)] $h(A^i) = h(A)^i$.
			\end{description}
		\end{thm}
	\end{screen}
	
	\begin{prf}\mbox{}
		\begin{description}
			\item[(1)]
				$A^i \subset A$より$A^{ic} \supset A^c$が従い,
				$A^{ic}$が閉であるから$A^{ic} \supset A^{ca}$となる.
				一方で$A^c \subset A^{ca}$より$A \supset A^{cac}$が従い,
				$A^{cac}$は開であるから$A^i \supset A^{cac}$すなわち
				$A^{ic} \subset A^{ca}$となる.
			
			\item[(2)]
				$h(A) \subset h(A^a)$かつ$h(A^a)$は閉であるから$h(A)^a \subset h(A^a)$が従う.一方で
				任意の$x \in h(A^a)$に対し$x = h(y)$を満たす
				$y \in A^a$と$x$の任意の近傍$V$を取れば,
				$h^{-1}(V) \cap A \neq \emptyset$より
				$V \cap h(A) \neq \emptyset$が成り立ち
				$x \in h(A)^a$となる.
				
			\item[(3)]
				$h(A^i) \subset h(A)$かつ$h(A^i)$は開であるから
				$h(A^i) \subset h(A)^i$が従う.一方で
				任意の開集合$O \subset h(A)$に対し
				$h^{-1}(O) \subset A$より
				$h^{-1}(O) \subset A^i$となり,
				$O \subset h(A^i)$が成り立つから
				$h(A)^i \subset h(A^i)$が得られる.
				\QED
		\end{description}
	\end{prf}
	
	\begin{screen}
		\begin{thm}[第一類集合の性質]
			$S$を位相空間とする.
			\begin{description}
				\item[(a)] $A \subset B \subset S$に対し$B$が第一類なら$A$も第一類である.
				\item[(b)] 第一類集合の可算和も第一類である.
				\item[(c)] 内核が空である閉集合は第一類である.
				\item[(d)] $S$から$S$への位相同型$h$と$E \subset S$に対し次が成り立つ:
					\begin{align}
						\mbox{$E$が第一類} \quad \Longleftrightarrow \quad
						\mbox{$h(E)$が第一類}.
					\end{align}
			\end{description}
		\end{thm}
	\end{screen}
	
	\begin{prf}\mbox{}
		\begin{description}
			\item[(a)] $B = \bigcup_{n=1}^\infty E_n$
				を満たす疎集合系$(E_n)_{n=1}^\infty$に対し
				$A \cap E_n$は疎であり$A = \bigcup_{n=1}^\infty (A \cap E_n)$となる.
			\item[(b)] $A_n \subset S,\ (n=1,2,\cdots)$が第一類集合とし
				$(E_{n,i})_{i=1}^\infty$を$A_n = \bigcup_{i=1}^\infty E_{n,i}$
				を満たす疎集合系とすれば
				\begin{align}
					\bigcup_{n=1}^\infty A_n
					= \bigcup_{n,i=1}^\infty E_{n,i}
				\end{align}
				が成り立つ.
				
			\item[(c)] 内核が空である閉集合はそれ自身が疎であり,自身の可算和に一致する.
			\item[(d)] $E$が第一類のとき,$E = \bigcup_{i=1}^\infty E_i$を満たす
				疎集合系$(E_i)_{i=1}^\infty$に対し定理\ref{thm:topology_note_closure_interior}より
				\begin{align}
					\emptyset = h(E_i^{ai})
					= h(E_i^a)^i
					= h(E_i)^{ai}
				\end{align}
				が成り立つから$h(E_i)$は疎であり,
				\begin{align}
					h(E) = \bigcup_{i=1}^\infty h(E_i)
				\end{align}
				となるから$h(E)$も第一類である.$h(E)$が第一類なら$E = h^{-1}(h(E))$も第一類である.
				\QED
		\end{description}
	\end{prf}
	
\subsection{有向点族}
\section{集合か位相的な}
\subsection{位相}
	位相空間$S$の部分集合$A$について,$A$の内核を$A^i$或は$A^{\mathrm{o}}$と書き,
	$A$の閉包を$A^a$或は$\overline{A}$と書く.
	また$A^{ca} = (A^c)^a,\ A^{ic} = (A^i)^c$と略記する.
	\begin{screen}
		\begin{thm}[閉包・内核]\label{thm:topology_note_closure_interior}
			$S$を位相空間,$h:S \longrightarrow S$を同相,$A$を部分集合とするとき次が成り立つ.
			\begin{description}
				\item[(1)] $A^{ic} = A^{ca}$.
				\item[(2)] $h(E^a) = h(E)^a$.
				\item[(3)] $h(E^i) = h(E)^i$.
			\end{description}
		\end{thm}
	\end{screen}
	
	\begin{prf}\mbox{}
		\begin{description}
			\item[(1)]
				$A^i \subset A$より$A^{ic} \supset A^c$が従い,
				$A^{ic}$が閉であるから$A^{ic} \supset A^{ca}$となる.
				一方で$A^c \subset A^{ca}$より$A \supset A^{cac}$が従い,
				$A^{cac}$は開であるから$A^i \supset A^{cac}$すなわち
				$A^{ic} \subset A^{ca}$となる.
			
			\item[(2)]
				$h(E) \subset h(E^a)$かつ$h(E^a)$は閉であるから$h(E)^a \subset h(E^a)$が従う.一方で
				任意の$x \in h(E^a)$に対し$x = h(y)$を満たす
				$y \in E^a$と$x$の任意の近傍$V$を取れば,
				$h^{-1}(V) \cap E \neq \emptyset$より
				$V \cap h(E) \neq \emptyset$が成り立ち
				$x \in h(E)^a$となる.
				
			\item[(3)]
				$h(E^i) \subset h(E)$かつ$h(E^i)$は開であるから
				$h(E^i) \subset h(E)^i$が従う.一方で
				任意の開集合$O \subset h(E)$に対し
				$h^{-1}(O) \subset E$より
				$h^{-1}(O) \subset E^i$となり,
				$O \subset h(E^i)$が成り立つから
				$h(E)^i \subset h(E^i)$が得られる.
				\QED
		\end{description}
	\end{prf}
	
	\begin{screen}
		\begin{thm}[有限交叉性]\label{thm:finite_intersection_property}
			位相空間$S$がコンパクトであることと,
			任意の閉集合系$(U_\lambda)_{\lambda \in \Lambda}$に対して
			\begin{align} 
				\bigcap_{\lambda \in F} U_\lambda \neq \emptyset
				\mbox{, for every finite subset $F \subset \Lambda$}
				\quad \Longrightarrow \quad \bigcap_{\lambda \in \Lambda} U_\lambda \neq \emptyset
				\label{eq:finite_intersection_property}
			\end{align}
			となることは同値である.
		\end{thm}
	\end{screen}
	
	\begin{prf}
		任意の閉集合系$(U_\lambda)_{\lambda \in \Lambda}$に対して
		$\bigcap_{\lambda \in \Lambda} U_\lambda = \emptyset$なら
		$(U_\lambda^c)_{\lambda \in \Lambda}$は$S$の開被覆となるから,
		$S$がコンパクトであることと(\refeq{eq:finite_intersection_property})は同値である.
		\QED
	\end{prf}
	
	\begin{screen}
		\begin{thm}[Cantorの共通部分定理]\label{thm:Cantor_intersection_theorem}
			$S$をHausdorff空間とし,
			$(K_n)_{n=1}^\infty$をコンパクト部分集合の列とする.
			このとき,任意の$n \geq 1$に対して$\bigcap_{i=1}^n K_i \neq \emptyset$なら
			$\bigcap_{i=1}^\infty K_i \neq \emptyset$が成り立つ.
		\end{thm}
	\end{screen}
	
	\begin{prf}
		$\bigcap_{i=1}^\infty K_i = \emptyset$と仮定すれば,
		$K_1 \subset \bigcup_{n=1}^\infty K_n^c = S$と$K_1$のコンパクト性より
		\begin{align}
			K_1 \subset \bigcup_{n=1}^N K_n^c = \Biggl( \bigcap_{n=1}^N K_n \Biggr)^c
		\end{align}
		を満たす$N \geq 1$が存在し,$\bigcap_{n=1}^N K_n \subset K_1$より$\bigcap_{n=1}^N K_n = \emptyset$が従う.
		\QED
	\end{prf}
	
	\begin{screen}
		\begin{thm}[局所コンパクトHausdorff空間の正則性]\label{thm:regularity_of_Hausdorff_spaces}
			局所コンパクトHausdorff空間は$T_3$である.
		\end{thm}
	\end{screen}
	
	\begin{prf}
		$X$を局所コンパクトHausdorff空間とし,$x \in X$,$x \notin F$を満たす閉集合$F$,及び
		$x$のコンパクトな近傍$K$を取る.Hausdorff性より$K \cap F$はコンパクトであるから
		\begin{align}
			U_0 \cap V_0 = \emptyset, \quad x \in U_0,  \quad K \cap F \subset V_0
		\end{align}
		を満たす開集合$U_0,V_0$が存在する.
		\begin{align}
			U \coloneqq U_0 \cap K^{\mathrm{o}},
			\quad V \coloneqq V_0 \cup (X \backslash K)
		\end{align}
		により開集合$U,V$を定めれば
		\begin{align}
			U \cap V = \emptyset,
			\quad x \in U,
			\quad F \subset V
		\end{align}
		が成立する.
		\QED
	\end{prf}
	
	\begin{screen}
		\begin{thm}\label{thm:T_3_space}
			$X$を$T_3$空間,$K$をコンパクト集合,$U$を開集合とするとき,或る開集合$V$が存在して
			\begin{align}
				K \subset V \subset \overline{V} \subset U.
				\label{eq:thm_regularity_of_Hausdorff_spaces}
			\end{align}
			を満たす.$X$が局所コンパクトなら$\overline{V}$がコンパクトとなるように$V$を取れる.
		\end{thm}
	\end{screen}
		
	\begin{prf}
		任意の$x \in K$に対し$V_x \subset \overline{V_x} \subset U$を満たす開近傍$V_x$が存在するから,
		\begin{align}
			K \subset V_{x_1} \cup \cdots \cup V_{x_n}
		\end{align}
		となるように$x_1,\cdots,x_n \in K$を取り$V \coloneqq V_{x_1} \cup \cdots \cup V_{x_n}$
		とおけば
		\begin{align}
			K \subset V \subset \overline{V}
			\subset \overline{V_{x_1}} \cup \cdots \overline{\cup V_{x_n}}
			\subset U
		\end{align}
		が成り立つ.$X$が局所コンパクトなら$x \in K$に対し閉包がコンパクトな開近傍$W_x$が存在するから,
		\begin{align}
			K \subset (W_{y_1} \cap V_{y_1}) \cup \cdots \cup (W_{y_m} \cap V_{y_m})
		\end{align}
		となるように$y_1,\cdots,y_n \in K$を取り$V \coloneqq \bigcup_{i=1}^m W_{y_i} \cap V_{y_i}$
		とおけば$\overline{V}$はコンパクトであり(\refeq{eq:thm_regularity_of_Hausdorff_spaces})を満たす.
		\QED
	\end{prf}
	
	\begin{screen}
		\begin{thm}[可算コンパクト性の同値条件]
		\end{thm}
	\end{screen}
	
	\begin{screen}
		\begin{thm}[第二可算空間の任意の基底は可算基を含む]\label{thm:countable_base_of_second_countable_space}
			$\mathscr{B}$を第二可算空間$S$の任意の基底とするとき,或る可算部分集合
			$\mathscr{B}_0 \subset \mathscr{B}$もまた$S$の基底となる.
		\end{thm}
	\end{screen}
	
	\begin{prf}
		$\mathscr{D}$を$S$の可算基とする.
		任意の開集合$U$に対し或る$\mathscr{B}_U \subset \mathscr{B}$が存在して
		$U = \bigcup_{V \in \mathscr{B}_U}V$を満たすから,
		\begin{align}
			\mathscr{D}_U \coloneqq
			\Set{W \in \mathscr{D}}{W \subset V,\ V \in \mathscr{B}_U}
			\label{eq:thm_countable_base_of_second_countable_space_1}
		\end{align}
		とおけば$U = \bigcup_{V \in \mathscr{B}_U} V
			= \bigcup_{V \in \mathscr{B}_U} \bigcup_{\substack{W \in \mathscr{D}_U \\ W \subset V}} W
			\subset \bigcup_{W \in \mathscr{D}_U} W
			\subset U$より
		\begin{align}
			U = \bigcup_{W \in \mathscr{D}_U} W
			\label{eq:thm_countable_base_of_second_countable_space_2}
		\end{align}
		が成り立つ.ここで(\refeq{eq:thm_countable_base_of_second_countable_space_1})より
		任意の$W \in \mathscr{D}_U$に対して
		$\Set{V \in \mathscr{B}}{W \subset V} \neq \emptyset$であるから
		\begin{align}
			\Phi_U \in \prod_{W \in \mathscr{D}_U} \Set{V \in \mathscr{B}}{W \subset V}
		\end{align}
		が取れる.$\mathscr{B}_U' \coloneqq \Set{\Phi_U(W)}{W \in \mathscr{D}_U}$とすれば
		$U = \bigcup_{W \in \mathscr{D}_U} W \subset \bigcup_{W \in \mathscr{D}_U} \Phi(W)
		\subset \bigcup_{V \in \mathscr{B}_U'} V \subset U$より
		\begin{align}
			U = \bigcup_{V \in \mathscr{B}_U'} V
			\label{eq:thm_countable_base_of_second_countable_space_3}
		\end{align}
		が満たされ,
		\begin{align}
			\mathscr{B}_0 \coloneqq \bigcup_{W \in \mathscr{D}} \mathscr{B}_W'
		\end{align}
		と定めれば$\mathscr{B}_0$は求める$S$の可算基となる.実際,任意の開集合$U$に対し
		(\refeq{eq:thm_countable_base_of_second_countable_space_2})と
		(\refeq{eq:thm_countable_base_of_second_countable_space_3})より
		\begin{align}
			U = \bigcup_{W \in \mathscr{D}_U} W
			= \bigcup_{W \in \mathscr{D}_U} \bigcup_{V \in \mathscr{B}_W'} V
		\end{align}
		となる.
		\QED
	\end{prf}
	
	\begin{screen}
		\begin{thm}[局所コンパクトHausdorff空間が第二可算なら$\sigma$-コンパクト]\label{thm:second_countable_Hausdorff_sigma_compact}
			$S$が第二可算公理を満たす局所コンパクトHausdorff空間なら,
			次を満たすコンパクト部分集合の列$(K_n)_{n=1}^\infty$が存在する:
			\begin{align}
				K_n \subset K_{n+1}^{\mathrm{o}},
				\quad S = \bigcup_{n=1}^\infty K_n.
			\end{align}
		\end{thm}
	\end{screen}
	
	\begin{prf}
		任意の$x \in S$に対して閉包がコンパクトな開近傍$U_x$を取っておく.
		$\mathscr{O}$を$S$の開集合系として
		\begin{align}
			\mathscr{B} \coloneqq
			\Set{U \in \mathscr{O}}{\mbox{$\overline{U}$がコンパクト}}
		\end{align}
		とおけば,$\mathscr{B}$は$\mathscr{O}$の基底となる.実際,
		任意の$O \in \mathscr{O}$に対し$O \cap U_x \in \mathscr{B}$かつ
		\begin{align}
			O = \bigcup_{x \in O} O \cap U_x
		\end{align}
		となる.従って定理\ref{thm:countable_base_of_second_countable_space}より
		或る可算部分集合$\{U_n\}_{n=1}^\infty \subset \mathscr{B}$が
		$\mathscr{O}$の基底となる.いま,$K_1 \coloneqq \overline{U_1}$として,
		またコンパクト集合$K_n$が選ばれたとして,
		$K_n$の有限被覆$\mathscr{U}_n \subset \mathscr{B}_0$を取り
		\begin{align}
			K_{n+1} \coloneqq \overline{U_{n+1}} \cup \bigcup_{V \in \mathscr{U}_n} \overline{V}
		\end{align}
		とすれば,$K_{n+1}$はコンパクトであり$K_n \subset K_{n+1}^{\mathrm{o}}$を満たす.
		この操作で$(K_n)_{n=1}^\infty$を構成すれば
		\begin{align}
			S = \bigcup_{n=1}^\infty U_n \subset \bigcup_{n=1}^\infty K_n \subset S
		\end{align}
		が成立する.
		\QED
	\end{prf}
	
	\begin{screen}
		\begin{dfn}[積$\sigma$-加法族]
			$\Lambda$を空でない任意濃度の添字集合、
			$\left((X_\lambda,\mathscr{F}_\lambda)\right)_{\lambda \in \Lambda}$を可測空間の族とし,
			$X \coloneqq \prod_{\lambda \in \Lambda} X_\lambda$とおく.$\lambda$射影を
			$p_\lambda:X \longrightarrow X_\lambda$と書くとき,
			\begin{align}
				\Set{p_\lambda^{-1}(A_\lambda)}{A_\lambda \in \mathscr{F}_\lambda,\ \lambda \in \Lambda}
			\end{align}
			が生成する$\sigma$-加法族を
			$(\mathscr{F}_\lambda)_{\lambda \in \Lambda}$の積$\sigma$-加法族と呼び,
			$\bigotimes_{\lambda \in \Lambda} \mathscr{F}_\lambda$で表す.
			$\Lambda = \{1,2,\cdots\}$の場合,
			\begin{align}
				\bigotimes_{\lambda \in \Lambda} \mathscr{F}_\lambda
				= \mathscr{F}_1 \otimes \mathscr{F}_2 \otimes \cdots
			\end{align}
			とも表記する.
		\end{dfn}
		\end{screen}
	
	\begin{screen}
		\begin{thm}[第二可算空間の直積Borel集合族]\label{thm:Borel_algebra_of_products_of_second_countable_spaces}
			$\Lambda$を空でない高々可算集合,
			$(S_\lambda)_{\lambda \in \Lambda}$を第二可算公理を満たす位相空間の族とする.
			$S \coloneqq \prod_{\lambda \in \Lambda} S_\lambda$とおいて
			直積位相を導入すれば,$S$は第二可算空間となりまた次が成立する:
			\begin{align}
				\borel{S} = \bigotimes_{\lambda \in \Lambda} \borel{S_\lambda}.
				\label{eq:lem1_for_chap_1_exercise_1_8_1}
			\end{align}
		\end{thm}
	\end{screen}

\begin{prf}
	各$S_\lambda$の開集合系及び可算基を$\mathscr{O}_\lambda,\ \mathscr{B}_\lambda$,
	$S$の開集合系を$\mathscr{O}$とし,
	また$\lambda$射影を$p_\lambda:S \longrightarrow S_\lambda$と書く.
	先ず,任意の$O_\lambda \in \mathscr{O}_\lambda$に対して
	$p_\lambda^{-1}(O_\lambda) \in \mathscr{O}$
	が満たされるから
	\begin{align}
		\mathscr{O}_\lambda 
		\subset \Set{A_\lambda \in \borel{S_\lambda}}{p_\lambda^{-1}(A_\lambda) \in \borel{S}}
	\end{align}
	が従い,右辺が$\sigma$-加法族であるから
	\begin{align}
		\bigotimes_{\lambda \in \Lambda} \borel{S_\lambda}
		= \sgmalg{\Set{p_\lambda^{-1}(A_\lambda)}{A_\lambda \in \borel{S_\lambda},\ \lambda \in \Lambda}} \subset \borel{S}
	\end{align}
	を得る.一方で
	\begin{align}
		\mathscr{B} \coloneqq 
		\Set{\bigcap_{\lambda \in \Lambda'} p_\lambda^{-1}(B_\lambda)}{B_\lambda \in \mathscr{B}_\lambda,\ \Lambda' \subset \Lambda:finite\ subset}
	\end{align}
	は$\mathscr{O}$の基底の一つである.実際,任意に$O \in \mathscr{O}$を取れば,
	任意の$x \in O$に対し或る有限集合$\Lambda' \subset \Lambda$が存在して
	\begin{align}
		x \in \bigcap_{\lambda \in \Lambda'} p_\lambda^{-1} (O_\lambda) \subset O
	\end{align}
	が成立するが,更に$S_\lambda$の第二可算性より或る
	$\mathscr{B}'_\lambda \subset \mathscr{B}_\lambda\ (\lambda \in \Lambda')$が存在して
	\begin{align}
		x \in \bigcap_{\lambda \in \Lambda'} p_\lambda^{-1} (O_\lambda)
		= \bigcap_{\lambda \in \Lambda'} \bigcup_{B_{\lambda} \in \mathscr{B}'_\lambda} p_\lambda^{-1} (B_{\lambda})
	\end{align}
	が満たされる.すなわち,任意の$O \in \mathscr{O}$は
	\begin{align}
		O = \bigcup_{E \in \mathscr{B}'} E,
		\quad (\exists\mathscr{B}' \subset \mathscr{B})
	\end{align}
	と表される.$\mathscr{B}$は高々可算の濃度を持ち
	\footnote{
		$L_0 \coloneqq 
		\Set{\Lambda'}{\Lambda' \subset \Lambda:finite\ subset}$は
		高々可算集合である.実際,$\Lambda_n \coloneqq \Lambda \times \cdots \times \Lambda\ 
		(\mbox{$n$ copies of $\Lambda$})$として
		$L \coloneqq \bigcup_{n=1}^{\# \Lambda} \Lambda_n$とおき,
		$(x_1,\cdots,x_n) \in L$に対し$\{x_1,\cdots,x_n\} \in L_0$を対応させる
		$f:L \longrightarrow L_0$を考えれば全射であるから
		$\card{L_0} \leq \card{L} \leq \aleph_0$が従う.
	},
	$\mathscr{B} \subset \prod_{\lambda \in \Lambda} \borel{S_\lambda}$が満たされるから
	\begin{align}
		\mathscr{O} \subset \bigotimes_{\lambda \in \Lambda} \borel{S_\lambda}
	\end{align}
	が従い(\refeq{eq:lem1_for_chap_1_exercise_1_8_1})を得る.
	\QED
\end{prf}
	
\subsection{範疇定理}
	\begin{screen}
		\begin{dfn}[疎集合・第一類集合・第二類集合]
			位相空間$S$の部分集合$A$が疎である(nowhere dense)とは
			$A$の閉包の内核が$\overline{A}^{\mathrm{o}} = \emptyset$を満たすことをいう.
			$S$が可算個の疎集合の合併で表せるとき$S$を第一類集合(the set of the first category)と呼び,
			そうでない場合はこれを第二類集合と呼ぶ.
		\end{dfn}
	\end{screen}
	
	\begin{screen}
		\begin{thm}[Baire]\label{thm:Baire_category_theorem}
			$S \neq \emptyset$が完備距離空間,或は局所コンパクトHausdorff空間なら
			$S$は第二類集合である.
		\end{thm}
	\end{screen}
	
	\begin{prf}\mbox{}
		\begin{description}
			\item[第一段]
				$(V_n)_{n=1}^\infty$を$S$で稠密な開集合系とするとき
				\begin{align}
					\overline{\bigcap_{n=1}^\infty V_n} = S,
					\label{eq:thm_Baire_category_theorem_1}
				\end{align}
				となることを示す.実際(\refeq{eq:thm_Baire_category_theorem_1})が満たされていれば,
				任意の疎集合系$(E_n)_{n=1}^\infty$に対して
				\begin{align}
					V_n \coloneqq \overline{E_n}^c,
					\quad n=1,2,\cdots
				\end{align}
				で開集合系$(V_n)$を定めると定理\ref{thm:topology_note_closure_interior}より
				\begin{align}
					\overline{V_n} = \overline{E_n}^{ca} = \overline{E_n}^{ic} = \emptyset^c = S
				\end{align}
				となるから,$\bigcap_{n=1}^\infty V_n \neq \emptyset$が従い
				$S \neq \bigcup_{n=1}^\infty \overline{E_n} \supset \bigcup_{n=1}^\infty E_n$
				が成り立つ.従って$S$は第二類である.
				
			\item[第二段]
				任意の空でない開集合$B_0$に対し$B_0 \cap \left( \bigcap_{n=1}^\infty V_n \right) \neq \emptyset$
				となることを示せば(\refeq{eq:thm_Baire_category_theorem_1})が従う.
				$V_1$の稠密性より或る点$x_1 \in B_0 \cap V_1$が存在し,
				定理\ref{thm:T_3_space}より次を満たす開集合$B_1$が取れる:
				\begin{align}
					x_1 \in \overline{B_1} \subset B_0 \cap V_1.
					\label{eq:thm_Baire_category_theorem_2}
				\end{align}
				このとき,$S$が距離空間なら$B_1$は半径$1$以下の開球,
				局所コンパクトHausdorff空間なら$\overline{B_1}$がコンパクトであるようにできる.
				繰り返して,半径$1/n$以下の開球,
				或は閉包がコンパクトな開集合$B_n$と$x_n \in S$が存在して
				\begin{align}
					x_n \in \overline{B_n} \subset B_{n-1} \cap V_n
				\end{align}
				を満たす.このとき$S$が完備距離空間なら$(x_n)_{n=1}^\infty$は
				Cauchy列をなし,その極限点$x_\infty$は
				\begin{align}
					x_\infty \in \bigcap_{n=1}^\infty \overline{B_n}
				\end{align}
				を満たす.$S$が局所コンパクトHausdorff空間なら定理\ref{thm:Cantor_intersection_theorem}より
				\begin{align}
					\bigcap_{n=1}^\infty \overline{B_n} \neq \emptyset
				\end{align}
				となるから,いずれの場合も
				\begin{align}
					\emptyset \neq \bigcap_{n=1}^\infty \overline{B_n} 
					\subset B_0 \cap \Biggl( \bigcap_{n=1}^\infty V_n \Biggr)
				\end{align}
				が従い定理の主張が得られる.
				\QED
		\end{description}
	\end{prf}
	
	\begin{screen}
		\begin{thm}[第一類集合の性質]
			$S$を位相空間とする.
			\begin{description}
				\item[(a)] $A \subset B \subset S$に対し$B$が第一類なら$A$も第一類である.
				\item[(b)] 第一類集合の可算和も第一類である.
				\item[(c)] 内核が空である閉集合は第一類である.
				\item[(d)] $S$から$S$への位相同型$h$と$E \subset S$に対し次が成り立つ:
					\begin{align}
						\mbox{$E$が第一類} \quad \Longleftrightarrow \quad
						\mbox{$h(E)$が第一類}.
					\end{align}
			\end{description}
		\end{thm}
	\end{screen}
	
	\begin{prf}\mbox{}
		\begin{description}
			\item[(a)] $B = \bigcup_{n=1}^\infty E_n$
				を満たす疎集合系$(E_n)_{n=1}^\infty$に対し
				$A \cap E_n$は疎であり$A = \bigcup_{n=1}^\infty (A \cap E_n)$となる.
			\item[(b)] $A_n \subset S,\ (n=1,2,\cdots)$が第一類集合とし
				$(E_{n,i})_{i=1}^\infty$を$A_n = \bigcup_{i=1}^\infty E_{n,i}$
				を満たす疎集合系とすれば
				\begin{align}
					\bigcup_{n=1}^\infty A_n
					= \bigcup_{n,i=1}^\infty E_{n,i}
				\end{align}
				が成り立つ.
				
			\item[(c)] 内核が空である閉集合はそれ自身が疎であり,自身の可算和に一致する.
			\item[(d)] $E$が第一類のとき,$E = \bigcup_{i=1}^\infty E_i$を満たす
				疎集合系$(E_i)_{i=1}^\infty$に対し定理\ref{thm:topology_note_closure_interior}より
				\begin{align}
					\emptyset = h(E_i^{ai})
					= h(E_i^a)^i
					= h(E_i)^{ai}
				\end{align}
				が成り立つから$h(E_i)$は疎であり,
				\begin{align}
					h(E) = \bigcup_{i=1}^\infty h(E_i)
				\end{align}
				となるから$h(E)$も第一類である.$h(E)$が第一類なら$E = h^{-1}(h(E))$も第一類である.
				\QED
		\end{description}
	\end{prf}
	
\subsection{有向点族}
\subsection{位相線型空間}
	
	位相線形空間$(X,\tau)$に対し,その部分集合$Y$上の相対位相を$\tau_Y$と書き,
	また$X$が或る距離$d$で距離付け可能なとき,
	$d$により導入する位相を$\tau_d$と書く.位相$\tau$に関する開集合,閉集合,近傍,
	Cauchy列は$\tau$-開集合(resp. 閉集合,近傍,Cauchy列)と書く.
	
	\begin{screen}
		\begin{thm}[部分空間が$F$-空間なら閉]
			$(X,\tau)$を位相線形空間,$Y \subset X$を部分空間とする.
			このとき$Y$が$F$-空間なら$Y$は$\tau$-閉である.
		\end{thm}
	\end{screen}
	
	\begin{prf}
		$Y$に対し或る平行移動不変な距離$d$が存在して$\tau_Y = \tau_d$を満たす.
		このとき
		\begin{align}
			B_{1/n} \coloneqq \Set{y \in Y}{d(y,0) < \frac{1}{n}},
			\quad n=1,2,\cdots
		\end{align}
		で$\tau_Y$-開集合を定めれば,$B_{1/n}$は$0$を含むから
		或る0の$\tau$-近傍$U_n$が存在して
		\begin{align}
			B_{1/n} = Y \cap U_n, \quad n=1,2,\cdots
		\end{align}
		を満たす.
	\end{prf}
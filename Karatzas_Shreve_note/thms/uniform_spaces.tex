\subsection{一様空間}
	一様空間は後述する距離空間や位相線型空間の上位概念である.
	距離空間では距離により,位相線型空間では$0$ベクトル周りの近傍を
	任意の点に移すことにより,空間全体で共通する点同士の`近さ'を規定することが出来る.
	一般の位相空間では点同士の`近さ'を相対的に比較することはできない
	(つまり点$x,y$の`近さ'と点$a,b$の`近さ'を比較する術がない)が,
	一様構造が導入された空間では各点に共通する近傍概念が定義されるため`近さ'の相対比較が可能になり,
	一様連続,一様収束,完備,全有界といった性質が定式化される.
	
	始めに次の集合演算を定義する.
	$S$を集合とするとき,任意の$V \subset S \times S$に対して,
	その反転$V^{-1}$を
	\begin{align}
		V^{-1} \coloneqq \Set{(y,x)}{(x,y) \in V}
	\end{align}
	により定め,また$S \times S$における二項演算$\circ$を
	\begin{align}
		U \circ V \coloneqq
		\Set{(x,z)}{\mbox{或る$y \in S$で$(x,y) \in U$かつ$(y,z) \in V$となる}},
		\quad (U,V \subset S \times S)
	\end{align}
	で定める.このとき演算$\circ$について次が成り立つ:
	$V,W \subset S \times S$を空でない部分集合とすれば
	\begin{align}
		W \circ W \subset V
		\quad \Longleftrightarrow \quad
		\mbox{任意の$x,y,z \in S$に対し$(x,y),(y,z) \in W$なら$(x,z) \in V$}.
	\end{align}
	
	\begin{screen}
		\begin{dfn}[近縁系]\label{dfn:uniform_structure}
			$S$を空でない集合とするとき,次の(US1)$\sim$(US5)を満たす
			$S \times S$の部分集合族$\mathscr{V}$を
			$S$の近縁系\index{きんえんけい@近縁系}(system of entourages)
			や一様構造\index{いちようこうぞう@一様構造}(uniform structure)と呼び,
			対$(S,\mathscr{V})$を一様空間\index{いちようくうかん@一様空間}
			(uniform space)と呼ぶ:
			\begin{description}
				\item[(US1)] $\mathscr{V} \neq \emptyset$かつ任意の$V \in \mathscr{V}$は
					$\Set{(x,x)}{x \in S} \subset V$を満たす.
					
				\item[(US2)] 任意の$V \subset S \times S$に対し
					$V \in \mathscr{V} \Longleftrightarrow V^{-1} \in \mathscr{V}$.
				\item[(US3)] 任意の$U,V \in \mathscr{V}$に対し$U \cap V \in \mathscr{V}$.
				\item[(US4)] 任意の$V \in \mathscr{V}$に対し或る$W \in \mathscr{V}$が存在して$W \circ W \subset V$.またこれは次と同値である:
					\begin{align}
						\forall V \in \mathscr{V};\ 
						\exists W \in \mathscr{V};\ 
						\forall x,y,z \in S;\quad
						(x,y),(y,z) \in W \Longrightarrow (x,z) \in V.
					\end{align}
					
				\item[(US5)] 任意の$V \subset \mathscr{V}$に対し
					$V \subset R$なら$R \in \mathscr{V}$.
			\end{description}
			$\mathscr{V}$の元を近縁\index{きんえん@近縁}(entourage)と呼び,
			近縁$V$が$V = V^{-1}$を満たすとき$V$は対称\index{たいしょう@対称}
			である(symmetric)という.また基本近傍系と同様に
			$\mathscr{V}$の部分集合$\mathscr{U}$で
			$\mathscr{V}$の任意の近縁に対しそれに含まれる$\mathscr{U}$の元が取れるとき,
			$\mathscr{U}$を$\mathscr{V}$の基本近縁系
			\index{きほんきんえんけい@基本近縁系}(fundamental system of entourages)と呼ぶ.
		\end{dfn}
	\end{screen}
	(US3)について,$V$に対し$W$を対称なものとして取ることができる.実際
	$U \in \mathscr{V}$が$U \circ U \subset V$を満たすとき,
	\begin{align}
		W \coloneqq U \cap U^{-1}
	\end{align}
	で$W \in \mathscr{V}$を定めれば,$W$は対称であり
	$W \circ W \subset U \circ U \subset V$となる.
	
	\begin{screen}
		\begin{thm}[近縁は対角線集合を覆う`ベルト'を持つ]\label{thm:uniform_structure}
			$(S,\mathscr{V})$を一様空間とするとき,
			任意の$V \in \mathscr{V}$に対し
			\begin{align}
				W_x \times W_x \subset V,\quad (\forall x \in S)
			\end{align}
			を満たす対称な$W \in \mathscr{V}$が存在する.
			ただし$W_x = \Set{y \in S}{(x,y) \in W}$である.
		\end{thm}
	\end{screen}
	
	\begin{prf}
		近縁系の定義より$U \circ U \subset V$を満たす
		$U \in \mathscr{V}$が存在する.ここで
		\begin{align}
			W \coloneqq U \cap U^{-1}
		\end{align}
		で対称な$W \in \mathscr{V}$を定めれば,任意の$x \in S$に対し
		\begin{align}
			y,z \in W_x \quad \Longrightarrow \quad
			(x,y),(x,z) \in W \quad \Longrightarrow \quad
			(y,x),(x,z) \in W \quad \Longrightarrow \quad
			(y,z) \in V
		\end{align}
		が成立し$W_x \times W_x \subset V$が得られる.
		\QED
	\end{prf}
	
	\begin{screen}
		\begin{thm}[近縁系で導入する位相]\label{thm:topology_induced_by_the_uniformity}
			$\mathscr{V}$を集合$S$の近縁系,$\mathscr{U}$を
			$\mathscr{V}$の基本近縁系とする.$V_x$を
			\begin{align}
				V_x \coloneqq \Set{y \in S}{(x,y) \in V},
				\quad (V \in \mathscr{V},\ x \in S)
			\end{align}
			で定義するとき,各$x \in S$で
			\begin{align}
				\mathscr{U}(x) \coloneqq \Set{U_x}{U \in \mathscr{U}}
			\end{align}
			とおけば$\{\mathscr{U}(x)\}_{x \in S}$は定理
			\ref{thm:a_local_base_restores_the_topology}
			の(LB1)(LB2)(LB3)を満たす.このとき$\{\mathscr{U}(x)\}_{x \in S}$が基本近傍系となる
			$S$の位相が唯一つ定まるが,別の基本近縁系を用いても同じ位相が定まる.
		\end{thm}
	\end{screen}
	
	\begin{prf}
		$\mathscr{U}$は空でないから$\mathscr{U}(x)$も空ではない.
		そして任意の$U \in \mathscr{U}$は$\Set{(x,x)}{x \in S}$を含むから
		$x \in U_x$となり(LB1)が満たされる.また任意の$U_x,V_x \in \mathscr{U}(x)$に対し
		或る$W \in \mathscr{U}$で$W \subset U \cap V$となるから,
		$W_x \subset U_x \cap V_x$が従い(LB2)も出る.
		任意の$U_x \in \mathscr{U}(x)$に対し
		定理\ref{thm:uniform_structure}より
		\begin{align}
			W_y \times W_y \subset U,\quad (\forall y \in S)
		\end{align}
		を満たす対称な$W \in \mathscr{V}$が存在する.
		$R \subset W$を満たす$R \in \mathscr{U}$を取れば
		\begin{align}
			y \in R_x \quad \Longrightarrow \quad
			y \in W_x \quad \Longrightarrow \quad
			(x,y) \in W_x \times W_x \subset U \quad \Longrightarrow \quad
			y \in U_x
		\end{align}
		となるから$R_x \subset U_x$が成り立ち,このとき任意の$y \in R_x$に対し
		\begin{align}
			z \in R_y \quad \Longrightarrow \quad
			(y,z) \in W = W^{-1} \quad \Longrightarrow \quad
			(x,z) \in W_y \times W_y \subset U \quad \Longrightarrow \quad
			z \in U_x
		\end{align}
		より$R_y \subset U_x$が満たされるから(LB3)も得られる.
		従って定理\ref{thm:local_base_defines_open_sets}と
		定理\ref{thm:a_local_base_restores_the_topology}より
		$\{\mathscr{U}(x)\}_{x \in S}$が基本近傍系となる$S$の位相
		$\tau_{\mathscr{U}}$が唯一つ定まる.
		いま,$\tilde{\mathscr{U}}$を$\mathscr{V}$の別の基本近縁系として
		\begin{align}
			\tilde{\mathscr{U}}(x) \coloneqq \Set{\tilde{U}_x}{\tilde{U} \in \tilde{\mathscr{U}}},
			\quad (\forall x \in S)
		\end{align}
		とおけば,$\left\{\tilde{\mathscr{U}}(x)\right\}_{x \in S}$は
		$(S,\tau_{\mathscr{U}})$における基本近傍系となる.
		実際,任意の$\tilde{U}_x \in \tilde{\mathscr{U}}(x)$に対し或る$U \in \mathscr{U}$で
		$U_x \subset \tilde{U}_x$となるから$\tilde{U}_x$は$x$の近傍であり,
		一方で任意の$V_x \in \mathscr{U}(x)$に対し
		或る$\tilde{V} \in \tilde{\mathscr{U}}$で
		$\tilde{V}_x \subset V_x$となるから
		$\tilde{\mathscr{U}}(x)$は$x$の基本近傍系をなしている.
		$\left\{\tilde{\mathscr{U}}(x)\right\}_{x \in S}$が基本近傍系となる位相
		は唯一つであるから$\tau_{\tilde{\mathscr{U}}} = \tau_{\mathscr{U}}$が成り立つ.
		\QED
	\end{prf}
	
	\begin{screen}
		\begin{dfn}[一様位相]
			$\mathscr{V}$を集合$S$の近縁系,$\mathscr{U}$を
			$\mathscr{V}$の基本近縁系とする.$U \in \mathscr{U}$と$x \in S$に対し$U_x$を
			\begin{align}
				U_x \coloneqq \Set{y \in S}{(x,y) \in U}
			\end{align}
			で定義するとき,定理\ref{thm:topology_induced_by_the_uniformity}より
			\begin{align}
				\mathscr{U}(x) \coloneqq \Set{U_x}{U \in \mathscr{U}}
			\end{align}
			を各点$x$の基本近傍系とする位相が定まるが,
			別の基本近縁系を取っても同じ位相が定まるので
			これを近縁系$\mathscr{V}$で導入する
			$S$の一様位相\index{いちよういそう@一様位相}(uniform topology)と呼ぶ.
			$S$が位相空間であるとき,$\mathscr{V}$で導入する位相と元の位相が一致することを
			$\mathscr{V}$と元の位相が両立\index{りょうりつ@両立}する(compatible)という.
		\end{dfn}
	\end{screen}
	
	\begin{screen}
		\begin{thm}[部分一様空間]
			$(S,\mathscr{V})$を一様空間とするとき,任意の空でない部分集合$A \subset S$に対し
			\begin{align}
				\mathscr{V}_A \coloneqq 
				\Set{(A \times A) \cap V}{V \in \mathscr{V}}
			\end{align}
			は$A$上の近縁系となる.また$S$に$\mathscr{V}$で位相を導入するとき,
			$A$上の相対位相と$\mathscr{V}_A$は両立する.
		\end{thm}
	\end{screen}
	
	\begin{prf}\mbox{}
		\begin{description}
			\item[第一段] $\mathscr{V}_A$が定義
				\ref{dfn:uniform_structure}の(US1)$\sim$(US5)を満たすことを示す.先ず
				$\mathscr{V} \neq \emptyset$より$\mathscr{V}_A \neq \emptyset$であり,
				\begin{align}
					V \in \mathscr{V} \quad \Longrightarrow \quad
					(a,a) \in V,\ (\forall a \in A) \quad \Longrightarrow \quad
					(a,a) \in (A \times A) \cap V,\ (\forall a \in A)
				\end{align}
				となるから(US1)が満たされる.また任意に$E \in \mathscr{V}_A$を取れば
				或る$V \in \mathscr{V}$で$E = (A \times A) \cap V$と表され,
				\begin{align}
					(x,y) \in E^{-1}
					\quad \Longleftrightarrow \quad
					(y,x) \in (A \times A) \cap V 
					\quad \Longleftrightarrow \quad
					(x,y) \in (A \times A) \cap V^{-1}
				\end{align}
				が成り立つから$E^{-1} \in \mathscr{V}_A$が従い(US2)も満たされる.
				任意の$U,V \in \mathscr{V}$に対し
				\begin{align}
					((A \times A) \cap U) \cap ((A \times A) \cap V)
					= (A \times A) \cap (U \cap V) \in \mathscr{V}_A
				\end{align}
				より(US3)が得られ,また$V \in \mathscr{V}$に対し
				$W \circ W \subset V$となる$W \in \mathscr{V}$を取れば
				\begin{align}
					(x,y),(y,z) \in (A \times A) \cap W
					\quad \Longrightarrow \quad
					x,z \in A,\ (x,z) \in V
					\quad \Longrightarrow \quad
					(x,z) \in (A \times A) \cap V
				\end{align}
				となるから(US4)が出る.
				$(A \times A) \cap V \subset R,\ (V \in \mathscr{V})$を満たす任意の
				$R \subset A \times A$に対し,
				$V \cup R \in \mathscr{V}$より
				\begin{align}
					R = (A \times A) \cap (V \cup R) \in \mathscr{V}_A
				\end{align}
				が成立し(US5)も従う.
			
			\item[第二段] $\mathscr{V}_A$で導入する$A$の位相を
				$\tau_A$と書く.任意の$a \in A$と$V \in \mathscr{V}$に対して
				\begin{align}
					[(A \times A) \cap V]_a \coloneqq
					&\Set{x \in A}{(a,x) \in (A \times A) \cap V} \\
					=& \Set{x \in S}{(a,x) \in V} \cap A
					\eqqcolon V_a \cap A
				\end{align}
				となる.$\Set{[(A \times A) \cap V]_a}{V \in \mathscr{V}}$
				は$\tau_A$における$a$の基本近傍系をなし,
				$\Set{V_a \cap A}{V \in \mathscr{V}}$
				は$A$の相対位相における$a$の基本近傍系をなすが,
				両者が一致するので位相も一致する.
				\QED
		\end{description}
	\end{prf}
	
	\begin{screen}
		\begin{thm}[一様位相空間において$T_0 \Longleftrightarrow T_2$]
		\label{thm:T_0_iff_T_2_on_uniform_topological_space}
			$(S,\mathscr{V})$を一様空間とし,$S$に一様位相を導入する.このとき
			\begin{align}
				\mbox{$S$が$T_0$} \quad \Longleftrightarrow \quad
				\bigcap_{V \in \mathscr{V}}V = \Set{(x,x)}{x \in S}
				\quad \Longleftrightarrow \quad
				\mbox{$S$が$T_2$}
				\label{eq:thm_T_0_iff_T_2_on_uniform_topological_space}
			\end{align}
		\end{thm}
	\end{screen}
	
	\begin{prf} 位相空間が$T_2$なら$T_0$であるから,二つの$\Longrightarrow$を示せば
		(\refeq{eq:thm_T_0_iff_T_2_on_uniform_topological_space})が従う.
		\begin{description}
			\item[一つ目の$\Longrightarrow$]
				$\bigcap_{V \in \mathscr{V}}V \neq \Set{(x,x)}{x \in S}$
				が満たされるとき,或る相異なる二点$x,y \in S$に対し
				\begin{align}
					(x,y),(y,x) \in V, \quad (\forall V \in \mathscr{V})
				\end{align}
				となる.$\Set{V_x \coloneqq \Set{s \in S}{(x,s) \in V}}{V \in \mathscr{V}}$は$x$の基本近傍系をなすから
				\begin{align}
					y \in V_x, \quad (\forall V \in \mathscr{V})
				\end{align}
				が成立し,定理\ref{thm:belongs_to_closure_iff_clusters}より
				$x \in \overline{\{y\}}$が従う.
				対称的に$y \in \overline{\{x\}}$も出るから
				$x$と$y$は位相的に区別不能である.
				
			\item[二つ目の$\Longrightarrow$]
				$\bigcap_{V \in \mathscr{V}}V = \Set{(x,x)}{x \in S}$
				が満たされるとき,任意の相異なる二点$x,y \in S$に対し
				\begin{align}
					(x,y) \in V
				\end{align}
				を満たす$V \in \mathscr{V}$が存在する.
				定理\ref{thm:uniform_structure}より或る対称な$W \in \mathscr{V}$で
				\begin{align}
					W \circ W \subset V,
					\quad W_x \times W_x \subset V,
					\quad W_y \times W_y \subset V
				\end{align}
				となるが,このとき$W_x \cap W_y = \emptyset$が成り立つ.実際,
				$W_x \cap W_y$が空でないとき,$z \in W_x \cap W_y$を取れば
				\begin{align}
					(x,z),(y,z) \in W \quad \Longrightarrow \quad
					(x,z),(z,y) \in W \quad \Longrightarrow \quad
					(x,y) \in V
				\end{align}
				が従い矛盾が生じる.$W_x,W_y$はそれぞれ$x,y$の近傍であるから二つ目の$\Longrightarrow$を得る.
				\QED
		\end{description}
	\end{prf}
	
	\begin{screen}
		\begin{dfn}[一様連続性]
			$(S,\mathscr{U})$と$(T,\mathscr{V})$を一様空間として
			$\mathscr{U},\mathscr{V}$により$S,T$に一様位相を導入し,
			$f:S \longrightarrow T$を連続写像とする.
			任意の$V \in \mathscr{V}$に対し或る$U \in \mathscr{U}$が存在して
			\begin{align}
				(x,y) \in U \quad \Longrightarrow \quad (f(x),f(y)) \in V
			\end{align}
			となるとき,$f$は一様連続\index{いちようれんぞく@一様連続}である(uniformly continuous)という.
		\end{dfn}
	\end{screen}
	
	\begin{screen}
		\begin{thm}[コンパクト集合上で連続写像は一様連続]
			$(S,\mathscr{U})$と$(T,\mathscr{V})$を一様空間として
			$\mathscr{U},\mathscr{V}$により$S,T$に一様位相を導入し,
			$f:S \longrightarrow T$を連続写像とする.
			$A \subset S$をコンパクト部分集合とするとき,
			$f$は$A$上で一様連続となる.つまり,
			任意の$V \in \mathscr{V}$に対し或る$U \in \mathscr{U}$が存在して
			\begin{align}
				(x,y) \in U \cap A \quad \Longrightarrow \quad (f(x),f(y)) \in V.
			\end{align}
		\end{thm}
	\end{screen}
	
	\begin{prf}
		任意の$M \in \mathscr{U}, s \in S$に対し
		\begin{align}
			M_s \coloneqq \Set{x \in S}{(s,x) \in M}
		\end{align}
		と定め,$W \in \mathscr{V},\ t \in T$に対しても同様に$W_t$を定める.
		任意に$V \in \mathscr{V}$を取れば,定理\ref{thm:uniform_structure}より
		或る$W \in \mathscr{V}$で
		\begin{align}
			W_t \times W_t \subset V,
			\quad (\forall t \in T)
		\end{align}
		となる.$f$は連続であるから任意の$s \in S$に対し或る$N(s) \in \mathscr{U}$が存在して
		\begin{align}
			(s,x) \in N(s) \quad \Longrightarrow \quad
			f(x) \in W_{f(s)}
		\end{align}
		が成り立ち,$M(s) \circ M(s) \subset N(s)$を満たす対称な$M(s) \in \mathscr{U}$を取れば,
		定理\ref{thm:subset_is_compact_iff_every_original_open_cover_contains_finite_subcover}より
		或る$a_1,\cdots,a_n \in A$で
		\begin{align}
			A \subset \bigcup_{i=1}^n M(a_i)_{a_i}
		\end{align}
		となる.近縁系は有限交叉で閉じるから
		\begin{align}
			U \coloneqq \bigcap_{i=1}^n M(a_i)
		\end{align}
		は$\mathscr{U}$の元であり,このとき任意に$(x,y) \in U \cap A$を取れば,
		或る$i$で$y \in M(a_i)_{a_i}$となり,
		\begin{align}
			(a_i,a_i),(a_i,y) \in M(a_i) \quad \Longrightarrow \quad
			(a_i,y) \in N(a_i)
		\end{align}
		及び$M(a_i)$の対称性から
		\begin{align}
			(a_i,y),(y,x) \in M(a_i) \quad \Longrightarrow \quad
			(a_i,x) \in N(a_i)
		\end{align}
		が満たされ,$f(x),f(y) \in W_{f(a_i)}$が従うから
		$(f(x),f(y)) \in V$が成立し$f$の$A$の上での一様連続性が出る.
		\QED
	\end{prf}
	
	\begin{screen}
		\begin{thm}[擬距離空間の一様構造]
		\label{thm:uniform_structure_on_pseudometric_spaces}
			$(S,d)$を擬距離空間とするとき,
			\begin{align}
				\mathscr{V} \coloneqq
				\Set{V(r)}{r > 0},
				\quad (V(r) \coloneqq \Set{(x,y) \in S \times S}{d(x,y) < r})
			\end{align}
			とおけば$\mathscr{V}$は$S$上の一様構造となり,
			$\mathscr{V}$で導入する一様位相は$d$-位相に一致する.
		\end{thm}
	\end{screen}
	
	\begin{screen}
		\begin{thm}[擬距離空間のCauchy列]
		\label{thm:Cauchy_sequences_on_pseudometric_spaces}
			$(S,d)$を擬距離空間とし,一様構造$\mathscr{V}$を
			定理\ref{thm:uniform_structure_on_pseudometric_spaces}の要領で定めるとき,
			$S$の任意の点列$(x_n)_{n \in \N}$に対し,$(x_n)_{n \in \N}$がCauchy列であることと
			\begin{align}
				\forall \epsilon > 0;\ 
				\exists N \in \N;\quad
				n,m \geq N \Longrightarrow d(x_n,x_m) < \epsilon
				\label{eq:thm_Cauchy_sequences_on_pseudometric_spaces}
			\end{align}
			が成り立つことは同値になる.
		\end{thm}
	\end{screen}
	
	\begin{prf}
		任意の$\epsilon$と$n,m \in \N$で
		\begin{align}
			(x_n,x_m) \in V(\epsilon) \quad \Longleftrightarrow \quad
			d(x_n,x_m) < \epsilon
		\end{align}
		となるから,$(x_n)_{n \in \N}$がCauchy列であるとき,任意の$\epsilon > 0$に対し
		或る$N \in \N$が存在して
		\begin{align}
			n,m \geq N \quad \Longrightarrow \quad
			(x_n,x_m) \in V(\epsilon) \quad \Longrightarrow \quad
			d(x_n,x_m) < \epsilon
		\end{align}
		が成り立つ.逆に$(x_n)_{n \in \N}$に対して
		(\refeq{eq:thm_Cauchy_sequences_on_pseudometric_spaces})が
		満たされているとき,任意の$V(\epsilon) \in \mathscr{V}$に対し
		或る$M \in \N$が存在して
		\begin{align}
			n,m \geq N \quad \Longrightarrow \quad
			d(x_n,x_m) < \epsilon \quad \Longrightarrow \quad
			(x_n,x_m) \in V(\epsilon)
		\end{align}
		となるから$(x_n)_{n \in \N}$はCauchy列である.
		\QED
	\end{prf}
	
	\begin{screen}
		\begin{thm}[点列の擬距離に関する収束]
			点列$(x_n)_{n \in \N}$が$a$に収束する
			ことと$d(x_n,a) \longrightarrow 0$は同値.
		\end{thm}
	\end{screen}
	
	\begin{screen}
		\begin{thm}[可算な基本近縁系が存在するとき,完備$\Longleftrightarrow$任意のCauchy列が収束する]
		\label{thm:complete_iff_every_Cauchy_seq_converges_if_entourage_contains_some_countable_subset}
			$(S,\mathscr{V})$を一様空間とする.
			$\mathscr{V}$に対して可算な基本近縁系$\{V_n\}_{n \in \N}$が存在するとき次が成立する:
			\begin{align}
				\mbox{$S$が完備である} \quad \Longleftrightarrow \quad
				\mbox{$S$の任意のCauchy列が収束する}.
			\end{align}
		\end{thm}
	\end{screen}
	
	\begin{prf}
		$\Longleftarrow$を示す.近縁系は有限交叉で閉じるから
		\begin{align}
			U_n \coloneqq V_1 \cap V_2 \cap \cdots \cap V_n,
			\quad (n = 1,2,\cdots)
		\end{align}
		により単調減少な$\mathscr{V}$の基本近縁系$\{U_n\}_{n \in \N}$が定まる.
		$(x_\lambda)_{\lambda \in \Lambda}$を$S$のCauchy有向点族として
		\begin{align}
			A_\lambda \coloneqq \Set{x_\mu}{\lambda \leq \mu},
			\quad (\forall \lambda \in \Lambda)
		\end{align}
		とおけば,任意の$n \in \N$で或る$\lambda_n \in \Lambda$が存在して
		\begin{align}
			A_{\lambda_n} \times A_{\lambda_n} \subset U_n
		\end{align}
		となる.任意の$V \in \mathscr{V}$に対し$W \circ W \subset V$
		を満たす$W \in \mathscr{V}$を取れば,或る$N \in \N$で$U_N \subset W$となるから
		\begin{align}
			U_N \circ U_N \subset V
		\end{align}
		が成り立つ.また任意の$n,m \geq N$に対し,有向集合の定義より
		$\lambda_n,\lambda_m \leq \mu$を満たす$\mu \in \Lambda$が存在して
		\begin{align}
			(x_{\lambda_n},x_\mu) \in U_n \subset U_N,
			\quad (x_\mu, x_{\lambda_m}) \in U_m \subset U_N
		\end{align}
		となり$(x_{\lambda_n},x_{\lambda_m}) \in V$が従うから,
		$(x_{\lambda_n})_{n \in \N}$はCauchy列であり或る$a \in S$に収束する.このとき
		\begin{align}
			\lim x_\lambda = a
			\label{eq:thm_complete_iff_every_Cauchy_seq_converges_if_entourage_contains_some_countable_subset}
		\end{align}
		が成立する.実際,任意に$a$の近傍$B$を取れば或る$\tilde{V} \in \mathscr{V}$で
		\begin{align}
			\tilde{V}_a \coloneqq \Set{x \in S}{(a,x) \in \tilde{V}} \subset B
		\end{align}
		となり,$\tilde{W} \circ \tilde{W} \subset V$を満たす$\tilde{W} \in \mathscr{V}$に対し
		或る$N_1 \in \N$が存在して
		\begin{align}
			n \geq N_1 \quad \Longrightarrow \quad
			x_{\lambda_n} \in \tilde{W}_a \quad \Longrightarrow \quad
			(a,x_{\lambda_n}) \in \tilde{W}
		\end{align}
		を満たす.また或る$N_2 \geq N_1$で$U_{N_2} \subset \tilde{W}$となるから
		\begin{align}
			A_{\lambda_{N_2}} \times A_{\lambda_{N_2}} \subset U_{N_2} \subset \tilde{W}
		\end{align}
		が従い,このとき$(a,x_{\lambda_{N_2}}) \in \tilde{W}$かつ
		$(x_{\lambda_{N_2}},x) \in \tilde{W},\ (\forall x \in A_{\lambda_{N_2}})$より
		$(a,x) \in \tilde{V},\ (\forall x \in A_{\lambda_{N_2}})$となるから
		\begin{align}
			A_{\lambda_{N_2}} \subset \tilde{V}_a 
		\end{align}
		が得られ(
		\refeq{eq:thm_complete_iff_every_Cauchy_seq_converges_if_entourage_contains_some_countable_subset})
		が出る.任意のCauchy有向点族が収束するから$S$は完備である.
		\QED
	\end{prf}
	
	\begin{screen}
		\begin{thm}[完備かつ全有界$\Longleftrightarrow$コンパクト]
			$(S,\mathscr{V})$を一様空間として$\mathscr{V}$で一様位相を導入するとき,
			\begin{align}
				\mbox{$S$が完備かつ全有界} \quad \Longleftrightarrow \quad
				\mbox{$S$がコンパクト}.
			\end{align}
		\end{thm}
	\end{screen}
	
	\begin{prf}\mbox{}
		\begin{description}
			\item[第一段]
				任意の有向点族が収束する部分有向点族を持てばコンパクトである.
				$(x_\lambda)_{\lambda \in \Lambda}$を$S$の有向点族とする.
				任意の$V \in \mathscr{V}$に対し或る$\{A_i\}_{i=1}^n$が存在して
				\begin{align}
					A_i \times A_i \subset U;\ (\forall i=1,\cdots,n),\quad
					\bigcup_{i=1}^n A_i = S
				\end{align}
				を満たす.この$\{A_i\}_{i=1}^n$が生成する$S$の位相を$\tau_V$とし,
				$\tau_V$を導入した$S$を$S_V$と書けば,$S_V$はコンパクトであるからTyconovの定理より
				\begin{align}
					T \coloneqq \prod_{V \in \mathscr{V}} S_V
				\end{align}
				はコンパクト空間であり,
			\item[第二段]
		\end{description}
	\end{prf}
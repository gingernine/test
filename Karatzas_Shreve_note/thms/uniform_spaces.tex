\subsection{一様空間}
	\begin{screen}
		\begin{thm}
			$(S,\mathscr{V})$を一様空間とするとき,
			任意の$V \in \mathscr{V}$に対し
			\begin{align}
				W_x \times W_x \subset V,\quad (\forall x \in S)
			\end{align}
			を満たす対称な$W \in \mathscr{V}$が存在する.
			ただし$W_x = \Set{y \in S}{(x,y) \in W}$である.
		\end{thm}
	\end{screen}
	
	\begin{prf}
		近縁系の定義より$U \circ U \subset V$を満たす
		$U \in \mathscr{V}$が存在する.ここで
		\begin{align}
			W \coloneqq U \cap U^{-1}
		\end{align}
		で$W \in \mathscr{V}$を定めれば,$W$は対称であるので任意の$x \in S$に対し
		\begin{align}
			y,z \in W_x \quad \Longrightarrow \quad
			(x,y),(x,z) \in W \quad \Longrightarrow \quad
			(y,x),(x,z) \in W \quad \Longrightarrow \quad
			(y,z) \in V
		\end{align}
		が成立し$W_x \times W_x \subset V$が得られる.
		\QED
	\end{prf}
	
	\begin{screen}
		\begin{thm}[一様位相空間において$T_0 \Longleftrightarrow T_2$]
		\label{thm:T_0_iff_T_2_on_uniform_topological_space}
			$(S,\mathscr{V})$を一様空間とし,$S$に一様位相を導入する.このとき
			\begin{align}
				\mbox{$S$が$T_0$} \quad \Longleftrightarrow \quad
				\bigcap_{V \in \mathscr{V}}V = \Set{(x,x)}{x \in S}
				\quad \Longleftrightarrow \quad
				\mbox{$S$が$T_2$}
				\label{eq:thm_T_0_iff_T_2_on_uniform_topological_space}
			\end{align}
		\end{thm}
	\end{screen}
	
	\begin{prf}\mbox{}
		\begin{description}
			\item[一つ目の$\Longrightarrow$]
				$\bigcap_{V \in \mathscr{V}}V \neq \Set{(x,x)}{x \in S}$
				が満たされるとき,或る相異なる二点$x,y \in S$に対し
				\begin{align}
					(x,y),(y,x) \in V, \quad (\forall V \in \mathscr{V})
				\end{align}
				となる.$\Set{V_x \coloneqq \Set{s \in S}{(x,s) \in V}}{V \in \mathscr{V}}$は$x$の基本近傍系をなすから
				\begin{align}
					y \in V_x, \quad (\forall V \in \mathscr{V})
				\end{align}
				が成立し,定理\ref{thm:belongs_to_closure_iff_clusters}より
				$x \in \overline{\{y\}}$が従う.
				対称的に$y \in \overline{\{x\}}$も出るから
				$x$と$y$は位相的に区別不能である.
				
			\item[二つ目の$\Longrightarrow$]
				$\bigcap_{V \in \mathscr{V}}V = \Set{(x,x)}{x \in S}$
				が満たされるとき,任意の相異なる二点$x,y \in S$に対し
				\begin{align}
					(x,y) \in V
				\end{align}
				を満たす$V \in \mathscr{V}$が存在する.
				定理より或る対称な$W \in \mathscr{V}$で
				\begin{align}
					W \circ W \subset V,
					\quad W_x \times W_x \subset V,
					\quad W_y \times W_y \subset V
				\end{align}
				となるが,このとき$W_x \cap W_y = \emptyset$が成り立つ.実際,
				$W_x \cap W_y$が空でないとき,$z \in W_x \cap W_y$を取れば
				\begin{align}
					(x,z),(y,z) \in W \quad \Longrightarrow \quad
					(x,z),(z,y) \in W \quad \Longrightarrow \quad
					(x,y) \in V
				\end{align}
				が従い矛盾が生じる.$W_x,W_y$はそれぞれ$x,y$の近傍であるから,
				二つ目の$\Longrightarrow$が得られた.
				位相空間が$T_2$なら$T_0$であるから
				(\refeq{eq:thm_T_0_iff_T_2_on_uniform_topological_space})が成り立つ.
				\QED
		\end{description}
	\end{prf}
\subsection{一様空間}
	一様空間は後述する距離空間や位相線型空間の上位概念である.
	距離空間では距離により,位相線型空間では$0$ベクトル周りの近傍を
	任意の点に移すことにより,空間全体で共通する点同士の`近さ'の尺度が与えられる.
	一般の位相空間では点同士の`近さ'を相対的に比較することはできない
	(つまり点$x,y$の`近さ'と点$a,b$の`近さ'を比較する基準がない)が,
	一様構造が導入された空間では各点に共通する近傍概念が定義されるため`近さ'の相対比較が可能になり,
	一様連続,一様収束,完備,全有界といった性質が定式化される.
	
	始めに次の集合演算を定義する.
	$S$を集合とするとき,任意の$V \subset S \times S$に対して,
	その反転$V^{-1}$を
	\begin{align}
		V^{-1} \coloneqq \Set{(y,x)}{(x,y) \in V}
	\end{align}
	により定め,また$S \times S$における二項演算$\circ$を
	\begin{align}
		U \circ V \coloneqq
		\Set{(x,z)}{\mbox{或る$y \in S$で$(x,y) \in U$かつ$(y,z) \in V$となる}},
		\quad (U,V \subset S \times S)
	\end{align}
	で定める.このとき演算$\circ$について次が成り立つ:
	$V,W \subset S \times S$を空でない部分集合とすれば
	\begin{align}
		W \circ W \subset V
		\quad \Longleftrightarrow \quad
		\mbox{任意の$x,y,z \in S$に対し$(x,y),(y,z) \in W$なら$(x,z) \in V$}.
	\end{align}
	
	\begin{screen}
		\begin{dfn}[近縁系]\label{dfn:uniform_structure}
			$S$を空でない集合とする.次の(US1)$\sim$(US5)を満たす
			$S \times S$の部分集合族$\mathscr{V}$を
			$S$の{\bf 近縁系}\index{きんえんけい@近縁系}{\bf (system of entourages)}
			や{\bf 一様構造}\index{いちようこうぞう@一様構造}{\bf (uniform structure)}と呼び,
			対$(S,\mathscr{V})$を{\bf 一様空間}\index{いちようくうかん@一様空間}
			{\bf (uniform space)}と呼ぶ:
			\begin{description}
				\item[(US1)] $\mathscr{V} \neq \emptyset$かつ任意の$V \in \mathscr{V}$は
					$\Set{(x,x)}{x \in S} \subset V$を満たす.
					
				\item[(US2)] 任意の$V \subset S \times S$に対し
					$V \in \mathscr{V} \Longleftrightarrow V^{-1} \in \mathscr{V}$.
				\item[(US3)] 任意の$U,V \in \mathscr{V}$に対し$U \cap V \in \mathscr{V}$.
				\item[(US4)] 任意の$V \in \mathscr{V}$に対し或る$W \in \mathscr{V}$が存在して$W \circ W \subset V$.またこれは次と同値である:
					\begin{align}
						\forall V \in \mathscr{V};\ 
						\exists W \in \mathscr{V};\ 
						\forall x,y,z \in S;\quad
						(x,y),(y,z) \in W \Longrightarrow (x,z) \in V.
					\end{align}
					
				\item[(US5)] 任意の$V \subset \mathscr{V}$に対し
					$V \subset R$なら$R \in \mathscr{V}$.
			\end{description}
			$\mathscr{V}$の元を{\bf 近縁}\index{きんえん@近縁}{\bf (entourage)}と呼び,
			近縁$V$が$V = V^{-1}$を満たすとき$V$は{\bf 対称}\index{たいしょう@対称}
			である{\bf (symmetric)}という.また基本近傍系と同様に
			$\mathscr{V}$の部分集合$\mathscr{U}$で
			$\mathscr{V}$の任意の近縁に対しそれに含まれる$\mathscr{U}$の元が取れるとき,
			$\mathscr{U}$を$\mathscr{V}$の{\bf 基本近縁系}
			\index{きほんきんえんけい@基本近縁系}
			{\bf (fundamental system of entourages)}と呼ぶ.
		\end{dfn}
	\end{screen}
	(US3)について,$V$に対し$W$を対称なものとして取ることができる.実際
	$U \in \mathscr{V}$が$U \circ U \subset V$を満たすとき,
	\begin{align}
		W \coloneqq U \cap U^{-1}
	\end{align}
	で$W \in \mathscr{V}$を定めれば,$W$は対称であり
	$W \circ W \subset U \circ U \subset V$となる.
	
	\begin{screen}
		\begin{thm}[近縁系は$\circ$により半群となる]
		\label{thm:uniform_structure_is_a_semigroup}
			二項演算$\circ$は結合法則を満たし,また近縁系の中で閉じる
			($U,V$が近縁なら$U \circ V$も近縁).
		\end{thm}
	\end{screen}
	
	\begin{prf}
		$S$を空でない集合,$\mathscr{V}$を$S$の近縁系とする.
		任意の$U,V,W \subset S \times S$で
		\begin{align}
			(a,b) \in (U \circ V) \circ W
			&\quad \Longleftrightarrow \quad
			\exists c \in S;\ (a,c) \in U \circ V,\ (c,b) \in W \\
			&\quad \Longleftrightarrow \quad
			\exists c,d \in S;\ (a,d) \in U,\ (d,c) \in V,\ (c,b) \in W \\
			&\quad \Longleftrightarrow \quad
			\exists d \in S;\ (a,d) \in U,\ (d,b) \in U \circ W \\
			&\quad \Longleftrightarrow \quad
			(a,b) \in U \circ (V \circ W)
		\end{align}
		となるから$\circ$は結合法測を満たす.また任意の$U,V \in \mathscr{V}$に対し
		\begin{align}
			(x,y) \in U \quad \Longrightarrow \quad
			(x,y) \in U,\ (y,y) \in V \quad \Longrightarrow \quad
			(x,y) \in U \circ V
		\end{align}
		より$U \subset U \circ V$となるから$U \circ V \in \mathscr{V}$が成り立つ.
		\QED
	\end{prf}
	
	従って$U_1,U_2,\cdots,U_n \subset S \times S$に対し
	\begin{align}
		U_1 \circ U_2 \circ \cdots \circ U_n
		\coloneqq (\cdots((U_1 \circ U_2) \circ U_3) \cdots ) \circ U_n
	\end{align}
	と定めれば,定理\ref{thm:generalized_associative_law_on_semigroup}より
	左辺の評価は演算$\circ$の順番に依らないから,
	任意に$1 \leq k \leq m \leq n$を取れば
	\begin{align}
		U_1 \circ U_2 \circ \cdots \circ U_n
		= (U_1 \circ \cdots \circ U_k) 
		\circ (U_{k+1} \circ \cdots \circ U_m)
		\circ (U_{m+1} \circ \cdots \circ U_n)
	\end{align}
	が成立する.
	
	\begin{screen}
		\begin{thm}\label{thm:uniform_structure}
			$(S,\mathscr{V})$を一様空間とするとき,
			任意の$V \in \mathscr{V}$に対し
			\begin{align}
				W_x \times W_x \subset V,\quad (\forall x \in S)
			\end{align}
			を満たす対称な$W \in \mathscr{V}$が存在する.
			ただし$W_x = \Set{y \in S}{(x,y) \in W}$である.
		\end{thm}
	\end{screen}
	
	\begin{prf}
		近縁系の定義より$U \circ U \subset V$を満たす
		$U \in \mathscr{V}$が存在する.ここで
		\begin{align}
			W \coloneqq U \cap U^{-1}
		\end{align}
		で対称な$W \in \mathscr{V}$を定めれば,任意の$x \in S$に対し
		\begin{align}
			y,z \in W_x \quad \Longrightarrow \quad
			(x,y),(x,z) \in W \quad \Longrightarrow \quad
			(y,x),(x,z) \in W \quad \Longrightarrow \quad
			(y,z) \in V
		\end{align}
		が成立し$W_x \times W_x \subset V$が得られる.
		\QED
	\end{prf}
	
	\begin{screen}
		\begin{thm}[近縁系で導入する位相]\label{thm:topology_induced_by_the_uniformity}
			$\mathscr{V}$を集合$S$の近縁系,$\mathscr{U}$を
			$\mathscr{V}$の基本近縁系とする.$V_x$を
			\begin{align}
				V_x \coloneqq \Set{y \in S}{(x,y) \in V},
				\quad (V \in \mathscr{V},\ x \in S)
			\end{align}
			で定義するとき,各$x \in S$で
			\begin{align}
				\mathscr{U}(x) \coloneqq \Set{U_x}{U \in \mathscr{U}}
			\end{align}
			とおけば$\{\mathscr{U}(x)\}_{x \in S}$は定理
			\ref{thm:a_local_base_restores_the_topology}
			の(LB1)(LB2)(LB3)を満たす.このとき$\{\mathscr{U}(x)\}_{x \in S}$が基本近傍系となる
			$S$の位相が唯一つ定まるが,別の基本近縁系を用いても同じ位相が定まる.
		\end{thm}
	\end{screen}
	
	\begin{prf}
		$\mathscr{U}$は空でないから$\mathscr{U}(x)$も空ではない.
		そして任意の$U \in \mathscr{U}$は$\Set{(x,x)}{x \in S}$を含むから
		$x \in U_x$となり(LB1)が満たされる.また任意の$U_x,V_x \in \mathscr{U}(x)$に対し
		或る$W \in \mathscr{U}$で$W \subset U \cap V$となるから,
		$W_x \subset U_x \cap V_x$が従い(LB2)も出る.
		任意の$U_x \in \mathscr{U}(x)$に対し
		定理\ref{thm:uniform_structure}より
		\begin{align}
			W_y \times W_y \subset U,\quad (\forall y \in S)
		\end{align}
		を満たす対称な$W \in \mathscr{V}$が存在する.
		$R \subset W$を満たす$R \in \mathscr{U}$を取れば
		\begin{align}
			y \in R_x \quad \Longrightarrow \quad
			y \in W_x \quad \Longrightarrow \quad
			(x,y) \in W_x \times W_x \subset U \quad \Longrightarrow \quad
			y \in U_x
		\end{align}
		となるから$R_x \subset U_x$が成り立ち,このとき任意の$y \in R_x$に対し
		\begin{align}
			z \in R_y \quad \Longrightarrow \quad
			(y,z) \in W = W^{-1} \quad \Longrightarrow \quad
			(x,z) \in W_y \times W_y \subset U \quad \Longrightarrow \quad
			z \in U_x
		\end{align}
		より$R_y \subset U_x$が満たされるから(LB3)も得られる.
		従って定理\ref{thm:local_base_defines_open_sets}と
		定理\ref{thm:a_local_base_restores_the_topology}より
		$\{\mathscr{U}(x)\}_{x \in S}$が基本近傍系となる$S$の位相
		$\tau_{\mathscr{U}}$が唯一つ定まる.
		いま,$\tilde{\mathscr{U}}$を$\mathscr{V}$の別の基本近縁系として
		\begin{align}
			\tilde{\mathscr{U}}(x) \coloneqq \Set{\tilde{U}_x}{\tilde{U} \in \tilde{\mathscr{U}}},
			\quad (\forall x \in S)
		\end{align}
		とおけば,$\left\{\tilde{\mathscr{U}}(x)\right\}_{x \in S}$は
		$(S,\tau_{\mathscr{U}})$における基本近傍系となる.
		実際,任意の$\tilde{U}_x \in \tilde{\mathscr{U}}(x)$に対し或る$U \in \mathscr{U}$で
		$U_x \subset \tilde{U}_x$となるから$\tilde{U}_x$は$x$の近傍であり,
		一方で任意の$V_x \in \mathscr{U}(x)$に対し
		或る$\tilde{V} \in \tilde{\mathscr{U}}$で
		$\tilde{V}_x \subset V_x$となるから
		$\tilde{\mathscr{U}}(x)$は$x$の基本近傍系をなしている.
		$\left\{\tilde{\mathscr{U}}(x)\right\}_{x \in S}$が基本近傍系となる位相
		は唯一つであるから$\tau_{\tilde{\mathscr{U}}} = \tau_{\mathscr{U}}$が成り立つ.
		\QED
	\end{prf}
	
	\begin{screen}
		\begin{dfn}[一様位相]
			$\mathscr{V}$を集合$S$の近縁系,$\mathscr{U}$を
			$\mathscr{V}$の基本近縁系とする.$U \in \mathscr{U}$と$x \in S$に対し$U_x$を
			\begin{align}
				U_x \coloneqq \Set{y \in S}{(x,y) \in U}
			\end{align}
			で定義するとき,定理\ref{thm:topology_induced_by_the_uniformity}より
			\begin{align}
				\mathscr{U}(x) \coloneqq \Set{U_x}{U \in \mathscr{U}}
			\end{align}
			を各点$x$の基本近傍系とする位相が定まるが,
			別の基本近縁系を取っても同じ位相が定まるので
			これを`近縁系$\mathscr{V}$で導入する
			$S$の{\bf 一様位相}\index{いちよういそう@一様位相}
			{\bf (uniform topology)}'と呼ぶ.$S$が元から位相空間であるとき,
			$\mathscr{V}$で導入する位相と元の位相が一致することを
			`$\mathscr{V}$と元の位相が{\bf 両立}\index{りょうりつ@両立}する
			{\bf (compatible)}'という.
		\end{dfn}
	\end{screen}
	
	\begin{screen}
		\begin{thm}[部分一様空間]
			$(S,\mathscr{V})$を一様空間とするとき,任意の空でない部分集合$A \subset S$に対し
			\begin{align}
				\mathscr{V}_A \coloneqq 
				\Set{(A \times A) \cap V}{V \in \mathscr{V}}
			\end{align}
			は$A$上の近縁系となる.また$S$に$\mathscr{V}$で位相を導入するとき,
			$A$上の相対位相と$\mathscr{V}_A$は両立する.
		\end{thm}
	\end{screen}
	
	\begin{prf}\mbox{}
		\begin{description}
			\item[第一段] $\mathscr{V}_A$が定義
				\ref{dfn:uniform_structure}の(US1)$\sim$(US5)を満たすことを示す.先ず
				$\mathscr{V} \neq \emptyset$より$\mathscr{V}_A \neq \emptyset$であり,
				\begin{align}
					V \in \mathscr{V} \quad \Longrightarrow \quad
					(a,a) \in V,\ (\forall a \in A) \quad \Longrightarrow \quad
					(a,a) \in (A \times A) \cap V,\ (\forall a \in A)
				\end{align}
				となるから(US1)が満たされる.また任意に$E \in \mathscr{V}_A$を取れば
				或る$V \in \mathscr{V}$で$E = (A \times A) \cap V$と表され,
				\begin{align}
					(x,y) \in E^{-1}
					\quad \Longleftrightarrow \quad
					(y,x) \in (A \times A) \cap V 
					\quad \Longleftrightarrow \quad
					(x,y) \in (A \times A) \cap V^{-1}
				\end{align}
				が成り立つから$E^{-1} \in \mathscr{V}_A$が従い(US2)も満たされる.
				任意の$U,V \in \mathscr{V}$に対し
				\begin{align}
					((A \times A) \cap U) \cap ((A \times A) \cap V)
					= (A \times A) \cap (U \cap V) \in \mathscr{V}_A
				\end{align}
				より(US3)が得られ,また$V \in \mathscr{V}$に対し
				$W \circ W \subset V$となる$W \in \mathscr{V}$を取れば
				\begin{align}
					(x,y),(y,z) \in (A \times A) \cap W
					\quad \Longrightarrow \quad
					x,z \in A,\ (x,z) \in V
					\quad \Longrightarrow \quad
					(x,z) \in (A \times A) \cap V
				\end{align}
				となるから(US4)が出る.
				$(A \times A) \cap V \subset R,\ (V \in \mathscr{V})$を満たす任意の
				$R \subset A \times A$に対し,
				$V \cup R \in \mathscr{V}$より
				\begin{align}
					R = (A \times A) \cap (V \cup R) \in \mathscr{V}_A
				\end{align}
				が成立し(US5)も従う.
			
			\item[第二段] $\mathscr{V}_A$で導入する$A$の位相を
				$\tau_A$と書く.任意の$a \in A$と$V \in \mathscr{V}$に対して
				\begin{align}
					[(A \times A) \cap V]_a \coloneqq
					&\Set{x \in A}{(a,x) \in (A \times A) \cap V} \\
					=& \Set{x \in S}{(a,x) \in V} \cap A
					\eqqcolon V_a \cap A
				\end{align}
				となる.$\Set{[(A \times A) \cap V]_a}{V \in \mathscr{V}}$
				は$\tau_A$における$a$の基本近傍系をなし,
				$\Set{V_a \cap A}{V \in \mathscr{V}}$
				は$A$の相対位相における$a$の基本近傍系をなすが,
				両者が一致するので位相も一致する.
				\QED
		\end{description}
	\end{prf}
	
	\begin{screen}
		\begin{thm}[可算な基本近縁系を持てば$\sigma$-局所有限な基底が存在する]
		\label{thm:if_uniformity_has_countable_base_then_has_topology_has_sigma_locally_finite_base}
			$(S,\mathscr{V})$を一様空間とし,$\mathscr{V}$が可算な基本近縁系$\{V_n\}_{n \in \N}$を
			持つとする.また$S$に$\mathscr{V}$で一様位相を導入する.このとき,$\mathscr{S}$を$S$の任意の開被覆
			($\emptyset \notin \mathscr{S}$)とすれば$\mathscr{S}$の開細分で
			$\sigma$-局所有限なものが存在する.特に,$S$は$\sigma$-局所有限な基底を持つ.
		\end{thm}
	\end{screen}
	
	\begin{prf}\mbox{}
		\begin{description}
			\item[第一段] $\mathscr{V}$の可算な基本近縁系$\{U_n\}_{n \in \N}$で,
				\begin{align}
					U_{n+1} \circ U_{n+1} \circ U_{n+1} \subset U_n
					\subset V_n,\quad (\forall n \in \N)
				\end{align}
				を満たし,かつ$U_n$が全て対称であるものが存在する.実際
				$V_1$に対し$\tilde{W}_1 \circ \tilde{W}_1 \subset V_1 \cap V_1^{-1}$を満たす
				$\tilde{W}_1 \in \mathscr{V}$が取れるが,
				$\tilde{W}_1$に対しても或る$W_1 \in \mathscr{V}$で
				$W_1 \circ W_1 \subset \tilde{W}_1$となり,このとき
				\begin{align}
					W_1 \circ W_1 \circ W_1
					\subset W_1 \circ W_1 \circ W_1 \circ W_1
					\subset V_1 \cap V_1^{-1}
				\end{align}
				が成り立つ.ここで
				\begin{align}
					U_1 \coloneqq V_1 \cap V_1^{-1},
					\quad U_2 \coloneqq V_2 \cap V_2^{-1} \cap W_1 \cap W_1^{-1}
				\end{align}
				とおく.$U_2$に対しても$W_2 \circ W_2 \circ W_2 \subset U_2$
				を満たす$W_2 \in \mathscr{V}$が取れるから,ここで
				\begin{align}
					U_3 \coloneqq V_3 \cap V_3^{-1} \cap W_2 \cap W_2^{-1}
				\end{align}
				とおく.帰納的に全ての$n \in \N$に対して$U_n$が定義されるから$\{U_n\}_{n \in \N}$を得る.
				
			\item[第二段]
				整列可能定理により$\mathscr{S}$を整列集合にする全順序$\preceq$が存在する.ここで
				\begin{align}
					E \prec F \quad \overset{\mathrm{def}}{\Longleftrightarrow} \quad
					\mbox{$E \preceq F$かつ$E \neq F$}
				\end{align}
				とする.任意の$x \in S$と$V \in \mathscr{V}$に対し
				\begin{align}
					V(x) \coloneqq \Set{y \in S}{(x,y) \in V}
				\end{align}
				と定義して,任意の$n \in \N$と$E \in \mathscr{S}$に対し
				\begin{align}
					I_n(E) &\coloneqq \Set{x \in S}{U_n(x) \subset E}, \\
					J_n(E) &\coloneqq I_n(E) \left\backslash \bigcup_{F \prec E}F \right.
				\end{align}
				として$\mathscr{S}_n \coloneqq \Set{E \in \mathscr{S}}{J_n(E) \neq \emptyset}$
				とおく.$\mathscr{S}$の$\preceq$に関する最小元を$M$として
				$x \in M$を一つ取れば,$M$は開集合であるから十分大きい$n_0$で
				$U_{n_0}(x) \subset M$となり,$M$の最小性より
				\begin{align}
					x \in I_{n_0}(M) = J_{n_0}(M)
				\end{align}
				が成り立つので,少なくとも$n \geq n_0$ならば$\mathscr{S}_n$は空でない.
				$\mathscr{S}_n \neq \emptyset$のとき,任意の$E \in \mathscr{S}_n$に対し
				\begin{align}
					K_n(E) \coloneqq \bigcup_{x \in J_n(E)} U_{n+1}(x)^{\mathrm{o}}
				\end{align}
				により開集合を定めて$\mathscr{K}_n \coloneqq 
				\Set{K_n(E)}{E \in \mathscr{S}_n}$とおけば次が成立する:
				\begin{description}
					\item[(1)] 任意の$E \in \mathscr{S}_n$に対し$K_n(E) \subset E$となる.
					
					\item[(2)] 相異なる二元$E, F \in \mathscr{S}_n$に対し
						\begin{align}
							(x,y) \notin U_{n+1},\quad
							(\forall x \in K_n(E),\ y \in K_n(F)).
						\end{align}
						
					\item[(3)] 任意の$x \in S$に対し,$U_{n+2}(x)$は$\mathscr{K}_n$の二個以上の元と
						交わることはない.
				\end{description}
				実際,任意の$x \in K_n(E)$に対し或る$x_0 \in J_n(E)$が存在して
				\begin{align}
					x \in U_{n+1}(x_0) \subset U_n(x_0) \subset E
				\end{align}
				となり(1)が出る.また任意に$x \in K_n(E)$と$y \in K_n(F)$を取れば
				或る$x_0 \in J_n(E)$と$y_0 \in J_n(F)$で
				\begin{align}
					(x_0,x),(y,y_0) \in U_{n+1}
				\end{align}
				となるが,このとき$E \neq F$とすると$E \prec F$又は$F \prec E$ということになり,
				$E \prec F$とすれば$x_0 \in E$かつ$y_0 \notin E$が満たされるから
				$(x_0,y_0) \notin U_n$が従う.そして$U_{n+1} \circ U_{n+1} \circ U_{n+1} \subset U_n$より
				\begin{align}
					(x,y) \notin U_{n+1}
				\end{align}
				が成立する.すなわち(2)も得られ,これと$y,z \in U_{n+2}(x) \Longrightarrow (y,z) \in U_{n+1}$
				を併せて(3)も出る.ここで
				\begin{align}
					\mathscr{K} \coloneqq 
					\bigcup_{\substack{n \in \N \\ \mathscr{S}_n \neq \emptyset}} \mathscr{K}_n
				\end{align}
				とおけば,$\mathscr{K}$は$\mathscr{S}$の$\sigma$-局所有限な開細分となる.実際
				(1)より$\mathscr{K}$の元は全て$\mathscr{S}$の元の部分集合であり,(3)より
				各$\mathscr{K}_n$は局所有限である.また
				任意の$x \in S$に対し,$x$を含む$\mathscr{S}$の元のうち$\preceq$に関する最小元を
				$E$とすれば,十分大きい$n$で$U_n(x) \subset E$となるから$x \in I_n(E)$
				が従い,最小性より$F \prec E$なら$x \notin F$が満たされ
				\begin{align}
					x \in J_n(E) \subset K_n(E)
				\end{align}
				が成立する.すなわち$\mathscr{K}$は$S$を覆っている.
				
			\item[第三段]
				任意の$n \in \N$に対し$\Set{U_n(x)^{\mathrm{o}}}{x \in S}$
				は$S$の開被覆をなすから,$\sigma$-局所有限な開細分$\mathscr{B}_n$が存在する.このとき
				任意に$B \in \mathscr{B}_{n+1}$を取れば,或る$z \in S$で$B \subset U_{n+1}(z)$となるから
				\begin{align}
					x,y \in B \quad\Longrightarrow\quad
					(x,z),(z,y) \in U_{n+1} \quad\Longrightarrow\quad
					(x,y) \in U_n
					\label{eq:thm_if_uniformity_has_countable_base_then_has_topology_has_sigma_locally_finite_base}
				\end{align}
				が満たされる.いま,任意に開集合$G$と$x \in G$を取れば,或る$n \in \N$で
				\begin{align}
					U_n(x) \subset G
				\end{align}
				となる.一方で$\mathscr{B}_{n+1}$は$S$を覆うから
				或る$B \in \mathscr{B}_{n+1}$もまた$x$を含み,このとき
				(\refeq{eq:thm_if_uniformity_has_countable_base_then_has_topology_has_sigma_locally_finite_base})
				より
				\begin{align}
					x \in B \subset U_n(x) \subset G
				\end{align}
				が成立する.従って$\mathscr{B} \coloneqq \bigcup_{n \in \N}\mathscr{B}_n$
				は$S$の基底となり,各$\mathscr{B}_n$が$\sigma$-局所有限であるから
				$\mathscr{B}$も$\sigma$-局所有限となる.
				すなわち$S$は$\sigma$-局所有限な基底を持つ.
				\QED
		\end{description}
	\end{prf}
	
	\begin{screen}
		\begin{thm}[一様位相空間は完全正則]
			近縁系で導入する位相は完全正則である.
		\end{thm}
	\end{screen}
	
	\begin{screen}
		\begin{thm}[可算な基本近縁系を持つ一様位相空間はパラコンパクト]
		\end{thm}
	\end{screen}
	
	\begin{prf}
		可算な基本近縁系を持つ一様位相空間は完全正則であるから正則である.
		また任意の開被覆は$\sigma$-局所有限な開被覆を持つが,
		正則空間においては局所有限な開被覆を持つことと同値になる.
		\QED
	\end{prf}
	
	\begin{screen}
		\begin{dfn}[一様化可能]
			空でない位相空間が{\bf 一様化可能}\index{いちようかかのう@一様化可能}
			{\bf (uniformizable)}であるとは,その位相と両立する近縁系が存在することをいう.
		\end{dfn}
	\end{screen}
	
	\begin{screen}
		\begin{thm}[完全正則なら一様化可能]
			$S$を空でない完全正則空間とし,$C(S,\R)$で$S$から$\R$への連続写像の全体を表す.
			任意の連続写像$f \in C(S,\R)$と$B_r \coloneqq \Set{(\alpha,\beta) \in \R \times \R}{|\alpha - \beta| < r},\ (r > 0)$に対し
			\begin{align}
				V(f,r) \coloneqq \Set{(x,y) \in S \times S}{(f(x),f(y)) \in B_r}
			\end{align}
			と定義して,これらの有限交叉の全体を$\mathscr{A} \coloneqq
				\Set{\bigcap_{i=1}^n V(f_i,r_i)}{f_i \in C(S,\R),\ r_i \in (0,\infty),\ n \in \N}$とおくとき,
			\begin{align}
				\mathscr{V} \coloneqq
				\Set{V \subset S \times S}{\mbox{$V$は$\mathscr{A}$の元を少なくとも一つ含む}}
			\end{align}
			は$S$上の近縁系となり$S$の位相と両立する.
		\end{thm}
	\end{screen}
	
	\begin{prf}\mbox{}
		\begin{description}
			\item[第一段] $\mathscr{V}$が近縁系であることを示す.$C(S,\R)$は空でないから
				$\mathscr{V}$も空でなく,また任意の$f \in C(S,\R)$と$r > 0$で
				\begin{align}
					(f(x),f(x)) \in B_r, \quad (\forall x \in S) 
				\end{align}
				となるから定義\ref{dfn:uniform_structure}の(US1)が満たされる.
				$B_r$の対称性より$V(f,r)$も対称となるから
				\begin{align}
					\bigcap_{i=1}^n V(f_i,r_i) \subset V
					\quad \Longleftrightarrow \quad
					\bigcap_{i=1}^n V(f_i,r_i) = \bigcap_{i=1}^n V(f_i,r_i)^{-1} \subset V^{-1}
				\end{align}
				が従い(US2)も満たされる.$\mathscr{A}$が有限交叉で閉じるから
				$\mathscr{V}$も有限交叉で閉じる.
				$\bigcap_{i=1}^n V(f_i,r_i) \subset V$のとき
				\begin{align}
					W \coloneqq \bigcap_{i=1}^n V(f_i,r_i/2)
				\end{align}
				とおけば
				\begin{align}
					(x,y),(y,z) \in W &\quad \Longrightarrow \quad
					|f(x) - f(y)|,|f(y)-f(z)| < r_i/2,\ (1 \leq i \leq n) \\
					&\quad \Longrightarrow \quad
					|f(x) - f(z)| < r_i,\ (1 \leq i \leq n)
				\end{align}
				となるから$W \circ W \subset \bigcap_{i=1}^n V(f_i,B_{r_i}) \subset V$
				が成立し,$\mathscr{V}$は上位の包含関係で閉じるから(US5)も満たされる.
				
			\item[第二段]
				任意の$x \in S$と$V \in \mathscr{V}$に対し
				\begin{align}
					V_x \coloneqq \Set{y \in S}{(x,y) \in V}
				\end{align}
				と定めるとき,$\mathscr{V}(x) \coloneqq \Set{V_x}{V \in \mathscr{V}}$
				がもとの位相において$x$の基本近傍系となることを示す.実際
				\begin{align}
					V(f,r)_x = \Set{y \in S}{(f(x),f(y)) \in B_r}
					= \Set{y \in S}{|f(x)-f(y)|< r}
				\end{align}
				より$V(f,r)_x$は$x$の開近傍となるから$\mathscr{U}(x)$の元は
				$x$の近傍であり,さらに$x$の任意の近傍$U$に対し
				\begin{align}
					g(y) = \begin{cases}
						0, & (x=y),\\
						1, & (y \in S \backslash U^{\mathrm{o}})
					\end{cases}
				\end{align}
				を満たす$g \in C(S,\R)$が取れるが(完全正則性による),このとき
				\begin{align}
					y \in V(g,1)_x \quad \Longrightarrow \quad
					|g(y)| = |g(y) - g(x)| < 1 \quad \Longrightarrow \quad
					y \in U^{\mathrm{o}}
				\end{align}
				となるから$V(g,B_1)_x \subset U$が従う.以上より
				$\mathscr{V}(x)$はもとの位相で$x$の基本近傍系となるが,
				一方で$\mathscr{V}$の導入する一様位相においても
				$\mathscr{V}(x)$は$x$の基本近傍系をなすから二つの位相は一致する.
				\QED
		\end{description}
	\end{prf}
	
	\begin{screen}
		\begin{thm}[一様位相空間において$T_0 \Longleftrightarrow T_{3{}^1{\mskip -5mu/\mskip -3mu}_2}$]
		\label{thm:T_0_iff_T_2_on_uniform_topological_space}
			$(S,\mathscr{V})$を一様空間とし,$S$に一様位相を導入する.このとき
			任意の$x,y \in S$に対して
			\begin{align}
				\mbox{$x,y$が位相的に識別可能}
				\quad \Longleftrightarrow \quad
				(x,y) \notin \bigcap \mathscr{V}
				\quad \Longleftrightarrow \quad
				\mbox{$x,y$が近傍で分離される}
				\label{eq:thm_T_0_iff_T_2_on_uniform_topological_space_0}
			\end{align}
			が成立する.特に,
			$S$が$T_0$であること,
			$\bigcap \mathscr{V} = \Set{(x,x)}{x \in S}$,
			$S$が$T_{3{}^1{\mskip -5mu/\mskip -3mu}_2}$であること,
			は全て同値になる.
		\end{thm}
	\end{screen}
	
	\begin{prf}
		任意の$V \in \mathscr{V}$と$s \in S$に対し
		\begin{align}
			V_s \coloneqq \Set{t \in S}{(s,t) \in V}
		\end{align}
		と定義する.$(x,y) \in \bigcap_{V \in \mathscr{V}}V$なら
		任意の$V \in \mathscr{V}$に対し
		\begin{align}
			x \in V_y,\quad y \in V_x
		\end{align}
		となる.$\{V_x\}_{V \in \mathscr{V}}$と$\{V_y\}_{V \in \mathscr{V}}$
		はそれぞれ$x,y$の基本近傍系となるから
		定理\ref{thm:belongs_to_closure_iff_clusters}より
		$x \in \overline{\{y\}}$かつ$y \in \overline{\{x\}}$が従い
		\begin{align}
			\mbox{$x,y$が位相的に識別可能}
			\quad \Longrightarrow \quad
			(x,y) \notin \bigcap \mathscr{V}
		\end{align}
		が出る.また或る$V \in \mathscr{V}$で$(x,y) \notin V$となるとき,
		$W \circ W \subset V$を満たす対称な$W \in \mathscr{V}$を取れば
		$W_x,W_y$はそれぞれ$x,y$の近傍となるが,これらは互いに素である.実際
		$W_x \cap W_y \neq \emptyset$とすると,
		$z \in W_x \cap W_y$を取れば$(x,z),(z,y) \in W$から
		$(x,y) \in V$が従い$(x,y) \notin V$に矛盾する.よって
		\begin{align}
			(x,y) \notin \bigcap \mathscr{V}
			\quad \Longrightarrow \quad
			\mbox{$x,y$が近傍で分離される}
		\end{align}
		を得る.近傍で分離される二点は位相的に識別可能であるから
		(\refeq{eq:thm_T_0_iff_T_2_on_uniform_topological_space_0})が成立する.
		$S$が$T_0$なら
		\begin{align}
			\begin{cases}
				(x,y) \notin \bigcap \mathscr{V}, & (x \neq y), \\
				(x,y) \in \bigcap \mathscr{V}, & (x = y),
			\end{cases}
			\quad (\forall x,y \in S)
		\end{align}
		となるから$\bigcap \mathscr{V} = \Set{(x,x)}{x \in S}$が従う.
		このとき相異なる任意の二点は近傍で分離されるから$S$はHausdorff性を持ち,
		完全正則性と併せて$T_{3{}^1{\mskip -5mu/\mskip -3mu}_2}$となる.
		$T_{3{}^1{\mskip -5mu/\mskip -3mu}_2}$なら$T_0$は満たされるから後半の主張を得る.
		\QED
	\end{prf}
	
	\begin{screen}
		\begin{dfn}[一様収束]
			$S,T$を空でない集合とし,$T$上に近縁系$\mathscr{V}$が定まっているとする.
			また$(f_\lambda)_{\lambda \in \Lambda}$を
			有向集合$(\Lambda,\leq)$で添数付けた$S$から$T$への写像族,
			$f$を$S$から$T$への写像,及び$A$を$S$の部分集合とする.
			\begin{itemize}
				\item 任意の近縁$V$に対し或る
					$\lambda_0 \in \Lambda$が存在して,全ての$y \in A$で
					\begin{align}
						\lambda_0 \leq \lambda \quad \Longrightarrow \quad
						(f(y),f_\lambda(y)) \in V
					\end{align}
					となるとき,$(f_\lambda)_{\lambda \in \Lambda}$は
					$A$上で$f$に{\bf 一様収束}\index{いちようしゅうそく@一様収束}する
					{\bf (uniformly converge)}という.
					
				\item 任意の一点集合において一様収束することを{\bf 各点収束}
					\index{かくてんしゅうそく@各点収束}{\bf (pointwise convergence)}という.
				
				\item $S$が位相空間であるとき,任意のコンパクト部分集合上で一様収束することを
					{\bf コンパクト一様収束}\index{こんぱくといちようしゅうそく@コンパクト一様収束}
					{\bf (compact convergence)}や{\bf 広義一様収束}
					\index{こうぎいちようしゅうそく@広義一様収束}という.
					
				\item $S$が位相空間であるとき,各点$x \in S$で或る近傍$U(x)$が取れて
					$U(x)$上で一様収束することを{\bf 局所一様収束}
					\index{きょくしょいちようしゅうそく@局所一様収束}
					{\bf (locally uniform convergence)}という.
			\end{itemize}
		\end{dfn}
	\end{screen}
	
	\begin{screen}
		\begin{thm}[連続写像が局所一様収束するなら極限写像も連続]
			$S$を位相空間,$(T,\mathscr{V})$を一様空間として$T$に$\mathscr{V}$で
			位相を導入する.また$(\Lambda,\leq)$を有向集合,
			$(f_\lambda)_{\lambda \in \Lambda}$を$S$から$T$への連続写像族とし,
			$f$を$S$から$T$への写像とする.このとき
			$(f_\lambda)_{\lambda \in \Lambda}$が$f$に局所一様収束しているなら
			$f$も連続写像である.
		\end{thm}
	\end{screen}
	
	\begin{prf}
		任意に$x \in S$を取れば,$f(x)$の任意の近傍$N$に対し或る近縁$V$が存在して
		\begin{align}
			V_{f(x)} \coloneqq \Set{t \in T}{(f(x),t) \in V} \subset N
		\end{align}
		を満たす.また$V$に対し或る対称な近縁$W$で
		\begin{align}
			(a,b),(b,c),(c,d) \in W \quad \Longrightarrow \quad
			(a,d) \in V,
			\quad (\forall a,b,c,d \in S)
		\end{align}
		を満たすものが取れる.この$W$に対して或る$\lambda_0 \in \Lambda$と
		$x$の近傍$U(x)$が存在し,全ての$y \in U(x)$で
		\begin{align}
			\lambda_0 \leq \lambda \quad \Longrightarrow \quad
			(f(y),f_\lambda(y)) \in W
		\end{align}
		となる.$f_{\lambda_0}$は連続であり$W_{f_{\lambda_0}(x)}$は
		$f_{\lambda_0}(x)$の近傍であるから,$x$の或る近傍$R(x)$で
		\begin{align}
			y \in R(x) \quad \Longrightarrow \quad
			f_{\lambda_0}(y) \in W_{f_{\lambda_0}(x)}
		\end{align}
		が成り立ち,以上より
		\begin{align}
			y \in U(x) \cap R(x) &\quad \Longrightarrow \quad
			(f(x),f_{\lambda_0}(x)),(f_{\lambda_0}(x),f_{\lambda_0}(y))
			,(f_{\lambda_0}(y),f(y)) \in W \\
			&\quad \Longrightarrow \quad (f(x),f(y)) \in V \\
			&\quad \Longrightarrow \quad f(y) \in N
		\end{align}
		が従い$f$の$x$での連続性が得られる.$x$の任意性と
		定理\ref{thm:continuous_on_every_point_iff_continuous}より$f$は$S$で連続である.
		\QED
	\end{prf}
	
	\begin{screen}
		\begin{thm}[局所コンパクト空間において,広義一様収束$\Longleftrightarrow$局所一様収束]
		\label{thm:compact_unif_conv_iff_locally_unif_conv_if_locally_compact}
			$S$を局所コンパクト空間,$(T,\mathscr{V})$を一様空間として$T$に$\mathscr{V}$で
			位相を導入する.また$(\Lambda,\leq)$を有向集合,
			$(f_\lambda)_{\lambda \in \Lambda}$を$S$から$T$への写像族とし,
			$f$を$S$から$T$への写像とする.このとき
			\begin{align}
				\mbox{$(f_\lambda)_{\lambda \in \Lambda}$が$f$に広義一様収束する} 
				\quad \Longleftrightarrow \quad
				\mbox{$(f_\lambda)_{\lambda \in \Lambda}$が$f$に局所一様収束する}.
			\end{align}
		\end{thm}
	\end{screen}
	
	\begin{prf}
		各点でコンパクトな近傍が取れるから$\Longrightarrow$が得られる.
		逆に$(f_\lambda)_{\lambda \in \Lambda}$が
		$f$に局所一様収束すると仮定する.$K$を$S$のコンパクト部分集合として
		各点$x \in K$のコンパクトな近傍$U(x)$を取れば,
		有限個の$x_1,x_2,\cdots,x_n \subset K$で
		\begin{align}
			K \subset U(x_1) \cup \cdots \cup U(x_n)
			\label{eq:thm_compact_unif_conv_iff_locally_unif_conv_if_locally_compact}
		\end{align}
		となる.任意に近縁$V \in \mathscr{V}$を取れば,
		各$i = 1,\cdots,n$で或る$\lambda_i \in \Lambda$が存在して
		\begin{align}
			y \in U(x_i),\ \lambda_i \leq \lambda \quad \Longrightarrow \quad
			(f(y),f_\lambda(y)) \in V
		\end{align}
		が成り立つが,このとき有向律より$\{\lambda_1,\cdots,\lambda_n\}$は
		上界$\lambda_0 \in \Lambda$
		を持つから,任意の$y \in \bigcup_{i=1}^n U(x_i)$で
		\begin{align}
			\lambda_0 \leq \lambda \quad \Longrightarrow \quad
			(f(y),f_\lambda(y)) \in V
		\end{align}
		が満たされる.
		(\refeq{eq:thm_compact_unif_conv_iff_locally_unif_conv_if_locally_compact})より
		$(f_\lambda)_{\lambda \in \Lambda}$は$K$上で$f$に一様収束し,
		$K$の任意性から$\Longleftarrow$が得られる.
		\QED
	\end{prf}
	
	\begin{screen}
		\begin{dfn}[一様連続性]
			$(S,\mathscr{U})$と$(T,\mathscr{V})$を一様空間として
			$\mathscr{U},\mathscr{V}$により$S,T$に一様位相を導入し,
			$f:S \longrightarrow T$を写像とする.また$A$を$S$の部分集合とする.
			任意の$V \in \mathscr{V}$に対し或る$U \in \mathscr{U}$が存在して
			\begin{align}
				(x,y) \in (A \times A) \cap U 
				\quad \Longrightarrow \quad (f(x),f(y)) \in V
			\end{align}
			となるとき,$f$は$A$上で{\bf 一様連続}\index{いちようれんぞく@一様連続}である
			{\bf (uniformly continuous)}という.
		\end{dfn}
	\end{screen}
	
	\begin{screen}
		\begin{thm}[一様連続なら連続]
			$(S,\mathscr{U})$と$(T,\mathscr{V})$を一様空間として
			$\mathscr{U},\mathscr{V}$により$S,T$に一様位相を導入し,
			$f:S \longrightarrow T$を写像とする.また$A$を$S$の部分集合とする.
			このとき$f$が$A$上で一様連続なら$f$は$A$上で連続
			($\left. f\right|_A$が連続)である.
		\end{thm}
	\end{screen}
	
	\begin{prf}
		任意の$U \in \mathscr{U},s \in S$に対し
		\begin{align}
			U_s \coloneqq \Set{y \in S}{(s,y) \in U}
		\end{align}
		と定義し,同様に$V \in \mathscr{V},t \in T$に対し$V_t$を定める.
		$x \in A$を取れば,$f(x)$の任意の近傍$N$に対し或る
		$V \in \mathscr{V}$で
		\begin{align}
			V_{f(x)} \subset N
		\end{align}
		となる.このとき$f$が$A$で一様連続であるなら
		$V$に対し或る$U \in \mathscr{U}$が存在して
		\begin{align}
			(x,y) \in (A \times A) \cap U 
			\quad \Longrightarrow \quad (f(x),f(y)) \in V
		\end{align}
		が満たされるから,
		\begin{align}
			y \in A \cap U_x \quad \Longrightarrow \quad
			f(y) \in V_{f(x)} \subset N
		\end{align}
		が従い$\left. f\right|_A$の$x$での連続性が出る.$x$の任意性より
		$f$は$A$上で連続である.
		\QED
	\end{prf}
	
	\begin{screen}
		\begin{thm}[連続写像はコンパクト集合上で一様連続]
			$(S,\mathscr{U})$と$(T,\mathscr{V})$を一様空間として
			$\mathscr{U},\mathscr{V}$で$S,T$に一様位相を導入し,
			$f:S \longrightarrow T$を連続写像とする.
			このとき$f$は任意のコンパクト部分集合上で一様連続となる.
		\end{thm}
	\end{screen}
	
	\begin{prf}
		$K$を$S$のコンパクト部分集合とする.
		任意の$M \in \mathscr{U}, s \in S$に対し
		\begin{align}
			M_s \coloneqq \Set{x \in S}{(s,x) \in M}
		\end{align}
		と定義し,$W \in \mathscr{V},\ t \in T$に対しても同様に$W_t$を定める.
		任意に$V \in \mathscr{V}$を取れば,定理\ref{thm:uniform_structure}より
		或る$W \in \mathscr{V}$で
		\begin{align}
			W_t \times W_t \subset V,
			\quad (\forall t \in T)
		\end{align}
		となる.$f$は連続であるから任意の$s \in S$に対し或る$N(s) \in \mathscr{U}$が存在して
		\begin{align}
			(s,x) \in N(s) \quad \Longrightarrow \quad
			f(x) \in W_{f(s)}
		\end{align}
		が成り立つ.$M(s) \circ M(s) \subset N(s)$を満たす対称な$M(s) \in \mathscr{U}$を取れば,
		定理\ref{thm:subset_is_compact_iff_every_original_open_cover_contains_finite_subcover}より
		或る$x_1,\cdots,x_n \in K$で
		\begin{align}
			K \subset \bigcup_{i=1}^n M(x_i)_{x_i}
		\end{align}
		となる.近縁系は有限交叉で閉じるから
		\begin{align}
			U \coloneqq \bigcap_{i=1}^n M(x_i)
		\end{align}
		は$\mathscr{U}$の元であり,このとき任意に$(x,y) \in (K \times K) \cap U$を取れば,
		或る$i$で$y \in M(x_i)_{x_i}$となり,$(x_i,x_i),(x_i,y) \in M(x_i)$より
		$(x_i,y) \in N(x_i)$となる.一方で$M(x_i)$が対称であるから
		$(x_i,y),(y,x) \in M(x_i)$となり$(x_i,x) \in N(x_i)$
		が満たされ,$f(x),f(y) \in W_{f(x_i)}$が従うから
		$(f(x),f(y)) \in V$が成立し$f$の$K$の上での一様連続性が出る.
		\QED
	\end{prf}
	
	\begin{screen}
		\begin{thm}[擬距離空間の一様構造]
		\label{thm:uniform_structure_on_pseudometric_spaces}
			$(S,d)$を擬距離空間とするとき,
			\begin{align}
				\mathscr{V} \coloneqq
				\Set{V(r)}{r > 0},
				\quad (V(r) \coloneqq \Set{(x,y) \in S \times S}{d(x,y) < r})
			\end{align}
			とおけば$\mathscr{V}$は$S$上の一様構造となり,
			$\mathscr{V}$で導入する一様位相は$d$-位相に一致する.
		\end{thm}
	\end{screen}
	
	\begin{screen}
		\begin{thm}[擬距離空間のCauchy列]
		\label{thm:Cauchy_sequences_on_pseudometric_spaces}
			$(S,d)$を擬距離空間とし,一様構造$\mathscr{V}$を
			定理\ref{thm:uniform_structure_on_pseudometric_spaces}の要領で定めるとき,
			$S$の任意の点列$(x_n)_{n \in \N}$に対し,$(x_n)_{n \in \N}$がCauchy列であることと
			\begin{align}
				\forall \epsilon > 0;\ 
				\exists N \in \N;\quad
				n,m \geq N \Longrightarrow d(x_n,x_m) < \epsilon
				\label{eq:thm_Cauchy_sequences_on_pseudometric_spaces}
			\end{align}
			が成り立つことは同値になる.
		\end{thm}
	\end{screen}
	
	\begin{prf}
		任意の$\epsilon$と$n,m \in \N$で
		\begin{align}
			(x_n,x_m) \in V(\epsilon) \quad \Longleftrightarrow \quad
			d(x_n,x_m) < \epsilon
		\end{align}
		となるから,$(x_n)_{n \in \N}$がCauchy列であるとき,任意の$\epsilon > 0$に対し
		或る$N \in \N$が存在して
		\begin{align}
			n,m \geq N \quad \Longrightarrow \quad
			(x_n,x_m) \in V(\epsilon) \quad \Longrightarrow \quad
			d(x_n,x_m) < \epsilon
		\end{align}
		が成り立つ.逆に$(x_n)_{n \in \N}$に対して
		(\refeq{eq:thm_Cauchy_sequences_on_pseudometric_spaces})が
		満たされているとき,任意の$V(\epsilon) \in \mathscr{V}$に対し
		或る$M \in \N$が存在して
		\begin{align}
			n,m \geq N \quad \Longrightarrow \quad
			d(x_n,x_m) < \epsilon \quad \Longrightarrow \quad
			(x_n,x_m) \in V(\epsilon)
		\end{align}
		となるから$(x_n)_{n \in \N}$はCauchy列である.
		\QED
	\end{prf}
	
	\begin{screen}
		\begin{thm}[点列の擬距離に関する収束]
			点列$(x_n)_{n \in \N}$が$a$に収束する
			ことと$d(x_n,a) \longrightarrow 0$は同値.
		\end{thm}
	\end{screen}
	
	\begin{screen}
		\begin{dfn}[Cauchy有向点族・完備性]
			$(S,\mathscr{V})$を一様空間,$(\Lambda,\leq)$を有向集合,
			$(x_\lambda)_{\lambda \in \Lambda}$を有向点族とする.
			$(x_\lambda)_{\lambda \in \Lambda}$が{\bf Cauchy有向点族}
			\index{Cauchyゆうこうてんぞく@Cauchy有向点族}
			{\bf (Cauchy net)}であるとは,任意の近縁$V \in \mathscr{V}$に対し
			或る$\lambda_0 \in \Lambda$が存在して
			\begin{align}
				\lambda_0 \leq \lambda,\mu
				\quad \Longrightarrow \quad
				(x_\lambda,x_\mu) \in V
			\end{align}
			となることをいう.点列がCauchy有向点族をなすときは
			これを{\bf Cauchy列}\index{Cauchy列@Cauchy列}{\bf (Cauchy sequence)}と呼ぶ.
			$S$の空でない部分集合$A$の上の任意のCauchy有向点族が$A$で収束するとき,
			$A$は$S$で{\bf 完備}である\index{かんび@完備}{\bf (complete)}という.
		\end{dfn}
	\end{screen}
	
	\begin{screen}
		\begin{thm}[収束する有向点族はCauchy有向点族]\label{thm:convergent_net_is_Cauchy}
			$(S,\mathscr{V})$を一様空間,$(\Lambda,\leq)$を有向集合,
			$(x_\lambda)_{\lambda \in \Lambda}$を有向点族とする.
			$(x_\lambda)_{\lambda \in \Lambda}$が$S$で収束するとき,
			$(x_\lambda)_{\lambda \in \Lambda}$はCauchy有向点族をなしている.
		\end{thm}
	\end{screen}
	
	\begin{prf}
		$a \in S$を$(x_\lambda)_{\lambda \in \Lambda}$の極限(の一つ)とする.
		任意の近縁$V \in \mathscr{V}$に対し或る対称な$W \in \mathscr{V}$で
		$W \circ W \subset V$を満たすものが取れるが,
		$W_a \coloneqq \Set{s \in S}{(a,s) \in W}$は$a$の近傍であるから
		或る$\lambda_0 \in \Lambda$が存在して
		\begin{align}
			\lambda_0 \leq \lambda \quad \Longrightarrow \quad
			x_\lambda \in W_a
		\end{align}
		となり,$W$の対称性と併せて
		\begin{align}
			\lambda_0 \leq \lambda,\mu \quad \Longrightarrow \quad
			(x_\lambda,a),(a,x_\mu) \in W \quad \Longrightarrow \quad
			(x_\lambda,x_\mu) \in V
		\end{align}
		が成り立つ.
		\QED
	\end{prf}
	
	\begin{screen}
		\begin{thm}[Hausdorff一様位相空間の完備部分集合は閉]
			$(S,\mathscr{V})$を一様空間として$\mathscr{V}$により$S$に位相を導入する.
			また$A$を$S$の完備な部分集合とする.このとき,$S$がHausdorffなら$A$は$S$で閉じている.
		\end{thm}
	\end{screen}
	
	\begin{prf}
		$A = \overline{A}$を示す.定理\ref{thm:closed_set_is_set_of_limits_of_some_net}より
		任意に$a \in \overline{A}$を取れば或る$A$上の有向点族
		$(x_\lambda)_{\lambda \in \Lambda}$が$a$に収束する.
		定理\ref{thm:convergent_net_is_Cauchy}より
		$(x_\lambda)_{\lambda \in \Lambda}$はCauchy有向点族となるから
		$A$で収束し,Hausdorff性から$a \in A$となる
		(定理\ref{thm:Hausdorff_iff_at_most_one_limit}).
		\QED
	\end{prf}
	
	\begin{screen}
		\begin{thm}[完備な一様空間の閉集合は完備]
		\end{thm}
	\end{screen}
	
	\begin{screen}
		\begin{thm}[Cauchy有向点族の部分有向点族もCauchy・部分点族の極限はもとの点族でも極限]
		\label{thm:subnet_of_a_Cauchy_net_is_Cauchy}
			$(S,\mathscr{V})$を一様空間,$(\Lambda,\leq)$を有向集合,
			$(x_\lambda)_{\lambda \in \Lambda}$をCauchy有向点族とする.
			また$(\Gamma,\preceq)$を有向集合,$f:\Gamma \longrightarrow \Lambda$を
			共終かつ単調な写像とする.このとき部分有向点族
			$(x_{f(\gamma)})_{\gamma \in \Gamma}$もまたCauchy有向点族となり,
			任意の$a \in S$に対し
			\begin{align}
				x_{f(\gamma)} \longrightarrow a
				\quad \Longrightarrow \quad
				x_\lambda \longrightarrow a.
			\end{align}
		\end{thm}
	\end{screen}
	定理\ref{thm:a_net_converges_iff_every_subnet_converges}より
	一般に有向点族が点$a$に収束するための必要十分条件は
	その任意の部分有向点族が$a$に収束することであるが,
	Cauchy有向点族の場合は半分収束点族をなしている(定理\ref{thm:convergent_net_is_Cauchy})から,
	一つでも収束する部分有向点族が得られれば元の点族の収束も判明する.
	
	\begin{prf}
		任意の近縁$V \in \mathscr{V}$に対し或る$\lambda_0 \in \Lambda$が存在して
		\begin{align}
			\lambda_0 \leq \lambda,\mu
			\quad \Longrightarrow \quad
			(x_\lambda,x_\mu) \in V
		\end{align}
		が成り立つ.共終性より$\lambda_0 \leq f(\gamma_0)$を満たす
		$\gamma_0 \in \Gamma$が取れて,
		$\gamma_0 \preceq \gamma, \xi$なら
		$f(\gamma_0) \leq f(\gamma),f(\xi)$となるから
		\begin{align}
			\gamma_0 \preceq \gamma,\xi
			\quad \Longrightarrow \quad
			\left(x_{f(\gamma)},x_{f(\xi)}\right) \in V
		\end{align}
		が従う.よって$(x_{f(\gamma)})_{\gamma \in \Gamma}$はCauchy有向点族である.
		いま,$x_{f(\gamma)} \longrightarrow a$とする.
		このとき任意の$U \in \mathscr{V}$に対し$W \circ W \subset U$を満たす
		$W \in \mathscr{V}$を取れば,或る$\gamma_1 \in \Gamma$が存在して
		\begin{align}
			\gamma_1 \preceq \gamma
			\quad \Longrightarrow \quad
			\left(a,x_{f(\gamma)}\right) \in W
		\end{align}
		となる.一方で或る$\lambda_1 \in \Lambda$が存在して
		\begin{align}
			\lambda_1 \leq \lambda,\mu
			\quad \Longrightarrow \quad
			(x_\lambda,x_\mu) \in W
		\end{align}
		となる.$\lambda_1 \leq f(\gamma_2)$を満たす$\gamma_2 \in \Gamma$を取り,
		$\{\gamma_1,\gamma_2\}$の上界を$\gamma_3 \in \Gamma$とすれば,
		\begin{align}
			\lambda_1 \leq \lambda \quad \Longrightarrow \quad
			\left(a,x_{f(\gamma_3)}\right),
			\left(x_{f(\gamma_3)},x_\lambda\right) \in W
			\quad \Longrightarrow \quad
			(a,x_\lambda) \in U
		\end{align}
		が成り立ち$x_\lambda \longrightarrow a$が従う.
		\QED
	\end{prf}
	
	\begin{screen}
		\begin{thm}[可算な基本近縁系が存在するとき,完備$\Longleftrightarrow$任意のCauchy列が収束する]
		\label{thm:complete_iff_every_Cauchy_seq_converges_if_entourage_contains_some_countable_subset}
			$(S,\mathscr{V})$を一様空間とし,$\mathscr{V}$で$S$に位相を導入する.
			また$A$を$S$の空でない部分集合とする.
			$\mathscr{V}$が可算な基本近縁系を持つとき,
			\begin{align}
				\mbox{$A$が$S$で完備である} \quad \Longleftrightarrow \quad
				\mbox{$A$上の任意のCauchy列が$A$で収束する}.
			\end{align}
		\end{thm}
	\end{screen}
	
	\begin{prf}
		$\Longleftarrow$を示す.$\mathscr{V}$が可算な基本近縁系
		$\{V_n\}_{n \in \N}$を持つとき,近縁系は有限交叉で閉じるから
		\begin{align}
			U_n \coloneqq V_1 \cap V_2 \cap \cdots \cap V_n,
			\quad (n = 1,2,\cdots)
		\end{align}
		により単調減少な$\mathscr{V}$の基本近縁系$\{U_n\}_{n \in \N}$が定まる.
		$(x_\lambda)_{\lambda \in \Lambda}$を$A$上のCauchy有向点族として
		\begin{align}
			X_\lambda \coloneqq \Set{x_\mu}{\lambda \leq \mu},
			\quad (\forall \lambda \in \Lambda)
		\end{align}
		とおけば,任意の$n \in \N$で或る$\lambda_n \in \Lambda$が存在して
		\begin{align}
			X_{\lambda_n} \times X_{\lambda_n} \subset U_n
		\end{align}
		となる.任意の$V \in \mathscr{V}$に対し$W \circ W \subset V$
		を満たす$W \in \mathscr{V}$を取れば,或る$N \in \N$で$U_N \subset W$となるから
		\begin{align}
			U_N \circ U_N \subset V
		\end{align}
		が成り立つ.また任意の$n,m \geq N$に対し,有向集合の定義より
		$\lambda_n,\lambda_m \leq \mu$を満たす$\mu \in \Lambda$が存在して
		\begin{align}
			(x_{\lambda_n},x_\mu) \in U_n \subset U_N,
			\quad (x_\mu, x_{\lambda_m}) \in U_m \subset U_N
		\end{align}
		となり$(x_{\lambda_n},x_{\lambda_m}) \in V$が従うから,
		$(x_{\lambda_n})_{n \in \N}$はCauchy列であり或る$a \in A$に収束する.このとき
		\begin{align}
			x_\lambda \longrightarrow a
			\label{eq:thm_complete_iff_every_Cauchy_seq_converges_if_entourage_contains_some_countable_subset}
		\end{align}
		が成立する.実際,任意に$a$の近傍$B$を取れば或る$\tilde{V} \in \mathscr{V}$で
		\begin{align}
			\tilde{V}_a \coloneqq \Set{x \in S}{(a,x) \in \tilde{V}} \subset B
		\end{align}
		となり,$\tilde{W} \circ \tilde{W} \subset V$を満たす$\tilde{W} \in \mathscr{V}$に対し
		或る$N_1 \in \N$が存在して
		\begin{align}
			n \geq N_1 \quad \Longrightarrow \quad
			x_{\lambda_n} \in \tilde{W}_a \quad \Longrightarrow \quad
			(a,x_{\lambda_n}) \in \tilde{W}
		\end{align}
		を満たす.また或る$N_2 \geq N_1$で$U_{N_2} \subset \tilde{W}$となるから
		\begin{align}
			X_{\lambda_{N_2}} \times X_{\lambda_{N_2}} \subset U_{N_2} \subset \tilde{W}
		\end{align}
		が従い,このとき$(a,x_{\lambda_{N_2}}) \in \tilde{W}$かつ
		$(x_{\lambda_{N_2}},x) \in \tilde{W},\ (\forall x \in X_{\lambda_{N_2}})$より
		$(a,x) \in \tilde{V},\ (\forall x \in X_{\lambda_{N_2}})$となるから
		\begin{align}
			X_{\lambda_{N_2}} \subset \tilde{V}_a \subset B 
		\end{align}
		が得られ
(\refeq{eq:thm_complete_iff_every_Cauchy_seq_converges_if_entourage_contains_some_countable_subset})
		が出る.$A$上の任意のCauchy有向点族が$A$で収束するから$A$は$S$で完備である.
		\QED
	\end{prf}
	
	\begin{screen}
		\begin{dfn}[全有界]
			$(S,\mathscr{V})$を一様空間とし,$A$を$S$の空でない部分集合とする.
			このとき,$A$が$S$で{\bf 全有界}\index{ぜんゆうかい@全有界}{\bf (totally bounded)}
			であるとは,任意の近縁$V \in \mathscr{V}$に対し或る
			有限集合$F_V \subset A$が存在して
			\begin{align}
				A \subset \bigcup_{x \in F_V} V_x
			\end{align}
			が満たされることを指す.ここで$V_x \coloneqq \Set{y \in S}{(x,y) \in V}$である.
		\end{dfn}
	\end{screen}
	
	\begin{screen}
		\begin{thm}[全有界性の同値条件]
		\label{thm:an_equivalent_condition_of_totally_boundedness}
			$(S,\mathscr{V})$を一様空間,$A$を$S$の空でない部分集合とする.
			このとき,$A$が$S$で全有界であることと,任意の近縁$V \in \mathscr{V}$に対し
			$A$の空でない部分集合の有限族$\mathscr{F}_V$で
			\begin{align}
				A = \bigcup \mathscr{F}_V;
				\quad F \times F \subset V,
				\ (\forall F \in \mathscr{F}_V)
				\label{eq:thm_an_equivalent_condition_of_totally_boundedness}
			\end{align}
			を満たすものが取れることは同値である.
		\end{thm}
	\end{screen}
	
	\begin{prf}
		$A$が$S$で全有界であるとする.任意に$V \in \mathscr{V}$を取れば,
		定理\ref{thm:uniform_structure}より或る$W \in \mathscr{V}$で
		\begin{align}
			W_x \times W_x \subset V,\quad (\forall x \in S)
		\end{align}
		が成り立つ.全有界性より$W$に対し或る有限集合$F_W \subset A$が存在して
		\begin{align}
			A \subset \bigcup_{x \in F_W} W_x
		\end{align}
		となるが,このとき$\mathscr{F}_V \coloneqq
		\{W_x \cap A\}_{x \in F_W}$は$A$の空でない部分集合の族で
		(\refeq{eq:thm_an_equivalent_condition_of_totally_boundedness})を満たす.
		逆に任意の近縁$V \in \mathscr{V}$に対して$A$の空でない部分集合の
		有限族$\mathscr{F}_V \subset \mathscr{A}$で
		(\refeq{eq:thm_an_equivalent_condition_of_totally_boundedness})を満たすものが
		取れるとき,$\mathscr{F}_V$の各元から一点ずつ選んで集めた$A$の有限部分集合を
		$F_V$とすれば,任意の$F \in \mathscr{F}_V$と$F$に属する$x \in F_V$で
		\begin{align}
			y \in F \quad \Longrightarrow \quad
			(x,y) \in F \times F \subset V \quad \Longrightarrow \quad
			y \in V_x
		\end{align}
		が成り立つから
		\begin{align}
			A = \bigcup \mathscr{F}_V \subset \bigcup_{x \in F_V} V_x
		\end{align}
		が従い$A$の$S$での全有界性が出る.
		\QED
	\end{prf}
	
	\begin{screen}
		\begin{thm}[全有界かつ可算な基本近縁系を持つ一様位相空間は可分かつ第二可算]
			
		\end{thm}
	\end{screen}
	
	\begin{screen}
		\begin{thm}[コンパクトなら全有界]\label{thm:compact_then_totally_bounded}
			$(S,\mathscr{V})$を一様空間とし,$\mathscr{V}$で$S$に位相を導入する.
			$A$を$S$の空でないコンパクト部分集合とするとき,$A$は全有界である.
		\end{thm}
	\end{screen}
	
	\begin{prf}
		任意の$V \in \mathscr{V}$に対し$\{V_x^{\mathrm{o}}\}_{x \in A}$は
		$A$の$S$における開被覆となるから,$A$がコンパクトであれば
		\begin{align}
			A \subset \bigcup_{x \in F_V} V_x^{\mathrm{o}} 
			\subset \bigcup_{x \in F_V} V_x
		\end{align}
		を満たす有限集合$F_V \subset A$が存在する.従って$A$は$S$で全有界である.
		\QED
	\end{prf}
	
	\begin{screen}
		\begin{thm}[完備かつ全有界$\Longleftrightarrow$コンパクト]
			$(S,\mathscr{V})$を一様空間として$\mathscr{V}$で一様位相を導入するとき,
			$S$の任意の空でない部分集合$A$に対し,
			\begin{align}
				\mbox{$A$が$S$で完備かつ全有界} \quad \Longleftrightarrow \quad
				\mbox{$A$がコンパクト}.
			\end{align}
		\end{thm}
	\end{screen}
	
	\begin{prf}\mbox{}
		\begin{description}
			\item[第一段]
				$\Longrightarrow$を示す.
				$A$が$S$で全有界なら,任意の$V \in \mathscr{V}$に対し
				$A$の空でない部分集合の有限族$\mathscr{F}_V$で
				\begin{align}
					A = \bigcup \mathscr{F}_V,\quad
					F \times F \subset V,\ (\forall F \in \mathscr{F}_V)
				\end{align}
				を満たすものが取れる.$\mathscr{F}_V$が生成する$A$の位相を$\tau_V$と書けば,
				Alexanderの定理より$(A,\tau_V)$はコンパクトとなり,
				Tychonoffの定理より$\left\{(A,\tau_V)\right\}_{V \in \mathscr{V}}$の
				直積位相空間$B$もまたコンパクトとなる.写像$\delta:A \longrightarrow B$を
				\begin{align}
					\delta(x)(V) = x,\quad (\forall V \in \mathscr{V},\ x \in A)
				\end{align}
				により定めれば,$A$上の任意の有向点族$(x_\lambda)_{\lambda \in \Lambda}$に対し
				$\left(\delta(x_\lambda)\right)_{\lambda \in \Lambda}$
				は$B$の有向点族となるから,$B$のコンパクト性より或る
				有向集合$(\Gamma,\preceq)$と共終かつ序列を保つ写像
				$f:\Gamma \longrightarrow \Lambda$,及び$b \in B$が存在して
				\begin{align}
					\delta(x_{f(\gamma)}) \longrightarrow b
				\end{align}
				となる.このとき$(x_{f(\gamma)})_{\gamma \in \Gamma}$は
				$A$上のCauchy有向点族をなす.実際,任意の$V \in \mathscr{V}$に対して
				$b(V) \in F$を満たす$F \in \mathscr{F}_V$を取れば
				\begin{align}
					b \in \operatorname{pr}_V^{-1}(F)
				\end{align}
				となり(ただし$\operatorname{pr}_V$は$B$から$(A,\tau_V)$への射影である),
				$\operatorname{pr}_V^{-1}(F)$は$b$の開近傍であるから
				或る$\gamma_V \in \Gamma$で
				\begin{align}
					\gamma_V \preceq \gamma
					\quad \Longrightarrow \quad
					\delta(x_{f(\gamma)}) \in \operatorname{pr}_V^{-1}(F)
					\quad \Longrightarrow \quad
					x_{f(\gamma)} \in F
				\end{align}
				が成り立つ.従って
				\begin{align}
					\gamma_V \preceq \gamma,\xi
					\quad \Longrightarrow \quad
					\left(x_{f(\gamma)},x_{f(\xi)}\right) \in F \times F \subset V
				\end{align}
				が得られる.いま$A$は$S$で完備であるから$(x_{f(\gamma)})_{\gamma \in \Gamma}$
				は$A$で収束し,定理\ref{thm:compact_iff_every_net_has_a_convergent_subnet}より$A$のコンパクト性が出る.
				
			\item[第二段]
				$A$がコンパクトであれば,定理\ref{thm:compact_then_totally_bounded}より
				$A$は全有界であり,また$A$上の任意のCauchy有向点族は$A$で収束する
				部分有向点族を持ち
				(定理\ref{thm:compact_iff_every_net_has_a_convergent_subnet}),
				このとき元の点族も$A$で収束する
				(定理\ref{thm:subnet_of_a_Cauchy_net_is_Cauchy}).
				\QED
		\end{description}
	\end{prf}
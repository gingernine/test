\subsection{一様空間}
	\begin{screen}
		\begin{thm}\label{thm:uniform_structure}
			$(S,\mathscr{V})$を一様空間とするとき,
			任意の$V \in \mathscr{V}$に対し
			\begin{align}
				W_x \times W_x \subset V,\quad (\forall x \in S)
			\end{align}
			を満たす対称な$W \in \mathscr{V}$が存在する.
			ただし$W_x = \Set{y \in S}{(x,y) \in W}$である.
		\end{thm}
	\end{screen}
	
	\begin{prf}
		近縁系の定義より$U \circ U \subset V$を満たす
		$U \in \mathscr{V}$が存在する.ここで
		\begin{align}
			W \coloneqq U \cap U^{-1}
		\end{align}
		で$W \in \mathscr{V}$を定めれば,$W$は対称であるので任意の$x \in S$に対し
		\begin{align}
			y,z \in W_x \quad \Longrightarrow \quad
			(x,y),(x,z) \in W \quad \Longrightarrow \quad
			(y,x),(x,z) \in W \quad \Longrightarrow \quad
			(y,z) \in V
		\end{align}
		が成立し$W_x \times W_x \subset V$が得られる.
		\QED
	\end{prf}
	
	\begin{screen}
		\begin{thm}[一様位相空間において$T_0 \Longleftrightarrow T_2$]
		\label{thm:T_0_iff_T_2_on_uniform_topological_space}
			$(S,\mathscr{V})$を一様空間とし,$S$に一様位相を導入する.このとき
			\begin{align}
				\mbox{$S$が$T_0$} \quad \Longleftrightarrow \quad
				\bigcap_{V \in \mathscr{V}}V = \Set{(x,x)}{x \in S}
				\quad \Longleftrightarrow \quad
				\mbox{$S$が$T_2$}
				\label{eq:thm_T_0_iff_T_2_on_uniform_topological_space}
			\end{align}
		\end{thm}
	\end{screen}
	
	\begin{prf} 位相空間が$T_2$なら$T_0$であるから,二つの$\Longrightarrow$を示せば
		(\refeq{eq:thm_T_0_iff_T_2_on_uniform_topological_space})が従う.
		\begin{description}
			\item[一つ目の$\Longrightarrow$]
				$\bigcap_{V \in \mathscr{V}}V \neq \Set{(x,x)}{x \in S}$
				が満たされるとき,或る相異なる二点$x,y \in S$に対し
				\begin{align}
					(x,y),(y,x) \in V, \quad (\forall V \in \mathscr{V})
				\end{align}
				となる.$\Set{V_x \coloneqq \Set{s \in S}{(x,s) \in V}}{V \in \mathscr{V}}$は$x$の基本近傍系をなすから
				\begin{align}
					y \in V_x, \quad (\forall V \in \mathscr{V})
				\end{align}
				が成立し,定理\ref{thm:belongs_to_closure_iff_clusters}より
				$x \in \overline{\{y\}}$が従う.
				対称的に$y \in \overline{\{x\}}$も出るから
				$x$と$y$は位相的に区別不能である.
				
			\item[二つ目の$\Longrightarrow$]
				$\bigcap_{V \in \mathscr{V}}V = \Set{(x,x)}{x \in S}$
				が満たされるとき,任意の相異なる二点$x,y \in S$に対し
				\begin{align}
					(x,y) \in V
				\end{align}
				を満たす$V \in \mathscr{V}$が存在する.
				定理\ref{thm:uniform_structure}より或る対称な$W \in \mathscr{V}$で
				\begin{align}
					W \circ W \subset V,
					\quad W_x \times W_x \subset V,
					\quad W_y \times W_y \subset V
				\end{align}
				となるが,このとき$W_x \cap W_y = \emptyset$が成り立つ.実際,
				$W_x \cap W_y$が空でないとき,$z \in W_x \cap W_y$を取れば
				\begin{align}
					(x,z),(y,z) \in W \quad \Longrightarrow \quad
					(x,z),(z,y) \in W \quad \Longrightarrow \quad
					(x,y) \in V
				\end{align}
				が従い矛盾が生じる.$W_x,W_y$はそれぞれ$x,y$の近傍であるから二つ目の$\Longrightarrow$を得る.
				\QED
		\end{description}
	\end{prf}
	
	\begin{screen}
		\begin{thm}[擬距離空間の一様構造]
		\label{thm:uniform_structure_on_pseudometric_spaces}
			$(S,d)$を擬距離空間とするとき,
			\begin{align}
				\mathscr{V} \coloneqq
				\Set{V(r)}{r > 0},
				\quad (V(r) \coloneqq \Set{(x,y) \in S \times S}{d(x,y) < r})
			\end{align}
			とおけば$\mathscr{V}$は$S$上の一様構造となり,
			$\mathscr{V}$で導入する一様位相は$d$-位相に一致する.
		\end{thm}
	\end{screen}
	
	\begin{screen}
		\begin{thm}[擬距離空間のCauchy列]
		\label{thm:Cauchy_sequences_on_pseudometric_spaces}
			$(S,d)$を擬距離空間とし,一様構造$\mathscr{V}$を
			定理\ref{thm:uniform_structure_on_pseudometric_spaces}の要領で定めるとき,
			$S$の任意の点列$(x_n)_{n \in \N}$に対し,$(x_n)_{n \in \N}$がCauchy列であることと
			\begin{align}
				\forall \epsilon > 0;\ 
				\exists N \in \N;\quad
				n,m \geq N \Longrightarrow d(x_n,x_m) < \epsilon
				\label{eq:thm_Cauchy_sequences_on_pseudometric_spaces}
			\end{align}
			が成り立つことは同値になる.
		\end{thm}
	\end{screen}
	
	\begin{prf}
		任意の$\epsilon$と$n,m \in \N$で
		\begin{align}
			(x_n,x_m) \in V(\epsilon) \quad \Longleftrightarrow \quad
			d(x_n,x_m) < \epsilon
		\end{align}
		となるから,$(x_n)_{n \in \N}$がCauchy列であるとき,任意の$\epsilon > 0$に対し
		或る$N \in \N$が存在して
		\begin{align}
			n,m \geq N \quad \Longrightarrow \quad
			(x_n,x_m) \in V(\epsilon) \quad \Longrightarrow \quad
			d(x_n,x_m) < \epsilon
		\end{align}
		が成り立つ.逆に$(x_n)_{n \in \N}$に対して
		(\refeq{eq:thm_Cauchy_sequences_on_pseudometric_spaces})が
		満たされているとき,任意の$V(\epsilon) \in \mathscr{V}$に対し
		或る$M \in \N$が存在して
		\begin{align}
			n,m \geq N \quad \Longrightarrow \quad
			d(x_n,x_m) < \epsilon \quad \Longrightarrow \quad
			(x_n,x_m) \in V(\epsilon)
		\end{align}
		となるから$(x_n)_{n \in \N}$はCauchy列である.
		\QED
	\end{prf}
	
	\begin{screen}
		\begin{thm}[点列の擬距離に関する収束]
			点列$(x_n)_{n \in \N}$が$a$に収束する
			ことと$d(x_n,a) \longrightarrow 0$は同値.
		\end{thm}
	\end{screen}
	
	\begin{screen}
		\begin{thm}[可算な基本近縁系が存在するとき,完備$\Longleftrightarrow$任意のCauchy列が収束する]
		\label{thm:complete_iff_every_Cauchy_seq_converges_if_entourage_contains_some_countable_subset}
			$(S,\mathscr{V})$を一様空間とする.
			$\mathscr{V}$に対して可算な基本近縁系$\{V_n\}_{n \in \N}$が存在するとき次が成立する:
			\begin{align}
				\mbox{$S$が完備である} \quad \Longleftrightarrow \quad
				\mbox{$S$の任意のCauchy列が収束する}.
			\end{align}
		\end{thm}
	\end{screen}
	
	\begin{prf}
		$\Longleftarrow$を示す.近縁系は有限交叉で閉じるから
		\begin{align}
			U_n \coloneqq V_1 \cap V_2 \cap \cdots \cap V_n,
			\quad (n = 1,2,\cdots)
		\end{align}
		により単調減少な$\mathscr{V}$の基本近縁系$\{U_n\}_{n \in \N}$が定まる.
		$(x_\lambda)_{\lambda \in \Lambda}$を$S$のCauchy有向点族として
		\begin{align}
			A_\lambda \coloneqq \Set{x_\mu}{\lambda \leq \mu},
			\quad (\forall \lambda \in \Lambda)
		\end{align}
		とおけば,任意の$n \in \N$で或る$\lambda_n \in \Lambda$が存在して
		\begin{align}
			A_{\lambda_n} \times A_{\lambda_n} \subset U_n
		\end{align}
		となる.任意の$V \in \mathscr{V}$に対し$W \circ W \subset V$
		を満たす$W \in \mathscr{V}$を取れば,或る$N \in \N$で$U_N \subset W$となるから
		\begin{align}
			U_N \circ U_N \subset V
		\end{align}
		が成り立つ.また任意の$n,m \geq N$に対し,有向集合の定義より
		$\lambda_n,\lambda_m \leq \mu$を満たす$\mu \in \Lambda$が存在して
		\begin{align}
			(x_{\lambda_n},x_\mu) \in U_n \subset U_N,
			\quad (x_\mu, x_{\lambda_m}) \in U_m \subset U_N
		\end{align}
		となり$(x_{\lambda_n},x_{\lambda_m}) \in V$が従うから,
		$(x_{\lambda_n})_{n \in \N}$はCauchy列であり或る$a \in S$に収束する.このとき
		\begin{align}
			\lim x_\lambda = a
			\label{eq:thm_complete_iff_every_Cauchy_seq_converges_if_entourage_contains_some_countable_subset}
		\end{align}
		が成立する.実際,任意に$a$の近傍$B$を取れば或る$\tilde{V} \in \mathscr{V}$で
		\begin{align}
			\tilde{V}_a \coloneqq \Set{x \in S}{(a,x) \in \tilde{V}} \subset B
		\end{align}
		となり,$\tilde{W} \circ \tilde{W} \subset V$を満たす$\tilde{W} \in \mathscr{V}$に対し
		或る$N_1 \in \N$が存在して
		\begin{align}
			n \geq N_1 \quad \Longrightarrow \quad
			x_{\lambda_n} \in \tilde{W}_a \quad \Longrightarrow \quad
			(a,x_{\lambda_n}) \in \tilde{W}
		\end{align}
		を満たす.また或る$N_2 \geq N_1$で$U_{N_2} \subset \tilde{W}$となるから
		\begin{align}
			A_{\lambda_{N_2}} \times A_{\lambda_{N_2}} \subset U_{N_2} \subset \tilde{W}
		\end{align}
		が従い,このとき$(a,x_{\lambda_{N_2}}) \in \tilde{W}$かつ
		$(x_{\lambda_{N_2}},x) \in \tilde{W},\ (\forall x \in A_{\lambda_{N_2}})$より
		$(a,x) \in \tilde{V},\ (\forall x \in A_{\lambda_{N_2}})$となるから
		\begin{align}
			A_{\lambda_{N_2}} \subset \tilde{V}_a 
		\end{align}
		が得られ(
		\refeq{eq:thm_complete_iff_every_Cauchy_seq_converges_if_entourage_contains_some_countable_subset})
		が出る.任意のCauchy有向点族が収束するから$S$は完備である.
		\QED
	\end{prf}
	
	\begin{screen}
		\begin{thm}[完備かつ全有界$\Longleftrightarrow$コンパクト]
			$(S,\mathscr{V})$を一様空間として$\mathscr{V}$で一様位相を導入するとき,
			\begin{align}
				\mbox{$S$が完備かつ全有界} \quad \Longleftrightarrow \quad
				\mbox{$S$がコンパクト}.
			\end{align}
		\end{thm}
	\end{screen}
	
	\begin{prf}\mbox{}
		\begin{description}
			\item[第一段]
				任意の有向点族が収束する部分有向点族を持てばコンパクトである.
				$(x_\lambda)_{\lambda \in \Lambda}$を$S$の有向点族とする.
				任意の$V \in \mathscr{V}$に対し或る$\{A_i\}_{i=1}^n$が存在して
				\begin{align}
					A_i \times A_i \subset U;\ (\forall i=1,\cdots,n),\quad
					\bigcup_{i=1}^n A_i = S
				\end{align}
				を満たす.この$\{A_i\}_{i=1}^n$が生成する$S$の位相を$\tau_V$とし,
				$\tau_V$を導入した$S$を$S_V$と書けば,$S_V$はコンパクトであるからTyconovの定理より
				\begin{align}
					T \coloneqq \prod_{V \in \mathscr{V}} S_V
				\end{align}
				はコンパクト空間であり,
			\item[第二段]
		\end{description}
	\end{prf}
\subsection{一様空間}
	\begin{screen}
		\begin{thm}\label{thm:uniform_structure}
			$(S,\mathscr{V})$を一様空間とするとき,
			任意の$V \in \mathscr{V}$に対し
			\begin{align}
				W_x \times W_x \subset V,\quad (\forall x \in S)
			\end{align}
			を満たす対称な$W \in \mathscr{V}$が存在する.
			ただし$W_x = \Set{y \in S}{(x,y) \in W}$である.
		\end{thm}
	\end{screen}
	
	\begin{prf}
		近縁系の定義より$U \circ U \subset V$を満たす
		$U \in \mathscr{V}$が存在する.ここで
		\begin{align}
			W \coloneqq U \cap U^{-1}
		\end{align}
		で$W \in \mathscr{V}$を定めれば,$W$は対称であるので任意の$x \in S$に対し
		\begin{align}
			y,z \in W_x \quad \Longrightarrow \quad
			(x,y),(x,z) \in W \quad \Longrightarrow \quad
			(y,x),(x,z) \in W \quad \Longrightarrow \quad
			(y,z) \in V
		\end{align}
		が成立し$W_x \times W_x \subset V$が得られる.
		\QED
	\end{prf}
	
	\begin{screen}
		\begin{thm}[一様位相空間において$T_0 \Longleftrightarrow T_2$]
		\label{thm:T_0_iff_T_2_on_uniform_topological_space}
			$(S,\mathscr{V})$を一様空間とし,$S$に一様位相を導入する.このとき
			\begin{align}
				\mbox{$S$が$T_0$} \quad \Longleftrightarrow \quad
				\bigcap_{V \in \mathscr{V}}V = \Set{(x,x)}{x \in S}
				\quad \Longleftrightarrow \quad
				\mbox{$S$が$T_2$}
				\label{eq:thm_T_0_iff_T_2_on_uniform_topological_space}
			\end{align}
		\end{thm}
	\end{screen}
	
	\begin{prf}\mbox{}
		\begin{description}
			\item[一つ目の$\Longrightarrow$]
				$\bigcap_{V \in \mathscr{V}}V \neq \Set{(x,x)}{x \in S}$
				が満たされるとき,或る相異なる二点$x,y \in S$に対し
				\begin{align}
					(x,y),(y,x) \in V, \quad (\forall V \in \mathscr{V})
				\end{align}
				となる.$\Set{V_x \coloneqq \Set{s \in S}{(x,s) \in V}}{V \in \mathscr{V}}$は$x$の基本近傍系をなすから
				\begin{align}
					y \in V_x, \quad (\forall V \in \mathscr{V})
				\end{align}
				が成立し,定理\ref{thm:belongs_to_closure_iff_clusters}より
				$x \in \overline{\{y\}}$が従う.
				対称的に$y \in \overline{\{x\}}$も出るから
				$x$と$y$は位相的に区別不能である.
				
			\item[二つ目の$\Longrightarrow$]
				$\bigcap_{V \in \mathscr{V}}V = \Set{(x,x)}{x \in S}$
				が満たされるとき,任意の相異なる二点$x,y \in S$に対し
				\begin{align}
					(x,y) \in V
				\end{align}
				を満たす$V \in \mathscr{V}$が存在する.
				定理\ref{thm:uniform_structure}より或る対称な$W \in \mathscr{V}$で
				\begin{align}
					W \circ W \subset V,
					\quad W_x \times W_x \subset V,
					\quad W_y \times W_y \subset V
				\end{align}
				となるが,このとき$W_x \cap W_y = \emptyset$が成り立つ.実際,
				$W_x \cap W_y$が空でないとき,$z \in W_x \cap W_y$を取れば
				\begin{align}
					(x,z),(y,z) \in W \quad \Longrightarrow \quad
					(x,z),(z,y) \in W \quad \Longrightarrow \quad
					(x,y) \in V
				\end{align}
				が従い矛盾が生じる.$W_x,W_y$はそれぞれ$x,y$の近傍であるから,
				二つ目の$\Longrightarrow$が得られた.
				位相空間が$T_2$なら$T_0$であるから
				(\refeq{eq:thm_T_0_iff_T_2_on_uniform_topological_space})が成り立つ.
				\QED
		\end{description}
	\end{prf}
	
	\begin{screen}
		\begin{thm}[擬距離空間の一様構造]
		\label{thm:uniform_structure_on_pseudometric_spaces}
			$(S,d)$を擬距離空間とするとき,
			\begin{align}
				\mathscr{V} \coloneqq
				\Set{V(r)}{r > 0},
				\quad (V(r) \coloneqq \Set{(x,y) \in S \times S}{d(x,y) < r})
			\end{align}
			とおけば$\mathscr{V}$は$S$上の一様構造となり,
			$\mathscr{V}$で導入する一様位相は$d$-位相に一致する.
		\end{thm}
	\end{screen}
	
	\begin{screen}
		\begin{thm}[擬距離空間のCauchy列]
		\label{thm:Cauchy_sequences_on_pseudometric_spaces}
			$(S,d)$を擬距離空間とし,一様構造$\mathscr{V}$を
			定理\ref{thm:uniform_structure_on_pseudometric_spaces}の要領で定めるとき,
			$S$の任意の点列$(x_n)_{n \in \N}$に対し,$(x_n)_{n \in \N}$がCauchy列であることと
			\begin{align}
				\forall \epsilon > 0;\ 
				\exists N \in \N;\quad
				n,m \geq N \Longrightarrow d(x_n,x_m) < \epsilon
				\label{eq:thm_Cauchy_sequences_on_pseudometric_spaces}
			\end{align}
			が成り立つことは同値になる.
		\end{thm}
	\end{screen}
	
	\begin{prf}
		任意の$\epsilon$と$n,m \in \N$で
		\begin{align}
			(x_n,x_m) \in V(\epsilon) \quad \Longleftrightarrow \quad
			d(x_n,x_m) < \epsilon
		\end{align}
		となるから,$(x_n)_{n \in \N}$がCauchy列であるとき,任意の$\epsilon > 0$に対し
		或る$N \in \N$が存在して
		\begin{align}
			n,m \geq N \quad \Longrightarrow \quad
			(x_n,x_m) \in V(\epsilon) \quad \Longrightarrow \quad
			d(x_n,x_m) < \epsilon
		\end{align}
		が成り立つ.逆に$(x_n)_{n \in \N}$に対して
		(\refeq{eq:thm_Cauchy_sequences_on_pseudometric_spaces})が
		満たされているとき,任意の$V(\epsilon) \in \mathscr{V}$に対し
		或る$M \in \N$が存在して
		\begin{align}
			n,m \geq N \quad \Longrightarrow \quad
			d(x_n,x_m) < \epsilon \quad \Longrightarrow \quad
			(x_n,x_m) \in V(\epsilon)
		\end{align}
		となるから$(x_n)_{n \in \N}$はCauchy列である.
		\QED
	\end{prf}
	
	\begin{screen}
		\begin{thm}[擬距離空間において,完備$\Longleftrightarrow$任意のCauchy列が収束する]
			$(S,d)$を擬距離空間として一様構造$\mathscr{V}$を
			定理\ref{thm:uniform_structure_on_pseudometric_spaces}の要領で定めるとき,
			\begin{align}
				\mbox{$S$が完備である} \quad \Longleftrightarrow \quad
				\mbox{$S$の任意のCauchy列が収束する}.
			\end{align}
		\end{thm}
	\end{screen}
	
	\begin{prf}
		$\Longleftarrow$を示せばよい.任意のCauchy列が収束するとき,$S$のCauchy有向点族
		$(x_\lambda)_{\lambda \in \Lambda}$に対し
		\begin{align}
			A_\lambda \coloneqq \Set{x_\mu}{\lambda \leq \mu}
		\end{align}
		とおけば,任意の$n \in \N$で
		\begin{align}
			A_{\lambda_n} \times A_{\lambda_n} \subset V(1/n)
		\end{align}
		を満たす$\lambda_n \in \Lambda$が存在する.
		このとき任意の$\epsilon > 0$に対し
		$1/N < \epsilon$となる$N \in \N$を取れば,
		$n,m \geq N$なら
		\begin{align}
			A_{\lambda_n} \times A_{\lambda_n},\
			A_{\lambda_m} \times A_{\lambda_m}
			\subset V(1/N) \subset V(\epsilon)
		\end{align}
		となり,有向集合の定義より$\lambda_n,\lambda_m \leq \mu$を満たす
		$\mu \in \Lambda$が存在して
		\begin{align}
			d(x_{\lambda_n},x_{\lambda_m})
			\leq d(x_{\lambda_n},x_\mu) + d(x_\mu,x_{\lambda_m})
			< 2\epsilon
		\end{align}
		が成り立つから,$(x_{\lambda_n})_{n \in \N}$は$S$のCauchy列であり
		或る$x_0 \in S$に収束する.このとき任意の$x \in A_{\lambda_n}$に対し
		\begin{align}
			d(x_0,x) \leq d(x_0,x_{\lambda_n}) + d(x_{\lambda_n},x)
			< d(x_0,x_{\lambda_n}) + \frac{1}{n}
		\end{align}
		となるから,各$n \in \N$で$r_n \coloneqq d(x_0,x_{\lambda_n}) + 1/n$
		とおけば
		\begin{align}
			A_{\lambda_n} \subset B_{r_n}(x_0)
			\coloneqq \Set{x \in S}{d(x_0,x) < r_n}
		\end{align}
		が成り立つ.$r_n \longrightarrow 0$より$B_{r_n}(x_0)$の全体は
		$x_0$の基本近傍系をなすから,$x_0$の任意の近傍$U$に対し
		\begin{align}
			A_{\lambda_n} \subset B_{r_n}(x_0) \subset U
		\end{align}
		を満たす$n \in \N$が取れる.すなわち
		$(x_\lambda)$は$x_0$に収束する.$S$の任意の有向点族が収束するから$S$は完備である.
		\QED
	\end{prf}
\subsection{ノルム}
	$\left(\left(X,\sigma_X\right),(\Phi,+,\bullet),s,\mathscr{O}_X\right)$を位相線型空間とし,
	$d$を$X$上の左不変擬距離とするとき,$\sigma_X$は可換であるから
	$X$の任意の要素$x,y,a$に対して
	\begin{align}
		d\left(x,y\right) = d\left(\sigma_X\left(x,a\right),\sigma_X\left(y,a\right)\right)
	\end{align}
	も成立する.つまり$d$は右不変でもあるから,位相線型空間においては左不変擬距離を
	{\bf 不変擬距離}\index{ふへんぎきょり@不変擬距離}{\bf (invariant pseudo metric)}と呼ぶ.
	
	\begin{screen}
		\begin{dfn}[$F$-空間]
			$\left(\left(X,\sigma_X\right),(\Phi,+,\bullet),s,\mathscr{O}_X\right)$を位相線型空間とする.
			$\left(X,\sigma_X\right)$に対して不変距離$d$が取れて,
			$\left(X,\mathscr{O}_X\right)$が$d$により距離化可能であり,
			かつ$(X,d)$が完備距離空間であるとき,$\left(\left(X,\sigma_X\right),(\Phi,+,\bullet),s,\mathscr{O}_X\right)$を
			{\bf $F$-空間}\index{Fくうかん@$F$-空間}{\bf ($F$-space)}と呼ぶ.
		\end{dfn}
	\end{screen}
	
	$\C$において絶対値は
	\begin{align}
		|\alpha \cdot x| = |\alpha| \cdot |x|
	\end{align}
	および
	\begin{align}
		|x + y| \leq |x| + |y|
	\end{align}
	を満たす関数であり,また
	\begin{align}
		\C \times \C \ni (x,y) \longmapsto |x - y|
	\end{align}
	なる関係は$\C$上の距離である.しかも二元の距離は
	\begin{align}
		x - y
	\end{align}
	という差によって定められているため,この距離は不変距離である.
	一般の線型空間においても絶対値と類似した性質を満たす写像が付与されれば不変(擬)距離が定められる.
	
	\begin{screen}
		\begin{dfn}[ノルム]
			$\left(\left(X,\sigma_X\right),(\Phi,+,\bullet),s\right)$を線型空間とし,
			$0_X$を$\left(X,\sigma_X\right)$の単位元とするとき,
			\begin{itemize}
				\item $p$は$X$上の実数値写像である:
					\begin{align}
						p:X \longrightarrow \R.
					\end{align}
					
				\item $p$は{\bf 絶対斉次的である}\index{ぜったいせいじてき@絶対斉次的}{\bf (absolutely homogeneous)}:
					\begin{align}
						\forall x \in X\, \forall \alpha \in \Phi\, 
						\left[\, p\left(s(\alpha,x)\right) = |\alpha| \cdot p(x)\, \right].
					\end{align}
					
				\item $p$は劣加法的である:
					\begin{align}
						\forall x,y \in X\, \left[\, p\left(\sigma_X\left(x,y\right)\right) 
						\leq p(x) + p(y)\, \right].
					\end{align}
			\end{itemize}
			を満たす集合$p$を$\left(\left(X,\sigma_X\right),(\Phi,+,\bullet),s\right)$の
			{\bf セミノルム}\index{セミノルム}{\bf (semi norm)}と呼ぶ.この$p$が
			\begin{align}
				\forall x \in X\, \left[\, p(x) = 0 \Longrightarrow x = 0_X\, \right]
			\end{align}
			を満たすとき,$p$を$\left(\left(X,\sigma_X\right),(\Phi,+,\bullet),s\right)$の
			{\bf ノルム}\index{ノルム}{\bf (norm)}と呼ぶ.
		\end{dfn}
	\end{screen}
	
	$\left(\left(X,\sigma_X\right),(\Phi,+,\bullet),s\right)$を線型空間とし,$p$をセミノルムとし,
	\begin{align}
		X \times X \ni (x,y) \longmapsto p(x-y)
	\end{align}
	なる関係を$d$と定めると,$d$は$X$上の擬距離である.$p$がノルムであれば$d$は距離である.
	
	\begin{screen}
		\begin{thm}[絶対斉次的な不変距離はノルムで導入する距離に限られる]
			ノルムで導入する距離は絶対斉次的か不変であり,
			かつそのような距離はノルムで導入する距離に限られる.
		\end{thm}
	\end{screen}
	
	\begin{prf}
		$\Norm{\cdot}{}$を線型空間$X$のノルムとするとき,
		\begin{align}
			d(x,y) \defeq \Norm{x-y}{}, \quad (\forall x,y \in X)
		\end{align}
		で距離を定めれば
		\begin{align}
			d(x+z,y+z) = \Norm{x+z-(y+z)}{} = \Norm{x-y}{} = d(x,y),
			\quad d(\alpha x, \alpha y)
			= \Norm{\alpha (x-y)}{} = |\alpha|\Norm{x-y}{} = |\alpha|d(x,y)
		\end{align}
		が成立する.逆に$X$上の距離$d$が絶対斉次的かつ平行移動不変であるとき,
		\begin{align}
			\Norm{x}{} \defeq d(x,0),\quad (\forall x \in X)
		\end{align}
		でノルムが定まる.実際$\Norm{\alpha x}{} = d(\alpha x,0) 
		= |\alpha|d(x,0) = |\alpha|\Norm{x}{}$かつ
		\begin{align}
			\quad \Norm{x+y}{} = d(x+y,0) = d(x,-y) 
			\leq d(x,0) + d(0,-y) = d(x,0) + d(y,0) = \Norm{x}{} + \Norm{y}{}
		\end{align}
		が成立する.
		\QED
	\end{prf}
	
	\begin{screen}
		\begin{thm}[ノルムで導入する距離位相は線型位相]
			$(X,\Norm{\cdot}{})$をノルム空間とするとき,
			$d(x,y) \defeq \Norm{x-y}{}$で定める距離$d$による距離位相は線型位相となる.
		\end{thm}
	\end{screen}
	
	\begin{prf}
		距離位相は$T_6$位相空間を定めるから$X$は定義\ref{def:topological_vector_space}の(tvs2)を満たす.また
		\begin{align}
			d(x+y,x'+y') \leq d(x+y,x'+y) + d(x'+y,x'+y') = d(x,x') + d(y,y')
		\end{align}
		より加法の連続性が得られ,
		\begin{align}
			d(\alpha x, \alpha'x') &\leq d(\alpha x, \alpha'x) + d(\alpha'x,\alpha'x') \\
			&= d((\alpha - \alpha') x, 0) + |\alpha'|d(x,x')
			= |\alpha-\alpha'|d(x,0) + |\alpha'|d(x,x')
		\end{align}
		よりスカラ倍の連続性も出る.
		\QED
	\end{prf}
	
	\begin{screen}
		\begin{thm}[ノルム空間において距離的な有界性と位相的な有界性は一致する]
			$(X,\Norm{}{})$をノルム空間とすると,
			$X$の部分集合のノルムに関する有界性と$\tau$-有界性は一致する.
		\end{thm}
	\end{screen}
	
	\begin{prf}
		任意の$\alpha>0,\ \delta>0$に対し,
		$B_{\delta}(0) \defeq \Set{x \in X}{d(x,0) < \delta}$とおけば斉次性より
		\begin{align}
			x \in \alpha B_{\delta}(0) 
			\quad \Longleftrightarrow \quad d\left( \alpha^{-1}x,0 \right) < \delta
			\quad \Longleftrightarrow \quad \alpha^{-1}d(x,0) < \delta
			\quad \Longleftrightarrow \quad x \in B_{\alpha\delta}(0)
		\end{align}
		が成立する.$\{B_r(0)\}_{r > 0}$は$X$の局所基となるから,
		$E \subset X$が$d$-有界のときも$\tau$-有界のときも
		$E \subset B_R(0)$を満たす$R > 0$が存在する.
		$E$が$d$-有界集合である場合,任意に0の近傍$V$を取れば
		或る$r > 0$で$B_r(0) \subset V$となり
		\begin{align}
			E \subset B_R(0) \subset B_t(0) = \frac{t}{r} B_r(0) \subset \frac{t}{r}V,
			\quad (\forall t > R)
		\end{align}
		が成立するから$E$は$\tau$-有界集合である.
		逆に$E$が$\tau$-有界集合であるとき,任意に$x \in X$を取れば
		\begin{align}
			E \subset B_R(0) \subset B_{d(x,0) + R}(x)
		\end{align}
		が成立するから$E$は$d$-有界集合である.
		\QED
	\end{prf}
	
\subsection{ノルム}
	$\left(\left(X,\sigma_X\right),(\Phi,+,\bullet),s,\mathscr{O}_X\right)$を位相線型空間とし,
	$d$を$X$上の左不変擬距離とするとき,$\sigma_X$は可換であるから
	$X$の任意の要素$x,y,a$に対して
	\begin{align}
		d\left(x,y\right) = d\left(\sigma_X\left(x,a\right),\sigma_X\left(y,a\right)\right)
	\end{align}
	も成立する.つまり$d$は右不変でもあるから,位相線型空間においては左不変擬距離を
	{\bf 不変擬距離}\index{ふへんぎきょり@不変擬距離}{\bf (invariant pseudo metric)}と呼ぶ.
	
	\begin{screen}
		\begin{thm}[絶対斉次的な不変距離はノルムで導入する距離に限られる]
			ノルムで導入する距離は絶対斉次的かつ平行移動不変であり,
			かつそのような距離はノルムで導入する距離に限られる.
		\end{thm}
	\end{screen}
	
	\begin{prf}
		$\Norm{\cdot}{}$を線型空間$X$のノルムとするとき,
		\begin{align}
			d(x,y) \defeq \Norm{x-y}{}, \quad (\forall x,y \in X)
		\end{align}
		で距離を定めれば
		\begin{align}
			d(x+z,y+z) = \Norm{x+z-(y+z)}{} = \Norm{x-y}{} = d(x,y),
			\quad d(\alpha x, \alpha y)
			= \Norm{\alpha (x-y)}{} = |\alpha|\Norm{x-y}{} = |\alpha|d(x,y)
		\end{align}
		が成立する.逆に$X$上の距離$d$が絶対斉次的かつ平行移動不変であるとき,
		\begin{align}
			\Norm{x}{} \defeq d(x,0),\quad (\forall x \in X)
		\end{align}
		でノルムが定まる.実際$\Norm{\alpha x}{} = d(\alpha x,0) 
		= |\alpha|d(x,0) = |\alpha|\Norm{x}{}$かつ
		\begin{align}
			\quad \Norm{x+y}{} = d(x+y,0) = d(x,-y) 
			\leq d(x,0) + d(0,-y) = d(x,0) + d(y,0) = \Norm{x}{} + \Norm{y}{}
		\end{align}
		が成立する.
		\QED
	\end{prf}
	
	\begin{screen}
		\begin{thm}[ノルムで導入する距離位相は線型位相]
			$(X,\Norm{\cdot}{})$をノルム空間とするとき,
			$d(x,y) \defeq \Norm{x-y}{}$で定める距離$d$による距離位相は線型位相となる.
		\end{thm}
	\end{screen}
	
	\begin{prf}
		距離位相は$T_6$位相空間を定めるから$X$は定義\ref{def:topological_vector_space}の(tvs2)を満たす.また
		\begin{align}
			d(x+y,x'+y') \leq d(x+y,x'+y) + d(x'+y,x'+y') = d(x,x') + d(y,y')
		\end{align}
		より加法の連続性が得られ,
		\begin{align}
			d(\alpha x, \alpha'x') &\leq d(\alpha x, \alpha'x) + d(\alpha'x,\alpha'x') \\
			&= d((\alpha - \alpha') x, 0) + |\alpha'|d(x,x')
			= |\alpha-\alpha'|d(x,0) + |\alpha'|d(x,x')
		\end{align}
		よりスカラ倍の連続性も出る.
		\QED
	\end{prf}
	
	\begin{screen}
		\begin{thm}[ノルム空間において距離的な有界性と位相的な有界性は一致する]
			位相線型空間$(X,\tau)$が斉次的な距離$d$で距離化可能である場合,
			$X$の部分集合の$d$-有界性と$\tau$-有界性は一致する.
		\end{thm}
	\end{screen}
	
	\begin{prf}
		任意の$\alpha>0,\ \delta>0$に対し,
		$B_{\delta}(0) \defeq \Set{x \in X}{d(x,0) < \delta}$とおけば斉次性より
		\begin{align}
			x \in \alpha B_{\delta}(0) 
			\quad \Longleftrightarrow \quad d\left( \alpha^{-1}x,0 \right) < \delta
			\quad \Longleftrightarrow \quad \alpha^{-1}d(x,0) < \delta
			\quad \Longleftrightarrow \quad x \in B_{\alpha\delta}(0)
		\end{align}
		が成立する.$\{B_r(0)\}_{r > 0}$は$X$の局所基となるから,
		$E \subset X$が$d$-有界のときも$\tau$-有界のときも
		$E \subset B_R(0)$を満たす$R > 0$が存在する.
		$E$が$d$-有界集合である場合,任意に0の近傍$V$を取れば
		或る$r > 0$で$B_r(0) \subset V$となり
		\begin{align}
			E \subset B_R(0) \subset B_t(0) = \frac{t}{r} B_r(0) \subset \frac{t}{r}V,
			\quad (\forall t > R)
		\end{align}
		が成立するから$E$は$\tau$-有界集合である.
		逆に$E$が$\tau$-有界集合であるとき,任意に$x \in X$を取れば
		\begin{align}
			E \subset B_R(0) \subset B_{d(x,0) + R}(x)
		\end{align}
		が成立するから$E$は$d$-有界集合である.
		\QED
	\end{prf}
	
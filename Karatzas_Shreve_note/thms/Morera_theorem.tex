\subsection{Moreraの定理}
	Moreraの定理とはGoursatの定理の逆の主張である.
	
	\begin{screen}
		\begin{thm}[正則ならば級数展開可能]
			$\gamma$を$[0,1]$上の路とし,$\gamma$で作る$\borel{[0,1]}$上の複素Stieltjes測度を
			$\mu_{\gamma}$とする.また$\varphi$を$[0,1]$上の$\C$値連続関数とする.このとき,
			$a$を$\C \backslash \ran{\gamma}$の要素として
			\begin{align}
				r \defeq \inf{}{\Set{|a-\gamma(t)|}{t \in [0,1]}}
			\end{align}
			とおくと,$\disc{a}{r}$の任意の要素$z$で
			\begin{align}
				\int_{[0,1]} \frac{\varphi}{\gamma - z}\ d\mu_{\gamma}
				= \sum_{n=0}^{\infty} \left[ \int_{[0,1]} \frac{\varphi}{(\gamma - a)^{n+1}}\ d\mu_{\gamma} \right] \cdot (z-a)^{n}
			\end{align}
			が成立する.
		\end{thm}
	\end{screen}
	
	定理より
	
	\begin{sketch}
		いま$z$を$\disc{a}{r}$の要素とする.このとき$[0,1]$の任意の要素$t$で
		\begin{align}
			\left|\frac{z-a}{|\gamma(t) - a|}\right|
			\leq \frac{|z-a|}{r}
			< 1
		\end{align}
		が成り立つので
		\begin{align}
			\frac{\varphi(t)}{\gamma(t) - a} = 
		\end{align}
	\end{sketch}
	
	\begin{screen}
		\begin{thm}[正則関数の導関数も正則]
			
		\end{thm}
	\end{screen}
	
	\begin{screen}
		\begin{thm}[Moreraの定理]
			
		\end{thm}
	\end{screen}
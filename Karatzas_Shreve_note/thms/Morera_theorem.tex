\subsection{Moreraの定理}
	Moreraの定理とはGoursatの定理の逆の主張である.
	
	\begin{screen}
		\begin{thm}[正則ならば級数展開可能]\label{thm:holomorphic_then_expanded}
			$\gamma$を$[\alpha,\beta]$上の路とし,$\gamma$で作る$\borel{[\alpha,\beta]}$上の複素Stieltjes測度を
			$\mu_{\gamma}$とする.また$\varphi$を$[\alpha,\beta]$上の$\C$値連続関数とする.このとき,
			$a$を$\C \backslash \ran{\gamma}$の要素とし,$r$を
			\begin{align}
				\disc{a}{r} \subset \C \backslash \ran{\gamma}
			\end{align}
			を満たす正の実数とすると,$\disc{a}{r}$の任意の要素$z$で
			\begin{align}
				\int_{[\alpha,\beta]} \frac{\varphi}{\gamma - z}\ d\mu_{\gamma}
				= \sum_{n=0}^{\infty} \left[ \int_{[\alpha,\beta]} \frac{\varphi}{(\gamma - a)^{n+1}}\ d\mu_{\gamma} \right] \cdot (z-a)^{n}
			\end{align}
			が成立する.
		\end{thm}
	\end{screen}
	
	\begin{sketch}
		いま$z$を$\disc{a}{r}$の要素とする.このとき$[\alpha,\beta]$の任意の要素$t$で
		\begin{align}
			\left|\frac{z-a}{\gamma(t) - a}\right|
			\leq \frac{|z-a|}{r}
			< 1
		\end{align}
		が成り立つので
		\begin{align}
			\frac{\varphi(t)}{\gamma(t) - a} = \varphi(t) \cdot \sum_{n=0}^{\infty} \frac{(z-a)^{n}}{(\gamma(t) - a)^{n+1}}
		\end{align}
		が成立する.また$[\alpha,\beta]$の任意の要素$t$で
		\begin{align}
			\sum_{n=0}^{\infty} \frac{|z-a|^{n}}{|\gamma(t) - a|^{n+1}}
			\leq \sum_{n=0}^{\infty} \frac{|z-a|^{n}}{r^{n+1}}
			= \frac{1}{r - |z-a|}
		\end{align}
		が成り立つので,Lebesgueの収束定理より
		\begin{align}
			\int_{[\alpha,\beta]} \frac{\varphi}{\gamma - z}\ d\mu_{\gamma}
			&= \int_{[\alpha,\beta]} \varphi \cdot \sum_{n=0}^{\infty} \frac{(z-a)^{n}}{(\gamma - a)^{n+1}}\ d\mu_{\gamma} \\
			&= \sum_{n=0}^{\infty} \int_{[\alpha,\beta]} (z-a)^{n} \cdot \frac{\varphi}{(\gamma - a)^{n+1}}\ d\mu_{\gamma} \\
			&= \sum_{n=0}^{\infty} \left[ \int_{[\alpha,\beta]} \frac{\varphi}{(\gamma - a)^{n+1}}\ d\mu_{\gamma} \right] \cdot (z-a)^{n}
		\end{align}
		が成立する.
		\QED
	\end{sketch}
		
	\begin{screen}
		\begin{thm}[Liouvilleの定理]\label{thm:Liouville_theorem}
			$\Omega$を$\C$の開集合とし,$f$を$\Omega$上の正則関数とし,
			$a$を$\Omega$の要素とし,$r$を
			\begin{align}
				\disc{a}{r} \subset \Omega
			\end{align}
			を満たす正の実数とする.このとき,
			\begin{align}
				[0,2\cdot\pi] \ni \theta \longmapsto a + r \cdot e^{\isym \cdot \theta}
			\end{align}
			なる写像を$c$とすれば,$\disc{a}{r}$の各要素$z$で
			\begin{align}
				f(z) = \sum_{n=0}^{\infty} \left[ \frac{1}{2\cdot\pi\cdot\isym} \cdot \int_{[0,2\cdot\pi]} \frac{f \circ c}{(c - a)^{n+1}}\ d\mu_{c} \right] \cdot (z-a)^{n}
			\end{align}
			が成立する.特に{\bf 有界な整関数は定数関数である.}
		\end{thm}
	\end{screen}
	
	\begin{sketch}\mbox{}
		\begin{description}
			\item[第一段]
				定理の設定の下で,定理\ref{thm:integral_formula_on_a_circle}より
				$\disc{a}{r}$の任意の要素$z$で
				\begin{align}
					f(z) = \frac{1}{2\cdot\pi\cdot\isym} \cdot \int_{[0,2\cdot\pi]} \frac{f \circ c}{c - z}\ d\mu_{c}
				\end{align}
				が成立する.他方で定理\ref{thm:holomorphic_then_expanded}より,
				$\disc{a}{r}$の任意の要素$z$で
				\begin{align}
					\int_{[0,2\cdot\pi]} \frac{f \circ c}{c - z}\ d\mu_{c}
					= \sum_{n=0}^{\infty} \left[ \int_{[0,2\cdot\pi]} \frac{f \circ c}{(c - a)^{n+1}}\ d\mu_{c} \right] \cdot (z-a)^{n}
				\end{align}
				が成立する.よって$\disc{a}{r}$の任意の要素$z$で
				\begin{align}
					f(z) = \sum_{n=0}^{\infty} \left[ \frac{1}{2\cdot\pi\cdot\isym} \cdot \int_{[0,2\cdot\pi]} \frac{f \circ c}{(c - a)^{n+1}}\ d\mu_{c} \right] \cdot (z-a)^{n}
				\end{align}
				が成立する.
			
			\item[第二段]
				$f$を有界な整関数とし,
				\begin{align}
					M \defeq \sup{z \in \C}{|f(z)|}
				\end{align}
				とおく.
				\begin{align}
					[0,2\cdot\pi] \ni \theta \longmapsto e^{\isym \cdot \theta}
				\end{align}
				なる写像を$c_{1}$とすれば,第一段の結果より$\disc{0}{1}$の任意の要素$z$で
				\begin{align}
					f(z) = \sum_{n=0}^{\infty} \left[ \frac{1}{2\cdot\pi\cdot\isym} \cdot \int_{[0,2\cdot\pi]} \frac{f \circ c_{1}}{c_{1}^{n+1}}\ d\mu_{c_{1}} \right] \cdot z^{n}
				\end{align}
				が成立する.また$R$を$1$より大きい任意の正数として
				\begin{align}
					[0,2\cdot\pi] \ni \theta \longmapsto R \cdot e^{\isym \cdot \theta}
				\end{align}
				なる写像を$c_{R}$とすれば,$\disc{0}{R}$の任意の要素$z$で
				\begin{align}
					f(z) = \sum_{n=0}^{\infty} \left[ \frac{1}{2\cdot\pi\cdot\isym} \cdot \int_{[0,2\cdot\pi]} \frac{f \circ c_{R}}{c_{R}^{n+1}}\ d\mu_{c_{R}} \right] \cdot z^{n}
				\end{align}
				が成立する.係数の一意性から任意の自然数$n$で
				\begin{align}
					\frac{1}{2\cdot\pi\cdot\isym} \cdot \int_{[0,2\cdot\pi]} \frac{f \circ c_{1}}{c_{1}^{n+1}}\ d\mu_{c_{1}}
					= \frac{1}{2\cdot\pi\cdot\isym} \cdot \int_{[0,2\cdot\pi]} \frac{f \circ c_{R}}{c_{R}^{n+1}}\ d\mu_{c_{R}}
				\end{align}
				が成り立つが,このとき
				\begin{align}
					\left|\frac{1}{2\cdot\pi\cdot\isym} \cdot \int_{[0,2\cdot\pi]} \frac{f \circ c_{1}}{c_{1}^{n+1}}\ d\mu_{c_{1}}\right|
					&\leq \frac{1}{2\cdot\pi} \cdot \int_{[0,2\cdot\pi]} \frac{\left|f \circ c_{R}\right|}{\left|c_{R}\right|^{n+1}}\ d\left|\mu_{c_{R}}\right| \\
					&\leq \frac{M}{R^{n+1}} 
				\end{align}
				が成立し,$R$の任意性から
				\begin{align}
					\frac{1}{2\cdot\pi\cdot\isym} \cdot \int_{[0,2\cdot\pi]} \frac{f \circ c_{1}}{c_{1}^{n+1}}\ d\mu_{c_{1}}
					= 0
				\end{align}
				が従う.ゆえに$\disc{0}{1}$の任意の要素$z$で
				\begin{align}
					f(z) = \frac{1}{2\cdot\pi\cdot\isym} \cdot \int_{[0,2\cdot\pi]} \frac{f \circ c_{1}}{c_{1}}\ d\mu_{c_{1}}
				\end{align}
				が成立する.ただし係数の一意性から
				\begin{align}
					f(z) = \frac{1}{2\cdot\pi\cdot\isym} \cdot \int_{[0,2\cdot\pi]} \frac{f \circ c_{1}}{c_{1}}\ d\mu_{c_{1}}
				\end{align}
				は任意の複素数$z$で成立する.
				\QED
		\end{description}
	\end{sketch}
	
	\begin{screen}
		\begin{thm}[正則関数の導関数も正則]\label{thm:derivative_of_holomorphic_function_is_holomorphic}
			$\Omega$を$\C$の開集合とするとき
			\begin{align}
				\forall f\, \left[\, f \in \Holomorphic{\Omega} \Longrightarrow f' \in \Holomorphic{\Omega}\, \right].
			\end{align}
		\end{thm}
	\end{screen}
	
	\begin{sketch}
		$f$を$\Omega$上の正則関数とし,$\Omega$の要素$a$を任意に取る.また
		\begin{align}
			\disc{a}{r} \subset \Omega
		\end{align}
		を満たす正の実数$r$を取って
		\begin{align}
			[0,2\cdot\pi] \ni \theta \longmapsto a + r \cdot e^{\isym \cdot \theta}
		\end{align}
		なる写像を$c$とする.このとき定理\ref{thm:Liouville_theorem}より$\disc{a}{r}$の任意の要素$z$で
		\begin{align}
			f(z) = \sum_{n=0}^{\infty} \left[ \frac{1}{2\cdot\pi\cdot\isym} \cdot \int_{[0,2\cdot\pi]} \frac{f \circ c}{(c - a)^{n+1}}\ d\mu_{c} \right] \cdot (z-a)^{n}
		\end{align}
		が成立する.よって定理\ref{thm:series_expanded_then_differentiable}より$\disc{a}{r}$の任意の要素$z$で
		\begin{align}
			f'(z) = \sum_{n=0}^{\infty} \left[ \frac{n+1}{2\cdot\pi\cdot\isym} \cdot \int_{[0,2\cdot\pi]} \frac{f \circ c}{(c - a)^{n+2}}\ d\mu_{c} \right] \cdot (z-a)^{n}
		\end{align}
		が成立し,$f'$の$a$での微分可能性が従う.$a$の任意性から
		\begin{align}
			f' \in \Holomorphic{\Omega}
		\end{align}
		が従う.
		\QED
	\end{sketch}
	
	\begin{screen}
		\begin{thm}[Moreraの定理]
			$\Omega$を$\C$の開集合とし,$f$を$\Omega$上の$\C$値連続関数とする.
			また$\Omega$から任意に三要素$a,b,c$が与えられたとき,この三点が作る三角集合が$\Omega$に含まれるなら,つまり
			\begin{align}
				\Set{z}{\exists t,s \in [0,1]\, 
				\left(\, z = (1-t) \cdot a 
				+ t \cdot (1-s) \cdot b 
				+ t \cdot s \cdot c\, \right)}
				\subset \Omega
			\end{align}
			であるならば
			\begin{align}
				\int_{\seg{a}{b}} f + \int_{\seg{b}{c}} f + \int_{\seg{c}{a}} f = 0
			\end{align}
			が成り立つとする.このとき$f$は$\Omega$上の正則関数である.
		\end{thm}
	\end{screen}
	
	\begin{sketch}
		いま$z$を$\Omega$の要素とし,
		\begin{align}
			\disc{z}{r} \subset \Omega
		\end{align}
		を満たす正の実数$r$を取る.このとき,定理\ref{thm:a_holomorphic_function_is_derivative_of_some_holomorphic_function_on_convex_open}より
		\begin{align}
			F \in \Holomorphic{\disc{z}{r}}
		\end{align}
		かつ
		\begin{align}
			F' = f|_{\disc{z}{r}}
		\end{align}
		を満たす$F$が取れて,さらに定理\ref{thm:derivative_of_holomorphic_function_is_holomorphic}より
		\begin{align}
			f|_{\disc{z}{r}} \in \Holomorphic{\disc{z}{r}}
		\end{align}
		が従う.ゆえに$f$は$z$で微分可能である.$z$の任意性から
		\begin{align}
			f \in \Holomorphic{\Omega}
		\end{align}
		が成立する.
		\QED
	\end{sketch}
	
\subsection{級数展開}
	
	先ず冪について,$z$を$0$でない複素数とし,$\alpha$を複素数とするとき,本稿では
	\begin{align}
		z^{\alpha} \defeq \exp{(\alpha \cdot \pvlog{z})}
	\end{align}
	と定める.また$z$が$0$であるときは
	\begin{align}
		z^{\alpha} \defeq 0
	\end{align}
	と定める.
	
	\begin{screen}
		\begin{dfn}[冪根]
			$x$を正の実数とし,$n$を$1$以上の自然数とするとき,
			\begin{align}
				\sqrt[n]{x} \defeq e^{\frac{1}{n} \cdot \pvlog{x}}
			\end{align}
			と定めてこれを$x$の{\bf $n$乗根}と呼ぶ.
		\end{dfn}
	\end{screen}
	
	$x$を正の実数とし,$n$を$1$以上の自然数とするとき,$\sqrt[n]{x}$は$n$乗すれば$x$に戻る.
	実際,定理\ref{thm:integer_exponentiation_of_exponential_function}より
	\begin{align}
		(\sqrt[n]{x})^{n} &= \left(e^{\frac{1}{n} \cdot \pvlog{x}}\right)^{n} \\
		&= e^{n \cdot \frac{1}{n} \cdot \pvlog{x}} \\
		&= e^{\pvlog{x}} \\
		&= x
	\end{align}
	が成立する.ここで$y$を正の実数とすれば
	\begin{align}
		\pvlog{(x \cdot y)} = \pvlog{x} + \pvlog{y}
	\end{align}
	が成り立つので
	\begin{align}
		\sqrt[n]{x \cdot y} = e^{\frac{1}{n} \cdot \pvlog{(x \cdot y)}}
		&= e^{\frac{1}{n} \cdot \pvlog{x} + \frac{1}{n} \cdot \pvlog{y}} \\
		&= e^{\frac{1}{n} \cdot \pvlog{x}} \cdot e^{\frac{1}{n} \cdot \pvlog{y}} \\
		&= \sqrt[n]{x} \cdot \sqrt[n]{y}
	\end{align}
	が成り立つ.また
	\begin{align}
		\pvlog{x^{n+1}} = \pvlog{(x^{n} \cdot x)} = \pvlog{x^{n}} + \pvlog{x}
	\end{align}
	と数学的帰納法の原理から
	\begin{align}
		\pvlog{x^{n}} = n \cdot \pvlog{x}
	\end{align}
	も示すことが出来る.
	
	\begin{screen}
		\begin{thm}[Cauchyの冪根判定法]
			$c$を複素数列とするとき,
			\begin{align}
				\inf{n \in \Natural}{\sup{\substack{k \in \Natural \\ n < k}}{\sqrt[k]{|c_{k}|}}} < 1
			\end{align}
			ならば$\sum_{n=0}^{\infty} c_{n}$は絶対収束し,
			\begin{align}
				1 < \inf{n \in \Natural}{\sup{\substack{k \in \Natural \\ n < k}}{\sqrt[k]{|c_{k}|}}}
			\end{align}
			ならば数列
			\begin{align}
				\Natural \ni n \longmapsto \sum_{k=0}^{n} c_{k}
			\end{align}
			は$\C$で収束しない.
		\end{thm}
	\end{screen}
	
	\begin{sketch}
		いま
		\begin{align}
			r \defeq \inf{n \in \Natural}{\sup{\substack{k \in \Natural \\ n < k}}{\sqrt[k]{|c_{k}|}}}
		\end{align}
		とおく.
		\begin{align}
			r < 1
		\end{align}
		であるとき,
		\begin{align}
			\sup{\substack{k \in \Natural \\ n < k}}{\sqrt[k]{|c_{k}|}} < \frac{1+r}{2}
		\end{align}
		を満たす自然数$n$が取れる.すなわち
		\begin{align}
			n < k
		\end{align}
		を満たす任意の自然数$k$に対して
		\begin{align}
			|c_{k}| \leq \left(\frac{1+r}{2}\right)^{k}
		\end{align}
		が成立する.ゆえに,$n$より大きい任意の自然数$N$に対して
		\begin{align}
			\sum_{k=0}^{N} |c_k|
			&= \sum_{k=0}^{n} |c_k| + \sum_{k=n+1}^{N} |c_k| \\
			&\leq \sum_{k=0}^{n} |c_k| + \sum_{k=n+1}^{N} \left(\frac{1+r}{2}\right)^{k} \\
			&= \sum_{k=0}^{n} |c_k| + \left(\frac{1+r}{2}\right)^{n+1} \cdot \sum_{k=0}^{N-n+1} \left(\frac{1+r}{2}\right)^{k} \\
			&\leq \sum_{k=0}^{n} |c_k| + \left(\frac{1+r}{2}\right)^{n+1} \cdot \frac{2}{1-r}
		\end{align}
		が成立する.ゆえに$\sum_{n=0}^{\infty} c_{n}$は絶対収束する.
		\begin{align}
			1 < r
		\end{align}
		であるとき,$n$を任意に与えられた自然数とすれば
		\begin{align}
			1 < \sup{\substack{k \in \Natural \\ n < k}}{\sqrt[k]{|c_{k}|}}
		\end{align}
		が成り立つので,
		\begin{align}
			n < k
		\end{align}
		かつ
		\begin{align}
			1 < \sqrt[k]{|c_{k}|}
		\end{align}
		を満たす自然数$k$が取れて,両辺$k$乗して
		\begin{align}
			1 < |c_{k}|
		\end{align}
		が成立する.他方で,
		\begin{align}
			\Natural \ni n \longmapsto \sum_{k=0}^{n} c_{k}
		\end{align}
		が$\C$で収束するときは
		\begin{align}
			\forall \epsilon \in \R_+\, \exists N \in \Natural\, \forall n \in \Natural\,
			\left[\, N \leq n \Longrightarrow |c_{n}| < \epsilon\, \right]
		\end{align}
		が成立するので,対偶命題から
		\begin{align}
			1 < r
		\end{align}
		ならば
		\begin{align}
			\Natural \ni n \longmapsto \sum_{k=0}^{n} c_{k}
		\end{align}
		は$\C$で収束しない.
		\QED
	\end{sketch}
	
	\begin{screen}
		\begin{dfn}[収束半径]
			複素数列$c$が与えられたとき,
			\begin{align}
				r \defeq \inf{n \in \Natural}{\sup{\substack{k \in \Natural \\ n < k}}{\sqrt[k]{|c_{k}|}}}
			\end{align}
			とおく.このとき
			\begin{align}
				R \defeq
				\begin{cases}
					0 & \mbox{if } r = \infty \\
					\displaystyle{\frac{1}{r}} & \mbox{if } \displaystyle{0 < r < \infty} \\
					\infty & \mbox{if } r = 0
				\end{cases}
			\end{align}
			により定める$R$を$c$の{\bf 収束半径}\index{しゅうそくはんけい@収束半径}{\bf (radius of convergence)}と呼ぶ.
		\end{dfn}
	\end{screen}
	
	\begin{screen}
		\begin{thm}[収束半径の内側では絶対収束,外側では収束しない]
		\label{thm:absolutely_converge_inside_the_convergence_circle}
			$z$を複素数とし,$c$を複素数列とし,$R$を$c$の収束半径とする.このとき,
			\begin{align}
				|z| < R
			\end{align}
			ならば$\sum_{n=0}^{\infty} c_n \cdot z^n$は絶対収束する.
			\begin{align}
				R < |z|
			\end{align}
			であれば
			\begin{align}
				\Natural \ni n \longmapsto \sum_{k=0}^{n} c_{k} \cdot z^{k}
			\end{align}
			なる数列は$\C$で収束しない.
		\end{thm}
	\end{screen}
	
	\begin{sketch}
		$z$を$0$でない複素数とすると,任意の自然数$n$で
		\begin{align}
			\sqrt[n]{|c_{n} \cdot z^{n}|}
			&= \sqrt[n]{|c_{n}| \cdot |z^{n}|} \\
			&= \sqrt[n]{|c_{n}| \cdot |z|^{n}} \\
			&= \sqrt[n]{|c_{n}|} \cdot \sqrt[n]{|z|^{n}} \\
			&= \sqrt[n]{|c_{n}|} \cdot e^{\frac{1}{n} \cdot \pvlog{|z|^{n}}} \\
			&= \sqrt[n]{|c_{n}|} \cdot e^{\frac{1}{n} \cdot n \cdot \pvlog{|z|}} \\
			&= \sqrt[n]{|c_{n}|} \cdot |z|
		\end{align}
		が成立する.$z$が$0$ならば
		\begin{align}
			\sqrt[n]{|c_{n} \cdot z^{n}|} = 0
		\end{align}
		である.よって,
		\begin{align}
			|z| < R
		\end{align}
		であるとき,
		\begin{align}
			\inf{n \in \Natural}{\sup{\substack{k \in \Natural \\ n < k}}{\sqrt[k]{|c_{k} \cdot z^{k}|}}}
			= |z| \cdot \inf{n \in \Natural}{\sup{\substack{k \in \Natural \\ n < k}}{\sqrt[k]{|c_{k}|}}}
			< 1
		\end{align}
		が成り立つ.ゆえにCauchyの冪根判定法より$\sum_{n=0}^{\infty} c_n \cdot z^n$は絶対収束する.
		\begin{align}
			R < |z|
		\end{align}
		であるとき,
		\begin{align}
			1 < \inf{n \in \Natural}{\sup{\substack{k \in \Natural \\ n < k}}{\sqrt[k]{|c_{k} \cdot z^{k}|}}}
			= |z| \cdot \inf{n \in \Natural}{\sup{\substack{k \in \Natural \\ n < k}}{\sqrt[k]{|c_{k}|}}}
		\end{align}
		が成り立つので,Cauchyの冪根判定法より
		\begin{align}
			\Natural \ni n \longmapsto \sum_{k=0}^{n} c_{k} \cdot z^{k}
		\end{align}
		なる数列は$\C$で収束しない.
		\QED
	\end{sketch}
	
	$z$を複素数とし,$r$を正数とするとき,
	\begin{align}
		\disc{z}{r} \defeq \Set{w \in \C}{|w - z| < r}
	\end{align}
	と定める.イメージとしては$\disc{z}{r}$とは中心$z$半径$r$の円板を表す.
	
	\begin{screen}
		\begin{thm}[級数で表される関数は微分可能]
			$r$を正の実数とし,$c$を複素数列とする.このとき,$\disc{0}{r}$の任意の要素$z$で
			\begin{align}
				\Natural \ni n \longmapsto \sum_{k=0}^{n} c_{k} \cdot z^{k}
			\end{align}
			が収束しているならば
			\begin{itemize}
				\item $\disc{0}{r}$の任意の要素$z$で$\sum_{n=0}^{\infty} c_{n} \cdot z^{n}$は絶対収束する.
				\item $\disc{0}{r}$上の写像$f$を
					\begin{align}
						z \longmapsto \sum_{n=0}^{\infty} c_{n} \cdot z^{n}
					\end{align}
					なる関係により定めると,$f$は$\disc{0}{r}$の各要素で微分可能であって,
					$f$の導関数を$f'$と書けば
					\begin{align}
						f'(z) = \sum_{n=0}^{\infty} (n+1) \cdot c(n+1) \cdot z^{n}
					\end{align}
					が$\disc{0}{r}$の任意の要素$z$で成立する.
			\end{itemize}
		\end{thm}
	\end{screen}
	
	\begin{sketch}\mbox{}
		\begin{description}
			\item[第一段]
				いま$R$を$c$の収束半径とし,
				\begin{align}
					\Natural \ni n \longmapsto \sum_{k=0}^{n} c_{k} \cdot z^{k}
				\end{align}
				が収束しているとする.$z$を$\disc{0}{r}$の要素とするとき,
				\begin{align}
					\rho \defeq \frac{|z| + r}{2}
				\end{align}
				とおけば
				\begin{align}
					|z| < \rho < r
				\end{align}
				が成り立つので
				\begin{align}
					\Natural \ni n \longmapsto \sum_{k=0}^{n} c_{k} \cdot \rho^{k}
				\end{align}
				は収束する.ゆえに定理\ref{thm:absolutely_converge_inside_the_convergence_circle}より
				\begin{align}
					\rho \leq R
				\end{align}
				が成り立つ.ゆえに
				\begin{align}
					|z| < R
				\end{align}
				が成り立つので,定理\ref{thm:absolutely_converge_inside_the_convergence_circle}より
				$\sum_{n=0}^{\infty} c_{n} \cdot z^{n}$は絶対収束する.
			
			\item[第二段]
				
		\end{description}
	\end{sketch}
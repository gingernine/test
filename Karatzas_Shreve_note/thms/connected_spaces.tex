\subsection{連結性}
	
	\begin{screen}
		\begin{thm}
			$\R$の任意の区間は連結である.
		\end{thm}
	\end{screen}
	
	\begin{screen}
		\begin{thm}
			連結集合の連続写像による像は連結である.
		\end{thm}
	\end{screen}
	
	\begin{screen}
		\begin{thm}[弧状連結なら連結]\label{thm:connected_path_connected}
			弧状連結位相空間は連結空間である.
		\end{thm}
	\end{screen}
	
	\begin{prf}
		$S$を連結でない位相空間とする.このとき
		或る空でない開集合$U_1,U_2$が存在して
		\begin{align}
			U_1 \cup U_2 = S,
			\quad U_1 \cap U_2 = \emptyset
		\end{align}
		を満たす.$x \in U_1,\ y \in U_2$に対し
		$f(0) = x,\ f(1) = y$を満たす連続写像
		$f:[0,1] \longrightarrow S$が存在する場合,
		\begin{align}
			f([0,1]) = \left( U_1 \cap f([0,1]) \right) \cup \left( U_2 \cap f([0,1]) \right),
			\quad \left( U_1 \cap f([0,1]) \right) \cap \left( U_2 \cap f([0,1]) \right) = \emptyset
		\end{align}
		となり$f([0,1])$の連結性に矛盾する.
		従って$x,y$を結ぶ道は存在しないから$S$は弧状連結ではない.
		\QED
	\end{prf}
	
\subsection{商位相}
	\begin{screen}
		\begin{thm}[商位相]
			位相空間$(S,\mathscr{O})$に同値関係$\sim$が定まっているとき,
			$x \in S$からその同値類$\pi(x)$への対応
			\begin{align}
				\pi: S \ni x \longmapsto \pi(x) \in S/\sim
			\end{align}
			を商写像(quotient mapping)という.また,商写像を連続にする$S/\sim$の最強の位相,つまり
			\begin{align}
				\mathscr{O}(S/\sim) \coloneqq
				\Set{V \subset S/\sim}{\pi^{-1}(V) \in \mathscr{O}}
			\end{align}
			で定まる位相を$S/\sim$の商位相(quotient topology)という.
		\end{thm}
	\end{screen}
	
	\begin{screen}
		\begin{thm}[商空間が$T_1 \Longleftrightarrow$同値類が元の空間で閉じている]
		\label{thm:quotient_space_T_1_iff_each_equivalence_class_closed}
			$S$を位相空間,$\sim$を$S$上の同値関係,$\pi:S \longrightarrow S/\sim$を商写像
			とする.このとき次が成り立つ:
			\begin{align}
				\mbox{$S/\sim$が$T_1$空間である}
				\quad \Longleftrightarrow \quad
				\mbox{任意の$x \in S$に対し$\pi(x)$が$S$の閉集合である}.
			\end{align}
		\end{thm}
	\end{screen}
	
	\begin{prf}
		任意の$F \subset S/\sim$に対し
		\begin{align}
			\mbox{$F$が閉} \quad \Longleftrightarrow \quad
			\mbox{$\pi^{-1}(F^c) = \pi^{-1}(F)^c$が開} \quad \Longleftrightarrow \quad
			\mbox{$\pi^{-1}(F)$が閉}
		\end{align}
		となる.いま任意の$x \in S$に対し
		$\pi(x) = \pi^{-1}(\pi(x))$が満たされているから定理の主張を得る.
		\QED
	\end{prf}
	
	\begin{screen}
		\begin{thm}[商写像が開なら,商空間がHausdorff$\Longleftrightarrow$対角線集合が閉]
		\label{thm:quotient_space_Hausdorff_iff_diagonal_set_closed}
			$S$を位相空間,$\sim$を$S$上の同値関係,$\pi:S \longrightarrow S/\sim$を商写像
			とする.このとき,$\pi$が開写像であれば次が成立する:
			\begin{align}
				\mbox{$S/\sim$がHausdorff} \quad \Longleftrightarrow \quad
				\mbox{$\Set{(x,y) \in S \times S}{x \sim y}$が閉}.
			\end{align}
		\end{thm}
	\end{screen}
	
	\begin{prf}
		$S/\sim$がHausdorffであるとき,$x \not\sim y$を満たす$(x,y) \in S \times S$に対し
		$\pi(x) \neq \pi(y)$となるから
		\begin{align}
			\pi(x) \in U,\quad \pi(y) \in V,\quad U \cap V = \emptyset
		\end{align}
		を満たす$S/\sim$の開集合$U,V$が取れる.このとき
		$\pi^{-1}(U) \times \pi^{-1}(V)$は$S \times S$の開集合であり
		\begin{align}
			(x,y) \in \pi^{-1}(U) \times \pi^{-1}(V)
			\subset \Set{(s,t) \in S \times S}{s \not\sim t}
		\end{align}
		が成り立つから$\Longrightarrow$が得られる.
		逆に$\Set{(s,t) \in S \times S}{s \not\sim t}$が開集合であるとき,
		$\pi(x) \neq \pi(y)$なら
		\begin{align}
			(x,y) \in U \times V \subset \Set{(s,t) \in S \times S}{s \not\sim t}
		\end{align}
		を満たす$S$の開集合$U,V$が存在し,このとき
		\begin{align}
			\pi(x) \in \pi(U),\quad \pi(y) \in \pi(V),
			\quad \pi(U) \cap \pi(V) = \emptyset
		\end{align}
		となりかつ$\pi$が開写像であるから$\Longleftarrow$が従う.
		\QED
	\end{prf}
	
	\begin{screen}
		\begin{cor}[Hausdorff$\Longleftrightarrow$対角線集合が閉]
		\label{cor:quotient_space_Hausdorff_iff_diagonal_set_closed}
			$S$を位相空間とするとき,
			\begin{align}
				\mbox{$S$がHausdorffである}
				\quad \Longleftrightarrow \quad
				\mbox{$\Set{(x,x)}{x \in S}$が$S \times S$で閉じている}.
			\end{align}
		\end{cor}
	\end{screen}
	
	\begin{prf}
		等号$=$を同値関係と見れば$S$と$S/=$は商写像により同相となるから,
		定理\ref{thm:quotient_space_Hausdorff_iff_diagonal_set_closed}より
		\begin{align}
			\mbox{$S$がHausdorff} \quad \Longleftrightarrow \quad
			\mbox{$S/=$がHausdorff} \quad \Longleftrightarrow \quad
			\mbox{$\Set{(x,x)}{x \in S}$が閉}
		\end{align}
		が成立する.
		\QED
	\end{prf}
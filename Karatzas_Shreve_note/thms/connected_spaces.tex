\section{連結性}
	
	\begin{screen}
		\begin{thm}
			$\R$の任意の区間は連結である.
		\end{thm}
	\end{screen}
	
	\begin{screen}
		\begin{thm}
			連結集合の連続写像による像は連結である.
		\end{thm}
	\end{screen}
	
	\begin{screen}
		\begin{thm}[弧状連結なら連結]\label{thm:connected_path_connected}
			弧状連結位相空間は連結空間である.
		\end{thm}
	\end{screen}
	
	\begin{prf}
		$S$を連結でない位相空間とする.このとき
		或る空でない開集合$U_1,U_2$が存在して
		\begin{align}
			U_1 \cup U_2 = S,
			\quad U_1 \cap U_2 = \emptyset
		\end{align}
		を満たす.$x \in U_1,\ y \in U_2$に対し
		$f(0) = x,\ f(1) = y$を満たす連続写像
		$f:[0,1] \longrightarrow S$が存在する場合,
		\begin{align}
			f([0,1]) = \left( U_1 \cap f([0,1]) \right) \cup \left( U_2 \cap f([0,1]) \right),
			\quad \left( U_1 \cap f([0,1]) \right) \cap \left( U_2 \cap f([0,1]) \right) = \emptyset
		\end{align}
		となり$f([0,1])$の連結性に矛盾する.
		従って$x,y$を結ぶ道は存在しないから$S$は弧状連結ではない.
		\QED
	\end{prf}
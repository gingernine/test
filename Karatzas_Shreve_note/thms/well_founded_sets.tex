\section{整礎集合}
	いま$\Univ$上の写像$G$を
	\begin{align}
		x \longmapsto
		\begin{cases}
			\emptyset & \mbox{if } \operatorname{dom}(x) = \emptyset \\
			\power{x(\beta)} & \mbox{if } \beta \in \ON \wedge \operatorname{dom}(x) = \beta \cup \{\beta\} \\
			\bigcup \operatorname{ran}(x) & \mathrm{o.w.}
		\end{cases}
	\end{align}
	なる関係により定めると,つまり正式には
	\begin{align}
		\{\, (x,y) \mid \quad &\left(\, \dom{x} = \emptyset \Longrightarrow y = \alpha\, \right) \\
		&\wedge \forall \beta \in \ON\, \left(\, \dom{x} = \beta \cup \{\beta\} \Longrightarrow y = \power{x(\beta)}\, \right) \\
		&\wedge \left[\, \dom{x} \neq \emptyset \wedge \forall \beta \in \ON\, \left(\, \dom{x} \neq \beta \cup \{\beta\}\, \right)
		\Longrightarrow y = \bigcup \ran{x}\, \right]\, \}
	\end{align}
	で定めると,
	\begin{align}
		\forall \alpha \in \ON\, (\, R(\alpha) = G(R|_\alpha)\, )
	\end{align}
	を満たす$\ON$上の写像$R$が取れる.$R$とは,その定義の仕方より
	\begin{align}
		\ON \ni \alpha \longmapsto
		\begin{cases}
			0 & \mbox{if } \alpha = 0 \\
			\power{R(\beta)} & \mbox{if } \beta \in \ON \wedge \alpha = \beta \cup \{\beta\} \\
			\bigcup_{\beta \in \alpha} R(\beta) & \mbox{if } \limo{\alpha}
		\end{cases}
	\end{align}
	を満たす写像である.本節ではこの$R$が考察対象となる.
	
	\begin{screen}
		\begin{dfn}[整礎集合]
			$\bigcup_{\alpha \in \ON} R(\alpha)$の要素を{\bf 整礎集合}
			\index{せいそしゅうごう@整礎集合}{\bf (well-founded set)}と呼ぶ.
		\end{dfn}
	\end{screen}
	
	この$R$を用いると次の美しい式が導かれる.ただしこれは偶然得られた訳ではなく,
	John Von Neumann はこの結果を予定して正則性公理を導入したのである.
	
	\begin{screen}
		\begin{thm}[すべての集合は整礎的である]\label{thm:every_set_is_well_founded}
			\begin{align}
				\Univ = \bigcup_{\alpha \in \ON} R(\alpha).
			\end{align}
		\end{thm}
	\end{screen}
	
	\begin{prf}
		$S$を類として,$S$が$\ON$の空でない部分類ならば
		\begin{align}
			\Univ \neq \bigcup_{\alpha \in S} R(\alpha)
			\Longrightarrow S \neq \ON
		\end{align}
		が成り立つことを示す.
		\begin{align}
			\Univ \neq \bigcup_{\alpha \in S} R(\alpha)
		\end{align}
		が成り立っているとすると,正則性公理より
		\begin{align}
			a \in \Univ \backslash \bigcup_{\alpha \in S} R(\alpha)
			\wedge a \cap \Univ \backslash \bigcup_{\alpha \in S} R(\alpha) = \emptyset
		\end{align}
		を満たす集合$a$が取れる.つまり
		\begin{align}
			a \notin \bigcup_{\alpha \in S} R(\alpha) \wedge a \subset \bigcup_{\alpha \in S} R(\alpha)
		\end{align}
		が成り立っている.ここで$a$の要素$s$に対して
		\begin{align}
			s \in R(\alpha)
		\end{align}
		を満たす順序数$\alpha$のうちで$\leq$に関して最小のものを対応させる関係を$f$とすると,つまり
		\begin{align}
			f \defeq \Set{x}{\exists s \in a\, \exists \alpha \in \ON\, 
			\left[\, x = (s,\alpha) \wedge s \in R(\alpha) \wedge
			\forall \beta \in \ON\, (\, s \in R(\beta) \Longrightarrow \alpha \leq \beta\, )\, \right]}
		\end{align}
		と定めれば,
		\begin{align}
			f:a \longrightarrow \ON
		\end{align}
		が成り立つ.従って
		\begin{align}
			\beta \defeq \bigcup f \ast a
		\end{align}
		とおけば$\beta$は順序数である.このとき
		\begin{align}
			t \in a \Longrightarrow t \in R(f(t)) \Longrightarrow t \in R(\beta)
		\end{align}
		が成り立つから
		\begin{align}
			a \subset R(\beta)
		\end{align}
		が成り立ち,そして
		\begin{align}
			R\left(\beta \cup \{\beta\}\right) = \power{R(\beta)}
		\end{align}
		であるから
		\begin{align}
			a \in R\left(\beta \cup \{\beta\}\right)
		\end{align}
		が従う.
		\begin{align}
			\forall \alpha \in S\ (\ a \notin R(\alpha)\ )
		\end{align}
		であったから
		\begin{align}
			\beta \cup \{\beta\} \notin S
		\end{align}
		が成り立つので
		\begin{align}
			S \neq \ON
		\end{align}
		である.定理の主張は対偶を取れば得られる.
		\QED
	\end{prf}
	
	\begin{screen}
		\begin{dfn}[集合の階数]
		\end{dfn}
	\end{screen}
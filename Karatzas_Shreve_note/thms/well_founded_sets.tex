\section{整礎集合}
	いま$\Univ$上の写像$G$を
	\begin{align}
		x \longmapsto
		\begin{cases}
			\emptyset & \mbox{if } \operatorname{dom}(x) = \emptyset \\
			x(\beta) \cup \power{x(\beta)} & \mbox{if } \beta \in \ON \wedge \operatorname{dom}(x) = \beta \cup \{\beta\} \\
			\bigcup \operatorname{ran}(x) & \mathrm{o.w.}
		\end{cases}
	\end{align}
	なる関係により定めると,つまり正式には
	\begin{align}
		\{\, (x,y) \mid \quad &\left(\, \dom{x} = \emptyset \Longrightarrow y = \alpha\, \right) \\
		&\wedge \forall \beta \in \ON\, \left(\, \dom{x} = \beta \cup \{\beta\} \Longrightarrow y = x(\beta) \cup \power{x(\beta)}\, \right) \\
		&\wedge \left[\, \dom{x} \neq \emptyset \wedge \forall \beta \in \ON\, \left(\, \dom{x} \neq \beta \cup \{\beta\}\, \right)
		\Longrightarrow y = \bigcup \ran{x}\, \right]\, \}
	\end{align}
	で定めると,
	\begin{align}
		\forall \alpha \in \ON\, (\, R(\alpha) = G(R|_\alpha)\, )
	\end{align}
	を満たす$\ON$上の写像$R$が取れる.本節ではこの$R$が考察対象となる.
	
	\begin{screen}
		\begin{thm}
			\begin{align}\label{thm:R_alpha_plus_1_equals_to_power_of_R_alpha}
				\forall \alpha \in \ON\ 
				\left(\ R(\alpha + 1) = \operatorname{P}(R(\alpha))\ \right)
			\end{align}
		\end{thm}
	\end{screen}
	
	\begin{prf}\mbox{}
		\begin{description}
			\item[第一段] $R(\alpha + 1) = R(\alpha) \cup \operatorname{P}(R(\alpha))$
				となることを示す.
				
			\item[第二段] $\alpha$を任意に与えられた空でない順序数とするとき,
				\begin{align}
					\forall \beta \in \alpha\ 
					\left(\ R(\beta + 1) \subset \operatorname{P}(R(\beta))\ \right)
					\Longrightarrow R(\alpha + 1) \subset \operatorname{P}(R(\alpha))
				\end{align}
				が成り立つことを示す.いま
				\begin{align}
					\forall \beta \in \alpha\ 
					\left(\ R(\beta + 1) \subset \operatorname{P}(R(\beta))\ \right)
					\label{eq:thm_R_alpha_plus_1_equals_to_power_of_R_alpha}
				\end{align}
				が成り立つと仮定する.$x$を$R(\alpha + 1)$の任意の要素とすれば,前段の結果より
				\begin{align}
					x \in R(\alpha) \vee x \subset R(\alpha)
				\end{align}
				となる.$x \in R(\alpha)$であるとき,$\alpha$の或る要素$\beta$に対し
				$x \in R(\beta)$となる.前段の結果より$x \in R(\beta + 1)$となり,
				(\refeq{eq:thm_R_alpha_plus_1_equals_to_power_of_R_alpha})より
				$x \subset R(\beta)$となるが,
				\begin{align}
					x \subset R(\beta) &\Longrightarrow x \subset R(\alpha), \\
					x \subset R(\alpha) &\Longrightarrow x \in \operatorname{P}(R(\alpha))
				\end{align}
				と併せて$x \in \operatorname{P}(R(\alpha))$が成り立つ.
				一方で$x \subset R(\alpha)$であるときも$x \in \operatorname{P}(R(\alpha))$
				となるから
				\begin{align}
					R(\alpha + 1) \subset \operatorname{P}(R(\alpha))
				\end{align}
				が従う.超限帰納法より定理の主張が得られる.
		\end{description}
	\end{prf}
	
	\begin{screen}
		\begin{dfn}[整礎集合]
			$\bigcup_{\alpha \in \ON} R(\alpha)$の要素を{\bf 整礎集合}
			\index{せいそしゅうごう@整礎集合}{\bf (well-founded set)}と呼ぶ.
		\end{dfn}
	\end{screen}
	
	この$R$を用いると次の美しい式が導かれる.ただしこれは偶然得られた訳ではなく,
	John Von Neumann はこの結果を予定して正則性公理を導入したのである.
	
	\begin{screen}
		\begin{thm}[すべての集合は整礎的である]\label{thm:every_set_is_well_founded}
			\begin{align}
				\Univ = \bigcup_{\alpha \in \ON} R(\alpha).
			\end{align}
		\end{thm}
	\end{screen}
	
	\begin{prf}
		$S$を類として,$S$が$\ON$の空でない部分類ならば
		\begin{align}
			\Univ \neq \bigcup_{\alpha \in S} R(\alpha)
			\Longrightarrow S \neq \ON
		\end{align}
		が成り立つことを示す.
		\begin{align}
			\Univ \neq \bigcup_{\alpha \in S} R(\alpha)
		\end{align}
		が成り立っているとすると,正則性公理より
		\begin{align}
			a \in \Univ \backslash \bigcup_{\alpha \in S} R(\alpha)
			\wedge a \cap \Univ \backslash \bigcup_{\alpha \in S} R(\alpha) = \emptyset
		\end{align}
		を満たす集合$a$が取れる.つまり
		\begin{align}
			a \notin \bigcup_{\alpha \in S} R(\alpha) \wedge a \subset \bigcup_{\alpha \in S} R(\alpha)
		\end{align}
		が成り立っている.ここで
		\begin{align}
			a \ni s \longmapsto \mu \alpha (s \in R(\alpha))
		\end{align}
		によって定める関係を$f$とすると,つまり
		\begin{align}
			f \defeq \Set{x}{\exists s \in a\ (\ x = (s,\mu \alpha (s \in R(\alpha)))\ )}
		\end{align}
		と定めれば,
		\begin{align}
			f:a \longrightarrow \ON
		\end{align}
		が成り立つ.従って
		\begin{align}
			\beta \defeq \bigcup f \ast a
		\end{align}
		とおけば$\beta$は順序数である.このとき
		\begin{align}
			t \in a \Longrightarrow t \in R(f(t)) \Longrightarrow t \in R(\beta)
		\end{align}
		が成り立つから
		\begin{align}
			a \subset R(\beta)
		\end{align}
		が成り立ち,そして定理\ref{thm:R_alpha_plus_1_equals_to_power_of_R_alpha}より
		\begin{align}
			a \in R\left(\beta \cup \{\beta\}\right)
		\end{align}
		が従う.
		\begin{align}
			\forall \alpha \in S\ (\ a \notin R(\alpha)\ )
		\end{align}
		であったから
		\begin{align}
			\beta \cup \{\beta\} \notin S
		\end{align}
		が成り立つので
		\begin{align}
			S \neq \ON
		\end{align}
		である.定理の主張は対偶を取れば得られる.
		\QED
	\end{prf}
	
	\begin{screen}
		\begin{dfn}[集合の階数]
		\end{dfn}
	\end{screen}
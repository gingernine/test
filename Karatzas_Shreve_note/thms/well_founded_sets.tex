\subsection{整礎集合}
	いま$\Univ$上の写像$G$を
	\begin{align}
		G(x) = 
		\begin{cases}
			\emptyset & (\operatorname{dom}(x) = \emptyset) \\
			x(\beta) \cup \operatorname{P}(x(\beta)) & (
			\exists \beta \in \ON\ (\ \operatorname{dom}(x) = \beta \cup \{\beta\}\ )) \\
			\bigcup \operatorname{ran}(x) & \mathrm{o.w.}
		\end{cases}
	\end{align}
	で定めると,定理\ref{thm:transfinite_recursion_theorem}より
	\begin{align}
		\forall \alpha \in \ON\ (\ R(\alpha) = G(R|_\alpha)\ )
	\end{align}
	を満たす$\ON$上の写像$R$が唯一つ存在する.以降しばらくはこの$R$が考察対象となる.
	
	\begin{screen}
		\begin{thm}
			\begin{align}\label{thm:R_alpha_plus_1_equals_to_power_of_R_alpha}
				\forall \alpha \in \ON\ 
				\left(\ R(\alpha + 1) = \operatorname{P}(R(\alpha))\ \right)
			\end{align}
		\end{thm}
	\end{screen}
	
	\begin{prf}\mbox{}
		\begin{description}
			\item[第一段] $R(\alpha + 1) = R(\alpha) \cup \operatorname{P}(R(\alpha))$
				となることを示す.
				
			\item[第二段] $\alpha$を任意に与えられた空でない順序数とするとき,
				\begin{align}
					\forall \beta \in \alpha\ 
					\left(\ R(\beta + 1) \subset \operatorname{P}(R(\beta))\ \right)
					\Longrightarrow R(\alpha + 1) \subset \operatorname{P}(R(\alpha))
				\end{align}
				が成り立つことを示す.いま
				\begin{align}
					\forall \beta \in \alpha\ 
					\left(\ R(\beta + 1) \subset \operatorname{P}(R(\beta))\ \right)
					\label{eq:thm_R_alpha_plus_1_equals_to_power_of_R_alpha}
				\end{align}
				が成り立つと仮定する.$x$を$R(\alpha + 1)$の任意の要素とすれば,前段の結果より
				\begin{align}
					x \in R(\alpha) \vee x \subset R(\alpha)
				\end{align}
				となる.$x \in R(\alpha)$であるとき,$\alpha$の或る要素$\beta$に対し
				$x \in R(\beta)$となる.前段の結果より$x \in R(\beta + 1)$となり,
				(\refeq{eq:thm_R_alpha_plus_1_equals_to_power_of_R_alpha})より
				$x \subset R(\beta)$となるが,
				\begin{align}
					x \subset R(\beta) &\Longrightarrow x \subset R(\alpha), \\
					x \subset R(\alpha) &\Longrightarrow x \in \operatorname{P}(R(\alpha))
				\end{align}
				と併せて$x \in \operatorname{P}(R(\alpha))$が成り立つ.
				一方で$x \subset R(\alpha)$であるときも$x \in \operatorname{P}(R(\alpha))$
				となるから
				\begin{align}
					R(\alpha + 1) \subset \operatorname{P}(R(\alpha))
				\end{align}
				が従う.超限帰納法より定理の主張が得られる.
		\end{description}
	\end{prf}
	
	\begin{screen}
		\begin{dfn}[整礎集合]
			$\bigcup_{\alpha \in \ON} R(\alpha)$の要素を{\bf 整礎集合}
			\index{せいそしゅうごう@整礎集合}{\bf (well-founded set)}と呼ぶ.
		\end{dfn}
	\end{screen}
	
	\begin{screen}
		\begin{thm}[すべての集合は整礎的である]\label{thm:every_set_is_well_founded}
			次は定理である:
			\begin{align}
				\Univ = \bigcup_{\alpha \in \ON} R(\alpha).
			\end{align}
		\end{thm}
	\end{screen}
	
	\begin{prf}
		いま,$S$を$\ON$の空でない部分集合として
		\begin{align}
			V \neq \bigcup_{\alpha \in S} R(\alpha)
			\Longrightarrow S \neq \ON
		\end{align}
		が成り立つことを示す.$V \neq \bigcup_{\alpha \in S} R(\alpha)$であれば
		正則性公理より或る集合$a$が存在して
		\begin{align}
			a \in V \backslash \bigcup_{\alpha \in S} R(\alpha)
			\wedge a \cap V \backslash \bigcup_{\alpha \in S} R(\alpha) = \emptyset
		\end{align}
		を満たす.このとき
		\begin{align}
			a \in \bigcup_{\alpha \in S} R(\alpha) \wedge a \subset \bigcup_{\alpha \in S} R(\alpha)
		\end{align}
		となる.ここで
		\begin{align}
			f = \Set{x}{\exists s \in a\ (\ x = (s,\mu \alpha (s \in R(\alpha)))\ )}
		\end{align}
		と定めれば$f:a \longrightarrow \ON$が成り立つ.
		$\beta = \bigcup f(a)$とおけば$\beta$は$\ON$に属する.このとき
		\begin{align}
			\forall t\ (\ t \in a \Longrightarrow t \in R(f(t))
			\Longrightarrow t \in R(\beta)\ )
		\end{align}
		となるから$a \subset R(\beta)$,そして定理\ref{thm:R_alpha_plus_1_equals_to_power_of_R_alpha}
		より$a \in R(\beta + 1)$が従う.
		\begin{align}
			\forall \alpha \in S\ (\ a \notin R(\alpha)\ )
		\end{align}
		であったから$\beta + 1 \notin S$であり,ゆえに$S \neq \ON$となる.
		定理の主張は対偶を取れば得られる.
		\QED
	\end{prf}
	
	\monologue{
		院生「\begin{align}
				\Univ = \bigcup_{\alpha \in \ON} R(\alpha)
			\end{align}
			という美しい式は偶然得られた訳ではありません.John Von Neumann はこの結果を
			予定して正則性公理を導入したのです.
			さて,超限帰納法による写像の構成を応用して
			次は順序数の足し算と掛け算を定義しましょう.」
	}
	
	\begin{screen}
		\begin{thm}[順序数の加法]\label{thm:the_definition_of_addition_of_ordinal_numbers}
			$\alpha$を$\ON$から任意に選ばれた順序数として,$\Univ$上の写像$G_\alpha$を
			\begin{align}
				G_\alpha(x) = 
				\begin{cases}
					\alpha & (\operatorname{dom}(x) = \emptyset) \\
					x(\beta) \cup \{x(\beta)\} & (
					\exists \beta \in \ON\, (\, \operatorname{dom}(x) = \beta \cup \{\beta\}\, )) \\
					\bigcup \operatorname{ran}(x) & \mathrm{o.w.}
				\end{cases}
			\end{align}
			で定めるとき,定理\ref{thm:transfinite_recursion_theorem}より
			\begin{align}
				\forall \beta \in \ON\, (\, A_\alpha(\beta) = G_\alpha(A_\alpha|_\beta)\, )
			\end{align}
			を満たす$\ON$上の写像$A_\alpha$が唯一つ存在する.ここで
			\begin{align}
				\alpha + \beta = A_\alpha (\beta)
			\end{align}
			と書くと,次が成立する:
			\begin{itemize}
				\item $\forall \alpha,\alpha' \in \ON\, \left(\, \alpha = \alpha' \Longrightarrow A_\alpha = A_{\alpha'}\, \right)$.
				\item $\forall \beta \in \ON\, (\, \alpha + \beta \in \ON\, )$.
				\item $\alpha \in {\bf \omega}$のとき,$\forall \beta \in {\bf \omega}\, (\, \alpha + \beta \in {\bf \omega}\, )$.
			\end{itemize}
		\end{thm}
	\end{screen}
	
	\begin{prf}
		いま$\beta$を任意に与えられた順序数とする.このとき,
		\begin{align}
			\forall \gamma \in \beta\ (\ \alpha + \gamma \in \ON\ )
		\end{align}
		が成り立っていると仮定すると,$\beta = \gamma + 1$と表せるとき
		\begin{align}
			\alpha + \beta 
			= G_\alpha (F_\alpha|_\beta)
			= F_\alpha(\gamma) + 1
			= (\alpha + \gamma) + 1 \in \ON
		\end{align}
		となり,$\beta$が極限数のときは
		\begin{align}
			\alpha + \beta = \operatorname*{sup}_{\gamma \in \beta} (\alpha + \gamma)
			= \bigcup \Set{\alpha + \gamma}{\gamma \in \beta}
			\in \ON
		\end{align}
		となるので,
		\begin{align}
			\forall \beta \in \ON\ \left(\ \forall \gamma \in \beta\ (\ \alpha + \gamma \in \ON\ ) \Longrightarrow \alpha + \beta \in \ON\ \right)
		\end{align}
		が得られた.超限帰納法により
		\begin{align}
			\forall \beta \in \ON\ (\ \alpha + \beta \in \ON\ )
		\end{align}
		が成立する.また$\alpha \in {\bf \omega}$のとき,
		\begin{align}
			a = \Set{\beta \in {\bf \omega}}{\alpha + \beta \in {\bf \omega}}
		\end{align}
		とおけば
		\begin{align}
			\emptyset \in a \wedge \forall x\ (\ x \in a \Longrightarrow x \cup \{x\} \in a\ )
		\end{align}
		となるので${\bf \omega} \subset a$が従う.よって
		\begin{align}
			\forall \beta \in {\bf \omega}\ 
			(\ \alpha + \beta \in {\bf \omega}\ )
		\end{align}
		も成り立つ.
		\QED
	\end{prf}
	
	\begin{screen}
		\begin{thm}[加法の性質]
		\label{thm:properties_of_addition_of_ordinal_numbers}
			定理\ref{thm:the_definition_of_addition_of_ordinal_numbers}で定めた
			加法は以下の性質を持つ:
			\begin{itemize}
				\item $\forall \alpha \in \ON\ (\ \alpha + 0 = 0 + \alpha = \alpha\ )$,
				
				\item $\forall \alpha \in \ON\ (\ \alpha + 1 = \alpha \cup \{\alpha\}\ )$,
				
				\item $\forall \alpha,\beta,\gamma \in \ON\ (\ (\alpha + \beta) + \gamma = \alpha + (\beta + \gamma)\ )$,
				
				\item $\forall \alpha,\beta \in {\bf \omega}\ (\ \alpha + \beta = \beta + \alpha\ )$,
				
				\item $\forall \alpha,\beta,\gamma \in \ON\ (\ \beta \in \gamma
					\Longrightarrow \alpha + \beta \in \alpha + \gamma\ )$,
				
				\item $\forall \alpha,\beta \in \beta\ (\ \alpha \in \beta
					\Longrightarrow \exists \gamma \in \ON\ (\ \alpha + \gamma = \beta\ )\ )$.
			\end{itemize}
		\end{thm}
	\end{screen}
	
	\begin{screen}
		\begin{thm}[順序数の乗法]
		\label{thm:the_definition_of_multiplication_of_ordinal_numbers}
			$\alpha$を$\ON$から任意に選ばれた順序数として,$\Univ$上の写像$G_\alpha$を
			\begin{align}
				G_\alpha(x) = 
				\begin{cases}
					0 & (\operatorname{dom}(x) = \emptyset) \\
					x(\beta) + \alpha & (
					\exists \beta \in \ON\ (\ \operatorname{dom}(x) = \beta \cup \{\beta\}\ )) \\
					\bigcup \operatorname{ran}(x) & \mathrm{o.w.}
				\end{cases}
			\end{align}
			で定めるとき,定理\ref{thm:transfinite_recursion_theorem}より
			\begin{align}
				\forall \beta \in \ON\ (\ M_\alpha(\beta) = G_\alpha(M_\alpha|_\beta)\ )
			\end{align}
			を満たす$\ON$上の写像$M_\alpha$が唯一つ存在する.ここで
			\begin{align}
				\alpha \cdot \beta = M_\alpha (\beta)
			\end{align}
			と書くと,次が成立する:
			\begin{itemize}
				\item $\forall \beta \in \ON\ (\ \alpha \cdot \beta \in \ON\ )$.
				\item $\alpha \in {\bf \omega}$のとき,$\forall \beta \in {\bf \omega}\ 
				(\ \alpha \cdot \beta \in {\bf \omega}\ )$.
			\end{itemize}
		\end{thm}
	\end{screen}
\section{整礎集合}
	いま$\Univ$上の写像$G$を
	\begin{align}
		G(x) = 
		\begin{cases}
			\emptyset & (\operatorname{dom}(x) = \emptyset) \\
			x(\beta) \cup \operatorname{P}(x(\beta)) & (
			\exists \beta \in \ON\ (\ \operatorname{dom}(x) = \beta \cup \{\beta\}\ )) \\
			\bigcup \operatorname{ran}(x) & \mathrm{o.w.}
		\end{cases}
	\end{align}
	で定めると,定理\ref{thm:transfinite_recursion_theorem}より
	\begin{align}
		\forall \alpha \in \ON\ (\ R(\alpha) = G(R|_\alpha)\ )
	\end{align}
	を満たす$\ON$上の写像$R$が唯一つ存在する.以降しばらくはこの$R$が考察対象となる.
	
	\begin{screen}
		\begin{thm}
			\begin{align}\label{thm:R_alpha_plus_1_equals_to_power_of_R_alpha}
				\forall \alpha \in \ON\ 
				\left(\ R(\alpha + 1) = \operatorname{P}(R(\alpha))\ \right)
			\end{align}
		\end{thm}
	\end{screen}
	
	\begin{prf}\mbox{}
		\begin{description}
			\item[第一段] $R(\alpha + 1) = R(\alpha) \cup \operatorname{P}(R(\alpha))$
				となることを示す.
				
			\item[第二段] $\alpha$を任意に与えられた空でない順序数とするとき,
				\begin{align}
					\forall \beta \in \alpha\ 
					\left(\ R(\beta + 1) \subset \operatorname{P}(R(\beta))\ \right)
					\Longrightarrow R(\alpha + 1) \subset \operatorname{P}(R(\alpha))
				\end{align}
				が成り立つことを示す.いま
				\begin{align}
					\forall \beta \in \alpha\ 
					\left(\ R(\beta + 1) \subset \operatorname{P}(R(\beta))\ \right)
					\label{eq:thm_R_alpha_plus_1_equals_to_power_of_R_alpha}
				\end{align}
				が成り立つと仮定する.$x$を$R(\alpha + 1)$の任意の要素とすれば,前段の結果より
				\begin{align}
					x \in R(\alpha) \vee x \subset R(\alpha)
				\end{align}
				となる.$x \in R(\alpha)$であるとき,$\alpha$の或る要素$\beta$に対し
				$x \in R(\beta)$となる.前段の結果より$x \in R(\beta + 1)$となり,
				(\refeq{eq:thm_R_alpha_plus_1_equals_to_power_of_R_alpha})より
				$x \subset R(\beta)$となるが,
				\begin{align}
					x \subset R(\beta) &\Longrightarrow x \subset R(\alpha), \\
					x \subset R(\alpha) &\Longrightarrow x \in \operatorname{P}(R(\alpha))
				\end{align}
				と併せて$x \in \operatorname{P}(R(\alpha))$が成り立つ.
				一方で$x \subset R(\alpha)$であるときも$x \in \operatorname{P}(R(\alpha))$
				となるから
				\begin{align}
					R(\alpha + 1) \subset \operatorname{P}(R(\alpha))
				\end{align}
				が従う.超限帰納法より定理の主張が得られる.
		\end{description}
	\end{prf}
	
	\begin{screen}
		\begin{dfn}[整礎集合]
			$\bigcup_{\alpha \in \ON} R(\alpha)$の要素を{\bf 整礎集合}
			\index{せいそしゅうごう@整礎集合}{\bf (well-founded set)}と呼ぶ.
		\end{dfn}
	\end{screen}
	
	\begin{screen}
		\begin{thm}[すべての集合は整礎的である]\label{thm:every_set_is_well_founded}
			次は定理である:
			\begin{align}
				\Univ = \bigcup_{\alpha \in \ON} R(\alpha).
			\end{align}
		\end{thm}
	\end{screen}
	
	\begin{prf}
		いま,$S$を$\ON$の空でない部分集合として
		\begin{align}
			V \neq \bigcup_{\alpha \in S} R(\alpha)
			\Longrightarrow S \neq \ON
		\end{align}
		が成り立つことを示す.$V \neq \bigcup_{\alpha \in S} R(\alpha)$であれば
		正則性公理より或る集合$a$が存在して
		\begin{align}
			a \in V \backslash \bigcup_{\alpha \in S} R(\alpha)
			\wedge a \cap V \backslash \bigcup_{\alpha \in S} R(\alpha) = \emptyset
		\end{align}
		を満たす.このとき
		\begin{align}
			a \in \bigcup_{\alpha \in S} R(\alpha) \wedge a \subset \bigcup_{\alpha \in S} R(\alpha)
		\end{align}
		となる.ここで
		\begin{align}
			f = \Set{x}{\exists s \in a\ (\ x = (s,\mu \alpha (s \in R(\alpha)))\ )}
		\end{align}
		と定めれば$f:a \longrightarrow \ON$が成り立つ.
		$\beta = \bigcup f(a)$とおけば$\beta$は$\ON$に属する.このとき
		\begin{align}
			\forall t\ (\ t \in a \Longrightarrow t \in R(f(t))
			\Longrightarrow t \in R(\beta)\ )
		\end{align}
		となるから$a \subset R(\beta)$,そして定理\ref{thm:R_alpha_plus_1_equals_to_power_of_R_alpha}
		より$a \in R(\beta + 1)$が従う.
		\begin{align}
			\forall \alpha \in S\ (\ a \notin R(\alpha)\ )
		\end{align}
		であったから$\beta + 1 \notin S$であり,ゆえに$S \neq \ON$となる.
		定理の主張は対偶を取れば得られる.
		\QED
	\end{prf}
	
	\monologue{
		院生「\begin{align}
				\Univ = \bigcup_{\alpha \in \ON} R(\alpha)
			\end{align}
			という美しい式は偶然得られた訳ではありません.John Von Neumann はこの結果を
			予定して正則性公理を導入したのです.」
	}
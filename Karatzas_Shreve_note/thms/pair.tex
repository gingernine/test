\section{対}
	$a$と$b$を類とするとき,$a$か$b$の少なくとも一方に等しい集合の全体,つまり
	\begin{align}
		a = x \vee b = x
	\end{align}
	を満たす全ての集合$x$を集めたものを$a$と$b$の対と呼び
	\begin{align}
		\{a,b\}
	\end{align}
	と書く.解釈としては``$a$と$b$のみを要素とする類''のことであり,当然$a$が集合であるならば
	\begin{align}
		a \in \{a,b\}
	\end{align}
	が成立する.しかし$a$と$b$が共に真類であるときは,いかなる集合も$a$にも$b$にも等しくないため
	\begin{align}
		\{a,b\} = \emptyset
	\end{align}
	となる.大雑把に対を紹介したが,この辺の事情の詳細は後述する.
	
	\begin{screen}
		\begin{dfn}[対]
			$a,b$を類とするとき,
			\begin{align}
				\{a,b\} \defeq \Set{x}{a = x \vee b = x}
			\end{align}
			で$\{a,b\}$を定義し,これを$a$と$b$の{\bf 対}\index{つい@対}{\bf (pair)}と呼ぶ.
			特に$\{a,a\}$を$\{a\}$と書く.
		\end{dfn}
	\end{screen}
	
	\begin{screen}
		\begin{thm}[対は表示されている要素しか持たない]
		\label{thm:pair_members_are_exactly_the_given_two}
			$a$と$b$を類とするとき次が成立する:
			\begin{align}
				\forall x\, (\, x \in \{a,b\} \Longleftrightarrow a=x \vee b=x\, ).
			\end{align}
		\end{thm}
	\end{screen}
	
	この定理はメタ的な定理\ref{metathm:rewritten_formula_is_equivalent}を適用しただけの主張であるが,
	直接確認することも出来る.実際,$a$が
	\begin{align}
		\Set{x}{A(x)}
	\end{align}
	で表される類で,$b$が
	\begin{align}
		\Set{x}{B(x)}
	\end{align}
	で表される類であるとき,$\chi$を$\mathcal{L}$の任意の対象とすれば
	\begin{align}
		\forall t\, (\, A(t) \Longleftrightarrow t \in \chi\, ) \Longleftrightarrow a = \chi
	\end{align}
	と
	\begin{align}
		\forall t\, (\, B(t) \Longleftrightarrow t \in \chi\, ) \Longleftrightarrow b = \chi
	\end{align}
	から
	\begin{align}
		\left[\, \forall t\, (\, A(t) \Longleftrightarrow t \in \chi\, ) \vee
		\forall t\, (\, B(t) \Longleftrightarrow t \in \chi\, )\, \right]
		\Longleftrightarrow
		\left[\, a = \chi \vee b = \chi\, \right]
	\end{align}
	が成り立つので,
	\begin{align}
		\forall x\, \left[\, \left[\, \forall t\, (\, A(t) \Longleftrightarrow t \in x\, ) \vee
		\forall t\, (\, B(t) \Longleftrightarrow t \in x\, )\, \right]
		\Longleftrightarrow
		\left[\, a = x \vee b = x\, \right]\, \right]
	\end{align}
	が成立する.
	
	\begin{screen}
		\begin{thm}[要素の表示の順番を入れ替えても対は等しい]\label{thm:commutative_law_of_pairs}
			$a$と$b$を類とするとき
			\begin{align}
				\{a,b\} = \{b,a\}.
			\end{align}
		\end{thm}
	\end{screen}
	
	\begin{sketch}
		$\tau$を$\mathcal{L}$の任意の対象とする.定理\ref{thm:pair_members_are_exactly_the_given_two}より
		\begin{align}
			\tau \in \{a,b\} \Longleftrightarrow \tau = a \vee \tau = b
		\end{align}
		が成立し,推論法則\ref{logicalthm:commutative_law_of_disjunction_and_conjunction}より
		\begin{align}
			\tau = a \vee \tau = b \Longleftrightarrow \tau = b \vee \tau = a
		\end{align}
		が成立し,定理\ref{thm:pair_members_are_exactly_the_given_two}と推論法則\ref{logicalthm:commutative_law_of_equivalence}より
		\begin{align}
			\tau = b \vee \tau = a \Longleftrightarrow \tau \in \{b,a\}
		\end{align}
		が成立する.そして含意の推移律から
		\begin{align}
			\tau \in \{a,b\} \Longleftrightarrow \tau \in \{b,a\}
		\end{align}
		が従う.$\tau$は任意に与えられていたから
		\begin{align}
			\forall t\, \left(\, t \in \{a,b\} \Longleftrightarrow t \in \{b,a\}\, \right)
		\end{align}
		が従い,外延性の公理より
		\begin{align}
			\{a,b\} = \{b,a\}
		\end{align}
		が出る.
		\QED
	\end{sketch}
		
	\begin{screen}
		\begin{axm}[対の公理]
			集合同士の対は集合である.つまり,$a$と$b$を集合とするとき
			\begin{align}
				\set{\{a,b\}}.
			\end{align}
		\end{axm}
	\end{screen}
	
	\monologue{
		対の公理の主張は,$a$と$b$を類とするとき
		\begin{align}
			\set{a} \wedge \set{b} \Longrightarrow \set{\{a,b\}}
		\end{align}
		が成り立つということですが,式にまとめてしまうと見づらいのではじめから$a$と$b$を集合としています.
	}
	
	\begin{screen}
		\begin{logicalthm}[量化記号の性質(ロ)]\label{logicalthm:properties_of_quantifiers_2}
			$A,B$を$\mathcal{L}'$の式とし,$x$を$A,B$に現れる文字とするとき,$x$のみが$A,B$で量化されていないならば以下は定理である:
			\begin{description}
				\item[(a)] $\exists x ( A(x) \vee B(x) ) \Longleftrightarrow \exists x A(x) \vee \exists x B(x)$.
				
				\item[(b)] $\forall x ( A(x) \wedge B(x) ) \Longleftrightarrow \forall x A(x) \wedge \forall x B(x)$.
			\end{description}
		\end{logicalthm}
	\end{screen}
	
	\begin{prf}\mbox{}
		\begin{description}
			\item[(a)]
				いま$c(x) \overset{\mathrm{def}}{\Longleftrightarrow} A(x) \vee B(x)$とおけば,
				$\exists x ( A(x) \vee B(x) )$と$\exists x ( C(x) )$は同じ記号列であるから
				\begin{align}
					\exists x ( A(x) \vee B(x) ) \Longrightarrow \exists x C(x)
					\label{eq:logicalthm_properties_of_quantifiers_1}
				\end{align}
				が成立する.また推論法則\ref{logicalthm:transitive_law_of_implication}より
				\begin{align}
					\exists x C(x) \Longrightarrow C(\varepsilon x C(x))
					\label{eq:logicalthm_properties_of_quantifiers_2}
				\end{align}
				が成立する.$C(\varepsilon x C(x))$と$A(\varepsilon x C(x)) \vee B(\varepsilon x C(x))$
				は同じ記号列であるから
				\begin{align}
					C(\varepsilon x C(x)) \Longrightarrow A(\varepsilon x C(x)) \vee B(\varepsilon x C(x))
					\label{eq:logicalthm_properties_of_quantifiers_3}
				\end{align}
				が成立する.ここで推論法則\ref{logicalthm:transitive_law_of_implication}と
				推論規則\ref{logicalaxm:fundamental_rules_of_inference}より
				\begin{align}
					A(\varepsilon x C(x)) &\Longrightarrow \exists x A(x) \\
						&\Longrightarrow \exists x A(x) \vee \exists x B(x), \\
					B(\varepsilon x C(x)) &\Longrightarrow \exists x B(x) \\
						&\Longrightarrow \exists x A(x) \vee \exists x B(x)
				\end{align}
				が成立するので,場合分け法則より
				\begin{align}
					A(\varepsilon x C(x)) \vee B(\varepsilon x C(x))
					\Longrightarrow \exists x A(x) \vee \exists x B(x)
					\label{eq:logicalthm_properties_of_quantifiers_4}
				\end{align}
				が成り立つ.(\refeq{eq:logicalthm_properties_of_quantifiers_1})
				(\refeq{eq:logicalthm_properties_of_quantifiers_2})
				(\refeq{eq:logicalthm_properties_of_quantifiers_3})
				(\refeq{eq:logicalthm_properties_of_quantifiers_4})
				に推論法則\ref{logicalthm:transitive_law_of_implication}を順次適用すれば
				\begin{align}
					\exists x ( A(x) \vee B(x) ) \Longrightarrow \exists x A(x) \vee \exists x B(x)
				\end{align}
				が得られる.他方,推論規則\ref{logicalaxm:rules_of_quantifiers}より
				\begin{align}
					\exists x A(x) &\Longrightarrow A(\varepsilon x A(x)) \\
						&\Longrightarrow A(\varepsilon x A(x)) \vee B(\varepsilon x A(x)) \\
						&\Longrightarrow C(\varepsilon x A(x)) \\
						&\Longrightarrow C(\varepsilon x C(x)) \\
						&\Longrightarrow \exists x C(x) \\
						&\Longrightarrow \exists x (A(x) \vee B(x))
				\end{align}
				が成立し,$A$を$B$に置き換えれば
				$\exists x B(x) \Longrightarrow \exists x (A(x) \vee B(x))$も成り立つので,
				場合分け法則より
				\begin{align}
					\exists x A(x) \vee \exists x B(x) \Longrightarrow \exists x (A(x) \vee B(x))
				\end{align}
				も得られる.
			
			\item[(b)]
				簡略して説明すれば
				\begin{align}
					\forall x \left( A(x) \wedge B(x) \right)
					&\Longleftrightarrow\ \rightharpoondown \exists x \rightharpoondown \left( A(x) \wedge B(x) \right) & (\mbox{推論法則\ref{logicalthm:De_Morgan_law_for_quantifiers}}) \\
					&\Longleftrightarrow\ \rightharpoondown \exists x \left( \rightharpoondown A(x) \vee \rightharpoondown B(x) \right) & (\mbox{De Morganの法則}) \\
					&\Longleftrightarrow\ \rightharpoondown \left( \exists x \rightharpoondown A(x) \vee \exists x \rightharpoondown B(x) \right) & (\mbox{前段の対偶}) \\
					&\Longleftrightarrow\ \rightharpoondown \left( \rightharpoondown \forall x A(x) \vee \rightharpoondown \forall x B(x) \right) & (\mbox{推論法則\ref{logicalthm:De_Morgan_law_for_quantifiers}}) \\
					&\Longleftrightarrow\ \rightharpoondown \rightharpoondown \forall x A(x) \wedge \rightharpoondown \rightharpoondown \forall x B(x) & (\mbox{De Morganの法則}) \\
					&\Longleftrightarrow \forall x A(x) \wedge \forall x B(x) &(\mbox{二重否定の法則})
				\end{align}
				となる.
				\QED
		\end{description}
	\end{prf}
	
	\begin{screen}
		\begin{thm}[集合は対の要素たりうる]\label{thm:set_is_an_element_of_its_pair}
			$a,b$を類とするとき,
			\begin{align}
				\set{a} \Longrightarrow a \in \{a,b\}.
			\end{align}
		\end{thm}
	\end{screen}
	
	\begin{sketch}
		いま
		\begin{align}
			\set{a}
		\end{align}
		が成り立っているとする.ここで
		\begin{align}
			\tau \defeq \varepsilon x\, (\, a = x\, )
		\end{align}
		とおけば,存在記号に関する規則より
		\begin{align}
			a = \tau
		\end{align}
		が成り立つ.ゆえに
		\begin{align}
			\tau = a \vee \tau = b
		\end{align}
		も成り立つ.ゆえに
		\begin{align}
			\tau \in \{a,b\}
		\end{align}
		が成り立ち,相当性の公理より
		\begin{align}
			a \in \{a,b\}
		\end{align}
		が従う.そして演繹法則より\
		\begin{align}
			\set{a} \Longrightarrow a \in \{a,b\}
		\end{align}
		が得られる.
		\QED
	\end{sketch}
	
	$a$を集合とすれば対の公理より$\{a\}$も集合となり,定理\ref{thm:set_is_an_element_of_its_pair}より
	\begin{align}
		a \in \{a\}
	\end{align}
	が成立する.
	
	\begin{screen}
		\begin{thm}[真類同士の対は空]\label{thm:pair_of_proper_classes_is_emptyset}
			$a,b$を類とするとき,
			\begin{align}
				\rightharpoondown \set{a} \wedge \rightharpoondown \set{b} \Longleftrightarrow \{a,b\} = \emptyset.
			\end{align}
		\end{thm}
	\end{screen}
	
	\begin{sketch}
		いま
		\begin{align}
			\rightharpoondown \set{a} \wedge \rightharpoondown \set{b}
		\end{align}
		が成り立っているとする.このとき
		\begin{align}
			\forall x\, (\, a \neq x\, ) \wedge \forall x\, (\, b \neq x\, )
		\end{align}
		が成立し,推論法則\ref{logicalthm:properties_of_quantifiers_2}より
		\begin{align}
			\forall x\, (\, a \neq x \wedge b \neq x\, )
		\end{align}
		が成立する.すなわち$\chi$を$\mathcal{L}$の任意の対象とすれば
		\begin{align}
			a \neq \chi \wedge b \neq \chi
		\end{align}
		が成立する.他方で
		\begin{align}
			a \neq \chi \wedge b \neq \chi \Longleftrightarrow \chi \notin \{a,b\}
		\end{align}
		も満たされているので,三段論法より
		\begin{align}
			\chi \notin \{a,b\}
		\end{align}
		が成立する.$\chi$の任意性より
		\begin{align}
			\forall x\, (\, x \notin \{a,b\}\, )
		\end{align}
		が成立し,定理\ref{thm:uniqueness_of_emptyset}
		\begin{align}
			\{a,b\} = \emptyset
		\end{align}
		が従う.そして演繹法則を適用して
		\begin{align}
			\rightharpoondown \set{a} \wedge \rightharpoondown \set{b} \Longrightarrow \{a,b\} = \emptyset
		\end{align}
		が得られる.逆に,定理\ref{thm:set_is_an_element_of_its_pair}から
		\begin{align}
			\set{a} \Longrightarrow a \in \{a,b\}
		\end{align}
		が成り立ち,定理\ref{thm:emptyset_does_not_contain_any_class}から
		\begin{align}
			a \in \{a,b\} \Longrightarrow \{a,b\} \neq \emptyset
		\end{align}
		が成り立つので,含意の推移律より
		\begin{align}
			\set{a} \Longrightarrow \{a,b\} \neq \emptyset
		\end{align}
		が成立する.同様に
		\begin{align}
			\set{b} \Longrightarrow \{a,b\} \neq \emptyset
		\end{align}
		も成り立つから,場合分け法則より
		\begin{align}
			\set{a} \vee \set{b} \Longrightarrow \{a,b\} \neq \emptyset
		\end{align}
		が成立し,この対偶を取って
		\begin{align}
			\{a,b\} = \emptyset \Longrightarrow\ \rightharpoondown \set{a} \wedge \rightharpoondown \set{b}
		\end{align}
		が得られる.
		\QED
	\end{sketch}
	
\subsection{対}
	\begin{screen}
		\begin{metathm}[量化記号の性質(ロ)]\label{metathm:properties_of_quantifiers_2}
			$A,B$を$\mathcal{L}'$の式とし,$x$を$A,B$に現れる文字とするとき,$x$のみが$A,B$で量化されていないならば以下は定理である:
			\begin{description}
				\item[(a)] $\exists x ( A(x) \vee B(x) ) \Longleftrightarrow \exists x A(x) \vee \exists x B(x)$.
				
				\item[(b)] $\forall x ( A(x) \wedge B(x) ) \Longleftrightarrow \forall x A(x) \wedge \forall x B(x)$.
			\end{description}
		\end{metathm}
	\end{screen}
	
	\begin{prf}\mbox{}
		\begin{description}
			\item[(a)]
				いま$c(x) \overset{\mathrm{def}}{\Longleftrightarrow} A(x) \vee B(x)$とおけば,
				$\exists x ( A(x) \vee B(x) )$と$\exists x ( C(x) )$は同じ記号列であるから
				\begin{align}
					\exists x ( A(x) \vee B(x) ) \Longrightarrow \exists x C(x)
					\label{eq:metathm_properties_of_quantifiers_1}
				\end{align}
				が成立する.また推論法則\ref{metathm:transitive_law_of_implication}より
				\begin{align}
					\exists x C(x) \Longrightarrow C(\varepsilon x C(x))
					\label{eq:metathm_properties_of_quantifiers_2}
				\end{align}
				が成立する.$C(\varepsilon x C(x))$と$A(\varepsilon x C(x)) \vee B(\varepsilon x C(x))$
				は同じ記号列であるから
				\begin{align}
					C(\varepsilon x C(x)) \Longrightarrow A(\varepsilon x C(x)) \vee B(\varepsilon x C(x))
					\label{eq:metathm_properties_of_quantifiers_3}
				\end{align}
				が成立する.ここで推論法則\ref{metathm:transitive_law_of_implication}と
				推論規則\ref{metaaxm:fundamental_rules_of_inference}より
				\begin{align}
					A(\varepsilon x C(x)) &\Longrightarrow \exists x A(x) \\
						&\Longrightarrow \exists x A(x) \vee \exists x B(x), \\
					B(\varepsilon x C(x)) &\Longrightarrow \exists x B(x) \\
						&\Longrightarrow \exists x A(x) \vee \exists x B(x)
				\end{align}
				が成立するので,場合分け法則より
				\begin{align}
					A(\varepsilon x C(x)) \vee B(\varepsilon x C(x))
					\Longrightarrow \exists x A(x) \vee \exists x B(x)
					\label{eq:metathm_properties_of_quantifiers_4}
				\end{align}
				が成り立つ.(\refeq{eq:metathm_properties_of_quantifiers_1})
				(\refeq{eq:metathm_properties_of_quantifiers_2})
				(\refeq{eq:metathm_properties_of_quantifiers_3})
				(\refeq{eq:metathm_properties_of_quantifiers_4})
				に推論法則\ref{metathm:transitive_law_of_implication}を順次適用すれば
				\begin{align}
					\exists x ( A(x) \vee B(x) ) \Longrightarrow \exists x A(x) \vee \exists x B(x)
				\end{align}
				が得られる.他方,推論規則\ref{metaaxm:rules_of_quantifiers}より
				\begin{align}
					\exists x A(x) &\Longrightarrow A(\varepsilon x A(x)) \\
						&\Longrightarrow A(\varepsilon x A(x)) \vee B(\varepsilon x A(x)) \\
						&\Longrightarrow C(\varepsilon x A(x)) \\
						&\Longrightarrow C(\varepsilon x C(x)) \\
						&\Longrightarrow \exists x C(x) \\
						&\Longrightarrow \exists x (A(x) \vee B(x))
				\end{align}
				が成立し,$A$を$B$に置き換えれば
				$\exists x B(x) \Longrightarrow \exists x (A(x) \vee B(x))$も成り立つので,
				場合分け法則より
				\begin{align}
					\exists x A(x) \vee \exists x B(x) \Longrightarrow \exists x (A(x) \vee B(x))
				\end{align}
				も得られる.
			
			\item[(b)]
				簡略して説明すれば
				\begin{align}
					\forall x \left( A(x) \wedge B(x) \right)
					&\Longleftrightarrow\ \rightharpoondown \exists x \rightharpoondown \left( A(x) \wedge B(x) \right) & (\mbox{推論法則\ref{metathm:properties_of_quantifiers}(c)の対偶}) \\
					&\Longleftrightarrow\ \rightharpoondown \exists x \left( \rightharpoondown A(x) \vee \rightharpoondown B(x) \right) & (\mbox{De Morganの法則}) \\
					&\Longleftrightarrow\ \rightharpoondown \left( \exists x \rightharpoondown A(x) \vee \exists x \rightharpoondown B(x) \right) & (\mbox{前段の対偶}) \\
					&\Longleftrightarrow\ \rightharpoondown \left( \rightharpoondown \forall x A(x) \vee \rightharpoondown \forall x B(x) \right) & (\mbox{推論法則\ref{metathm:properties_of_quantifiers}(c)}) \\
					&\Longleftrightarrow\ \rightharpoondown \rightharpoondown \forall x A(x) \wedge \rightharpoondown \rightharpoondown \forall x B(x) & (\mbox{De Morganの法則}) \\
					&\Longleftrightarrow \forall x A(x) \wedge \forall x B(x) &(\mbox{二重否定の法則})
				\end{align}
				となる.
				\QED
		\end{description}
	\end{prf}
	
	\monologue{
		院生「我々は$\mathcal{L}$の式$A$を用いて$\Set{x}{A(x)}$の記法を導入しましたが,
			$\mathcal{L}'$の式$B$に対しても$\Set{x}{B(x)}$の形で書けると便利なことが多いです.
			ただし後者の記法は$B$と同値な$\mathcal{L}$の式$B'$によって
			\begin{align}
				\Set{x}{B(x)} \coloneqq \Set{x}{B'(x)}
			\end{align}
			で定められるものとします.$\mathcal{L}'$の式が与えられたらそれを
			或る手続きで$\mathcal{L}$の式に書き換えていくのですが,
			そこで真価を発揮するのは$\varepsilon$記号です.」
	}
	
	$a$を類とするとき,$a$は$\mathcal{L}$の対象であるか$\Set{x}{A(x)}$の形をしている.そこで,文字$x$に対し
	\begin{itemize}
		\item $a$が$\mathcal{L}$の対象ならば$\varepsilon a(x) \overset{\mathrm{def}}{\Longleftrightarrow} x \in a$,
		\item $a$が$\Set{x}{A(x)}$の形をしていれば$\varepsilon a(x) \overset{\mathrm{def}}{\Longleftrightarrow} A(x)$,
	\end{itemize}
	として記号列$\varepsilon a(x)$を定める.この記法は
	\begin{align}
		\forall x\, (\, \varepsilon a(x) \Longleftrightarrow x \in a\, )
		\label{eq:a_meaning_of_epsilon_notation}
	\end{align}
	を満たすことを意図している.$\varepsilon$記号を用いているのは,
	量化記号に関する推論規則で$\varepsilon$記号を定めたときと導入の動機が似ているためである.
	
	次に$B$を$\mathcal{L}'$の式として,$B$を$\mathcal{L}$の式に書き換える手続きを指定する.
	\begin{description}
		\item[step1] $B$が$\mathcal{L}$の式であるとき,
			\begin{align}
				\mathcal{L}B \overset{\mathrm{def}}{\Longleftrightarrow} B
			\end{align}
			と定める.そうでない場合の対応を以下に示す.
			
		\item[step2] $s,t$を$\mathcal{L}'$の項として,$B$が
			\begin{align}
				s \in t
			\end{align}
			であるとき,
			\begin{itemize}
				\item $s,t$が共に$\mathcal{L}$の項であるとき
					\begin{align}
						\mathcal{L}B \overset{\mathrm{def}}{\Longleftrightarrow}
						s \in t,
					\end{align}
				
				\item $s$が$\mathcal{L}$の項ではなく,$t$が$\mathcal{L}$の項であるとき
					\begin{align}
						\mathcal{L}B \overset{\mathrm{def}}{\Longleftrightarrow}
						\varepsilon x\, (\, s=x\, ) \in t,
					\end{align}
				
				\item $s$が$\mathcal{L}$の項であり,$t$が$\mathcal{L}$の項でないとき
					\begin{align}
						\mathcal{L}B \overset{\mathrm{def}}{\Longleftrightarrow}
						\varepsilon t(s),
					\end{align}
					
				\item $s$も$t$も$\mathcal{L}$の項でないとき
					\begin{align}
						\mathcal{L}B \overset{\mathrm{def}}{\Longleftrightarrow}
						\varepsilon t(\varepsilon x\, (\, s=x\, )),
					\end{align}
			\end{itemize}
			と定める.一方で$B$が
			\begin{align}
				s = t
			\end{align}
			であるとき,
			\begin{itemize}
				\item $s,t$が共に$\mathcal{L}$の項であるとき
					\begin{align}
						\mathcal{L}B \overset{\mathrm{def}}{\Longleftrightarrow}
						s = t,
					\end{align}
				
				\item $s$が$\mathcal{L}$の項ではなく,$t$が$\mathcal{L}$の項であるとき,
					$s$が集合なら
					\begin{align}
						\mathcal{L}B \overset{\mathrm{def}}{\Longleftrightarrow}
						\varepsilon x\, (\, s=x\, ) = t,
					\end{align}
					$s$が真類なら
					\begin{align}
						\mathcal{L}B \overset{\mathrm{def}}{\Longleftrightarrow}
						\forall u\, \left(\, \varepsilon s(u) \Longleftrightarrow u \in t\, \right),
					\end{align}
				
				\item $s$が$\mathcal{L}$の項であり,$t$が$\mathcal{L}$の項でないとき,
					$t$が集合なら
					\begin{align}
						\mathcal{L}B \overset{\mathrm{def}}{\Longleftrightarrow}
						s = \varepsilon x\, (\, t=x\, ),
					\end{align}
					$t$が真類なら
					\begin{align}
						\mathcal{L}B \overset{\mathrm{def}}{\Longleftrightarrow}
						\forall u\, \left(\, u \in s \Longleftrightarrow \varepsilon t(u)\, \right),
					\end{align}
					
				\item $s$も$t$も$\mathcal{L}$の項でないとき,
					$s,t$がどちらも集合なら
					\begin{align}
						\mathcal{L}B \overset{\mathrm{def}}{\Longleftrightarrow}
						\varepsilon x\, (\, s=x\, ) = \varepsilon x\, (\, t=x\, ),
					\end{align}
					$s,t$の一方でも真類なら
					\begin{align}
						\mathcal{L}B \overset{\mathrm{def}}{\Longleftrightarrow}
						\forall u\, \left(\, \varepsilon s(u) \Longleftrightarrow \varepsilon t(u)\, \right),
					\end{align}
			\end{itemize}
			と定める.
			
		\item[step3] $P,Q$を$\mathcal{L}'$の式として,$B$が
			\begin{align}
				P \vee Q,\quad P \wedge Q,\quad P \Longrightarrow Q,\quad \rightharpoondown P
			\end{align}
			であるとき,
			\begin{align}
				\mathcal{L}P,\mathcal{L}Q
			\end{align}
			をそれぞれ$P,Q$から得られた$\mathcal{L}$の式として,各場合に応じて
			\begin{align}
				\mathcal{L}B \overset{\mathrm{def}}{\Longleftrightarrow}
				\begin{cases}
					\mathcal{L}P \vee \mathcal{L}Q & \\
					\mathcal{L}P \wedge \mathcal{L}Q & \\
					\mathcal{L}P \Longrightarrow \mathcal{L}Q & \\
					\rightharpoondown \mathcal{L}P &
				\end{cases}
			\end{align}
			と定める.
	\end{description}
	
	
	
	\begin{screen}
		\begin{dfn}[対]
			$a,b$を類とするとき,
			\begin{align}
				\{a,b\} \coloneqq \Set{x}{
					\forall s\, \left(\, s \in x \Longleftrightarrow \varepsilon a(s)\, \right) \vee 
					\forall t\, \left(\, t \in x \Longleftrightarrow \varepsilon b(t)\, \right)}
			\end{align}
			で$\{a,b\}$を定義し,これを$a$と$b$の{\bf 対}\index{つい@対}{\bf (pair)}と呼ぶ.
			特に$\{a,a\}$を$\{a\}$と書く.
		\end{dfn}
	\end{screen}
	
	$a,b$を類とするとき
	\begin{align}
		\forall x\, \left(\, x = a \vee x = b \Longleftrightarrow 
			\forall s\, \left(\, s \in x \Longleftrightarrow \varepsilon a(s)\, \right) \vee 
			\forall t\, \left(\, t \in x \Longleftrightarrow \varepsilon b(t)\, \right)\, 
		\right)
	\end{align}
	が成立する.従って
	\begin{align}
		\forall x\, (\, x=a \vee x=b \Longleftrightarrow x \in \{a,b\}\, )
		\label{eq:definition_of_a_pair_of_classes}
	\end{align}
	が成立する.これで次の定理が得られた:
	
	
	\begin{screen}
		\begin{thm}[対はそこに書かれている要素しか持たない]
		\label{thm:pair_members_are_exactly_the_given_two}
			$a$と$b$を類とするとき次が成立する:
			\begin{align}
				\forall x\, (\, x \in \{a,b\} \Longleftrightarrow x=a \vee x=b\, ).
			\end{align}
		\end{thm}
	\end{screen}
	
	
	以上を根拠にして,$\{a,b\}$を
	\begin{align}
		\Set{x}{x = a \vee x = b}
	\end{align}
	と表しても良いことにする.もとより,対の定義はこちらの表記を正当化することを予期したものである.
	
	\monologue{
		院生「一般の類$a,b$に対して,本来
			\begin{align}
				\Set{x}{x = a \vee x = b}
			\end{align}
			は類として失格です.なぜならば,$a,b$の一方でも$\mathcal{L}$の対象ではない場合はそもそも
			\begin{align}
				x = a \vee x = b
			\end{align}
			が$\mathcal{L}$の式でないからです.しかしいちいち$\varepsilon$記号を使っていては見た目が煩雑になりますから,
			表記上は
			\begin{align}
				\Set{x}{x = a \vee x = b}
			\end{align}
			も認めるのです.以後もこのように妥協する場面に直面するでしょうが,
			しかし$\varepsilon$記号を用いれば正式な形に書き直せるのですから解釈上の不具合は無いのです.」
	}
	
	\begin{screen}
		\begin{axm}[対の公理]
			集合同士の対は集合である.つまり,$a,b$を類とするとき次が成り立つ:
			\begin{align}
				\set{a} \wedge \set{b} \Longrightarrow 
				\set{\{a,b\}}.
			\end{align}
		\end{axm}
	\end{screen}
	
	\begin{screen}
		\begin{thm}[真類の対は空]
		\label{thm:pair_of_proper_classes_is_emptyset}
			$a,b$を類とするとき次が成り立つ:
			\begin{description}
				\item[(イ)] $\set{a} \Longrightarrow a \in \{a,b\}.$
				
				\item[(ロ)] $\rightharpoondown \set{a} \wedge \rightharpoondown \set{b} \Longleftrightarrow \{a,b\} = \emptyset.$
			\end{description}
		\end{thm}
	\end{screen}
	
	\begin{prf}\mbox{}
		\begin{description}
			\item[(イ)]
				まず存在記号に関する規則より
				\begin{align}
					\set{a} \Longrightarrow a = \varepsilon x(\, a = x\, )
				\end{align}
				も成り立つ.ここで$\set{a}$が成り立っていると仮定して$\tau \coloneqq \varepsilon x(\, a = x\, )$とおけば,
				三段論法より$\tau = a$が成立し,$\vee$の導入より$\tau = a \vee \tau = b$が成り立つ.
				(\refeq{eq:definition_of_a_pair_of_classes})と
				推論法則\ref{metathm:fundamental_law_of_universal_quantifier}より
				$\mathcal{L}$の対象である$\tau$に対しては
				\begin{align}
					\tau = a \vee \tau = b \Longleftrightarrow \tau \in \{a,b\}
				\end{align}
				が満たされるので,三段論法より$\tau \in \{a,b\}$が成り立ち,相等性の公理より
				\begin{align}
					a \in \{a,b\}
				\end{align}
				が従う.ここに演繹法則を適用すれば(i)が得られる.
			
			\item[(ロ)]
				いま$\rightharpoondown \set{a} \wedge \rightharpoondown \set{b}$が成り立っているとする.
				このとき推論法則\ref{metathm:properties_of_quantifiers}より
				\begin{align}
					\forall x\, (\, a \neq x\, ) \wedge \forall x\, (\, b \neq x\, )
				\end{align}
				が成り立ち,推論法則\ref{metathm:properties_of_quantifiers_2}より
				\begin{align}
					\forall x\, (\, a \neq x \wedge b \neq x\, )
				\end{align}
				が成立する.ここで$\chi$を$\mathcal{L}$の任意の対象とすれば,
				(\refeq{eq:definition_of_a_pair_of_classes})と
				推論法則\ref{metathm:fundamental_law_of_universal_quantifier}より
				\begin{align}
					a = \chi \vee b = \chi \Longleftrightarrow \chi \in \{a,b\}
				\end{align}
				が成立し,$\wedge$の除去と対偶命題の同値性から
				\begin{align}
					a \neq \chi \wedge b \neq \chi \Longrightarrow \chi \notin \{a,b\}
				\end{align}
				が成り立つ.いま$a \neq \chi \wedge b \neq \chi$が満たされているので三段論法より
				$\chi \notin \{a,b\}$が成立し,$\chi$の任意性と
				推論法則\ref{metathm:fundamental_law_of_universal_quantifier}より
				\begin{align}
					\forall x\, (\, x \notin \{a,b\}\, )
				\end{align}
				が成立する.このとき定理\ref{thm:uniqueness_of_emptyset}より$\{a,b\} = \emptyset$が従うので,演繹法則を適用して
				\begin{align}
					\rightharpoondown \set{a} \wedge \rightharpoondown \set{b} \Longrightarrow \{a,b\} = \emptyset
				\end{align}
				が得られる.一方で(i)の結果と定理\ref{thm:emptyset_does_not_contain_any_class}より
				\begin{align}
					\set{a} \Longrightarrow a \in \{a,b\} \Longrightarrow \{a,b\} \neq \emptyset
				\end{align}
				が成り立ち,同様に$\set{b} \Longrightarrow \{a,b\} \neq \emptyset$も成り立つので
				場合分け法則より
				\begin{align}
					\set{a} \vee \set{b} \Longrightarrow \{a,b\} \neq \emptyset
				\end{align}
				が成立する.この対偶を取りDe Morganの法則を適用すれば
				\begin{align}
					\{a,b\} = \emptyset \Longrightarrow\, \rightharpoondown \set{a} \wedge \rightharpoondown \set{b}
				\end{align}
				も得られる.
				\QED
		\end{description}
	\end{prf}
	
	\monologue{
		院生「上の定理から{\bf 集合は或る類の要素である}という真な言明が得られます.
			実際,$a$を集合とすれば$\{a\}$も集合となり,そして$a \in \{a\}$が成り立ちますね.」
	}
\section{対}
	\begin{screen}
		\begin{dfn}[対]
			$a,b$を類とするとき,
			\begin{align}
				\{a,b\} \defeq \Set{x}{a = x \vee b = x}
			\end{align}
			で$\{a,b\}$を定義し,これを$a$と$b$の{\bf 対}\index{つい@対}{\bf (pair)}と呼ぶ.
			特に$\{a,a\}$を$\{a\}$と書く.
		\end{dfn}
	\end{screen}
	
	対の定義の
	\begin{align}
		a = x \vee b = x \label{form:definition_of_pairings}
	\end{align}
	の部分は$\mathcal{L}$の式とは限らない.例えば$a$が
	\begin{align}
		\Set{x}{A(x)}
	\end{align}
	で表される類で,$b$が
	\begin{align}
		\Set{x}{B(x)}
	\end{align}
	で表される類であるとき,(\refeq{form:definition_of_pairings})は次の$\mathcal{L}$の式
	\begin{align}
		\forall t\, (\, A(t) \Longleftrightarrow t \in x\, ) \vee
		\forall t\, (\, B(t) \Longleftrightarrow t \in x\, )
	\end{align}
	と書き換えられる.そして
	\begin{align}
		\Set{x}{a = x \vee b = x}
	\end{align}
	は
	\begin{align}
		\Set{x}{\forall t\, (\, A(t) \Longleftrightarrow t \in x\, ) \vee
		\forall t\, (\, B(t) \Longleftrightarrow t \in x\, )}
	\end{align}
	によって定められているのであった.
	
	\begin{screen}
		\begin{thm}[対は表示されている要素しか持たない]
		\label{thm:pair_members_are_exactly_the_given_two}
			$a$と$b$を類とするとき次が成立する:
			\begin{align}
				\forall x\, (\, x \in \{a,b\} \Longleftrightarrow a=x \vee b=x\, ).
			\end{align}
		\end{thm}
	\end{screen}
	
	この定理はメタ的な定理\ref{metathm:rewritten_formula_is_equivalent}を適用しただけの主張であるが,
	直接確認することも出来る.実際,$a$が
	\begin{align}
		\Set{x}{A(x)}
	\end{align}
	で表される類で,$b$が
	\begin{align}
		\Set{x}{B(x)}
	\end{align}
	で表される類であるとき,$\chi$を$\mathcal{L}$の任意の対象とすれば
	\begin{align}
		\forall t\, (\, A(t) \Longleftrightarrow t \in \chi\, ) \Longleftrightarrow a = \chi
	\end{align}
	と
	\begin{align}
		\forall t\, (\, B(t) \Longleftrightarrow t \in \chi\, ) \Longleftrightarrow b = \chi
	\end{align}
	から
	\begin{align}
		\forall t\, (\, A(t) \Longleftrightarrow t \in \chi\, ) \vee
		\forall t\, (\, B(t) \Longleftrightarrow t \in \chi\, )
		\Longleftrightarrow
		a = \chi \vee b = \chi
	\end{align}
	が成り立つので,
	\begin{align}
		\forall x\, \left(\, \forall t\, (\, A(t) \Longleftrightarrow t \in x\, ) \vee
		\forall t\, (\, B(t) \Longleftrightarrow t \in x\, )
		\Longleftrightarrow
		a = x \vee a = x\, \right)
	\end{align}
	が成立する.
	
	\begin{screen}
		\begin{axm}[対の公理]
			集合同士の対は集合である.つまり,$a,b$を類とするとき次が成り立つ:
			\begin{align}
				\set{a} \wedge \set{b} \Longrightarrow 
				\set{\{a,b\}}.
			\end{align}
		\end{axm}
	\end{screen}
	
	\begin{screen}
		\begin{logicalthm}[量化記号の性質(ロ)]\label{logicalthm:properties_of_quantifiers_2}
			$A,B$を$\mathcal{L}'$の式とし,$x$を$A,B$に現れる文字とするとき,$x$のみが$A,B$で量化されていないならば以下は定理である:
			\begin{description}
				\item[(a)] $\exists x ( A(x) \vee B(x) ) \Longleftrightarrow \exists x A(x) \vee \exists x B(x)$.
				
				\item[(b)] $\forall x ( A(x) \wedge B(x) ) \Longleftrightarrow \forall x A(x) \wedge \forall x B(x)$.
			\end{description}
		\end{logicalthm}
	\end{screen}
	
	\begin{prf}\mbox{}
		\begin{description}
			\item[(a)]
				いま$c(x) \overset{\mathrm{def}}{\Longleftrightarrow} A(x) \vee B(x)$とおけば,
				$\exists x ( A(x) \vee B(x) )$と$\exists x ( C(x) )$は同じ記号列であるから
				\begin{align}
					\exists x ( A(x) \vee B(x) ) \Longrightarrow \exists x C(x)
					\label{eq:logicalthm_properties_of_quantifiers_1}
				\end{align}
				が成立する.また推論法則\ref{logicalthm:transitive_law_of_implication}より
				\begin{align}
					\exists x C(x) \Longrightarrow C(\varepsilon x C(x))
					\label{eq:logicalthm_properties_of_quantifiers_2}
				\end{align}
				が成立する.$C(\varepsilon x C(x))$と$A(\varepsilon x C(x)) \vee B(\varepsilon x C(x))$
				は同じ記号列であるから
				\begin{align}
					C(\varepsilon x C(x)) \Longrightarrow A(\varepsilon x C(x)) \vee B(\varepsilon x C(x))
					\label{eq:logicalthm_properties_of_quantifiers_3}
				\end{align}
				が成立する.ここで推論法則\ref{logicalthm:transitive_law_of_implication}と
				推論規則\ref{logicalaxm:fundamental_rules_of_inference}より
				\begin{align}
					A(\varepsilon x C(x)) &\Longrightarrow \exists x A(x) \\
						&\Longrightarrow \exists x A(x) \vee \exists x B(x), \\
					B(\varepsilon x C(x)) &\Longrightarrow \exists x B(x) \\
						&\Longrightarrow \exists x A(x) \vee \exists x B(x)
				\end{align}
				が成立するので,場合分け法則より
				\begin{align}
					A(\varepsilon x C(x)) \vee B(\varepsilon x C(x))
					\Longrightarrow \exists x A(x) \vee \exists x B(x)
					\label{eq:logicalthm_properties_of_quantifiers_4}
				\end{align}
				が成り立つ.(\refeq{eq:logicalthm_properties_of_quantifiers_1})
				(\refeq{eq:logicalthm_properties_of_quantifiers_2})
				(\refeq{eq:logicalthm_properties_of_quantifiers_3})
				(\refeq{eq:logicalthm_properties_of_quantifiers_4})
				に推論法則\ref{logicalthm:transitive_law_of_implication}を順次適用すれば
				\begin{align}
					\exists x ( A(x) \vee B(x) ) \Longrightarrow \exists x A(x) \vee \exists x B(x)
				\end{align}
				が得られる.他方,推論規則\ref{logicalaxm:rules_of_quantifiers}より
				\begin{align}
					\exists x A(x) &\Longrightarrow A(\varepsilon x A(x)) \\
						&\Longrightarrow A(\varepsilon x A(x)) \vee B(\varepsilon x A(x)) \\
						&\Longrightarrow C(\varepsilon x A(x)) \\
						&\Longrightarrow C(\varepsilon x C(x)) \\
						&\Longrightarrow \exists x C(x) \\
						&\Longrightarrow \exists x (A(x) \vee B(x))
				\end{align}
				が成立し,$A$を$B$に置き換えれば
				$\exists x B(x) \Longrightarrow \exists x (A(x) \vee B(x))$も成り立つので,
				場合分け法則より
				\begin{align}
					\exists x A(x) \vee \exists x B(x) \Longrightarrow \exists x (A(x) \vee B(x))
				\end{align}
				も得られる.
			
			\item[(b)]
				簡略して説明すれば
				\begin{align}
					\forall x \left( A(x) \wedge B(x) \right)
					&\Longleftrightarrow\ \rightharpoondown \exists x \rightharpoondown \left( A(x) \wedge B(x) \right) & (\mbox{推論法則\ref{logicalthm:properties_of_quantifiers}(c)の対偶}) \\
					&\Longleftrightarrow\ \rightharpoondown \exists x \left( \rightharpoondown A(x) \vee \rightharpoondown B(x) \right) & (\mbox{De Morganの法則}) \\
					&\Longleftrightarrow\ \rightharpoondown \left( \exists x \rightharpoondown A(x) \vee \exists x \rightharpoondown B(x) \right) & (\mbox{前段の対偶}) \\
					&\Longleftrightarrow\ \rightharpoondown \left( \rightharpoondown \forall x A(x) \vee \rightharpoondown \forall x B(x) \right) & (\mbox{推論法則\ref{logicalthm:properties_of_quantifiers}(c)}) \\
					&\Longleftrightarrow\ \rightharpoondown \rightharpoondown \forall x A(x) \wedge \rightharpoondown \rightharpoondown \forall x B(x) & (\mbox{De Morganの法則}) \\
					&\Longleftrightarrow \forall x A(x) \wedge \forall x B(x) &(\mbox{二重否定の法則})
				\end{align}
				となる.
				\QED
		\end{description}
	\end{prf}
	
	\begin{screen}
		\begin{thm}[真類の対は空]
		\label{thm:pair_of_proper_classes_is_emptyset}
			$a,b$を類とするとき次が成り立つ:
			\begin{description}
				\item[(イ)] $\set{a} \Longrightarrow a \in \{a,b\}.$
				
				\item[(ロ)] $\rightharpoondown \set{a} \wedge \rightharpoondown \set{b} \Longleftrightarrow \{a,b\} = \emptyset.$
			\end{description}
		\end{thm}
	\end{screen}
	
	\begin{prf}\mbox{}
		\begin{description}
			\item[(イ)]
				まず存在記号に関する規則より
				\begin{align}
					\set{a} \Longrightarrow a = \varepsilon x(\, a = x\, )
				\end{align}
				も成り立つ.ここで$\set{a}$が成り立っていると仮定して$\tau \coloneqq \varepsilon x(\, a = x\, )$とおけば,
				三段論法より$\tau = a$が成立し,$\vee$の導入より$\tau = a \vee \tau = b$が成り立つ.
				(\refeq{eq:definition_of_a_pair_of_classes})と
				推論法則\ref{logicalthm:fundamental_law_of_universal_quantifier}より
				$\mathcal{L}$の対象である$\tau$に対しては
				\begin{align}
					\tau = a \vee \tau = b \Longleftrightarrow \tau \in \{a,b\}
				\end{align}
				が満たされるので,三段論法より$\tau \in \{a,b\}$が成り立ち,相等性の公理より
				\begin{align}
					a \in \{a,b\}
				\end{align}
				が従う.ここに演繹法則を適用すれば(i)が得られる.
			
			\item[(ロ)]
				いま$\rightharpoondown \set{a} \wedge \rightharpoondown \set{b}$が成り立っているとする.
				このとき推論法則\ref{logicalthm:properties_of_quantifiers}より
				\begin{align}
					\forall x\, (\, a \neq x\, ) \wedge \forall x\, (\, b \neq x\, )
				\end{align}
				が成り立ち,推論法則\ref{logicalthm:properties_of_quantifiers_2}より
				\begin{align}
					\forall x\, (\, a \neq x \wedge b \neq x\, )
				\end{align}
				が成立する.ここで$\chi$を$\mathcal{L}$の任意の対象とすれば,
				(\refeq{eq:definition_of_a_pair_of_classes})と
				推論法則\ref{logicalthm:fundamental_law_of_universal_quantifier}より
				\begin{align}
					a = \chi \vee b = \chi \Longleftrightarrow \chi \in \{a,b\}
				\end{align}
				が成立し,$\wedge$の除去と対偶命題の同値性から
				\begin{align}
					a \neq \chi \wedge b \neq \chi \Longrightarrow \chi \notin \{a,b\}
				\end{align}
				が成り立つ.いま$a \neq \chi \wedge b \neq \chi$が満たされているので三段論法より
				$\chi \notin \{a,b\}$が成立し,$\chi$の任意性と
				推論法則\ref{logicalthm:fundamental_law_of_universal_quantifier}より
				\begin{align}
					\forall x\, (\, x \notin \{a,b\}\, )
				\end{align}
				が成立する.このとき定理\ref{thm:uniqueness_of_emptyset}より$\{a,b\} = \emptyset$が従うので,演繹法則を適用して
				\begin{align}
					\rightharpoondown \set{a} \wedge \rightharpoondown \set{b} \Longrightarrow \{a,b\} = \emptyset
				\end{align}
				が得られる.一方で(i)の結果と定理\ref{thm:emptyset_does_not_contain_any_class}より
				\begin{align}
					\set{a} \Longrightarrow a \in \{a,b\} \Longrightarrow \{a,b\} \neq \emptyset
				\end{align}
				が成り立ち,同様に$\set{b} \Longrightarrow \{a,b\} \neq \emptyset$も成り立つので
				場合分け法則より
				\begin{align}
					\set{a} \vee \set{b} \Longrightarrow \{a,b\} \neq \emptyset
				\end{align}
				が成立する.この対偶を取りDe Morganの法則を適用すれば
				\begin{align}
					\{a,b\} = \emptyset \Longrightarrow\, \rightharpoondown \set{a} \wedge \rightharpoondown \set{b}
				\end{align}
				も得られる.
				\QED
		\end{description}
	\end{prf}
	
	\monologue{
		上の定理から{\bf 集合は或る類の要素である}という真な言明が得られます.
		実際,$a$を集合とすれば$\{a\}$も集合となり,そして$a \in \{a\}$が成り立ちますね.
	}
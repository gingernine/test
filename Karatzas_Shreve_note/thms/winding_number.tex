\subsection{回転数}
	
	路$\gamma$が与えられたとき,$\gamma$が$a$の周りを``何周するか''という問題を考える.
	図を描けば明らかと言ってしまえば
	
	\begin{screen}
		\begin{thm}[路の偏角の連続選択]\label{thm:continuous_choice_of_arguments}
			$\gamma$を$[0,1]$上の$\C$値連続写像とし,
			\begin{align}
				\forall t \in [0,1]\, (\, |\gamma(t)| = 1\, )
			\end{align}
			であるとする.このとき$[0,1]$上の$\R$値連続写像$\theta$で
			\begin{align}
				\forall t \in [0,1]\, \left(\, \gamma(t) = e^{\isym\cdot\theta(t)}\, \right)
			\end{align}
			を満たすものが取れる.
		\end{thm}
	\end{screen}
	
	\begin{sketch}
		$\gamma$は$[0,1]$上で一様連続であるから,自然数$n$で
		\begin{align}
			\forall s,t \in [0,1]\, \left(\, |t-s| \leq \frac{1}{n} \Longrightarrow |\gamma(t) - \gamma(s)| < 1\, \right)
		\end{align}
		を満たすものが取れる.このとき$n$の各要素$k$で
		\begin{align}
			\gamma \ast \left[k/n,(k+1)/n\right] \subset \disc{\gamma(k/n)}{1}
		\end{align}
		が成り立つので,
		\begin{align}
			\alpha_{k} \defeq \pvarg{(\gamma(k/n))}
		\end{align}
		とおいて
		\begin{align}
			\left[k/n,(k+1)/n\right] \ni t \longmapsto \pvarg_{\alpha_{k} + \pi}(\gamma(t))
		\end{align}
		なる写像を
		\begin{align}
			\theta_{k}
		\end{align}
		と定めれば,$\theta_{k}$は$\left[k/n,(k+1)/n\right]$上の$\R$値連続写像であって,
		$\left[k/n,(k+1)/n\right]$の任意の要素$t$で
		\begin{align}
			\gamma(t) = e^{\isym \cdot \theta_{k}(t)}
		\end{align}
		を満たす.$[0,1]$上の写像$\theta$を
		\begin{align}
			[0,1] \ni t \longmapsto
			\begin{cases}
				\theta_{0}(t) & \mbox{if } {\displaystyle 0 \leq t \leq \frac{1}{n}} \\
				{\displaystyle \theta_{k}(t) + \sum_{j=0}^{k-1} \left\{\theta_{j}((j+1)/n) - \theta_{j+1}((j+1)/n)\right\}}
				& \mbox{if } 1 \leq k \wedge {\displaystyle \frac{k}{n} < t \leq \frac{k+1}{n}} \\
			\end{cases}
		\end{align}
		により定めれば,$\theta$は$[0,1]$上で連続である.また$n$の各要素$j$で
		\begin{align}
			e^{\isym \cdot \left(\theta_{j}((j+1)/n) - \theta_{j+1}((j+1)/n)\right)}
			= \frac{e^{\isym \cdot \theta_{j}((j+1)/n)}}{e^{\isym \cdot \theta_{j+1}((j+1)/n)}}
			= \frac{\gamma((j+1)/n)}{\gamma((j+1)/n)}
			= 1
		\end{align}
		が成り立つから,$[0,1]$の任意の要素$t$で
		\begin{align}
			\gamma(t) = e^{\isym \cdot \theta(t)}
		\end{align}
		が満たされる.
		\QED
	\end{sketch}
	
	いま$a$を複素数とし,$r$を正の実数として,
	\begin{align}
		[0,2 \cdot \pi] \ni \theta \longmapsto a + r \cdot e^{\isym \cdot \theta}
	\end{align}
	なる写像を$\gamma$とする.$\gamma$は$a$を中心に半径$r$の円周を描くが,
	このとき
	\begin{align}
		\Ind_{\gamma}(a) 
		&= \frac{1}{2\cdot\pi\cdot\isym} \cdot \int_{\gamma} \frac{1}{z-a}\ dz \\
		&= \frac{1}{2\cdot\pi\cdot\isym} \cdot \int_{[0,2\cdot\pi]} \frac{\isym \cdot e^{\isym \cdot \theta}}{e^{\isym \cdot \theta}}\ d\theta \\
		&= 1
	\end{align}
	が成り立ち,$\Ind_{\gamma}(a)$はちょうど$\gamma$が$a$の周りを回った回数に一致する.では次に
	\begin{align}
		[0,4 \cdot \pi] \ni \theta \longmapsto a + r \cdot e^{\isym \cdot \theta}
	\end{align}
	なる写像を$\eta$としてみる.$\eta$は$a$を中心に半径$r$の円周を描くが,$\gamma$とは違って
	$a$の周りを二周する.そして
	\begin{align}
		\Ind_{\eta}(a) 
		&= \frac{1}{2\cdot\pi\cdot\isym} \cdot \int_{\eta} \frac{1}{z-a}\ dz \\
		&= \frac{1}{2\cdot\pi\cdot\isym} \cdot \int_{[0,4\cdot\pi]} \frac{\isym \cdot e^{\isym \cdot \theta}}{e^{\isym \cdot \theta}}\ d\theta \\
		&= 2
	\end{align}
	が成り立つのだから,今度もまた$\Ind_{\eta}(a)$はちょうど$\eta$が$a$の周りを回った回数に一致した.
	同様に$a$の周りを$3$周する路の指数は$3$になり,$4$周すれば指数は$4$になる.
	
	これは単純な例であるが,実際に任意の閉路$\gamma$に対して,その$z$周りの指数は
	$\gamma$が$z$の周りを回転した回数に一致する.そして指数と回転数が等しいことを示すことが本節の主題である.
	
	\begin{screen}
		\begin{dfn}[回転数]
			$\gamma$を$[\alpha,\beta]$上の閉路として,$a$を
			$\ran{\gamma}$に属さない複素数とする.このとき$[\alpha,\beta]$の任意の要素$t$で
			\begin{align}
				\frac{\gamma(t) - a}{|\gamma(t) - a|} = e^{\isym\cdot\theta(t)}
			\end{align}
			を満たす$[\alpha,\beta]$上の$\R$値連続写像$\theta$を取って
			\begin{align}
				\frac{\theta(\beta) - \theta(\alpha)}{2\cdot\pi}
			\end{align}
			により定める整数を,$\gamma$の$a$周りの{\bf 回転数}\index{かいてんすう@回転数}{\bf (winding number)}と呼ぶ.
		\end{dfn}
	\end{screen}
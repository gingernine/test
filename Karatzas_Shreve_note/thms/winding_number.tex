\subsection{回転数}
	
	複素数$a$と,$a$を通らない路$\gamma$が与えられたとき,$\gamma$が$a$の周りを``何周するか''という問題を考える.
	なぜそのようなことを考えるのかというと,実は$\gamma$が閉路であるときは$a$の周りを回る`回数'は$a$周りの指数に一致して,
	しかも回転数の性質を分析することでCauchyの積分定理の実用的なステートメントが得られるためである.
	思い返せばCauchyの積分定理には
	\begin{quote}
		$\C$の開集合$\Psi$で
		\begin{align}
			\C \backslash \Omega \subset \Psi \subset \Ind_{\gamma}^{-1} \ast \{0\}
		\end{align}
		を満たすものが取れるなら
	\end{quote}
	という条件が付いていたが,これを愚直に確認するのは無謀である.
	
	では何をもって`回転'を数えるのかというと,{\bf 偏角の変動量}を使うのである.
	イメージとしては,路の偏角が$2 \cdot \pi$だけ増えるごとに``$a$の周りを一回転した''と数える.
	ただし,便宜上の呼び方であるが,もし偏角が$2\cdot\pi$減ったら``$a$の周りをマイナス一回転した''と数えることにする.
	
	\begin{center}
		\begin{tikzpicture}
			\node [anchor=north] at (0,0) {$a$};
			\draw [densely dotted] (0,0) -- ({1+(pi/6)*cos(pi/6 r)},{(pi/6)*sin(pi/6 r)});
			\draw [densely dotted] (0,0) -- ({1+(17*pi/6)*cos(17*pi/6 r)},{(17*pi/6)*sin(17*pi/6 r)});
			\draw [thick, domain=(pi/6):(17*pi/6), samples=200] plot({1+\x*cos(\x r)},{\x*sin(\x r)});
			\draw [domain=(pi/15):pi, samples=200, color=blue] plot({0.5*cos(\x r)},{0.5*sin(\x r)});
			\draw [domain=pi:(2*pi), samples=200, color=blue] plot({0.1+0.6*cos(\x r)},{0.6*sin(\x r)});
			\draw [domain=(2*pi):(17*pi/6.05), samples=200, color=blue, ->] plot({0.7*cos(\x r)},{0.7*sin(\x r)});
			
			\foreach \Point in {(0,0)}{
				\node at \Point {\textbullet};
			}
		\end{tikzpicture}
	\end{center}
	
	しかし偏角を測るからといって安直に$\pvarg$を使ってはいけない.$\pvarg$を使ってしまっては,路が負の実軸をまたぐごとに
	偏角が$2 \cdot \pi$だけ巻き戻されてしまうからである.これは$\pvarg$が負の実軸を挟んでジャンプすることが原因であり,
	他の偏角の枝を取ってもこの問題は避けられない.解決するには,路の挙動に合わせて臨機応変に偏角の枝を選択し繋ぎ合わせる必要がある.
	
	いま$\alpha$を実数として
	\begin{align}
		L_{\alpha} \defeq \Set{t \cdot e^{\isym\cdot\alpha}}{t \in \R \wedge t \leq 0}
	\end{align}
	とおき,$\C \backslash L_{\alpha}$上の写像$\pvarg_{\alpha}$を
	\begin{align}
		z \longmapsto \alpha + \pvarg{(z \cdot e^{-\isym\cdot\alpha})}
	\end{align}
	により定める.すると$\pvarg_{\alpha}$は$\C \backslash L_{\alpha}$上の実連続写像であって,
	$\C \backslash L_{\alpha}$の各要素$z$に対して
	\begin{align}
		\frac{z}{|z|} = e^{\isym \cdot \pvarg_{\alpha}{(z)}}
	\end{align}
	を満たす.実際,
	\begin{align}
		\C \ni z \longmapsto z \cdot e^{-\isym \cdot \alpha}
	\end{align}
	は連続写像であり,$\pvarg$は$\C \backslash L_{0}$上の実連続写像であり,また
	任意の複素数$z$で
	\begin{align}
		z \in \C \backslash L_{\alpha} \Longrightarrow z \cdot e^{-\isym \cdot \alpha} \in \C \backslash L_{0}
	\end{align}
	が成り立つので,$\pvarg_{\alpha}$は$\C \backslash L_{\alpha}$の各点で連続である.
	また$z$を$\C \backslash L_{\alpha}$の要素とすれば
	\begin{align}
		e^{\isym \cdot \pvarg_{\alpha}(z)}
		= e^{\isym \cdot \alpha} \cdot e^{\isym \cdot \pvarg{(z \cdot e^{-\isym\cdot\alpha})}}
		= e^{\isym \cdot \alpha} \cdot \frac{z \cdot e^{-\isym\cdot\alpha}}{\left|z \cdot e^{-\isym\cdot\alpha}\right|}
		= \frac{z}{|z|}
	\end{align}
	が成り立つ.
	
	\begin{screen}
		\begin{thm}[曲線の偏角の連続選択]\label{thm:continuous_choice_of_arguments}
			$\gamma$を$[0,1]$上の$\C$値連続写像とし,
			\begin{align}
				\forall t \in [0,1]\, (\, |\gamma(t)| = 1\, )
			\end{align}
			であるとする.このとき$[0,1]$上の$\R$値連続写像$\theta$で
			\begin{align}
				\forall t \in [0,1]\, \left(\, \gamma(t) = e^{\isym\cdot\theta(t)}\, \right)
			\end{align}
			を満たすものが取れる.
		\end{thm}
	\end{screen}
	
	\begin{sketch}
		$\gamma$は$[0,1]$上で一様連続であるから,自然数$n$で
		\begin{align}
			\forall s,t \in [0,1]\, \left(\, |t-s| \leq \frac{1}{n} \Longrightarrow |\gamma(t) - \gamma(s)| < 1\, \right)
		\end{align}
		を満たすものが取れる.このとき$n$の各要素$k$で
		\begin{align}
			\gamma \ast \left[k/n,(k+1)/n\right] \subset \disc{\gamma(k/n)}{1}
		\end{align}
		が成り立つので($D$の定義はP. \pageref{definition_of_disc_on_plane}),
		\begin{align}
			\alpha_{k} \defeq \pvarg{(\gamma(k/n))}
		\end{align}
		とおいて
		\begin{align}
			\left[k/n,(k+1)/n\right] \ni t \longmapsto \pvarg_{\alpha_{k} + \pi}(\gamma(t))
		\end{align}
		なる写像を
		\begin{align}
			\theta_{k}
		\end{align}
		と定めれば,$\theta_{k}$は$\left[k/n,(k+1)/n\right]$上の$\R$値連続写像であって,
		$\left[k/n,(k+1)/n\right]$の任意の要素$t$で
		\begin{align}
			\gamma(t) = e^{\isym \cdot \theta_{k}(t)}
		\end{align}
		を満たす.$[0,1]$上の写像$\theta$を
		\begin{align}
			[0,1] \ni t \longmapsto
			\begin{cases}
				\theta_{0}(t) & \mbox{if } {\displaystyle 0 \leq t \leq \frac{1}{n}} \\
				{\displaystyle \theta_{k}(t) + \sum_{j=0}^{k-1} \left\{\theta_{j}((j+1)/n) - \theta_{j+1}((j+1)/n)\right\}}
				& \mbox{if } 1 \leq k \mbox{ and } {\displaystyle \frac{k}{n} < t \leq \frac{k+1}{n}} \\
			\end{cases}
		\end{align}
		により定めれば,$\theta$は$[0,1]$上で連続である.また$n$の各要素$j$で
		\begin{align}
			e^{\isym \cdot \left(\theta_{j}((j+1)/n) - \theta_{j+1}((j+1)/n)\right)}
			= \frac{e^{\isym \cdot \theta_{j}((j+1)/n)}}{e^{\isym \cdot \theta_{j+1}((j+1)/n)}}
			= \frac{\gamma((j+1)/n)}{\gamma((j+1)/n)}
			= 1
		\end{align}
		が成り立つから,$[0,1]$の任意の要素$t$で
		\begin{align}
			\gamma(t) = e^{\isym \cdot \theta(t)}
		\end{align}
		が満たされる.
		\QED
	\end{sketch}
	
	$\gamma$を$[\alpha,\beta]$上の路として,$a$を$\ran{\gamma}$に属さない複素数とする.
	このとき$\varphi$を
	\begin{align}
		[0,1] \ni t \longmapsto \alpha + t \cdot (\beta - \alpha)
	\end{align}
	なる$[0,1]$上の写像とすると,
	\begin{align}
		[0,1] \ni t \longmapsto \frac{\gamma(\varphi(t)) - a}{\left|\gamma(\varphi(t)) - a\right|}
	\end{align}
	なる写像は$[0,1]$上の連続写像であるから,定理\ref{thm:continuous_choice_of_arguments}より,
	$[0,1]$の任意の要素$t$で
	\begin{align}
		\frac{\gamma(\varphi(t)) - a}{\left|\gamma(\varphi(t)) - a\right|}
		= e^{\isym \cdot \eta(t)}
	\end{align}
	を満たす$[0,1]$上の$\R$値連続写像$\eta$が取れる.$\varphi$は$[0,1]$から$[\alpha,\beta]$への同相写像であるから,
	\begin{align}
		\theta \defeq \eta \circ \varphi^{-1}
	\end{align}
	により定める$\theta$は$[\alpha,\beta]$上の$\R$値連続写像であって,$[\alpha,\beta]$の任意の要素$t$で
	\begin{align}
		\frac{\gamma(t) - a}{\left|\gamma(t) - a\right|}
		= \frac{\gamma(\varphi(\varphi^{-1}(t))) - a}{\left|\gamma(\varphi(\varphi^{-1}(t))) - a\right|}
		= e^{\isym \cdot \eta(\varphi^{-1}(t))}
		= e^{\isym \cdot \theta(t)}
	\end{align}
	が成立する.ここで$\xi$を$[\alpha,\beta]$上の$\R$値連続写像で,$[\alpha,\beta]$の任意の要素$t$で
	\begin{align}
		\frac{\gamma(t) - a}{\left|\gamma(t) - a\right|} = e^{\isym \cdot \xi(t)}
	\end{align}
	を満たすものとすると,定理\ref{thm:period_of_exponential_function_is_2_pi_i}より
	\begin{align}
		[\alpha,\beta] \ni t \longmapsto \frac{\theta(t) - \xi(t)}{2 \cdot \pi}
	\end{align}
	は整数値で,かつ連続である.ゆえにこの写像は定数値である.ゆえに
	\begin{align}
		\frac{\theta(\beta) - \theta(\alpha)}{2 \cdot \pi} = \frac{\xi(\beta) - \xi(\alpha)}{2 \cdot \pi}
	\end{align}
	が成立する.ちなみに,$\gamma$が閉路であるとき
	\begin{align}
		e^{\isym \cdot \theta(\alpha)} = \gamma(\alpha) = \gamma(\beta) = e^{\isym \cdot \theta(\beta)}
	\end{align}
	が成り立つので,再び定理\ref{thm:period_of_exponential_function_is_2_pi_i}より
	\begin{align}
		\frac{\theta(\beta) - \theta(\alpha)}{2 \cdot \pi}
	\end{align}
	は整数である.
	
	\begin{screen}
		\begin{dfn}[回転数]
			$\gamma$を$[\alpha,\beta]$上の閉路として,$a$を
			$\ran{\gamma}$に属さない複素数とする.このとき$[\alpha,\beta]$の任意の要素$t$で
			\begin{align}
				\frac{\gamma(t) - a}{|\gamma(t) - a|} = e^{\isym\cdot\theta(t)}
			\end{align}
			を満たす$[\alpha,\beta]$上の$\R$値連続写像$\theta$を取って
			\begin{align}
				\frac{\theta(\beta) - \theta(\alpha)}{2\cdot\pi}
			\end{align}
			により定める整数を,$\gamma$の$a$周りの{\bf 回転数}\index{かいてんすう@回転数}{\bf (winding number)}と呼ぶ.
		\end{dfn}
	\end{screen}
	
	閉路$\gamma$が与えられたとき,$\ran{\gamma}$に属さない複素数に対して,その周りの$\gamma$の回転数を対応させる写像を
	\begin{align}
		\Wnd_{\gamma}
	\end{align}
	と書く.
	
	{\bf 本節の主題は指数と回転数が等しいということである.}導入として簡単な例により確認してみる.いま
	\begin{align}
		[0,2 \cdot \pi] \ni \theta \longmapsto e^{\isym \cdot \theta}
	\end{align}
	なる写像を$\gamma$とする.$\gamma$は$0$を中心に半径$1$の円周を描くが,
	このとき
	\begin{align}
		\Ind_{\gamma}(0) 
		&= \frac{1}{2\cdot\pi\cdot\isym} \cdot \int_{\gamma} \frac{1}{z}\ dz \\
		&= \frac{1}{2\cdot\pi\cdot\isym} \cdot \int_{[0,2\cdot\pi]} \frac{\isym \cdot e^{\isym \cdot \theta}}{e^{\isym \cdot \theta}}\ d\theta \\
		&= 1
	\end{align}
	が成り立つから,$\Ind_{\gamma}(0)$はちょうど$\gamma$が$0$の周りを回った回数に一致する.では次に
	\begin{align}
		[0,4 \cdot \pi] \ni \theta \longmapsto e^{\isym \cdot \theta}
	\end{align}
	なる写像を$\eta$としてみる.$\eta$も$0$を中心に半径$1$の円周を描くが,$\gamma$とは違って
	$0$の周りを二周する.そして
	\begin{align}
		\Ind_{\eta}(0) 
		&= \frac{1}{2\cdot\pi\cdot\isym} \cdot \int_{\eta} \frac{1}{z}\ dz \\
		&= \frac{1}{2\cdot\pi\cdot\isym} \cdot \int_{[0,4\cdot\pi]} \frac{\isym \cdot e^{\isym \cdot \theta}}{e^{\isym \cdot \theta}}\ d\theta \\
		&= 2
	\end{align}
	が成り立つのだから,今度もまた$\Ind_{\eta}(0)$はちょうど$\eta$が$0$の周りを回った回数に一致した.
	同様に$0$の周りを$3$周する路の指数は$3$になり,$4$周すれば指数は$4$になる.
	
	\begin{screen}
		\begin{thm}[分数関数に対する微分積分学の基本定理]
		\label{thm:fundamental_theorem_of_calculus_for_fractional_function_complex_line_integral}
			$\gamma$を$[0,1]$上の路とし,
			\begin{align}
				0 \notin \ran{\gamma}
			\end{align}
			であるとする.また$\theta$を$[0,1]$の任意の要素$t$で
			\begin{align}
				\gamma(t) = |\gamma(t)| \cdot e^{\isym \cdot \theta(t)}
			\end{align}
			を満たす$[0,1]$上の$\R$値連続写像とする.このとき
			\begin{align}
				\int_{[0,1]} \frac{1}{\gamma}\ d\mu_{\gamma}
				= \pvlog{|\gamma(1)|} - \pvlog{|\gamma(0)|} + \isym \cdot (\theta(1) - \theta(0)).
			\end{align}
		\end{thm}
	\end{screen}
	
	\begin{sketch}\mbox{}
		\begin{description}
			\item[第一段]
				いま
				\begin{align}
					[0,1] \ni t \longmapsto \pvlog{|\gamma(t)|} + \isym \cdot \theta(t)
				\end{align}
				なる写像を$\ell$とし,
				\begin{align}
					[0,1] \ni t \longmapsto \frac{1}{\gamma(t)}
				\end{align}
				なる写像を$f$とし,
				\begin{align}
					\C \ni z \longmapsto e^{z} - 1 - z
				\end{align}
				なる写像を$\rho$とし,
				\begin{align}
					R \defeq \inf{t \in [0,1]}{|\gamma(t)|}
				\end{align}
				とおき,
				\begin{align}
					L \defeq |\mu_{\gamma}|([0,1])
				\end{align}
				とおく.また$\epsilon$を任意に与えられた正の実数とする.このとき自然数$n$で
				\begin{align}
					|t - s| \leq \frac{1}{n}
				\end{align}
				を満たす$[0,1]$の任意の要素$s$と$t$に対して
				\begin{align}
					|f(t) - f(s)| < \frac{\epsilon}{L}
				\end{align}
				かつ
				\begin{align}
					\left|\rho(\ell(t) - \ell(s))\right| \leq \frac{R \cdot \epsilon}{2 \cdot L} \cdot |\ell(t) - \ell(s)| 
				\end{align}
				かつ(第二段)
				\begin{align}
					|\ell(t) - \ell(s)| \leq \frac{2}{R} \cdot |\gamma(t) - \gamma(s)|
				\end{align}
				を満たす(第三段)ものが取れる.ここで
				\begin{align}
					[0,1] \ni t \longmapsto f(0) \cdot \defunc_{\{0\}}(t)
					+ \sum_{k=0}^{n-1} f(k/n) \cdot \defunc_{(k/n,(k+1)/n]}(t)
				\end{align}
				なる写像を$g$とおけば
				\begin{align}
					\left|\int_{[0,1]} f\ d\mu_{\gamma} - \int_{[0,1]} g\ d\mu_{\gamma}\right|
					\leq \int_{[0,1]} |f-g|\ d|\mu_{\gamma}|
					\leq \frac{\epsilon}{L} \cdot |\mu_{\gamma}|([0,1])
					= \epsilon
				\end{align}
				が成り立つ.他方で
				\begin{align}
					\int_{[0,1]} g\ d\mu_{\gamma}
					&= \sum_{k=0}^{n-1} f(k/n) \cdot \mu_{\gamma}\left((k/n,(k+1)/n]\right) \\
					&= \sum_{k=0}^{n-1} \frac{\gamma((k+1)/n) - \gamma(k/n)}{\gamma(k/n)} \\
					&= \sum_{k=0}^{n-1} \frac{e^{\ell((k+1)/n)} - e^{\ell(k/n)}}{e^{\ell(k/n)}} \\
					&= \sum_{k=0}^{n-1} \left(e^{\ell((k+1)/n) - \ell(k/n)} - 1\right) \\
					&= \sum_{k=0}^{n-1} \left[\ell((k+1)/n) - \ell(k/n) + \rho\left(\ell((k+1)/n) - \ell(k/n)\right)\right] \\
					&= \ell(1) - \ell(0) + \sum_{k=0}^{n-1} \rho\left(\ell((k+1)/n) - \ell(k/n)\right)
				\end{align}
				が成り立つので
				\begin{align}
					\left|\int_{[0,1]} g\ d\mu_{\gamma} - [\ell(1) - \ell(0)]\right|
					&\leq \sum_{k=0}^{n-1} \left|\rho\left(\ell((k+1)/n) - \ell(k/n)\right)\right| \\
					&\leq \sum_{k=0}^{n-1} \frac{R \cdot \epsilon}{2 \cdot L} \cdot |\ell((k+1)/n) - \ell(k/n)| \\
					&\leq \sum_{k=0}^{n-1} \frac{R \cdot \epsilon}{2 \cdot L} \cdot \frac{2}{R} \cdot |\gamma((k+1)/n) - \gamma(k/n)| \\
					&= \frac{\epsilon}{L} \sum_{k=0}^{n-1} |\gamma((k+1)/n) - \gamma(k/n)| \\
					&\leq \epsilon
				\end{align}
				が成立する.ゆえに
				\begin{align}
					&\left|\int_{[0,1]} f\ d\mu_{\gamma} - [\ell(1) - \ell(0)]\right| \\
					&\leq \left|\int_{[0,1]} f\ d\mu_{\gamma} - \int_{[0,1]} g\ d\mu_{\gamma}\right|
					+ \left|\int_{[0,1]} g\ d\mu_{\gamma} - [\ell(1) - \ell(0)]\right| \\
					&\leq 2 \cdot \epsilon
				\end{align}
				が従い,$\epsilon$の任意性から
				\begin{align}
					\int_{[0,1]} f\ d\mu_{\gamma} = \ell(1) - \ell(0)
					= \pvlog{|\gamma(1)|} - \pvlog{|\gamma(0)|} + \isym \cdot \left(\theta(1) - \theta(0)\right)
				\end{align}
				が得られる.
				
			\item[第二段]
				正の実数$\delta$で,
				\begin{align}
					|t - s| < \delta
				\end{align}
				を満たす$[0,1]$の任意の要素$s$と$t$に対して
				\begin{align}
					\left|\rho(\ell(t) - \ell(s))\right| \leq \frac{R \cdot \epsilon}{2 \cdot L} \cdot |\ell(t) - \ell(s)| 
				\end{align}
				を満たすものが取れる.実際,
				\begin{align}
					\frac{\rho(z)}{z} \longrightarrow 0 \quad (z \longrightarrow 0)
				\end{align}
				であるから
				\begin{align}
					\forall z \in \C\,
					\left(\, |z| < \eta \Longrightarrow |\rho(z)| < \frac{R \cdot \epsilon}{2 \cdot L} \cdot |z|\, \right)
				\end{align}
				を満たす正の実数$\eta$が取れて,$\ell$の一様連続性から
				正の実数$\delta$で,
				\begin{align}
					|t - s| < \delta
				\end{align}
				を満たす$[0,1]$の任意の要素$s$と$t$に対して
				\begin{align}
					|\ell(t) - \ell(s)| < \eta
				\end{align}
				を満たすものが取れる.
				
			\item[第三段]
				正の実数$\delta$で,
				\begin{align}
					|t - s| < \delta
				\end{align}
				を満たす$[0,1]$の任意の要素$s$と$t$に対して
				\begin{align}
					|\ell(t) - \ell(s)| \leq \frac{2}{R} \cdot |\gamma(t) - \gamma(s)|
				\end{align}
				を満たすものが取れる.実際,
				\begin{align}
					\forall z \in \C\,
					\left(\, |z| < \eta \Longrightarrow |\rho(z)| < \frac{|z|}{2}\, \right)
				\end{align}
				を満たす正の実数$\eta$が取れて,
				$\ell$の一様連続性から
				正の実数$\delta$で,
				\begin{align}
					|t - s| < \delta
				\end{align}
				を満たす$[0,1]$の任意の要素$s$と$t$に対して
				\begin{align}
					|\ell(t) - \ell(s)| < \eta
				\end{align}
				を満たすものが取れる.このとき,$s$と$t$を
				\begin{align}
					|t - s| < \delta
				\end{align}
				を満たす$[0,1]$の要素とすれば
				\begin{align}
					|\gamma(t) - \gamma(s)|
					&= \left| e^{\ell(t)} - e^{\ell(s)} \right| \\
					&= |\gamma(s)| \cdot \left| e^{\ell(t) - \ell(s)} - 1 \right| \\
					&= |\gamma(s)| \cdot \left| \ell(t) - \ell(s) + \rho(\ell(t) - \ell(s)) \right|
				\end{align}
				が成立する.ここで
				\begin{align}
					R \leq |\gamma(s)|
				\end{align}
				かつ
				\begin{align}
					\frac{|\ell(t) - \ell(s)|}{2} \leq \left| \ell(t) - \ell(s) + \rho(\ell(t) - \ell(s)) \right|
				\end{align}
				なので
				\begin{align}
					\frac{R}{2} \cdot |\ell(t) - \ell(s)| \leq |\gamma(t) - \gamma(s)|
				\end{align}
				が成立する.
				\QED
		\end{description}
	\end{sketch}
	
	\begin{screen}
		\begin{thm}[閉路の指数と回転数は一致する]
			$\alpha$と$\beta$を$\alpha < \beta$なる実数とし,$\gamma$を$[\alpha,\beta]$上の閉路とする.このとき
			\begin{align}
				\Ind_{\gamma} = \Wnd_{\gamma}.
			\end{align}
		\end{thm}
	\end{screen}
	
	\begin{sketch}
		いま$a$を$\C \backslash \ran{\gamma}$の要素とする.
		\begin{description}
			\item[第一段]
				定義より
				\begin{align}
					\Ind_{\gamma}(a) = \frac{1}{2\cdot\pi\cdot\isym} \cdot \int_{[\alpha,\beta]} \frac{1}{\gamma - a}\ d\mu_{\gamma}
				\end{align}
				である.ここで
				\begin{align}
					[\alpha,\beta] \ni t \longmapsto \gamma(t) - a
				\end{align}
				なる$[\alpha,\beta]$上の写像を$\eta$とすれば,
				\begin{align}
					\mu_{\gamma} = \mu_{\eta}
				\end{align}
				が成り立つ.実際$s$と$t$を
				\begin{align}
					s < t
				\end{align}
				を満たす$[\alpha,\beta]$の要素とすれば
				\begin{align}
					\mu_{\gamma}(]s,t]) &= \gamma(t) - \gamma(s) \\
					&= (\gamma(t) - a) - (\gamma(s) - a) \\
					&= \eta(t) - \eta(s) \\
					&= \mu_{\eta}(]s,t])
				\end{align}
				が成り立つので,測度の一致の定理より
				\begin{align}
					\mu_{\gamma} = \mu_{\eta}
				\end{align}
				が得られる.ゆえに
				\begin{align}
					\int_{[\alpha,\beta]} \frac{1}{\gamma - a}\ d\mu_{\gamma}
					= \int_{[\alpha,\beta]} \frac{1}{\eta}\ d\mu_{\gamma}
					= \int_{[\alpha,\beta]} \frac{1}{\eta}\ d\mu_{\eta}
				\end{align}
				が成り立つ.
				
			\item[第二段]
				$\varphi$を
				\begin{align}
					[0,1] \ni t \longmapsto \alpha + t \cdot (\beta - \alpha)
				\end{align}
				なる$[0,1]$上の写像とすれば
				\begin{align}
					\int_{[\alpha,\beta]} \frac{1}{\eta}\ d\mu_{\eta}
					= \int_{[0,1]} \frac{1}{\eta \circ \varphi}\ d\mu_{\eta \circ \varphi}
				\end{align}
				が成立する(パラメータ区間変項,P \pageref{fom:change_of_parameter_interval_complex_contour_integral}).
				ここで$[0,1]$の任意の要素$t$に対して
				\begin{align}
					\frac{\eta \circ \varphi(t)}{|\eta \circ \varphi(t)|} = e^{\isym \cdot \xi(t)}
				\end{align}
				を満たす,$[0,1]$上の$\R$値連続写像$\xi$を取ると,定理\ref{thm:fundamental_theorem_of_calculus_for_fractional_function_complex_line_integral}より
				\begin{align}
					\int_{[0,1]} \frac{1}{\eta \circ \varphi}\ d\mu_{\eta \circ \varphi}
					&= \pvlog{|\eta \circ \varphi(1)|} - \pvlog{|\eta \circ \varphi(0)|}
					+ \isym \cdot (\xi(1) - \xi(0)) \\
					&= \isym \cdot (\xi(1) - \xi(0))
				\end{align}
				が成り立つ.
			
			\item[第三段]
				$[\alpha,\beta]$の任意の要素$t$で
				\begin{align}
					\frac{\gamma(t)-a}{|\gamma(t) - a|}
					= \frac{\eta(t)}{|\eta(t)|}
					= e^{\isym \cdot \xi(\varphi^{-1}(t))}
				\end{align}
				が成り立つ.$\xi \circ \varphi^{-1}$は$[\alpha,\beta]$から$\R$への連続写像であるから
				\begin{align}
					\Wnd_{\gamma}(a) = \frac{\xi(\varphi^{-1}(\beta)) - \xi(\varphi^{-1}(\alpha))}{2 \cdot \pi}
					= \frac{\xi(1) - \xi(0)}{2 \cdot \pi}
				\end{align}
				が成立する.以上より
				\begin{align}
					\Ind_{\gamma}(a)
					&= \frac{1}{2\cdot\pi\cdot\isym} \cdot \int_{[0,1]} \frac{1}{\eta \circ \varphi}\ d\mu_{\eta \circ \varphi} \\
					&= \frac{\xi(1) - \xi(0)}{2 \cdot \pi} \\
					&= \Wnd_{\gamma}(a)
				\end{align}
				が得られた.
				\QED
		\end{description}
	\end{sketch}
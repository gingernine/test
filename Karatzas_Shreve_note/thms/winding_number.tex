\subsection{回転数}
	\begin{screen}
		\begin{dfn}[閉路]
			$\gamma$を路とし,$\alpha$と$\beta$を
			\begin{align}
				[\alpha,\beta] = \dom{\gamma}
			\end{align}
			を満たす実数とする.
			\begin{align}
				\gamma(\alpha) = \gamma(\beta)
			\end{align}
			であるとき$\gamma$を{\bf 閉路}\index{へいろ@閉路}{\bf (closed contour)}と呼ぶ.
			つまり閉路とは始点と終点が一致する路のことである.
		\end{dfn}
	\end{screen}
	
	いま$a$を複素数とし,$r$を正の実数として,
	\begin{align}
		[0,2 \cdot \pi] \ni \theta \longmapsto a + r \cdot e^{\isym \cdot \theta}
	\end{align}
	なる写像を$\gamma$とする.$\gamma$は$a$を中心に半径$r$の円周を描くが,ここで$f$をその円周上の連続写像とすれば
	\begin{align}
		\int_{\gamma} f 
		= r \cdot \isym \cdot \int_{[0,2 \cdot \pi]} f(a + r \cdot e^{\isym \cdot \theta}) \cdot e^{\isym \cdot \theta}\ d\theta
	\end{align}
	が成り立つ.特に$f$が
	\begin{align}
		\C \ni z \longmapsto \frac{1}{z - a}
	\end{align}
	なる写像であるとき,
	\begin{align}
		\int_{\gamma} \frac{1}{z - a}\ dz = 2 \cdot \pi \cdot \isym
	\end{align}
	が成り立つ.すなわち
	\begin{align}
		\frac{1}{2 \cdot \pi \cdot \isym} \cdot \int_{\gamma} \frac{1}{z - a}\ dz
	\end{align}
	は$1$であるが,これはちょうど$\gamma$が$a$の周りを回った数に一致する.
	
	\begin{screen}
		\begin{dfn}[指数]
			$\gamma$を路とするとき,$\C \backslash \ran{\gamma}$上で定義された写像で,
			\begin{align}
				z \notin \ran{\gamma}
			\end{align}
			を満たす複素数$z$に対して
			\begin{align}
				\frac{1}{2 \cdot \pi \cdot \isym} \cdot \int_{\gamma} \frac{1}{\zeta - z}\ d\zeta
			\end{align}
			を対応させるものを
			\begin{align}
				\Ind_{\gamma}
			\end{align}
			と書く.$\Ind_{\gamma}(z)$を$\gamma$の$z$周りの{\bf 指数}\index{しすう@指数}{\bf (index)}と呼ぶ.
		\end{dfn}
	\end{screen}
	
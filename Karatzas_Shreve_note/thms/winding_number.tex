\subsection{回転数}
	
	いま$a$を複素数とし,$r$を正の実数として,
	\begin{align}
		[0,2 \cdot \pi] \ni \theta \longmapsto a + r \cdot e^{\isym \cdot \theta}
	\end{align}
	なる写像を$\gamma$とする.$\gamma$は$a$を中心に半径$r$の円周を描くが,
	このとき
	\begin{align}
		\Ind_{\gamma}(a) 
		&= \frac{1}{2\cdot\pi\cdot\isym} \cdot \int_{\gamma} \frac{1}{z-a}\ dz \\
		&= \frac{1}{2\cdot\pi\cdot\isym} \cdot \int_{[0,2\cdot\pi]} \frac{\isym \cdot e^{\isym \cdot \theta}}{e^{\isym \cdot \theta}}\ d\theta \\
		&= 1
	\end{align}
	が成り立ち,$\Ind_{\gamma}(a)$はちょうど$\gamma$が$a$の周りを回った回数に一致する.では次に
	\begin{align}
		[0,4 \cdot \pi] \ni \theta \longmapsto a + r \cdot e^{\isym \cdot \theta}
	\end{align}
	なる写像を$\eta$としてみる.$\eta$は$a$を中心に半径$r$の円周を描くが,$\gamma$とは違って
	$a$の周りを二周する.そして
	\begin{align}
		\Ind_{\eta}(a) 
		&= \frac{1}{2\cdot\pi\cdot\isym} \cdot \int_{\eta} \frac{1}{z-a}\ dz \\
		&= \frac{1}{2\cdot\pi\cdot\isym} \cdot \int_{[0,4\cdot\pi]} \frac{\isym \cdot e^{\isym \cdot \theta}}{e^{\isym \cdot \theta}}\ d\theta \\
		&= 2
	\end{align}
	が成り立つのだから,今度もまた$\Ind_{\eta}(a)$はちょうど$\eta$が$a$の周りを回った回数に一致した.
	同様に$a$の周りを$3$周する路の指数は$3$になり,$4$周すれば指数は$4$になる.
	
	これは単純な例であるが,実際に任意の閉路$\gamma$に対して,その$z$周りの指数は
	$\gamma$が$z$の周りを回転した回数に一致する.そして指数と回転数が等しいことを示すことが本節の主題である.
	
	\begin{screen}
		\begin{dfn}[回転数]
			
		\end{dfn}
	\end{screen}
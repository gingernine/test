\subsection{回転数}
	
	路$\gamma$が与えられたとき,$\gamma$が$a$の周りを``何周するか''という問題を考える.
	図を描けば明らかと言ってしまえば
	
	\begin{screen}
		\begin{thm}[路の偏角の連続選択]\label{thm:continuous_choice_of_arguments}
			$\gamma$を$[0,1]$上の$\C$値連続写像とし,
			\begin{align}
				\forall t \in [0,1]\, (\, |\gamma(t)| = 1\, )
			\end{align}
			であるとする.このとき$[0,1]$上の$\R$値連続写像$\theta$で
			\begin{align}
				\forall t \in [0,1]\, \left(\, \gamma(t) = e^{\isym\cdot\theta(t)}\, \right)
			\end{align}
			を満たすものが取れる.
		\end{thm}
	\end{screen}
	
	\begin{sketch}
		$\gamma$は$[0,1]$上で一様連続であるから,自然数$n$で
		\begin{align}
			\forall s,t \in [0,1]\, \left(\, |t-s| \leq \frac{1}{n} \Longrightarrow |\gamma(t) - \gamma(s)| < 1\, \right)
		\end{align}
		を満たすものが取れる.このとき$n$の各要素$k$で
		\begin{align}
			\gamma \ast \left[k/n,(k+1)/n\right] \subset \disc{\gamma(k/n)}{1}
		\end{align}
		が成り立つので,
		\begin{align}
			\alpha_{k} \defeq \pvarg{(\gamma(k/n))}
		\end{align}
		とおいて
		\begin{align}
			\left[k/n,(k+1)/n\right] \ni t \longmapsto \pvarg_{\alpha_{k} + \pi}(\gamma(t))
		\end{align}
		なる写像を
		\begin{align}
			\theta_{k}
		\end{align}
		と定めれば,$\theta_{k}$は$\left[k/n,(k+1)/n\right]$上の$\R$値連続写像であって,
		$\left[k/n,(k+1)/n\right]$の任意の要素$t$で
		\begin{align}
			\gamma(t) = e^{\isym \cdot \theta_{k}(t)}
		\end{align}
		を満たす.$[0,1]$上の写像$\theta$を
		\begin{align}
			[0,1] \ni t \longmapsto
			\begin{cases}
				\theta_{0}(t) & \mbox{if } {\displaystyle 0 \leq t \leq \frac{1}{n}} \\
				{\displaystyle \theta_{k}(t) + \sum_{j=0}^{k-1} \left\{\theta_{j}((j+1)/n) - \theta_{j+1}((j+1)/n)\right\}}
				& \mbox{if } 1 \leq k \wedge {\displaystyle \frac{k}{n} < t \leq \frac{k+1}{n}} \\
			\end{cases}
		\end{align}
		により定めれば,$\theta$は$[0,1]$上で連続である.また$n$の各要素$j$で
		\begin{align}
			e^{\isym \cdot \left(\theta_{j}((j+1)/n) - \theta_{j+1}((j+1)/n)\right)}
			= \frac{e^{\isym \cdot \theta_{j}((j+1)/n)}}{e^{\isym \cdot \theta_{j+1}((j+1)/n)}}
			= \frac{\gamma((j+1)/n)}{\gamma((j+1)/n)}
			= 1
		\end{align}
		が成り立つから,$[0,1]$の任意の要素$t$で
		\begin{align}
			\gamma(t) = e^{\isym \cdot \theta(t)}
		\end{align}
		が満たされる.
		\QED
	\end{sketch}
	
	いま$a$を複素数とし,$r$を正の実数として,
	\begin{align}
		[0,2 \cdot \pi] \ni \theta \longmapsto a + r \cdot e^{\isym \cdot \theta}
	\end{align}
	なる写像を$\gamma$とする.$\gamma$は$a$を中心に半径$r$の円周を描くが,
	このとき
	\begin{align}
		\Ind_{\gamma}(a) 
		&= \frac{1}{2\cdot\pi\cdot\isym} \cdot \int_{\gamma} \frac{1}{z-a}\ dz \\
		&= \frac{1}{2\cdot\pi\cdot\isym} \cdot \int_{[0,2\cdot\pi]} \frac{\isym \cdot e^{\isym \cdot \theta}}{e^{\isym \cdot \theta}}\ d\theta \\
		&= 1
	\end{align}
	が成り立ち,$\Ind_{\gamma}(a)$はちょうど$\gamma$が$a$の周りを回った回数に一致する.では次に
	\begin{align}
		[0,4 \cdot \pi] \ni \theta \longmapsto a + r \cdot e^{\isym \cdot \theta}
	\end{align}
	なる写像を$\eta$としてみる.$\eta$は$a$を中心に半径$r$の円周を描くが,$\gamma$とは違って
	$a$の周りを二周する.そして
	\begin{align}
		\Ind_{\eta}(a) 
		&= \frac{1}{2\cdot\pi\cdot\isym} \cdot \int_{\eta} \frac{1}{z-a}\ dz \\
		&= \frac{1}{2\cdot\pi\cdot\isym} \cdot \int_{[0,4\cdot\pi]} \frac{\isym \cdot e^{\isym \cdot \theta}}{e^{\isym \cdot \theta}}\ d\theta \\
		&= 2
	\end{align}
	が成り立つのだから,今度もまた$\Ind_{\eta}(a)$はちょうど$\eta$が$a$の周りを回った回数に一致した.
	同様に$a$の周りを$3$周する路の指数は$3$になり,$4$周すれば指数は$4$になる.
	
	これは単純な例であるが,実際に任意の閉路$\gamma$に対して,その$z$周りの指数は
	$\gamma$が$z$の周りを回転した回数に一致する.そして指数と回転数が等しいことを示すことが本節の主題である.
	
	\begin{screen}
		\begin{dfn}[回転数]
			$\gamma$を$[\alpha,\beta]$上の閉路として,$a$を
			$\ran{\gamma}$に属さない複素数とする.このとき$[\alpha,\beta]$の任意の要素$t$で
			\begin{align}
				\frac{\gamma(t) - a}{|\gamma(t) - a|} = e^{\isym\cdot\theta(t)}
			\end{align}
			を満たす$[\alpha,\beta]$上の$\R$値連続写像$\theta$を取って
			\begin{align}
				\frac{\theta(\beta) - \theta(\alpha)}{2\cdot\pi}
			\end{align}
			により定める整数を,$\gamma$の$a$周りの{\bf 回転数}\index{かいてんすう@回転数}{\bf (winding number)}と呼ぶ.
		\end{dfn}
	\end{screen}
	
	\begin{screen}
		\begin{thm}[分数関数に対する微分積分学の基本定理]
			$\gamma$を$[0,1]$上の路とし,
			\begin{align}
				0 \notin \ran{\gamma}
			\end{align}
			であるとする.また$\theta$を$[0,1]$の任意の要素$t$で
			\begin{align}
				\gamma(t) = |\gamma(t)| \cdot e^{\isym \cdot \theta(t)}
			\end{align}
			を満たす$[0,1]$上の$\R$値連続写像とする.このとき
			\begin{align}
				\int_{[0,1]} \frac{1}{\gamma}\ d\mu_{\gamma}
				= \pvlog{|\gamma(1)|} - \pvlog{|\gamma(0)|}
				+ \isym \cdot \left(\theta(1) - \theta(0)\right).
			\end{align}
		\end{thm}
	\end{screen}
	
	\begin{sketch}\mbox{}
		\begin{description}
			\item[第一段]
				いま
				\begin{align}
					[0,1] \ni t \longmapsto \pvlog{|\gamma(t)|} + \isym \cdot \theta(t)
				\end{align}
				なる写像を$\ell$とし,
				\begin{align}
					[0,1] \ni t \longmapsto \frac{1}{\gamma(t)}
				\end{align}
				なる写像を$f$とし,
				\begin{align}
					\C \ni z \longmapsto e^{z} - 1 - z
				\end{align}
				なる写像を$\rho$とし,
				\begin{align}
					R \defeq \inf{t \in [0,1]}{|\gamma(t)|}
				\end{align}
				とおき,
				\begin{align}
					L \defeq |\mu_{\gamma}|([0,1])
				\end{align}
				とおく.また$\epsilon$を任意に与えられた正の実数とする.このとき自然数$n$で
				\begin{align}
					|t - s| \leq \frac{1}{n}
				\end{align}
				を満たす$[0,1]$の任意の要素$s$と$t$に対して
				\begin{align}
					|f(t) - f(s)| < \frac{\epsilon}{L}
				\end{align}
				かつ
				\begin{align}
					\left|\rho(\ell(t) - \ell(s))\right| \leq \frac{R \cdot \epsilon}{2 \cdot L} \cdot |\ell(t) - \ell(s)| 
				\end{align}
				かつ(第二段)
				\begin{align}
					|\ell(t) - \ell(s)| \leq \frac{2}{R} \cdot |\gamma(t) - \gamma(s)|
				\end{align}
				を満たす(第三段)ものが取れる.ここで
				\begin{align}
					[0,1] \ni t \longmapsto f(0) \cdot \defunc_{\{0\}}(t)
					+ \sum_{k=0}^{n-1} f(k/n) \cdot \defunc_{(k/n,(k+1)/n]}(t)
				\end{align}
				なる写像を$g$とおけば
				\begin{align}
					\left|\int_{[0,1]} f\ d\mu_{\gamma} - \int_{[0,1]} g\ d\mu_{\gamma}\right|
					\leq \int_{[0,1]} |f-g|\ d|\mu_{\gamma}|
					\leq \frac{\epsilon}{L} \cdot |\mu_{\gamma}|([0,1])
					= \epsilon
				\end{align}
				が成り立つ.他方で
				\begin{align}
					\int_{[0,1]} g\ d\mu_{\gamma}
					&= \sum_{k=0}^{n-1} g(k/n) \cdot \mu_{\gamma}\left((k/n,(k+1)/n]\right) \\
					&= \sum_{k=0}^{n-1} \frac{\gamma((k+1)/n) - \gamma(k/n)}{\gamma(k/n)} \\
					&= \sum_{k=0}^{n-1} \frac{e^{\ell((k+1)/n)} - e^{\ell(k/n)}}{e^{\ell(k/n)}} \\
					&= \sum_{k=0}^{n-1} \left(e^{\ell((k+1)/n) - \ell(k/n)} - 1\right) \\
					&= \sum_{k=0}^{n-1} \left[\ell((k+1)/n) - \ell(k/n) + \rho\left(\ell((k+1)/n) - \ell(k/n)\right)\right] \\
					&= \ell(1) - \ell(0) + \sum_{k=0}^{n-1} \rho\left(\ell((k+1)/n) - \ell(k/n)\right)
				\end{align}
				が成り立つので
				\begin{align}
					\left|\int_{[0,1]} g\ d\mu_{\gamma} - [\ell(1) - \ell(0)]\right|
					&\leq \sum_{k=0}^{n-1} \left|\rho\left(\ell((k+1)/n) - \ell(k/n)\right)\right| \\
					&\leq \sum_{k=0}^{n-1} \frac{R \cdot \epsilon}{2 \cdot L} \cdot |\ell((k+1)/n) - \ell(k/n)| \\
					&\leq \sum_{k=0}^{n-1} \frac{R \cdot \epsilon}{2 \cdot L} \cdot \frac{2}{R} \cdot |\gamma((k+1)/n) - \gamma(k/n)| \\
					&= \frac{\epsilon}{L} \sum_{k=0}^{n-1} |\gamma((k+1)/n) - \gamma(k/n)| \\
					&\leq \epsilon
				\end{align}
				が成立する.ゆえに
				\begin{align}
					&\left|\int_{[0,1]} f\ d\mu_{\gamma} - [\ell(1) - \ell(0)]\right| \\
					&\leq \left|\int_{[0,1]} f\ d\mu_{\gamma} - \int_{[0,1]} g\ d\mu_{\gamma}\right|
					+ \left|\int_{[0,1]} g\ d\mu_{\gamma} - [\ell(1) - \ell(0)]\right|
					\leq 2 \cdot \epsilon
				\end{align}
				が従い,$\epsilon$の任意性から
				\begin{align}
					\int_{[0,1]} f\ d\mu_{\gamma} = \ell(1) - \ell(0)
					= \pvlog{|\gamma(1)|} - \pvlog{|\gamma(0)|} + \isym \cdot \left(\theta(1) - \theta(0)\right)
				\end{align}
				が得られる.
				
			\item[第二段]
				正の実数$\delta$で,
				\begin{align}
					|t - s| < \delta
				\end{align}
				を満たす$[0,1]$の任意の要素$s$と$t$に対して
				\begin{align}
					\left|\rho(\ell(t) - \ell(s))\right| \leq \frac{R \cdot \epsilon}{2 \cdot L} \cdot |\ell(t) - \ell(s)| 
				\end{align}
				を満たすものが取れる.実際,
				\begin{align}
					\frac{\rho(z)}{z} \longrightarrow 0 \quad (z \longrightarrow 0)
				\end{align}
				であるから
				\begin{align}
					\forall z \in \C\,
					\left(\, |z| < \eta \Longrightarrow |\rho(z)| < \frac{R \cdot \epsilon}{2 \cdot L} \cdot |z|\, \right)
				\end{align}
				を満たす正の実数$\eta$が取れて,$\ell$の一様連続性から
				正の実数$\delta$で,
				\begin{align}
					|t - s| < \delta
				\end{align}
				を満たす$[0,1]$の任意の要素$s$と$t$に対して
				\begin{align}
					|\ell(t) - \ell(s)| < \eta
				\end{align}
				を満たすものが取れる.
				
			\item[第三段]
				正の実数$\delta$で,
				\begin{align}
					|t - s| < \delta
				\end{align}
				を満たす$[0,1]$の任意の要素$s$と$t$に対して
				\begin{align}
					|\ell(t) - \ell(s)| \leq \frac{2}{R} \cdot |\gamma(t) - \gamma(s)|
				\end{align}
				を満たすものが取れる.実際,
				\begin{align}
					\forall z \in \C\,
					\left(\, |z| < \eta \Longrightarrow |\rho(z)| < \frac{|z|}{2}\, \right)
				\end{align}
				を満たす正の実数$\eta$が取れて,
				$\ell$の一様連続性から
				正の実数$\delta$で,
				\begin{align}
					|t - s| < \delta
				\end{align}
				を満たす$[0,1]$の任意の要素$s$と$t$に対して
				\begin{align}
					|\ell(t) - \ell(s)| < \eta
				\end{align}
				を満たすものが取れる.このとき,$s$と$t$を
				\begin{align}
					|t - s| < \delta
				\end{align}
				を満たす$[0,1]$の要素とすれば
				\begin{align}
					|\gamma(t) - \gamma(s)|
					&= \left| e^{\ell(t)} - e^{\ell(s)} \right| \\
					&= |\gamma(s)| \cdot \left| e^{\ell(t) - \ell(s)} - 1 \right| \\
					&= |\gamma(s)| \cdot \left| \ell(t) - \ell(s) + \rho(\ell(t) - \ell(s)) \right|
				\end{align}
				が成立する.ここで
				\begin{align}
					R \leq |\gamma(s)|
				\end{align}
				かつ
				\begin{align}
					\frac{|\ell(t) - \ell(s)|}{2} \leq \left| \ell(t) - \ell(s) + \rho(\ell(t) - \ell(s)) \right|
				\end{align}
				なので
				\begin{align}
					R \cdot \frac{|\ell(t) - \ell(s)|}{2} \leq |\gamma(t) - \gamma(s)|
				\end{align}
				が成立する.
				\QED
		\end{description}
	\end{sketch}
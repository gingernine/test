\section{偏角}
	
	$0$でない複素数$z$に対して
	\begin{align}
		z = \exp{w}
	\end{align}
	を満たす複素数$w$を$z$の対数と呼んだが,このとき
	\begin{align}
		\frac{z}{|z|} = e^{\isym \cdot \Im{w}}
	\end{align}
	が成立する.$\Im{w}$の様に
	\begin{align}
		\frac{z}{|z|} = e^{\isym \cdot \theta}
	\end{align}
	を満たす実数$\theta$のことを$z$の{\bf 偏角}\index{へんかく@偏角}{\bf (argument)}と呼ぶ.
	対数と同様に偏角も整数の個数だけ存在する.
	
	\begin{screen}
		\begin{dfn}[偏角]
			複素数$z$に対して,その偏角の全体
			\begin{align}
				\Set{\theta \in \R}{z = |z| \cdot \exp{(\isym \cdot \theta)}}
			\end{align}
			を対応させる$\C$上の写像を
			\begin{align}
				\arg
			\end{align}
			と書く.$\arg$もまた多価関数であるが,何らかの条件によって偏角を抜き取れば``関数''となり,その抜き取る操作を{\bf 偏角の枝を取る}という.
		\end{dfn}
	\end{screen}
	
	$z$を$0$でない複素数とするとき,
	\begin{align}
		\arg{z}
	\end{align}
	は$z$の対数の虚部の全体
	\begin{align}
		\Set{\theta}{\exists w \in \C\, \left(\, z = \exp{w} \wedge \theta = \Im{w}\, \right)}
	\end{align}
	に一致する.実際,$\theta$を$z$の偏角とすれば
	\begin{align}
		\frac{z}{|z|} = e^{\isym \cdot \theta}
	\end{align}
	が成り立つので,
	\begin{align}
		w \defeq \pvlog{|z|} + \isym \cdot \theta
	\end{align}
	により複素数$w$を定めれば
	\begin{align}
		e^{w} = e^{\pvlog{|z|}} \cdot e^{\isym \cdot \theta} = |z| \cdot \frac{z}{|z|} = z
	\end{align}
	が成立する.つまり$w$は$z$の対数であり,$\pvlog{|z|}$が実数であるから$\theta$は$w$の虚部である.逆に,
	$\theta$を実数とし,
	\begin{align}
		z = \exp{w} \wedge \theta = \Im{w}
	\end{align}
	を満たす複素数$w$が取れるとする.この場合は冒頭に書いた内容から
	\begin{align}
		\frac{z}{|z|} = e^{\isym \cdot \Im{w}} = e^{\isym \cdot \theta}
	\end{align}
	が成立するので$\theta$は$z$の偏角である.
	
	$z$を$0$でない複素数とするとき,
	\begin{align}
		-\pi < \theta \leq \pi
	\end{align}
	を満たす$z$の偏角を$\arg{z}$の{\bf 主値}\index{しゅち@主値}{\bf (principal value)}
	と呼ぶ.$z$の偏角の主値を
	\begin{align}
		\pv{\arg{z}}
	\end{align}
	と書くとき,
	\begin{align}
		\C \backslash \{0\} \ni z \longmapsto \pv{\arg{z}}
	\end{align}
	なる対応関係で定める$\C \backslash \{0\}$上の写像を
	\begin{align}
		\pvarg
	\end{align}
	と書く.正式には$\pvarg$とは
	\begin{align}
		\pvarg \defeq \Set{x}{\exists z \in \C\, \exists \theta \in \R\, 
		\left[\, x=(z,\theta) \wedge z \neq 0 \wedge z = |z| \cdot \exp{(\isym \cdot \theta)} \wedge
		-\pi < \theta \leq \pi\, \right]}
	\end{align}
	で定められる関係である.
	
	\begin{screen}
		\begin{thm}[偏角の主値は対数の主値の虚部]
			\begin{align}
				\pvarg = \Im \circ \pvlog.
			\end{align}
		\end{thm}
	\end{screen}
	
	\begin{sketch}
		$x$を$\pvarg$の要素とすると,
		\begin{align}
			z = |z| \cdot \exp{(\isym \cdot \theta)} \wedge -\pi < \theta \leq \pi
		\end{align}
		を満たす$0$でない複素数$z$,および実数$\theta$が取れて,
		\begin{align}
			x = (z,\theta)
		\end{align}
		が成り立つ.ここで
		\begin{align}
			w \defeq \pvlog{|z|} + \isym \cdot \theta
		\end{align}
		により複素数$w$を定めれば,
		\begin{align}
			z = \exp{w}
		\end{align}
		かつ
		\begin{align}
			\theta = \Im{w}
		\end{align}
		が成り立つので
		\begin{align}
			(z,w) \in \pvlog
		\end{align}
		かつ
		\begin{align}
			(w,\theta) \in \Im
		\end{align}
		が成立する.ゆえに
		\begin{align}
			x = (z,\theta) \in \Im \circ \pvlog
		\end{align}
		が成立する.逆に$x$を$\Im \circ \pvlog$の要素とすると,
		\begin{align}
			(z,w) \in \pvlog
		\end{align}
		と
		\begin{align}
			(w,\theta) \in \Im
		\end{align}
		を満たす複素数$z,w$と実数$\theta$が取れて,
		\begin{align}
			x = (z,\theta)
		\end{align}
		が成り立つ.このとき
		\begin{align}
			z = \exp{w}
		\end{align}
		かつ
		\begin{align}
			\theta = \Im{w}
		\end{align}
		であるから
		\begin{align}
			e^{\isym \cdot \theta} = e^{\isym \cdot \Im{w}} = \frac{z}{|z|}
		\end{align}
		が成立し,また
		\begin{align}
			-\pi < \Im{w} \leq \pi
		\end{align}
		であるから
		\begin{align}
			-\pi < \theta \leq \pi
		\end{align}
		も満たされる.ゆえに
		\begin{align}
			x = (z,\theta) \in \pvarg
		\end{align}
		が従う.
		\QED
	\end{sketch}
	
	{\bf 特に$\pvarg$は$\C \backslash \{0\}$から$\R$への連続写像である.}
	
	\begin{itembox}[l]{偏角の群論的な視点}
		いま
		\begin{align}
			U \defeq \Set{z \in \C}{|z| = 1}
		\end{align}
		とおいて,$\varphi$を
		\begin{align}
			\R \ni y \longmapsto e^{\isym \cdot y}
		\end{align}
		なる写像とする.この$\varphi$は$(\R,+)$から$(U,\cdot)$への全射群準同型である.$\varphi$の核は
		\begin{align}
			\Set{2 \cdot n \cdot \pi}{n \in \Z}
		\end{align}
		に一致し,この集合を
		\begin{align}
			2\pi\Z
		\end{align}
		と書けば同型定理より商$\R/2\pi\Z$から$U$への全単射$\psi$が得られる.このとき,$z$を$0$でない任意の複素数とすれば
		\begin{align}
			\arg{z} = \psi^{-1}(z/|z|)
		\end{align}
		が成立する.
	\end{itembox}
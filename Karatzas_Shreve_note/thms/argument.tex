\section{偏角}
	\begin{itembox}[l]{以降の流れを忘れないうちにメモ}
		\begin{itemize}
			\item 偏角をカルタン流に定義するか,或いはカルタン流とのつながりを書く.
			\item $\pvarg = \Im \circ \pvlog$を示す.
			\item 偏角の連続選択.
			\item $\gamma$を$0$を通らない路とするとき,$\gamma(t) = r(t) e^{\isym \theta(t)}$を満たす
				実連続関数$r$と$\theta$が取れる.
			\item 任意の$\epsilon$に対して
				\begin{align}
					|\Ind_{\gamma}(0) - \Ind_{\eta}(0)| < \epsilon
				\end{align}
				かつ
				\begin{align}
					\Wnd_{\gamma}(0) = \Wnd_{\eta}(0)
				\end{align}
				を満たす絶対連続な$\eta$が取れる.
			\item $\eta$は$\Ind_{\eta}(0) = \Wnd_{\eta}(0)$を満たす.
			\item $\Ind_\gamma(0) = \Wnd_{\eta}(0) = \Wnd_{\gamma}(a)$.
		\end{itemize}
	\end{itembox}
	
	$\gamma$を$0$を通らない路とするとき,
	\begin{align}
		\theta:[0,1] \longrightarrow \R
	\end{align}
	かつ
	\begin{align}
		\forall t \in [0,1]\, \left(\, \theta(t) \in \arg{(\gamma(t))}\, \right)
	\end{align}
	を満たす連続写像$\theta$を$\gamma$の偏角の連続選択関数と呼ぶ.
	
	$\theta$と$\phi$を$\gamma$の偏角の連続選択関数とするとき,
	\begin{align}
		t \longmapsto \frac{\theta(t) - \phi(t)}{2\pi}
	\end{align}
	は連続で整数値なので定数値関数である.ゆえに
	\begin{align}
		\theta(0) - \phi(0) = \theta(1) - \phi(1)
	\end{align}
	ゆえに
	\begin{align}
		\frac{\theta(1) - \theta(0)}{2\pi} = \frac{\phi(1) - \phi(0)}{2\pi}
	\end{align}
	が成立する.
	
	\begin{screen}
		\begin{thm}
			$\eta$を$[0,1]$上の絶対連続関数とするとき,$\eta$が$0$を通らなければ
			\begin{align}
				\Ind_{\eta}(0) = \Wnd_{\eta}(0)
			\end{align}
			が成り立つ.
		\end{thm}
	\end{screen}
	
	\begin{sketch}
		
	\end{sketch}
	
	\begin{screen}
		\begin{thm}
			任意の$\epsilon$に対して
			\begin{align}
				|\Ind_{\gamma}(0) - \Ind_{\eta}(0)| < \epsilon
			\end{align}
			かつ
			\begin{align}
				\Wnd_{\gamma}(0) = \Wnd_{\eta}(0)
			\end{align}
			を満たす絶対連続な$\eta$が取れる.
		\end{thm}
	\end{screen}
	
	\begin{sketch}
		いま$\epsilon$を任意に与えられた正の実数とする.
		\begin{align}
			\C \backslash \{0\} \ni z \longmapsto \frac{1}{z}
		\end{align}
		なる写像を$f$とおく.$d$を$\ran{\gamma}$と$0$との距離とする:
		\begin{align}
			d \defeq \inf{t \in [0,1]}{|\gamma(t)|}.
		\end{align}
		そして$\ran{\gamma}$の$d/2$近傍を
		\begin{align}
			N \defeq \bigcup_{t \in [0,1]} \disc{\gamma(t)}{d/2}
		\end{align}
		とおく.すると$\closure{N}$はコンパクトなので$f$は$\closure{N}$上で一様連続.ゆえに,
		\begin{align}
			\delta < \frac{d}{2}
		\end{align}
		かつ
		\begin{align}
			\forall z,w \in N\, 
			\left(\, |z-w| < \delta \Longrightarrow |f(z) - f(w)| < \epsilon\, \right)
		\end{align}
		を満たす正の実数$\delta$が取れる.
		\begin{align}
			0 = t_0 < t_1 < \cdots < t_n = 1
		\end{align}
		を,各$k$で
		\begin{align}
			\gamma \ast \left[t_k,t_{k+1}\right] \subset \disc{\gamma(t)}{\delta}
		\end{align}
		が満たされるように取る.$\eta$を
		\begin{align}
			\left[t_k,t_{k+1}\right] \ni t \longmapsto \gamma(t_k) + \frac{t - t_k}{t_{k+1} - t_k} \left(\gamma(t_{k+1}) - \gamma(t_k)\right)
		\end{align}
		なる写像として取る.このとき
		\begin{description}
			\item[(1)]
				\begin{align}
					\left|\int_{\gamma} f  - \int_{\eta} f\right| < 2 \epsilon V(\gamma)
				\end{align}
				が成り立つ.
			\item[(2)]
				\begin{align}
					\Wnd_{\gamma}(0) = \Wnd_{\eta}(0)
				\end{align}
				が成り立つ.各$k$
				\begin{align}
					\disc{\gamma(t_k)}{\delta}
				\end{align}
				上の連続選択を
				\begin{align}
					\pvarg_k
				\end{align}
				とする.
				\begin{align}
					\theta^{\gamma}_k: t \longmapsto \pvarg_k(\gamma(t))
				\end{align}
				とし,
				\begin{align}
					\theta^{\eta}_k: t \longmapsto \pvarg_k(\eta(t))
				\end{align}
				とすると,$\theta^{\gamma}_k$と$\theta^{\eta}_k$は$\left[t_k,t_{k+1}\right]$上で連続.
				$\theta^{\gamma}$と$\theta^{\eta}$を連接すれば
				\begin{align}
					\theta^{\gamma}(t_k) = \theta^{\eta}(t_k)
				\end{align}
				なので
				\begin{align}
					\theta^{\gamma}(1) - \theta^{\gamma}(0) = \theta^{\eta}(1) - \theta^{\eta}(0)
				\end{align}
				が成り立つ.ゆえに
				\begin{align}
					\Wnd_{\gamma}(0) = \Wnd_{\eta}(0)
				\end{align}
				が得られた.
				\QED
		\end{description}
	\end{sketch}
	
	ゆえに,任意の$\epsilon$に対して
	\begin{align}
		|\Ind_{\gamma}(0) - \Wnd_{\gamma}(0)| < \epsilon
	\end{align}
	が成立する.ゆえに
	\begin{align}
		\Ind_{\gamma}(0) = \Wnd_{\gamma}(0)
	\end{align}
	が得られた.
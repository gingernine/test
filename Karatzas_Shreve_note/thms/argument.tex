\section{偏角}
	\begin{itembox}[l]{以降の流れを忘れないうちにメモ}
		\begin{itemize}
			\item 偏角をカルタン流に定義するか,或いはカルタン流とのつながりを書く.
			\item $\pvarg = \Im \circ \pvlog$を示す.
		\end{itemize}
	\end{itembox}
	
	$0$でない複素数$z$に対して
	\begin{align}
		z = \exp{w}
	\end{align}
	を満たす複素数$w$を$z$の対数と呼んだが,このとき
	\begin{align}
		\frac{z}{|z|} = e^{\isym \cdot \Im{w}}
	\end{align}
	が成立する.$\Im{w}$の様に
	\begin{align}
		\frac{z}{|z|} = e^{\isym \cdot \theta}
	\end{align}
	を満たす実数$\theta$のことを$z$の{\bf 偏角}\index{へんかく@偏角}{\bf (argument)}と呼ぶ.
	対数と同様に偏角も整数の個数だけ存在する.
	
	\begin{screen}
		\begin{dfn}[偏角]
			複素数$z$に対して,その偏角の全体
			\begin{align}
				\Set{\theta \in \R}{z = |z| \cdot \exp{(\isym \cdot \theta)}}
			\end{align}
			を対応させる$\C$上の写像を
			\begin{align}
				\arg
			\end{align}
			と書く.
		\end{dfn}
	\end{screen}
	
	$z$を$0$でない複素数とするとき,
	\begin{align}
		\arg{z}
	\end{align}
	は$z$の対数の虚部の全体
	\begin{align}
		\Set{\theta}{\exists w \in \C\, \left(\, z = \exp{w} \wedge \theta = \Im{w}\, \right)}
	\end{align}
	に一致する.実際,$\theta$を$z$の偏角とすれば
	\begin{align}
		\frac{z}{|z|} = e^{\isym \cdot \theta}
	\end{align}
	が成り立つので,
	\begin{align}
		w \defeq \pvlog{|z|} + \isym \cdot \theta
	\end{align}
	により複素数$w$を定めれば
	\begin{align}
		e^{w} = e^{\pvlog{|z|}} \cdot e^{\isym \cdot \theta} = |z| \cdot \frac{z}{|z|} = z
	\end{align}
	が成立する.つまり$w$は$z$の対数であり,$\pvlog{|z|}$が実数であるから$\theta$は$w$の虚部である.逆に,
	$\theta$を実数とし,
	\begin{align}
		z = \exp{w} \wedge \theta = \Im{w}
	\end{align}
	を満たす複素数$w$が取れるとする.この場合は冒頭に書いた内容から
	\begin{align}
		\frac{z}{|z|} = e^{\isym \cdot \Im{w}} = e^{\isym \cdot \theta}
	\end{align}
	が成立するので$\theta$は$z$の偏角である.
	
	
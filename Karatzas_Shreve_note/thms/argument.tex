\section{偏角}
	
	$0$でない複素数$z$に対して
	\begin{align}
		z = \exp{w}
	\end{align}
	を満たす複素数$w$を$z$の対数と呼んだが,このとき
	\begin{align}
		\frac{z}{|z|} = e^{\isym \cdot \Im{w}}
	\end{align}
	が成立する.$\Im{w}$の様に
	\begin{align}
		\frac{z}{|z|} = e^{\isym \cdot \theta}
	\end{align}
	を満たす実数$\theta$のことを$z$の{\bf 偏角}\index{へんかく@偏角}{\bf (argument)}と呼ぶ.
	対数と同様に偏角も整数の個数だけ存在する.
	
	\begin{screen}
		\begin{dfn}[偏角]
			複素数$z$に対して,その偏角の全体
			\begin{align}
				\Set{\theta \in \R}{z = |z| \cdot \exp{(\isym \cdot \theta)}}
			\end{align}
			を対応させる$\C$上の写像を
			\begin{align}
				\arg
			\end{align}
			と書く.$\arg$もまた多価関数であるが,何らかの条件によって偏角を抜き取れば``関数''となり,その抜き取る操作を{\bf 偏角の枝を取る}という.
		\end{dfn}
	\end{screen}
	
	$z$を$0$でない複素数とするとき,
	\begin{align}
		\arg{z}
	\end{align}
	は$z$の対数の虚部の全体
	\begin{align}
		\Set{\theta}{\exists w \in \C\, \left(\, z = \exp{w} \wedge \theta = \Im{w}\, \right)}
	\end{align}
	に一致する.実際,$\theta$を$z$の偏角とすれば
	\begin{align}
		\frac{z}{|z|} = e^{\isym \cdot \theta}
	\end{align}
	が成り立つので,
	\begin{align}
		w \defeq \pvlog{|z|} + \isym \cdot \theta
	\end{align}
	により複素数$w$を定めれば
	\begin{align}
		e^{w} = e^{\pvlog{|z|}} \cdot e^{\isym \cdot \theta} = |z| \cdot \frac{z}{|z|} = z
	\end{align}
	が成立する.つまり$w$は$z$の対数であり,$\pvlog{|z|}$が実数であるから$\theta$は$w$の虚部である.逆に,
	$\theta$を実数とし,
	\begin{align}
		z = \exp{w} \wedge \theta = \Im{w}
	\end{align}
	を満たす複素数$w$が取れるとする.この場合は冒頭に書いた内容から
	\begin{align}
		\frac{z}{|z|} = e^{\isym \cdot \Im{w}} = e^{\isym \cdot \theta}
	\end{align}
	が成立するので$\theta$は$z$の偏角である.
	
	$z$を$0$でない複素数とするとき,
	\begin{align}
		-\pi < \theta \leq \pi
	\end{align}
	を満たす$z$の偏角を$\arg{z}$の{\bf 主値}\index{しゅち@主値}{\bf (principal value)}
	と呼ぶ.$z$の偏角の主値を
	\begin{align}
		\pv{\arg{z}}
	\end{align}
	と書くとき,
	\begin{align}
		\C \backslash \{0\} \ni z \longmapsto \pv{\arg{z}}
	\end{align}
	なる対応関係で定める$\C \backslash \{0\}$上の写像を
	\begin{align}
		\pvarg
	\end{align}
	と書く.正式には$\pvarg$とは
	\begin{align}
		\pvarg \defeq \Set{x}{\exists z \in \C\, \exists \theta \in \R\, 
		\left[\, x=(z,\theta) \wedge z \neq 0 \wedge z = |z| \cdot \exp{(\isym \cdot \theta)} \wedge
		-\pi < \theta \leq \pi\, \right]}
	\end{align}
	で定められる関係である.
	
	\begin{screen}
		\begin{thm}[偏角の主値は対数の主値の虚部]
			\begin{align}
				\pvarg = \Im \circ \pvlog.
			\end{align}
		\end{thm}
	\end{screen}
	
	\begin{sketch}
		
	\end{sketch}
	
	{\bf 特に$\pvarg$は$\C \backslash \{0\}$から$\R$への連続写像である.}
	
	\begin{itembox}[l]{偏角の群論的な視点}
		
	\end{itembox}
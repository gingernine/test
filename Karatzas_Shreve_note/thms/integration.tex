\section{積分}
\subsection{積分}
	\begin{screen}
		\begin{thm}[複素数値可測$\Longleftrightarrow$実部虚部が可測]\label{thm:measurability_of_complex_measurable_functions}
			$(X,\mathscr{F})$を可測空間,$f:X \longrightarrow \C$とするとき,
			$f$が$\mathscr{F}/\borel{\C}$-可測であることと
			$f$の実部$u$と虚部$v$がそれぞれ$\mathscr{F}/\borel{\R}$-可測であることは同値である.
		\end{thm}
	\end{screen}
	
	\begin{prf}
		$z \in \C$に対し$x,y \in \C$の組が唯一つ対応し$z = x + i y$を満たす.この対応関係により定める写像
		\begin{align}
			\varphi:\C \ni z \longmapsto (x,y) \in \R^2
		\end{align}
		は位相同型である.射影を$p_1:\R^2 \ni (x,y) \longmapsto x,
		\ p_2:\R^2 \ni (x,y) \longmapsto y$とすれば
		\begin{align}
			u = p_1 \circ \varphi \circ f,
			\quad v = p_2 \circ \varphi \circ f
		\end{align}
		となるから,$f$が$\mathscr{F}/\borel{\C}$-可測であるなら
		$p_1,p_2,\varphi$の連続性より
		\begin{align}
			u^{-1}(A) = f^{-1} \circ \varphi^{-1} \circ p_1^{-1}(A) \in \mathscr{F},
			\quad v^{-1}(A) = f^{-1} \circ \varphi^{-1} \circ p_2^{-1}(A) \in \mathscr{F},
			\quad (\forall A \in \borel{\R})
		\end{align}
		が成り立ち$u,v$の$\mathscr{F}/\borel{\R}$-可測性が従う.逆に$u,v$が$\mathscr{F}/\borel{\R}$-可測であるとき,
		\begin{align}
			f^{-1}(B) = \Set{x \in X}{(u(x),v(x)) \in \varphi(B)} \in \mathscr{F},
			\quad (\forall B \in \borel{\C})
		\end{align}
		が成り立ち$f$の$\mathscr{F}/\borel{\C}$-可測性が出る.
		\QED
	\end{prf}
	
	\begin{screen}
		\begin{thm}[和・積・商の可測性]
			
		\end{thm}
	\end{screen}
	
	\begin{screen}
		\begin{thm}[相対位相のBorel集合族]\label{thm:Borel_algebra_of_relative_topology}
			$(S,\mathscr{O})$を位相空間とする.部分集合$A \subset S$に対して
			\begin{align}
				\borel{A} \coloneqq \sigma\left[ \Set{A \cap O}{O \in \mathscr{O}} \right]
			\end{align}
			とおくとき次が成り立つ:
			\begin{align}
				\borel{A} = \Set{A \cap E}{E \in \borel{S}}.
			\end{align}
			また$A \in \borel{S}$なら$\borel{A} \subset \borel{S}$となる.
		\end{thm}
	\end{screen}
	
	$\R$-値可測関数は$\C$-値可測関数でもある.
	
	\begin{screen}
		\begin{thm}[単関数近似列の存在]
			$(X,\mathscr{F})$を可測空間とする.
			\begin{description} 
				\item[(1)] 任意の$\mathscr{F}/\borel{[0,\infty]}$-可測写像$f$に対し
					\begin{align}
						0 \leq f_1 \leq f_2 \leq \cdots \leq f;
						\quad \lim_{n \to \infty} f_n(x) = f(x),\ (\forall x \in X)
					\end{align}
					を満たす$\mathscr{F}/\borel{[0,\infty)}$-可測単関数列$(f_n)_{n=1}^\infty$が存在する.
					
				\item[(2)]
					 任意の$\mathscr{F}/\borel{\C}$-可測写像$f$に対し
					\begin{align}
						0 \leq |f_1| \leq |f_2| \leq \cdots \leq |f|;
						\quad \lim_{n \to \infty} f_n(x) = f(x),\ (\forall x \in X)
					\end{align}
					を満たす$\mathscr{F}/\borel{\C}$-可測単関数列$(f_n)_{n=1}^\infty$が存在する.
				
				\item[(3)] (1)または(2)において,$f$が$E \in \mathscr{F}$上で有界なら
					$f_n \defunc_E$は一様に$f \defunc_E$を近似する:
					\begin{align}
						\sup{x \in E}{\left| f_n(x) - f(x) \right|} \longrightarrow 0
						\quad (n \longrightarrow \infty).
					\end{align}
			\end{description}
		\end{thm}
	\end{screen}
	
	\begin{screen}
		\begin{dfn}[複素数値可測関数の正値測度に関する積分]
			$(X,\mathscr{F},\mu)$を正値測度空間,
			$f$を$\mathscr{F}/\borel{\C}$-可測関数とする.
			$u \coloneqq \Re{f},\ v \coloneqq \Im{f}$とおけば
			$|u|,|v| \leq |f| \leq |u| + |v|$より
			\begin{align}
				\mbox{$|f|$が可積分} \quad \Longleftrightarrow \quad
				\mbox{$u,v$が共に可積分}
			\end{align}
			が成り立つ.$|f|$が可積分のとき,$f$は可積分であるといい$f$の$\mu$に関する積分を次で定める:
			\begin{align}
				\int_X f\ d\mu
				\coloneqq \int_X u\ d\mu + i \int_X v\ d\mu.
			\end{align}
		\end{dfn}
	\end{screen}
	
	\begin{screen}
		\begin{thm}[Lebesgueの収束定理]
			$(X,\mathscr{F},\mu)$を正値測度空間,
			$f,\ f_n\ (n=1,2,\cdots)$を$\mathscr{F}/\borel{\C}$-可測な可積分関数とする.
			このとき,$f = \lim_{n \to \infty} f_n\ \mbox{$\mu$-a.e.}$かつ
			\begin{align}
				|f_n| \leq g, \quad \mbox{$\mu$-a.e.}
			\end{align}
			を満たす可積分関数$g$が存在するとき
			\begin{align}
				\int_X |f - f_n|\ d\mu \longrightarrow 0
				\quad (n \longrightarrow \infty).
			\end{align}
		\end{thm}
	\end{screen}
	
	\begin{screen}
		\begin{thm}[積分の線形性・積分作用素の有界性]
			$(X,\mathscr{F})$を可測空間とし,$\mu$を$\mathscr{F}$上の正値測度とする.
			\begin{description}
				\item[(1)] 任意の$\mathscr{F}/\borel{\C}$-可測可積分関数$f,g$と
					$\alpha,\beta \in \C$に対して次が成り立つ:
					\begin{align}
						\int_X \alpha f + \beta g\ d\mu
						= \alpha \int_X f\ d\mu + \beta \int_X g\ d\mu.
					\end{align}
					
				\item[(2)] 任意の$\mathscr{F}/\borel{\C}$-可測可積分関数$f$に対して次が成り立つ:
					\begin{align}
						\left| \int_X f\ d\mu \right| \leq \int_X |f|\ d\mu.
					\end{align}
			\end{description}	
		\end{thm}
	\end{screen}
	
	\begin{prf}\mbox{}
		\begin{description}
			\item[(1)] 
			
			\item[(2)]
				$\alpha \coloneqq \int_X f\ d\mu$とおけば,$\alpha \neq 0$の場合
				\begin{align}
					|\alpha|
					= \frac{\overline{\alpha}}{|\alpha|} \int_X f\ d\mu
					= \int_X \frac{\overline{\alpha}}{|\alpha|} f\ d\mu
				\end{align}
				が成り立ち
				\begin{align}
					|\alpha| = \Re{|\alpha|}
					= \Re{\int_X \frac{\overline{\alpha}}{|\alpha|} f\ d\mu}
					= \int_X \Re{\frac{\overline{\alpha}}{|\alpha|} f}\ d\mu
					\leq \int_X |f|\ d\mu
				\end{align}
				が従う.$\alpha = 0$の場合も不等式は成立する.
				\QED
		\end{description}
	\end{prf}
	
	\begin{screen}
		\begin{lem}
			$S$を実数の集合とする.$-S \coloneqq \Set{-s}{s \in S}$とおくとき次が成り立つ:
			\begin{align}
				\inf{}{S} = -\sup{}{(-S)},
				\quad \sup{}{S} = -\inf{}{(-S)}.
			\end{align}
		\end{lem}
	\end{screen}
	
	\begin{prf}
		任意の$s \in S$に対して$-s \leq \sup{}{(-S)}$より
		$\inf{}{S} \geq -\sup{}{(-S)}$となる.一方で任意の$s \in S$に対し
		$\inf{}{S} \leq s$より$-s \leq -\inf{}{S}$となり
		$\sup{}{(-S)} \leq -\inf{}{S}$が従うから
		$-\sup{}{(-S)} \geq \inf{}{S}$も成り立ち
		$\inf{}{S} = -\sup{}{(-S)}$が出る.
		\QED
	\end{prf}
	
	\begin{screen}
		\begin{thm}[写像の値域は積分の平均値の範囲を出ない]\label{thm:mean_value_of_integral_and_closed_set}
			$(X,\mathscr{F},\mu)$を$\sigma$-有限測度空間,
			$f:X \longrightarrow \C$を$\mathscr{F}/\borel{\C}$-可測かつ$\mu$-可積分な関数,
			$C \subset \C$を閉集合とする.このとき
			\begin{align}
				\frac{1}{\mu(E)}\int_E f\ d\mu \in C,
				\quad (\forall E \in \mathscr{F},\ 0 < \mu(E) < \infty)
				\label{eq:thm_mean_value_of_integral_and_closed_set}
			\end{align}
			なら次が成り立つ:
			\begin{align}
				f(x) \in C \quad \mbox{$\mu$-a.e.}x \in X.
			\end{align}
		\end{thm}
	\end{screen}
	$C=\R$なら$f$は殆ど至る所$\R$値であり,
	$C=\{0\}$なら殆ど至る所$f=0$である.
	\begin{prf}
		$\sigma$-有限の仮定より次を満たす$\{X_n\}_{n=1}^\infty \subset \mathscr{F}$が存在する:
		\begin{align}
			\mu(X_n) < \infty,\ (\forall n \geq 1);
			\quad X = \bigcup_{n=1}^\infty X_n.
		\end{align}
		$C = \C$なら$f(x) \in C\ (\forall x \in X)$である.
		$C \neq \C$の場合,任意の$\alpha \in \C \backslash C$に対し
		或る$r > 0$が存在して
		\begin{align}
			B_r(\alpha) \coloneqq \Set{z \in \C}{|z - \alpha| \leq r} \subset \C \backslash C
		\end{align}
		を満たす.ここで
		\begin{align}
			E \coloneqq f^{-1}\left( B_r(\alpha) \right),
			\quad E_n \coloneqq E \cap X_n
		\end{align}
		とおけば,任意の$n \geq 1$について$\mu(E_n) > 0$なら
		\begin{align}
			\left| \frac{1}{\mu(E_n)}\int_{E_n} f\ d\mu - \alpha \right|
			= \left| \frac{1}{\mu(E_n)}\int_{E_n} f - \alpha\ d\mu \right|
			\leq \frac{1}{\mu(E_n)}\int_{E_n} |f - \alpha|\ d\mu
			\leq r
		\end{align}
		となり(\refeq{eq:thm_mean_value_of_integral_and_closed_set})に反するから,
		$\mu(E_n) = 0\ (\forall n \geq 1)$及び
		\begin{align}
			\mu(E) = \mu\Biggl( \bigcup_{n=1}^\infty E_n \Biggr) 
			\leq \sum_{n=1}^\infty \mu(E_n) = 0
		\end{align}
		が従う.$\C \backslash C$は開集合であり$B_r(\alpha)$の形の集合の可算和で表せるから
		\begin{align}
			\mu\left( f^{-1}\left( \C \backslash C \right) \right) = 0
		\end{align}
		が成り立ち主張が得られる.
		\QED
	\end{prf}
	
	\begin{screen}
		\begin{thm}[可積分なら積分値を一様に小さくできる]\label{thm:integrable_intvalue_uniformly_shrinking}
			$(X,\mathscr{F},\mu)$を正値測度空間,$f:X \longrightarrow \C$
			を$\mathscr{F}/\borel{\C}$-可測関数とするとき,
			$f$が可積分なら,任意の$\epsilon > 0$に対して或る$\delta > 0$が存在し次を満たす:
			\begin{align}
				\mu(E) < \delta \quad \Longrightarrow \quad \int_E |f|\ d\mu < \epsilon.
			\end{align}
		\end{thm}
	\end{screen}
	
	\begin{prf}
		$X_n \coloneqq \{|f| \leq n\}$により増大列$(X_n)_{n=1}^\infty$を定めれば
		単調収束定理より
		\begin{align}
			\int_X |f|\ d\mu = \lim_{n \to \infty} \int_{X_n} |f|\ d\mu
		\end{align}
		となるから,任意の$\epsilon > 0$に対し或る$n_0 \geq 1$が存在して
		\begin{align}
			\int_{X \backslash X_{n_0}} |f|\ d\mu < \epsilon
		\end{align}
		が成り立つ.このとき$\mu(E) < \delta \coloneqq \epsilon/n_0$なら
		\begin{align}
			\int_E |f|\ d\mu
			= \int_{E \cap X_{n_0}} |f|\ d\mu + \int_{E \cap (X \backslash X_{n_0})} |f|\ d\mu
			\leq n_0 \mu(E) + \int_{X \backslash X_{n_0}} |f|\ d\mu
			< 2\epsilon
		\end{align}
		が従う.
		\QED
	\end{prf}
	
	\subsection{関数列の収束}
		\begin{screen}
			\begin{dfn}[概収束すれば測度収束する]
				$(X,\mathscr{F},\mu)$を正値有限測度空間とする.
				$(f_n)_{n=1}^\infty,f$を全て$\mathscr{F}/\borel{\C}$-可測関数とするとき,
				$\lim_{n \to \infty} f_n = f,\ \mbox{$\mu$-a.e.}$なら
				$(f_n)_{n=1}^\infty$は$f$に測度収束する.
			\end{dfn}
		\end{screen}
		
		\begin{prf}
			任意の$\epsilon > 0$に対し
			\begin{align}
				A^n_\epsilon \coloneqq \left\{ |f_n - f| > \epsilon \right\}
			\end{align}
			とおけば,Lebesgueの収束定理より任意の$k \geq 1$で
			\begin{align}
				\epsilon \mu\left(A^n_\epsilon\right)
				\leq \int_{A^n_\epsilon} |f_n - f| \wedge \epsilon\ d\mu
				\leq \int_{X} |f_n - f| \wedge \epsilon\ d\mu
				\longrightarrow 0
				\quad (n \longrightarrow \infty)
			\end{align}
			が成立する.
			\QED
		\end{prf}
		
		上の定理で有限性を外すときの反例を示す.
		$X = \R$,$\mu$を一次元Lebesgue測度とするとき,
		\begin{align}
			f_n \coloneqq \defunc_{\R \backslash (-n,n)}
		\end{align}
		で定める関数列$(f_n)_{n=1}^\infty$は零写像に各点収束するが,$0 < \epsilon < 1$に対し
		\begin{align}
			\mu\left( f_n > \epsilon \right) = \mu((-\infty,-n] \cup [n,\infty)) = \infty,
			\quad (\forall n \geq 1)
		\end{align}
		を満たすから測度収束しない.
		
		\begin{screen}
			\begin{thm}[測度収束列の概収束部分列]\label{thm:convergence_in_measure_then_convergence_almost_everywhere}
				$(X,\mathscr{F},\mu)$を正値測度空間,
				$(f_n)_{n=1}^\infty,f$を全て$\mathscr{F}/\borel{\C}$-可測関数とするとき,
				$(f_n)_{n=1}^\infty$が$f$に測度収束するなら
				或る部分列$(f_{n_k})_{k=1}^\infty$は$f$に概収束する.
			\end{thm}
		\end{screen}
		
		\begin{prf}
			$(f_n)_{n=1}^\infty$が$f$に測度収束するとき,任意の$k \geq 1$に対し
			\begin{align}
				\mu\left( |f_{n_k} - f| > \frac{1}{2^k}\right) < \frac{1}{2^k}
			\end{align}
			を満たす添数列$n_1 < n_2 < n_3 < \cdots$が取れる.
			\begin{align}
				A_k \coloneqq \left\{|f_{n_k} - f| > \frac{1}{2^k}\right\},
				\quad A \coloneqq \bigcup_{k\geq1} \bigcap_{j>k} A_j^c
			\end{align}
			とおけば,$\mu(A^c) \leq \mu\left(\bigcup_{j>k} A_j\right),\ (\forall k \geq 1)$かつ
			\begin{align}
				\mu\Biggl(\bigcup_{j>k} A_j\Biggr) \leq \sum_{j>k} \frac{1}{2^j}
				= \frac{1}{2^k}
			\end{align}
			より$\mu(A^c) = 0$が従い,$x \in A$なら或る$k = k(x)$が存在して
			\begin{align}
				|f_{n_j}(x) - f(x)| \leq \frac{1}{2^j}, \quad (\forall j > k)
			\end{align}
			となるから$\lim_{k \to \infty} f_{n_k}(x) = f(x)$が満たされる.
			\QED
		\end{prf}
		
		\begin{screen}
			\begin{thm}[平均収束すれば測度収束する]
				$p \in (0,\infty)$,$(X,\mathscr{F},\mu)$を正値測度空間,
				$(f_n)_{n=1}^\infty,f$を全て$\mathscr{F}/\borel{\C}$-可測関数とするとき,
				\begin{align}
					\int_X |f_n - f|^p\ d\mu \longrightarrow 0
					\quad (n \longrightarrow \infty)
				\end{align}
				なら$(f_n)_{n=1}^\infty$は$f$に測度収束する.
			\end{thm}
		\end{screen}
		
		\begin{prf}
			任意の$\epsilon > 0$に対し
			\begin{align}
				\epsilon^p \mu\left(|f_n - f| > \epsilon\right)
				\leq \int_{\left\{|f_n - f| > \epsilon\right\}} |f_n-f|^p\ d\mu
				\leq \int_X |f_n - f|^p\ d\mu \longrightarrow 0
				\quad (n \longrightarrow \infty)
			\end{align}
			が成立する.
			\QED
		\end{prf}
		
		\begin{screen}
			\begin{thm}[Egorov]
			\end{thm}
		\end{screen}
		
	\subsection{Radon測度}
		\begin{screen}
			\begin{thm}[Riesz-Markov-Kakutaniの表現定理]
			\end{thm}
		\end{screen}
		
		\begin{screen}
			\begin{thm}[正値Borel測度の正則性定理]
				
			\end{thm}
		\end{screen}
	
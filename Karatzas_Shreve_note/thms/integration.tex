\section{積分}
	\begin{screen}
		\begin{thm}[複素数値関数の可測性]\label{thm:measurability_of_complex_measurable_functions}
			$(X,\mathscr{F})$を可測空間,$f:X \longrightarrow \C$とする.このとき,
			$f$が$\mathscr{F}/\borel{\C}$-可測であることと
			$f$の実部$u$と虚部$v$がそれぞれ$\mathscr{F}/\borel{\R}$-可測であることは同値である.
		\end{thm}
	\end{screen}
	
	\begin{prf}
		$z \in \C$に対し$x,y \in \C$の組が唯一つ対応し$z = x + i y$を満たす.この対応関係により定める写像
		\begin{align}
			\varphi:\C \ni z \longmapsto (x,y) \in \R^2
		\end{align}
		は位相同型である.射影を$p_1:\R^2 \ni (x,y) \longmapsto x,
		\ p_2:\R^2 \ni (x,y) \longmapsto y$とすれば
		\begin{align}
			u = p_1 \circ \varphi \circ f,
			\quad v = p_2 \circ \varphi \circ f
		\end{align}
		となるから,$f$が$\mathscr{F}/\borel{\C}$-可測であるなら
		$p_1,p_2,\varphi$の連続性より
		\begin{align}
			u^{-1}(A) = f^{-1} \circ \varphi^{-1} \circ p_1^{-1}(A) \in \mathscr{F},
			\quad v^{-1}(A) = f^{-1} \circ \varphi^{-1} \circ p_2^{-1}(A) \in \mathscr{F},
			\quad (\forall A \in \borel{\R})
		\end{align}
		が成り立ち$u,v$の$\mathscr{F}/\borel{\R}$-可測性が従う.逆に$u,v$が$\mathscr{F}/\borel{\R}$-可測であるとき,
		\begin{align}
			f^{-1}(B) = \Set{x \in X}{(u(x),v(x)) \in \varphi(B)} \in \mathscr{F},
			\quad (\forall B \in \borel{\C})
		\end{align}
		が成り立ち$f$の$\mathscr{F}/\borel{\C}$-可測性が出る.
		\QED
	\end{prf}
	
	\begin{screen}
		\begin{thm}[単関数近似列の存在]
			$(X,\mathscr{F})$を可測空間とする.
			\begin{description} 
				\item[(1)] 任意の$\mathscr{F}/\borel{[0,\infty]}$-可測写像$f$に対し
					\begin{align}
						0 \leq f_1 \leq f_2 \leq \cdots \leq f;
						\quad \lim_{n \to \infty} f_n(x) = f(x),\ (\forall x \in X)
					\end{align}
					を満たす$\mathscr{F}/\borel{[0,\infty)}$-可測単関数列$(f_n)_{n=1}^\infty$が存在する.
					
				\item[(2)]
					 任意の$\mathscr{F}/\borel{\C}$-可測写像$f$に対し
					\begin{align}
						0 \leq |f_1| \leq |f_2| \leq \cdots \leq |f|;
						\quad \lim_{n \to \infty} f_n(x) = f(x),\ (\forall x \in X)
					\end{align}
					を満たす$\mathscr{F}/\borel{\C}$-可測単関数列$(f_n)_{n=1}^\infty$が存在する.
				
				\item[(3)] (1)または(2)において,$f$が$E \in \mathscr{F}$上で有界なら
					$f_n \defunc_E$は一様に$f \defunc_E$を近似する:
					\begin{align}
						\sup{x \in E}{\left| f_n(x) - f(x) \right|} \longrightarrow 0
						\quad (n \longrightarrow \infty).
					\end{align}
			\end{description}
		\end{thm}
	\end{screen}
		
	\begin{screen}
		\begin{lem}
			$S$を実数の集合とする.$-S \coloneqq \Set{-s}{s \in S}$とおくとき次が成り立つ:
			\begin{align}
				\inf{}{S} = -\sup{}{(-S)},
				\quad \sup{}{S} = -\inf{}{(-S)}.
			\end{align}
		\end{lem}
	\end{screen}
	
	\begin{prf}
		任意の$s \in S$に対して$-s \leq \sup{}{(-S)}$より
		$\inf{}{S} \geq -\sup{}{(-S)}$となる.一方で任意の$s \in S$に対し
		$\inf{}{S} \leq s$より$-s \leq -\inf{}{S}$となり
		$\sup{}{(-S)} \leq -\inf{}{S}$が従うから
		$-\sup{}{(-S)} \geq \inf{}{S}$も成り立ち
		$\inf{}{S} = -\sup{}{(-S)}$が出る.
		\QED
	\end{prf}
	
	\begin{screen}
		\begin{thm}[積分の平均値と写像の値域の関係]\label{thm:mean_value_of_integral_and_closed_set}
			$(X,\mathscr{F},\mu)$を$\sigma$-有限測度空間,
			$f:X \longrightarrow \C$を$\mathscr{F}/\borel{\C}$-可測可積分関数,
			$C \subset \C$を閉集合とする.このとき
			\begin{align}
				\frac{1}{\mu(E)}\int_E f\ d\mu \in C,
				\quad (\forall E \in \mathscr{F},\ 0 < \mu(E) < \infty)
				\label{eq:thm_mean_value_of_integral_and_closed_set}
			\end{align}
			なら次が成り立つ:
			\begin{align}
				f(x) \in C \quad \mbox{$\mu$-a.e.}x \in X.
			\end{align}
		\end{thm}
	\end{screen}
	$C=\R$なら$f$は殆ど至る所$\R$値であり,
	$C=\{0\}$なら殆ど至る所$f=0$である.
	\begin{prf}
		$\sigma$-有限の仮定より次を満たす$\{X_n\}_{n=1}^\infty \subset \mathscr{F}$が存在する:
		\begin{align}
			\mu(X_n) < \infty,\ (\forall n \geq 1);
			\quad X = \bigcup_{n=1}^\infty X_n.
		\end{align}
		$C = \C$なら$f(x) \in C\ (\forall x \in X)$である.
		$C \neq \C$の場合,任意の$\alpha \in \C \backslash C$に対し
		或る$r > 0$が存在して
		\begin{align}
			B_r(\alpha) \coloneqq \Set{z \in \C}{|z - \alpha| \leq r} \subset \C \backslash C
		\end{align}
		を満たす.ここで
		\begin{align}
			E \coloneqq f^{-1}\left( B_r(\alpha) \right),
			\quad E_n \coloneqq E \cap X_n
		\end{align}
		とおけば,任意の$n \geq 1$について$\mu(E_n) > 0$なら
		\begin{align}
			\left| \frac{1}{\mu(E_n)}\int_{E_n} f\ d\mu - \alpha \right|
			= \left| \frac{1}{\mu(E_n)}\int_{E_n} f - \alpha\ d\mu \right|
			\leq \frac{1}{\mu(E_n)}\int_{E_n} |f - \alpha|\ d\mu
			\leq r
		\end{align}
		となり(\refeq{eq:thm_mean_value_of_integral_and_closed_set})に反するから,
		$\mu(E_n) = 0\ (\forall n \geq 1)$及び
		\begin{align}
			\mu(E) = \mu\Biggl( \bigcup_{n=1}^\infty E_n \Biggr) 
			\leq \sum_{n=1}^\infty \mu(E_n) = 0
		\end{align}
		が従う.$\C \backslash C$は開集合であり$B_r(\alpha)$の形の集合の可算和で表せるから
		\begin{align}
			\mu\left( f^{-1}\left( \C \backslash C \right) \right) = 0
		\end{align}
		が成り立ち主張が得られる.
		\QED
	\end{prf}
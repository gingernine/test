\section{積分}
	\begin{screen}
		\begin{thm}[積分の平均値と写像の値域の関係]\label{thm:mean_value_of_integral_and_closed_set}
			$(X,\mathscr{F},\mu)$を$\sigma$-有限測度空間,
			$f:X \longrightarrow \C$を$\mathscr{F}/\borel{\C}$-可測可積分関数,
			$C \subset \C$を閉集合とする.このとき
			\begin{align}
				\frac{1}{\mu(E)}\int_E f\ d\mu \in C,
				\quad (\forall E \in \mathscr{F},\ 0 < \mu(E) < \infty)
				\label{eq:thm_mean_value_of_integral_and_closed_set}
			\end{align}
			なら次が成り立つ:
			\begin{align}
				f(x) \in C \quad \mbox{$\mu$-a.e.}x \in X.
			\end{align}
		\end{thm}
	\end{screen}
	$C=\R$なら$f$は殆ど至る所$\R$値であり,
	$C=\{0\}$なら殆ど至る所$f=0$である.
	\begin{prf}
		$\sigma$-有限の仮定より次を満たす$\{X_n\}_{n=1}^\infty \subset \mathscr{F}$が存在する:
		\begin{align}
			\mu(X_n) < \infty,\ (\forall n \geq 1);
			\quad X = \bigcup_{n=1}^\infty X_n.
		\end{align}
		$C = \C$なら$f(x) \in C\ (\forall x \in X)$である.
		$C \neq \C$の場合,任意の$\alpha \in \C \backslash C$に対し
		或る$r > 0$が存在して
		\begin{align}
			B_r(\alpha) \coloneqq \Set{z \in \C}{|z - \alpha| \leq r} \subset \C \backslash C
		\end{align}
		を満たす.ここで
		\begin{align}
			E \coloneqq f^{-1}\left( B_r(\alpha) \right),
			\quad E_n \coloneqq E \cap X_n
		\end{align}
		とおけば,任意の$n \geq 1$について$\mu(E_n) > 0$なら
		\begin{align}
			\left| \frac{1}{\mu(E_n)}\int_{E_n} f\ d\mu - \alpha \right|
			= \left| \frac{1}{\mu(E_n)}\int_{E_n} f - \alpha\ d\mu \right|
			\leq \frac{1}{\mu(E_n)}\int_{E_n} |f - \alpha|\ d\mu
			\leq r
		\end{align}
		となり(\refeq{eq:thm_mean_value_of_integral_and_closed_set})に反するから,
		$\mu(E_n) = 0\ (\forall n \geq 1)$及び
		\begin{align}
			\mu(E) = \mu\Biggl( \bigcup_{n=1}^\infty E_n \Biggr) 
			\leq \sum_{n=1}^\infty \mu(E_n) = 0
		\end{align}
		が従う.$\C \backslash C$は開集合であり$B_r(\alpha)$の形の集合の可算和で表せるから
		\begin{align}
			\mu\left( f^{-1}\left( \C \backslash C \right) \right) = 0
		\end{align}
		が成り立ち主張が得られる.
		\QED
	\end{prf}
\subsection{Cauchyの定理}
	
	\begin{screen}
		\begin{thm}[複素線積分のFubiniの定理]
			
		\end{thm}
	\end{screen}
	
	\begin{screen}
		\begin{dfn}[閉路]
			始点と終点が一致する路を{\bf 閉路}\index{へいろ@閉路}{\bf (closed contour)}と呼ぶ.
			つまり,$\gamma$が閉路であるとは
			\begin{align}
				[\alpha,\beta] = \dom{\gamma}
			\end{align}
			を満たす実数$\alpha$と$\beta$を取ったときに
			\begin{align}
				\gamma(\alpha) = \gamma(\beta)
			\end{align}
			が成り立つということである.
		\end{dfn}
	\end{screen}
	
	例えば,いままで扱ってきた
	\begin{align}
		[0,2\cdot\pi] \ni \theta \longmapsto a + r \cdot e^{\isym \cdot \theta}
	\end{align}
	なる路は閉路である.
	
	\begin{screen}
		\begin{dfn}[指数]
			$\gamma$を閉路とし,$z$を
			\begin{align}
				z \notin \ran{\gamma}
			\end{align}
			を満たす複素数とするとき,
			\begin{align}
				\frac{1}{2 \cdot \pi \cdot \isym} \cdot \int_{\gamma} \frac{1}{\zeta - z}\ d\zeta
			\end{align}
			を$\gamma$の$z$周りの{\bf 指数}\index{しすう@指数}{\bf (index)}と呼ぶ.
		\end{dfn}
	\end{screen}
	
	閉路$\gamma$が与えられたとき,$\ran{\gamma}$に属さない複素数に対して,その周りの$\gamma$の指数を対応させる写像を
	\begin{align}
		\Ind_{\gamma}
	\end{align}
	と書く.
	
	定理\ref{thm:holomorphic_then_expanded}と定理\ref{thm:series_expanded_then_differentiable}より
	$\Ind_{\gamma}$は$\C \backslash \ran{\gamma}$の各要素において微分可能である.すなわち
	\begin{align}
		\Ind_{\gamma} \in \Holomorphic{\C \backslash \ran{\gamma}}
	\end{align}
	である.
	
	\begin{screen}
		\begin{thm}[Cauchyの積分定理]
			$\Omega$を開集合とし,$f$を$\Omega$上の正則関数とし,$\gamma$を閉路とする.このとき,
			\begin{align}
				\ran{\gamma} \subset \Omega
			\end{align}
			かつ,$\C$の開集合$\Psi$で
			\begin{align}
				\C \backslash \Omega \subset \Psi \subset \Ind_{\gamma}^{-1} \ast \{0\}
			\end{align}
			を満たすものが取れるなら,$\Omega \backslash \ran{\gamma}$の任意の要素$z$に対して
			\begin{align}
				f(z) \cdot \Ind_{\gamma}(z) = \frac{1}{2\cdot\pi\cdot\isym} \cdot \int_{\gamma} \frac{f(w)}{w - z}\ dw
			\end{align}
			が成立し,このとき特に
			\begin{align}
				\int_\gamma f = 0
			\end{align}
			が成立する.
		\end{thm}
	\end{screen}
	
	\begin{sketch}[大雑把]
		$\Omega \times \Omega$上の写像$g$を
		\begin{align}
			\Omega \times \Omega \ni (z,w) \longmapsto
			\begin{cases}
				{\displaystyle \frac{f(w) - f(z)}{w-z}} & \mbox{if } z \neq w \\
				f'(z) & \mbox{if } z = w
			\end{cases}
		\end{align}
		により定めれば,$g$は連続である.
		\begin{align}
			\Omega \ni z \longmapsto \frac{1}{2\cdot\pi\cdot\isym} \cdot \int_{\gamma} g(z,w)\ dw
		\end{align}
		なる写像を$h$とすれば,$h$は$\Omega$上で連続である.$a,b,c$を$\Omega$の要素として,それらがなす三角集合が$\Omega$に含まれるとき,つまり
		\begin{align}
			\Set{z}{\exists t,s \in [0,1]\, 
			\left(\, z = (1-t) \cdot a 
			+ t \cdot (1-s) \cdot b 
			+ t \cdot s \cdot c\, \right)} \subset \Omega
		\end{align}
		であるとき,Fubiniの定理とGoursatの定理より
		\begin{align}
			&\int_{\seg{a}{b}} h + \int_{\seg{b}{c}} h + \int_{\seg{c}{a}} h \\
			&= \frac{1}{2 \cdot \pi \cdot \isym} \cdot \int_{\gamma} 
			\left(\int_{\seg{a}{b}} g_w + \int_{\seg{b}{c}} g_w + \int_{\seg{c}{a}} g_w\right)\ dw \\
			&= 0
		\end{align}
		が成立するので,Moreraの定理より
		\begin{align}
			h \in \Holomorphic{\Omega}
		\end{align}
		が従う.
		%\begin{align}
		%	\Psi \defeq \left(\Ind_{\gamma}^{-1} \ast \{0\}\right)^{\mathrm{o}}
		%\end{align}
		%により$\C$の開集合を定めて
		%\footnote{
		%	一見すると閉集合を引き戻しているので$\Psi$は閉集合であるかと思えるが,
		%	$\Ind_{\gamma}$とは$\Omega \backslash \ran{\gamma}$上の連続写像であり,
		%	$\Psi$はその相対位相で閉集合であるにすぎない.しかも$\Ind_{\gamma}$は整数値であるから,
		%	\begin{align}
		%		\Ind_{\gamma}^{-1} \ast \{0\} = \Ind_{\gamma}^{-1} \ast \Set{x \in \R}{-1 < x < 1}
		%	\end{align}
		%	が成り立つ.つまり$\Psi$は$\Omega \backslash \ran{\gamma}$の開集合でもあることになるが,
		%	この現象は$\Omega \backslash \ran{\gamma}$が連結でないことに起因する.そして
		%	$\Omega \backslash \ran{\gamma}$自体は$\C$の開集合なので$\Psi$は$\C$の開集合である.
		%}
		\begin{align}
			\C \ni z \longmapsto
			\begin{cases}
				{\displaystyle \frac{1}{2\cdot\pi\cdot\isym} \cdot \int_{\gamma} \frac{f(w)-f(z)}{w-z}\ dw} & \mbox{if } z \in \Omega \\
				{\displaystyle \frac{1}{2\cdot\pi\cdot\isym} \cdot \int_{\gamma} \frac{f(w)}{w-z}\ dw} & \mbox{if } z \in \Psi
			\end{cases}
		\end{align}
		なる写像を$\varphi$とすると,$\varphi$は整関数であって,
		\begin{align}
			\lim_{|z| \longrightarrow \infty} \varphi(z) = 0 
		\end{align}
		であるから
		\begin{align}
			\forall z \in \C\, (\, \varphi(z) = 0\, )
		\end{align}
		が成立する.ゆえに$\Omega \backslash \ran{\gamma}$の任意の要素$z$に対して
		\begin{align}
			\frac{1}{2\cdot\pi\cdot\isym} \cdot \int_{\gamma} \frac{f(w)-f(z)}{w-z}\ dw = 0
		\end{align}
		が成立するから,移項して
		\begin{align}
			\frac{1}{2\cdot\pi\cdot\isym} \cdot \int_{\gamma} \frac{f(w)}{w-z}\ dw
			= f(z) \cdot \Ind_{\gamma}(z)
		\end{align}
		を得る.$\Omega \backslash \ran{\gamma}$の要素$z$を取り
		\begin{align}
			\Omega \ni w \longmapsto f(w) \cdot (w - z)
		\end{align}
		なる写像を$F$と定めれば,
		\begin{align}
			F \in \Holomorphic{\Omega}
		\end{align}
		であるから
		\begin{align}
			\int_{\gamma} f = \int_{\gamma} \frac{F(w)}{w-z}\ dw = F(z) \cdot \Ind_{\gamma}(z) = 0
		\end{align}
		が成立する.
		\QED
	\end{sketch}
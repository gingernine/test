	\monologue{
		院生「現代的な数学では,数や関数など数学に関するあらゆるものは集合で構成されます.
			そして集合そのものは述語論理を基礎にして公理的に規定されます.
			この意味で集合論の勉強には論理学の知識が必要であると聞きますけれども,
			真に受けて論理学の本を眺めてみれば,はじめから集合そのものが出てきたり,
			変数に数で添え字をつけたり,述語関数などといったものを取り扱っていたりしているものばかりで残念です.
			論理学を基に集合論を展開しようというのですから,集合論の諸概念を予定して論理学を説明するのは本末転倒です.
			とはいえ集合論と論理学とは切っても切り離せないのですから,いっそ同時並行でそつなく理解してやりましょう.
			(いわゆるメタ数学についてはいまのところ手を出すつもりはありません.)」
	}
\subsection{言語}
	\begin{quote}
		初めに言(ことば)があった。言は神と共にあった。言は神であった。\\
		この言は、初めに神と共にあった。\\
		万物は言によって成った。成ったもので、言によらずに成ったものは何一つなかった。
	\end{quote}
	ヨハネによる福音書の冒頭である.本稿の世界もまた数学のことば,言い換えれば論理のみによって創られる(予定).

	\monologue{
		院生「私の指導教官に``新約聖書がはじめにギリシア語で書かれたとき,`ことば'にはlogosが充てられた.
			logosは`言語'の意味を持つと同時に`論理'の意味も持つ''と教わりました.
			つまり,ギリシア語版の福音書では``初めに論理があった''とも解釈できるのですね.
			一方で日本語訳では言葉ではなく言と書かれています.なぜ``言葉''ではなく``言''と書くのでしょうか.
			一説によれば言葉の葉の字の由来は万葉古今集仮名序にあり,
			現代的に説明すれば,見聞きしたり感動したりしたところを種にして生じる語彙のことを木の葉に喩えているらしいです.
			言葉は人が発するものであり,たいていの場合食い違いなく通用するのですから,すなわち
			葉が付かない``言''とは,人為の介入する前から世界を認識し,人が自覚する前から人の心に通底している
			コードとでも解釈されるでしょうか.聖書の引用文の通り%は森羅万象はことばによって成り,ことばによって尽くされるという意味であるから,
			キリスト教においてことばとは神であり森羅万象を超越しているのですから,言の字に神性を伴わせても良いですよね.
			本稿の世界もまた数学のことばによって創られますが,``はじめにことばありき''の名句が国籍や文化を問わず
			現代まで受け入れられてきたという事実を鑑みれば,ことばから始めようというのは人が生来持っている直観に対して自然な起こりなのでしょう.」
			%しかしながら,神なることばが世界の悉くを尽くせる一方で,人が創造する数学の世界は論理のみによっては完結し得ないという事実もあります.
	}
	
	まず数学の言語として$\mathcal{L}$というものを用意する.
	\begin{description}
		\item[使用文字] 自然言語から借用する文字は表にあるものに限る.
		\item[対象領域] $V$
		\item[述語記号] $=,\ \in$
		\item[論理記号] $\bot,\ \Longrightarrow,\ \wedge,\ \vee,\ \rightharpoondown$
		\item[量化記号] $\forall,\ \exists$
	\end{description}
	
	日常言語において,``あmt後右所sごぐふぉsdあじお''のように無作為に文字を並べただけでは意味不明な
	文字列が出来上がる.文字列は,何らかの規則に従って並ぶことで単語や文章として成立するのである.
	数学も同じで,一定の規則に従って並ぶ記号列のみを数学における文章として扱う.
	数学語において,名詞にあたるものは{\bf 対象}\index{たいしょう@対象}{\bf (individual)}と呼ばれる.
	$V$を対象領域といったが,これは$V$が$\mathcal{L}$の対象の全体を表しているという意味である.
	述語とは対象同士を結ぶものであり,最小単位の文章を形作る.例えば,対象$s,t$に対し
	\begin{align}
		s \in t
	\end{align}
	は数学の文章となり,日本語には``$s$は$t$の要素である''と翻訳される.
	数学における文章を{\bf 式}\index{しき@式}
	{\bf (formula)}或は{\bf 論理式}\index{ろんりしき@論理式}と呼ぶ.
	論理記号とは式同士を繋ぐ役割を持つ.
	
	\begin{screen}
		\begin{dfn}[変項]
			言語で使用される文字のうち,対象領域にもいずれの対象にも
			あてがわれておらず,かつデフォルトのフォントであるものを
			{\bf 変項}\index{へんこう@変項}{\bf (variable)}と呼ぶ.
		\end{dfn}
	\end{screen}
	
	仮に対象が$x,y,z$のみであるような言語を考えるときは,変項は$a$でも
	$b$でもよい.ただし$x$は変項として使えない.$x$は対象$x$を指すためである.
	ギリシア文字全体が対象領域なら,変項はこう言うであろう:
	\begin{align}
		\mbox{私はアルファでありオメガである.}
	\end{align}
	
	\monologue{
		院生「今後は$\emptyset$と${\boldsymbol\omega}$が特別な記号として出てくるでしょう.
			これらは対象として扱われるので変項として用いてはいけません.
			(しかしながら本編では変項として$\omega$がやたらと使われます.
			先に書いた${\boldsymbol\omega}$は最小の極限数(後述)を表す記号で,
			違いを出すために太字にしているのですが見分けづらい...
			でも確率論での$\omega$と極限数としての${\boldsymbol\omega}$が
			同時に登場する場面はあまりないと思われますからそれほど気にしないで良さそうですね.)
			また$\N$や$\R$など太字で特別な意味を持つようになる文字もありますから,
			変項は`デフォルトのフォントであるもの'と表示技巧上の制限を課しています.
			大抵の文字は特別な意味を持っていないので変項として扱って問題ありません.
			変項としてよく使われるものは$a,b,c,\ell,m,n,x,y,z,\alpha,\beta,\gamma,\delta,\epsilon$
			(大文字込)です.」
	}
	
	対象と変項を併せて{\bf 項}\index{こう@項}{\bf (term)}と呼び,
	対象を用いて作られていた式は対象を項に替えても式と呼ぶことにする.
	
	\begin{screen}
		\begin{dfn}[量化]
			$A$を言語の式とし(式は対象や述語記号,論理記号を組み合わせた記号列のはずであるが,いまは具体的な形は気にしないので$A$で表す)
			,記号$x$を$A$に現れる変項とするとき,`$\forall x A$'や`$\exists x A$'と書けば新しい記号列が得られる.
			このとき変項$x$は`$\forall x A$'で,或は`$\exists x A$'で{\bf 量化されている}\index{りょうか@量化}{\bf (quantified)}という.
			記号列に現れる変項のうち量化されていないものをその記号列の{\bf 自由変項}\index{じゆうへんこう@自由変項}{\bf (free variable)}と呼ぶ.
		\end{dfn}
	\end{screen}
	
	項と式の定義を形式的に書き直すと次のようになる.
	\begin{description}
		\item[項] 言語$\mathcal{L}$の対象は$\mathcal{L}$の項であり,
			変項も$\mathcal{L}$の項である.
			またそれらのみが$\mathcal{L}$の項である.
			
		\item[式] 
			\begin{itemize}
				\item $\bot$は$\mathcal{L}$の式である.
				
				\item $s,t$を言語$\mathcal{L}$の項とするとき,
					$s=t,\ s \in t$は$\mathcal{L}$の式である.
					
				\item $A,B$を$\mathcal{L}$の式とするとき,
					$A$で量化されている変項と$B$に現れる自由変項が同じ記号でなく,かつ
					$A$に現れる自由変項と$B$で量化されている変項が同じ記号でないとき,
					$A \wedge B,\ A \vee B,\ A\Longrightarrow B$は$\mathcal{L}$の式である.
				
				\item $A$を$\mathcal{L}$の式とするとき,
					$\rightharpoondown A$は$\mathcal{L}$の式である.
				
				\item $A$が$\mathcal{L}$の式であり,項$x$が
					$A$に現れる自由変項であるとき,$\forall x A,\ \exists x A$は$\mathcal{L}$の式である.
				
				\item 以上の操作を繰り返して得られる記号列のみが$\mathcal{L}$の式である.
					ただし,繰り返しの操作は無制限に行われるものではない.
					得られる記号列は左端から辿っていけば必ず右端が見つかるものとする
					\footnotemark.
			\end{itemize}
	\end{description}
	\footnotetext{式は有限長であると言えば済みそうであるが,有限とは何かを未だ規定していないので
		ナイーブな書き方になってしまうのは致し方ない.}
	
	\monologue{
		院生「式の定義では,始めに最も簡単な形の式($\bot$や$s=t$)を提示して,
			以降の段階で新しい式を作り出す手段(論理記号による式の接合)を指定しています.
			このような定義を{\bf 帰納的な定義}\index{きのうてきなていぎ@帰納的な定義}{\bf (inductive definition)}と呼びます.
			プログラミングで言うところのfor文の操作と同じですね.
			また量化記号が付くのは自由変項に限られますから,既に量化されている変項が再び量化されるということは起こり得ません.」
	}
	
	\begin{screen}
		\begin{dfn}[閉式・命題]
			自由変項を含まない式を{\bf 閉式}\index{へいしき@閉式}{\bf (closed formula)}や{\bf 命題}\index{めいだい@命題}{\bf (proposition)}と呼ぶ.
		\end{dfn}
	\end{screen}
	
	\monologue{
		院生「命題とは真偽が定まったものであるという釈然としない説明をよく目にしますが,
			どうやら命題や真偽の哲学的議論には決着が付いていないようで,
			本によっては``異論が続出するから深い言及を避ける''と書いてあるものもあります.
			しかし本稿では命題も真偽もその概念を明確に定義して,
			つまり概念を本稿で必要な分に制限するということになりますが,
			扱うことにいたします.」
	}
	たったいま言語$\mathcal{L}$を作ったばかりであるが,$\mathcal{L}$を
	次の言語$\mathcal{L}'$に拡張する.理由は,$\Set{x}{A(x)}$のような形の
	記法を導入したいためである.
	
	\begin{description}
		\item[対象]
			\begin{itemize}
				\item $\mathcal{L}$の対象は$\mathcal{L}'$の対象である.
				\item $\mathcal{L}$の対象領域$V$は$\mathcal{L}'$の対象である.
				\item 項$x$のみを自由変項とする$\mathcal{L}$の論理式$A$に対し
					$\Set{x}{A}$は$\mathcal{L}'$の対象である.
				\item 以上で得られる記号列のみが$\mathcal{L}'$の対象である.
			\end{itemize}
			
		\item[項] 
			言語$\mathcal{L}'$の対象は$\mathcal{L}'$の項であり,
			変項も$\mathcal{L}'$の項である.
			またそれらのみが$\mathcal{L}'$の項である.
			
		\item[述語記号] $=,\in$
		\item[論理記号] $\bot,\ \Longrightarrow,\ \wedge,\ \vee,\ \rightharpoondown$
	\end{description}
	
	述語記号も論理記号も$\mathcal{L}$のものと同じであるが,こうすることで
	$\mathcal{L}$の式が$\mathcal{L}'$においても式となる.
	%言語を拡張したことに応じて記号の適用範囲が広がったと見ればよい.
	
	\monologue{
		院生「今回は新造語を取り入れて言語を拡張しました.
			身近な例として最近ではネットスラングが国語辞書に載ったりしていますね.
			日常生活で新造語が出来ても文法まではなかなか変わらないように,
			述語記号や論理記号は拡張後も変えませんでした.むしろ変えてしまうと不便で不都合です.
			対象領域$V$が$\mathcal{L}'$の対象になっているのは,
			新造語が出現して初めて旧来の言語というカテゴリーが考察の俎上に上がったと解釈できるでしょうか.」
	}
	
	\begin{screen}
		\begin{dfn}[宇宙]
			言語$\mathcal{L}$の対象領域$V$を{\bf 宇宙}\index{うちゅう@宇宙}{\bf (Universe)}と呼ぶ.
		\end{dfn}
	\end{screen}
	
	\monologue{
		院生「宇宙という壮大な言葉が出てきてしまいました.
			定理\ref{}まで読めばわかるように集合論の世界は$V$の範囲内で語り尽くせてしまうのですから,
			現代数学にとって$V$は宇宙そのものなのですね.
			ところで,現実世界において人間が把握し得る最大の世界は宇宙空間でしょうが,
			数学の世界では宇宙の外側を見ることが出来るのです.宇宙の外側に在るものは真類と呼ばれます.
			実は宇宙そのものも真類の一つなのですが(宇宙が宇宙の外側に在るとは奇妙です),
			その話は後述にまかせましょう.」
	}
	
	数学の式を日本語に翻訳するとき,慣習上よく使われる訳し方があるので列挙する.
	いま,$a,b$を$\mathcal{L}'$の対象とし,$A,B$を$\mathcal{L}'$の式とするとき,
	\begin{itemize}
		\item 式$a = b$を``$a$は$b$に等しい''や``$a$と$b$は等しい''と翻訳する.
		\item 式$a \in b$を``$a$は$b$の要素である''や``$a$は$b$に属する''と翻訳する.
		\item 式$A \Longrightarrow B$を``$A$が成り立つならば$B$が成り立つ''と翻訳する.
		\item 式$\rightharpoondown A$を%``$A$でない''と翻訳する.
	\end{itemize}
	
	\begin{screen}
		\begin{dfn}[類・集合]
			言語$\mathcal{L}'$の対象を{\bf 類}\index{るい@類}{\bf (class)}と呼び,
			特に$V$の要素である類を{\bf 集合}\index{しゅうごう@集合}{\bf (set)}と呼ぶ.
			また$V$の要素でない類のことを{\bf 真類}\index{しんるい@真類}{\bf (proper class)}と呼ぶ.
		\end{dfn}
	\end{screen}
	
	\monologue{
		院生「言語$\mathcal{L}$の対象は$V$の要素なのでしょうか.$V$を対象領域と呼んでいましたから
			当然それが満たされていて欲しいような気もいたしますが,実はそれを確かめる術はありません.
			もともと$\mathcal{L}$において$V$と対象との関係を式で表せなかったためです.
			しかしそれで問題ありません.今後は類が$V$の要素であるか否かが焦点となるので,
			類が$\mathcal{L}$の対象であったか否かにはあまり興味が無いのです.
			$\mathcal{L}$とは,あくまで形式上の便宜のために導入したものであったということです.」
	}
	
	\monologue{
		院生「話を進めましょう.
			類は$V$の要素であれば集合と呼ばれ,$V$の要素でなければ真類と呼ばれます.
			では集合であり真類でもある類や,集合でも真類でもない類はあるのでしょうか?
			答えは``現段階では確定したことは何も言えない''です.
			質問を変えましょう.集合であり真類でもある類や集合でも真類でもない類の存在を禁止するにはどうしたら良いでしょうか?
			我々は,数学において中庸が無いということや矛盾が起きるということをどう表現しようかという問題に直面しているのです.
			この問題の解決への方便として{\bf 推論規則}\index{すいろんきそく@推論規則}
			{\bf (rule of inference)}と呼ばれるものを導入します.」
	}
	
	\begin{screen}
		\begin{axm}[排中律]
			$\mathcal{L}'$の任意の論理式$A$に対し,$A \vee \rightharpoondown A$が成り立つ.
		\end{axm}
	\end{screen}
	
	\begin{screen}
		\begin{thm}[集合でも真類でもない類は存在しない]
			\begin{align}
				\forall a\ \left(\ \rightharpoondown (\ a \in V \wedge a \notin V\ )\ \right)
			\end{align}
		\end{thm}
	\end{screen}
	
	\begin{screen}
		\begin{axm}[空集合の存在公理]
			いかなる集合も要素に持たない集合が存在する:
			\begin{align}
				\exists x \in V\ \forall y \in V\ (\ y \notin x\ ).
			\end{align}
		\end{axm}
	\end{screen}
	
	\begin{screen}
		\begin{thm}[空集合はただ一つ]
			空集合の存在公理を満たす集合はただ一つである:
			\begin{align}
				\forall x \in V\ \forall y \in V
				\ (\ (\ \forall z \in V\ (\ z \notin x\ ) \wedge \forall z \in V
				\ (\ z \notin y\ )\ )
				\Longrightarrow x=y\ ).
			\end{align}
			この何も持たない空の集合を{\bf 空集合}\index{くうしゅうごう@空集合}{\bf (empty set)}と呼び$\emptyset$という記号で表す.
		\end{thm}
	\end{screen}
	
	\monologue{
		院生「ようやく存在が約束された本物の集合が出てきましたね.
			あらゆる集合は空集合を元に作られていくのですから,空集合は集合の親とでもいえるのでしょうか.」
	}
	
	\begin{screen}
		\begin{axm}[類の公理]
			$\Set{x}{A(x)}$とは$A$という性質を持つ集合の全体である:
			\begin{align}
				\forall t\ (\ t \in \Set{x}{A(x)} \Longleftrightarrow t \in V \wedge A(t)\ ).
			\end{align}
		\end{axm}
	\end{screen}
	
	\begin{screen}
		\begin{axm}[外延性の公理]
			全く同じ要素からなる類は等しい:
			\begin{align}
				\forall a\ \forall b\ \left(\ \forall t\ (\ t \in a  \Longleftrightarrow t \in b\ )
				\Longrightarrow a=b\ \right).
			\end{align}
		\end{axm}
	\end{screen}
	
	\begin{screen}
		\begin{thm}\mbox{}
			\begin{description}
				\item[(1)] $\forall a\ (\ a=a\ )$
				\item[(2)] $\forall a \in V\ (\ a = \Set{x}{x \in a}\ )$
				\item[(3)] $V=\Set{x}{x=x}$
				\item[(4)] $\forall a\ (\ a \subset V\ )$
				\item[(5)] $\Set{x}{A} = \Set{y}{(y\, |\, x)A}$
				\item[(6)] $\Set{x}{A(x)} \cup \Set{x}{\rightharpoondown A(x)} = V$.
			\end{description}
		\end{thm}
	\end{screen}
	
	\begin{screen}
		\begin{axm}[相等性の公理]\mbox{}
			\begin{description}
				\item[(1)] $\forall a\ \forall b\ \forall c\ \left(\ a=b \Longrightarrow (\ c \in a \Longleftrightarrow c \in b\ )\ \right).$
				\item[(2)] $\forall a\ \forall b\ \forall c\ \left(\ a=b \Longrightarrow (\ c = a \Longleftrightarrow c = b\ )\ \right).$
				\item[(3)] $\forall a\ \forall b\ \forall c\ \left(\ a=b \Longrightarrow (\ a \in c \Longleftrightarrow b \in c\ )\ \right).$
			\end{description}
		\end{axm}
	\end{screen}
	
	\begin{screen}
		\begin{axm}[正則性公理]
			空でない類は自分自身と交わらない集合を要素に持つ:
			\begin{align}
				\forall a\ \left(\ a \neq \emptyset \Longrightarrow 
				\exists x \in a\ (\ x \cap a = \emptyset\ )\ \right)
			\end{align}
		\end{axm}
	\end{screen}
	
	\begin{screen}
		\begin{thm}[いかなる集合も自分自身を要素に持たない]
			次が成り立つ:
			\begin{align}
				\forall x \in V\ (\ x \notin x\ ).
			\end{align}
		\end{thm}
	\end{screen}
	
	\begin{prf}
		$a$を任意に選ばれた集合とするとき,類$\Set{x}{x=a}$は$a$を要素に持つから空ではなく,
		正則性公理より
		\begin{align}
			\exists t\ (\ t = a \wedge t \cap \Set{x}{x=a} = \emptyset\ )
		\end{align}
		が成立する.すなわち$a \cap \Set{x}{x=a} = \emptyset$となり,
		$a \in \Set{x}{x=a}$と併せて$a \notin a$が従う.
		\QED
	\end{prf}
	
	\begin{screen}
		\begin{thm}
			
		\end{thm}
	\end{screen}
	
	\begin{prf}\mbox{}
		\begin{description}
			\item[(1)] $a^{-1}$の任意の要素$t$に対し或る$V$の要素$x,y$が存在して
				\begin{align}
					(x,y) \in a \wedge t = (y,x)
				\end{align}
				を満たす.$((x,y),(y,x)) \in f$より$((x,y),t) \in f$が成り立つから
				$t \in f \ast a$となる.逆に$f \ast a$の任意の要素$t$に対して
				$a$の或る要素$x$が存在して
				\begin{align}
					x \in a \wedge (x,t) \in f
				\end{align}
				となる.$x$に対し$V$の或る要素$a,b$が存在して$x=(a,b)$となるので
				\begin{align}
					((a,b),t) \in f
				\end{align}
				となり,$V$の或る要素$c,d$が存在して
				\begin{align}
					((a,b),t) = ((c,d),(d,c))
				\end{align}
				となる.$(a,b) = (c,d)$より$a=c$かつ$b=d$となり,
				$t = (d,c)$かつ$(d,c)=(b,a)$より$t=(b,a)$,従って
				$t \in a^{-1}$が成り立つ.
		\end{description}
	\end{prf}
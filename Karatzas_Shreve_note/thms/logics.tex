	\begin{quote}
		太初に言あり、言は神と偕にあり、言は神なりき。\\
		この言は太初に神とともに在り、\\
		萬の物これに由りて成り、成りたる物に一つとして之によらで成りたるはなし。
	\end{quote}
	ヨハネによる福音書の冒頭である。森羅万象は言葉によって成り,言葉によって尽くされるという意味であるから,
	キリスト教においては言葉とはこの宇宙の悉くを超越しているのである.
	\begin{comment}
		実際に自然言語の発生が事物の観測なしに起こり得たかという問題は言語哲学上も決着がついていないらしいが,
		少なくとも
	\end{comment}
	本稿の世界もまた数学の言葉,言い換えれば論理のみによって創られるという点でキリスト教的であるといえるが,
	一方で論理のみによっては完結し得ないという点でキリスト教の世界観と決定的に違っている.
	
	\begin{screen}
		\begin{axm}[命題論理の公理]\mbox{}
			\begin{description}
				\item[(1)] 任意の命題$A,B$に対し,
					$A $
				\item[(1)] 証明可能な命題は真である.
				\item[(2)] 任意の命題$A,B$に対し,$A$も$A \Longrightarrow B$も真であるとき$B$は真である.
				\item[(3)] 任意の命題$A$に対し,$\rightharpoondown A$が真であるとき$A$は偽である.
			\end{description}
		\end{axm}
	\end{screen}
	
	\begin{screen}
		\begin{thm}
			$A$を任意の命題とするとき,
			\begin{align}
				\mbox{$\rightharpoondown A$が真である} \Longleftrightarrow \mbox{$A$が偽である}.
			\end{align}
		\end{thm}
	\end{screen}
	\begin{quote}
		初めに言(ことば)があった。言は神と共にあった。言は神であった。\\
		この言は、初めに神と共にあった。\\
		万物は言によって成った。成ったもので、言によらずに成ったものは何一つなかった。
	\end{quote}
	ヨハネによる福音書の冒頭である.数学の世界もまたことばが支配する.
	ただし数学の世界におけることばには二つの階層がある.一つは記号や記号の並べ方を規定する下位のことばであり,
	もう一つは何が定理であるかを規定する上位のことばである.
	前者は我々が神の視点で創る世界のことばであり,後者は神である我々の世界のことばであるが,
	こちらは論理と言い換える方が適当である.
	我々が創る世界は集合論と呼ばれ,数や関数など高校まで初等的に与えられてきたあらゆる概念がその世界の中で説明し直されることになる.
	
\subsection{言語}
	
	まず言語$\mathcal{L}$というものを設定する.これは我々が創る世界のことばである.以下は$\mathcal{L}$を構成する要素である:
	\begin{description}
		\item[使用文字] 自然言語から借用する文字は表にあるものに限る.
		\item[定数記号] $\emptyset$
		\item[述語記号] $=,\ \in$
		\item[論理記号] $\bot,\ \Longrightarrow,\ \wedge,\ \vee,\ \rightharpoondown$
		\item[量化記号] $\forall,\ \exists$
		\item[補助記号] $[\ ,\ ]\ ,\ (\ ,\ )\ ,\ \{\ ,\ \}\ ,\ <\ ,\ >\ ,\ |$
	\end{description}
	
	日常言語において,``あmt後右所sごぐふぉsdあじお''のように無作為に文字を並べただけでは意味不明な
	文字列が出来上がる.文字列は,何らかの規則に従って並ぶことで単語や文章として成立するのである.
	数学も同じで,一定の規則に従って並ぶ記号列のみを数学における文章として扱う.
	数学語において,名詞にあたるものは{\bf 対象}\index{たいしょう@対象}{\bf (individual)}と呼ばれる.
	述語とは対象同士を結ぶものであり,最小単位の文章を形作る.例えば,対象$s,t$に対し
	\begin{align}
		s \in t
	\end{align}
	は数学の文章となり,日本語には``$s$は$t$の要素である''と翻訳される.
	数学における文章を{\bf 式}\index{しき@式}
	{\bf (formula)}或は{\bf 論理式}\index{ろんりしき@論理式}と呼ぶ.
	論理記号とは式同士を繋ぐ役割を持つ.
	
	\monologue{
		院生「定数$\emptyset$は$\mathcal{L}$の対象の一つです.
			他の対象がどういうものであるかは後で判明しますが,
			今のところはその正体は伏せておいて,とりあえず対象とは何かしらの予め定められた記号列であると認めて話を進めます.
			また文字は対象ではないということも認めます.」
	}
	
	対象および文字を{\bf 項}\index{こう@項}{\bf (term)}と呼び,
	対象を用いて作られていた式は対象を項に替えても式と呼ぶことにする.
	
	$A$を言語の式とし(式は対象や述語記号,論理記号を組み合わせた記号列のはずであるが,いまは具体的な形は気にしないので$A$で表す)
	,$A$の中に文字$x$が現れるとき,`$\forall x A$'や`$\exists x A$'と書けば新しい記号列が得られる.
	このとき文字$x$は`$\forall x A$'で,或は`$\exists x A$'で{\bf 量化されている}\index{りょうか@量化}{\bf (quantified)}という.
	
	項と式の構成法を形式的に書き直すと次のようになる.
	\begin{description}
		\item[項] 言語$\mathcal{L}$の対象は$\mathcal{L}$の項であり,
			文字も$\mathcal{L}$の項である.
			またそれらのみが$\mathcal{L}$の項である.
			
		\item[式] 
			\begin{itemize}
				\item $\bot$は$\mathcal{L}$の式である.
				
				\item $s,t$を項とするとき,$s=t,\ s \in t$は式である.
					
				\item $A,B$を式とするとき,
					$A$では量化されていないが$B$で量化されているといった文字が無いときに限り,
					$(A) \wedge (B),\ (A) \vee (B),\ (A)\Longrightarrow (B)$は式である.
				
				\item $A$を式とするとき,$\rightharpoondown (A)$は式である.
				
				\item $A$を式とするとき,文字$x$が$A$に現れ,かつ$x$が$A$で量化されていないときに限り
					$\forall x (A),\ \exists x (A)$は式である.
				
				\item 以上の操作を繰り返して得られる記号列のみが式である.
			\end{itemize}
	\end{description}
	
	\monologue{
		院生「`$A$では量化されていないが$B$で量化されているといった文字が無いときに限り'という
			制限は何のためにあるのでしょうか.例えばこの制限を外すと
			\begin{align}
				\forall x ((x \in x) \vee (\forall y (\exists x ( y = x ))))
			\end{align}
			は式となりますが,同じ式で文字$x$は二回量化されています.
			同じ文字が複数回量化されてしまうと式を解釈するときに厄介なので,
			そのような状況を排除するために制限を課しているのですね.
			では,`以上の操作を繰り返して得られる記号列のみが式である'はどういう意味でしょうか.
			例えば,最後の制限を外してしまうと
			\begin{align}
				\exists (\rightharpoondown (\exists x(\forall y (x = y))))
			\end{align}
			という記号列が式であるか式でないかは判別できませんが,
			最後の規制によりこれは式ではないと判断できます.具体的な問題を考える際は上述のような式は扱わないので
			形式上の規制と思われますが,体系の完全性など考察する際には必要な規制でしょう.」
	}
	
	\monologue{
		院生「式の定義では,始めに最も簡単な形の式($\bot$や$s=t$)を提示して,
			以降の段階で新しい式を作り出す手段(論理記号による式の接合)を指定しています.
			このような定義を{\bf 帰納的な定義}\index{きのうてきなていぎ@帰納的な定義}{\bf (inductive definition)}と呼びます.
			プログラミングで言うところのfor文の操作と同じですね.
			また既に量化されている文字が再び量化されるということは起こり得ません.」
	}
	
	$A$を式とする.
	$A$に$a$という文字が現れるとき,$A$に現れる全ての$a$を$x$に置き換えた式を
	\begin{align}
		(x \mid a)\, A
	\end{align}
	で表す.特に$A$に現れる文字で量化されていないものが$a$のみであるとき,
	$(x \mid a)\, A$を
	\begin{align}
		A(x)
	\end{align}
	で表す.このとき式$A$自体は$(a \mid a)\, A$とも$A(a)$とも書ける.
	
	いま文字から成る式を作ったが,例えば$x$のみが量化されていない式$A$に対して
	\begin{align}
		\Set{x}{A(x)}
	\end{align}
	という記法を導入し,これを対象として
	\begin{align}
		s \in \Set{x}{A(x)},\quad t = \Set{x}{A(x)}
	\end{align}
	などと式に組み込んで扱いたい.そこで$\mathcal{L}$を次の言語$\mathcal{L}'$に拡張する.
	
	\begin{description}
		\item[対象]
			\begin{itemize}
				\item 式$A$において文字$x$が現れ,かつ$x$のみが$A$で量化されていないとき,
					\begin{align}
						\Set{x}{A(x)}
					\end{align}
					は$\mathcal{L}'$の対象である.
					
				\item $\mathcal{L}$の対象は$\mathcal{L}'$の対象である.
				
				\item 以上のみが$\mathcal{L}'$の対象である.
			\end{itemize}
			
		\item[項] 言語$\mathcal{L}'$の対象は$\mathcal{L}'$の項であり,
			文字も$\mathcal{L}'$の項である.
			またそれらのみが$\mathcal{L}'$の項である.
	
		\item[式] 
			\begin{itemize}
				\item $\bot$は$\mathcal{L}'$の式である.
				
				\item $s,t$を項とするとき,$s=t,\ s \in t$は式である.
					
				\item $A,B$を式とするとき,
					$A$では量化されていないが$B$で量化されているといった文字が無いときに限り,
					$(A) \wedge (B),\ (A) \vee (B),\ (A)\Longrightarrow (B)$は式である.
				
				\item $A$を式とするとき,$\rightharpoondown (A)$は式である.
				
				\item $A$を式とするとき,文字$x$が$A$に現れ,かつ$x$が$A$で量化されていないときに限り
					$\forall x (A),\ \exists x (A)$は式である.
				
				\item 以上の操作を繰り返して得られる記号列のみが式である.
			\end{itemize}
	\end{description}
	
	\begin{screen}
		\begin{dfn}[閉式]
			量化されていない文字を含まない式を{\bf 閉式}\index{へいしき@閉式}{\bf (closed formula)}と呼ぶ.
		\end{dfn}
	\end{screen}
	
	\begin{screen}
		\begin{dfn}[宇宙]
			$\Univ \coloneqq \Set{x}{x=x}$で定める$\Univ$を{\bf 宇宙}\index{うちゅう@宇宙}{\bf (Universe)}と呼ぶ.
		\end{dfn}
	\end{screen}
	
	\monologue{
		院生「記号$\coloneqq$は定義記号と呼ばれ,右辺の類に左辺の記号列で名前を付けるという意味で使われます.
			言語$\mathcal{L}'$の世界においては,式に出てくる$\Set{x}{x=x}$の部分を$\Univ$で置き換えられるようになります.
			$\Univ$は項として使われる文字ではありませんでしたが,$\coloneqq$の推薦で$\Set{x}{x=x}$の代理人として市民権を得たのですね.
			さて宇宙という壮大な言葉が出てきましたが,後述する通り$\Univ$は集合の全体のことですから,
			あらゆるものが集合で説明される現代数学にとって$\Univ$はまさしく宇宙なのですね.
			また定理\ref{thm:Universe_and_ordinal_numbers}で$\Univ$の実態が明らかになるでしょう.
			我々はこの定理で集合とは何者かという問いへの完全な答えを得ることになります.
			ところで,現実世界において人間が把握し得る最大の世界は宇宙空間でしょうが,
			数学の世界では宇宙の外側を見ることが出来るのです.そこは真類と呼ばれるものの世界です.
			実は宇宙そのものも真類の一つなのですが(宇宙が宇宙の外側に在るとは奇妙です),
			その話も後述にまかせましょう.」
	}
	
	数学の式を日本語に翻訳するとき,慣習上よく使われる訳し方があるので列挙する.
	\begin{itemize}
		\item 式$a = b$を``$a$は$b$に等しい''や``$a$と$b$は等しい''と翻訳する.
		\item 式$a \in b$を``$a$は$b$の要素である''や``$a$は$b$に属する''と翻訳する.
		\item 式$(A) \Longrightarrow (B)$を``$A$が成り立つならば$B$が成り立つ''と翻訳する.
		\item 式$\rightharpoondown (A)$を%``$A$でない''と翻訳する.
	\end{itemize}
	
	\begin{screen}
		\begin{dfn}[類・集合]
			$\mathcal{L}'$の対象のことを{\bf 類}\index{るい@類}{\bf (class)}と呼び直し,
			特に$\Univ$の要素である類を{\bf 集合}\index{しゅうごう@集合}{\bf (set)}と呼ぶ.
			また$\Univ$の要素でない類のことを{\bf 真類}\index{しんるい@真類}{\bf (proper class)}と呼ぶ.
		\end{dfn}
	\end{screen}

	\monologue{
		院生「類は$V$の要素であれば集合と呼ばれ,$V$の要素でなければ真類と呼ばれます.
			では集合であり真類でもある類や,集合でも真類でもない類はあるのでしょうか?
			答えは``現段階では確定したことは何も言えない''です.
			質問を変えましょう.集合であり真類でもある類や集合でも真類でもない類の存在を禁止するにはどうしたら良いでしょうか?
			我々は,数学において中庸が無いということや矛盾が起きるということをどう表現しようかという問題に直面しているのです.
			この問題の解決への方便として{\bf 推論規則}\index{すいろんきそく@推論規則}
			{\bf (rule of inference)}と呼ばれるものを導入します.」
	}
	
	\begin{screen}
		\begin{axm}[排中律]
			任意の論理式$A$に対し,$A \vee \rightharpoondown A$が成り立つ.
		\end{axm}
	\end{screen}
	
	いま$a,b$を類とするとき,
	\begin{align}
		a \notin b \overset{\mathrm{def}}{\Longleftrightarrow}\ \rightharpoondown a \in b
	\end{align}
	で$a \notin b$を定める.同様に
	\begin{align}
		a \neq b \overset{\mathrm{def}}{\Longleftrightarrow}\ \rightharpoondown a = b
	\end{align}
	で$a \neq b$を定める.
	
	\monologue{
		院生「定義記号$\coloneqq$と同様に,`$A \overset{\mathrm{def}}{\Longleftrightarrow} B$'とは
			式$B$を記号列$A$で置き換えて良いという意味で使われます.また,式中に記号列$A$が出てくるときは,
			暗黙裡にその$A$を$B$に戻して式の内容を解釈します.
			$\coloneqq$も$\overset{\mathrm{def}}{\Longleftrightarrow}$も略記することと同じですね.」
	}
	
	\begin{screen}
		\begin{thm}[集合でも真類でもない類は存在しない]
			$a$を類とするとき次が成り立つ:
			\begin{align}
				\rightharpoondown (\ a \in \Univ \wedge a \notin \Univ\ ).
			\end{align}
		\end{thm}
	\end{screen}
	
	\begin{screen}
		\begin{axm}[外延性の公理]
			$a,b$を類とするとき,次が成り立つ:
			\begin{align}
				\forall t\ (\ t \in a  \Longleftrightarrow t \in b\ )
				\Longrightarrow a=b.
			\end{align}
		\end{axm}
	\end{screen}
	
	\begin{screen}
		\begin{thm}[任意の類は自分自身と等しい]\label{thm:any_class_equals_to_itself}
			$a$を類とするとき次が成り立つ:
			\begin{align}
				a = a.
			\end{align}
		\end{thm}
	\end{screen}
	
	\begin{prf}
		$\mathcal{L}$の任意の対象$\tau$に対して
		\begin{align}
			\tau \in a \Longleftrightarrow \tau \in a
		\end{align}
		となるから
		\begin{align}
			\forall t\ (\ t \in a  \Longleftrightarrow t \in b\ )
		\end{align}
		が成り立つ.従って外延性の公理より$a = a$が出る.
		\QED
	\end{prf}
	
	\begin{screen}
		\begin{axm}[類の公理]\mbox{}
			\begin{description}
				\item[(i)] $a,b$を類とするとき,次が成り立つ:
					\begin{align}
						a \in b \wedge a \neq b \Longrightarrow \exists x\ (\ a = x\ ).
					\end{align}
					つまり類が別の類の要素であるならば$\mathcal{L}$の或る対象に等しい.
				
				\item[(ii)] 
					\begin{align}
						\forall t\ (\ t \in \Set{x}{A(x)} \Longleftrightarrow A(t)\ ).
					\end{align}
			\end{description}
		\end{axm}
	\end{screen}
	
	\begin{screen}
		\begin{thm}[$\mathcal{L}$の対象も$\Set{x}{A(x)}$の形で表せる]
			次が成り立つ:
			\begin{align}
				\forall t\ (\ t = \Set{x}{x \in t}\ ).
			\end{align}
		\end{thm}
	\end{screen}
	
	\begin{prf}
		$\tau$を$\mathcal{L}$の任意の対象とすると類の公理より
		\begin{align}
			\forall s\ (\ s \in \tau \Longleftrightarrow s \in \Set{x}{x \in \tau}\ )
		\end{align}
		が成り立つが,このとき外延性の公理より
		\begin{align}
			\tau = \Set{x}{x \in \tau}
		\end{align}
		が従う.$\tau$の任意性より
		\begin{align}
			\forall t\ (\ t = \Set{x}{x \in t}\ )
		\end{align}
		を得る.
		\QED
	\end{prf}
	
	\begin{screen}
		\begin{axm}[相等性の公理]
			$a,b,c$を類とするとき以下は全て公理である:
			\begin{description}
				\item[(i)] $a=b \Longrightarrow (\ c \in a \Longleftrightarrow c \in b\ ).$
				\item[(ii)] $a=b \Longrightarrow (\ c = a \Longleftrightarrow c = b\ ).$
				\item[(iii)] $a=b \Longrightarrow (\ a \in c \Longleftrightarrow b \in c\ ).$
			\end{description}
		\end{axm}
	\end{screen}
	
	\begin{screen}
		\begin{thm}[$\exists x\ (\ a = x\ )$は$a$が集合であるということを意味する]
			$a$を類とするとき次が成り立つ:
			\begin{align}
				\exists x\ (\ a = x\ ) \Longleftrightarrow a \in \Univ.
			\end{align}
		\end{thm}
	\end{screen}
	
	\begin{prf}
		$a$を類に対して,まず類の公理より
		\begin{align}
			a \in \Univ \Longrightarrow \exists x\ (\ a = x\ )
		\end{align}
		が得られる.逆に
		\begin{align}
			\exists x\ (\ a = x\ )
		\end{align}
		が成り立っていると仮定する.$\mathcal{L}$の対象$\varepsilon x (a = x)$を$\tau$と書けば,
		定理\ref{thm:any_class_equals_to_itself}より$\tau = \tau$となるので
		\begin{align}
			\tau \in \Univ
		\end{align}
		が成り立つ.相等性の公理より
		\begin{align}
			a \in \Univ
		\end{align}
		が従うから$\exists x\ (\ a = x\ ) \Longrightarrow a \in \Univ$も得られる.
		\QED
	\end{prf}
	
	\monologue{
		院生「つまり,$\mathcal{L}$の対象か,或いは$\mathcal{L}$の対象と等しい類のみが集合であるということですね.」
	}
	
	\begin{screen}
		\begin{axm}[空集合の公理]
			次は公理である:
			\begin{align}
				\forall x\ (\ x \notin \emptyset\ ).
			\end{align}
			つまり$\emptyset$は$\mathcal{L}$のいかなる対象も要素に持たない.
			$\emptyset$を{\bf 空集合}\index{くうしゅうごう@空集合}{\bf (empty set)}と呼ぶ.
		\end{axm}
	\end{screen}
	
	\monologue{
		院生「空集合とは集合の系譜の起点となります.聖書物語でいうところのアダムです.」
	}
	
	\begin{screen}
		\begin{thm}[$\mathcal{L}$のいかなる対象も要素に持たない類は空集合に等しい]
			$a$を類とするとき次が成り立つ:
			\begin{align}
				\forall x\ (\ x \notin a\ ) \Longrightarrow a = \emptyset.
			\end{align}
		\end{thm}
	\end{screen}
	
	\begin{screen}
		\begin{thm}[空集合はいかなる類も要素に持たない]
			$a$を類とするとき次が成り立つ:
			\begin{align}
				a \notin \emptyset.
			\end{align}
		\end{thm}
	\end{screen}
	
	\begin{prf}
		$a,b$を類とするとき
		\begin{align}
			a \in b \Longrightarrow \exists x\ (\ x \in b\ )
		\end{align}
		が成り立つから,対偶を取れば
		\begin{align}
			\forall x\ (\ x \notin b\ ) \Longrightarrow a \notin b
		\end{align}
		が成り立つ.$b$を$\emptyset$に置き換えれば$\forall x\ (\ x \notin \emptyset\ )$は真であるから
		$a \notin \emptyset$が従う.
		\QED
	\end{prf}
	
	\begin{screen}
		\begin{dfn}[部分類]
			$a,b$を類とするとき,
			\begin{align}
				a \subset b \overset{\mathrm{def}}{\Longleftrightarrow}
				\forall x\ (\ x \in a \Longrightarrow x \in b\ )
			\end{align}
			で記号列$a \subset b$を定める.また類$a,b$に対して$a \subset b$が成り立っているとき
			$a$を$b$の{\bf 部分類}\index{ぶぶんるい@部分類}{\bf (subclass)}と呼び,
			特に$a$が集合である場合は$a$を$b$の{\bf 部分集合}\index{ぶぶんしゅうごう@部分集合}{\bf (subset)}と呼ぶ.
			また類$a,b$に対して
			\begin{align}
				a \subsetneq b \overset{\mathrm{def}}{\Longleftrightarrow}
				a \subset b \wedge a \neq b
			\end{align}
			と定める.
		\end{dfn}
	\end{screen}
	
	\begin{screen}
		\begin{thm}[空集合は全ての類に含まれる]
			$a$を類とするとき次が成り立つ:
			\begin{align}
				\emptyset \subset a.
			\end{align}
		\end{thm}
	\end{screen}
	
	\begin{prf}
		$a$を類とする.$\tau$を$\mathcal{L}$の任意の対象とすれば,$\tau \notin \emptyset$が成り立つから
		\begin{align}
			\tau \notin \emptyset \vee \tau \in a
		\end{align}
		が成り立つ.これは$\tau \in \emptyset \Longrightarrow \tau \in a$が成り立つことと同値であるから,
		$\tau$の任意性より$\emptyset \subset a$が得られる.
		\QED
	\end{prf}
	
	\monologue{
		院生「この結果は空虚な真の一例ですね.」
	}
	
	\begin{screen}
		\begin{thm}[類はその部分類の全ての要素を含む]\label{thm:subclass_contains_all_elements}
			$a,b$を類として,$a$を$b$の部分類とする.このとき,任意の類$c$に対して次が成り立つ:
			\begin{align}
				c \in a \Longrightarrow c \in b.
			\end{align}
		\end{thm}
	\end{screen}
	
	\begin{prf}
		$c \in a$が成り立っていると仮定する.このとき$c$に対して
		\begin{align}
			c = \tau
		\end{align}
		を満たす$\mathcal{L}$の対象$\tau$が存在する.
		相等性の公理より$\tau \in a$となり,部分類の定義より$\tau \in b$となり,
		再び相等性の公理より$c \in b$が従う.ゆえに
		\begin{align}
			c \in a \Longrightarrow c \in b
		\end{align}
		が得られる.
		\QED
	\end{prf}
	
	\begin{screen}
		\begin{thm}[$\Univ$は最大の類である]
			$a$を類とするとき次が成り立つ:
			\begin{align}
				a \subset \Univ.
			\end{align}
		\end{thm}
	\end{screen}
	
	\begin{prf}
		$\Univ$は$\mathcal{L}$の対象の全てを要素として持つから
		\begin{align}
			\forall x\ (\ x \in a \Longrightarrow x \in \Univ\ )
		\end{align}
		が成り立つ.
		\QED
	\end{prf}
	
	\begin{screen}
		\begin{thm}[互いに互いの部分類となる類同士は等しい]\label{thm:mutually_sub_classes_are_equivalent}
			$a,b$を類とするとき次が成り立つ:
			\begin{align}
				a \subset b \wedge b \subset a \Longleftrightarrow a = b.
			\end{align}
		\end{thm}
	\end{screen}
	
	\begin{prf}
		$a \subset b \wedge b \subset a$が成り立っていると仮定する.
		このとき$\mathcal{L}$の任意の対象$\tau$に対して
		\begin{align}
			\tau \in a \Longleftrightarrow \tau \in b
		\end{align}
		が成り立つから$a = b$が従う.逆に$a = b$が成り立っていると仮定すれば,
		$\mathcal{L}$の任意の対象$\tau$に対して
		\begin{align}
			\tau \in a \Longleftrightarrow \tau \in b
		\end{align}
		が成り立つので$a \subset b$と$b \subset a$が共に従う.
		\QED
	\end{prf}
	
	\monologue{
		院生「定理\ref{thm:subclass_contains_all_elements}と定理\ref{thm:mutually_sub_classes_are_equivalent}より,
			類$a,b$が$a = b$を満たすならば,$a$と$b$は要素に持つ$\mathcal{L}$の対象のみならず,
			要素に持つ類までも一致するのですね.これは`類は別の類の要素であるならば$\mathcal{L}$の或る要素に等しい'
			という決まりの帰結です.」
	}
	
	\begin{screen}
		\begin{dfn}[合併]
			$a$を集合とするとき,$\mathcal{L}$の或る対象$\tau$が存在して$a = \tau$を満たすが,このとき
			\begin{align}
				\bigcup a \coloneqq \Set{x}{\exists \sigma \in \tau\ (\ x \in \sigma\ )}
			\end{align}
			で$\bigcup a$を定め,これを$a$の{\bf 合併}\index{がっぺい@合併}{\bf (union)}と呼ぶ.
		\end{dfn}
	\end{screen}
	
	\monologue{
		院生「$a$が集合であるから,$a$に一致する$\mathcal{L}$の対象を用いて$\bigcup a$を定義できるのですね.
			つまり,記号列$\exists \sigma \in \tau\ (\ x \in \sigma\ )$が$x$の式として妥当となるのです.」
	}
	
	\begin{screen}
		\begin{axm}[合併集合の公理]
			$a$を集合とするとき次が成り立つ:
			\begin{align}
				\bigcup a \in \Univ.
			\end{align}
		\end{axm}
	\end{screen}
	
	\begin{screen}
		\begin{thm}\mbox{}
			\begin{description}
				\item[(2)] $\forall a\ (\ a = \Set{x}{x \in a}\ )$
				\item[(4)] $a \subset V$
				\item[(5)] $\Set{x}{A} = \Set{y}{(y\, |\, x)A}$
				\item[(6)] $\Set{x}{A(x)} \cup \Set{x}{\rightharpoondown A(x)} = V$.
			\end{description}
		\end{thm}
	\end{screen}
	
	\begin{screen}
		\begin{thm}
			
		\end{thm}
	\end{screen}
	
	\begin{prf}\mbox{}
		\begin{description}
			\item[(1)] $a^{-1}$の任意の要素$t$に対し或る$V$の要素$x,y$が存在して
				\begin{align}
					(x,y) \in a \wedge t = (y,x)
				\end{align}
				を満たす.$((x,y),(y,x)) \in f$より$((x,y),t) \in f$が成り立つから
				$t \in f \ast a$となる.逆に$f \ast a$の任意の要素$t$に対して
				$a$の或る要素$x$が存在して
				\begin{align}
					x \in a \wedge (x,t) \in f
				\end{align}
				となる.$x$に対し$V$の或る要素$a,b$が存在して$x=(a,b)$となるので
				\begin{align}
					((a,b),t) \in f
				\end{align}
				となり,$V$の或る要素$c,d$が存在して
				\begin{align}
					((a,b),t) = ((c,d),(d,c))
				\end{align}
				となる.$(a,b) = (c,d)$より$a=c$かつ$b=d$となり,
				$t = (d,c)$かつ$(d,c)=(b,a)$より$t=(b,a)$,従って
				$t \in a^{-1}$が成り立つ.
		\end{description}
	\end{prf}
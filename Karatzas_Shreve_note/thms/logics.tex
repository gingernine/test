	\begin{comment}
	\begin{quote}
		初めに言(ことば)があった。言は神と共にあった。言は神であった。\\
		この言は、初めに神と共にあった。\\
		万物は言によって成った。成ったもので、言によらずに成ったものは何一つなかった。
	\end{quote}
	ヨハネによる福音書の冒頭である.数学の世界もまたことばが支配する.
	ただし数学の世界におけることばには二つの階層がある.一つは記号や記号の並べ方を規定する下位のことばであり,
	もう一つは何が定理であるかを規定する上位のことばである.
	前者は我々が神の視点で創る世界のことばであり,後者は神である我々の世界のことばであるが,
	後者は論理と言い換える方が適当である.
	我々が創る世界は集合論と呼ばれ,数や関数など高校まで初等的に与えられてきたあらゆる概念がその世界の中で説明し直されることになる.
	\end{comment}
	
\subsection{言語}
	
	まず言語$\mathcal{L}$というものを設定する.これは我々が創る世界のことばである.以下は$\mathcal{L}$を構成する要素である:
	\begin{description}
		\item[使用文字] 使う文字は表(P.\pageref{tab:alphabet})にあるものに限る.
		\item[定数記号] $\emptyset$
		\item[述語記号] $=,\ \in$
		\item[論理記号] $\bot,\ \Longrightarrow,\ \wedge,\ \vee,\ \rightharpoondown$
		\item[量化記号] $\forall,\ \exists$
		\item[補助記号] $[\ ,\ ]\ ,\ (\ ,\ )\ ,\ \{\ ,\ \}\ ,\ <\ ,\ >\ ,\ |,\coloneqq,\ \overset{\mathrm{def}}{\Longleftrightarrow}$
	\end{description}
	
	日常言語において,``あmt後右所sごぐふぉsd''のように無作為に文字を並べただけでは意味不明な
	文字列が出来上がる.文字列は,何らかの規則に従って並ぶことで単語や文章として成立するのである.
	数学も同じで,一定の規則に従って並ぶ記号列のみを数学における文章として扱う.
	数学語において,名詞にあたるものは{\bf 対象}\index{たいしょう@対象}{\bf (individual)}と呼ばれる.
	述語とは対象同士を結ぶものであり,最小単位の文章を形成する.例えば$s,t$を対象とするとき
	\begin{align}
		s \in t
	\end{align}
	は数学の文章となり,日本語には``$s$は$t$の要素である''と翻訳される.
	数学における文章を{\bf 式}\index{しき@式}
	{\bf (formula)}或は{\bf 論理式}\index{ろんりしき@論理式}と呼ぶ.
	論理記号とは式同士を繋ぐ役割を持つ.
	
	\monologue{
		院生「定数$\emptyset$は$\mathcal{L}$の対象の一つです.
			他の対象がどういうものであるかは後で判明しますが,
			今のところはその正体は伏せておいて,とりあえず対象は予め存在しているものとして話を進めます.
			また{\bf 文字は対象ではない}ということも認めます.説明中は``$s$を対象とする''のように書くことが多いですが,
			これは一時的に`$s$'を対象の一つに代用しているだけで,文字`$s$'が対象であると言っているのではありません.
			このような代用記号のことを{\bf 超記号}と呼びます.対象のみならず式にも超記号を宣言することが多いです.」
	}
	
	対象および文字を{\bf 項}\index{こう@項}{\bf (term)}と呼び,
	対象を用いて作られていた式は対象を項に替えても式と呼ぶことにする.
	
	$A$を式とし(上述の通り$A$とは超記号である)
	,$A$の中に文字$x$が現れるとき,`$\forall x A$'や`$\exists x A$'と書けば新しい記号列が得られる.
	このとき文字$x$は`$\forall x A$'で,或は`$\exists x A$'で{\bf 量化されている}\index{りょうか@量化}{\bf (quantified)}という.
	
	項と式の構成法を形式的に書き直すと次のようになる.
	\begin{description}
		\item[項] 言語$\mathcal{L}$の対象は$\mathcal{L}$の項であり,
			文字も$\mathcal{L}$の項である.
			またそれらのみが$\mathcal{L}$の項である.
			
		\item[式] 
			\begin{itemize}
				\item `$\bot$'は$\mathcal{L}$の式である.
				
				\item $s,t$を項とするとき,`$s=t$'と`$s \in t$'はどちらも式である.
					
				\item $A,B$を式とするとき,
					$A$では量化されていないが$B$で量化されているといった文字が無いときに限り,
					`$(A) \wedge (B)$',`$(A) \vee (B)$',`$(A) \Longrightarrow (B)$'はいずれも式である.
				
				\item $A$を式とするとき,`$\rightharpoondown (A)$'は式である.
				
				\item $A$を式とし,$x$を$A$に現れる文字とするとき,$x$が$A$で量化されていないときに限り
					`$\forall x (A)$'と`$\exists x (A)$'はどちらも式である.
				
				\item 以上の操作を繰り返して得られる記号列のみが式である.
			\end{itemize}
	\end{description}
	
	\monologue{
		院生「`$A$では量化されていないが$B$で量化されているといった文字が無いときに限り'という
			制限は何のためにあるのでしょうか.例えばこの制限を外すと
			\begin{align}
				\forall x ((x \in x) \vee (\forall y (\exists x ( y = x ))))
			\end{align}
			は式となりますが,同じ式で文字$x$は二回量化されています.
			これでは式を解釈するときに厄介なので,このような状況を排除するために制約を設けているのですね.
			では,`以上の操作を繰り返して得られる記号列のみが式である'はどういう意味でしょうか.
			例えば,最後の制限を外してしまうと
			\begin{align}
				\exists (\rightharpoondown (\exists x(\forall y (x = y))))
			\end{align}
			という記号列が式であるか式でないかは判別できませんが,
			最後の規制によりこれは式ではないと判断できます.具体的な問題を考える際は上の例のような式は扱わないので,
			本稿においては殆どご利益の無い形式上の規制となりますが,体系の完全性など考察する際には必要な規制でしょう.」
	}
	
	\monologue{
		院生「式の定義では,始めに最も簡単な形の式(`$\bot$'や`$s=t$')を提示して,
			以降の段階で新しい式を作り出す手段(論理記号による式の接合)を指定しています.
			このような定義を{\bf 帰納的な定義}\index{きのうてきなていぎ@帰納的な定義}{\bf (inductive definition)}と呼びます.
			プログラミングで言うところのfor文の操作と同じですね.」
	}
	
	$A$を式とする.$A$が`$t = a$'という式である場合など,$A$に文字$a$が現れるとき,$A$に現れる全ての文字$a$を文字$x$に置き換えた式を
	\begin{align}
		(x \mid a)\, A
	\end{align}
	で表す.特に$A$に現れる文字で量化されていないものが$a$のみであるとき,
	$(x \mid a)\, A$を
	\begin{align}
		A(x)
	\end{align}
	で表す.このとき式$A$自体は$(a \mid a)\, A$とも$A(a)$とも書ける.
	
	いま言語$\mathcal{L}$を設定したばかりであるが,例えば$x$のみが量化されていない式$A$に対して
	\begin{align}
		\Set{x}{A(x)}
	\end{align}
	という記法を導入し,これを対象として
	\begin{align}
		s \in \Set{x}{A(x)},\quad t = \Set{x}{A(x)}
	\end{align}
	のように式に組み込んで扱いたい.そこで$\mathcal{L}$を言語$\mathcal{L}'$に拡張する.
	$\mathcal{L}'$の使用文字,定数記号,述語記号,論理記号,量化記号,補助記号は$\mathcal{L}$のものをそのまま継承し,
	対象・項・式は次のように定める:
	
	\begin{description}
		\item[対象]
			\begin{itemize}
				\item $A$を$\mathcal{L}$の式とし,$x$を$A$に現れる文字とし,$x$のみが$A$で量化されていないとするとき,
					$\Set{x}{A(x)}$は$\mathcal{L}'$の対象である.
					
				\item $\mathcal{L}$の対象は$\mathcal{L}'$の対象である.
				
				\item 以上のみが$\mathcal{L}'$の対象である.
			\end{itemize}
			
		\item[項] 言語$\mathcal{L}'$の対象は$\mathcal{L}'$の項であり,
			文字も$\mathcal{L}'$の項である.
			またそれらのみが$\mathcal{L}'$の項である.
	
		\item[式] 
			\begin{itemize}
				\item $\bot$は$\mathcal{L}'$の式である.
				
				\item $s,t$を$\mathcal{L}'$の項とするとき,`$s=t$'と`$s \in t$'はどちらも$\mathcal{L}'$の式である.
					
				\item $A,B$を$\mathcal{L}'$の式とするとき,
					$A$では量化されていないが$B$で量化されているといった文字が無いときに限り,
					`$(A) \wedge (B)$',`$(A) \vee (B)$',`$(A) \Longrightarrow (B)$'はいずれも$\mathcal{L}'$の式である.
				
				\item $A$を$\mathcal{L}'$の式とするとき,`$\rightharpoondown (A)$'は$\mathcal{L}'$の式である.
				
				\item $A$を$\mathcal{L}'$の式とし,$x$を$A$に現れる文字とするとき,$x$が$A$で量化されていないときに限り
					`$\forall x (A(x))$'と`$\exists x (A(x))$'はどちらも$\mathcal{L}'$の式である.
				
				\item 以上の操作を繰り返して得られる記号列のみが$\mathcal{L}'$の式である.
			\end{itemize}
	\end{description}
	
	\monologue{
		院生「$A$を$\mathcal{L}$の式とするとき,$x$が$A$に現れ,かつ$x$のみが$A$で量化されていないならば
			\begin{align}
				\Set{x}{A(x)}
			\end{align}
			は$\mathcal{L}'$の対象であると決めましたが,いま$A(x)$の中に文字$y$が現れないと仮定すれば,
			式$A(y)$には文字$y$が現れ,かつ$y$のみが量化されていないことになりますから
			\begin{align}
				\Set{y}{A(y)}
			\end{align}
			もまた$\mathcal{L}'$の対象となります.このとき`$\Set{x}{A(x)}$'と`$\Set{y}{A(y)}$'は$x$と$y$の違いを除いて同じ記号列になりますから,
			これらを同等な対象として扱いたいものです.同等とは等号で結ばれることですが,
			このことは後述する``類の公理''と``外延性の公理''により保証されます.」
	}
	
	$\mathcal{L}'$の式のうち,量化されていない文字を含まないものを{\bf 閉式}\index{へいしき@閉式}{\bf (closed formula)}と呼ぶ.
	定理として考察するものは全て閉式である.また数などの特別な対象や概念は{\bf 枠線付きの定義}により名前を付けていく.
	一度枠線付きの定義で名前を付けられた対象や概念は,それ以後は本稿においてその名前で通用する.
	他方,枠線付きの定義という手続きを踏まなくても,便宜のために説明や証明の途中で対象に名前を付けることがある.それは次のようなものである: 
	\begin{prf}
		$\cdots$いま$P \coloneqq \Set{x}{\forall t\ (\ x = t \vee x \in t\ )}$とおく.このとき$\cdots$
	\end{prf}
	記号`$\coloneqq$'は{\bf 定義記号}と呼ばれ,右辺の類に左辺の記号列で名前を付けるという意味で使われる.
	このような文言は多くの説明や証明に出てくるが,実際上の効果として,
	以後の式に出てくる`$\Set{x}{\forall t\ (\ x = t \vee x \in t\ )}$'の部分を`$P$'で置き換えられるようになる.
	ただしその場合の定義はその説明や証明の中でのみ通用するものと約束する.
	
	\begin{screen}
		\begin{dfn}[宇宙]
			$\Univ \coloneqq \Set{x}{x=x}$で定める$\Univ$を{\bf 宇宙}\index{うちゅう@宇宙}{\bf (Universe)}と呼ぶ.
		\end{dfn}
	\end{screen}
	
	\monologue{
		院生「$\Univ$はそもそも式に現れる記号ではありませんでしたが,$\coloneqq$の推薦で$\Set{x}{x=x}$の代理人として市民権を得たのですね.
			このように既定の用語から定められる記号を{\bf 派生記号}と呼びます.
			さて宇宙という壮大な言葉が出てきましたが,後述する通り$\Univ$は集合の全体のことですから,
			あらゆるものが集合で説明される現代数学にとって$\Univ$はまさしく宇宙なのですね.
			また定理\ref{thm:Universe_and_ordinal_numbers}で$\Univ$の実態が明らかになるでしょう.
			我々はこの定理で集合とは何者かという問いへの完全な答えを得ることになります.
			ところで,現実世界において人間が把握し得る最大の世界は宇宙空間でしょうが,
			数学の世界では宇宙の外側を見ることが出来るのです.そこは真類と呼ばれるものの世界です.
			実は宇宙そのものも真類の一つなのですが,その話も後述にまかせましょう.
			ちなみに,宇宙が$\Univ$で表されるのはJohn Von NeumannのVに由来していると思われます.」
	}
	
	数学の式を日本語に翻訳するとき,慣習上よく使われる訳し方があるので列挙する.
	\begin{itemize}
		\item 式$a = b$を``$a$は$b$に等しい''や``$a$と$b$は等しい''と翻訳する.
		\item 式$a \in b$を``$a$は$b$の要素である''や``$a$は$b$に属する''と翻訳する.
		\item 式$(A) \Longrightarrow (B)$を``$A$が成り立つならば$B$が成り立つ''と翻訳する.
		\item 式$\rightharpoondown (A)$を%``$A$でない''と翻訳する.
	\end{itemize}
	
	\begin{screen}
		\begin{dfn}[類・集合]
			$\mathcal{L}'$の対象のことを{\bf 類}\index{るい@類}{\bf (class)}と呼ぶ.また$a$を類とするとき
			\begin{align}
				\set{a} \overset{\mathrm{def}}{\Longleftrightarrow} \exists x\ (\ a = x\ )
			\end{align}
			と定め,$\set{a}$が成立すれば$a$を{\bf 集合}\index{しゅうごう@集合}{\bf (set)}と呼び,
			$\rightharpoondown \set{a}$が成立すれば$a$を
			{\bf 真類}\index{しんるい@真類}{\bf (proper class)}と呼ぶ.
		\end{dfn}
	\end{screen}
	
	\monologue{
		院生「集合とは$\mathcal{L}$の対象か$\mathcal{L}$の対象に等しい類のことを指すのですね.
			また後述することですが宇宙は集合の全体に一致します.つまり$\Univ$の要素である類は集合であり,
			逆に類が集合であるならば$\Univ$の要素であるのです.
			さてここで次の問題を考えましょう.集合であり真類でもある類や,集合でも真類でもない類は存在するのかという問題です.
			実は我々はまだこれに答える術を持っていません.
			質問を変えましょう.集合であり真類でもある類や集合でも真類でもない類の存在を禁止するにはどうしたら良いでしょうか?
			我々は,数学において中庸が無いということや矛盾が起きるということをどう表現しようかという問題に直面しているのです.
			この問題の解決への方便として{\bf 推論規則}\index{すいろんきそく@推論規則}
			{\bf (rule of inference)}を導入します.」
	}
	
	\begin{screen}
		\begin{metaaxm}[排中律]
			$A$を$\mathcal{L}'$の閉式とするとき次は定理である:
			\begin{align}
				A \vee \rightharpoondown A.
			\end{align}
		\end{metaaxm}
	\end{screen}
	
	\monologue{
		院生「排中律の言明は``どんな閉式でも持ってくれば,その式に対して排中律が適用される''という意味です.
		このように無際限に存在し得る定理を一括して表現する書き方を{\bf 公理図式}\index{こうりずしき@公理図式}{\bf (schema)}と呼びます.」
	}
	
	いま$a,b$を類とするとき,
	\begin{align}
		a \notin b \overset{\mathrm{def}}{\Longleftrightarrow}\ \rightharpoondown a \in b
	\end{align}
	で$a \notin b$を定める.同様に
	\begin{align}
		a \neq b \overset{\mathrm{def}}{\Longleftrightarrow}\ \rightharpoondown a = b
	\end{align}
	で$a \neq b$を定める.
	
	\monologue{
		院生「定義記号$\coloneqq$と同様に,`$A \overset{\mathrm{def}}{\Longleftrightarrow} B$'とは
			式$B$を記号列$A$で置き換えて良いという意味で使われます.また,式中に記号列$A$が出てくるときは,
			暗黙裡にその$A$を$B$に戻して式を解釈します.
			$\coloneqq$も$\overset{\mathrm{def}}{\Longleftrightarrow}$も略記することと同じですね.」
	}
	
	\begin{screen}
		\begin{thm}[類は集合であるか真類であるかのいずれかに定まる]
			$a$を類とするとき次は定理である:
			\begin{align}
				\set{a} \vee \rightharpoondown \set{a}.
			\end{align}
		\end{thm}
	\end{screen}
	
	\begin{prf}
		排中律を適用することにより従う.
		\QED
	\end{prf}
	
	排中律をそのまま適用することにより上の定理は導かれたが,``集合でも真類でもない類は存在しない''という主張はまだ得られない.
	以下はこの言明を証明することを目標にしてしばらく推論規則の話が続くが,提示される規則はどれも基本的すぎるあまり
	通常は無断で使用されてしまうものである.
	
	\begin{screen}
		\begin{metaaxm}[基本的な推論規則]\label{metaaxm:fundamental_rules_of_inference}
			$A,B,C$を$\mathcal{L}'$の閉式とするとき,次の規則を認める:
			\begin{description}
				\item[三段論法] $A$ならびに$A \Longrightarrow B$が定理なら$B$は定理である.
				\item[演繹法則] $A$を公理に追加した下で$B$が定理であるなら,
					$A$を外した公理系で$A \Longrightarrow B$は定理である.
				\item[$\vee$の導入1] $A \Longrightarrow (A \vee B)$は定理である.
				\item[$\vee$の導入2] $A \Longrightarrow (B \vee A)$は定理である.
				\item[$\wedge$の導入] $A,B$が共に定理なら$A \wedge B$は定理である.
				\item[$\wedge$の除去1] $(A \wedge B) \Longrightarrow A$は定理である.
				\item[$\wedge$の除去2] $(A \wedge B) \Longrightarrow B$は定理である.
				\item[場合分け法則] $A \Longrightarrow C$と$B \Longrightarrow C$が共に定理であるとき
					$(A \vee B) \Longrightarrow C$は定理である.
			\end{description}	
		\end{metaaxm}
	\end{screen}
	
	\monologue{
		院生「演繹法則について,``$A$を公理に追加する''ことを``$A$が成り立っていると仮定する''
		などの言明により示唆することが多いです.」
	}
	
	\begin{screen}
		\begin{metathm}[反射律]\label{metathm:reflective_law_of_implication}
			$A$を$\mathcal{L}'$の閉式とするとき$A \Longrightarrow A$は定理である.
		\end{metathm}
	\end{screen}
	
	\begin{prf}
		$A$を公理に追加すれば$A$は成立するので,演繹法則より$A \Longrightarrow A$は定理である.
		\QED
	\end{prf}
	
	\begin{screen}
		\begin{metathm}[可換律]
			$A,B$を$\mathcal{L}'$の閉式とするとき次は定理である:
			\begin{itemize}
				\item $(A \vee B) \Longrightarrow (B \vee A)$.
				\item $(A \wedge B) \Longrightarrow (B \wedge A)$.
			\end{itemize}
		\end{metathm}
	\end{screen}
	
	\begin{prf}
		$\vee$の導入により$A \Longrightarrow (B \vee A)$と$B \Longrightarrow (A \vee B)$は定理であるから,場合分け法則より
		\begin{align}
			(A \vee B) \Longrightarrow (B \vee A)
		\end{align}
		は定理である.また,$\wedge$の除去より
		$A \wedge B \Longrightarrow A$と$A \wedge B \Longrightarrow B$は定理であるから,
		いま$A \wedge B$が成り立っていると仮定すれば三段論法により$B$も$A$も定理となる.このとき
		$B \wedge A$が定理となるので,演繹法則より
		\begin{align}
			(A \wedge B) \Longrightarrow (B \wedge A)
		\end{align}
		は定理である.
		\QED
	\end{prf}
	
	\begin{screen}
		\begin{metathm}[推移律]\label{metathm:transitive_law_of_implication}
			$A,B,C$を$\mathcal{L}'$の閉式とするとき,
			$A \Longrightarrow B$と$B \Longrightarrow C$が共に定理ならば
			$A \Longrightarrow C$は定理である.
		\end{metathm}
	\end{screen}
	
	\begin{prf}
		$A \Longrightarrow B$と$B \Longrightarrow C$が共に定理であるとして,
		$A$が成り立っていると仮定する.このとき三段論法より$B$が定理となり,
		再び三段論法より$C$が定理となる.ゆえに$A \Longrightarrow C$は定理である.
		\QED
	\end{prf}
	
	\begin{screen}
		\begin{metathm}[含意は遺伝する]\label{metathm:rule_of_inference_1}
			$A,B,C$を$\mathcal{L}'$の閉式とするとき以下が成り立つ:
			\begin{description}
				\item[(a)] $(A \Longrightarrow B) \Longrightarrow ( (A \vee C) \Longrightarrow (B \vee C) )$.
				
				\item[(b)] $(A \Longrightarrow B) \Longrightarrow ( (A \wedge C) \Longrightarrow (B \wedge C) )$.
				
				\item[(c)] $(A \Longrightarrow B) \Longrightarrow ( (B \Longrightarrow C) \Longrightarrow (A \Longrightarrow C) )$.
				
				\item[(c)] $(A \Longrightarrow B) \Longrightarrow ( (C \Longrightarrow A) \Longrightarrow (C \Longrightarrow B) )$.
			\end{description}
		\end{metathm}
	\end{screen}
	
	\begin{prf}\mbox{}
		\begin{description}
			\item[(a)]
				いま$A \Longrightarrow B$と$A$が成り立っていると仮定する.
				このとき三段論法より$B$が成立し,他方で$\vee$の導入より
				$B \Longrightarrow (B \vee C)$が成り立つので再び三段論法から$B \vee C$が成立する.
				ここに演繹法則を適用すれば,$A \Longrightarrow B$を成り立っている下で
				\begin{align}
					A \Longrightarrow (B \vee C)
				\end{align}
				が成り立つ.$\vee$の導入より$C \Longrightarrow (B \vee C)$も成り立っているから,
				場合分け法則より$A \Longrightarrow B$が成り立っている下で
				\begin{align}
					(A \vee C) \Longrightarrow (B \vee C)
				\end{align}
				が定理となり,ここに演繹法則を適用して
				\begin{align}
					(A \Longrightarrow B) \Longrightarrow 
					( (A \vee C) \Longrightarrow (B \vee C) )
				\end{align}
				が得られる.
				
			\item[(b)]
				いま$A \Longrightarrow B$と$A \wedge C$が成り立っていると仮定する.このとき
				$\wedge$の除去より$A$と$C$が成り立ち,三段論法より$B$が成り立つので,$\wedge$の導入より
				$B \wedge C$が成立する.ここに演繹法則を適用すれば,$A \Longrightarrow B$が成り立っている下で
				\begin{align}
					(A \wedge C) \Longrightarrow (B \wedge C)
				\end{align}
				が成立する.再び演繹法則を適用すれば
				\begin{align}
					(A \Longrightarrow B) \Longrightarrow ( (A \wedge C) \Longrightarrow (B \wedge C) )
				\end{align}
				が得られる.
				
			\item[(c)]
				いま$A \Longrightarrow B$,$B \Longrightarrow C$および
				$A$が成り立っていると仮定する.このとき三段論法より$B$が成り立つので再び三段論法より
				$C$が成立する.ゆえに演繹法則より$A \Longrightarrow B$と$B \Longrightarrow C$が
				成り立っている下で
				\begin{align}
					A \Longrightarrow C
				\end{align}
				が成立し,演繹法則を更に順次適用すれば
				\begin{align}
					(A \Longrightarrow B) \Longrightarrow ( (B \Longrightarrow C) \Longrightarrow (A \Longrightarrow C) )
				\end{align}
				が得られる.
				
			\item[(d)]
				いま$A \Longrightarrow B$,$C \Longrightarrow A$および
				$C$が成り立っていると仮定する.このとき三段論法より$A$が成り立つので再び三段論法より$B$が成立し,
				ここに演繹法則を適用すれば,$A \Longrightarrow B$と$C \Longrightarrow A$が成立している下で
				\begin{align}
					C \Longrightarrow B
				\end{align}
				が成立する.演繹法則を更に順次適用すれば
				\begin{align}
					(A \Longrightarrow B) \Longrightarrow ( (C \Longrightarrow A) \Longrightarrow (C \Longrightarrow B) )
				\end{align}
				が得られる.
				\QED
		\end{description}
	\end{prf}
	
	\begin{screen}
		\begin{metathm}\label{metathm:rule_of_inference_2}
			$A,B$を$\mathcal{L}'$の閉式とするとき,
			$B \Longrightarrow (A \Longrightarrow B)$は定理である.
		\end{metathm}
	\end{screen}
	
	\begin{prf}
		$B$を公理に追加した場合,$A$を公理に追加しても$B$は真であるから,このとき
		\begin{align}
			A \Longrightarrow B
		\end{align}
		は定理となる.従って演繹法則より$B \Longrightarrow (A \Longrightarrow B)$は定理である.
		\QED
	\end{prf}
	
	\begin{screen}
		\begin{dfn}[矛盾]
			論理式$\bot$を{\bf 矛盾}\index{むじゅん@矛盾}{\bf (contradiction)}と呼ぶ.
		\end{dfn}
	\end{screen}
	
	\begin{screen}
		\begin{metaaxm}[矛盾に関する規則]\label{metaaxm:rules_of_contradiction}
			$A$を$\mathcal{L}'$の閉式とするとき以下の式が成り立つ:
			\begin{description}
				\item[矛盾の発生] $A$と$\rightharpoondown A$が共に成り立つなら$\bot$が成り立つ:
					$(A \wedge \rightharpoondown A) \Longrightarrow \bot$.
				\item[否定の導出] $A$が$\bot$を導くなら$\rightharpoondown A$が成り立つ:
					$(A \Longrightarrow \bot) \Longrightarrow\ \rightharpoondown A$.
				\item[矛盾からはあらゆる式が導かれる] $\bot \Longrightarrow A$.
			\end{description}
		\end{metaaxm}
	\end{screen}
	
	\monologue{
		院生「$A$を$\mathcal{L}'$の閉式とするとき,式$A \Longrightarrow \bot$を
			``$A$は{\bf 偽である}\index{ぎ@偽}{\bf (false)}''と翻訳します.」
	}
	
	$A$と$B$を$\mathcal{L}'$の式とするとき,
	\begin{align}
		(A \Longleftrightarrow B) \overset{\mathrm{def}}{\Longleftrightarrow}
		(A \Longrightarrow B \wedge B \Longrightarrow A)
	\end{align}
	により$\Longleftrightarrow$を定め,式`$A \Longleftrightarrow B$'を
	``$A$と$B$は{\bf 同値である}\index{どうち@同値}{\bf (equivalent)}''と翻訳する.
	
	\begin{screen}
		\begin{metathm}[同値関係の遺伝性質]\label{metathm:hereditary_of_equivalence}
			$A,B,C$を$\mathcal{L}'$の閉式とするとき以下の式が成り立つ:
			\begin{description}
				\item[(a)] $(A \Longleftrightarrow B) \Longrightarrow ((A \vee C) \Longleftrightarrow (B \vee C))$.
				\item[(b)] $(A \Longleftrightarrow B) \Longrightarrow ((A \wedge C) \Longleftrightarrow (B \wedge C))$.
				\item[(c)] $(A \Longleftrightarrow B) \Longrightarrow ((B \Longrightarrow C) \Longleftrightarrow (A \Longrightarrow C))$.
				
				\item[(d)] $(A \Longleftrightarrow B) \Longrightarrow ((C \Longrightarrow A) \Longleftrightarrow (C \Longrightarrow B))$.
			\end{description}
		\end{metathm}
	\end{screen}
	
	\begin{prf}
		まず(a)を示す.いま$A \Longleftrightarrow B$が成り立っていると仮定する.このとき$A \Longrightarrow B$と
		$B \Longrightarrow A$が共に成立し,他方で含意の遺伝性質より
		\begin{align}
			&(A \Longrightarrow B) \Longrightarrow ((A \vee C) \Longrightarrow (B \vee C)), \\
			&(B \Longrightarrow A) \Longrightarrow ((B \vee C) \Longrightarrow (A \vee C))
		\end{align}
		が成立するから三段論法より$(A \vee C) \Longrightarrow (B \vee C)$と
		$(B \vee C) \Longrightarrow (A \vee C)$が共に成立する.ここに$\wedge$の導入を適用すれば
		\begin{align}
			(A \vee C) \Longleftrightarrow (B \vee C)
		\end{align}
		が成立し,演繹法則を適用すれば
		\begin{align}
			(A \Longleftrightarrow B) \Longrightarrow ((A \vee C) \Longleftrightarrow (B \vee C))
		\end{align}
		が得られる.(b)(c)(d)も含意の遺伝性を適用すれば得られる.
		\QED
	\end{prf}
	
	\begin{screen}
		\begin{metathm}[偽であることと否定は同値]\label{metathm:false_and_negation_are_equivalent}
			$A$を$\mathcal{L}'$の閉式とするとき次が成り立つ:
			\begin{align}
				(A \Longrightarrow \bot) \Longleftrightarrow\ \rightharpoondown A.
			\end{align}
		\end{metathm}
	\end{screen}
	
	\begin{prf}
		$\rightharpoondown A$が成り立っていると仮定する.このとき$A$が成り立っていれば
		推論規則\ref{metaaxm:rules_of_contradiction}より$\bot$が成立するから,演繹法則より
		\begin{align}
			\rightharpoondown A \Longrightarrow (A \Longrightarrow \bot)
		\end{align}
		が成り立つ.一方で推論規則\ref{metaaxm:rules_of_contradiction}より
		\begin{align}
			(A \Longrightarrow \bot) \Longrightarrow\ \rightharpoondown A.
		\end{align}
		が満たされているので$(A \Longrightarrow \bot)$と$\rightharpoondown A$は同値である.
		\QED
	\end{prf}
	
	\begin{screen}
		\begin{metathm}[二重否定の法則]
			$A$を$\mathcal{L}'$の閉式とするとき,
			$A \Longleftrightarrow\ \rightharpoondown \rightharpoondown A$は定理である.
		\end{metathm}
	\end{screen}
	
	\begin{prf}
		$A$が成り立っていると仮定する.このとき$\rightharpoondown A$が成り立っていれば
		$\bot$が定理となるので
		\begin{align}
			\rightharpoondown A \Longrightarrow \bot
		\end{align}
		が成り立ち,否定の導出より$\rightharpoondown \rightharpoondown A$が定理となる.従って
		\begin{align}
			A \Longrightarrow\ \rightharpoondown \rightharpoondown A
		\end{align}
		が得られる.逆に$\rightharpoondown \rightharpoondown A$が成り立っていると仮定すると,
		$\rightharpoondown A$が成り立っているなら$\bot$が定理となり,
		また$\bot$からはあらゆる閉式が導かれるので$A$も定理となる.
		ここに推論法則\ref{metathm:transitive_law_of_implication}を適用すれば
		\begin{align}
			\rightharpoondown A \Longrightarrow A
		\end{align}
		が成り立つ.反射律より$A \Longrightarrow A$も定理であるから場合分け法則より
		\begin{align}
			(A \vee \rightharpoondown A) \Longrightarrow A
		\end{align}
		が成り立ち,排中律より$A \vee \rightharpoondown A$は正しいので三段論法より$A$は定理となる.
		これにより
		\begin{align}
			\rightharpoondown \rightharpoondown A \Longrightarrow\ A
		\end{align}
		も得られた.
		\QED
	\end{prf}
	
	\begin{screen}
		\begin{metathm}[背理法の原理]
			$A$を$\mathcal{L}'$の閉式とするとき,
			$\rightharpoondown A \Longrightarrow \bot$が成り立つならば$A$は定理である.
		\end{metathm}
	\end{screen}
	
	\begin{prf}
		$\rightharpoondown A \Longrightarrow \bot$が成り立つとき,否定の導出より
		$\rightharpoondown \rightharpoondown A$が成り立つが,二重否定の法則より
		$A$も成立する.
		\QED
	\end{prf}
	
	\begin{screen}
		\begin{metathm}[ならばとまたは]\label{metathm:rule_of_inference_3}
			$A,B$を$\mathcal{L}'$の閉式とするとき,次が成り立つ:
			\begin{align}
				(A \Longrightarrow B) \Longleftrightarrow (\rightharpoondown A \vee B).
			\end{align}
		\end{metathm}
	\end{screen}
	
	\begin{prf}
		$A \Longrightarrow B$が成り立っていると仮定する.含意の遺伝性質より
		\begin{align}
			(A \Longrightarrow B) \Longrightarrow 
			((A \vee \rightharpoondown A) \Longrightarrow (B \vee \rightharpoondown A))
		\end{align}
		が満たされているから三段論法より$(A \vee \rightharpoondown A) \Longrightarrow (B \vee \rightharpoondown A)$
		は定理となり,ここに排中律と三段論法を適用すれば$B \vee \rightharpoondown A$が定理となる.
		ここで可換律より$\rightharpoondown A \vee B$が成り立つので,演繹法則を適用して
		\begin{align}
			(A \Longrightarrow B) \Longrightarrow (\rightharpoondown A \vee B)
		\end{align}
		が得られる.また$\rightharpoondown A$が成り立っていると仮定すると,
		この下で$A$を公理に追加すれば$\bot$が定理となり,$\bot$からは$B$が導かれるので$B$も定理となる.従って
		演繹法則より
		\begin{align}
			(\rightharpoondown A) \Longrightarrow (A \Longrightarrow B)
		\end{align}
		が成り立つ.一方で推論法則\ref{metathm:rule_of_inference_2}より
		$B \Longrightarrow (A \Longrightarrow B)$が定理であるから,場合分けの法則より
		\begin{align}
			(\rightharpoondown A \vee B) \Longrightarrow (A \Longrightarrow B)
		\end{align}
		が成り立つ.以上で$(A \Longrightarrow B) \Longleftrightarrow (\rightharpoondown A \vee B)$が得られた.
		\QED
	\end{prf}
	
	\monologue{
		院生「$A,B$を$\mathcal{L}'$の閉式とするとき,$A$が偽であれば$\rightharpoondown A$が成立する
			(推論規則\ref{metaaxm:rules_of_contradiction})ので
			$\rightharpoondown A \vee B$が成立します(推論規則\ref{metaaxm:fundamental_rules_of_inference}).
			すなわちこのとき$A \Longrightarrow B$が成り立つのですが,式の解釈としては
			``偽な式からはあらゆる式が導かれる''となりますね.この現象を
			{\bf 空虚な真}\index{くうきょなしん@空虚な真}{\bf (vacuous truth)}と呼びます.」
	}
	
	\begin{screen}
		\begin{metathm}[対偶命題は同値]
			$A,B$を$\mathcal{L}'$の閉式とするとき,次が成り立つ:
			\begin{align}
				(A \Longrightarrow B) \Longleftrightarrow (\rightharpoondown B \Longrightarrow\ \rightharpoondown A).
			\end{align}
		\end{metathm}
	\end{screen}
	
	\begin{prf}
		推論法則\ref{metathm:rule_of_inference_3},可換律,二重否定の法則を順に用いれば
		\begin{align}
			(A \Longrightarrow B) &\Longleftrightarrow (\rightharpoondown A \vee B) \\
			&\Longleftrightarrow (B \vee \rightharpoondown A) \\
			&\Longleftrightarrow (\rightharpoondown \rightharpoondown B \vee \rightharpoondown A) \\
			&\Longleftrightarrow (\rightharpoondown B \Longrightarrow\ \rightharpoondown A)
		\end{align}
		が成り立つ.
		\QED
	\end{prf}
	
	\monologue{
		院生「対偶命題を述べるときには``対偶を取る''と表現することが多いです.」
	}
	
	\begin{screen}
		\begin{metathm}[De Morganの法則]
			$A,B$を$\mathcal{L}'$の閉式とするとき,次が成り立つ:
			\begin{itemize}
				\item $\rightharpoondown (A \vee B) \Longleftrightarrow\ \rightharpoondown A \wedge \rightharpoondown B$.
			
				\item $\rightharpoondown (A \wedge B) \Longleftrightarrow\ \rightharpoondown A \vee \rightharpoondown B$.
			\end{itemize}
		\end{metathm}
	\end{screen}
	
	\begin{prf}
		$A \Longrightarrow (A \vee B)$は定理であるから,その対偶命題
		\begin{align}
			\rightharpoondown (A \vee B) \Longrightarrow\ \rightharpoondown A
		\end{align}
		も定理となる.同様に$\rightharpoondown (A \vee B) \Longrightarrow\ \rightharpoondown B$は定理となるので,
		$\rightharpoondown (A \vee B)$が成り立っていると仮定すれば$\rightharpoondown A \wedge \rightharpoondown B$が成り立つ.
		ゆえに
		\begin{align}
			\rightharpoondown (A \vee B) \Longrightarrow\ \rightharpoondown A \wedge \rightharpoondown B
		\end{align}
		が得られる.また$A$が成り立っていると仮定すれば,この下で$\rightharpoondown A \wedge \rightharpoondown B$が成り立っているなら
		$A$と$\rightharpoondown A$が同時に成り立つことになるので$\bot$が成立する.つまり
		$A$が成り立っているとき
		\begin{align}
			\rightharpoondown A \wedge \rightharpoondown B \Longrightarrow \bot
		\end{align}
		が成り立つが,このとき$\rightharpoondown(\rightharpoondown A \wedge \rightharpoondown B)$が成り立つので
		\begin{align}
			A \Longrightarrow\ \rightharpoondown(\rightharpoondown A \wedge \rightharpoondown B)
		\end{align}
		が得られる.同様にして
		\begin{align}
			B \Longrightarrow\ \rightharpoondown(\rightharpoondown A \wedge \rightharpoondown B)
		\end{align}
		も得られるから,場合分け法則より
		\begin{align}
			(A \vee B) \Longrightarrow\ \rightharpoondown(\rightharpoondown A \wedge \rightharpoondown B)
		\end{align}
		が成立する.この対偶を取れば
		\begin{align}
			\rightharpoondown A \wedge \rightharpoondown B
			\Longrightarrow\ \rightharpoondown (A \vee B)
		\end{align}
		が出る.以上で一つ目の式が示された.一つ目の式で$A$を$\rightharpoondown A$に,
		$B$を$\rightharpoondown B$に置き換えると
		\begin{align}
			\rightharpoondown \rightharpoondown A \wedge \rightharpoondown \rightharpoondown B
			\Longleftrightarrow\ \rightharpoondown (\rightharpoondown A \vee \rightharpoondown B)
		\end{align}
		が得られるが,このとき二重否定の法則より
		\begin{align}
			A \wedge B
			\Longleftrightarrow\ \rightharpoondown (\rightharpoondown A \vee \rightharpoondown B)
		\end{align}
		が成立し,対偶命題の同値性から
		\begin{align}
			\rightharpoondown (A \wedge B)
			\Longleftrightarrow\ (\rightharpoondown A \vee \rightharpoondown B)
		\end{align}
		は定理となる.
		\QED
	\end{prf}
	
	\monologue{
		院生「以上で``集合でも真類でもない類は存在しない''という言明を証明する準備が整いました.」
	}
	
	\begin{screen}
		\begin{thm}[集合でも真類でもない類は存在しない]
			$a$を類とするとき次が成り立つ:
			\begin{align}
				\rightharpoondown (\ \set{a} \wedge \rightharpoondown \set{a}\ ).
			\end{align}
		\end{thm}
	\end{screen}
	
	\begin{prf}
		$a$を類とするとき,排中律より$\set{a} \vee \rightharpoondown \set{a}$
		が成り立ち,可換律より
		\begin{align}
			\rightharpoondown \set{a} \vee \set{a}
		\end{align}
		も成立する.そしてDe Morganの法則より
		\begin{align}
			\rightharpoondown (\ \rightharpoondown \rightharpoondown \set{a} \wedge \rightharpoondown \set{a}\ )
		\end{align}
		が成り立つが,二重否定の法則より$\rightharpoondown \rightharpoondown \set{a}$と
		$\set{a}$は同値となるので
		\begin{align}
			\rightharpoondown (\ \set{a} \wedge \rightharpoondown \set{a}\ )
		\end{align}
		が成り立つ.
		\QED
	\end{prf}
	
	\monologue{
		院生「我々は$\exists$の意味には触れずにここまで来ましたが,
			次は量化記号が推論操作の上でどのような働きを持つのかを規定しましょう.」
	}
	
	\begin{screen}
		\begin{metaaxm}[量化記号に関する規則]\label{metaaxm:rules_of_quantifiers}
			$A$を$\mathcal{L}'$の式とし,$x$を$A$に現れる文字とするとき,$x$のみが$A$で量化されていないならば以下を認める:
			\begin{description}
				\item[$\varepsilon$記号の導入] $\varepsilon x A(x)$は$\mathcal{L}$の或る対象に代用される.
				\item[存在記号の規則] $A (\varepsilon x A(x)) \Longleftrightarrow \exists x A(x)$が成り立つ.
				\item[全称記号の規則] $A (\varepsilon x \rightharpoondown A(x)) \Longleftrightarrow \forall x A(x)$が成り立つ.
				\item[存在記号の基本性質] $\tau$を$\mathcal{L}$の対象とするとき
					$A(\tau) \Longrightarrow \exists x A(x)$が成り立つ.
			\end{description}
		\end{metaaxm}
	\end{screen}
	
	\monologue{
		院生「$\varepsilon$記号はHilbertのイプシロン関数と呼ばれる概念を参考にしたもので,
			量化記号の働きを形式的に表現するのに便利です.また
			``$\varepsilon x A(x)$は$\mathcal{L}$の対象である''という言明において,
			$\mathcal{L}$の対象であるという点がミソです...
			なぜミソなのか忘れましたがこうでないと後で不具合が起きた記憶があります.」
	}
	
	\begin{screen}
		\begin{metathm}[全称記号の基本性質]\label{metathm:fundamental_law_of_universal_quantifier}
			$A$を$\mathcal{L}'$の式とし,$x$を$A$に現れる文字とし,$x$のみが$A$で量化されていないとする.このとき
			$\forall x A(x)$が成り立つならば$\mathcal{L}$のいかなる対象$\tau$に対しても$ A(\tau)$が成り立つ.
			逆に,$\mathcal{L}$のいかなる対象$\tau$に対しても$A(\tau)$が成り立てば$\forall x A(x)$が成り立つ.
		\end{metathm}
	\end{screen}
	
	\begin{prf}
		$\tau$を$\mathcal{L}$の任意の対象とすれば,存在記号に関する推論規則より
		\begin{align}
			\begin{gathered}
				\rightharpoondown A(\tau) \Longrightarrow\ \exists x \rightharpoondown A(x), \\
				\exists x \rightharpoondown A(x) \Longrightarrow\ \rightharpoondown A
				\left( \varepsilon x \rightharpoondown A(x) \right)
			\end{gathered}
		\end{align}
		が成り立つから,推論法則\ref{metathm:transitive_law_of_implication}より
		$\rightharpoondown A(\tau) \Longrightarrow\ \rightharpoondown A
				\left( \varepsilon x \rightharpoondown A(x) \right)$が成り立つ.
		そして対偶を取って
		\begin{align}
			A \left( \varepsilon x \rightharpoondown A(x) \right)
			\Longrightarrow A(\tau)
		\end{align}
		が成り立つ.全称記号に関する推論規則より$A \left( \varepsilon x \rightharpoondown A(x) \right)$と
		$\forall x A(x)$は同値であるから
		\begin{align}
			\forall x A(x) \Longrightarrow A(\tau)
		\end{align}
		が成り立つ.逆にいかなる対象$\tau$に対しても$A(\tau)$が成り立つとき,特に
		\begin{align}
			A \left( \varepsilon x \rightharpoondown A(x) \right)
		\end{align}
		が成り立つので$\forall x A(x)$も成り立つ.
		\QED
	\end{prf}
	
	\monologue{
		院生「推論法則\ref{metathm:fundamental_law_of_universal_quantifier}を根拠にして,
			当面は$\forall x A(x)$という式を``$\mathcal{L}$の任意の対象$x$に対して
			$A(x)$が成立する''と翻訳することにします.また後述する相等性の公理によれば,
			これは``任意の集合$x$に対して$A(x)$が成立する''と翻訳しても同義です.」
	}
	
	\begin{screen}
		\begin{metathm}[量化記号の性質(イ)]\label{metathm:properties_of_quantifiers}
			$A,B$を$\mathcal{L}'$の式とし,$x$を$A,B$に現れる文字とするとき,$x$のみが$A,B$で量化されていないならば以下は定理である:
			\begin{description}
				\item[(a)] $\mathcal{L}$の任意の対象$\tau$に対して
					$A(\tau) \Longleftrightarrow B(\tau)$となるとき次が成り立つ:
					\begin{align}
						\exists x A(x) \Longleftrightarrow \exists x B(x).
					\end{align}
				
				\item[(b)] $\mathcal{L}$の任意の対象$\tau$に対して
					$A(\tau) \Longleftrightarrow B(\tau)$となるとき次が成り立つ:
					\begin{align}
						\forall x A(x) \Longleftrightarrow \forall x B(x).
					\end{align}
					
				\item[(c)] $\exists x \rightharpoondown A(x) \Longleftrightarrow\ \rightharpoondown \forall x A(x)$.
				
				\item[(d)] $\forall x \rightharpoondown A(x) \Longleftrightarrow\ \rightharpoondown \exists x A(x)$.
			\end{description}
		\end{metathm}
	\end{screen}
	
	\begin{prf}\mbox{}
		\begin{description}
			\item[(a)]
				いま,$\mathcal{L}$の任意の対象$\tau$に対して
				$A(\tau) \Longleftrightarrow B(\tau)$が満たされているとする.
				ここで$\exists x A(x)$が成り立っていると仮定すると,
				\begin{align}
					\tau \coloneqq \varepsilon x A(x)
				\end{align}
				とおけば推論規則\ref{metaaxm:rules_of_quantifiers}より$A(\tau)$が成立し,
				他方で推論規則\ref{metaaxm:fundamental_rules_of_inference}により
				\begin{align}
					A(\tau) \Longrightarrow B(\tau)
				\end{align}
				が満たされるので$B(\tau)$が成立する.
				再び推論規則\ref{metaaxm:rules_of_quantifiers}より$\exists x B(x)$が成り立つので
				\begin{align}
					\exists x A(x) \Longrightarrow \exists x B(x)
				\end{align}
				が得られる.$A$と$B$の立場を入れ替えれば$\exists x B(x) \Longrightarrow \exists x A(x)$も得られる.
				
			\item[(b)]
				いま,$\mathcal{L}$の任意の対象$\tau$に対して
				$A(\tau) \Longleftrightarrow B(\tau)$が満たされているとする.
				ここで$\forall x A(x)$が成り立っていると仮定すると,
				推論法則\ref{metathm:fundamental_law_of_universal_quantifier}より
				$\mathcal{L}$の任意の対象$\tau$に対して$A(\tau)$が成立し,
				他方で
				\begin{align}
					A(\tau) \Longrightarrow B(\tau)
				\end{align}
				が満たされるので$B(\tau)$が成立する.このとき$\tau$の任意性と
				推論法則\ref{metathm:fundamental_law_of_universal_quantifier}より$\forall x B(x)$が成り立つから
				\begin{align}
					\forall x A(x) \Longrightarrow \forall x B(x)
				\end{align}
				が得られる.$A$と$B$の立場を入れ替えれば$\forall x B(x) \Longrightarrow \forall x A(x)$も得られる.
				
			\item[(c)]
				推論規則\ref{metaaxm:rules_of_quantifiers}より
				\begin{align}
					\exists x \rightharpoondown A(x) \Longleftrightarrow\ 
					\rightharpoondown A(\varepsilon x \rightharpoondown A(x))
				\end{align}
				は定理である.同様に推論規則\ref{metaaxm:rules_of_quantifiers}より
				\begin{align}
					A(\varepsilon x \rightharpoondown A(x)) \Longleftrightarrow \forall x A(x) 
				\end{align}
				もまた定理であり,対偶を取れば
				\begin{align}
					\rightharpoondown A(\varepsilon x \rightharpoondown A(x)) \Longleftrightarrow\ 
					\rightharpoondown \forall x A(x)
				\end{align}
				が定理となるので$\exists x \rightharpoondown A(x) \Longleftrightarrow\ \rightharpoondown \forall x A(x)$を得る.
			
			\item[(d)]
				前段の結果より
				\begin{align}
					\forall x \rightharpoondown A(x) \Longleftrightarrow\ 
					\rightharpoondown \exists x \rightharpoondown \rightharpoondown A(x)
				\end{align}
				が成り立ち,また二重否定の法則と(a)より
				\begin{align}
					\exists x \rightharpoondown \rightharpoondown A(x)
					\Longleftrightarrow \exists x A(x)
				\end{align}
				も成り立つから,推論法則\ref{metathm:transitive_law_of_implication}と併せて
				\begin{align}
					\forall x \rightharpoondown A(x) \Longleftrightarrow\ 
					\rightharpoondown \exists x A(x)
				\end{align}
				が得られる.
				\QED
		\end{description}
	\end{prf}
	\begin{quote}
		初めに言があった。言は神と共にあった。言は神であった。\\
		この言は、初めに神と共にあった。\\
		万物は言によって成った。成ったもので、言によらずに成ったものは何一つなかった。
	\end{quote}
	ヨハネによる福音書の冒頭である。本稿の世界もまた数学のことば,言い換えれば論理のみによって創られる.
	まず数学の言語として$\mathcal{L}$というものを用意する.
	\begin{comment}
	言と書いてことばと読む.なぜ``言葉''ではなく``言''と書くのだろうか.
	一説によれば言葉の葉の字の由来は万葉古今集仮名序にあり,
	現代的に説明すれば,見聞きしたり感動したりしたところを種にして生じる語彙のことを木の葉に喩えているらしい.
	言葉は人が発するものであり,たいていの場合食い違いなく通用する.ならばすなわち,
	葉が付かない``言''とは,人為の介入する前から世界を認識し,人が自覚する前から人の心に通底している
	コードと解釈すれば良いのであろうか.一気に言の字が神格を帯びてくるが,
	``言葉''ではなく``言''と書くことの謎解きとしてはあながち見当はずれでもないらしく,
	上の引用文の通り%は森羅万象はことばによって成り,ことばによって尽くされるという意味であるから,
	キリスト教においてことばとは神であり森羅万象を超越しているのである.
	しかもキリスト教に限らずとも,信仰心が薄いと言われる日本でさえ``はじめにことばありき''の文句が闊歩している程,
	それはこの世の真理として2000年以上も国籍や文化を問わず多くの人に受け入れられてきた.
	
		実際に自然言語の発生が事物の観測なしに起こり得たかという問題は言語哲学上も決着がついていないらしいが,
		少なくとも
	
	そして本稿の世界もまた数学のことば,言い換えれば論理のみによって創られるという点でキリスト教的であり,それ以上に
	ことばから始めようというのは,人が生来持っている直観に対して自然な起りなのであろう.
	しかしながら,神なることばが世界の悉くを尽くせる一方で,人が創造する数学の世界は論理のみによっては完結し得ないという事実もある.
	\end{comment}
	
	
	\begin{description}
		\item[対象領域] $V$
		\item[述語記号] $=,\ \in$
		\item[論理記号] $\bot,\ \Longrightarrow,\ \wedge,\ \vee,\ \rightharpoondown$
		\item[限定作用素] $\forall,\ \exists$
	\end{description}
	何が$V$の構成員となりうるかは以後公理によって規定していく.
	
	日常言語において,``あmt後右所sごぐふぉsdあじお''のように無作為に文字を並べただけでは意味不明な
	文字列が出来上がる.文字列は,何らかの規則に従って並ぶことで単語や文章として成立するのである.
	数学も同じで,一定の規則に従って並ぶ記号列のみを数学における文章として扱う.
	数学語において,単語にあたるものは{\bf 対象}\index{たいしょう@対象}{\bf (individual)}と呼ばれる.
	$V$を対象領域といったが,これは$V$が$\mathcal{L}$の対象の全体を表しているという意味である.
	述語とは対象同士を結ぶものであり,最小単位の文章を形作る.例えば,対象$s,t$に対し
	\begin{align}
		s \in t
	\end{align}
	は数学の文章となり,日本語には``$s$は$t$の要素である''と翻訳される.
	数学における文章を{\bf 式}\index{しき@式}
	{\bf (formula)}或は{\bf 論理式}\index{ろんりしき@論理式}と呼ぶ.
	論理記号とは式同士を繋ぐ役割を持つ.
	
	\begin{screen}
		\begin{dfn}[変項]
			言語において対象を代表する記号を{\bf 変項}\index{へんこう@変項}{\bf (variable)}
			と呼ぶ.変項は特定の対象を指すものではない.
		\end{dfn}
	\end{screen}
	
	仮に対象が$x,y,z$のみであるような言語を考えるとして,変項は$a$でも
	$b$でもよい.ただし$x$は変項として使えない.$x$は対象$x$を指すためである.
	ギリシア文字全体が対象領域なら,変項はこう言うであろう:
	\begin{align}
		\mbox{私はアルファでありオメガである.}
	\end{align}
	
	対象と変項を併せて{\bf 項}\index{こう@項}{\bf (term)}と呼び,
	対象を用いて作られていた式は対象を項に替えても式と呼ぶことにする.
	いま述べたことを形式的に書き直すと次のようになる.
	\begin{description}
		\item[項] 言語$\mathcal{L}$の対象は$\mathcal{L}$の項であり,
			変項も$\mathcal{L}$の項である.
			またそれらのみが$\mathcal{L}$の項である.
			
		\item[式] 
			\begin{itemize}
				\item $\bot$は$\mathcal{L}$の式である.
				
				\item $s,t$を言語$\mathcal{L}$の項とするとき,
					$s=t,s \in t$は$\mathcal{L}$の式である.
					
				\item $A,B$を$\mathcal{L}$の式とするとき,
					$A \wedge B,A \vee B,A\Longrightarrow B,
					\rightharpoondown A$は$\mathcal{L}$の式である.
				
				\item $A$が$\mathcal{L}$の式であり,項$x$が
					$A$においてただ一つの自由変項であるとき,
					$\forall x A(x),\ \exists x A(x)$は
					$\mathcal{L}$の式である.
				
				\item 以上の操作を繰り返して得られる記号列のみが$\mathcal{L}$の式である.
					ただし,繰り返しの操作は無制限に行われるものではない.
					得られる記号列は左端から辿っていけば必ず右端が見つかるものとする
					\footnotemark.
			\end{itemize}
	\end{description}
	\footnotetext{式は有限長であると言えば済みそうであるが,有限とは何かを未だ規定していないので
		ナイーブな書き方になってしまうのは致し方ない.}
	
	たったいま言語$\mathcal{L}$を作ったばかりであるが,$\mathcal{L}$を
	次の言語$\mathcal{L}'$に拡張する.理由は,$\Set{x}{A(x)}$のような形の
	記法を導入したいためである.
	
	\begin{description}
		\item[対象]
			\begin{itemize}
				\item $\mathcal{L}$の対象は$\mathcal{L}'$の対象である.
				\item $\mathcal{L}$の対象領域$V$は$\mathcal{L}'$の対象である.
				\item 項$x$のみを自由変数とする$\mathcal{L}$の論理式$A(x)$に対し
					$\Set{x}{A(x)}$は$\mathcal{L}'$の対象である.
				\item 以上で得られる記号列のみが$\mathcal{L}'$の対象である.
			\end{itemize}
			
		\item[項] 
			言語$\mathcal{L}'$の対象は$\mathcal{L}'$の項であり,
			変項も$\mathcal{L}'$の項である.
			またそれらのみが$\mathcal{L}'$の項である.
			
		\item[述語記号] $=,\in$
		\item[論理記号] $\bot,\ \Longrightarrow,\ \wedge,\ \vee,\ \rightharpoondown$
	\end{description}
	
	述語記号も論理記号も$\mathcal{L}$のものと同じであるが,こうすることで
	$\mathcal{L}$の式が$\mathcal{L}'$においても式となる.
	%言語を拡張したことに応じて記号の適用範囲が広がったと見ればよい.
	\begin{comment}
	\begin{itembox}[l]{研究室にて}
		ポスドク「今回の言語の拡張では新造語を取り入れた.
			身近な例として最近ではネットスラングが国語辞書に載ったりしているね.
			日常生活で新造語が出来ても文法まではなかなか変わらないように,
			述語記号や論理記号は言語を拡張しても変えなかった.むしろ変えてしまうと不便で不都合だ.
			対象領域$V$が$\mathcal{L}'$の項になっているのは,
			新造語が出現して初めて旧来の言語というカテゴリーが意識されたと解釈できるかな.
			まあ実際は別の意図があるのだけれど…」
	\end{itembox}
	\end{comment}
	
	\begin{screen}
		\begin{dfn}[類・集合]
			言語$\mathcal{L}'$の対象を{\bf 類}\index{るい@類}{\bf (class)}と呼び,
			特に言語$\mathcal{L}$の対象であった類を
			{\bf 集合}\index{しゅうごう@集合}{\bf (set)}と呼ぶ.
		\end{dfn}
	\end{screen}
	
	\begin{comment}
	\begin{itembox}[l]{研究室にて}
		ポスドク「現代的な数学では,数や関数など数学に関するあらゆるものは集合で構成されるのだね.
		そして集合そのものは述語論理を基礎にして公理的に規定される.
		この意味で集合論の勉強には論理学の知識が必要であると聞くけれども,
		真に受けて論理学の本を眺めてみれば,変数に数で添え字をつけたり,
		関数や$k$項述語などといったものを取り扱っているのが散見するのが残念だ.
		モノの数え方すら知らない生まれたばかりの世界に立って色々説明しようというのに,
		その前準備の段階で数の存在を予定したり関数の概念を前借りしてしまうのは好ましくないし,
		そのように集合論と論理学が循環しているのは勉強する側にとってはただの障壁にしかならない.
		どのみち両者は切り離せないのだから,同時に並行して記述する方がやりやすいのではないだろうか.」
	\end{itembox}
	\end{comment}
	
	\begin{screen}
		\begin{axm}[類の公理]
			\begin{align}
				\forall t(A(t) \Longleftrightarrow t \in \Set{x}{A(x)}).
			\end{align}
		\end{axm}
	\end{screen}
	
	\begin{screen}
		\begin{axm}[外延性の公理]
			\begin{align}
				\forall t(t \in A  \Longleftrightarrow t \in B)
				\Longrightarrow A=B.
			\end{align}
		\end{axm}
	\end{screen}
	
	\begin{screen}
		\begin{thm}
			\begin{description}
				\item[(1)] $A=A$
				\item[(2)] $a = \Set{x}{x \in a}$
				\item[(3)] $V=\Set{x}{x=x}$
				\item[(4)] $\Set{x}{A(x)} = \Set{y}{A(y)}$
			\end{description}
		\end{thm}
	\end{screen}
	
	\begin{screen}
		\begin{axm}[相等性の公理]
			\begin{align}
				A=B \Longrightarrow (\varphi(A) \Longleftrightarrow \varphi(B)).
			\end{align}
		\end{axm}
	\end{screen}
	
	\begin{screen}
		\begin{axm}[論理の公理]\mbox{}
			\begin{description}
				\item[(1)] 証明可能な命題は真である.
				\item[(2)] 任意の命題$A,B$に対し,$A$も$A \Longrightarrow B$も真であるとき$B$は真である.
				\item[(3)] 任意の命題$A$に対し,$\rightharpoondown A$が真であるとき$A$は偽である.
			\end{description}
		\end{axm}
	\end{screen}
	
	\begin{screen}
		\begin{thm}
			$A$を任意の命題とするとき,
			\begin{align}
				\mbox{$\rightharpoondown A$が真である} \Longleftrightarrow \mbox{$A$が偽である}.
			\end{align}
		\end{thm}
	\end{screen}
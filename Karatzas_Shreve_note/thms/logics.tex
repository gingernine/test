	\begin{screen}
		\begin{dfn}[$\Longrightarrow,\ \Longleftrightarrow,\ \wedge,\ \vee$の定義]
			$A,B$を命題とするとき,
			\begin{itemize}
				\item `$A$が真であるとき$B$は真である'を$A \Longrightarrow B$で表す.
				\item `$A$かつ$B$'を$A \wedge B$で表す.
				\item `$A$または$B$'を$A \vee B$で表す.
				\item $A \Longrightarrow B \wedge B \Longrightarrow A$を
					$A \Longleftrightarrow B$と略記する.
			\end{itemize}
		\end{dfn}
	\end{screen}
	
	\begin{screen}
		\begin{axm}[命題論理の公理]\mbox{}
			\begin{description}
				\item[(1)] 任意の命題$A,B$に対し,
					$A $
				\item[(1)] 証明可能な命題は真である.
				\item[(2)] 任意の命題$A,B$に対し,$A$も$A \Longrightarrow B$も真であるとき$B$は真である.
				\item[(3)] 任意の命題$A$に対し,$\rightharpoondown A$が真であるとき$A$は偽である.
			\end{description}
		\end{axm}
	\end{screen}
	
	\begin{screen}
		\begin{thm}
			$A$を任意の命題とするとき,
			\begin{align}
				\mbox{$\rightharpoondown A$が真である} \Longleftrightarrow \mbox{$A$が偽である}.
			\end{align}
		\end{thm}
	\end{screen}
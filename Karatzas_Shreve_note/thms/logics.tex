	\begin{quote}
		初めに言(ことば)があった。言は神と共にあった。言は神であった。\\
		この言は、初めに神と共にあった。\\
		万物は言によって成った。成ったもので、言によらずに成ったものは何一つなかった。
	\end{quote}
	ヨハネによる福音書の冒頭である.数学の世界もまたことばが支配する.
	ただし数学の世界におけることばには二つの階層がある.一つは記号や記号の並べ方を規定する下位のことばであり,
	もう一つは何が定理であるかを規定する上位のことばである.
	前者は我々が神の視点で創る世界のことばであり,後者は神である我々の世界のことばであるが,
	こちらは論理と言い換える方が適当である.
	我々が創る世界は集合論と呼ばれ,数や関数など高校まで初等的に与えられてきたあらゆる概念がその世界の中で説明し直されることになる.
	
\subsection{言語}
	
	まず言語$\mathcal{L}$というものを設定する.これは我々が創る世界のことばである.以下は$\mathcal{L}$を構成する要素である:
	\begin{description}
		\item[使用文字] 自然言語から借用する文字は表にあるものに限る.
		\item[定数記号] $\emptyset$
		\item[述語記号] $=,\ \in$
		\item[論理記号] $\bot,\ \Longrightarrow,\ \wedge,\ \vee,\ \rightharpoondown$
		\item[量化記号] $\forall,\ \exists$
		\item[補助記号] $[\ ,\ ]\ ,\ (\ ,\ )\ ,\ \{\ ,\ \}\ ,\ <\ ,\ >\ ,\ |,\coloneqq,\ \overset{\mathrm{def}}{\Longleftrightarrow}$
	\end{description}
	
	日常言語において,``あmt後右所sごぐふぉsd''のように無作為に文字を並べただけでは意味不明な
	文字列が出来上がる.文字列は,何らかの規則に従って並ぶことで単語や文章として成立するのである.
	数学も同じで,一定の規則に従って並ぶ記号列のみを数学における文章として扱う.
	数学語において,名詞にあたるものは{\bf 対象}\index{たいしょう@対象}{\bf (individual)}と呼ばれる.
	述語とは対象同士を結ぶものであり,最小単位の文章を形成する.例えば$s,t$を対象とするとき
	\begin{align}
		s \in t
	\end{align}
	は数学の文章となり,日本語には``$s$は$t$の要素である''と翻訳される.
	数学における文章を{\bf 式}\index{しき@式}
	{\bf (formula)}或は{\bf 論理式}\index{ろんりしき@論理式}と呼ぶ.
	論理記号とは式同士を繋ぐ役割を持つ.
	
	\monologue{
		院生「定数$\emptyset$は$\mathcal{L}$の対象の一つです.
			他の対象がどういうものであるかは後で判明しますが,
			今のところはその正体は伏せておいて,とりあえず対象は予め定められているものとして話を進めます.
			また文字は対象ではないということも認めます.説明中は``$s$を対象とする''のように書くことが多いですが,
			これは一時的に$s$を対象の一つに代用しているだけで,文字$s$が対象であると言っているのではありません.
			このような代用記号のことを{\bf 超記号}と呼びます.対象のみならず式にも超記号を宣言することが多いです.」
	}
	
	対象および文字を{\bf 項}\index{こう@項}{\bf (term)}と呼び,
	対象を用いて作られていた式は対象を項に替えても式と呼ぶことにする.
	
	$A$を式とし(上述の通り$A$とは超記号である)
	,$A$の中に文字$x$が現れるとき,`$\forall x A$'や`$\exists x A$'と書けば新しい記号列が得られる.
	このとき文字$x$は`$\forall x A$'で,或は`$\exists x A$'で{\bf 量化されている}\index{りょうか@量化}{\bf (quantified)}という.
	
	項と式の構成法を形式的に書き直すと次のようになる.
	\begin{description}
		\item[項] 言語$\mathcal{L}$の対象は$\mathcal{L}$の項であり,
			文字も$\mathcal{L}$の項である.
			またそれらのみが$\mathcal{L}$の項である.
			
		\item[式] 
			\begin{itemize}
				\item `$\bot$'は$\mathcal{L}$の式である.
				
				\item $s,t$を項とするとき,`$s=t$'と`$s \in t$'はどちらも式である.
					
				\item $A,B$を式とするとき,
					$A$では量化されていないが$B$で量化されているといった文字が無いときに限り,
					`$(A) \wedge (B)$',`$(A) \vee (B)$',`$(A) \Longrightarrow (B)$'はいずれも式である.
				
				\item $A$を式とするとき,`$\rightharpoondown (A)$'は式である.
				
				\item $A$を式とし,$x$を$A$に現れる文字とするとき,$x$が$A$で量化されていないときに限り
					`$\forall x (A)$'と`$\exists x (A)$'はどちらも式である.
				
				\item 以上の操作を繰り返して得られる記号列のみが式である.
			\end{itemize}
	\end{description}
	
	\monologue{
		院生「`$A$では量化されていないが$B$で量化されているといった文字が無いときに限り'という
			制限は何のためにあるのでしょうか.例えばこの制限を外すと
			\begin{align}
				\forall x ((x \in x) \vee (\forall y (\exists x ( y = x ))))
			\end{align}
			は式となりますが,同じ式で文字$x$は二回量化されています.
			これでは式を解釈するときに厄介なので,このような状況を排除するために制約を設けているのですね.
			では,`以上の操作を繰り返して得られる記号列のみが式である'はどういう意味でしょうか.
			例えば,最後の制限を外してしまうと
			\begin{align}
				\exists (\rightharpoondown (\exists x(\forall y (x = y))))
			\end{align}
			という記号列が式であるか式でないかは判別できませんが,
			最後の規制によりこれは式ではないと判断できます.具体的な問題を考える際は上の例のような式は扱わないので,
			本稿においては殆どご利益の無い形式上の規制となりますが,体系の完全性など考察する際には必要な規制でしょう.」
	}
	
	\monologue{
		院生「式の定義では,始めに最も簡単な形の式(`$\bot$'や`$s=t$')を提示して,
			以降の段階で新しい式を作り出す手段(論理記号による式の接合)を指定しています.
			このような定義を{\bf 帰納的な定義}\index{きのうてきなていぎ@帰納的な定義}{\bf (inductive definition)}と呼びます.
			プログラミングで言うところのfor文の操作と同じですね.」
	}
	
	$A$を式とする.
	`$t = a$'など$A$に文字$a$が現れるとき,$A$に現れる全ての文字$a$を文字$x$に置き換えた式を
	\begin{align}
		(x \mid a)\, A
	\end{align}
	で表す.特に$A$に現れる文字で量化されていないものが$a$のみであるとき,
	$(x \mid a)\, A$を
	\begin{align}
		A(x)
	\end{align}
	で表す.このとき式$A$自体は$(a \mid a)\, A$とも$A(a)$とも書ける.
	
	いま言語$\mathcal{L}$を設定したばかりであるが,例えば$x$のみが量化されていない式$A$に対して
	\begin{align}
		\Set{x}{A(x)}
	\end{align}
	という記法を導入し,これを対象として
	\begin{align}
		s \in \Set{x}{A(x)},\quad t = \Set{x}{A(x)}
	\end{align}
	のように式に組み込んで扱いたい.そこで$\mathcal{L}$を言語$\mathcal{L}'$に拡張する.
	$\mathcal{L}'$の使用文字,定数記号,述語記号,論理記号,量化記号,補助記号は$\mathcal{L}$のものをそのまま継承し,
	対象・項・式は次のように定める:
	
	\begin{description}
		\item[対象]
			\begin{itemize}
				\item $A$を式とし,$x$を$A$に現れる文字とするとき,$x$のみが$A$で量化されていないならば
					記号列`$\Set{x}{A(x)}$'は$\mathcal{L}'$の対象である.
					
				\item $\mathcal{L}$の対象は$\mathcal{L}'$の対象である.
				
				\item 以上のみが$\mathcal{L}'$の対象である.
			\end{itemize}
			
		\item[項] 言語$\mathcal{L}'$の対象は$\mathcal{L}'$の項であり,
			文字も$\mathcal{L}'$の項である.
			またそれらのみが$\mathcal{L}'$の項である.
	
		\item[式] 
			\begin{itemize}
				\item $\bot$は$\mathcal{L}'$の式である.
				
				\item $s,t$を$\mathcal{L}'$の項とするとき,`$s=t$'と`$s \in t$'はどちらも$\mathcal{L}'$の式である.
					
				\item $A,B$を$\mathcal{L}'$の式とするとき,
					$A$では量化されていないが$B$で量化されているといった文字が無いときに限り,
					`$(A) \wedge (B)$',`$(A) \vee (B)$',`$(A) \Longrightarrow (B)$'はいずれも$\mathcal{L}'$の式である.
				
				\item $A$を$\mathcal{L}'$の式とするとき,`$\rightharpoondown (A)$'は$\mathcal{L}'$の式である.
				
				\item $A$を$\mathcal{L}'$の式とし,$x$を$A$に現れる文字とするとき,$x$が$A$で量化されていないときに限り
					`$\forall x (A(x))$'と`$\exists x (A(x))$'はどちらも$\mathcal{L}'$の式である.
				
				\item 以上の操作を繰り返して得られる記号列のみが$\mathcal{L}'$の式である.
			\end{itemize}
	\end{description}
	
	\monologue{
		院生「$A$を$\mathcal{L}$の式とするとき,$x$が$A$に現れ,かつ$x$のみが$A$で量化されていないならば
			\begin{align}
				\Set{x}{A(x)}
			\end{align}
			は$\mathcal{L}'$の対象であると決めましたが,いま$A(x)$の中に文字$y$が現れないと仮定すれば,
			式$A(y)$には文字$y$が現れ,かつ$y$のみが量化されていないことになりますから
			\begin{align}
				\Set{y}{A(y)}
			\end{align}
			もまた$\mathcal{L}'$の対象となります.このとき`$\Set{x}{A(x)}$'と`$\Set{y}{A(y)}$'は$x$と$y$の違いを除いて同じ記号列になりますから,
			これらを同等な対象として扱いたいものです.同等とは等号で結ばれることですが,
			このことは後述する``類の公理''と``外延性の公理''により保証されます.」
	}
	
	$\mathcal{L}'$の式のうち,量化されていない文字を含まないものを{\bf 閉式}\index{へいしき@閉式}{\bf (closed formula)}と呼ぶ.
	定理として考察するものは全て閉式である.また数などの特別な対象や概念は{\bf 枠線付きの定義}により名前を付けていく.
	一度枠線付きの定義で名前を付けられた対象や概念は,それ以後は本稿においてその名前で通用する.
	他方,枠線付きの定義という手続きを踏まなくても,便宜のために説明や証明の途中で対象に名前を付けることがある.それは次のようなものである: 
	\begin{prf}
		$\cdots$いま$P \coloneqq \Set{x}{\forall t\ (\ x = t \vee x \in t\ )}$とおく.このとき$\cdots$
	\end{prf}
	記号`$\coloneqq$'は{\bf 定義記号}と呼ばれ,右辺の類に左辺の記号列で名前を付けるという意味で使われる.
	このような文言は多くの説明や証明に出てくるが,実際上の効果として,
	以後の式に出てくる`$\Set{x}{\forall t\ (\ x = t \vee x \in t\ )}$'の部分を`$P$'で置き換えられるようになる.
	ただしその場合の定義はその説明や証明の中でのみ通用するものと約束する.
	
	\begin{screen}
		\begin{dfn}[宇宙]
			$\Univ \coloneqq \Set{x}{x=x}$で定める$\Univ$を{\bf 宇宙}\index{うちゅう@宇宙}{\bf (Universe)}と呼ぶ.
		\end{dfn}
	\end{screen}
	
	\monologue{
		院生「$\Univ$はそもそも式に現れる文字ではありませんでしたが,$\coloneqq$の推薦で$\Set{x}{x=x}$の代理人として市民権を得たのですね.
			さて宇宙という壮大な言葉が出てきましたが,後述する通り$\Univ$は集合の全体のことですから,
			あらゆるものが集合で説明される現代数学にとって$\Univ$はまさしく宇宙なのですね.
			また定理\ref{thm:Universe_and_ordinal_numbers}で$\Univ$の実態が明らかになるでしょう.
			我々はこの定理で集合とは何者かという問いへの完全な答えを得ることになります.
			ところで,現実世界において人間が把握し得る最大の世界は宇宙空間でしょうが,
			数学の世界では宇宙の外側を見ることが出来るのです.そこは真類と呼ばれるものの世界です.
			実は宇宙そのものも真類の一つなのですが(宇宙が宇宙の外側に在るとは奇妙です),
			その話も後述にまかせましょう.ちなみに,宇宙が$\Univ$で表されるのはJohn Von NeumannのVに由来していると思われます.」
	}
	
	数学の式を日本語に翻訳するとき,慣習上よく使われる訳し方があるので列挙する.
	\begin{itemize}
		\item 式$a = b$を``$a$は$b$に等しい''や``$a$と$b$は等しい''と翻訳する.
		\item 式$a \in b$を``$a$は$b$の要素である''や``$a$は$b$に属する''と翻訳する.
		\item 式$(A) \Longrightarrow (B)$を``$A$が成り立つならば$B$が成り立つ''と翻訳する.
		\item 式$\rightharpoondown (A)$を%``$A$でない''と翻訳する.
	\end{itemize}
	
	\begin{screen}
		\begin{dfn}[類・集合]
			$\mathcal{L}'$の対象のことを{\bf 類}\index{るい@類}{\bf (class)}と呼び,
			特に$\Univ$の要素である類を{\bf 集合}\index{しゅうごう@集合}{\bf (set)}と呼ぶ.
			また$\Univ$の要素でない類のことを{\bf 真類}\index{しんるい@真類}{\bf (proper class)}と呼ぶ.
		\end{dfn}
	\end{screen}

	\monologue{
		院生「類は$\Univ$の要素であれば集合と呼ばれ,$\Univ$の要素でなければ真類と呼ばれます.
			では集合であり真類でもある類や,集合でも真類でもない類はあるのでしょうか?
			答えは``現段階では確定したことは何も言えない''です.
			質問を変えましょう.集合であり真類でもある類や集合でも真類でもない類の存在を禁止するにはどうしたら良いでしょうか?
			我々は,数学において中庸が無いということや矛盾が起きるということをどう表現しようかという問題に直面しているのです.
			この問題の解決への方便として{\bf 推論規則}\index{すいろんきそく@推論規則}
			{\bf (rule of inference)}と呼ばれるものを導入します.」
	}
	
	\begin{screen}
		\begin{metaaxm}[排中律]
			$A$を$\mathcal{L}'$の閉式とするとき次が成り立つ:
			\begin{align}
				A \vee \rightharpoondown A.
			\end{align}
		\end{metaaxm}
	\end{screen}
	
	いま$a,b$を類とするとき,
	\begin{align}
		a \notin b \overset{\mathrm{def}}{\Longleftrightarrow}\ \rightharpoondown a \in b
	\end{align}
	で$a \notin b$を定める.同様に
	\begin{align}
		a \neq b \overset{\mathrm{def}}{\Longleftrightarrow}\ \rightharpoondown a = b
	\end{align}
	で$a \neq b$を定める.
	
	\monologue{
		院生「定義記号$\coloneqq$と同様に,`$A \overset{\mathrm{def}}{\Longleftrightarrow} B$'とは
			式$B$を記号列$A$で置き換えて良いという意味で使われます.また,式中に記号列$A$が出てくるときは,
			暗黙裡にその$A$を$B$に戻して式を解釈します.
			$\coloneqq$も$\overset{\mathrm{def}}{\Longleftrightarrow}$も略記することと同じですね.」
	}
	
	\begin{screen}
		\begin{thm}[集合でも真類でもない類は存在しない]
			$a$を類とするとき次が成り立つ:
			\begin{align}
				\rightharpoondown (\ a \in \Univ \wedge a \notin \Univ\ ).
			\end{align}
		\end{thm}
	\end{screen}
	
	\begin{screen}
		\begin{axm}[外延性の公理]
			$a,b$を類とするとき,次が成り立つ:
			\begin{align}
				\forall t\ (\ t \in a  \Longleftrightarrow t \in b\ )
				\Longrightarrow a=b.
			\end{align}
		\end{axm}
	\end{screen}
	
	\begin{screen}
		\begin{thm}[任意の類は自分自身と等しい]\label{thm:any_class_equals_to_itself}
			$a$を類とするとき次が成り立つ:
			\begin{align}
				a = a.
			\end{align}
		\end{thm}
	\end{screen}
	
	\begin{prf}
		$\mathcal{L}$の任意の対象$\tau$に対して
		\begin{align}
			\tau \in a \Longleftrightarrow \tau \in a
		\end{align}
		となるから
		\begin{align}
			\forall t\ (\ t \in a  \Longleftrightarrow t \in b\ )
		\end{align}
		が成り立つ.従って外延性の公理より$a = a$が出る.
		\QED
	\end{prf}
	
	\begin{screen}
		\begin{axm}[類の公理]\mbox{}
			\begin{description}
				\item[(i)] $a,b$を類とするとき,次が成り立つ:
					\begin{align}
						a \in b \Longrightarrow \exists x\ (\ a = x\ ).
					\end{align}
				
				\item[(ii)] 
					\begin{align}
						\forall t\ (\ t \in \Set{x}{A(x)} \Longleftrightarrow A(t)\ ).
					\end{align}
			\end{description}
		\end{axm}
	\end{screen}
	
	\begin{screen}
		\begin{thm}[$\mathcal{L}$の対象も$\Set{x}{A(x)}$の形で表せる]
			次が成り立つ:
			\begin{align}
				\forall t\ (\ t = \Set{x}{x \in t}\ ).
			\end{align}
		\end{thm}
	\end{screen}
	
	\begin{prf}
		$\tau$を$\mathcal{L}$の任意の対象とすると類の公理より
		\begin{align}
			\forall s\ (\ s \in \tau \Longleftrightarrow s \in \Set{x}{x \in \tau}\ )
		\end{align}
		が成り立つが,このとき外延性の公理より
		\begin{align}
			\tau = \Set{x}{x \in \tau}
		\end{align}
		が従う.$\tau$の任意性より
		\begin{align}
			\forall t\ (\ t = \Set{x}{x \in t}\ )
		\end{align}
		を得る.
		\QED
	\end{prf}
	
	\begin{screen}
		\begin{axm}[相等性の公理]
			$a,b,c$を類とするとき以下は全て公理である:
			\begin{description}
				\item[(i)] $a=b \Longrightarrow (\ c \in a \Longleftrightarrow c \in b\ ).$
				\item[(ii)] $a=b \Longrightarrow (\ c = a \Longleftrightarrow c = b\ ).$
				\item[(iii)] $a=b \Longrightarrow (\ a \in c \Longleftrightarrow b \in c\ ).$
			\end{description}
		\end{axm}
	\end{screen}
	
	\begin{screen}
		\begin{thm}[外延性の公理は同値関係で成立する]\label{thm:axiom_of_extensionality_equivalent}
			$a,b$を類とするとき,次が成り立つ:
			\begin{align}
				\forall t\ (\ t \in a  \Longleftrightarrow t \in b\ )
				\Longleftrightarrow a=b.
			\end{align}
		\end{thm}
	\end{screen}
	
	\begin{prf}
		外延性の公理より$\forall t\ (\ t \in a  \Longrightarrow t \in b\ ) \Longrightarrow a=b$は成り立つので,ここでは
		\begin{align}
			a = b \Longrightarrow \forall t\ (\ t \in a  \Longrightarrow t \in b\ )
			\label{eq:thm_axiom_of_extensionality_equivalent_1}
		\end{align}
		が成り立つことを示す.実際,$a = b$であるとき,相等性の公理より$\mathcal{L}$の任意の対象$\tau$に対して
		\begin{align}
			\tau \in a \Longleftrightarrow \tau \in b
		\end{align}
		が満たされるから
		\begin{align}
			\forall t\ (\ t \in a  \Longrightarrow t \in b\ )
		\end{align}
		が成立する.よって(\refeq{eq:thm_axiom_of_extensionality_equivalent_1})が出る.
		\QED
	\end{prf}
	
	\monologue{
		院生「この定理では等しい類同士は同じ$\mathcal{L}$の対象を要素に持つと示されましたが,
			実はこのとき要素に持つ類まで一致しているのです.そのことは部分類の箇所で説明いたしましょう.」
	}
	
	\begin{screen}
		\begin{thm}[集合とは,$\mathcal{L}$の対象か,$\mathcal{L}$の対象と等しい類のことを指す]
		\label{thm:every_set_is_equivalent_to_some_individual_in_L}
			$a$を類とするとき次が成り立つ:
			\begin{align}
				\exists x\ (\ a = x\ ) \Longleftrightarrow a \in \Univ.
			\end{align}
		\end{thm}
	\end{screen}
	
	\begin{prf}
		$a$を類とするとき,まず類の公理より
		\begin{align}
			a \in \Univ \Longrightarrow \exists x\ (\ a = x\ )
		\end{align}
		が得られる.逆に$\exists x\ (\ a = x\ )$が成り立っていると仮定する.
		\begin{align}
			\tau \coloneqq \varepsilon x (a = x)
		\end{align}
		とおけば,定理\ref{thm:any_class_equals_to_itself}より$\tau = \tau$となるので
		\begin{align}
			\tau \in \Univ
		\end{align}
		が成り立つ.そして相等性の公理より
		\begin{align}
			a \in \Univ
		\end{align}
		が従うから$\exists x\ (\ a = x\ ) \Longrightarrow a \in \Univ$も得られる.
		\QED
	\end{prf}
	
	\monologue{
		院生「集合は$\mathcal{L}$の対象であるとは限りません.例えば
			$\mathcal{L}$の対象$\tau$に対して$\Set{x}{x \in \tau}$は集合ですが$\mathcal{L}$の対象ではありませんね.
			また類の公理の一つ目の主張は{\bf ``要素となりうる類は集合である''}と言い換えられますね.」
	}
	
	\begin{screen}
		\begin{axm}[空集合の公理]
			次は公理である:
			\begin{align}
				\forall x\ (\ x \notin \emptyset\ ).
			\end{align}
			つまり$\emptyset$は$\mathcal{L}$のいかなる対象も要素に持たない.
			$\emptyset$を{\bf 空集合}\index{くうしゅうごう@空集合}{\bf (empty set)}と呼ぶ.
		\end{axm}
	\end{screen}
	
	\monologue{
		院生「空集合は集合の系譜の起点となります.聖書物語でいうところのアダムです.」
	}
	
	\begin{screen}
		\begin{thm}[$\mathcal{L}$のいかなる対象も要素に持たない類は空集合に等しい]
			$a$を類とするとき次が成り立つ:
			\begin{align}
				\forall x\ (\ x \notin a\ ) \Longrightarrow a = \emptyset.
			\end{align}
		\end{thm}
	\end{screen}
	
	\begin{screen}
		\begin{thm}[空集合はいかなる類も要素に持たない]
			$a$を類とするとき次が成り立つ:
			\begin{align}
				a \notin \emptyset.
			\end{align}
		\end{thm}
	\end{screen}
	
	\begin{prf}
		$a,b$を類とするとき
		\begin{align}
			a \in b \Longrightarrow \exists x\ (\ x \in b\ )
		\end{align}
		が成り立つから,対偶を取れば
		\begin{align}
			\forall x\ (\ x \notin b\ ) \Longrightarrow a \notin b
		\end{align}
		が成り立つ.$b$を$\emptyset$に置き換えれば$\forall x\ (\ x \notin \emptyset\ )$は真であるから
		$a \notin \emptyset$が従う.
		\QED
	\end{prf}
	
	\begin{screen}
		\begin{dfn}[部分類]
			$a,b$を類とするとき,
			\begin{align}
				a \subset b \overset{\mathrm{def}}{\Longleftrightarrow}
				\forall x\ (\ x \in a \Longrightarrow x \in b\ )
			\end{align}
			で記号列$a \subset b$を定める.また類$a,b$に対して$a \subset b$が成り立っているとき
			$a$を$b$の{\bf 部分類}\index{ぶぶんるい@部分類}{\bf (subclass)}と呼び,
			特に$a$が集合である場合は$a$を$b$の{\bf 部分集合}\index{ぶぶんしゅうごう@部分集合}{\bf (subset)}と呼ぶ.
			また類$a,b$に対して
			\begin{align}
				a \subsetneq b \overset{\mathrm{def}}{\Longleftrightarrow}
				a \subset b \wedge a \neq b
			\end{align}
			と定める.
		\end{dfn}
	\end{screen}
	
	\begin{screen}
		\begin{thm}[空集合は全ての類に含まれる]
			$a$を類とするとき次が成り立つ:
			\begin{align}
				\emptyset \subset a.
			\end{align}
		\end{thm}
	\end{screen}
	
	\begin{prf}
		$a$を類とする.$\tau$を$\mathcal{L}$の任意の対象とすれば,$\tau \notin \emptyset$が成り立つから
		\begin{align}
			\tau \notin \emptyset \vee \tau \in a
		\end{align}
		が成り立つ.これは$\tau \in \emptyset \Longrightarrow \tau \in a$が成り立つことと同値であるから,
		$\tau$の任意性より$\emptyset \subset a$が得られる.
		\QED
	\end{prf}
	
	\monologue{
		院生「この結果は空虚な真の一例ですね.」
	}
	
	\begin{screen}
		\begin{thm}[類はその部分類に属する全ての類を要素に持つ]\label{thm:subclass_contains_all_elements}
			$a,b$を類として,$a$を$b$の部分類とする.このとき,任意の類$c$に対して次が成り立つ:
			\begin{align}
				c \in a \Longrightarrow c \in b.
			\end{align}
		\end{thm}
	\end{screen}
	
	\begin{prf}
		$c \in a$が成り立っていると仮定すれば$\exists x\ (\ c = x\ )$が成り立つ.このとき
		\begin{align}
			\tau \coloneqq \varepsilon x(c=x)
		\end{align}
		とおくと,相等性の公理より$\tau \in a$となり,
		$a \subset b$より$\tau \in b$となり,再び相等性の公理より$c \in b$となるから
		\begin{align}
			c \in a \Longrightarrow c \in b
		\end{align}
		が得られる.
		\QED
	\end{prf}
	
	\begin{screen}
		\begin{thm}[$\Univ$は最大の類である]
			$a$を類とするとき次が成り立つ:
			\begin{align}
				a \subset \Univ.
			\end{align}
		\end{thm}
	\end{screen}
	
	\begin{prf}
		$\Univ$は$\mathcal{L}$の対象の全てを要素として持つから
		\begin{align}
			\forall x\ (\ x \in a \Longrightarrow x \in \Univ\ )
		\end{align}
		が成り立つ.
		\QED
	\end{prf}
	
	\begin{screen}
		\begin{thm}[互いに互いの部分類となる類同士は等しい]\label{thm:mutually_sub_classes_are_equivalent}
			$a,b$を類とするとき次が成り立つ:
			\begin{align}
				a \subset b \wedge b \subset a \Longleftrightarrow a = b.
			\end{align}
		\end{thm}
	\end{screen}
	
	\begin{prf}
		$a \subset b \wedge b \subset a$が成り立っていると仮定する.
		このとき$\mathcal{L}$の任意の対象$\tau$に対して
		\begin{align}
			\tau \in a \Longleftrightarrow \tau \in b
		\end{align}
		が成り立つから$a = b$が従う.逆に$a = b$が成り立っていると仮定すれば,
		$\mathcal{L}$の任意の対象$\tau$に対して
		\begin{align}
			\tau \in a \Longleftrightarrow \tau \in b
		\end{align}
		が成り立つので$a \subset b$と$b \subset a$が共に従う.
		\QED
	\end{prf}
	
	\monologue{
		院生「定理\ref{thm:subclass_contains_all_elements}と定理\ref{thm:mutually_sub_classes_are_equivalent}より,
			類$a,b$が$a = b$を満たすならば,$a$と$b$は要素に持つ$\mathcal{L}$の対象のみならず,
			要素に持つ類までも一致するのですね.これは
			\begin{itemize}
				\item 要素となりうる類は集合である
				\item 集合は$\mathcal{L}$の或る要素に等しい
			\end{itemize}
			からの帰結です.」
	}
	
	\begin{screen}
		\begin{dfn}[合併]
			$a$を類とするとき
			\begin{align}
				\bigcup a \coloneqq \Set{x}{\exists t\ (\ \varepsilon a(t) \wedge x \in t\ )}
			\end{align}
			で$\bigcup a$を定め,これを$a$の{\bf 合併}\index{がっぺい@合併}{\bf (union)}と呼ぶ.
		\end{dfn}
	\end{screen}
	
	$a$を類とするとき,
	\begin{align}
		\forall x\ \left(\ x \in \bigcup a \Longleftrightarrow \exists t \in a\ (\ x \in t\ )\ \right)
	\end{align}
	が成立する.実際,$\varepsilon$記号についての規則と類の公理から
	\begin{align}
		\forall t\ \left(\ \varepsilon a(t) \Longleftrightarrow t \in a\ \right)
	\end{align}
	が満たされるので,$x$を$\mathcal{L}$の任意の対象とすれば
	\begin{align}
		\exists t\ (\ \varepsilon a(t) \wedge x \in t\ )
		\Longleftrightarrow \exists t\ (\ t \in a \wedge x \in t\ )
	\end{align}
	となる.これを根拠にして,$\bigcup a$を
	\begin{align}
		\Set{x}{\exists t \in a\ (\ x \in t\ )}
	\end{align}
	で表しても良いことにする.もとより,合併の定義はこちらの表記を正当化することを予定したものである.
	
	\monologue{
		院生「一般の類$a$に対して,本来
			\begin{align}
				\Set{x}{\exists t \in a\ (\ x \in t\ )}
			\end{align}
			は類として失格です.なぜならば,$a$が$\mathcal{L}$の対象ではない場合はそもそも
			\begin{align}
				\exists t \in a\ (\ x \in t\ )
			\end{align}
			が$\mathcal{L}$の式でないからです.しかしいちいち$\varepsilon$記号を使っていては見た目が煩雑になりますから,表記上は
			\begin{align}
				\Set{x}{\exists t \in a\ (\ x \in t\ )}
			\end{align}
			も認めるのです.以後もこのように妥協する場面に直面するでしょう.
			しかし$\varepsilon$記号のおかげで解釈上の不具合は無いのです.」
	}
	
	\begin{screen}
		\begin{axm}[合併の公理]
			集合の合併は集合である.つまり,$a$を類とするとき次が成り立つ:
			\begin{align}
				a \in \Univ \Longrightarrow \bigcup a \in \Univ.
			\end{align}
		\end{axm}
	\end{screen}
	
	\begin{screen}
		\begin{thm}[合併の性質]\mbox{}
			\begin{description}
				\item[(1)] $\bigcup \emptyset = \emptyset$
				\item[(2)] $\forall x\ (\ x \in a \cup b \Longleftrightarrow x \in a \vee x \in b\ )$
				\item[(3)] $\Set{x}{A(x)} \cup \Set{x}{\rightharpoondown A(x)} = \Univ$.
			\end{description}
		\end{thm}
	\end{screen}
	
	\begin{screen}
		\begin{dfn}[交叉]
			$a$を類とするとき,
			\begin{align}
				\bigcap a \coloneqq \Set{x}{\forall t \in a\ (\ x \in t\ )}
			\end{align}
			で$\bigcap a$を定め,これを$a$の{\bf 交叉}\index{こうさ@交叉}{\bf (intersection)}と呼ぶ.
		\end{dfn}
	\end{screen}
	
	\begin{screen}
		\begin{thm}[交叉の性質]\mbox{}
			\begin{description}
				\item[(1)] $\bigcap \emptyset = \Univ$
				\item[(2)] $\forall x\ (\ x \in a \cap b \Longleftrightarrow x \in a \wedge x \in b\ )$
				\item[(3)] $\Set{x}{A(x)} \cap \Set{x}{\rightharpoondown A(x)} = \emptyset$.
			\end{description}
		\end{thm}
	\end{screen}
	
	\begin{prf}
		$x$を$\mathcal{L}$の任意の対象とするとき,空虚な真より
		\begin{align}
			t \in \emptyset \Longrightarrow x \in t
		\end{align}
		は$\mathcal{L}$のいかなる対象$t$に対してもに真となる.ゆえに$\forall t \in \emptyset\ (\ x \in t\ )$が成立し
		\begin{align}
			\forall x\ (\ x \in \bigcap \emptyset\ )
		\end{align}
		が従う.$\forall x\ (\ x \in \Univ\ )$と併せて$\bigcap \emptyset = \Univ$を得る.
		\QED
	\end{prf}
	
	\monologue{
		院生「$\bigcup \emptyset$が$\emptyset$に等しいのは受け容れられますが,
			$\bigcap \emptyset$が$\Univ$に等しいというのは直感に合いません.空虚な真おそるべしです.」
	}
	
	\begin{screen}
		\begin{thm}
			
		\end{thm}
	\end{screen}
	
	\begin{prf}\mbox{}
		\begin{description}
			\item[(1)] $a^{-1}$の任意の要素$t$に対し或る$V$の要素$x,y$が存在して
				\begin{align}
					(x,y) \in a \wedge t = (y,x)
				\end{align}
				を満たす.$((x,y),(y,x)) \in f$より$((x,y),t) \in f$が成り立つから
				$t \in f \ast a$となる.逆に$f \ast a$の任意の要素$t$に対して
				$a$の或る要素$x$が存在して
				\begin{align}
					x \in a \wedge (x,t) \in f
				\end{align}
				となる.$x$に対し$V$の或る要素$a,b$が存在して$x=(a,b)$となるので
				\begin{align}
					((a,b),t) \in f
				\end{align}
				となり,$V$の或る要素$c,d$が存在して
				\begin{align}
					((a,b),t) = ((c,d),(d,c))
				\end{align}
				となる.$(a,b) = (c,d)$より$a=c$かつ$b=d$となり,
				$t = (d,c)$かつ$(d,c)=(b,a)$より$t=(b,a)$,従って
				$t \in a^{-1}$が成り立つ.
		\end{description}
	\end{prf}
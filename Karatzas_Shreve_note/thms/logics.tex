	\begin{quote}
		初めに言があった。言は神と共にあった。言は神であった。\\
		この言は、初めに神と共にあった。\\
		万物は言によって成った。成ったもので、言によらずに成ったものは何一つなかった。
	\end{quote}
	ヨハネによる福音書の冒頭である。言と書いてことばと読む.なぜ``言葉''ではなく``言''と書くのだろうか.
	一説によれば言葉の葉の字の由来は万葉古今集仮名序にあり,
	現代的に表現すれば,見聞きしたり感動したりしたところを種にして生じる語彙のことを木の葉に喩えているらしい.
	ならばすなわち,葉が付かない``言''とは人為の介入する前から世界を認識している
	と解釈すれば良いのであろうか.一気に言の字が神格を帯びてくるが,解釈としてはあながち見当外れでもないらしく,
	上の引用文の通り%は森羅万象はことばによって成り,ことばによって尽くされるという意味であるから,
	キリスト教においてことばとはこの宇宙の悉くを超越しているのである.
	\begin{comment}
		実際に自然言語の発生が事物の観測なしに起こり得たかという問題は言語哲学上も決着がついていないらしいが,
		少なくとも
	\end{comment}
	そして本稿の世界もまた数学のことば,言い換えれば論理のみによって創られるという点でキリスト教的である.
	しかしながら,論理のみによっては完結し得ないという点でキリスト教の世界観と決定的に違っている.
	
	
	\begin{description}
		\item[定数] $\emptyset$
		\item[変数] $a,b,c,\cdots,x,y,z,\cdots,\alpha,\beta,\gamma,\cdots,A,B,C,\cdots,$
			ただし変数の対象領域を$V$と表す.
		\item[述語] $=,\ \Longrightarrow,\ \Longleftarrow,\ \Longleftrightarrow,
			\ \wedge,\ \vee,\ \rightharpoondown$
	\end{description}
	
	我々は言語$L$の論理式という記号の列により世界を創造しているのである.
	解釈は後付けという立場であるから,記号の使用履歴から来る先入観は排除しなければならない.
	\begin{screen}
		\begin{dfn}[集合]
			言語$L$の対象式を{\bf 集合}\index{しゅうごう@集合}{\bf (set)}と呼ぶ.
			ただし$V$に属さないものは集合とは呼ばない.
		\end{dfn}
	\end{screen}
	
	\begin{screen}
		\begin{axm}[等号の公理]
			述語記号$=$を{\bf 等号}\index{とうごう@等号}{\bf (equal sign)}と呼ぶが,
			等号について次が成り立つ:
			\begin{align}
				\forall x (x \in V \Longleftrightarrow x=x).
			\end{align}
		\end{axm}
	\end{screen}
	
	\begin{screen}
		\begin{dfn}[クラス]
			$x$のみを自由変数とする論理式を$A(x)$と表すとき,
			$\Set{x}{A(x)}$なる形式の記号列を{\bf クラス}\index{くらす@クラス}{\bf (class)}と呼び,$y \in \Set{x}{A(x)}$を
			\begin{align}
				y \in \Set{x}{A(x)} \Longleftrightarrow A(y)
			\end{align}
			で解釈する.
		\end{dfn}
	\end{screen}
	
	\begin{screen}
		\begin{axm}[命題論理の公理]\mbox{}
			\begin{description}
				\item[(1)] 任意の命題$A,B$に対し,
					$A $
				\item[(1)] 証明可能な命題は真である.
				\item[(2)] 任意の命題$A,B$に対し,$A$も$A \Longrightarrow B$も真であるとき$B$は真である.
				\item[(3)] 任意の命題$A$に対し,$\rightharpoondown A$が真であるとき$A$は偽である.
			\end{description}
		\end{axm}
	\end{screen}
	
	\begin{screen}
		\begin{thm}
			$A$を任意の命題とするとき,
			\begin{align}
				\mbox{$\rightharpoondown A$が真である} \Longleftrightarrow \mbox{$A$が偽である}.
			\end{align}
		\end{thm}
	\end{screen}
	\monologue{
		院生「現代的な数学では,数や関数など数学に関するあらゆるものは集合で構成されます.
			そして集合そのものは述語論理を基礎にして公理的に規定されます.
			この意味で集合論の勉強には論理学の知識が必要であると聞きますけれども,
			真に受けて論理学の本を眺めてみれば,はじめから集合そのものが出てきたり,
			変数に数で添え字をつけたり,述語関数などといったものを取り扱っていたりしているものばかりで残念です.
			論理学を基に集合論を展開しようというのですから,集合論の諸概念を予定して論理学を説明するのは本末転倒です.
			とはいえ集合論と論理学とは切っても切り離せないのですから,いっそ同時並行でそつなく理解してやりましょう.
			(いわゆるメタ数学についてはいまのところ手を出すつもりはありません.)」
	}
\subsection{言語}
	\begin{quote}
		初めに言(ことば)があった。言は神と共にあった。言は神であった。\\
		この言は、初めに神と共にあった。\\
		万物は言によって成った。成ったもので、言によらずに成ったものは何一つなかった。
	\end{quote}
	ヨハネによる福音書の冒頭である.本稿の世界もまた数学のことば,言い換えれば論理のみによって創られる(予定).

	\monologue{
		院生「私の指導教官に``新約聖書がはじめにギリシア語で書かれたとき,`ことば'にはlogosが充てられた.
			logosは`言語'の意味を持つと同時に`論理'の意味も持つ''と教わりました.
			つまり,ギリシア語版の福音書では``初めに論理があった''とも解釈できるのですね.
			一方で日本語訳では言葉ではなく言と書かれています.なぜ``言葉''ではなく``言''と書くのでしょうか.
			一説によれば言葉の葉の字の由来は万葉古今集仮名序にあり,
			現代的に説明すれば,見聞きしたり感動したりしたところを種にして生じる語彙のことを木の葉に喩えているらしいです.
			言葉は人が発するものであり,たいていの場合食い違いなく通用するのですから,すなわち
			葉が付かない``言''とは,人為の介入する前から世界を認識し,人が自覚する前から人の心に通底している
			コードとでも解釈されるでしょうか.聖書の引用文の通り%は森羅万象はことばによって成り,ことばによって尽くされるという意味であるから,
			キリスト教においてことばとは神であり森羅万象を超越しているのですから,言の字に神性を伴わせても良いですよね.
			本稿の世界もまた数学のことばによって創られますが,``はじめにことばありき''の名句が国籍や文化を問わず
			現代まで受け入れられてきたという事実を鑑みれば,ことばから始めようというのは人が生来持っている直観に対して自然な起こりなのでしょう.」
			%しかしながら,神なることばが世界の悉くを尽くせる一方で,人が創造する数学の世界は論理のみによっては完結し得ないという事実もあります.
	}
	
	\monologue{
		院生「集合論の言語の設定は思いの外厄介ですね.いや,私にとって厄介というだけですが.
			一旦言語を設定してはみるものの,行き詰れば設定をやり直すことになりますから,
			以下記述する内容はあくまで仮の形です.それから,私自身集合論も論理学も
			ド素人ですから,言語に対する認識が専門家とズレていることも十分あり得ます.
			勉強を進める中で自分の誤解に気付けばその時点で全てやり直しです.
			見る人が見れば滑稽千万な破綻が見つかるかもしれませんが,
			しかし私としてはHilbertの形式主義,つまり文字と特殊記号を一定の法則で並べただけの
			無意味な記号列に対して推論規則や公理により形式上の意味を付けるという姿勢を
			貫いているつもりです.予防的な言い訳はこの程度にして,本論に入りましょう.」
	}
	
	言語における使用文字と特殊記号は以下に指定するものである:
	\begin{description}
		\item[使用文字] 自然言語から借用する文字は表にあるものに限る.
		\item[述語記号] $=,\ \in$
		\item[論理記号] $\bot,\ \Longrightarrow,\ \wedge,\ \vee,\ \rightharpoondown$
		\item[量化記号] $\forall,\ \exists$
		\item[補助記号] $[\ ,\ ]\ ,\ (\ ,\ )\ ,\ \{\ ,\ \}\ ,\ <\ ,\ >\ ,\ |$
	\end{description}
	
	日常言語において,``あmt後右所sごぐふぉsdあじお''のように無作為に文字を並べただけでは意味不明な
	文字列が出来上がる.文字列は,何らかの規則に従って並ぶことで単語や文章として成立するのである.
	数学も同じで,一定の規則に従って並ぶ記号列のみを数学における文章として扱う.
	述語記号とは,今のところは文字同士を繋ぎ最小単位の文章を成すものとする.例えば,文字$s,t$に対し
	\begin{align}
		s \in t
	\end{align}
	は数学の文章となり,日本語には``$s$は$t$の要素である''と翻訳される.
	数学における文章を{\bf 式}\index{しき@式}
	{\bf (formula)}或は{\bf 論理式}\index{ろんりしき@論理式}と呼ぶ.
	論理記号とは式同士を繋ぐ役割を持つ.
	
	式の構成法を形式的に書き直すと次のようになる.
	\begin{description}
		\item[式] 
			\begin{itemize}
				\item $\bot$は$\mathcal{L}$の式である.
				
				\item 文字$s,t$に対して,$s=t,\ s \in t$は式である.
					
				\item $A,B$を式とするとき,
					$A$では量化されていないが$B$で量化されているといった文字が無いときに限り,
					$(A) \wedge (B),\ (A) \vee (B),\ (A)\Longrightarrow (B)$は式である.
				
				\item $A$を式とするとき,$\rightharpoondown (A)$は式である.
				
				\item $A$を式とするとき,文字$x$が$A$に現れ,かつ$x$が$A$で量化されていないときに限り
					$\forall x (A),\ \exists x (A)$は式である.
				
				\item 以上の操作を繰り返して得られる記号列のみが式である.
					ただし,繰り返しの操作は無制限に行われるものではない.
					得られる記号列は左端から辿っていけば必ず右端が見つかるものとする.
			\end{itemize}
	\end{description}
	
	\monologue{
		院生「`$A$では量化されていないが$B$で量化されているといった文字が無いときに限り'という
			制限は何のためにあるのでしょうか.例えばこの制限を外すと
			\begin{align}
				\forall x ((x \in x) \vee (\forall y (\exists x ( y = x ))))
			\end{align}
			は式となりますが,同じ式で文字$x$は二回量化されています.
			同じ文字が複数回量化されてしまうと式を解釈するときに厄介なので,
			そのような状況を排除するために制限を課しているのですね.
			では,`以上の操作を繰り返して得られる記号列のみが式である'はどういう意味でしょうか.
			例えば,最後の制限を外してしまうと
			\begin{align}
				\exists (\rightharpoondown (\exists x(\forall y (x = y))))
			\end{align}
			という記号列が式であるか式でないかは判別できませんが,
			最後の規制によりこれは式ではないと判断できます.
			また`得られる記号列は左端から辿っていけば必ず右端が見つかるものとする'というのは,
			式の長さは有限であるということを伝えているのですね.しかし未だ有限とは何かを
			規定していないのでこう書くほか術が見当たらないのです.」
	}
	
	\monologue{
		院生「式の定義では,始めに最も簡単な形の式($\bot$や$s=t$)を提示して,
			以降の段階で新しい式を作り出す手段(論理記号による式の接合)を指定しています.
			このような定義を{\bf 帰納的な定義}\index{きのうてきなていぎ@帰納的な定義}{\bf (inductive definition)}と呼びます.
			プログラミングで言うところのfor文の操作と同じですね.
			また既に量化されている文字が再び量化されるということは起こり得ません.」
	}
	
	$A$を式とする.
	$A$に$a$という文字が現れるとき,$A$に現れる全ての$a$を$x$に置き換えた式を
	\begin{align}
		(x \mid a)\, A
	\end{align}
	で表す.特に$A$に現れる文字で量化されていないものが$a$のみであるとき,
	$(x \mid a)\, A$を
	\begin{align}
		A(x)
	\end{align}
	で表す.このとき式$A$自体は$(a \mid a)\, A$とも$A(a)$とも書ける.
	
	いま文字から成る式を作ったが,例えば$x$のみが量化されていない式$A$に対して
	\begin{align}
		\Set{x}{A(x)}
	\end{align}
	という記法を導入し,これも文字同様に
	\begin{align}
		s \in \Set{x}{A(x)},\quad t = \Set{x}{A(x)}
	\end{align}
	などと式に組み込んで扱いたい.そこで{\bf 対象}\index{たいしょう@対象}{\bf (individual)}
	と{\bf 項}\index{こう@項}{\bf (term)}という概念を使う.
	
	\begin{description}
		\item[対象]
			\begin{itemize}
				\item 式$A$において文字$x$が現れ,かつ$x$のみが$A$で量化されていないとき,
					\begin{align}
						\Set{x}{A(x)}
					\end{align}
					は対象である.
					
				\item 式$A$において文字$x$が現れ,かつ$x$のみが$A$で量化されていないとき,
					\begin{align}
						\varepsilon x A(x)
					\end{align}
					は対象である.
			\end{itemize}
			
		\item[項] 対象は項である.文字も項である.またこれらのみが項である.
	\end{description}
	
	\monologue{
		院生「唐突に出てきた$\varepsilon x A(x)$という記号列は何なのでしょう.
			後述することですが,これは$\forall$と$\exists$の意味を
			公理化するための方便として導入するものなのです.ちなみに,以下で式を拡張することにより
			\begin{align}
				A(\varepsilon x A(x))
			\end{align}
			という形の記号列も式として扱えることになります.」
	}
	
	項を用いて,先ほど定義した式を拡張する.
	\begin{description}
		\item[式] 
			\begin{itemize}
				\item $\bot$は$\mathcal{L}$の式である.
				
				\item $s,t$を項とするとき,$s=t,\ s \in t$は式である.
					
				\item $A,B$を式とするとき,
					$A$では量化されていないが$B$で量化されているといった文字が無いときに限り,
					$(A) \wedge (B),\ (A) \vee (B),\ (A)\Longrightarrow (B)$は式である.
				
				\item $A$を式とするとき,$\rightharpoondown (A)$は式である.
				
				\item $A$を式とするとき,文字$x$が$A$に現れ,かつ$x$が$A$で量化されていないときに限り
					$\forall x (A),\ \exists x (A)$は式である.
				
				\item 以上の操作を繰り返して得られる記号列のみが式である.
					ただし,繰り返しの操作は無制限に行われるものではない.
					得られる記号列は左端から辿っていけば必ず右端が見つかるものとする.
			\end{itemize}
	\end{description}
	
	\monologue{
		院生「式の概念を拡張したことで
			\begin{align}
				s \in \Set{x}{A(x)},\quad t = \Set{x}{A(x)}
			\end{align}
			は式として扱えるようになりましたが,
			\begin{align}
				\forall \Set{x}{A(x)}(s \in \Set{x}{A(x)})
			\end{align}
			は式として認められないのですね.量化記号が付くのは文字に限られます.」
	}
	
	日常使用している言語と同じく,名詞にあたる対象と文法にあたる式の形成手順を
	合わせて{\bf 言語}\index{げんご@言語}{\bf (language)}と呼ぶ.
	以降は言語そのものを意識することは殆ど無いが,形式上の出発点として宣言しておく.
	
	\begin{screen}
		\begin{dfn}[閉式・命題]
			自由変項を含まない式を{\bf 閉式}\index{へいしき@閉式}{\bf (closed formula)}や{\bf 命題}\index{めいだい@命題}{\bf (proposition)}と呼ぶ.
		\end{dfn}
	\end{screen}
	
	\monologue{
		院生「命題とは真偽が定まったものであるという釈然としない説明をよく目にしますが,
			どうやら命題や真偽の哲学的議論には決着が付いていないようで,
			本によっては``異論が続出するから深い言及を避ける''と書いてあるものもあります.
			しかし本稿では命題も真偽もその概念を明確に定義して,
			つまり概念を本稿で必要な分に制限するということになりますが,
			扱うことにいたします.」
	}
	
	\begin{screen}
		\begin{dfn}[宇宙]
			文字$V$を{\bf 宇宙}\index{うちゅう@宇宙}{\bf (Universe)}と呼ぶ.
		\end{dfn}
	\end{screen}
	
	\monologue{
		院生「文字$V$を特別扱いするということですね(笑).宇宙という壮大な言葉が出てきてしまいましたが,
			集合論の世界は$V$の範囲内で語り尽くせてしまうのですから,
			現代数学にとって$V$は宇宙そのものなのですね.
			ところで,現実世界において人間が把握し得る最大の世界は宇宙空間でしょうが,
			数学の世界では宇宙の外側を見ることが出来るのです.宇宙の外側に在るものは真類と呼ばれます.
			実は宇宙そのものも真類の一つなのですが(宇宙が宇宙の外側に在るとは奇妙です),
			その話は後述にまかせましょう.」
	}
	
	数学の式を日本語に翻訳するとき,慣習上よく使われる訳し方があるので列挙する.
	\begin{itemize}
		\item 式$a = b$を``$a$は$b$に等しい''や``$a$と$b$は等しい''と翻訳する.
		\item 式$a \in b$を``$a$は$b$の要素である''や``$a$は$b$に属する''と翻訳する.
		\item 式$(A) \Longrightarrow (B)$を``$A$が成り立つならば$B$が成り立つ''と翻訳する.
		\item 式$\rightharpoondown (A)$を%``$A$でない''と翻訳する.
	\end{itemize}
	
	\begin{screen}
		\begin{dfn}[類・集合]
			対象のことを{\bf 類}\index{るい@類}{\bf (class)}と呼び直し,
			特に$V$の要素である類を{\bf 集合}\index{しゅうごう@集合}{\bf (set)}と呼ぶ.
			また$V$の要素でない類のことを{\bf 真類}\index{しんるい@真類}{\bf (proper class)}と呼ぶ.
		\end{dfn}
	\end{screen}

	\monologue{
		院生「類は$V$の要素であれば集合と呼ばれ,$V$の要素でなければ真類と呼ばれます.
			では集合であり真類でもある類や,集合でも真類でもない類はあるのでしょうか?
			答えは``現段階では確定したことは何も言えない''です.
			質問を変えましょう.集合であり真類でもある類や集合でも真類でもない類の存在を禁止するにはどうしたら良いでしょうか?
			我々は,数学において中庸が無いということや矛盾が起きるということをどう表現しようかという問題に直面しているのです.
			この問題の解決への方便として{\bf 推論規則}\index{すいろんきそく@推論規則}
			{\bf (rule of inference)}と呼ばれるものを導入します.」
	}
	
	\begin{screen}
		\begin{axm}[排中律]
			任意の論理式$A$に対し,$A \vee \rightharpoondown A$が成り立つ.
		\end{axm}
	\end{screen}
	
	\begin{screen}
		\begin{thm}[集合でも真類でもない類は存在しない]
			\begin{align}
				\forall a\ \left(\ \rightharpoondown (\ a \in V \wedge a \notin V\ )\ \right)
			\end{align}
		\end{thm}
	\end{screen}
	
	\begin{screen}
		\begin{axm}[空集合の存在公理]
			いかなる集合も要素に持たない集合が存在する:
			\begin{align}
				\exists x \in V\ \forall y \in V\ (\ y \notin x\ ).
			\end{align}
		\end{axm}
	\end{screen}
	
	\begin{screen}
		\begin{thm}[空集合はただ一つ]
			空集合の存在公理を満たす集合はただ一つである:
			\begin{align}
				\forall x \in V\ \forall y \in V
				\ (\ (\ \forall z \in V\ (\ z \notin x\ ) \wedge \forall z \in V
				\ (\ z \notin y\ )\ )
				\Longrightarrow x=y\ ).
			\end{align}
			この何も持たない空の集合を{\bf 空集合}\index{くうしゅうごう@空集合}{\bf (empty set)}と呼び$\emptyset$という記号で表す.
		\end{thm}
	\end{screen}
	
	\monologue{
		院生「ようやく存在が約束された本物の集合が出てきましたね.
			あらゆる集合は空集合を元に作られていくのですから,空集合は集合の親とでもいえるのでしょうか.」
	}
	
	\begin{screen}
		\begin{axm}[類の公理]\mbox{}
			\begin{description}
				\item[(i)] 類の要素は集合である:
					\begin{align}
						\forall a\ \forall x\ (\ x \in a \Longrightarrow x \in V\ ).
					\end{align}
				
				\item[(ii)] $\Set{x}{A(x)}$とは$A(x)$を成り立たせる集合$x$の全体である:
					\begin{align}
						\forall t\ (\ t \in \Set{x}{A(x)} \Longleftrightarrow t \in V \wedge A(t)\ ).
					\end{align}
			\end{description}
		\end{axm}
	\end{screen}
	
	\begin{screen}
		\begin{axm}[外延性の公理]
			全く同じ要素からなる類は等しい:
			\begin{align}
				\forall a\ \forall b\ \left(\ \forall t\ (\ t \in a  \Longleftrightarrow t \in b\ )
				\Longrightarrow a=b\ \right).
			\end{align}
		\end{axm}
	\end{screen}
	
	\begin{screen}
		\begin{thm}\mbox{}
			\begin{description}
				\item[(1)] $\forall a\ (\ a=a\ )$
				\item[(2)] $\forall a \in V\ (\ a = \Set{x}{x \in a}\ )$
				\item[(3)] $V=\Set{x}{x=x}$
				\item[(4)] $\forall a\ (\ a \subset V\ )$
				\item[(5)] $\Set{x}{A} = \Set{y}{(y\, |\, x)A}$
				\item[(6)] $\Set{x}{A(x)} \cup \Set{x}{\rightharpoondown A(x)} = V$.
			\end{description}
		\end{thm}
	\end{screen}
	
	\begin{screen}
		\begin{axm}[相等性の公理]\mbox{}
			\begin{description}
				\item[(1)] $\forall a\ \forall b\ \forall c\ \left(\ a=b \Longrightarrow (\ c \in a \Longleftrightarrow c \in b\ )\ \right).$
				\item[(2)] $\forall a\ \forall b\ \forall c\ \left(\ a=b \Longrightarrow (\ c = a \Longleftrightarrow c = b\ )\ \right).$
				\item[(3)] $\forall a\ \forall b\ \forall c\ \left(\ a=b \Longrightarrow (\ a \in c \Longleftrightarrow b \in c\ )\ \right).$
			\end{description}
		\end{axm}
	\end{screen}
	
	\begin{screen}
		\begin{thm}
			
		\end{thm}
	\end{screen}
	
	\begin{prf}\mbox{}
		\begin{description}
			\item[(1)] $a^{-1}$の任意の要素$t$に対し或る$V$の要素$x,y$が存在して
				\begin{align}
					(x,y) \in a \wedge t = (y,x)
				\end{align}
				を満たす.$((x,y),(y,x)) \in f$より$((x,y),t) \in f$が成り立つから
				$t \in f \ast a$となる.逆に$f \ast a$の任意の要素$t$に対して
				$a$の或る要素$x$が存在して
				\begin{align}
					x \in a \wedge (x,t) \in f
				\end{align}
				となる.$x$に対し$V$の或る要素$a,b$が存在して$x=(a,b)$となるので
				\begin{align}
					((a,b),t) \in f
				\end{align}
				となり,$V$の或る要素$c,d$が存在して
				\begin{align}
					((a,b),t) = ((c,d),(d,c))
				\end{align}
				となる.$(a,b) = (c,d)$より$a=c$かつ$b=d$となり,
				$t = (d,c)$かつ$(d,c)=(b,a)$より$t=(b,a)$,従って
				$t \in a^{-1}$が成り立つ.
		\end{description}
	\end{prf}
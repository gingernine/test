\chapter{言語$\mathcal{L}_{\in}$}

	本稿の世界を展開するために使用する言語は二つ種類がある.
	一つは自然言語の日本語であり,もう一つは記号のみで作られた人工的な言語である.
	その人工的な言語は記号列が数学の式となるための文法を指定し,
	そこで組み立てられた式のみが考察対象となる.
	日本語は式を解釈したり人工言語を補助するために使われる.
	
	まず,人工的な言語である$\mathcal{L}_{\in}$を設定する.
	以下は$\mathcal{L}_{\in}$を構成する要素である:
	\begin{description}
		\item[矛盾記号] $\bot$
		\item[論理記号] $\rightharpoondown$, $\vee$, $\wedge$, $\Longrightarrow$
		\item[量化子] $\forall$, $\exists$
		\item[述語記号] $=$, $\in$
		\item[使用文字] ローマ字及びギリシア文字.
		\item[接項子] $\natural$
	\end{description}
	
	日本語と同様に,決められた規則に従って並ぶ記号列のみを$\mathcal{L}_{\in}$の単語や文章として扱う.
	$\mathcal{L}_{\in}$において,名詞にあたるものは{\bf 項}\index{こう@項}{\bf (term)}と呼ばれる.
	文字は最もよく使われる項である.述語とは項同士を結ぶものであり,最小単位の文章を形成する.例えば
	\begin{align}
		\in st
	\end{align}
	は$\mathcal{L}_{\in}$の文章となり,日本語には``$s$は$t$の要素である''と翻訳される.
	$\mathcal{L}_{\in}$の文章を{\bf 式}\index{しき@式}{\bf (formula)}或いは
	{\bf 論理式}\index{ろんりしき@論理式}と呼ぶ.論理記号は主に式同士を繋ぐ役割を持つ.
	
	論理学的な言語とは論理記号と変項記号を除く記号をすべて集めたものである.
	本稿で用意した記号で言うと,論理記号とは
	\begin{align}
		\bot,\ \rightharpoondown,\ \vee,\ \wedge,\ \Longrightarrow,\ \forall,\ \exists,\ =
	\end{align}
	であり,変項記号とは文字であって,$\mathcal{L}_{\in}$の語彙は
	\begin{align}
		\in,\ \natural
	\end{align}
	しかない.だが本稿の目的は集合論の構築であって一般の言語について考察するわけではないので,
	論理記号も文字もすべて$\mathcal{L}_{\in}$の一員と見做す方が自然である.
	ついでに記号の分類も主流の論理学とは変えていて,
	\begin{itemize}
		\item $\bot$はそれ単体で式であるので他の記号とは分ける.
		\item 論理記号とは式に作用するものとして$\rightharpoondown,\vee,\wedge,\Longrightarrow$のみとする.
		\item $\forall$と$\exists$は項に作用するものであるから量化子として分類する.
		\item 等号$=$は'等しい'という述語になっているから,論理記号ではなく述語記号に入れる.
	\end{itemize}
	以上の変更点は殆ど無意味であるが,
	いかに``直観的な''集合論を構築するかという目的を勘案すれば良いスタートであるように思える.
	
\section{項}
	
	文字は項として使われるが,文字だけを項とするのは不十分であり,
	例えば$1000$個の相異なる項が必要であるといった場合には異体字まで駆使しても不足する.
	そこで,文字$x$に対して
	\begin{align}
		\natural x
	\end{align}
	もまた項であると約束する.
	また,$\tau$を項とするときに
	\begin{align}
		\natural \tau
	\end{align}
	も項であると約束する.この約束に従えば,文字$x$だけを用いたとしても
	\begin{align}
		x,\quad \natural x, \quad \natural \natural x, \quad \natural \natural \natural x
	\end{align}
	はいずれも項ということになる.極端なことを言えば,「$1000$個の項を用意してくれ」と頼まれたとしても
	$\natural$と$x$だけで$1000$個の項を作り出すことが可能なのだ.
	
	大切なのは,$\natural$を用いれば理屈の上では項に不足しないということであって,
	具体的な数式を扱うときに$\natural$が出てくるかと言えば否である.
	$\natural$が必要になるほどに長い式を読解するのは困難であるから,
	通常は何らかの略記法を導入して複雑なところを覆い隠してしまう.
	
	\begin{itembox}[l]{超記号}
		上で「$\tau$を項とするときに」と書いたが,これは一時的に
		$\tau$を或る項に代用しているだけであって,
		$\tau$が指している項の本来の字面は$x$であるかもしれない.
		この場合の$\tau$を{\bf 超記号}\index{ちょうきごう@超記号}と呼ぶ.
		「$A$を式とする」など式にも超記号が宣言される.
	\end{itembox}
	
	項は形式的には次のよう定義される:
	
	\begin{description}
		\item[項]
			\begin{itemize}
				\item 文字は項である.
				\item $\tau$を項とするとき,$\natural \tau$は項である.
				\item 以上のみが項である.
			\end{itemize}
	\end{description}
	
	上の定義では,はじめに発端を決めて,次に新しい項を作り出す手段を指定している.こういった定義の仕方を
	{\bf 帰納的定義}\index{きのうてきていぎ@帰納的定義}{\bf (inductive definition)}と呼ぶ.
	ただしそれだけでは項の範囲が定まらないので,最後に「以上のみが項である」と加えている.
	
	「以上のみが項である」という約束によって,例えば「$\tau$が項である」という言明が与えられたとき,この言明が
	``$\tau$は或る文字に代用されている''か
	``項$\sigma$が取れて(超記号),$\tau$は$\natural \sigma$に代用されている''
	のどちらか一方にしか解釈され得ないのは,言うまでもない,であろうか.直感的にはそうであっても
	直感を万人が共有している保証はないから,やはりここは明示的に,「$\tau$が項である」という言明の解釈は
	\begin{itemize}
		\item $\tau$は或る文字に代用されている
		\item 項$\sigma$が取れて(超記号),$\tau$は$\natural \sigma$に代用されている
	\end{itemize}
	に限られると決めてしまおう.主張はストレートな方が後々使いやすい.
	
	\begin{itembox}[l]{暗に宣言された超記号}
		上で「項$x$が取れて」と書いたが,この$x$は唐突に出てきたので,
		それが表す文字そのものでしかないのか,或いは超記号であるのか,一見判然しない.
		本来は「項,これを$x$で表す,が取れて」などと書くのが
		良いのかもしれないが,はじめの書き方でも文脈上は超記号として解釈するのが自然であるし,
		何より言い方がまどろこくない.このように見た目の簡潔さのために超記号の宣言を省略する場合もある.
	\end{itembox}
	
\section{式}
	式も項と同様に帰納的に定義される:
	
	\begin{description}
		\item[式]
			\begin{itemize}
				\item $\bot$は式である.
				\item $\sigma$と$\tau$を項とするとき,$\in st$と$=st$は式である.
					これを{\bf 原子式}\index{げんししき@原子式}{\bf (atomic formula)}と呼ぶ.
				\item $\varphi$を式とするとき,$\rightharpoondown \varphi$は式である.
				\item $\varphi$と$\psi$を式とするとき,$\vee \varphi \psi,\ 
					\wedge \varphi \psi,\ \Longrightarrow \varphi \psi$はいずれも式である.
			
				\item $x$を項とし,$\varphi$を式とするとき,$\forall x \varphi$と$\exists x \varphi$は式である.
				
				\item 以上のみが式である.
			\end{itemize}
	\end{description}
	
	例えば「$\varphi$が式である」という言明の解釈は,
	\begin{itemize}
		\item $\varphi$は$\bot$である
		\item 項$s$と項$t$が得られて,$\varphi$は$\in s t$である
		\item 項$s$と項$t$が得られて,$\varphi$は$= s t$である
		\item 式$\psi$が得られて,$\varphi$は$\rightharpoondown \psi$である
		\item 式$\psi$と式$\xi$が得られて,$\varphi$は$\vee \psi \xi$である
		\item 式$\psi$と式$\xi$が得られて,$\varphi$は$\wedge \psi \xi$である
		\item 式$\psi$と式$\xi$が得られて,$\varphi$は$\Longrightarrow \psi \xi$である
		\item 項$x$と式$\psi$が得られて,$\varphi$は$\forall x \psi$である
		\item 項$x$と式$\psi$が得られて,$\varphi$は$\exists x \psi$である
	\end{itemize}
	に限られる.
	
\chapter{量化}
	例えば
	\begin{align}
		\forall x \in x y
	\end{align}
	なる式を考える.中置記法(後述)で
	\begin{align}
		\forall x\, (\, x \in y\, )
	\end{align}
	と書けば若干見やすくなるであろうか.冠頭詞$\forall$は直後の$x$に係って「任意の$x$に対し...」の意味を持ち,
	この式は「任意の$x$に対して$x$は$y$の要素である」と読むのであるが,
	このとき$x$は$\forall x \in x y$で{\bf 束縛されている}{\bf (bound)}や
	或いは{\bf 量化されている}{\bf (quantified)}と言う.
	$\forall$が$\exists$に代わっても,今度は``$x$は$\exists x \in x y$で束縛されている''と言う.
	まあつまり,{\bf 量化子の直後に続く項(量化子が係っている項)は,その量化子から始まる式の中で束縛されている}
	と解釈することになっているのだ.
	
	では
	\begin{align}
		\Longrightarrow \forall x \in x y \in x z
	\end{align}
	という式はどうであるか.$\forall x$の後ろには$x$が二か所に現れているが,
	どちらの$x$も$\forall$によって束縛されているのであろうか?
	結論を言えば$\in x y$の$x$は束縛されていて,$\in x z$の$x$は束縛されていない.
	というのも式の構成法を思い返せば,$\forall x \varphi$が式であると言ったら$\varphi$は式であるはずで,
	今の例で$\forall x$に後続する式は
	\begin{align}
		\in x y
	\end{align}
	しかないのだから,$\forall$から始まる式は
	\begin{align}
		\forall x \in x y
	\end{align}
	しかないのである.$\forall$が係る$x$が束縛されている範囲は
	``$\forall$から始まる式''であるので,$\in x z$の$x$とは
	量化子$\forall$による``束縛''から漏れた``自由な''$x$ということになる.
	
	上の例でみたように,量化はその範囲が重要になる.
	量化子$\forall$が式$\varphi$に現れたとき,
	その$\forall$から始まる$\varphi$の部分式を
	$\forall$の{\bf スコープ}と呼ぶが,
	いつでもスコープが取れることは明白であるとして,
	$\forall$のスコープは唯一つでないと都合が悪いだろう.
	もしも異なるスコープが存在したら,同じ式なのに全く違う解釈に分かれてしまうのだから.
	実際そのような心配は無用であると後で保証するわけだが,
	その準備として{\bf 始切片}という概念について取り掛かる.
	
	ではさらにグレードアップさせて,
	\begin{align}
		\forall x \Longrightarrow \forall x \in x y \in x z
	\end{align}
	なる式における量化はどうであろうか.
	
\section{部分項と部分式}
	\begin{description}
		\item[部分項]
			項から切り取ったひとつづきの部分列で,それ自体が項であるものを元の項に対して
			{\bf 部分項}\index{ぶぶんこう@部分項}{\bf (sub term)}と呼ぶ.
			元の項全体も部分項と捉えるが,自分自身を除く部分項を特に
			{\bf 真部分項}\index{しんぶぶんこう@真部分項}{\bf (proper sub term)}と呼ぶ.
			例えば,文字$x$の部分項は$x$自身のみであって,また$\tau$を項とすると$\tau$は$\natural \tau$の部分項である.
		
		\item[部分式]
			式から切り取ったひとつづきの部分列で,それ自体が式であるものを元の式に対して
			{\bf 部分式}\index{ぶぶんしき@部分式}{\bf (sub formula)}と呼ぶ.
			例えば$\varphi$と$\psi$を式とするとき,$\varphi$と$\psi$は$\vee \varphi \psi$の部分式である.
			元の式全体も部分式と捉えるが,自分自身を除く部分式を特に
			{\bf 真部分式}\index{しんぶぶんしき@真部分式}{\bf (proper sub formula)}と呼ぶ.
	\end{description}
	
\section{始切片}
	$\varphi$を$\mathcal{L}_{\in}$の式とするとき,$\varphi$の左端から切り取るひとつづきの部分列を
	$\varphi$の{\bf 始切片}\index{しせっぺん@始切片}{\bf (initial segment)}と呼ぶ.
	例えば$\varphi$が
	\begin{align}
		\Longrightarrow \forall x \wedge \Longrightarrow \in xy \in xz \Longrightarrow \in xz \in xy = yz
	\end{align}
	である場合,
	\begin{align}
		\textcolor{red}{\Longrightarrow \forall x \wedge \Longrightarrow \in xy \in xz \Longrightarrow \in xz \in x}y = yz
	\end{align}
	や
	\begin{align}
		\textcolor{red}{\Longrightarrow \forall x \wedge \Longrightarrow \in xy} \in xz \Longrightarrow \in xz \in xy = yz
	\end{align}
	など赤字で分けられた部分は$\varphi$の始切片である.また$\varphi$自身も$\varphi$の始切片である.
	
	項についても同様に,項の左端から切り取るひとつづきの部分列をその項の始切片と呼ぶ.
	
	本節の主題は次である.
	\begin{screen}
		\begin{metathm}[始切片の一意性]\label{metathm:initial_segment_L_in}
			$\tau$を$\mathcal{L}_{\in}$の項とするとき,$\tau$の始切片で$\mathcal{L}_{\in}$の項であるものは$\tau$自身に限られる.
			また$\varphi$を$\mathcal{L}_{\in}$の式とするとき,$\varphi$の始切片で$\mathcal{L}_{\in}$の式であるものは$\varphi$自身に限られる.
		\end{metathm}
	\end{screen}
	
	「項の始切片で項であるものはその項自身に限られる.また,式の始切片で式であるものはその式自身に限られる.」という言明を(★)と書くことにする.
	このメタ定理を示すには次の原理を用いる:
	
	\begin{screen}
		\begin{metaaxm}[$\mathcal{L}_{\in}$の項に対する構造的帰納法]
			$\mathcal{L}_{\in}$の項に対する言明Xに対し(Xとは,例えば上の(★)),
			\begin{itemize}
				\item 文字に対してXが言える.
				\item 無作為に与えられた項が与えられたとき,その全ての真部分項に対してXが言えるならば,
					その項に対してもXが言える.
			\end{itemize}
			ならば,いかなる項に対してもXが言える.
		\end{metaaxm}
	\end{screen}
	
	\begin{screen}
		\begin{metaaxm}[$\mathcal{L}_{\in}$の式に対する構造的帰納法]
			$\mathcal{L}_{\in}$の式に対する言明Xに対し(Xとは,例えば上の(★)),
			\begin{itemize}
				\item $\bot$に対してXが言える.
				\item 原子式に対してXが言える.
				\item 無作為に与えられた式が与えられたとき,その全ての真部分式に対してXが言えるならば,
					その式に対してもXが言える.
			\end{itemize}
			ならば,いかなる式に対してもXが言える.
		\end{metaaxm}
	\end{screen}
	
	では定理を示す.
	
	\begin{metaprf}\mbox{}
		\begin{description}
			\item[項について]
				$s$を項とするとき,$s$が文字ならば$s$の始切片は$s$のみである.つまり(★)が言える.
				$s$が文字でないならば,$s$の全ての真部分項に対して(★)が言えると仮定する.
				(項の構成法より)項$t$が取れて$s$は
				\begin{align}
					\natural t
				\end{align}
				と表せる.$u$を$s$の始切片で項であるものとすると
				$u$に対しても(項の構成法より)項$v$が取れて,$u$は
				\begin{align}
					\natural v
				\end{align}
				と表せる.このとき$v$は$t$の始切片であり,
				$t$については(★)が言えるので,$t$と$v$は一致する.
				ゆえに$s$と$u$は一致する.ゆえに$s$に対しても(★)が言える.
				
			\item[式について]
			
	$\bot$については,その始切片は$\bot$に限られる.
	$\in st$なる原子式については,その始切片は
	\begin{align}
		\in, \quad \in s, \quad \in st
	\end{align}
	のいずれかとなるが,このうち式であるものは$\in st$のみである.
	$=st$なる原子式についても,その始切片で式であるものは$=st$に限られる.
	
	いま$\varphi$を任意に与えられた式とし,
	$\varphi$の真部分式に対しては(★)が当てはまっているとする.
	\begin{description}
		\item[ケース1] 式$\psi$が得られて$\varphi$が
			\begin{align}
				\rightharpoondown \psi
			\end{align}
			であるとき,$\psi$は$\varphi$の真部分式であるので(★)は当てはまる.
			$\varphi$の始切片で式であるものは,
			式$\xi$を用いて$\rightharpoondown \xi$と表せるが,
			$\xi$は$\psi$の始切片であるから,帰納法の仮定より$\xi$と$\psi$は一致する.
			ゆえに$\varphi$の始切片で式であるものは$\varphi$自身に限られる.
			
		\item[ケース2] 式$\psi$と$\xi$が得られて$\varphi$が
			\begin{align}
				\vee \psi \xi
			\end{align}
			であるとする.$\varphi$の始切片で式であるものも$\vee$が左端に来るので,
			式$\eta$と式$\zeta$が得られて始切片は
			\begin{align}
				\vee \eta \zeta
			\end{align}
			と表せる.$\psi$と$\eta$,$\xi$と$\zeta$は
			いずれも$\varphi$の真部分式であるので(★)が当てはまる.
			そして$\psi$と$\eta$は一方が他方の始切片であるので,(★)より一致する.
			すると$\xi$と$\zeta$も一方が他方の始切片ということになり,(★)より一致する.
			ゆえに$\vee \psi \xi$と$\vee \eta \zeta$は一致する.
			つまり$\varphi$の始切片で式であるものは$\varphi$自身に限られる.
			$\varphi$が$\wedge \psi \xi$や$\Longrightarrow \psi \xi$である場合も同じである.
			
		\item[ケース3] 項$x$と式$\psi$が得られて,$\varphi$が
			\begin{align}
				\forall x \psi
			\end{align}
			であるとき,$\varphi$の始切片で式であるものは,式$\xi$が取れて
			\begin{align}
				\forall x \xi
			\end{align}
			と表せる.このとき$\xi$は$\psi$の始切片であるし,
			また$\psi$は$\varphi$の真部分式であるから,(★)より$\psi$と$\xi$は一致する.
			ゆえに$\varphi$の始切片で式であるものは$\varphi$自身に限られる.
			$\varphi$が$\forall x \psi$である場合も同じである.
			\QED
	\end{description}

		\end{description}
	\end{metaprf}
	
\section{スコープ}
	$\varphi$を式とし,$s$を「$\natural,\in,\bot,\rightharpoondown,\vee,\wedge,\Longrightarrow,\exists,\forall$」
	のいずれかの記号とし,$\varphi$に$s$が現れたとする.このとき,$s$のその出現位置から始まる$\varphi$の部分式,
	或いは$s$が$\natural$である場合は部分項,を
	$s$の{\bf スコープ}\index{スコープ}{\bf (scope)}と呼ぶ.具体的に,$\varphi$を
	\begin{align}
		\Longrightarrow \forall x \wedge \Longrightarrow \in xy \in xz \Longrightarrow \in xz \in xy = yz
	\end{align}
	としよう.このとき$\varphi$の左から$6$番目に$\in$が現れるが,この$\in$から
	\begin{align}
		\in xy
	\end{align}
	なる原子式が$\varphi$の上に現れている:
	\begin{align}
		\Longrightarrow \forall x \wedge \Longrightarrow \textcolor{red}{\in xy} \in xz \Longrightarrow \in xz \in xy = yz.
	\end{align}
	これは{\bf $\varphi$における左から6番目の$\in$のスコープ}である.他にも,$\varphi$の左から$4$番目に$\wedge$が現れるが,この右側に
	\begin{align}
		\Longrightarrow \in xy \in xz
	\end{align}
	と
	\begin{align}
		\Longrightarrow \in xz \in xy
	\end{align}
	の二つの式が続いていて,$\wedge$を起点に
	\begin{align}
		\wedge \Longrightarrow \in xy \in xz \Longrightarrow \in xz \in xy
	\end{align}
	なる式が$\varphi$の上に現れている:
	\begin{align}
		\Longrightarrow \forall x \textcolor{red}{\wedge \Longrightarrow \in xy \in xz \Longrightarrow \in xz \in xy} = yz.
	\end{align}
	これは{\bf $\varphi$における左から4番目の$\wedge$のスコープ}である.$\varphi$の左から$2$番目には$\forall$が現れて,
	この$\forall$に対して項$x$と
	\begin{align}
		\wedge \Longrightarrow \in xy \in xz \Longrightarrow \in xz \in xy
	\end{align}
	なる式が続き,
	\begin{align}
		\forall x \wedge \Longrightarrow \in xy \in xz \Longrightarrow \in xz \in xy
	\end{align}
	なる式が$\varphi$の上に現れている:
	\begin{align}
		\Longrightarrow \textcolor{red}{\forall x \wedge \Longrightarrow \in xy \in xz \Longrightarrow \in xz \in xy} = yz.
	\end{align}
	
	しかも$\in,\wedge,\forall$のスコープは上にあげた部分式のほかに取りようが無い.
	上の具体例を見れば,直感的に「現れた記号のスコープはただ一つだけ,必ず取ることが出来る」
	が一般の式に対しても当てはまるであるように思えるが,直感を排除してこれを認めるには構造的帰納法の原理が必要になる.
	
	当然ながら$\mathcal{L}_{\in}$の式には同じ記号が何か所にも出現しうるので,
	式$\varphi$に記号$s$が現れたと言ってもそれがどこの$s$を指定しているのかはっきりしない.
	しかし{\bf スコープを考える際には,$\varphi$に複数現れうる$s$のどれか一つを選んで,
	その$s$に終始注目している}のであり,
	「その$s$の...」や「$s$のその出現位置から...」のように限定詞を付けてそのことを示唆することにする.
	
	\begin{screen}
		\begin{metathm}[スコープの存在]\label{metathm:existence_of_scopes_L_in}
		$\varphi$を式とするとき,
		\begin{description}
			\item[(a)] $\natural$が$\varphi$に現れたとき,項$t$が得られて,
				$\natural$のその出現位置から$\natural t$なる項が$\varphi$の上に現れる.
				
			\item[(b)] $\in$が$\varphi$に現れたとき,項$\sigma$と項$\tau$が得られて,
				$\in$のその出現位置から$\in \sigma \tau$なる式が$\varphi$の上に現れる.
				
			\item[(c)] $\rightharpoondown$が$\varphi$に現れたとき,式$\psi$が得られて,
				$\rightharpoondown$のその出現位置から$\rightharpoondown \psi$なる式が
				$\varphi$の上に現れる.
				
			\item[(d)] $\vee$が$\varphi$に現れたとき,式$\psi$と式$\xi$が得られて,
				$\vee$のその出現位置から$\vee \psi \xi$なる式が$\varphi$の上に現れる.
				
			\item[(e)] $\exists$が$\varphi$に現れたとき,項$x$と式$\psi$が得られて,
				$\exists$のその出現位置から$\exists x \psi$なる式が$\varphi$の上に現れる.
		\end{description}
		\end{metathm}
	\end{screen}
	
	(b)では$\in$を$=$に替えたって同じ主張が成り立つし,(d)では$\vee$を$\wedge$や$\Longrightarrow$に替えても同じである.
	(e)では$\exists$を$\forall$に替えても同じことが言える.
	
	\begin{metaprf}\mbox{}
		\begin{description}
			\item[項について]
				「項に$\natural$が現れたとき,項$t$が取れて,
				その$\natural$の出現位置から$\natural t$がその項の部分項として現れる」---(※),を示す.
				$s$を項とするとき,$s$が文字ならば$s$に対して(※)が言える.
				$s$が文字でないとき,$s$の全ての真部分項に対して(※)が言えるとする.
				$s$は文字ではないので,(項の構成法より)項$t$が取れて$s$は
				\begin{align}
					\natural t
				\end{align}
				と表せる.$s$に現れる$\natural$とは$s$の左端のものであるか
				$t$の中に現れるものであるが,$t$は$s$の真部分項であって,
				$t$については(※)が言えるので,結局$s$に対しても(※)が言えるのである.
			
			\item[case1]
				$\bot$に対しては上の言明は当てはまる.
			
			\item[case2]
				$\in s t$なる式に対しては,$\in$のスコープは$\in s t$に他ならない.
				実際,$\in$から始まる$\in s t$の部分式は,項$u,v$が取れて
				\begin{align}
					\in u v
				\end{align}
				と書けるが,このとき$u$と$s$は一方が他方の始切片となっているので,
				メタ定理\ref{metathm:initial_segment_L_in}より$u$と$s$は一致する.
				すると今度は$v$と$t$について一方が他方の始切片となるので,
				メタ定理\ref{metathm:initial_segment_L_in}より$v$と$t$も一致する.
				
				$\in s t$に$\natural$が現れた場合,これが$s$に現れているとすると,
				前段より項$u$が取れて,この$\natural$の出現位置から$\natural u$なる項が$s$の上に現れる.
				また項$v$が取れて,この$\natural$の出現位置から$\natural v$なる項が
				$\in s t$の上に現れているとしても,$u$と$v$は一方が他方の始切片となるから
				メタ定理\ref{metathm:initial_segment_L_in}より
				$u$と$v$は一致する.$\natural$が$t$に現れたときも同じである.
				以上より$\in s t$に対して定理の主張が当てはまる.
					
			\item[case3]
				$\varphi$を任意に与えられた式とし,$\varphi$の全ての真部分式に対しては
				定理の主張が当てはまっているとする.
		
				式$\varphi$と$\psi$に対して上の言明が当てはまるとする.
				式$\rightharpoondown \varphi$に対して,
				$\sigma$が左端の$\rightharpoondown$であるとき
				$\sigma \varphi$は$\rightharpoondown \varphi$の部分式である.
				また$\sigma \psi$が$\sigma$のその出現位置から始まる$\rightharpoondown \varphi$の部分式
				であるとすると,
				$\psi$は$\varphi$の左端から始まる$\varphi$の部分式ということになるので
				メタ定理\ref{metathm:initial_segment_L_in}より$\varphi$と$\psi$は一致する.
				$\sigma$が$\varphi$に現れる記号であれば,帰納法の仮定より
				$\sigma$から始まる$\varphi$の部分式が一意的に得られる.
				その部分式は$\rightharpoondown \varphi$の部分式でもあるし,
				$\rightharpoondown \varphi$の部分式としての一意性は
				メタ定理\ref{metathm:initial_segment_L_in}より従う.
	
				式$\vee \varphi \psi$に対して,
				$\sigma$が左端の$\vee$であるとき,式$\xi$と$\eta$が得られて$\sigma \xi \eta$が
				$\vee \varphi \psi$の部分式となったとすると,
				$\xi$と$\varphi$は左端を同じくし,どちらか一方は他方の部分式である.
				$\xi$が$\varphi$の部分式であるならば,
				メタ定理\ref{metathm:initial_segment_L_in}より$\xi$と$\varphi$は一致する.
				$\varphi$が$\xi$の部分式であるならば,$\xi$と$\psi$が重なるとなると
				$\psi$の左端の記号から始まる$\xi$の部分式と$\psi$は一致しなくてはならない.
				\QED
		\end{description}
	\end{metaprf}
	
	始切片に関する定理からスコープの一意性を示すことが出来る.
	
	\begin{screen}
		\begin{metathm}[スコープの一意性]\label{metathm:uniqueness_of_scopes_L_in}
			$\varphi$を式とし,$s$を
			$\natural,\in,\bot,\rightharpoondown,\vee,\wedge,\Longrightarrow,\exists,\forall$
			のいずれかの記号とし,$\varphi$に$s$が現れたとする.
			このとき$\varphi$におけるその$s$のスコープは唯一つである.
		\end{metathm}
	\end{screen}
	
	\begin{metaprf}\mbox{}
		\begin{description}
			\item[case1]
				$\natural$が$\varphi$に現れた場合,スコープの存在定理\ref{metathm:existence_of_scopes_L_in}
				より項$\tau$が取れて
				\begin{align}
					\natural \tau
				\end{align}
				なる形の項が$\natural$のその出現位置から$\varphi$の上に現れるわけだが,
				\begin{align}
					\natural \sigma
				\end{align}
				なる項も$\natural$のその出現位置から$\varphi$の上に出現しているといった場合,
				$\tau$と$\sigma$は一方が他方の始切片となるわけで,
				始切片のメタ定理\ref{metathm:initial_segment_L_in}より
				$\tau$と$\sigma$は一致する.
			
			\item[case2]
				$\rightharpoondown$が$\varphi$に現れた場合,
				これはcase1において項であったところが式に替わるだけで殆ど同じ証明となる.
			
			\item[case3]
				$\vee$が$\varphi$に現れた場合,定理\ref{metathm:existence_of_scopes_L_in}
				より式$\psi,\xi$が取れて
				\begin{align}
					\vee \psi \xi
				\end{align}
				なる形の式が$\vee$のその出現位置から$\varphi$の上に現れる.ここで
				\begin{align}
					\vee \eta \Gamma
				\end{align}
				なる式も$\vee$のその出現位置から$\varphi$の上に出現しているといった場合,
				まず$\psi$と$\eta$は一方が他方の始切片となるわけで,
				メタ定理\ref{metathm:initial_segment_L_in}より
				$\psi$と$\eta$は一致する.すると今度は$\xi$と$\Gamma$について
				一方が他方の始切片となるので,同様に$\xi$と$\Gamma$も一致する.
				$\wedge$や$\Longrightarrow$のスコープの一意性も同様に示される.
				
			\item[case4]
				$\exists$が$\varphi$に現れた場合,定理\ref{metathm:existence_of_scopes_L_in}
				より項$x$と式$\psi$が取れて
				\begin{align}
					\exists x \psi
				\end{align}
				なる形の式が$\exists$のその出現位置から$\varphi$の上に現れる.ここで
				\begin{align}
					\exists y \xi
				\end{align}
				なる式も$\exists$のその出現位置から$\varphi$の上に出現しているといった場合,
				まず項$x$と項$y$は一方が他方の始切片となるわけで,
				メタ定理\ref{metathm:initial_segment_L_in}より
				$x$と$y$は一致する.すると今度は$\psi$と$\xi$が
				一方が他方の始切片の関係となるので,この両者も一致する.
				$\forall$のスコープの一意性も同様に示される.
				\QED
		\end{description}
	\end{metaprf}
	
	\begin{itembox}[l]{量化}
		$\varphi$に$\forall$が現れるとき,
		その$\forall$に後続する項$x$が取れるが,このとき項$x$は$\forall$のスコープ内で
		{\bf 量化されている}\index{りょうか@量化}{\bf(quantified)}という.
		詳しく言い直せば,項$x$と式$\psi$が取れて,その$\forall$のスコープは
		\begin{align}
			\forall x \psi
		\end{align}
		なる式で表されるが,このとき$x$は$\forall x \psi$において量化されているという.
	\end{itembox}
	
	$A$を式とし,$a$を$A$に現れる項とする.このとき$A$の中の項$a$を全て項$x$に置き換えた式を
	\begin{align}
		(x \mid a)A
	\end{align}
	で表す.特に項$a$と項$x$が同一の項である場合は$(x \mid a)A$は$A$自身に一致する.
	また$A$の中で自由に現れる項が$a$のみであって,かつ$a$が自由に現れる箇所がどれも項$x$の量化スコープではないとき,
	$A$に現れる項$a$のうち,{\bf 自由に現れる箇所}を全て項$x$に置き換えた式を
	\begin{align}
		A(x)
	\end{align}
	と書く.$A$に現れる項$a$が全て自由であるときは$A(a)$は$A$自身に一致する.

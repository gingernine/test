	\begin{screen}
		\begin{metathm}[書き換えへの代入は代入した式の書き換え]
		\label{metathm:substitution_to_rewritten_formula}
			$\varphi$を$\lang{\varepsilon}$の式ではない$\mathcal{L}$の式とし,
			$\varphi$には変項$x$が自由に現れているとし,$\tau$を主要$\varepsilon$項とし,
			$\widehat{\varphi}$を$\varphi$の書き換えとする.このとき
			$\widehat{\varphi}(x/\tau)$は$\varphi(x/\tau)$の書き換えである.
		\end{metathm}
	\end{screen}
	
	\begin{metaprf}\mbox{}
		\begin{description}
			\item[step1] $\varphi$が原子式であるとする.
				\begin{description}
					\item[case1] $\varphi$が
						\begin{align}
							x = \Set{z}{\psi}
						\end{align}
						なる式のとき,$\widehat{\varphi}$は
						\begin{align}
							\forall v\, (\, v \in x \lrarrow \psi(z/v)\, )
						\end{align}
						なる式である.
						\begin{itemize}
							\item $x$と$z$が同じならば$\widehat{\varphi}(x/\tau)$は
								\begin{align}
									\forall v\, (\, v \in \tau \lrarrow \psi(z/v)\, )
								\end{align}
								となる.他方で$\varphi(x/\tau)$は
								\begin{align}
									\tau = \Set{z}{\psi}
								\end{align}
								であるから$\widehat{\varphi}(x/\tau)$は
								$\varphi(x/\tau)$の書き換えである.
								
							\item $x$と$z$が違うとき$\widehat{\varphi}(x/\tau)$は,
								$\psi$に$x$が自由に現れていれば
								\begin{align}
									\forall v\, (\, v \in \tau \lrarrow \psi(z/v)(x/\tau)\, )
								\end{align}
								となるが,書き換えの変項条件より$x$は$v$とも違うので
								$\psi(z/v)(x/\tau)$と$\psi(x/\tau)(z/v)$は
								同じである.従って$\widehat{\varphi}(x/\tau)$は
								\begin{align}
									\forall v\, (\, v \in \tau \lrarrow \psi(x/\tau)(z/v)\, )
								\end{align}
								と同じである.他方で$\varphi(x/\tau)$は
								\begin{align}
									\tau = \Set{z}{\psi(x/\tau)}
								\end{align}
								であるから,この場合は$\widehat{\varphi}(x/\tau)$は
								$\varphi(x/\tau)$の書き換えである.
								$\psi$に$x$が自由に現れていない場合,
								$\widehat{\varphi}(x/\tau)$は
								\begin{align}
									\forall v\, (\, v \in \tau \lrarrow \psi(z/v)\, )
								\end{align}
								となるが,$\varphi(x/\tau)$は
								\begin{align}
									\tau = \Set{z}{\psi}
								\end{align}
								であるからこの場合も$\widehat{\varphi}(x/\tau)$は
								$\varphi(x/\tau)$の書き換えである.
						\end{itemize}
						
					\item[case2] $\varphi$が
						\begin{align}
							a = \Set{z}{\psi}
						\end{align}
						なる式のとき($a$と$x$は違う項),$\widehat{\varphi}$は
						\begin{align}
							\forall v\, (\, v \in a \lrarrow \psi(z/v)\, )
						\end{align}
						なる式である.$\varphi$には$x$が自由に現れているので,つまり
						$x$は$\psi$に自由に現れている.従って$\widehat{\varphi}(x/\tau)$は
						\begin{align}
							\forall v\, (\, v \in a \lrarrow \psi(z/v)(x/\tau)\, )
						\end{align}
						となるが,書き換えの変項条件より$x$は$v$とも違うので
						$\psi(z/v)(x/\tau)$と$\psi(x/\tau)(z/v)$は
						同じである.従って$\widehat{\varphi}(x/\tau)$は
						\begin{align}
							\forall v\, (\, v \in a \lrarrow \psi(x/\tau)(z/v)\, )
						\end{align}
						と同じである.他方で$\varphi(x/\tau)$は
						\begin{align}
							a = \Set{z}{\psi(x/\tau)}
						\end{align}
						であるから$\widehat{\varphi}(x/\tau)$は
						$\varphi(x/\tau)$の書き換えである.
					
					\item[case3] $\varphi$が
						\begin{align}
							\Set{y}{\chi} = x
						\end{align}
						なる式のとき,$\widehat{\varphi}$は
						\begin{align}
							\forall u\, (\, \chi(y/u) \lrarrow u \in x\, )
						\end{align}
						なる式である.
						\begin{itemize}
							\item $x$と$y$が同じならば$\widehat{\varphi}(x/\tau)$は
								\begin{align}
									\forall u\, (\, \chi(y/u) \lrarrow u \in \tau\, )
								\end{align}
								となる.他方で$\varphi(x/\tau)$は
								\begin{align}
									\Set{y}{\chi} = \tau
								\end{align}
								であるから$\widehat{\varphi}(x/\tau)$は
								$\varphi(x/\tau)$の書き換えである.
								
							\item $x$と$y$が違うとき$\widehat{\varphi}(x/\tau)$は,
								$\chi$に$x$が自由に現れていれば
								\begin{align}
									\forall u\, (\, \chi(y/u)(x/\tau) \lrarrow u \in \tau\, )
								\end{align}
								となるが,書き換えの変項条件より$x$は$u$とも違うので
								$\chi(y/u)(x/\tau)$と$\chi(x/\tau)(y/u)$は
								同じである.従って$\widehat{\varphi}(x/\tau)$は
								\begin{align}
									\forall u\, (\, \chi(x/\tau)(y/u) \lrarrow u \in \tau\, )
								\end{align}
								と同じである.他方で$\varphi(x/\tau)$は
								\begin{align}
									\Set{y}{\chi(x/\tau)} = \tau
								\end{align}
								であるから,この場合は$\widehat{\varphi}(x/\tau)$は
								$\varphi(x/\tau)$の書き換えである.
								$\chi$に$x$が自由に現れていない場合,
								$\widehat{\varphi}(x/\tau)$は
								\begin{align}
									\forall u\, (\, \chi(y/u) \lrarrow u \in \tau\, )
								\end{align}
								となるが,$\varphi(x/\tau)$は
								\begin{align}
									\Set{y}{\chi} = \tau
								\end{align}
								であるからこの場合も$\widehat{\varphi}(x/\tau)$は
								$\varphi(x/\tau)$の書き換えである.
						\end{itemize}
						
					\item[case4] $\varphi$が
						\begin{align}
							\Set{y}{\chi} = b
						\end{align}
						なる式のとき($b$は$x$と違う項),$\widehat{\varphi}$は
						\begin{align}
							\forall u\, (\, \chi(y/u) \lrarrow u \in b\, )
						\end{align}
						なる式である.$\varphi$には$x$が自由に現れているので,つまり
						$x$は$\chi$に自由に現れている.従って$\widehat{\varphi}(x/\tau)$は
						\begin{align}
							\forall u\, (\, \chi(y/u)(x/\tau) \lrarrow u \in b\, )
						\end{align}
						となるが,書き換えの変項条件より$x$は$u$とも違うので
						$\chi(y/u)(x/\tau)$と$\chi(x/\tau)(y/u)$は
						同じである.従って$\widehat{\varphi}(x/\tau)$は
						\begin{align}
							\forall u\, (\, \chi(x/\tau)(y/u) \lrarrow u \in b\, )
						\end{align}
						と同じである.他方で$\varphi(x/\tau)$は
						\begin{align}
							\Set{y}{\chi(x/\tau)} = b
						\end{align}
						であるから$\widehat{\varphi}(x/\tau)$は
						$\varphi(x/\tau)$の書き換えである.
					
					\item[case5] $\varphi$が
						\begin{align}
							\Set{y}{\chi} = \Set{z}{\psi}
						\end{align}
						なる式のとき,$\widehat{\varphi}$は
						\begin{align}
							\forall u\, (\, \chi(y/u) \lrarrow \psi(z/u)\, )
						\end{align}
						なる式である.
						\begin{itemize}
							\item $x$と$y$が同じならば,$x$は$\Set{y}{\chi}$には自由に
								現れないので,$x$が$\varphi$に自由に現れている以上
								$\psi$に自由に現れることになる.すなわち$x$と$z$は違う項である.
								このとき$\widehat{\varphi}(x/\tau)$は
								\begin{align}
									\forall u\, (\, \chi(y/u) \lrarrow \psi(z/u)(x/\tau)\, )
								\end{align}
								となるが,書き換えの変項条件より$x$は$u$とも違うので
								$\psi(z/u)(x/\tau)$と$\psi(x/\tau)(z/u)$は
								同じである.従って$\widehat{\varphi}(x/\tau)$は
								\begin{align}
									\forall u\, (\, \chi(y/u) \lrarrow \psi(x/\tau)(z/u)\, )
								\end{align}
								と同じである.他方で$\varphi(x/\tau)$は
								\begin{align}
									\Set{y}{\chi} = \Set{z}{\psi(x/\tau)}
								\end{align}
								であるから$\widehat{\varphi}(x/\tau)$は
								$\varphi(x/\tau)$の書き換えである.
								
							\item $x$と$z$が同じならば,$x$は$\Set{z}{\psi}$には自由に
								現れないので,$x$が$\varphi$に自由に現れている以上
								$\chi$に自由に現れることになる.すなわち$x$と$y$は違う項である.
								このとき$\widehat{\varphi}(x/\tau)$は
								\begin{align}
									\forall u\, (\, \chi(y/u)(x/\tau) \lrarrow \psi(z/u)\, )
								\end{align}
								となるが,書き換えの変項条件より$x$は$u$とも違うので
								$\chi(y/u)(x/\tau)$と$\chi(x/\tau)(y/u)$は
								同じである.従って$\widehat{\varphi}(x/\tau)$は
								\begin{align}
									\forall u\, (\, \chi(x/\tau)(y/u) \lrarrow \psi(z/u)\, )
								\end{align}
								と同じである.他方で$\varphi(x/\tau)$は
								\begin{align}
									\Set{y}{\chi(x/\tau)} = \Set{z}{\psi}
								\end{align}
								であるから$\widehat{\varphi}(x/\tau)$は
								$\varphi(x/\tau)$の書き換えである.
							
							\item $x$が$y$とも$z$とも違うならば,$x$は$\chi$か$\psi$の
								少なくとも一方には自由に現れている.
								このとき$\widehat{\varphi}(x/\tau)$は
								\begin{align}
									\forall u\, (\, \chi(y/u)(x/\tau) \lrarrow \psi(z/u)(x/\tau)\, )
								\end{align}
								となるが,書き換えの変項条件より
								$\widehat{\varphi}(x/\tau)$は
								\begin{align}
									\forall u\, (\, \chi(x/\tau)(y/u) \lrarrow \psi(x/\tau)(z/u)\, )
								\end{align}
								と同じである.他方で$\varphi(x/\tau)$は
								\begin{align}
									\Set{y}{\chi(x/\tau)} = \Set{z}{\psi(x/\tau)}
								\end{align}
								であるから$\widehat{\varphi}(x/\tau)$は
								$\varphi(x/\tau)$の書き換えである.
						\end{itemize}
						
					\item[case6] $\varphi$が
						\begin{align}
							x \in \Set{z}{\psi}
						\end{align}
						なる式のとき,必要ならば$\psi$の変項の
						名前替えをしたものを$\widetilde{\psi}$とする.ただし
						名前替えをしなかったら$\widetilde{\psi}$は$\psi$とする.
						$\widehat{\varphi}$は$\widetilde{\psi}(z/x)$なる式であるから
						$\widehat{\varphi}(x/\tau)$は$\widetilde{\psi}(z/x)(x/\tau)$
						となる.
						\begin{itemize}
							\item $x$と$z$が同じならば$\psi$の変項の名前替えは必要ない.
								$\psi(z/x)$とは$\psi$そのものであるから,
								$\psi(z/x)(x/\tau)$は$\psi(x/\tau)$となる.
								他方で$\varphi(x/\tau)$は
								\begin{align}
									\tau \in \Set{z}{\psi}
								\end{align}
								となるから,$\psi(z/\tau)$は$\varphi(x/\tau)$の書き換えである.
								
							\item $x$と$z$が違うとき,$\widetilde{\psi}(z/x)(x/\tau)$は
								$\widetilde{\psi}(x/\tau)(z/\tau)$である.他方で
								$\varphi(x/\tau)$は
								\begin{align}
									\tau \in \Set{z}{\psi(x/\tau)}
								\end{align}
								となるから,$\psi(x/\tau)(z/\tau)$は$\varphi(x/\tau)$の書き換えとなる.
								$\psi(x/\tau)(z/\tau)$と$\widetilde{\psi}(x/\tau)(z/\tau)$は,
								もし違う式でも$\forall x$か$\exists x$から始まる或る部分式が違うだけであるから,
								$\widetilde{\psi}(x/\tau)(z/\tau)$もまた$\varphi(x/\tau)$の書き換えである.
						\end{itemize}
						
					\item[case7] $\varphi$が
						\begin{align}
							a \in \Set{z}{\psi}
						\end{align}
						なる式のとき($a$は$x$とは違う項),必要ならば$\psi$の変項の
						名前替えをしたものを$\widetilde{\psi}$とする.ただし
						名前替えをしなかったら$\widetilde{\psi}$は$\psi$とする.
						$\widehat{\varphi}$は$\widetilde{\psi}(z/a)$なる式であるから
						$\widehat{\varphi}(x/\tau)$は$\widetilde{\psi}(z/a)(x/\tau)$
						となる.$\varphi$には$x$が自由に現れているので,つまり
						$x$は$\psi$に自由に現れているから$x$は$z$とも違う変項である.
						従って$\widetilde{\psi}(z/a)(x/\tau)$は
						$\widetilde{\psi}(x/\tau)(z/a)$である.他方で
						$\varphi(x/\tau)$は
						\begin{align}
							a \in \Set{z}{\psi(x/\tau)}
						\end{align}
						となるから,$\widetilde{\psi}(x/\tau)(z/a)$は$\varphi(x/\tau)$の書き換えとなる.
					
					\item[case8] $\varphi$が
						\begin{align}
							\Set{y}{\chi} \in x
						\end{align}
						なる式のとき,$\widehat{\varphi}$は
						\begin{align}
							\exists s\, (\, \forall u\, (\, \chi(y/u) \lrarrow u \in s\, ) \wedge s \in x\, )
						\end{align}
						なる式である.
						\begin{itemize}
							\item $x$と$y$が同じならば$\widehat{\varphi}(x/\tau)$は
								\begin{align}
									\exists s\, (\, \forall u\, (\, \chi(y/u) \lrarrow u \in s\, ) \wedge s \in \tau\, )
								\end{align}
								となる.他方で$\varphi(x/\tau)$は
								\begin{align}
									\Set{y}{\chi} \in \tau
								\end{align}
								であるから$\widehat{\varphi}(x/\tau)$は
								$\varphi(x/\tau)$の書き換えである.
								
							\item $x$と$y$が違うとき$\widehat{\varphi}(x/\tau)$は,
								$x$が$\chi$に自由に現れているならば
								\begin{align}
									\exists s\, (\, \forall u\, (\, \chi(y/u)(x/\tau) \lrarrow u \in s\, ) \wedge s \in \tau\, )
								\end{align}
								となるが,書き換えの変項条件より$x$は$u$とも違うので
								$\chi(y/u)(x/\tau)$と$\chi(x/\tau)(y/u)$は
								同じである.従って$\widehat{\varphi}(x/\tau)$は
								\begin{align}
									\exists s\, (\, \forall u\, (\, \chi(x/\tau)(y/u) \lrarrow u \in s\, ) \wedge s \in \tau\, )
								\end{align}
								と同じである.他方で$\varphi(x/\tau)$は
								\begin{align}
									\Set{y}{\chi(x/\tau)} \in \tau
								\end{align}
								であるから,この場合は$\widehat{\varphi}(x/\tau)$は
								$\varphi(x/\tau)$の書き換えである.
								$x$が$\chi$に自由に現れていない場合,
								$\widehat{\varphi}(x/\tau)$は
								\begin{align}
									\exists s\, (\, \forall u\, (\, \chi(y/u) \lrarrow u \in s\, ) \wedge s \in \tau\, )
								\end{align}
								となり,$\varphi(x/\tau)$は
								\begin{align}
									\Set{y}{\chi} \in \tau
								\end{align}
								であるからこの場合も$\widehat{\varphi}(x/\tau)$は
								$\varphi(x/\tau)$の書き換えである.
						\end{itemize}
					
					\item[case9] $\varphi$が
						\begin{align}
							\Set{y}{\chi} \in b
						\end{align}
						なる式のとき($b$は$x$と違う項),$\widehat{\varphi}$は
						\begin{align}
							\exists s\, (\, \forall u\, (\, \chi(y/u) \lrarrow u \in s\, ) \wedge s \in b\, )
						\end{align}
						なる式である.$\varphi$には$x$が自由に現れているので,つまり
						$x$は$\chi$に自由に現れている.従って$\widehat{\varphi}(x/\tau)$は
						\begin{align}
							\exists s\, (\, \forall u\, (\, \chi(y/u)(x/\tau) \lrarrow u \in s\, ) \wedge s \in b\, )
						\end{align}
						となるが,書き換えの変項条件より$x$は$u$とも違うので
						$\chi(y/u)(x/\tau)$と$\chi(x/\tau)(y/u)$は
						同じである.従って$\widehat{\varphi}(x/\tau)$は
						\begin{align}
							\exists s\, (\, \forall u\, (\, \chi(x/\tau)(y/u) \lrarrow u \in s\, ) \wedge s \in b\, )
						\end{align}
						と同じである.他方で$\varphi(x/\tau)$は
						\begin{align}
							\Set{y}{\chi(x/\tau)} \in b
						\end{align}
						であるから,$\widehat{\varphi}(x/\tau)$は
						$\varphi(x/\tau)$の書き換えである.
						
					\item[case10] $\varphi$が
						\begin{align}
							\Set{y}{\chi} \in \Set{z}{\psi}
						\end{align}
						なる式のとき,$\widehat{\varphi}$は
						\begin{align}
							\exists s\, (\, \forall u\, (\, \chi(y/u) \lrarrow u \in s\, ) \wedge \psi(z/s)\, )
						\end{align}
						なる式である.
						\begin{itemize}
							\item $x$と$y$が同じならば,$x$は$\Set{y}{\chi}$には自由に
								現れないので,$x$が$\varphi$に自由に現れている以上
								$\psi$に自由に現れることになる.すなわち$x$と$z$は違う項である.
								このとき$\widehat{\varphi}(x/\tau)$は
								\begin{align}
									\exists s\, (\, \forall u\, (\, \chi(y/u) \lrarrow u \in s\, ) \wedge \psi(z/s)(x/\tau)\, )
								\end{align}
								となるが,書き換えの変項条件より$x$は$s$とも違うので
								$\psi(z/s)(x/\tau)$と$\psi(x/\tau)(z/s)$は
								同じである.従って$\widehat{\varphi}(x/\tau)$は
								\begin{align}
									\exists s\, (\, \forall u\, (\, \chi(y/u) \lrarrow u \in s\, ) \wedge \psi(x/\tau)(z/s)\, )
								\end{align}
								と同じである.他方で$\varphi(x/\tau)$は
								\begin{align}
									\Set{y}{\chi} \in \Set{z}{\psi(x/\tau)}
								\end{align}
								であるから,$\widehat{\varphi}(x/\tau)$は
								$\varphi(x/\tau)$の書き換えである.
								
							\item $x$と$z$が同じならば,$x$は$\Set{z}{\psi}$には自由に
								現れないので,$x$が$\varphi$に自由に現れている以上
								$\chi$に自由に現れることになる.すなわち$x$と$y$は違う項である.
								このとき$\widehat{\varphi}(x/\tau)$は
								\begin{align}
									\exists s\, (\, \forall u\, (\, \chi(y/u)(x/\tau) \lrarrow u \in s\, ) \wedge \psi(z/s)\, )
								\end{align}
								となるが,書き換えの変項条件より$x$は$s$とも違うので
								$\chi(y/u)(x/\tau)$と$\chi(x/\tau)(y/u)$は
								同じである.従って$\widehat{\varphi}(x/\tau)$は
								\begin{align}
									\exists s\, (\, \forall u\, (\, \chi(x/\tau)(y/u) \lrarrow u \in s\, ) \wedge \psi(z/s)\, )
								\end{align}
								と同じである.他方で$\varphi(x/\tau)$は
								\begin{align}
									\Set{y}{\chi(x/\tau)} \in \Set{z}{\psi}
								\end{align}
								であるから,$\widehat{\varphi}(x/\tau)$は
								$\varphi(x/\tau)$の書き換えである.
								
							\item $x$が$y$とも$z$とも違うならば,$x$は$\chi$か$\psi$の
								少なくとも一方にはには自由に現れている.
								このとき$\widehat{\varphi}(x/\tau)$は
								\begin{align}
									\exists s\, (\, \forall u\, (\, \chi(y/u)(x/\tau) \lrarrow u \in s\, ) \wedge \psi(z/s)(x/\tau)\, )
								\end{align}
								となるが,書き換えの変項条件より
								$\widehat{\varphi}(x/\tau)$は
								\begin{align}
									\exists s\, (\, \forall u\, (\, \chi(x/\tau)(y/u) \lrarrow u \in s\, ) \wedge \psi(x/\tau)(z/s)\, )
								\end{align}
								と同じである.他方で$\varphi(x/\tau)$は
								\begin{align}
									\Set{y}{\chi(x/\tau)} \in \Set{z}{\psi(x/\tau)}
								\end{align}
								であるから,$\widehat{\varphi}(x/\tau)$は
								$\varphi(x/\tau)$の書き換えである.
						\end{itemize}
				\end{description}
			
			\item[step2] $\varphi$が一般の式であるとき,
				\begin{itembox}[l]{IH (帰納法の仮定)}
					$\varphi$の任意の真部分式$\psi$に対して,
					$\psi$が$\lang{\varepsilon}$の式でなければ,
					$\widehat{\psi}$を$\psi$の書き換えとすれば
					$\widehat{\psi}(x/\tau)$は$\psi(x/\tau)$の書き換えである.
				\end{itembox}
				と仮定する
				\footnote{
					メタ定理\ref{metathm:variables_unchanged_after_rewriting}より
					$\psi$に$x$が自由に現れていなければ$\widehat{\psi}$にも
					$x$は自由に現れないので,$\psi$に$x$が自由に現れていない場合は
					$\psi(x/\tau)$は$\psi$であり,$\widehat{\psi}(x/\tau)$は
					$\widehat{\psi}$である.
				}.
				
				\begin{description}
					\item[case1] $\varphi$が
						\begin{align}
							\negation \psi
						\end{align}
						なる式である場合,メタ定理\ref{metathm:relation_to_subformula_rewriting}より$\widehat{\varphi}$は
						\begin{align}
							\negation \widehat{\psi}
						\end{align}
						なる形で書けて,$\widehat{\psi}$は$\psi$の書き換えである.
						(IH)より$\widehat{\psi}(x/\tau)$は
						$\psi(x/\tau)$の書き換えであるから,
						再びメタ定理\ref{metathm:relation_to_subformula_rewriting}より
						$\negation \widehat{\psi}(x/\tau)$は
						$\negation \psi(x/\tau)$の書き換えである.
						$\negation \widehat{\psi}(x/\tau)$とは
						$\widehat{\varphi}(x/\tau)$のことであり,
						$\negation \psi(x/\tau)$とは$\varphi(x/\tau)$のことであるから,
						$\widehat{\varphi}(x/\tau)$は$\varphi(x/\tau)$の書き換えである.
					
					\item[case2] $\varphi$が
						\begin{align}
							\vee \psi \chi
						\end{align}
						なる式である場合,メタ定理\ref{metathm:relation_to_subformula_rewriting}より$\widehat{\varphi}$は
						\begin{align}
							\vee \widehat{\psi} \widehat{\chi}
						\end{align}
						なる形で書けて,$\widehat{\psi}$は$\psi$の書き換えであり,
						$\widehat{\chi}$は$\chi$の書き換えである.
						(IH)より$\widehat{\psi}(x/\tau)$は
						$\psi(x/\tau)$の書き換えであり,また$\widehat{\chi}(x/\tau)$は
						$\chi(x/\tau)$の書き換えであるから,
						再びメタ定理\ref{metathm:relation_to_subformula_rewriting}より
						$\vee \widehat{\psi}(x/\tau)\widehat{\chi}(x/\tau)$は
						$\vee \psi(x/\tau)\chi(x/\tau)$の書き換えである..
						$\vee \widehat{\psi}(x/\tau)\widehat{\chi}(x/\tau)$とは
						$\widehat{\varphi}(x/\tau)$のことであり,
						$\vee \psi(x/\tau)\chi(x/\tau)$とは
						$\varphi(x/\tau)$のことであるから,
						$\widehat{\varphi}(x/\tau)$は$\varphi(x/\tau)$の書き換えである.
					
					\item[case3] $\varphi$が
						\begin{align}
							\exists x \psi
						\end{align}
						なる式である場合,メタ定理\ref{metathm:relation_to_subformula_rewriting}より$\widehat{\varphi}$は
						\begin{align}
							\exists y \widehat{\psi}
						\end{align}
						なる形で書けて,$\widehat{\psi}$は$\psi$の書き換えである.
						(IH)より$\widehat{\psi}(x/\tau)$は
						$\psi(x/\tau)$の書き換えであるから,
						再びメタ定理\ref{metathm:relation_to_subformula_rewriting}より
						$\exists y \widehat{\psi}(x/\tau)$は
						$\exists y \psi(x/\tau)$の書き換えである.
						$\exists y \widehat{\psi}(x/\tau)$とは
						$\widehat{\varphi}(x/\tau)$のことであり,
						$\exists y \psi(x/\tau)$とは$\varphi(x/\tau)$のことであるから,
						$\widehat{\varphi}(x/\tau)$は$\varphi(x/\tau)$の書き換えである.
						\QED
				\end{description}
		\end{description}
	\end{metaprf}
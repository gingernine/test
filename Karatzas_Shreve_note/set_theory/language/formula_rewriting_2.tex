	\begin{screen}
		\begin{metathm}[部分式の差し替えと代入]
		\label{metathm:subformula_replacing_and_substitution}
			$\varphi$を$\lang{\varepsilon}$の式とし,$x$を$\varphi$に自由に現れる変項とする.
			また$\varphi$に$\forall z \xi$ (resp. $\exists z \xi$)の
			形の部分式が現れているとし,$y$を$\xi$に自由に現れない変項で$\xi$の中で$z$への
			代入について自由であるものとし,$\varphi$の$\forall z \xi$ 
			(resp. $\exists z \xi$)の部分を一か所だけ$\forall y \xi(z/y)$ 
			(resp. $\exists y \xi(z/y)$)に差し替えた式を$\widetilde{\varphi}$とする.
			それから$\tau$を主要$\varepsilon$項とする.このとき,
			\begin{description}
				\item[(1)] $\varphi$における$x$の自由な出現が,差し替えられる$\forall z \xi$ 
					(resp. $\exists z \xi$)の中にある場合\footnotemark,
					$\widetilde{\varphi}(x/\tau)$は$\varphi(x/\tau)$の部分式$\forall z \xi(x/\tau)$ 
					(resp. $\exists z \xi(x/\tau)$)を$\forall y \xi(x/\tau)(z/y)$ 
					(resp. $\exists y \xi(x/\tau)(z/y)$)に差し替えた式である.
					
				\item[(2)] $\varphi$における$x$の自由な出現が,差し替えられる$\forall z \xi$ 
					(resp. $\exists z \xi$)の中に無い場合,$\widetilde{\varphi}(x/\tau)$
					は$\varphi(x/\tau)$の部分式$\forall z \xi$ 
					(resp. $\exists z \xi$)を$\forall y \xi(z/y)$ 
					(resp. $\exists y \xi(z/y)$)に差し替えた式である.
			\end{description}
		\end{metathm}
	\end{screen}
	
	\footnotetext{
		$\varphi$における$x$の自由な出現が差し替えられる$\forall z \xi$ (resp. $\exists z \xi$)の中にある場合,
		$\xi$は$\forall x$或いは$\exists x$から始まる$\varphi$の部分式の内部には無い.
		従って$\xi$に自由に現れる$x$は全て$\varphi$にも自由に現れる.
	}
	
	\begin{metaprf} 差し替えられる$\varphi$の部分式が$\forall z \xi$だとして示すが,
		$\exists z \xi$に替えても同じである.
		\begin{description}
			\item[step1] $\varphi$が$\forall z \xi$なる式である場合,
				$x$は$\varphi$に自由に現れているので$x$は$z$ではない.
				$\widetilde{\varphi}$とは$\forall y \xi(z/y)$なる式であるが,
				$y$の選び方より$x$は$y$でもない.
				すなわち$x$は$\widetilde{\varphi}$にも自由に現れている.
				%また代入条件より$\tau$もまた$z$でも$y$でもない.
				$\widetilde{\varphi}(x/\tau)$とは
				\begin{align}
					\forall y \xi(z/y)(x/\tau)
				\end{align}
				なる式であるが,いま$\xi(z/y)(x/\tau)$と$\xi(x/\tau)(z/y)$は同じ式なので,
				$\widetilde{\varphi}(x/\tau)$は
				\begin{align}
					\forall y \xi(x/\tau)(z/y)
				\end{align}
				と同じ式である.いまの場合$\varphi(x/\tau)$は$\forall z \xi(x/\tau)$であるから
				(1)の主張が成り立つ.
				
			\item[step2]\mbox{}
				\begin{itembox}[l]{IH (帰納法の仮定)}
					$\varphi$の任意の真部分式$\psi$に対して,
					差し替えられる$\forall z \xi$が$\psi$に部分式として現れているとき,
					$\psi$の$\forall z \xi$を$\forall y \xi(z/y)$ 
					に差し替えた式を$\widetilde{\psi}$とする.このとき
					\begin{description}
						\item[(1)] $\varphi$における$x$の自由な出現が,
							差し替えられる$\forall z \xi$の中にある場合,
							$\widetilde{\psi}(x/\tau)$は$\psi(x/\tau)$の部分式
							$\forall z \xi(x/\tau)$ 
							を$\forall y \xi(x/\tau)(z/y)$に差し替えた式である.
						
						\item[(2)] $\varphi$における$x$の自由な出現が,
							差し替えられる$\forall z \xi$の中に無い場合,
							$\widetilde{\psi}(x/\tau)$は$\psi(x/\tau)$の部分式
							$\forall z \xi$ を$\forall y \xi(z/y)$に差し替えた式である.
					\end{description}
				\end{itembox}
				と仮定する.このとき,
				\begin{description}
					\item[case1] $\varphi$が
						\begin{align}
							\negation \psi
						\end{align}
						なる式である場合,差し替えられる$\forall z \xi$は$\psi$に現れる.
						$\varphi(x/\tau)$は
						\begin{align}
							\negation \psi(x/\tau)
						\end{align}
						なる式であり,$\widetilde{\varphi}$は
						\begin{align}
							\negation \widetilde{\psi}
						\end{align}
						なる式であり,$\widetilde{\varphi}(x/\tau)$は
						\begin{align}
							\negation \widetilde{\psi}(x/\tau)
						\end{align}
						なる式である.
						\begin{description}
							\item[(1)] $\varphi$における$x$の自由な出現が
								差し替えられる$\forall z \xi$の中にある場合,
								$\varphi(x/\tau)$の部分式
								$\forall z \xi(x/\tau)$を
								$\forall y \xi(x/\tau)(z/y)$に差し替えた式は,
								(IH)より
								\begin{align}
									\negation \widetilde{\psi}(x/\tau)
								\end{align}
								と同じ式である.従ってその式は
								$\widetilde{\varphi}(x/\tau)$とも同じ式である.
								
							\item[(2)] $\varphi$における$x$の自由な出現が
								差し替えられる$\forall z \xi$の中に無い場合,
								$\varphi(x/\tau)$の部分式$\forall z \xi$を
								$\forall y \xi(z/y)$に差し替えた式は,(IH)より
								\begin{align}
									\negation \widetilde{\psi}(x/\tau)
								\end{align}
								と同じ式である.従ってその式は
								$\widetilde{\varphi}(x/\tau)$とも同じ式である.
						\end{description}
						
					\item[case2] $\varphi$が
						\begin{align}
							\vee \psi \chi
						\end{align}
						なる式である場合,差し替えられる$\forall z \xi$は$\psi$か$\chi$
						のどちらか一方に現れるが,ここでは
						$\psi$の側に現れているとする.$\varphi(x/\tau)$は
						\begin{align}
							\vee \psi(x/\tau) \chi(x/\tau)
						\end{align}
						なる式であり,$\widetilde{\varphi}$は
						\begin{align}
							\vee \widetilde{\psi} \chi
						\end{align}
						なる式であり,$\widetilde{\varphi}(x/\tau)$は
						\begin{align}
							\vee \widetilde{\psi}(x/\tau) \chi(x/\tau)
						\end{align}
						なる式である.
						\begin{description}
							\item[(1)] $\varphi$における$x$の自由な出現が
								差し替えられる$\forall z \xi$の中にある場合,
								$\varphi(x/\tau)$の部分式
								$\forall z \xi(x/\tau)$を
								$\forall y \xi(x/\tau)(z/y)$に差し替えた式は,
								(IH)より
								\begin{align}
									\vee \widetilde{\psi}(x/\tau) \chi(x/\tau)
								\end{align}
								と同じ式である.従ってその式は
								$\widetilde{\varphi}(x/\tau)$とも同じ式である.
								
							\item[(2)] $\varphi$における$x$の自由な出現が
								差し替えられる$\forall z \xi$の中に無い場合,
								$\varphi(x/\tau)$の部分式$\forall z \xi$を
								$\forall y \xi(z/y)$に差し替えた式は,(IH)より
								\begin{align}
									\vee \widetilde{\psi}(x/\tau) \chi(x/\tau)
								\end{align}
								と同じ式である.従ってその式は
								$\widetilde{\varphi}(x/\tau)$とも同じ式である.
						\end{description}
						
					\item[case3] $\varphi$が
						\begin{align}
							\exists w \psi
						\end{align}
						なる式である場合,差し替えられる$\forall z \xi$は$\psi$に現れる.
						$\varphi(x/\tau)$は
						\begin{align}
							\exists w \psi(x/\tau)
						\end{align}
						なる式であり,$\widetilde{\varphi}$は
						\begin{align}
							\exists w \widetilde{\psi}
						\end{align}
						なる式であり,$\widetilde{\varphi}(x/\tau)$は
						\begin{align}
							\exists w \widetilde{\psi}(x/\tau)
						\end{align}
						なる式である.
						\begin{description}
							\item[(1)] $\varphi$における$x$の自由な出現が
								差し替えられる$\forall z \xi$の中にある場合,
								$\varphi(x/\tau)$の部分式
								$\forall z \xi(x/\tau)$を
								$\forall y \xi(x/\tau)(z/y)$に差し替えた式は,
								(IH)より
								\begin{align}
									\exists w \widetilde{\psi}(x/\tau)
								\end{align}
								と同じ式である.従ってその式は
								$\widetilde{\varphi}(x/\tau)$とも同じ式である.
								
							\item[(2)] $\varphi$における$x$の自由な出現が
								差し替えられる$\forall z \xi$の中に無い場合,
								$\varphi(x/\tau)$の部分式$\forall z \xi$を
								$\forall y \xi(z/y)$に差し替えた式は,(IH)より
								\begin{align}
									\exists w \widetilde{\psi}(x/\tau)
								\end{align}
								と同じ式である.従ってその式は
								$\widetilde{\varphi}(x/\tau)$とも同じ式である.
								\QED
						\end{description}
				\end{description}
		\end{description}
	\end{metaprf}
	
	%\footnotetext{
	%	「任意の真部分式に対し,それが直部分式ならば...」と書き改めれば構造的帰納法の原理を適用できる
	%	が,始めから直部分式と限定しても実質的には変わらない.
	%}
	
	\begin{screen}
		\begin{metathm}[書き換えへの代入は代入した式の書き換え]
		\label{metathm:substitution_to_rewritten_formula}
			$\varphi$を$\lang{\varepsilon}$の式ではない$\mathcal{L}$の式とし,
			$\varphi$には変項$x$が自由に現れているとし,$\tau$を主要$\varepsilon$項とし,
			$\widehat{\varphi}$を$\varphi$の書き換えとする\footnotemark
			.このとき$\widehat{\varphi}(x/\tau)$は$\varphi(x/\tau)$の書き換えである.
		\end{metathm}
	\end{screen}
	
	\footnotetext{
		定理\ref{metathm:variables_unchanged_after_rewriting}より
		$\widehat{\varphi}$にも$x$は自由に現れている.
	}
	
	証明が長いので第一証明と第二証明に分割する.第一証明では$\widehat{\varphi}$が$\varphi$の
	部分式で原子式であるものを全て表\ref{tab:formula_rewriting}の通りに直した式である場合を扱い,
	第二証明では「式の書き換えによる構造的帰納法」のセカンドステップを扱う.
	
	\begin{metaprf}[第一] $\widehat{\varphi}$が$\varphi$の部分式で原子式であるものを全て
		表\ref{tab:formula_rewriting}の通りに直した式であるとき,$\widehat{\varphi}(x/\tau)$
		が$\varphi(x/\tau)$の書き換えであることを示す.
		\begin{description}
			\item[step1] $\varphi$が原子式であるとする.
				\begin{description}
					\item[case1] $\varphi$が
						\begin{align}
							x = \Set{z}{\psi}
						\end{align}
						なる式のとき,$\widehat{\varphi}$は
						\begin{align}
							\forall v\, (\, v \in x \lrarrow \psi(z/v)\, )
						\end{align}
						なる式である.
						\begin{itemize}
							\item $x$と$z$が同じであるとする.
								このとき$\widehat{\varphi}(x/\tau)$は
								\begin{align}
									\forall v\, (\, v \in \tau \lrarrow \psi(z/v)\, )
								\end{align}
								となる.他方で$\varphi(x/\tau)$は
								\begin{align}
									\tau = \Set{z}{\psi}
								\end{align}
								であるから$\widehat{\varphi}(x/\tau)$は
								$\varphi(x/\tau)$の書き換えである.
								
							\item $x$と$z$が違うとする.このとき
								\begin{itemize}
									\item $x$が$\Set{z}{\psi}$に自由に現れている場合,
										$\widehat{\varphi}(x/\tau)$は
										\begin{align}
											\forall v\, (\, v \in \tau \lrarrow \psi(z/v)(x/\tau)\, )
										\end{align}
										となるが,書き換えの変項条件より$x$は$v$とも違うので,
										%代入条件より$\tau$もまた$z$とも$v$とも違うので,
										$\psi(z/v)(x/\tau)$と$\psi(x/\tau)(z/v)$は
										同じ式である.従って$\widehat{\varphi}(x/\tau)$は
										\begin{align}
											\forall v\, (\, v \in \tau \lrarrow \psi(x/\tau)(z/v)\, )
										\end{align}
										と同じ式である.他方で$\varphi(x/\tau)$は
										\begin{align}
											\tau = \Set{z}{\psi(x/\tau)}
										\end{align}
										であるから,この場合は
										$\widehat{\varphi}(x/\tau)$は
										$\varphi(x/\tau)$の書き換えである.
										
									\item $x$が$\Set{z}{\psi}$に自由に現れていない場合,
										$\widehat{\varphi}(x/\tau)$は
										\begin{align}
											\forall v\, (\, v \in \tau \lrarrow \psi(z/v)\, )
										\end{align}
										となるが,$\varphi(x/\tau)$は
										\begin{align}
											\tau = \Set{z}{\psi}
										\end{align}
										であるからこの場合も
										$\widehat{\varphi}(x/\tau)$は
										$\varphi(x/\tau)$の書き換えである.
								\end{itemize}
						\end{itemize}
						
					\item[case2] $\varphi$が
						\begin{align}
							a = \Set{z}{\psi}
						\end{align}
						なる式のとき($a$と$x$は違う$\lang{\varepsilon}$の項),
						$\widehat{\varphi}$は
						\begin{align}
							\forall v\, (\, v \in a \lrarrow \psi(z/v)\, )
						\end{align}
						なる式である.$\varphi$には$x$が自由に現れているので,つまり
						$x$は$z$ではなく,また$\Set{z}{\psi}$に自由に現れている.従って
						$\widehat{\varphi}(x/\tau)$は
						\begin{align}
							\forall v\, (\, v \in a \lrarrow \psi(z/v)(x/\tau)\, )
						\end{align}
						となるが,書き換えの変項条件より$x$は$v$とも違うので,
						%代入条件より$\tau$もまた$z$とも$v$とも違うので,
						$\psi(z/v)(x/\tau)$と$\psi(x/\tau)(z/v)$は
						同じ式である.従って$\widehat{\varphi}(x/\tau)$は
						\begin{align}
							\forall v\, (\, v \in a \lrarrow \psi(x/\tau)(z/v)\, )
						\end{align}
						と同じ式である.他方で$\varphi(x/\tau)$は
						\begin{align}
							a = \Set{z}{\psi(x/\tau)}
						\end{align}
						であるから$\widehat{\varphi}(x/\tau)$は
						$\varphi(x/\tau)$の書き換えである.
					
					\item[case3] $\varphi$が
						\begin{align}
							\Set{y}{\xi} = x
						\end{align}
						なる式のとき,$\widehat{\varphi}$は
						\begin{align}
							\forall u\, (\, \xi(y/u) \lrarrow u \in x\, )
						\end{align}
						なる式である.
						\begin{itemize}
							\item $x$と$y$が同じであるとする.このとき
								$\widehat{\varphi}(x/\tau)$は
								\begin{align}
									\forall u\, (\, \xi(y/u) \lrarrow u \in \tau\, )
								\end{align}
								となる.他方で$\varphi(x/\tau)$は
								\begin{align}
									\Set{y}{\xi} = \tau
								\end{align}
								であるから$\widehat{\varphi}(x/\tau)$は
								$\varphi(x/\tau)$の書き換えである.
								
							\item $x$と$y$が違うとする.このとき
								\begin{itemize}
									\item $x$が$\Set{y}{\xi}$に自由に現れていれば,
										$\widehat{\varphi}(x/\tau)$は
										\begin{align}
											\forall u\, (\, \xi(y/u)(x/\tau) \lrarrow u \in \tau\, )
										\end{align}
										となるが,書き換えの変項条件より$x$は$u$とも違うので,
										%代入条件より$\tau$もまた$y$とも$u$とも違うので,
										$\xi(y/u)(x/\tau)$と$\xi(x/\tau)(y/u)$は
										同じ式である.従って$\widehat{\varphi}(x/\tau)$は
										\begin{align}
											\forall u\, (\, \xi(x/\tau)(y/u) \lrarrow u \in \tau\, )
										\end{align}
										と同じ式である.他方で$\varphi(x/\tau)$は
										\begin{align}
											\Set{y}{\xi(x/\tau)} = \tau
										\end{align}
										であるから,この場合は
										$\widehat{\varphi}(x/\tau)$は
										$\varphi(x/\tau)$の書き換えである.
								
									\item $x$が$\Set{y}{\xi}$に自由に現れていない場合,
										$\widehat{\varphi}(x/\tau)$は
										\begin{align}
											\forall u\, (\, \xi(y/u) \lrarrow u \in \tau\, )
										\end{align}
										となるが,$\varphi(x/\tau)$は
										\begin{align}
											\Set{y}{\xi} = \tau
										\end{align}
										であるからこの場合も$\widehat{\varphi}(x/\tau)$は
										$\varphi(x/\tau)$の書き換えである.
								\end{itemize}
						\end{itemize}
						
					\item[case4] $\varphi$が
						\begin{align}
							\Set{y}{\xi} = b
						\end{align}
						なる式のとき($b$は$x$と違う$\lang{\varepsilon}$の項),
						$\widehat{\varphi}$は
						\begin{align}
							\forall u\, (\, \xi(y/u) \lrarrow u \in b\, )
						\end{align}
						なる式である.$\varphi$には$x$が自由に現れているので,つまり
						$x$は$y$ではなく,また$\Set{y}{\xi}$に自由に現れている.
						従って$\widehat{\varphi}(x/\tau)$は
						\begin{align}
							\forall u\, (\, \xi(y/u)(x/\tau) \lrarrow u \in b\, )
						\end{align}
						となるが,書き換えの変項条件より$x$は$u$とも違うので,
						%代入条件より$\tau$もまた$y$とも$u$とも違うので,
						$\xi(y/u)(x/\tau)$と$\xi(x/\tau)(y/u)$は
						同じ式である.従って$\widehat{\varphi}(x/\tau)$は
						\begin{align}
							\forall u\, (\, \xi(x/\tau)(y/u) \lrarrow u \in b\, )
						\end{align}
						と同じ式である.他方で$\varphi(x/\tau)$は
						\begin{align}
							\Set{y}{\xi(x/\tau)} = b
						\end{align}
						であるから$\widehat{\varphi}(x/\tau)$は
						$\varphi(x/\tau)$の書き換えである.
					
					\item[case5] $\varphi$が
						\begin{align}
							\Set{y}{\xi} = \Set{z}{\psi}
						\end{align}
						なる式のとき,$\widehat{\varphi}$は
						\begin{align}
							\forall u\, (\, \xi(y/u) \lrarrow \psi(z/u)\, )
						\end{align}
						なる式である.
						\begin{itemize}
							\item $x$と$y$が同じであるとする.このとき
								$x$は$\Set{y}{\xi}$には自由に
								現れないので,$x$が$\varphi$に自由に現れている以上
								$\Set{z}{\psi}$に自由に現れることになる.
								すなわち$x$と$z$は違う項である.
								このとき$\widehat{\varphi}(x/\tau)$は
								\begin{align}
									\forall u\, (\, \xi(y/u) \lrarrow \psi(z/u)(x/\tau)\, )
								\end{align}
								となるが,書き換えの変項条件より$x$は$u$とも違うので,
								%代入条件より$\tau$もまた$z$とも$u$とも違うので,
								$\psi(z/u)(x/\tau)$と$\psi(x/\tau)(z/u)$は
								同じ式である.従って$\widehat{\varphi}(x/\tau)$は
								\begin{align}
									\forall u\, (\, \xi(y/u) \lrarrow \psi(x/\tau)(z/u)\, )
								\end{align}
								と同じ式である.他方で$\varphi(x/\tau)$は
								\begin{align}
									\Set{y}{\xi} = \Set{z}{\psi(x/\tau)}
								\end{align}
								であるから$\widehat{\varphi}(x/\tau)$は
								$\varphi(x/\tau)$の書き換えである.
								
							\item $x$と$y$と違い,$x$と$z$が同じであるとする.
								$x$が$\varphi$に自由に現れている以上
								$x$は$\Set{y}{\xi}$に自由に現れることになるから,
								$\widehat{\varphi}(x/\tau)$は
								\begin{align}
									\forall u\, (\, \xi(y/u)(x/\tau) \lrarrow \psi(z/u)\, )
								\end{align}
								となるが,書き換えの変項条件より$x$は$u$とも違うので,
								%代入条件より$\tau$もまた$y$とも$u$とも違うので,
								$\xi(y/u)(x/\tau)$と$\xi(x/\tau)(y/u)$は
								同じ式である.従って$\widehat{\varphi}(x/\tau)$は
								\begin{align}
									\forall u\, (\, \xi(x/\tau)(y/u) \lrarrow \psi(z/u)\, )
								\end{align}
								と同じ式である.他方で$\varphi(x/\tau)$は
								\begin{align}
									\Set{y}{\xi(x/\tau)} = \Set{z}{\psi}
								\end{align}
								であるから$\widehat{\varphi}(x/\tau)$は
								$\varphi(x/\tau)$の書き換えである.
							
							\item $x$が$y$とも$z$とも違うとする,このとき
								$x$は$\Set{y}{\xi}$か$\Set{z}{\psi}$の少なくとも
								一方には自由に現れている.
								このとき$\widehat{\varphi}(x/\tau)$は
								\begin{align}
									\forall u\, (\, \xi(y/u)(x/\tau) \lrarrow \psi(z/u)(x/\tau)\, )
								\end{align}
								となるが,書き換えの変項条件より$x$は$u$とも違うので,
								$\widehat{\varphi}(x/\tau)$は
								\begin{align}
									\forall u\, (\, \xi(x/\tau)(y/u) \lrarrow \psi(x/\tau)(z/u)\, )
								\end{align}
								と同じ式である.他方で$\varphi(x/\tau)$は
								\begin{align}
									\Set{y}{\xi(x/\tau)} = \Set{z}{\psi(x/\tau)}
								\end{align}
								であるから$\widehat{\varphi}(x/\tau)$は
								$\varphi(x/\tau)$の書き換えである.
						\end{itemize}
						
					\item[case6] $\varphi$が
						\begin{align}
							x \in \Set{z}{\psi}
						\end{align}
						なる式のとき,$\widehat{\varphi}$を得るために必要ならば$\psi$の変項の
						名前替えをしたものを$\widetilde{\psi}$とする.ただし
						名前替えをしなかったら$\widetilde{\psi}$は$\psi$とする.
						$\widehat{\varphi}$は$\widetilde{\psi}(z/x)$なる式であり,
						$\widehat{\varphi}(x/\tau)$は$\widetilde{\psi}(z/x)(x/\tau)$
						となる.
						\begin{itemize}
							\item $x$と$z$が同じであるとする.
								このときは$\psi$の変項の名前替えは必要ない.
								$\widehat{\varphi}$とは$\psi$そのものであり,
								$\widehat{\varphi}(x/\tau)$は$\psi(z/\tau)$と同じ式である.
								他方で$\varphi(x/\tau)$は
								\begin{align}
									\tau \in \Set{z}{\psi}
								\end{align}
								となるから,$\psi(z/\tau)$は$\varphi(x/\tau)$の書き換えである.
								従って$\widehat{\varphi}(x/\tau)$は$\varphi(x/\tau)$の書き換えである.
								
							\item $x$と$z$が違うとする.このとき
								\begin{itemize}
									\item $x$が$\Set{z}{\psi}$に自由に現れている場合.
										$\widetilde{\psi}(z/x)(x/\tau)$は
										$\widetilde{\psi}(x/\tau)(z/\tau)$と同じ式である.
										他方で$\varphi(x/\tau)$は
										\begin{align}
											\tau \in \Set{z}{\psi(x/\tau)}
										\end{align}
										となるから,$\varphi(x/\tau)$の書き換えは$\psi(x/\tau)(z/\tau)$となる.
										
										ここで$\widetilde{\psi}(x/\tau)(z/\tau)$が
										$\psi(x/\tau)(z/\tau)$の量化部分式を
										(ゼロ回乃至数回だけ)差し替えた式であることを示す.
										ゼロ回というのは$\widetilde{\psi}$が$\psi$であるということだから,
										既に$\widehat{\varphi}(x/\tau)$が
										$\varphi(x/\tau)$の書き換えであると判ってしまう.
										以下では$\widetilde{\psi}$は$\psi$ではないと仮定して話を進める.
										そもそも$\widetilde{\psi}$とはどのように出来ていたかというと,
										表\ref{tab:formula_rewriting}の下の変項条件で書いたように,
										$\psi$の量化部分式を一回乃至数回差し替えているのである.
										それが$n$回あったとして,$\psi$から$\widetilde{\psi}$に至るまでの
										差し替えの履歴を
										\begin{align}
											\psi_{1},\ \psi_{2},\ \cdots,\ \psi_{n}
										\end{align}
										としよう.$\psi_{n}$とは$\widetilde{\psi}$のことである.
										メタ定理\ref{metathm:subformula_replacing_and_substitution}より,
										$\psi_{1}(x/\tau)$とは$\psi(x/\tau)$の量化部分式を一つ差し替えた式である.
										すると再びメタ定理\ref{metathm:subformula_replacing_and_substitution}より
										$\psi_{1}(x/\tau)(z/\tau)$は$\psi(x/\tau)(z/\tau)$
										の量化部分式を一つ差し替えた式となる.
										同様に$\psi_{i+1}(x/\tau)(z/\tau)$は$\psi_{i}(x/\tau)(z/\tau)$の
										量化部分式を一つ差し替えた式であるから,$\widetilde{\psi}(x/\tau)(z/\tau)$
										は$\psi(x/\tau)(z/\tau)$の量化部分式を$n$回差し替えた式なのである.
										$\psi(x/\tau)(z/\tau)$とは$\varphi(x/\tau)$の書き換えなのだから,
										書き換えの定義によって$\widetilde{\psi}(x/\tau)(z/\tau)$は
										$\varphi(x/\tau)$の書き換えである.つまり$\widehat{\varphi}(x/\tau)$は
										$\varphi(x/\tau)$の書き換えである.
										
										
									\item $x$が$\Set{z}{\psi}$に自由に現れていない場合.
										$\psi$に$x$は自由に現れないので
										$\widetilde{\psi}$にも
										$x$は自由に現れない($\widetilde{\psi}$は
										$\psi$に自由に現れる変項に影響しないように
										部分式を差し替えて作られているため).
										従って$\widetilde{\psi}(z/x)(x/\tau)$は
										$\widetilde{\psi}(z/\tau)$と同じ式である.
										他方で$\varphi(x/\tau)$は
										\begin{align}
											\tau \in \Set{z}{\psi}
										\end{align}
										となるから,$\psi(z/\tau)$は$\varphi(x/\tau)$の書き換えとなる.
										先と同じ論法で$\widetilde{\psi}(z/\tau)$が
										$\psi(z/\tau)$の量化部分式をゼロ回乃至数回だけ差し替えた
										式であると判るので,書き換えの定義より
										$\widehat{\varphi}(x/\tau)$は$\varphi(x/\tau)$の書き換えである.
								\end{itemize}
						\end{itemize}
						
					\item[case7] $\varphi$が
						\begin{align}
							a \in \Set{z}{\psi}
						\end{align}
						なる式のとき($a$は$x$とは違う$\lang{\varepsilon}$の項),
						$x$は$\varphi$に自由に現れているので,$x$は$z$とは違う変項であり,
						$\psi$に自由に現れている.$\widehat{\varphi}$を得るために
						必要ならば$\psi$の変項の名前替えをして$\widetilde{\psi}$を作る.ただし
						名前替えをしなかったら$\widetilde{\psi}$は$\psi$とする.
						$\widehat{\varphi}$は$\widetilde{\psi}(z/a)$なる式であり,
						$\widehat{\varphi}(x/\tau)$は$\widetilde{\psi}(z/a)(x/\tau)$
						となるが,$x$は$z$とも$a$とも違うので$\widehat{\varphi}(x/\tau)$は
						\begin{align}
							\widetilde{\psi}(x/\tau)(z/a)
						\end{align}
						と一致する.他方で$\varphi(x/\tau)$は
						\begin{align}
							a \in \Set{z}{\psi(x/\tau)}
						\end{align}
						となる.ところで,$\widetilde{\psi}$は$\psi$の量化部分式を
						或る$n$回だけ差し替えた式であるから,
						メタ定理\ref{metathm:subformula_replacing_and_substitution}
						とcase6の説明より,$\widetilde{\psi}(x/\tau)$もまた
						$\psi(x/\tau)$の量化部分式を$n$回だけ差し替えた式である.
						すなわち$\widetilde{\psi}(x/\tau)(z/a)$は
						$\varphi(x/\tau)$の書き換えとなっている.ゆえに
						$\widehat{\varphi}(x/\tau)$は$\varphi(x/\tau)$の書き換えである.
						
					\item[case8] $\varphi$が
						\begin{align}
							\Set{y}{\xi} \in x
						\end{align}
						なる式のとき,$\widehat{\varphi}$は
						\begin{align}
							\exists s\, (\, \forall u\, (\, \xi(y/u) \lrarrow u \in s\, ) \wedge s \in x\, )
						\end{align}
						なる式である.
						\begin{itemize}
							\item $x$と$y$が同じであるとする.このとき
								$\widehat{\varphi}(x/\tau)$は
								\begin{align}
									\exists s\, (\, \forall u\, (\, \xi(y/u) \lrarrow u \in s\, ) \wedge s \in \tau\, )
								\end{align}
								となる.他方で$\varphi(x/\tau)$は
								\begin{align}
									\Set{y}{\xi} \in \tau
								\end{align}
								であるから$\widehat{\varphi}(x/\tau)$は
								$\varphi(x/\tau)$の書き換えである.
								
							\item $x$と$y$が違うとする.このとき
								\begin{itemize}
									\item $x$が$\Set{y}{\xi}$に自由に現れているならば,
										$\widehat{\varphi}(x/\tau)$は
										\begin{align}
											\exists s\, (\, \forall u\, (\, \xi(y/u)(x/\tau) \lrarrow u \in s\, ) \wedge s \in \tau\, )
										\end{align}
										となるが,書き換えの変項条件より$x$は$u$とも違うので,
										%代入条件より$\tau$もまた$y$とも$u$とも違うので,
										$\xi(y/u)(x/\tau)$と$\xi(x/\tau)(y/u)$は
										同じ式である.従って$\widehat{\varphi}(x/\tau)$は
										\begin{align}
											\exists s\, (\, \forall u\, (\, \xi(x/\tau)(y/u) \lrarrow u \in s\, ) \wedge s \in \tau\, )
										\end{align}
										と同じ式である.他方で$\varphi(x/\tau)$は
										\begin{align}
											\Set{y}{\xi(x/\tau)} \in \tau
										\end{align}
										であるから,この場合は
										$\widehat{\varphi}(x/\tau)$は
										$\varphi(x/\tau)$の書き換えである.
										
									\item $x$が$\Set{y}{\xi}$に自由に現れていない場合,
										$\widehat{\varphi}(x/\tau)$は
										\begin{align}
											\exists s\, (\, \forall u\, (\, \xi(y/u) \lrarrow u \in s\, ) \wedge s \in \tau\, )
										\end{align}
										となり,$\varphi(x/\tau)$は
										\begin{align}
											\Set{y}{\xi} \in \tau
										\end{align}
										であるからこの場合も
										$\widehat{\varphi}(x/\tau)$は
										$\varphi(x/\tau)$の書き換えである.
								\end{itemize}
						\end{itemize}
					
					\item[case9] $\varphi$が
						\begin{align}
							\Set{y}{\xi} \in b
						\end{align}
						なる式のとき($b$は$x$と違う$\lang{\varepsilon}$の項),
						$\widehat{\varphi}$は
						\begin{align}
							\exists s\, (\, \forall u\, (\, \xi(y/u) \lrarrow u \in s\, ) \wedge s \in b\, )
						\end{align}
						なる式である.$\varphi$には$x$が自由に現れているので,つまり
						$x$は$y$ではなく,また$\xi$に自由に現れている.
						従って$\widehat{\varphi}(x/\tau)$は
						\begin{align}
							\exists s\, (\, \forall u\, (\, \xi(y/u)(x/\tau) \lrarrow u \in s\, ) \wedge s \in b\, )
						\end{align}
						となるが,書き換えの変項条件より$x$は$u$とも違うので,
						%代入条件より$\tau$もまた$y$とも$u$とも違うので,
						$\xi(y/u)(x/\tau)$と$\xi(x/\tau)(y/u)$は
						同じ式である.従って$\widehat{\varphi}(x/\tau)$は
						\begin{align}
							\exists s\, (\, \forall u\, (\, \xi(x/\tau)(y/u) \lrarrow u \in s\, ) \wedge s \in b\, )
						\end{align}
						と同じ式である.他方で$\varphi(x/\tau)$は
						\begin{align}
							\Set{y}{\xi(x/\tau)} \in b
						\end{align}
						であるから,$\widehat{\varphi}(x/\tau)$は
						$\varphi(x/\tau)$の書き換えである.
						
					\item[case10] $\varphi$が
						\begin{align}
							\Set{y}{\xi} \in \Set{z}{\psi}
						\end{align}
						なる式のとき,$\widehat{\varphi}$は
						\begin{align}
							\exists s\, (\, \forall u\, (\, \xi(y/u) \lrarrow u \in s\, ) \wedge \psi(z/s)\, )
						\end{align}
						なる式である.
						\begin{itemize}
							\item $x$と$y$が同じであるとする.このとき
								$x$は$\Set{y}{\xi}$には自由に
								現れないので,$x$が$\varphi$に自由に現れている以上
								$\Set{z}{\psi}$に自由に現れることになる.
								すなわち$x$と$z$は違う項である.
								このとき$\widehat{\varphi}(x/\tau)$は
								\begin{align}
									\exists s\, (\, \forall u\, (\, \xi(y/u) \lrarrow u \in s\, ) \wedge \psi(z/s)(x/\tau)\, )
								\end{align}
								となるが,書き換えの変項条件より$x$は$s$とも違うので,
								%代入条件より$\tau$もまた$z$とも$s$とも違うので,
								$\psi(z/s)(x/\tau)$と$\psi(x/\tau)(z/s)$は
								同じ式である.従って$\widehat{\varphi}(x/\tau)$は
								\begin{align}
									\exists s\, (\, \forall u\, (\, \xi(y/u) \lrarrow u \in s\, ) \wedge \psi(x/\tau)(z/s)\, )
								\end{align}
								と同じ式である.他方で$\varphi(x/\tau)$は
								\begin{align}
									\Set{y}{\xi} \in \Set{z}{\psi(x/\tau)}
								\end{align}
								であるから,$\widehat{\varphi}(x/\tau)$は
								$\varphi(x/\tau)$の書き換えである.
								
							\item $x$と$y$が違い,$x$と$z$が同じであるとする.
								$x$は$\Set{z}{\psi}$には自由に
								現れないので,$x$が$\varphi$に自由に現れている以上
								$\Set{y}{\xi}$に自由に現れることになる.
								このとき$\widehat{\varphi}(x/\tau)$は
								\begin{align}
									\exists s\, (\, \forall u\, (\, \xi(y/u)(x/\tau) \lrarrow u \in s\, ) \wedge \psi(z/s)\, )
								\end{align}
								となるが,書き換えの変項条件より$x$は$u$とも違うので,
								%代入条件より$\tau$もまた$y$とも$u$とも違うので,
								$\xi(y/u)(x/\tau)$と$\xi(x/\tau)(y/u)$は
								同じ式である.従って$\widehat{\varphi}(x/\tau)$は
								\begin{align}
									\exists s\, (\, \forall u\, (\, \xi(x/\tau)(y/u) \lrarrow u \in s\, ) \wedge \psi(z/s)\, )
								\end{align}
								と同じ式である.他方で$\varphi(x/\tau)$は
								\begin{align}
									\Set{y}{\xi(x/\tau)} \in \Set{z}{\psi}
								\end{align}
								であるから,$\widehat{\varphi}(x/\tau)$は
								$\varphi(x/\tau)$の書き換えである.
								
							\item $x$が$y$とも$z$とも違うとする.このとき$x$は
								$\Set{y}{\xi}$か$\Set{z}{\psi}$の
								少なくとも一方にはには自由に現れている.
								このとき$\widehat{\varphi}(x/\tau)$は
								\begin{align}
									\exists s\, (\, \forall u\, (\, \xi(y/u)(x/\tau) \lrarrow u \in s\, ) \wedge \psi(z/s)(x/\tau)\, )
								\end{align}
								となるが,書き換えの変項条件より$x$は$u$とも$s$とも違うので,
								$\widehat{\varphi}(x/\tau)$は
								\begin{align}
									\exists s\, (\, \forall u\, (\, \xi(x/\tau)(y/u) \lrarrow u \in s\, ) \wedge \psi(x/\tau)(z/s)\, )
								\end{align}
								と同じ式である.他方で$\varphi(x/\tau)$は
								\begin{align}
									\Set{y}{\xi(x/\tau)} \in \Set{z}{\psi(x/\tau)}
								\end{align}
								であるから,$\widehat{\varphi}(x/\tau)$は
								$\varphi(x/\tau)$の書き換えである.
						\end{itemize}
				\end{description}
			
			\item[step2] $\varphi$が一般の式であるとき,
				\begin{itembox}[l]{IH (帰納法の仮定)}
					$\varphi$の任意の真部分式$\psi$に対し,$\widehat{\psi}$が
					$\psi$の部分式で原子式であるものを全て
					表\ref{tab:formula_rewriting}の通りに直した式
					であるとすれば($\psi$が$\lang{\varepsilon}$の
					式ならば$\widehat{\psi}$は$\psi$とする),
					$\widehat{\psi}(x/\tau)$は$\psi(x/\tau)$の書き換えである.
				\end{itembox}
				と仮定する
				\footnote{
					メタ定理\ref{metathm:variables_unchanged_after_rewriting}より
					$\psi$に$x$が自由に現れていなければ$\widehat{\psi}$にも
					$x$は自由に現れないので,$\psi$に$x$が自由に現れていない場合は
					$\psi(x/\tau)$は$\psi$であり,$\widehat{\psi}(x/\tau)$は
					$\widehat{\psi}$である.
				}.
				
				\begin{description}
					\item[case1] $\varphi$が
						\begin{align}
							\negation \psi
						\end{align}
						なる式である場合,メタ定理\ref{metathm:relation_to_subformula_rewriting_1}より$\widehat{\varphi}$は
						\begin{align}
							\negation \widehat{\psi}
						\end{align}
						なる形で書けて,$\widehat{\psi}$は$\psi$の書き換えである.
						(IH)より$\widehat{\psi}(x/\tau)$は
						$\psi(x/\tau)$の書き換えであるから,
						再びメタ定理\ref{metathm:relation_to_subformula_rewriting_1}より
						$\negation \widehat{\psi}(x/\tau)$は
						$\negation \psi(x/\tau)$の書き換えである.
						$\negation \widehat{\psi}(x/\tau)$とは
						$\widehat{\varphi}(x/\tau)$のことであり,
						$\negation \psi(x/\tau)$とは$\varphi(x/\tau)$のことであるから,
						$\widehat{\varphi}(x/\tau)$は$\varphi(x/\tau)$の書き換えである.
					
					\item[case2] $\varphi$が
						\begin{align}
							\vee \psi \xi
						\end{align}
						なる式である場合,メタ定理\ref{metathm:relation_to_subformula_rewriting_2}より$\widehat{\varphi}$は
						\begin{align}
							\vee \widehat{\psi} \widehat{\xi}
						\end{align}
						なる形で書けて,$\widehat{\psi}$は$\psi$の書き換えであり,
						$\widehat{\xi}$は$\xi$の書き換えである.
						(IH)より$\widehat{\psi}(x/\tau)$は
						$\psi(x/\tau)$の書き換えであり,また$\widehat{\xi}(x/\tau)$は
						$\xi(x/\tau)$の書き換えであるから,
						再びメタ定理\ref{metathm:relation_to_subformula_rewriting_2}より
						$\vee \widehat{\psi}(x/\tau)\widehat{\xi}(x/\tau)$は
						$\vee \psi(x/\tau)\xi(x/\tau)$の書き換えである..
						$\vee \widehat{\psi}(x/\tau)\widehat{\xi}(x/\tau)$とは
						$\widehat{\varphi}(x/\tau)$のことであり,
						$\vee \psi(x/\tau)\xi(x/\tau)$とは
						$\varphi(x/\tau)$のことであるから,
						$\widehat{\varphi}(x/\tau)$は$\varphi(x/\tau)$の書き換えである.
					
					\item[case3] $\varphi$が
						\begin{align}
							\exists y \psi
						\end{align}
						なる式である場合,メタ定理\ref{metathm:relation_to_subformula_rewriting_3}より$\widehat{\varphi}$は
						\begin{align}
							\exists y \widehat{\psi}
						\end{align}
						なる形で書けて,$\widehat{\psi}$は$\psi$の書き換えである.
						(IH)より$\widehat{\psi}(x/\tau)$は
						$\psi(x/\tau)$の書き換えであるから,
						再びメタ定理\ref{metathm:relation_to_subformula_rewriting_3}より
						$\exists y \widehat{\psi}(x/\tau)$は
						$\exists y \psi(x/\tau)$の書き換えである.
						$\exists y \widehat{\psi}(x/\tau)$とは
						$\widehat{\varphi}(x/\tau)$のことであり,
						$\exists y \psi(x/\tau)$とは$\varphi(x/\tau)$のことであるから,
						$\widehat{\varphi}(x/\tau)$は$\varphi(x/\tau)$の書き換えである.
						\QED
				\end{description}
		\end{description}
	\end{metaprf}
	
	\begin{metaprf}[第二]
		$\widehat{\varphi}$を$\varphi$の書き換えとし,
		\begin{itembox}[l]{IH (帰納法の仮定)}
			$\widehat{\varphi}(x/\tau)$は$\varphi(x/\tau)$の書き換えである
		\end{itembox}
		と仮定する.このとき,$\widehat{\varphi}$に$\forall z \xi$ (resp. $\exists z \xi$)の
		形の部分式が現れているとし,$y$を$\xi$に自由に現れない変項で$\xi$の中で$z$への代入について
		自由であるものとし,$\widehat{\varphi}$の$\forall z \xi$ (resp. $\exists z \xi$)
		の部分を一か所だけ$\forall y \xi(z/y)$ (resp. $\exists y \xi(z/y)$)
		に差し替えた式を$\widetilde{\varphi}$とする(つまり$\widetilde{\varphi}$も$\varphi$の書き換えである).
		メタ定理\ref{metathm:subformula_replacing_and_substitution}より
		$\widetilde{\varphi}(x/\tau)$とは$\widehat{\varphi}(x/\tau)$の部分式
		$\forall z \xi(x/\tau)$ (resp. $\exists z \xi(x/\tau)$)を
		$\forall y \xi(x/\tau)(z/y)$ (resp. $\exists y \xi(x/\tau)(z/y)$)に差し替えた式であり,
		(IH)より$\widehat{\varphi}(x/\tau)$は$\varphi(x/\tau)$の書き換えであるから,
		$\widetilde{\varphi}(x/\tau)$もまた$\varphi(x/\tau)$の書き換えである.
		\QED
	\end{metaprf}
	\begin{screen}
		\begin{metathm}[部分式の取り替えと代入]
		\label{metathm:subformula_replacing_and_substitution}
			$\varphi$を$\lang{\varepsilon}$の式とし,$x$を$\varphi$に自由に現れる変項とする.
			また$\varphi$に$\forall z \xi$ (resp. $\exists z \xi$)の
			形の部分式が現れているとし,$y$を$\xi$に自由に現れない変項で$\xi$の中で$z$への
			代入について自由であるものとし,$\varphi$のその$\forall z \xi$ 
			(resp. $\exists z \xi$)の部分を(一か所)$\forall y \xi(z/y)$ 
			(resp. $\exists y \xi(z/y)$)に置き換えた式を$\widetilde{\varphi}$とする.
			それから$\tau$を$\lang{\varepsilon}$の項とし,$\varphi$と$\widetilde{\varphi}$
			の中で$x$への代入について自由であるとする.このとき,
			\begin{description}
				\item[(1)] $\varphi$における$x$の自由な出現がその$\forall z \xi$ 
					(resp. $\exists z \xi$)の中にある場合,$\xi$の中のその$x$
					($\varphi$で自由に出現している$x$)を全て$\tau$に置き換えた式を
					$\widetilde{\xi}$とすれば,$\widetilde{\varphi}(x/\tau)$
					は$\varphi(x/\tau)$の部分式$\forall z \widetilde{\xi}$ 
					(resp. $\exists z \widetilde{\xi}$)を
					$\forall y \widetilde{\xi}(z/y)$ 
					(resp. $\exists y \widetilde{\xi}(z/y)$)に置き換えた式である.
					
				\item[(2)] $\varphi$における$x$の自由な出現がその$\forall z \xi$ 
					(resp. $\exists z \xi$)の中に無い場合,$\widetilde{\varphi}(x/\tau)$
					は$\varphi(x/\tau)$の部分式$\forall z \xi$ 
					(resp. $\exists z \xi$)を$\forall y \xi(z/y)$ 
					(resp. $\exists y \xi(z/y)$)に置き換えた式である.
			\end{description}
		\end{metathm}
	\end{screen}
	
	\begin{metaprf} $\varphi$に現れているのが$\forall z \xi$だとして示すが,$\exists z \xi$
		に替えても同じである.
		\begin{description}
			\item[step1] $\varphi$が$\forall z \xi$なる式である場合,$\tilde{\varphi}$とは
				$\forall y \xi(z/y)$なる式であり,$x$は$\varphi$に自由に現れているので
				$\tilde{\varphi}$にも自由に現れている.ゆえに$x$は$z$でも$y$でもなく,
				代入条件より$\tau$もまた$z$でも$y$でもない.
				$\tilde{\varphi}(x/\tau)$とは
				\begin{align}
					\forall y \xi(z/y)(x/\tau)
				\end{align}
				なる式であるが,いま$\xi(z/y)(x/\tau)$と$\xi(x/\tau)(z/y)$は同じ式なので,
				$\tilde{\varphi}(x/\tau)$は
				\begin{align}
					\forall y \xi(x/\tau)(z/y)
				\end{align}
				と同じ式である.ゆえにいまの場合では(1)の主張が成り立つ.
				
			\item[step2] 
				\begin{itembox}[l]{IH (帰納法の仮定)}
					$\varphi$の任意の真部分式$\psi$に対して,$\psi$にその
					$\forall z \xi$が部分式とし現れているとするとき,
					$\varphi$で自由に出現している$x$で$\psi$の中にあるものを
					全て$\tau$に置き換えた式を$\widetilde{\psi}$とすれば,
					\begin{description}
						\item[(1)] $\widetilde{\psi}$は$\psi$の
							$\forall z \widetilde{\xi}$を
							$\forall y \widetilde{\xi}(z/y)$に置き換えた式である.
					\end{description}
				\end{itembox}
		\end{description}
	\end{metaprf}
	
	\begin{screen}
		\begin{metathm}[書き換えへの代入は代入した式の書き換え]
		\label{metathm:substitution_to_rewritten_formula}
			$\varphi$を$\lang{\varepsilon}$の式ではない$\mathcal{L}$の式とし,
			$\varphi$には変項$x$が自由に現れているとし,$\tau$を$\lang{\varepsilon}$の項とし,
			$\widehat{\varphi}$を$\varphi$の書き換えとし,$\tau$は$\varphi$と
			$\widehat{\varphi}$の中で$x$への代入について自由であるとする\footnotemark
			.このとき$\widehat{\varphi}(x/\tau)$は$\varphi(x/\tau)$の書き換えである.
		\end{metathm}
	\end{screen}
	
	\footnotetext{
		定理\ref{metathm:variables_unchanged_after_rewriting}より
		$\widehat{\varphi}$にも$x$は自由に現れている.
	}
	
	証明が長いので第一証明と第二証明に分割する.第一証明では$\widehat{\varphi}$が$\varphi$の
	部分式で原子式であるものを全て表\ref{tab:formula_rewriting}の通りに直した式である場合を扱い,
	第二証明では「式の書き換えによる構造的帰納法」のセカンドステップを扱う.
	
	\begin{metaprf}[第一] $\widehat{\varphi}$が$\varphi$の部分式で原子式であるものを全て
		表\ref{tab:formula_rewriting}の通りに直した式であるとき,$\widehat{\varphi}(x/\tau)$
		が$\varphi(x/\tau)$の書き換えであることを示す.
		\begin{description}
			\item[step1] $\varphi$が原子式であるとする.
				\begin{description}
					\item[case1] $\varphi$が
						\begin{align}
							x = \Set{z}{\psi}
						\end{align}
						なる式のとき,$\widehat{\varphi}$は
						\begin{align}
							\forall v\, (\, v \in x \lrarrow \psi(z/v)\, )
						\end{align}
						なる式である.
						\begin{itemize}
							\item $x$と$z$が同じであるとする.
								このとき$\widehat{\varphi}(x/\tau)$は
								\begin{align}
									\forall v\, (\, v \in \tau \lrarrow \psi(z/v)\, )
								\end{align}
								となる.他方で$\varphi(x/\tau)$は
								\begin{align}
									\tau = \Set{z}{\psi}
								\end{align}
								であるから$\widehat{\varphi}(x/\tau)$は
								$\varphi(x/\tau)$の書き換えである.
								
							\item $x$と$z$が違うとする.このとき
								\begin{itemize}
									\item $x$が$\Set{z}{\psi}$に自由に現れている場合,
										$\widehat{\varphi}(x/\tau)$は
										\begin{align}
											\forall v\, (\, v \in \tau \lrarrow \psi(z/v)(x/\tau)\, )
										\end{align}
										となるが,書き換えの変項条件より$x$は$v$とも違い,
										代入条件より$\tau$もまた$z$とも$v$とも違うので,
										$\psi(z/v)(x/\tau)$と$\psi(x/\tau)(z/v)$は
										同じである.従って$\widehat{\varphi}(x/\tau)$は
										\begin{align}
											\forall v\, (\, v \in \tau \lrarrow \psi(x/\tau)(z/v)\, )
										\end{align}
										と同じである.他方で$\varphi(x/\tau)$は
										\begin{align}
											\tau = \Set{z}{\psi(x/\tau)}
										\end{align}
										であるから,この場合は
										$\widehat{\varphi}(x/\tau)$は
										$\varphi(x/\tau)$の書き換えである.
										
									\item $x$が$\Set{z}{\psi}$に自由に現れていない場合,
										$\widehat{\varphi}(x/\tau)$は
										\begin{align}
											\forall v\, (\, v \in \tau \lrarrow \psi(z/v)\, )
										\end{align}
										となるが,$\varphi(x/\tau)$は
										\begin{align}
											\tau = \Set{z}{\psi}
										\end{align}
										であるからこの場合も
										$\widehat{\varphi}(x/\tau)$は
										$\varphi(x/\tau)$の書き換えである.
								\end{itemize}
						\end{itemize}
						
					\item[case2] $\varphi$が
						\begin{align}
							a = \Set{z}{\psi}
						\end{align}
						なる式のとき($a$と$x$は違う$\lang{\varepsilon}$の項),
						$\widehat{\varphi}$は
						\begin{align}
							\forall v\, (\, v \in a \lrarrow \psi(z/v)\, )
						\end{align}
						なる式である.$\varphi$には$x$が自由に現れているので,つまり
						$x$は$z$ではなく,また$\Set{z}{\psi}$に自由に現れている.従って
						$\widehat{\varphi}(x/\tau)$は
						\begin{align}
							\forall v\, (\, v \in a \lrarrow \psi(z/v)(x/\tau)\, )
						\end{align}
						となるが,書き換えの変項条件より$x$は$v$とも違い,
						代入条件より$\tau$もまた$z$とも$v$とも違うので,
						$\psi(z/v)(x/\tau)$と$\psi(x/\tau)(z/v)$は
						同じである.従って$\widehat{\varphi}(x/\tau)$は
						\begin{align}
							\forall v\, (\, v \in a \lrarrow \psi(x/\tau)(z/v)\, )
						\end{align}
						と同じである.他方で$\varphi(x/\tau)$は
						\begin{align}
							a = \Set{z}{\psi(x/\tau)}
						\end{align}
						であるから$\widehat{\varphi}(x/\tau)$は
						$\varphi(x/\tau)$の書き換えである.
					
					\item[case3] $\varphi$が
						\begin{align}
							\Set{y}{\xi} = x
						\end{align}
						なる式のとき,$\widehat{\varphi}$は
						\begin{align}
							\forall u\, (\, \xi(y/u) \lrarrow u \in x\, )
						\end{align}
						なる式である.
						\begin{itemize}
							\item $x$と$y$が同じであるとする.このとき
								$\widehat{\varphi}(x/\tau)$は
								\begin{align}
									\forall u\, (\, \xi(y/u) \lrarrow u \in \tau\, )
								\end{align}
								となる.他方で$\varphi(x/\tau)$は
								\begin{align}
									\Set{y}{\xi} = \tau
								\end{align}
								であるから$\widehat{\varphi}(x/\tau)$は
								$\varphi(x/\tau)$の書き換えである.
								
							\item $x$と$y$が違うとする.このとき
								\begin{itemize}
									\item $x$が$\Set{y}{\xi}$に自由に現れていれば,
										$\widehat{\varphi}(x/\tau)$は
										\begin{align}
											\forall u\, (\, \xi(y/u)(x/\tau) \lrarrow u \in \tau\, )
										\end{align}
										となるが,書き換えの変項条件より$x$は$u$とも違い,
										代入条件より$\tau$もまた$y$とも$u$とも違うので,
										$\xi(y/u)(x/\tau)$と$\xi(x/\tau)(y/u)$は
										同じである.従って$\widehat{\varphi}(x/\tau)$は
										\begin{align}
											\forall u\, (\, \xi(x/\tau)(y/u) \lrarrow u \in \tau\, )
										\end{align}
										と同じである.他方で$\varphi(x/\tau)$は
										\begin{align}
											\Set{y}{\xi(x/\tau)} = \tau
										\end{align}
										であるから,この場合は
										$\widehat{\varphi}(x/\tau)$は
										$\varphi(x/\tau)$の書き換えである.
								
									\item $x$が$\Set{y}{\xi}$に自由に現れていない場合,
										$\widehat{\varphi}(x/\tau)$は
										\begin{align}
											\forall u\, (\, \xi(y/u) \lrarrow u \in \tau\, )
										\end{align}
										となるが,$\varphi(x/\tau)$は
										\begin{align}
											\Set{y}{\xi} = \tau
										\end{align}
										であるからこの場合も$\widehat{\varphi}(x/\tau)$は
										$\varphi(x/\tau)$の書き換えである.
								\end{itemize}
						\end{itemize}
						
					\item[case4] $\varphi$が
						\begin{align}
							\Set{y}{\xi} = b
						\end{align}
						なる式のとき($b$は$x$と違う$\lang{\varepsilon}$の項),
						$\widehat{\varphi}$は
						\begin{align}
							\forall u\, (\, \xi(y/u) \lrarrow u \in b\, )
						\end{align}
						なる式である.$\varphi$には$x$が自由に現れているので,つまり
						$x$は$y$ではなく,また$\Set{y}{\xi}$に自由に現れている.
						従って$\widehat{\varphi}(x/\tau)$は
						\begin{align}
							\forall u\, (\, \xi(y/u)(x/\tau) \lrarrow u \in b\, )
						\end{align}
						となるが,書き換えの変項条件より$x$は$u$とも違い,
						代入条件より$\tau$もまた$y$とも$u$とも違うので,
						$\xi(y/u)(x/\tau)$と$\xi(x/\tau)(y/u)$は
						同じである.従って$\widehat{\varphi}(x/\tau)$は
						\begin{align}
							\forall u\, (\, \xi(x/\tau)(y/u) \lrarrow u \in b\, )
						\end{align}
						と同じである.他方で$\varphi(x/\tau)$は
						\begin{align}
							\Set{y}{\xi(x/\tau)} = b
						\end{align}
						であるから$\widehat{\varphi}(x/\tau)$は
						$\varphi(x/\tau)$の書き換えである.
					
					\item[case5] $\varphi$が
						\begin{align}
							\Set{y}{\xi} = \Set{z}{\psi}
						\end{align}
						なる式のとき,$\widehat{\varphi}$は
						\begin{align}
							\forall u\, (\, \xi(y/u) \lrarrow \psi(z/u)\, )
						\end{align}
						なる式である.
						\begin{itemize}
							\item $x$と$y$が同じであるとする.このとき
								$x$は$\Set{y}{\xi}$には自由に
								現れないので,$x$が$\varphi$に自由に現れている以上
								$\Set{z}{\psi}$に自由に現れることになる.
								すなわち$x$と$z$は違う項である.
								このとき$\widehat{\varphi}(x/\tau)$は
								\begin{align}
									\forall u\, (\, \xi(y/u) \lrarrow \psi(z/u)(x/\tau)\, )
								\end{align}
								となるが,書き換えの変項条件より$x$は$u$とも違い,
								代入条件より$\tau$もまた$z$とも$u$とも違うので,
								$\psi(z/u)(x/\tau)$と$\psi(x/\tau)(z/u)$は
								同じである.従って$\widehat{\varphi}(x/\tau)$は
								\begin{align}
									\forall u\, (\, \xi(y/u) \lrarrow \psi(x/\tau)(z/u)\, )
								\end{align}
								と同じである.他方で$\varphi(x/\tau)$は
								\begin{align}
									\Set{y}{\xi} = \Set{z}{\psi(x/\tau)}
								\end{align}
								であるから$\widehat{\varphi}(x/\tau)$は
								$\varphi(x/\tau)$の書き換えである.
								
							\item $x$と$y$と違い,$x$と$z$が同じであるとする.
								$x$が$\varphi$に自由に現れている以上
								$x$は$\Set{y}{\xi}$に自由に現れることになるから,
								$\widehat{\varphi}(x/\tau)$は
								\begin{align}
									\forall u\, (\, \xi(y/u)(x/\tau) \lrarrow \psi(z/u)\, )
								\end{align}
								となるが,書き換えの変項条件より$x$は$u$とも違い,
								代入条件より$\tau$もまた$y$とも$u$とも違うので,
								$\xi(y/u)(x/\tau)$と$\xi(x/\tau)(y/u)$は
								同じである.従って$\widehat{\varphi}(x/\tau)$は
								\begin{align}
									\forall u\, (\, \xi(x/\tau)(y/u) \lrarrow \psi(z/u)\, )
								\end{align}
								と同じである.他方で$\varphi(x/\tau)$は
								\begin{align}
									\Set{y}{\xi(x/\tau)} = \Set{z}{\psi}
								\end{align}
								であるから$\widehat{\varphi}(x/\tau)$は
								$\varphi(x/\tau)$の書き換えである.
							
							\item $x$が$y$とも$z$とも違うとする,このとき
								$x$は$\Set{y}{\xi}$か$\Set{z}{\psi}$の少なくとも
								一方には自由に現れている.
								このとき$\widehat{\varphi}(x/\tau)$は
								\begin{align}
									\forall u\, (\, \xi(y/u)(x/\tau) \lrarrow \psi(z/u)(x/\tau)\, )
								\end{align}
								となるが,書き換えの変項条件および代入条件より
								$\widehat{\varphi}(x/\tau)$は
								\begin{align}
									\forall u\, (\, \xi(x/\tau)(y/u) \lrarrow \psi(x/\tau)(z/u)\, )
								\end{align}
								と同じである.他方で$\varphi(x/\tau)$は
								\begin{align}
									\Set{y}{\xi(x/\tau)} = \Set{z}{\psi(x/\tau)}
								\end{align}
								であるから$\widehat{\varphi}(x/\tau)$は
								$\varphi(x/\tau)$の書き換えである.
						\end{itemize}
						
					\item[case6] $\varphi$が
						\begin{align}
							x \in \Set{z}{\psi}
						\end{align}
						なる式のとき,必要ならば$\psi$の変項の
						名前替えをしたものを$\widetilde{\psi}$とする.ただし
						名前替えをしなかったら$\widetilde{\psi}$は$\psi$とする.
						$\widehat{\varphi}$は$\widetilde{\psi}(z/x)$なる式であるから
						$\widehat{\varphi}(x/\tau)$は$\widetilde{\psi}(z/x)(x/\tau)$
						となる.
						\begin{itemize}
							\item $x$と$z$が同じであるとする.
								このときは$\psi$の変項の名前替えは必要ない.
								$\widehat{\varphi}$とは$\psi$そのものであり,
								$\tau$は$\psi$の中で$z$への代入について自由である.
								$\psi(z/x)$とは$\psi$そのものであるから,
								$\psi(z/x)(x/\tau)$は$\psi(z/\tau)$となる.
								他方で$\varphi(x/\tau)$は
								\begin{align}
									\tau \in \Set{z}{\psi}
								\end{align}
								となるから,$\psi(z/\tau)$は$\varphi(x/\tau)$の書き換えである.
								
							\item $x$と$z$が違うとする.このとき
								\begin{itemize}
									\item $x$が$\Set{z}{\psi}$に自由に現れている場合.
										$\widetilde{\psi}(z/x)(x/\tau)$は
										$\widetilde{\psi}(x/\tau)(z/\tau)$である.
										他方で$\varphi(x/\tau)$は
										\begin{align}
											\tau \in \Set{z}{\psi(x/\tau)}
										\end{align}
										となるから,$\psi(x/\tau)(z/\tau)$は
										$\varphi(x/\tau)$の書き換えとなる.
										$\psi(x/\tau)(z/\tau)$と
										$\widetilde{\psi}(x/\tau)(z/\tau)$は,
										もし違う式でも$\forall x$か$\exists x$から
										始まる或る部分式が違うだけであるから,
										$\widetilde{\psi}(x/\tau)(z/\tau)$もまた
										$\varphi(x/\tau)$の書き換えである.
										
									\item $x$が$\Set{z}{\psi}$に自由に現れていない場合.
										このときは$\psi$に$x$は自由に現れないので
										$\widetilde{\psi}$にも
										$x$は自由に現れない.従って
										$\widetilde{\psi}(z/x)(x/\tau)$は
										$\widetilde{\psi}(z/\tau)$である.
										他方で$\varphi(x/\tau)$は
										\begin{align}
											\tau \in \Set{z}{\psi}
										\end{align}
										となるから,$\widetilde{\psi}(z/\tau)$は
										$\varphi(x/\tau)$の書き換えとなる.
								\end{itemize}
						\end{itemize}
						
					\item[case7] $\varphi$が
						\begin{align}
							a \in \Set{z}{\psi}
						\end{align}
						なる式のとき($a$は$x$とは違う$\lang{\varepsilon}$の項),
						必要ならば$\psi$の変項の
						名前替えをしたものを$\widetilde{\psi}$とする.ただし
						名前替えをしなかったら$\widetilde{\psi}$は$\psi$とする.
						$\widehat{\varphi}$は$\widetilde{\psi}(z/a)$なる式であるから
						$\widehat{\varphi}(x/\tau)$は$\widetilde{\psi}(z/a)(x/\tau)$
						となる.$\varphi$には$x$が自由に現れているので,つまり
						$x$は$\Set{z}{\psi}$に自由に現れているから$x$は$z$とも違う変項である.
						また代入条件より$\tau$も$z$と違う項である.
						従って$\widetilde{\psi}(z/a)(x/\tau)$は
						$\widetilde{\psi}(x/\tau)(z/a)$である.他方で
						$\varphi(x/\tau)$は
						\begin{align}
							a \in \Set{z}{\psi(x/\tau)}
						\end{align}
						となるから,$\widetilde{\psi}(x/\tau)(z/a)$は$\varphi(x/\tau)$の書き換えとなる.
					
					\item[case8] $\varphi$が
						\begin{align}
							\Set{y}{\xi} \in x
						\end{align}
						なる式のとき,$\widehat{\varphi}$は
						\begin{align}
							\exists s\, (\, \forall u\, (\, \xi(y/u) \lrarrow u \in s\, ) \wedge s \in x\, )
						\end{align}
						なる式である.
						\begin{itemize}
							\item $x$と$y$が同じであるとする.このとき
								$\widehat{\varphi}(x/\tau)$は
								\begin{align}
									\exists s\, (\, \forall u\, (\, \xi(y/u) \lrarrow u \in s\, ) \wedge s \in \tau\, )
								\end{align}
								となる.他方で$\varphi(x/\tau)$は
								\begin{align}
									\Set{y}{\xi} \in \tau
								\end{align}
								であるから$\widehat{\varphi}(x/\tau)$は
								$\varphi(x/\tau)$の書き換えである.
								
							\item $x$と$y$が違うとする.このとき
								\begin{itemize}
									\item $x$が$\Set{y}{\xi}$に自由に現れているならば,
										$\widehat{\varphi}(x/\tau)$は
										\begin{align}
											\exists s\, (\, \forall u\, (\, \xi(y/u)(x/\tau) \lrarrow u \in s\, ) \wedge s \in \tau\, )
										\end{align}
										となるが,書き換えの変項条件より$x$は$u$とも違い,
										代入条件より$\tau$もまた$y$とも$u$とも違うので,
										$\xi(y/u)(x/\tau)$と$\xi(x/\tau)(y/u)$は
										同じである.従って$\widehat{\varphi}(x/\tau)$は
										\begin{align}
											\exists s\, (\, \forall u\, (\, \xi(x/\tau)(y/u) \lrarrow u \in s\, ) \wedge s \in \tau\, )
										\end{align}
										と同じである.他方で$\varphi(x/\tau)$は
										\begin{align}
											\Set{y}{\xi(x/\tau)} \in \tau
										\end{align}
										であるから,この場合は
										$\widehat{\varphi}(x/\tau)$は
										$\varphi(x/\tau)$の書き換えである.
										
									\item $x$が$\Set{y}{\xi}$に自由に現れていない場合,
										$\widehat{\varphi}(x/\tau)$は
										\begin{align}
											\exists s\, (\, \forall u\, (\, \xi(y/u) \lrarrow u \in s\, ) \wedge s \in \tau\, )
										\end{align}
										となり,$\varphi(x/\tau)$は
										\begin{align}
											\Set{y}{\xi} \in \tau
										\end{align}
										であるからこの場合も
										$\widehat{\varphi}(x/\tau)$は
										$\varphi(x/\tau)$の書き換えである.
								\end{itemize}
						\end{itemize}
					
					\item[case9] $\varphi$が
						\begin{align}
							\Set{y}{\xi} \in b
						\end{align}
						なる式のとき($b$は$x$と違う$\lang{\varepsilon}$の項),
						$\widehat{\varphi}$は
						\begin{align}
							\exists s\, (\, \forall u\, (\, \xi(y/u) \lrarrow u \in s\, ) \wedge s \in b\, )
						\end{align}
						なる式である.$\varphi$には$x$が自由に現れているので,つまり
						$x$は$y$ではなく,また$\xi$に自由に現れている.
						従って$\widehat{\varphi}(x/\tau)$は
						\begin{align}
							\exists s\, (\, \forall u\, (\, \xi(y/u)(x/\tau) \lrarrow u \in s\, ) \wedge s \in b\, )
						\end{align}
						となるが,書き換えの変項条件より$x$は$u$とも違い,
						代入条件より$\tau$もまた$y$とも$u$とも違うので,
						$\xi(y/u)(x/\tau)$と$\xi(x/\tau)(y/u)$は
						同じである.従って$\widehat{\varphi}(x/\tau)$は
						\begin{align}
							\exists s\, (\, \forall u\, (\, \xi(x/\tau)(y/u) \lrarrow u \in s\, ) \wedge s \in b\, )
						\end{align}
						と同じである.他方で$\varphi(x/\tau)$は
						\begin{align}
							\Set{y}{\xi(x/\tau)} \in b
						\end{align}
						であるから,$\widehat{\varphi}(x/\tau)$は
						$\varphi(x/\tau)$の書き換えである.
						
					\item[case10] $\varphi$が
						\begin{align}
							\Set{y}{\xi} \in \Set{z}{\psi}
						\end{align}
						なる式のとき,$\widehat{\varphi}$は
						\begin{align}
							\exists s\, (\, \forall u\, (\, \xi(y/u) \lrarrow u \in s\, ) \wedge \psi(z/s)\, )
						\end{align}
						なる式である.
						\begin{itemize}
							\item $x$と$y$が同じであるとする.このとき
								$x$は$\Set{y}{\xi}$には自由に
								現れないので,$x$が$\varphi$に自由に現れている以上
								$\Set{z}{\psi}$に自由に現れることになる.
								すなわち$x$と$z$は違う項である.
								このとき$\widehat{\varphi}(x/\tau)$は
								\begin{align}
									\exists s\, (\, \forall u\, (\, \xi(y/u) \lrarrow u \in s\, ) \wedge \psi(z/s)(x/\tau)\, )
								\end{align}
								となるが,書き換えの変項条件より$x$は$s$とも違い,
								代入条件より$\tau$もまた$z$とも$s$とも違うので,
								$\psi(z/s)(x/\tau)$と$\psi(x/\tau)(z/s)$は
								同じである.従って$\widehat{\varphi}(x/\tau)$は
								\begin{align}
									\exists s\, (\, \forall u\, (\, \xi(y/u) \lrarrow u \in s\, ) \wedge \psi(x/\tau)(z/s)\, )
								\end{align}
								と同じである.他方で$\varphi(x/\tau)$は
								\begin{align}
									\Set{y}{\xi} \in \Set{z}{\psi(x/\tau)}
								\end{align}
								であるから,$\widehat{\varphi}(x/\tau)$は
								$\varphi(x/\tau)$の書き換えである.
								
							\item $x$と$y$が違い,$x$と$z$が同じであるとする.
								$x$は$\Set{z}{\psi}$には自由に
								現れないので,$x$が$\varphi$に自由に現れている以上
								$\Set{y}{\xi}$に自由に現れることになる.
								このとき$\widehat{\varphi}(x/\tau)$は
								\begin{align}
									\exists s\, (\, \forall u\, (\, \xi(y/u)(x/\tau) \lrarrow u \in s\, ) \wedge \psi(z/s)\, )
								\end{align}
								となるが,書き換えの変項条件より$x$は$u$とも違い,
								代入条件より$\tau$もまた$y$とも$u$とも違うので,
								$\xi(y/u)(x/\tau)$と$\xi(x/\tau)(y/u)$は
								同じである.従って$\widehat{\varphi}(x/\tau)$は
								\begin{align}
									\exists s\, (\, \forall u\, (\, \xi(x/\tau)(y/u) \lrarrow u \in s\, ) \wedge \psi(z/s)\, )
								\end{align}
								と同じである.他方で$\varphi(x/\tau)$は
								\begin{align}
									\Set{y}{\xi(x/\tau)} \in \Set{z}{\psi}
								\end{align}
								であるから,$\widehat{\varphi}(x/\tau)$は
								$\varphi(x/\tau)$の書き換えである.
								
							\item $x$が$y$とも$z$とも違うとする.このとき$x$は
								$\Set{y}{\xi}$か$\Set{z}{\psi}$の
								少なくとも一方にはには自由に現れている.
								このとき$\widehat{\varphi}(x/\tau)$は
								\begin{align}
									\exists s\, (\, \forall u\, (\, \xi(y/u)(x/\tau) \lrarrow u \in s\, ) \wedge \psi(z/s)(x/\tau)\, )
								\end{align}
								となるが,書き換えの変項条件および代入条件より
								$\widehat{\varphi}(x/\tau)$は
								\begin{align}
									\exists s\, (\, \forall u\, (\, \xi(x/\tau)(y/u) \lrarrow u \in s\, ) \wedge \psi(x/\tau)(z/s)\, )
								\end{align}
								と同じである.他方で$\varphi(x/\tau)$は
								\begin{align}
									\Set{y}{\xi(x/\tau)} \in \Set{z}{\psi(x/\tau)}
								\end{align}
								であるから,$\widehat{\varphi}(x/\tau)$は
								$\varphi(x/\tau)$の書き換えである.
						\end{itemize}
				\end{description}
			
			\item[step2] $\varphi$が一般の式であるとき,
				\begin{itembox}[l]{IH (帰納法の仮定)}
					$\varphi$の任意の真部分式$\psi$に対し,$\widehat{\psi}$が
					$\psi$の部分式で原子式であるものを全て
					表\ref{tab:formula_rewriting}の通りに直した式
					であるとすれば($\psi$が$\lang{\varepsilon}$の
					式ならば$\widehat{\psi}$は$\psi$とする),
					$\widehat{\psi}(x/\tau)$は$\psi(x/\tau)$の書き換えである.
				\end{itembox}
				と仮定する
				\footnote{
					メタ定理\ref{metathm:variables_unchanged_after_rewriting}より
					$\psi$に$x$が自由に現れていなければ$\widehat{\psi}$にも
					$x$は自由に現れないので,$\psi$に$x$が自由に現れていない場合は
					$\psi(x/\tau)$は$\psi$であり,$\widehat{\psi}(x/\tau)$は
					$\widehat{\psi}$である.
				}.
				
				\begin{description}
					\item[case1] $\varphi$が
						\begin{align}
							\negation \psi
						\end{align}
						なる式である場合,メタ定理\ref{metathm:relation_to_subformula_rewriting}より$\widehat{\varphi}$は
						\begin{align}
							\negation \widehat{\psi}
						\end{align}
						なる形で書けて,$\widehat{\psi}$は$\psi$の書き換えである.
						(IH)より$\widehat{\psi}(x/\tau)$は
						$\psi(x/\tau)$の書き換えであるから,
						再びメタ定理\ref{metathm:relation_to_subformula_rewriting}より
						$\negation \widehat{\psi}(x/\tau)$は
						$\negation \psi(x/\tau)$の書き換えである.
						$\negation \widehat{\psi}(x/\tau)$とは
						$\widehat{\varphi}(x/\tau)$のことであり,
						$\negation \psi(x/\tau)$とは$\varphi(x/\tau)$のことであるから,
						$\widehat{\varphi}(x/\tau)$は$\varphi(x/\tau)$の書き換えである.
					
					\item[case2] $\varphi$が
						\begin{align}
							\vee \psi \xi
						\end{align}
						なる式である場合,メタ定理\ref{metathm:relation_to_subformula_rewriting}より$\widehat{\varphi}$は
						\begin{align}
							\vee \widehat{\psi} \widehat{\xi}
						\end{align}
						なる形で書けて,$\widehat{\psi}$は$\psi$の書き換えであり,
						$\widehat{\xi}$は$\xi$の書き換えである.
						(IH)より$\widehat{\psi}(x/\tau)$は
						$\psi(x/\tau)$の書き換えであり,また$\widehat{\xi}(x/\tau)$は
						$\xi(x/\tau)$の書き換えであるから,
						再びメタ定理\ref{metathm:relation_to_subformula_rewriting}より
						$\vee \widehat{\psi}(x/\tau)\widehat{\xi}(x/\tau)$は
						$\vee \psi(x/\tau)\xi(x/\tau)$の書き換えである..
						$\vee \widehat{\psi}(x/\tau)\widehat{\xi}(x/\tau)$とは
						$\widehat{\varphi}(x/\tau)$のことであり,
						$\vee \psi(x/\tau)\xi(x/\tau)$とは
						$\varphi(x/\tau)$のことであるから,
						$\widehat{\varphi}(x/\tau)$は$\varphi(x/\tau)$の書き換えである.
					
					\item[case3] $\varphi$が
						\begin{align}
							\exists y \psi
						\end{align}
						なる式である場合,メタ定理\ref{metathm:relation_to_subformula_rewriting}より$\widehat{\varphi}$は
						\begin{align}
							\exists y \widehat{\psi}
						\end{align}
						なる形で書けて,$\widehat{\psi}$は$\psi$の書き換えである.
						(IH)より$\widehat{\psi}(x/\tau)$は
						$\psi(x/\tau)$の書き換えであるから,
						再びメタ定理\ref{metathm:relation_to_subformula_rewriting}より
						$\exists y \widehat{\psi}(x/\tau)$は
						$\exists y \psi(x/\tau)$の書き換えである.
						$\exists y \widehat{\psi}(x/\tau)$とは
						$\widehat{\varphi}(x/\tau)$のことであり,
						$\exists y \psi(x/\tau)$とは$\varphi(x/\tau)$のことであるから,
						$\widehat{\varphi}(x/\tau)$は$\varphi(x/\tau)$の書き換えである.
						\QED
				\end{description}
		\end{description}
	\end{metaprf}
	
	\begin{metaprf}[第二]
		$\widehat{\varphi}$を$\varphi$の書き換えとし,
		\begin{itembox}[l]{IH (帰納法の仮定)}
			$\widehat{\varphi}(x/\tau)$は$\varphi(x/\tau)$の書き換えである
		\end{itembox}
		と仮定する.このとき,$\widehat{\varphi}$に$\forall z \xi$ (resp. $\exists z \xi$)の
		形の部分式が現れているとし,$y$を$\xi$に自由に現れない変項で$\xi$の中で$z$への代入について
		自由であるものとし,$\widehat{\varphi}$のその$\forall z \xi$ (resp. $\exists z \xi$)
		の部分を(一か所)$\forall y \xi(z/y)$ (resp. $\exists y \xi(z/y)$)
		に置き換えた式を$\widetilde{\varphi}$とする.
	\end{metaprf}
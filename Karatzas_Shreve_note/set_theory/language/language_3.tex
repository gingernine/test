\subsection{言語$\mathcal{L}$}
	本稿における主流の言語は,次に定める$\mathcal{L}$である.$\mathcal{L}$の最大の特徴は
	\begin{align}
		\Set{x}{\varphi(x)}
	\end{align}
	なる形のオブジェクトが``正式に''項として用いられることである.
	他の多くの集合論の本では$\Set{x}{\varphi(x)}$なる項はインフォーマルに導入されるものであるが,
	インフォーマルなものでありながらこの種のオブジェクトはいたるところで堂々と登場するので,
	やはりフォーマルに導入して然るべきである.
	
	$\mathcal{L}$の構成要素は以下のものである.
	
	\begin{description}
		\item[矛盾記号] $\bot$
		\item[論理記号] $\negation,\ \vee,\ \wedge,\ \rarrow$
		\item[量化子] $\forall,\ \exists$
		\item[述語記号] $=,\ \in$
		\item[変項] \ref{sec:variables}節のもの.
		\item[補助記号] $\{,\ |,\ \}$
	\end{description}
	
	$\mathcal{L}$の項と式の構成規則は$\lang{\in}$のものと大差ない.
	
	\begin{description}
		\item[項] 
			\begin{itemize}
				\item 変項は$\mathcal{L}$の項である.
				\item $\lang{\varepsilon}$の項は$\mathcal{L}$の項である.
				\item $x$を$\mathcal{L}$の変項とし,$\varphi$を
					$\lang{\varepsilon}$の式とするとき,
					$\Set{x}{\varphi}$なる記号列は$\mathcal{L}$の項である.
				\item 以上のみが$\mathcal{L}$の項である.
			\end{itemize}
	\end{description}
	
	によって正式に定義される.
	
	\begin{description}
		\item[式] 
			\begin{itemize}
				\item $\bot$は$\mathcal{L}$の式である.
				\item $\sigma$と$\tau$を$\mathcal{L}$の項とするとき,
					$\in st$と$=st$は$\mathcal{L}$の式である.
				\item $\varphi$を$\mathcal{L}$の式とするとき,
					$\negation \varphi$は$\mathcal{L}$の式である.
				\item $\varphi$と$\psi$を$\mathcal{L}$の式とするとき,
					$\vee \varphi \psi,\ \wedge \varphi \psi,\ \rarrow \varphi \psi$は
					いずれも$\mathcal{L}$の式である.
				\item $x$を$\mathcal{L}$の{\bf 変項}とし,$\varphi$を
					$\mathcal{L}$の式とするとき,$\forall x \varphi$と
					$\exists x \varphi$は$\mathcal{L}$の式である.
			\end{itemize}
	\end{description}
	
	\begin{screen}
		\begin{dfn}[内包項]
			$\Set{x}{\varphi}$なる項を{\bf 内包項}\index{ないほうこう@内包項}
			と呼ぶ.ここで$x$は変項であり,$\varphi$は$\mathcal{L}$の式である.
		\end{dfn}
	\end{screen}
	
	定義通りなら,式$\varepsilon$に$x$が自由に現れていない場合でも$\Set{x}{\varphi}$は
	$\mathcal{L}$の項である.ただしそのような項は全く無用であるから,
	後で実際に集合論を構築する際には排除してしまう(\ref{sec:restriction_of_formulas}節参照).
	
	言語の拡張の仕方より明らかであるが,次が成り立つ:
	
	\begin{screen}
		\begin{metathm}
			$\lang{\in}$の式は$\lang{\varepsilon}$の式であり,
			また$\lang{\varepsilon}$の式は$\mathcal{L}$の式である.
		\end{metathm}
	\end{screen}
	
	\begin{metaprf}\mbox{}
		\begin{description}
			\item[step1]
				式の構成法より$\lang{\in}$の原子式は$\lang{\varepsilon}$の式である.
				また$\varphi$を任意に与えられた$\lang{\in}$の式とするとき,
				\begin{itembox}[l]{IH (帰納法の仮定)}
					$\varphi$のすべての真部分式は$\lang{\varepsilon}$の式である
				\end{itembox}
				と仮定すると,$\varphi$が
				\begin{description}
					\item[case1] $\negation \psi$
					\item[case2] $\vee \psi \chi$
					\item[case3] $\exists x \psi$
				\end{description}
				のいずれの形の式であっても,$\psi$も$\chi$も(IH)より$\lang{\varepsilon}$の式
				であるから,式の構成法より$\varphi$自信も$\lang{\varepsilon}$の式である.
				ゆえに$\lang{\in}$の式は$\lang{\varepsilon}$の式である.
				
			\item[step2]
				$\lang{\varepsilon}$の式が$\mathcal{L}$の式であることを示す.
				まず,$\mathcal{L}$の式の構成において使える項を変項に制限すれば
				全ての$\lang{\in}$の式が作られるのだから
				$\lang{\in}$の式は$\mathcal{L}$の式である.
				また$\varphi$を任意に与えられた$\lang{\varepsilon}$の式とするとき,
				\begin{itembox}[l]{IH (帰納法の仮定)}
					$\varphi$のすべての真部分式は$\mathcal{L}$の式である
				\end{itembox}
				と仮定すると(今回は予め$\lang{\varepsilon}$の項は
				$\mathcal{L}$の項とされているので,真部分式に対する仮定のみで十分である),
				\begin{description}
					\item[case1] $\varphi$が$\in \sigma \tau$なる形の原子式であるとき,
						$\sigma$も$\tau$も$\mathcal{L}$の項であるから
						$\in \sigma \tau$は$\mathcal{L}$の式である.
						
					\item[case2] $\varphi$が$\negation \psi$なる形の式であるとき,
						(IH)より$\psi$は$\mathcal{L}$の式であるから
						$\negation \psi$も$\mathcal{L}$の式である.
						
					\item[case3] $\varphi$が$\vee \psi \chi$なる形の式であるとき,
						(IH)より$\psi$も$\chi$も$\mathcal{L}$の式であるから
						$\vee \psi \chi$も$\mathcal{L}$の式である.
						
					\item[case4] $\varphi$が$\exists x \psi$なる形の式であるとき,
						(IH)より$\psi$は$\mathcal{L}$の式であるから
						$\exists x \psi$も$\mathcal{L}$の式である.
				\end{description}
				となる.ゆえに$\lang{\varepsilon}$の式は$\mathcal{L}$の式である.
				\QED
		\end{description}
	\end{metaprf}
	
	$\varphi$を$\mathcal{L}$の式とし,$s$を$\varphi$に現れる記号とすると,
	\begin{description}
		\item[(1)] $s$は文字である.
		\item[(2)] $s$は$\natural$である.
		\item[(2)] $s$は$\{$である.
		\item[(3)] $s$は$|$である.
		\item[(4)] $s$は$\}$である.
		\item[(5)] $s$は$\bot$である.
		\item[(6)] $s$は$\in$か$=$である.
		\item[(7)] $s$は$\negation$である.
		\item[(8)] $s$は$\vee,\wedge,\rightarrow$のいずれかである.
	\end{description}
	
	\begin{screen}
		(★★) いま,$\varphi$を任意に与えられた式としよう.
		\begin{itemize}
			\item $\natural$が$\varphi$に現れたとき,$\lang{\in}$の項$\tau$と$\sigma$が得られて,$\natural \tau \sigma$は
				$\natural$のその出現位置から始まる$\lang{\in}$の項となる.
				また$\natural$のその出現位置から始まる$\lang{\in}$の項は$\natural \tau \sigma$のみである.
				
			\item $\{$が$\varphi$に現れたとき,$\lang{\in}$の変項$x$及び$\lang{\in}$の式$A$が得られて,
				$\{ x|A\}$は$\{$のその出現位置から始まる項となる.
				また$\{$のその出現位置から始まる項は$\{x|A\}$のみである.
				
			\item $|$が$\varphi$に現れたとき,,変項$x$と$\lang{\in}$の式$A$が得られて,
				$\{x|A\}$は$|$のその出現位置から広がる項となる.
				また$|$のその出現位置から広がる項は$\{x|A\}$のみである.
				
			\item $\}$が$\varphi$に現れたとき,変項$x$と式$A$が得られて,
				$\{x|A\}$は$\}$のその出現位置を終点とする項となる.
				また$\}$のその出現位置を終点とする項は$\{x|A\}$のみである.
		\end{itemize}
	\end{screen}
	
	\begin{description}
		\item[$\natural$に対して$\natural \tau \sigma$なる変項$\tau$と$\sigma$が得られること]
			$\natural$が原子項に現れたら,原子項とは文字$x,y$によって
			\begin{align}
				\natural xy
			\end{align}
			と表されるものであるから,$\natural$に対して変項$\tau,\sigma$ (すなわち文字$x,y$)が取れたことになる.
			$\natural$が項に現れたとする.項とは,変項$x,y$によって
			\begin{align}
				\natural xy
			\end{align}
			で表されるものであり,$\natural$は左端の$\natural$であるか,$x$に現れるか,$y$に現れる.
			$\natural$が$x$か$y$に現れるときは帰納法の仮定により,
			$\natural$が左端のものである場合は$x$が$\tau$,$y$が$\sigma$ということになる.
			
		\item[変項の始切片で変項であるものは自分自身のみ]
			$x$が文字である場合はそう.$x$の任意の部分変項が言明を満たしているなら,
			$x$は$\natural st$なる変項である(生成規則)から,$x$の始切片は$\natural uv$なる変項である.
			$s,t,u,v$はいずれも$x$の部分変項なので仮定が適用されている.
			ゆえに$s$と$u$は一方が他方の始切片であり,一致する.すなわち$t$と$v$も一方が他方の始切片であり一致する.
			ゆえに$x$の始切片で変項であるものは$x$自身である.
			
		\item[$\natural$に対して得られる変項の一意性]
			$\natural xy$と$\natural st$が共に変項であるとき,$x$と$s$,$y$と$t$は一致するか.
			$\natural xy$が原子項であるときは明らかである.
			$x$の始切片で変項であるものは$x$自身に限られるので,
			$x$と$s$は一致する.ゆえに$t$は$y$の始切片であり,$t$と$y$も一致する.
		
		\item[生成規則より$x$と$A$が得られるか]
			$\varphi$が原子式であるとき,
			$\{$が現れるとすれば項の中である.項とは$\lang{\in}$の項であるか$\{x|A\}$なるものであるので
			$\{$が現れたならば$\{$とは$\{x|A\}$の$\{$である.
			
			$\varphi$の任意の部分式に対して言明が満たされているとする.
			$\varphi$とは$\negation \psi,\vee \psi \xi,...$の形であるから,
			$\varphi$に現れた$\{$とは$\psi$や$\xi$に現れるのである.ゆえに
			仮定より$x$と$A$が取れるわけである.
			
		\item[$\{$に対して]
			項の生成規則より$x$と$A$が得られる.
			$\{y|B\}$もまた$\{$から始まる項である場合,順番に見ていって
			$x$と$y$は一方が他方の始切片という関係になるから一致する.
			すると$A$と$B$は一方が他方の始切片という関係になり,(★)より$A$と$B$は一致する.
			
		\item[$|$について]
			項の生成規則より$x$と$A$が得られる.
			$\{y|B\}$もまた$|$から広がる項である場合,順番に見ていって
			$x$にも$y$にも$\{$という記号は現れないので$x$と$y$は一致する.
			$A$と$B$は一方が他方の始切片という関係になるので(★)より$A$と$B$は一致する.
			
		\item[$\}$について]
			項の生成規則より$x$と$A$が得られる.
			$\{y|B\}$もまた$\}$のその出現位置を終点とする変項である場合,
			$A$と$B$は$\lang{\in}$の式なので$|$という記号は現れない.ゆえに
			$A$と$B$は一致する.すると$x$と$y$は右端で揃うが,
			$x$にも$y$にも$\{$という記号は現れないので$x$と$y$は一致する.
	\end{description}
	
\subsection{量化}
	自由の意味,束縛とスコープ,$\varepsilon$項と内包項の束縛
	
	$\varphi$を$\mathcal{L}$の式とする.もし$\varphi$に$\forall$が現れたら,
	その$\forall$に後続する変項$x$と式$\psi$が取れるが,そのとき$x$は
	\begin{align}
		\forall x \psi
	\end{align}
	の中で{\bf 「量化されている」}\index{りょうか@量化}{\bf(quantified)}や
	{\bf 「束縛されている」}\index{そくばく@束縛}{\bf (bound)}という.
	同様に$\varphi$の中に$\exists$や$\varepsilon$が現れたら,
	その$\exists,\varepsilon$の直後にくる変項は,
	「その$\exists,\varepsilon$のスコープの中で量化されている」といい,
	また$\varphi$の中に
	\begin{align}
		\Set{x}{\psi}
	\end{align}
	なる内包項が現れたら,$x$は「この内包項の中で量化されている」という.
	まとめれば,$\forall,\exists,\varepsilon,$そして$\{$は
	直後の変項をそのスコープ内で量化しているのである.たとえば
	\begin{align}
		\forall x\, (\, x \in y\, )
	\end{align}
	においては$x$は量化されているし,
	\begin{align}
		\Set{u}{u = z}
	\end{align}
	において$u$は量化されている.
	
	この二つの例では$y$と$z$は量化されていないが,考えている項や式の中で量化されていない変項
	を{\bf 自由な}\index{じゆう@自由}{\bf (free)}変項と呼ぶ.
	ただし現れる変項が自由であるか否かは当然その出現位置に依存している.特に,
	\begin{align}
		\forall x\, (\, x \in y\, ) \rarrow (\, x \in z\, )
	\end{align}
	なる式においては左の二つの$x$は量化されている一方で右の$x$は自由であるように,
	同じ変項が複数個所に現れる場合はその変項が量化されているか自由であるかは一概には言えない.
	
	{\bf 自由な出現}\index{じゆうなしゅつげん@自由な出現}{\bf (free occurrence)}と呼ぶ.
	
\subsection{代入}
	変項とは束縛されうる項であったが,別の項を代入されうる項でもある.
	代入とは別の項で置き換えるということであり,また代入されうるのは式の中で自由な変項のみである.
	ただし,代入には「{\bf 式の中の自由な変項を別の変項に取り替えても式の意味を変えてはならない}」という
	大前提がある.たとえば
	\begin{align}
		\forall u\, (\, u \in x\, )
	\end{align}
	という式で考察すると,この式で$x$は自由であるから別の項を代入して良いのであり,$z$を代入すれば
	\begin{align}
		\forall u\, (\, u \in z\, )
	\end{align}
	となる.そしてこの場合はどちらの式も意味は同じである.意味が同じであるとは
	量化してみれば一目瞭然であって,両式を全称記号で量化すれば
	\begin{align}
		&\forall x\, \forall u\, (\, u \in x\, ), \\
		&\forall z\, \forall u\, (\, u \in z\, )
	\end{align}
	はどちらも「どの集合も,全ての集合を要素に持つ」と解釈され,
	両式を存在記号で量化すれば
	\begin{align}
		&\exists x\, \forall u\, (\, u \in x\, ), \\
		&\exists z\, \forall u\, (\, u \in z\, )
	\end{align}
	はどちらも「或る集合は,全ての集合を要素に持つ」と解釈される.
	ところが$x$に$u$を代入すると
	\begin{align}
		\forall u\, (\, u \in u\, )
	\end{align}
	となり,これは「全ての集合は自分自身を要素に持つ」という意味に変わる.
	つまり先の大前提に立てば,代入する際には{\bf 代入後に束縛されてしまう変項は使ってはいけない}のである.
	
	代入するのは変項だけではない.$\varepsilon$項や内包項だって上の$x$に代入して良い.
	ただし上と同様の注意が必要で,$\varepsilon$項や内包項に$u$が自由に現れている場合と
	そうでない場合では代入後の式の意味が分かれてしまうので,
	代入して良い項は$u$が自由に現れていないものに限る.
	
	以上の考察を一般的な代入規則に敷衍して言えば,
	
	\begin{itembox}[l]{代入可能な項}
		$\varphi$を$\mathcal{L}$の式とし,$x$を$\varphi$に自由に現れる変項とするとき,
		$\varphi$に自由に現れる$x$に代入して良いのは
		\begin{itemize}
			\item 代入した後で束縛されない変項
			\item 代入するのが$\varepsilon$項や内包項である場合は,
				その項の中に自由に現れる変項が,代入した箇所で束縛されないもの
		\end{itemize}
		に限る.$\mathcal{L}$の項$\tau$がこの条件を満たすとき,
		{\bf 「$\tau$は$\varphi$の中で$x$への代入について自由である」}という.
	\end{itembox}
	
	ちなみに$\varphi$に自由に現れる$x$に$\tau$を代入する際は,特筆が無い限り
	$\varphi$に自由に現れる\underline{{\bf 全ての$x$}} に$\tau$を代入する.
	そして「$\tau$が$\varphi$の中で$x$への代入について自由である」とは,
	$\tau$は$\varphi$に代入されたどの箇所でも束縛の影響を受けないということである.
	$\varphi$に自由に現れる$x$に$\tau$を代入した後の式を
	\begin{align}
		\varphi(x/\tau)
	\end{align}
	と書く($x/\tau$とは``replace $x$ by $\tau$''の順である).
	特に,$\varphi$の中に自由に現れている変項が$x$だけである場合は,$\varphi(x/\tau)$を
	\begin{align}
		\varphi(\tau)
	\end{align}
	とも書く.$\tau$が$x$自身である場合は$\varphi(x)$は$\varphi$そのものであるが,
	「$\varphi$に自由に現れているのは$x$だけである」ということを強調するために
	\begin{align}
		\varphi(x)
	\end{align}
	と書くことも多い.
	
\subsection{類}
	\begin{comment}
	\begin{screen}
		\begin{dfn}[閉項]
			どの変項も自由に現れない$\varepsilon$項を
			{\bf 閉${\boldsymbol \varepsilon}$項}\index{
			へいイプシロンこう@閉$\varepsilon$項}{\bf (closed epsilon term)}と呼び,
			どの変項も自由に現れない内包項を{\bf 閉内包項}\index{
			へいないほうこう@閉内包項}{\bf (closed comprehension term)}と呼ぶ.
			また閉$\varepsilon$項と閉内包項は以上のみである.
		\end{dfn}
	\end{screen}
	\end{comment}
	
	元々の意図としては,例えば$x$のみが自由に現れる式$\varphi(x)$に対して
	``$\varphi(x)$を満たすいずれかの集合$x$''という意味を込めて
	\begin{align}
		\varepsilon x \varphi(x)
	\end{align}
	を作ったのだし,``$\varphi(x)$を満たす集合$x$の全体''という意味を込めて
	\begin{align}
		\Set{x}{\varphi(x)}
	\end{align}
	を作ったのである.つまりこの場合の$\varepsilon x \varphi(x)$と
	$\Set{x}{\varphi(x)}$は``意味を持っている''わけである.
	これが,もし$x$とは別の変項$y$が$\varphi$に自由に現れているとすれば,
	$\varepsilon x \varphi$も$\Set{x}{\varphi}$も$y$に依存してしまい
	意味が定まらなくなる.というのも,変項とは代入可能な項であるから,$y$に代入する項ごとに
	$\varepsilon x \varphi$と$\Set{x}{\varphi}$は別の意味を持ち得るのである.
	また項が閉じていても意味不明な場合がある.たとえば,$\psi$が文であるときに
	\begin{align}
		\varepsilon y \psi
	\end{align}
	や
	\begin{align}
		\Set{y}{\psi}
	\end{align}
	なる項は閉じてはいるが,導入の意図には適っていない.意味不明ながらこういった項が存在しているのは
	導入時にこれらを排除する面倒を避けたからであり,また一旦すべてを作り終えた後で
	余計なものを排除するほうが楽である.
	
	とりあえず,導入の意図に適っている項は特別の名前を持っていて然るべきである.
	
	\begin{screen}
		\begin{dfn}[類]
			$\varphi$を$\lang{\varepsilon}$の式とし,$x$を$\varphi$に自由に現れる変項とし,
			$\varphi$に自由に現れる項は$x$のみであるとするとき,$\varepsilon x \varphi$
			と$\Set{x}{\varphi}$を{\bf 類}\index{るい@類}{\bf (class)}と呼ぶ.
			またこれらのみが類である.
		\end{dfn}
	\end{screen}
	
	類には二種類あるので,それらも名前を分けておく.
	\begin{screen}
		\begin{dfn}[主要$\varepsilon$項]
			類である$\varepsilon$項を{\bf 主要${\boldsymbol \varepsilon}$項}
			\index{しゅよういぷしんろんこう@主要$\varepsilon$項}
			{\bf (critical epsilon term)}と呼ぶ.
		\end{dfn}
	\end{screen}
	
	\begin{screen}
		\begin{dfn}[主要内包項]
			類である内包項を{\bf 主要内包項}\index{しゅようないほうこう@主要内包項}と呼ぶ.
		\end{dfn}
	\end{screen}
	
	内包項に関しては便宜上自由な変項の出現も許すことにするが,
	たとえば$\Set{x}{\varphi}$と書いたら少なくとも$x$は$\varphi$に自由に現れているべきであり,
	この意味で性質の良い内包項に対しても特別な名前を付けておく.
	
	\begin{screen}
		\begin{dfn}[正則内包項]
			$\varphi$を$\lang{\varepsilon}$の式とし,$x$を変項とし,
			$\varphi$に$x$が自由に現れているとするとき,
			$\Set{x}{\varphi}$を{\bf 正則内包項}\index{せいそくないほうこう@正則内包項}と呼ぶ.
		\end{dfn}
	\end{screen}
	
\subsection{扱う式の制限}
\label{sec:restriction_of_formulas}
	{\bf 以降では扱う式は,そこに現れる$\varepsilon$項は全て主要$\varepsilon$項であり,
	現れる内包項は全て正則内包項であるとする.}

\subsection{式の書き換え}
	$\varepsilon$項を取り入れた当初の目的は,「$\varphi(x)$を満たす集合$x$が存在するならば
	$\varepsilon x \varphi(x)$はその$x$の一つである」という意味で
	{\bf 存在文}\index{そんざいぶん@存在文}{\bf (existential sentence)}に対して
	{\bf 証人}\index{しょうにん@証人}{\bf (witness)}を付けることであった.
	ただし$\varepsilon$項を作れる式は$\lang{\varepsilon}$の式のみであるので,
	$\varphi$が$\mathcal{L}$の式であると$\varepsilon x \varphi(x)$を使うことが出来ない.
	だが$\mathcal{L}$の式の存在文も往々にして登場するから
	そういった場合でも証人を用意できると便利である.
	そこで$\mathcal{L}$の式を``同値''な$\lang{\varepsilon}$の式に書き換えて,
	その書き換えた式で作る$\varepsilon$項を使うことにする.つまり
	$\varphi$が$\mathcal{L}$の式である場合は,$\varphi$を
	$\lang{\varepsilon}$の式$\hat{\varphi}$に書き換えてから
	\begin{align}
		\exists x \varphi(x) \rarrow \varphi(\varepsilon x \hat{\varphi}(x))
	\end{align}
	を保証するのである.書き換える必要があるのは内包項を含んでいる式のみであり,
	式を書き換える際にはその式の中で{\bf 内包項が使われている原子式だけを書き換えれば十分である}.
	書き換えが``同値''というのは後述の\ref{sec:equivalence_of_formula_rewriting}節
	で述べてあるような意味であるが,それは直感的に妥当なものである.原子式の書き換えは次の要領で行う:
	
	\begin{table}[H]
		\begin{center}
		\begin{tabular}{c|c}
			元の式 & 書き換え後 \\ \hline \hline
			$a = \Set{z}{\psi}$ & $\forall v\, (\, v \in a \lrarrow \psi(z/v)\, )$ \\ \hline
			$\Set{y}{\varphi} = b$ & $\forall u\, (\, \varphi(y/u) \lrarrow u \in b\, )$ \\ \hline
			$\Set{y}{\varphi} = \Set{z}{\psi}$ & $\forall u\, (\, \varphi(y/u) \lrarrow \psi(z/u)\, )$ \\ \hline
			$a \in \Set{z}{\psi}$ & $\psi(z/a)$ \\ \hline
			$\Set{y}{\varphi} \in b$ & $\exists s\, (\, \forall u\, (\, \varphi(y/u) \lrarrow u \in s\, ) \wedge s \in b\, )$ \\ \hline
			$\Set{y}{\varphi} \in \Set{z}{\psi}$ & $\exists s\, (\, \forall u\, (\, \varphi(y/u) \lrarrow u \in s\, ) \wedge \psi(z/s)\, )$ \\ \hline
		\end{tabular}
		\end{center}
	\end{table}
	
	ただし上の記号に課している条件は
	\begin{itemize}
		\item $a,b$は$\lang{\varepsilon}$の項である
			(\ref{sec:restriction_of_formulas}節の約束によって
			$a,b$は変項であるか主要$\varepsilon$項である).
		\item $\varphi,\psi$は$\lang{\varepsilon}$の式である.
		\item $\varphi$には$y$が自由に現れ,$\psi$には$z$が自由に現れている
			(\ref{sec:restriction_of_formulas}節の約束によってどちらも正則内包項である).
		\item $u$は$\varphi$の中で$y$への代入について自由であり,
			$v$は$\psi$の中で$z$への代入について自由である.
		\item 注意が必要なのは$a$が変項である場合の$a \in \Set{z}{\psi}$の書き換えであり,
			$a$を$\psi$の中の自由な$z$に代入した後で$a$が束縛される場合,
			束縛変項の名前替えをしなくてはならない.たとえば
			\begin{align}
				a \in \Set{z}{\forall a\, (\, z \in a\, )}
			\end{align}
			という式では,左辺の$a$は自由であるのに,書き換えの規則を直接適用すると
			\begin{align}
				\forall a\, (\, a \in a\, )
			\end{align}
			となり束縛されてしまう.代入後の$a$が束縛されないためには
			\begin{align}
				a \in \Set{z}{\forall b\, (\, z \in b\, )}
			\end{align}
			のように束縛変項$a$を別の変項$b$に替えて
			\begin{align}
				\forall b\, (\, a \in b\, )
			\end{align}
			とすればよい.
	\end{itemize}
	
\subsection{中置記法}
	たとえば$\in s t$なる原子式は「$s$は$t$の要素である」と読むのだから,語順通りに
	\begin{align}
		s \in t
	\end{align}
	と書きかえる方が見やすくなる.同じように,$\vee \varphi \psi$なる式も
	「$\varphi$または$\psi$」と読むのだから
	\begin{align}
		\varphi \vee \psi
	\end{align}
	と書きかえる方が見やすくなる.ところで$\rarrow \vee \varphi \psi \wedge \chi \xi$なる式は,
	上の作法に倣えば
	\begin{align}
		\begin{gathered}
			\rarrow \vee \varphi \psi \wedge \chi \xi \\
			\rarrow \color{red}{\varphi \vee \psi} \color{blue}{\chi \wedge \xi} \\
			\color{red}{\varphi \vee \psi} \color{black}{\rarrow} \color{blue}{\chi \wedge \xi}
		\end{gathered}
	\end{align}
	と書きかえることになるが,一々色分けするわけにもいかないので``(''と``)''を使って
	\begin{align}
		(\varphi \vee \psi) \rarrow (\chi \wedge \xi)
	\end{align}
	と書くようにすれば良い.
	
	\begin{screen}
		\begin{metadfn}[中置記法]
			$\mathcal{L}$の式は以下の手順で中置記法に書き換える.
			\begin{enumerate}
				\item $\in s t$なる形の原子式は$s \in t$と書きかえる.
					$= s t$も同様に書き換える.
					
				\item $\negation \varphi$なる形の式はそのままにする.
				
				\item $\vee \varphi \psi$なる形の式は$(\varphi \vee \psi)$と書きかえる.
					$\wedge \varphi \psi$と$\rarrow \varphi \psi$の形の式も同様に書き換える.
				
				\item $\exists x \varphi$なる形の式はそのままにする.
					$\forall x \varphi$なる形の式も同様にする.
			\end{enumerate}
		\end{metadfn}
	\end{screen}
	
	上の書き換え法では,たとえば$\rarrow \vee \varphi \psi \wedge \chi \xi$なる式は
	\begin{align}
		((\varphi \vee \psi) \rarrow (\chi \wedge \xi))
	\end{align}
	となるが,括弧はあくまで式の境界の印として使うものであるから,一番外側の括弧は外して
	\begin{align}
		(\varphi \vee \psi) \rarrow (\chi \wedge \xi)
	\end{align}
	と書く方が良い.よって{\bf 中置記法に書き換え終わったときに一番外側にある括弧は外す}ことにする.
	
	$\wedge \vee \exists x \varphi \psi \negation \rarrow \chi \in s t$なる式は
	\begin{align}
		\begin{gathered}
			\wedge \vee \exists x \varphi \psi \negation \rarrow \chi s \in t \\
			\wedge (\exists x \varphi \vee \psi) \negation (\chi \rarrow s \in t) \\
			(\exists x \varphi \vee \psi) \wedge \negation (\chi \rarrow s \in t)
		\end{gathered}
	\end{align}
	となる.
	
	ただしあまり括弧が連なると読みづらくなるので,
	\begin{align}
		(\varphi \vee \psi) \rarrow \chi
	\end{align}
	なる形の式は
	\begin{align}
		\varphi \vee \psi \rarrow \chi
	\end{align}
	に,同様に
	\begin{align}
		\varphi \rarrow (\psi \vee \chi)
	\end{align}
	なる形の式は
	\begin{align}
		\varphi \rarrow \psi \vee \chi
	\end{align}
	とも書く.また$\vee$が$\wedge$であっても同じように括弧を省く.
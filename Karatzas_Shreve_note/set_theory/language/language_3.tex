\subsection{内包項}
	本稿における主流の言語は,次に定める$\mathcal{L}$である.$\mathcal{L}$の最大の特徴は
	\begin{align}
		\Set{x}{\varphi(x)}
	\end{align}
	なる形のオブジェクトが``正式に''項として用いられることである.
	他の多くの集合論の本では$\Set{x}{\varphi(x)}$なる項はインフォーマルに導入されるものであるが,
	インフォーマルなものでありながらこの種のオブジェクトはいたるところで堂々と登場するので,
	やはりフォーマルに導入して然るべきである.
	
	$\mathcal{L}$の構成要素は以下のものである.
	
	\begin{description}
		\item[矛盾記号] $\bot$
		\item[論理記号] $\negation,\ \vee,\ \wedge,\ \rarrow$
		\item[量化子] $\forall,\ \exists$
		\item[述語記号] $=,\ \in$
		\item[変項] \ref{sec:variables}節のもの.
		\item[補助記号] $\{,\ |,\ \}$
	\end{description}
	
	$\mathcal{L}$の項と式の構成規則は$\lang{\in}$のものと大差ない.
	
	\begin{description}
		\item[項] 
			\begin{itemize}
				\item 変項は$\mathcal{L}$の項である.
				\item $\lang{\varepsilon}$の項は$\mathcal{L}$の項である.
				\item $x$を$\mathcal{L}$の変項とし,$\varphi$を
					$\lang{\varepsilon}$の式とするとき,
					$\Set{x}{\varphi}$なる記号列は$\mathcal{L}$の項である.
				\item 以上のみが$\mathcal{L}$の項である.
			\end{itemize}
	\end{description}
	
	によって正式に定義される.
	
	\begin{description}
		\item[式] 
			\begin{itemize}
				\item $\bot$は$\mathcal{L}$の式である.
				\item $\sigma$と$\tau$を$\mathcal{L}$の項とするとき,
					$\in st$と$=st$は$\mathcal{L}$の式である.
					これらは$\mathcal{L}$の{\bf 原子式}\index{げんししき@原子式}
					{\bf (atomic formula)}である.
				\item $\varphi$を$\mathcal{L}$の式とするとき,
					$\negation \varphi$は$\mathcal{L}$の式である.
				\item $\varphi$と$\psi$を$\mathcal{L}$の式とするとき,
					$\vee \varphi \psi,\ \wedge \varphi \psi,\ \rarrow \varphi \psi$は
					いずれも$\mathcal{L}$の式である.
				\item $x$を$\mathcal{L}$の{\bf 変項}とし,$\varphi$を
					$\mathcal{L}$の式とするとき,$\forall x \varphi$と
					$\exists x \varphi$は$\mathcal{L}$の式である.
			\end{itemize}
	\end{description}
	
	\begin{screen}
		\begin{dfn}[内包項]
			$\Set{x}{\varphi}$なる項を{\bf 内包項}\index{ないほうこう@内包項}
			と呼ぶ.ここで$x$は変項であり,$\varphi$は$\mathcal{L}$の式である.
		\end{dfn}
	\end{screen}
	
	定義通りなら,$\Set{x}{y=y}$のように式$\varphi$に$x$が自由に現れていない場合でも
	$\Set{x}{\varphi}$は$\mathcal{L}$の項である.ただしそのような項は全く無用であるから,
	後で実際に集合論を構築する際には排除してしまう(\ref{sec:restriction_of_formulas}節参照).
	
	\begin{screen}
		\begin{metathm}
			$\lang{\in}$の式は$\lang{\varepsilon}$の式であり,
			また$\lang{\varepsilon}$の式は$\mathcal{L}$の式である.
		\end{metathm}
	\end{screen}
	
	\begin{metaprf}\mbox{}
		\begin{description}
			\item[step1]
				式の構成法より$\lang{\in}$の原子式は$\lang{\varepsilon}$の式である.
				また$\varphi$を任意に与えられた$\lang{\in}$の式とするとき,
				\begin{itembox}[l]{IH (帰納法の仮定)}
					$\varphi$のすべての真部分式は$\lang{\varepsilon}$の式である
				\end{itembox}
				と仮定すると,$\varphi$が
				\begin{description}
					\item[case1] $\negation \psi$
					\item[case2] $\vee \psi \chi$
					\item[case3] $\exists x \psi$
				\end{description}
				のいずれの形の式であっても,$\psi$も$\chi$も(IH)より$\lang{\varepsilon}$の式
				であるから,式の構成法より$\varphi$自信も$\lang{\varepsilon}$の式である.
				ゆえに$\lang{\in}$の式は$\lang{\varepsilon}$の式である.
				
			\item[step2]
				$\lang{\varepsilon}$の式が$\mathcal{L}$の式であることを示す.
				まず,$\mathcal{L}$の式の構成において使える項を変項に制限すれば
				全ての$\lang{\in}$の式が作られるのだから
				$\lang{\in}$の式は$\mathcal{L}$の式である.
				また$\varphi$を任意に与えられた$\lang{\varepsilon}$の式とするとき,
				\begin{itembox}[l]{IH (帰納法の仮定)}
					$\varphi$のすべての真部分式は$\mathcal{L}$の式である
				\end{itembox}
				と仮定すると(今回は予め$\lang{\varepsilon}$の項は
				$\mathcal{L}$の項とされているので,真部分式に対する仮定のみで十分である),
				\begin{description}
					\item[case1] $\varphi$が$\in \sigma \tau$なる形の原子式であるとき,
						$\sigma$も$\tau$も$\mathcal{L}$の項であるから
						$\in \sigma \tau$は$\mathcal{L}$の式である.
						
					\item[case2] $\varphi$が$\negation \psi$なる形の式であるとき,
						(IH)より$\psi$は$\mathcal{L}$の式であるから
						$\negation \psi$も$\mathcal{L}$の式である.
						
					\item[case3] $\varphi$が$\vee \psi \chi$なる形の式であるとき,
						(IH)より$\psi$も$\chi$も$\mathcal{L}$の式であるから
						$\vee \psi \chi$も$\mathcal{L}$の式である.
						
					\item[case4] $\varphi$が$\exists x \psi$なる形の式であるとき,
						(IH)より$\psi$は$\mathcal{L}$の式であるから
						$\exists x \psi$も$\mathcal{L}$の式である.
				\end{description}
				となる.ゆえに$\lang{\varepsilon}$の式は$\mathcal{L}$の式である.
				\QED
		\end{description}
	\end{metaprf}
	
	\begin{screen}
		\begin{metaaxm}[$\mathcal{L}$の式に対する構造的帰納法]
			$\mathcal{L}$の式に対する言明Xに対し,
			\begin{itemize}
				\item 原子式に対してXが言える.
				\item 無作為に選ばれた式$\varphi$について,その全ての真部分式に対してXが言える
					と仮定すれば,$\varphi$に対してもXが言える.
			\end{itemize}
			ならば,いかなる式に対してもXが言える.
		\end{metaaxm}
	\end{screen}
	
	$\mathcal{L}$の項は帰納的な構成になっていないので構造的帰納法は不要である.
	
	\begin{screen}
		\begin{metathm}[$\mathcal{L}$の始切片の一意性]
		\label{metathm:initial_segment_L}
			$\tau$を$\mathcal{L}$の項とするとき,$\tau$の始切片で$\mathcal{L}$の項であるものは
			$\tau$自信に限られる.また$\varphi$を$\mathcal{L}$の式とするとき,$\varphi$の
			始切片で$\mathcal{L}$の式であるものは$\varphi$自信に限られる.
		\end{metathm}
	\end{screen}
	
	\begin{metaprf}\mbox{}
		\begin{description}
			\item[項について]
				$\tau$を項とするとき,$\tau$が変項ならば
				メタ定理\ref{metathm:initial_segment_L_in}によって,
				$\tau$が$\lang{\varepsilon}$の項ならば
				メタ定理\ref{metathm:initial_segment_L_epsilon}によって,
				$\tau$の始切片で$\mathcal{L}$の項であるものは$\tau$自身に限られる.
				$\tau$が
				\begin{align}
					\Set{x}{\varphi}
				\end{align}
				なる内包項である場合,$\tau$の始切片で項であるものも
				\begin{align}
					\Set{y}{\psi}
				\end{align}
				なる形をしている.メタ定理\ref{metathm:initial_segment_L_in}より
				$x$と$y$が一致し,メタ定理\ref{metathm:initial_segment_L_epsilon}より
				$\varphi$と$\psi$も一致するので,この場合も$\tau$の始切片で項であるものは
				$\tau$自身に限られる.
				
			\item[式について]
				$\in st$なる原子式については,その始切片で式であるものは
				\begin{align}
					\in uv
				\end{align}
				なる形をしているが,前段の結果より$s$と$u$,$t$と$v$は一致する.
				$=st$なる原子式についても,その始切片で$\mathcal{L}$の式であるものは
				$=st$に限られる.
				
				いま$\varphi$を任意に与えられた$\mathcal{L}$の式とし,
				\begin{itembox}[l]{IH (帰納法の仮定)}
					$\varphi$に現れる任意の真部分式$\psi$に対して,
					その始切片で式であるものは$\psi$に限られる.
				\end{itembox}
				と仮定する.このとき
				\begin{description}
					\item[case1] $\varphi$が
						\begin{align}
							\negation \psi
						\end{align}
						なる形の式であるとき,$\varphi$の始切片で式であるものもまた
						\begin{align}
							\negation \xi
						\end{align}
						なる形をしている.このとき$\xi$は$\psi$の始切片であるから,
						(IH)より$\xi$と$\psi$は一致する.
						ゆえに$\varphi$の始切片で式であるものは$\varphi$自身に限られる.
			
					\item[case2] $\varphi$が
						\begin{align}
							\vee \psi \xi
						\end{align}
						なる形の式であるとき,$\varphi$の始切片で式であるものもまた
						\begin{align}
							\vee \eta \zeta
						\end{align}
						なる形をしている.このとき$\psi$と$\eta$は一方が他方の始切片であるので
						(IH)より一致する.すると$\xi$と$\zeta$も一方が他方の始切片ということに
						なり,(IH)より一致する.ゆえに$\varphi$の始切片で式であるものは
						$\varphi$自身に限られる.
						
					\item[case3] $\varphi$が
						\begin{align}
							\exists x \psi
						\end{align}
						なる形の式であるとき,$\varphi$の始切片で式であるものもまた
						\begin{align}
							\exists y \xi
						\end{align}
						なる形の式である.このとき$x$と$y$は一方が他方の始切片であり,これらは
						変項であるからメタ定理\ref{metathm:initial_segment_L_in}より
						一致する.すると$\psi$と$\chi$も一方が他方の始切片ということになり,
						(IH)より一致する.
						ゆえに$\varphi$の始切片で式であるものは$\varphi$自身に限られる.
						\QED
				\end{description}
		\end{description}
	\end{metaprf}
	
	$\varphi$を$\mathcal{L}$の式とし,$s$を
	\begin{align}
		\natural,\ \{,\ \in,\ \negation,\ \vee,
		\ \wedge,\ \rarrow,\ \exists,\ \forall,\ \varepsilon
	\end{align}
	のいずれかの記号とするとき,$s$が$\varphi$に現れたら$s$のその出現位置から始まる$\varphi$の部分式
	(ただし$s$が``$\natural,\{,\varepsilon$''である場合は部分項)を$s$の
	{\bf スコープ}\index{スコープ}{\bf (scope)}と呼ぶ.ところで$\varphi$には
	\begin{align}
		|, \quad \}
	\end{align}
	も現れるので,これらにもスコープを割り当てるために
	\begin{itemize}
		\item $\varphi$に``$|$''が現れたら,``$|$''のその出現位置を跨いで$\varphi$の上に
			現れる内包項$\Set{x}{\psi}$をその``$|$''のスコープと呼ぶ.
			つまり現れた``$|$''とは$\Set{x}{\psi}$の中心線``$|$''のことである.
			
		\item $\varphi$に``$\}$''が現れたら,``$\}$''のその出現位置を右端にして$\varphi$の上に
			現れる内包項$\Set{x}{\psi}$をその``$\}$''のスコープと呼ぶ.
			つまり現れた``$\}$''とは$\Set{x}{\psi}$の右端の``$\}$''のことである.
	\end{itemize}
	と定める.すると,次のメタ定理によって``$\natural,\ \{,\ |,\ \},\ \in,\ \negation,\ \vee,
	\ \wedge,\ \rarrow,\ \exists,\ \forall,\ \varepsilon$''の全ての記号に対して
	スコープが取れることが保証される.
	
	取れるスコープの唯一性はメタ定理\ref{metathm:initial_segment_L}からすぐに従い,
	その証明は$\lang{\in}$や$\lang{\varepsilon}$の場合と殆ど同様であるが,
	``$|$''と``$\}$''のスコープの唯一性について書いておくと
	\begin{itemize}
		\item $\varphi$の中で``$|$''のスコープ$\Set{x}{\psi}$と$\Set{y}{\chi}$が取れたとすれば,
			$\psi$と$\chi$は$\varphi$の中で同じ位置から始まる式であるから
			メタ定理\ref{metathm:initial_segment_L_epsilon}より一致する.
			また$x$と$y$は変項であるからその中に``$\{$''が現れるはずはなく,$x$と$y$も一致すると判る.
			
		\item $\varphi$の中で``$\}$''のスコープ$\Set{x}{\psi}$と$\Set{y}{\chi}$が取れたとすれば,
			$\psi$と$\chi$は$\lang{\varepsilon}$の式であるからその中に``$|$''が現れるはずはなく,
			両者は一致していなくてはならない.すると上と同様に$x$と$y$も一致していなくてはならない.
	\end{itemize}
	
	\begin{screen}
		\begin{metathm}[$\mathcal{L}$のスコープの存在]
		\label{metathm:existence_of_scopes_L}
			$\varphi$を$\mathcal{L}$の式,或いは$\mathcal{L}$の項とするとき,
			\begin{description}
				\item[(a)] $\natural$が$\varphi$に現れたとき,変項$t$が得られて,
					$\natural$のその位置から$\natural t$なる項が$\varphi$の上に現れる.
					
				\item[(b)] $\{$が$\varphi$に現れたとき,変項$x$と$\mathcal{L}$の式$\psi$が得られて,
					$\{$のその出現位置から$\Set{x}{\psi}$なる項が$\varphi$の上に現れる.
					
				\item[(c)] $|$が$\varphi$に現れたとき,変項$x$と$\mathcal{L}$の式$\psi$が得られて,
					$|$のその出現位置を跨いで$\Set{x}{\psi}$なる項が$\varphi$の上に現れる.
					
				\item[(d)] $\}$が$\varphi$に現れたとき,変項$x$と$\mathcal{L}$の式$\psi$が得られて,
					$\}$のその出現位置右端にして$\Set{x}{\psi}$なる項が$\varphi$の上に現れる.
					
				\item[(e)] $\in$が$\varphi$に現れたとき,$\mathcal{L}$の項$\sigma,\tau$が得られて,
					$\in$のその出現位置から$\in \sigma \tau$なる式が$\varphi$の上に現れる.
				
				\item[(f)] $\negation$が$\varphi$に現れたとき,$\mathcal{L}$の式$\psi$が得られて,
					$\negation$のその出現位置から$\negation \psi$なる式が
					$\varphi$の上に現れる.	
				
				\item[(g)] $\vee$が$\varphi$に現れたとき,$\mathcal{L}$の式$\psi,\xi$が得られて,
					$\vee$のその出現位置から$\vee \psi \xi$なる式が$\varphi$の上に現れる.
				
				\item[(h)] $\exists$が$\varphi$に現れたとき,変項$x$と$\mathcal{L}$の式$\psi$が得られて,
					$\exists$のその出現位置から$\exists x \psi$なる式が$\varphi$の上に現れる.
			\end{description}
		\end{metathm}
	\end{screen}
	
	\begin{metaprf}\mbox{}
		\begin{description}
			\item[case1] $\in st$なる原子式に対しては,
				\begin{itemize}
					\item $\natural,\negation,\vee,\exists$が現れたとすれば,
						それらは$s$か$t$の中に現れているのであり,
						メタ定理\ref{metathm:existence_of_scopes_L_in}と
						メタ定理\ref{metathm:existence_of_scopes_L_epsilon}より
						それらのスコープは取れる.仮に$s$と$t$の一方が
						\begin{align}
							\Set{x}{\psi}
						\end{align}
						なる内包項であるとしても,$\natural,\negation,\vee,\exists$が
						現れうるのは$x$或いは$\psi$の中であるから,
						スコープの存在は上記のメタ定理に訴えればよい.
				
					\item $\in st$に$\in$が現れたとすれば,それが$s,t$の中のものならば
						上記の定理によってスコープは取れるし,それが$\in st$の左端の
						$\in$を指しているなら$\in st$自身をスコープとして取れば良い.
						
					\item $\in st$に$\{,\ |,\ \}$が現れたとすれば,$s$と$t$の少なくとも一方は
						\begin{align}
							\Set{x}{\psi}
						\end{align}
						なる項であることになるので,スコープとしてこの内包項を取れば良い.
				\end{itemize}
				
			\item[case2] $\varphi$を任意に与えられた$\mathcal{L}$の式として
				$\varphi$を任意に与えられた式として
				\begin{itembox}[l]{IH (帰納法の仮定)}
					$\varphi$の全ての真部分式に対しては(a)から(h)の主張が当てはまる
				\end{itembox}
				と仮定する.このとき,
				\begin{itemize}
					\item $\varphi$が
						\begin{align}
							\negation \psi
						\end{align}
						なる形の式であるとき,$\natural,\{,|,\},\in,\vee,\exists$が
						$\varphi$に現れたなら,それらは$\psi$の中に現れているのだから
						(IH)よりスコープが取れる.また$\varphi$に$\negation$が現れた場合,
						その$\negation$が$\psi$の中のものならば(IH)に訴えれば良いし,
						$\varphi$の左端の$\negation$を指しているなら
						スコープとして$\varphi$自身を取れば良い.
						
					\item $\varphi$が
						\begin{align}
							\vee \psi \chi
						\end{align}
						なる形の式であるとき,$\natural,\{,|,\},\in,\negation,\exists$が
						$\varphi$に現れたなら,それらは$\psi$か$\chi$の中に現れているのだから
						(IH)よりスコープが取れる.また$\varphi$に$\vee$が現れた場合,
						その$\vee$が$\psi,\chi$の中のものならば(IH)に訴えれば良いし,
						$\varphi$の左端の$\vee$を指しているなら
						スコープとして$\varphi$自身を取れば良い.
						
					\item $\varphi$が
						\begin{align}
							\exists x \psi
						\end{align}
						なる形の式であるとき,$\natural,\{,|,\}\in,\negation,\vee$が
						$\varphi$に現れたなら,それらは$\psi$の中に現れているのだから
						(IH)よりスコープが取れる.また$\varphi$に$\exists$が現れた場合,
						その$\exists$が$\psi$の中のものならば(IH)に訴えれば良いし,
						$\varphi$の左端の$\exists$を指しているなら
						スコープとして$\varphi$自身を取れば良い.
						\QED
				\end{itemize}
		\end{description}
	\end{metaprf}
	
\subsection{量化}
	$\varphi$を$\mathcal{L}$の式とする.もし$\varphi$に$\forall$が現れたら,
	その$\forall$に後続する変項$x$と式$\psi$が取れるが,そのとき$x$は
	\begin{align}
		\forall x \psi
	\end{align}
	の中で{\bf 「量化されている」}\index{りょうか@量化}{\bf(quantified)}や
	{\bf 「束縛されている」}\index{そくばく@束縛}{\bf (bound)}という.
	同様に$\varphi$の中に$\exists$や$\varepsilon$が現れたら,
	その$\exists$ (または$\varepsilon$)の直後にくる変項は,
	「その$\exists$ (または$\varepsilon$)のスコープの中で量化されている」といい,
	また$\varphi$の中に
	\begin{align}
		\Set{x}{\psi}
	\end{align}
	なる内包項が現れたら,$x$は「この内包項の中で量化されている」という.
	他方で$\psi$の中に$x$とは別の変項が現れていても,その変項は
	$\forall x \psi,\ \exists x \psi,\ \varepsilon x \psi,\ \Set{x}{\psi}$
	の中では「量化されていない」と解釈する.
	まとめれば,$\forall,\exists,\varepsilon,$そして$\{$は
	直後に来る変項のみをそのスコープ内で量化しているのである.たとえば
	\begin{align}
		\forall x\, (\, x \in y\, )
	\end{align}
	においては$x$は量化されているし,
	\begin{align}
		\Set{u}{u = z}
	\end{align}
	において$u$は量化されている.量化は二重に行われることもある.例えば
	\begin{align}
		\forall x\, (\, \forall x\, (\, x \in y\, ) \rarrow (\, x \in z\, )\, )
	\end{align}
	なる式においては,$\forall x\, (\, x \in y\, )$にある$x$は
	上式で一番左の$\forall$のスコープ内の$x$でもあるので,これらの$x$は二重に量化されていることになる.
	仮に「何重にも量化されている場合は最も狭いスコープで量化されていることにする」と決めても良いが,
	ただし重要なのは変項が量化されているか否かであって,それが二重でも三重でもどうでも構わない.
	
	上の例では$y$と$z$は量化されていないが,考えている項や式の中で量化されていない変項
	を{\bf 自由な}\index{じゆう@自由}{\bf (free)}変項と呼ぶ.
	現れる変項が自由であるか否かは当然その出現位置に依存しているのであり,たとえば
	\begin{align}
		\forall x\, (\, x \in y\, ) \rarrow (\, x \in z\, )
	\end{align}
	なる式では左の二つの$x$が量化されている一方で右の$x$は自由であるように,
	同じ変項が複数個所に現れる場合はその変項が量化されているか自由であるかは一概には言えない.
	式$\varphi$の中に量化されていない変項が現れている場合は,
	その変項が``その位置''に現れていることを
	{\bf 自由な出現}\index{じゆうなしゅつげん@自由な出現}{\bf (free occurrence)}と呼ぶ.
	
	\begin{screen}
		\begin{metadfn}[文]
			自由な変項が現れない$\mathcal{L}$の式を{\bf 文}\index{ぶん@文}{\bf (sentence)}
			や{\bf 閉式}\index{へいしき@閉式}{\bf (closed formula)}と呼ぶ.
		\end{metadfn}
	\end{screen}
	
\subsection{代入}
	変項とは束縛されうる項であったが,別の項を代入されうる項でもある.
	代入とは別の項で置き換えるということであり,また代入されうるのは式の中で自由な変項のみである.
	ただし,代入には「{\bf 式の中の自由な変項を別の変項に取り替えても式の意味を変えてはならない}」という
	大前提がある.たとえば
	\begin{align}
		\forall u\, (\, u \in x\, )
	\end{align}
	という式で考察すると,この式で$x$は自由であるから別の項を代入して良いのであり,$z$を代入すれば
	\begin{align}
		\forall u\, (\, u \in z\, )
	\end{align}
	となる.そしてこの場合はどちらの式も意味は同じである.意味が同じであるとは
	量化してみれば一目瞭然であって,両式を全称記号で量化すれば
	\begin{align}
		&\forall x\, \forall u\, (\, u \in x\, ), \\
		&\forall z\, \forall u\, (\, u \in z\, )
	\end{align}
	はどちらも「どの集合も,全ての集合を要素に持つ」と解釈され,
	両式を存在記号で量化すれば
	\begin{align}
		&\exists x\, \forall u\, (\, u \in x\, ), \\
		&\exists z\, \forall u\, (\, u \in z\, )
	\end{align}
	はどちらも「或る集合は,全ての集合を要素に持つ」と解釈される.
	ところが$x$に$u$を代入すると
	\begin{align}
		\forall u\, (\, u \in u\, )
	\end{align}
	となり,これは「全ての集合は自分自身を要素に持つ」という意味に変わる.
	つまり先の大前提に立てば,代入する際には{\bf 代入後に束縛されてしまう変項は使ってはいけない}のである.
	
	代入するのは変項だけではない.$\varepsilon$項や内包項だって上の$x$に代入して良い.
	ただし上と同様の注意が必要で,$\varepsilon$項や内包項に$u$が自由に現れている場合と
	そうでない場合では代入後の式の意味が分かれてしまうので,
	代入して良い項は$u$が自由に現れていないものに限る.
	
	以上の考察を一般的な代入規則に敷衍して言えば,
	
	\begin{itembox}[l]{代入可能な項}
		$\varphi$を$\mathcal{L}$の式とし,$x$を$\varphi$に自由に現れる変項とし,
		$\tau$を$\mathcal{L}$の項とする.このとき「$\varphi$に自由に現れる$x$に$\tau$を
		代入する」とは,特筆が無い限り$\varphi$に自由に現れる全ての$x$に
		$\tau$を代入することであって,その際に$\tau$が満たすべき条件は
		\begin{itemize}
			\item $\tau$が変項ならば$\tau$は$\varphi$に代入されたどの箇所でも自由である
			\item $\tau$が$\varepsilon$項や内包項である場合は,
				$\tau$の中に自由に現れる変項があったとしても,
				それらは全て$\tau$が代入されたどの箇所でも束縛されない
		\end{itemize}
		とする.$\tau$がこの条件を満たすとき,
		{\bf 「$\tau$は$\varphi$の中で$x$への代入について自由である」}という.
	\end{itembox}
	
	$\varphi$に自由に現れる$x$に$\tau$を代入した後の式を
	\begin{align}
		\varphi(x/\tau)
	\end{align}
	と書く($x/\tau$は``replace $x$ by $\tau$''の順).
	特に$\varphi$の中に自由に現れている変項が$x$だけである場合は,$\varphi(x/\tau)$を
	\begin{align}
		\varphi(\tau)
	\end{align}
	とも書く.$\tau$が$x$自身である場合は$\varphi(x)$は$\varphi$そのものであるが,
	「$\varphi$に自由に現れているのは$x$だけである」ということを強調するために
	\begin{align}
		\varphi(x)
	\end{align}
	と書くことも多い.$\varphi$に別の変項$y$が現れていて,$y$に項$\sigma$を代入するときは,
	\begin{align}
		\varphi(x/\tau)(y/\sigma)
	\end{align}
	を
	\begin{align}
		\varphi(x/\tau,y/\sigma)
	\end{align}
	とも書く.特に$\varphi$の中に自由に現れている変項が$x$と$y$だけである場合は,
	$\tau$と$\sigma$の代入先がはっきりしていれば
	\begin{align}
		\varphi(\tau,\sigma)
	\end{align}
	とも書くし,「$\varphi$に自由に現れているのは$x$と$y$だけである」ということを強調するために
	\begin{align}
		\varphi(x,y)
	\end{align}
	と書くことも多い.$\varphi$に$x$が自由に現れていない場合でも$\varphi(x/\tau)$などと書かれていたら,
	その式は$\varphi$のことであると理解する.
	
\subsection{類}
	\begin{comment}
	\begin{screen}
		\begin{dfn}[閉項]
			どの変項も自由に現れない$\varepsilon$項を
			{\bf 閉${\boldsymbol \varepsilon}$項}\index{
			へいイプシロンこう@閉$\varepsilon$項}{\bf (closed epsilon term)}と呼び,
			どの変項も自由に現れない内包項を{\bf 閉内包項}\index{
			へいないほうこう@閉内包項}{\bf (closed comprehension term)}と呼ぶ.
			また閉$\varepsilon$項と閉内包項は以上のみである.
		\end{dfn}
	\end{screen}
	\end{comment}
	
	元々の意図としては,例えば$x$のみが自由に現れる式$\varphi(x)$に対して
	``$\varphi(x)$を満たすいずれかの集合$x$''という意味を込めて
	\begin{align}
		\varepsilon x \varphi(x)
	\end{align}
	を作ったのだし,``$\varphi(x)$を満たす集合$x$の全体''という意味を込めて
	\begin{align}
		\Set{x}{\varphi(x)}
	\end{align}
	を作ったのである.つまりこの場合の$\varepsilon x \varphi(x)$と
	$\Set{x}{\varphi(x)}$は``意味を持っている''わけである.
	これが,もし$x$とは別の変項$y$が$\varphi$に自由に現れているとすれば,
	$\varepsilon x \varphi$も$\Set{x}{\varphi}$も$y$に依存してしまい
	意味が定まらなくなる.というのも,変項とは代入可能な項であるから,$y$に代入する項ごとに
	$\varepsilon x \varphi$と$\Set{x}{\varphi}$は別の意味を持ち得るのである.
	また項が閉じていても意味不明な場合がある.たとえば,$\psi$が文であるときに
	\begin{align}
		\varepsilon y \psi
	\end{align}
	や
	\begin{align}
		\Set{y}{\psi}
	\end{align}
	なる項は閉じてはいるが,導入の意図には適っていない.意味不明ながらこういった項が存在しているのは
	導入時にこれらを排除する面倒を避けたからであり,また一旦すべてを作り終えた後で余計なものを捨てる方が
	楽だからである.
	
	とりあえず,導入の意図に適っている項は特別の名前を持っていて然るべきである.
	
	\begin{screen}
		\begin{dfn}[類]
			$\varphi$を$\lang{\varepsilon}$の式とし,$x$を$\varphi$に自由に現れる変項とし,
			$\varphi$に自由に現れる項は$x$のみであるとするとき,$\varepsilon x \varphi$
			と$\Set{x}{\varphi}$を{\bf 類}\index{るい@類}{\bf (class)}と呼ぶ.
			またこれらのみが類である.
		\end{dfn}
	\end{screen}
	
	類には二種類あるので,それらも名前を分けておく.
	\begin{screen}
		\begin{dfn}[主要$\varepsilon$項]
			類である$\varepsilon$項を{\bf 主要${\boldsymbol \varepsilon}$項}
			\index{しゅよういぷしんろんこう@主要$\varepsilon$項}
			{\bf (critical epsilon term)}と呼ぶ.
		\end{dfn}
	\end{screen}
	
	\begin{screen}
		\begin{dfn}[主要内包項]
			類である内包項を{\bf 主要内包項}\index{しゅようないほうこう@主要内包項}と呼ぶ.
		\end{dfn}
	\end{screen}
	
	内包項に関しては便宜上自由な変項の出現も許すことにするが,
	たとえば$\Set{x}{\varphi}$と書いたら少なくとも$x$は$\varphi$に自由に現れているべきであり,
	この意味で性質の良い内包項に対しても特別な名前を付けておく.
	
	\begin{screen}
		\begin{dfn}[正則内包項]
			$\varphi$を$\lang{\varepsilon}$の式とし,$x$を変項とし,
			$\varphi$に$x$が自由に現れているとするとき,
			$\Set{x}{\varphi}$を{\bf 正則内包項}\index{せいそくないほうこう@正則内包項}と呼ぶ.
		\end{dfn}
	\end{screen}
	
\subsection{扱う式の制限}
\label{sec:restriction_of_formulas}
	{\bf 以降では扱う式は,特筆が無い限りそこに現れる$\varepsilon$項は全て
	主要$\varepsilon$項であり,現れる内包項は全て正則内包項であるとする.}

\subsection{式の書き換え}
	$\varepsilon$項を取り入れた目的は{\bf 存在文}\index{そんざいぶん@存在文}
	{\bf (existential sentence)}に対して{\bf 証人}\index{しょうにん@証人}{\bf (witness)}
	を与えることであり,それは
	\begin{align}
		\exists x \varphi(x) \rarrow \varphi(\varepsilon x \varphi(x))
	\end{align}
	なる式を公理とすることで実質的に裏付けされる.
	ただし$\varepsilon$項を作れる式は$\lang{\varepsilon}$の式のみであるから,
	$\varphi$が内包項を含んだ式であると$\varepsilon x \varphi(x)$を使うことが出来ない.
	とはいえ$\mathcal{L}$の式の存在文も往々にして登場するので
	それらに対しても証人を用意できると便利である.
	そこで$\mathcal{L}$の式を``同値''な$\lang{\varepsilon}$の式に書き換えて,
	その書き換えた式で作る$\varepsilon$項を使うことにする.つまり
	$\varphi$が$\mathcal{L}$の式である場合は,$\varphi$を
	``同値''な$\lang{\varepsilon}$の式$\hat{\varphi}$に書き換えてから
	\begin{align}
		\exists x \varphi(x) \rarrow \varphi(\varepsilon x \hat{\varphi}(x))
	\end{align}
	を保証するのである.書き換える必要があるのは内包項を含んでいる式のみであり,
	また原子式だけを書き換えれば十分である.
	書き換えが``同値''というのは後述の\ref{sec:equivalence_of_formula_rewriting}節
	で述べてあるような意味であるが,それは直感的に妥当なものである.原子式の書き換えは次の要領で行う:
	
	\begin{table}[H]
		\begin{center}
		\begin{tabular}{c|c|c}
			元の式 & 書き換え後 & 付記 \\ \hline \hline
			$a = \Set{z}{\psi}$ & $\forall v\, (\, v \in a \lrarrow \psi(z/v)\, )$ & \\ \hline
			$\Set{y}{\varphi} = b$ & $\forall u\, (\, \varphi(y/u) \lrarrow u \in b\, )$ & \\ \hline
			$\Set{y}{\varphi} = \Set{z}{\psi}$ & $\forall u\, (\, \varphi(y/u) \lrarrow \psi(z/u)\, )$ & \\ \hline
			$a \in \Set{z}{\psi}$ & $\psi(z/a)$ & 必要なら束縛変項の名前替えをする\footnotemark \\ \hline
			$\Set{y}{\varphi} \in b$ & $\exists s\, (\, \forall u\, (\, \varphi(y/u) \lrarrow u \in s\, ) \wedge s \in b\, )$ & \\ \hline
			$\Set{y}{\varphi} \in \Set{z}{\psi}$ & $\exists s\, (\, \forall u\, (\, \varphi(y/u) \lrarrow u \in s\, ) \wedge \psi(z/s)\, )$ & \\ \hline
		\end{tabular}
		\end{center}
	\end{table}
	
	ただし上の記号に課している条件は
	\begin{itemize}
		\item $a,b$は$\lang{\varepsilon}$の項である
			(\ref{sec:restriction_of_formulas}節より
			$a,b$は変項か主要$\varepsilon$項).
		\item $\Set{y}{\varphi}$と$\Set{z}{\psi}$を正則内包項である.
		\item $u$は$\varphi$の中で$y$への代入について自由であり,
			$u,v,s$は$\psi$の中で$z$への代入について自由である.
			上の式の書き換えにおいては変項$u,v,s$を追加したが,
			代入について自由である限りどの変項を選んでも構わない.
			従って式の書き換えは一つに決まらないということになるが,
			違う変項を選んでも式の意味は変わらない.
	\end{itemize}
	
	\footnotetext{
			$a$を$\psi$の中の自由な$z$に代入した後で$a$が束縛される場合,
			束縛変項の名前替えをしなくてはならない.たとえば
			\begin{align}
				a \in \Set{z}{\forall a\, (\, z \in a\, )}
			\end{align}
			という式では左辺の$a$は自由であるのに,書き換えの規則をそのまま適用すると
			\begin{align}
				\forall a\, (\, a \in a\, )
			\end{align}
			となり束縛されてしまう.代入後の$a$が束縛されないためには
			\begin{align}
				a \in \Set{z}{\forall b\, (\, z \in b\, )}
			\end{align}
			のように束縛変項$a$を別の変項$b$に替えて
			\begin{align}
				\forall b\, (\, a \in b\, )
			\end{align}
			とすればよい.
	}
	
	原子式に対する書き換えが掲示されたので,$\mathcal{L}$の一般の式$\varphi$から
	$\lang{\varepsilon}$の式$\hat{\varphi}$を得る操作は次の帰納的な要領で行えばよい.
	\begin{description}
		\item[case1] $\varphi$が
			\begin{align}
				\negation \psi
			\end{align}
			なる式であるとき,$\psi$を$\lang{\varepsilon}$の式に書き換えた$\hat{\psi}$を用いて
			\begin{align}
				\negation \hat{\psi}
			\end{align}
			を$\hat{\varphi}$とする.
			
		\item[case2] $\varphi$が
			\begin{align}
				\vee \psi \chi
			\end{align}
			なる式であるとき,$\psi,\chi$を$\lang{\varepsilon}$の式に書き換えた
			$\hat{\psi},\hat{\chi}$を用いて
			\begin{align}
				\vee \hat{\psi} \hat{\chi}
			\end{align}
			を$\hat{\varphi}$とする.$\varphi$が$\wedge \psi \chi$や$\rarrow \psi \chi$
			の形の時も同様にする.
			
		\item[case3] $\varphi$が
			\begin{align}
				\exists x \psi
			\end{align}
			なる式であるとき,$\psi$を$\lang{\varepsilon}$の式に書き換えた
			$\hat{\psi}$を用いて
			\begin{align}
				\exists x \hat{\psi}
			\end{align}
			を$\hat{\varphi}$とする.$\varphi$が$\forall x \psi$の形の時も同様にする.
	\end{description}
	
	もちろん$\varphi$が$\lang{\varepsilon}$の式ならわざわざ書き換える必要は無い.
	
	\begin{screen}
		\begin{metathm}[書き換え後も自由な変項は増減しない]
			$\varphi$を$\mathcal{L}$の式とし,これを$\lang{\varepsilon}$の式に
			書き換えたものを$\hat{\varphi}$とする.このとき
			$\varphi$に自由に現れる変項と$\hat{\varphi}$に自由に現れる変項は一致する.
		\end{metathm}
	\end{screen}
	
	\begin{metaprf}\mbox{}
		\begin{description}
			\item[step1] $\varphi$が原子式であるときは上の書き換え表より一目瞭然である.
			
			\item[step2]
				$\varphi$が一般の式であるとき
				\begin{itembox}[l]{IH (帰納法の仮定)}
					$\varphi$の任意の真部分式$\psi$と,それを$\lang{\varepsilon}$の式
					に書き換えた$\hat{\psi}$は,自由に現れる変項が一致する
				\end{itembox}
				と仮定する.すると
				\begin{description}
					\item[case1] $\varphi$が
						\begin{align}
							\negation \psi
						\end{align}
						なる式の場合,$\varphi$に自由に現れる変項は
						$\psi$に自由に現れる変項と一致するが,それは
						$\hat{\psi}$に自由に現れる変項と一致するので,
						$\negation \hat{\psi}$に自由に現れる変項とも一致する.
						
					\item[case2] $\varphi$が
						\begin{align}
							\vee \psi \chi
						\end{align}
						なる式の場合,$\varphi$に自由に現れる変項は$\psi,\chi$に自由に現れる
						変項と一致するが,それは$\hat{\psi},\hat{\chi}$に
						自由に現れる変項と一致するので,
						$\negation \hat{\psi} \hat{\chi}$に自由に現れる変項とも一致する.
					
					\item $\varphi$が
						\begin{align}
							\exists x \psi
						\end{align}
						なる式の場合,$\varphi$に自由に現れる変項は
						$\psi$に自由に現れる$x$以外の変項と一致するが,それは
						$\hat{\psi}$に自由に現れる$x$以外の変項変項と一致するので,
						$\negation \hat{\psi}$に自由に現れる変項とも一致する.
						\QED
				\end{description}
		\end{description}
	\end{metaprf}
	
\subsection{中置記法}
	たとえば$\in s t$なる原子式は「$s$は$t$の要素である($s$ is in $t$)」と読むのだから,語順通りに,
	或いは$s$が$t$の中にあるというイメージ通りに
	\begin{align}
		s \in t
	\end{align}
	と書きかえる方が見やすくなる.同じように,$\vee \varphi \psi$なる式も
	「$\varphi$または$\psi$」と読むのだから
	\begin{align}
		\varphi \vee \psi
	\end{align}
	と書きかえる方が見やすくなる.$\rarrow \vee \varphi \psi \wedge \chi \xi$のように長い式も,
	上の作法に倣えば
	\begin{align}
		\begin{gathered}
			\rarrow \vee \varphi \psi \wedge \chi \xi \\
			\rarrow \color{red}{\varphi \vee \psi} \color{blue}{\chi \wedge \xi} \\
			\color{red}{\varphi \vee \psi} \color{black}{\rarrow} \color{blue}{\chi \wedge \xi}
		\end{gathered}
	\end{align}
	と書きかえることになるが,一々色分けするわけにもいかないので``(''と``)''を使って
	\begin{align}
		(\varphi \vee \psi) \rarrow (\chi \wedge \xi)
	\end{align}
	と書くようにすれば良い.
	
	\begin{itembox}[l]{{\bf 中置記法}\index{ちゅうちきほう@中置記法}{\bf (infix notation)}}
			$\mathcal{L}$の式は以下の手順で中置記法に書き換える.
			\begin{enumerate}
				\item $\in s t$なる形の原子式は$s \in t$と書きかえる.
					$= s t$も同様に書き換える.
					
				\item $\negation \varphi$なる形の式はそのままにする.
				
				\item $\vee \varphi \psi$なる形の式は$(\varphi \vee \psi)$と書きかえる.
					$\wedge \varphi \psi$と$\rarrow \varphi \psi$の形の式も同様に書き換える.
				
				\item $\exists x \varphi$なる形の式はそのままにする.
					$\forall x \varphi$なる形の式も同様にする.
			\end{enumerate}
	\end{itembox}
	
	上の書き換え法では,たとえば$\rarrow \vee \varphi \psi \wedge \chi \xi$なる式は
	\begin{align}
		((\varphi \vee \psi) \rarrow (\chi \wedge \xi))
	\end{align}
	となるが,括弧はあくまで式の境界の印として使うものであるから,一番外側の括弧は外して
	\begin{align}
		(\varphi \vee \psi) \rarrow (\chi \wedge \xi)
	\end{align}
	と書く方が良い.よって{\bf 中置記法に書き換え終わったときに一番外側にある括弧は外す}ことにする.
	
	$\wedge \vee \exists x \varphi \psi \negation \rarrow \chi \in s t$なる式は
	\begin{align}
		\begin{gathered}
			\wedge \vee \exists x \varphi \psi \negation \rarrow \chi s \in t \\
			\wedge (\exists x \varphi \vee \psi) \negation (\chi \rarrow s \in t) \\
			(\exists x \varphi \vee \psi) \wedge \negation (\chi \rarrow s \in t)
		\end{gathered}
	\end{align}
	となる.
	
	ただしあまり括弧が連なると読みづらくなるので,
	\begin{align}
		(\varphi \vee \psi) \rarrow \chi
	\end{align}
	なる形の式は
	\begin{align}
		\varphi \vee \psi \rarrow \chi
	\end{align}
	に,同様に
	\begin{align}
		\varphi \rarrow (\psi \vee \chi)
	\end{align}
	なる形の式は
	\begin{align}
		\varphi \rarrow \psi \vee \chi
	\end{align}
	とも書く.また$\vee$が$\wedge$であっても同じように括弧を省く.
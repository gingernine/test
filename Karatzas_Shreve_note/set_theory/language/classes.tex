\subsection{類}
	\begin{comment}
	\begin{screen}
		\begin{dfn}[閉項]
			どの変項も自由に現れない$\varepsilon$項を
			{\bf 閉${\boldsymbol \varepsilon}$項}\index{
			へいイプシロンこう@閉$\varepsilon$項}{\bf (closed epsilon term)}と呼び,
			どの変項も自由に現れない内包項を{\bf 閉内包項}\index{
			へいないほうこう@閉内包項}{\bf (closed comprehension term)}と呼ぶ.
			また閉$\varepsilon$項と閉内包項は以上のみである.
		\end{dfn}
	\end{screen}
	\end{comment}
	
	元々の意図としては,例えば$x$のみが自由に現れる式$\varphi(x)$に対して
	``$\varphi(x)$を満たすいずれかの集合$x$''という意味を込めて
	\begin{align}
		\varepsilon x \varphi(x)
	\end{align}
	を作ったのだし,``$\varphi(x)$を満たす集合$x$の全体''という意味を込めて
	\begin{align}
		\Set{x}{\varphi(x)}
	\end{align}
	を作ったのである.つまりこの場合の$\varepsilon x \varphi(x)$と
	$\Set{x}{\varphi(x)}$は``意味を持っている''わけである.
	これが,もし$x$とは別の変項$y$が$\varphi$に自由に現れているとすれば,
	$\varepsilon x \varphi$も$\Set{x}{\varphi}$も$y$に依存してしまい
	意味が定まらなくなる.というのも,変項とは代入可能な項であるから,$y$に代入する項ごとに
	$\varepsilon x \varphi$と$\Set{x}{\varphi}$は別の意味を持ち得るのである.
	また項が閉じていても意味不明な場合がある.たとえば,$\psi$が文であるときに
	\begin{align}
		\varepsilon y \psi
	\end{align}
	や
	\begin{align}
		\Set{y}{\psi}
	\end{align}
	なる項は閉じてはいるが,導入の意図には適っていない.意味不明ながらこういった項が存在しているのは
	導入時にこれらを排除する面倒を避けたからであり,また一旦すべてを作り終えた後で余計なものを捨てる方が
	楽だからである.
	
	とりあえず,導入の意図に適っている項は特別の名前を持っていて然るべきである.
	
	\begin{screen}
		\begin{dfn}[類]
			$\varphi$を$\lang{\varepsilon}$の式とし,$x$を$\varphi$に自由に現れる変項とし,
			$\varphi$に自由に現れる項は$x$のみであるとするとき,$\varepsilon x \varphi$
			と$\Set{x}{\varphi}$を{\bf 類}\index{るい@類}{\bf (class)}と呼ぶ.
			またこれらのみが類である.
		\end{dfn}
	\end{screen}
	
	類には二種類あるので,それらも名前を分けておく.
	\begin{screen}
		\begin{dfn}[主要$\varepsilon$項]
			類である$\varepsilon$項を{\bf 主要${\boldsymbol \varepsilon}$項}
			\index{しゅよういぷしんろんこう@主要$\varepsilon$項}
			{\bf (critical epsilon term)}と呼ぶ.
		\end{dfn}
	\end{screen}
	
	\begin{screen}
		\begin{dfn}[主要内包項]
			類である内包項を{\bf 主要内包項}\index{しゅようないほうこう@主要内包項}と呼ぶ.
		\end{dfn}
	\end{screen}
	
	内包項に関しては便宜上自由な変項の出現も許すことにするが,
	たとえば$\Set{x}{\varphi}$と書いたら少なくとも$x$は$\varphi$に自由に現れているべきであり,
	この意味で性質の良い内包項に対しても特別な名前を付けておく.
	
	\begin{screen}
		\begin{dfn}[正則内包項]
			$\varphi$を$\lang{\varepsilon}$の式とし,$x$を変項とし,
			$\varphi$に$x$が自由に現れているとするとき,
			$\Set{x}{\varphi}$を{\bf 正則内包項}\index{せいそくないほうこう@正則内包項}と呼ぶ.
		\end{dfn}
	\end{screen}
	
\subsection{扱う式の制限}
\label{sec:restriction_of_formulas}
	\begin{itembox}[l]{式の制限}
		以降で扱う$\mathcal{L}$の項と式に対して,特筆が無い限り次が満たされていることを約束する:
		\begin{itemize}
			\item 式に現れる$\varepsilon$項は全て主要$\varepsilon$項である.
			\item 式に現れる内包項は全て正則内包項である.
			\item 項或いは式の上に現れる$\forall x \psi,\exists x \psi$なる形の式は,$\psi$の中に$x$が自由に現れている.
		\end{itemize}
	\end{itembox}
	
	項の中に現れる$\varepsilon$項も,項の中の項の中に現れる$\varepsilon$項も,
	現れうる$\varepsilon$項は全て主要$\varepsilon$項である.
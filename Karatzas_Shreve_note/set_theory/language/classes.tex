\subsection{類}
	\begin{comment}
	\begin{screen}
		\begin{dfn}[閉項]
			どの変項も自由に現れない$\varepsilon$項を
			{\bf 閉${\boldsymbol \varepsilon}$項}\index{
			へいイプシロンこう@閉$\varepsilon$項}{\bf (closed epsilon term)}と呼び,
			どの変項も自由に現れない内包項を{\bf 閉内包項}\index{
			へいないほうこう@閉内包項}{\bf (closed comprehension term)}と呼ぶ.
			また閉$\varepsilon$項と閉内包項は以上のみである.
		\end{dfn}
	\end{screen}
	\end{comment}
	
	元々の意図としては,例えば$x$のみが自由に現れる式$\varphi(x)$に対して
	「$\varphi(x)$を満たすいずれかの集合$x$」という意味を込めて
	\begin{align}
		\varepsilon x \varphi(x)
	\end{align}
	を作ったのだし,「$\varphi(x)$を満たす集合$x$の全体」という意味を込めて
	\begin{align}
		\Set{x}{\varphi(x)}
	\end{align}
	を作ったのである.つまりこの場合の$\varepsilon x \varphi(x)$と
	$\Set{x}{\varphi(x)}$は``意味を持っている''わけである.
	これが,もし$x$とは別の変項$y$が$\varphi$に自由に現れているとすれば,
	$\varepsilon x \varphi$も$\Set{x}{\varphi}$も$y$に依存してしまい
	意味が定まらなくなる.というのも,変項とは代入可能な項であるから,$y$に代入する項ごとに
	$\varepsilon x \varphi$と$\Set{x}{\varphi}$は別の意味を持ち得るのである.
	また項が閉じていても意味不明な場合がある.たとえば
	\begin{align}
		\varepsilon y \forall x\, (\, x = x\, )
	\end{align}
	や
	\begin{align}
		\Set{y}{\forall x\, (\, x = x\, )}
	\end{align}
	なる項は閉じてはいるが,導入の意図には適っていない.意味不明ながらこういった項が存在しているのは
	導入時にこれらを排除する面倒を避けたからであり,また一旦すべてを作り終えた後で余計なものを捨てる方が
	楽だからである.とりあえず,導入の意図に適っている項は特別の名前を持っているべきである.
	
	\begin{screen}
		\begin{dfn}[類]
			$\varphi$を$\lang{\varepsilon}$の式とし,$x$を変項とし,
			$\varphi$には$x$のみ自由に現れているとするとき,$\varepsilon x \varphi(x)$
			と$\Set{x}{\varphi(x)}$を{\bf 類}\index{るい@類}{\bf (class)}と呼ぶ.
			またこれらのみが類である.
		\end{dfn}
	\end{screen}
	
	類には二種類あるので,それらも名前を分けておく.
	\begin{screen}
		\begin{dfn}[主要$\varepsilon$項]
			類である$\varepsilon$項を{\bf 主要${\boldsymbol \varepsilon}$項}
			\index{しゅよういぷしんろんこう@主要$\varepsilon$項}
			{\bf (critical epsilon term)}と呼ぶ.
		\end{dfn}
	\end{screen}
	
	後述することだが,本稿における集合とは,主要$\varepsilon$項か
	主要$\varepsilon$項に等しい類のことである
	(定理\ref{thm:critical_epsilon_term_is_set}).
	
	
	\begin{screen}
		\begin{dfn}[主要内包項]
			類である内包項を{\bf 主要内包項}\index{しゅようないほうこう@主要内包項}と呼ぶ.
		\end{dfn}
	\end{screen}
	
	内包項に関しては便宜上自由な変項の出現も許すことにするが,
	たとえば$\Set{x}{\varphi}$と書いたら少なくとも$x$は$\varphi$に自由に現れているべきであり,
	この意味で性質の良い内包項に対しても特別な名前を付けておく.
	
	\begin{screen}
		\begin{dfn}[正則内包項]
			$\varphi$を$\lang{\varepsilon}$の式とし,$x$を変項とし,
			$\varphi$に$x$が自由に現れているとするとき,
			$\Set{x}{\varphi}$を{\bf 正則内包項}\index{せいそくないほうこう@正則内包項}と呼ぶ.
		\end{dfn}
	\end{screen}
	
\subsection{扱う式の制限}
\label{sec:restriction_of_formulas}
	\begin{itembox}[l]{式の制限}
		以降で扱う$\mathcal{L}$の項と式に対して,特筆が無い限り次が満たされていることを約束する:
		\begin{itemize}
			\item 式に現れる$\varepsilon$項は全て主要$\varepsilon$項である.
			\item 式に現れる内包項は全て正則内包項である.
			\item 項或いは式の上に現れる$\forall x \psi,\exists x \psi$なる形の式は,$\psi$の中に$x$が自由に現れている.
		\end{itemize}
	\end{itembox}
	
	項の中に現れる$\varepsilon$項も,項の中の項の中に現れる$\varepsilon$項も,
	現れうる$\varepsilon$項は全て主要$\varepsilon$項である.
	
	\begin{screen}
		\begin{metathm}[$\lang{\varepsilon}$の式に代入後も$\lang{\varepsilon}$の式]
		\label{metathm:substitutions_into_L_epsilon_formulas}
			$\varphi$を$\lang{\varepsilon}$の式とし,$x$を$\varphi$に自由に現れる変項とし,
			$a$を$\varphi$の中で$x$への代入について自由である$\lang{\varepsilon}$の項
			(変項または主要$\varepsilon$項)とする.
			このとき$\varphi(x/a)$は$\lang{\varepsilon}$の式である.
		\end{metathm}
	\end{screen}
	
	\begin{metaprf}\mbox{}
		\begin{description}
			\item[step1] $\varphi$が原子式であるとき,例えば$\varphi$が
				\begin{align}
					x \in b
				\end{align}
				なる式であれば,$\varphi(x/a)$は
				\begin{align}
					a \in b
				\end{align}
				なる式である($b$が$x$ならば$\varphi(x/a)$は$a \in a$となる.$b$が$x$でない
				ならば$b$は変項か主要$\varepsilon$項なので$\varphi(x/a)$は$a \in b$となる).
				他の場合も同様に$\varphi(x/a)$が$\lang{\in}$の式であると判る.
			
			\item[step2] $\varphi$が原子式でないとき,
				\begin{itembox}[l]{IH (帰納法の仮定)}
					$\varphi$の任意の真部分式$\psi$に対して
					$\psi(x/a)$は$\lang{\varepsilon}$の式である
				\end{itembox}
				と仮定する.
				\begin{description}
					\item[case1] $\varphi$が
						\begin{align}
							\negation \psi
						\end{align}
						なる式のとき,$\varphi(x/a)$は
						\begin{align}
							\negation \psi(x/a)
						\end{align}
						なる式であって,(IH)より$\psi(x/a)$は$\lang{\varepsilon}$の式であるから
						$\varphi(x/a)$は$\lang{\varepsilon}$の式である.
						
					\item[case2] $\varphi$が
						\begin{align}
							\vee \psi \xi
						\end{align}
						なる式のとき,$\varphi(x/a)$は
						\begin{align}
							\vee \psi(x/a) \xi(x/a)
						\end{align}
						なる式であって,(IH)より$\psi(x/a),\xi(x/a)$は$\lang{\varepsilon}$の式
						であるから$\varphi(x/a)$は$\lang{\varepsilon}$の式である.
					
					\item[case3] $\varphi$が
						\begin{align}
							\exists z \psi
						\end{align}
						なる式のとき,$\varphi(x/a)$は
						\begin{align}
							\exists z \psi(x/a)
						\end{align}
						なる式であって,(IH)より$\psi(x/a)$は$\lang{\varepsilon}$の式であるから
						$\varphi(x/a)$は$\lang{\varepsilon}$の式である.
						\QED
				\end{description}
		\end{description}
	\end{metaprf}
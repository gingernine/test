\subsection{量化}
	$\varphi$を$\mathcal{L}$の式とする.もし$\varphi$に$\forall$が現れたら,
	その$\forall$に後続する変項$x$と式$\psi$が取れるが,そのとき$x$は
	\begin{align}
		\forall x \psi
	\end{align}
	の中で{\bf 「量化されている」}\index{りょうか@量化}{\bf(quantified)}や
	{\bf 「束縛されている」}\index{そくばく@束縛}{\bf (bound)}という.
	同様に$\varphi$の中に$\exists$や$\varepsilon$が現れたら,
	その$\exists$ (または$\varepsilon$)の直後にくる変項は,
	「その$\exists$ (または$\varepsilon$)のスコープの中で量化されている」といい,
	また$\varphi$の中に
	\begin{align}
		\Set{x}{\psi}
	\end{align}
	なる内包項が現れたら,$x$は「この内包項の中で量化されている」という.
	他方で$\psi$の中に$x$とは別の変項が現れていても,その変項は
	$\forall x \psi,\ \exists x \psi,\ \varepsilon x \psi,\ \Set{x}{\psi}$
	の中では「量化されていない」と解釈する.
	まとめれば,$\forall,\exists,\varepsilon,$そして$\{$は
	直後に来る変項のみをそのスコープ内で量化しているのである.たとえば
	\begin{align}
		\forall x\, (\, x \in y\, )
	\end{align}
	においては$x$は量化されているし,
	\begin{align}
		\Set{u}{u = z}
	\end{align}
	において$u$は量化されている.量化は二重に行われることもある.例えば
	\begin{align}
		\forall x\, (\, \forall x\, (\, x \in y\, ) \rarrow (\, x \in z\, )\, )
	\end{align}
	なる式においては,$\forall x\, (\, x \in y\, )$にある$x$は
	上式で一番左の$\forall$のスコープ内の$x$でもあるので,これらの$x$は二重に量化されていることになる.
	仮に「何重にも量化されている場合は最も狭いスコープで量化されていることにする」と決めても良いが,
	ただし重要なのは変項が量化されているか否かであって,それが二重でも三重でもどうでも構わない.
	
	上の例では$y$と$z$は量化されていないが,考えている項や式の中で量化されていない変項
	を{\bf 自由な}\index{じゆう@自由}{\bf (free)}変項と呼ぶ.
	現れる変項が自由であるか否かは当然その出現位置に依存しているのであり,たとえば
	\begin{align}
		\forall x\, (\, x \in y\, ) \rarrow (\, x \in z\, )
	\end{align}
	なる式では左の二つの$x$が量化されている一方で右の$x$は自由であるように,
	同じ変項が複数個所に現れる場合はその変項が量化されているか自由であるかは一概には言えない.
	式$\varphi$の中に量化されていない変項が現れている場合は,
	その変項が``その位置''に現れていることを
	{\bf 自由な出現}\index{じゆうなしゅつげん@自由な出現}{\bf (free occurrence)}と呼ぶ.
	
	\begin{screen}
		\begin{metadfn}[文]
			自由な変項が現れない$\mathcal{L}$の式を{\bf 文}\index{ぶん@文}{\bf (sentence)}
			や{\bf 閉式}\index{へいしき@閉式}{\bf (closed formula)}と呼ぶ.
		\end{metadfn}
	\end{screen}
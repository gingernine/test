\section{中置記法}
	たとえば$\in s t$なる原子式は「$s$は$t$の要素である($s$ is in $t$)」と読むのだから,語順通りに,
	或いは$s$が$t$の中にあるというイメージ通りに
	\begin{align}
		s \in t
	\end{align}
	と書きかえる方が見やすくなる.同じように,$\vee \varphi \psi$なる式も
	「$\varphi$または$\psi$」と読むのだから
	\begin{align}
		\varphi \vee \psi
	\end{align}
	と書きかえる方が見やすくなる.$\rarrow \vee \varphi \psi \wedge \chi \xi$のように長い式も,
	上の作法に倣えば
	\begin{align}
		\begin{gathered}
			\rarrow \vee \varphi \psi \wedge \chi \xi \\
			\rarrow \color{red}{\varphi \vee \psi} \color{blue}{\chi \wedge \xi} \\
			\color{red}{\varphi \vee \psi} \color{black}{\rarrow} \color{blue}{\chi \wedge \xi}
		\end{gathered}
	\end{align}
	と書きかえることになるが,一々色分けするわけにもいかないので``(''と``)''を使って
	\begin{align}
		(\varphi \vee \psi) \rarrow (\chi \wedge \xi)
	\end{align}
	と書くようにすれば良い.
	
	\begin{itembox}[l]{{\bf 中置記法}\index{ちゅうちきほう@中置記法}{\bf (infix notation)}}
			$\mathcal{L}$の式は以下の手順で中置記法に変換する.
			\begin{enumerate}
				\item $\in s t$なる形の原子式は$s \in t$と書きかえる.
					$= s t$も同様に書き換える.
					
				\item $\negation \varphi$なる形の式は,$\varphi$の中置記法への変換
					$\widehat{\varphi}$を用いて$\negation (\widehat{\varphi})$と変換する.
				
				\item $\vee \varphi \psi$なる形の式は,$\varphi,\psi$の中置記法への変換
					$\widehat{\varphi},\widehat{\psi}$を用いて
					$(\widehat{\varphi}) \vee (\widehat{\psi})$と変換する.
					$\wedge \varphi \psi$と$\rarrow \varphi \psi$の形の式も同様に変換する.
				
				\item $\exists x \varphi$なる形の式は,$\varphi$の中置記法への変換
					$\widehat{\varphi}$を用いて$\exists x (\widehat{\varphi})$と変換する.
					$\forall x \varphi,\varepsilon x \varphi$なる形の式や項も同様にする.
					
				\item $\Set{x}{\varphi}$なる形の項は,$\varphi$の中置記法への変換
					$\widehat{\varphi}$を用いて$\Set{x}{\widehat{\varphi}}$と変換する.
			\end{enumerate}
			中置記法は表示用の記法であって,扱う項や式の``本来の姿''は前節までの{\bf 前置記法}
			\index{ぜんちきほう@前置記法}{\bf (prefix notation)}で書かれたものである.
	\end{itembox}
	
	上の変換法では,たとえば$\rarrow \vee \varphi \psi \wedge \chi \xi$なる式は
	\begin{align}
		(\, (\widehat{\varphi}) \vee (\widehat{\psi})\, ) 
		\rarrow (\, (\widehat{\chi}) \wedge (\widehat{\xi})\, )
	\end{align}
	となるが,括弧はあくまで式の境界の印として使うものであるから,内側の括弧は外して
	\begin{align}
		(\, \widehat{\varphi} \vee \widehat{\psi}\, ) 
		\rarrow (\, \widehat{\chi} \wedge \widehat{\xi}\, )
	\end{align}
	と書く方が良い.
	
	$\wedge \vee \exists x \varphi \psi \negation \rarrow \chi \in s t$なる式を変換すると
	\begin{align}
		(\, (\exists x (\widehat{\varphi})) \vee (\widehat{\psi})\, ) \wedge (\negation (\, (\widehat{\chi}) \rarrow (s \in t)\, ))
	\end{align}
	となるが,これも内側の括弧および$\negation ...$を囲う括弧は外して
	\begin{align}
		(\, \exists x (\widehat{\varphi}) \vee \widehat{\psi}\, ) \wedge \negation (\, \widehat{\chi} \rarrow s \in t\, )
	\end{align}
	と書く.
	
	あまり括弧が連なると読みづらくなるので,
	\begin{align}
		(\, \varphi \vee \psi\, ) \rarrow (\chi)
	\end{align}
	なる形の式は
	\begin{align}
		\varphi \vee \psi \rarrow \chi
	\end{align}
	に,同様に
	\begin{align}
		(\varphi) \rarrow (\, \psi \vee \chi\, )
	\end{align}
	なる形の式は
	\begin{align}
		\varphi \rarrow \psi \vee \chi
	\end{align}
	とも書く.また$\vee$が$\wedge$であっても同じように括弧を省く.
	\begin{align}
		\exists x (\negation (\varphi))
	\end{align}
	なる式は
	\begin{align}
		\exists x \negation (\varphi)
	\end{align}
	とも書き,$\exists$が$\forall$や$\varepsilon$であっても同じように括弧を省く.
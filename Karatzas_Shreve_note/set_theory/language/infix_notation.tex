\subsection{中置記法}
	たとえば$\in s t$なる原子式は「$s$は$t$の要素である($s$ is in $t$)」と読むのだから,語順通りに,
	或いは$s$が$t$の中にあるというイメージ通りに
	\begin{align}
		s \in t
	\end{align}
	と書きかえる方が見やすくなる.同じように,$\vee \varphi \psi$なる式も
	「$\varphi$または$\psi$」と読むのだから
	\begin{align}
		\varphi \vee \psi
	\end{align}
	と書きかえる方が見やすくなる.$\rarrow \vee \varphi \psi \wedge \chi \xi$のように長い式も,
	上の作法に倣えば
	\begin{align}
		\begin{gathered}
			\rarrow \vee \varphi \psi \wedge \chi \xi \\
			\rarrow \color{red}{\varphi \vee \psi} \color{blue}{\chi \wedge \xi} \\
			\color{red}{\varphi \vee \psi} \color{black}{\rarrow} \color{blue}{\chi \wedge \xi}
		\end{gathered}
	\end{align}
	と書きかえることになるが,一々色分けするわけにもいかないので``(''と``)''を使って
	\begin{align}
		(\varphi \vee \psi) \rarrow (\chi \wedge \xi)
	\end{align}
	と書くようにすれば良い.
	
	\begin{itembox}[l]{{\bf 中置記法}\index{ちゅうちきほう@中置記法}{\bf (infix notation)}}
			$\mathcal{L}$の式は以下の手順で中置記法に書き換える.
			\begin{enumerate}
				\item $\in s t$なる形の原子式は$s \in t$と書きかえる.
					$= s t$も同様に書き換える.
					
				\item $\negation \varphi$なる形の式はそのままにする.
				
				\item $\vee \varphi \psi$なる形の式は$(\varphi \vee \psi)$と書きかえる.
					$\wedge \varphi \psi$と$\rarrow \varphi \psi$の形の式も同様に書き換える.
				
				\item $\exists x \varphi$なる形の式はそのままにする.
					$\forall x \varphi$なる形の式も同様にする.
			\end{enumerate}
	\end{itembox}
	
	上の書き換え法では,たとえば$\rarrow \vee \varphi \psi \wedge \chi \xi$なる式は
	\begin{align}
		((\varphi \vee \psi) \rarrow (\chi \wedge \xi))
	\end{align}
	となるが,括弧はあくまで式の境界の印として使うものであるから,一番外側の括弧は外して
	\begin{align}
		(\varphi \vee \psi) \rarrow (\chi \wedge \xi)
	\end{align}
	と書く方が良い.よって{\bf 中置記法に書き換え終わったときに一番外側にある括弧は外す}ことにする.
	
	$\wedge \vee \exists x \varphi \psi \negation \rarrow \chi \in s t$なる式は
	\begin{align}
		\begin{gathered}
			\wedge \vee \exists x \varphi \psi \negation \rarrow \chi s \in t \\
			\wedge (\exists x \varphi \vee \psi) \negation (\chi \rarrow s \in t) \\
			(\exists x \varphi \vee \psi) \wedge \negation (\chi \rarrow s \in t)
		\end{gathered}
	\end{align}
	となる.
	
	ただしあまり括弧が連なると読みづらくなるので,
	\begin{align}
		(\varphi \vee \psi) \rarrow \chi
	\end{align}
	なる形の式は
	\begin{align}
		\varphi \vee \psi \rarrow \chi
	\end{align}
	に,同様に
	\begin{align}
		\varphi \rarrow (\psi \vee \chi)
	\end{align}
	なる形の式は
	\begin{align}
		\varphi \rarrow \psi \vee \chi
	\end{align}
	とも書く.また$\vee$が$\wedge$であっても同じように括弧を省く.
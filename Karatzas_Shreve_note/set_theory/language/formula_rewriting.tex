\subsection{式の書き換え}
\label{subsec:formula_rewriting}
	$\varepsilon$項を取り入れたのは{\bf 存在文}\index{そんざいぶん@存在文}
	{\bf (existential sentence)}に対して{\bf 証人}\index{しょうにん@証人}
	{\bf (witnessing term)}を与えるためであり,それは
	\begin{align}
		\exists x \varphi(x) \rarrow \varphi(\varepsilon x \varphi(x))
	\end{align}
	なる式を公理とすることで裏付けされる.ただし$\varepsilon$項を作れるのは$\lang{\varepsilon}$
	の式のみであるから,$\varphi$が内包項を含んだ式であると$\varepsilon x \varphi(x)$を
	使うことが出来ない.とはいえ内包項を含んだ存在文も往々にして登場するので,それらに対しても
	証人を用意できると便利である.そこで$\varphi$を内包項を含んだ$\mathcal{L}$の式とするとき,
	$\varphi$を``同値''な$\lang{\varepsilon}$の式$\widehat{\varphi}$に書き換えて
	\begin{align}
		\exists x \varphi(x) \rarrow \varphi(\varepsilon x \widehat{\varphi}(x))
	\end{align}
	を公理とする(量化公理\ref{logicalaxm:rules_of_quantifiers}).
	注意点は\underline{同値な書き換えはいくらでも作れる}ということであり,
	$\check{\varphi}$も$\varphi$の書き換えならば
	\begin{align}
		\exists x \varphi(x) \rarrow \varphi(\varepsilon x \check{\varphi}(x))
	\end{align}
	も公理とする.書き換える必要があるのは内包項を含んでいる式のみであり,
	またそのような式の{\bf 原子式の部分,つまり$\in$と$=$のスコープ,だけを書き換えれば十分}である.
	書き換えが``同値''というのは後述の\ref{sec:equivalence_of_formula_rewriting}節
	で述べてあるような意味であり,直感的には妥当な範囲でしかないのだが,
	書き換え時に課す条件が多いため読むだけでも嫌気がさすかもしれない.
	実際のところは,とりあえず「式の書き換えが可能である」ということと「式の書き換えの形」だけを認識されれば,
	その他の些末事項を当面無視しても大して問題は無い.
	原子式の書き換えは次の要領で行う:
	
	\begin{table}[H]
		\begin{center}
		\begin{tabular}{c|c|c}
			元の式 & 書き換え後 & 付記 \\ \hline \hline
			$a = \Set{z}{\psi}$ & $\forall v\, (\, v \in a \lrarrow \psi(z/v)\, )$ & $a$と$v$は違う項である \\ \hline
			$\Set{y}{\varphi} = b$ & $\forall u\, (\, \varphi(y/u) \lrarrow u \in b\, )$ & $b$と$u$は違う項である \\ \hline
			$\Set{y}{\varphi} = \Set{z}{\psi}$ & $\forall u\, (\, \varphi(y/u) \lrarrow \psi(z/u)\, )$ & \\ \hline
			$a \in \Set{z}{\psi}$ & $\psi(z/a)$ & 必要なら束縛変項の名前替えをする \\ \hline
			$\Set{y}{\varphi} \in b$ & $\exists s\, (\, \forall u\, (\, \varphi(y/u) \lrarrow u \in s\, ) \wedge s \in b\, )$ & $s$は$b$とも$u$とも違う項である \\ \hline
			$\Set{y}{\varphi} \in \Set{z}{\psi}$ & $\exists s\, (\, \forall u\, (\, \varphi(y/u) \lrarrow u \in s\, ) \wedge \psi(z/s)\, )$ & $s$と$u$は違う項である \\ \hline
		\end{tabular}
		\end{center}
	\end{table}
	
	ただし上の項が満たすべき条件は
	\begin{itemize}
		\item $a,b$は$\lang{\varepsilon}$の項である
			(\ref{sec:restriction_of_formulas}節より
			$a,b$は変項か主要$\varepsilon$項).
		
		\item $\Set{y}{\varphi}$と$\Set{z}{\psi}$を正則内包項である.
		
		\item 一つ目の書き換えでは$v$は$\psi$に自由に現れる$z$以外のどの変項とも違う.
		
		\item 二つ目の書き換えでは$u$は$\varphi$に自由に現れる$y$以外のどの変項とも違う.
		
		\item 三つ目の書き換えでは$u$は,$\varphi$に自由に現れる$y$以外のどの変項とも,
			$\psi$に自由に現れる$z$以外のどの変項とも違う.
		
		\item 五つ目の書き換えでは$s$と$u$は$\varphi$に自由に現れる$y$以外のどの変項とも違う.
		
		\item 六つ目の書き換えでは,$s$は$\varphi$に自由に現れる$y$以外のどの変項とも,
			$\psi$に自由に現れる$z$以外のどの変項とも違う.
			$u$は$\varphi$に自由に現れる$y$以外のどの変項とも違う.
			
		\item $u$は$\varphi$の中で$y$への代入について自由であり,
			$u,v,s$は$\psi$の中で$z$への代入について自由である.
			
		\item 上の式の書き換えにおいては変項$u,v,s$を追加したが,
			以上の条件を満たす限りどの変項を選んでも構わない.
			従って\underline{式の書き換えは一つに決まらない}ということになるが,
			違う変項を選んでも式の意味は変わらない.
			
		\item 付記「束縛変項の名前替え」について.
			$a$を$\psi$の中の自由な$z$に代入した後で$a$が束縛される場合,
			束縛変項の名前替えをしなくてはならない.たとえば
			\begin{align}
				a \in \Set{z}{\forall a\, (\, z \in a\, )}
			\end{align}
			という式では左辺の$a$は自由であるのに,書き換えの規則をそのまま適用すると
			\begin{align}
				\forall a\, (\, a \in a\, )
			\end{align}
			となり束縛されてしまう.代入後の$a$が束縛されないためには
			\begin{align}
				a \in \Set{z}{\forall b\, (\, z \in b\, )}
			\end{align}
			のように束縛変項$a$を別の変項$b$に替えて
			\begin{align}
				\forall b\, (\, a \in b\, )
			\end{align}
			とすればよい.ただし新しく選ぶ束縛変項は,
			当該のスコープに元々自由に現れていた$a$以外の項を束縛してはならない.
	\end{itemize}
	
	\begin{screen}
		\begin{metadfn}[式の書き換え]
			$\varphi$を$\lang{\varepsilon}$の式ではない$\mathcal{L}$の式とするとき,
			$\varphi$の{\bf 書き換え}を,$\varphi$の原子式の部分を全て
			上表に従って直した式のこととする.
		\end{metadfn}
	\end{screen}
	
	\begin{screen}
		\begin{metathm}[書き換えは$\lang{\varepsilon}$の式]
		\label{metathm:rewritten_formulas_are_of_L_epsilon}
			$\varphi$を$\lang{\varepsilon}$の式ではない$\mathcal{L}$の式とし,
			$\widehat{\varphi}$を$\varphi$の書き換えとするとき,
			$\widehat{\varphi}$は$\lang{\varepsilon}$の式である.
		\end{metathm}
	\end{screen}
	
	\begin{metaprf}\mbox{}
		\begin{description}
			\item[step1] $\varphi$が原子式なら,表の通り$\widehat{\varphi}$は
				$\lang{\varepsilon}$の式である.
			
			\item[step2] $\varphi$が原子式でないとき,
				\begin{itembox}[l]{IH (帰納法の仮定)}
					$\varphi$の任意の真部分式は,それが$\lang{\varepsilon}$の式でない場合
					その書き換えは$\lang{\varepsilon}$の式である.
				\end{itembox}
				と仮定する.
				\begin{description}
					\item[case1] $\varphi$が
						\begin{align}
							\negation \psi
						\end{align}
						なる式のとき,$\varphi$の$\in,=$のスコープはいずれも
						$\psi$の部分原子式であり,逆に$\psi$の$\in,=$のどのスコープも
						$\varphi$の部分原子式であるから,$\varphi$の原子式の部分を
						全て書き換えるということは$\psi$の原子式の部分を全て書き換える
						ということになる.$\widehat{\varphi}$は
						\begin{align}
							\negation \widehat{\psi}
						\end{align}
						なる形の式であるが,$\widehat{\psi}$は$\psi$の書き換えであり,
						(IH)より$\lang{\varepsilon}$の式である.ゆえに
						$\widehat{\varphi}$も$\lang{\varepsilon}$の式である.
						
					\item[case2] $\varphi$が
						\begin{align}
							\vee \psi \chi
						\end{align}
						なる式のとき,$\varphi$の$\in,=$のスコープはいずれも
						$\psi$か$\chi$の一方の部分原子式であり(始切片の一意性
						のメタ定理\ref{metathm:initial_segment_L}より
						$\varphi$の真部分式が$\psi$と$\chi$の境を跨ぐことはない),
						逆に$\psi,\chi$の$\in,=$のどのスコープも
						$\varphi$の部分原子式であるから,$\varphi$の原子式の部分を
						全て書き換えるということは$\psi$と$\chi$の原子式の部分を全て書き換える
						ということになる.$\widehat{\varphi}$は
						\begin{align}
							\vee \widehat{\psi} \widehat{\chi}
						\end{align}
						なる形の式であるが,$\widehat{\psi},\widehat{\chi}$はそれぞれ$\psi,\chi$の
						書き換えであり(もしくは,$\psi,\chi$の一方は元から
						$\lang{\varepsilon}$の式かもしれない),(IH)よりどちらも
						$\lang{\varepsilon}$の式である.ゆえに
						$\widehat{\varphi}$も$\lang{\varepsilon}$の式である.
						
					\item[case3] $\varphi$が
						\begin{align}
							\exists x \psi
						\end{align}
						なる式のとき,case1 と同様の理由で$\varphi$の原子式の部分を
						全て書き換えるということは$\psi$の原子式の部分を全て書き換える
						ということになる.$\widehat{\varphi}$は
						\begin{align}
							\exists x \widehat{\psi}
						\end{align}
						なる形の式であるが,$\widehat{\psi}$は$\psi$の書き換えであり,
						(IH)より$\lang{\varepsilon}$の式である.ゆえに
						$\widehat{\varphi}$も$\lang{\varepsilon}$の式である.
						\QED
				\end{description}
		\end{description}
	\end{metaprf}
	
	\begin{screen}
		\begin{metathm}[部分式の書き換えとの関係]
		\label{metathm:relation_to_subformula_rewriting}
			$\varphi$を$\lang{\varepsilon}$の式ではない$\mathcal{L}$の式とするとき,
			\begin{description}
				\item[case1] $\varphi$が$\negation \psi$なる式のとき,
					$\varphi$の書き換え$\widehat{\varphi}$は$\negation \widehat{\psi}$
					なる形の式であるが,このとき$\widehat{\psi}$は$\psi$の書き換えである.
					逆に$\check{\psi}$を$\psi$の書き換えとすれば$\negation \check{\psi}$
					は$\varphi$の書き換えである.
					
				\item[case2] $\varphi$が$\vee \psi \chi$なる式のとき,
					$\varphi$の書き換え$\widehat{\varphi}$は$\vee \widehat{\psi} \widehat{\chi}$
					なる形の式であるが,このとき$\widehat{\psi},\widehat{\chi}$はそれぞれ$\psi,\chi$の
					書き換えである.逆に$\check{\psi},\check{\chi}$をそれぞれ$\psi,\chi$の
					書き換えとすれば$\vee \check{\psi} \check{\chi}$は$\varphi$の
					書き換えである.なお,$\psi,\chi$の一方は元から
					$\lang{\varepsilon}$の式かもしれないが,たとえば$\psi$がそうなら
					$\widehat{\psi}$も$\check{\psi}$も$\psi$であるとする.
					
				\item[case3] $\varphi$が$\exists x \psi$なる式のとき,
					$\varphi$の書き換え$\widehat{\varphi}$は$\exists x \widehat{\psi}$
					なる形の式であるが,このとき$\widehat{\psi}$は$\psi$の書き換えである.
					逆に$\check{\psi}$を$\psi$の書き換えとすれば$\exists x \check{\psi}$
					は$\varphi$の書き換えである.
			\end{description}
		\end{metathm}
	\end{screen}
	
	\begin{metaprf}
		証明は前定理の説明と大方被ってしまうがもう一度載せて置く.
		\begin{description}
			\item[case1] $\varphi$の$\in,=$のスコープはいずれも
				$\psi$の部分原子式であり,逆に$\psi$の$\in,=$のどのスコープも
				$\varphi$の部分原子式であるから,$\varphi$の原子式の部分を
				全て書き換えることと$\psi$の原子式の部分を全て書き換えることは同じである.
				従って$\widehat{\psi}$は$\psi$の書き換えであり,$\negation \check{\psi}$は
				$\varphi$の書き換えである.
			
			\item[case2] $\varphi$の$\in,=$のスコープはいずれも
				$\psi$か$\chi$の一方の部分原子式であり(始切片の一意性
				のメタ定理\ref{metathm:initial_segment_L}より
				$\varphi$の真部分式が$\psi$と$\chi$の境を跨ぐことはない),
				逆に$\psi,\chi$の$\in,=$のどのスコープも
				$\varphi$の部分原子式であるから,$\varphi$の原子式の部分を
				全て書き換えることと$\psi$と$\chi$の原子式の部分を全て書き換えることは同じである.
				従って$\widehat{\psi},\widehat{\chi}$はそれぞれ$\psi,\chi$の書き換えであり,
				$\negation \check{\psi} \check{\chi}$は$\varphi$の書き換えである.
				
			\item[case3] case1 と同じ理由によって,$\widehat{\psi}$は$\psi$の書き換えであり,
				$\exists x \check{\psi}$は$\varphi$の書き換えである.
				\QED
		\end{description}
	\end{metaprf}
	
	\begin{screen}
		\begin{metathm}[書き換え後も自由な変項は増減しない]
		\label{metathm:variables_unchanged_after_rewriting}
			$\varphi$を$\lang{\varepsilon}$の式ではない$\mathcal{L}$の式とし,
			この書き換えを$\widehat{\varphi}$とする.このとき
			$\varphi$に自由に現れる変項は$\widehat{\varphi}$にも自由に現れ,
			逆に$\widehat{\varphi}$に自由に現れる変項は$\varphi$にも自由に現れる.
			特に,$\varphi$が文ならば$\widehat{\varphi}$も文である.
		\end{metathm}
	\end{screen}
	
	\begin{metaprf}\mbox{}
		\begin{description}
			\item[step1] $\varphi$が原子式であるときは上の書き換え表より一目瞭然である.
			
			\item[step2]
				$\varphi$が一般の式であるとき
				\begin{itembox}[l]{IH (帰納法の仮定)}
					$\varphi$の任意の真部分式$\psi$に対し,その書き換えを
					$\widehat{\psi}$とすれば($\psi$が$\lang{\varepsilon}$の式ならば
					$\widehat{\psi}$は$\psi$とする),
					$\psi$に自由に現れる変項は$\widehat{\psi}$にも自由に現れ,
					逆に$\widehat{\psi}$に自由に現れる変項は$\psi$にも自由に現れる.
				\end{itembox}
				と仮定する.すると
				\begin{description}
					\item[case1] $\varphi$が
						\begin{align}
							\negation \psi
						\end{align}
						なる式の場合,$\widehat{\varphi}$は
						\begin{align}
							\negation \widehat{\psi}
						\end{align}
						なる形の式であるが,
						メタ定理\ref{metathm:relation_to_subformula_rewriting}より
						$\widehat{\psi}$は$\psi$の書き換えである.
						$\varphi$に自由に現れる変項は$\psi$に自由に現れる変項と一致するが,
						(IH)よりそれは$\widehat{\psi}$に自由に現れる変項と一致するので,
						$\widehat{\varphi}$に自由に現れる変項とも一致する.
						
					\item[case2] $\varphi$が
						\begin{align}
							\vee \psi \chi
						\end{align}
						なる式の場合,$\widehat{\varphi}$は
						\begin{align}
							\vee \widehat{\psi} \widehat{\chi}
						\end{align}
						なる形の式であるが,
						メタ定理\ref{metathm:relation_to_subformula_rewriting}より
						$\widehat{\psi},\widehat{\chi}$はそれぞれ$\psi,\chi$の書き換えである
						($\psi,\chi$の一方は元から$\lang{\varepsilon}$の式かもしれないが,
						たとえば$\psi$がそうなら$\widehat{\psi}$も$\check{\psi}$も$\psi$
						であるとする).
						$\varphi$に自由に現れる変項は$\psi,\chi$に自由に現れる変項と
						一致するが,(IH)よりそれは$\widehat{\psi},\widehat{\chi}$に自由に現れる
						変項と一致するので,$\widehat{\varphi}$に自由に現れる変項とも一致する.
					
					\item[case3] $\varphi$が
						\begin{align}
							\exists x \psi
						\end{align}
						なる式の場合,$\widehat{\varphi}$は
						\begin{align}
							\exists x \widehat{\psi}
						\end{align}
						なる形の式であるが,
						メタ定理\ref{metathm:relation_to_subformula_rewriting}より
						$\widehat{\psi}$は$\psi$の書き換えである.
						$\varphi$に自由に現れる変項は$\psi$に自由に現れる$x$以外の
						変項と一致するが,(IH)よりそれは$\widehat{\psi}$に自由に現れる$x$以外の
						変項と一致するので,$\widehat{\varphi}$に自由に現れる変項とも一致する.
						\QED
				\end{description}
		\end{description}
	\end{metaprf}
	
	\begin{screen}
		\begin{metathm}[書き換えへの代入は代入した式の書き換え]
		\label{metathm:substitution_to_rewritten_formula}
			$\varphi$を$\lang{\varepsilon}$の式ではない$\mathcal{L}$の式とし,
			$\varphi$には変項$x$が自由に現れているとし,$\tau$を主要$\varepsilon$項とし,
			$\widehat{\varphi}$を$\varphi$の書き換えとする.このとき
			$\widehat{\varphi}(x/\tau)$は$\varphi(x/\tau)$の書き換えである.
		\end{metathm}
	\end{screen}
	
	\begin{metaprf}\mbox{}
		\begin{description}
			\item[step1] $\varphi$が原子式であるとする.
				\begin{description}
					\item[case1] $\varphi$が
						\begin{align}
							x = \Set{z}{\psi}
						\end{align}
						なる式のとき,$\widehat{\varphi}$は
						\begin{align}
							\forall v\, (\, v \in x \lrarrow \psi(z/v)\, )
						\end{align}
						なる式である.
						\begin{itemize}
							\item $x$と$z$が同じならば$\widehat{\varphi}(x/\tau)$は
								\begin{align}
									\forall v\, (\, v \in \tau \lrarrow \psi(z/v)\, )
								\end{align}
								となる.他方で$\varphi(x/\tau)$は
								\begin{align}
									\tau = \Set{z}{\psi}
								\end{align}
								であるから$\widehat{\varphi}(x/\tau)$は
								$\varphi(x/\tau)$の書き換えである.
								
							\item $x$と$z$が違うとき$\widehat{\varphi}(x/\tau)$は,
								$\psi$に$x$が自由に現れていれば
								\begin{align}
									\forall v\, (\, v \in \tau \lrarrow \psi(z/v)(x/\tau)\, )
								\end{align}
								となるが,書き換えの変項条件より$x$は$v$とも違うので
								$\psi(z/v)(x/\tau)$と$\psi(x/\tau)(z/v)$は
								同じである.従って$\widehat{\varphi}(x/\tau)$は
								\begin{align}
									\forall v\, (\, v \in \tau \lrarrow \psi(x/\tau)(z/v)\, )
								\end{align}
								と同じである.他方で$\varphi(x/\tau)$は
								\begin{align}
									\tau = \Set{z}{\psi(x/\tau)}
								\end{align}
								であるから,この場合は$\widehat{\varphi}(x/\tau)$は
								$\varphi(x/\tau)$の書き換えである.
								$\psi$に$x$が自由に現れていない場合,
								$\widehat{\varphi}(x/\tau)$は
								\begin{align}
									\forall v\, (\, v \in \tau \lrarrow \psi(z/v)\, )
								\end{align}
								となるが,$\varphi(x/\tau)$は
								\begin{align}
									\tau = \Set{z}{\psi}
								\end{align}
								であるからこの場合も$\widehat{\varphi}(x/\tau)$は
								$\varphi(x/\tau)$の書き換えである.
						\end{itemize}
						
					\item[case2] $\varphi$が
						\begin{align}
							a = \Set{z}{\psi}
						\end{align}
						なる式のとき($a$と$x$は違う項),$\widehat{\varphi}$は
						\begin{align}
							\forall v\, (\, v \in a \lrarrow \psi(z/v)\, )
						\end{align}
						なる式である.$\varphi$には$x$が自由に現れているので,つまり
						$x$は$\psi$に自由に現れている.従って$\widehat{\varphi}(x/\tau)$は
						\begin{align}
							\forall v\, (\, v \in a \lrarrow \psi(z/v)(x/\tau)\, )
						\end{align}
						となるが,書き換えの変項条件より$x$は$v$とも違うので
						$\psi(z/v)(x/\tau)$と$\psi(x/\tau)(z/v)$は
						同じである.従って$\widehat{\varphi}(x/\tau)$は
						\begin{align}
							\forall v\, (\, v \in a \lrarrow \psi(x/\tau)(z/v)\, )
						\end{align}
						と同じである.他方で$\varphi(x/\tau)$は
						\begin{align}
							a = \Set{z}{\psi(x/\tau)}
						\end{align}
						であるから$\widehat{\varphi}(x/\tau)$は
						$\varphi(x/\tau)$の書き換えである.
					
					\item[case3] $\varphi$が
						\begin{align}
							\Set{y}{\chi} = x
						\end{align}
						なる式のとき,$\widehat{\varphi}$は
						\begin{align}
							\forall u\, (\, \chi(y/u) \lrarrow u \in x\, )
						\end{align}
						なる式である.
						\begin{itemize}
							\item $x$と$y$が同じならば$\widehat{\varphi}(x/\tau)$は
								\begin{align}
									\forall u\, (\, \chi(y/u) \lrarrow u \in \tau\, )
								\end{align}
								となる.他方で$\varphi(x/\tau)$は
								\begin{align}
									\Set{y}{\chi} = \tau
								\end{align}
								であるから$\widehat{\varphi}(x/\tau)$は
								$\varphi(x/\tau)$の書き換えである.
								
							\item $x$と$y$が違うとき$\widehat{\varphi}(x/\tau)$は,
								$\chi$に$x$が自由に現れていれば
								\begin{align}
									\forall u\, (\, \chi(y/u)(x/\tau) \lrarrow u \in \tau\, )
								\end{align}
								となるが,書き換えの変項条件より$x$は$u$とも違うので
								$\chi(y/u)(x/\tau)$と$\chi(x/\tau)(y/u)$は
								同じである.従って$\widehat{\varphi}(x/\tau)$は
								\begin{align}
									\forall u\, (\, \chi(x/\tau)(y/u) \lrarrow u \in \tau\, )
								\end{align}
								と同じである.他方で$\varphi(x/\tau)$は
								\begin{align}
									\Set{y}{\chi(x/\tau)} = \tau
								\end{align}
								であるから,この場合は$\widehat{\varphi}(x/\tau)$は
								$\varphi(x/\tau)$の書き換えである.
								$\chi$に$x$が自由に現れていない場合,
								$\widehat{\varphi}(x/\tau)$は
								\begin{align}
									\forall u\, (\, \chi(y/u) \lrarrow u \in \tau\, )
								\end{align}
								となるが,$\varphi(x/\tau)$は
								\begin{align}
									\Set{y}{\chi} = \tau
								\end{align}
								であるからこの場合も$\widehat{\varphi}(x/\tau)$は
								$\varphi(x/\tau)$の書き換えである.
						\end{itemize}
						
					\item[case4] $\varphi$が
						\begin{align}
							\Set{y}{\chi} = b
						\end{align}
						なる式のとき($b$は$x$と違う項),$\widehat{\varphi}$は
						\begin{align}
							\forall u\, (\, \chi(y/u) \lrarrow u \in b\, )
						\end{align}
						なる式である.$\varphi$には$x$が自由に現れているので,つまり
						$x$は$\chi$に自由に現れている.従って$\widehat{\varphi}(x/\tau)$は
						\begin{align}
							\forall u\, (\, \chi(y/u)(x/\tau) \lrarrow u \in b\, )
						\end{align}
						となるが,書き換えの変項条件より$x$は$u$とも違うので
						$\chi(y/u)(x/\tau)$と$\chi(x/\tau)(y/u)$は
						同じである.従って$\widehat{\varphi}(x/\tau)$は
						\begin{align}
							\forall u\, (\, \chi(x/\tau)(y/u) \lrarrow u \in b\, )
						\end{align}
						と同じである.他方で$\varphi(x/\tau)$は
						\begin{align}
							\Set{y}{\chi(x/\tau)} = b
						\end{align}
						であるから$\widehat{\varphi}(x/\tau)$は
						$\varphi(x/\tau)$の書き換えである.
					
					\item[case5] $\varphi$が
						\begin{align}
							\Set{y}{\chi} = \Set{z}{\psi}
						\end{align}
						なる式のとき,$\widehat{\varphi}$は
						\begin{align}
							\forall u\, (\, \chi(y/u) \lrarrow \psi(z/u)\, )
						\end{align}
						なる式である.
						\begin{itemize}
							\item $x$と$y$が同じならば,$x$は$\Set{y}{\chi}$には自由に
								現れないので,$x$が$\varphi$に自由に現れている以上
								$\psi$に自由に現れることになる.すなわち$x$と$z$は違う項である.
								このとき$\widehat{\varphi}(x/\tau)$は
								\begin{align}
									\forall u\, (\, \chi(y/u) \lrarrow \psi(z/u)(x/\tau)\, )
								\end{align}
								となるが,書き換えの変項条件より$x$は$u$とも違うので
								$\psi(z/u)(x/\tau)$と$\psi(x/\tau)(z/u)$は
								同じである.従って$\widehat{\varphi}(x/\tau)$は
								\begin{align}
									\forall u\, (\, \chi(y/u) \lrarrow \psi(x/\tau)(z/u)\, )
								\end{align}
								と同じである.他方で$\varphi(x/\tau)$は
								\begin{align}
									\Set{y}{\chi} = \Set{z}{\psi(x/\tau)}
								\end{align}
								であるから$\widehat{\varphi}(x/\tau)$は
								$\varphi(x/\tau)$の書き換えである.
								
							\item $x$と$z$が同じならば,$x$は$\Set{z}{\psi}$には自由に
								現れないので,$x$が$\varphi$に自由に現れている以上
								$\chi$に自由に現れることになる.すなわち$x$と$y$は違う項である.
								このとき$\widehat{\varphi}(x/\tau)$は
								\begin{align}
									\forall u\, (\, \chi(y/u)(x/\tau) \lrarrow \psi(z/u)\, )
								\end{align}
								となるが,書き換えの変項条件より$x$は$u$とも違うので
								$\chi(y/u)(x/\tau)$と$\chi(x/\tau)(y/u)$は
								同じである.従って$\widehat{\varphi}(x/\tau)$は
								\begin{align}
									\forall u\, (\, \chi(x/\tau)(y/u) \lrarrow \psi(z/u)\, )
								\end{align}
								と同じである.他方で$\varphi(x/\tau)$は
								\begin{align}
									\Set{y}{\chi(x/\tau)} = \Set{z}{\psi}
								\end{align}
								であるから$\widehat{\varphi}(x/\tau)$は
								$\varphi(x/\tau)$の書き換えである.
							
							\item $x$が$y$とも$z$とも違うならば,$x$は$\chi$か$\psi$の
								少なくとも一方には自由に現れている.
								このとき$\widehat{\varphi}(x/\tau)$は
								\begin{align}
									\forall u\, (\, \chi(y/u)(x/\tau) \lrarrow \psi(z/u)(x/\tau)\, )
								\end{align}
								となるが,書き換えの変項条件より
								$\widehat{\varphi}(x/\tau)$は
								\begin{align}
									\forall u\, (\, \chi(x/\tau)(y/u) \lrarrow \psi(x/\tau)(z/u)\, )
								\end{align}
								と同じである.他方で$\varphi(x/\tau)$は
								\begin{align}
									\Set{y}{\chi(x/\tau)} = \Set{z}{\psi(x/\tau)}
								\end{align}
								であるから$\widehat{\varphi}(x/\tau)$は
								$\varphi(x/\tau)$の書き換えである.
						\end{itemize}
						
					\item[case6] $\varphi$が
						\begin{align}
							x \in \Set{z}{\psi}
						\end{align}
						なる式のとき,必要ならば$\psi$の束縛変項の
						名前替えをしたものを$\widetilde{\psi}$とする.ただし
						名前替えをしなかったら$\widetilde{\psi}$は$\psi$とする.
						$\widehat{\varphi}$は$\widetilde{\psi}(z/x)$なる式であるから
						$\widehat{\varphi}(x/\tau)$は$\tilde{\psi}(z/x)(x/\tau)$
						となる.
						\begin{itemize}
							\item $x$と$z$が同じならば$\psi(z/x)(x/\tau)$は
								$\psi(z/\tau)$となる.他方で$\varphi(x/\tau)$は
								\begin{align}
									\tau \in \Set{z}{\psi}
								\end{align}
								となるから,$\psi(z/\tau)$は$\varphi(x/\tau)$の書き換えである.
								
							\item $x$と$z$が違うとき,$\widetilde{\psi}(z/x)(x/\tau)$は
								$\widetilde{\psi}(x/\tau)(z/\tau)$である.他方で
								$\varphi(x/\tau)$は
								\begin{align}
									\tau \in \Set{z}{\psi(x/\tau)}
								\end{align}
								となるから,$\widetilde{\psi}(x/\tau)(z/\tau)$は
								$\varphi(x/\tau)$の書き換えである.
						\end{itemize}
						
					\item[case7] $\varphi$が
						\begin{align}
							a \in \Set{z}{\psi}
						\end{align}
						なる式のとき($a$は$x$とは違う項),必要ならば$\psi$の束縛変項の
						名前替えをしたものを$\widetilde{\psi}$とする.ただし
						名前替えをしなかったら$\widetilde{\psi}$は$\psi$とする.
						$\widehat{\varphi}$は$\widetilde{\psi}(z/a)$なる式であるから
						$\widehat{\varphi}(x/\tau)$は$\widetilde{\psi}(z/a)(x/\tau)$
						となる.ここで,
						\begin{itemize}
							\item $x$と$z$が同じならば$\widetilde{\psi}(z/a)(x/\tau)$は
								$\widetilde{\psi}(z/a)$となる.
								他方で$\varphi(x/\tau)$は
								\begin{align}
									a \in \Set{z}{\psi}
								\end{align}
								となるから,$\widetilde{\psi}(z/a)$は
								$\varphi(x/\tau)$の書き換えである.
								
							\item $x$と$z$が違うとき,$\tilde{\psi}(z/a)(x/\tau)$は
								$\widetilde{\psi}(x/\tau)(z/a)$である.他方で
								$\varphi(x/\tau)$は
								\begin{align}
									a \in \Set{z}{\psi(x/\tau)}
								\end{align}
								となるから,$\widetilde{\psi}(x/\tau)(z/a)$は
								$\varphi(x/\tau)$の書き換えである.
						\end{itemize}
					
					\item[case8] $\varphi$が
						\begin{align}
							\Set{y}{\chi} \in x
						\end{align}
						なる式のとき,$\widehat{\varphi}$は
						\begin{align}
							\exists s\, (\, \forall u\, (\, \chi(y/u) \lrarrow u \in s\, ) \wedge s \in x\, )
						\end{align}
						なる式である.
						\begin{itemize}
							\item $x$と$y$が同じならば$\widehat{\varphi}(x/\tau)$は
								\begin{align}
									\exists s\, (\, \forall u\, (\, \chi(y/u) \lrarrow u \in s\, ) \wedge s \in \tau\, )
								\end{align}
								となる.他方で$\varphi(x/\tau)$は
								\begin{align}
									\Set{y}{\chi} \in \tau
								\end{align}
								であるから$\widehat{\varphi}(x/\tau)$は
								$\varphi(x/\tau)$の書き換えである.
								
							\item $x$と$y$が違うとき$\widehat{\varphi}(x/\tau)$は,
								$x$が$\chi$に自由に現れているならば
								\begin{align}
									\exists s\, (\, \forall u\, (\, \chi(y/u)(x/\tau) \lrarrow u \in s\, ) \wedge s \in \tau\, )
								\end{align}
								となるが,書き換えの変項条件より$x$は$u$とも違うので
								$\chi(y/u)(x/\tau)$と$\chi(x/\tau)(y/u)$は
								同じである.従って$\widehat{\varphi}(x/\tau)$は
								\begin{align}
									\exists s\, (\, \forall u\, (\, \chi(x/\tau)(y/u) \lrarrow u \in s\, ) \wedge s \in \tau\, )
								\end{align}
								と同じである.他方で$\varphi(x/\tau)$は
								\begin{align}
									\Set{y}{\chi(x/\tau)} \in \tau
								\end{align}
								であるから,この場合は$\widehat{\varphi}(x/\tau)$は
								$\varphi(x/\tau)$の書き換えである.
								$x$が$\chi$に自由に現れていない場合,
								$\widehat{\varphi}(x/\tau)$は
								\begin{align}
									\exists s\, (\, \forall u\, (\, \chi(y/u) \lrarrow u \in s\, ) \wedge s \in \tau\, )
								\end{align}
								となり,$\varphi(x/\tau)$は
								\begin{align}
									\Set{y}{\chi} \in \tau
								\end{align}
								であるからこの場合も$\widehat{\varphi}(x/\tau)$は
								$\varphi(x/\tau)$の書き換えである.
						\end{itemize}
					
					\item[case9] $\varphi$が
						\begin{align}
							\Set{y}{\chi} \in b
						\end{align}
						なる式のとき($b$は$x$と違う項),$\widehat{\varphi}$は
						\begin{align}
							\exists s\, (\, \forall u\, (\, \chi(y/u) \lrarrow u \in s\, ) \wedge s \in b\, )
						\end{align}
						なる式である.$\varphi$には$x$が自由に現れているので,つまり
						$x$は$\chi$に自由に現れている.従って$\widehat{\varphi}(x/\tau)$は
						\begin{align}
							\exists s\, (\, \forall u\, (\, \chi(y/u)(x/\tau) \lrarrow u \in s\, ) \wedge s \in b\, )
						\end{align}
						となるが,書き換えの変項条件より$x$は$u$とも違うので
						$\chi(y/u)(x/\tau)$と$\chi(x/\tau)(y/u)$は
						同じである.従って$\widehat{\varphi}(x/\tau)$は
						\begin{align}
							\exists s\, (\, \forall u\, (\, \chi(x/\tau)(y/u) \lrarrow u \in s\, ) \wedge s \in b\, )
						\end{align}
						と同じである.他方で$\varphi(x/\tau)$は
						\begin{align}
							\Set{y}{\chi(x/\tau)} \in b
						\end{align}
						であるから,$\widehat{\varphi}(x/\tau)$は
						$\varphi(x/\tau)$の書き換えである.
						
					\item[case10] $\varphi$が
						\begin{align}
							\Set{y}{\chi} \in \Set{z}{\psi}
						\end{align}
						なる式のとき,$\widehat{\varphi}$は
						\begin{align}
							\exists s\, (\, \forall u\, (\, \chi(y/u) \lrarrow u \in s\, ) \wedge \psi(z/s)\, )
						\end{align}
						なる式である.
						\begin{itemize}
							\item $x$と$y$が同じならば,$x$は$\Set{y}{\chi}$には自由に
								現れないので,$x$が$\varphi$に自由に現れている以上
								$\psi$に自由に現れることになる.すなわち$x$と$z$は違う項である.
								このとき$\widehat{\varphi}(x/\tau)$は
								\begin{align}
									\exists s\, (\, \forall u\, (\, \chi(y/u) \lrarrow u \in s\, ) \wedge \psi(z/s)(x/\tau)\, )
								\end{align}
								となるが,書き換えの変項条件より$x$は$s$とも違うので
								$\psi(z/s)(x/\tau)$と$\psi(x/\tau)(z/s)$は
								同じである.従って$\widehat{\varphi}(x/\tau)$は
								\begin{align}
									\exists s\, (\, \forall u\, (\, \chi(y/u) \lrarrow u \in s\, ) \wedge \psi(x/\tau)(z/s)\, )
								\end{align}
								と同じである.他方で$\varphi(x/\tau)$は
								\begin{align}
									\Set{y}{\chi} \in \Set{z}{\psi(x/\tau)}
								\end{align}
								であるから,$\widehat{\varphi}(x/\tau)$は
								$\varphi(x/\tau)$の書き換えである.
								
							\item $x$と$z$が同じならば,$x$は$\Set{z}{\psi}$には自由に
								現れないので,$x$が$\varphi$に自由に現れている以上
								$\chi$に自由に現れることになる.すなわち$x$と$y$は違う項である.
								このとき$\widehat{\varphi}(x/\tau)$は
								\begin{align}
									\exists s\, (\, \forall u\, (\, \chi(y/u)(x/\tau) \lrarrow u \in s\, ) \wedge \psi(z/s)\, )
								\end{align}
								となるが,書き換えの変項条件より$x$は$s$とも違うので
								$\chi(y/u)(x/\tau)$と$\chi(x/\tau)(y/u)$は
								同じである.従って$\widehat{\varphi}(x/\tau)$は
								\begin{align}
									\exists s\, (\, \forall u\, (\, \chi(x/\tau)(y/u) \lrarrow u \in s\, ) \wedge \psi(z/s)\, )
								\end{align}
								と同じである.他方で$\varphi(x/\tau)$は
								\begin{align}
									\Set{y}{\chi(x/\tau)} \in \Set{z}{\psi}
								\end{align}
								であるから,$\widehat{\varphi}(x/\tau)$は
								$\varphi(x/\tau)$の書き換えである.
								
							\item $x$が$y$とも$z$とも違うならば,$x$は$\chi$か$\psi$の
								少なくとも一方にはには自由に現れている.
								このとき$\widehat{\varphi}(x/\tau)$は
								\begin{align}
									\exists s\, (\, \forall u\, (\, \chi(y/u)(x/\tau) \lrarrow u \in s\, ) \wedge \psi(z/s)(x/\tau)\, )
								\end{align}
								となるが,書き換えの変項条件より
								$\widehat{\varphi}(x/\tau)$は
								\begin{align}
									\exists s\, (\, \forall u\, (\, \chi(x/\tau)(y/u) \lrarrow u \in s\, ) \wedge \psi(x/\tau)(z/s)\, )
								\end{align}
								と同じである.他方で$\varphi(x/\tau)$は
								\begin{align}
									\Set{y}{\chi(x/\tau)} \in \Set{z}{\psi(x/\tau)}
								\end{align}
								であるから,$\widehat{\varphi}(x/\tau)$は
								$\varphi(x/\tau)$の書き換えである.
						\end{itemize}
				\end{description}
		\end{description}
	\end{metaprf}
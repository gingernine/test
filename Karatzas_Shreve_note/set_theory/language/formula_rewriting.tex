\subsection{式の書き換え}
\label{subsec:formula_rewriting}
	$\varepsilon$項を取り入れたのは{\bf 存在文}\index{そんざいぶん@存在文}
	{\bf (existential sentence)}に対して{\bf 証人}\index{しょうにん@証人}
	{\bf (witnessing term)}を与えるためであり,そのために
	\begin{align}
		\exists x \varphi(x) \rarrow \varphi(\varepsilon x \varphi(x))
	\end{align}
	なる式を公理とする.ただし$\varepsilon$項を作れるのは$\lang{\varepsilon}$
	の式のみであるから,$\varphi$が内包項を含んだ式であると$\varepsilon x \varphi(x)$を
	使うことが出来ない.とはいえ内包項を含んだ存在文も往々にして登場するので,それらに対しても
	証人を用意できると便利である.そこで$\varphi$を内包項を含んだ$\mathcal{L}$の式とするとき,
	$\varphi$を``同値''な$\lang{\varepsilon}$の式$\widehat{\varphi}$に書き換えて
	\begin{align}
		\exists x \varphi(x) \rarrow \varphi(\varepsilon x \widehat{\varphi}(x))
	\end{align}
	を公理とする(量化公理\ref{logicalaxm:rules_of_quantifiers}).
	注意点は\underline{同値な書き換えはいくらでも作れる}ということであり,
	$\check{\varphi}$も$\varphi$の書き換えならば
	\begin{align}
		\exists x \varphi(x) \rarrow \varphi(\varepsilon x \check{\varphi}(x))
	\end{align}
	も公理とする.書き換える必要があるのは内包項を含んでいる式のみであり,
	またそのような式の{\bf 原子式の部分だけを書き換えれば十分}である.
	書き換えが``同値''というのは後述の\ref{sec:equivalence_of_formula_rewriting}節
	で述べてあるような意味であり,直感的に当然な範囲でしかないのだが,
	書き換え時に課す変項の条件が多いから読むだけでも疲労困憊する.
	これらは全て\underline{自由に現れる変項を書き換え後も自由ならしめるため}のものであるが,
	とりあえずは「式の書き換えが可能である」ということと「式の書き換えの形」だけを認識されれば,
	その他の些末事項を当面無視しても大して問題は無い.
	原子式の書き換えは次の要領で行う:
	
	\begin{table}[H]
		\begin{center}
		\caption{式の書き換え表}
		\begin{tabular}{c|c|c|c}
			 & 元の式 & 書き換え後 & 付記 \\ \hline \hline
			(1) & $a = \Set{z}{\psi}$ & $\forall v\, (\, v \in a \lrarrow \psi(z/v)\, )$ & \\ \hline
			(2) & $\Set{y}{\varphi} = b$ & $\forall u\, (\, \varphi(y/u) \lrarrow u \in b\, )$ & \\ \hline
			(3) & $\Set{y}{\varphi} = \Set{z}{\psi}$ & $\forall u\, (\, \varphi(y/u) \lrarrow \psi(z/u)\, )$ & \\ \hline
			(4) & $a \in \Set{z}{\psi}$ & $\psi(z/a)$ & 必要なら変項の名前替え \\ \hline
			(5) & $\Set{y}{\varphi} \in b$ & $\exists s\, (\, \forall u\, (\, \varphi(y/u) \lrarrow u \in s\, ) \wedge s \in b\, )$ & \\ \hline
			(6) & $\Set{y}{\varphi} \in \Set{z}{\psi}$ & $\exists s\, (\, \forall u\, (\, \varphi(y/u) \lrarrow u \in s\, ) \wedge \psi(z/s)\, )$ & \\ \hline
		\end{tabular}
		\label{tab:formula_rewriting}
		\end{center}
	\end{table}
	
	上の表の書き換えでは変項$u,v,s$を追加したが,以下の条件を満たす限りどの変項を選んでも構わない.
	従って\underline{式の書き換えは一つに決まらない}ということになるが,違う変項を選んでも式の意味は変わらない
	(定理\ref{logicalthm:equivalence_by_replacing_bound_variables}).
	上の表にある項が満たしている条件は次である:
			
	\begin{itemize}
		\item $a,b$は$\lang{\varepsilon}$の項である
			(\ref{sec:restriction_of_formulas}節より
			$a,b$は変項か主要$\varepsilon$項).
		
		\item $\Set{y}{\varphi}$と$\Set{z}{\psi}$を正則内包項である.
		
		\item (1)の$v$は$\psi$に自由に現れる$z$以外のどの変項とも$a$とも違い,
			また$\psi$の中で$z$への代入について自由である.
		
		\item (2)の$u$は$\varphi$に自由に現れる$y$以外のどの変項とも$b$とも違い,
			また$\varphi$の中で$y$への代入について自由である.
			
		\item (3)の$u$は$\varphi$に自由に現れる$y$以外のどの変項とも,
			$\psi$に自由に現れる$z$以外のどの変項とも違い,また
			$\varphi$の中で$y$への代入について自由であり,
			$\psi$の中で$z$への代入について自由である.
		
		\item (5)の$s$は$\varphi$に自由に現れる$y$以外のどの変項とも$b,u$とも違う.
			$u$は$\varphi$に自由に現れる$y$以外のどの変項とも違い,
			また$\varphi$の中で$y$への代入について自由である.
		
		\item (6)の$s$は,$\varphi$に自由に現れる$y$以外のどの変項とも,
			$\psi$に自由に現れる$z$以外のどの変項とも,$u$とも違い,
			また$\psi$の中で$z$への代入について自由である.
			$u$は$\varphi$に自由に現れる$y$以外のどの変項とも違い,
			また$\varphi$の中で$y$への代入について自由である.
			
		\item (4)の「変項の名前替え」について.$\psi$の中の$z$の自由な出現が,$\psi$の
			$\forall a \chi$或いは$\exists a \chi$なる形の部分式の中にある場合,
			その部分式を$\forall b \chi(a/b)$或いは$\exists b \chi(a/b)$で置き換える.
			この$b$としては,\underline{$\chi$に自由に現れず,かつ$\chi$の中で$a$への代入について
			自由である変項}を取るが,これは$\forall a \chi$或いは$\exists a \chi$に
			自由に現れる変項を束縛しないためである.たとえば
			\begin{align}
				a \in \Set{z}{\exists a\, (\, z \in a\, )}
			\end{align}
			という式では左辺の$a$は自由であるのに,書き換えの規則をそのまま適用すると
			\begin{align}
				\exists a\, (\, a \in a\, )
			\end{align}
			となり束縛されてしまう.こうならないためには
			\begin{align}
				a \in \Set{z}{\exists b\, (\, z \in b\, )}
			\end{align}
			のように$a$を別の変項$b$に替えて
			\begin{align}
				\exists b\, (\, a \in b\, )
			\end{align}
			とすればよい.ただし,
			\begin{align}
				\forall a\, (\, ...\exists a\, (\, ...\, z\, ...\, )...\, )
			\end{align}
			のように$\forall a,\exists a$の多層構造の内部に$z$の自由な出現がある場合,
			$z$のその自由な出現がある$\forall a,\exists a$から始まる$\psi$の部分式
			(定義\ref{metadfn:L_subformula})のうちまずは最も狭いものを取り,
			その直部分式(定義\ref{metadfn:L_immediate_subformula})に
			自由に現れる$a$に$b$を代入して$b$の量化式とする.次は二番目に狭い部分式を取り,
			その直部分式に自由に現れる$a$に$b$を代入して$b$の量化式とする.
			同じ操作を$\forall a,\exists a$のスコープから抜け出るまで繰り返す.
			たとえば$\psi$が
			\begin{align}
				\forall a (\, \eta \vee \forall a (\, \xi \wedge \exists a \chi\, )\, )
			\end{align}
			なる式で(他に$a$の量化はないとする),$z$の自由な出現が$\chi$の中にある状況では,まず
			\begin{align}
				\forall a (\, \eta \vee \forall a (\, \xi \wedge \exists b \chi(a/b)\, )\, )
			\end{align}
			とし,次に
			\begin{align}
				\forall a (\, \eta \vee \forall b (\, \xi(a/b) \wedge \exists b \chi(a/b)\, )\, )
			\end{align}
			とし,最後に
			\begin{align}
				\forall b (\, \eta(a/b) \vee \forall b (\, \xi(a/b) \wedge \exists b \chi(a/b)\, )\, )
			\end{align}
			とするということである.
	\end{itemize}
	
	ここで「部分式を取り替えても式である」ということを示しておく.
	
	\begin{screen}
		\begin{metathm}[部分式を取り替えても式]
		\label{metathm:replace_subformula_with_some_formula}
			$\varphi$を$\mathcal{L}$の式とし,$\psi$を$\varphi$の部分式
			(定義\ref{metadfn:L_subformula})とし,$\chi$を$\mathcal{L}$の式とする.
			このとき,$\varphi$のその$\psi$の部分(一か所)を$\chi$に置き換えて得られる記号列
			$\widetilde{\varphi}$は$\mathcal{L}$の式である.
			また$\varphi$と$\chi$が$\lang{\in}$の式ならば$\widetilde{\varphi}$も
			$\lang{\in}$の式であるし,$\varphi$と$\chi$が$\lang{\varepsilon}$の式ならば
			$\widetilde{\varphi}$も$\lang{\varepsilon}$の式である.
		\end{metathm}
	\end{screen}
	
	\begin{metaprf} $\varphi$と$\chi$が$\lang{\in}$或いは$\lang{\varepsilon}$の式の場合は,
		下の証明の「$\mathcal{L}$」の部分を「$\lang{\in}$」或いは「$\lang{\varepsilon}$」と
		読み替えればよい.
		\begin{description}
			\item[step1] $\varphi$が原子式である場合,$\psi$とは$\varphi$自身のことである.
				従って$\widetilde{\varphi}$は$\chi$となり,これは$\mathcal{L}$の式である.
				
			\item[step2] $\varphi$が原子式でないとき,
				\begin{itembox}[l]{IH (帰納法の仮定)}
					$\varphi$の任意の真部分式に対して,その部分式を一か所$\mathcal{L}$の
					他の式に置き換えて得られる記号列は$\mathcal{L}$の式である
				\end{itembox}
				と仮定する.このとき
				\begin{description}
					\item[case1] $\varphi$が
						\begin{align}
							\negation \xi
						\end{align}
						なる式のとき,$\psi$は$\varphi$自身であるか,$\xi$の部分式である.
						前者の場合は$\widetilde{\varphi}$は$\chi$となる.
						後者の場合は,$\xi$のその$\psi$の部分を$\chi$に置き換えた記号列
						$\widetilde{\xi}$は(IH)より$\mathcal{L}$の式であって,
						$\widetilde{\varphi}$は
						\begin{align}
							\negation \widetilde{\xi}
						\end{align}
						なる記号列となる.ゆえにいずれの場合も$\widetilde{\varphi}$は
						$\mathcal{L}$の式である.
					
					\item[case2] $\varphi$が
						\begin{align}
							\vee \xi \zeta
						\end{align}
						なる式のとき,$\psi$は$\varphi$自身であるか,$\xi$或いは
						$\zeta$の部分式である.
						前者の場合は$\widetilde{\varphi}$は$\chi$となる.
						後者の場合は,その$\psi$が$\xi$に現れているとして,
						$\xi$のその$\psi$の部分を$\chi$に置き換えた記号列
						$\widetilde{\xi}$は(IH)より$\mathcal{L}$の式であって,
						$\widetilde{\varphi}$は
						\begin{align}
							\vee \widetilde{\xi} \zeta
						\end{align}
						なる記号列となる.その$\psi$が$\zeta$に現れていれば
						$\widetilde{\varphi}$は
						\begin{align}
							\vee \xi \widetilde{\zeta}
						\end{align}
						なる記号列となる.ゆえにいずれの場合も$\widetilde{\varphi}$は
						$\mathcal{L}$の式である.
						
					\item[case3] $\varphi$が
						\begin{align}
							\exists x \xi
						\end{align}
						なる式のとき,$\psi$は$\varphi$自身であるか,$\xi$の部分式である.
						前者の場合は$\widetilde{\varphi}$は$\chi$となる.
						後者の場合は,$\xi$のその$\psi$の部分を$\chi$に置き換えた記号列
						$\widetilde{\xi}$は(IH)より$\mathcal{L}$の式であって,
						$\widetilde{\varphi}$は
						\begin{align}
							\exists x \widetilde{\xi}
						\end{align}
						なる記号列となる.ゆえにいずれの場合も$\widetilde{\varphi}$は
						$\mathcal{L}$の式である.
						\QED
				\end{description}
		\end{description}
	\end{metaprf}
	
	\begin{screen}
		\begin{metadfn}[式の書き換え]
			$\varphi$を$\lang{\varepsilon}$の式ではない$\mathcal{L}$の式とするとき,
			$\varphi$の{\bf 書き換え}\index{かきかえ@書き換え}とは次の手順で得られる式のこととする:
			\begin{enumerate}
				\item $\varphi$の部分式のうち原子式であるものを全て
					表\ref{tab:formula_rewriting}に従って直した式は
					$\varphi$の書き換えである.
					
				\item $\widehat{\varphi}$を$\varphi$の書き換えとし,$\widehat{\varphi}$に
					$\forall x \xi$ (resp. $\exists x \xi$)の形の部分式\footnotemark
					が現れているとし,$y$を$\xi$に自由に現れない変項で
					$\xi$の中で$x$への代入について自由であるものとするとき,
					$\widehat{\varphi}$のその$\forall x \xi$ (resp. $\exists x \xi$)
					の部分を(一か所) $\forall y \xi(x/y)$ (resp. $\exists y \xi(x/y)$)
					に置き換えた式は$\varphi$の書き換えである.
			\end{enumerate}
		\end{metadfn}
	\end{screen}
	
	\footnotetext{
		第\ref{sec:restriction_of_formulas}節の約束により,$\xi$には$x$が自由に現れている.
	}
	
	\begin{screen}
		\begin{metaaxm}[式の書き換えに関する構造的帰納法の原理]
			$\lang{\varepsilon}$の式ではない$\mathcal{L}$の式の書き換えに対する言明Xに対し,
			\begin{itemize}
				\item $\varphi$を$\lang{\varepsilon}$の式ではない$\mathcal{L}$の式
					とするとき,$\varphi$の部分式で原子式であるものを全て
					表\ref{tab:formula_rewriting}の通り
					に直した式に対してXが言える.
					
				\item $\varphi$を$\lang{\varepsilon}$の式ではない$\mathcal{L}$の式とし,
					$\widehat{\varphi}$を$\varphi$の書き換えとし,$\widehat{\varphi}$
					に対してXが言えると仮定する.このとき,$\widehat{\varphi}$に
					$\forall x \xi$ (resp. $\exists x \xi$)の形の部分式が現れているとし,
					$y$を$\xi$に自由に現れない変項で
					$\xi$の中で$x$への代入について自由であるものとすれば,
					$\widehat{\varphi}$のその$\forall x \xi$ (resp. $\exists x \xi$)
					の部分を(一か所)$\forall y \xi(x/y)$ (resp. $\exists y \xi(x/y)$)
					に置き換えた式に対してXが言える.
			\end{itemize}
			ならば,$\lang{\varepsilon}$の式ではない$\mathcal{L}$の任意の式に対して
			Xが言える.
		\end{metaaxm}
	\end{screen}
	
	\begin{screen}
		\begin{metathm}[書き換えは$\lang{\varepsilon}$の式]
		\label{metathm:rewritten_formulas_are_of_L_epsilon}
			$\varphi$を$\lang{\varepsilon}$の式ではない$\mathcal{L}$の式とするとき,
			$\varphi$の任意の書き換えは$\lang{\varepsilon}$の式である.
		\end{metathm}
	\end{screen}
	
	\begin{metaprf}\mbox{}
		\begin{description}
			\item[step1] $\varphi$を$\lang{\varepsilon}$の式ではない$\mathcal{L}$の式
				とするとき,$\varphi$の部分式で原子式であるものを全て
				表\ref{tab:formula_rewriting}の通り
				に直した式が$\lang{\varepsilon}$の式であることを示す.
				\begin{description}
					\item[case1] $\varphi$が原子式であるとき,
						$\varphi$を表\ref{tab:formula_rewriting}の通りに書き換えた式は,
						表の通り$\lang{\varepsilon}$の式である.
						ちなみに表\ref{tab:formula_rewriting}の(4)の場合では
						$\psi$の部分式を取り替えることもあるが,
						メタ定理\ref{metathm:replace_subformula_with_some_formula}より
						取り替えた後の式$\widetilde{\psi}$も$\lang{\varepsilon}$の式
						であり,$\widetilde{\psi}(z/a)$は$\lang{\varepsilon}$の式である
						(メタ定理\ref{metathm:substitutions_into_L_epsilon_formulas}).
						
					\item[case2] $\varphi$が原子式でないとき,
						\begin{itembox}[l]{IH (帰納法の仮定)}
							$\varphi$の任意の真部分式に対して,
							その部分式で原子式であるものを全て
							表\ref{tab:formula_rewriting}の通りに直した式は
							$\lang{\varepsilon}$の式である
						\end{itembox}
						と仮定する.
						\begin{description}
							\item[case2-1] $\varphi$が
								\begin{align}
									\negation \psi
								\end{align}
								なる式であるとき,$\varphi$の部分式で原子式であるものは
								$\psi$の部分式で原子式であり,逆に$\psi$の部分式で
								原子式であるものは$\varphi$の部分式で原子式であるから,
								$\varphi$の部分式で原子式であるものを全て
								表\ref{tab:formula_rewriting}の通りに直した式は
								次のように表される:
								\begin{align}
									\negation \widehat{\psi}
								\end{align}
								ここで$\widehat{\psi}$とは,$\psi$の部分式で
								原子式であるものを全て
								表\ref{tab:formula_rewriting}の通りに直した式である.
								(IH)より$\widehat{\psi}$は$\lang{\varepsilon}$の式
								なので,$\widehat{\varphi}$も$\lang{\varepsilon}$の式
								である.
								
							\item[case2-2] $\varphi$が
								\begin{align}
									\vee \psi \xi
								\end{align}
								なる式であるとき,$\varphi$の部分式で原子式であるものは
								$\psi$か$\xi$の一方の部分式で原子式であり
								(始切片の一意性のメタ定理
								\ref{metathm:initial_segment_L}より
								$\varphi$の真部分式が$\psi$と$\xi$の境を跨ぐことはない),
								逆に$\psi,\xi$の部分式で原子式であるものは$\varphi$の
								部分式で原子式であるから,
								$\varphi$の部分式で原子式であるものを全て
								表\ref{tab:formula_rewriting}の通りに直した式は
								次のように表される:
								\begin{align}
									\vee \widehat{\psi} \widehat{\xi}
								\end{align}
								ここで$\widehat{\psi},\widehat{\xi}$とは,
								それぞれ$\psi,\xi$の部分式で原子式であるものを全て
								表\ref{tab:formula_rewriting}の通りに直した式である.
								(IH)より$\widehat{\psi},\widehat{\xi}$は
								$\lang{\varepsilon}$の式なので,
								$\widehat{\varphi}$も$\lang{\varepsilon}$の式
								である.
								
							\item[case2-3] $\varphi$が
								\begin{align}
									\exists x \psi
								\end{align}
								なる式であるとき,$\varphi$の部分式で原子式であるものは
								$\psi$の部分式で原子式であり,逆に$\psi$の部分式で
								原子式であるものは$\varphi$の部分式で原子式であるから,
								$\varphi$の部分式で原子式であるものを全て
								表\ref{tab:formula_rewriting}の通りに直した式は
								次のように表される:
								\begin{align}
									\exists x \widehat{\psi}
								\end{align}
								ここで$\widehat{\psi}$とは,$\psi$の部分式で
								原子式であるものを全て
								表\ref{tab:formula_rewriting}の通りに直した式である.
								(IH)より$\widehat{\psi}$は$\lang{\varepsilon}$の式
								なので,$\widehat{\varphi}$も$\lang{\varepsilon}$の式
								である.
						\end{description}
				\end{description}
				
			\item[step2] $\widehat{\varphi}$を$\varphi$の書き換えとするとき,
				\begin{itembox}[l]{IH (帰納法の仮定)}
					$\widehat{\varphi}$は$\lang{\varepsilon}$の式である
				\end{itembox}
				と仮定する.このとき,$\widehat{\varphi}$に$\forall x \xi$ 
				(resp. $\exists x \xi$)の形の部分式が現れているとし,
				$y$を$\xi$に自由に現れない変項で$\xi$の中で$x$への代入について
				自由であるものとすれば,
				メタ定理\ref{metathm:substitutions_into_L_epsilon_formulas}より
				$\forall y \xi(x/y)$ (resp. $\exists y \xi(x/y)$)は
				$\lang{\varepsilon}$の式である.従って,
				メタ定理\ref{metathm:replace_subformula_with_some_formula}より
				$\widehat{\varphi}$のその$\forall x \xi$ (resp. $\exists x \xi$)の
				部分を(一か所)
				$\forall y \xi(x/y)$ (resp. $\exists y \xi(x/y)$)に置き換えた式は
				$\lang{\varepsilon}$の式である.
				\QED
		\end{description}
	\end{metaprf}
	
	\begin{screen}
		\begin{metathm}[部分式の書き換えとの関係1]
		\label{metathm:relation_to_subformula_rewriting}
		\label{metathm:relation_to_subformula_rewriting_1}
			$\varphi$を$\lang{\varepsilon}$の式ではない$\mathcal{L}$の式とし,
			\begin{align}
				\negation \psi
			\end{align}
			なる形であるとする.このとき
			\begin{description}
				\item[(1)] $\varphi$の書き換え$\widehat{\varphi}$は
					$\negation \widehat{\psi}$なる形の式であり,
					この$\widehat{\psi}$は$\psi$の書き換えである.
					
				\item[(2)] $\check{\psi}$を$\psi$の書き換えとすれば
					$\negation \check{\psi}$は$\varphi$の書き換えである.
			\end{description}
		\end{metathm}
	\end{screen}
	
	\begin{metaprf}
		[注] $\varphi$の部分式で原子式であるものは
		$\psi$の部分式で原子式であり,逆に部分式で原子式であるものは
		$\varphi$の部分式で原子式であるから,$\varphi$の原子式の部分を
		全て書き換えることと$\psi$の原子式の部分を全て書き換えることは同じである.
		\begin{description}
			\item[(1)] 
				\begin{description}
					\item[step1]
						$\varphi$の部分式で原子式であるものを全て
						表\ref{tab:formula_rewriting}の通りに書き換えた式は,
						[注]より次の形で表される:
						\begin{align}
							\negation \hat{\psi},
						\end{align}
						ここで$\hat{\psi}$とは,$\psi$の部分式で
						原子式であるものを全て
						表\ref{tab:formula_rewriting}の通りに直した式である.
						ゆえに$\hat{\psi}$は$\psi$の書き換えである.
						
					\item[step2]
						$\widehat{\varphi}$を$\varphi$の書き換えとし,
						\begin{itembox}[l]{IH (帰納法の仮定)}
							$\widehat{\varphi}$は$\negation \dot{\psi}$
							なる形で書けて,この$\dot{\psi}$は$\psi$の
							書き換えである
						\end{itembox}
						と仮定する.このとき,$\widehat{\varphi}$に$\forall x \xi$ 
						(resp. $\exists x \xi$)の形の部分式が現れているとし,
						$y$を$\xi$に自由に現れない変項で$\xi$の中で$x$への代入に
						ついて自由であるものとすれば,$\widehat{\varphi}$のその
						$\forall x \xi$ (resp. $\exists x \xi$)の部分を
						(一か所) $\forall y \xi(x/y)$
						(resp. $\exists y \xi(x/y)$)に置き換えた式は,
						(IH)より$\dot{\psi}$のその$\forall x \xi$
						(resp. $\exists x \xi$)の部分を
						(一か所) $\forall y \xi(x/y)$
						(resp. $\exists y \xi(x/y)$)に置き換えた式
						$\ddot{\psi}$によって
						\begin{align}
							\negation \ddot{\psi}
						\end{align}
						と表される.(IH)より$\dot{\psi}$は$\psi$の書き換えである
						から,$\ddot{\psi}$は$\psi$の書き換えである.
				\end{description}
				
			\item[(2)]		
				\begin{description}
					\item[step1]
						$\psi$の部分式で原子式であるものを全て
						表\ref{tab:formula_rewriting}の通りに書き換えた
						式を$\check{\psi}$とすれば,[注]より
						\begin{align}
							\negation \check{\psi}
						\end{align}
						は$\varphi$の部分式で原子式であるものを全て
						表\ref{tab:formula_rewriting}の通りに書き換えた式である.
						従ってこれは$\varphi$の書き換えである.
								
					\item[step2]
						$\widetilde{\psi}$を$\psi$の書き換えとし,
						\begin{itembox}[l]{IH (帰納法の仮定)}
							$\negation \widetilde{\psi}$は$\varphi$の
							書き換えである
						\end{itembox}
						と仮定する.このとき,$\widetilde{\psi}$に$\forall x \xi$ 
						(resp. $\exists x \xi$)の形の部分式が現れているとし,
						$y$を$\xi$に自由に現れない変項で$\xi$の中で$x$への代入に
						ついて自由であるものとし,$\widetilde{\psi}$のその
						$\forall x \xi$ (resp. $\exists x \xi$)の部分を
						(一か所) $\forall y \xi(x/y)$
						(resp. $\exists y \xi(x/y)$)に置き換えた式を
						$\overline{\psi}$とすれば,
						\begin{align}
							\negation \overline{\psi}
						\end{align}
						は$\negation \widetilde{\psi}$のその
						$\forall x \xi$ (resp. $\exists x \xi$)の部分を
						(一か所) $\forall y \xi(x/y)$
						(resp. $\exists y \xi(x/y)$)に置き換えた式と同じである.
						(IH)より$\negation \widetilde{\psi}$は$\varphi$の
						書き換えであるから,$\negation \overline{\psi}$もまた$\varphi$の
						書き換えである.
						\QED
				\end{description}
		\end{description}
	\end{metaprf}
	
	\begin{screen}
		\begin{metathm}[部分式の書き換えとの関係2]
		\label{metathm:relation_to_subformula_rewriting_2}
			$\varphi$を$\lang{\varepsilon}$の式ではない$\mathcal{L}$の式とし,
			\begin{align}
				\vee \psi \xi
			\end{align}
			なる形であるとする.
			\begin{description}
				\item[(1)] $\varphi$の書き換え$\widehat{\varphi}$は
					$\vee \widehat{\psi} \widehat{\xi}$なる形の式であり,
					この$\widehat{\psi},\widehat{\xi}$はそれぞれ$\psi,\xi$の
					書き換えである.
					
				\item[(2)] $\check{\psi},\check{\xi}$をそれぞれ$\psi,\xi$の
					書き換えとすれば$\vee \check{\psi} \check{\xi}$は$\varphi$の
					書き換えである.なお,$\psi,\xi$の一方は元から
					$\lang{\varepsilon}$の式かもしれないが,たとえば$\psi$がそうなら
					$\widehat{\psi}$も$\check{\psi}$も$\psi$であるとする.
			\end{description}
		\end{metathm}
	\end{screen}
	
	\begin{metaprf}
		[注] $\varphi$の部分式で原子式であるものは$\psi$か$\xi$の一方の部分式で原子式であり,
		(始切片の一意性のメタ定理\ref{metathm:initial_segment_L}より
		$\varphi$の真部分式が$\psi$と$\xi$の境を跨ぐことはない),
		逆に$\psi,\xi$の部分式で原子式であるものは$\varphi$の
		部分式で原子式であるから,$\varphi$の原子式の部分を
		全て書き換えることと$\psi,\xi$の原子式の部分を全て書き換えることは同じである.
		\begin{description}
			\item[(1)] 
				\begin{description}
					\item[step1]
						$\varphi$の部分式で原子式であるものを全て
						表\ref{tab:formula_rewriting}の通りに書き換えた式は,
						[注]より次の形で表される:
						\begin{align}
							\vee \hat{\psi} \hat{\xi},
						\end{align}
						ここで$\hat{\psi},\hat{\xi}$とは,それぞれ$\psi,\xi$の部分式で
						原子式であるものを全て表\ref{tab:formula_rewriting}の通りに
						直した式である.ゆえに$\hat{\psi},\hat{\xi}$は
						それぞれ$\psi,\xi$の書き換えである.
						
					\item[step2]
						$\widehat{\varphi}$を$\varphi$の書き換えとし,
						\begin{itembox}[l]{IH (帰納法の仮定)}
							$\widehat{\varphi}$は$\negation \dot{\psi} \dot{\xi}$
							なる形で書けて,この$\dot{\psi},\dot{\xi}$はそれぞれ
							$\psi,\xi$の書き換えである
						\end{itembox}
						と仮定する.このとき,$\widehat{\varphi}$に$\forall x \xi$ 
						(resp. $\exists x \xi$)の形の部分式が現れているとし,
						$y$を$\xi$に自由に現れない変項で$\xi$の中で$x$への代入に
						ついて自由であるものとすれば,$\widehat{\varphi}$のその
						$\forall x \xi$ (resp. $\exists x \xi$)の部分を
						(一か所) $\forall y \xi(x/y)$
						(resp. $\exists y \xi(x/y)$)に置き換えた式は,
						(IH)より$\dot{\psi}$或いは$\dot{\xi}$のその$\forall x \xi$
						(resp. $\exists x \xi$)の部分を
						(一か所) $\forall y \xi(x/y)$
						(resp. $\exists y \xi(x/y)$)に置き換えた式
						$\ddot{\psi},\ddot{\xi}$によって
						\begin{align}
							\vee \ddot{\psi} \ddot{\xi}
						\end{align}
						と表される(たとえば$\dot{\xi}$にその$\forall x \xi$
						(resp. $\exists x \xi$)が現れていなければ
						$\ddot{\xi}$は$\dot{\xi}$であるとする).
						(IH)より$\dot{\psi},\dot{\xi}$はそれぞれ
						$\psi,\xi$の書き換えであるから,
						$\ddot{\psi},\ddot{\xi}$はそれぞれ$\psi,\xi$の書き換えである.
				\end{description}
				
			\item[(2)]		
				\begin{description}
					\item[step1]
						$\psi,\xi$の部分式で原子式であるものを全て
						表\ref{tab:formula_rewriting}の通りに書き換えた
						式をそれぞれ$\check{\psi},\check{\xi}$とすれば,[注]より
						\begin{align}
							\vee \check{\psi} \check{\xi}
						\end{align}
						は$\varphi$の部分式で原子式であるものを全て
						表\ref{tab:formula_rewriting}の通りに書き換えた式である.
						従ってこれは$\varphi$の書き換えである.
								
					\item[step2]
						$\widetilde{\psi},\widetilde{\xi}$をそれぞれ$\psi,\xi$の
						書き換えとし,
						\begin{itembox}[l]{IH (帰納法の仮定)}
							$\vee \widetilde{\psi} \widetilde{\xi}$は$\varphi$の
							書き換えである
						\end{itembox}
						と仮定する.このとき,$\widetilde{\psi}$に$\forall x \xi$ 
						(resp. $\exists x \xi$)の形の部分式が現れているとし,
						$y$を$\xi$に自由に現れない変項で$\xi$の中で$x$への代入に
						ついて自由であるものとし,$\widetilde{\psi}$のその
						$\forall x \xi$ (resp. $\exists x \xi$)の部分を
						(一か所) $\forall y \xi(x/y)$
						(resp. $\exists y \xi(x/y)$)に置き換えた式を
						$\overline{\psi}$とすれば,
						\begin{align}
							\vee \overline{\psi} \widetilde{\xi}
						\end{align}
						は$\vee \widetilde{\psi} \widetilde{\xi}$のその
						$\forall x \xi$ (resp. $\exists x \xi$)の部分を
						(一か所) $\forall y \xi(x/y)$
						(resp. $\exists y \xi(x/y)$)に置き換えた式と同じである.
						(IH)より$\vee \widetilde{\psi} \widetilde{\xi}$は$\varphi$の
						書き換えであるから,$\vee \overline{\psi} \widetilde{\xi}$もまた
						$\varphi$の書き換えである.$\forall x \xi$ 
						(resp. $\exists x \xi$)が$\widetilde{\xi}$の部分式であったとしても
						同様のことが言える.
						\QED
				\end{description}
		\end{description}
	\end{metaprf}
	
	\begin{screen}
		\begin{metathm}[部分式の書き換えとの関係3]
		\label{metathm:relation_to_subformula_rewriting_1}
			$\varphi$を$\lang{\varepsilon}$の式ではない$\mathcal{L}$の式とし,
			\begin{align}
				\exists x \psi
			\end{align}
			なる形であるとする.。
			\begin{description}
				\item[(1)] $\varphi$の書き換え$\widehat{\varphi}$は
					$\exists x \widehat{\psi}$
					なる形の式であり,この$\widehat{\psi}$は$\psi$の書き換えである.
			
				\item[(2)] $\check{\psi}$を$\psi$の書き換えとすれば
					$\exists x \check{\psi}$は$\varphi$の書き換えである.
			\end{description}
		\end{metathm}
	\end{screen}
	
	\begin{metaprf}
		[注] $\varphi$の部分式で原子式であるものは
		$\psi$の部分式で原子式であり,逆に部分式で原子式であるものは
		$\varphi$の部分式で原子式であるから,$\varphi$の原子式の部分を
		全て書き換えることと$\psi$の原子式の部分を全て書き換えることは同じである.
		\begin{description}
			\item[(1)] 
				\begin{description}
					\item[step1]
						$\varphi$の部分式で原子式であるものを全て
						表\ref{tab:formula_rewriting}の通りに書き換えた式は,
						[注]より次の形で表される:
						\begin{align}
							\exists x \hat{\psi},
						\end{align}
						ここで$\hat{\psi}$とは,$\psi$の部分式で
						原子式であるものを全て
						表\ref{tab:formula_rewriting}の通りに直した式である.
						ゆえに$\hat{\psi}$は$\psi$の書き換えである.
						
					\item[step2]
						$\widehat{\varphi}$を$\varphi$の書き換えとし,
						\begin{itembox}[l]{IH (帰納法の仮定)}
							$\widehat{\varphi}$は$\exists x \dot{\psi}$
							なる形で書けて,この$\dot{\psi}$は$\psi$の
							書き換えである
						\end{itembox}
						と仮定する.このとき,$\widehat{\varphi}$に$\forall z \xi$ 
						(resp. $\exists z \xi$)の形の部分式が現れているとし,
						$y$を$\xi$に自由に現れない変項で$\xi$の中で$z$への代入に
						ついて自由であるものとすれば,$\widehat{\varphi}$のその
						$\forall z \xi$ (resp. $\exists z \xi$)の部分を
						(一か所) $\forall y \xi(z/y)$
						(resp. $\exists y \xi(z/y)$)に置き換えた式は,
						(IH)より$\dot{\psi}$のその$\forall z \xi$
						(resp. $\exists z \xi$)の部分を
						(一か所) $\forall y \xi(z/y)$
						(resp. $\exists y \xi(z/y)$)に置き換えた式
						$\ddot{\psi}$によって
						\begin{align}
							\exists x \ddot{\psi}
						\end{align}
						と表される.(IH)より$\dot{\psi}$は$\psi$の書き換えである
						から,$\ddot{\psi}$は$\psi$の書き換えである.
				\end{description}
				
			\item[(2)]		
				\begin{description}
					\item[step1]
						$\psi$の部分式で原子式であるものを全て
						表\ref{tab:formula_rewriting}の通りに書き換えた
						式を$\check{\psi}$とすれば,[注]より
						\begin{align}
							\exists x \check{\psi}
						\end{align}
						は$\varphi$の部分式で原子式であるものを全て
						表\ref{tab:formula_rewriting}の通りに書き換えた式である.
						従ってこれは$\varphi$の書き換えである.
								
					\item[step2]
						$\widetilde{\psi}$を$\psi$の書き換えとし,
						\begin{itembox}[l]{IH (帰納法の仮定)}
							$\exists x \widetilde{\psi}$は$\varphi$の
							書き換えである
						\end{itembox}
						と仮定する.このとき,$\widetilde{\psi}$に$\forall z \xi$ 
						(resp. $\exists z \xi$)の形の部分式が現れているとし,
						$y$を$\xi$に自由に現れない変項で$\xi$の中で$z$への代入に
						ついて自由であるものとし,$\widetilde{\psi}$のその
						$\forall z \xi$ (resp. $\exists z \xi$)の部分を
						(一か所) $\forall y \xi(z/y)$
						(resp. $\exists y \xi(z/y)$)に置き換えた式を
						$\overline{\psi}$とすれば,
						\begin{align}
							\exists x \overline{\psi}
						\end{align}
						は$\exists x \widetilde{\psi}$のその
						$\forall z \xi$ (resp. $\exists z \xi$)の部分を
						(一か所) $\forall y \xi(z/y)$
						(resp. $\exists y \xi(z/y)$)に置き換えた式と同じである.
						(IH)より$\exists x \widetilde{\psi}$は$\varphi$の
						書き換えであるから,$\exists x \overline{\psi}$もまた$\varphi$の
						書き換えである.
						\QED
				\end{description}
		\end{description}
	\end{metaprf}
	
	\begin{screen}
		\begin{metathm}[書き換え後も自由な変項は増減しない]
		\label{metathm:variables_unchanged_after_rewriting}
			$\varphi$を$\lang{\varepsilon}$の式ではない$\mathcal{L}$の式とし,
			この書き換えを$\widehat{\varphi}$とする.このとき
			$\varphi$に自由に現れる変項は$\widehat{\varphi}$にも自由に現れ,
			逆に$\widehat{\varphi}$に自由に現れる変項は$\varphi$にも自由に現れる.
			特に,$\varphi$が文ならば$\widehat{\varphi}$も文である.
		\end{metathm}
	\end{screen}
	
	\begin{metaprf}\mbox{}
		\begin{description}
			\item[step1] $\varphi$が原子式であるときは,書き換え時の変項条件より
				$\varphi$と$\widehat{\varphi}$に現れる変項は一致する.
			
			\item[step2]
				$\varphi$が一般の式であるとき
				\begin{itembox}[l]{IH (帰納法の仮定)}
					$\varphi$の任意の真部分式$\psi$に対し,その書き換えを
					$\widehat{\psi}$とすれば($\psi$が$\lang{\varepsilon}$の式ならば
					$\widehat{\psi}$は$\psi$とする),
					$\psi$に自由に現れる変項は$\widehat{\psi}$にも自由に現れ,
					逆に$\widehat{\psi}$に自由に現れる変項は$\psi$にも自由に現れる.
				\end{itembox}
				と仮定する.すると
				\begin{description}
					\item[case1] $\varphi$が
						\begin{align}
							\negation \psi
						\end{align}
						なる式の場合,$\widehat{\varphi}$は
						\begin{align}
							\negation \widehat{\psi}
						\end{align}
						なる形の式であるが,
						メタ定理\ref{metathm:relation_to_subformula_rewriting}より
						$\widehat{\psi}$は$\psi$の書き換えである.
						$\varphi$に自由に現れる変項は$\psi$に自由に現れる変項と一致するが,
						(IH)よりそれは$\widehat{\psi}$に自由に現れる変項と一致するので,
						$\widehat{\varphi}$に自由に現れる変項とも一致する.
						
					\item[case2] $\varphi$が
						\begin{align}
							\vee \psi \chi
						\end{align}
						なる式の場合,$\widehat{\varphi}$は
						\begin{align}
							\vee \widehat{\psi} \widehat{\chi}
						\end{align}
						なる形の式であるが,
						メタ定理\ref{metathm:relation_to_subformula_rewriting}より
						$\widehat{\psi},\widehat{\chi}$はそれぞれ$\psi,\chi$の書き換えである
						($\psi,\chi$の一方は元から$\lang{\varepsilon}$の式かもしれないが,
						たとえば$\psi$がそうなら$\widehat{\psi}$も$\check{\psi}$も$\psi$
						であるとする).
						$\varphi$に自由に現れる変項は$\psi,\chi$に自由に現れる変項と
						一致するが,(IH)よりそれは$\widehat{\psi},\widehat{\chi}$に自由に現れる
						変項と一致するので,$\widehat{\varphi}$に自由に現れる変項とも一致する.
					
					\item[case3] $\varphi$が
						\begin{align}
							\exists x \psi
						\end{align}
						なる式の場合,$\widehat{\varphi}$は
						\begin{align}
							\exists x \widehat{\psi}
						\end{align}
						なる形の式であるが,
						メタ定理\ref{metathm:relation_to_subformula_rewriting}より
						$\widehat{\psi}$は$\psi$の書き換えである.
						$\varphi$に自由に現れる変項は$\psi$に自由に現れる$x$以外の
						変項と一致するが,(IH)よりそれは$\widehat{\psi}$に自由に現れる$x$以外の
						変項と一致するので,$\widehat{\varphi}$に自由に現れる変項とも一致する.
						\QED
				\end{description}
		\end{description}
	\end{metaprf}
\subsection{式の書き換え}
\label{subsec:formula_rewriting}
	$\varepsilon$項を取り入れたのは{\bf 存在文}\index{そんざいぶん@存在文}
	{\bf (existential sentence)}に対して{\bf 証人}\index{しょうにん@証人}
	{\bf (witnessing term)}を与えるためであり,それは
	\begin{align}
		\exists x \varphi(x) \rarrow \varphi(\varepsilon x \varphi(x))
	\end{align}
	なる式を公理とすることで裏付けされる.ただし$\varepsilon$項を作れるのは$\lang{\varepsilon}$
	の式のみであるから,$\varphi$が内包項を含んだ式であると$\varepsilon x \varphi(x)$を
	使うことが出来ない.とはいえ内包項を含んだ存在文も往々にして登場するので,それらに対しても
	証人を用意できると便利である.そこで$\varphi$を内包項を含んだ$\mathcal{L}$の式とするとき,
	$\varphi$を``同値''な$\lang{\varepsilon}$の式$\hat{\varphi}$に書き換えて
	\begin{align}
		\exists x \varphi(x) \rarrow \varphi(\varepsilon x \hat{\varphi}(x))
	\end{align}
	を公理とする(量化公理\ref{logicalaxm:rules_of_quantifiers}).
	注意点は\underline{同値な書き換えはいくらでも作れる}ということであり,
	$\check{\varphi}$も$\varphi$の書き換えならば
	\begin{align}
		\exists x \varphi(x) \rarrow \varphi(\varepsilon x \check{\varphi}(x))
	\end{align}
	も公理とする.書き換える必要があるのは内包項を含んでいる式のみであり,
	またそのような式の原子式の部分,つまり$\in$と$=$のスコープ,だけを書き換えれば十分である.
	書き換えが``同値''というのは後述の\ref{sec:equivalence_of_formula_rewriting}節
	で述べてあるような意味であるが,直感的に妥当な範囲でしかない.原子式の書き換えは次の要領で行う:
	
	\begin{table}[H]
		\begin{center}
		\begin{tabular}{c|c|c}
			元の式 & 書き換え後 & 付記 \\ \hline \hline
			$a = \Set{z}{\psi}$ & $\forall v\, (\, v \in a \lrarrow \psi(z/v)\, )$ & \\ \hline
			$\Set{y}{\varphi} = b$ & $\forall u\, (\, \varphi(y/u) \lrarrow u \in b\, )$ & \\ \hline
			$\Set{y}{\varphi} = \Set{z}{\psi}$ & $\forall u\, (\, \varphi(y/u) \lrarrow \psi(z/u)\, )$ & \\ \hline
			$a \in \Set{z}{\psi}$ & $\psi(z/a)$ & 必要なら束縛変項の名前替えをする \\ \hline
			$\Set{y}{\varphi} \in b$ & $\exists s\, (\, \forall u\, (\, \varphi(y/u) \lrarrow u \in s\, ) \wedge s \in b\, )$ & \\ \hline
			$\Set{y}{\varphi} \in \Set{z}{\psi}$ & $\exists s\, (\, \forall u\, (\, \varphi(y/u) \lrarrow u \in s\, ) \wedge \psi(z/s)\, )$ & \\ \hline
		\end{tabular}
		\end{center}
	\end{table}
	
	ただし上の記号に課している条件は
	\begin{itemize}
		\item $a,b$は$\lang{\varepsilon}$の項である
			(\ref{sec:restriction_of_formulas}節より
			$a,b$は変項か主要$\varepsilon$項).
		
		\item $\Set{y}{\varphi}$と$\Set{z}{\psi}$を正則内包項である.
		
		\item $u$は$\varphi$の中で$y$への代入について自由であり,
			$u,v,s$は$\psi$の中で$z$への代入について自由である.
			上の式の書き換えにおいては変項$u,v,s$を追加したが,
			代入について自由である限りどの変項を選んでも構わない.
			従って\underline{式の書き換えは一つに決まらない}ということになるが,
			違う変項を選んでも式の意味は変わらない.
			
		\item 付記「束縛変項の名前替え」について.
			$a$を$\psi$の中の自由な$z$に代入した後で$a$が束縛される場合,
			束縛変項の名前替えをしなくてはならない.たとえば
			\begin{align}
				a \in \Set{z}{\forall a\, (\, z \in a\, )}
			\end{align}
			という式では左辺の$a$は自由であるのに,書き換えの規則をそのまま適用すると
			\begin{align}
				\forall a\, (\, a \in a\, )
			\end{align}
			となり束縛されてしまう.代入後の$a$が束縛されないためには
			\begin{align}
				a \in \Set{z}{\forall b\, (\, z \in b\, )}
			\end{align}
			のように束縛変項$a$を別の変項$b$に替えて
			\begin{align}
				\forall b\, (\, a \in b\, )
			\end{align}
			とすればよい.
	\end{itemize}
	
	\begin{screen}
		\begin{metadfn}[式の書き換え]
			$\varphi$を$\lang{\varepsilon}$の式ではない$\mathcal{L}$の式とするとき,
			$\varphi$の原子式の部分,つまり$\in$と$=$のスコープを全て上表に従って直した式を
			$\varphi$の{\bf 書き換え}と呼ぶ.
		\end{metadfn}
	\end{screen}
	
	\begin{screen}
		\begin{metathm}[書き換えは$\lang{\varepsilon}$の式]
		\label{metathm:rewritten_formulas_are_of_L_epsilon}
			$\varphi$を$\lang{\varepsilon}$の式ではない$\mathcal{L}$の式とし,
			$\hat{\varphi}$を$\varphi$の書き換えとするとき,
			$\hat{\varphi}$は$\lang{\varepsilon}$の式である.
		\end{metathm}
	\end{screen}
	
	\begin{metaprf}\mbox{}
		\begin{description}
			\item[step1] $\varphi$が原子式なら,表の通り$\hat{\varphi}$は
				$\lang{\varepsilon}$の式である.
			
			\item[step2] $\varphi$が原子式ないとき,
				\begin{itembox}[l]{IH (帰納法の仮定)}
					$\varphi$の任意の真部分式は,それが$\lang{\varepsilon}$の式でない場合
					その書き換えは$\lang{\varepsilon}$の式である.
				\end{itembox}
				と仮定する.
				\begin{description}
					\item[case1] $\varphi$が
						\begin{align}
							\negation \psi
						\end{align}
						なる式のとき,$\varphi$の$\in,=$のスコープはいずれも
						$\psi$の部分原子式であり,逆に$\psi$の$\in,=$のどのスコープも
						$\varphi$の部分原子式であるから,$\varphi$の原子式の部分を
						全て書き換えるということは$\psi$の原子式の部分を全て書き換える
						ということになる.$\hat{\varphi}$は
						\begin{align}
							\negation \hat{\psi}
						\end{align}
						なる形の式であるが,$\hat{\psi}$は$\psi$の書き換えであり,
						(IH)より$\lang{\varepsilon}$の式である.ゆえに
						$\hat{\varphi}$も$\lang{\varepsilon}$の式である.
						
					\item[case2] $\varphi$が
						\begin{align}
							\vee \psi \chi
						\end{align}
						なる式のとき,$\varphi$の$\in,=$のスコープはいずれも
						$\psi$か$\chi$の一方の部分原子式であり(始切片の一意性
						のメタ定理\ref{metathm:initial_segment_L}より
						$\varphi$の真部分式が$\psi$と$\chi$の境を跨ぐことはない),
						逆に$\psi,\chi$の$\in,=$のどのスコープも
						$\varphi$の部分原子式であるから,$\varphi$の原子式の部分を
						全て書き換えるということは$\psi$と$\chi$の原子式の部分を全て書き換える
						ということになる.$\hat{\varphi}$は
						\begin{align}
							\vee \hat{\psi} \hat{\chi}
						\end{align}
						なる形の式であるが,$\hat{\psi},\hat{\chi}$はそれぞれ$\psi,\chi$の
						書き換えであり(もしくは,$\psi,\chi$の一方は元から
						$\lang{\varepsilon}$の式かもしれない),(IH)よりどちらも
						$\lang{\varepsilon}$の式である.ゆえに
						$\hat{\varphi}$も$\lang{\varepsilon}$の式である.
						
					\item[case3] $\varphi$が
						\begin{align}
							\exists x \psi
						\end{align}
						なる式のとき,case1 と同様の理由で$\varphi$の原子式の部分を
						全て書き換えるということは$\psi$の原子式の部分を全て書き換える
						ということになる.$\hat{\varphi}$は
						\begin{align}
							\exists x \hat{\psi}
						\end{align}
						なる形の式であるが,$\hat{\psi}$は$\psi$の書き換えであり,
						(IH)より$\lang{\varepsilon}$の式である.ゆえに
						$\hat{\varphi}$も$\lang{\varepsilon}$の式である.
						\QED
				\end{description}
		\end{description}
	\end{metaprf}
	
	\begin{screen}
		\begin{metathm}[部分式の書き換えとの関係]
		\label{metathm:relation_to_subformula_rewriting}
			$\varphi$を$\lang{\varepsilon}$の式ではない$\mathcal{L}$の式とするとき,
			\begin{description}
				\item[case1] $\varphi$が$\negation \psi$なる式のとき,
					$\varphi$の書き換え$\hat{\varphi}$は$\negation \hat{\psi}$
					なる形の式であるが,このとき$\hat{\psi}$は$\psi$の書き換えである.
					逆に$\check{\psi}$を$\psi$の書き換えとすれば$\negation \check{\psi}$
					は$\varphi$の書き換えである.
					
				\item[case2] $\varphi$が$\vee \psi \chi$なる式のとき,
					$\varphi$の書き換え$\hat{\varphi}$は$\vee \hat{\psi} \hat{\chi}$
					なる形の式であるが,このとき$\hat{\psi},\hat{\chi}$はそれぞれ$\psi,\chi$の
					書き換えである.逆に$\check{\psi},\check{\chi}$をそれぞれ$\psi,\chi$の
					書き換えとすれば$\vee \check{\psi} \check{\chi}$は$\varphi$の
					書き換えである.なお,$\psi,\chi$の一方は元から
					$\lang{\varepsilon}$の式かもしれないが,たとえば$\psi$がそうなら
					$\hat{\psi}$も$\check{\psi}$も$\psi$であるとする.
					
				\item[case3] $\varphi$が$\exists x \psi$なる式のとき,
					$\varphi$の書き換え$\hat{\varphi}$は$\exists x \hat{\psi}$
					なる形の式であるが,このとき$\hat{\psi}$は$\psi$の書き換えである.
					逆に$\check{\psi}$を$\psi$の書き換えとすれば$\exists x \check{\psi}$
					は$\varphi$の書き換えである.
			\end{description}
		\end{metathm}
	\end{screen}
	
	\begin{metaprf}
		証明は前定理の説明と大方被ってしまうがもう一度載せて置く.
		\begin{description}
			\item[case1] $\varphi$の$\in,=$のスコープはいずれも
				$\psi$の部分原子式であり,逆に$\psi$の$\in,=$のどのスコープも
				$\varphi$の部分原子式であるから,$\varphi$の原子式の部分を
				全て書き換えることと$\psi$の原子式の部分を全て書き換えることは同じである.
				従って$\hat{\psi}$は$\psi$の書き換えであり,$\negation \check{\psi}$は
				$\varphi$の書き換えである.
			
			\item[case2] $\varphi$の$\in,=$のスコープはいずれも
				$\psi$か$\chi$の一方の部分原子式であり(始切片の一意性
				のメタ定理\ref{metathm:initial_segment_L}より
				$\varphi$の真部分式が$\psi$と$\chi$の境を跨ぐことはない),
				逆に$\psi,\chi$の$\in,=$のどのスコープも
				$\varphi$の部分原子式であるから,$\varphi$の原子式の部分を
				全て書き換えることと$\psi$と$\chi$の原子式の部分を全て書き換えることは同じである.
				従って$\hat{\psi},\hat{\chi}$はそれぞれ$\psi,\chi$の書き換えであり,
				$\negation \check{\psi} \check{\chi}$は$\varphi$の書き換えである.
				
			\item[case3] case1 と同じ理由によって,$\hat{\psi}$は$\psi$の書き換えであり,
				$\exists x \check{\psi}$は$\varphi$の書き換えである.
				\QED
		\end{description}
	\end{metaprf}
	
	\begin{screen}
		\begin{metathm}[書き換え後も自由な変項は増減しない]
		\label{metathm:variables_unchanged_after_rewriting}
			$\varphi$を$\lang{\varepsilon}$の式ではない$\mathcal{L}$の式とし,
			この書き換えを$\hat{\varphi}$とする.このとき
			$\varphi$に自由に現れる変項は$\hat{\varphi}$にも自由に現れ,
			逆に$\hat{\varphi}$に自由に現れる変項は$\varphi$にも自由に現れる.
			特に,$\varphi$が文ならば$\hat{\varphi}$も文である.
		\end{metathm}
	\end{screen}
	
	\begin{metaprf}\mbox{}
		\begin{description}
			\item[step1] $\varphi$が原子式であるときは上の書き換え表より一目瞭然である.
			
			\item[step2]
				$\varphi$が一般の式であるとき
				\begin{itembox}[l]{IH (帰納法の仮定)}
					$\varphi$の任意の真部分式$\psi$に対し,その書き換えを
					$\hat{\psi}$とすれば($\psi$が$\lang{\varepsilon}$の式ならば
					$\hat{\psi}$は$\psi$とする),
					$\psi$に自由に現れる変項は$\hat{\psi}$にも自由に現れ,
					逆に$\hat{\psi}$に自由に現れる変項は$\psi$にも自由に現れる.
				\end{itembox}
				と仮定する.すると
				\begin{description}
					\item[case1] $\varphi$が
						\begin{align}
							\negation \psi
						\end{align}
						なる式の場合,$\hat{\varphi}$は
						\begin{align}
							\negation \hat{\psi}
						\end{align}
						なる形の式であるが,
						メタ定理\ref{metathm:relation_to_subformula_rewriting}より
						$\hat{\psi}$は$\psi$の書き換えである.
						$\varphi$に自由に現れる変項は$\psi$に自由に現れる変項と一致するが,
						(IH)よりそれは$\hat{\psi}$に自由に現れる変項と一致するので,
						$\hat{\varphi}$に自由に現れる変項とも一致する.
						
					\item[case2] $\varphi$が
						\begin{align}
							\vee \psi \chi
						\end{align}
						なる式の場合,$\hat{\varphi}$は
						\begin{align}
							\vee \hat{\psi} \hat{\chi}
						\end{align}
						なる形の式であるが,
						メタ定理\ref{metathm:relation_to_subformula_rewriting}より
						$\hat{\psi},\hat{\chi}$はそれぞれ$\psi,\chi$の書き換えである
						($\psi,\chi$の一方は元から$\lang{\varepsilon}$の式かもしれないが,
						たとえば$\psi$がそうなら$\hat{\psi}$も$\check{\psi}$も$\psi$
						であるとする).
						$\varphi$に自由に現れる変項は$\psi,\chi$に自由に現れる変項と
						一致するが,(IH)よりそれは$\hat{\psi},\hat{\chi}$に自由に現れる
						変項と一致するので,$\hat{\varphi}$に自由に現れる変項とも一致する.
					
					\item[case3] $\varphi$が
						\begin{align}
							\exists x \psi
						\end{align}
						なる式の場合,$\hat{\varphi}$は
						\begin{align}
							\exists x \hat{\psi}
						\end{align}
						なる形の式であるが,
						メタ定理\ref{metathm:relation_to_subformula_rewriting}より
						$\hat{\psi}$は$\psi$の書き換えである.
						$\varphi$に自由に現れる変項は$\psi$に自由に現れる$x$以外の
						変項と一致するが,(IH)よりそれは$\hat{\psi}$に自由に現れる$x$以外の
						変項と一致するので,$\hat{\varphi}$に自由に現れる変項とも一致する.
						\QED
				\end{description}
		\end{description}
	\end{metaprf}
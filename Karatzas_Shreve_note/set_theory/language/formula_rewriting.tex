\subsection{式の書き換え}
	$\varepsilon$項を取り入れた目的は{\bf 存在文}\index{そんざいぶん@存在文}
	{\bf (existential sentence)}に対して{\bf 証人}\index{しょうにん@証人}{\bf (witness)}
	を与えることであり,それは
	\begin{align}
		\exists x \varphi(x) \rarrow \varphi(\varepsilon x \varphi(x))
	\end{align}
	なる式を公理とすることで実質的に裏付けされる.
	ただし$\varepsilon$項を作れる式は$\lang{\varepsilon}$の式のみであるから,
	$\varphi$が内包項を含んだ式であると$\varepsilon x \varphi(x)$を使うことが出来ない.
	とはいえ$\mathcal{L}$の式の存在文も往々にして登場するので
	それらに対しても証人を用意できると便利である.
	そこで$\mathcal{L}$の式を``同値''な$\lang{\varepsilon}$の式に書き換えて,
	その書き換えた式で作る$\varepsilon$項を使うことにする.つまり
	$\varphi$が$\mathcal{L}$の式である場合は,$\varphi$を
	``同値''な$\lang{\varepsilon}$の式$\hat{\varphi}$に書き換えてから
	\begin{align}
		\exists x \varphi(x) \rarrow \varphi(\varepsilon x \hat{\varphi}(x))
	\end{align}
	を保証するのである.書き換える必要があるのは内包項を含んでいる式のみであり,
	また原子式だけを書き換えれば十分である.
	書き換えが``同値''というのは後述の\ref{sec:equivalence_of_formula_rewriting}節
	で述べてあるような意味であるが,それは直感的に妥当なものである.原子式の書き換えは次の要領で行う:
	
	\begin{table}[H]
		\begin{center}
		\begin{tabular}{c|c|c}
			元の式 & 書き換え後 & 付記 \\ \hline \hline
			$a = \Set{z}{\psi}$ & $\forall v\, (\, v \in a \lrarrow \psi(z/v)\, )$ & \\ \hline
			$\Set{y}{\varphi} = b$ & $\forall u\, (\, \varphi(y/u) \lrarrow u \in b\, )$ & \\ \hline
			$\Set{y}{\varphi} = \Set{z}{\psi}$ & $\forall u\, (\, \varphi(y/u) \lrarrow \psi(z/u)\, )$ & \\ \hline
			$a \in \Set{z}{\psi}$ & $\psi(z/a)$ & 必要なら束縛変項の名前替えをする\footnotemark \\ \hline
			$\Set{y}{\varphi} \in b$ & $\exists s\, (\, \forall u\, (\, \varphi(y/u) \lrarrow u \in s\, ) \wedge s \in b\, )$ & \\ \hline
			$\Set{y}{\varphi} \in \Set{z}{\psi}$ & $\exists s\, (\, \forall u\, (\, \varphi(y/u) \lrarrow u \in s\, ) \wedge \psi(z/s)\, )$ & \\ \hline
		\end{tabular}
		\end{center}
	\end{table}
	
	ただし上の記号に課している条件は
	\begin{itemize}
		\item $a,b$は$\lang{\varepsilon}$の項である
			(\ref{sec:restriction_of_formulas}節より
			$a,b$は変項か主要$\varepsilon$項).
		\item $\Set{y}{\varphi}$と$\Set{z}{\psi}$を正則内包項である.
		\item $u$は$\varphi$の中で$y$への代入について自由であり,
			$u,v,s$は$\psi$の中で$z$への代入について自由である.
			上の式の書き換えにおいては変項$u,v,s$を追加したが,
			代入について自由である限りどの変項を選んでも構わない.
			従って式の書き換えは一つに決まらないということになるが,
			違う変項を選んでも式の意味は変わらない.
	\end{itemize}
	
	\footnotetext{
			$a$を$\psi$の中の自由な$z$に代入した後で$a$が束縛される場合,
			束縛変項の名前替えをしなくてはならない.たとえば
			\begin{align}
				a \in \Set{z}{\forall a\, (\, z \in a\, )}
			\end{align}
			という式では左辺の$a$は自由であるのに,書き換えの規則をそのまま適用すると
			\begin{align}
				\forall a\, (\, a \in a\, )
			\end{align}
			となり束縛されてしまう.代入後の$a$が束縛されないためには
			\begin{align}
				a \in \Set{z}{\forall b\, (\, z \in b\, )}
			\end{align}
			のように束縛変項$a$を別の変項$b$に替えて
			\begin{align}
				\forall b\, (\, a \in b\, )
			\end{align}
			とすればよい.
	}
	
	原子式に対する書き換えが掲示されたので,$\mathcal{L}$の一般の式$\varphi$から
	$\lang{\varepsilon}$の式$\hat{\varphi}$を得る操作は次の帰納的な要領で行えばよい.
	\begin{description}
		\item[case1] $\varphi$が
			\begin{align}
				\negation \psi
			\end{align}
			なる式であるとき,$\psi$を$\lang{\varepsilon}$の式に書き換えた$\hat{\psi}$を用いて
			\begin{align}
				\negation \hat{\psi}
			\end{align}
			を$\hat{\varphi}$とする.
			
		\item[case2] $\varphi$が
			\begin{align}
				\vee \psi \chi
			\end{align}
			なる式であるとき,$\psi,\chi$を$\lang{\varepsilon}$の式に書き換えた
			$\hat{\psi},\hat{\chi}$を用いて
			\begin{align}
				\vee \hat{\psi} \hat{\chi}
			\end{align}
			を$\hat{\varphi}$とする.$\varphi$が$\wedge \psi \chi$や$\rarrow \psi \chi$
			の形の時も同様にする.
			
		\item[case3] $\varphi$が
			\begin{align}
				\exists x \psi
			\end{align}
			なる式であるとき,$\psi$を$\lang{\varepsilon}$の式に書き換えた
			$\hat{\psi}$を用いて
			\begin{align}
				\exists x \hat{\psi}
			\end{align}
			を$\hat{\varphi}$とする.$\varphi$が$\forall x \psi$の形の時も同様にする.
	\end{description}
	
	もちろん$\varphi$が$\lang{\varepsilon}$の式ならわざわざ書き換える必要は無い.
	
	\begin{screen}
		\begin{metathm}[書き換え後も自由な変項は増減しない]
			$\varphi$を$\mathcal{L}$の式とし,これを$\lang{\varepsilon}$の式に
			書き換えたものを$\hat{\varphi}$とする.このとき
			$\varphi$に自由に現れる変項と$\hat{\varphi}$に自由に現れる変項は一致する.
		\end{metathm}
	\end{screen}
	
	\begin{metaprf}\mbox{}
		\begin{description}
			\item[step1] $\varphi$が原子式であるときは上の書き換え表より一目瞭然である.
			
			\item[step2]
				$\varphi$が一般の式であるとき
				\begin{itembox}[l]{IH (帰納法の仮定)}
					$\varphi$の任意の真部分式$\psi$と,それを$\lang{\varepsilon}$の式
					に書き換えた$\hat{\psi}$は,自由に現れる変項が一致する
				\end{itembox}
				と仮定する.すると
				\begin{description}
					\item[case1] $\varphi$が
						\begin{align}
							\negation \psi
						\end{align}
						なる式の場合,$\varphi$に自由に現れる変項は
						$\psi$に自由に現れる変項と一致するが,それは
						$\hat{\psi}$に自由に現れる変項と一致するので,
						$\negation \hat{\psi}$に自由に現れる変項とも一致する.
						
					\item[case2] $\varphi$が
						\begin{align}
							\vee \psi \chi
						\end{align}
						なる式の場合,$\varphi$に自由に現れる変項は$\psi,\chi$に自由に現れる
						変項と一致するが,それは$\hat{\psi},\hat{\chi}$に
						自由に現れる変項と一致するので,
						$\negation \hat{\psi} \hat{\chi}$に自由に現れる変項とも一致する.
					
					\item $\varphi$が
						\begin{align}
							\exists x \psi
						\end{align}
						なる式の場合,$\varphi$に自由に現れる変項は
						$\psi$に自由に現れる$x$以外の変項と一致するが,それは
						$\hat{\psi}$に自由に現れる$x$以外の変項変項と一致するので,
						$\negation \hat{\psi}$に自由に現れる変項とも一致する.
						\QED
				\end{description}
		\end{description}
	\end{metaprf}
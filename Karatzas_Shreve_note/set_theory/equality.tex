\section{相等性}
	本稿において``等しい''とは項に対する言明であって,$a$と$b$を項とするとき
	\begin{align}
		a = b
	\end{align}
	なる式で表される.この記号
	\begin{align}
		=
	\end{align}
	は{\bf 等号}\index{とうごう@等号}{\bf (equal sign)}と呼ばれるが,
	現時点では述語として導入されているだけで,推論操作における働きはまだ明文化していない.
	本節では,いつ類は等しくなるのか,そして,等しい場合に何が起きるのか,の二つが主題となる.
	
	\begin{screen}
		\begin{axm}[外延性の公理]
			$a,b$を類とするとき,次が成り立つ:
			\begin{align}
				\forall x\, (\, x \in a \Longleftrightarrow x \in b\, )
				\Longrightarrow a=b.
			\end{align}
		\end{axm}
	\end{screen}
	
	\begin{screen}
		\begin{thm}[任意の類は自分自身と等しい]\label{thm:any_class_equals_to_itself}
			$a$を類とするとき次が成り立つ:
			\begin{align}
				a = a.
			\end{align}
		\end{thm}
	\end{screen}
	
	\begin{sketch}
		任意の$\varepsilon$項$\tau$に対して,
		推論法則\ref{logicalthm:reflective_law_of_implication}より
		\begin{align}
			\tau \in a \Longleftrightarrow \tau \in a
		\end{align}
		が成り立つから,$\tau$の任意性より
		\begin{align}
			\forall x\, (\, x \in a  \Longleftrightarrow x \in a\, )
		\end{align}
		が成り立つ.外延性の公理と三段論法より
		\begin{align}
			a = a
		\end{align}が得られる.
		\QED
	\end{sketch}
	
	\begin{screen}
		\begin{thm}[$\varepsilon$項は集合である]
			任意の$\varepsilon$項$\varepsilon x A(x)$に対して
			\begin{align}
				\set{\varepsilon x A(x)}.
			\end{align}
		\end{thm}
	\end{screen}
	
	\begin{sketch}
		定理\ref{thm:any_class_equals_to_itself}より
		\begin{align}
			\varepsilon x A(x) = \varepsilon x A(x)
		\end{align}
		が成立するので,存在記号の推論規則より
		\begin{align}
			\exists y\, \left(\, \varepsilon x A(x) = y\, \right)
		\end{align}
		が成立する.
		\QED
	\end{sketch}
	
	$A$を$\mathcal{L}_{\in}$の式とし,$x$を$A$に現れる変項とし,$x$のみが$A$で自由であるとし,かつ
	\begin{align}
		\set{\Set{x}{A(x)}}
	\end{align}
	が満たされているとする.つまり
	\begin{align}
		\exists y\, \left(\, \Set{x}{A(x)} = y\, \right)
	\end{align}
	が成り立っているということであるが,$\Set{x}{A(x)} = y$を
	\begin{align}
		\forall x\, \left(\, A(x) \Longleftrightarrow x \in y\, \right)
	\end{align}
	と書き換えれば,存在記号の推論規則より
	\begin{align}
		\Set{x}{A(x)} = \varepsilon y \forall x\, \left(\, A(x) \Longleftrightarrow x \in y\, \right)
	\end{align}
	が得られる.
	
	\begin{screen}
		\begin{thm}[集合である内包項は$\varepsilon$項で書ける]
			任意の内包項$\Set{x}{A(x)}$に対して,$\Set{x}{A(x)}$が集合であれば
			\begin{align}
				\Set{x}{A(x)} = \varepsilon y \forall x\, \left(\, A(x) \Longleftrightarrow x \in y\, \right).
			\end{align}
		\end{thm}
	\end{screen}
	
	ブルバキでは$\tau$項を,島内では$\varepsilon$項のみを導入して
	$\varepsilon y \forall x\, \left(\, A(x) \Longleftrightarrow x \in y\, \right)$
	によって$\Set{x}{A(x)}$を定めている.本稿と同じくブルバキの$\tau$項も島内の$\varepsilon$項も
	集合を表すものであるから,
	\begin{align}
		\exists y\, \forall x\, \left(\, A(x) \Longleftrightarrow x \in y\, \right)
	\end{align}
	を満たさないような性質$A$に対しては$\varepsilon y \forall x\, \left(\, A(x) \Longleftrightarrow x \in y\, \right)$
	は不定の集合を指す.本稿では
	
	\begin{screen}
		\begin{axm}[要素の公理]
			要素となりうる類は集合である.つまり,$a,b$を類とするとき
			\begin{align}
				a \in b \Longrightarrow \set{a}.
			\end{align}
		\end{axm}
	\end{screen}
	
	\begin{screen}
		\begin{axm}[内包性公理] 
			$A$を$\mathcal{L}_{\in}$の式とし,$x$を$A$に現れる変項とし,$y$を$A(x)$に現れない変項とし,
			$x$のみが$A$で自由であるとする.このとき
			\begin{align}
				\forall y\, \left(\, y \in \Set{x}{A(x)} \Longleftrightarrow A(y)\, \right).
			\end{align}
		\end{axm}
	\end{screen}
	
	要素の公理で要求していることは{\bf 類を構成できるのは集合に限られる}ということであり,
	内包性公理は{\bf 甲種項はその固有の性質を持つ集合の全体である}という意味を持つ.
	
	
	例えば
	\begin{align}
		a = b
	\end{align}
	と書いてあったら``$a$と$b$は等しい''と読めるわけだが,明らかに$a$は$b$とは違うではないではないか!
	こんなことはしょっちゅう起こることであって,上で述べたように$\Set{x}{A(x)}$が集合なら
	\begin{align}
		\Set{x}{A(x)} = \varepsilon y \forall x\, \left(\, A(x) \Longleftrightarrow x \in y\, \right)
	\end{align}
	が成り立ったりする.そこで``数学的に等しいとは何事か''という疑問が浮かぶのは至極自然であって,
	それに答えるのが次の相等性公理である.
	
	\begin{screen}
		\begin{axm}[相等性公理]
			$A$を$\mathcal{L}'$の式とし,$x$を$A$に現れる文字とし,
			$x$のみが$A$で量化されていないとする.このとき$a,b$を類とすれば次が成り立つ:
			\begin{align}
				a = b \Longrightarrow (\, A(a) \Longleftrightarrow A(b)\, ).
			\end{align}
		\end{axm}
	\end{screen}
	
	\begin{screen}
		\begin{thm}[外延性の公理の逆も成り立つ]\label{thm:axiom_of_extensionality_equivalent}
			$a$と$b$を類とするとき
			\begin{align}
				a=b \Longrightarrow \forall x\, (\, x \in a  \Longleftrightarrow x \in b\, ).
			\end{align}
		\end{thm}
	\end{screen}
	
	\begin{prf}
		$a = b$が成り立っていると仮定すれば,相等性の公理より$\mathcal{L}$の任意の対象$\tau$に対して
		\begin{align}
			\tau \in a \Longleftrightarrow \tau \in b
		\end{align}
		が満たされるから,推論法則\ref{logicalthm:fundamental_law_of_universal_quantifier}より
		\begin{align}
			\forall x\, (\, x \in a  \Longleftrightarrow x \in b\, )
		\end{align}
		が成立する.よって演繹法則より
		\begin{align}
			a = b \Longrightarrow \forall x\, (\, x \in a  \Longleftrightarrow x \in b\, )
		\end{align}
		が成り立つ.
		\QED
	\end{prf}
	
	\monologue{
		等しい類同士は同じ$\mathcal{L}$の対象を要素に持つと示されましたが,
		このとき要素に持つ集合まで一致します.これは相等性の公理から明らかでしょうが,
		詳しくは部分類の箇所で説明いたしましょう.
	}
	
	\begin{screen}
		\begin{thm}[条件を満たす集合は要素である]\label{thm:satisfactory_set_is_an_element}
			$A$を$\mathcal{L}$の式とし,$x$を$A$に現れる文字とし,$t$を$A(x)$に現れない文字とし,
			$x$のみが$A$で量化されていないとする.このとき,$a$を類とすると
			\begin{align}
				A(a) \Longrightarrow 
				\left(\, \set{a} \Longrightarrow a \in \Set{x}{A(x)}\, \right).
			\end{align}
		\end{thm}
	\end{screen}
	
	\begin{sketch}
		いま
		\begin{align}
			A(a)
		\end{align}
		と
		\begin{align}
			\set{a}
		\end{align}
		が成立していると仮定する.このとき要素の公理から
		\begin{align}
			\exists x\, (\, a = x\, )
		\end{align}
		が成立するので,
		\begin{align}
			\tau \defeq \varepsilon x\, (\, a = x\, )
		\end{align}
		とおけば
		\begin{align}
			a = \tau
		\end{align}
		が成り立ち,相等性の公理より
		\begin{align}
			A(\tau)
		\end{align}
		が成立する.よって類の公理より
		\begin{align}
			\tau \in \Set{x}{A(x)}
		\end{align}
		が従い,相等性の公理から
		\begin{align}
			a \in \Set{x}{A(x)}
		\end{align}
		が成立する.
		\QED
	\end{sketch}
	
	\begin{screen}
		\begin{thm}[$\Univ$は集合の全体である]
		\label{thm:V_is_the_whole_of_sets}
			$a$を類とするとき次が成り立つ:
			\begin{align}
				\set{a} \Longleftrightarrow a \in \Univ.
			\end{align}
		\end{thm}
	\end{screen}
	
	\begin{prf}
		$a$を類とするとき,まず要素の公理より
		\begin{align}
			a \in \Univ \Longrightarrow \set{a}
		\end{align}
		が得られる.逆に
		\begin{align}
			\set{a}
		\end{align}
		が成り立っていると仮定する.このとき
		\begin{align}
			\tau \defeq \varepsilon x (\ a = x\ )
		\end{align}
		とおけば,定理\ref{thm:any_class_equals_to_itself}より
		\begin{align}
			\tau = \tau
		\end{align}
		となるので,類の公理より
		\begin{align}
			\tau \in \Univ
		\end{align}
		が成り立つ.そして相等性の公理より
		\begin{align}
			a \in \Univ
		\end{align}
		が従うから
		\begin{align}
			\set{a} \Longrightarrow a \in \Univ
		\end{align}
		も得られる.
		\QED
	\end{prf}
	
	\begin{screen}
		\begin{dfn}[空集合]
			$\emptyset \defeq \Set{x}{x \neq x}$で定める類$\emptyset$を{\bf 空集合}\index{くうしゅうごう@空集合}{\bf (empty set)}と呼ぶ.
		\end{dfn}
	\end{screen}
	
	\begin{screen}
		\begin{axm}[置換公理]
			\begin{align}
				\forall a\, \left[\, \forall x \in a\, \exists!y \varphi(x,y)
				\Longrightarrow \exists z\, \forall y\,
				(\, y \in z \Longleftrightarrow \exists x\, (\, x \in a \wedge 
				\varphi(x,y)\, )\, )\, \right].
			\end{align}
		\end{axm}
	\end{screen}
	
	$\Set{x}{\varphi(x)}$は集合であるとは限らないが,
	集合$a$に対して
	\begin{align}
		a \cap \Set{x}{\varphi(x)}
	\end{align}
	なる類は当然$a$より``小さい集まり''なのだから,集合であってほしいものである.
	置換公理によってそのこと保証され,分出定理として知られている.
	
	\begin{screen}
		\begin{thm}[分出定理]
			\begin{align}
				\forall a\, \exists s\, \forall x\,
				(\, x \in s \Longleftrightarrow x \in a \wedge \varphi(x)\, ).
			\end{align}
		\end{thm}
	\end{screen}
	
	\begin{sketch}
		$x$と$y$が自由に現れる式$\psi(x,y)$を
		\begin{align}
			x = y \wedge \varphi(x)
		\end{align}
		と設定すると,これは
		\begin{align}
			\forall x \in a\, \exists!y \psi(x,y)
		\end{align}
		を満たすので,置換公理より
		\begin{align}
			\forall y\, (\, y \in z \Longleftrightarrow \exists x\, (\, x \in a \wedge 
			\psi(x,y)\, )\, )
		\end{align}
		を満たす集合$z$が取れる.このとき
		\begin{align}
			\Set{y}{y \in a \wedge \varphi(y)} = z
		\end{align}
		が成立する.実際,
		\begin{align}
			y \in z
		\end{align}
		ならば
		\begin{align}
			x \in a \wedge x = y \wedge \varphi(x)
		\end{align}
		を満たす$x$が取れるが,このとき相等性から
		\begin{align}
			y \in a \wedge \varphi(y)
		\end{align}
		が成立する.逆に
		\begin{align}
			y \in a \wedge \varphi(y)
		\end{align}
		であれば
		\begin{align}
			x \defeq y
		\end{align}
		によって
		\begin{align}
			\exists x\, (\, x \in a \wedge \psi(x,y)\, )
		\end{align}
		が成り立つので
		\begin{align}
			y \in z
		\end{align}
		となる.
		\QED
	\end{sketch}
	
	\begin{screen}
		\begin{thm}[$\emptyset$は集合]\label{thm:emptyset_is_a_set}
			$\emptyset$は集合である:
			\begin{align}
				\set{\emptyset}.
			\end{align}
		\end{thm}
	\end{screen}
	
	\begin{sketch}
		分出定理より
		\begin{align}
			\forall z\, \exists y\, \forall x\,
			(\, x \in y \Longleftrightarrow x \in z \wedge x \neq x\, )
			\label{fom:thm_emptyset_is_a_set_1}
		\end{align}
		が成立するが,この式から
		\begin{align}
			\exists y\, \forall x\, (\, x \in y \Longleftrightarrow x \neq x\, )
			\label{fom:thm_emptyset_is_a_set_2}
		\end{align}
		を示せる.これはすなわち$\emptyset$が集合であるということを示唆する.
		$\zeta$を勝手な$\varepsilon$項として,後々の便宜のために
		\begin{align}
			\sigma &\defeq \varepsilon y\, \forall x\,
			(\, x \in y \Longleftrightarrow x \in \zeta \wedge x \neq x\, ), \\
			\tau &\defeq \varepsilon x \rightharpoondown
			(\, x \in \sigma \Longleftrightarrow x \neq x\, )
		\end{align}
		とおけば,(\refeq{fom:thm_emptyset_is_a_set_1})より
		\begin{align}
			\tau \in \sigma \Longleftrightarrow \tau \in \zeta \wedge \tau \neq \tau
		\end{align}
		が成立する.論理和の規則より
		\begin{align}
			\tau \in \zeta \wedge \tau \neq \tau \Longrightarrow \tau \neq \tau
		\end{align}
		が満たされるので,まずは
		\begin{align}
			\tau \in \sigma \Longrightarrow \tau \neq \tau
		\end{align}
		が得られる.また
		\begin{align}
			\tau = \tau
		\end{align}
		は正しいので,
		\begin{align}
			\tau = \tau \Longrightarrow (\, \tau \notin \sigma \Longrightarrow
			\tau = \tau\, )
		\end{align}
		と併せて
		\begin{align}
			\tau \notin \sigma \Longrightarrow \tau = \tau
		\end{align}
		が成り立ち,対偶を取れば
		\begin{align}
			\tau \neq \tau \Longrightarrow \tau \in \sigma
		\end{align}
		も得られる.ゆえに
		\begin{align}
			\forall x\, (\, x \in \sigma \Longleftrightarrow x \neq x\, )
		\end{align}
		が得られ,(\refeq{fom:thm_emptyset_is_a_set_2})が従う.
		\QED
	\end{sketch}
	
	\begin{screen}
		\begin{thm}[空集合は$\mathcal{L}$のいかなる対象も要素に持たない]\label{thm:emptyset_has_nothing}
			\begin{align}
				\forall x\, (\, x \notin \emptyset\, ).
			\end{align}
		\end{thm}
	\end{screen}
	
	\begin{sketch}
		$\tau$を$\mathscr{L}$の対象とするとき,類の公理より
		\begin{align}
			\tau \in \emptyset \Longrightarrow \tau \neq \tau
		\end{align}
		が成り立つから,対偶を取れば
		\begin{align}
			\tau = \tau \Longrightarrow \tau \notin \emptyset
		\end{align}
		が成り立つ(推論法則\ref{thm:contraposition_is_true}).定理\ref{thm:any_class_equals_to_itself}より
		\begin{align}
			\tau = \tau
		\end{align}
		は正しいので,三段論法より
		\begin{align}
			\tau \notin \emptyset
		\end{align}
		が成り立つ.そして$\tau$の任意性より
		\begin{align}
			\forall x\, (\, x \notin \emptyset\, )
		\end{align}
		が得られる.
		\QED
	\end{sketch}
	
	\begin{screen}
		\begin{thm}[$\mathcal{L}$のいかなる対象も要素に持たない類は空集合に等しい]
		\label{thm:uniqueness_of_emptyset}
			$a$を類とするとき次が成り立つ:
			\begin{align}
				\forall x\, (\, x \notin a\, ) \Longleftrightarrow a = \emptyset.
			\end{align}
		\end{thm}
	\end{screen}
	
	\begin{prf}
		$a$を類として$\forall x\, (\, x \notin a\, )$が成り立っていると仮定する.このとき
		$\tau$を$\mathcal{L}$の任意の対象とすれば
		\begin{align}
			\tau \notin a \vee \tau \in \emptyset
		\end{align}
		と
		\begin{align}
			\tau \notin \emptyset \vee \tau \in a
		\end{align}
		が共に成り立つので,推論法則\ref{logicalthm:rule_of_inference_3}より
		\begin{align}
			\tau \in a \Longrightarrow \tau \in \emptyset
		\end{align}
		と
		\begin{align}
			\tau \in \emptyset \Longrightarrow \tau \in a
		\end{align}
		が共に成り立つ.よって
		\begin{align}
			\tau \in a \Longleftrightarrow \tau \in \emptyset
		\end{align}
		が成立し,$\tau$の任意性と推論法則\ref{logicalthm:fundamental_law_of_universal_quantifier}から
		\begin{align}
			\forall x\, (\, x \in a \Longleftrightarrow x \in \emptyset\, )
		\end{align}
		が得られる.ゆえに外延性の公理より
		\begin{align}
			a = \emptyset
		\end{align}
		が成立し,演繹法則より
		\begin{align}
			\forall x\, (\, x \notin a\, ) \Longrightarrow a = \emptyset
		\end{align}
		が得られる.逆に
		\begin{align}
			a = \emptyset
		\end{align}
		が成り立っていると仮定する.ここで$\chi$を$\mathcal{L}$の任意の対象とすれば,
		相等性の公理より
		\begin{align}
			\chi \in a \Longrightarrow \chi \in \emptyset
		\end{align}
		が成立するので,対偶を取れば
		\begin{align}
			\chi \notin \emptyset \Longrightarrow \chi \notin a
		\end{align}
		が成り立つ.定理\ref{thm:emptyset_has_nothing}より
		\begin{align}
			\chi \notin \emptyset
		\end{align}
		が満たされているので,三段論法より
		\begin{align}
			\chi \notin a
		\end{align}
		が成立し,$\chi$の任意性と推論法則\ref{logicalthm:fundamental_law_of_universal_quantifier}より
		\begin{align}
			\forall x\, (\, x \notin a\, )
		\end{align}
		が成立する.ここに演繹法則を適用して
		\begin{align}
			a = \emptyset \Longrightarrow \forall x\, (\, x \notin a\, )
		\end{align}
		も得られる.
		\QED
	\end{prf}
	
	\begin{screen}
		\begin{thm}[空集合はいかなる類も要素に持たない]
		\label{thm:emptyset_does_not_contain_any_class}
			$a,b$を類とするとき次が成り立つ:
			\begin{align}
				b = \emptyset \Longrightarrow a \notin b.
			\end{align}
		\end{thm}
	\end{screen}
	
	\begin{prf}
		いま$a \in b$が成り立っていると仮定する.このとき要素の公理と三段論法より
		\begin{align}
			\set{a}
		\end{align}
		が成立する.ここで
		\begin{align}
			\tau \defeq \varepsilon x\, (\, a = x\, )
		\end{align}
		とおけば,存在記号に関する規則から
		\begin{align}
			a = \tau
		\end{align}
		が成り立つので,相等性の公理より
		\begin{align}
			\tau \in b
		\end{align}
		が従い,存在記号に関する規則より
		\begin{align}
			\exists x\, (\, x \in b\, )
		\end{align}
		が成り立つ.よって演繹法則から
		\begin{align}
			a \in b \Longrightarrow \exists x\, (\, x \in b\, )
		\end{align}
		が成り立つ.この対偶を取り推論法則\ref{logicalthm:De_Morgan_law_for_quantifiers}を適用すれば
		\begin{align}
			\forall x\, (\, x \notin b\, ) \Longrightarrow a \notin b
		\end{align}
		が得られる.定理\ref{thm:uniqueness_of_emptyset}より
		\begin{align}
			b = \emptyset \Longrightarrow \forall x\, (\, x \notin b\, )
		\end{align}
		も正しいので,含意の推移律から
		\begin{align}
			b = \emptyset \Longrightarrow a \notin b
		\end{align}
		が得られる.
		\QED
	\end{prf}
	
	\begin{screen}
		\begin{dfn}[部分類]
			$a,b$を$\mathcal{L}'$の項とするとき,
			\begin{align}
				a \subset b \overset{\mathrm{def}}{\Longleftrightarrow}
				\forall x\ (\ x \in a \Longrightarrow x \in b\ )
			\end{align}
			と定める.式$a \subset b$を``$a$は$b$の{\bf 部分類}\index{ぶぶんるい@部分類}{\bf (subclass)}である''
			と翻訳し,特に$a$が集合である場合は``$a$は$b$の{\bf 部分集合}\index{ぶぶんしゅうごう@部分集合}{\bf (subset)}である''と翻訳する.
			また次の記号も定める:
			\begin{align}
				a \subsetneq b \defarrow a \subset b \wedge a \neq b.
			\end{align}
		\end{dfn}
	\end{screen}
	
	空虚な真の一例として次の結果を得る.
	
	\begin{screen}
		\begin{thm}[空集合は全ての類に含まれる]\label{thm:emptyset_if_a_subset_of_every_class}
			$a$を類とするとき次が成り立つ:
			\begin{align}
				\emptyset \subset a.
			\end{align}
		\end{thm}
	\end{screen}
	
	\begin{prf}
		$a$を類とする.$\tau$を$\mathcal{L}$の任意の対象とすれば
		\begin{align}
			\tau \notin \emptyset
		\end{align}
		が成り立つから,推論規則\ref{logicalaxm:fundamental_rules_of_inference}を適用して
		\begin{align}
			\tau \notin \emptyset \vee \tau \in a
		\end{align}
		が成り立つ.従って
		\begin{align}
			\tau \in \emptyset \Longrightarrow \tau \in a
		\end{align}
		が成り立ち,$\tau$の任意性と推論法則\ref{logicalthm:fundamental_law_of_universal_quantifier}より
		\begin{align}
			\forall x\, (\, x \in \emptyset \Longrightarrow x \in a\, )
		\end{align}
		が成立する.
		\QED
	\end{prf}
	
	$a \subset b$とは$a$に属する全ての``$\mathcal{L}$の対象''は$b$に属するという定義であったが,
	要素となりうる類は集合であるという公理から,$a$に属する全ての``類''もまた$b$に属する.
	
	\begin{screen}
		\begin{thm}[類はその部分類に属する全ての類を要素に持つ]\label{thm:subclass_contains_all_elements}
			$a,b,c$を類とすれば次が成り立つ:
			\begin{align}
				a \subset b \Longrightarrow (\, c \in a \Longrightarrow c \in b\, ).
			\end{align}
		\end{thm}
	\end{screen}
	
	\begin{prf}	
		いま$a \subset b$が成り立っているとする.このとき
		\begin{align}
			c \in a
		\end{align}
		が成り立っていると仮定すれば,要素の公理より
		\begin{align}
			\set{c}
		\end{align}
		が成り立つ.ここで
		\begin{align}
			\tau \defeq \varepsilon x\, (\, c=x\, )
		\end{align}
		とおくと
		\begin{align}
			c = \tau
		\end{align}
		が成り立つので,相等性の公理より
		\begin{align}
			\tau \in a
		\end{align}
		が成り立ち,$a \subset b$と推論法則\ref{logicalthm:fundamental_law_of_universal_quantifier}から
		\begin{align}
			\tau \in b
		\end{align}
		が従う.再び相等性の公理を適用すれば
		\begin{align}
			c \in b
		\end{align}
		が成り立つので,演繹法則より,$a \subset b$が成り立っている下で
		\begin{align}
			c \in a \Longrightarrow c \in b
		\end{align}
		が成立する.再び演繹法則を適用すれば定理の主張が得られる.
		\QED
	\end{prf}
	
	宇宙$\Univ$は類の一つであった.当然のようであるが,それは最大の類である.
	\begin{screen}
		\begin{thm}[$\Univ$は最大の類である]
			$a$を類とするとき次が成り立つ:
			\begin{align}
				a \subset \Univ.
			\end{align}
		\end{thm}
	\end{screen}
	
	\begin{prf}
		$\tau$を$\mathcal{L}$の任意の対象とすれば,定理\ref{thm:any_class_equals_to_itself}と類の公理より
		\begin{align}
			\tau \in \Univ
		\end{align}
		が成立するので,推論規則\ref{logicalaxm:fundamental_rules_of_inference}より
		\begin{align}
			\tau \notin a \vee \tau \in \Univ
		\end{align}
		が成立する.このとき推論法則\ref{logicalthm:rule_of_inference_3}より
		\begin{align}
			\tau \in a \Longrightarrow \tau \in \Univ
		\end{align}
		が成立し,$\tau$の任意性と推論法則\ref{logicalthm:fundamental_law_of_universal_quantifier}から
		\begin{align}
			\forall x\, (\, x \in a \Longrightarrow x \in \Univ\, )
		\end{align}
		が従う.
		\QED
	\end{prf}
	
	\begin{screen}
		\begin{thm}[互いに互いの部分類となる類同士は等しい]\label{thm:mutually_sub_classes_are_equivalent}
			$a,b$を類とするとき次が成り立つ:
			\begin{align}
				a \subset b \wedge b \subset a \Longleftrightarrow a = b.
			\end{align}
		\end{thm}
	\end{screen}
	
	\begin{sketch}
		$a \subset b \wedge b \subset a$が成り立っていると仮定する.
		このとき$\tau$を$\mathcal{L}$の任意の対象とすれば,
		$a \subset b$と推論法則\ref{logicalthm:fundamental_law_of_universal_quantifier}より
		\begin{align}
			\tau \in a \Longrightarrow \tau \in b
		\end{align}
		が成立し,$b \subset a$と推論法則\ref{logicalthm:fundamental_law_of_universal_quantifier}より
		\begin{align}
			\tau \in b \Longrightarrow \tau \in a
		\end{align}
		が成立するので,
		\begin{align}
			\tau \in a \Longleftrightarrow \tau \in b
		\end{align}
		が成り立つ.$\tau$の任意性と推論法則\ref{logicalthm:fundamental_law_of_universal_quantifier}および
		外延性の公理より
		\begin{align}
			a = b
		\end{align}
		が出るので,演繹法則より
		\begin{align}
			a \subset b \wedge b \subset a \Longrightarrow a = b
		\end{align}
		が得られる.逆に$a = b$が満たされていると仮定するとき,$\tau$を$\mathcal{L}$の任意の対象とすれば
		\begin{align}
			\tau \in a \Longrightarrow \tau \in b
		\end{align}
		と
		\begin{align}
			\tau \in b \Longrightarrow \tau \in a
		\end{align}
		が共に成り立つ. よって推論法則\ref{logicalthm:fundamental_law_of_universal_quantifier}より
		\begin{align}
			a \subset b
		\end{align}
		と
		\begin{align}
			b \subset a
		\end{align}
		が共に従う.よって演繹法則より
		\begin{align}
			a = b \Longrightarrow a \subset b \wedge b \subset a
		\end{align}
		も得られる.
		\QED
	\end{sketch}
	
	\monologue{
		定理\ref{thm:subclass_contains_all_elements}と定理\ref{thm:mutually_sub_classes_are_equivalent}より,
			類$a,b$が$a = b$を満たすならば,$a$と$b$は要素に持つ$\mathcal{L}$の対象のみならず,
			要素に持つ類までも一致するのですね.
	}
	
\section{順序型について}
	$(A,R)$を整列集合とするとき,
	\begin{align}
		x \longmapsto 
		\begin{cases}
			\min{A \backslash \ran{x}} & \mbox{if } \ran{x} \subsetneq A \\
			A & \mbox{o.w.} \\
		\end{cases}
	\end{align}
	なる写像$G$に対して
	\begin{align}
		\forall \alpha\, F(\alpha) = G(\rest{F}{\alpha})
	\end{align}
	なる写像$F$を取り
	\begin{align}
		\alpha \defeq \min{\Set{\alpha \in \ON}{F(\alpha) = A}}
	\end{align}
	とおけば,$\alpha$は$(A,R)$の順序型.
	
\section{超限再帰について}
	$\Univ$上の写像$G$が与えられたら,
	\begin{align}
		F \defeq \Set{(\alpha,x)}{\ord{\alpha} \wedge
		\exists f\, \left(\, f \fon \alpha \wedge
		\forall \beta \in \alpha\, \left(\, f(\beta) = G(\rest{f}{\beta})\, \right)
		\wedge x = G(f)\, \right)}
	\end{align}
	により$F$を定めれば
	\begin{align}
		\forall \alpha\, F(\alpha) = G(\rest{F}{\alpha})
	\end{align}
	が成立する.
	
	\begin{screen}
		任意の順序数$\alpha$および$\alpha$上の写像$f$と$g$に対して,
		\begin{align}
			\forall \beta \in \alpha\,
			\left(\, f(\beta) = G(\rest{f}{\beta})\, \right)
		\end{align}
		かつ
		\begin{align}
			\forall \beta \in \alpha\,
			\left(\, g(\beta) = G(\rest{g}{\beta})\, \right)
		\end{align}
		ならば$f = g$である.
	\end{screen}
	
	まず
	\begin{align}
		f(0) = G(\rest{f}{0}) = G(0) = G(\rest{g}{0}) = g(0)
	\end{align}
	が成り立つ.また
	\begin{align}
		\forall \delta \in \beta\, \left(\, 
		\delta \in \alpha \Longrightarrow f(\delta) = g(\delta)\, \right)
	\end{align}
	ならば,$\beta \in \alpha$であるとき
	\begin{align}
		\rest{f}{\beta} = \rest{g}{\beta}
	\end{align}
	となるので
	\begin{align}
		\beta \in \alpha \Longrightarrow f(\beta) = g(\beta)
	\end{align}
	が成り立つ.ゆえに
	\begin{align}
		f = g
	\end{align}
	が得られる.
	
	\begin{screen}
		任意の順序数$\alpha$に対して,$\alpha$上の写像$f$で
		\begin{align}
			\forall \beta \in \alpha\, \left(\, 
			f(\beta) = G(\rest{f}{\beta})\, \right)
		\end{align}
		を満たすものが取れる.
	\end{screen}
	
	$\alpha = 0$のとき$f \defeq 0$とすればよい.$\alpha$の任意の要素$\beta$に対して
	\begin{align}
		g \fon \beta \wedge \forall \gamma\in \beta\, \left(\, 
		g(\gamma) = G(\rest{g}{\gamma})\, \right)
	\end{align}
	なる$g$が存在するとき,
	\begin{align}
		f \defeq \Set{(\beta,x)}{\beta \in \alpha \wedge
		\exists g\, \left(\, g \fon \beta \wedge
		\forall \gamma \in \beta\, \left(\, g(\gamma) = G(\rest{g}{\gamma})\, \right)
		\wedge x = G(g)\, \right)}
	\end{align}
	と定めれば,$f$は$\alpha$上の写像であって
	\begin{align}
		\forall \beta \in \alpha\, \left(\, 
		f(\beta) = G(\rest{f}{\beta})\, \right)
	\end{align}
	を満たす.
	
	\begin{screen}
		任意の順序数$\alpha$に対して$F(\alpha) = G(\rest{F}{\alpha})$が成り立つ.
	\end{screen}
	
	$\alpha = 0$ならば,$0$上の写像は$0$のみなので
	\begin{align}
		F(0) = G(0) = G(\rest{F}{0})
	\end{align}
	である.
	\begin{align}
		\forall \beta \in \alpha\, F(\beta) = G(\rest{F}{\beta})
	\end{align}
	が成り立っているとき,
	\begin{align}
		\forall \beta \in \alpha\, f(\beta) = G(\rest{f}{\beta})
	\end{align}
	を満たす$\alpha$上の写像$f$を取れば,前の一意性より
	\begin{align}
		f = \rest{F}{\alpha}
	\end{align}
	が成立する.よって
	\begin{align}
		F(\alpha) = G(f) = G(\rest{F}{\alpha})
	\end{align}
	となる.
	\QED
	
\section{自然数の全体について}
	$\Natural$を
	\begin{align}
		\Natural \defeq \Set{\beta}{\mbox{$\alpha \leq \beta$である$\alpha$は
		$0$であるか後続型順序数}}
	\end{align}
	によって定めれば,無限公理より
	\begin{align}
		\set{\Natural}
	\end{align}
	である.また$\ord{\Natural}$と$\limo{\Natural}$も証明できるはず.
	$\Natural$が最小の極限数であることは$\Natural$を定義した論理式より従う.
\section{推論}
	いま$x,y$を$\mathcal{L}$の項とするとき,
	\begin{align}
		x \notin y \defarrow\ \negation x \in y
	\end{align}
	で$x \notin y$を定める.同様に
	\begin{align}
		x \neq y \defarrow\ \negation x = y
	\end{align}
	で$x \neq y$を定める.
	
	定義記号$\defeq$と同様に,`$A \defarrow B$'とは
	式$B$を記号列$A$で置き換えて良いという意味で使われる.また,式中に記号列$A$が出てくるときは,
	暗黙裡にその$A$を$B$に戻して式を読む.$\defeq$も$\defarrow$も略記法を定めるための記号である.
	
	ここで論理記号の名称を書いておく.
	\begin{itemize}
		\item $\bot$を{\bf 矛盾}\index{むじゅん@矛盾}{\bf (contradiction)}と呼ぶ.
		\item $\vee$を{\bf 論理和}\index{ろんりわ@論理和}{\bf (logical disjunction)}と呼ぶ.
		\item $\wedge$を{\bf 論理積}\index{ろんりせき@論理積}{\bf (logical conjunction)}と呼ぶ.
		\item $\rarrow$を{\bf 含意}\index{がんい@含意}{\bf (implication)}と呼ぶ.
		\item $\negation$を{\bf 否定}\index{ひてい@否定}{\bf (negation)}と呼ぶ.
	\end{itemize}
	
	\begin{screen}
		\begin{logicalthm}[論理和・論理積の可換律]
		\label{logicalthm:commutative_law_of_disjunction_and_conjunction}
			$A,B$を文とするとき
			\begin{itemize}
				\item $\vdash (A \vee B) \rarrow (B \vee A)$.
				\item $\vdash (A \wedge B) \rarrow (B \wedge A)$.
			\end{itemize}
		\end{logicalthm}
	\end{screen}
	
	\begin{prf}
		$\vee$の導入により
		\begin{align}
			\vdash A \rarrow (B \vee A)
		\end{align}
		と
		\begin{align}
			\vdash B \rarrow (A \vee B)
		\end{align}
		が成り立つので,場合分け法則より
		\begin{align}
			\vdash (A \vee B) \rarrow (B \vee A)
		\end{align}
		が成り立つ.また,$\wedge$の除去より
		\begin{align}
			A \wedge B \vdash A
		\end{align}
		と
		\begin{align}
			A \wedge B \vdash B
		\end{align}
		となるので,$\wedge$の導入により
		\begin{align}
			A \wedge B \vdash B \wedge A
		\end{align}
		が成り立つ.よって演繹法則より
		\begin{align}
			\vdash (A \wedge B) \rarrow (B \wedge A)
		\end{align}
		が成り立つ.
		\QED
	\end{prf}
	
	\begin{screen}
		\begin{logicalthm}[含意の推移律]\label{logicalthm:transitive_law_of_implication}
			$A,B,C$を文とするとき
			\begin{align}
				\vdash ((A \rarrow B) \wedge (B \rarrow C)) 
				\rarrow (A \rarrow C).
			\end{align}
		\end{logicalthm}
	\end{screen}
	
	\begin{prf}
		\begin{align}
			(A \rarrow B) \wedge (B \rarrow C),A \vdash 
			(A \rarrow B) \wedge (B \rarrow C)
		\end{align}
		であるから,$\wedge$の除去より
		\begin{align}
			(A \rarrow B) \wedge (B \rarrow C),A \vdash A \rarrow B
		\end{align}
		となる.また
		\begin{align}
			(A \rarrow B) \wedge (B \rarrow C),A \vdash A
		\end{align}
		でもあるから,三段論法より
		\begin{align}
			(A \rarrow B) \wedge (B \rarrow C),A \vdash B
		\end{align}
		となる.$\wedge$の除去より
		\begin{align}
			(A \rarrow B) \wedge (B \rarrow C),A \vdash B \rarrow C
		\end{align}
		も成り立つから,再び三段論法より
		\begin{align}
			(A \rarrow B) \wedge (B \rarrow C),A \vdash C
		\end{align}
		となる.よって演繹法則より
		\begin{align}
			(A \rarrow B) \wedge (B \rarrow C) \vdash A \rarrow C
		\end{align}
		となり,
		\begin{align}
			\vdash ((A \rarrow B) \wedge (B \rarrow C)) 
			\rarrow (A \rarrow C)
		\end{align}
		を得る.
		\QED
	\end{prf}
	
	\begin{screen}
		\begin{logicalthm}[二式が同時に導かれるならその論理積が導かれる]
		\label{logicalthm:conjunction_of_consequences}
			$A,B,C$を文とするとき
			\begin{align}
				\vdash ((A \rarrow B) \wedge (A \rarrow C))
				\rarrow (A \rarrow (B \wedge C))
			\end{align}
		\end{logicalthm}
	\end{screen}
	
	\begin{prf}
		\begin{align}
			(A \rarrow B) \wedge (A \rarrow C),A \vdash
			(A \rarrow B) \wedge (A \rarrow C)
		\end{align}
		であるから,$\wedge$の除去より
		\begin{align}
			(A \rarrow B) \wedge (A \rarrow C),A \vdash
			A \rarrow B
		\end{align}
		が成り立つ.
		\begin{align}
			(A \rarrow B) \wedge (A \rarrow C),A \vdash A
		\end{align}
		でもあるから
		\begin{align}
			(A \rarrow B) \wedge (A \rarrow C),A \vdash B
		\end{align}
		となる.同様にして
		\begin{align}
			(A \rarrow B) \wedge (A \rarrow C),A \vdash C
		\end{align}
		となるので,$\wedge$の導入により
		\begin{align}
			(A \rarrow B) \wedge (A \rarrow C),A \vdash B \wedge C
		\end{align}
		となり,演繹法則より
		\begin{align}
			(A \rarrow B) \wedge (A \rarrow C) \vdash
			A \rarrow (B \wedge C)
		\end{align}
		が成り立つ.ゆえに
		\begin{align}
			\vdash ((A \rarrow B) \wedge (A \rarrow C))
			\rarrow (A \rarrow (B \wedge C))
		\end{align}
		が得られる.
		\QED
	\end{prf}
	
	\begin{screen}
		\begin{logicalthm}[含意は遺伝する]\label{logicalthm:rule_of_inference_1}
			$A,B,C$を$\mathcal{L}'$の閉式とするとき以下が成り立つ:
			\begin{description}
				\item[(a)] $(A \rarrow B) \rarrow ( (A \vee C) \rarrow (B \vee C) )$.
				
				\item[(b)] $(A \rarrow B) \rarrow ( (A \wedge C) \rarrow (B \wedge C) )$.
				
				\item[(c)] $(A \rarrow B) \rarrow ( (B \rarrow C) \rarrow (A \rarrow C) )$.
				
				\item[(c)] $(A \rarrow B) \rarrow ( (C \rarrow A) \rarrow (C \rarrow B) )$.
			\end{description}
		\end{logicalthm}
	\end{screen}
	
	\begin{prf}\mbox{}
		\begin{description}
			\item[(a)]
				いま$A \rarrow B$が成り立っていると仮定する.
				論理和の導入により
				\begin{align}
					C \rarrow (B \vee C)
				\end{align}
				は定理であるから,含意の推移律より
				\begin{align}
					A \rarrow (B \vee C)
				\end{align}
				が従い,場合分け法則より
				\begin{align}
					(A \vee C) \rarrow (B \vee C)
				\end{align}
				が成立する.ここに演繹法則を適用して
				\begin{align}
					(A \rarrow B) \rarrow 
					( (A \vee C) \rarrow (B \vee C) )
				\end{align}
				が得られる.
				
			\item[(b)]
				いま$A \rarrow B$が成り立っていると仮定する.論理積の除去より
				\begin{align}
					(A \wedge C) \rarrow A
				\end{align}
				は定理であるから,含意の推移律より
				\begin{align}
					(A \wedge C) \rarrow B
				\end{align}
				が従い,他方で論理積の除去より
				\begin{align}
					(A \wedge C) \rarrow C
				\end{align}
				も満たされる.そして推論法則\ref{logicalthm:conjunction_of_consequences}から
				\begin{align}
					(A \wedge C) \rarrow (B \wedge C)
				\end{align}
				が成り立ち,演繹法則より
				\begin{align}
					(A \rarrow B) \rarrow ((A \wedge C) \rarrow (B \wedge C))
				\end{align}
				が得られる.
				
			\item[(c)]
				いま$A \rarrow B$,$B \rarrow C$および
				$A$が成り立っていると仮定する.このとき三段論法より$B$が成り立つので再び三段論法より
				$C$が成立する.ゆえに演繹法則より$A \rarrow B$と$B \rarrow C$が
				成り立っている下で
				\begin{align}
					A \rarrow C
				\end{align}
				が成立し,演繹法則を更に順次適用すれば
				\begin{align}
					(A \rarrow B) \rarrow ( (B \rarrow C) \rarrow (A \rarrow C) )
				\end{align}
				が得られる.
				
			\item[(d)]
				いま$A \rarrow B$,$C \rarrow A$および
				$C$が成り立っていると仮定する.このとき三段論法より$A$が成り立つので再び三段論法より$B$が成立し,
				ここに演繹法則を適用すれば,$A \rarrow B$と$C \rarrow A$が成立している下で
				\begin{align}
					C \rarrow B
				\end{align}
				が成立する.演繹法則を更に順次適用すれば
				\begin{align}
					(A \rarrow B) \rarrow ( (C \rarrow A) \rarrow (C \rarrow B) )
				\end{align}
				が得られる.
				\QED
		\end{description}
	\end{prf}
	
	$A$と$B$を$\mathcal{L}'$の式とするとき,
	\begin{align}
		(A \lrarrow B) \defarrow
		(A \rarrow B \wedge B \rarrow A)
	\end{align}
	により$\lrarrow$を定め,式`$A \lrarrow B$'を
	``$A$と$B$は{\bf 同値である}\index{どうち@同値}{\bf (equivalent)}''と翻訳する.
	
	\begin{screen}
		\begin{logicalthm}[同値記号の可換律]\label{logicalthm:commutative_law_of_equivalence}
			$A$と$B$を$\mathcal{L}'$の閉式とするとき
			\begin{align}
				(A \lrarrow B) \rarrow (B \lrarrow A).
			\end{align}
		\end{logicalthm}
	\end{screen}
	
	\begin{sketch}
		$A \lrarrow B$が成り立っているならば,推論法則\ref{logicalthm:commutative_law_of_disjunction_and_conjunction}より
		\begin{align}
			B \rarrow A \wedge A \rarrow B
		\end{align}
		が成立する.すなわち
		\begin{align}
			B \lrarrow A
		\end{align}
		が成立する.そして演繹法則から
		\begin{align}
			(A \lrarrow B) \rarrow (B \lrarrow A)
		\end{align}
		が成立する.
		\QED
	\end{sketch}
	
	\begin{screen}
		\begin{logicalthm}[同値記号の遺伝性質]\label{logicalthm:hereditary_of_equivalence}
			$A,B,C$を$\mathcal{L}'$の閉式とするとき以下の式が成り立つ:
			\begin{description}
				\item[(a)] $(A \lrarrow B) \rarrow ((A \vee C) \lrarrow (B \vee C))$.
				\item[(b)] $(A \lrarrow B) \rarrow ((A \wedge C) \lrarrow (B \wedge C))$.
				\item[(c)] $(A \lrarrow B) \rarrow ((B \rarrow C) \lrarrow (A \rarrow C))$.
				
				\item[(d)] $(A \lrarrow B) \rarrow ((C \rarrow A) \lrarrow (C \rarrow B))$.
			\end{description}
		\end{logicalthm}
	\end{screen}
	
	\begin{prf}
		まず(a)を示す.いま$A \lrarrow B$が成り立っていると仮定する.このとき$A \rarrow B$と
		$B \rarrow A$が共に成立し,他方で含意の遺伝性質より
		\begin{align}
			&(A \rarrow B) \rarrow ((A \vee C) \rarrow (B \vee C)), \\
			&(B \rarrow A) \rarrow ((B \vee C) \rarrow (A \vee C))
		\end{align}
		が成立するから三段論法より$(A \vee C) \rarrow (B \vee C)$と
		$(B \vee C) \rarrow (A \vee C)$が共に成立する.ここに$\wedge$の導入を適用すれば
		\begin{align}
			(A \vee C) \lrarrow (B \vee C)
		\end{align}
		が成立し,演繹法則を適用すれば
		\begin{align}
			(A \lrarrow B) \rarrow ((A \vee C) \lrarrow (B \vee C))
		\end{align}
		が得られる.(b)(c)(d)も含意の遺伝性を適用すれば得られる.
		\QED
	\end{prf}
	
	\begin{screen}
		\begin{logicalaxm}[矛盾と否定に関する規則]\label{logicalaxm:rules_of_contradiction}
			$A$を文とするとき以下が成り立つ:
			\begin{description}
				\item[矛盾の発生] 否定が共に成り立つとき矛盾が起きる:
					\begin{align}
						A, \negation A \vdash \bot.
					\end{align}
				\item[否定の導出] 矛盾が導かれるとき否定が成り立つ:
					\begin{align}
						A \rarrow \bot \vdash \negation A.
					\end{align}
				\item[二重否定の法則] 二重に否定された式は元の式を導く:
					\begin{align}
						\negation \negation A \vdash A.
					\end{align}
			\end{description}
		\end{logicalaxm}
	\end{screen}
	
	文$A$が$\mathscr{S} \vdash \negation A$を満たすとき,
	$A$は公理系$\mathscr{S}$において{\bf 偽である}\index{ぎ@偽}{\bf (false)}という.
	
	\begin{screen}
		\begin{logicalthm}[偽な式は矛盾を導く]\label{logicalthm:false_and_negation_are_equivalent}
			$A$を文とするとき
			\begin{align}
				\vdash \negation A \rarrow (A \rarrow \bot).
			\end{align}
		\end{logicalthm}
	\end{screen}
	
	\begin{prf}
		矛盾の規則より
		\begin{align}
			A,\negation A \vdash \bot
		\end{align}
		である.演繹法則より
		\begin{align}
			\negation A \vdash A \rarrow \bot
		\end{align}
		が成り立ち,再び演繹法則より
		\begin{align}
			\vdash \negation A \rarrow (A \rarrow \bot)
		\end{align}
		が得られる.
		\QED
	\end{prf}
	
	$A,B \vdash C$と$A \wedge B \vdash C$は同値である.$A,B \vdash C$が成り立っているとすると,
	$\vdash A \rarrow (B \rarrow C)$が成り立つ.
	$A \wedge B \vdash A$と三段論法より
	$A \wedge B \vdash B \rarrow C$が成り立ち,
	$A \wedge B \vdash B$と三段論法より
	$A \wedge B \vdash C$が成り立つ.逆に$A \wedge B \vdash C$が成り立っているとすると
	$\vdash (A \wedge B) \rarrow C$が成り立つ.
	$A,B \vdash A \wedge B$と三段論法より
	$A, B \vdash C$が成り立つ.
	
	\begin{screen}
		\begin{logicalthm}[背理法の原理]
			$A$を文とするとき
			\begin{align}
				\vdash (\negation A \rarrow \bot) \rarrow A.
			\end{align}
		\end{logicalthm}
	\end{screen}
	
	\begin{prf}
		否定の導出より
		\begin{align}
			\vdash (\negation A \rarrow \bot) \rarrow \negation \negation A
		\end{align}
		が成り立ち,二重否定の法則より
		\begin{align}
			\vdash \negation \negation A \rarrow A
		\end{align}
		が成り立つので,$\wedge$の導入より
		\begin{align}
			\vdash ((\negation A \rarrow \bot) \rarrow \negation \negation A)
			\wedge (\negation \negation A \rarrow A)
		\end{align}
		が成り立つ.含意の推移律(推論法則\ref{logicalthm:transitive_law_of_implication})より
		\begin{align}
			\vdash ((\negation A \rarrow \bot) \rarrow \negation \negation A)
			\wedge (\negation \negation A \rarrow A)
			\rarrow ((\negation A \rarrow \bot) \rarrow A)
		\end{align}
		が成り立つので,三段論法より
		\begin{align}
			\vdash (\negation A \rarrow \bot) \rarrow A
		\end{align}
		が得られる.
		\QED
	\end{prf}
	
	\begin{screen}
		\begin{logicalthm}[排中律]\label{logicalthm:law_of_excluded_middle}
			$A$を文とするとき
			\begin{align}
				\vdash A \vee \negation A.
			\end{align}
		\end{logicalthm}
	\end{screen}
	
	\begin{prf}
		\begin{align}
			\negation (A \vee \negation A), A \vdash A
		\end{align}
		と$\vee$の導入より
		\begin{align}
			\negation (A \vee \negation A), A \vdash A \vee \negation A
		\end{align}
		が成り立つ.一方で
		\begin{align}
			\negation (A \vee \negation A), A \vdash \negation (A \vee \negation A)
		\end{align}
		も成り立つので
		\begin{align}
			\negation (A \vee \negation A), A \vdash \bot
		\end{align}
		が成り立つ.演繹法則より
		\begin{align}
			\negation (A \vee \negation A) \vdash A \rarrow \bot
		\end{align}
		が成り立つので,否定の導出より
		\begin{align}
			\negation (A \vee \negation A) \vdash \negation A
		\end{align}
		が成り立つ.再び$\vee$の導入によって
		\begin{align}
			\negation (A \vee \negation A) \vdash A \vee \negation A
		\end{align}
		が成り立つ.再び否定の導出より
		\begin{align}
			\negation (A \vee \negation A) \vdash \bot
		\end{align}
		が成り立つ.ゆえに
		\begin{align}
			\vdash \negation (A \vee \negation A) \rarrow \bot
		\end{align}
		が成り立ち,背理法の原理より
		\begin{align}
			\vdash A \vee \negation A
		\end{align}
		が得られる.
		\QED
	\end{prf}
	
	\begin{screen}
		\begin{thm}[類は集合であるか真類であるかのいずれかに定まる]
			$a$を類とするとき
			\begin{align}
				\vdash \set{a} \vee \negation \set{a}.
			\end{align}
		\end{thm}
	\end{screen}
	
	\begin{prf}
		排中律を適用することにより従う.
		\QED
	\end{prf}
	
	\begin{screen}
		\begin{logicalthm}[矛盾からはあらゆる式が導かれる]\label{logicalthm:contradiction_derives_any_formula}
			$A$を文とするとき
			\begin{align}
				\vdash \bot \rarrow A.
			\end{align}
		\end{logicalthm}
	\end{screen}
	
	\begin{prf}
		推論法則\ref{logicalthm:rule_of_inference_2}より
		\begin{align}
			\vdash \bot \rarrow (\negation A \rarrow \bot)
		\end{align}
		が成り立つ.また背理法の原理より
		\begin{align}
			\vdash (\negation A \rarrow \bot) \rarrow A
		\end{align}
		が成り立つので,含意の推移律を適用すれば
		\begin{align}
			\vdash \bot \rarrow A
		\end{align}
		が得られる.
		\QED
	\end{prf}
	
	\begin{screen}
		\begin{logicalthm}[矛盾を導く式はあらゆる式を導く]\label{logicalthm:formula_leading_to_contradiction_derives_any_formula}
			$A,B$を文とするとき
			\begin{align}
				\vdash (A \rarrow \bot) \rarrow (A \rarrow B).
			\end{align}
		\end{logicalthm}
	\end{screen}
	
	\begin{prf}
		推論法則\ref{logicalthm:contradiction_derives_any_formula}より
		\begin{align}
			\vdash \bot \rarrow B
		\end{align}
		が成り立つので
		\begin{align}
			A \rarrow \bot \vdash \bot \rarrow B
		\end{align}
		が成り立つ.
		\begin{align}
			A \rarrow \bot \vdash A \rarrow \bot
		\end{align}
		も成り立つので,含意の推移律(推論法則\ref{logicalthm:transitive_law_of_implication})より
		\begin{align}
			A \rarrow \bot \vdash A \rarrow B
		\end{align}
		が成り立つ.そして演繹法則より
		\begin{align}
			\vdash (A \rarrow \bot) \rarrow (A \rarrow B)
		\end{align}
		が得られる.
		\QED
	\end{prf}
	
	\begin{itembox}[l]{空虚な真}
		$A,B$を文とするとき,偽な式は矛盾を導くので(推論法則\ref{logicalthm:false_and_negation_are_equivalent})
		\begin{align}
			\vdash \negation A \rarrow (A \rarrow \bot)
		\end{align}
		が成り立ち,矛盾を導く式はあらゆる式を導くから(推論法則\ref{logicalthm:formula_leading_to_contradiction_derives_any_formula})
		\begin{align}
			\vdash (A \rarrow \bot) \rarrow (A \rarrow B)
		\end{align}
		が成り立つ.以上と含意の推移律より
		\begin{align}
			\vdash \negation A \rarrow (A \rarrow B)
		\end{align}
		が得られる.つまり``偽な式はあらゆる式を導く''のであり,この現象を
		{\bf 空虚な真}\index{くうきょなしん@空虚な真}{\bf (vacuous truth)}と呼ぶ.
	\end{itembox}
	
	\begin{screen}
		\begin{logicalthm}[含意は否定と論理和で表せる]\label{logicalthm:rule_of_inference_3}
			$A,B$を文とするとき
			\begin{align}
				\vdash (A \rarrow B) \lrarrow (\negation A \vee B).
			\end{align}
		\end{logicalthm}
	\end{screen}
	
	\begin{prf}\mbox{}
		\begin{description}
			\item[第一段]
				含意の遺伝性質より
				\begin{align}
					\vdash (A \rarrow B) \rarrow 
					((A \vee \negation A) \rarrow (B \vee \negation A))
				\end{align}
				が成り立つので
				\begin{align}
					A \rarrow B \vdash (A \vee \negation A) \rarrow (B \vee \negation A)
				\end{align}
				となる.排中律より
				\begin{align}
					A \rarrow B \vdash A \vee \negation A
				\end{align}
				も成り立つので,三段論法より
				\begin{align}
					A \rarrow B \vdash B \vee \negation A
				\end{align}
				が成り立ち,論理和の可換律(推論法則\ref{logicalthm:commutative_law_of_disjunction_and_conjunction})より
				\begin{align}
					A \rarrow B \vdash \negation A \vee B
				\end{align}
				が得られ,演繹法則より
				\begin{align}
					\vdash (A \rarrow B) \rarrow (\negation A \vee B)
				\end{align}
				が得られる.
				
			\item[第二段]
				推論法則\ref{logicalthm:false_and_negation_are_equivalent}より
				\begin{align}
					\vdash \negation A \rarrow (A \rarrow \bot)
				\end{align}
				が成り立ち,一方で推論法則\ref{logicalthm:formula_leading_to_contradiction_derives_any_formula}より
				\begin{align}
					\vdash (A \rarrow \bot) \rarrow (A \rarrow B)
				\end{align}
				も成り立つので,含意の推移律より
				\begin{align}
					\vdash \negation A \rarrow (A \rarrow B)
				\end{align}
				が成立する.推論法則\ref{logicalthm:rule_of_inference_2}より
				\begin{align}
					\vdash B \rarrow (A \rarrow B)
				\end{align}
				も成り立つから,場合分けの法則より
				\begin{align}
					\vdash (\negation A \vee B) \rarrow (A \rarrow B)
				\end{align}
				が成り立つ.
				\QED
		\end{description}
	\end{prf}
	
	\begin{screen}
		\begin{logicalthm}[二重否定の法則の逆が成り立つ]\label{logicalthm:converse_of_law_of_double_negative}
			$A$を文とするとき
			\begin{align}
				\vdash A \rarrow \negation \negation A.
			\end{align}
		\end{logicalthm}
	\end{screen}
	
	\begin{prf}
		排中律より
		\begin{align}
			\vdash \negation A \vee \negation \negation A
		\end{align}
		が成立し,また推論法則\ref{logicalthm:rule_of_inference_3}より
		\begin{align}
			\vdash (\negation A \vee \negation \negation A)
			\rarrow (A \rarrow \negation \negation A)
		\end{align}
		も成り立つので,三段論法より
		\begin{align}
			\vdash A \rarrow \negation \negation A
		\end{align}
		が成立する.
		\QED
	\end{prf}
	
	\begin{screen}
		\begin{logicalthm}[対偶命題は同値]\label{thm:contraposition_is_true}
			$A,B$を文とするとき
			\begin{align}
				\vdash (A \rarrow B) \lrarrow (\negation B \rarrow \negation A).
			\end{align}
		\end{logicalthm}
	\end{screen}
	
	\begin{prf}\mbox{}
		\begin{description}
			\item[第一段]
				含意は否定と論理和で表せるので(推論法則\ref{logicalthm:rule_of_inference_3})
				\begin{align}
					\vdash (A \rarrow B) \rarrow (\negation A \vee B)
					\label{eq:thm_contraposition_is_true_1}
				\end{align}
				が成り立つ.また論理和は可換であるから(推論法則\ref{logicalthm:commutative_law_of_disjunction_and_conjunction})
				\begin{align}
					\vdash (\negation A \vee B) \rarrow (B \vee \negation A)
					\label{eq:thm_contraposition_is_true_2}
				\end{align}
				が成り立つ.ところで二重否定の法則の逆(推論法則\ref{logicalthm:converse_of_law_of_double_negative})より
				\begin{align}
					\vdash B \rarrow \negation \negation B
				\end{align}
				が成り立ち,また含意の遺伝性質より
				\begin{align}
					\vdash (B \rarrow \negation \negation B)
					\rarrow ((B \vee \negation A) 
					\rarrow (\negation \negation B \vee \negation A))
				\end{align}
				も成り立つから,三段論法より
				\begin{align}
					\vdash (B \vee \negation A) 
					\rarrow (\negation \negation B \vee \negation A)
					\label{eq:thm_contraposition_is_true_3}
				\end{align}
				が成立する.再び推論法則\ref{logicalthm:rule_of_inference_3}によって
				\begin{align}
					\vdash (\negation \negation B \vee \negation A)
					\rarrow (\negation B \rarrow \negation A)
					\label{eq:thm_contraposition_is_true_4}
				\end{align}
				が成り立つ.(\refeq{eq:thm_contraposition_is_true_1})と(\refeq{eq:thm_contraposition_is_true_2})と
				(\refeq{eq:thm_contraposition_is_true_3})と(\refeq{eq:thm_contraposition_is_true_4})に含意の推移律を適用すれば
				\begin{align}
					\vdash (A \rarrow B) \rarrow (\negation B \rarrow \negation A)
				\end{align}
				が得られる.
				
			\item[第二段]
				含意は否定と論理和で表せるので(推論法則\ref{logicalthm:rule_of_inference_3})
				\begin{align}
					\vdash (\negation B \rarrow \negation A)
					\rarrow (\negation \negation B \vee \negation A)
					\label{eq:thm_contraposition_is_true_5}
				\end{align}
				が成り立つ.ところで二重否定の法則より
				\begin{align}
					\vdash \negation \negation B \rarrow B
				\end{align}
				が成り立ち,また含意の遺伝性質より
				\begin{align}
					\vdash (\negation \negation B \rarrow B)
					\rarrow ((\negation \negation B \vee \negation A)
					\rarrow (B \vee \negation A))
				\end{align}
				も成り立つから,三段論法より
				\begin{align}
					\vdash (\negation \negation B \vee \negation A)
					\rarrow (B \vee \negation A)
					\label{eq:thm_contraposition_is_true_6}
				\end{align}
				が成立する.論理和は可換であるから(推論法則\ref{logicalthm:commutative_law_of_disjunction_and_conjunction})
				\begin{align}
					\vdash (B \vee \negation A) \rarrow (\negation A \vee B)
					\label{eq:thm_contraposition_is_true_7}
				\end{align}
				が成り立つ.再び推論法則\ref{logicalthm:rule_of_inference_3}によって
				\begin{align}
					\vdash (\negation A \vee B)
					\rarrow (A \rarrow B)
					\label{eq:thm_contraposition_is_true_8}
				\end{align}
				が成り立つ.(\refeq{eq:thm_contraposition_is_true_5})と(\refeq{eq:thm_contraposition_is_true_6})と
				(\refeq{eq:thm_contraposition_is_true_7})と(\refeq{eq:thm_contraposition_is_true_8})に含意の推移律を適用すれば
				\begin{align}
					\vdash (\negation B \rarrow \negation A) \rarrow (A \rarrow B)
				\end{align}
				が得られる.
				\QED
		\end{description}
	\end{prf}
	
	上の証明は簡単に書けば
	\begin{align}
		(A \rarrow B) &\lrarrow (\negation A \vee B) \\
		&\lrarrow (B \vee \negation A) \\
		&\lrarrow (\negation \negation B \vee \negation A) \\
		&\lrarrow (\negation B \rarrow \negation A)
	\end{align}
	で足りる.
	
	\begin{screen}
		\begin{logicalthm}[De Morganの法則]
			$A,B$を文とするとき
			\begin{itemize}
				\item $\vdash \negation (A \vee B) \lrarrow \negation A \wedge \negation B$.
			
				\item $\vdash \negation (A \wedge B) \lrarrow \negation A \vee \negation B$.
			\end{itemize}
		\end{logicalthm}
	\end{screen}
	
	\begin{prf}\mbox{}
		\begin{description}
			\item[第一段]	論理和の導入の対偶を取れば
				\begin{align}
					\vdash \negation (A \vee B) \rarrow \negation A
				\end{align}
				と
				\begin{align}
					\vdash \negation (A \vee B) \rarrow \negation B
				\end{align}
				が成り立つ(推論法則\ref{thm:contraposition_is_true}).
				二式が同時に導かれるならその論理積も導かれるので(推論法則\ref{logicalthm:conjunction_of_consequences})
				\begin{align}
					\vdash \negation (A \vee B) \rarrow (\negation A \wedge \negation B)
				\end{align}
				が得られる.また
				\begin{align}
					A, \negation A \wedge \negation B \vdash A
				\end{align}
				かつ
				\begin{align}
					A, \negation A \wedge \negation B \vdash \negation A
				\end{align}
				より
				\begin{align}
					A, \negation A \wedge \negation B \vdash \bot
				\end{align}
				が成り立つので,演繹法則より
				\begin{align}
					A \vdash (\negation A \wedge \negation B) \rarrow \bot
				\end{align}
				が従い,否定の導入により
				\begin{align}
					A \vdash \negation (\negation A \wedge \negation B)
				\end{align}
				が成り立つ.同様にして
				\begin{align}
					B \vdash \negation (\negation A \wedge \negation B)
				\end{align}
				も成り立つので,場合分け法則より
				\begin{align}
					\vdash (A \vee B) \rarrow \negation (\negation A \wedge \negation B)
				\end{align}
				が成立する.この対偶を取れば
				\begin{align}
					\vdash (\negation A \wedge \negation B) \rarrow \negation (A \vee B)
				\end{align}
				が得られる(推論法則\ref{thm:contraposition_is_true}).
				
			\item[第二段]
				前段の結果より
				\begin{align}
					\vdash (\negation \negation A \wedge \negation \negation B)
					\lrarrow \negation (\negation A \vee \negation B)
				\end{align}
				が成り立つ.ところで二重否定の法則とその逆(推論法則\ref{logicalthm:converse_of_law_of_double_negative})より
				\begin{align}
					\vdash (\negation \negation A \wedge \negation \negation B)
					\lrarrow (A \wedge B)
				\end{align}
				が成り立つので
				\begin{align}
					\vdash (A \wedge B) 
					\lrarrow \negation (\negation A \vee \negation B)
				\end{align}
				が成り立つ.対偶命題の同値性(推論法則\ref{thm:contraposition_is_true})から
				\begin{align}
					\vdash \negation (A \wedge B)
					\lrarrow (\negation A \vee \negation B)
				\end{align}
				が得られる.
				\QED
		\end{description}
	\end{prf}
	
	\monologue{
		以上で``集合であり真類でもある類は存在しない''という言明を証明する準備が整いました.
	}
	
	\begin{screen}
		\begin{thm}[集合であり真類でもある類は存在しない]
			$a$を類とするとき
			\begin{align}
				\vdash \negation (\ \set{a} \wedge \negation \set{a}\ ).
			\end{align}
		\end{thm}
	\end{screen}
	
	\begin{prf}
		$a$を類とするとき,排中律より
		\begin{align}
			\vdash \set{a} \vee \negation \set{a}
		\end{align}
		が成り立ち,論理和の可換律より
		\begin{align}
			\vdash \negation \set{a} \vee \set{a}
		\end{align}
		も成立する.そしてDe Morganの法則より
		\begin{align}
			\vdash \negation (\ \negation \negation \set{a} \wedge \negation \set{a}\ )
		\end{align}
		が成り立つが,二重否定の法則より$\negation \negation \set{a}$と
		$\set{a}$は同値となるので
		\begin{align}
			\vdash \negation (\ \set{a} \wedge \negation \set{a}\ )
		\end{align}
		が成り立つ.
		\QED
	\end{prf}
	
	「集合であり真類でもある類は存在しない」とは言ったものの,それはあくまで
	\begin{align}
		\Sigma \vdash \negation (\ \set{a} \wedge \negation \set{a}\ )
	\end{align}
	を翻訳したに過ぎないのであって,もしかすると
	\begin{align}
		\Sigma \vdash \set{a} \wedge \negation \set{a}
	\end{align}
	も導かれるかもしれない.この場合$\Sigma$は矛盾することになるが,$\Sigma$の無矛盾性が不明であるため
	この事態が起こらないとは言えない.
	
	\begin{screen}
		\begin{logicalaxm}[量化記号に関する規則]\label{logicalaxm:rules_of_quantifiers}
			$A$を$\mathcal{L}$の式とし,$x$を$A$に自由に現れる変項とし,
			$A$に自由に現れる項が$x$のみであるとする.
			また$\tau$を任意の$\varepsilon$項とする.このとき以下を推論規則とする.
			\begin{align}
				A(\tau) &\vdash \exists x A(x), \\
				\exists x A(x) &\vdash A(\varepsilon x A(x)), \\
				\forall x A(x) &\vdash A(\tau), \\
				A(\varepsilon x \negation A(x)) &\vdash \forall x A(x).
			\end{align}
		\end{logicalaxm}
	\end{screen}
	
	どれでも一つ,$A$を成り立たせるような$\varepsilon$項$\tau$が取れれば
	$\exists x A(x)$が成り立つのだし,逆に$\exists x A(x)$が成り立つならば
	$\varepsilon x A(x)$なる$\epsilon$項が$A$を満たすのであるから,
	$\exists x A(x)$が成り立つということと$A$を満たす$\varepsilon$項が取れるということは
	同じ意味になる.
	
	$\forall x A(x)$が成り立つならばいかなる$\varepsilon$項も$A$を満たすし,
	逆にいかなる$\varepsilon$項も$A$を満たすならば,特に$\varepsilon x \negation A(x)$
	なる$\varepsilon$項も$A$を満たすのだから,$\forall x A(x)$が成立する.
	つまり,$\forall x A(x)$が成り立つということと,任意の$\varepsilon$項が$A$を満たすということは
	同じ意味になる.
	
	後述することであるが,$\varepsilon$項はどれも集合であって,また集合である類は
	いずれかの$\varepsilon$項と等しい.ゆえに,量化子の亘る範囲は集合に制限されるのである.
	
	\begin{screen}
		\begin{logicalthm}[量化記号の性質(イ)]\label{logicalthm:properties_of_quantifiers}
			$A,B$を$\mathcal{L}'$の式とし,$x$を$A,B$に現れる文字とし,$x$のみが$A,B$で量化されていないとする.
			$\mathcal{L}$の任意の対象$\tau$に対して
			\begin{align}
				A(\tau) \lrarrow B(\tau)
			\end{align}
			が成り立っているとき,
			\begin{align}
				\exists x A(x) \lrarrow \exists x B(x)
			\end{align}
			および
			\begin{align}
				\forall x A(x) \lrarrow \forall x B(x)
			\end{align}
			が成り立つ.
		\end{logicalthm}
	\end{screen}
	
	\begin{prf}
		いま,$\mathcal{L}$の任意の対象$\tau$に対して
		\begin{align}
			A(\tau) \lrarrow B(\tau)
			\label{logicalthm:properties_of_quantifiers_1}
		\end{align}
		が成り立っているとする.
		ここで
		\begin{align}
			\exists x A(x)
		\end{align}
		が成り立っていると仮定すると,
		\begin{align}
			\tau \defeq \varepsilon x A(x)
		\end{align}
		とおけば存在記号に関する規則より
		\begin{align}
			A(\tau)
		\end{align}
		が成立し,(\refeq{logicalthm:properties_of_quantifiers_1})と併せて
		\begin{align}
			B(\tau)
		\end{align}
		が成立する.再び存在記号に関する規則より
		\begin{align}
			\exists x B(x)
		\end{align}
		が成り立つので,演繹法則から
		\begin{align}
			\exists x A(x) \rarrow \exists x B(x)
		\end{align}
		が得られる.$A$と$B$の立場を入れ替えれば
		\begin{align}
			\exists x B(x) \rarrow \exists x A(x)
		\end{align}
		も得られる.今度は
		\begin{align}
			\forall x A(x)
		\end{align}
		が成り立っていると仮定すると,
		推論法則\ref{logicalthm:fundamental_law_of_universal_quantifier}より
		$\mathcal{L}$の任意の対象$\tau$に対して
		\begin{align}
			A(\tau)
		\end{align}
		が成立し,(\refeq{logicalthm:properties_of_quantifiers_1})と併せて
		\begin{align}
			B(\tau)
		\end{align}
		が成立する.$\tau$の任意性と推論法則\ref{logicalthm:fundamental_law_of_universal_quantifier}より
		\begin{align}
			\forall x B(x)
		\end{align}
		が成り立つので,演繹法則から
		\begin{align}
			\forall x A(x) \rarrow \forall x B(x)
		\end{align}
		が得られる.$A$と$B$の立場を入れ替えれば
		\begin{align}
			\forall x B(x) \rarrow \forall x A(x)
		\end{align}
		も得られる.
		\QED
	\end{prf}
	
	\begin{screen}
		\begin{logicalthm}[量化記号に対する De Morgan の法則]\label{logicalthm:De_Morgan_law_for_quantifiers}
			$A$を$\mathcal{L}'$の式とし,$x$を$A$に現れる文字とし,$x$のみが$A$で量化されていないとする.このとき
			\begin{align}
				\exists x \negation A(x) \lrarrow \negation \forall x A(x)
			\end{align}
			および
			\begin{align}
				\forall x \negation A(x) \lrarrow \negation \exists x A(x)
			\end{align}
			が成り立つ.
		\end{logicalthm}
	\end{screen}
	
	\begin{sketch}
		推論規則\ref{logicalaxm:rules_of_quantifiers}より
		\begin{align}
			\exists x \negation A(x) \lrarrow 
			\negation A(\varepsilon x \negation A(x))
		\end{align}
		は定理である.他方で推論規則\ref{logicalaxm:rules_of_quantifiers}より
		\begin{align}
			A(\varepsilon x \negation A(x)) \lrarrow \forall x A(x) 
		\end{align}
		もまた定理であり,この対偶を取れば
		\begin{align}
			\negation A(\varepsilon x \negation A(x)) \lrarrow 
			\negation \forall x A(x)
		\end{align}
		が成り立つ.ゆえに
		\begin{align}
			\exists x \negation A(x) \lrarrow \negation \forall x A(x)
		\end{align}
		が従う.$A$を$\negation A$に置き換えれば
		\begin{align}
			\forall x \negation A(x) \lrarrow 
			\negation \exists x \negation \negation A(x)
		\end{align}
		が成り立ち,また$\mathcal{L}$の任意の対象$\tau$に対して
		\begin{align}
			A(\tau) \lrarrow \negation \negation A(\tau)
		\end{align}
		が成り立つので,推論法則\ref{logicalthm:properties_of_quantifiers}より
		\begin{align}
			\exists x \negation \negation A(x)
			\lrarrow \exists x A(x)
		\end{align}
		も成り立つ.ゆえに
		\begin{align}
			\forall x \negation A(x) \lrarrow 
			\negation \exists x A(x)
		\end{align}
		が従う.
		\QED
	\end{sketch}
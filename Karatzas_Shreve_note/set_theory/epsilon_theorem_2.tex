\section{第二イプシロン定理}
	$\exists x \forall y \exists z B(x,y,z)$を$L(PC)$の冠頭標準形
	(つまり$B(x,y,z)$は$L(EC)$の式である)とし,$B$に自由に現れる変項は$x,y,z$のみであるとする.
	また
	\begin{align}
		\provable{\mbox{{\bf PC}${}_{\varepsilon}$}} 
		\exists x \forall y \exists z B(x,y,z)
	\end{align}
	であるとする.
	
	$f$を$L(PC)$には無い一変数関数記号とし,
	\begin{align}
		L'(PC) &\defeq L(PC) \cup \{f\}, \\
		L'(EC) &\defeq L(EC) \cup \{f\}, \\
		L'(PC_{\varepsilon}) &\defeq L(PC_{\varepsilon}) \cup \{f\}, \\
		L'(EC_{\varepsilon}) &\defeq L(EC_{\varepsilon}) \cup \{f\}
	\end{align}
	とする.そして,$L'(PC_{\varepsilon})$の式を用いた証明
	(公理と推論規則は{\bf PC}のもの)が存在することを
	\begin{align}
		\provable{\mbox{{\bf PC'}${}_{\varepsilon}$}}
	\end{align}
	と書く(同様に$\provable{\mbox{{\bf EC'}}},
	\provable{\mbox{{\bf EC'}${}_{\varepsilon}$}},
	\provable{\mbox{{\bf PC'}}}$を使う.).
	このとき,{\bf PC}${}_{\varepsilon}$の証明は{\bf PC'}${}_{\varepsilon}$の証明でもあるから
	\begin{align}
		\provable{\mbox{{\bf PC'}${}_{\varepsilon}$}}
		\exists x \forall y \exists z B(x,y,z)
	\end{align}
	である.また
	\begin{align}
		&\provable{\mbox{{\bf PC'}${}_{\varepsilon}$}} \exists x \forall y \exists z B(x,y,z), \\
		&\provable{\mbox{{\bf PC'}${}_{\varepsilon}$}} \exists x \forall y \exists z B(x,y,z)
		\rightarrow \forall y \exists z B(\tau,y,z), && 
		(\tau \defeq \varepsilon x \forall y \exists z B(x,y,z)) \\
		&\provable{\mbox{{\bf PC'}${}_{\varepsilon}$}} \forall y \exists z B(\tau,y,z), \\
		&\provable{\mbox{{\bf PC'}${}_{\varepsilon}$}} \forall y \exists z B(\tau,y,z)
		\rightarrow \exists z B(\tau,f\tau,z), \\
		&\provable{\mbox{{\bf PC'}${}_{\varepsilon}$}} \exists z B(\tau,f\tau,z), \\
		&\provable{\mbox{{\bf PC'}${}_{\varepsilon}$}} \exists z B(\tau,f\tau,z)
		\rightarrow \exists x \exists z B(x,fx,z), \\
		&\provable{\mbox{{\bf PC'}${}_{\varepsilon}$}} \exists x \exists z B(x,fx,z)
	\end{align}
	が成り立つ.すると埋め込み定理より,
	\begin{align}
		\zeta &\defeq \varepsilon z B(x,fx,z), \\
		\tau &\defeq \varepsilon x B(x,fx,\zeta)
	\end{align}
	とおけば
	\begin{align}
		\provable{\mbox{{\bf EC'}${}_{\varepsilon}$}} B(\tau,f\tau,\zeta)
	\end{align}
	が成り立つ.そして,或る$p>1$と,$p$個の$L'(EC)$の項$r_{i}$と,
	同じく$p$個の$L'(EC)$の項$s_{i}$が取れて,
	\begin{align}
		\provable{\mbox{{\bf EC'}}} \bigvee_{i=1}^{p} B(r_{i},fr_{i},s_{i})
	\end{align}
	が成り立つ{\bf (後述,次の小節)}.同じ証明で
	\begin{align}
		\provable{\mbox{{\bf PC'}}} \bigvee_{i=1}^{p} B(r_{i},fr_{i},s_{i})
	\end{align}
	であることも言える.
	\begin{align}
		\provable{\mbox{{\bf PC'}}} \bigvee_{i=1}^{p-1} B(r_{i},fr_{i},s_{i})
		\vee B(r_{p},fr_{p},s_{p})
	\end{align}
	より,まず
	\begin{align}
		\provable{\mbox{{\bf PC'}}} \bigvee_{i=1}^{p-1} B(r_{i},fr_{i},s_{i})
		\vee \exists z B(r_{p},fr_{p},z)
	\end{align}
	となる.続いて,$fr_{p}$は$\bigvee_{i=1}^{p-1} B(r_{i},fr_{i},s_{i})$には現れないので
	\begin{align}
		\provable{\mbox{{\bf PC'}}} \bigvee_{i=1}^{p-1} B(r_{i},fr_{i},s_{i})
		\vee \forall y \exists z B(r_{p},y,z)
	\end{align}
	となる.最後に
	\begin{align}
		\provable{\mbox{{\bf PC'}}} \bigvee_{i=1}^{p-1} B(r_{i},fr_{i},s_{i})
		\vee \exists x \forall y \exists z B(x,y,z)
	\end{align}
	となる.これを繰り返せば
	\begin{align}
		\provable{\mbox{{\bf PC'}}} \exists x \forall y \exists z B(x,y,z)
		\vee \cdots \vee \exists x \forall y \exists z B(x,y,z)
	\end{align}
	が得られるので
	\begin{align}
		\provable{\mbox{{\bf PC'}}} \exists x \forall y \exists z B(x,y,z)
	\end{align}
	となる.最後に,$\exists x \forall y \exists z B(x,y,z)$への証明に残っている
	$f$を含む項を$L(PC)$の項に置き換えれば,{\bf PC}から
	$\exists x \forall y \exists z B(x,y,z)$への証明が得られる.
	
\subsection{$\bigvee_{i=1}^{p} B(r_{i},fr_{i},s_{i})$への証明}
	第一イプシロン定理と証明は殆ど変わらない.
	
	\begin{enumerate}
		\item $B(\tau,f\tau,\zeta)$への$EC_{\varepsilon}$の証明を
			$\pi_{1},\pi_{2},\cdots,\pi_{n}$とする.
			
		\item $e$を,この証明の主要$\varepsilon$項のうち階数が最大であって,かつ
			その階数を持つ$\pi$の主要$\varepsilon$項の中で極大である(
			他の$\varepsilon$項の部分項ではない)ものとする.
		
		\item この証明から,$e$が属する主要論理式の一つ
			\begin{align}
				A(t/x) \rightarrow A(e/x)
			\end{align}
			を取る.以下では
			\begin{align}
				B(\tau,f\tau,\zeta) \vee B(\tau',f\tau',\zeta')
			\end{align}
			への$A(t/x) \rightarrow A(e/x)$を用いない証明を構成する.
			ただし$\tau',\zeta'$はそれぞれ$\tau,\zeta$に現れる$e$を$t$に置き換えた項である.
			
		\item それ以降の流れは第一イプシロン定理と同様であって,
			階数が大きい$\varepsilon$項から順番に,それが属する主要論理式を一本ずつ排除しながら
			証明を構成するが,証明される式は
			\begin{align}
				B(\tau,f\tau,\zeta) \vee B(\tau',f\tau',\zeta') \vee
				B(\tau'',f\tau'',\zeta'') \vee \cdots
			\end{align}
			のように増えていく.そして,最終的に
			\begin{align}
				\bigvee_{i=1}^{p} B(\tau_{i},f\tau_{i},\zeta_{i})
			\end{align}
			への{\bf EC'}${}_{\varepsilon}$の証明が得られる.最後に証明に残っている
			$\varepsilon$項を$L(EC)$の項に置き換える.
	\end{enumerate}
	
	\begin{description}
		\item[step1]
			$\varphi_{i}$が$A(t/x) \rightarrow A(e/x)$でない
			{\bf EC}${}_{\varepsilon}$の公理ならば,
			$\varphi_{i}$と$\varphi_{i+1}$の間に
			\begin{align}
				&\varphi_{i} \rightarrow 
				(\rightharpoondown A(t/x) \rightarrow \varphi_{i}), \\
				&\rightharpoondown A(t/x) \rightarrow \varphi_{i}
			\end{align}
			を挿入する(この時点で,$A(t/x) \rightarrow A(e/x)$でない主要論理式は
			無傷のまま残ることが判る).$\varphi_{i}$が$\varphi_{j}$と$\varphi_{k}
			\equiv \varphi_{j} \rightarrow \varphi_{i}$から三段論法で得られる場合は,
			$\varphi_{i}$を
			\begin{align}
				&(\rightharpoondown A(t/x) \rightarrow (\varphi_{j} \rightarrow \varphi_{i})) \rightarrow
				[(\rightharpoondown A(t/x) \rightarrow \varphi_{j})
				\rightarrow (\rightharpoondown A(t/x) \rightarrow \varphi_{i})], \\
				&(\rightharpoondown A(t/x) \rightarrow \varphi_{j})
				\rightarrow (\rightharpoondown A(t/x) \rightarrow \varphi_{i}), \\
				&\rightharpoondown A(t/x) \rightarrow \varphi_{i}
			\end{align}
			で置き換える.$\varphi_{i}$が$A(t/x) \rightarrow A(e/x)$であるときは,
			$\varphi_{i}$を
			\begin{align}
				\rightharpoondown A(t/x) \rightarrow (A(t/x) \rightarrow A(e/x))
			\end{align}
			で置き換える.以上で$\rightharpoondown A(t/x) \rightarrow 
			B(\tau,f\tau,\zeta)$への証明が得られた.
	
		\item[step2]
			$\varphi_{1},\varphi_{2},\cdots,\varphi_{n}$に現れる$e$を
			($e$が部分項として現れる場合も)$t$に置き換えた式を
			\begin{align}
				\tilde{\varphi}_{1},\ \tilde{\varphi}_{2},\cdots, \tilde{\varphi}_{n}
			\end{align}
			と書く.このとき,任意の$i \in \{1,2,\cdots,n\}$で
			\begin{enumerate}
				\item $\varphi_{i}$が主要論理式でない{\bf EC}${}_{\varepsilon}$の定理なら
					$\tilde{\varphi}_{i}$も主要論理式でない{\bf EC}${}_{\varepsilon}$の定理である.
				\item $\varphi_{i}$が$A(u/x) \rightarrow A(e/x)$なる形の主要論理式
					ならば,$\tilde{\varphi}_{i}$は$A(v/x) \rightarrow A(t/x)$
					の形の式である.ただし$v$とは,$u$に$e$が現れていたらそれを$t$
					に置き換えた項である.
				\item $\varphi_{i}$が主要論理式で,$e$が属していないならば,
					$\tilde{\varphi}_{i}$も主要論理式である(置換定理).
			\end{enumerate}
	
			$\varphi_{i}$が$A(t/x) \rightarrow A(e/x)$でない$EC_{\varepsilon}$の公理
			ならば,$\tilde{\varphi}_{i}$と$\tilde{\varphi}_{i+1}$の間に
			\begin{align}
				&\tilde{\varphi}_{i} \rightarrow 
				\left( A(t/x) \rightarrow \tilde{\varphi}_{i} \right), \\
				&A(t/x) \rightarrow \tilde{\varphi}_{i}
			\end{align}
			を挿入する.$\varphi_{i}$が$\varphi_{j}$と$\varphi_{k}
			\equiv \varphi_{j} \rightarrow \varphi_{i}$からモーダスポンネスで得られる場合は,
			$\tilde{\varphi}_{i}$を
			\begin{align}
				&\left( A(t/x) \rightarrow \left(\tilde{\varphi}_{j} \rightarrow
				\tilde{\varphi}_{i} \right) \right)
				\rightarrow \left[ \left( A(t/x) \rightarrow \tilde{\varphi}_{j} \right)
				\rightarrow \left( A(t/x) \rightarrow \tilde{\varphi}_{i} \right) \right], \\
				&\left( A(t/x) \rightarrow \tilde{\varphi}_{j} \right)
				\rightarrow \left( A(t/x) \rightarrow \tilde{\varphi}_{i} \right), \\
				&A(t) \rightarrow \tilde{\varphi}_{i}
			\end{align}
			で置き換える.$\varphi_{i}$が$e$が属する主要論理式
			$A(u/x) \rightarrow A(e/x)$であるときは,$\tilde{\varphi}_{i}$とは
			\begin{align}
				A(v/x) \rightarrow A(t/x)
			\end{align}
			なる形の式であるが,$\tilde{\varphi}_{i}$を
			\begin{align}
				A(t/x) \rightarrow (A(v/x) \rightarrow A(t/x))
			\end{align}
			で置き換える.以上で$A(t/x) \rightarrow B(\tau',f\tau',\zeta')$への
			証明が得られた.
	
		\item[$B$への証明の構成]
			$B(\tau,f\tau,\zeta) \vee B(\tau',f\tau',\zeta')$を$C$とおく.
			まずは$\rightharpoondown A(t/x) \rightarrow C$への証明を得るために
			\begin{align}
				(\rightharpoondown A(t/x) \rightarrow B(\tau,f\tau,\zeta))
				\rightarrow [(B(\tau,f\tau,\zeta) \rightarrow C)
				\rightarrow (\rightharpoondown A(t/x) \rightarrow C)]
			\end{align}
			(これは$(\varphi \rightarrow \psi) \rightarrow
			((\psi \rightarrow \chi) \rightarrow (\varphi \rightarrow \chi))$
			の形の{\bf EC}${}_{\varepsilon}$の定理である)への証明と
			\begin{align}
				&(B(\tau,f\tau,\zeta) \rightarrow C)
					\rightarrow (\rightharpoondown A(t/x) \rightarrow C), \\
				&B(\tau,f\tau,\zeta) \rightarrow C, \\
				&\rightharpoondown A(t/x) \rightarrow C
			\end{align}
			を追加する.同様にして
			\begin{align}
				A(t/x) \rightarrow C
			\end{align}
			への証明も得られる.あとは,
			\begin{align}
				(A(t/x) \rightarrow C) \rightarrow
				\left[ (\rightharpoondown A(t/x) \rightarrow C) \rightarrow
				(A(t/x) \vee \rightharpoondown A(t/x) \rightarrow C)\right]
			\end{align}
			への証明を追加し(これは$(\varphi \rightarrow \chi) \rightarrow
			((\psi \rightarrow \chi) \rightarrow 
			(\varphi \vee \psi \rightarrow \chi))$
			の形の{\bf EC}${}_{\varepsilon}$の定理である),
			\begin{align}
				&(\rightharpoondown A(t/x) \rightarrow C) \rightarrow
				(A(t/x) \vee \rightharpoondown A(t/x) \rightarrow C), \\
				&A(t/x) \vee \rightharpoondown A(t/x) \rightarrow C, \\
				&A(t) \vee \rightharpoondown A(t), \\
				&C
			\end{align}
			を追加すれば,$A(t/x) \rightarrow A(e/x)$を用いない
			$B$への{\bf EC}${}_{\varepsilon}$の証明となる.
			\QED
	\end{description}
\section{第二イプシロン定理}
	$\exists x \forall y \exists z B(x,y,z)$を$L(PC)$の冠頭標準形とする.
	つまり$B(x,y,z)$には量化子が現れないので,$B(x,y,z)$は$L(EC)$の式ということである.
	また
	\begin{align}
		PC_{\varepsilon} \vdash \exists x \forall y \exists z B(x,y,z)
	\end{align}
	であるとする.
	
	$f$を$L(PC)$には無い一変数関数記号とし,
	\begin{align}
		L'(PC) &\defeq L(PC) \cup \{f\}, \\
		L'(EC) &\defeq L(EC) \cup \{f\}, \\
		L'(PC_{\varepsilon}) &\defeq L(PC_{\varepsilon}) \cup \{f\}, \\
		L'(EC_{\varepsilon}) &\defeq L(EC_{\varepsilon}) \cup \{f\}
	\end{align}
	とする.このとき明らかに
	\begin{align}
		{PC'}_{\varepsilon} \vdash \exists x \forall y \exists z B(x,y,z)
	\end{align}
	であるが(ただし${PC'}_{\varepsilon} \vdash$とは$L'(PC_{\varepsilon})$の
	式からなる証明が存在するという意味),
	\begin{align}
		{PC'}_{\varepsilon} &\vdash \exists x \forall y \exists z B(x,y,z), \\
		{PC'}_{\varepsilon} &\vdash \exists x \forall y \exists z B(x,y,z)
		\rightarrow \forall y \exists z B(\tau,y,z), && 
		(\tau \defeq \varepsilon x \forall y \exists z B(x,y,z)) \\
		{PC'}_{\varepsilon} &\vdash \forall y \exists z B(\tau,y,z), \\
		{PC'}_{\varepsilon} &\vdash \forall y \exists z B(\tau,y,z)
		\rightarrow \exists z B(\tau,f(\tau),z), \\
		{PC'}_{\varepsilon} &\vdash \exists z B(\tau,f(\tau),z), \\
		{PC'}_{\varepsilon} &\vdash \exists z B(\tau,f(\tau),z)
		\rightarrow \exists x \exists z B(x,f(x),z), \\
		{PC'}_{\varepsilon} &\vdash \exists x \exists z B(x,f(x),z)
	\end{align}
	が成り立つ.すると拡張第一イプシロン定理より,$p$個の$L'(EC)$の項$r_{i}$
	と,同じく$p$個の$L'(EC)$の項$s_{i}$が取れて,
	\begin{align}
		{EC'}_{\varepsilon} \vdash \bigvee_{i=1}^{p} B(r_{i},f(r_{i}),s_{i})
	\end{align}
	となる.同じ証明で
	\begin{align}
		{PC'}_{\varepsilon} \vdash \bigvee_{i=1}^{p} B(r_{i},f(r_{i}),s_{i})
	\end{align}
	であることも言える.
	\begin{align}
		{PC'}_{\varepsilon} \vdash \bigvee_{i=1}^{p-1} B(r_{i},f(r_{i}),s_{i})
		\vee B(r_{p},f(r_{p}),s_{p})
	\end{align}
	より,まず
	\begin{align}
		{PC'}_{\varepsilon} \vdash \bigvee_{i=1}^{p-1} B(r_{i},f(r_{i}),s_{i})
		\vee \exists z B(r_{p},f(r_{p}),z)
	\end{align}
	となる.続いて,$f(r_{p})$は$\bigvee_{i=1}^{p-1} B(r_{i},f(r_{i}),s_{i})$には現れないので
	\begin{align}
		{PC'}_{\varepsilon} \vdash \bigvee_{i=1}^{p-1} B(r_{i},f(r_{i}),s_{i})
		\vee \forall y \exists z B(r_{p},y,z)
	\end{align}
	となる.最後に
	\begin{align}
		{PC'}_{\varepsilon} \vdash \bigvee_{i=1}^{p-1} B(r_{i},f(r_{i}),s_{i})
		\vee \exists x \forall y \exists z B(x,y,z)
	\end{align}
	となる.これを繰り返せば
	\begin{align}
		{PC'}_{\varepsilon} \vdash \exists x \forall y \exists z B(x,y,z)
		\vee \cdots \vee \exists x \forall y \exists z B(x,y,z)
	\end{align}
	が得られるので
	\begin{align}
		{PC'}_{\varepsilon} \vdash \exists x \forall y \exists z B(x,y,z)
	\end{align}
	となる.最後に,$\exists x \forall y \exists z B(x,y,z)$への証明に残っている
	$f$を含む項を$L(PC)$の項に置き換えれば,$L(PC)$から$\exists x \forall y \exists z B(x,y,z)$
	への証明が得られる.
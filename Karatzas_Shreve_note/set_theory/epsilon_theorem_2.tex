\section{第二イプシロン定理}
	$\exists x \forall y \exists z B(x,y,z)$を$L(PC)$の冠頭標準形とする.
	つまり$B(x,y,z)$は$L(EC)$の式である.
	また
	\begin{align}
		PC_{\varepsilon} \vdash \exists x \forall y \exists z B(x,y,z)
	\end{align}
	であるとする.
	
	$f$を$L(PC)$には無い一変数関数記号とし,
	\begin{align}
		L'(PC) &\defeq L(PC) \cup \{f\}, \\
		L'(EC) &\defeq L(EC) \cup \{f\}, \\
		L'(PC_{\varepsilon}) &\defeq L(PC_{\varepsilon}) \cup \{f\}, \\
		L'(EC_{\varepsilon}) &\defeq L(EC_{\varepsilon}) \cup \{f\}
	\end{align}
	とする.このとき明らかに
	\begin{align}
		{PC'}_{\varepsilon} \vdash \exists x \forall y \exists z B(x,y,z)
	\end{align}
	であるが(ただし${PC'}_{\varepsilon} \vdash$とは$L'(PC_{\varepsilon})$の
	式からなる証明が存在するという意味),
	\begin{align}
		{PC'}_{\varepsilon} &\vdash \exists x \forall y \exists z B(x,y,z), \\
		{PC'}_{\varepsilon} &\vdash \exists x \forall y \exists z B(x,y,z)
		\rightarrow \forall y \exists z B(\tau,y,z), && 
		(\tau \defeq \varepsilon x \forall y \exists z B(x,y,z)) \\
		{PC'}_{\varepsilon} &\vdash \forall y \exists z B(\tau,y,z), \\
		{PC'}_{\varepsilon} &\vdash \forall y \exists z B(\tau,y,z)
		\rightarrow \exists z B(\tau,f(\tau),z), \\
		{PC'}_{\varepsilon} &\vdash \exists z B(\tau,f(\tau),z), \\
		{PC'}_{\varepsilon} &\vdash \exists z B(\tau,f(\tau),z)
		\rightarrow \exists x \exists z B(x,f(x),z), \\
		{PC'}_{\varepsilon} &\vdash \exists x \exists z B(x,f(x),z)
	\end{align}
	が成り立つ.すると拡張第一イプシロン定理より,$p$個の$L'(EC)$の項$r_{i}$
	と,同じく$p$個の$L'(EC)$の項$s_{i}$が取れて,
	\begin{align}
		{EC'}_{\varepsilon} \vdash \bigvee_{i=1}^{p} B(r_{i},f(r_{i}),s_{i})
	\end{align}
	となる.同じ証明で
	\begin{align}
		{PC'}_{\varepsilon} \vdash \bigvee_{i=1}^{p} B(r_{i},f(r_{i}),s_{i})
	\end{align}
	であることも言える.
	\begin{align}
		{PC'}_{\varepsilon} \vdash \bigvee_{i=1}^{p-1} B(r_{i},f(r_{i}),s_{i})
		\vee B(r_{p},f(r_{p}),s_{p})
	\end{align}
	より,まず
	\begin{align}
		{PC'}_{\varepsilon} \vdash \bigvee_{i=1}^{p-1} B(r_{i},f(r_{i}),s_{i})
		\vee \exists z B(r_{p},f(r_{p}),z)
	\end{align}
	となる.続いて,$f(r_{p})$は$\bigvee_{i=1}^{p-1} B(r_{i},f(r_{i}),s_{i})$には現れないので
	\begin{align}
		{PC'}_{\varepsilon} \vdash \bigvee_{i=1}^{p-1} B(r_{i},f(r_{i}),s_{i})
		\vee \forall y \exists z B(r_{p},y,z)
	\end{align}
	となる.最後に
	\begin{align}
		{PC'}_{\varepsilon} \vdash \bigvee_{i=1}^{p-1} B(r_{i},f(r_{i}),s_{i})
		\vee \exists x \forall y \exists z B(x,y,z)
	\end{align}
	となる.これを繰り返せば
	\begin{align}
		{PC'}_{\varepsilon} \vdash \exists x \forall y \exists z B(x,y,z)
		\vee \cdots \vee \exists x \forall y \exists z B(x,y,z)
	\end{align}
	が得られるので
	\begin{align}
		{PC'}_{\varepsilon} \vdash \exists x \forall y \exists z B(x,y,z)
	\end{align}
	となる.最後に,$\exists x \forall y \exists z B(x,y,z)$への証明に残っている
	$f$を含む項を$L(PC)$の項に置き換えれば,$L(PC)$から$\exists x \forall y \exists z B(x,y,z)$
	への証明が得られる.
	
\subsection{{\bf EC}${}_{\varepsilon}$から$B(\tau,f\tau,\zeta)$への証明}
	$\pi$を$\varphi_{0},\varphi_{1},\cdots,\varphi_{n}$とし,
	$\varphi_{0},\varphi_{1},\cdots,\varphi_{n}$に現れる$e$を$t$に置き換えた式を
	\begin{align}
		\tilde{\varphi}_{0},\ \tilde{\varphi}_{1},\cdots, \tilde{\varphi}_{n}
	\end{align}
	と書く($e$は,どれかの項の部分項であるときも置き換える).
	このとき,任意の$0 \leq i \leq n$で
	\begin{enumerate}
		\item $\varphi_{i}$がトートロジーなら$\tilde{\varphi}_{i}$もトートロジーである.
		\item $\varphi_{i}$が主要論理式で,$e$が$\varphi_{i}$の主要項であるならば,
			$\tilde{\varphi}_{i}$は$A(u) \Longrightarrow A(t)$なる形の式である
			\footnotemark.
		\item $\varphi_{i}$が主要論理式で,$e$が$\varphi_{i}$の主要項ではないならば,
			$\tilde{\varphi}_{i}$も主要論理式である.
	\end{enumerate}
	
	\footnotetext{
		$\varepsilon x A$と$\varepsilon y B$が記号列として一致すれば,
		$x$と$y$は一致するし,式$A$と式$B$も一致するので
		$A(\varepsilon x A)$と$B(\varepsilon y B)$も記号列として一致する.
	}
	
	$\varphi$が$A(t) \Longrightarrow A(e)$でない$EC_{\varepsilon}$の公理ならば,
	$\tilde{\varphi}_{i}$と$\tilde{\varphi}_{i+1}$の間に
	\begin{align}
		&\tilde{\varphi}_{i} \Longrightarrow 
		\left( A(t) \Longrightarrow \tilde{\varphi}_{i} \right), \\
		&A(t) \Longrightarrow \tilde{\varphi}_{i}
	\end{align}
	を挿入する.$\varphi_{i}$が$\varphi_{j}$と$\varphi_{k}$からモーダスポンネスで得られる場合は,
	$\tilde{\varphi}_{i}$を
	\begin{align}
		&\left( A(t) \Longrightarrow \tilde{\varphi}_{j} \right)
		\Longrightarrow \left[ \left( A(t) \Longrightarrow 
		\left( \tilde{\varphi}_{j}\Longrightarrow \tilde{\varphi}_{i} \right) \right)
		\Longrightarrow \left( A(t) \Longrightarrow \tilde{\varphi}_{i} \right) \right], \\
		&\left( A(t) \Longrightarrow 
		\left( \tilde{\varphi}_{j}\Longrightarrow \tilde{\varphi}_{i} \right) \right)
		\Longrightarrow \left( A(t) \Longrightarrow \tilde{\varphi}_{i} \right), \\
		&A(t) \Longrightarrow \tilde{\varphi}_{i}
	\end{align}
	で置き換える.すると,$A(t) \Longrightarrow A(e)$を使わない
	$EC_{\varepsilon}$から$A(t) \Longrightarrow B$への証明が得られる.
	$\varphi_{i}$が$e$が属する主要論理式$A(s) \Longrightarrow A(e)$であるときは,
	$\tilde{\varphi}_{i}$とは
	\begin{align}
		A(s') \Longrightarrow A(t)
	\end{align}
	なる形の式であるが
	\footnote{
		$x$を$A$に現れている自由な変項とすれば,$e$とは$\varepsilon x A$のことであるし,
		$A(\varepsilon x A)$とは$A$に自由に現れる$x$を$\varepsilon x A$に置換した式である.
		$A$には$\varepsilon x A$は現れていないので,というのも$\varepsilon x A$が登場するのは
		$A$が作られた後であるからだが,$A(e)$に現れる$e$を$t$に変換した式は
		$A(t)$になる.同様に,$A(s)$に$e$が現れるとすれば,その$e$は$y$に代入された$s$の
		部分項でしかありえない.すなわち,$A(s)$に現れる$e$を$t$で置換した式は,
		$s'$を$s$に現れる$e$を$t$に変換した項として ($s$に$e$が現れなければ$s'$は$s$である)
		$A(s')$となるわけである.
	},$\tilde{\varphi}_{i}$を
	\begin{align}
		&A(t) \Longrightarrow (A(s') \Longrightarrow A(t)), \\
		&A(s') \Longrightarrow A(t)
	\end{align}
	で置き換える.
	
	同様に$A(t) \Longrightarrow A(e)$を使わない$EC_{\varepsilon})$から
	$\rightharpoondown A(t) \Longrightarrow B$への証明を構成する.
	今度は$\pi$に現れる$e$を$t$に置き換える必要はない.
	$\varphi_{i}$が$A(t) \Longrightarrow A(e)$でない$EC_{\varepsilon}$の公理ならば,
	$\varphi_{i}$と$\varphi_{i+1}$の間に
	\begin{align}
		&\varphi_{i} \Longrightarrow (\rightharpoondown A(t) \Longrightarrow \varphi_{i}), \\
		&\rightharpoondown A(t) \Longrightarrow \varphi_{i}
	\end{align}
	を挿入する.$\varphi_{i}$が$\varphi_{j}$と$\varphi_{k}$からモーダスポンネスで得られる場合は,
	$\varphi_{i}$を
	\begin{align}
		&(\rightharpoondown A(t) \Longrightarrow \varphi_{j}) \Longrightarrow
		[(\rightharpoondown A(t) \Longrightarrow 
		(\varphi_{j}\Longrightarrow \varphi_{i}))
		\Longrightarrow (\rightharpoondown A(t) \Longrightarrow \varphi_{i})], \\
		&(\rightharpoondown A(t) \Longrightarrow 
		(\varphi_{j} \Longrightarrow \varphi_{i}))
		\Longrightarrow (\rightharpoondown A(t) \Longrightarrow \varphi_{i}), \\
		&\rightharpoondown A(t) \Longrightarrow \varphi_{i}
	\end{align}
	で置き換える.$\varphi_{i}$が$A(t) \Longrightarrow A(e)$であるときは,$\varphi_{i}$を
	\begin{align}
		\rightharpoondown A(t) \Longrightarrow (A(t) \Longrightarrow A(e))
	\end{align}
	で置き換える.
	
	以上で$A(t) \Longrightarrow B$と$\rightharpoondown A(t) \Longrightarrow B$に対して
	$A(t) \Longrightarrow A(e)$を用いない$EC_{\varepsilon}$からの証明が得られた.後はこれに
	\begin{align}
		&(A(t) \Longrightarrow B) \Longrightarrow
		((\rightharpoondown A(t) \Longrightarrow B) \Longrightarrow
		((A(t) \Longrightarrow B) \wedge (\rightharpoondown A(t) \Longrightarrow B))), \\
		&(\rightharpoondown A(t) \Longrightarrow B) \Longrightarrow
		((A(t) \Longrightarrow B) \wedge (\rightharpoondown A(t) \Longrightarrow B)), \\
		&(A(t) \Longrightarrow B) \wedge (\rightharpoondown A(t) \Longrightarrow B), \\
		&((A(t) \Longrightarrow B) \wedge (\rightharpoondown A(t) \Longrightarrow B))
		\Longrightarrow ((A(t) \vee \rightharpoondown A(t)) \Longrightarrow B), \\
		&(A(t) \vee \rightharpoondown A(t)) \Longrightarrow B, \\
		&A(t) \vee \rightharpoondown A(t), \\
		&B
	\end{align}
	を追加すれば,$A(t) \Longrightarrow A(e)$を用いない$EC_{\varepsilon}$から$B$への証明となる.
	\begin{screen}
		\begin{thm}[$\ON$の整列性]\label{thm:On_is_wellordered}
			$\leq$は$\ON$上の整列順序である.また次が成り立つ.
			\begin{align}
				\forall \alpha,\beta \in \ON\,
				\left(\, \alpha \in \beta \vee \alpha = \beta \vee \beta \in \alpha\, \right).
			\end{align}
		\end{thm}
	\end{screen}
	
	\begin{prf}\mbox{}
		\begin{description}
			\item[第一段]
				$\alpha,\beta,\gamma$を順序数とすれば
				\begin{align}
					\alpha \subset \alpha
				\end{align}
				かつ
				\begin{align}
					\alpha \subset \beta \wedge \beta \subset \alpha \Longrightarrow \alpha = \beta
				\end{align}
				かつ
				\begin{align}
					\alpha \subset \beta \wedge \beta \subset \gamma \Longrightarrow \alpha \subset \gamma
				\end{align}
				が成り立つ.ゆえに$\leq$は$\ON$上の順序である.
				
			\item[第二段]
				
			
			\item[第三段]
				$\leq$が整列順序であることを示す.$a$を$\ON$の空でない部分集合とする.このとき正則性公理より
				\begin{align}
					x \in a \wedge x \cap a = \emptyset
				\end{align}
				を満たす集合$x$が取れるが,この$x$が$a$の最小限である.実際,任意に$a$から要素$y$を取ると
				\begin{align}
					x \cap a = \emptyset
				\end{align}
				から
				\begin{align}
					y \notin x
				\end{align}
				が従い,また前段の結果より
				\begin{align}
					x \in y \vee x = y \vee y \in x
				\end{align}
				も成り立つので,選言三段論法より
				\begin{align}
					x \in y \vee x = y
					\label{eq:thm_On_is_wellordered_6}
				\end{align}
				が成り立つ.$y$は推移的であるから
				\begin{align}
					x \in y \Longrightarrow x \subset y
				\end{align}
				が成立して,また
				\begin{align}
					x = y \Longrightarrow x \subset y
				\end{align}
				も成り立つから,(\refeq{eq:thm_On_is_wellordered_6})と場合分け法則から
				\begin{align}
					(x,y) \in\ \leq
				\end{align}
				が従う.$y$の任意性より
				\begin{align}
					\forall y \in a\, \left[\, (x,y) \in\ \leq\, \right]
				\end{align}
				が成立するので$x$は$a$の最小限である.
				\QED
		\end{description}
	\end{prf}
	
	\begin{screen}
		\begin{thm}[$\ON$の部分集合の合併は順序数となる]\label{thm:union_of_set_of_ordinal_numbers_is_ordinal}
			\begin{align}
				\forall a\,
				\left(\, a \subset \ON \Longrightarrow \bigcup a \in \ON\, \right).
			\end{align}
		\end{thm}
	\end{screen}
	
	\begin{prf}
		和集合の公理より$\bigcup a \in \Univ$となる.また順序数の推移性より
		$\bigcup a$の任意の要素は順序数であるから,定理\ref{thm:On_is_wellordered}より
		\begin{align}
			\forall x,y \in \bigcup a\ (\ x \in y \vee x = y \vee y \in x\ )
		\end{align}
		も成り立つ.最後に$\operatorname{Tran}(\bigcup a)$が成り立つことを示す.
		$b$を$\bigcup a$の任意の要素とすれば,$a$の或る要素$x$に対して
		\begin{align}
			b \in x
		\end{align}
		となるが,$x$の推移性より$b \subset x$となり,$x \subset \bigcup a$と併せて
		\begin{align}
			b \subset \bigcup a
		\end{align}
		が従う.
		\QED
	\end{prf}
	
	\begin{screen}
		\begin{dfn}[後者]
			$x$を集合とするとき,
			\begin{align}
				x \cup \{x\}
			\end{align}
			を$x$の{\bf 後者}\index{こうしゃ@後者}{\bf (latter)}と呼ぶ.
		\end{dfn}
	\end{screen}
	
	\begin{screen}
		\begin{thm}[順序数の後者は順序数である]\label{thm:latter_element_is_ordinal}
			$\alpha$が順序数であるということと$\alpha \cup \{\alpha\}$が順序数であるということは同値である.
			\begin{align}
				\forall \alpha\, \left(\, \alpha \in \ON \Longleftrightarrow \alpha \cup \{\alpha\} \in \ON\, \right).
			\end{align}
		\end{thm}
	\end{screen}
	
	\begin{sketch}\mbox{}
		\begin{description}
			\item[第一段]
				$\alpha$を順序数とする.そして$x$を
				\begin{align}
					x \in \alpha \cup \{\alpha\}
					\label{fom:thm_latter_element_is_ordinal_3}
				\end{align}
				なる任意の集合とすると,
				\begin{align}
					y \in x
				\end{align}
				なる任意の集合$y$に対して定理\ref{thm:union_of_pair_is_union_of_their_elements}より
				\begin{align}
					y \in \alpha \vee y \in \{\alpha\}
					\label{fom:thm_latter_element_is_ordinal_5}
				\end{align}
				が成立する.$\alpha$が順序数であるから
				\begin{align}
					y \in \alpha \Longrightarrow y \subset \alpha
					\label{fom:thm_latter_element_is_ordinal_1}
				\end{align}
				が成立する.他方で定理\ref{thm:pair_members_are_exactly_the_given_two}より
				\begin{align}
					y \in \{\alpha\} \Longrightarrow y = \alpha
				\end{align}
				が成立し,
				\begin{align}
					y = \alpha \Longrightarrow y \subset \alpha
				\end{align}
				であるから
				\begin{align}
					y \in \{\alpha\} \Longrightarrow y \subset \alpha
					\label{fom:thm_latter_element_is_ordinal_2}
				\end{align}
				が従う.定理\ref{thm:union_is_bigger_than_any_member}より
				\begin{align}
					y \subset \alpha \Longrightarrow y \subset \alpha \cup \{\alpha\}
				\end{align}
				が成り立つので,(\refeq{fom:thm_latter_element_is_ordinal_1})と
				(\refeq{fom:thm_latter_element_is_ordinal_2})と併せて
				\begin{align}
					y \in \alpha \Longrightarrow y \subset \alpha \cup \{\alpha\}
				\end{align}
				かつ
				\begin{align}
					y \in \{\alpha\} \Longrightarrow y \subset \alpha \cup \{\alpha\}
				\end{align}
				が成立し,場合分け法則より
				\begin{align}
					y \in \alpha \vee y \in \{\alpha\} \Longrightarrow y \subset \alpha \cup \{\alpha\}
				\end{align}
				が従う.そして(\refeq{fom:thm_latter_element_is_ordinal_5})と併せて
				\begin{align}
					y \subset \alpha \cup \{\alpha\}
				\end{align}
				が成立する.$y$の任意性ゆえに(\refeq{fom:thm_latter_element_is_ordinal_3})の下で
				\begin{align}
					\forall y\, \left(\, y \in x \Longrightarrow y \subset \alpha \cup \{\alpha\}\, \right)
				\end{align}
				が成り立ち,演繹法則と$x$の任意性から
				\begin{align}
					\forall x\, \left(\, x \in \alpha \cup \{\alpha\} \Longrightarrow x \subset \alpha \cup \{\alpha\}\, \right)
				\end{align}
				が従う.ゆえにいま
				\begin{align}
					\tran{\alpha \cup \{\alpha\}}
					\label{fom:thm_latter_element_is_ordinal_4}
				\end{align}
				が得られた.また$s$と$t$を$\alpha \cup \{\alpha\}$の任意の要素とすると
				\begin{align}
					s \in \alpha \vee s = \alpha
				\end{align}
				と
				\begin{align}
					t \in \alpha \vee t = \alpha
				\end{align}
				が成り立つが,
				\begin{align}
					s \in \alpha \Longrightarrow s \in \ON
				\end{align}
				かつ
				\begin{align}
					s = \alpha \Longrightarrow s \in \ON
				\end{align}
				から
				\begin{align}
					s \in \alpha \vee s = \alpha \Longrightarrow s \in \ON
				\end{align}
				が従い,同様にして
				\begin{align}
					t \in \alpha \vee t = \alpha \Longrightarrow t \in \ON
				\end{align}
				も成り立つので,
				\begin{align}
					s \in \ON
				\end{align}
				かつ
				\begin{align}
					t \in \ON
				\end{align}
				となる.このとき定理\ref{thm:On_is_wellordered}より
				\begin{align}
					s \in t \vee s = t \vee t \in s
				\end{align}
				が成り立つので,$s$および$t$の任意性より
				\begin{align}
					\forall s,t \in \alpha \cup \{\alpha\}\,
					\left(\, s \in t \vee s = t \vee t \in s\, \right)
				\end{align}
				が得られた.(\refeq{fom:thm_latter_element_is_ordinal_4})と併せて
				\begin{align}
					\ord{\alpha \cup \{\alpha\}}
				\end{align}
				が従い,演繹法則より
				\begin{align}
					\alpha \in \ON \Longrightarrow \alpha \cup \{\alpha\} \in \ON
				\end{align}
				を得る.
				
			\item[第二段]
		\end{description}
	\end{sketch}
	
	\begin{screen}
		\begin{thm}[順序数は後者が直後の数となる]
			$\alpha$を順序数とすれば,$\ON$において$\alpha \cup \{\alpha\}$は$\alpha$の直後の数である:
			\begin{align}
				\forall \alpha \in \ON\, 
				\left[\, \forall \beta \in \ON\, (\, \alpha < \beta 
				\Longrightarrow \alpha \cup \{\alpha\} \leq \beta\, )
				\, \right].
			\end{align}
		\end{thm}
	\end{screen}
	
	\begin{sketch}
		$\alpha$と$\beta$を任意に与えられた順序数とし,
		\begin{align}
			\alpha < \beta
		\end{align}
		であるとする.定理\ref{thm:element_and_proper_subset_correspond_between_ordinal_numbers}より,このとき
		\begin{align}
			\alpha \in \beta
		\end{align}
		が成り立ち,$\leq$の定義より
		\begin{align}
			\alpha \subset \beta
		\end{align}
		も成り立つ.ところで,いま$t$を任意の集合とすると
		\begin{align}
			t \in \{\alpha\} \Longrightarrow t = \alpha
		\end{align}
		かつ
		\begin{align}
			t = \alpha \Longrightarrow t \in \beta
		\end{align}
		が成り立つので,
		\begin{align}
			\{\alpha\} \subset \beta
		\end{align}
		が成り立つ.ゆえに
		\begin{align}
			\forall x\, \left(\, x \in \left\{ \alpha, \{\alpha\} \right\} \Longrightarrow x \subset \beta\, \right)
		\end{align}
		が成り立つ.ゆえに定理\ref{thm:union_of_subsets_is_subclass}より
		\begin{align}
			\alpha \cup \{\alpha\} \subset \beta.
		\end{align}
		すなわち
		\begin{align}
			\alpha \cup \{\alpha\} \leq \beta
		\end{align}
		が成り立つ.
		\QED
	\end{sketch}
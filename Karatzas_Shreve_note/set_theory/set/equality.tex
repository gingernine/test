	$a,b$を$\mathcal{L}$の項とするとき,
	\begin{align}
		a \notin b \defarrow\ \negation a \in b
	\end{align}
	で$a \notin b$を定める.同様に
	\begin{align}
		a \neq b \defarrow\ \negation a = b
	\end{align}
	で$a \neq b$を定める.
	
	類とされた項の多くは集合であるが,{\bf 類が全て集合であると考えると矛盾が起こる}.
	たとえばRussellのパラドックスで有名な
	\begin{align}
		R \defeq \Set{x}{x \notin x}
	\end{align}
	なる類が集合であるとすると($\defarrow$は``式''に対する略記の導入に使ったが,
	$\defeq$とは``類''に対する略記を導入するために使う定義記号である)
	\begin{align}
		\Sigma \vdash R \notin R \lrarrow R \in R
	\end{align}
	が成り立ってしまい(正式な推論は無視してラフに考えれば),これは$\Sigma \vdash \bot$を導く.
	この種の矛盾を回避するために類を導入したのであり,
	集合とは類の中で特定の性質をもつものに限られる.
	
	\begin{screen}
		\begin{dfn}[集合]
			$a$を類とするとき,$a$が集合であるという言明を
			\begin{align}
				\set{a} \defarrow \exists x\, (\, a = x\, )
			\end{align}
			で定める.$\Sigma \vdash \set{a}$を満たす類$a$を
			{\bf 集合}\index{しゅうごう@集合}{\bf (set)}と呼び,
			$\Sigma \vdash\ \negation \set{a}$を満たす類$a$を
			{\bf 真類}\index{しんるい@真類}{\bf (proper class)}と呼ぶ.
		\end{dfn}
	\end{screen}
	
	$\varphi$を$\mathcal{L}$の式とし,$x$を$\varphi$に自由に現れる変項とし,
	$x$のみが$\varphi$で自由であるとする.このとき
	\begin{align}
		\set{\Set{x}{\varphi(x)}} \vdash \set{\Set{x}{\varphi(x)}}
	\end{align}
	が満たされている.つまり
	\begin{align}
		\set{\Set{x}{\varphi(x)}}
		\vdash \exists y\, \left(\, \Set{x}{\varphi(x)} = y\, \right)
	\end{align}
	が成り立っているということであるが,$\Set{x}{\varphi(x)} = y$を
	\begin{align}
		\forall x\, (\, \varphi(x) \lrarrow x \in y\, )
	\end{align}
	と書き換えれば,存在記号の推論規則より
	\begin{align}
		\set{\Set{x}{\varphi(x)}} \vdash \Set{x}{\varphi(x)} = 
		\varepsilon y\, \forall x\, (\, \varphi(x) \lrarrow x \in y\, )
	\end{align}
	が得られる.
	
	\begin{screen}
		\begin{thm}[集合である内包項は$\varepsilon$項で書ける]
		\label{thm:if_a_class_is_a_set_then_equal_to_some_epsilon_term}
			$\varphi$を$\mathcal{L}$の式とし,$x$を$\varphi$に自由に現れる変項とし,
			$x$のみが$\varphi$で自由であるとする.このとき
			\begin{align}
				\set{\Set{x}{\varphi(x)}} \vdash \Set{x}{\varphi(x)} 
				= \varepsilon y\, \forall x\, (\, \varphi(x) \lrarrow x \in y\, ).
			\end{align}
		\end{thm}
	\end{screen}
	
	ブルバキ\cite{key4}では$\tau$項を,島内\cite{key6}では$\varepsilon$項のみを導入して
	$\varepsilon y \forall x\, (\, \varphi(x) \lrarrow x \in y\, )$
	によって$\Set{x}{\varphi(x)}$を定めているが,この定め方には欠点がある.
	というのも,本稿と同じくブルバキ\cite{key4}の$\tau$項も島内\cite{key6}の$\varepsilon$項も
	集合であるから,
	\begin{align}
		\exists y\, \forall x\, (\, \varphi(x) \lrarrow x \in y\, )
	\end{align}
	が成立しない場合は$\varepsilon y \forall x\, (\, \varphi(x) \lrarrow x \in y\, )$
	は正体不明になってしまい,$\Set{x}{\varphi(x)}$が「性質$\varphi$を持つ集合の全体」
	の意味を持たないのである.本稿では内包項と$\varepsilon$項を別々に
	生成しているのでこの欠点は解消される.
	
\section{相等性}
	本稿において``等しい''とは項に対する言明であって,$a$と$b$を項とするとき
	\begin{align}
		a = b
	\end{align}
	なる式で表される.この記号
	\begin{align}
		=
	\end{align}
	は{\bf 等号}\index{とうごう@等号}{\bf (equal sign)}と呼ばれるが,
	現時点では述語として導入されているだけで,推論操作における働きは不明のままである.
	本節では,いつ類は等しくなるのか,そして,等しい場合に何が起きるのか,の二つが主題となる.
	
	\begin{screen}
		\begin{axm}[外延性の公理 (Extensionality)]
			任意の類$a,b$に対して
			\begin{align}
				\EXTAX \defarrow \forall x\, (\, x \in a \lrarrow x \in b\, ) 
				\rarrow a=b.
			\end{align}
		\end{axm}
	\end{screen}
	
	\begin{screen}
		\begin{thm}[任意の類は自分自身と等しい]\label{thm:any_class_equals_to_itself}
			任意の類$\tau$に対して
			\begin{align}
				\EXTAX \vdash \tau = \tau.
			\end{align}
		\end{thm}
	\end{screen}
	
	\begin{sketch}
		いま
		\begin{align}
			\sigma \defeq 
			\varepsilon s \negation (\, s \in \tau \lrarrow s \in \tau\, )
		\end{align}
		とおく.推論法則\ref{logicalthm:reflective_law_of_implication}より
		\begin{align}
			\vdash \sigma \in \tau \lrarrow \sigma \in \tau
		\end{align}
		が成り立つから,全称記号の推論規則より
		\begin{align}
			\vdash \forall s\, (\, s \in \tau  \lrarrow s \in \tau\, )
		\end{align}
		が成り立つ.外延性の公理より
		\begin{align}
			\EXTAX \vdash \forall s\, (\, s \in \tau  \lrarrow s \in \tau\, )
			\rarrow \tau = \tau
		\end{align}
		となるので,三段論法より
		\begin{align}
			\EXTAX \vdash \tau = \tau
		\end{align}
		が得られる.
		\QED
	\end{sketch}
	
	\begin{screen}
		\begin{thm}[主要$\varepsilon$項は集合である]
		\label{thm:critical_epsilon_term_is_set}
			$\tau$を類である$\varepsilon$項とするとき
			\begin{align}
				\EXTAX \vdash \set{\tau}.
			\end{align}
		\end{thm}
	\end{screen}
	
	\begin{sketch}
		定理\ref{thm:any_class_equals_to_itself}より
		\begin{align}
			\EXTAX \vdash \tau = \tau
		\end{align}
		が成立するので,存在記号の推論規則より
		\begin{align}
			\EXTAX \vdash \exists x\, \left(\, \tau = x\, \right)
		\end{align}
		が成立する.
		\QED
	\end{sketch}
	
	例えば
	\begin{align}
		a = b
	\end{align}
	と書いてあったら``$a$と$b$は等しい''と読めるわけだが,明らかに$a$は$b$とは違うではないではないか!
	こんなことはしょっちゅう起こることであって,上で述べたように$\Set{x}{A(x)}$が集合なら
	\begin{align}
		\Set{x}{A(x)} = \varepsilon y \forall x\, \left(\, A(x) \lrarrow x \in y\, \right)
	\end{align}
	が成り立ったりする.そこで``数学的に等しいとは何事か''という疑問が浮かぶのは至極自然であって,
	それに答えるのが次の相等性公理である.
	
	\begin{screen}
		\begin{axm}[相等性公理]
			$a,b,c$を類とするとき
			\begin{align}
				\EQAX \defarrow
				\begin{cases}
					a = b \rarrow b = a, & \\
					a = b \rarrow (\, a \in c \rarrow b \in c\, ), & \\
					a = b \rarrow (\, c \in a \rarrow c \in b\, ). & 
				\end{cases}
			\end{align}
		\end{axm}
	\end{screen}
	
	\begin{screen}
		\begin{thm}[外延性の公理の逆も成り立つ]
		\label{thm:inverse_of_axiom_of_extensionality}
			$a$と$b$を類とするとき
			\begin{align}
				\EQAX \vdash 
				a = b \rarrow \forall x\, (\, x \in a  \lrarrow x \in b\, ).
			\end{align}
		\end{thm}
	\end{screen}
	
	\begin{prf}
		いま
		\begin{align}
			\tau \defeq \varepsilon x \negation (\, x \in a  \lrarrow x \in b\, )
		\end{align}
		とおく.相等性公理より
		\begin{align}
			\EQAX \vdash a = b \rarrow (\, \tau \in a \rarrow \tau \in b\, )
		\end{align}
		となるので,演繹法則の逆より
		\begin{align}
			a = b,\ \EQAX \vdash \tau \in a \rarrow \tau \in b
			\label{fom:inverse_of_axiom_of_extensionality_1}
		\end{align}
		となる.また相等性公理と演繹法則の逆により
		\begin{align}
			a = b,\ \EQAX \vdash b = a
		\end{align}
		が成り立ち,同じく相等性公理より
		\begin{align}
			\EQAX \vdash b = a \rarrow (\, \tau \in b \rarrow \tau \in a\, )
		\end{align}
		も成り立つので,三段論法より
		\begin{align}
			a = b,\ \EQAX \vdash \tau \in b \rarrow \tau \in a
			\label{fom:inverse_of_axiom_of_extensionality_2}
		\end{align}
		も得られる.論理積の導入により
		\begin{align}
			a = b,\ \EQAX \vdash (\, \tau \in a \rarrow \tau \in b\, )
			\rarrow (\, (\, \tau \in b \rarrow \tau \in a\, )
			\rarrow (\, \tau \in a \lrarrow \tau \in b\, )\, )
		\end{align}
		が成り立つので,(\refeq{fom:inverse_of_axiom_of_extensionality_1})との三段論法より
		\begin{align}
			a = b,\ \EQAX \vdash (\, \tau \in b \rarrow \tau \in a\, )
			\rarrow (\, \tau \in a \lrarrow \tau \in b\, )
		\end{align}
		が従い,(\refeq{fom:inverse_of_axiom_of_extensionality_2})との三段論法より
		\begin{align}
			a = b,\ \EQAX \vdash \tau \in a \lrarrow \tau \in b
		\end{align}
		が従う.全称記号の推論規則より
		\begin{align}
			a = b,\ \EQAX \vdash \forall x\, (\, x \in a  \lrarrow x \in b\, )
		\end{align}
		が成立し,演繹法則より
		\begin{align}
			\EQAX \vdash a = b \rarrow \forall x\, (\, x \in a  \lrarrow x \in b\, )
		\end{align}
		が得られる.
		\QED
	\end{prf}
	
	\begin{comment}
	\begin{screen}
		\begin{thm}[(ボツ!!!)等号の対称律]\label{thm:symmetry_of_equality}
			$a,b$を類とするとき
			\begin{align}
				\EXTAX,\EQAX \vdash a = b \rarrow b = a.
			\end{align}
		\end{thm}
	\end{screen}
	
	\begin{prf}
		定理\ref{thm:axiom_of_extensionality_equivalent}より
		\begin{align}
			a=b,\ \EQAX \vdash \forall x\, (\, x \in a  \lrarrow x \in b\, )
		\end{align}
		となるが,ここで類である任意の$\varepsilon$項$\tau$に対して
		\begin{align}
			a=b,\ \EQAX \vdash \tau \in a \lrarrow \tau \in b
		\end{align}
		となるが,他方で推論法則\ref{logicalthm:symmetry_of_equivalence_arrows}より
		\begin{align}
			a=b,\ \EQAX \vdash (\, \tau \in a \lrarrow \tau \in b\, )
				\rarrow (\, \tau \in b \lrarrow \tau \in a\, )
		\end{align}
		が成り立つので,三段論法より
		\begin{align}
			a=b,\ \EQAX \vdash \tau \in b \lrarrow \tau \in a
		\end{align}
		となる.そして$\tau$の任意性より
		\begin{align}
			a=b,\ \EQAX \vdash \forall x\, (\, x \in b  \lrarrow x \in a\, )
		\end{align}
		が成り立つ.外延性の公理より
		\begin{align}
			a=b,\ \EXTAX,\EQAX \vdash \forall x\, (\, x \in b  \lrarrow x \in a\, )
			\rarrow b = a
		\end{align}
		となるので,三段論法より
		\begin{align}
			a=b,\ \EXTAX,\EQAX \vdash b = a
		\end{align}
		となる.最後に演繹法則より
		\begin{align}
			\EXTAX,\EQAX \vdash a = b \rarrow b = a
		\end{align}
		が得られる.
		\QED
	\end{prf}
	\end{comment}
	
	\begin{screen}
		\begin{axm}[内包性公理] 
			$\varphi$を$\mathcal{L}$の式とし,$y$を$\varphi$に自由に現れる変項とし,
			$\varphi$に自由に現れる項は$y$のみであるとし,
			$x$は$\varphi$で$y$への代入について自由であるとするとき,
			\begin{align}
				\COMAX \defarrow \forall x\, (\, x \in \Set{y}{\varphi(y)} \lrarrow \varphi(x)\, ).
			\end{align}
		\end{axm}
	\end{screen}
	
	\begin{screen}
		\begin{thm}[条件を満たす集合は要素である]\label{thm:satisfactory_set_is_an_element}
			$\varphi$を$\mathcal{L}$の式とし,$x$を$\varphi$に自由に現れる変項とし,
			$x$のみが$\varphi$で束縛されていないとする.このとき,任意の類$a$に対して
			\begin{align}
				\EQAX,\COMAX \vdash \varphi(a) \rarrow 
				\left(\, \set{a} \rarrow a \in \Set{x}{\varphi(x)}\, \right).
			\end{align}
		\end{thm}
	\end{screen}
	
	\begin{sketch}
		\begin{align}
			\set{a} \vdash \exists x\, (\, a = x\, )
		\end{align}
		より,
		\begin{align}
			\tau \defeq \varepsilon x\, (\, a = x\, )
		\end{align}
		とおけば
		\begin{align}
			\set{a} \vdash a = \tau
		\end{align}
		となる.相等性の公理より
		\begin{align}
			\set{a},\EQAX \vdash 
			a = \tau \rarrow (\, \varphi(a) \rarrow \varphi(\tau)\, )
		\end{align}
		となるので,三段論法と演繹法則の逆より
		\begin{align}
			\varphi(a),\set{a},\EQAX \vdash \varphi(\tau)
		\end{align}
		となる.内包性公理より
		\begin{align}
			\varphi(a),\set{a},\EQAX,\COMAX \vdash \tau \in \Set{x}{A(x)}
		\end{align}
		が従い,相等性の公理から
		\begin{align}
			\varphi(a),\set{a},\EQAX,\COMAX \vdash a \in \Set{x}{A(x)}
		\end{align}
		が成立する.演繹法則より
		\begin{align}
			\varphi(a),\EQAX,\COMAX &\vdash \set{a} \rarrow a \in \Set{x}{A(x)}, \\
			\EQAX,\COMAX &\vdash \varphi(a) \rarrow 
			\left(\, \set{a} \rarrow a \in \Set{x}{\varphi(x)}\, \right)
		\end{align}
		が従う.
		\QED
	\end{sketch}
	
	\begin{screen}
		\begin{dfn}[宇宙]
			$\Univ \defeq \Set{x}{x=x}$で定める類$\Univ$を{\bf 宇宙}\index{うちゅう@宇宙}
			{\bf (Universe)}と呼ぶ.
		\end{dfn}
	\end{screen}
	
	定理\ref{thm:V_is_the_whole_of_sets}の通り宇宙とは集合の全体を表すが,
	これ自体は集合ではない.また$\Univ$のより具体的な構造ものちに判る.
	ちなみに名前のVとはVon NeumannのVである.
	
	\begin{screen}
		\begin{axm}[要素の公理]
			要素となりうる類は集合である.つまり,$a,b$を類とするとき
			\begin{align}
				\ELEAX \defarrow a \in b \rarrow \set{a}.
			\end{align}
		\end{axm}
	\end{screen}
	
	\begin{screen}
		\begin{thm}[$\Univ$は集合の全体である]
		\label{thm:V_is_the_whole_of_sets}
			$a$を類とするとき次が成り立つ:
			\begin{align}
				\EXTAX,\EQAX,\ELEAX,\COMAX \vdash \set{a} \lrarrow a \in \Univ.
			\end{align}
		\end{thm}
	\end{screen}
	
	\begin{prf}
		$a$を類とするとき,まず要素の公理より
		\begin{align}
			\ELEAX \vdash a \in \Univ \rarrow \set{a}
		\end{align}
		が得られる.逆を示す.いま
		\begin{align}
			\tau \defeq \varepsilon x\, (\, a = x\, )
		\end{align}
		とおくと,
		\begin{align}
			\set{a} \vdash \exists x\, (\, a = x\, )
		\end{align}
		と
		\begin{align}
			\set{a} \vdash \exists x\, (\, a = x\, ) \rarrow a = \tau
		\end{align}
		(存在記号の推論規則)より
		\begin{align}
			\set{a} \vdash a = \tau
			\label{fom:thm_V_is_the_whole_of_sets_1}
		\end{align}
		が成り立つ.他方で定理\ref{thm:any_class_equals_to_itself}と内包性公理より
		\begin{align}
			\EXTAX &\vdash \tau = \tau, \\
			\COMAX &\vdash \tau = \tau \rarrow \tau \in \Univ
		\end{align}
		が成り立つので,三段論法より
		\begin{align}
			\EXTAX,\COMAX \vdash \tau \in \Univ
			\label{fom:thm_V_is_the_whole_of_sets_2}
		\end{align}
		となる.ここで相等性公理より
		\begin{align}
			\EQAX \vdash a = \tau \rarrow \tau = a
		\end{align}
		が成り立つので,(\refeq{fom:thm_V_is_the_whole_of_sets_1})と三段論法より
		\begin{align}
			\set{a},\EQAX \vdash \tau = a
			\label{fom:thm_V_is_the_whole_of_sets_3}
		\end{align}
		となる.同じく相等性公理より
		\begin{align}
			\EQAX \vdash \tau = a \rarrow (\, \tau \in \Univ \rarrow a \in \Univ\, )
		\end{align}
		が成り立つので,(\refeq{fom:thm_V_is_the_whole_of_sets_3})と三段論法より
		\begin{align}
			\set{a},\ \EQAX \vdash \tau \in \Univ \rarrow a \in \Univ
		\end{align}
		となり,(\refeq{fom:thm_V_is_the_whole_of_sets_2})と三段論法より
		\begin{align}
			\set{a},\ \EXTAX,\EQAX,\COMAX \vdash a \in \Univ
		\end{align}
		が成り立つ.最後に演繹法則より
		\begin{align}
			\EXTAX,\EQAX,\COMAX \vdash \set{a} \rarrow a \in \Univ
		\end{align}
		が得られる.
		\QED
	\end{prf}
	
	\begin{screen}
		\begin{logicalthm}[同値関係の可換律]
		\label{logicalthm:commutative_law_of_equivalence_symbol}
			$A,B$を$\mathcal{L}$の文とするとき
			\begin{align}
				\vdash (A \lrarrow B) \rarrow (B \lrarrow A).
			\end{align}
		\end{logicalthm}
	\end{screen}
	
	\begin{sketch}
		論理積の除去規則より
		\begin{align}
			A \lrarrow B &\vdash A \rarrow B, 
			\label{fom:logicalthm_commutative_law_of_equivalence_symbol_1} \\
			A \lrarrow B &\vdash B \rarrow A
			\label{fom:logicalthm_commutative_law_of_equivalence_symbol_2}
		\end{align}
		となる.他方で論理積の導入規則より
		\begin{align}
			\vdash (B \rarrow A) \rarrow ((A \rarrow B) \rarrow (B \lrarrow A))
		\end{align}
		が成り立つので
		\begin{align}
			A \lrarrow B \vdash (B \rarrow A) \rarrow ((A \rarrow B) \rarrow (B \lrarrow A))
		\end{align}
		も成り立つ.これと(\refeq{fom:logicalthm_commutative_law_of_equivalence_symbol_1})
		との三段論法より
		\begin{align}
			A \lrarrow B \vdash (A \rarrow B) \rarrow (B \lrarrow A)
		\end{align}
		となり,(\refeq{fom:logicalthm_commutative_law_of_equivalence_symbol_2})
		との三段論法より
		\begin{align}
			A \lrarrow B \vdash B \lrarrow A
		\end{align}
		が得られる.
		\QED
	\end{sketch}
	
	\begin{screen}
		\begin{logicalthm}[同値関係の推移律]
		\label{logicalthm:transitive_law_of_equivalence_symbol}
			$A,B,C$を$\mathcal{L}$の文とするとき
			\begin{align}
				\vdash (A \lrarrow B) \rarrow ((B \lrarrow C) \rarrow 
				(A \lrarrow C)).
			\end{align}
		\end{logicalthm}
	\end{screen}
	
	\begin{sketch}
		論理積の除去法則より
		\begin{align}
			A \lrarrow B &\vdash A \rarrow B, \\
			A \lrarrow B &\vdash B \rarrow A
		\end{align}
		が成り立つので
		\begin{align}
			A \lrarrow B,\ B \lrarrow C &\vdash A \rarrow B, 
			\label{fom:transitive_law_of_equivalence_symbol_1} \\
			A \lrarrow B,\ B \lrarrow C &\vdash B \rarrow A
		\end{align}
		も成り立つし,対称的に
		\begin{align}
			A \lrarrow B,\ B \lrarrow C &\vdash B \rarrow C, 
			\label{fom:transitive_law_of_equivalence_symbol_2} \\
			A \lrarrow B,\ B \lrarrow C &\vdash C \rarrow B
		\end{align}
		も成り立つ.含意の推移律(推論法則\ref{logicalthm:transitive_law_of_implication})より
		\begin{align}
			\vdash (A \rarrow B) \rarrow ((B \rarrow C) \rarrow (A \rarrow C))
		\end{align}
		となるので,(\refeq{fom:transitive_law_of_equivalence_symbol_1})との三段論法より
		\begin{align}
			A \lrarrow B,\ B \lrarrow C \vdash (B \rarrow C) \rarrow (A \rarrow C)
		\end{align}
		が成り立ち,(\refeq{fom:transitive_law_of_equivalence_symbol_2})との三段論法より
		\begin{align}
			A \lrarrow B,\ B \lrarrow C \vdash A \rarrow C
			\label{fom:transitive_law_of_equivalence_symbol_3}
		\end{align}
		が成り立つ.同様にして
		\begin{align}
			A \lrarrow B,\ B \lrarrow C \vdash C \rarrow A
			\label{fom:transitive_law_of_equivalence_symbol_4}
		\end{align}
		も得られる.論理積の導入規則より
		\begin{align}
			\vdash (A \rarrow C) \rarrow ((C \rarrow A) \rarrow (A \lrarrow C))
		\end{align}
		が成り立つので,(\refeq{fom:transitive_law_of_equivalence_symbol_3})との三段論法より
		\begin{align}
			A \lrarrow B,\ B \lrarrow C \vdash (C \rarrow A) \rarrow (A \lrarrow C)
		\end{align}
		となり,(\refeq{fom:transitive_law_of_equivalence_symbol_4})との三段論法より
		\begin{align}
			A \lrarrow B,\ B \lrarrow C \vdash A \lrarrow C
		\end{align}
		となる.あとは演繹規則を二回適用すれば
		\begin{align}
			\vdash (A \lrarrow B) \rarrow ((B \lrarrow C) \rarrow (A \lrarrow C))
		\end{align}
		が得られる.
		\QED
	\end{sketch}
	
	\begin{screen}
		\begin{thm}[等号の推移律]\label{thm:transitive_law_of_equality}
			$a,b,c$を類とするとき
			\begin{align}
				\EXTAX,\EQAX \vdash a = b \rarrow (\, a = c \rarrow b = c\, ).
			\end{align}
		\end{thm}
	\end{screen}
	
	\begin{sketch}
		まずは
		\begin{align}
			a = b,\ a = c,\ \EQAX \vdash \forall x\, (\, x \in b \lrarrow x \in c\, )
		\end{align}
		を示したいので
		\begin{align}
			\tau \defeq \varepsilon x \negation (\, x \in b \lrarrow x \in c\, )
		\end{align}
		とおく($b,c$が$\lang{\varepsilon}$の項でなければ
		$x \in b \lrarrow x \in c$を書き換える).相等性公理より
		\begin{align}
			a = b,\ a = c,\ \EQAX \vdash a = b \rarrow (\, \tau \in a \rarrow \tau \in b\, )
		\end{align}
		が成り立つので,
		\begin{align}
			a = b,\ a = c,\ \EQAX \vdash a = b
			\label{fom:thm_transitive_law_of_equality_0}
		\end{align}
		との三段論法より
		\begin{align}
			a = b,\ a = c,\ \EQAX \vdash \tau \in a \rarrow \tau \in b
			\label{fom:thm_transitive_law_of_equality_1}
		\end{align}
		となる.同じく相等性公理より
		\begin{align}
			a = b,\ a = c,\ \EQAX \vdash a = b \rarrow b = a, \\
		\end{align}
		が成り立つので,(\refeq{fom:thm_transitive_law_of_equality_0})との三段論法より
		\begin{align}
			a = b,\ a = c,\ \EQAX \vdash b = a
		\end{align}
		となり,同様に相等性公理から
		\begin{align}
			a = b,\ a = c,\ \EQAX \vdash b = a \rarrow (\, \tau \in b \rarrow \tau \in a\, )
		\end{align}
		が成り立つので,三段論法より
		\begin{align}
			a = b,\ a = c,\ \EQAX \vdash \tau \in b \rarrow \tau \in a
			\label{fom:thm_transitive_law_of_equality_2}
		\end{align}
		となる.論理積の導入規則より
		\begin{align}
			a = b,\ a = c,\ \EQAX \vdash (\tau \in a \rarrow \tau \in b)
			\rarrow ((\tau \in b \rarrow \tau \in a) \rarrow 
			(\tau \in a \lrarrow \tau \in b))
		\end{align}
		が成り立つので,(\refeq{fom:thm_transitive_law_of_equality_1})との三段論法より
		\begin{align}
			a = b,\ a = c,\ \EQAX \vdash (\tau \in b \rarrow \tau \in a) \rarrow 
			(\tau \in a \lrarrow \tau \in b)
		\end{align}
		となり,(\refeq{fom:thm_transitive_law_of_equality_2})との三段論法より
		\begin{align}
			a = b,\ a = c,\ \EQAX \vdash \tau \in a \lrarrow \tau \in b
			\label{fom:thm_transitive_law_of_equality_4}
		\end{align}
		となる.対称的に
		\begin{align}
			a = b,\ a = c,\ \EQAX \vdash \tau \in a \lrarrow \tau \in c
			\label{fom:thm_transitive_law_of_equality_3}
		\end{align}
		も得られる.ここで含意の可換律
		(推論法則\ref{logicalthm:commutative_law_of_equivalence_symbol})より
		\begin{align}
			a = b,\ a = c,\ \EQAX \vdash (\, \tau \in a \lrarrow \tau \in b\, )
			\rarrow (\, \tau \in b \lrarrow \tau \in a\, ) 
		\end{align}
		が成り立つので,(\refeq{fom:thm_transitive_law_of_equality_4})との三段論法より
		\begin{align}
			a = b,\ a = c,\ \EQAX \vdash \tau \in b \lrarrow \tau \in a
			\label{fom:thm_transitive_law_of_equality_5}
		\end{align}
		となる.また含意の推移律
		(推論法則\ref{logicalthm:transitive_law_of_equivalence_symbol})より
		\begin{align}
			a = b,\ a = c,\ \EQAX \vdash (\, \tau \in b \lrarrow \tau \in a\, )
			\rarrow ((\, \tau \in a \lrarrow \tau \in c\, )
			\rarrow (\, \tau \in b \lrarrow \tau \in c\, )) 
		\end{align}
		が成り立つので,(\refeq{fom:thm_transitive_law_of_equality_5})との三段論法より
		\begin{align}
			a = b,\ a = c,\ \EQAX \vdash (\, \tau \in a \lrarrow \tau \in c\, )
			\rarrow (\, \tau \in b \lrarrow \tau \in c\, )
		\end{align}
		となり,(\refeq{fom:thm_transitive_law_of_equality_3})との三段論法より
		\begin{align}
			a = b,\ a = c,\ \EQAX \vdash \tau \in b \lrarrow \tau \in c
		\end{align}
		が得られる.全称記号の推論規則より
		\begin{align}
			a = b,\ a = c,\ \EQAX \vdash (\tau \in b \lrarrow \tau \in c)
			\rarrow \forall x\, (\, x \in b \lrarrow x \in c\, )
		\end{align}
		となるので,三段論法より
		\begin{align}
			a = b,\ a = c,\ \EQAX \vdash \forall x\, (\, x \in b \lrarrow x \in c\, )
		\end{align}
		となり,外延性公理より
		\begin{align}
			a = b,\ a = c,\ \EXTAX,\EQAX \vdash \forall x\, (\, x \in b \lrarrow x \in c\, )
			\rarrow b = c
		\end{align}
		となるので,三段論法より
		\begin{align}
			a = b,\ a = c,\ \EXTAX,\EQAX \vdash b = c
		\end{align}
		が得られる.
		\QED
	\end{sketch}
	
	\begin{itembox}[l]{等号の対称律と推移律について}
		本稿では等号の対称律
		\begin{align}
			a = b \rarrow b = a
		\end{align}
		を公理としたが,逆に推移律を公理にすれば
		\begin{align}
			\EXTAX,\EQAX \vdash a = b \rarrow b = a
		\end{align}
		が成立する.実際
		\begin{align}
			a = b,\ \EQAX &\vdash a = a \rarrow b = a, && \\
			\EXTAX &\vdash a = a, 
			&& (\mbox{定理\ref{thm:any_class_equals_to_itself}}), \\
			a = b,\ \EXTAX,\EQAX &\vdash b = a
			&& (\mbox{三段論法})
		\end{align}
		となる.つまり等号の対称律と推移律は外延性公理の下で同値なのである.
	\end{itembox}
\section{関係}
	\begin{screen}
		\begin{dfn}[順序対]
			$x$と$y$を$\mathcal{L}$の項とするとき,
			\begin{align}
				(x,y) \defeq \{\{x\},\{x,y\}\}
			\end{align}
			で定める項$(x,y)$を$x$と$y$の{\bf 順序対}\index{じゅんじょつい@順序対}
			{\bf (ordered pair)}と呼ぶ.
		\end{dfn}
	\end{screen}
	
	\begin{screen}
		\begin{thm}[集合の順序対は集合]
		\label{thm:ordered_pair_of_sets_is_a_set}
			$a$と$b$を類とするとき
			\begin{align}
				\EXTAX,\EQAX,\COMAX,\PAIAX \vdash
				\set{a} \wedge \set{b} \rarrow \set{(a,b)}.
			\end{align}
		\end{thm}
	\end{screen}
	
	\begin{prf}
		集合の対は集合(定理\ref{thm:pair_of_sets_is_a_set})であるから
		\begin{align}
			\set{a},\ \set{b},\ \EXTAX,\EQAX,\COMAX,\PAIAX &\vdash \set{\{a\}}, \\
			\set{a},\ \set{b},\ \EXTAX,\EQAX,\COMAX,\PAIAX &\vdash \set{\{a,b\}}
		\end{align}
		が成り立つので
		\begin{align}
			\set{a},\ \set{b},\ \EXTAX,\EQAX,\COMAX,\PAIAX \vdash 
			\set{\{a\}} \wedge \set{\{a,b\}}
		\end{align}
		が従い,再び定理\ref{thm:pair_of_sets_is_a_set}より
		\begin{align}
			\set{a},\ \set{b},\ \EXTAX,\EQAX,\COMAX,\PAIAX \vdash \set{(a,b)}
		\end{align}
		となる.
		\QED
	\end{prf}
	
	\begin{screen}
		\begin{thm}[順序対の相等性]
		\label{thm:equality_of_ordered_pairs}
			$a,b,c,d$を集合とするとき
			\begin{align}
				(a,b) = (c,d) \rarrow a=c \wedge b=d.
			\end{align}
		\end{thm}
	\end{screen}
	
	\begin{sketch}
		$(a,b) = (c,d)$と仮定すると,
		\begin{align}
			\{a\} \in \{\{c\},\{c,d\}\}
		\end{align}
		より
		\begin{align}
			\{a\} = \{c\} \vee \{a\} = \{c,d\}
		\end{align}
		が成り立つ.
		\begin{align}
			\{a\} = \{c\} \rarrow a \in \{c\} \rarrow a = c
		\end{align}
		となるし,
		\begin{align}
			\{a\} = \{c,d\} \rarrow c \in \{a\} \rarrow a = c
		\end{align}
		となるので
		\begin{align}
			a = c
		\end{align}
		が成り立つ.ゆえに
		\begin{align}
			\{\{a\},\{a,b\}\} = \{\{a\},\{a,d\}\}
		\end{align}
		である.$(a,b) = (c,d)$に加えて
		\begin{align}
			a = d
		\end{align}
		と仮定すると,
		\begin{align}
			\{a,b\} = \{a\} \vee \{a,b\} = \{a,d\}
		\end{align}
		と
		\begin{align}
			\{a,b\} = \{a\} \rarrow b = a = d
		\end{align}
		となり,
		\begin{align}
			\{a,b\} = \{a,d\} &\rarrow b = a \vee b = d, \\
			b = a &\rarrow b = d, \\
			b = d &\rarrow b = d
		\end{align}
		より
		\begin{align}
			\{a,b\} = \{a,d\} \rarrow b = d
		\end{align}
		も成り立つ.ゆえに
		\begin{align}
			a = d \rarrow b = d
		\end{align}
		である.今度は$(a,b) = (c,d)$に加えて
		\begin{align}
			a \neq d
		\end{align}
		と仮定する.
		\begin{align}
			\{a,d\} = \{a\} \vee \{a,d\} = \{a,b\}
		\end{align}
		と
		\begin{align}
			\{a,d\} \neq \{a\}
		\end{align}
		より
		\begin{align}
			\{a,d\} = \{a,b\}
		\end{align}
		が成り立ち,
		\begin{align}
			d = a \vee d = b
		\end{align}
		が成り立つ.$d \neq a$より
		\begin{align}
			d = b
		\end{align}
		が従う.ゆえに
		\begin{align}
			a \neq d \rarrow b = d
		\end{align}
		でもある.
		\QED
	\end{sketch}
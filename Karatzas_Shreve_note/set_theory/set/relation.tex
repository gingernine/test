\section{関係}
	\begin{screen}
		\begin{dfn}[順序対]
			$x$と$y$を$\mathcal{L}$の項とするとき,
			\begin{align}
				(x,y) \defeq \{\{x\},\{x,y\}\}
			\end{align}
			で定める項$(x,y)$を$x$と$y$の{\bf 順序対}\index{じゅんじょつい@順序対}
			{\bf (ordered pair)}と呼ぶ.
		\end{dfn}
	\end{screen}
	
	\begin{screen}
		\begin{thm}[集合の順序対は集合]
		\label{thm:ordered_pair_of_sets_is_a_set}
			$a$と$b$を類とするとき
			\begin{align}
				\EXTAX,\EQAX,\COMAX,\PAIAX \vdash
				\set{a} \wedge \set{b} \rarrow \set{(a,b)}.
			\end{align}
		\end{thm}
	\end{screen}
	
	\begin{prf}
		集合の対は集合(定理\ref{thm:pair_of_sets_is_a_set})であるから
		\begin{align}
			\set{a},\ \set{b},\ \EXTAX,\EQAX,\COMAX,\PAIAX &\vdash \set{\{a\}}, \\
			\set{a},\ \set{b},\ \EXTAX,\EQAX,\COMAX,\PAIAX &\vdash \set{\{a,b\}}
		\end{align}
		が成り立つので
		\begin{align}
			\set{a},\ \set{b},\ \EXTAX,\EQAX,\COMAX,\PAIAX \vdash 
			\set{\{a\}} \wedge \set{\{a,b\}}
		\end{align}
		が従い,再び定理\ref{thm:pair_of_sets_is_a_set}より
		\begin{align}
			\set{a},\ \set{b},\ \EXTAX,\EQAX,\COMAX,\PAIAX \vdash \set{(a,b)}
		\end{align}
		となる.
		\QED
	\end{prf}
	
	\begin{screen}
		\begin{thm}[順序対の相等性]
		\label{thm:equality_of_ordered_pairs}
			$a,b,c,d$を集合とするとき
			\begin{align}
				(a,b) = (c,d) \rarrow a=c \wedge b=d.
			\end{align}
		\end{thm}
	\end{screen}
	
	\begin{sketch}\mbox{}
		\begin{description}
			\item[step1] 集合は自分自身の対の要素である(定理\ref{thm:set_is_an_element_of_its_pair})から
				\begin{align}
					\set{\{a\}},\ \EXTAX,\EQAX,\COMAX \vdash \{a\} \in (a,b)
				\end{align}
				が成り立つ.従って$(a,b) = (c,d)$と仮定すると,相等性公理より
				\begin{align}
					(a,b) = (c,d),\ \set{\{a\}},\ \EXTAX,\EQAX,\COMAX \vdash \{a\} \in (c,d)
				\end{align}
				が成り立つ.定理\ref{cor:pair_members_are_exactly_the_given_two}
				(対の要素は表示されている要素の一方には等しい)より
				\begin{align}
					\EXTAX,\EQAX,\COMAX,\ELEAX \vdash \{a\} \in (c,d) \rarrow \{a\} = \{c\} \vee \{a\} = \{c,d\}
					\label{fom:equality_of_ordered_pairs_1}
				\end{align}
				となるから,三段論法より
				\begin{align}
					(a,b) = (c,d),\ \set{\{a\}},\ \EXTAX,\EQAX,\COMAX,\ELEAX \vdash \{a\} = \{c\} \vee \{a\} = \{c,d\}
				\end{align}
				が従い,演繹定理より
				\begin{align}
					(a,b) = (c,d),\ \EXTAX,\EQAX,\COMAX,\ELEAX \vdash
					\set{\{a\}} \rarrow \{a\} = \{c\} \vee \{a\} = \{c,d\}
				\end{align}
				が従う.ところで集合の対は集合(定理\ref{thm:pair_of_sets_is_a_set})なので
				\begin{align}
					\set{a},\ \EXTAX,\EQAX,\COMAX,\PAIAX \vdash \set{\{a\}}
				\end{align}
				が成り立つから,三段論法より
				\begin{align}
					(a,b) = (c,d),\ \set{a},\ \EXTAX,\EQAX,\COMAX,\ELEAX \vdash \{a\} = \{c\} \vee \{a\} = \{c,d\}
					\label{fom:equality_of_ordered_pairs_2}
				\end{align}
				が従う.
		
		
			\item[step2] 定理\ref{thm:set_is_an_element_of_its_pair}より
				\begin{align}
					\set{a},\ \EXTAX,\EQAX,\COMAX \vdash a \in \{a\}
				\end{align}
				が成り立つから,相等性公理より
				\begin{align}
					\{a\} = \{c\},\ \set{a},\ \EXTAX,\EQAX,\COMAX \vdash a \in \{c\}
				\end{align}
				となる.また定理\ref{cor:pair_members_are_exactly_the_given_two}より
				\begin{align}
					\EXTAX,\EQAX,\COMAX,\ELEAX \vdash a \in \{c\} \rarrow a = c
				\end{align}
				となるから,三段論法より
				\begin{align}
					\{a\} = \{c\},\ \set{a},\ \EXTAX,\EQAX,\COMAX,\ELEAX \vdash a = c
					\label{fom:equality_of_ordered_pairs_3}
				\end{align}
				が得られる.
				
			\item[step3] 同様に,定理\ref{thm:set_is_an_element_of_its_pair}より
				\begin{align}
					\set{c},\ \EXTAX,\EQAX,\COMAX \vdash c \in \{c,d\}
				\end{align}
				が成り立つので,相等性公理より
				\begin{align}
					\{a\} = \{c,d\},\ \set{c},\ \EXTAX,\EQAX,\COMAX \vdash c \in \{a\}
				\end{align}
				となる.また定理\ref{cor:pair_members_are_exactly_the_given_two}より
				\begin{align}
					\EXTAX,\EQAX,\COMAX,\ELEAX \vdash c \in \{a\} \rarrow a = c
				\end{align}
				となるから,三段論法より
				\begin{align}
					\{a\} = \{c,d\},\ \set{c},\ \EXTAX,\EQAX,\COMAX,\ELEAX \vdash a = c
					\label{fom:equality_of_ordered_pairs_4}
				\end{align}
				が得られる.
				
			\item[step4] (\refeq{fom:equality_of_ordered_pairs_3})と(\refeq{fom:equality_of_ordered_pairs_4})と
				演繹定理より
				\begin{align}
					\set{a},\ \EXTAX,\EQAX,\COMAX,\ELEAX &\vdash \{a\} = \{c\} \rarrow a = c, \\
					\set{c},\ \EXTAX,\EQAX,\COMAX,\ELEAX &\vdash \{a\} = \{c,d\} \rarrow a = c
				\end{align}
				が成り立つので,論理和の除去より
				\begin{align}
					\set{a},\ \set{c},\ \EXTAX,\EQAX,\COMAX,\ELEAX \vdash 
					\{a\} = \{c\} \vee \{a\} = \{c,d\} \rarrow a = c
				\end{align}
				が従い,(\refeq{fom:equality_of_ordered_pairs_2})との三段論法より
				\begin{align}
					(a,b) = (c,d),\ \set{a},\ \set{c},\ \EXTAX,\EQAX,\COMAX,\ELEAX \vdash a = c
				\end{align}
				が出る.
				
			\item[step5]
		ゆえに
		\begin{align}
			\{\{a\},\{a,b\}\} = \{\{a\},\{a,d\}\}
		\end{align}
		である.$(a,b) = (c,d)$に加えて
		\begin{align}
			a = d
		\end{align}
		と仮定すると,
		\begin{align}
			\{a,b\} = \{a\} \vee \{a,b\} = \{a,d\}
		\end{align}
		と
		\begin{align}
			\{a,b\} = \{a\} \rarrow b = a = d
		\end{align}
		となり,
		\begin{align}
			\{a,b\} = \{a,d\} &\rarrow b = a \vee b = d, \\
			b = a &\rarrow b = d, \\
			b = d &\rarrow b = d
		\end{align}
		より
		\begin{align}
			\{a,b\} = \{a,d\} \rarrow b = d
		\end{align}
		も成り立つ.ゆえに
		\begin{align}
			a = d \rarrow b = d
		\end{align}
		である.今度は$(a,b) = (c,d)$に加えて
		\begin{align}
			a \neq d
		\end{align}
		と仮定する.
		\begin{align}
			\{a,d\} = \{a\} \vee \{a,d\} = \{a,b\}
		\end{align}
		と
		\begin{align}
			\{a,d\} \neq \{a\}
		\end{align}
		より
		\begin{align}
			\{a,d\} = \{a,b\}
		\end{align}
		が成り立ち,
		\begin{align}
			d = a \vee d = b
		\end{align}
		が成り立つ.$d \neq a$より
		\begin{align}
			d = b
		\end{align}
		が従う.ゆえに
		\begin{align}
			a \neq d \rarrow b = d
		\end{align}
		でもある.
		\QED
		\end{description}
	\end{sketch}
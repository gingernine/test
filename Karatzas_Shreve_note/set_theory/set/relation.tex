\section{関係}
	\begin{screen}
		\begin{dfn}[順序対]
			$x$と$y$を$\mathcal{L}$の項とするとき,
			\begin{align}
				(x,y) \defeq \{\{x\},\{x,y\}\}
			\end{align}
			で定める項$(x,y)$を$x$と$y$の{\bf 順序対}\index{じゅんじょつい@順序対}
			{\bf (ordered pair)}と呼ぶ.
		\end{dfn}
	\end{screen}
	
	\begin{screen}
		\begin{thm}[集合の順序対は集合]
		\label{thm:ordered_pair_of_sets_is_a_set}
			$a$と$b$を類とするとき
			\begin{align}
				\EXTAX,\EQAX,\COMAX,\PAIAX \vdash
				\set{a} \wedge \set{b} \rarrow \set{(a,b)}.
			\end{align}
		\end{thm}
	\end{screen}
	
	\begin{prf}
		集合の対は集合(定理\ref{thm:pair_of_sets_is_a_set})であるから
		\begin{align}
			\set{a},\ \EXTAX,\EQAX,\COMAX,\PAIAX &\vdash \set{\{a\}}, \\
			\set{a},\ \set{b},\ \EXTAX,\EQAX,\COMAX,\PAIAX &\vdash \set{\{a,b\}}
		\end{align}
		が成り立つので
		\begin{align}
			\set{a},\ \set{b},\ \EXTAX,\EQAX,\COMAX,\PAIAX \vdash 
			\set{\{a\}} \wedge \set{\{a,b\}}
		\end{align}
		が従い,再び定理\ref{thm:pair_of_sets_is_a_set}より
		\begin{align}
			\set{a},\ \set{b},\ \EXTAX,\EQAX,\COMAX,\PAIAX \vdash \set{(a,b)}
		\end{align}
		となる.
		\QED
	\end{prf}
	
	\begin{screen}
		\begin{logicalthm}[選言三段論法]
		\label{logicalthm:disjunctive_syllogism}
			$A$と$B$を文とするとき
			\begin{align}
				\vdash A \vee B \rarrow (\, \negation A \rarrow B\, ), \\
				\vdash A \vee B \rarrow (\, \negation B \rarrow A\, ).
			\end{align}
		\end{logicalthm}
	\end{screen}
	
	\begin{sketch}
		まず含意の導入より
		\begin{align}
			\vdash B \rarrow (\, \negation A \rarrow B\, )
		\end{align}
		が成り立つ.また矛盾の導入より
		\begin{align}
			A,\ \negation A \vdash \bot
		\end{align}
		が成り立つが,爆発律(論理的定理\ref{logicalthm:principle_of_explosion})より
		\begin{align}
			A,\ \negation A \vdash B
		\end{align}
		が従い,演繹定理より
		\begin{align}
			\vdash A \rarrow (\, \negation A \rarrow B\, )
		\end{align}
		も得られる.そして論理和の除去より
		\begin{align}
			\vdash A \vee B \rarrow (\, \negation A \rarrow B\, )
		\end{align}
		が出る.演繹定理の逆より
		\begin{align}
			A \vee B \vdash\ \negation A \rarrow B
		\end{align}
		となるが,ここで対偶律$3$ (論理的定理\ref{logicalthm:contraposition_3})より
		\begin{align}
			A \vee B \vdash\ \negation B \rarrow A
		\end{align}
		が従い,演繹定理より
		\begin{align}
			\vdash A \vee B \rarrow (\, \negation B \rarrow A\, )
		\end{align}
		も得られる.
		\QED
	\end{sketch}
	
	\begin{screen}
		\begin{thm}[順序対の相等性]
		\label{thm:equality_of_ordered_pairs}
			$a,b,c,d$を類とするとき
			\begin{align}
				\begin{gathered}
					\set{a},\ \set{b},\ \set{c},\ \set{d},\ \EXTAX,\EQAX,\COMAX,\ELEAX,\PAIAX \\
					\vdash (a,b) = (c,d) \rarrow a = c \wedge b = d
				\end{gathered}.
			\end{align}
		\end{thm}
	\end{screen}
	
	\begin{sketch}\mbox{}
		\begin{description}
			\item[step1] 集合は自分自身の対の要素である(定理\ref{thm:set_is_an_element_of_its_pair})から
				\begin{align}
					\set{\{a\}},\ \EXTAX,\EQAX,\COMAX \vdash \{a\} \in (a,b)
				\end{align}
				が成り立つ.従って$(a,b) = (c,d)$と仮定すると,相等性公理より
				\begin{align}
					(a,b) = (c,d),\ \set{\{a\}},\ \EXTAX,\EQAX,\COMAX \vdash \{a\} \in (c,d)
				\end{align}
				が成り立つ.定理\ref{cor:pair_members_are_exactly_the_given_two}
				(対の要素は表示されている要素の一方には等しい)より
				\begin{align}
					\EXTAX,\EQAX,\COMAX,\ELEAX \vdash \{a\} \in (c,d) \rarrow \{a\} = \{c\} \vee \{a\} = \{c,d\}
					\label{fom:equality_of_ordered_pairs_1}
				\end{align}
				となるから,三段論法より
				\begin{align}
					(a,b) = (c,d),\ \set{\{a\}},\ \EXTAX,\EQAX,\COMAX,\ELEAX \vdash \{a\} = \{c\} \vee \{a\} = \{c,d\}
				\end{align}
				が従い,演繹定理より
				\begin{align}
					(a,b) = (c,d),\ \EXTAX,\EQAX,\COMAX,\ELEAX \vdash
					\set{\{a\}} \rarrow \{a\} = \{c\} \vee \{a\} = \{c,d\}
				\end{align}
				が従う.ところで集合の対は集合(定理\ref{thm:pair_of_sets_is_a_set})なので
				\begin{align}
					\set{a},\ \EXTAX,\EQAX,\COMAX,\PAIAX \vdash \set{\{a\}}
				\end{align}
				が成り立ち,三段論法より
				\begin{align}
					(a,b) = (c,d),\ \set{a},\ \EXTAX,\EQAX,\COMAX,\ELEAX,\PAIAX \vdash \{a\} = \{c\} \vee \{a\} = \{c,d\}
					\label{fom:equality_of_ordered_pairs_2}
				\end{align}
				が従う.	
		
			\item[step2] この段では
				\begin{align}
					\set{a},\ \EXTAX,\EQAX,\COMAX,\ELEAX \vdash \{a\} = \{c\} \rarrow a = c 
				\end{align}
				を示す.定理\ref{thm:set_is_an_element_of_its_pair} (集合は自分自身の対の要素)より
				\begin{align}
					\set{a},\ \EXTAX,\EQAX,\COMAX \vdash a \in \{a\}
				\end{align}
				が成り立ち,相等性公理より
				\begin{align}
					\{a\} = \{c\},\ \set{a},\ \EXTAX,\EQAX,\COMAX \vdash a \in \{c\}
				\end{align}
				となる.また定理\ref{cor:pair_members_are_exactly_the_given_two}
				(対の要素は表示されている要素の一方には等しい)より
				\begin{align}
					\EXTAX,\EQAX,\COMAX,\ELEAX \vdash a \in \{c\} \rarrow a = c
				\end{align}
				も成り立つので,三段論法より
				\begin{align}
					\{a\} = \{c\},\ \set{a},\ \EXTAX,\EQAX,\COMAX,\ELEAX \vdash a = c
					\label{fom:equality_of_ordered_pairs_3}
				\end{align}
				が得られる.
				
			\item[step3] この段では
				\begin{align}
					\set{a},\ \EXTAX,\EQAX,\COMAX,\ELEAX \vdash \{a\} = \{c,d\} \rarrow a = c
				\end{align}
				同様に,定理\ref{thm:set_is_an_element_of_its_pair}より
				\begin{align}
					\set{c},\ \EXTAX,\EQAX,\COMAX \vdash c \in \{c,d\}
				\end{align}
				が成り立ち,相等性公理より
				\begin{align}
					\{a\} = \{c,d\},\ \set{c},\ \EXTAX,\EQAX,\COMAX \vdash c \in \{a\}
				\end{align}
				となる.また定理\ref{cor:pair_members_are_exactly_the_given_two}より
				\begin{align}
					\EXTAX,\EQAX,\COMAX,\ELEAX \vdash c \in \{a\} \rarrow a = c
				\end{align}
				も成り立つので,三段論法より
				\begin{align}
					\{a\} = \{c,d\},\ \set{c},\ \EXTAX,\EQAX,\COMAX,\ELEAX \vdash a = c
					\label{fom:equality_of_ordered_pairs_4}
				\end{align}
				が得られる.
				
			\item[step4] (\refeq{fom:equality_of_ordered_pairs_3})と(\refeq{fom:equality_of_ordered_pairs_4})と
				演繹定理より
				\begin{align}
					\set{a},\ \EXTAX,\EQAX,\COMAX,\ELEAX &\vdash \{a\} = \{c\} \rarrow a = c, \\
					\set{c},\ \EXTAX,\EQAX,\COMAX,\ELEAX &\vdash \{a\} = \{c,d\} \rarrow a = c
				\end{align}
				が成り立つので,論理和の除去より
				\begin{align}
					\set{a},\ \set{c},\ \EXTAX,\EQAX,\COMAX,\ELEAX \vdash 
					\{a\} = \{c\} \vee \{a\} = \{c,d\} \rarrow a = c
				\end{align}
				が従い,(\refeq{fom:equality_of_ordered_pairs_2})との三段論法より
				\begin{align}
					(a,b) = (c,d),\ \set{a},\ \set{c},\ \EXTAX,\EQAX,\COMAX,\ELEAX \vdash a = c
					\label{fom:equality_of_ordered_pairs_5}
				\end{align}
				が出る.
				
			\item[step5]
				定理\ref{thm:set_is_an_element_of_its_pair}より
				\begin{align}
					\set{b},\ \EXTAX,\EQAX,\COMAX \vdash b \in \{a,b\}
					\label{fom:equality_of_ordered_pairs_8}
				\end{align}
				が成り立つので,相等性公理より
				\begin{align}
					\{a,b\} = \{c\},\ \set{b},\ \EXTAX,\EQAX,\COMAX \vdash b \in \{c\}
				\end{align}
				となる.また定理\ref{cor:pair_members_are_exactly_the_given_two}より
				\begin{align}
					\EXTAX,\EQAX,\COMAX,\ELEAX \vdash b \in \{c\} \rarrow c = b
				\end{align}
				となるから,三段論法より
				\begin{align}
					\{a,b\} = \{c\},\ \set{b},\ \EXTAX,\EQAX,\COMAX,\ELEAX \vdash c = b
					\label{fom:equality_of_ordered_pairs_6}
				\end{align}
				が従う.他方で等号の推移律(定理\ref{thm:transitive_law_of_equality})より
				\begin{align}
					\EXTAX,\EQAX \vdash a = c \rarrow (\, c = b \rarrow a = b\, )
				\end{align}
				が成り立つから,(\refeq{fom:equality_of_ordered_pairs_5})との三段論法より
				\begin{align}
					(a,b) = (c,d),\ \set{a},\ \set{c},\ \EXTAX,\EQAX,\COMAX,\ELEAX \vdash c = b \rarrow a = b
					\label{fom:equality_of_ordered_pairs_11}
				\end{align}
				となり,(\refeq{fom:equality_of_ordered_pairs_6})との三段論法より
				\begin{align}
					\begin{gathered}
						\{a,b\} = \{c\},\ (a,b) = (c,d),\ \set{a},\ \set{b},\ \set{c},\ \EXTAX,\EQAX,\COMAX,\ELEAX \\
						\vdash a = b
					\end{gathered}
				\end{align}
				となる.再び等号の推移律(定理\ref{thm:transitive_law_of_equality})より
				\begin{align}
					\begin{gathered}
						\{a,b\} = \{c\},\ (a,b) = (c,d),\ \set{a},\ \set{b},\ \set{c},\ 
						\EXTAX,\EQAX,\COMAX,\ELEAX \\
						\vdash a = d \rarrow b = d
					\end{gathered}
					\label{fom:equality_of_ordered_pairs_7}
				\end{align}
				が従う.演繹定理より
				\begin{align}
					\begin{gathered}
						(a,b) = (c,d),\ \set{a},\ \set{b},\ \set{c},\ 
						\EXTAX,\EQAX,\COMAX,\ELEAX \\ 
						\vdash \{a,b\} = \{c\} \rarrow (\, a = d \rarrow b = d\, )
					\end{gathered}
					\label{fom:equality_of_ordered_pairs_9}
				\end{align}
				となる.
				
			\item[step6] 定理\ref{thm:set_is_an_element_of_its_pair}より
				(\refeq{fom:equality_of_ordered_pairs_8})と相等性公理より
				\begin{align}
					\{a,b\} = \{c,d\},\ \set{b},\ \EXTAX,\EQAX,\COMAX \vdash b \in \{c,d\}
				\end{align}
				となり,また定理\ref{cor:pair_members_are_exactly_the_given_two}より
				\begin{align}
					\EXTAX,\EQAX,\COMAX,\ELEAX \vdash b \in \{c,d\} \rarrow c = b \vee d = b
				\end{align}
				となるから,三段論法より
				\begin{align}
					\{a,b\} = \{c,d\},\ \set{b},\ \EXTAX,\EQAX,\COMAX,\ELEAX \vdash c = b \vee d = b
					\label{fom:equality_of_ordered_pairs_10}
				\end{align}
				が従う.ここで(\refeq{fom:equality_of_ordered_pairs_11})と演繹定理の逆より
				\begin{align}
					c = b,\ (a,b) = (c,d),\ \set{a},\ \set{c},\ \EXTAX,\EQAX,\COMAX,\ELEAX \vdash a = b
				\end{align}
				となり,また等号の推移律(定理\ref{thm:transitive_law_of_equality})より
				\begin{align}
					\EXTAX,\EQAX \vdash a = b \rarrow (\, a = d \rarrow b = d\, )
				\end{align}
				が成り立つので,三段論法より
				\begin{align}
					c = b,\ (a,b) = (c,d),\ \set{a},\ \set{c},\ \EXTAX,\EQAX,\COMAX,\ELEAX \vdash a = d \rarrow b = d
				\end{align}
				となる.すなわち
				\begin{align}
					a = d,\ (a,b) = (c,d),\ \set{a},\ \set{c},\ \EXTAX,\EQAX,\COMAX,\ELEAX \vdash c = b \rarrow b = d
				\end{align}
				となる.他方で
				\begin{align}
					\EQAX \vdash d = b \rarrow b = d
				\end{align}
				も成り立つから,論理和の除去より
				\begin{align}
					\begin{gathered}
						a = d,\ (a,b) = (c,d),\ \set{a},\ \set{c},\ \EXTAX,\EQAX,\COMAX,\ELEAX \\
						\vdash c = b \vee d = b \rarrow b = d
					\end{gathered}
				\end{align}
				が従う.(\refeq{fom:equality_of_ordered_pairs_10})との三段論法より
				\begin{align}
					\begin{gathered}
						a = d,\ \{a,b\} = \{c,d\},\ (a,b) = (c,d),\ \set{a},\ \set{b},\ \set{c},\ \EXTAX,\EQAX,\COMAX,\ELEAX \\
						\vdash b = d
					\end{gathered}
				\end{align}
				が成り立ち,演繹定理より
				\begin{align}
					\begin{gathered}
						(a,b) = (c,d),\ \set{a},\ \set{b},\ \set{c},\ \EXTAX,\EQAX,\COMAX,\ELEAX \\
						\vdash \{a,b\} = \{c,d\} \rarrow (\, a = d \rarrow b = d\, )
					\end{gathered}
					\label{fom:equality_of_ordered_pairs_12}
				\end{align}
				が得られる.
				
			\item[step7] (\refeq{fom:equality_of_ordered_pairs_9})と(\refeq{fom:equality_of_ordered_pairs_12})と
				論理和の除去より
				\begin{align}
					\begin{gathered}
						(a,b) = (c,d),\ \set{a},\ \set{b},\ \set{c},\ \EXTAX,\EQAX,\COMAX,\ELEAX \\
						\vdash \{a,b\} = \{c\} \vee \{a,b\} = \{c,d\} \rarrow (\, a = d \rarrow b = d\, )
					\end{gathered}
					\label{fom:equality_of_ordered_pairs_13}
				\end{align}
				が成り立つ.ところで集合の対は集合(定理\ref{thm:pair_of_sets_is_a_set})なので
				\begin{align}
					\set{a},\ \set{b},\ \EXTAX,\EQAX,\COMAX,\PAIAX \vdash \set{\{a,b\}}
				\end{align}
				となり,また定理\ref{thm:set_is_an_element_of_its_pair}(集合は自分自身の対の要素)より
				\begin{align}
					\EXTAX,\EQAX,\COMAX \vdash \set{\{a,b\}} \rarrow \{a,b\} \in (a,b)	
				\end{align}
				となる.よって三段論法より
				\begin{align}
					\set{a},\ \set{b},\ \EXTAX,\EQAX,\COMAX,\PAIAX \vdash \{a,b\} \in (a,b)
				\end{align}
				となり,相等性公理より
				\begin{align}
					(a,b) = (c,d),\ \set{a},\ \set{b},\ \EXTAX,\EQAX,\COMAX,\PAIAX \vdash \{a,b\} \in (c,d)
				\end{align}
				となり,定理\ref{cor:pair_members_are_exactly_the_given_two}より
				\begin{align}
					(a,b) = (c,d),\ \set{a},\ \set{b},\ \EXTAX,\EQAX,\COMAX,\PAIAX 
					\vdash \{a,b\} = \{c\} \vee \{a,b\} = \{c,d\}
				\end{align}
				が従う.これと(\refeq{fom:equality_of_ordered_pairs_13})より
				\begin{align}
					\begin{gathered}
						(a,b) = (c,d),\ \set{a},\ \set{b},\ \set{c},\ \EXTAX,\EQAX,\COMAX,\ELEAX,\PAIAX \\
						\vdash a = d \rarrow b = d
					\end{gathered}
					\label{fom:equality_of_ordered_pairs_14}
				\end{align}
				が従う.
				
			\item[step8] 集合の対は集合(定理\ref{thm:pair_of_sets_is_a_set})なので
				\begin{align}
					\set{c},\ \set{d},\ \EXTAX,\EQAX,\COMAX,\PAIAX \vdash \set{\{c,d\}}
				\end{align}
				となり,また定理\ref{thm:set_is_an_element_of_its_pair}(集合は自分自身の対の要素)より
				\begin{align}
					\EXTAX,\EQAX,\COMAX \vdash \set{\{c,d\}} \rarrow \{c,d\} \in (c,d)	
				\end{align}
				となる.よって三段論法より
				\begin{align}
					\set{c},\ \set{d},\ \EXTAX,\EQAX,\COMAX,\PAIAX \vdash \{c,d\} \in (c,d)
				\end{align}
				が従い,相等性公理より
				\begin{align}
					(a,b) = (c,d),\ \set{c},\ \set{d},\ \EXTAX,\EQAX,\COMAX,\PAIAX \vdash \{c,d\} \in (a,b)
				\end{align}
				となり,定理\ref{cor:pair_members_are_exactly_the_given_two}より
				\begin{align}
					(a,b) = (c,d),\ \set{c},\ \set{d},\ \EXTAX,\EQAX,\COMAX,\PAIAX 
					\vdash \{c,d\} = \{a\} \vee \{c,d\} = \{a,b\}
					\label{fom:equality_of_ordered_pairs_15}
				\end{align}
				が従う.ところで
				\begin{align}
					\set{d},\ \EXTAX,\EQAX,\COMAX \vdash d \in \{c,d\}
					\label{fom:equality_of_ordered_pairs_17}
				\end{align}
				であるから,相等性公理より
				\begin{align}
					\{c,d\} = \{a\},\ \set{d},\ \EXTAX,\EQAX,\COMAX \vdash d \in \{a\}
				\end{align}
				が従い,定理\ref{cor:pair_members_are_exactly_the_given_two} (対の要素は表示されている要素の一方に等しい)より
				\begin{align}
					\{c,d\} = \{a\},\ \set{d},\ \EXTAX,\EQAX,\COMAX,\ELEAX \vdash a = d
				\end{align}
				が従う.つまり演繹定理より
				\begin{align}
					\set{d},\ \EXTAX,\EQAX,\COMAX,\ELEAX \vdash \{c,d\} = \{a\} \rarrow a = d
				\end{align}
				となるが,対偶律1 (論理的定理\ref{logicalthm:introduction_of_contraposition})より
				\begin{align}
					\set{d},\ \EXTAX,\EQAX,\COMAX,\ELEAX \vdash a \neq d \rarrow \{c,d\} \neq \{a\}
				\end{align}
				となり,演繹定理の逆より
				\begin{align}
					a \neq d,\ \set{d},\ \EXTAX,\EQAX,\COMAX,\ELEAX \vdash \{c,d\} \neq \{a\}
					\label{fom:equality_of_ordered_pairs_16}
				\end{align}
				が従う.(\refeq{fom:equality_of_ordered_pairs_15})と選言三段論法
				(論理的定理\ref{logicalthm:disjunctive_syllogism})より
				\begin{align}
					(a,b) = (c,d),\ \set{c},\ \set{d},\ \EXTAX,\EQAX,\COMAX,\PAIAX 
					\vdash \{c,d\} \neq \{a\} \rarrow \{c,d\} = \{a,b\}
				\end{align}
				が成り立つから,(\refeq{fom:equality_of_ordered_pairs_16})との三段論法より
				\begin{align}
					a \neq d,\ (a,b) = (c,d),\ \set{c},\ \set{d},\ \EXTAX,\EQAX,\COMAX,\ELEAX,\PAIAX
					\vdash \{c,d\} = \{a,b\} 
				\end{align}
				が従う.相等性公理より
				\begin{align}
					\begin{gathered}
						a \neq d,\ (a,b) = (c,d),\ \set{c},\ \set{d},\ \EXTAX,\EQAX,\COMAX,\ELEAX,\PAIAX \\
						\vdash d \in \{c,d\} \rarrow d \in \{a,b\}
					\end{gathered}
				\end{align}
				となり,(\refeq{fom:equality_of_ordered_pairs_17})との三段論法より
				\begin{align}
					\begin{gathered}
						a \neq d,\ (a,b) = (c,d),\ \set{c},\ \set{d},\ \EXTAX,\EQAX,\COMAX,\ELEAX,\PAIAX \\
						\vdash d \in \{a,b\}
					\end{gathered}
				\end{align}
				となり,定理\ref{cor:pair_members_are_exactly_the_given_two} (対の要素は表示されている要素の一方に等しい)より
				\begin{align}
					\begin{gathered}
						a \neq d,\ (a,b) = (c,d),\ \set{c},\ \set{d},\ \EXTAX,\EQAX,\COMAX,\ELEAX,\PAIAX \\
						\vdash a = d \vee b = d
					\end{gathered}
				\end{align}
				が従う.再び選言三段論法(論理的定理\ref{logicalthm:disjunctive_syllogism})より
				\begin{align}
					\begin{gathered}
						a \neq d,\ (a,b) = (c,d),\ \set{c},\ \set{d},\ \EXTAX,\EQAX,\COMAX,\ELEAX,\PAIAX \\
						\vdash a \neq d \rarrow b = d
					\end{gathered}
				\end{align}
				が成り立つので
				\begin{align}
					\begin{gathered}
						a \neq d,\ (a,b) = (c,d),\ \set{c},\ \set{d},\ \EXTAX,\EQAX,\COMAX,\ELEAX,\PAIAX \\
						\vdash b = d
					\end{gathered}
				\end{align}
				が従い,演繹定理より
				\begin{align}
					\begin{gathered}
						(a,b) = (c,d),\ \set{c},\ \set{d},\ \EXTAX,\EQAX,\COMAX,\ELEAX,\PAIAX \\
						\vdash a \neq d \rarrow b = d
					\end{gathered}
					\label{fom:equality_of_ordered_pairs_18}
				\end{align}
				が得られる.(\refeq{fom:equality_of_ordered_pairs_14})と(\refeq{fom:equality_of_ordered_pairs_18})と
				論理和の除去より
				\begin{align}
					\begin{gathered}
						(a,b) = (c,d),\ \set{a},\ \set{b},\ \set{c},\ \set{d},\ \EXTAX,\EQAX,\COMAX,\ELEAX,\PAIAX \\
						\vdash a = d \vee a \neq d \rarrow b = d
					\end{gathered}
				\end{align}
				が成り立つが,排中律(論理的定理\ref{logicalthm:law_of_excluded_middle})より
				\begin{align}
					\vdash a = d \vee a \neq d
				\end{align}
				となるので,三段論法より
				\begin{align}
					\begin{gathered}
						(a,b) = (c,d),\ \set{a},\ \set{b},\ \set{c},\ \set{d},\ \EXTAX,\EQAX,\COMAX,\ELEAX,\PAIAX \\
						\vdash b = d
					\end{gathered}
				\end{align}
				が得られる.これと(\refeq{fom:equality_of_ordered_pairs_5})と論理積の導入より
				\begin{align}
					\begin{gathered}
						(a,b) = (c,d),\ \set{a},\ \set{b},\ \set{c},\ \set{d},\ \EXTAX,\EQAX,\COMAX,\ELEAX,\PAIAX \\
						\vdash a = c \wedge b = d
					\end{gathered}
				\end{align}
				が成り立つ.
				\QED
		\end{description}
	\end{sketch}
	
	\begin{screen}
		\begin{dfn}[Cartesian積]
			$a$と$b$を類とするとき,
			\begin{align}
				a \times b \defeq \Set{x}{\exists s\, (\, s \in a \wedge \exists t\, (\,t \in b \wedge x = (s,t) \, )\, )}
			\end{align}
			により定める類$a \times b$を$a$と$b$の{\bf Cartesian積}
			\index{Cartesianせき@Cartesian積}{\bf (Cartesian product)}と呼ぶ.
		\end{dfn}
	\end{screen}
	
	\begin{screen}
		\begin{dfn}[関係]
			$R \subset \Univ \times \Univ$を満たす類$R$を{\bf 関係}\index{かんけい@関係}{\bf (relation)}と呼ぶ.
		\end{dfn}
	\end{screen}
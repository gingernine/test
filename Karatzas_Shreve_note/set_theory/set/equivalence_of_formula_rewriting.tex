\section{書き換えの同値性}
\label{sec:equivalence_of_formula_rewriting}
	$\mathcal{L}$の式を$\lang{\varepsilon}$の式に変換するときは,まず部分式のうち原子式に対して
	\begin{table}[H]
		\begin{center}
		\caption{式の書き換え表(再掲)}
		\begin{tabular}{c|c|c|c}
			 & 元の式 & 書き換え後 & 付記 \\ \hline \hline
			(1) & $a = \Set{z}{\psi}$ & $\forall v\, (\, v \in a \lrarrow \psi(z/v)\, )$ & \\ \hline
			(2) & $\Set{y}{\varphi} = b$ & $\forall u\, (\, \varphi(y/u) \lrarrow u \in b\, )$ & \\ \hline
			(3) & $\Set{y}{\varphi} = \Set{z}{\psi}$ & $\forall u\, (\, \varphi(y/u) \lrarrow \psi(z/u)\, )$ & \\ \hline
			(4) & $a \in \Set{z}{\psi}$ & $\psi(z/a)$ & 必要なら変項の名前替え \\ \hline
			(5) & $\Set{y}{\varphi} \in b$ & $\exists s\, (\, \forall u\, (\, \varphi(y/u) \lrarrow u \in s\, ) \wedge s \in b\, )$ & \\ \hline
			(6) & $\Set{y}{\varphi} \in \Set{z}{\psi}$ & $\exists s\, (\, \forall u\, (\, \varphi(y/u) \lrarrow u \in s\, ) \wedge \psi(z/s)\, )$ & \\ \hline
		\end{tabular}
		\label{tab:formula_rewriting_repeat}
		\end{center}
	\end{table}
	
	に則って変換したのであった.また量化部分式を逐次的に差し替えることによって得られる式も書き換えと呼んだ
	(定義\ref{def:formula_rewriting}).
	以下では表\ref{tab:formula_rewriting}の直後で述べた細かい変項条件等は全て満たされているとする.
	この節では,$\varphi$が$\lang{\varepsilon}$の式ではない$\mathcal{L}$の文であるとして,
	その書き換え$\widehat{\varphi}$に対して
	\begin{align}
		\EXTAX,\EQAX,\COMAX,\ELEAX \vdash \varphi \lrarrow \widehat{\varphi}
		\label{formula_rewriting_1}
	\end{align}
	が成り立つことを示す.
	これが示されれば,仮に$\varphi$に変項$x$が自由に現れていても
	\begin{align}
		\EXTAX,\EQAX,\COMAX,\ELEAX \vdash \forall x\, (\, \varphi(x) \lrarrow \widehat{\varphi}(x)\, )
	\end{align}
	が成り立つし,$x$に加えて$y$が自由に現れていても
	\begin{align}
		\EXTAX,\EQAX,\COMAX,\ELEAX \vdash \forall x\, \forall y\, (\, \varphi(x,y) \lrarrow \widehat{\varphi}(x,y)\, )
	\end{align}
	が成り立つ.前者の場合は
	\begin{align}
		\tau \defeq \varepsilon x \negation (\, \widehat{\varphi}(x) \lrarrow \widehat{\varphi}(x)\, )
	\end{align}
	に対して
	\begin{align}
		\EXTAX,\EQAX,\COMAX,\ELEAX \vdash \varphi(\tau) \lrarrow \widehat{\varphi}(\tau)
	\end{align}
	が成り立つのだし,後者の場合は
	\begin{align}
		\sigma &\defeq \varepsilon x \negation \forall y\, (\, \widehat{\varphi}(x,y) \lrarrow \widehat{\varphi}(x,y)\, ), \\
		\rho &\defeq \varepsilon y \negation (\, \widehat{\varphi}(\sigma,y) \lrarrow \widehat{\varphi}(\sigma,y)\, )
	\end{align}
	に対して
	\begin{align}
		\EXTAX,\EQAX,\COMAX,\ELEAX \vdash \varphi(\sigma,\rho) \lrarrow \widehat{\varphi}(\sigma,\rho)
	\end{align}
	が成り立つので,全称の導出(論理的定理\ref{logicalthm:derivation_of_universal_by_epsilon})を適用すればいい.
	一般の式$\varphi$に対して(\refeq{formula_rewriting_1})を示すには
	$\varphi$が原子式であるときに(\refeq{formula_rewriting_1})が成り立つことを言えば十分であり,
	それについて先に結論を書いておくと
	\begin{align}
		\EQAX,\COMAX &\vdash a = \Set{z}{\psi(z)} 
			\rarrow \forall v\, (\, v \in a \lrarrow \psi(v)\, ), \\
		\EXTAX,\COMAX &\vdash \forall v\, (\, v \in a \lrarrow \psi(v)\, )
			\rarrow a = \Set{z}{\psi(z)}, \\
		\EQAX,\COMAX &\vdash \Set{y}{\varphi(y)} = b 
			\rarrow \forall u\, (\, \varphi(u) \lrarrow u \in b\, ), \\
		\EXTAX,\COMAX &\vdash \forall u\, (\, \varphi(u) \lrarrow u \in b\, )
			\rarrow \Set{y}{\varphi(y)} = b, \\
		\EQAX,\COMAX &\vdash \Set{y}{\varphi(y)} = \Set{z}{\psi(z)}
			\rarrow \forall u\, (\, \varphi(u) \lrarrow \psi(u)\, ), \\
		\EXTAX,\COMAX &\vdash \forall u\, (\, \varphi(u) \lrarrow \psi(u)\, )
			\rarrow \Set{y}{\varphi(y)} = \Set{z}{\psi(z)}, \\
		\COMAX &\vdash a \in \Set{z}{\psi(z)} \rarrow \widetilde{\psi}(a), \\
		\COMAX &\vdash \widetilde{\psi}(a) \rarrow a \in \Set{z}{\psi(z)}, \\
		\EQAX,\COMAX,\ELEAX &\vdash \Set{y}{\varphi(y)} \in b
			\rarrow \exists s\, (\, \forall u\, (\, \varphi(u) \lrarrow u \in s\, ) \wedge s \in b\, ), \\
		\EXTAX,\EQAX,\COMAX &\vdash \exists s\, (\, \forall u\, (\, \varphi(u) \lrarrow u \in s\, ) \wedge s \in b\, ) \rarrow \Set{y}{\varphi(y)} \in b, \\
		\EQAX,\COMAX,\ELEAX &\vdash \Set{y}{\varphi(y)} \in \Set{z}{\psi(z)}
			\rarrow \exists s\, (\, \forall u\, (\, \varphi(u) \lrarrow u \in s\, ) \wedge \psi(s)\, ), \\
		\EXTAX,\EQAX,\COMAX &\vdash \exists s\, (\, \forall u\, (\, \varphi(u) \lrarrow u \in s\, ) \wedge \psi(s)\, ) \rarrow \Set{y}{\varphi(y)} \in \Set{z}{\psi(z)}
	\end{align}
	が成立する.
	
	\begin{screen}
		\begin{logicalthm}[同値記号の対称律]
		\label{logicalthm:symmetry_of_equivalence_arrows}
			$A,B$を$\mathcal{L}$の文とするとき
			\begin{align}
				\vdash (\, A \lrarrow B\, ) \rarrow (\, B \lrarrow A\, ).
			\end{align}
		\end{logicalthm}
	\end{screen}
	
	\begin{prf}
		論理積の除去より
		\begin{align}
			A \lrarrow B &\vdash A \rarrow B, \\
			A \lrarrow B &\vdash B \rarrow A
		\end{align}
		となる.他方で論理積の導入より
		\begin{align}
			\vdash (\, B \rarrow A\, ) \rarrow (\, (\, A \rarrow B\, ) \rarrow 
			(\, B \rarrow A\, ) \wedge (\, A \rarrow B\, )\, )
		\end{align}
		が成り立つので,三段論法を二回適用すれば
		\begin{align}
			A \lrarrow B \vdash (\, B \rarrow A\, ) \wedge (\, A \rarrow B\, )
		\end{align}
		となる.つまり
		\begin{align}
			A \lrarrow B \vdash B \lrarrow A
		\end{align}
		が得られた.
		\QED
	\end{prf}
	
	\begin{screen}
		\begin{thm}
		\label{thm:equivalent_formula_rewriting_1}
			$a$を主要$\varepsilon$項とし,$\psi$を$\lang{\varepsilon}$の式とし,
			$\psi$には$z$のみ自由に現れているとする.このとき
			\begin{align}
				\EQAX,\COMAX \vdash a = \Set{z}{\psi(z)} 
				\rarrow \forall v\, (\, v \in a \lrarrow \psi(v)\, ).
			\end{align}
		\end{thm}
	\end{screen}
	
	\begin{sketch}
		いま
		\begin{align}
			\tau \defeq \varepsilon v \negation (\, v \in a \lrarrow \psi(v)\, )
		\end{align}
		とおく.外延性公理の逆(定理\ref{thm:inverse_of_axiom_of_extensionality})より
		\begin{align}
			a = \Set{z}{\psi(z)},\ \EQAX \vdash 
			\tau \in a \lrarrow \tau \in \Set{z}{\psi(z)}
		\end{align}
		が成り立ち,他方で内包性公理より
		\begin{align}
			\COMAX \vdash \tau \in \Set{z}{\psi(z)} \lrarrow \psi(\tau)
		\end{align}
		が成り立つので,同値記号の推移律
		(論理的定理\ref{logicalthm:transitive_law_of_equivalence_symbol})より
		\begin{align}
			a = \Set{z}{\psi(z)},\ \EQAX,\COMAX \vdash \tau \in a \lrarrow \psi(\tau)
		\end{align}
		が従う.そして全称の導出(論理的定理\ref{logicalthm:derivation_of_universal_by_epsilon})より
		\begin{align}
			a = \Set{z}{\psi(z)},\ \EQAX,\COMAX \vdash 
			\forall v\, (\, v \in a \lrarrow \psi(v)\, )
		\end{align}
		が得られる.
		\QED
	\end{sketch}
	
	\begin{screen}
		\begin{thm}
		\label{thm:equivalent_formula_rewriting_2}
			$a$を主要$\varepsilon$項とし,$\psi$を$\lang{\varepsilon}$の式とし,
			$\psi$には$z$のみ自由に現れているとする.このとき
			\begin{align}
				\EXTAX,\COMAX \vdash \forall v\, (\, v \in a \lrarrow \psi(v)\, )
				\rarrow a = \Set{z}{\psi(z)}.
			\end{align}
		\end{thm}
	\end{screen}
	
	\begin{sketch}
		いま
		\begin{align}
			\tau \defeq 
			\varepsilon x \negation (\, x \in a \lrarrow x \in \Set{z}{\psi(z)}\, )
		\end{align}
		とおく.まず全称記号の論理的公理より
		\begin{align}
			\forall v\, (\, v \in a \lrarrow \psi(v)\, )
			\vdash \tau \in a \lrarrow \psi(\tau)
			\label{fom:equivalent_formula_rewriting_2_1}
		\end{align}
		が成り立つ.また内包性公理より
		\begin{align}
			\COMAX \vdash \tau \in \Set{z}{\psi(z)} \lrarrow \psi(\tau)
		\end{align}
		となるので,同値記号の対称律(\ref{logicalthm:symmetry_of_equivalence_arrows})より
		\begin{align}
			\COMAX \vdash \psi(\tau) \lrarrow \tau \in \Set{z}{\psi(z)}
			\label{fom:equivalent_formula_rewriting_2_2}
		\end{align}
		が成り立つ.(\refeq{fom:equivalent_formula_rewriting_2_1})と
		(\refeq{fom:equivalent_formula_rewriting_2_2})と同値記号の推移律
		(論理的定理\ref{logicalthm:transitive_law_of_equivalence_symbol})より
		\begin{align}
			\forall v\, (\, v \in a \lrarrow \psi(v)\, ),\ \COMAX \vdash
			\tau \in a \lrarrow \tau \in \Set{z}{\psi(z)}
		\end{align}
		となり,全称の導出(論理的定理\ref{logicalthm:derivation_of_universal_by_epsilon})より
		\begin{align}
			\forall v\, (\, v \in a \lrarrow \psi(v)\, ),\ \COMAX \vdash
			\forall x\, (\, x \in a \lrarrow x \in \Set{z}{\psi(z)}\, )
		\end{align}
		となり,外延性公理より
		\begin{align}
			\forall v\, (\, v \in a \lrarrow \psi(v)\, ),\ \EXTAX,\COMAX \vdash
			a = \Set{z}{\psi(z)}
		\end{align}
		が得られる.
		\QED
	\end{sketch}
	
	\begin{screen}
		\begin{thm}
		\label{thm:equivalent_formula_rewriting_3}
			$b$を主要$\varepsilon$項とし,$\varphi$を$\lang{\varepsilon}$の式とし,
			$\varphi$には$y$のみ自由に現れているとする.このとき
			\begin{align}
				\EQAX,\COMAX \vdash \Set{y}{\varphi(y)} = b 
				\rarrow \forall u\, (\, \varphi(u) \lrarrow u \in b\, ).
			\end{align}
		\end{thm}
	\end{screen}
	
	\begin{sketch}
		いま
		\begin{align}
			\tau \defeq \varepsilon u \negation (\, \varphi(u) \lrarrow u \in b\, )
		\end{align}
		とおけば,まず外延性公理の逆(定理\ref{thm:inverse_of_axiom_of_extensionality})より
		\begin{align}
			\Set{y}{\varphi(y)} = b,\ \EQAX \vdash 
			\tau \in \Set{y}{\varphi(z)} \lrarrow \tau \in b
			\label{fom:equivalent_formula_rewriting_3_1}
		\end{align}
		が成り立つ.他方で内包性公理より
		\begin{align}
			\COMAX \vdash \tau \in \Set{y}{\varphi(y)} \lrarrow \varphi(\tau)
		\end{align}
		となり,同値記号の対称律(\ref{logicalthm:symmetry_of_equivalence_arrows})より
		\begin{align}
			\COMAX \vdash \varphi(\tau) \lrarrow \tau \in \Set{y}{\varphi(y)}
			\label{fom:equivalent_formula_rewriting_3_2}
		\end{align}
		が成り立つ.(\refeq{fom:equivalent_formula_rewriting_3_1})と
		(\refeq{fom:equivalent_formula_rewriting_3_2})と同値記号の推移律
		(論理的定理\ref{logicalthm:transitive_law_of_equivalence_symbol})より
		\begin{align}
			\Set{y}{\varphi(y)} = b,\ \EQAX,\COMAX \vdash 
			\varphi(\tau) \lrarrow \tau \in b 
		\end{align}
		が成り立ち,全称の導出(論理的定理\ref{logicalthm:derivation_of_universal_by_epsilon})より
		\begin{align}
			\Set{y}{\varphi(y)} = b,\ \EQAX,\COMAX \vdash 
			\forall u\, (\, \varphi(u) \lrarrow u \in b\, )
		\end{align}
		が得られる.
		\QED
	\end{sketch}
	
	\begin{screen}
		\begin{thm}
		\label{thm:equivalent_formula_rewriting_4}
			$b$を主要$\varepsilon$項とし,$\varphi$を$\lang{\varepsilon}$の式とし,
			$\varphi$には$y$のみ自由に現れているとする.このとき
			\begin{align}
				\EXTAX,\COMAX \vdash \forall u\, (\, \varphi(u) \lrarrow u \in b\, )
				\rarrow \Set{y}{\varphi(y)} = b.
			\end{align}
		\end{thm}
	\end{screen}
	
	\begin{sketch}
		いま
		\begin{align}
			\tau \defeq 
			\varepsilon x \negation (\, x \in \Set{y}{\varphi(y)} \lrarrow x \in b\, )
		\end{align}
		とおく.まず全称記号の論理的公理より
		\begin{align}
			\forall u\, (\, \varphi(u) \lrarrow u \in b\, )
			\vdash \varphi(\tau) \lrarrow \tau \in b
			\label{fom:equivalent_formula_rewriting_4_1}
		\end{align}
		が成り立ち,また内包性公理より
		\begin{align}
			\COMAX \vdash \tau \in \Set{y}{\varphi(y)} \lrarrow \varphi(\tau)
			\label{fom:equivalent_formula_rewriting_4_2}
		\end{align}
		が成り立つので,(\refeq{fom:equivalent_formula_rewriting_4_1})と
		(\refeq{fom:equivalent_formula_rewriting_4_2})と同値記号の推移律
		(論理的定理\ref{logicalthm:transitive_law_of_equivalence_symbol})より
		\begin{align}
			\forall u\, (\, \varphi(u) \lrarrow u \in b\, ),\ \COMAX \vdash
			\tau \in \Set{y}{\varphi(y)} \lrarrow \tau \in b
		\end{align}
		が成り立つ.全称の導出(論理的定理\ref{logicalthm:derivation_of_universal_by_epsilon})より
		\begin{align}
			\forall u\, (\, \varphi(u) \lrarrow u \in b\, ),\ \COMAX \vdash
			\forall x\, (\, x \in \Set{y}{\varphi(y)} \lrarrow x \in b\, )
		\end{align}
		となり,外延性公理より
		\begin{align}
			\forall u\, (\, \varphi(u) \lrarrow u \in b\, ),\ \EXTAX,\COMAX \vdash
			\Set{y}{\varphi(y)} = b
		\end{align}
		が得られる.
		\QED
	\end{sketch}
	
	\begin{screen}
		\begin{thm}
		\label{thm:equivalent_formula_rewriting_5}
			$\varphi$と$\psi$を$\lang{\varepsilon}$の式とし,
			$\varphi$には$y$のみ自由に現れ,
			$\psi$には$z$のみ自由に現れているとする.このとき
			\begin{align}
				\EQAX,\COMAX \vdash \Set{y}{\varphi(y)} = \Set{z}{\psi(z)}
				\rarrow \forall u\, (\, \varphi(u) \lrarrow \psi(u)\, ).
			\end{align}
		\end{thm}
	\end{screen}
	
	\begin{sketch}
		いま
		\begin{align}
			\tau \defeq \varepsilon u \negation (\, \varphi(u) \lrarrow \psi(u)\, )
		\end{align}
		とおけば,まず外延性公理の逆(定理\ref{thm:inverse_of_axiom_of_extensionality})より
		\begin{align}
			\Set{y}{\varphi(y)} = \Set{z}{\psi(z)},\ \EQAX \vdash 
			\tau \in \Set{y}{\varphi(z)} \lrarrow \tau \in \Set{z}{\psi(z)}
			\label{fom:equivalent_formula_rewriting_5_1}
		\end{align}
		が成り立つ.また内包性公理より
		\begin{align}
			\COMAX &\vdash \tau \in \Set{y}{\varphi(y)} \lrarrow \varphi(\tau), 
			\label{fom:equivalent_formula_rewriting_5_2} \\
			\COMAX &\vdash \tau \in \Set{z}{\varphi(z)} \lrarrow \psi(\tau)
			\label{fom:equivalent_formula_rewriting_5_3}
		\end{align}
		が成り立つが,(\refeq{fom:equivalent_formula_rewriting_5_2})と
		同値記号の対称律(\ref{logicalthm:symmetry_of_equivalence_arrows})より
		\begin{align}
			\COMAX \vdash \varphi(\tau) \lrarrow \tau \in \Set{y}{\varphi(y)}
			\label{fom:equivalent_formula_rewriting_5_4}
		\end{align}
		も成り立つ.(\refeq{fom:equivalent_formula_rewriting_5_4})と
		(\refeq{fom:equivalent_formula_rewriting_5_1})と同値記号の推移律
		(論理的定理\ref{logicalthm:transitive_law_of_equivalence_symbol})より
		\begin{align}
			\Set{y}{\varphi(y)} = \Set{z}{\psi(z)},\ \EQAX,\COMAX \vdash
			 \varphi(\tau) \lrarrow \tau \in \Set{z}{\psi(z)}
			\label{fom:equivalent_formula_rewriting_5_5}
		\end{align}
		が従い,(\refeq{fom:equivalent_formula_rewriting_5_5})と
		(\refeq{fom:equivalent_formula_rewriting_5_3})と同値記号の推移律より
		\begin{align}
			\Set{y}{\varphi(y)} = \Set{z}{\psi(z)},\ \EQAX,\COMAX \vdash
			\varphi(\tau) \lrarrow \psi(\tau)
		\end{align}
		が従う.そして全称の導出(論理的定理\ref{logicalthm:derivation_of_universal_by_epsilon})より
		\begin{align}
			\Set{y}{\varphi(y)} = \Set{z}{\psi(z)},\ \EQAX,\COMAX \vdash
			\forall u\, (\, \varphi(u) \lrarrow \psi(u)\, )
		\end{align}
		が得られる.
		\QED
	\end{sketch}
	
	\begin{screen}
		\begin{thm}
		\label{thm:equivalent_formula_rewriting_6}
			$\varphi$と$\psi$を$\lang{\varepsilon}$の式とし,
			$\varphi$には$y$のみ自由に現れ,
			$\psi$には$z$のみ自由に現れているとする.このとき
			\begin{align}
				\EXTAX,\COMAX \vdash \forall u\, (\, \varphi(u) \lrarrow \psi(u)\, )
				\rarrow \Set{y}{\varphi(y)} = \Set{z}{\psi(z)}.
			\end{align}
		\end{thm}
	\end{screen}
	
	\begin{sketch}
		いま
		\begin{align}
			\tau \defeq \varepsilon x \negation (\, x \in \Set{y}{\varphi(y)} \lrarrow x \in \Set{z}{\psi(z)}\, )
		\end{align}
		とおく.まず全称記号の論理的公理より
		\begin{align}
			\forall u\, (\, \varphi(u) \lrarrow \psi(u)\, )
			\vdash \varphi(\tau) \lrarrow \psi(\tau)
			\label{fom:equivalent_formula_rewriting_6_1}
		\end{align}
		が成り立つ.また内包性公理より
		\begin{align}
			\COMAX &\vdash \tau \in \Set{y}{\varphi(y)} \lrarrow \varphi(\tau), 
			\label{fom:equivalent_formula_rewriting_6_2} \\
			\COMAX &\vdash \tau \in \Set{z}{\psi(z)} \lrarrow \psi(\tau)
		\end{align}
		となり,同値記号の対称律(\ref{logicalthm:symmetry_of_equivalence_arrows})より
		\begin{align}
			\COMAX \vdash \psi(\tau) \lrarrow \tau \in \Set{z}{\psi(z)}
			\label{fom:equivalent_formula_rewriting_6_3}
		\end{align}
		も成り立つ.(\refeq{fom:equivalent_formula_rewriting_6_1})と
		(\refeq{fom:equivalent_formula_rewriting_6_2})と同値記号の推移律
		(論理的定理\ref{logicalthm:transitive_law_of_equivalence_symbol})より
		\begin{align}
			\forall u\, (\, \varphi(u) \lrarrow \psi(u)\, ),\ \COMAX \vdash
			\tau \in \Set{y}{\varphi(y)} \lrarrow \psi(\tau)
			\label{fom:equivalent_formula_rewriting_6_4}
		\end{align}
		となり,(\refeq{fom:equivalent_formula_rewriting_6_3})と
		(\refeq{fom:equivalent_formula_rewriting_6_4})と同値記号の推移律より
		\begin{align}
			\forall u\, (\, \varphi(u) \lrarrow \psi(u)\, ),\ \COMAX \vdash
			\tau \in \Set{y}{\varphi(y)} \lrarrow \tau \in \Set{z}{\psi(z)}
		\end{align}
		となり,全称の導出(論理的定理\ref{logicalthm:derivation_of_universal_by_epsilon})より
		\begin{align}
			\forall u\, (\, \varphi(u) \lrarrow \psi(u)\, ),\ \COMAX \vdash
			\forall x\, (\, x \in \Set{y}{\varphi(y)} \lrarrow x \in \Set{z}{\psi(z)}\, )
		\end{align}
		となり,外延性公理より
		\begin{align}
			\forall u\, (\, \varphi(u) \lrarrow \psi(u)\, ),\ \EXTAX,\COMAX \vdash
			\Set{y}{\varphi(y)} = \Set{z}{\psi(z)}
		\end{align}
		が得られる.
		\QED
	\end{sketch}
	
	\begin{screen}
		\begin{thm}
		\label{thm:equivalent_formula_rewriting_7}
			$a$を主要$\varepsilon$項とし,$\psi$を$\lang{\varepsilon}$の式とし,
			$\psi$には$z$のみ自由に現れているとする.このとき
			\begin{align}
				\COMAX \vdash a \in \Set{z}{\psi(z)} \rarrow \psi(a).
			\end{align}
		\end{thm}
	\end{screen}
	
	\begin{sketch}
		$a$は主要$\varepsilon$項であるから,内包性公理より
		\begin{align}
			\COMAX \vdash a \in \Set{z}{\psi(z)} \rarrow \psi(a)
		\end{align}
		が成り立つ.
		\QED
	\end{sketch}
	
	\begin{screen}
		\begin{thm}
		\label{thm:equivalent_formula_rewriting_8}
			$a$を主要$\varepsilon$項とし,$\psi$を$\lang{\varepsilon}$の式とし,
			$\psi$には$z$のみ自由に現れているとする.このとき
			\begin{align}
				\COMAX \vdash \psi(a) \rarrow a \in \Set{z}{\psi(z)}.
			\end{align}
		\end{thm}
	\end{screen}
	
	\begin{sketch}
		$a$は主要$\varepsilon$項であるから,内包性公理より
		\begin{align}
			\COMAX \vdash \psi(a) \rarrow a \in \Set{z}{\psi(z)}
		\end{align}
		が成り立つ.
		\QED
	\end{sketch}
	
	\begin{screen}
		\begin{thm}
		\label{thm:equivalent_formula_rewriting_9}
			$b$を主要$\varepsilon$項とし,$\varphi$を$\lang{\varepsilon}$の式とし,
			$\varphi$には$y$のみ自由に現れているとする.このとき
			\begin{align}
				\EQAX,\COMAX,\ELEAX \vdash \Set{y}{\varphi(y)} \in b
				\rarrow \exists s\, (\, 
				\forall u\, (\, \varphi(u) \lrarrow u \in s\, )
				\wedge s \in b\, ).
			\end{align}
		\end{thm}
	\end{screen}
	
	\begin{sketch}
		要素の公理より
		\begin{align}
			\Set{y}{\varphi(y)} \in b,\ \ELEAX \vdash 
			\exists s\, (\, \Set{y}{\varphi(y)} = s\, )
		\end{align}
		が成り立つので,
		\begin{align}
			\sigma \defeq 
			\varepsilon s\, \forall u\, (\, \varphi(u) \lrarrow u \in s\, )
		\end{align}
		とおけば存在記号の論理的公理より
		\begin{align}
			\Set{y}{\varphi(y)} \in b,\ \ELEAX \vdash \Set{y}{\varphi(y)} = \sigma
			\label{fom:equivalent_formula_rewriting_9_1}
		\end{align}
		となる.ここで相等性公理より
		\begin{align}
			\EQAX \vdash \Set{y}{\varphi(y)} = \sigma
			\rarrow (\, \Set{y}{\varphi(y)} \in b \rarrow \sigma \in b\, )
		\end{align}
		が成り立つので,(\refeq{fom:equivalent_formula_rewriting_9_1})と三段論法より
		\begin{align}
			\Set{y}{\varphi(y)} \in b,\ \EQAX,\ELEAX \vdash \sigma \in b
			\label{fom:equivalent_formula_rewriting_9_2}
		\end{align}
		が得られる.他方で定理\ref{thm:equivalent_formula_rewriting_3}より
		\begin{align}
			\EQAX,\COMAX \vdash \Set{y}{\varphi(y)} = \sigma
			\rarrow \forall u\, (\, \varphi(u) \lrarrow u \in \sigma\, )
		\end{align}
		が成り立つので,(\refeq{fom:equivalent_formula_rewriting_9_1})と三段論法より
		\begin{align}
			\Set{y}{\varphi(y)} \in b,\ \EQAX,\COMAX,\ELEAX \vdash
			\forall u\, (\, \varphi(u) \lrarrow u \in \sigma\, )
			\label{fom:equivalent_formula_rewriting_9_3}
		\end{align}
		も得られる.(\refeq{fom:equivalent_formula_rewriting_9_2})と
		(\refeq{fom:equivalent_formula_rewriting_9_3})と論理積の導入より
		\begin{align}
			\Set{y}{\varphi(y)} \in b,\ \EQAX,\COMAX,\ELEAX \vdash
			\forall u\, (\, \varphi(u) \lrarrow u \in \sigma\, ) \wedge \sigma \in b
		\end{align}
		が成り立つので,存在記号の論理的公理より
		\begin{align}
			\Set{y}{\varphi(y)} \in b,\ \EQAX,\COMAX,\ELEAX \vdash
			\exists s\, (\, \forall u\, (\, \varphi(u) \lrarrow u \in s\, ) \wedge s \in b\, )
		\end{align}
		が得られる.
		\QED
	\end{sketch}
	
	\begin{screen}
		\begin{thm}
		\label{thm:equivalent_formula_rewriting_10}
			$b$を主要$\varepsilon$項とし,$\varphi$を$\lang{\varepsilon}$の式とし,
			$\varphi$には$y$のみ自由に現れているとする.このとき
			\begin{align}
				\EXTAX,\EQAX,\COMAX \vdash \exists s\, (\, \forall u\, (\, \varphi(u) \lrarrow u \in s\, ) \wedge s \in b\, ) \rarrow \Set{y}{\varphi(y)} \in b.
			\end{align}
		\end{thm}
	\end{screen}
	
	\begin{sketch}
		いま
		\begin{align}
			\sigma \defeq \varepsilon s\, (\, \forall u\, (\, \varphi(u) \lrarrow u \in s\, ) \wedge s \in b\, )
		\end{align}
		とおけば,存在記号の論理的公理と論理積の除去より
		\begin{align}
			\exists s\, (\, \forall u\, (\, \varphi(u) \lrarrow u \in s\, ) \wedge s \in b\, )
			&\vdash \forall u\, (\, \varphi(u) \lrarrow u \in \sigma\, ), 
			\label{fom:equivalent_formula_rewriting_10_1} \\
			\exists s\, (\, \forall u\, (\, \varphi(u) \lrarrow u \in s\, ) \wedge s \in b\, )
			&\vdash \sigma \in b
			\label{fom:equivalent_formula_rewriting_10_2}
		\end{align}
		が成り立つ.ここで定理\ref{thm:equivalent_formula_rewriting_4}より
		\begin{align}
			\EXTAX,\COMAX \vdash \forall u\, (\, \varphi(u) \lrarrow u \in \sigma\, )
			\rarrow \Set{y}{\varphi(y)} = \sigma
		\end{align}
		が成り立つので,(\refeq{fom:equivalent_formula_rewriting_10_1})との三段論法より
		\begin{align}
			\exists s\, (\, \forall u\, (\, \varphi(u) \lrarrow u \in s\, ) \wedge s \in b\, ),\ \EXTAX,\COMAX \vdash \Set{y}{\varphi(y)} = \sigma
			\label{fom:equivalent_formula_rewriting_10_3}
		\end{align}
		が得られる.また相等性公理より
		\begin{align}
			\EQAX &\vdash \Set{y}{\varphi(y)} = \sigma \rarrow \sigma = \Set{y}{\varphi(y)}, \\
			\EQAX &\vdash \sigma = \Set{y}{\varphi(y)} \rarrow
			(\, \sigma \in b \rarrow \Set{y}{\varphi(y)} \in b\, )
		\end{align}
		が成り立つので,(\refeq{fom:equivalent_formula_rewriting_10_2})と
		(\refeq{fom:equivalent_formula_rewriting_10_3})との三段論法より
		\begin{align}
			\exists s\, (\, \forall u\, (\, \varphi(u) \lrarrow u \in s\, ) \wedge s \in b\, ),\ \EXTAX,\EQAX,\COMAX \vdash \Set{y}{\varphi(y)} \in b
		\end{align}
		が従う.
		\QED
	\end{sketch}
	
	\begin{screen}
		\begin{thm}
		\label{thm:equivalent_formula_rewriting_11}
			$\varphi$と$\psi$を$\lang{\varepsilon}$の式とし,
			$\varphi$には$y$のみ自由に現れ,
			$\psi$には$z$のみ自由に現れているとする.このとき
			\begin{align}
				\EQAX,\COMAX,\ELEAX \vdash \Set{y}{\varphi(y)} \in \Set{z}{\psi(z)}
				\rarrow \exists s\, (\, 
				\forall u\, (\, \varphi(u) \lrarrow u \in s\, )
				\wedge \psi(s)\, ).
			\end{align}
		\end{thm}
	\end{screen}
	
	\begin{sketch}
		まず(\refeq{fom:equivalent_formula_rewriting_9_1})と
		(\refeq{fom:equivalent_formula_rewriting_9_3})と同様に,
		\begin{align}
			\sigma \defeq 
			\varepsilon s\, \forall u\, (\, \varphi(u) \lrarrow u \in s\, )
		\end{align}
		とおけば
		\begin{align}
			\Set{y}{\varphi(y)} \in \Set{z}{\psi(z)},\ \ELEAX \vdash 
			\Set{y}{\varphi(y)} = \sigma
			\label{fom:equivalent_formula_rewriting_11_1}
		\end{align}
		と
		\begin{align}
			\Set{y}{\varphi(y)} \in \Set{z}{\psi(z)},\ \EQAX,\COMAX,\ELEAX \vdash
			\forall u\, (\, \varphi(u) \lrarrow u \in \sigma\, )
			\label{fom:equivalent_formula_rewriting_11_2}
		\end{align}
		が成り立つ.また相等性公理より
		\begin{align}
			\EQAX \vdash \Set{y}{\varphi(y)} = \sigma
			\rarrow (\, \Set{y}{\varphi(y)} \in \Set{z}{\psi(z)}
			\rarrow \sigma \in \Set{z}{\psi(z)}\, )
		\end{align}
		となるので,(\refeq{fom:equivalent_formula_rewriting_11_1})との三段論法より
		\begin{align}
			\Set{y}{\varphi(y)} \in \Set{z}{\psi(z)},\ \EQAX,\ELEAX \vdash 
			\sigma \in \Set{z}{\psi(z)}
		\end{align}
		が成り立ち,内包性公理より
		\begin{align}
			\COMAX \vdash \sigma \in \Set{z}{\psi(z)} \rarrow \psi(\sigma)
		\end{align}
		が成り立つので
		\begin{align}
			\Set{y}{\varphi(y)} \in \Set{z}{\psi(z)},\ \EQAX,\ELEAX \vdash 
			\psi(\sigma)
			\label{fom:equivalent_formula_rewriting_11_3}
		\end{align}
		が得られる.(\refeq{fom:equivalent_formula_rewriting_11_2})と
		(\refeq{fom:equivalent_formula_rewriting_11_3})と論理積の導入より
		\begin{align}
			\Set{y}{\varphi(y)} \in \Set{z}{\psi(z)},\ \EQAX,\COMAX,\ELEAX \vdash
			\forall u\, (\, \varphi(u) \lrarrow u \in \sigma\, ) \wedge \psi(\sigma)
		\end{align}
		が成り立ち,存在記号の論理的公理より
		\begin{align}
			\Set{y}{\varphi(y)} \in \Set{z}{\psi(z)},\ \EQAX,\COMAX,\ELEAX \vdash
			\exists s\, (\forall u\, (\, \varphi(u) \lrarrow u \in x\, ) \wedge \psi(x)\, )
		\end{align}
		が得られる.
		\QED
	\end{sketch}
	
	\begin{screen}
		\begin{thm}
		\label{thm:equivalent_formula_rewriting_12}
			$\varphi$と$\psi$を$\lang{\varepsilon}$の式とし,
			$\varphi$には$y$のみ自由に現れ,
			$\psi$には$z$のみ自由に現れているとする.このとき
			\begin{align}
				\EXTAX,\EQAX,\COMAX \vdash \exists s\, (\, \forall u\, (\, \varphi(u) \lrarrow u \in s\, ) \wedge \psi(s)\, ) \rarrow \Set{y}{\varphi(y)} \in \Set{z}{\psi(z)}.
			\end{align}
		\end{thm}
	\end{screen}
	
	\begin{sketch}
		いま
		\begin{align}
			\sigma \defeq \varepsilon s\, (\, \forall u\, (\, \varphi(u) \lrarrow u \in s\, ) \wedge \psi(s)\, )
		\end{align}
		とおけば,存在記号の論理的公理と論理積の除去より
		\begin{align}
			\exists s\, (\, \forall u\, (\, \varphi(u) \lrarrow u \in s\, ) \wedge \psi(s)\, )
			&\vdash \forall u\, (\, \varphi(u) \lrarrow u \in \sigma\, ), 
			\label{fom:equivalent_formula_rewriting_12_1} \\
			\exists s\, (\, \forall u\, (\, \varphi(u) \lrarrow u \in s\, ) \wedge \psi(s)\, )
			&\vdash \psi(\sigma)
			\label{fom:equivalent_formula_rewriting_12_2}
		\end{align}
		が成り立つ.ここで定理\ref{thm:equivalent_formula_rewriting_4}より
		\begin{align}
			\EXTAX,\COMAX \vdash \forall u\, (\, \varphi(u) \lrarrow u \in \sigma\, )
			\rarrow \Set{y}{\varphi(y)} = \sigma
		\end{align}
		が成り立つので,(\refeq{fom:equivalent_formula_rewriting_12_1})との三段論法より
		\begin{align}
			\exists s\, (\, \forall u\, (\, \varphi(u) \lrarrow u \in s\, ) \wedge \psi(s)\, ),\ \EXTAX,\COMAX \vdash \Set{y}{\varphi(y)} = \sigma
			\label{fom:equivalent_formula_rewriting_12_3}
		\end{align}
		が得られる.また内包性公理より
		\begin{align}
			\COMAX \vdash \psi(\sigma) \rarrow \sigma \in \Set{z}{\psi(z)}
		\end{align}
		が成り立つので,(\refeq{fom:equivalent_formula_rewriting_12_1})との三段論法より
		\begin{align}
			\exists s\, (\, \forall u\, (\, \varphi(u) \lrarrow u \in s\, ) \wedge \psi(s)\, ),\ \COMAX \vdash \sigma \in \Set{z}{\psi(z)}
			\label{fom:equivalent_formula_rewriting_12_4}
		\end{align}
		が得られる.相等性公理より
		\begin{align}
			\EQAX &\vdash \Set{y}{\varphi(y)} = \sigma \rarrow \sigma = \Set{y}{\varphi(y)}, \\
			\EQAX &\vdash \sigma = \Set{y}{\varphi(y)} \rarrow
			(\, \sigma \in \Set{z}{\psi(z)} \rarrow \Set{y}{\varphi(y)} \in \Set{z}{\psi(z)}\, )
		\end{align}
		が成り立つので,(\refeq{fom:equivalent_formula_rewriting_12_3})と
		(\refeq{fom:equivalent_formula_rewriting_12_4})との三段論法より
		\begin{align}
			\exists s\, (\, \forall u\, (\, \varphi(u) \lrarrow u \in s\, ) \wedge \psi(s)\, ),\ \EXTAX,\EQAX,\COMAX \vdash \Set{y}{\varphi(y)} \in \Set{z}{\psi(z)}
		\end{align}
		が従う.
		\QED
	\end{sketch}
	
	一般の式に対して書き換えの同値性を示す前に,いくつか必要な論理的定理を準備する.
	
	\begin{screen}
		\begin{logicalthm}[含意の論理和への遺伝性2]
		\label{logicalthm:heredity_of_implication_to_disjunction_2}
			$A,B,C,D$を文とするとき
			\begin{align}
				A \rarrow C,\ B \rarrow D \vdash A \vee B \rarrow C \vee D.
			\end{align}
		\end{logicalthm}
	\end{screen}
	
	\begin{sketch}
		三段論法より
		\begin{align}
			A,\ A \rarrow C \vdash C
		\end{align}
		が成り立ち,論理和の導入より
		\begin{align}
			A,\ A \rarrow C \vdash C \vee D
		\end{align}
		が成り立つので,
		\begin{align}
			A \rarrow C \vdash A \rarrow C \vee D
		\end{align}
		が従う.同様に
		\begin{align}
			B \rarrow D \vdash B \rarrow C \vee D
		\end{align}
		も成り立ち,論理和の除去より
		\begin{align}
			A \rarrow C,\ B \rarrow D \vdash A \vee B \rarrow C \vee D
		\end{align}
		が得られる.
		\QED
	\end{sketch}
	
	\begin{screen}
		\begin{logicalthm}[含意の論理積への遺伝性2]
		\label{logicalthm:heredity_of_implication_to_conjunction_2}
			$A,B,C,D$を文とするとき,
			\begin{align}
				A \rarrow C,\ B \rarrow D \vdash A \wedge B \rarrow C \wedge D.
			\end{align}
		\end{logicalthm}
	\end{screen}
	
	\begin{sketch}
		論理積の除去より
		\begin{align}
			A \wedge B \vdash A
		\end{align}
		が成り立つので,三段論法より
		\begin{align}
			A \wedge B,\ A \rarrow C \vdash C
		\end{align}
		が従う.同様に
		\begin{align}
			A \wedge B,\ B \rarrow D \vdash D
		\end{align}
		も成り立ち,論理積の導入より
		\begin{align}
			A \wedge B,\ A \rarrow C,\ B \rarrow D \vdash C \wedge D
		\end{align}
		が従う.
		\QED
	\end{sketch}
	
	\begin{screen}
		\begin{logicalthm}[含意の含意への遺伝性2]
		\label{logicalthm:heredity_of_implication_to_implication_2}
			$A,B,C,D$を文とするとき,
			\begin{align}
				C \rarrow A,\ B \rarrow D \vdash (\, A \rarrow B\, ) 
				\rarrow (\, C \rarrow D\, ).
			\end{align}
		\end{logicalthm}
	\end{screen}
	
	\begin{sketch}
		三段論法より
		\begin{align}
			C,\ C \rarrow A \vdash A
		\end{align}
		が成り立つので,再び三段論法より
		\begin{align}
			C,\ A \rarrow B,\ C \rarrow A \vdash B
		\end{align}
		が成り立ち,再び三段論法より
		\begin{align}
			C,\ A \rarrow B,\ C \rarrow A,\ B \rarrow D \vdash D
		\end{align}
		が従う.そして演繹定理より
		\begin{align}
			C \rarrow A,\ B \rarrow D \vdash 
			(\, A \rarrow B\, ) \rarrow (\, C \rarrow D\, )
		\end{align}
		が得られる.
		\QED
	\end{sketch}
	
	\begin{screen}
		\begin{thm}[書き換えの同値性]
		\label{thm:equivalence_of_formula_rewritings}
			$\varphi$を$\lang{\varepsilon}$の文ではない$\mathcal{L}$の文とし,
			$\widehat{\varphi}$を$\varphi$の書き換えとするとき,
			\begin{align}
				\EXTAX,\EQAX,\COMAX,\ELEAX \vdash \varphi \lrarrow \widehat{\varphi}.
			\end{align}
		\end{thm}
	\end{screen}
	
	\begin{sketch}\mbox{}
		\begin{description}
			\item[step1] $\widehat{\varphi}$が,$\varphi$の部分式で原子式であるものを全て
				表\ref{tab:formula_rewriting_repeat}の通りに書き換えた式である場合を扱う.
				\begin{description}
					\item[step1-1] $\varphi$が原子式であるならば,すでに示した通り
						\begin{align}
							\EXTAX,\EQAX,\COMAX,\ELEAX \vdash \varphi \lrarrow \widehat{\varphi}
						\end{align}
						が成立する.
							
					\item[step1-2] $\varphi$が原子式でないとき,
						\begin{itembox}[l]{IH (帰納法の仮定)}
							$\varphi$の任意の真部分式$\psi$に対して,
							$\psi$に自由に現れている変項が$x_{1},\cdots,x_{n}$で全てであるとき,
							任意に主要$\varepsilon$項$\tau_{1},\cdots,\tau_{n}$を取って
							$\psi(x_{1}/\tau_{1})\cdots(x_{n}/\tau_{n})$なる文を
							$\psi^{*}$とする.$\psi$が文なら$\psi^{*}$を$\psi$とする.
							このとき,$\psi^{*}$の部分式で原子式であるものを全て
							表\ref{tab:formula_rewriting_repeat}の通りに書き換えた式\footnotemark
							$\widehat{\psi^{*}}$に対して
							\begin{align}
								\EXTAX,\EQAX,\COMAX,\ELEAX \vdash \psi^{*} \lrarrow \widehat{\psi^{*}}
							\end{align}
							となる
						\end{itembox}
						と仮定する(帰納法の考え方としては,$\varphi$よりも``前の段階''で作られた
						任意の文に対して同値性が満たされていると仮定している).
						
						\footnotetext{
							文の書き換えは文である(メタ定理\ref{metathm:variables_unchanged_after_rewriting}).
						}
						
						\begin{description}
							\item[case1] $\varphi$が
								\begin{align}
									\negation \psi
								\end{align}
								なる文であるとき,$\widehat{\varphi}$は
								\begin{align}
									\negation \widehat{\psi}
								\end{align}
								なる形で書ける.ただし$\widehat{\psi}$とは$\psi$の部分式で原子式であるものを全て
								表\ref{tab:formula_rewriting_repeat}の通りに書き換えた式である.
								(IH)より
								\begin{align}
									\EXTAX,\EQAX,\COMAX,\ELEAX \vdash \psi \lrarrow \widehat{\psi}
								\end{align}
								が成り立つので,対偶律1 (論理的定理\ref{logicalthm:introduction_of_contraposition})より
								\begin{align}
									\EXTAX,\EQAX,\COMAX,\ELEAX \vdash\ \negation \psi \lrarrow\ \negation \widehat{\psi}
								\end{align}
								が従う.すなわち
								\begin{align}
									\EXTAX,\EQAX,\COMAX,\ELEAX \vdash \varphi \lrarrow \widehat{\varphi}
								\end{align}
								が成り立つ.
								
							\item[case2] $\varphi$が
								\begin{align}
									\psi \vee \chi
								\end{align}
								なる文であるとき,$\widehat{\varphi}$は
								\begin{align}
									\widehat{\psi} \vee \widehat{\chi}
								\end{align}
								なる形で書ける.ただし$\widehat{\psi},\widehat{\chi}$とは
								それぞれ$\psi,\chi$の部分式で原子式であるものを全て
								表\ref{tab:formula_rewriting_repeat}の通りに書き換えた式である.
								(IH)より
								\begin{align}
									\EXTAX,\EQAX,\COMAX,\ELEAX &\vdash \psi \lrarrow \widehat{\psi}, \\
									\EXTAX,\EQAX,\COMAX,\ELEAX &\vdash \chi \lrarrow \widehat{\chi}
								\end{align}
								が成り立つので,含意の論理和への遺伝性
								(論理的定理\ref{logicalthm:heredity_of_implication_to_conjunction_2})より
								\begin{align}
									\EXTAX,\EQAX,\COMAX,\ELEAX \vdash \psi \vee \chi 
									\lrarrow \widehat{\psi} \vee \widehat{\chi}
								\end{align}
								が従う.すなわち
								\begin{align}
									\EXTAX,\EQAX,\COMAX,\ELEAX \vdash \varphi \lrarrow \widehat{\varphi}
								\end{align}
								が成り立つ.$\varphi$が$\eta \wedge \chi$や$\eta \rarrow \chi$
								なる文の場合も論理的定理
								\ref{logicalthm:heredity_of_implication_to_conjunction_2}
								或いは論理的定理
								\ref{logicalthm:heredity_of_implication_to_implication_2}を使えば
								同様にして書き換えの同値性が得られる.
								
							\item[case3] $\varphi$が
								\begin{align}
									\exists x \psi
								\end{align}
								なる文であるとき,$\widehat{\varphi}$は
								\begin{align}
									\exists x \widehat{\psi}
								\end{align}
								なる形で書ける.ただし$\widehat{\psi}$とは$\psi$の部分式で原子式であるものを全て
								表\ref{tab:formula_rewriting_repeat}の通りに書き換えた式である.
								\begin{align}
									\tau \defeq \varepsilon x \widehat{\psi}
								\end{align}
								とおけば,存在記号の論理的公理より
								\begin{align}
									\exists x \widehat{\psi} \vdash \widehat{\psi}(x/\tau)
								\end{align}
								が成り立つが,いま$\widehat{\psi}$は$\psi$の書き換えであるから,
								メタ定理\ref{metathm:substitution_to_rewritten_formula}より
								$\widehat{\psi}(x/\tau)$は$\psi(x/\tau)$の書き換えである.従って(IH)より
								\begin{align}
									\EXTAX,\EQAX,\COMAX,\ELEAX \vdash \widehat{\psi}(x/\tau) \rarrow \psi(x/\tau)
								\end{align}
								が成り立つ.ここについて間違いの内容に述べておくと,実はこの場合$\widehat{\psi}(x/\tau)$は
								$\psi(x/\tau)$の原子式を書き換えただけの式であるかは
								定かではない(メタ定理\ref{metathm:substitution_to_rewritten_formula}
								証明中のcase 6,7参照).
								しかし書き換えである以上は,原子式を書き換えただけでなくとも
								せいぜい何回かの量化部分式の差し替えが行われただけであり,
								その差し替え前後の同値性は定理\ref{logicalthm:equivalence_by_replacing_bound_variables}
								より保証される.従って,$\psi(x/\tau)$とその原子式を書き換えただけの式との同値性が
								(IH)により保証されれば,$\psi(x/\tau)$と$\widehat{\psi}(x/\tau)$も同値となるのである.
								話を戻せば,三段論法より
								\begin{align}
									\exists x \widehat{\psi},\ \EXTAX,\EQAX,\COMAX,\ELEAX \vdash \psi(x/\tau)
								\end{align}
								となり,存在記号の論理的公理より
								\begin{align}
									\exists x \widehat{\psi},\ \EXTAX,\EQAX,\COMAX,\ELEAX \vdash \exists x \psi
								\end{align}
								が従う.同様にして(いまの$\tau$で$\exists x \psi \rarrow \psi(x/\tau)$が公理であることを使う)
								\begin{align}
									\EXTAX,\EQAX,\COMAX,\ELEAX \vdash \exists x \psi \rarrow \exists x \widehat{\psi}
								\end{align}
								も成り立つので,論理積の導入より
								\begin{align}
									\EXTAX,\EQAX,\COMAX,\ELEAX \vdash \exists x \psi \lrarrow \exists x \widehat{\psi}
								\end{align}
								が従う.すなわち
								\begin{align}
									\EXTAX,\EQAX,\COMAX,\ELEAX \vdash \varphi \lrarrow \widehat{\varphi}
								\end{align}
								が成り立つ.$\varphi$が$\forall y \psi$なる式であっても,全称の導出
								(論理的定理\ref{logicalthm:derivation_of_universal_by_epsilon})を利用すれば
								同様にして書き換えの同値性が得られる.
						\end{description}
				\end{description}
				
			\item[step2] $\widehat{\varphi}$を$\varphi$の書き換えとして,
				\begin{itembox}[l]{IH (帰納法の仮定)}
					$\varphi$と$\widehat{\varphi}$は$\EXTAX,\EQAX,\COMAX,\ELEAX$の下で同値である:
					\begin{align}
						\EXTAX,\EQAX,\COMAX,\ELEAX \vdash \varphi \lrarrow \widehat{\varphi}
					\end{align}
				\end{itembox}
				であると仮定する.このとき,$\widehat{\varphi}$の量化部分式を
				一つ差し替えた式を$\widetilde{\varphi}$とする.
				つまり$\widetilde{\varphi}$とは,
				$\widehat{\varphi}$に現れる$\forall x \xi$ (resp. $\exists x \xi$)の形の部分式を
				一つだけ$\forall y \xi(x/y)$ (resp. $\exists y \xi(x/y)$)に差し替えた式である.
				ここで$y$は$\xi$に自由に現れない変項で,$\xi$の中で$x$への代入について自由であるものとする.
				このとき定理\ref{logicalthm:equivalence_by_replacing_bound_variables}より
				\begin{align}
					\vdash \widehat{\varphi} \lrarrow \widetilde{\varphi}
				\end{align}
				が成り立つので,(IH)と同値関係の推移律
				(論理的定理\ref{logicalthm:transitive_law_of_equivalence_symbol})より
				\begin{align}
					\EXTAX,\EQAX,\COMAX,\ELEAX \vdash \varphi \lrarrow \widetilde{\varphi}
				\end{align}
				が成立する.ゆえに構造的帰納法の原理より,$\varphi$とその任意の書き換えは
				$\EXTAX,\EQAX,\COMAX,\ELEAX$の下で同値である.
				\QED
		\end{description}
	\end{sketch}
	
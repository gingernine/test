\section{変換の同値性}
	\begin{screen}
		\begin{logicalthm}[同値記号の対称律]
		\label{logicalthm:symmetry_of_equivalence_arrows}
			$A,B$を$\mathcal{L}$の文とするとき
			\begin{align}
				\vdash (A \lrarrow B) \rarrow (B \lrarrow A).
			\end{align}
		\end{logicalthm}
	\end{screen}
	
	\begin{prf}
		$\wedge$の除去(推論規則\ref{logicalaxm:elimination_of_conjunction})より
		\begin{align}
			A \lrarrow B &\vdash A \rarrow B, \\
			A \lrarrow B &\vdash B \rarrow A
		\end{align}
		となる.他方で論理積の導入(推論規則\ref{logicalaxm:introduction_of_conjunction})より
		\begin{align}
			\vdash (B \rarrow A) \rarrow ((A \rarrow B) \rarrow 
			(B \rarrow A) \wedge (A \rarrow B))
		\end{align}
		が成り立つので,三段論法を二回適用すれば
		\begin{align}
			A \lrarrow B \vdash (B \rarrow A) \wedge (A \rarrow B)
		\end{align}
		となる.つまり
		\begin{align}
			A \lrarrow B \vdash B \lrarrow A
		\end{align}
		が得られた.
		\QED
	\end{prf}
	
	\begin{itemize}
		\item $a = \Set{z}{\psi(z)} \rarrow \forall v\, (\, v \in a \lrarrow \psi(v)\, )$
		\item $\Set{y}{\varphi(y)} = b \rarrow \forall u\, (\, \varphi(u) \lrarrow u \in b\, )$
		\item $\Set{y}{\varphi(y)} = \Set{z}{\psi(z)} \rarrow \forall u\, (\, \varphi(u) \lrarrow \psi(u)\, )$
		\item $a \in \Set{z}{\psi(z)} \rarrow \psi(a)$
		\item $\Set{y}{\varphi(y)} \in b
			\rarrow \exists s\, (\, \forall u\, (\, \varphi(u) \lrarrow u \in s\, )
			\wedge s \in b\, )$
		\item $\Set{y}{\varphi(y)} \in \Set{z}{\psi(z)} 
			\rarrow \exists s\, (\, \forall u\, (\, \varphi(u) \lrarrow u \in s\, )
			\wedge \psi(s)\, )$
	\end{itemize}
	
	\begin{screen}
		\begin{thm}
		\label{thm:equivalent_formula_rewriting_1}
			$a$を主要$\varepsilon$項とし,$\psi$を$\lang{\varepsilon}$の式とし,
			$z$を$\psi$に自由に現れる変項とし,$\psi$に自由に現れる変項は$z$のみであるとする.このとき
			\begin{align}
				\EQAX,\COMAX \vdash a = \Set{z}{\psi(z)} 
				\rarrow \forall v\, (\, v \in a \lrarrow \psi(v)\, ).
			\end{align}
		\end{thm}
	\end{screen}
	
	\begin{sketch}
		いま
		\begin{align}
			\tau \defeq \varepsilon v \negation (\, v \in a \lrarrow \psi(v)\, )
		\end{align}
		とおく.外延性公理の逆(定理\ref{thm:inverse_of_axiom_of_extensionality})より
		\begin{align}
			a = \Set{z}{\psi(z)},\ \EQAX \vdash 
			\tau \in a \lrarrow \tau \in \Set{z}{\psi(z)}
		\end{align}
		が成り立ち,他方で内包性公理より
		\begin{align}
			\COMAX \vdash \tau \in \Set{z}{\psi(z)} \lrarrow \psi(\tau)
		\end{align}
		が成り立つので,同値記号の推移律
		(推論法則\ref{logicalthm:transitive_law_of_equivalence_symbol})より
		\begin{align}
			a = \Set{z}{\psi(z)},\ \EQAX,\COMAX \vdash \tau \in a \lrarrow \psi(\tau)
		\end{align}
		が従う.そして全称記号の推論規則より
		\begin{align}
			a = \Set{z}{\psi(z)},\ \EQAX,\COMAX \vdash 
			\forall v\, (\, v \in a \lrarrow \psi(v)\, )
		\end{align}
		が得られる.
		\QED
	\end{sketch}
	
	\begin{screen}
		\begin{thm}
		\label{thm:equivalent_formula_rewriting_3}
			$b$を主要$\varepsilon$項とし,$\varphi$を$\lang{\varepsilon}$の式とし,
			$y$を$\varphi$に自由に現れる変項とし,$\varphi$に自由に現れる変項は$y$のみ
			であるとする.このとき
			\begin{align}
				\EQAX,\COMAX \vdash \Set{y}{\varphi(y)} = b 
				\rarrow \forall u\, (\, \varphi(u) \lrarrow u \in b\, ).
			\end{align}
		\end{thm}
	\end{screen}
	
	\begin{sketch}
		いま
		\begin{align}
			\tau \defeq \varepsilon u \negation (\, \varphi(u) \lrarrow u \in b\, )
		\end{align}
		とおけば,まず外延性公理の逆(定理\ref{thm:inverse_of_axiom_of_extensionality})より
		\begin{align}
			\Set{y}{\varphi(y)} = b,\ \EQAX \vdash 
			\tau \in \Set{z}{\psi(z)} \lrarrow \tau \in b
			\label{fom:equivalent_formula_rewriting_3_1}
		\end{align}
		が成り立つ.他方で内包性公理より
		\begin{align}
			\COMAX \vdash \tau \in \Set{y}{\varphi(y)} \lrarrow \varphi(\tau)
		\end{align}
		となり,同値記号の対称律(\ref{logicalthm:symmetry_of_equivalence_arrows})より
		\begin{align}
			\COMAX \vdash \varphi(\tau) \lrarrow \tau \in \Set{y}{\varphi(y)}
			\label{fom:equivalent_formula_rewriting_3_2}
		\end{align}
		が成り立つ.(\refeq{fom:equivalent_formula_rewriting_3_1})と
		(\refeq{fom:equivalent_formula_rewriting_3_2})と同値記号の推移律
		(推論法則\ref{logicalthm:transitive_law_of_equivalence_symbol})より
		\begin{align}
			\Set{y}{\varphi(y)} = b,\ \EQAX,\COMAX \vdash 
			\varphi(\tau) \lrarrow \tau \in b 
		\end{align}
		が成り立ち,全称記号の推論規則より
		\begin{align}
			\Set{y}{\varphi(y)} = b,\ \EQAX,\COMAX \vdash 
			\forall u\, (\, \varphi(u) \lrarrow u \in b\, )
		\end{align}
		が得られる.
		\QED
	\end{sketch}
	
	\begin{screen}
		\begin{thm}
			$a$を主要$\varepsilon$項とし,$\psi$を$\lang{\varepsilon}$の式とし,
			$z$を$\psi$に自由に現れる変項とし,$\psi$に自由に現れる変項は$z$のみであるとする.このとき
			\begin{align}
				\COMAX \vdash a \in \Set{z}{\psi(z)} \rarrow \psi(a).
			\end{align}
		\end{thm}
	\end{screen}
	
	\begin{sketch}
		$a$は主要$\varepsilon$項であるから,内包性公理より
		\begin{align}
			\COMAX \vdash a \in \Set{z}{\psi(z)} \rarrow \psi(a)
		\end{align}
		が成り立つ.
		\QED
	\end{sketch}
	
	\begin{screen}
		\begin{thm}
			$a$を主要$\varepsilon$項とし,$\psi$を$\lang{\varepsilon}$の式とし,
			$z$を$\psi$に自由に現れる変項とし,$\psi$に自由に現れる変項は$z$のみであるとする.このとき
			\begin{align}
				\COMAX \vdash \psi(a) \rarrow a \in \Set{z}{\psi(z)}.
			\end{align}
		\end{thm}
	\end{screen}
	
	\begin{sketch}
		$a$は主要$\varepsilon$項であるから,内包性公理より
		\begin{align}
			\COMAX \vdash \psi(a) \rarrow a \in \Set{z}{\psi(z)}
		\end{align}
		が成り立つ.
		\QED
	\end{sketch}
	
	\begin{screen}
		\begin{thm}
		\label{thm:equivalent_formula_rewriting_9}
			$b$を主要$\varepsilon$項とし,$\varphi$を$\lang{\varepsilon}$の式とし,
			$y$を$\varphi$に自由に現れる変項とし,
			$\varphi$に自由に現れる変項は$y$のみであるとする.このとき
			\begin{align}
				\EQAX,\COMAX,\ELEAX \vdash \Set{y}{\varphi(y)} \in b
				\rarrow \exists s\, (\, 
				\forall u\, (\, \varphi(u) \lrarrow u \in s\, )
				\wedge s \in b\, ).
			\end{align}
		\end{thm}
	\end{screen}
	
	\begin{sketch}
		要素の公理より
		\begin{align}
			\Set{y}{\varphi(y)} \in b,\ \ELEAX \vdash 
			\exists s\, (\, \Set{y}{\varphi(y)} = s\, )
		\end{align}
		が成り立つので,
		\begin{align}
			\sigma \defeq 
			\varepsilon s\, \forall u\, (\, \varphi(u) \lrarrow u \in s\, )
		\end{align}
		とおけば存在記号の推論規則より
		\begin{align}
			\Set{y}{\varphi(y)} \in b,\ \ELEAX \vdash \Set{y}{\varphi(y)} = \sigma
			\label{fom:equivalent_formula_rewriting_9_1}
		\end{align}
		となる.ここで相等性公理より
		\begin{align}
			\EQAX \vdash \Set{y}{\varphi(y)} = \sigma
			\rarrow (\, \Set{y}{\varphi(y)} \in b \rarrow \sigma \in b\, )
		\end{align}
		が成り立つので,(\refeq{fom:equivalent_formula_rewriting_9_1})と三段論法より
		\begin{align}
			\Set{y}{\varphi(y)} \in b,\ \EQAX,\ELEAX \vdash \sigma \in b
			\label{fom:equivalent_formula_rewriting_9_2}
		\end{align}
		が得られる.他方で定理\ref{thm:equivalent_formula_rewriting_3}より
		\begin{align}
			\EQAX,\COMAX \vdash \Set{y}{\varphi(y)} = \sigma
			\rarrow \forall u\, (\, \varphi(u) \lrarrow u \in \sigma\, )
		\end{align}
		が成り立つので,(\refeq{fom:equivalent_formula_rewriting_9_1})と三段論法より
		\begin{align}
			\Set{y}{\varphi(y)} \in b,\ \EQAX,\COMAX,\ELEAX \vdash
			\forall u\, (\, \varphi(u) \lrarrow u \in \sigma\, )
			\label{fom:equivalent_formula_rewriting_9_3}
		\end{align}
		も得られる.(\refeq{fom:equivalent_formula_rewriting_9_2})と
		(\refeq{fom:equivalent_formula_rewriting_9_3})と論理積の導入規則より
		\begin{align}
			\Set{y}{\varphi(y)} \in b,\ \EQAX,\COMAX,\ELEAX \vdash
			\forall u\, (\, \varphi(u) \lrarrow u \in \sigma\, ) \wedge \sigma \in b
		\end{align}
		が成り立つので,存在記号の推論規則より
		\begin{align}
			\Set{y}{\varphi(y)} \in b,\ \EQAX,\COMAX,\ELEAX \vdash
			\exists s\, (\, \forall u\, (\, \varphi(u) \lrarrow u \in s\, ) \wedge s \in b\, )
		\end{align}
		が得られる.
		\QED
	\end{sketch}
	
	\begin{screen}
		\begin{thm}
		\label{thm:equivalent_formula_rewriting_11}
			$\varphi$と$\psi$を$\lang{\varepsilon}$の式とし,
			$y$を$\varphi$に自由に現れる変項とし,
			$z$を$\psi$に自由に現れる変項とし,
			$\varphi$に自由に現れる変項は$y$のみであるとし,
			$\psi$に自由に現れる変項は$z$のみであるとし,する.このとき
			\begin{align}
				\EQAX,\COMAX,\ELEAX \vdash \Set{y}{\varphi(y)} \in \Set{z}{\psi(z)}
				\rarrow \exists s\, (\, 
				\forall u\, (\, \varphi(u) \lrarrow u \in s\, )
				\wedge \psi(s)\, ).
			\end{align}
		\end{thm}
	\end{screen}
	
	\begin{sketch}
		まず(\refeq{fom:equivalent_formula_rewriting_9_1})と
		(\refeq{fom:equivalent_formula_rewriting_9_3})と同様に,
		\begin{align}
			\sigma \defeq 
			\varepsilon s\, \forall u\, (\, \varphi(u) \lrarrow u \in s\, )
		\end{align}
		とおけば
		\begin{align}
			\Set{y}{\varphi(y)} \in \Set{z}{\psi(z)},\ \ELEAX \vdash 
			\Set{y}{\varphi(y)} = \sigma
			\label{fom:equivalent_formula_rewriting_11_1}
		\end{align}
		と
		\begin{align}
			\Set{y}{\varphi(y)} \in \Set{z}{\psi(z)},\ \EQAX,\COMAX,\ELEAX \vdash
			\forall u\, (\, \varphi(u) \lrarrow u \in \sigma\, )
			\label{fom:equivalent_formula_rewriting_11_2}
		\end{align}
		が成り立つ.また相等性公理より
		\begin{align}
			\EQAX \vdash \Set{y}{\varphi(y)} = \sigma
			\rarrow (\, \Set{y}{\varphi(y)} \in \Set{z}{\psi(z)}
			\rarrow \sigma \in \Set{z}{\psi(z)}\, )
		\end{align}
		となるので,(\refeq{fom:equivalent_formula_rewriting_11_1})との三段論法より
		\begin{align}
			\Set{y}{\varphi(y)} \in \Set{z}{\psi(z)},\ \EQAX,\ELEAX \vdash 
			\sigma \in \Set{z}{\psi(z)}
		\end{align}
		が成り立ち,内包性公理より
		\begin{align}
			\COMAX \vdash \sigma \in \Set{z}{\psi(z)} \rarrow \psi(\sigma)
		\end{align}
		が成り立つので
		\begin{align}
			\Set{y}{\varphi(y)} \in \Set{z}{\psi(z)},\ \EQAX,\ELEAX \vdash 
			\psi(\sigma)
			\label{fom:equivalent_formula_rewriting_11_3}
		\end{align}
		が得られる.(\refeq{fom:equivalent_formula_rewriting_11_2})と
		(\refeq{fom:equivalent_formula_rewriting_11_3})と論理積の導入規則より
		\begin{align}
			\Set{y}{\varphi(y)} \in \Set{z}{\psi(z)},\ \EQAX,\COMAX,\ELEAX \vdash
			\forall u\, (\, \varphi(u) \lrarrow u \in \sigma\, ) \wedge \psi(\sigma)
		\end{align}
		が成り立ち,存在記号の推論規則より
		\begin{align}
			\Set{y}{\varphi(y)} \in \Set{z}{\psi(z)},\ \EQAX,\COMAX,\ELEAX \vdash
			\exists s\, (\forall u\, (\, \varphi(u) \lrarrow u \in x\, ) \wedge \psi(x)\, )
		\end{align}
		が得られる.
		\QED
	\end{sketch}
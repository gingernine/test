\section{書き換えの同値性}
	
	$\mathcal{L}$の式を$\lang{\varepsilon}$の式に変換するときは,まず原子式に対して
	\begin{table}[h]
		\begin{center}
		\begin{tabular}{c|c}
			元の式 & 書き換え後 \\ \hline \hline
			$a = \Set{z}{\psi}$ & $\forall v\, (\, v \in a \lrarrow \psi(v/z)\, )$ \\ \hline
			$\Set{y}{\varphi} = b$ & $\forall u\, (\, \varphi(u/y) \lrarrow u \in b\, )$ \\ \hline
			$\Set{y}{\varphi} = \Set{z}{\psi}$ & $\forall u\, (\, \varphi(u/y) \lrarrow \psi(u/z)\, )$ \\ \hline
			$a \in \Set{z}{\psi}$ & $\psi(a/z)$ \\ \hline
			$\Set{y}{\varphi} \in b$ & $\exists s\, (\, \forall u\, (\, \varphi(u/y) \lrarrow u \in s\, ) \wedge s \in b\, )$ \\ \hline
			$\Set{y}{\varphi} \in \Set{z}{\psi}$ & $\exists s\, (\, \forall u\, (\, \varphi(u/y) \lrarrow u \in s\, ) \wedge \psi(s/z)\, )$ \\ \hline
		\end{tabular}
		\end{center}
	\end{table}
	
	とし,あとは帰納的に式を書き換えることによって$\mathcal{L}$の式$\varphi$から
	$\lang{\varepsilon}$の式$\hat{\varphi}$を得たのであった.
	この節では$\varphi$が文であるときに
	\begin{align}
		\varphi \lrarrow \hat{\varphi}
	\end{align}
	が成り立つことを示す.
	これが示されれば,仮に$\varphi$に変項$x$が自由に現れていても
	\begin{align}
		\forall x\, (\, \varphi(x) \lrarrow \hat{\varphi}(x)\, )
	\end{align}
	が成り立つし,$x$に加えて$y$が自由に現れていても
	\begin{align}
		\forall x\, \forall y\, (\, \varphi(x,y) \lrarrow \hat{\varphi}(x,y)\, )
	\end{align}
	が成り立つ.前者の場合は
	\begin{align}
		\tau \defeq \varepsilon x \negation (\, \hat{\varphi}(x) \lrarrow \hat{\varphi}(x)\, )
	\end{align}
	に対して
	\begin{align}
		\varphi(\tau) \lrarrow \hat{\varphi}(\tau)
	\end{align}
	が成り立つのだし,後者の場合は
	\begin{align}
		\sigma &\defeq \varepsilon x \negation \forall y\, (\, \hat{\varphi}(x,y) \lrarrow \hat{\varphi}(x,y)\, ), \\
		\rho &\defeq \varepsilon y \negation (\, \hat{\varphi}(\sigma,y) \lrarrow \hat{\varphi}(\sigma,y)\, )
	\end{align}
	に対して
	\begin{align}
		\varphi(\sigma,\rho) \lrarrow \hat{\varphi}(\sigma,\rho)
	\end{align}
	が成り立つので,全称記号の推論規則を適用すればいい.
	
	\begin{screen}
		\begin{logicalthm}[同値記号の対称律]
		\label{logicalthm:symmetry_of_equivalence_arrows}
			$A,B$を$\mathcal{L}$の文とするとき
			\begin{align}
				\vdash (A \lrarrow B) \rarrow (B \lrarrow A).
			\end{align}
		\end{logicalthm}
	\end{screen}
	
	\begin{prf}
		$\wedge$の除去(推論規則\ref{logicalaxm:elimination_of_conjunction})より
		\begin{align}
			A \lrarrow B &\vdash A \rarrow B, \\
			A \lrarrow B &\vdash B \rarrow A
		\end{align}
		となる.他方で論理積の導入(推論規則\ref{logicalaxm:introduction_of_conjunction})より
		\begin{align}
			\vdash (B \rarrow A) \rarrow ((A \rarrow B) \rarrow 
			(B \rarrow A) \wedge (A \rarrow B))
		\end{align}
		が成り立つので,三段論法を二回適用すれば
		\begin{align}
			A \lrarrow B \vdash (B \rarrow A) \wedge (A \rarrow B)
		\end{align}
		となる.つまり
		\begin{align}
			A \lrarrow B \vdash B \lrarrow A
		\end{align}
		が得られた.
		\QED
	\end{prf}
	
	\begin{screen}
		\begin{thm}
		\label{thm:equivalent_formula_rewriting_1}
			$a$を主要$\varepsilon$項とし,$\psi$を$\lang{\varepsilon}$の式とし,
			$z$を$\psi$に自由に現れる変項とし,$\psi$に自由に現れる変項は$z$のみであるとする.このとき
			\begin{align}
				\EQAX,\COMAX \vdash a = \Set{z}{\psi(z)} 
				\rarrow \forall v\, (\, v \in a \lrarrow \psi(v)\, ).
			\end{align}
		\end{thm}
	\end{screen}
	
	\begin{sketch}
		いま
		\begin{align}
			\tau \defeq \varepsilon v \negation (\, v \in a \lrarrow \psi(v)\, )
		\end{align}
		とおく.外延性公理の逆(定理\ref{thm:inverse_of_axiom_of_extensionality})より
		\begin{align}
			a = \Set{z}{\psi(z)},\ \EQAX \vdash 
			\tau \in a \lrarrow \tau \in \Set{z}{\psi(z)}
		\end{align}
		が成り立ち,他方で内包性公理より
		\begin{align}
			\COMAX \vdash \tau \in \Set{z}{\psi(z)} \lrarrow \psi(\tau)
		\end{align}
		が成り立つので,同値記号の推移律
		(推論法則\ref{logicalthm:transitive_law_of_equivalence_symbol})より
		\begin{align}
			a = \Set{z}{\psi(z)},\ \EQAX,\COMAX \vdash \tau \in a \lrarrow \psi(\tau)
		\end{align}
		が従う.そして全称記号の推論規則より
		\begin{align}
			a = \Set{z}{\psi(z)},\ \EQAX,\COMAX \vdash 
			\forall v\, (\, v \in a \lrarrow \psi(v)\, )
		\end{align}
		が得られる.
		\QED
	\end{sketch}
	
	\begin{screen}
		\begin{thm}
		\label{thm:equivalent_formula_rewriting_2}
			$a$を主要$\varepsilon$項とし,$\psi$を$\lang{\varepsilon}$の式とし,
			$z$を$\psi$に自由に現れる変項とし,$\psi$に自由に現れる変項は$z$のみであるとする.このとき
			\begin{align}
				\EXTAX,\COMAX \vdash \forall v\, (\, v \in a \lrarrow \psi(v)\, )
				\rarrow a = \Set{z}{\psi(z)}.
			\end{align}
		\end{thm}
	\end{screen}
	
	\begin{sketch}
		いま
		\begin{align}
			\tau \defeq 
			\varepsilon x \negation (\, x \in a \lrarrow x \in \Set{z}{\psi(z)}\, )
		\end{align}
		とおく.まず全称記号の推論規則より
		\begin{align}
			\forall v\, (\, v \in a \lrarrow \psi(v)\, )
			\vdash \tau \in a \lrarrow \psi(\tau)
			\label{fom:equivalent_formula_rewriting_2_1}
		\end{align}
		が成り立つ.また内包性公理より
		\begin{align}
			\COMAX \vdash \tau \in \Set{z}{\psi(z)} \lrarrow \psi(\tau)
		\end{align}
		となるので,同値記号の対称律(\ref{logicalthm:symmetry_of_equivalence_arrows})より
		\begin{align}
			\COMAX \vdash \psi(\tau) \lrarrow \tau \in \Set{z}{\psi(z)}
			\label{fom:equivalent_formula_rewriting_2_2}
		\end{align}
		が成り立つ.(\refeq{fom:equivalent_formula_rewriting_2_1})と
		(\refeq{fom:equivalent_formula_rewriting_2_2})と同値記号の推移律
		(推論法則\ref{logicalthm:transitive_law_of_equivalence_symbol})より
		\begin{align}
			\forall v\, (\, v \in a \lrarrow \psi(v)\, ),\ \COMAX \vdash
			\tau \in a \lrarrow \tau \in \Set{z}{\psi(z)}
		\end{align}
		となり,全称記号の推論規則より
		\begin{align}
			\forall v\, (\, v \in a \lrarrow \psi(v)\, ),\ \COMAX \vdash
			\forall x\, (\, x \in a \lrarrow x \in \Set{z}{\psi(z)}\, )
		\end{align}
		となり,外延性公理より
		\begin{align}
			\forall v\, (\, v \in a \lrarrow \psi(v)\, ),\ \EXTAX,\COMAX \vdash
			a = \Set{z}{\psi(z)}
		\end{align}
		が得られる.
		\QED
	\end{sketch}
	
	\begin{screen}
		\begin{thm}
		\label{thm:equivalent_formula_rewriting_3}
			$b$を主要$\varepsilon$項とし,$\varphi$を$\lang{\varepsilon}$の式とし,
			$y$を$\varphi$に自由に現れる変項とし,$\varphi$に自由に現れる変項は$y$のみ
			であるとする.このとき
			\begin{align}
				\EQAX,\COMAX \vdash \Set{y}{\varphi(y)} = b 
				\rarrow \forall u\, (\, \varphi(u) \lrarrow u \in b\, ).
			\end{align}
		\end{thm}
	\end{screen}
	
	\begin{sketch}
		いま
		\begin{align}
			\tau \defeq \varepsilon u \negation (\, \varphi(u) \lrarrow u \in b\, )
		\end{align}
		とおけば,まず外延性公理の逆(定理\ref{thm:inverse_of_axiom_of_extensionality})より
		\begin{align}
			\Set{y}{\varphi(y)} = b,\ \EQAX \vdash 
			\tau \in \Set{y}{\varphi(z)} \lrarrow \tau \in b
			\label{fom:equivalent_formula_rewriting_3_1}
		\end{align}
		が成り立つ.他方で内包性公理より
		\begin{align}
			\COMAX \vdash \tau \in \Set{y}{\varphi(y)} \lrarrow \varphi(\tau)
		\end{align}
		となり,同値記号の対称律(\ref{logicalthm:symmetry_of_equivalence_arrows})より
		\begin{align}
			\COMAX \vdash \varphi(\tau) \lrarrow \tau \in \Set{y}{\varphi(y)}
			\label{fom:equivalent_formula_rewriting_3_2}
		\end{align}
		が成り立つ.(\refeq{fom:equivalent_formula_rewriting_3_1})と
		(\refeq{fom:equivalent_formula_rewriting_3_2})と同値記号の推移律
		(推論法則\ref{logicalthm:transitive_law_of_equivalence_symbol})より
		\begin{align}
			\Set{y}{\varphi(y)} = b,\ \EQAX,\COMAX \vdash 
			\varphi(\tau) \lrarrow \tau \in b 
		\end{align}
		が成り立ち,全称記号の推論規則より
		\begin{align}
			\Set{y}{\varphi(y)} = b,\ \EQAX,\COMAX \vdash 
			\forall u\, (\, \varphi(u) \lrarrow u \in b\, )
		\end{align}
		が得られる.
		\QED
	\end{sketch}
	
	\begin{screen}
		\begin{thm}
		\label{thm:equivalent_formula_rewriting_4}
			$b$を主要$\varepsilon$項とし,$\varphi$を$\lang{\varepsilon}$の式とし,
			$y$を$\varphi$に自由に現れる変項とし,$\varphi$に自由に現れる変項は$y$のみ
			であるとする.このとき
			\begin{align}
				\EXTAX,\COMAX \vdash \forall u\, (\, \varphi(u) \lrarrow u \in b\, )
				\rarrow \Set{y}{\varphi(y)} = b.
			\end{align}
		\end{thm}
	\end{screen}
	
	\begin{sketch}
		いま
		\begin{align}
			\tau \defeq 
			\varepsilon x \negation (\, x \in \Set{y}{\varphi(y)} \lrarrow x \in b\, )
		\end{align}
		とおく.まず全称記号の推論規則より
		\begin{align}
			\forall u\, (\, \varphi(u) \lrarrow u \in b\, )
			\vdash \varphi(\tau) \lrarrow \tau \in b
			\label{fom:equivalent_formula_rewriting_4_1}
		\end{align}
		が成り立ち,また内包性公理より
		\begin{align}
			\COMAX \vdash \tau \in \Set{y}{\varphi(y)} \lrarrow \varphi(\tau)
			\label{fom:equivalent_formula_rewriting_4_2}
		\end{align}
		が成り立つので,(\refeq{fom:equivalent_formula_rewriting_4_1})と
		(\refeq{fom:equivalent_formula_rewriting_4_2})と同値記号の推移律
		(推論法則\ref{logicalthm:transitive_law_of_equivalence_symbol})より
		\begin{align}
			\forall u\, (\, \varphi(u) \lrarrow u \in b\, ),\ \COMAX \vdash
			\tau \in \Set{y}{\varphi(y)} \lrarrow \tau \in b
		\end{align}
		が成り立つ.全称記号の推論規則より
		\begin{align}
			\forall u\, (\, \varphi(u) \lrarrow u \in b\, ),\ \COMAX \vdash
			\forall x\, (\, x \in \Set{y}{\varphi(y)} \lrarrow x \in b\, )
		\end{align}
		となり,外延性公理より
		\begin{align}
			\forall u\, (\, \varphi(u) \lrarrow u \in b\, ),\ \EXTAX,\COMAX \vdash
			\Set{y}{\varphi(y)} = b
		\end{align}
		が得られる.
		\QED
	\end{sketch}
	
	\begin{screen}
		\begin{thm}
		\label{thm:equivalent_formula_rewriting_5}
			$\varphi$と$\psi$を$\lang{\varepsilon}$の式とし,
			$y$を$\varphi$に自由に現れる変項とし,
			$z$を$\psi$に自由に現れる変項とし,
			$\varphi$に自由に現れる変項は$y$のみであるとし,
			$\psi$に自由に現れる変項は$z$のみであるとし,する.このとき
			\begin{align}
				\EQAX,\COMAX \vdash \Set{y}{\varphi(y)} = \Set{z}{\psi(z)}
				\rarrow \forall u\, (\, \varphi(u) \lrarrow \psi(u)\, ).
			\end{align}
		\end{thm}
	\end{screen}
	
	\begin{sketch}
		いま
		\begin{align}
			\tau \defeq \varepsilon u \negation (\, \varphi(u) \lrarrow \psi(u)\, )
		\end{align}
		とおけば,まず外延性公理の逆(定理\ref{thm:inverse_of_axiom_of_extensionality})より
		\begin{align}
			\Set{y}{\varphi(y)} = \Set{z}{\psi(z)},\ \EQAX \vdash 
			\tau \in \Set{y}{\varphi(z)} \lrarrow \tau \in \Set{z}{\psi(z)}
			\label{fom:equivalent_formula_rewriting_5_1}
		\end{align}
		が成り立つ.また内包性公理より
		\begin{align}
			\COMAX &\vdash \tau \in \Set{y}{\varphi(y)} \lrarrow \varphi(\tau), 
			\label{fom:equivalent_formula_rewriting_5_2} \\
			\COMAX &\vdash \tau \in \Set{z}{\varphi(z)} \lrarrow \psi(\tau)
			\label{fom:equivalent_formula_rewriting_5_3}
		\end{align}
		が成り立つが,(\refeq{fom:equivalent_formula_rewriting_5_2})と
		同値記号の対称律(\ref{logicalthm:symmetry_of_equivalence_arrows})より
		\begin{align}
			\COMAX \vdash \varphi(\tau) \lrarrow \tau \in \Set{y}{\varphi(y)}
			\label{fom:equivalent_formula_rewriting_5_4}
		\end{align}
		も成り立つ.(\refeq{fom:equivalent_formula_rewriting_5_4})と
		(\refeq{fom:equivalent_formula_rewriting_5_1})と同値記号の推移律
		(推論法則\ref{logicalthm:transitive_law_of_equivalence_symbol})より
		\begin{align}
			\Set{y}{\varphi(y)} = \Set{z}{\psi(z)},\ \EQAX,\COMAX \vdash
			 \varphi(\tau) \lrarrow \tau \in \Set{z}{\psi(z)}
			\label{fom:equivalent_formula_rewriting_5_5}
		\end{align}
		が従い,(\refeq{fom:equivalent_formula_rewriting_5_5})と
		(\refeq{fom:equivalent_formula_rewriting_5_3})と同値記号の推移律より
		\begin{align}
			\Set{y}{\varphi(y)} = \Set{z}{\psi(z)},\ \EQAX,\COMAX \vdash
			\varphi(\tau) \lrarrow \psi(\tau)
		\end{align}
		が従う.そして全称記号の推論規則より
		\begin{align}
			\Set{y}{\varphi(y)} = \Set{z}{\psi(z)},\ \EQAX,\COMAX \vdash
			\forall u\, (\, \varphi(u) \lrarrow \psi(u)\, )
		\end{align}
		が得られる.
		\QED
	\end{sketch}
	
	\begin{screen}
		\begin{thm}
		\label{thm:equivalent_formula_rewriting_6}
			$\varphi$と$\psi$を$\lang{\varepsilon}$の式とし,
			$y$を$\varphi$に自由に現れる変項とし,
			$z$を$\psi$に自由に現れる変項とし,
			$\varphi$に自由に現れる変項は$y$のみであるとし,
			$\psi$に自由に現れる変項は$z$のみであるとし,する.このとき
			\begin{align}
				\EXTAX,\COMAX \vdash \forall u\, (\, \varphi(u) \lrarrow \psi(u)\, )
				\lrarrow \Set{y}{\varphi(y)} = \Set{z}{\psi(z)}.
			\end{align}
		\end{thm}
	\end{screen}
	
	\begin{sketch}
		いま
		\begin{align}
			\tau \defeq \varepsilon x \negation (\, x \in \Set{y}{\varphi(y)} \lrarrow x \in \Set{z}{\psi(z)}\, )
		\end{align}
		とおく.まず全称記号の推論規則より
		\begin{align}
			\forall u\, (\, \varphi(u) \lrarrow \psi(u)\, )
			\vdash \varphi(\tau) \lrarrow \psi(\tau)
			\label{fom:equivalent_formula_rewriting_6_1}
		\end{align}
		が成り立つ.また内包性公理より
		\begin{align}
			\COMAX &\vdash \tau \in \Set{y}{\varphi(y)} \lrarrow \varphi(\tau), 
			\label{fom:equivalent_formula_rewriting_6_2} \\
			\COMAX &\vdash \tau \in \Set{z}{\psi(z)} \lrarrow \psi(\tau)
		\end{align}
		となり,同値記号の対称律(\ref{logicalthm:symmetry_of_equivalence_arrows})より
		\begin{align}
			\COMAX \vdash \psi(\tau) \lrarrow \tau \in \Set{z}{\psi(z)}
			\label{fom:equivalent_formula_rewriting_6_3}
		\end{align}
		も成り立つ.(\refeq{fom:equivalent_formula_rewriting_6_1})と
		(\refeq{fom:equivalent_formula_rewriting_6_2})と同値記号の推移律
		(推論法則\ref{logicalthm:transitive_law_of_equivalence_symbol})より
		\begin{align}
			\forall u\, (\, \varphi(u) \lrarrow \psi(u)\, ),\ \COMAX \vdash
			\tau \in \Set{y}{\varphi(y)} \lrarrow \psi(\tau)
			\label{fom:equivalent_formula_rewriting_6_4}
		\end{align}
		となり,(\refeq{fom:equivalent_formula_rewriting_6_3})と
		(\refeq{fom:equivalent_formula_rewriting_6_4})と同値記号の推移律より
		\begin{align}
			\forall u\, (\, \varphi(u) \lrarrow \psi(u)\, ),\ \COMAX \vdash
			\tau \in \Set{y}{\varphi(y)} \lrarrow \tau \in \Set{z}{\psi(z)}
		\end{align}
		となり,全称記号の推論規則より
		\begin{align}
			\forall u\, (\, \varphi(u) \lrarrow \psi(u)\, ),\ \COMAX \vdash
			\forall x\, (\, x \in \Set{y}{\varphi(y)} \lrarrow x \in \Set{z}{\psi(z)}\, )
		\end{align}
		となり,外延性公理より
		\begin{align}
			\forall u\, (\, \varphi(u) \lrarrow \psi(u)\, ),\ \EXTAX,\COMAX \vdash
			\Set{y}{\varphi(y)} = \Set{z}{\psi(z)}
		\end{align}
		が得られる.
		\QED
	\end{sketch}
	
	\begin{screen}
		\begin{thm}
		\label{thm:equivalent_formula_rewriting_7}
			$a$を主要$\varepsilon$項とし,$\psi$を$\lang{\varepsilon}$の式とし,
			$z$を$\psi$に自由に現れる変項とし,$\psi$に自由に現れる変項は$z$のみであるとする.このとき
			\begin{align}
				\COMAX \vdash a \in \Set{z}{\psi(z)} \rarrow \psi(a).
			\end{align}
		\end{thm}
	\end{screen}
	
	\begin{sketch}
		$a$は主要$\varepsilon$項であるから,内包性公理より
		\begin{align}
			\COMAX \vdash a \in \Set{z}{\psi(z)} \rarrow \psi(a)
		\end{align}
		が成り立つ.
		\QED
	\end{sketch}
	
	\begin{screen}
		\begin{thm}
		\label{thm:equivalent_formula_rewriting_8}
			$a$を主要$\varepsilon$項とし,$\psi$を$\lang{\varepsilon}$の式とし,
			$z$を$\psi$に自由に現れる変項とし,$\psi$に自由に現れる変項は$z$のみであるとする.このとき
			\begin{align}
				\COMAX \vdash \psi(a) \rarrow a \in \Set{z}{\psi(z)}.
			\end{align}
		\end{thm}
	\end{screen}
	
	\begin{sketch}
		$a$は主要$\varepsilon$項であるから,内包性公理より
		\begin{align}
			\COMAX \vdash \psi(a) \rarrow a \in \Set{z}{\psi(z)}
		\end{align}
		が成り立つ.
		\QED
	\end{sketch}
	
	\begin{screen}
		\begin{thm}
		\label{thm:equivalent_formula_rewriting_9}
			$b$を主要$\varepsilon$項とし,$\varphi$を$\lang{\varepsilon}$の式とし,
			$y$を$\varphi$に自由に現れる変項とし,
			$\varphi$に自由に現れる変項は$y$のみであるとする.このとき
			\begin{align}
				\EQAX,\COMAX,\ELEAX \vdash \Set{y}{\varphi(y)} \in b
				\rarrow \exists s\, (\, 
				\forall u\, (\, \varphi(u) \lrarrow u \in s\, )
				\wedge s \in b\, ).
			\end{align}
		\end{thm}
	\end{screen}
	
	\begin{sketch}
		要素の公理より
		\begin{align}
			\Set{y}{\varphi(y)} \in b,\ \ELEAX \vdash 
			\exists s\, (\, \Set{y}{\varphi(y)} = s\, )
		\end{align}
		が成り立つので,
		\begin{align}
			\sigma \defeq 
			\varepsilon s\, \forall u\, (\, \varphi(u) \lrarrow u \in s\, )
		\end{align}
		とおけば存在記号の推論規則より
		\begin{align}
			\Set{y}{\varphi(y)} \in b,\ \ELEAX \vdash \Set{y}{\varphi(y)} = \sigma
			\label{fom:equivalent_formula_rewriting_9_1}
		\end{align}
		となる.ここで相等性公理より
		\begin{align}
			\EQAX \vdash \Set{y}{\varphi(y)} = \sigma
			\rarrow (\, \Set{y}{\varphi(y)} \in b \rarrow \sigma \in b\, )
		\end{align}
		が成り立つので,(\refeq{fom:equivalent_formula_rewriting_9_1})と三段論法より
		\begin{align}
			\Set{y}{\varphi(y)} \in b,\ \EQAX,\ELEAX \vdash \sigma \in b
			\label{fom:equivalent_formula_rewriting_9_2}
		\end{align}
		が得られる.他方で定理\ref{thm:equivalent_formula_rewriting_3}より
		\begin{align}
			\EQAX,\COMAX \vdash \Set{y}{\varphi(y)} = \sigma
			\rarrow \forall u\, (\, \varphi(u) \lrarrow u \in \sigma\, )
		\end{align}
		が成り立つので,(\refeq{fom:equivalent_formula_rewriting_9_1})と三段論法より
		\begin{align}
			\Set{y}{\varphi(y)} \in b,\ \EQAX,\COMAX,\ELEAX \vdash
			\forall u\, (\, \varphi(u) \lrarrow u \in \sigma\, )
			\label{fom:equivalent_formula_rewriting_9_3}
		\end{align}
		も得られる.(\refeq{fom:equivalent_formula_rewriting_9_2})と
		(\refeq{fom:equivalent_formula_rewriting_9_3})と論理積の導入規則より
		\begin{align}
			\Set{y}{\varphi(y)} \in b,\ \EQAX,\COMAX,\ELEAX \vdash
			\forall u\, (\, \varphi(u) \lrarrow u \in \sigma\, ) \wedge \sigma \in b
		\end{align}
		が成り立つので,存在記号の推論規則より
		\begin{align}
			\Set{y}{\varphi(y)} \in b,\ \EQAX,\COMAX,\ELEAX \vdash
			\exists s\, (\, \forall u\, (\, \varphi(u) \lrarrow u \in s\, ) \wedge s \in b\, )
		\end{align}
		が得られる.
		\QED
	\end{sketch}
	
	\begin{screen}
		\begin{thm}
		\label{thm:equivalent_formula_rewriting_10}
			$b$を主要$\varepsilon$項とし,$\varphi$を$\lang{\varepsilon}$の式とし,
			$y$を$\varphi$に自由に現れる変項とし,
			$\varphi$に自由に現れる変項は$y$のみであるとする.このとき
			\begin{align}
				\EXTAX,\EQAX,\COMAX \vdash \exists s\, (\, \forall u\, (\, \varphi(u) \lrarrow u \in s\, ) \wedge s \in b\, ) \rarrow \Set{y}{\varphi(y)} \in b.
			\end{align}
		\end{thm}
	\end{screen}
	
	\begin{sketch}
		いま
		\begin{align}
			\sigma \defeq \varepsilon s\, (\, \forall u\, (\, \varphi(u) \lrarrow u \in s\, ) \wedge s \in b\, )
		\end{align}
		とおけば,存在記号の推論規則と論理積の除去より
		\begin{align}
			\exists s\, (\, \forall u\, (\, \varphi(u) \lrarrow u \in s\, ) \wedge s \in b\, )
			&\vdash \forall u\, (\, \varphi(u) \lrarrow u \in \sigma\, ), 
			\label{fom:equivalent_formula_rewriting_10_1} \\
			\exists s\, (\, \forall u\, (\, \varphi(u) \lrarrow u \in s\, ) \wedge s \in b\, )
			&\vdash \sigma \in b
			\label{fom:equivalent_formula_rewriting_10_2}
		\end{align}
		が成り立つ.ここで定理\ref{thm:equivalent_formula_rewriting_4}より
		\begin{align}
			\EXTAX,\COMAX \vdash \forall u\, (\, \varphi(u) \lrarrow u \in \sigma\, )
			\rarrow \Set{y}{\varphi(y)} = \sigma
		\end{align}
		が成り立つので,(\refeq{fom:equivalent_formula_rewriting_10_1})との三段論法より
		\begin{align}
			\exists s\, (\, \forall u\, (\, \varphi(u) \lrarrow u \in s\, ) \wedge s \in b\, ),\ \EXTAX,\COMAX \vdash \Set{y}{\varphi(y)} = \sigma
			\label{fom:equivalent_formula_rewriting_10_3}
		\end{align}
		が得られる.また相等性公理より
		\begin{align}
			\EQAX &\vdash \Set{y}{\varphi(y)} = \sigma \rarrow \sigma = \Set{y}{\varphi(y)}, \\
			\EQAX &\vdash \sigma = \Set{y}{\varphi(y)} \rarrow
			(\, \sigma \in b \rarrow \Set{y}{\varphi(y)} \in b\, )
		\end{align}
		が成り立つので,(\refeq{fom:equivalent_formula_rewriting_10_2})と
		(\refeq{fom:equivalent_formula_rewriting_10_3})との三段論法より
		\begin{align}
			\exists s\, (\, \forall u\, (\, \varphi(u) \lrarrow u \in s\, ) \wedge s \in b\, ),\ \EXTAX,\EQAX,\COMAX \vdash \Set{y}{\varphi(y)} \in b
		\end{align}
		が従う.
		\QED
	\end{sketch}
	
	\begin{screen}
		\begin{thm}
		\label{thm:equivalent_formula_rewriting_11}
			$\varphi$と$\psi$を$\lang{\varepsilon}$の式とし,
			$y$を$\varphi$に自由に現れる変項とし,
			$z$を$\psi$に自由に現れる変項とし,
			$\varphi$に自由に現れる変項は$y$のみであるとし,
			$\psi$に自由に現れる変項は$z$のみであるとし,する.このとき
			\begin{align}
				\EQAX,\COMAX,\ELEAX \vdash \Set{y}{\varphi(y)} \in \Set{z}{\psi(z)}
				\rarrow \exists s\, (\, 
				\forall u\, (\, \varphi(u) \lrarrow u \in s\, )
				\wedge \psi(s)\, ).
			\end{align}
		\end{thm}
	\end{screen}
	
	\begin{sketch}
		まず(\refeq{fom:equivalent_formula_rewriting_9_1})と
		(\refeq{fom:equivalent_formula_rewriting_9_3})と同様に,
		\begin{align}
			\sigma \defeq 
			\varepsilon s\, \forall u\, (\, \varphi(u) \lrarrow u \in s\, )
		\end{align}
		とおけば
		\begin{align}
			\Set{y}{\varphi(y)} \in \Set{z}{\psi(z)},\ \ELEAX \vdash 
			\Set{y}{\varphi(y)} = \sigma
			\label{fom:equivalent_formula_rewriting_11_1}
		\end{align}
		と
		\begin{align}
			\Set{y}{\varphi(y)} \in \Set{z}{\psi(z)},\ \EQAX,\COMAX,\ELEAX \vdash
			\forall u\, (\, \varphi(u) \lrarrow u \in \sigma\, )
			\label{fom:equivalent_formula_rewriting_11_2}
		\end{align}
		が成り立つ.また相等性公理より
		\begin{align}
			\EQAX \vdash \Set{y}{\varphi(y)} = \sigma
			\rarrow (\, \Set{y}{\varphi(y)} \in \Set{z}{\psi(z)}
			\rarrow \sigma \in \Set{z}{\psi(z)}\, )
		\end{align}
		となるので,(\refeq{fom:equivalent_formula_rewriting_11_1})との三段論法より
		\begin{align}
			\Set{y}{\varphi(y)} \in \Set{z}{\psi(z)},\ \EQAX,\ELEAX \vdash 
			\sigma \in \Set{z}{\psi(z)}
		\end{align}
		が成り立ち,内包性公理より
		\begin{align}
			\COMAX \vdash \sigma \in \Set{z}{\psi(z)} \rarrow \psi(\sigma)
		\end{align}
		が成り立つので
		\begin{align}
			\Set{y}{\varphi(y)} \in \Set{z}{\psi(z)},\ \EQAX,\ELEAX \vdash 
			\psi(\sigma)
			\label{fom:equivalent_formula_rewriting_11_3}
		\end{align}
		が得られる.(\refeq{fom:equivalent_formula_rewriting_11_2})と
		(\refeq{fom:equivalent_formula_rewriting_11_3})と論理積の導入規則より
		\begin{align}
			\Set{y}{\varphi(y)} \in \Set{z}{\psi(z)},\ \EQAX,\COMAX,\ELEAX \vdash
			\forall u\, (\, \varphi(u) \lrarrow u \in \sigma\, ) \wedge \psi(\sigma)
		\end{align}
		が成り立ち,存在記号の推論規則より
		\begin{align}
			\Set{y}{\varphi(y)} \in \Set{z}{\psi(z)},\ \EQAX,\COMAX,\ELEAX \vdash
			\exists s\, (\forall u\, (\, \varphi(u) \lrarrow u \in x\, ) \wedge \psi(x)\, )
		\end{align}
		が得られる.
		\QED
	\end{sketch}
	
	\begin{screen}
		\begin{thm}
		\label{thm:equivalent_formula_rewriting_12}
			$\varphi$と$\psi$を$\lang{\varepsilon}$の式とし,
			$y$を$\varphi$に自由に現れる変項とし,
			$z$を$\psi$に自由に現れる変項とし,
			$\varphi$に自由に現れる変項は$y$のみであるとし,
			$\psi$に自由に現れる変項は$z$のみであるとし,する.このとき
			\begin{align}
				\EXTAX,\EQAX,\COMAX \vdash \exists s\, (\, \forall u\, (\, \varphi(u) \lrarrow u \in s\, ) \wedge \psi(s)\, ) \rarrow \Set{y}{\varphi(y)} \in \Set{z}{\psi(z)}.
			\end{align}
		\end{thm}
	\end{screen}
	
	\begin{sketch}
		いま
		\begin{align}
			\sigma \defeq \varepsilon s\, (\, \forall u\, (\, \varphi(u) \lrarrow u \in s\, ) \wedge \psi(s)\, )
		\end{align}
		とおけば,存在記号の推論規則と論理積の除去より
		\begin{align}
			\exists s\, (\, \forall u\, (\, \varphi(u) \lrarrow u \in s\, ) \wedge \psi(s)\, )
			&\vdash \forall u\, (\, \varphi(u) \lrarrow u \in \sigma\, ), 
			\label{fom:equivalent_formula_rewriting_12_1} \\
			\exists s\, (\, \forall u\, (\, \varphi(u) \lrarrow u \in s\, ) \wedge \psi(s)\, )
			&\vdash \psi(\sigma)
			\label{fom:equivalent_formula_rewriting_12_2}
		\end{align}
		が成り立つ.ここで定理\ref{thm:equivalent_formula_rewriting_4}より
		\begin{align}
			\EXTAX,\COMAX \vdash \forall u\, (\, \varphi(u) \lrarrow u \in \sigma\, )
			\rarrow \Set{y}{\varphi(y)} = \sigma
		\end{align}
		が成り立つので,(\refeq{fom:equivalent_formula_rewriting_12_1})との三段論法より
		\begin{align}
			\exists s\, (\, \forall u\, (\, \varphi(u) \lrarrow u \in s\, ) \wedge \psi(s)\, ),\ \EXTAX,\COMAX \vdash \Set{y}{\varphi(y)} = \sigma
			\label{fom:equivalent_formula_rewriting_12_3}
		\end{align}
		が得られる.また内包性公理より
		\begin{align}
			\COMAX \vdash \psi(\sigma) \rarrow \sigma \in \Set{z}{\psi(z)}
		\end{align}
		が成り立つので,(\refeq{fom:equivalent_formula_rewriting_12_1})との三段論法より
		\begin{align}
			\exists s\, (\, \forall u\, (\, \varphi(u) \lrarrow u \in s\, ) \wedge \psi(s)\, ),\ \COMAX \vdash \sigma \in \Set{z}{\psi(z)}
			\label{fom:equivalent_formula_rewriting_12_4}
		\end{align}
		が得られる.相等性公理より
		\begin{align}
			\EQAX &\vdash \Set{y}{\varphi(y)} = \sigma \rarrow \sigma = \Set{y}{\varphi(y)}, \\
			\EQAX &\vdash \sigma = \Set{y}{\varphi(y)} \rarrow
			(\, \sigma \in \Set{z}{\psi(z)} \rarrow \Set{y}{\varphi(y)} \in \Set{z}{\psi(z)}\, )
		\end{align}
		が成り立つので,(\refeq{fom:equivalent_formula_rewriting_12_3})と
		(\refeq{fom:equivalent_formula_rewriting_12_4})との三段論法より
		\begin{align}
			\exists s\, (\, \forall u\, (\, \varphi(u) \lrarrow u \in s\, ) \wedge \psi(s)\, ),\ \EXTAX,\EQAX,\COMAX \vdash \Set{y}{\varphi(y)} \in \Set{z}{\psi(z)}
		\end{align}
		が従う.
		\QED
	\end{sketch}
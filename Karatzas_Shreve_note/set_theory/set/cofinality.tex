\section{共終数}
	順序数$\alpha,\beta$に対して,$\alpha$から$\beta$への写像$f$で
	\begin{align}
		\forall \delta,\zeta \in \alpha\, (\, \delta < \zeta \rarrow f(\delta) < f(\zeta)\, )
	\end{align}
	と
	\begin{align}
		\forall \gamma \in \beta\, \exists \delta \in \alpha\, (\, \gamma < f(\delta)\, )
	\end{align}
	を満たすもの({\bf 共終写像}\index{きょうしゅうしゃぞう@共終写像}{\bf (cofinal function)})
	が存在することを
	\begin{align}
		\cof{\alpha}{\beta}
	\end{align}
	と書く.つまり
	\begin{align}
		\cof{\alpha}{\beta} \defarrow
		\exists f\, (\, &f:\alpha \rarrow \beta \\
		&\wedge \forall \delta,\zeta \in \alpha\, (\, \delta < \zeta \rarrow f(\delta) < f(\zeta)\, ) \\
		&\wedge \forall \gamma \in \beta\, \exists \delta \in \alpha\, (\, \gamma < f(\delta)\, )
		 \, )
	\end{align}
	である.また$\beta$に対して$\cof{\alpha}{\beta}$を満たす最小の順序数$\alpha$を
	\begin{align}
		\cf{\beta}
	\end{align}
	と書き,$\beta$の{\bf 共終数}\index{きょうしゅうすう@共終数}{\bf (cofinality)}と呼ぶ.
	
	\begin{screen}
		\begin{thm}[共終の概念は極限数のみに使われる]
			\begin{align}
				&\forall \alpha,\beta \in \ON\, 
				(\, \cof{\alpha}{\beta} \rarrow \limo{\beta}\, ), \\
				&\forall \alpha,\beta \in \ON\, 
				(\, \cof{\alpha}{\beta} \rarrow \limo{\alpha}\, )
			\end{align}
		\end{thm}
	\end{screen}
	
	\begin{sketch}
		$\cof{\alpha}{\beta}$であるとし,$f$を$\alpha$から$\beta$への共終写像とする.
		$\beta$の任意の要素$\gamma$に対して
		\begin{align}
			\gamma < f(\delta) < \beta
		\end{align}
		なる$\alpha$の要素$\delta$が取れるので,
		\begin{align}
			\forall \gamma \in \beta\, (\, \gamma + 1 \neq \beta\, )
		\end{align}
		である.ゆえに$\beta$は極限数である.
		$\alpha$の任意の要素$\delta$に対して
		\begin{align}
			f(\delta) < f(\zeta)
		\end{align}
		なる$\alpha$の要素$\zeta$が取れて,$f$の単調増大性より
		\begin{align}
			\delta < \zeta
		\end{align}
		となるので,
		\begin{align}
			\forall \delta \in \alpha\, (\, \delta+ 1 \neq \alpha\, )
		\end{align}
		である.ゆえに$\alpha$も極限数である.
		\QED
	\end{sketch}
	
	\begin{screen}
		\begin{thm}[極限数は自分自身と共終]
			\begin{align}
				\forall \alpha \in \ON\, 
				(\, \limo{\alpha} \rarrow \cof{\alpha}{\alpha}\, )
			\end{align}
		\end{thm}
	\end{screen}
	
	\begin{sketch}
		$\alpha$上の恒等写像が共終写像となる.
	\end{sketch}
	
	\begin{screen}
		\begin{thm}[共終数は基数である]
			\begin{align}
				\forall \alpha \in \ON\, (\, \limo{\alpha} \rarrow \cf{\alpha} \in \CN\, ).
			\end{align}
		\end{thm}
	\end{screen}
	
	\begin{screen}
		\begin{dfn}[正則基数]
			$\cf{\alpha} = \alpha$を満たす基数$\alpha$を{\bf 正則基数}
			\index{せいそくきすう@正則基数}{\bf (regular cardinal)}と呼ぶ.
			そうでない基数を{\bf 特異基数}\index{とくいきすう@特異基数}
			{\bf (singular cardinal)}と呼ぶ.
		\end{dfn}
	\end{screen}
		
	\begin{screen}
		\begin{thm}[$\Natural$は正則である]
			\begin{align}
				\cf{\Natural} = \Natural.
			\end{align}
		\end{thm}
	\end{screen}
	
	\begin{sketch}
		$\limo{\Natural}$より
		\begin{align}
			\cof{\Natural}{\Natural}
		\end{align}
		が成り立つので
		\begin{align}
			\cf{\Natural} \leq \Natural
		\end{align}
		となる.他方で
		\begin{align}
			\limo{\cf{\Natural}}
		\end{align}
		より
		\begin{align}
			\Natural \leq \cf{\Natural}
		\end{align}
		である.
		\QED
	\end{sketch}
	
	\begin{screen}
		\begin{thm}[共終数は正則である]
			\begin{align}
				\forall \alpha \in \ON\, 
				(\, \limo{\alpha} \rarrow \cf{\cf{\alpha}} = \cf{\alpha}\, ).
			\end{align}
		\end{thm}
	\end{screen}
	
	\begin{sketch}
		\begin{align}
			\cof{\cf{\alpha}}{\cf{\alpha}}
		\end{align}
		より
		\begin{align}
			\cf{\cf{\alpha}} \leq \cf{\alpha}
		\end{align}
		となる.また
		\begin{align}
			&\cof{\cf{\cf{\alpha}}}{\cf{\alpha}}, \\
			&\cof{\cf{\alpha}}{\alpha}
		\end{align}
		より
		\begin{align}
			\cof{\cf{\cf{\alpha}}}{\alpha}
		\end{align}
		が従い,
		\begin{align}
			\cf{\cf{\alpha}} \leq \cf{\alpha}
		\end{align}
		も得られる.
		\QED
	\end{sketch}
	
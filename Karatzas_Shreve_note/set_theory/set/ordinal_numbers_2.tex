	\begin{screen}
		\begin{dfn}[推移的類]
			$\mathcal{L}$の項$x$に対して,$x$が{\bf 推移的}\index{すいいてき@推移的}
			{\bf (transitive)}であるということを
			\begin{align}
				\tran{x} \defarrow
				\forall s\, (\, s \in x \rarrow s \subset x\, )
			\end{align}
			で定める.
		\end{dfn}
	\end{screen}
	
	$x$が推移的であるとは,「$x$の要素の要素が$x$の要素となる」という意味である.
	
	\begin{screen}
		\begin{dfn}[順序数]
			$\mathcal{L}$の項$x$に対して
			\begin{align}
				\ord{x} \defarrow \tran{x} \wedge 
				\forall t,u \in x\, (\, t \in u \vee t = u \vee u \in t\, )
			\end{align}
			と定め(ただし$t \in u \vee t = u \vee u \in t$は
			$(\, t \in u \vee t = u\, ) \vee u \in t$の略記とする),
			\begin{align}
				\ON \defeq \Set{x}{\ord{x}}
			\end{align}
			とおく.$\ON$の要素を{\bf 順序数}\index{じゅんじょすう@順序数}
			{\bf (ordinal number)}と呼ぶ.
		\end{dfn}
	\end{screen}
	
	空虚な真の一例であるが,例えば$0$は順序数の性質を満たす.
	ここに一つの順序数が得られたが,いま仮に$\alpha$を順序数とすれば
	\begin{align}
		\alpha \cup \{\alpha\}
	\end{align}
	もまた順序数となることが直ちに判明する.数字の定め方から
	\begin{align}
		1 &= 0 \cup \{0\}, \\
		2 &= 1 \cup \{1\}, \\
		3 &= 2 \cup \{2\}, \\
		&\vdots
	\end{align}
	が成り立つから,数字は全て順序数である.
	
	いま$\ON$上の関係を
	\begin{align}
		\leq\ \defeq \Set{x}{\exists \alpha,\beta \in \ON\, 
		(\, x=(\alpha,\beta) \wedge \alpha \subset \beta\, )}
	\end{align}
	と定める.そして
	\begin{align}
		x \leq y &\defarrow (x,y) \in\ \leq, \\
		x < y &\defarrow (x,y) \in\ \leq \wedge x \neq y
	\end{align}
	と書く(中置記法).
	
	以下順序数の性質を列挙するが,長いので主張だけ先に述べておく.
	\begin{itemize}
		\item $\ON$は推移的類である.
		\item $\leq$は$\ON$において整列順序となる.
		%\item $a$を$a \subset \ON$なる集合とすると,$\bigcup a$は$a$の$\leq$に関する上限となる.
		\item $\ON$は集合ではない.
	\end{itemize}
	
	\begin{screen}
		\begin{thm}[推移的で$\in$が全順序となる類は$\ON$に含まれる]
		\label{thm:transitive_totally_ordered_class}
			$a$を類とするとき
			\begin{align}
				\EXTAX,\EQAX,\COMAX,\PAIAX,\UNIAX,\REGAX \vdash 
				\ord{a} \rarrow a \subset \ON.
			\end{align}
		\end{thm}
	\end{screen}
	
	\begin{sketch}
		いま
		\begin{align}
			\chi \defeq \varepsilon x \negation 
			(\, x \in a \rarrow x \in \ON\, )
		\end{align}
		とおく.
		\begin{description}
			\item[step1] まず
				\begin{align}
					\chi \in a,\ \ord{a} \vdash 
					\forall s,t \in \chi\, (\, s \in t \vee s = t \vee t \in s\, )
					\label{fom:thm_transitive_totally_ordered_class_1}
				\end{align}
				を示す.$a$の推移性より
				\begin{align}
					\chi \in a,\ \ord{a} \vdash \chi \subset a
				\end{align}
				が成り立つから,
				\begin{align}
					\sigma &\defeq \varepsilon s \negation 
					(\, s \in \chi \rarrow \forall t\, (\, t \in \chi \rarrow 
					(\, s \in t \vee s = t \vee t \in s\, )\, )\, ), \\
					\tau &\defeq \varepsilon t \negation (\, t \in \chi \rarrow 
					(\, \sigma \in t \vee \sigma = t \vee t \in \sigma\, )\, )
				\end{align}
				とおけば,
				\begin{align}
					\tau \in \chi,\ \sigma \in \chi,\ \chi \in a,\ \ord{a} 
					&\vdash \sigma \in a, 
					\label{fom:thm_transitive_totally_ordered_class_2} \\
					\tau \in \chi,\ \sigma \in \chi,\ \chi \in a,\ \ord{a} 
					&\vdash \tau \in a
					\label{fom:thm_transitive_totally_ordered_class_3}
				\end{align}
				となる.他方で$\ord{a}$の定義式より
				\begin{align}
					\ord{a} \vdash 
					\forall s\, (\, s \in a \rarrow 
					\forall t\, (\, t \in a \rarrow (\, \sigma \in t \vee \sigma = t \vee t \in \sigma\, )\, )\, )
					\label{fom:thm_transitive_totally_ordered_class_4}
				\end{align}
				が成り立つので,全称記号の論理的公理および
				(\refeq{fom:thm_transitive_totally_ordered_class_2})と
				(\refeq{fom:thm_transitive_totally_ordered_class_3})との三段論法により
				\begin{align}
					\tau \in \chi,\ \sigma \in \chi,\ \chi \in a,\ \ord{a} \vdash 
					\sigma \in \tau \vee \sigma = \tau \vee \tau \in \sigma
				\end{align}
				が従う.演繹定理より
				\begin{align}
					\sigma \in \chi,\ \chi \in a,\ \ord{a} \vdash 
					\tau \in \chi \rarrow (\, \sigma \in \tau \vee \sigma = \tau \vee \tau \in \sigma\, )
				\end{align}
				が成り立つので,全称の導出
				(論理的定理\ref{logicalthm:derivation_of_universal_by_epsilon})より
				\begin{align}
					\sigma \in \chi,\ \chi \in a,\ \ord{a} \vdash 
					\forall t\, (\, t \in \chi \rarrow (\, \sigma \in t \vee \sigma = t \vee t \in \sigma\, )
				\end{align}
				となり,再び演繹定理と全称の導出によって
				\begin{align}
					\chi \in a,\ \ord{a} \vdash 
					\forall s\, (\, s \in \chi \rarrow \forall t\, (\, t \in \chi \rarrow (\, s \in t \vee s = t \vee t \in s\, )\, )
				\end{align}
				が得られる.
				
			\item[step2] 次に
				\begin{align}
					\chi \in a,\ \ord{a},\ 
					\EXTAX,\EQAX,\COMAX,\PAIAX,\UNIAX,\REGAX \vdash \tran{\chi}
				\end{align}
				を示す.いま
				\begin{align}
					\eta &\defeq \varepsilon y \negation (\, y \in \chi \rarrow y \subset \chi\, ), \\
					\zeta &\defeq \varepsilon z \negation (\, z \in \eta \rarrow z \in \chi\, )
				\end{align}
				とおく.
				\begin{description}
					\item[step2-1]
						この段では
						\begin{align}
							\zeta \in \eta,\ \eta \in \chi,\ \chi \in a,\ \ord{a} 
							\vdash 
							\zeta \in \chi \vee \zeta = \chi \vee \chi \in \zeta
						\end{align}
						を示す.$a$の推移性より
						\begin{align}
							\chi \in a,\ \ord{a} \vdash \chi \subset a
						\end{align}
						が成り立つので,
						\begin{align}
							\eta \in \chi,\ \chi \in a,\ \ord{a} \vdash \eta \in a
						\end{align}
						が成り立ち,再び$a$の推移性より
						\begin{align}
							\eta \in \chi,\ \chi \in a,\ \ord{a} \vdash 
							\eta \subset a
						\end{align}
						となる.従って
						\begin{align}
							\zeta \in \eta,\ \eta \in \chi,\ \chi \in a,\ \ord{a} 
							\vdash \zeta \in a
						\end{align}
						も成り立ち,
						(\refeq{fom:thm_transitive_totally_ordered_class_4})と
						全称記号の論理的公理より
						\begin{align}
							\zeta \in \eta,\ \eta \in \chi,\ \chi \in a,\ \ord{a} 
							\vdash 
							\zeta \in \chi \vee \zeta = \chi \vee \chi \in \zeta
							\label{fom:thm_transitive_totally_ordered_class_5}
						\end{align}
						が得られる.
						
					\item[step2-2] この段では
						\begin{align}
							\zeta \in \eta,\ \eta \in \chi,\ 
							\EXTAX,\EQAX,\COMAX,\PAIAX,\REGAX 
							\vdash \zeta \neq \chi
						\end{align}
						を示す.定理\ref{thm:no_pair_of_sets_go_round}
						(所属関係で堂々巡りしない)より
						\begin{align}
							\EXTAX,\EQAX,\COMAX,\PAIAX,\REGAX \vdash 
							\zeta \in \eta \rarrow \eta \notin \zeta
						\end{align}
						が成り立つので,演繹定理の逆より
						\begin{align}
							\zeta \in \eta,\ \EXTAX,\EQAX,\COMAX,\PAIAX,\REGAX 
							\vdash \eta \notin \zeta
						\end{align}
						となる.他方で
						\begin{align}
							\EQAX \vdash \eta \notin \zeta 
							\rarrow \zeta \neq \chi \vee \eta \notin \chi
						\end{align}
						が成り立つので($\zeta = \chi \wedge \eta \in \chi 
						\rarrow \eta \in \zeta$の対偶を取ってDe Morganの法則)
						\begin{align}
							\zeta \in \eta,\ \EXTAX,\EQAX,\COMAX,\PAIAX,\REGAX 
							\vdash \zeta \neq \chi \vee \eta \notin \chi
						\end{align}
						が従い,選言三段論法
						(論理的定理\ref{logicalthm:disjunctive_syllogism})より
						\begin{align}
							\zeta \in \eta,\ \EXTAX,\EQAX,\COMAX,\PAIAX,\REGAX 
							\vdash\ \negation \eta \notin \chi
							\rarrow \zeta \neq \chi
						\end{align}
						となる.ここで二重否定の導入
						(論理的定理\ref{logicalthm:introduction_of_double_negation})
						より
						\begin{align}
							\eta \in \chi \vdash\ \negation \eta \notin \chi
						\end{align}
						となるので
						\begin{align}
							\zeta \in \eta,\ \eta \in \chi,\ 
							\EXTAX,\EQAX,\COMAX,\PAIAX,\REGAX 
							\vdash \zeta \neq \chi
							\label{fom:thm_transitive_totally_ordered_class_6}
						\end{align}
						が得られる.
						
					\item[step2-3] この段では
						\begin{align}
							\zeta \in \eta,\ \eta \in \chi,\ \chi \in a,\ \ord{a},\ 
							\EXTAX,\EQAX,\COMAX,\PAIAX,\UNIAX,\REGAX \vdash 
							\zeta \in \chi
						\end{align}
						を示す.定理\ref{thm:no_three_sets_go_round}
						(所属関係で堂々巡りしない)より
						\begin{align}
							\EXTAX,\EQAX,\COMAX,\PAIAX,\UNIAX,\REGAX \vdash 
							\zeta \in \eta \wedge \eta \in \chi \rarrow 
							\chi \notin \zeta
						\end{align}
						が成り立つので,
						\begin{align}
							\zeta \in \eta,\ \eta \in \chi,\ 
							\EXTAX,\EQAX,\COMAX,\PAIAX,\UNIAX,\REGAX \vdash 
							\chi \notin \zeta
						\end{align}
						が従う.ここで
						(\refeq{fom:thm_transitive_totally_ordered_class_6})
						と論理積の導入より
						\begin{align}
							\zeta \in \eta,\ \eta \in \chi,\ 
							\EXTAX,\EQAX,\COMAX,\PAIAX,\UNIAX,\REGAX \vdash 
							\zeta \neq \chi \wedge \chi \notin \zeta
						\end{align}
						となり,De Morganの法則
						(論理的定理\ref{logicalthm:weak_De_Morgan_law_1})より
						\begin{align}
							\zeta \in \eta,\ \eta \in \chi,\ 
							\EXTAX,\EQAX,\COMAX,\PAIAX,\UNIAX,\REGAX \vdash\ 
							\negation (\, \zeta = \chi \vee \chi \in \zeta\, )
							\label{fom:thm_transitive_totally_ordered_class_7}
						\end{align}
						が成り立つ.ところで
						(\refeq{fom:thm_transitive_totally_ordered_class_5})と
						論理和の結合律
						(論理的定理\ref{logicalthm:associative_law_of_disjunctions})
						より
						\begin{align}
							\zeta \in \eta,\ \eta \in \chi,\ \chi \in a,\ \ord{a} 
							\vdash 
							\zeta \in \chi \vee (\, \zeta = \chi \vee \chi \in \zeta\, )
						\end{align}
						が成り立つので,
						(\refeq{fom:thm_transitive_totally_ordered_class_7})と
						選言三段論法
						(論理的定理\ref{logicalthm:disjunctive_syllogism})より
						\begin{align}
							\zeta \in \eta,\ \eta \in \chi,\ \chi \in a,\ \ord{a},\ 
							\EXTAX,\EQAX,\COMAX,\PAIAX,\UNIAX,\REGAX \vdash 
							\zeta \in \chi
							\label{fom:thm_transitive_totally_ordered_class_8}
						\end{align}
						が出る.
				\end{description}
				(\refeq{fom:thm_transitive_totally_ordered_class_8})と演繹定理より
				\begin{align}
					\eta \in \chi,\ \chi \in a,\ \ord{a},\ 
					\EXTAX,\EQAX,\COMAX,\PAIAX,\UNIAX,\REGAX \vdash 
					\zeta \in \eta \rarrow \zeta \in \chi
				\end{align}
				が成り立つので,全称の導出
				(論理的定理\ref{logicalthm:derivation_of_universal_by_epsilon})より
				\begin{align}
					\eta \in \chi,\ \chi \in a,\ \ord{a},\ 
					\EXTAX,\EQAX,\COMAX,\PAIAX,\UNIAX,\REGAX \vdash 
					\eta \subset \chi
				\end{align}
				が得られ,再び演繹定理と全称の導出により
				\begin{align}
					\chi \in a,\ \ord{a},\ 
					\EXTAX,\EQAX,\COMAX,\PAIAX,\UNIAX,\REGAX \vdash 
					\forall y\, (\, y \in \chi \rarrow y \subset \chi\, )
				\end{align}
				が得られる.すなわち
				\begin{align}
					\chi \in a,\ \ord{a},\ 
					\EXTAX,\EQAX,\COMAX,\PAIAX,\UNIAX,\REGAX \vdash \tran{\chi}
				\end{align}
				が成立する.
				
			\item[step3] step1 と step2 の結果を併せれば
				\begin{align}
					\chi \in a,\ \ord{a},\ 
					\EXTAX,\EQAX,\COMAX,\PAIAX,\UNIAX,\REGAX \vdash \ord{\chi}
				\end{align}
				が成り立つので,演繹定理より
				\begin{align}
					\ord{a},\ \EXTAX,\EQAX,\COMAX,\PAIAX,\UNIAX,\REGAX \vdash 
					\chi \in a \rarrow \chi \in \ON
				\end{align}
				となり,全称の導出
				(論理的定理\ref{logicalthm:derivation_of_universal_by_epsilon})より
				\begin{align}
					\ord{a},\ \EXTAX,\EQAX,\COMAX,\PAIAX,\UNIAX,\REGAX \vdash 
					a \subset \ON
				\end{align}
				が出る.
				\QED
		\end{description}
	\end{sketch}
	
	\begin{screen}
		\begin{thm}[$\ON$は推移的]\label{thm:On_is_transitive}
			\begin{align}
				\EXTAX,\EQAX,\COMAX,\PAIAX,\UNIAX,\REGAX \vdash \tran{\ON}.
			\end{align}
		\end{thm}
	\end{screen}
	
	\begin{prf}
		いま
		\begin{align}
			\chi \defeq \varepsilon x \negation 
			(\, x \in \ON \rarrow x \subset \ON\, )
		\end{align}
		とおけば,定理\ref{thm:transitive_totally_ordered_class}より
		\begin{align}
			\EXTAX,\EQAX,\COMAX,\PAIAX,\UNIAX,\REGAX \vdash 
			\chi \in \ON \rarrow \chi \subset \ON
		\end{align}
		が成り立つので,全称の導出
		(論理的定理\ref{logicalthm:derivation_of_universal_by_epsilon})より
		\begin{align}
			\EXTAX,\EQAX,\COMAX,\PAIAX,\UNIAX,\REGAX \vdash 
			\forall x\, (\, x \in \ON \rarrow x \subset \ON\, )
		\end{align}
		が従う.
		\QED
	\end{prf}
	
	\begin{screen}
		\begin{dfn}[類の差]
			$x$と$y$を$\mathcal{L}$の項とするとき,
			\begin{align}
				x \backslash y \defeq \Set{z}{z \in x \wedge z \notin y}
			\end{align}
			と定める.$x$と$y$が類であれば$x \backslash y$を{\bf 差類}\index{さるい@差類}{\bf (class difference)}と呼び,
			$x$と$y$が集合であれば$x \backslash y$を{\bf 差集合}\index{さしゅうごう@差集合}{\bf (set difference)}と呼ぶ.
		\end{dfn}
	\end{screen}
	
	$a$と$b$を類とするとき,任意の主要$\varepsilon$項$\tau$に対して
	\begin{align}
		\EXTAX,\EQAX,\COMAX,\ELEAX \vdash \tau \in b \backslash a \lrarrow \tau \in b \wedge \tau \notin a
	\end{align}
	が成り立つので(注意\ref{rem:epsilon_terms_of_not_L_epsilon_formula}),
	\begin{align}
		\EXTAX,\EQAX,\COMAX,\ELEAX \vdash b \backslash a \subset b
	\end{align}
	は常に満たされる.
	
	\begin{screen}
		\begin{thm}[差集合は集合]\label{thm:set_difference_is_set}
			$a$と$b$を主要$\varepsilon$項とするとき
			\begin{align}
				\EXTAX,\EQAX,\COMAX,\REPAX \vdash \set{b \backslash a}.
			\end{align}
		\end{thm}
	\end{screen}
	
	\begin{sketch}
		分出定理(定理\ref{thm:axiom_of_separation})より
		\begin{align}
			\EXTAX,\EQAX,\REPAX \vdash \exists s\, \forall x\, (\, x \in s \lrarrow x \in b \wedge (\, x \in b \wedge x \notin a\, )\, )
		\end{align}
		が成り立つ.ここで
		\begin{align}
			\sigma &\defeq \varepsilon s\, \forall x\, (\, x \in s \lrarrow x \in b \wedge (\, x \in b \wedge x \notin a\, )\, ), \\
			\tau &\defeq \varepsilon x\, (\, x \in \sigma \lrarrow x \in b \backslash a\, )
		\end{align}
		とおけば,
		\begin{align}
			\EXTAX,\EQAX,\REPAX \vdash \tau \in \sigma \lrarrow \tau \in b \wedge (\, \tau \in b \wedge \tau \notin a\, )
		\end{align}
		が成り立つ.ところで
		\begin{align}
			\vdash \tau \in b \wedge (\, \tau \in b \wedge \tau \notin a\, ) \lrarrow \tau \in b \wedge \tau \notin a
		\end{align}
		であるから,同値関係の推移律(論理的定理\ref{logicalthm:transitive_law_of_equivalence_symbol})より
		\begin{align}
			\EXTAX,\EQAX,\REPAX \vdash \tau \in \sigma \lrarrow \tau \in b \wedge \tau \notin a
		\end{align}
		が従う.また
		\begin{align}
			\COMAX \vdash \tau \in b \wedge \tau \notin a \lrarrow \tau \in b \backslash a
		\end{align}
		も成り立つので,再び同値関係の推移律によって
		\begin{align}
			\EXTAX,\EQAX,\COMAX,\REPAX \vdash \tau \in \sigma \lrarrow \tau \in b \backslash a
		\end{align}
		となり,全称の導出(論理的定理\ref{logicalthm:derivation_of_universal_by_epsilon})より
		\begin{align}
			\EXTAX,\EQAX,\COMAX,\REPAX \vdash \forall x\, (\, x \in \sigma \lrarrow x \in b \backslash a\, )
		\end{align}
		が従い,外延性公理と相等性公理(等号の対称性)より
		\begin{align}
			\EXTAX,\EQAX,\COMAX,\REPAX \vdash b \backslash a = \sigma
		\end{align}
		が出る.従って存在記号の論理的公理より
		\begin{align}
			\EXTAX,\EQAX,\COMAX,\REPAX \vdash \exists s\, (\, b \backslash a = s\, )
		\end{align}
		が得られる.
		\QED
	\end{sketch}
	
	\begin{screen}
		\begin{thm}[$\ON$において$\in$と$<$は同義]
		\label{thm:element_and_proper_subset_correspond}
			\begin{align}
				\EXTAX,\EQAX,\COMAX,\REPAX,\PAIAX,\UNIAX,\REGAX 
				\vdash \forall \alpha,\beta \in \ON\, (\, \alpha \in \beta \lrarrow \alpha < \beta\, ).
			\end{align}
		\end{thm}
	\end{screen}
	
	\begin{prf}
		いま
		\begin{align}
			a &\defeq \varepsilon \alpha \negation 
			(\, \alpha \in \ON \rarrow \forall \beta\, (\, \beta \in \ON \rarrow 
			(\, \alpha \in \beta \lrarrow \alpha < \beta\, )\,) \,), \\
			b &\defeq \varepsilon \beta \negation (\, \beta \in \ON \rarrow 
			(\, a \in \beta \lrarrow a < \beta\, )\,)
		\end{align}
		とおく.
		\begin{description}
			\item[step1] まず
				\begin{align}
					\ord{b},\ \EXTAX,\EQAX,\COMAX,\PAIAX,\UNIAX,\REGAX \vdash 
					a \in b \rarrow a < b
				\end{align}
				を示す.順序数の推移性より
				\begin{align}
					a \in b,\ \ord{b} \vdash a \subset b
					\label{fom:element_and_proper_subset_correspond_1}
				\end{align}
				が成り立つから,
				\begin{align}
					\tau \defeq \varepsilon x\, (\, (a,b) = x\, )
				\end{align}
				とおけば
				\begin{align}
					a \in b,\ \ord{b},\ \COMAX 
					\vdash b \in \ON \wedge (\, \tau = (a,b) \wedge a \subset b\, )
				\end{align}
				となり
				\begin{align}
					a \in b,\ \ord{b},\ \COMAX  
					\vdash \exists \beta\, (\, \beta \in \ON \wedge 
					(\, \tau = (a,\beta) \wedge a \subset \beta\, )\, )
					\label{fom:element_and_proper_subset_correspond_2}
				\end{align}
				が従う.ところで定理\ref{thm:transitive_totally_ordered_class}より
				\begin{align}
					\ord{b},\ \EXTAX,\EQAX,\COMAX,\PAIAX,\UNIAX,\REGAX \vdash 
					b \subset \ON
				\end{align}
				が成り立つので,
				\begin{align}
					a \in b,\ \ord{b},\ \EXTAX,\EQAX,\COMAX,\PAIAX,\UNIAX,\REGAX 
					\vdash a \in \ON
				\end{align}
				となる.すなわち,(\refeq{fom:element_and_proper_subset_correspond_2})と
				論理積の導入より
				\begin{align}
					&a \in b,\ \ord{b},\ \EXTAX,\EQAX,\COMAX,\PAIAX,\UNIAX,\REGAX \\
					&\vdash a \in \ON \wedge \exists \beta\, (\, \beta \in \ON \wedge 
					(\, \tau = (a,\beta) \wedge a \subset \beta\, )\, )
				\end{align}
				が成り立ち,
				\begin{align}
					&a \in b,\ \ord{b},\ \EXTAX,\EQAX,\COMAX,\PAIAX,\UNIAX,\REGAX \\
					&\vdash \exists \alpha\, (\, \alpha \in \ON \wedge 
					\exists \beta\, (\, \beta \in \ON \wedge 
					(\, \tau = (\alpha,\beta) \wedge \alpha \subset \beta\, )\, )\, )
				\end{align}
				が成り立ち,ゆえに
				\begin{align}
					a \in b,\ \ord{b},\ \EXTAX,\EQAX,\COMAX,\PAIAX,\UNIAX,\REGAX 
					\vdash \tau \in\ \leq
				\end{align}
				となり,
				\begin{align}
					a \in b,\ \ord{b},\ \EXTAX,\EQAX,\COMAX,\PAIAX,\UNIAX,\REGAX 
					\vdash (a,b) \in\ \leq
					\label{fom:element_and_proper_subset_correspond_3}
				\end{align}
				が従う.他方で
				\begin{align}
					\EQAX \vdash a \in b \rarrow (\, a \notin a \rarrow a \neq b\, )
				\end{align}
				と演繹定理の逆より
				\begin{align}
					a \in b,\ \EQAX \vdash a \notin a \rarrow a \neq b
				\end{align}
				となり,定理\ref{thm:no_class_contains_itself}
				(自分自身は要素ではない)と併せて
				\begin{align}
					a \in b,\ \EXTAX,\EQAX,\COMAX,\PAIAX,\REGAX \vdash 
					a \neq b
					\label{fom:element_and_proper_subset_correspond_4}
				\end{align}
				が従う.(\refeq{fom:element_and_proper_subset_correspond_3})と
				(\refeq{fom:element_and_proper_subset_correspond_4})と論理積の導入より
				\begin{align}
					a \in b,\ \ord{b},\ \EXTAX,\EQAX,\COMAX,\PAIAX,\UNIAX,\REGAX 
					\vdash (a,b) \in\ \leq \wedge a \neq b
				\end{align}
				となるので,
				\begin{align}
					a \in b,\ \ord{b},\ \EXTAX,\EQAX,\COMAX,\PAIAX,\UNIAX,\REGAX \vdash a < b
					\label{fom:element_and_proper_subset_correspond_21}
				\end{align}
				が得られる.
				
			\item[step2]
				外延性公理と対偶律1 
				(論理的定理\ref{logicalthm:introduction_of_contraposition})より
				\begin{align}
					a \neq b,\ \EXTAX \vdash\ \negation\forall x\, (\, x \in a \lrarrow x \in b\, )
				\end{align}
				となり,量化子の論理的公理より
				\begin{align}
					a \neq b,\ \EXTAX \vdash \exists x \negation (\, x \in a \lrarrow x \in b\, )
				\end{align}
				が成り立つので,
				\begin{align}
					\tau \defeq \varepsilon x \negation (\, x \in a \lrarrow x \in b\, )
				\end{align}
				とおけば,存在記号の論理的公理より
				\begin{align}
					a \neq b,\ \EXTAX \vdash\ \negation (\, \tau \in a \lrarrow \tau \in b\, )
				\end{align}
				が成り立ち,De Morganの法則(論理的定理\ref{logicalthm:strong_De_Morgan_law_2})より
				\begin{align}
					a \neq b,\ \EXTAX \vdash\ \negation (\, \tau \in a \rarrow \tau \in b\, ) \vee
					\negation (\, \tau \in b \rarrow \tau \in a\, )
				\end{align}
				が従う.よって選言三段論法(論理的定理\ref{logicalthm:disjunctive_syllogism})より
				\begin{align}
					a \neq b,\ \EXTAX \vdash\ \negation \negation (\, \tau \in a \rarrow \tau \in b\, )
					\rarrow\ \negation (\, \tau \in b \rarrow \tau \in a\, )
				\end{align}
				が成り立つが,二重否定の導入
				(論理的定理\ref{logicalthm:introduction_of_double_negation})より
				\begin{align}
					\tau \in a \rarrow \tau \in b \vdash\ \negation \negation (\, \tau \in a \rarrow \tau \in b\, )
				\end{align}
				となるので,三段論法および演繹定理より
				\begin{align}
					a \neq b,\ \EXTAX \vdash (\, \tau \in a \rarrow \tau \in b\, )
					\rarrow\ \negation (\, \tau \in b \rarrow \tau \in a\, )
				\end{align}
				が得られる.他方で
				\begin{align}
					a < b \vdash a \subset b \wedge a \neq b
				\end{align}
				であるから
				\begin{align}
					a < b,\ \EXTAX \vdash (\, \tau \in a \rarrow \tau \in b\, )
					\rarrow\ \negation (\, \tau \in b \rarrow \tau \in a\, )
					\label{fom:element_and_proper_subset_correspond_6}
				\end{align}
				が成り立ち,また
				\begin{align}
					a < b \vdash \tau \in a \rarrow \tau \in b
				\end{align}
				も成り立つので,(\refeq{fom:element_and_proper_subset_correspond_6})との三段論法より
				\begin{align}
					a < b,\ \EXTAX \vdash\ \negation (\, \tau \in b \rarrow \tau \in a\, )
					\label{fom:element_and_proper_subset_correspond_7}
				\end{align}
				となる.ところで論理的定理\ref{logicalthm:disjunction_of_negation_rewritable_by_implication}
				(否定の論理和は含意で書ける)と対偶律1 (論理的定理\ref{logicalthm:introduction_of_contraposition})より
				\begin{align}
					\vdash\ \negation (\, \tau \in b \rarrow \tau \in a\, ) \rarrow\ 
					\negation (\, \tau \notin b \vee \tau \in a\, )
				\end{align}
				が成り立つので,(\refeq{fom:element_and_proper_subset_correspond_7})との三段論法より
				\begin{align}
					a < b,\ \EXTAX \vdash\ \negation (\, \tau \notin b \vee \tau \in a\, )
				\end{align}
				が従い,De Morganの法則(論理的定理\ref{logicalthm:weak_De_Morgan_law_2})と二重否定の除去より
				\begin{align}
					a < b,\ \EXTAX \vdash \tau \in b \wedge \tau \notin a
				\end{align}
				が従う.そして
				\begin{align}
					a < b,\ \EXTAX,\COMAX \vdash \tau \in b \backslash a
					\label{fom:element_and_proper_subset_correspond_5}
				\end{align}
				が得られる.
				
			\item[step3] 定理\ref{thm:set_difference_is_set}より
				\begin{align}
					\EXTAX,\EQAX,\COMAX,\REPAX \vdash \set{b \backslash a}
				\end{align}
				が成り立つので,
				\begin{align}
					\sigma \defeq \exists s\, (\, b \backslash a = s\, )
				\end{align}
				とおけば
				\begin{align}
					\EXTAX,\EQAX,\COMAX,\REPAX \vdash b \backslash a = \sigma
					\label{fom:element_and_proper_subset_correspond_8}
				\end{align}
				となる.(\refeq{fom:element_and_proper_subset_correspond_5})と併せて
				\begin{align}
					a < b,\ \EXTAX,\EQAX,\COMAX,\REPAX \vdash \tau \in \sigma
				\end{align}
				となり,存在記号の論理的公理より
				\begin{align}
					a < b,\ \EXTAX,\EQAX,\COMAX,\REPAX \vdash \exists x\, (\, x \in \sigma\, )
				\end{align}
				となるが,正則性公理より
				\begin{align}
					\REGAX \vdash \exists x\, (\, x \in \sigma\, ) 
					\rarrow \exists y\, (\, y \in \sigma \wedge \forall z\, (\, z \in \sigma \rarrow z \notin y\, )\, )
				\end{align}
				が成り立つので
				\begin{align}
					a < b,\ \EXTAX,\EQAX,\COMAX,\REPAX,\REGAX \vdash 
					\exists y\, (\, y \in \sigma \wedge \forall z\, (\, z \in \sigma \rarrow z \notin y\, )\, )
				\end{align}
				が従う.ここで
				\begin{align}
					\eta \defeq \varepsilon y\, (\, y \in \sigma \wedge 
					\forall z\, (\, z \in \sigma \rarrow z \notin y\, )\, )
				\end{align}
				とおけば
				\begin{align}
					a < b,\ \EXTAX,\EQAX,\COMAX,\REPAX,\REGAX &\vdash \eta \in \sigma, 
					\label{fom:element_and_proper_subset_correspond_9} \\
					a < b,\ \EXTAX,\EQAX,\COMAX,\REPAX,\REGAX &\vdash 
					\forall z\, (\, z \in \sigma \rarrow z \notin \eta\, )
					\label{fom:element_and_proper_subset_correspond_10}
				\end{align}
				が成り立つ.
				
			\item[step4] この段では
				\begin{align}
					a < b,\ \ord{a},\ \ord{b} \EXTAX,\EQAX,\COMAX,\REPAX,\REGAX \vdash a \subset \eta 
				\end{align}
				を示す.いま
				\begin{align}
					\chi \defeq \varepsilon x \negation (\, x \in a \rarrow x \in \eta\, )
				\end{align}
				とおく.(\refeq{fom:element_and_proper_subset_correspond_8})と
				(\refeq{fom:element_and_proper_subset_correspond_9})より
				\begin{align}
					a < b,\ \EXTAX,\EQAX,\COMAX,\REPAX,\REGAX &\vdash \eta \in b, 
					\label{fom:element_and_proper_subset_correspond_11} \\
					a < b,\ \EXTAX,\EQAX,\COMAX,\REPAX,\REGAX &\vdash \eta \notin a
					\label{fom:element_and_proper_subset_correspond_12}
				\end{align}
				となり,他方で
				\begin{align}
					\chi \in a,\ a < b \vdash \chi \in b
				\end{align}
				となるから,$\ord{b}$を公理に追加すれば
				\begin{align}
					\chi \in a,\ a < b,\ \ord{b},\ \EXTAX,\EQAX,\COMAX,\REPAX,\REGAX 
					\vdash \chi \in \eta \vee \chi = \eta \vee \eta \in \chi
					\label{fom:element_and_proper_subset_correspond_13}
				\end{align}
				が成り立つ.ところで
				\begin{align}
					\EQAX \vdash \chi = \eta \wedge \chi \in a \rarrow \eta \in a
				\end{align}
				が成り立つので,対偶律1 (論理的定理\ref{logicalthm:introduction_of_contraposition})より
				\begin{align}
					\EQAX \vdash \eta \notin a \rarrow \chi \neq \eta \vee \chi \notin a
				\end{align}
				となる.また
				\begin{align}
					\ord{a} \vdash \eta \in \chi \wedge \chi \in a \rarrow \eta \in a
				\end{align}
				が成り立つので,対偶律1 (論理的定理\ref{logicalthm:introduction_of_contraposition})より
				\begin{align}
					\ord{a} \vdash \eta \notin a \rarrow \eta \notin \chi \vee \chi \notin a
				\end{align}
				となる.よって,(\refeq{fom:element_and_proper_subset_correspond_12})と併せて
				\begin{align}
					a < b,\ \ord{a},\ \EXTAX,\EQAX,\COMAX,\REPAX,\REGAX &\vdash \chi \neq \eta \vee \chi \notin a, \\
					a < b,\ \ord{a},\ \EXTAX,\EQAX,\COMAX,\REPAX,\REGAX &\vdash \eta \notin \chi \vee \chi \notin a
				\end{align}
				が成り立ち,選言三段論法(論理的定理\ref{logicalthm:disjunctive_syllogism})より
				\begin{align}
					a < b,\ \ord{a},\ \EXTAX,\EQAX,\COMAX,\REPAX,\REGAX 
					&\vdash \chi \in a \rarrow \chi \neq \eta, \\
					a < b,\ \ord{a},\ \EXTAX,\EQAX,\COMAX,\REPAX,\REGAX 
					&\vdash \chi \in a \rarrow \eta \notin \chi
				\end{align}
				が成り立ち,演繹定理の逆により
				\begin{align}
					\chi \in a,\ a < b,\ \ord{a},\ \EXTAX,\EQAX,\COMAX,\REPAX,\REGAX &\vdash \chi \neq \eta,
					\label{fom:element_and_proper_subset_correspond_14} \\
					\chi \in a,\ a < b,\ \ord{a},\ \EXTAX,\EQAX,\COMAX,\REPAX,\REGAX &\vdash \eta \notin \chi
					\label{fom:element_and_proper_subset_correspond_15}
				\end{align}
				となり,論理積の導入とDe Morganの法則(論理的定理\ref{logicalthm:weak_De_Morgan_law_1})より
				\begin{align}
					\chi \in a,\ a < b,\ \ord{a},\ \EXTAX,\EQAX,\COMAX,\REPAX,\REGAX \vdash\ 
					\negation (\, \chi = \eta \vee \eta \in \chi\, )
				\end{align}
				が従う.これと(\refeq{fom:element_and_proper_subset_correspond_13})と
				選言三段論法(論理的定理\ref{logicalthm:disjunctive_syllogism})より
				\begin{align}
					\chi \in a,\ a < b,\ \ord{a},\ \ord{b},\ \EXTAX,\EQAX,\COMAX,\REPAX,\REGAX \vdash \chi \in \eta
				\end{align}
				が成立するので,演繹定理より
				\begin{align}
					a < b,\ \ord{a},\ \ord{b},\ \EXTAX,\EQAX,\COMAX,\REPAX,\REGAX \vdash 
					\chi \in a \rarrow \chi \in \eta
				\end{align}
				となり,全称の導出(論理的定理\ref{logicalthm:derivation_of_universal_by_epsilon})より
				\begin{align}
					a < b,\ \ord{a},\ \ord{b},\ \EXTAX,\EQAX,\COMAX,\REPAX,\REGAX \vdash a \subset \eta
					\label{fom:element_and_proper_subset_correspond_16}
				\end{align}
				が出る.
			
			\item[step5] この段では
				\begin{align}
					a < b,\ \ord{b},\ \EXTAX,\EQAX,\COMAX,\REPAX,\REGAX \vdash \eta \subset a
				\end{align}
				を示す.いま
				\begin{align}
					\chi \defeq \varepsilon x \negation (\, x \in \eta \rarrow x \in a\, )
				\end{align}
				とおく.(\refeq{fom:element_and_proper_subset_correspond_11})と順序数の推移性より
				\begin{align}
					\chi \in \eta,\ a < b,\ \ord{b},\ \EXTAX,\EQAX,\COMAX,\REPAX,\REGAX \vdash \chi \in b
					\label{fom:element_and_proper_subset_correspond_17}
				\end{align}
				となる.他方で(\refeq{fom:element_and_proper_subset_correspond_10})と
				対偶律2 (論理的定理\ref{logicalthm:contraposition_2})より
				\begin{align}
					a < b,\ \EXTAX,\EQAX,\COMAX,\REPAX,\REGAX \vdash 
					\chi \in \eta \rarrow \chi \notin \sigma
				\end{align}
				となり,演繹定理の逆より
				\begin{align}
					\chi \in \eta,\ a < b,\ \EXTAX,\EQAX,\COMAX,\REPAX,\REGAX \vdash \chi \notin \sigma
				\end{align}
				となるが,(\refeq{fom:element_and_proper_subset_correspond_8})より
				\begin{align}
					\chi \in \eta,\ a < b,\ \EXTAX,\EQAX,\COMAX,\REPAX,\REGAX \vdash \chi \notin b \backslash a
					\label{fom:element_and_proper_subset_correspond_18}
				\end{align}
				が従う.ところで
				\begin{align}
					\COMAX \vdash \chi \in b \backslash a \lrarrow \chi \in b \wedge \chi \notin a
				\end{align}
				が成り立つので,対偶を取って
				\begin{align}
					\COMAX \vdash \chi \notin b \backslash a \lrarrow \chi \notin b \vee \chi \in a
				\end{align}
				となる.よって(\refeq{fom:element_and_proper_subset_correspond_18})から
				\begin{align}
					\chi \in \eta,\ a < b,\ \EXTAX,\EQAX,\COMAX,\REPAX,\REGAX \vdash \chi \notin b \vee \chi \in a
				\end{align}
				が従い,論理的定理\ref{logicalthm:disjunction_of_negation_rewritable_by_implication}
				(否定の論理和は含意で書ける)より
				\begin{align}
					\chi \in \eta,\ a < b,\ \EXTAX,\EQAX,\COMAX,\REPAX,\REGAX \vdash \chi \in b \rarrow \chi \in a
				\end{align}
				が成り立つ.これと(\refeq{fom:element_and_proper_subset_correspond_17})との三段論法より
				\begin{align}
					\chi \in \eta,\ a < b,\ \ord{b},\ \EXTAX,\EQAX,\COMAX,\REPAX,\REGAX \vdash \chi \in a
				\end{align}
				が成り立ち,演繹定理と全称の導出(論理的定理\ref{logicalthm:derivation_of_universal_by_epsilon})より
				\begin{align}
					a < b,\ \ord{b},\ \EXTAX,\EQAX,\COMAX,\REPAX,\REGAX \vdash \eta \subset a
					\label{fom:element_and_proper_subset_correspond_19}
				\end{align}
				が出る.
			
			\item[step5] (\refeq{fom:element_and_proper_subset_correspond_16})と
				(\refeq{fom:element_and_proper_subset_correspond_19})および
				定理\ref{thm:mutually_including_classes_are_equivalent} (互いに相手を包含する類同士は等しい)より
				\begin{align}
					a < b,\ \ord{a},\ \ord{b},\ \EXTAX,\EQAX,\COMAX,\REPAX,\PAIAX,\UNIAX,\REGAX \vdash \eta = a
				\end{align}
				が成り立ち,(\refeq{fom:element_and_proper_subset_correspond_11})と相等性公理より
				\begin{align}
					a < b,\ \ord{a},\ \ord{b},\ \EXTAX,\EQAX,\COMAX,\REPAX,\PAIAX,\UNIAX,\REGAX \vdash a \in b
					\label{fom:element_and_proper_subset_correspond_20}
				\end{align}
				が従う.(\refeq{fom:element_and_proper_subset_correspond_21})と
				(\refeq{fom:element_and_proper_subset_correspond_20})と演繹定理および論理積の導入より
				\begin{align}
					\ord{a},\ \EXTAX,\EQAX,\COMAX,\REPAX,\PAIAX,\UNIAX,\REGAX \vdash 
					\ord{b} \rarrow (\, a \in b \lrarrow a < b\, )
				\end{align}
				が成り立つが,ここで全称の導出(論理的定理\ref{logicalthm:derivation_of_universal_by_epsilon})より
				\begin{align}
					&\ord{a},\ \EXTAX,\EQAX,\COMAX,\REPAX,\PAIAX,\UNIAX,\REGAX \\
					&\vdash \forall \beta\, (\, \beta \in \ON \rarrow (\, a \in \beta \lrarrow a < \beta\, )\, )
				\end{align}
				が従い,同様にして
				\begin{align}
					&\EXTAX,\EQAX,\COMAX,\REPAX,\PAIAX,\UNIAX,\REGAX \\
					&\vdash \forall \alpha\, (\, \alpha \in \ON \rarrow
					\forall \beta\, (\, \beta \in \ON \rarrow (\, \alpha \in \beta \lrarrow \alpha < \beta\, )\, )\, )
				\end{align}
				が得られる.
				\QED
		\end{description}
	\end{prf}
	
	\begin{screen}
		\begin{thm}[$\ON$の整列性]\label{thm:On_is_wellordered}
			$\leq$は$\ON$上の整列順序である.また次が成り立つ.
			\begin{align}
				\forall \alpha,\beta \in \ON\,
				\left(\, \alpha \in \beta \vee \alpha = \beta \vee \beta \in \alpha\, \right).
			\end{align}
		\end{thm}
	\end{screen}
	
	\begin{prf}\mbox{}
		\begin{description}
			\item[第一段]
				$\alpha,\beta,\gamma$を順序数とすれば
				\begin{align}
					\alpha \subset \alpha
				\end{align}
				かつ
				\begin{align}
					\alpha \subset \beta \wedge \beta \subset \alpha \Longrightarrow \alpha = \beta
				\end{align}
				かつ
				\begin{align}
					\alpha \subset \beta \wedge \beta \subset \gamma \Longrightarrow \alpha \subset \gamma
				\end{align}
				が成り立つ.ゆえに$\leq$は$\ON$上の順序である.
				
			\item[第二段]
				$\leq$が全順序であることを示す.$\alpha$と$\beta$を順序数とする.このとき
				\begin{align}
					\ord{\alpha \cap \beta}
				\end{align}
				が成り立ち,他方で定理\ref{thm:no_set_is_an_element_of_itself}より
				\begin{align}
					\alpha \cap \beta \notin \alpha \cap \beta
				\end{align}
				が満たされるので
				\begin{align}
					\alpha \cap \beta \notin \alpha \vee \alpha \cap \beta \notin \beta
					\label{eq:thm_On_is_wellordered_5}
				\end{align}
				が成立する.ところで
				\begin{align}
					\alpha \cap \beta \subset \alpha
				\end{align}
				は正しいので定理\ref{thm:element_and_proper_subset_correspond_between_ordinal_numbers}から
				\begin{align}
					\alpha \cap \beta \in \alpha \vee \alpha \cap \beta = \alpha
				\end{align}
				が成立する.従って
				\begin{align}
					\alpha \cap \beta \notin \alpha \Longrightarrow 
					(\alpha \cap \beta \in \alpha \vee \alpha \cap \beta = \alpha) \wedge \alpha \cap \beta \notin \alpha
					\label{eq:thm_On_is_wellordered_2}
				\end{align}
				が成り立ち,他方で選言三段論法より
				\begin{align}
					(\alpha \cap \beta \in \alpha \vee \alpha \cap \beta = \alpha) \wedge \alpha \cap \beta \notin \alpha
					\Longrightarrow \alpha \cap \beta = \alpha
					\label{eq:thm_On_is_wellordered_3}
				\end{align}
				も成り立ち,かつ
				\begin{align}
					\alpha \cap \beta = \alpha \Longrightarrow \alpha \subset \beta
					\label{eq:thm_On_is_wellordered_4}
				\end{align}
				も成り立つので,(\refeq{eq:thm_On_is_wellordered_2})と(\refeq{eq:thm_On_is_wellordered_3})と
				(\refeq{eq:thm_On_is_wellordered_4})から
				\begin{align}
					\alpha \cap \beta \notin \alpha \Longrightarrow \alpha \subset \beta
				\end{align}
				が得られる.同様にして
				\begin{align}
					\alpha \cap \beta \notin \beta \Longrightarrow \beta \subset \alpha
				\end{align}
				も得られる.さらに論理和の規則から
				\begin{align}
					\alpha \cap \beta \notin \alpha \Longrightarrow \alpha \subset \beta \vee \beta \subset \alpha
				\end{align}
				と
				\begin{align}
					\alpha \cap \beta \notin \beta \Longrightarrow \alpha \subset \beta \vee \beta \subset \alpha
				\end{align}
				が従うので,(\refeq{eq:thm_On_is_wellordered_5})と場合分け法則より
				\begin{align}
					\alpha \subset \beta \vee \beta \subset \alpha
				\end{align}
				が成立して
				\begin{align}
					(\alpha,\beta) \in\ \leq \vee (\beta,\alpha) \in\ \leq
				\end{align}
				が成立する.ゆえに$\leq$は全順序である.
			
			\item[第三段]
				$\leq$が整列順序であることを示す.$a$を$\ON$の空でない部分集合とする.このとき正則性公理より
				\begin{align}
					x \in a \wedge x \cap a = \emptyset
				\end{align}
				を満たす集合$x$が取れるが,この$x$が$a$の最小限である.実際,任意に$a$から要素$y$を取ると
				\begin{align}
					x \cap a = \emptyset
				\end{align}
				から
				\begin{align}
					y \notin x
				\end{align}
				が従い,また前段の結果より
				\begin{align}
					x \in y \vee x = y \vee y \in x
				\end{align}
				も成り立つので,選言三段論法より
				\begin{align}
					x \in y \vee x = y
					\label{eq:thm_On_is_wellordered_6}
				\end{align}
				が成り立つ.$y$は推移的であるから
				\begin{align}
					x \in y \Longrightarrow x \subset y
				\end{align}
				が成立して,また
				\begin{align}
					x = y \Longrightarrow x \subset y
				\end{align}
				も成り立つから,(\refeq{eq:thm_On_is_wellordered_6})と場合分け法則から
				\begin{align}
					(x,y) \in\ \leq
				\end{align}
				が従う.$y$の任意性より
				\begin{align}
					\forall y \in a\, \left[\, (x,y) \in\ \leq\, \right]
				\end{align}
				が成立するので$x$は$a$の最小限である.
				\QED
		\end{description}
	\end{prf}
	
	\begin{screen}
		\begin{thm}[$\ON$の部分集合の合併は順序数となる]\label{thm:union_of_set_of_ordinal_numbers_is_ordinal}
			\begin{align}
				\forall a\,
				\left(\, a \subset \ON \Longrightarrow \bigcup a \in \ON\, \right).
			\end{align}
		\end{thm}
	\end{screen}
	
	\begin{prf}
		和集合の公理より$\bigcup a \in \Univ$となる.また順序数の推移性より
		$\bigcup a$の任意の要素は順序数であるから,定理\ref{thm:On_is_wellordered}より
		\begin{align}
			\forall x,y \in \bigcup a\ (\ x \in y \vee x = y \vee y \in x\ )
		\end{align}
		も成り立つ.最後に$\operatorname{Tran}(\bigcup a)$が成り立つことを示す.
		$b$を$\bigcup a$の任意の要素とすれば,$a$の或る要素$x$に対して
		\begin{align}
			b \in x
		\end{align}
		となるが,$x$の推移性より$b \subset x$となり,$x \subset \bigcup a$と併せて
		\begin{align}
			b \subset \bigcup a
		\end{align}
		が従う.
		\QED
	\end{prf}
	
	\begin{screen}
		\begin{thm}[Burali-Forti]\label{thm:Burali_Forti}
			順序数の全体は集合ではない.
			\begin{align}
				\rightharpoondown \set{\ON}.
			\end{align}
		\end{thm}
	\end{screen}
	
	\begin{prf}
		$a$を類とするとき,定理\ref{thm:satisfactory_set_is_an_element}より
		\begin{align}
			\ord{a} \Longrightarrow \left(\, \set{a} \Longrightarrow a \in \ON\, \right)
		\end{align}
		が成り立つ.定理\ref{thm:On_is_transitive}と定理\ref{thm:On_is_wellordered}より
		\begin{align}
			\ord{\ON}
		\end{align}
		が成り立つから
		\begin{align}
			\set{\ON} \Longrightarrow \ON \in \ON
			\label{eq:Burali_Forti_1}
		\end{align}
		が従い,また定理\ref{thm:no_set_is_an_element_of_itself}より
		\begin{align}
			\ON \notin \ON
		\end{align}
		も成り立つので,(\refeq{eq:Burali_Forti_1})の対偶から
		\begin{align}
			\rightharpoondown \set{\ON}
		\end{align}
		が成立する.
		\QED
	\end{prf}
	
	\begin{screen}
		\begin{dfn}[後者]
			$x$を集合とするとき,
			\begin{align}
				x \cup \{x\}
			\end{align}
			を$x$の{\bf 後者}\index{こうしゃ@後者}{\bf (latter)}と呼ぶ.
		\end{dfn}
	\end{screen}
	
	\begin{screen}
		\begin{thm}[順序数の後者は順序数である]\label{thm:latter_element_is_ordinal}
			$\alpha$が順序数であるということと$\alpha \cup \{\alpha\}$が順序数であるということは同値である.
			\begin{align}
				\forall \alpha\, \left(\, \alpha \in \ON \Longleftrightarrow \alpha \cup \{\alpha\} \in \ON\, \right).
			\end{align}
		\end{thm}
	\end{screen}
	
	\begin{sketch}\mbox{}
		\begin{description}
			\item[第一段]
				$\alpha$を順序数とする.そして$x$を
				\begin{align}
					x \in \alpha \cup \{\alpha\}
					\label{fom:thm_latter_element_is_ordinal_3}
				\end{align}
				なる任意の集合とすると,
				\begin{align}
					y \in x
				\end{align}
				なる任意の集合$y$に対して定理\ref{thm:union_of_pair_is_union_of_their_elements}より
				\begin{align}
					y \in \alpha \vee y \in \{\alpha\}
					\label{fom:thm_latter_element_is_ordinal_5}
				\end{align}
				が成立する.$\alpha$が順序数であるから
				\begin{align}
					y \in \alpha \Longrightarrow y \subset \alpha
					\label{fom:thm_latter_element_is_ordinal_1}
				\end{align}
				が成立する.他方で定理\ref{thm:pair_members_are_exactly_the_given_two}より
				\begin{align}
					y \in \{\alpha\} \Longrightarrow y = \alpha
				\end{align}
				が成立し,
				\begin{align}
					y = \alpha \Longrightarrow y \subset \alpha
				\end{align}
				であるから
				\begin{align}
					y \in \{\alpha\} \Longrightarrow y \subset \alpha
					\label{fom:thm_latter_element_is_ordinal_2}
				\end{align}
				が従う.定理\ref{thm:union_is_bigger_than_any_member}より
				\begin{align}
					y \subset \alpha \Longrightarrow y \subset \alpha \cup \{\alpha\}
				\end{align}
				が成り立つので,(\refeq{fom:thm_latter_element_is_ordinal_1})と
				(\refeq{fom:thm_latter_element_is_ordinal_2})と併せて
				\begin{align}
					y \in \alpha \Longrightarrow y \subset \alpha \cup \{\alpha\}
				\end{align}
				かつ
				\begin{align}
					y \in \{\alpha\} \Longrightarrow y \subset \alpha \cup \{\alpha\}
				\end{align}
				が成立し,場合分け法則より
				\begin{align}
					y \in \alpha \vee y \in \{\alpha\} \Longrightarrow y \subset \alpha \cup \{\alpha\}
				\end{align}
				が従う.そして(\refeq{fom:thm_latter_element_is_ordinal_5})と併せて
				\begin{align}
					y \subset \alpha \cup \{\alpha\}
				\end{align}
				が成立する.$y$の任意性ゆえに(\refeq{fom:thm_latter_element_is_ordinal_3})の下で
				\begin{align}
					\forall y\, \left(\, y \in x \Longrightarrow y \subset \alpha \cup \{\alpha\}\, \right)
				\end{align}
				が成り立ち,演繹法則と$x$の任意性から
				\begin{align}
					\forall x\, \left(\, x \in \alpha \cup \{\alpha\} \Longrightarrow x \subset \alpha \cup \{\alpha\}\, \right)
				\end{align}
				が従う.ゆえにいま
				\begin{align}
					\tran{\alpha \cup \{\alpha\}}
					\label{fom:thm_latter_element_is_ordinal_4}
				\end{align}
				が得られた.また$s$と$t$を$\alpha \cup \{\alpha\}$の任意の要素とすると
				\begin{align}
					s \in \alpha \vee s = \alpha
				\end{align}
				と
				\begin{align}
					t \in \alpha \vee t = \alpha
				\end{align}
				が成り立つが,
				\begin{align}
					s \in \alpha \Longrightarrow s \in \ON
				\end{align}
				かつ
				\begin{align}
					s = \alpha \Longrightarrow s \in \ON
				\end{align}
				から
				\begin{align}
					s \in \alpha \vee s = \alpha \Longrightarrow s \in \ON
				\end{align}
				が従い,同様にして
				\begin{align}
					t \in \alpha \vee t = \alpha \Longrightarrow t \in \ON
				\end{align}
				も成り立つので,
				\begin{align}
					s \in \ON
				\end{align}
				かつ
				\begin{align}
					t \in \ON
				\end{align}
				となる.このとき定理\ref{thm:On_is_wellordered}より
				\begin{align}
					s \in t \vee s = t \vee t \in s
				\end{align}
				が成り立つので,$s$および$t$の任意性より
				\begin{align}
					\forall s,t \in \alpha \cup \{\alpha\}\,
					\left(\, s \in t \vee s = t \vee t \in s\, \right)
				\end{align}
				が得られた.(\refeq{fom:thm_latter_element_is_ordinal_4})と併せて
				\begin{align}
					\ord{\alpha \cup \{\alpha\}}
				\end{align}
				が従い,演繹法則より
				\begin{align}
					\alpha \in \ON \Longrightarrow \alpha \cup \{\alpha\} \in \ON
				\end{align}
				を得る.
				
			\item[第二段]
		\end{description}
	\end{sketch}
	
	\begin{screen}
		\begin{thm}[順序数は後者が直後の数となる]
			$\alpha$を順序数とすれば,$\ON$において$\alpha \cup \{\alpha\}$は$\alpha$の直後の数である:
			\begin{align}
				\forall \alpha \in \ON\, 
				\left[\, \forall \beta \in \ON\, (\, \alpha < \beta 
				\Longrightarrow \alpha \cup \{\alpha\} \leq \beta\, )
				\, \right].
			\end{align}
		\end{thm}
	\end{screen}
	
	\begin{sketch}
		$\alpha$と$\beta$を任意に与えられた順序数とし,
		\begin{align}
			\alpha < \beta
		\end{align}
		であるとする.定理\ref{thm:element_and_proper_subset_correspond_between_ordinal_numbers}より,このとき
		\begin{align}
			\alpha \in \beta
		\end{align}
		が成り立ち,$\leq$の定義より
		\begin{align}
			\alpha \subset \beta
		\end{align}
		も成り立つ.ところで,いま$t$を任意の集合とすると
		\begin{align}
			t \in \{\alpha\} \Longrightarrow t = \alpha
		\end{align}
		かつ
		\begin{align}
			t = \alpha \Longrightarrow t \in \beta
		\end{align}
		が成り立つので,
		\begin{align}
			\{\alpha\} \subset \beta
		\end{align}
		が成り立つ.ゆえに
		\begin{align}
			\forall x\, \left(\, x \in \left\{ \alpha, \{\alpha\} \right\} \Longrightarrow x \subset \beta\, \right)
		\end{align}
		が成り立つ.ゆえに定理\ref{thm:union_of_subsets_is_subclass}より
		\begin{align}
			\alpha \cup \{\alpha\} \subset \beta.
		\end{align}
		すなわち
		\begin{align}
			\alpha \cup \{\alpha\} \leq \beta
		\end{align}
		が成り立つ.
		\QED
	\end{sketch}
	\begin{screen}
		\begin{dfn}[推移的クラス]
			$\mathcal{L}$の項$x$に対して,$x$が{\bf 推移的}\index{すいいてき@推移的}
			{\bf (transitive)}であるということを
			\begin{align}
				\tran{x} \defarrow
				\forall s\, (\, s \in x \rarrow s \subset x\, )
			\end{align}
			で定める.
		\end{dfn}
	\end{screen}
	
	$x$が推移的であるとは,「$x$の要素の要素が$x$の要素となる」という意味である.
	
	\begin{screen}
		\begin{dfn}[順序数]
			$\mathcal{L}$の項$x$に対して
			\begin{align}
				\ord{x} \defarrow \tran{x} \wedge 
				\forall t,u \in x\, (\, t \in u \vee t = u \vee u \in t\, )
			\end{align}
			と定め(ただし$t \in u \vee t = u \vee u \in t$は
			$(\, t \in u \vee t = u\, ) \vee u \in t$の略記とする),
			\begin{align}
				\ON \defeq \Set{x}{\ord{x}}
			\end{align}
			とおく.$\ON$の要素を{\bf 順序数}\index{じゅんじょすう@順序数}
			{\bf (ordinal number)}と呼ぶ.
		\end{dfn}
	\end{screen}
	
	空虚な真の一例であるが,例えば$0$は順序数の性質を満たす.
	ここに一つの順序数が得られたが,いま仮に$\alpha$を順序数とすれば
	\begin{align}
		\alpha \cup \{\alpha\}
	\end{align}
	もまた順序数となることが直ちに判明する.数字の定め方から
	\begin{align}
		1 &= 0 \cup \{0\}, \\
		2 &= 1 \cup \{1\}, \\
		3 &= 2 \cup \{2\}, \\
		&\vdots
	\end{align}
	が成り立つから,数字は全て順序数である.
	
	いま関係を
	\begin{align}
		\leq\ \defeq \Set{x}{\exists \alpha,\beta\, 
		(\, x=(\alpha,\beta) \wedge \alpha \subset \beta\, )}
	\end{align}
	と定める.そして
	\begin{align}
		x \leq y &\defarrow (x,y) \in\ \leq, \\
		x < y &\defarrow x \leq y \wedge x \neq y
	\end{align}
	と書く(中置記法).
	
	以下順序数の性質を列挙するが,長いので主張だけ先に述べておく.
	\begin{itemize}
		\item $\ON$は推移的クラスである.
		\item $\ON$上で$\in$と$<$は同義になる.
		\item $\leq$は$\ON$において整列順序となる.
		%\item $a$を$a \subset \ON$なる集合とすると,$\bigcup a$は$a$の$\leq$に関する上限となる.
		\item $\ON$は集合ではない.
	\end{itemize}
	
	\begin{screen}
		\begin{thm}[推移的で$\in$が全順序となるクラスは$\ON$に含まれる]
		\label{thm:transitive_totally_ordered_class}
			$a$をクラスとするとき
			\begin{align}
				\EXTAX,\EQAX,\COMAX,\PAIAX,\UNIAX,\REGAX \vdash 
				\ord{a} \rarrow a \subset \ON.
			\end{align}
		\end{thm}
	\end{screen}
	
	\begin{sketch}
		いま
		\begin{align}
			\chi \defeq \varepsilon x \negation 
			(\, x \in a \rarrow x \in \ON\, )
		\end{align}
		とおく.
		\begin{description}
			\item[step1] まず
				\begin{align}
					\chi \in a,\ \ord{a} \vdash 
					\forall s,t \in \chi\, (\, s \in t \vee s = t \vee t \in s\, )
					\label{fom:thm_transitive_totally_ordered_class_1}
				\end{align}
				を示す.$a$の推移性より
				\begin{align}
					\chi \in a,\ \ord{a} \vdash \chi \subset a
				\end{align}
				が成り立つから,
				\begin{align}
					\sigma &\defeq \varepsilon s \negation 
					(\, s \in \chi \rarrow \forall t\, (\, t \in \chi \rarrow 
					(\, s \in t \vee s = t \vee t \in s\, )\, )\, ), \\
					\tau &\defeq \varepsilon t \negation (\, t \in \chi \rarrow 
					(\, \sigma \in t \vee \sigma = t \vee t \in \sigma\, )\, )
				\end{align}
				とおけば,
				\begin{align}
					\tau \in \chi,\ \sigma \in \chi,\ \chi \in a,\ \ord{a} 
					&\vdash \sigma \in a, 
					\label{fom:thm_transitive_totally_ordered_class_2} \\
					\tau \in \chi,\ \sigma \in \chi,\ \chi \in a,\ \ord{a} 
					&\vdash \tau \in a
					\label{fom:thm_transitive_totally_ordered_class_3}
				\end{align}
				となる.他方で$\ord{a}$の定義式より
				\begin{align}
					\ord{a} \vdash 
					\forall s\, (\, s \in a \rarrow 
					\forall t\, (\, t \in a \rarrow (\, \sigma \in t \vee \sigma = t \vee t \in \sigma\, )\, )\, )
					\label{fom:thm_transitive_totally_ordered_class_4}
				\end{align}
				が成り立つので,全称記号の論理的公理および
				(\refeq{fom:thm_transitive_totally_ordered_class_2})と
				(\refeq{fom:thm_transitive_totally_ordered_class_3})との三段論法により
				\begin{align}
					\tau \in \chi,\ \sigma \in \chi,\ \chi \in a,\ \ord{a} \vdash 
					\sigma \in \tau \vee \sigma = \tau \vee \tau \in \sigma
				\end{align}
				が従う.演繹定理より
				\begin{align}
					\sigma \in \chi,\ \chi \in a,\ \ord{a} \vdash 
					\tau \in \chi \rarrow (\, \sigma \in \tau \vee \sigma = \tau \vee \tau \in \sigma\, )
				\end{align}
				が成り立つので,全称の導出
				(論理的定理\ref{logicalthm:derivation_of_universal_by_epsilon})より
				\begin{align}
					\sigma \in \chi,\ \chi \in a,\ \ord{a} \vdash 
					\forall t\, (\, t \in \chi \rarrow (\, \sigma \in t \vee \sigma = t \vee t \in \sigma\, )
				\end{align}
				となり,再び演繹定理と全称の導出によって
				\begin{align}
					\chi \in a,\ \ord{a} \vdash 
					\forall s\, (\, s \in \chi \rarrow \forall t\, (\, t \in \chi \rarrow (\, s \in t \vee s = t \vee t \in s\, )\, )
				\end{align}
				が得られる.
				
			\item[step2] 次に
				\begin{align}
					\chi \in a,\ \ord{a},\ 
					\EXTAX,\EQAX,\COMAX,\PAIAX,\UNIAX,\REGAX \vdash \tran{\chi}
				\end{align}
				を示す.いま
				\begin{align}
					\eta &\defeq \varepsilon y \negation (\, y \in \chi \rarrow y \subset \chi\, ), \\
					\zeta &\defeq \varepsilon z \negation (\, z \in \eta \rarrow z \in \chi\, )
				\end{align}
				とおく.
				\begin{description}
					\item[step2-1]
						この段では
						\begin{align}
							\zeta \in \eta,\ \eta \in \chi,\ \chi \in a,\ \ord{a} 
							\vdash 
							\zeta \in \chi \vee \zeta = \chi \vee \chi \in \zeta
						\end{align}
						を示す.$a$の推移性より
						\begin{align}
							\chi \in a,\ \ord{a} \vdash \chi \subset a
						\end{align}
						が成り立つので,
						\begin{align}
							\eta \in \chi,\ \chi \in a,\ \ord{a} \vdash \eta \in a
						\end{align}
						が成り立ち,再び$a$の推移性より
						\begin{align}
							\eta \in \chi,\ \chi \in a,\ \ord{a} \vdash 
							\eta \subset a
						\end{align}
						となる.従って
						\begin{align}
							\zeta \in \eta,\ \eta \in \chi,\ \chi \in a,\ \ord{a} 
							\vdash \zeta \in a
						\end{align}
						も成り立ち,
						(\refeq{fom:thm_transitive_totally_ordered_class_4})と
						全称記号の論理的公理より
						\begin{align}
							\zeta \in \eta,\ \eta \in \chi,\ \chi \in a,\ \ord{a} 
							\vdash 
							\zeta \in \chi \vee \zeta = \chi \vee \chi \in \zeta
							\label{fom:thm_transitive_totally_ordered_class_5}
						\end{align}
						が得られる.
						
					\item[step2-2] この段では
						\begin{align}
							\zeta \in \eta,\ \eta \in \chi,\ 
							\EXTAX,\EQAX,\COMAX,\PAIAX,\REGAX 
							\vdash \zeta \neq \chi
						\end{align}
						を示す.定理\ref{thm:no_pair_of_sets_go_round}
						(所属関係で堂々巡りしない)より
						\begin{align}
							\EXTAX,\EQAX,\COMAX,\PAIAX,\REGAX \vdash 
							\zeta \in \eta \rarrow \eta \notin \zeta
						\end{align}
						が成り立つので,演繹定理の逆より
						\begin{align}
							\zeta \in \eta,\ \EXTAX,\EQAX,\COMAX,\PAIAX,\REGAX 
							\vdash \eta \notin \zeta
						\end{align}
						となる.他方で
						\begin{align}
							\EQAX \vdash \eta \notin \zeta 
							\rarrow \zeta \neq \chi \vee \eta \notin \chi
						\end{align}
						が成り立つので($\zeta = \chi \wedge \eta \in \chi 
						\rarrow \eta \in \zeta$の対偶を取ってDe Morganの法則)
						\begin{align}
							\zeta \in \eta,\ \EXTAX,\EQAX,\COMAX,\PAIAX,\REGAX 
							\vdash \zeta \neq \chi \vee \eta \notin \chi
						\end{align}
						が従い,論理的定理\ref{logicalthm:disjunction_of_negation_rewritable_by_implication}
						(含意の論理和は否定で書ける)より
						\begin{align}
							\zeta \in \eta,\ \EXTAX,\EQAX,\COMAX,\PAIAX,\REGAX 
							\vdash \eta \in \chi
							\rarrow \zeta \neq \chi
						\end{align}
						となる.そして演繹定理の逆より
						\begin{align}
							\zeta \in \eta,\ \eta \in \chi,\ 
							\EXTAX,\EQAX,\COMAX,\PAIAX,\REGAX 
							\vdash \zeta \neq \chi
							\label{fom:thm_transitive_totally_ordered_class_6}
						\end{align}
						が得られる.
						
					\item[step2-3] この段では
						\begin{align}
							\zeta \in \eta,\ \eta \in \chi,\ \chi \in a,\ \ord{a},\ 
							\EXTAX,\EQAX,\COMAX,\PAIAX,\UNIAX,\REGAX \vdash 
							\zeta \in \chi
						\end{align}
						を示す.定理\ref{thm:no_three_sets_go_round}
						(所属関係で堂々巡りしない)より
						\begin{align}
							\EXTAX,\EQAX,\COMAX,\PAIAX,\UNIAX,\REGAX \vdash 
							\zeta \in \eta \wedge \eta \in \chi \rarrow 
							\chi \notin \zeta
						\end{align}
						が成り立つので,
						\begin{align}
							\zeta \in \eta,\ \eta \in \chi,\ 
							\EXTAX,\EQAX,\COMAX,\PAIAX,\UNIAX,\REGAX \vdash 
							\chi \notin \zeta
						\end{align}
						が従う.ここで
						(\refeq{fom:thm_transitive_totally_ordered_class_6})
						と論理積の導入より
						\begin{align}
							\zeta \in \eta,\ \eta \in \chi,\ 
							\EXTAX,\EQAX,\COMAX,\PAIAX,\UNIAX,\REGAX \vdash 
							\zeta \neq \chi \wedge \chi \notin \zeta
						\end{align}
						となり,De Morganの法則
						(論理的定理\ref{logicalthm:weak_De_Morgan_law_1})より
						\begin{align}
							\zeta \in \eta,\ \eta \in \chi,\ 
							\EXTAX,\EQAX,\COMAX,\PAIAX,\UNIAX,\REGAX \vdash\ 
							\negation (\, \zeta = \chi \vee \chi \in \zeta\, )
							\label{fom:thm_transitive_totally_ordered_class_7}
						\end{align}
						が成り立つ.ところで
						(\refeq{fom:thm_transitive_totally_ordered_class_5})と
						論理和の結合律
						(論理的定理\ref{logicalthm:associative_law_of_disjunctions})
						より
						\begin{align}
							\zeta \in \eta,\ \eta \in \chi,\ \chi \in a,\ \ord{a} 
							\vdash 
							\zeta \in \chi \vee (\, \zeta = \chi \vee \chi \in \zeta\, )
						\end{align}
						が成り立つので,
						(\refeq{fom:thm_transitive_totally_ordered_class_7})と
						選言三段論法
						(論理的定理\ref{logicalthm:disjunctive_syllogism})より
						\begin{align}
							\zeta \in \eta,\ \eta \in \chi,\ \chi \in a,\ \ord{a},\ 
							\EXTAX,\EQAX,\COMAX,\PAIAX,\UNIAX,\REGAX \vdash 
							\zeta \in \chi
							\label{fom:thm_transitive_totally_ordered_class_8}
						\end{align}
						が出る.
				\end{description}
				(\refeq{fom:thm_transitive_totally_ordered_class_8})と演繹定理より
				\begin{align}
					\eta \in \chi,\ \chi \in a,\ \ord{a},\ 
					\EXTAX,\EQAX,\COMAX,\PAIAX,\UNIAX,\REGAX \vdash 
					\zeta \in \eta \rarrow \zeta \in \chi
				\end{align}
				が成り立つので,全称の導出
				(論理的定理\ref{logicalthm:derivation_of_universal_by_epsilon})より
				\begin{align}
					\eta \in \chi,\ \chi \in a,\ \ord{a},\ 
					\EXTAX,\EQAX,\COMAX,\PAIAX,\UNIAX,\REGAX \vdash 
					\eta \subset \chi
				\end{align}
				が得られ,再び演繹定理と全称の導出により
				\begin{align}
					\chi \in a,\ \ord{a},\ 
					\EXTAX,\EQAX,\COMAX,\PAIAX,\UNIAX,\REGAX \vdash 
					\forall y\, (\, y \in \chi \rarrow y \subset \chi\, )
				\end{align}
				が得られる.すなわち
				\begin{align}
					\chi \in a,\ \ord{a},\ 
					\EXTAX,\EQAX,\COMAX,\PAIAX,\UNIAX,\REGAX \vdash \tran{\chi}
				\end{align}
				が成立する.
				
			\item[step3] step1 と step2 の結果を併せれば
				\begin{align}
					\chi \in a,\ \ord{a},\ 
					\EXTAX,\EQAX,\COMAX,\PAIAX,\UNIAX,\REGAX \vdash \ord{\chi}
				\end{align}
				が成り立つので,演繹定理より
				\begin{align}
					\ord{a},\ \EXTAX,\EQAX,\COMAX,\PAIAX,\UNIAX,\REGAX \vdash 
					\chi \in a \rarrow \chi \in \ON
				\end{align}
				となり,全称の導出
				(論理的定理\ref{logicalthm:derivation_of_universal_by_epsilon})より
				\begin{align}
					\ord{a},\ \EXTAX,\EQAX,\COMAX,\PAIAX,\UNIAX,\REGAX \vdash 
					a \subset \ON
				\end{align}
				が出る.
				\QED
		\end{description}
	\end{sketch}
	
	\begin{screen}
		\begin{thm}[$\ON$は推移的]\label{thm:On_is_transitive}
			\begin{align}
				\EXTAX,\EQAX,\COMAX,\PAIAX,\UNIAX,\REGAX \vdash \tran{\ON}.
			\end{align}
		\end{thm}
	\end{screen}
	
	\begin{prf}
		いま
		\begin{align}
			\chi \defeq \varepsilon x \negation 
			(\, x \in \ON \rarrow x \subset \ON\, )
		\end{align}
		とおけば,定理\ref{thm:transitive_totally_ordered_class}より
		\begin{align}
			\EXTAX,\EQAX,\COMAX,\PAIAX,\UNIAX,\REGAX \vdash 
			\chi \in \ON \rarrow \chi \subset \ON
		\end{align}
		が成り立つので,全称の導出
		(論理的定理\ref{logicalthm:derivation_of_universal_by_epsilon})より
		\begin{align}
			\EXTAX,\EQAX,\COMAX,\PAIAX,\UNIAX,\REGAX \vdash 
			\forall x\, (\, x \in \ON \rarrow x \subset \ON\, )
		\end{align}
		が従う.
		\QED
	\end{prf}
	
	\begin{screen}
		\begin{dfn}[クラスの差]
			$x$と$y$を$\mathcal{L}$の項とするとき,
			\begin{align}
				x \backslash y \defeq \Set{z}{z \in x \wedge z \notin y}
			\end{align}
			と定める.この$x \backslash y$を$x$と$y$の{\bf 差}\index{さ@差}{\bf (difference)}と呼び,
			$x$と$y$が集合であれば$x \backslash y$を{\bf 差集合}\index{さしゅうごう@差集合}{\bf (set difference)}と呼ぶ.
		\end{dfn}
	\end{screen}
	
	$a$と$b$をクラスとするとき,任意の主要$\varepsilon$項$\tau$に対して
	\begin{align}
		\EXTAX,\EQAX,\COMAX,\ELEAX \vdash \tau \in b \backslash a \lrarrow \tau \in b \wedge \tau \notin a
	\end{align}
	が成り立つので(注意\ref{rem:epsilon_terms_of_not_L_epsilon_formula}),
	\begin{align}
		\EXTAX,\EQAX,\COMAX,\ELEAX \vdash b \backslash a \subset b
	\end{align}
	は常に満たされる.
	
	\begin{screen}
		\begin{thm}[差集合は集合]\label{thm:set_difference_is_set}
			$a$と$b$を主要$\varepsilon$項とするとき
			\begin{align}
				\EXTAX,\EQAX,\COMAX,\REPAX \vdash \set{b \backslash a}.
			\end{align}
		\end{thm}
	\end{screen}
	
	\begin{sketch}
		分出定理(定理\ref{thm:axiom_of_separation})より
		\begin{align}
			\EXTAX,\EQAX,\REPAX \vdash \exists s\, \forall x\, (\, x \in s \lrarrow x \in b \wedge (\, x \in b \wedge x \notin a\, )\, )
		\end{align}
		が成り立つ.ここで
		\begin{align}
			\sigma &\defeq \varepsilon s\, \forall x\, (\, x \in s \lrarrow x \in b \wedge (\, x \in b \wedge x \notin a\, )\, ), \\
			\tau &\defeq \varepsilon x\, (\, x \in \sigma \lrarrow x \in b \backslash a\, )
		\end{align}
		とおけば,
		\begin{align}
			\EXTAX,\EQAX,\REPAX \vdash \tau \in \sigma \lrarrow \tau \in b \wedge (\, \tau \in b \wedge \tau \notin a\, )
		\end{align}
		が成り立つ.ところで
		\begin{align}
			\vdash \tau \in b \wedge (\, \tau \in b \wedge \tau \notin a\, ) \lrarrow \tau \in b \wedge \tau \notin a
		\end{align}
		であるから,同値関係の推移律(論理的定理\ref{logicalthm:transitive_law_of_equivalence_symbol})より
		\begin{align}
			\EXTAX,\EQAX,\REPAX \vdash \tau \in \sigma \lrarrow \tau \in b \wedge \tau \notin a
		\end{align}
		が従う.また
		\begin{align}
			\COMAX \vdash \tau \in b \wedge \tau \notin a \lrarrow \tau \in b \backslash a
		\end{align}
		も成り立つので,再び同値関係の推移律によって
		\begin{align}
			\EXTAX,\EQAX,\COMAX,\REPAX \vdash \tau \in \sigma \lrarrow \tau \in b \backslash a
		\end{align}
		となり,全称の導出(論理的定理\ref{logicalthm:derivation_of_universal_by_epsilon})より
		\begin{align}
			\EXTAX,\EQAX,\COMAX,\REPAX \vdash \forall x\, (\, x \in \sigma \lrarrow x \in b \backslash a\, )
		\end{align}
		が従い,外延性公理と相等性公理(等号の対称性)より
		\begin{align}
			\EXTAX,\EQAX,\COMAX,\REPAX \vdash b \backslash a = \sigma
		\end{align}
		が出る.従って存在記号の論理的公理より
		\begin{align}
			\EXTAX,\EQAX,\COMAX,\REPAX \vdash \exists s\, (\, b \backslash a = s\, )
		\end{align}
		が得られる.
		\QED
	\end{sketch}
	
	\begin{screen}
		\begin{thm}[$\ON$において$\in$と$<$は同義]
		\label{thm:element_and_proper_subset_correspond}
			\begin{align}
				\EXTAX,\EQAX,\COMAX,\ELEAX,\REPAX,\PAIAX,\REGAX 
				\vdash \forall \alpha,\beta \in \ON\, (\, \alpha \in \beta \lrarrow \alpha < \beta\, ).
			\end{align}
		\end{thm}
	\end{screen}
	
	\begin{prf}
		いま
		\begin{align}
			a &\defeq \varepsilon \alpha \negation 
			(\, \alpha \in \ON \rarrow \forall \beta\, (\, \beta \in \ON \rarrow 
			(\, \alpha \in \beta \lrarrow \alpha < \beta\, )\,) \,), \\
			b &\defeq \varepsilon \beta \negation (\, \beta \in \ON \rarrow 
			(\, a \in \beta \lrarrow a < \beta\, )\,)
		\end{align}
		とおく.
		\begin{description}
			\item[step1] まず
				\begin{align}
					\ord{b},\ \EXTAX,\EQAX,\COMAX,\ELEAX,\PAIAX,\REGAX \vdash 
					a \in b \rarrow a < b
				\end{align}
				を示す.定理\ref{thm:critical_epsilon_term_is_set} (主要$\varepsilon$項は集合)と
				定理\ref{thm:ordered_pair_of_sets_is_a_set} (集合の順序対は集合)より
				\begin{align}
					\EXTAX,\EQAX,\COMAX,\PAIAX \vdash \set{(a,b)}
				\end{align}
				が成り立つので,
				\begin{align}
					\tau \defeq \varepsilon x\, (\, (a,b) = x\, )
				\end{align}
				とおけば
				\begin{align}
					\EXTAX,\EQAX,\COMAX,\PAIAX \vdash (a,b) = \tau
					\label{fom:element_and_proper_subset_correspond_1}
				\end{align}
				となる.ところで順序数の推移性より
				\begin{align}
					a \in b,\ \ord{b} \vdash a \subset b
				\end{align}
				が成り立つから,
				\begin{align}
					a \in b,\ \ord{b},\ \EXTAX,\EQAX,\COMAX,\PAIAX 
					\vdash \tau = (a,b) \wedge a \subset b
				\end{align}
				となり,存在記号の論理的公理より
				\begin{align}
					a \in b,\ \ord{b},\ \EXTAX,\EQAX,\COMAX,\PAIAX 
					\vdash \exists \alpha\, \exists \beta\, (\, \tau = (\alpha,\beta) \wedge \alpha \subset \beta\, )
				\end{align}
				が従う.よって注意\ref{rem:epsilon_terms_of_not_L_epsilon_formula}より
				\begin{align}
					a \in b,\ \ord{b},\ \EXTAX,\EQAX,\COMAX,\ELEAX,\PAIAX \vdash \tau \in\ \leq
					\label{fom:element_and_proper_subset_correspond_2}
				\end{align}
				となり,
				%\begin{align}
				%	a \in b,\ \ord{b},\ \EQAX,\COMAX 
				%	\vdash a \leq b
				%\end{align}
				%ところで定理\ref{thm:transitive_totally_ordered_class}より
				%\begin{align}
				%	\ord{b},\ \EXTAX,\EQAX,\COMAX,\PAIAX,\UNIAX,\REGAX \vdash 
				%	b \subset \ON
				%\end{align}
				%が成り立つので,
				%\begin{align}
				%	a \in b,\ \ord{b},\ \EXTAX,\EQAX,\COMAX,\PAIAX,\UNIAX,\REGAX 
				%	\vdash a \in \ON
				%\end{align}
				%となり,(\refeq{fom:element_and_proper_subset_correspond_2})と
				%論理積の導入より
				%\begin{align}
				%	&a \in b,\ \ord{b},\ \EXTAX,\EQAX,\COMAX,\PAIAX,\UNIAX,\REGAX \\
				%	&\vdash a \in \ON \wedge \exists \beta\, (\, \beta \in \ON \wedge 
				%	(\, \tau = (a,\beta) \wedge a \subset \beta\, )\, )
				%\end{align}
				%が成り立ち,存在記号の論理的公理より
				%\begin{align}
				%	&a \in b,\ \ord{b},\ \EXTAX,\EQAX,\COMAX,\PAIAX,\UNIAX,\REGAX \\
				%	&\vdash \exists \alpha\, (\, \alpha \in \ON \wedge 
				%	\exists \beta\, (\, \beta \in \ON \wedge 
				%	(\, \tau = (\alpha,\beta) \wedge \alpha \subset \beta\, )\, )\, )
				%\end{align}
				%が成り立つ.ゆえに
				%\begin{align}
				%	a \in b,\ \ord{b},\ \EXTAX,\EQAX,\COMAX,\PAIAX,\UNIAX,\REGAX 
				%	\vdash \tau \in\ \leq
				%\end{align}
				%となり,
				(\refeq{fom:element_and_proper_subset_correspond_1})と相等性公理より
				\begin{align}
					a \in b,\ \ord{b},\ \EXTAX,\EQAX,\COMAX,\ELEAX,\PAIAX 
					\vdash a \leq b
					\label{fom:element_and_proper_subset_correspond_3}
				\end{align}
				が従う.他方で
				\begin{align}
					a \in b,\ \EQAX \vdash a \notin a \rarrow a \neq b
				\end{align}
				が成り立つので,定理\ref{thm:no_class_contains_itself}
				(自分自身は要素ではない)と併せて
				\begin{align}
					a \in b,\ \EXTAX,\EQAX,\COMAX,\PAIAX,\REGAX \vdash 
					a \neq b
					\label{fom:element_and_proper_subset_correspond_4}
				\end{align}
				が従う.(\refeq{fom:element_and_proper_subset_correspond_3})と
				(\refeq{fom:element_and_proper_subset_correspond_4})と論理積の導入より
				\begin{align}
					a \in b,\ \ord{b},\ \EXTAX,\EQAX,\COMAX,\ELEAX,\PAIAX,\REGAX 
					\vdash a \leq b \wedge a \neq b
				\end{align}
				となるので,
				\begin{align}
					a \in b,\ \ord{b},\ \EXTAX,\EQAX,\COMAX,\ELEAX,\PAIAX,\REGAX \vdash a < b
					\label{fom:element_and_proper_subset_correspond_21}
				\end{align}
				が得られる.
				
			\item[step2]
				外延性公理と対偶律1 
				(論理的定理\ref{logicalthm:introduction_of_contraposition})より
				\begin{align}
					a \neq b,\ \EXTAX \vdash\ \negation\forall x\, (\, x \in a \lrarrow x \in b\, )
				\end{align}
				となり,量化子の論理的公理より
				\begin{align}
					a \neq b,\ \EXTAX \vdash \exists x \negation (\, x \in a \lrarrow x \in b\, )
				\end{align}
				が成り立つので,
				\begin{align}
					\tau \defeq \varepsilon x \negation (\, x \in a \lrarrow x \in b\, )
				\end{align}
				とおけば,存在記号の論理的公理より
				\begin{align}
					a \neq b,\ \EXTAX \vdash\ \negation (\, \tau \in a \lrarrow \tau \in b\, )
				\end{align}
				が成り立ち,De Morganの法則(論理的定理\ref{logicalthm:strong_De_Morgan_law_2})より
				\begin{align}
					a \neq b,\ \EXTAX \vdash\ \negation (\, \tau \in a \rarrow \tau \in b\, ) \vee
					\negation (\, \tau \in b \rarrow \tau \in a\, )
				\end{align}
				が従う.よって論理的定理\ref{logicalthm:disjunction_of_negation_rewritable_by_implication}
				(否定の論理和は含意で書ける)より
				\begin{align}
					a \neq b,\ \EXTAX \vdash (\, \tau \in a \rarrow \tau \in b\, )
					\rarrow\ \negation (\, \tau \in b \rarrow \tau \in a\, )
					\label{fom:element_and_proper_subset_correspond_22}
				\end{align}
				が成り立つ.他方で
				\begin{align}
					a < b \vdash a \neq b
				\end{align}
				であるから,(\refeq{fom:element_and_proper_subset_correspond_22})より
				\begin{align}
					a < b,\ \EXTAX \vdash (\, \tau \in a \rarrow \tau \in b\, )
					\rarrow\ \negation (\, \tau \in b \rarrow \tau \in a\, )
					\label{fom:element_and_proper_subset_correspond_6}
				\end{align}
				が成り立ち,また
				\begin{align}
					a < b \vdash \tau \in a \rarrow \tau \in b
				\end{align}
				も成り立つので,(\refeq{fom:element_and_proper_subset_correspond_6})との三段論法より
				\begin{align}
					a < b,\ \EXTAX \vdash\ \negation (\, \tau \in b \rarrow \tau \in a\, )
					\label{fom:element_and_proper_subset_correspond_7}
				\end{align}
				となる.ところで論理的定理\ref{logicalthm:disjunction_of_negation_rewritable_by_implication}
				(否定の論理和は含意で書ける)と対偶律1 (論理的定理\ref{logicalthm:introduction_of_contraposition})より
				\begin{align}
					\vdash\ \negation (\, \tau \in b \rarrow \tau \in a\, ) \rarrow\ 
					\negation (\, \tau \notin b \vee \tau \in a\, )
				\end{align}
				が成り立つので,(\refeq{fom:element_and_proper_subset_correspond_7})との三段論法より
				\begin{align}
					a < b,\ \EXTAX \vdash\ \negation (\, \tau \notin b \vee \tau \in a\, )
				\end{align}
				が従い,De Morganの法則(論理的定理\ref{logicalthm:weak_De_Morgan_law_2})と二重否定の除去より
				\begin{align}
					a < b,\ \EXTAX \vdash \tau \in b \wedge \tau \notin a
				\end{align}
				が従う.そして
				\begin{align}
					a < b,\ \EXTAX,\COMAX \vdash \tau \in b \backslash a
					\label{fom:element_and_proper_subset_correspond_5}
				\end{align}
				が得られる.
				
			\item[step3] 定理\ref{thm:set_difference_is_set}より
				\begin{align}
					\EXTAX,\EQAX,\COMAX,\REPAX \vdash \set{b \backslash a}
				\end{align}
				が成り立つので,
				\begin{align}
					\sigma \defeq \varepsilon s\, (\, b \backslash a = s\, )
				\end{align}
				とおけば
				\begin{align}
					\EXTAX,\EQAX,\COMAX,\REPAX \vdash b \backslash a = \sigma
					\label{fom:element_and_proper_subset_correspond_8}
				\end{align}
				となる.(\refeq{fom:element_and_proper_subset_correspond_5})と併せて
				\begin{align}
					a < b,\ \EXTAX,\EQAX,\COMAX,\REPAX \vdash \tau \in \sigma
				\end{align}
				となり,存在記号の論理的公理より
				\begin{align}
					a < b,\ \EXTAX,\EQAX,\COMAX,\REPAX \vdash \exists x\, (\, x \in \sigma\, )
				\end{align}
				となるが,正則性公理より
				\begin{align}
					\REGAX \vdash \exists x\, (\, x \in \sigma\, ) 
					\rarrow \exists y\, (\, y \in \sigma \wedge \forall z\, (\, z \in \sigma \rarrow z \notin y\, )\, )
				\end{align}
				が成り立つので
				\begin{align}
					a < b,\ \EXTAX,\EQAX,\COMAX,\REPAX,\REGAX \vdash 
					\exists y\, (\, y \in \sigma \wedge \forall z\, (\, z \in \sigma \rarrow z \notin y\, )\, )
				\end{align}
				が従う.ここで
				\begin{align}
					\eta \defeq \varepsilon y\, (\, y \in \sigma \wedge 
					\forall z\, (\, z \in \sigma \rarrow z \notin y\, )\, )
				\end{align}
				とおけば
				\begin{align}
					a < b,\ \EXTAX,\EQAX,\COMAX,\REPAX,\REGAX &\vdash \eta \in \sigma, 
					\label{fom:element_and_proper_subset_correspond_9} \\
					a < b,\ \EXTAX,\EQAX,\COMAX,\REPAX,\REGAX &\vdash 
					\forall z\, (\, z \in \sigma \rarrow z \notin \eta\, )
					\label{fom:element_and_proper_subset_correspond_10}
				\end{align}
				が成り立つ.正則性公理の式の意味を考えれば,ここで取られた$\eta$は
				「$b \backslash a$の要素であり,$b \backslash a$とは交わらない」
				という性質を持っている.$a$と$b$が順序数であれば$\eta$は$a$に等しくなる
				ことが示されるが,それは次段以降で解説する.
				
			\item[step4] この段では
				\begin{align}
					a < b,\ \ord{a},\ \ord{b},\ \EXTAX,\EQAX,\COMAX,\REPAX,\REGAX \vdash a \subset \eta 
				\end{align}
				を示す.いま
				\begin{align}
					\chi \defeq \varepsilon x \negation (\, x \in a \rarrow x \in \eta\, )
				\end{align}
				とおく.(\refeq{fom:element_and_proper_subset_correspond_8})と
				(\refeq{fom:element_and_proper_subset_correspond_9})より
				\begin{align}
					a < b,\ \EXTAX,\EQAX,\COMAX,\REPAX,\REGAX &\vdash \eta \in b, 
					\label{fom:element_and_proper_subset_correspond_11} \\
					a < b,\ \EXTAX,\EQAX,\COMAX,\REPAX,\REGAX &\vdash \eta \notin a
					\label{fom:element_and_proper_subset_correspond_12}
				\end{align}
				となり,他方で
				\begin{align}
					\chi \in a,\ a < b \vdash \chi \in b
				\end{align}
				となるから,$\ord{b}$を公理に追加すれば
				\begin{align}
					\chi \in a,\ a < b,\ \ord{b},\ \EXTAX,\EQAX,\COMAX,\REPAX,\REGAX 
					\vdash \chi \in \eta \vee \chi = \eta \vee \eta \in \chi
					\label{fom:element_and_proper_subset_correspond_13}
				\end{align}
				が成り立つ.ところで
				\begin{align}
					\EQAX \vdash \chi = \eta \wedge \chi \in a \rarrow \eta \in a
				\end{align}
				が成り立つので,対偶律1 (論理的定理\ref{logicalthm:introduction_of_contraposition})より
				\begin{align}
					\EQAX \vdash \eta \notin a \rarrow \chi \neq \eta \vee \chi \notin a
				\end{align}
				となる.また順序数の推移性より
				\begin{align}
					\ord{a} \vdash \eta \in \chi \wedge \chi \in a \rarrow \eta \in a
				\end{align}
				が成り立つので,対偶律1 (論理的定理\ref{logicalthm:introduction_of_contraposition})より
				\begin{align}
					\ord{a} \vdash \eta \notin a \rarrow \eta \notin \chi \vee \chi \notin a
				\end{align}
				となる.よって,(\refeq{fom:element_and_proper_subset_correspond_12})と併せて
				\begin{align}
					a < b,\ \EXTAX,\EQAX,\COMAX,\REPAX,\REGAX &\vdash \chi \neq \eta \vee \chi \notin a, \\
					a < b,\ \ord{a},\ \EXTAX,\EQAX,\COMAX,\REPAX,\REGAX &\vdash \eta \notin \chi \vee \chi \notin a
				\end{align}
				が成り立ち,論理的定理\ref{logicalthm:disjunction_of_negation_rewritable_by_implication}
				(否定の論理和は含意で書ける)より
				\begin{align}
					a < b,\ \EXTAX,\EQAX,\COMAX,\REPAX,\REGAX 
					&\vdash \chi \in a \rarrow \chi \neq \eta, \\
					a < b,\ \ord{a},\ \EXTAX,\EQAX,\COMAX,\REPAX,\REGAX 
					&\vdash \chi \in a \rarrow \eta \notin \chi
				\end{align}
				が成り立ち,演繹定理の逆により
				\begin{align}
					\chi \in a,\ a < b,\ \EXTAX,\EQAX,\COMAX,\REPAX,\REGAX &\vdash \chi \neq \eta,
					\label{fom:element_and_proper_subset_correspond_14} \\
					\chi \in a,\ a < b,\ \ord{a},\ \EXTAX,\EQAX,\COMAX,\REPAX,\REGAX &\vdash \eta \notin \chi
					\label{fom:element_and_proper_subset_correspond_15}
				\end{align}
				となり,論理積の導入とDe Morganの法則(論理的定理\ref{logicalthm:weak_De_Morgan_law_1})より
				\begin{align}
					\chi \in a,\ a < b,\ \ord{a},\ \EXTAX,\EQAX,\COMAX,\REPAX,\REGAX \vdash\ 
					\negation (\, \chi = \eta \vee \eta \in \chi\, )
				\end{align}
				が従う.これと(\refeq{fom:element_and_proper_subset_correspond_13})と
				選言三段論法(論理的定理\ref{logicalthm:disjunctive_syllogism})より
				\begin{align}
					\chi \in a,\ a < b,\ \ord{a},\ \ord{b},\ \EXTAX,\EQAX,\COMAX,\REPAX,\REGAX \vdash \chi \in \eta
				\end{align}
				が成立するので,演繹定理より
				\begin{align}
					a < b,\ \ord{a},\ \ord{b},\ \EXTAX,\EQAX,\COMAX,\REPAX,\REGAX \vdash 
					\chi \in a \rarrow \chi \in \eta
				\end{align}
				となり,全称の導出(論理的定理\ref{logicalthm:derivation_of_universal_by_epsilon})より
				\begin{align}
					a < b,\ \ord{a},\ \ord{b},\ \EXTAX,\EQAX,\COMAX,\REPAX,\REGAX \vdash a \subset \eta
					\label{fom:element_and_proper_subset_correspond_16}
				\end{align}
				が出る.
			
			\item[step5] この段では
				\begin{align}
					a < b,\ \ord{b},\ \EXTAX,\EQAX,\COMAX,\REPAX,\REGAX \vdash \eta \subset a
				\end{align}
				を示す.いま
				\begin{align}
					\chi \defeq \varepsilon x \negation (\, x \in \eta \rarrow x \in a\, )
				\end{align}
				とおく.(\refeq{fom:element_and_proper_subset_correspond_11})と順序数の推移性より
				\begin{align}
					\chi \in \eta,\ a < b,\ \ord{b},\ \EXTAX,\EQAX,\COMAX,\REPAX,\REGAX \vdash \chi \in b
					\label{fom:element_and_proper_subset_correspond_17}
				\end{align}
				となる.他方で(\refeq{fom:element_and_proper_subset_correspond_10})と
				対偶律2 (論理的定理\ref{logicalthm:contraposition_2})より
				\begin{align}
					a < b,\ \EXTAX,\EQAX,\COMAX,\REPAX,\REGAX \vdash 
					\chi \in \eta \rarrow \chi \notin \sigma
				\end{align}
				となり,演繹定理の逆より
				\begin{align}
					\chi \in \eta,\ a < b,\ \EXTAX,\EQAX,\COMAX,\REPAX,\REGAX \vdash \chi \notin \sigma
				\end{align}
				となるが,(\refeq{fom:element_and_proper_subset_correspond_8})より
				\begin{align}
					\chi \in \eta,\ a < b,\ \EXTAX,\EQAX,\COMAX,\REPAX,\REGAX \vdash \chi \notin b \backslash a
					\label{fom:element_and_proper_subset_correspond_18}
				\end{align}
				が従う.ところで
				\begin{align}
					\COMAX \vdash \chi \in b \backslash a \lrarrow \chi \in b \wedge \chi \notin a
				\end{align}
				が成り立つので,対偶を取って
				\begin{align}
					\COMAX \vdash \chi \notin b \backslash a \lrarrow \chi \notin b \vee \chi \in a
				\end{align}
				となるから,(\refeq{fom:element_and_proper_subset_correspond_18})より
				\begin{align}
					\chi \in \eta,\ a < b,\ \EXTAX,\EQAX,\COMAX,\REPAX,\REGAX \vdash \chi \notin b \vee \chi \in a
				\end{align}
				が従い,論理的定理\ref{logicalthm:disjunction_of_negation_rewritable_by_implication}
				(否定の論理和は含意で書ける)より
				\begin{align}
					\chi \in \eta,\ a < b,\ \EXTAX,\EQAX,\COMAX,\REPAX,\REGAX \vdash \chi \in b \rarrow \chi \in a
				\end{align}
				が成り立つ.これと(\refeq{fom:element_and_proper_subset_correspond_17})との三段論法より
				\begin{align}
					\chi \in \eta,\ a < b,\ \ord{b},\ \EXTAX,\EQAX,\COMAX,\REPAX,\REGAX \vdash \chi \in a
				\end{align}
				が成り立ち,演繹定理と全称の導出(論理的定理\ref{logicalthm:derivation_of_universal_by_epsilon})より
				\begin{align}
					a < b,\ \ord{b},\ \EXTAX,\EQAX,\COMAX,\REPAX,\REGAX \vdash \eta \subset a
					\label{fom:element_and_proper_subset_correspond_19}
				\end{align}
				が出る.
			
			\item[step6] (\refeq{fom:element_and_proper_subset_correspond_16})と
				(\refeq{fom:element_and_proper_subset_correspond_19})および
				定理\ref{thm:mutually_including_classes_are_equivalent} (互いに相手を包含するクラス同士は等しい)より
				\begin{align}
					a < b,\ \ord{a},\ \ord{b},\ \EXTAX,\EQAX,\COMAX,\REPAX,\REGAX \vdash \eta = a
				\end{align}
				が成り立ち,(\refeq{fom:element_and_proper_subset_correspond_11})と相等性公理より
				\begin{align}
					a < b,\ \ord{a},\ \ord{b},\ \EXTAX,\EQAX,\COMAX,\REPAX,\REGAX \vdash a \in b
					\label{fom:element_and_proper_subset_correspond_20}
				\end{align}
				が従う.(\refeq{fom:element_and_proper_subset_correspond_21})と
				(\refeq{fom:element_and_proper_subset_correspond_20})と演繹定理および論理積の導入より
				\begin{align}
					\ord{a},\ \EXTAX,\EQAX,\COMAX,\ELEAX,\REPAX,\PAIAX,\REGAX \vdash 
					\ord{b} \rarrow (\, a \in b \lrarrow a < b\, )
				\end{align}
				が成り立つが,ここで全称の導出(論理的定理\ref{logicalthm:derivation_of_universal_by_epsilon})より
				\begin{align}
					&\ord{a},\ \EXTAX,\EQAX,\COMAX,\ELEAX,\REPAX,\PAIAX,\REGAX \\
					&\vdash \forall \beta\, (\, \beta \in \ON \rarrow (\, a \in \beta \lrarrow a < \beta\, )\, )
				\end{align}
				が従い,同様にして
				\begin{align}
					&\EXTAX,\EQAX,\COMAX,\ELEAX,\REPAX,\PAIAX,\REGAX \\
					&\vdash \forall \alpha\, (\, \alpha \in \ON \rarrow
					\forall \beta\, (\, \beta \in \ON \rarrow (\, \alpha \in \beta \lrarrow \alpha < \beta\, )\, )\, )
				\end{align}
				が得られる.
				\QED
		\end{description}
	\end{prf}
	
	\begin{screen}
		\begin{thm}[$\leq$は$\ON$の全順序]
		\label{thm:ON_is_totally_ordered}
			\begin{align}
				\EXTAX,\EQAX,\COMAX,\ELEAX,\REPAX,\PAIAX,\REGAX \vdash 
				\forall \alpha,\beta \in \ON\,
				(\, \alpha \in \beta \vee \alpha = \beta \vee \beta \in \alpha\, ).
			\end{align}
		\end{thm}
	\end{screen}
	
	\begin{sketch}
		いま
		\begin{align}
			a &\defeq \varepsilon \alpha \negation 
			(\, \alpha \in \ON \rarrow \forall \beta\, (\, \beta \in \ON \rarrow 
			(\, \alpha \in \beta \vee \alpha = \beta \vee \beta \in \alpha\, )\,) \,), \\
			b &\defeq \varepsilon \beta \negation (\, \beta \in \ON \rarrow 
			(\, a \in \beta \vee a = \beta \vee \beta \in a\, )\,)
		\end{align}
		とおく.
		\begin{description}
			\item[step1] この段では
				\begin{align}
					\ord{a},\ \ord{b} \vdash \ord{a \cap b}
					\label{fom:ON_is_totally_ordered_1}
				\end{align}
				を示す.実際,
				\begin{align}
					\chi &\defeq \varepsilon x \negation 
					(\, x \in a \cap b \rarrow \forall y\, (\, 
					y \in a \cap b \rarrow (\, x \in y \vee x = y \vee y \in x\, )\, )\, ), \\
					\eta &\defeq \varepsilon y \negation (\, 
					y \in a \cap b \rarrow (\, \chi \in y \vee \chi = y \vee y \in \chi\, )\, )
				\end{align}
				とおけば,
				\begin{align}
					\chi \in a \cap b \vdash \chi \in a
				\end{align}
				および
				\begin{align}
					\eta \in a \cap b \vdash \eta \in a
				\end{align}
				および
				\begin{align}
					\ord{a} \vdash \chi \in a \rarrow (\, 
					\eta \in a \rarrow (\, \chi \in \eta \vee \chi = \eta \vee \eta \in \chi\, )\, )
				\end{align}
				より
				\begin{align}
					\eta \in a \cap b,\ \chi \in a \cap b,\ \ord{a} 
					\vdash \chi \in \eta \vee \chi = \eta \vee \eta \in \chi
				\end{align}
				が成り立つので,演繹定理と全称の導出
				(論理的定理\ref{logicalthm:derivation_of_universal_by_epsilon})より
				\begin{align}
					\ord{a} \vdash \forall x,y \in a \cap b\, 
					(\, x \in y \vee x = y \vee y \in x\, )
				\end{align}
				が得られる.今度は
				\begin{align}
					\chi \defeq \varepsilon x \negation 
					(\, x \in a \cap b \rarrow x \subset a \cap b\, )
				\end{align}
				とおき直せば,順序数の推移性より
				\begin{align}
					\chi \in a \cap b,\ \ord{a} &\vdash \chi \subset a, \\
					\chi \in a \cap b,\ \ord{b} &\vdash \chi \subset b
				\end{align}
				が成り立つので
				\begin{align}
					\chi \in a \cap b,\ \ord{a},\ \ord{b} \vdash \chi \subset a \cap b
				\end{align}
				となり,演繹定理と全称の導出
				(論理的定理\ref{logicalthm:derivation_of_universal_by_epsilon})より
				\begin{align}
					\ord{a},\ \ord{b} \vdash 
					\forall x\, (\, x \in a \cap b \rarrow x \subset a \cap b\, )
				\end{align}
				も得られる.
				
			\item[step2] 
				%\ELEAX \vdash a \cap b \in a \rarrow \exists x\, (\, a \cap b = x\, )
				%\tau \defeq \varepsilon x\, (\, a \cap b = x\, )
				%a \cap b \in a,\ \ELEAX \vdash a \cap b = \tau
				%a \cap b \in a,\ \ELEAX,\EQAX \vdash a \cap b = \tau \rarrow (\, a \cap b \in a \rarrow \tau \in a\, )
				%a \cap b \in a,\ \ELEAX,\EQAX \vdash \tau \in a
				%a \cap b \in b,\ \ELEAX,\EQAX \vdash \tau \in b
				%a \cap b \in a,\ a \cap b \in b,\ \ELEAX,\EQAX \vdash \tau \in a \wedge \tau \in b
				%a \cap b \in a,\ a \cap b \in b,\ \ELEAX,\EQAX,\COMAX \vdash \tau \in a \cap b
				%a \cap b \in a,\ a \cap b \in b,\ \ELEAX,\EQAX,\COMAX \vdash a \cap b \in a \cap b
				%\ELEAX,\EQAX,\COMAX \vdash a \cap b \in a \wedge a \cap b \in b \rarrow a \cap b \in a \cap b
				定理\ref{thm:no_class_contains_itself} (自分自身は要素に持たない)より
				\begin{align}
					\EXTAX,\EQAX,\COMAX,\PAIAX,\REGAX \vdash a \cap b \notin a \cap b
				\end{align}
				が成り立ち,他方で
				\begin{align}
					\EQAX,\COMAX,\ELEAX \vdash a \cap b \notin a \cap b
					\rarrow a \cap b \notin a \vee a \cap b \notin b
				\end{align}
				も成り立つので,三段論法より
				\begin{align}
					\EXTAX,\EQAX,\COMAX,\ELEAX,\PAIAX,\REGAX \vdash 
					a \cap b \notin a \vee a \cap b \notin b
					\label{fom:ON_is_totally_ordered_2}
				\end{align}
				が従う.ところで
				\begin{align}
					\COMAX \vdash a \cap b \subset a
				\end{align}
				と定理\ref{thm:element_and_proper_subset_correspond}
				および(\refeq{fom:ON_is_totally_ordered_1})より
				\begin{align}
					\ord{a},\ \ord{b},\ \EXTAX,\EQAX,\COMAX,\ELEAX,\REPAX,\PAIAX,\REGAX \vdash 
					a \cap b \in a \vee a \cap b = a
				\end{align}
				が成り立つので,選言三段論法
				(論理的定理\ref{logicalthm:disjunctive_syllogism})より
				\begin{align}
					\ord{a},\ \ord{b},\ \EXTAX,\EQAX,\COMAX,\ELEAX,\REPAX,\PAIAX,\REGAX \vdash 
					a \cap b \notin a \rarrow a \cap b = a
				\end{align}
				が成り立つ.演繹定理の逆より
				\begin{align}
					a \cap b \notin a,\ \ord{a},\ \ord{b},\ \EXTAX,\EQAX,\COMAX,\ELEAX,\REPAX,\PAIAX,\REGAX \vdash 
					a \cap b = a
				\end{align}
				となるが,ここで
				\begin{align}
					a \cap b = a,\ \COMAX \vdash a \subset b
				\end{align}
				が成り立つので
				\begin{align}
					a \cap b \notin a,\ \ord{a},\ \ord{b},\ \EXTAX,\EQAX,\COMAX,\ELEAX,\REPAX,\PAIAX,\REGAX \vdash 
					a \subset b
				\end{align}
				が従い,定理\ref{thm:element_and_proper_subset_correspond}より
				\begin{align}
					a \cap b \notin a,\ \ord{a},\ \ord{b},\ \EXTAX,\EQAX,\COMAX,\ELEAX,\REPAX,\PAIAX,\REGAX \vdash 
					a \in b \vee a = b
				\end{align}
				が成り立つ.論理和の導入より
				\begin{align}
					&a \cap b \notin a,\ \ord{a},\ \ord{b},\ \EXTAX,\EQAX,\COMAX,\ELEAX,\REPAX,\PAIAX,\REGAX \\
					&\vdash a \in b \vee a = b \vee b \in a
				\end{align}
				となり,演繹定理より
				\begin{align}
					&\ord{a},\ \ord{b},\ \EXTAX,\EQAX,\COMAX,\ELEAX,\REPAX,\PAIAX,\REGAX \\
					&\vdash a \cap b \notin a \rarrow a \in b \vee a = b \vee b \in a
				\end{align}
				が従う.同様にして
				\begin{align}
					&\ord{a},\ \ord{b},\ \EXTAX,\EQAX,\COMAX,\ELEAX,\REPAX,\PAIAX,\REGAX \\
					&\vdash a \cap b \notin b \rarrow a \in b \vee a = b \vee b \in a
				\end{align}
				も成り立つので,論理和の除去より
				\begin{align}
					&\ord{a},\ \ord{b},\ \EXTAX,\EQAX,\COMAX,\ELEAX,\REPAX,\PAIAX,\REGAX \\
					&\vdash 
					a \cap b \notin a \vee a \cap b \notin b
					\rarrow a \in b \vee a = b \vee b \in a
				\end{align}
				が成り立ち,(\refeq{fom:ON_is_totally_ordered_2})との三段論法より
				\begin{align}
					\ord{a},\ \ord{b},\ \EXTAX,\EQAX,\COMAX,\ELEAX,\REPAX,\PAIAX,\REGAX \vdash 
					\rarrow a \in b \vee a = b \vee b \in a
				\end{align}
				が従う.あとは演繹定理と全称の導出
				(論理的定理\ref{logicalthm:derivation_of_universal_by_epsilon})より
				\begin{align}
					\EXTAX,\EQAX,\COMAX,\ELEAX,\REPAX,\PAIAX,\REGAX \vdash 
					\forall \alpha,\beta \in \ON\, (\, \alpha \in \beta \vee \alpha = \beta \vee \beta \in \alpha\, )
				\end{align}
				が出る.
				\QED
		\end{description}
	\end{sketch}
	
	\begin{screen}
		\begin{thm}[Burali-Forti]\label{thm:Burali_Forti}
			順序数の全体は集合ではない.
			\begin{align}
				\EXTAX,\EQAX,\COMAX,\ELEAX,\REPAX,\PAIAX,\UNIAX,\REGAX
				\vdash\ \negation \set{\ON}.
			\end{align}
		\end{thm}
	\end{screen}
	
	\begin{prf}
		定理\ref{thm:satisfactory_set_is_an_element}より
		\begin{align}
			\EQAX,\COMAX \vdash \ord{\ON} \rarrow (\, \set{\ON} \rarrow \ON \in \ON\, )
			\label{eq:Burali_Forti_1}
		\end{align}
		が成り立つ.定理\ref{thm:On_is_transitive}と定理\ref{thm:ON_is_totally_ordered}より
		\begin{align}
			\EXTAX,\EQAX,\COMAX,\ELEAX,\REPAX,\PAIAX,\UNIAX,\REGAX
			\vdash \ord{\ON}
		\end{align}
		が成り立つから,(\refeq{eq:Burali_Forti_1})との三段論法より
		\begin{align}
			\EXTAX,\EQAX,\COMAX,\ELEAX,\REPAX,\PAIAX,\UNIAX,\REGAX
			\vdash \set{\ON} \rarrow \ON \in \ON
		\end{align}
		となり,対偶律1 (論理的定理\ref{logicalthm:introduction_of_contraposition})より
		\begin{align}
			\EXTAX,\EQAX,\COMAX,\ELEAX,\REPAX,\PAIAX,\UNIAX,\REGAX
			\vdash \ON \notin \ON \rarrow\ \negation \set{\ON}
			\label{eq:Burali_Forti_2}
		\end{align}
		が従う.他方で定理\ref{thm:no_class_contains_itself} (自分自身は要素に持たない)より
		\begin{align}
			\EXTAX,\EQAX,\COMAX,\ELEAX,\PAIAX,\REGAX \vdash \ON \notin \ON
		\end{align}
		も成り立つので,(\refeq{eq:Burali_Forti_2})との三段論法から
		\begin{align}
			\EXTAX,\EQAX,\COMAX,\ELEAX,\REPAX,\PAIAX,\UNIAX,\REGAX
			\vdash\ \negation \set{\ON}
		\end{align}
		が得られる.
		\QED
	\end{prf}
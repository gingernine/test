	\begin{screen}
		\begin{dfn}[推移的類]
			$\mathcal{L}$の項$x$に対して,$x$が{\bf 推移的}\index{すいいてき@推移的}
			{\bf (transitive)}であるということを
			\begin{align}
				\tran{x} \defarrow
				\forall s\, (\, s \in x \rarrow s \subset x\, )
			\end{align}
			で定める.
		\end{dfn}
	\end{screen}
	
	$x$が推移的であるとは,「$x$の要素の要素が$x$の要素となる」という意味である.
	
	\begin{screen}
		\begin{dfn}[順序数]
			$\mathcal{L}$の項$x$に対して
			\begin{align}
				\ord{x} \defarrow \tran{x} \wedge 
				\forall t,u \in x\, (\, t \in u \vee t = u \vee u \in t\, )
			\end{align}
			と定め,
			\begin{align}
				\ON \defeq \Set{x}{\ord{x}}
			\end{align}
			とおく.$\ON$の要素を{\bf 順序数}\index{じゅんじょすう@順序数}
			{\bf (ordinal number)}と呼ぶ.
		\end{dfn}
	\end{screen}
	
	空虚な真の一例であるが,例えば$0$は順序数の性質を満たす.
	ここに一つの順序数が得られたが,いま仮に$\alpha$を順序数とすれば
	\begin{align}
		\alpha \cup \{\alpha\}
	\end{align}
	もまた順序数となることが直ちに判明する.数字の定め方から
	\begin{align}
		1 &= 0 \cup \{0\}, \\
		2 &= 1 \cup \{1\}, \\
		3 &= 2 \cup \{2\}, \\
		&\vdots
	\end{align}
	が成り立つから,数字は全て順序数である.
	
	いま$\ON$上の関係を
	\begin{align}
		\leq\ \defeq \Set{x}{\exists \alpha,\beta \in \ON\, 
		(\, x=(\alpha,\beta) \wedge \alpha \subset \beta\, )}
	\end{align}
	と定める.そして
	\begin{align}
		x \leq y &\defarrow (x,y) \in\ \leq, \\
		x < y &\defarrow (x,y) \in\ \leq \wedge x \neq y
	\end{align}
	と書く(中置記法).
	
	以下順序数の性質を列挙するが,長いので主張だけ先に述べておく.
	\begin{itemize}
		\item $\ON$は推移的類である.
		\item $\leq$は$\ON$において整列順序となる.
		%\item $a$を$a \subset \ON$なる集合とすると,$\bigcup a$は$a$の$\leq$に関する上限となる.
		\item $\ON$は集合ではない.
	\end{itemize}
	
	\begin{screen}
		\begin{thm}[推移的で$\in$が全順序となる類は$\ON$に含まれる]
		\label{thm:transitive_totally_ordered_class_is_contained_in_ON}
			$S$を類とするとき
			\begin{align}
				\ord{S} \rarrow S \subset \ON.
			\end{align}
		\end{thm}
	\end{screen}
	
	\begin{sketch}
		$x$を$S$の要素とする.まず
		\begin{align}
			\forall s,t \in x\, (\, s \in t \vee s = t \vee t \in s\, )
			\label{fom:thm_transitive_totally_ordered_class_is_contained_in_ON_1}
		\end{align}
		が成り立つことを示す.実際$S$の推移性より
		\begin{align}
			x \subset S
		\end{align}
		が成り立つので,$x$の要素は全て$S$の要素となり
		(\refeq{fom:thm_transitive_totally_ordered_class_is_contained_in_ON_1})が満たされる.次に
		\begin{align}
			\tran{x}
		\end{align}
		が成り立つことを示す.$y$を$x$の要素とする.また$z$を$y$の要素とする.このとき
		\begin{align}
			x \subset S
		\end{align}
		から
		\begin{align}
			y \in S
		\end{align}
		が成り立つので
		\begin{align}
			y \subset S
		\end{align}
		が成り立ち,ゆえに
		\begin{align}
			z \in S
		\end{align}
		となる.従って
		\begin{align}
			z \in x \vee z = x \vee x \in z
			\label{fom:thm_transitive_totally_ordered_class_is_contained_in_ON_2}
		\end{align}
		が成立する.ところで定理\ref{thm:no_set_is_an_element_of_itself}より
		\begin{align}
			z \in y \Longrightarrow y \notin z
		\end{align}
		が成り立つから
		\begin{align}
			y \notin z
			\label{fom:thm_transitive_totally_ordered_class_is_contained_in_ON_3}
		\end{align}
		が成立する.また相当性の公理から
		\begin{align}
			z = x \vee y \in x \Longrightarrow y \in z
		\end{align}
		が成り立つので,その対偶と(\refeq{fom:thm_transitive_totally_ordered_class_is_contained_in_ON_2})から
		\begin{align}
			z \neq x \vee y \notin x
		\end{align}
		も満たされる.いま
		\begin{align}
			y \in x
		\end{align}
		が成り立っていて,さらに選言三段論法より
		\begin{align}
			(\, z \neq x \vee y \notin x\, ) \wedge y \in x \Longrightarrow z \neq x
		\end{align}
		も成り立つから,
		\begin{align}
			z \neq x
		\end{align}
		が成立する.他方で定理\ref{thm:no_set_is_an_element_of_itself}より
		\begin{align}
			z \in y \wedge y \in x \Longrightarrow x \notin z
		\end{align}
		が成立するから,ゆえにいま
		\begin{align}
			z \neq x \wedge x \notin z
		\end{align}
		が,つまり
		\begin{align}
			\rightharpoondown (\, z = x \vee x \in z\, )
			\label{fom:thm_transitive_totally_ordered_class_is_contained_in_ON_4}
		\end{align}
		が成立している.ここで選言三段論法より
		\begin{align}
			(\, z \in x \vee z = x \vee x \in z\, ) \wedge 
			\rightharpoondown (\, z = x \vee x \in z\, )
			\Longrightarrow z \in x
		\end{align}
		も成り立つので,(\refeq{fom:thm_transitive_totally_ordered_class_is_contained_in_ON_3})と
		(\refeq{fom:thm_transitive_totally_ordered_class_is_contained_in_ON_4})と併せて
		\begin{align}
			z \in x
		\end{align}
		が従う.以上より,$y$を$x$の要素とすれば
		\begin{align}
			\forall z \in y\, (\, z \in y \Longrightarrow z \in x\, )
		\end{align}
		が成り立ち,ゆえに
		\begin{align}
			y \subset x
		\end{align}
		が成り立つ.ゆえに$x$は推移的である.ゆえに
		\begin{align}
			\ord{x}
		\end{align}
		が成立し
		\begin{align}
			x \in \ON
		\end{align}
		となる.$x$の任意性より
		\begin{align}
			S \subset \ON
		\end{align}
		が得られる.
		\QED
	\end{sketch}
	
	\begin{screen}
		\begin{thm}[$\ON$は推移的]\label{thm:On_is_transitive}
			$\tran{\ON}$が成立する.
		\end{thm}
	\end{screen}
	
	\begin{prf} 
		$x$を順序数とすると
		\begin{align}
			\ord{x}
		\end{align}
		が成り立つので,定理\ref{thm:transitive_totally_ordered_class_is_contained_in_ON}から
		\begin{align}
			x \subset \ON
		\end{align}
		が成立する.ゆえに$\ON$は推移的である.
		\QED
	\end{prf}
	
	\begin{screen}
		\begin{thm}[$\ON$において$\in$と$<$は同義]
		\label{thm:element_and_proper_subset_correspond_between_ordinal_numbers}
			\begin{align}
				\forall \alpha,\beta \in \ON\,
				(\, \alpha \in \beta \Longleftrightarrow \alpha < \beta\, ).
			\end{align}
		\end{thm}
	\end{screen}
	
	\begin{prf}
		$\alpha,\beta$を任意に与えられた順序数とする.
		\begin{align}
			\alpha \in \beta
		\end{align}
		が成り立っているとすると,順序数の推移性より
		\begin{align}
			\alpha \subset \beta
		\end{align}
		が成り立つ.定理\ref{thm:no_set_is_an_element_of_itself}より
		\begin{align}
			\alpha \neq \beta
		\end{align}
		も成り立つから
		\begin{align}
			\alpha < \beta
		\end{align}
		が成り立つ.ゆえに
		\begin{align}
			\alpha \in \beta \Longrightarrow \alpha < \beta
		\end{align}
		が成立する.逆に
		\begin{align}
			\alpha < \beta
		\end{align}
		が成り立っているとすると
		\begin{align}
			\beta \backslash \alpha \neq \emptyset
		\end{align}
		が成り立つので,正則性公理より
		\begin{align}
			\gamma \in \beta \backslash \alpha \wedge \gamma \cap (\beta \backslash \alpha) = \emptyset
		\end{align}
		を満たす$\gamma$が取れる.このとき
		\begin{align}
			\alpha = \gamma
		\end{align}
		が成り立つことを示す.$x$を$\alpha$の任意の要素とすれば,
		$x,\gamma$は共に$\beta$に属するから
		\begin{align}
			x \in \gamma \vee x = \gamma \vee \gamma \in x
			\label{eq:thm_element_and_proper_subset_correspond_between_ordinal_numbers_1}
		\end{align}
		が成り立つ.ところで相等性の公理から
		\begin{align}
			x = \gamma \wedge x \in \alpha \Longrightarrow \gamma \in \alpha
		\end{align}
		が成り立ち,$\alpha$の推移性から
		\begin{align}
			\gamma \in x \wedge x \in \alpha \Longrightarrow \gamma \in \alpha
		\end{align}
		が成り立つから,それぞれ対偶を取れば
		\begin{align}
			\gamma \notin \alpha \Longrightarrow x \neq \gamma \vee x \notin \alpha
		\end{align}
		と
		\begin{align}
			\gamma \notin \alpha \Longrightarrow \gamma \notin x \vee x \notin \alpha
		\end{align}
		が成立する.いま
		\begin{align}
			\gamma \notin \alpha
		\end{align}
		が成り立っているので
		\begin{align}
			x \neq \gamma \vee x \notin \alpha
		\end{align}
		と
		\begin{align}
			\gamma \notin x \vee x \notin \alpha
		\end{align}
		が共に成り立ち,また
		\begin{align}
			x \in \alpha
		\end{align}
		でもあるから選言三段論法より
		\begin{align}
			x \neq \gamma
		\end{align}
		と
		\begin{align}
			\gamma \notin x
		\end{align}
		が共に成立する.そして(\refeq{eq:thm_element_and_proper_subset_correspond_between_ordinal_numbers_1})と
		選言三段論法より
		\begin{align}
			x \in \gamma
		\end{align}
		が従うので
		\begin{align}
			\alpha \subset \gamma
		\end{align}
		が得られる.逆に$x$を$\gamma$に任意の要素とすると
		\begin{align}
			x \in \beta \wedge x \notin \beta \backslash \alpha
		\end{align}
		が成り立つから,すなわち
		\begin{align}
			x \in \beta \wedge (\, x \notin \beta \vee x \in \alpha\, )
		\end{align}
		が成立する.ゆえに選言三段論法より
		\begin{align}
			x \in \alpha
		\end{align}
		が成り立ち,$x$の任意性より
		\begin{align}
			\gamma \subset \alpha
		\end{align}
		となる.従って
		\begin{align}
			\gamma = \alpha
		\end{align}
		が成立し,
		\begin{align}
			\gamma \in \beta
		\end{align}
		なので
		\begin{align}
			\alpha \in \beta
		\end{align}
		が成り立つ.以上で
		\begin{align}
			\alpha < \beta \Longrightarrow \alpha \in \beta
		\end{align}
		も得られた.
		\QED
	\end{prf}
	
	\begin{screen}
		\begin{thm}[$\ON$の整列性]\label{thm:On_is_wellordered}
			$\leq$は$\ON$上の整列順序である.また次が成り立つ.
			\begin{align}
				\forall \alpha,\beta \in \ON\,
				\left(\, \alpha \in \beta \vee \alpha = \beta \vee \beta \in \alpha\, \right).
			\end{align}
		\end{thm}
	\end{screen}
	
	\begin{prf}\mbox{}
		\begin{description}
			\item[第一段]
				$\alpha,\beta,\gamma$を順序数とすれば
				\begin{align}
					\alpha \subset \alpha
				\end{align}
				かつ
				\begin{align}
					\alpha \subset \beta \wedge \beta \subset \alpha \Longrightarrow \alpha = \beta
				\end{align}
				かつ
				\begin{align}
					\alpha \subset \beta \wedge \beta \subset \gamma \Longrightarrow \alpha \subset \gamma
				\end{align}
				が成り立つ.ゆえに$\leq$は$\ON$上の順序である.
				
			\item[第二段]
				$\leq$が全順序であることを示す.$\alpha$と$\beta$を順序数とする.このとき
				\begin{align}
					\ord{\alpha \cap \beta}
				\end{align}
				が成り立ち,他方で定理\ref{thm:no_set_is_an_element_of_itself}より
				\begin{align}
					\alpha \cap \beta \notin \alpha \cap \beta
				\end{align}
				が満たされるので
				\begin{align}
					\alpha \cap \beta \notin \alpha \vee \alpha \cap \beta \notin \beta
					\label{eq:thm_On_is_wellordered_5}
				\end{align}
				が成立する.ところで
				\begin{align}
					\alpha \cap \beta \subset \alpha
				\end{align}
				は正しいので定理\ref{thm:element_and_proper_subset_correspond_between_ordinal_numbers}から
				\begin{align}
					\alpha \cap \beta \in \alpha \vee \alpha \cap \beta = \alpha
				\end{align}
				が成立する.従って
				\begin{align}
					\alpha \cap \beta \notin \alpha \Longrightarrow 
					(\alpha \cap \beta \in \alpha \vee \alpha \cap \beta = \alpha) \wedge \alpha \cap \beta \notin \alpha
					\label{eq:thm_On_is_wellordered_2}
				\end{align}
				が成り立ち,他方で選言三段論法より
				\begin{align}
					(\alpha \cap \beta \in \alpha \vee \alpha \cap \beta = \alpha) \wedge \alpha \cap \beta \notin \alpha
					\Longrightarrow \alpha \cap \beta = \alpha
					\label{eq:thm_On_is_wellordered_3}
				\end{align}
				も成り立ち,かつ
				\begin{align}
					\alpha \cap \beta = \alpha \Longrightarrow \alpha \subset \beta
					\label{eq:thm_On_is_wellordered_4}
				\end{align}
				も成り立つので,(\refeq{eq:thm_On_is_wellordered_2})と(\refeq{eq:thm_On_is_wellordered_3})と
				(\refeq{eq:thm_On_is_wellordered_4})から
				\begin{align}
					\alpha \cap \beta \notin \alpha \Longrightarrow \alpha \subset \beta
				\end{align}
				が得られる.同様にして
				\begin{align}
					\alpha \cap \beta \notin \beta \Longrightarrow \beta \subset \alpha
				\end{align}
				も得られる.さらに論理和の規則から
				\begin{align}
					\alpha \cap \beta \notin \alpha \Longrightarrow \alpha \subset \beta \vee \beta \subset \alpha
				\end{align}
				と
				\begin{align}
					\alpha \cap \beta \notin \beta \Longrightarrow \alpha \subset \beta \vee \beta \subset \alpha
				\end{align}
				が従うので,(\refeq{eq:thm_On_is_wellordered_5})と場合分け法則より
				\begin{align}
					\alpha \subset \beta \vee \beta \subset \alpha
				\end{align}
				が成立して
				\begin{align}
					(\alpha,\beta) \in\ \leq \vee (\beta,\alpha) \in\ \leq
				\end{align}
				が成立する.ゆえに$\leq$は全順序である.
			
			\item[第三段]
				$\leq$が整列順序であることを示す.$a$を$\ON$の空でない部分集合とする.このとき正則性公理より
				\begin{align}
					x \in a \wedge x \cap a = \emptyset
				\end{align}
				を満たす集合$x$が取れるが,この$x$が$a$の最小限である.実際,任意に$a$から要素$y$を取ると
				\begin{align}
					x \cap a = \emptyset
				\end{align}
				から
				\begin{align}
					y \notin x
				\end{align}
				が従い,また前段の結果より
				\begin{align}
					x \in y \vee x = y \vee y \in x
				\end{align}
				も成り立つので,選言三段論法より
				\begin{align}
					x \in y \vee x = y
					\label{eq:thm_On_is_wellordered_6}
				\end{align}
				が成り立つ.$y$は推移的であるから
				\begin{align}
					x \in y \Longrightarrow x \subset y
				\end{align}
				が成立して,また
				\begin{align}
					x = y \Longrightarrow x \subset y
				\end{align}
				も成り立つから,(\refeq{eq:thm_On_is_wellordered_6})と場合分け法則から
				\begin{align}
					(x,y) \in\ \leq
				\end{align}
				が従う.$y$の任意性より
				\begin{align}
					\forall y \in a\, \left[\, (x,y) \in\ \leq\, \right]
				\end{align}
				が成立するので$x$は$a$の最小限である.
				\QED
		\end{description}
	\end{prf}
	
	\begin{screen}
		\begin{thm}[$\ON$の部分集合の合併は順序数となる]\label{thm:union_of_set_of_ordinal_numbers_is_ordinal}
			\begin{align}
				\forall a\,
				\left(\, a \subset \ON \Longrightarrow \bigcup a \in \ON\, \right).
			\end{align}
		\end{thm}
	\end{screen}
	
	\begin{prf}
		和集合の公理より$\bigcup a \in \Univ$となる.また順序数の推移性より
		$\bigcup a$の任意の要素は順序数であるから,定理\ref{thm:On_is_wellordered}より
		\begin{align}
			\forall x,y \in \bigcup a\ (\ x \in y \vee x = y \vee y \in x\ )
		\end{align}
		も成り立つ.最後に$\operatorname{Tran}(\bigcup a)$が成り立つことを示す.
		$b$を$\bigcup a$の任意の要素とすれば,$a$の或る要素$x$に対して
		\begin{align}
			b \in x
		\end{align}
		となるが,$x$の推移性より$b \subset x$となり,$x \subset \bigcup a$と併せて
		\begin{align}
			b \subset \bigcup a
		\end{align}
		が従う.
		\QED
	\end{prf}
	
	\begin{screen}
		\begin{thm}[Burali-Forti]\label{thm:Burali_Forti}
			順序数の全体は集合ではない.
			\begin{align}
				\rightharpoondown \set{\ON}.
			\end{align}
		\end{thm}
	\end{screen}
	
	\begin{prf}
		$a$を類とするとき,定理\ref{thm:satisfactory_set_is_an_element}より
		\begin{align}
			\ord{a} \Longrightarrow \left(\, \set{a} \Longrightarrow a \in \ON\, \right)
		\end{align}
		が成り立つ.定理\ref{thm:On_is_transitive}と定理\ref{thm:On_is_wellordered}より
		\begin{align}
			\ord{\ON}
		\end{align}
		が成り立つから
		\begin{align}
			\set{\ON} \Longrightarrow \ON \in \ON
			\label{eq:Burali_Forti_1}
		\end{align}
		が従い,また定理\ref{thm:no_set_is_an_element_of_itself}より
		\begin{align}
			\ON \notin \ON
		\end{align}
		も成り立つので,(\refeq{eq:Burali_Forti_1})の対偶から
		\begin{align}
			\rightharpoondown \set{\ON}
		\end{align}
		が成立する.
		\QED
	\end{prf}
	
	\begin{screen}
		\begin{dfn}[後者]
			$x$を集合とするとき,
			\begin{align}
				x \cup \{x\}
			\end{align}
			を$x$の{\bf 後者}\index{こうしゃ@後者}{\bf (latter)}と呼ぶ.
		\end{dfn}
	\end{screen}
	
	\begin{screen}
		\begin{thm}[順序数の後者は順序数である]\label{thm:latter_element_is_ordinal}
			$\alpha$が順序数であるということと$\alpha \cup \{\alpha\}$が順序数であるということは同値である.
			\begin{align}
				\forall \alpha\, \left(\, \alpha \in \ON \Longleftrightarrow \alpha \cup \{\alpha\} \in \ON\, \right).
			\end{align}
		\end{thm}
	\end{screen}
	
	\begin{sketch}\mbox{}
		\begin{description}
			\item[第一段]
				$\alpha$を順序数とする.そして$x$を
				\begin{align}
					x \in \alpha \cup \{\alpha\}
					\label{fom:thm_latter_element_is_ordinal_3}
				\end{align}
				なる任意の集合とすると,
				\begin{align}
					y \in x
				\end{align}
				なる任意の集合$y$に対して定理\ref{thm:union_of_pair_is_union_of_their_elements}より
				\begin{align}
					y \in \alpha \vee y \in \{\alpha\}
					\label{fom:thm_latter_element_is_ordinal_5}
				\end{align}
				が成立する.$\alpha$が順序数であるから
				\begin{align}
					y \in \alpha \Longrightarrow y \subset \alpha
					\label{fom:thm_latter_element_is_ordinal_1}
				\end{align}
				が成立する.他方で定理\ref{thm:pair_members_are_exactly_the_given_two}より
				\begin{align}
					y \in \{\alpha\} \Longrightarrow y = \alpha
				\end{align}
				が成立し,
				\begin{align}
					y = \alpha \Longrightarrow y \subset \alpha
				\end{align}
				であるから
				\begin{align}
					y \in \{\alpha\} \Longrightarrow y \subset \alpha
					\label{fom:thm_latter_element_is_ordinal_2}
				\end{align}
				が従う.定理\ref{thm:union_is_bigger_than_any_member}より
				\begin{align}
					y \subset \alpha \Longrightarrow y \subset \alpha \cup \{\alpha\}
				\end{align}
				が成り立つので,(\refeq{fom:thm_latter_element_is_ordinal_1})と
				(\refeq{fom:thm_latter_element_is_ordinal_2})と併せて
				\begin{align}
					y \in \alpha \Longrightarrow y \subset \alpha \cup \{\alpha\}
				\end{align}
				かつ
				\begin{align}
					y \in \{\alpha\} \Longrightarrow y \subset \alpha \cup \{\alpha\}
				\end{align}
				が成立し,場合分け法則より
				\begin{align}
					y \in \alpha \vee y \in \{\alpha\} \Longrightarrow y \subset \alpha \cup \{\alpha\}
				\end{align}
				が従う.そして(\refeq{fom:thm_latter_element_is_ordinal_5})と併せて
				\begin{align}
					y \subset \alpha \cup \{\alpha\}
				\end{align}
				が成立する.$y$の任意性ゆえに(\refeq{fom:thm_latter_element_is_ordinal_3})の下で
				\begin{align}
					\forall y\, \left(\, y \in x \Longrightarrow y \subset \alpha \cup \{\alpha\}\, \right)
				\end{align}
				が成り立ち,演繹法則と$x$の任意性から
				\begin{align}
					\forall x\, \left(\, x \in \alpha \cup \{\alpha\} \Longrightarrow x \subset \alpha \cup \{\alpha\}\, \right)
				\end{align}
				が従う.ゆえにいま
				\begin{align}
					\tran{\alpha \cup \{\alpha\}}
					\label{fom:thm_latter_element_is_ordinal_4}
				\end{align}
				が得られた.また$s$と$t$を$\alpha \cup \{\alpha\}$の任意の要素とすると
				\begin{align}
					s \in \alpha \vee s = \alpha
				\end{align}
				と
				\begin{align}
					t \in \alpha \vee t = \alpha
				\end{align}
				が成り立つが,
				\begin{align}
					s \in \alpha \Longrightarrow s \in \ON
				\end{align}
				かつ
				\begin{align}
					s = \alpha \Longrightarrow s \in \ON
				\end{align}
				から
				\begin{align}
					s \in \alpha \vee s = \alpha \Longrightarrow s \in \ON
				\end{align}
				が従い,同様にして
				\begin{align}
					t \in \alpha \vee t = \alpha \Longrightarrow t \in \ON
				\end{align}
				も成り立つので,
				\begin{align}
					s \in \ON
				\end{align}
				かつ
				\begin{align}
					t \in \ON
				\end{align}
				となる.このとき定理\ref{thm:On_is_wellordered}より
				\begin{align}
					s \in t \vee s = t \vee t \in s
				\end{align}
				が成り立つので,$s$および$t$の任意性より
				\begin{align}
					\forall s,t \in \alpha \cup \{\alpha\}\,
					\left(\, s \in t \vee s = t \vee t \in s\, \right)
				\end{align}
				が得られた.(\refeq{fom:thm_latter_element_is_ordinal_4})と併せて
				\begin{align}
					\ord{\alpha \cup \{\alpha\}}
				\end{align}
				が従い,演繹法則より
				\begin{align}
					\alpha \in \ON \Longrightarrow \alpha \cup \{\alpha\} \in \ON
				\end{align}
				を得る.
				
			\item[第二段]
		\end{description}
	\end{sketch}
	
	\begin{screen}
		\begin{thm}[順序数は後者が直後の数となる]
			$\alpha$を順序数とすれば,$\ON$において$\alpha \cup \{\alpha\}$は$\alpha$の直後の数である:
			\begin{align}
				\forall \alpha \in \ON\, 
				\left[\, \forall \beta \in \ON\, (\, \alpha < \beta 
				\Longrightarrow \alpha \cup \{\alpha\} \leq \beta\, )
				\, \right].
			\end{align}
		\end{thm}
	\end{screen}
	
	\begin{sketch}
		$\alpha$と$\beta$を任意に与えられた順序数とし,
		\begin{align}
			\alpha < \beta
		\end{align}
		であるとする.定理\ref{thm:element_and_proper_subset_correspond_between_ordinal_numbers}より,このとき
		\begin{align}
			\alpha \in \beta
		\end{align}
		が成り立ち,$\leq$の定義より
		\begin{align}
			\alpha \subset \beta
		\end{align}
		も成り立つ.ところで,いま$t$を任意の集合とすると
		\begin{align}
			t \in \{\alpha\} \Longrightarrow t = \alpha
		\end{align}
		かつ
		\begin{align}
			t = \alpha \Longrightarrow t \in \beta
		\end{align}
		が成り立つので,
		\begin{align}
			\{\alpha\} \subset \beta
		\end{align}
		が成り立つ.ゆえに
		\begin{align}
			\forall x\, \left(\, x \in \left\{ \alpha, \{\alpha\} \right\} \Longrightarrow x \subset \beta\, \right)
		\end{align}
		が成り立つ.ゆえに定理\ref{thm:union_of_subsets_is_subclass}より
		\begin{align}
			\alpha \cup \{\alpha\} \subset \beta.
		\end{align}
		すなわち
		\begin{align}
			\alpha \cup \{\alpha\} \leq \beta
		\end{align}
		が成り立つ.
		\QED
	\end{sketch}
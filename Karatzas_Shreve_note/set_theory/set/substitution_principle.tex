\section{代入原理}
	$a$と$b$を類とし,$\varphi$を$x$のみが自由に現れる式とするとき,
	\begin{align}
		a = b
	\end{align}
	ならば$a$と$b$をそれぞれ$\varphi$の自由な$x$に代入しても
	\begin{align}
		\varphi(a) \lrarrow \varphi(b)
	\end{align}
	が成立するというのは{\bf 代入原理}\index{だいにゅうげんり@代入原理}
	{\bf (the principle of substitution)}と呼ばれる.
	この原理の証明は相等性公理に負うところが多いが,
	本稿では$\varepsilon$項という厄介なものを抱え込んでいるため
	$\EQAX$だけでは不十分であり,次に追加する公理が必要になる.
	
	\begin{screen}
		\begin{axm}[$\varepsilon$項に対する相等性公理]
			$a,b$を類とし,$\varphi$を$\lang{\varepsilon}$の式とし,$\varphi$には変項$x,y$が
			自由に現れ,また$\varphi$に自由に現れる変項はこれらのみであるとする.このとき
			\begin{align}
				\EQAXEP \defarrow
				a = b \rarrow \varepsilon x \varphi(x,a) = \varepsilon x \varphi(x,b).
			\end{align}
		\end{axm}
	\end{screen}
	
	代入原理を示すには構造的帰納法の原理が必要になるので,証明はメタなものとなる.
	
	\begin{screen}
		\begin{thm}[代入原理]\label{thm:the_principle_of_substitution}
			$a,b$を類とし,$\varphi$を$\mathcal{L}$の式とし,$x$を$\varphi$に自由に現れる変項
			とし,$\varphi$に自由に現れる変項は$x$のみであるとする.このとき
			\begin{align}
				\EXTAX,\EQAX,\EQAXEP \vdash a = b \rarrow 
				(\, \varphi(a) \lrarrow \varphi(b)\, ).
			\end{align}
		\end{thm}
	\end{screen}
	
	\begin{sketch}\mbox{}
		\begin{description}
			\item[step1]
				$c$を類として,$\varphi$が
				\begin{align}
					x \in c
				\end{align}
				なる式であるとき,
				\begin{align}
					a = b,\ \EQAX \vdash a \in c \rarrow b \in c
				\end{align}
				となる.また
				\begin{align}
					a = b,\ \EXTAX,\EQAX \vdash b = a
				\end{align}
				より
				\begin{align}
					a = b,\ \EXTAX,\EQAX \vdash b \in c \rarrow a \in c
				\end{align}
				も成り立つ.ゆえに
				\begin{align}
					a = b,\ \EXTAX,\EQAX \vdash a \in c \lrarrow b \in c
				\end{align}
				が得られる.同様に
				\begin{align}
					a = b,\ \EXTAX,\EQAX \vdash c \in a \lrarrow c \in b
				\end{align}
				も得られる.
				
			\item[step2]
				$\varphi$が
				\begin{align}
					x \in \varepsilon y\, R(x,y)
				\end{align}
				なる式であるとき,
				\begin{align}
					a = b,\ \EQAXEP \vdash \varepsilon y\, R(a,y) = \varepsilon y\, R(b,y)
				\end{align}
				となる.
				\begin{align}
					a = b,\ \EXTAX,\EQAX 
					\vdash (\, \varepsilon y\, R(a,y) = \varepsilon y\, R(b,y)\, )
					\rarrow (\, a \in \varepsilon y\, R(a,y) \lrarrow a \in \varepsilon y\, R(b,y)\, )
				\end{align}
				なので
				\begin{align}
					a = b,\ \EXTAX,\EQAX,\EQAXEP \vdash 
					a \in \varepsilon y\, R(a,y) \lrarrow a \in \varepsilon y\, R(b,y)
				\end{align}
				となる.
				\begin{align}
					a=b,\ \EXTAX,\EQAX \vdash 
					a \in \varepsilon y\, R(b,y) \lrarrow b \in \varepsilon y\, R(b,y)
				\end{align}
				も成り立つので
				\begin{align}
					a=b,\ \EXTAX,\EQAX,\EQAXEP \vdash 
					a \in \varepsilon y\, R(a,y) \lrarrow b \in \varepsilon y\, R(b,y)
				\end{align}
				が得られる.
				
			\item[step3]
				$\varphi$が
				\begin{align}
					\varepsilon y R(x,y) \in \varepsilon z T(x,z)
				\end{align}
				なる形のとき,
				\begin{align}
					a = b,\ \EQAXEP \vdash 
					\varepsilon y R(a,y) = \varepsilon y R(b,y)
				\end{align}
				と
				\begin{align}
					a = b,\ \EQAX, \EQAXEP \vdash 
					(\, \varepsilon y R(a,y) = \varepsilon y R(b,y)\, )
					\rarrow (\, \varepsilon y R(a,y) \in \varepsilon z T(a,z)
					\lrarrow \varepsilon y R(b,y) \in \varepsilon z T(a,z)\, )
				\end{align}
				より
				\begin{align}
					a = b,\ \EQAX, \EQAXEP \vdash 
					\varepsilon y R(a,y) \in \varepsilon z T(a,z)
					\lrarrow \varepsilon y R(b,y) \in \varepsilon z T(a,z)
				\end{align}
				が成り立つ.他方で
				\begin{align}
					a = b,\ \EQAXEP \vdash 
					\varepsilon z T(a,z) = \varepsilon z T(b,z)
				\end{align}
				と
				\begin{align}
					a = b,\ \EQAX, \EQAXEP \vdash 
					(\, \varepsilon z T(a,z) = \varepsilon z T(b,z)\, )
					\rarrow (\, \varepsilon y R(b,y) \in \varepsilon z T(a,z)
					\lrarrow \varepsilon y R(b,y) \in \varepsilon z T(b,z)\, )
				\end{align}
				より
				\begin{align}
					a = b,\ \EQAX, \EQAXEP \vdash 
					\varepsilon y R(b,y) \in \varepsilon z T(a,z)
					\lrarrow \varepsilon y R(b,y) \in \varepsilon z T(b,z)
				\end{align}
				が成り立つ.同値関係の推移律
				(\ref{logicalthm:transitive_law_of_equivalence_symbol})より
				\begin{align}
					a = b,\ \EQAX, \EQAXEP \vdash 
					\varepsilon y R(a,y) \in \varepsilon z T(a,z)
					\lrarrow \varepsilon y R(b,y) \in \varepsilon z T(b,z)
				\end{align}
				が成立する.
		\end{description}
	\end{sketch}
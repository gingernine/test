\section{代入原理}
	$a$と$b$を主要$\varepsilon$項とし,$\varphi$を$x$のみが自由に現れる式とするとき,
	\begin{align}
		a = b
	\end{align}
	ならば$a$と$b$をそれぞれ$\varphi$の自由な$x$に代入しても
	\begin{align}
		\varphi(a) \lrarrow \varphi(b)
	\end{align}
	が成立するというのは{\bf 代入原理}\index{だいにゅうげんり@代入原理}
	{\bf (the principle of substitution)}と呼ばれる.
	第\ref{sec:restriction_of_formulas}節で決めたことをここでも注意しておくと,
	{\bf 式に現れる$\varepsilon$項は全て主要$\varepsilon$項であり,
	式に現れる内包項は全て正則内包項であり,
	項や式の上に現れる$\forall x \psi,\exists x \psi$なる形の式は,
	$\psi$の中に$x$が自由に現れている.}
	
	%この原理の証明は相等性公理に負うところが多いが,
	%本稿では$\varepsilon$項という厄介なものを抱え込んでいるため
	%$\EQAX$だけでは不十分であり,次に追加する公理が必要になる.
	
	%\begin{screen}
	%	\begin{axm}[$\varepsilon$項に対する相等性公理]
	%		$a,b$を類とし,$\varphi$を$\lang{\varepsilon}$の式とし,$\varphi$には変項$x,y$が
	%		自由に現れ,また$\varphi$に自由に現れる変項はこれらのみであるとする.このとき
	%		\begin{align}
	%			\EQAXEP \defarrow
	%			a = b \rarrow \varepsilon x \varphi(x,a) 
	%			= \varepsilon x \varphi(x,b).
	%		\end{align}
	%	\end{axm}
	%\end{screen}
	
	\begin{screen}
		\begin{thm}[代入原理]\label{thm:the_principle_of_substitution}
			$a,b$を主要$\varepsilon$項とし,$\varphi$を$\mathcal{L}$の式とし,$x$を変項とし,
			$\varphi$には$x$のみ自由に現れるとする.このとき
			\begin{align}
				\EXTAX,\ \EQAX,\ \COMAX,\ \ELEAX \vdash a = b \rarrow 
				(\, \varphi(a) \lrarrow \varphi(b)\, )
			\end{align}
			が成り立つ.ただし$\varphi$が$\lang{\varepsilon}$の式であるときは
			\begin{align}
				\EXTAX,\EQAX \vdash a = b \rarrow 
				(\, \varphi(a) \lrarrow \varphi(b)\, ).
			\end{align}
		\end{thm}
	\end{screen}
	
	$\varphi$が$\lang{\varepsilon}$の式であるとして証明すれば十分である.
	実際$\varphi$を$x$のみが自由に現れる$\mathcal{L}$の式とし,
	$\varphi$を$\lang{\varepsilon}$の式に書き直したものを$\widehat{\varphi}$と書くと,
	$\lang{\varepsilon}$の式に対して代入原理が成り立つのであれば
	\begin{align}
		a = b,\ \EXTAX,\EQAX \vdash \widehat{\varphi}(a) \lrarrow \widehat{\varphi}(b)
	\end{align}
	となる.ここでメタ定理\ref{metathm:substitution_to_rewritten_formula}より
	$\widehat{\varphi}(a)$は$\varphi(a)$の書き換えであって,
	$\widehat{\varphi}(b)$は$\varphi(b)$の書き換えであるから,
	書き換えの同値性(定理\ref{thm:equivalence_of_formula_rewritings})より
	\begin{align}
		\EXTAX,\EQAX,\COMAX,\ELEAX &\vdash \varphi(a) \lrarrow \widehat{\varphi}(a), \\
		\EXTAX,\EQAX,\COMAX,\ELEAX &\vdash \widehat{\varphi}(b) \lrarrow \varphi(b)
	\end{align}
	が成り立つ.従って
	\begin{align}
		a = b,\ \EXTAX,\EQAX,\COMAX,\ELEAX \vdash \varphi(a) \lrarrow \varphi(b)
	\end{align}
	が出る.
	
	\begin{sketch} 代入原理を示すには構造的帰納法の原理が必要になるので,証明はメタなものとなる.
		\begin{description}
			\item[step1-1]
				始めの$3$ステップでは$\varphi$が原子式であるとして考察する.
				$c$を主要$\varepsilon$項とすると,相等性公理から直接
				\begin{align}
					a = b,\ \EQAX \vdash a \in c \rarrow b \in c
				\end{align}
				となる.また
				\begin{align}
					a = b,\ \EQAX \vdash b = a
				\end{align}
				より
				\begin{align}
					a = b,\ \EQAX \vdash b \in c \rarrow a \in c
				\end{align}
				も成り立つ.従って
				\begin{align}
					a = b,\ \EQAX \vdash a \in c \lrarrow b \in c
				\end{align}
				が得られる.同様に
				\begin{align}
					a = b,\ \EQAX \vdash c \in a \lrarrow c \in b
				\end{align}
				も得られるので,$\varphi$が$x \in c$や$c \in x$なる式であるときは
				\begin{align}
					\EQAX \vdash a = b \rarrow (\, \varphi(a) \lrarrow \varphi(b)\, )
				\end{align}
				が成り立つ.
			
			\item[step1-2]
				$c$を主要$\varepsilon$項とすると,等号の推移律(定理\ref{thm:transitive_law_of_equality})より
				\begin{align}
					a = b,\ \EXTAX,\EQAX \vdash a = c \rarrow b = c
				\end{align}
				となる.また
				\begin{align}
					a = b,\ \EQAX \vdash b = a
				\end{align}
				と等号の推移律(定理\ref{thm:transitive_law_of_equality})より
				\begin{align}
					a = b,\ \EXTAX,\EQAX \vdash b = c \rarrow a = c
				\end{align}
				も成り立つ.従って
				\begin{align}
					a = b,\ \EXTAX,\EQAX \vdash a = c \lrarrow b = c
				\end{align}
				が得られる.つまり$\varphi$が
				\begin{align}
					x = c
				\end{align}
				なる式であるときは
				\begin{align}
					\EQAX \vdash a = b \rarrow (\, \varphi(a) \lrarrow \varphi(b)\, )
				\end{align}
				が成り立つ.
				
			\item[step1-3]
				$c$を主要$\varepsilon$項とすると,等号の推移律(定理\ref{thm:transitive_law_of_equality})より
				\begin{align}
					a = b,\ \EXTAX,\EQAX \vdash a = c \rarrow b = c
				\end{align}
				となるが,ここで
				\begin{align}
					c = a,\ \EQAX \vdash a = c
				\end{align}
				なので
				\begin{align}
					c = a,\ a = b,\ \EXTAX,\EQAX \vdash b = c
				\end{align}
				が成り立ち,また
				\begin{align}
					\EQAX \vdash b = c \rarrow c = b
				\end{align}
				より
				\begin{align}
					c = a,\ a = b,\ \EXTAX,\EQAX \vdash c = b
				\end{align}
				が従い,演繹定理より
				\begin{align}
					a = b,\ \EXTAX,\EQAX \vdash c = a \rarrow c = b
					\label{fom:the_principle_of_substitution_1}
				\end{align}
				が得られる.
				\begin{align}
					a = b,\ \EXTAX,\EQAX \vdash b = a
				\end{align}
				と(\refeq{fom:the_principle_of_substitution_1})より
				\begin{align}
					a = b,\ \EXTAX,\EQAX \vdash c = b \rarrow c = a
				\end{align}
				も得られるので
				\begin{align}
					a = b,\ \EXTAX,\EQAX \vdash c = a \lrarrow c = b
				\end{align}
				が成り立つ.従って$\varphi$が
				\begin{align}
					c = x
				\end{align}
				なる式であるときも
				\begin{align}
					\EXTAX,\EQAX \vdash a = b \rarrow (\, \varphi(a) \lrarrow \varphi(b)\, )
				\end{align}
				が成り立つ.
			
			\item[step2]
				$\varphi$を$x$のみが自由に現れる$\lang{\varepsilon}$の式として
				\begin{itembox}[l]{IH (帰納法の仮定)}
					$\varphi$の任意の真部分式$\psi$に対して,$\psi$に$x$が自由に現れていて,
					また$\psi$に自由に現れている$x$以外の変項が$x_{1},\cdots,x_{n}$で全て
					であるとき,任意に主要$\varepsilon$項$\tau_{1},\cdots,\tau_{n}$を取って
					$\psi(x_{1}/\tau_{1})\cdots(x_{n}/\tau_{n})$なる文を
					$\psi^{*}$とする.$\psi$に自由に現れる変項が$x$のみなら$\psi^{*}$
					を$\psi$とする.このとき
					\begin{align}
						\EXTAX,\EQAX \vdash a = b \rarrow 
						(\, \psi^{*}(a) \lrarrow \psi^{*}(b)\, )
					\end{align}
					が成り立つ.
				\end{itembox}
				と仮定する(帰納法の考え方としては,$\varphi$よりも``前の段階''で作られた
				任意の式に対して代入原理が満たされていると仮定している).このとき
				\begin{description}
					\item[case1] $\varphi$が
						\begin{align}
							\negation \psi
						\end{align}
						なる式であるとき,(IH)より
						\begin{align}
							a = b,\ \EXTAX,\EQAX \vdash \psi(a) \lrarrow \psi(b)
						\end{align}
						が成り立つので,対偶を取れば
						\begin{align}
							a = b,\ \EXTAX,\EQAX \vdash\ 
							\negation \psi(a) \lrarrow\ \negation \psi(b)
						\end{align}
						が成り立つ.
						
					\item[case2]
						$\varphi$が
						\begin{align}
							\psi \vee \chi
						\end{align}
						なる式であるとき,
						\begin{align}
							\psi_{a} \defarrow
							\begin{cases}
								\psi(a) & \mbox{if $\psi$に$x$が自由に現れている} \\
								\psi & \mbox{if $\psi$に$x$が自由に現れていない}
							\end{cases}
						\end{align}
						と定め,同様に$\psi_{b},\chi_{a},\chi_{b}$も定めれば,(IH)
						或いは含意の反射律(論理的定理
						\ref{logicalthm:reflective_law_of_implication})より
						\begin{align}
							a = b,\ \EXTAX,\EQAX &\vdash \psi_{a} \lrarrow \psi_{b}, \\
							a = b,\ \EXTAX,\EQAX &\vdash \chi_{a} \lrarrow \chi_{b}
						\end{align}
						が満たされる.従って論理的定理
						\ref{logicalthm:heredity_of_implication_to_disjunction_2}より
						\begin{align}
							a = b,\ \EXTAX,\EQAX \vdash 
							\psi_{a} \vee \chi_{a} \lrarrow \psi_{b} \vee \chi_{b}
						\end{align}
						が成り立つ.$\varphi$が$\psi \wedge \chi$や$\psi \rarrow \chi$
						なる式であるときも,論理的定理
						\ref{logicalthm:heredity_of_implication_to_conjunction_2}
						或いは論理的定理
						\ref{logicalthm:heredity_of_implication_to_implication_2}
						を用いれば同様にして$a = b,\ \EXTAX,\EQAX \vdash 
						\varphi(a) \lrarrow \varphi(b)$が示される.
					
					\item[case3]
						$\varphi$が
						\begin{align}
							\exists y \psi
						\end{align}
						なる式であるとき,$x$は$\varphi$に自由に現れているので
						$x$は$y$とは違う変項である.いま
						\begin{align}
							\tau \defeq \varepsilon y \psi(x/a)
						\end{align}
						と主要$\varepsilon$項を定めると,存在記号の論理的公理より
						\begin{align}
							\exists y \psi(x/a) \vdash \psi(x/a)(y/\tau)
						\end{align}
						が成り立つ($\varphi(a)$と$\exists y \psi(x/a)$は一致する).(IH)より
						\begin{align}
							a = b,\ \EXTAX,\EQAX \vdash 
							\psi(y/\tau)(x/a) \rarrow \psi(y/\tau)(x/b)
						\end{align}
						が成り立つが,$x$と$y$は違う変項であるから,$\psi(x/a)(y/\tau)$と
						$\psi(y/\tau)(x/a)$,$\psi(x/b)(y/\tau)$と
						$\psi(y/\tau)(x/b)$はそれぞれ同じ式である.従って
						\begin{align}
							a = b,\ \EXTAX,\EQAX \vdash 
							\psi(x/a)(y/\tau) \rarrow \psi(x/b)(y/\tau)
						\end{align}
						が成り立つ.三段論法より
						\begin{align}
							\exists y \psi(x/a),\ a = b,\ \EXTAX,\EQAX \vdash 
							\psi(x/b,y/\tau)
						\end{align}
						が従い,存在記号の論理的公理より
						\begin{align}
							\exists y \psi(x/a),\ a = b,\ \EXTAX,\EQAX \vdash 
							\exists y \psi(x/b)
						\end{align}
						となり,演繹定理より
						\begin{align}
							a = b,\ \EXTAX,\EQAX \vdash 
							\exists y \psi(x/a) \rarrow \exists y \psi(x/b)
						\end{align}
						が出る.$a$と$b$を入れ替えれば
						\begin{align}
							a = b,\ \EXTAX,\EQAX \vdash 
							\exists y \psi(x/b) \rarrow \exists y \psi(x/a)
						\end{align}
						も成り立つので,論理積の導入より
						\begin{align}
							a = b,\ \EXTAX,\EQAX \vdash 
							\exists y \psi(x/a) \lrarrow \exists y \psi(x/b)
						\end{align}
						が得られる.$\varphi$が$\forall y \psi$なる式であっても,全称の導出
						(論理的定理
						\ref{logicalthm:derivation_of_universal_by_epsilon})
						を利用すれば同様にして$a = b,\ \EXTAX,\EQAX \vdash 
						\varphi(a) \lrarrow \varphi(b)$が示される.
						\QED
					%\item[case4]
					%	$\varphi$が
					%	\begin{align}
					%		\forall y \psi
					%	\end{align}
					%	なる式の場合は,まず$\tau$を
					%	\begin{align}
					%		\varepsilon y \negation \psi(x/b)
					%	\end{align}
					%	なる主要$\varepsilon$項とする.このとき全称記号の論理的公理より
					%	\begin{align}
					%		\forall y \psi(x/a) \vdash \psi(x/a,y/\tau)
					%	\end{align}
					%	が成り立ち($\varphi(a)$と$\forall y \psi(x/a)$は一致する),(IH)より
					%	\begin{align}
					%		a = b,\ \EXTAX,\EQAX \vdash 
					%		\psi(x/a,y/\tau) \rarrow \psi(x/b,y/\tau)
					%	\end{align}
					%	が成り立つので三段論法より
					%	\begin{align}
					%		\forall y \psi(x/a),\ a = b,\ \EXTAX,\EQAX \vdash 
					%		\psi(x/b,y/\tau)
					%	\end{align}
					%	が従い,全称の導出
					%	\ref{logicalthm:derivation_of_universal_by_epsilon}より
					%	\begin{align}
					%		\forall y \psi(x/a),\ a = b,\ \EXTAX,\EQAX \vdash 
					%		\forall y \psi(x/b)
					%	\end{align}
					%	となり
					%	\begin{align}
					%		a = b,\ \EXTAX,\EQAX \vdash 
					%		\forall y \psi(x/a) \rarrow \forall y \psi(x/b)
					%	\end{align}
					%	が出る.今度は
					%	\begin{align}
					%		\tau \defeq \varepsilon \negation y \psi(x/a)
					%	\end{align}
					%	とおけば上と同様に
					%	\begin{align}
					%		a = b,\ \EXTAX,\EQAX \vdash 
					%		\forall y \psi(x/b) \rarrow \forall y \psi(x/a)
					%	\end{align}
					%	が成り立ち,
					%	\begin{align}
					%		a = b,\ \EXTAX,\EQAX \vdash 
					%		\forall y \psi(x/a) \lrarrow \forall y \psi(x/b)
					%	\end{align}
					%	が得られる.
					%	\QED
				\end{description}
				
			%\item[step2]
			%	$\varphi$が
			%	\begin{align}
			%		x \in \varepsilon y\, R(x,y)
			%	\end{align}
			%	なる式であるとき,
			%	\begin{align}
			%		a = b,\ \EQAXEP \vdash \varepsilon y\, R(a,y) = \varepsilon y\, R(b,y)
			%	\end{align}
			%	となる.
			%	\begin{align}
			%		a = b,\ \EXTAX,\EQAX 
			%		\vdash (\, \varepsilon y\, R(a,y) = \varepsilon y\, R(b,y)\, )
			%		\rarrow (\, a \in \varepsilon y\, R(a,y) \lrarrow a \in \varepsilon y\, R(b,y)\, )
			%	\end{align}
			%	なので
			%	\begin{align}
			%		a = b,\ \EXTAX,\EQAX,\EQAXEP \vdash 
			%		a \in \varepsilon y\, R(a,y) \lrarrow a \in \varepsilon y\, R(b,y)
			%	\end{align}
			%	となる.
			%	\begin{align}
			%		a=b,\ \EXTAX,\EQAX \vdash 
			%		a \in \varepsilon y\, R(b,y) \lrarrow b \in \varepsilon y\, R(b,y)
			%	\end{align}
			%	も成り立つので
			%	\begin{align}
			%		a=b,\ \EXTAX,\EQAX,\EQAXEP \vdash 
			%		a \in \varepsilon y\, R(a,y) \lrarrow b \in \varepsilon y\, R(b,y)
			%	\end{align}
			%	が得られる.
				
			%\item[step3]
			%	$\varphi$が
			%	\begin{align}
			%		\varepsilon y R(x,y) \in \varepsilon z T(x,z)
			%	\end{align}
			%	なる形のとき,
			%	\begin{align}
			%		a = b,\ \EQAXEP \vdash 
			%		\varepsilon y R(a,y) = \varepsilon y R(b,y)
			%	\end{align}
			%	と
			%	\begin{align}
			%		a = b,\ \EQAX, \EQAXEP \vdash 
			%		(\, \varepsilon y R(a,y) = \varepsilon y R(b,y)\, )
			%		\rarrow (\, \varepsilon y R(a,y) \in \varepsilon z T(a,z)
			%		\lrarrow \varepsilon y R(b,y) \in \varepsilon z T(a,z)\, )
			%	\end{align}
			%	より
			%	\begin{align}
			%		a = b,\ \EQAX, \EQAXEP \vdash 
			%		\varepsilon y R(a,y) \in \varepsilon z T(a,z)
			%		\lrarrow \varepsilon y R(b,y) \in \varepsilon z T(a,z)
			%	\end{align}
			%	が成り立つ.他方で
			%	\begin{align}
			%		a = b,\ \EQAXEP \vdash 
			%		\varepsilon z T(a,z) = \varepsilon z T(b,z)
			%	\end{align}
			%	と
			%	\begin{align}
			%		a = b,\ \EQAX, \EQAXEP \vdash 
			%		(\, \varepsilon z T(a,z) = \varepsilon z T(b,z)\, )
			%		\rarrow (\, \varepsilon y R(b,y) \in \varepsilon z T(a,z)
			%		\lrarrow \varepsilon y R(b,y) \in \varepsilon z T(b,z)\, )
			%	\end{align}
			%	より
			%	\begin{align}
			%		a = b,\ \EQAX, \EQAXEP \vdash 
			%		\varepsilon y R(b,y) \in \varepsilon z T(a,z)
			%		\lrarrow \varepsilon y R(b,y) \in \varepsilon z T(b,z)
			%	\end{align}
			%	が成り立つ.同値関係の推移律
			%	(\ref{logicalthm:transitive_law_of_equivalence_symbol})より
			%	\begin{align}
			%		a = b,\ \EQAX, \EQAXEP \vdash 
			%		\varepsilon y R(a,y) \in \varepsilon z T(a,z)
			%		\lrarrow \varepsilon y R(b,y) \in \varepsilon z T(b,z)
			%	\end{align}
			%	が成立する.
		\end{description}
	\end{sketch}
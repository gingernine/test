\section{代入原理}
	$a$と$b$を類とし,$\varphi$を$x$のみが自由に現れる式とするとき,
	\begin{align}
		a = b
	\end{align}
	ならば$a$と$b$をそれぞれ$\varphi$の自由な$x$に代入しても
	\begin{align}
		\varphi(a) \lrarrow \varphi(b)
	\end{align}
	が成立するというのは{\bf 代入原理}\index{だいにゅうげんり@代入原理}
	{\bf (the principle of substitution)}と呼ばれる.
	第\ref{sec:restriction_of_formulas}節で決めたことをここでも注意しておくと,
	{\bf 扱う式は全て,そこに現れる$\varepsilon$項は全て主要$\varepsilon$項であり,
	現れる内包項は全て正則内包項であるとする.}始めにいくつか必要な定理を示しておく.
	
	\begin{screen}
		\begin{logicalthm}
		\label{logicalthm:lemma_for_principle_of_substitution}
			$\psi,\chi,\psi',\chi'$を文とするとき,
			\begin{align}
				\psi \rarrow \psi',\ \chi \rarrow \chi' &\vdash
				\psi \vee \chi \rarrow \psi' \vee \chi', 
				\label{fom:lemma_for_principle_of_substitution_1} \\
				\psi \rarrow \psi',\ \chi \rarrow \chi' &\vdash
				\psi \wedge \chi \rarrow \psi' \wedge \chi', 
				\label{fom:lemma_for_principle_of_substitution_2} \\
				\psi' \rarrow \psi,\ \chi \rarrow \chi' &\vdash
				(\, \psi \rarrow \chi\, ) \rarrow (\, \psi' \rarrow \chi'\, )
				\label{fom:lemma_for_principle_of_substitution_3}.
			\end{align}
		\end{logicalthm}
	\end{screen}
	
	\begin{sketch}\mbox{}
		\begin{itemize}
			\item (\refeq{fom:lemma_for_principle_of_substitution_1})を示す.
				\begin{align}
					\psi,\ \psi \rarrow \psi',\ \chi \rarrow \chi' \vdash \psi'
				\end{align}
				と論理和の導入より
				\begin{align}
					\psi,\ \psi \rarrow \psi',\ \chi \rarrow \chi' \vdash \psi' \vee \chi'
				\end{align}
				が成り立つので
				\begin{align}
					\psi \rarrow \psi',\ \chi \rarrow \chi' \vdash 
					\psi \rarrow \psi' \vee \chi'
				\end{align}
				が従う.同様に
				\begin{align}
					\psi \rarrow \psi',\ \chi \rarrow \chi' \vdash 
					\chi \rarrow \psi' \vee \chi'
				\end{align}
				も成り立ち,論理和の除去より
				\begin{align}
					\psi \rarrow \psi',\ \chi \rarrow \chi' \vdash 
					\psi \vee \chi \rarrow \psi' \vee \chi'
				\end{align}
				が得られる.
				
			\item (\refeq{fom:lemma_for_principle_of_substitution_2})を示す.
				論理積の除去より
				\begin{align}
					\psi \wedge \chi,\ \psi \rarrow \psi',\ \chi \rarrow \chi' 
					\vdash \psi
				\end{align}
				が成り立つので三段論法より
				\begin{align}
					\psi \wedge \chi,\ \psi \rarrow \psi',\ \chi \rarrow \chi' 
					\vdash \psi'
				\end{align}
				が従う.同様に
				\begin{align}
					\psi \wedge \chi,\ \psi \rarrow \psi',\ \chi \rarrow \chi' 
					\vdash \chi'
				\end{align}
				も成り立ち,論理積の導入より
				\begin{align}
					\psi \wedge \chi,\ \psi \rarrow \psi',\ \chi \rarrow \chi' 
					\vdash \psi' \wedge \chi'
				\end{align}
				が得られる.
				
			\item (\refeq{fom:lemma_for_principle_of_substitution_3})を示す.
				\begin{align}
					\psi',\ \psi \rarrow \chi,\ \psi' \rarrow \psi,\ \chi \rarrow \chi' &\vdash \psi, \\
					\psi',\ \psi \rarrow \chi,\ \psi' \rarrow \psi,\ \chi \rarrow \chi' &\vdash \psi \rarrow \chi
				\end{align}
				より
				\begin{align}
					\psi',\ \psi \rarrow \chi,\ \psi' \rarrow \psi,\ \chi \rarrow \chi' \vdash \chi
				\end{align}
				が成り立ち,再び三段論法より
				\begin{align}
					\psi',\ \psi \rarrow \chi,\ \psi' \rarrow \psi,\ \chi \rarrow \chi' \vdash \chi'
				\end{align}
				が従う.演繹法則より
				\begin{align}
					\psi' \rarrow \psi,\ \chi \rarrow \chi' \vdash
					(\, \psi \rarrow \chi\, ) \rarrow (\, \psi' \rarrow \chi'\, )
				\end{align}
				が得られる.
				\QED
		\end{itemize}
	\end{sketch}
	
	代入原理を示すには構造的帰納法の原理が必要になるので,証明はメタなものとなる.
	
	\begin{comment}
	
	この原理の証明は相等性公理に負うところが多いが,
	本稿では$\varepsilon$項という厄介なものを抱え込んでいるため
	$\EQAX$だけでは不十分であり,次に追加する公理が必要になる.
	
	\begin{screen}
		\begin{axm}[$\varepsilon$項に対する相等性公理]
			$a,b$を類とし,$\varphi$を$\lang{\varepsilon}$の式とし,$\varphi$には変項$x,y$が
			自由に現れ,また$\varphi$に自由に現れる変項はこれらのみであるとする.このとき
			\begin{align}
				\EQAXEP \defarrow
				a = b \rarrow \varepsilon x \varphi(x,a) = \varepsilon x \varphi(x,b).
			\end{align}
		\end{axm}
	\end{screen}
	
	\end{comment}
	
	\begin{screen}
		\begin{thm}[代入原理]\label{thm:the_principle_of_substitution}
			$a,b$を類とし,$\varphi$を$\mathcal{L}$の式とし,$x$を変項とし,
			$\varphi$には$x$のみ自由に現れるとする.このとき
			\begin{align}
				\EXTAX,\ \EQAX,\ \COMAX,\ \ELEAX \vdash a = b \rarrow 
				(\, \varphi(a) \lrarrow \varphi(b)\, )
			\end{align}
			が成り立つ.ただし$\varphi$が$\lang{\varepsilon}$の式であるときは
			\begin{align}
				\EXTAX,\EQAX \vdash a = b \rarrow 
				(\, \varphi(a) \lrarrow \varphi(b)\, ).
			\end{align}
		\end{thm}
	\end{screen}
	
	$\varphi$が$\lang{\varepsilon}$の式であるとして証明すれば十分である.
	実際$\varphi$を$x$のみが自由に現れる$\mathcal{L}$の式とし,
	$\varphi$を$\lang{\varepsilon}$の式に書き直したものを$\hat{\varphi}$と書くと,
	$\lang{\varepsilon}$の式に対して代入原理が成り立つのであれば
	\begin{align}
		a = b,\ \EXTAX,\EQAX \vdash \hat{\varphi}(a) \lrarrow \hat{\varphi}(b)
	\end{align}
	がとなるが,書き換えの同値性(\ref{sec:equivalence_of_formula_rewriting}節)より
	\begin{align}
		\EXTAX,\ \EQAX,\ \COMAX,\ \ELEAX &\vdash \varphi(a) \rarrow \hat{\varphi}(a), \\
		\EXTAX,\ \EQAX,\ \COMAX,\ \ELEAX &\vdash \hat{\varphi}(b) \rarrow \varphi(b)
	\end{align}
	が成り立つので
	\begin{align}
		a = b,\ \EXTAX,\ \EQAX,\ \COMAX,\ \ELEAX \vdash \varphi(a) \lrarrow \varphi(b)
	\end{align}
	が従う.
	
	\begin{sketch}\mbox{}
		\begin{description}
			\item[step1]
				始めの$3$ステップでは$\varphi$が原子式であるとして考察する.
				$c$を類とすると,相等性公理から直接
				\begin{align}
					a = b,\ \EQAX \vdash a \in c \rarrow b \in c
				\end{align}
				となる.また
				\begin{align}
					a = b,\ \EQAX \vdash b = a
				\end{align}
				より
				\begin{align}
					a = b,\ \EQAX \vdash b \in c \rarrow a \in c
				\end{align}
				も成り立つ.従って
				\begin{align}
					a = b,\ \EQAX \vdash a \in c \lrarrow b \in c
				\end{align}
				が得られる.同様に
				\begin{align}
					a = b,\ \EQAX \vdash c \in a \lrarrow c \in b
				\end{align}
				も得られるので,$\varphi$が
				\begin{align}
					x \in c
				\end{align}
				や
				\begin{align}
					c \in x
				\end{align}
				なる式であるときは
				\begin{align}
					\EQAX \vdash a = b \rarrow (\, \varphi(a) \lrarrow \varphi(b)\, )
				\end{align}
				が成り立つ.
			
			\item[step2]
				$c$を類とすると,等号の推移律(定理\ref{thm:transitive_law_of_equality})より
				\begin{align}
					a = b,\ \EXTAX,\EQAX \vdash a = c \rarrow b = c
				\end{align}
				となる.また
				\begin{align}
					a = b,\ \EXTAX,\EQAX b = a
				\end{align}
				と等号の推移律(定理\ref{thm:transitive_law_of_equality})より
				\begin{align}
					a = b,\ \EXTAX,\EQAX \vdash b = c \rarrow a = c
				\end{align}
				も成り立つ.従って
				\begin{align}
					a = b,\ \EXTAX,\EQAX \vdash a = c \lrarrow b = c
				\end{align}
				が得られる.つまり$\varphi$が
				\begin{align}
					x = c
				\end{align}
				なる式であるときは
				\begin{align}
					\EQAX \vdash a = b \rarrow (\, \varphi(a) \lrarrow \varphi(b)\, )
				\end{align}
				が成り立つ.
				
			\item[step3]
				$c$を類とすると,等号の推移律(定理\ref{thm:transitive_law_of_equality})より
				\begin{align}
					a = b,\ \EXTAX,\EQAX \vdash a = c \rarrow b = c
				\end{align}
				となるが,ここで
				\begin{align}
					c = a,\ \EQAX \vdash a = c
				\end{align}
				なので
				\begin{align}
					c = a,\ a = b,\ \EXTAX,\EQAX \vdash b = c
				\end{align}
				が成り立ち,また
				\begin{align}
					\EQAX \vdash b = c \rarrow c = b
				\end{align}
				より
				\begin{align}
					c = a,\ a = b,\ \EXTAX,\EQAX \vdash c = b
				\end{align}
				が従い,演繹法則より
				\begin{align}
					a = b,\ \EXTAX,\EQAX \vdash c = a \rarrow c = b
					\label{fom:the_principle_of_substitution_1}
				\end{align}
				が得られる.
				\begin{align}
					a = b,\ \EXTAX,\EQAX \vdash b = a
				\end{align}
				と(\refeq{fom:the_principle_of_substitution_1})より
				\begin{align}
					a = b,\ \EXTAX,\EQAX \vdash c = b \rarrow c = a
				\end{align}
				も得られるので
				\begin{align}
					a = b,\ \EXTAX,\EQAX \vdash c = a \lrarrow c = b
				\end{align}
				が成り立つ.従って$\varphi$が
				\begin{align}
					c = x
				\end{align}
				なる式であるときも
				\begin{align}
					\EXTAX,\EQAX \vdash a = b \rarrow (\, \varphi(a) \lrarrow \varphi(b)\, )
				\end{align}
				が成り立つ.
			
			\item[step4]
				$\varphi$を$x$のみが自由に現れる$\lang{\varepsilon}$の式として
				\begin{itembox}[l]{IH (帰納法の仮定)}
					$\varphi$の任意の真部分式$\psi$に対して,
					$\psi$に$x$が自由に現れているならば
					\begin{align}
						\EXTAX,\EQAX \vdash a = b \rarrow (\, \psi(a) \lrarrow \psi(b)\, )
					\end{align}
				\end{itembox}
				と仮定する.このとき
				\begin{description}
					\item[case1] $\varphi$が
						\begin{align}
							\negation \psi
						\end{align}
						なる式であるとき,(IH)より
						\begin{align}
							a = b,\ \EXTAX,\EQAX \vdash \psi(a) \lrarrow \psi(b)
						\end{align}
						が成り立つので,対偶を取れば
						\begin{align}
							a = b,\ \EXTAX,\EQAX \vdash\ 
							\negation \psi(a) \lrarrow\ \negation \psi(b)
						\end{align}
						が成り立つ.
						
					\item[case2]
						$\varphi$が
						\begin{align}
							\psi \vee \chi
						\end{align}
						なる式であるとき,
						\begin{align}
							\psi_{a} \defarrow
							\begin{cases}
								\psi(a) & \mbox{if $\psi$に$x$が自由に現れる} \\
								\psi & \mbox{if $\psi$に$x$が自由に現れない}
							\end{cases}
						\end{align}
						と定め,同様に$\psi_{b},\chi_{a},\chi_{b}$も定めれば,(IH)より
						\begin{align}
							a = b,\ \EXTAX,\EQAX &\vdash \psi_{a} \lrarrow \psi_{b}, \\
							a = b,\ \EXTAX,\EQAX &\vdash \chi_{a} \lrarrow \chi_{b}
						\end{align}
						が満たされる.
						(\refeq{fom:lemma_for_principle_of_substitution_1})より
						\begin{align}
							a = b,\ \EXTAX,\EQAX \vdash\ 
							\negation \psi_{a} \vee \chi_{a} \lrarrow\ \negation \psi_{b} \vee \chi_{b}
						\end{align}
						が成り立つ.$\varphi$が
						\begin{align}
							\psi \wedge \chi
						\end{align}
						や
						\begin{align}
							\psi \rarrow \chi
						\end{align}
						なる式であるときも同様である.
					
					\item[case3]
						
						
				\end{description}
				
			\begin{comment}
			%%%%%%%%%%%%%%%%%%%%%コメント%%%%%%%%%%%%%%%%%%%%%%
			\item[step2]
				$\varphi$が
				\begin{align}
					x \in \varepsilon y\, R(x,y)
				\end{align}
				なる式であるとき,
				\begin{align}
					a = b,\ \EQAXEP \vdash \varepsilon y\, R(a,y) = \varepsilon y\, R(b,y)
				\end{align}
				となる.
				\begin{align}
					a = b,\ \EXTAX,\EQAX 
					\vdash (\, \varepsilon y\, R(a,y) = \varepsilon y\, R(b,y)\, )
					\rarrow (\, a \in \varepsilon y\, R(a,y) \lrarrow a \in \varepsilon y\, R(b,y)\, )
				\end{align}
				なので
				\begin{align}
					a = b,\ \EXTAX,\EQAX,\EQAXEP \vdash 
					a \in \varepsilon y\, R(a,y) \lrarrow a \in \varepsilon y\, R(b,y)
				\end{align}
				となる.
				\begin{align}
					a=b,\ \EXTAX,\EQAX \vdash 
					a \in \varepsilon y\, R(b,y) \lrarrow b \in \varepsilon y\, R(b,y)
				\end{align}
				も成り立つので
				\begin{align}
					a=b,\ \EXTAX,\EQAX,\EQAXEP \vdash 
					a \in \varepsilon y\, R(a,y) \lrarrow b \in \varepsilon y\, R(b,y)
				\end{align}
				が得られる.
				
			\item[step3]
				$\varphi$が
				\begin{align}
					\varepsilon y R(x,y) \in \varepsilon z T(x,z)
				\end{align}
				なる形のとき,
				\begin{align}
					a = b,\ \EQAXEP \vdash 
					\varepsilon y R(a,y) = \varepsilon y R(b,y)
				\end{align}
				と
				\begin{align}
					a = b,\ \EQAX, \EQAXEP \vdash 
					(\, \varepsilon y R(a,y) = \varepsilon y R(b,y)\, )
					\rarrow (\, \varepsilon y R(a,y) \in \varepsilon z T(a,z)
					\lrarrow \varepsilon y R(b,y) \in \varepsilon z T(a,z)\, )
				\end{align}
				より
				\begin{align}
					a = b,\ \EQAX, \EQAXEP \vdash 
					\varepsilon y R(a,y) \in \varepsilon z T(a,z)
					\lrarrow \varepsilon y R(b,y) \in \varepsilon z T(a,z)
				\end{align}
				が成り立つ.他方で
				\begin{align}
					a = b,\ \EQAXEP \vdash 
					\varepsilon z T(a,z) = \varepsilon z T(b,z)
				\end{align}
				と
				\begin{align}
					a = b,\ \EQAX, \EQAXEP \vdash 
					(\, \varepsilon z T(a,z) = \varepsilon z T(b,z)\, )
					\rarrow (\, \varepsilon y R(b,y) \in \varepsilon z T(a,z)
					\lrarrow \varepsilon y R(b,y) \in \varepsilon z T(b,z)\, )
				\end{align}
				より
				\begin{align}
					a = b,\ \EQAX, \EQAXEP \vdash 
					\varepsilon y R(b,y) \in \varepsilon z T(a,z)
					\lrarrow \varepsilon y R(b,y) \in \varepsilon z T(b,z)
				\end{align}
				が成り立つ.同値関係の推移律
				(\ref{logicalthm:transitive_law_of_equivalence_symbol})より
				\begin{align}
					a = b,\ \EQAX, \EQAXEP \vdash 
					\varepsilon y R(a,y) \in \varepsilon z T(a,z)
					\lrarrow \varepsilon y R(b,y) \in \varepsilon z T(b,z)
				\end{align}
				が成立する.
			%%%%%%%%%%%%%%%%%%%%%コメント%%%%%%%%%%%%%%%%%%%%%%
			\end{comment}
			
		\end{description}
	\end{sketch}
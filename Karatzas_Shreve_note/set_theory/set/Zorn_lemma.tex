\section{極大原理}
	いま,$P$が集合で,$O$が$P$上の順序関係であるとする.$c$が$P$の部分集合で
	\begin{align}
		\forall x,y \in c\, (\, (x,y) \in O \vee (y,x) \in O\, )
	\end{align}
	を満たすとき,つまり$c$のどの$2$要素も比較可能であるということだが,
	$c$を順序集合$(P,O)$の{\bf 鎖}\index{さ@鎖}{\bf (chain)}と呼ぶ.
	ここで$(P,O)$の鎖の全体を
	\begin{align}
		\mathscr{C}
	\end{align}
	とする.つまり$\mathscr{C}$とは
	\begin{align}
		\mathscr{C} \defeq \Set{c}{c \subset P \wedge \forall x,y \in c\, (\, (x,y) \in O \vee (y,x) \in O\, )}
	\end{align}	
	で定められた集合である.$\mathscr{C}$の要素$c$で
	\begin{align}
		\forall u \in \mathscr{C}\, (\, c \subset u \Longrightarrow c=u\, )
	\end{align}
	を満たすものを$(P,O)$の{\bf 極大鎖}\index{きょくだいさ@極大鎖}{\bf (maximal chain)}と呼ぶ.
	つまり,$c$が$(P,O)$の極大鎖であるとは$c$を含む鎖が$c$の他に無いということである.
	
	\begin{screen}
		\begin{thm}[極大鎖は必ず存在する]
		\label{thm:existence_of_maximal_chain}
			$P$を集合とし,$O$を$P$上の順序関係とするとき,$(P,O)$の極大鎖を取ることが出来る.
		\end{thm}
	\end{screen}
	
	$P$が空ならば$O$は当然空であるから(空虚な真により$\emptyset$は$P$上の順序関係である),
	\begin{align}
		\mathscr{C} = \{\emptyset\}
	\end{align}
	が成り立つ.従って$(P,O)$の極大鎖は$\emptyset$ということになる.以下では$P$は空でないとする.
	つまり$P$の要素$x$が取れるわけだが,
	\begin{align}
		\{x\}
	\end{align}
	はこれで一つの鎖をなすので$\mathscr{C}$は空ではない.
	
	\begin{sketch}\mbox{}
		\begin{description}
			\item[第一段]
				始めに全体のあらすじを書いておく.$\mathscr{C}$上の写像$h$を
				\begin{align}
					\mathscr{C} \ni c \longmapsto
					\begin{cases}
						\Set{u \in \mathscr{C}}{ c \subset u \wedge c \neq u} 
						& \mbox{if } \exists u \in \mathscr{C}\, (\, c \subset u \wedge c \neq u\, ) \\
						\{c\} &\mbox{if } \forall u \in \mathscr{C}\, (\, \rightharpoondown c \subset u \vee c=u\, )
					\end{cases}
				\end{align}
				なる関係により定めると,定理(\ref{thm:direct_product_of_non_empty_sets_is_not_empty})より
				$\mathscr{C}$上の写像$f$で,$\mathscr{C}$の任意の要素$c$に対して
				\begin{align}
					\begin{cases}
						c \subset f(c) \wedge c \neq f(c) 
						& \mbox{if } \exists u \in \mathscr{C}\, (\, c \subset u \wedge c \neq u\, ) \\
						c=f(c) & \mbox{if } \forall u \in \mathscr{C}\, (\, \rightharpoondown c \subset u \vee c=u\, )
					\end{cases}
				\end{align}
				を満たすものが取れる.ここで$\mathscr{C}$から要素$c$を選び,また$\Univ$の上の写像$G$を
				\begin{align}
					x \longmapsto
					\begin{cases}
						c & \mbox{if } \dom{x} = \emptyset \\
						f(x(\beta)) & \mbox{if } \beta \in \ON \wedge \dom{x} = \beta+1 \wedge x(\beta) \in \mathscr{C} \\
						\bigcup \ran{x} & \mbox{otherwise}
					\end{cases}
				\end{align}
				なる関係により定めると,$\ON$上の写像$F$で
				\begin{itemize}
					\item $F(0) = c$
					\item 任意の順序数$\alpha$に対して,$F(\alpha) \in \mathscr{C}$ならば$F(\alpha+1)=f(F(\alpha))$
					\item $\alpha$が極限数ならば$F(\alpha)=\bigcup_{\beta \in \alpha} F(\beta)$
				\end{itemize}
				を満たすものが取れる.以降はこの$F$をメインに考察する.
				
				まず,次の第二段では,任意の順序数$\alpha$に対して$F(\alpha)$が$(P,O)$の鎖であること,つまり
				\begin{align}
					\forall \alpha \in \ON\, (\, F(\alpha) \in \mathscr{C}\, )
				\end{align}
				が成り立つことを示す.
				
				第三段では,$\alpha$を極限数としたときに,$F(\alpha)$が極大鎖でないならば
				\begin{align}
					\card{\alpha} \leq \card{F(\alpha)}
				\end{align}
				が成り立つことを示す.
				
				第四段では,任意の順序数$\beta$に対して,
				\begin{align}
					\card{\beta} \leq \card{F(\beta)}
				\end{align}
				が満たされていてかつ$F(\beta)$が極大鎖ではないときに
				\begin{align}
					\card{(\beta+1)} \leq \card{F(\beta+1)}
				\end{align}
				が成り立つこと,つまり
				\begin{align}
					\card{\beta} \leq \card{F(\beta)} \wedge
					\exists u \in \mathscr{C}\, \left(\, F(\beta) \subset u \wedge F(\beta) \neq u\, \right)
					\Longrightarrow \card{(\beta+1)} \leq \card{F(\beta+1)}
				\end{align}
				が成り立つことを示す.
				
				以上を踏まえて,いま
				\begin{align}
					\delta \defeq \card{P}
				\end{align}
				とおけば,定理\ref{thm:no_ordinal_number_is_bigger_than_its_aleph_number}より
				\begin{align}
					\card{F(\aleph_{\delta+1})} \leq \card{P}
					=\delta < \aleph_{\delta+1}
				\end{align}
				が成り立つから,
				\begin{align}
					\card{F(\alpha)} < \card{\alpha}
				\end{align}
				を満たす最小の順序数$\alpha$を取ることが出来る.
				\begin{align}
					0 \leq \card{F(0)}
				\end{align}
				なので$\alpha$は$0$ではない.
				$\alpha$が極限数であるとき,第三段の対偶より$F(\alpha)$は極大鎖である.
				$\alpha$が極限数でないとき,つまり
				\begin{align}
					\alpha = \beta+1
				\end{align}
				を満たす順序数$\beta$が取れるとき,$\alpha$の取り方より
				\begin{align}
					\card{\beta} \leq \card{F(\beta)}
				\end{align}
				が成り立つので,第四段の対偶と選言三段論法(P. \pageref{logicalthm:disjunctive_syllogism})より
				\begin{align}
					\rightharpoondown \exists u \in \mathscr{C}\, (\, F(\beta) \subset u \wedge F(\beta) \neq u\, )
				\end{align}
				が従う.つまり$F(\beta)$は$(P,O)$の極大鎖である.以上より,いずれの場合も極大鎖は取れる.
			
			\item[第二段]
				 始めに
				 \begin{align}
					\forall \beta,\gamma \in \ON\,
					\left(\, \beta < \gamma \wedge \forall \delta \in \gamma\, (\, F(\delta) \in \mathscr{C}\, )
					\Longrightarrow F(\beta) \subset F(\gamma)\, \right)
				\end{align}
				が成り立つことを示す.これは,$\beta$を任意の順序数として
				\begin{align}
					\forall \gamma \in \ON\, \left(\, \beta < \gamma \wedge
					\forall \delta \in \gamma\, (\, F(\delta) \in \mathscr{C}\, )
					\Longrightarrow F(\beta) \subset F(\gamma)\, \right)
				\end{align}
				が成り立つことを超限帰納法で示せばよい.
				\begin{align}
					\gamma = 0
				\end{align}
				の場合は
				\begin{align}
					\beta < \gamma \wedge \forall \delta \in \gamma\, (\, F(\delta) \in \mathscr{C}\, )
				\end{align}
				の仮定が偽になるので
				\begin{align}
					\beta < \gamma \wedge \forall \delta \in \gamma\, (\, F(\delta) \in \mathscr{C}\, )
					\Longrightarrow F(\beta) \subset F(\gamma)
				\end{align}
				は成立する.次に$\gamma$を$0$でない順序数として,$\gamma$の任意の要素$\rho$に対して
				\begin{align}
					\beta < \rho \wedge \forall \delta \in \rho\, (\, F(\delta) \in \mathscr{C}\, )
					\Longrightarrow F(\beta) \subset F(\rho)
					\label{fom:thm_existence_of_maximal_chain_1}
				\end{align}
				が成り立っているとし,この下で
				\begin{align}
					\beta < \gamma \wedge \forall \delta \in \gamma\, (\, F(\delta) \in \mathscr{C}\, )
				\end{align}
				が満たされているとする.
				\begin{align}
					\gamma = \rho + 1
				\end{align}
				を満たす順序数$\rho$が取れるとき,
				\begin{align}
					\beta = \rho
				\end{align}
				ならば
				\begin{align}
					F(\beta) \subset f(F(\beta)) = F(\gamma)
				\end{align}
				が成り立ち,
				\begin{align}
					\beta < \rho
				\end{align}
				ならば(\refeq{fom:thm_existence_of_maximal_chain_1})より
				\begin{align}
					F(\beta) \subset F(\rho) \subset f(F(\rho)) = F(\gamma)
				\end{align}
				が成り立つ.$\gamma$が極限数であるとき,
				\begin{align}
					F(\beta) \subset \bigcup_{\delta \in \gamma} F(\delta) = F(\gamma)
				\end{align}
				が成り立つ.以上と超限帰納法より
				\begin{align}
					\forall \gamma \in \ON\, \left(\, \beta < \gamma \wedge
					\forall \delta \in \gamma\, (\, F(\delta) \in \mathscr{C}\, )
					\Longrightarrow F(\beta) \subset F(\gamma)\, \right)
				\end{align}
				を得る.
				
				では,任意の順序数$\alpha$に対して$F(\alpha)$が鎖であることを超限帰納法により示す.まず
				\begin{align}
					F(0) = c
				\end{align}
				なので
				\begin{align}
					F(0) \in \mathscr{C}
				\end{align}
				が成り立つ.次に$\alpha$を$0$でない任意の順序数として
				\begin{align}
					\forall \beta \in \alpha\, (\, F(\beta) \in \mathscr{C}\, )
				\end{align}
				が満たされているとする.このとき
				\begin{align}
					\alpha = \beta + 1
				\end{align}
				を満たす順序数$\beta$が取れるならば
				\begin{align}
					F(\alpha) = f(F(\beta))
				\end{align}
				が成り立つので$F(\alpha)$は鎖である.
				$\alpha$が極限数である場合,$x$と$y$を$F(\alpha)$の任意の要素とすれば,
				\begin{align}
					F(\alpha) = \bigcup_{\beta \in \alpha} F(\beta)
				\end{align}
				より
				\begin{align}
					x \in F(\beta)
				\end{align}
				を満たす$\alpha$の要素$\beta$と
				\begin{align}
					y \in F(\gamma)
				\end{align}
				を満たす$\alpha$の要素$\gamma$が取れて,
				\begin{align}
					F(\beta) \subset F(\gamma)
				\end{align}
				または
				\begin{align}
					F(\gamma) \subset F(\beta)
				\end{align}
				が成り立つが,いずれの場合も
				\begin{align}
					(x,y) \in O \vee (y,x) \in O
				\end{align}
				が従う.つまりこの場合も$F(\alpha)$は鎖である.よって超限帰納法より
				\begin{align}
					\forall \alpha \in \ON\, (\, F(\alpha) \in \mathscr{C}\, )
				\end{align}
				が成立する.
				
			\item[第三段]
				$\alpha$が極限数であるとし,$F(\alpha)$が極大鎖でないとする.$\alpha$の任意の要素$\beta$に対して
				\begin{align}
					F(\beta) \subset F(\alpha)
				\end{align}
				が成り立つので$F(\beta)$もまた極大鎖ではなく,ゆえに
				\begin{align}
					F(\beta) \subset F(\beta + 1) \wedge F(\beta) \neq F(\beta + 1)
				\end{align}
				が成立する.定理(\ref{thm:direct_product_of_non_empty_sets_is_not_empty})より
				$\alpha$上の写像$g$で,$\alpha$の任意の要素$\beta$に対して
				\begin{align}
					g(\beta) \in F(\beta+1) \wedge g(\beta) \notin F(\beta)
				\end{align}
				を満たすものが取れる.この$g$は単射である.実際,$\alpha$の二要素$\beta$と$\gamma$に対して,
				\begin{align}
					\beta < \gamma
				\end{align}
				ならば
				\begin{align}
					F(\beta+1) \subset F(\gamma)
				\end{align}
				が成り立つが,
				\begin{align}
					g(\beta) \in F(\beta + 1)
				\end{align}
				かつ
				\begin{align}
					g(\gamma) \notin F(\gamma)
				\end{align}
				が満たされているので
				\begin{align}
					g(\beta) \neq g(\gamma)
				\end{align}
				が従う.$\alpha$から$F(\alpha)$への単射が得られたので
				\begin{align}
					\card{\alpha} \leq \card{F(\alpha)}
				\end{align}
				が成立する.
			
			\item[第四段]
				いま$\beta$を順序数として
				\begin{align}
					\card{\beta} \leq \card{F(\beta)}
				\end{align}
				が満たされていて,かつ$F(\beta)$は極大鎖ではないとする.
				ここで$\beta$から$\card{\beta}$への全単射を$p$とし,
				$\card{F(\beta)}$から$F(\beta)$への全単射を$q$とすると,
				$q \circ p$は$\beta$から$F(\beta)$への単射である.
				いま$F(\beta)$は極大鎖ではないので
				\begin{align}
					x \notin F(\beta) \wedge x \in F(\beta + 1)
				\end{align}
				を満たす$x$が取れて,
				\begin{align}
					g \defeq (q \circ p) \cup \{(\beta,x)\}
				\end{align}
				と$g$を定めると,$g$は$\beta+1$から$F(\beta+1)$への単射となる.すなわち
				\begin{align}
					\card{(\beta+1)} \leq \card{F(\beta+1)}
				\end{align}
				が成立する.
				\QED
		\end{description}
	\end{sketch}
\section{対}
	$a$と$b$を類とするとき,$a$か$b$の少なくとも一方に等しい集合の全体,つまり
	\begin{align}
		a = x \vee b = x
	\end{align}
	を満たす全ての集合$x$を集めたものを$a$と$b$の対と呼び
	\begin{align}
		\{a,b\}
	\end{align}
	と書く.解釈としては``$a$と$b$のみを要素とする類''のことであり,当然$a$が集合であるならば
	\begin{align}
		a \in \{a,b\}
	\end{align}
	が成立する.しかし$a$と$b$が共に真類であるときは,いかなる集合も$a$にも$b$にも等しくないため
	\begin{align}
		\{a,b\} = \emptyset
	\end{align}
	となる.以上が大雑把な対の説明である.
	
	\begin{screen}
		\begin{dfn}[対]
			$x,y$を$\mathcal{L}$の項とし,$z$を$x$にも$y$にも自由に現れない変項とするとき,
			\begin{align}
				\{x,y\} \defeq \Set{z}{x = z \vee y = z}
			\end{align}
			で$\{x,y\}$を定義し,これを$x$と$y$の{\bf 対}\index{つい@対}{\bf (pair)}と呼ぶ.
			特に$\{x,x\}$を$\{x\}$と書く.
		\end{dfn}
	\end{screen}
	
	上の定義では省略したが,$x$や$y$が内包項である場合は$z = x \vee z = y$を
	$\lang{\varepsilon}$の式に書き換えてから$\{x,y\}$を定めるのである.つまり
	\begin{align}
		\varphi \defarrow x = z \vee y = z
	\end{align}
	とおけば,$\varphi$を$\lang{\varepsilon}$の式に書き換えた式$\hat{\varphi}$によって
	\begin{align}
		\{x,y\} \defeq \Set{z}{\hat{\varphi}(z)}
	\end{align}
	と定めるのである.たとえば$a$や$b$を類として
	\begin{align}
		\varphi \defarrow a = z \vee b = z
	\end{align}
	とおけば,
	\begin{align}
		\COMAX \vdash \forall z\, (\, z \in \{a,b\} \lrarrow \hat{\varphi}(z)\, )
	\end{align}
	が成立するし,同時に定理\ref{thm:equivalent_formula_rewriting_1}と
	定理\ref{thm:equivalent_formula_rewriting_2}より
	\begin{align}
		\EXTAX,\EQAX,\COMAX \vdash 
		\forall z\, (\, \hat{\varphi}(z) \lrarrow \varphi(z)\, )
	\end{align}
	も成り立つので
	\begin{align}
		\EXTAX,\EQAX,\COMAX \vdash 
		\forall z\, (\, z \in \{a,b\} \lrarrow a = z \vee b = z\, )
	\end{align}
	が得られる.
	
	\begin{screen}
		\begin{thm}[対は表示されている要素しか持たない]
		\label{thm:pair_members_are_exactly_the_given_two}
			$a$と$b$を類とするとき次が成立する:
			\begin{align}
				\EXTAX,\EQAX,\COMAX \vdash 
				\forall x\, (\, x \in \{a,b\} \lrarrow a = x \vee b = x\, ).
			\end{align}
		\end{thm}
	\end{screen}
	
	$\ELEAX$を加えれば次が得られる.
	
	\begin{screen}
		\begin{thm}[対の要素は表示されている要素の一方には等しい]
		\label{cor:pair_members_are_exactly_the_given_two}
			$a,b,c$を類とするとき次が成立する:
			\begin{align}
				\EXTAX,\EQAX,\COMAX,\ELEAX \vdash 
				c \in \{a,b\} \rarrow a = c \vee b = c.
			\end{align}
		\end{thm}
	\end{screen}
	
	\begin{sketch}
		要素の公理より
		\begin{align}
			c \in \{a,b\},\ \ELEAX \vdash \set{c}
		\end{align}
		が成り立つので
		\begin{align}
			\tau \defeq \varepsilon s\, (\, c = s\, )
		\end{align}
		とおけば
		\begin{align}
			c \in \{a,b\},\ \ELEAX \vdash c = \tau
			\label{fom:pair_members_are_exactly_the_given_two_2}
		\end{align}
		となる.$\tau$に対しては定理\ref{thm:pair_members_are_exactly_the_given_two}より 
		\begin{align}
			\EXTAX,\EQAX,\COMAX \vdash 
			\tau \in \{a,b\} \rarrow a = \tau \vee b = \tau
		\end{align}
		が成り立つが,ここで(\refeq{fom:pair_members_are_exactly_the_given_two_2})より
		\begin{align}
			c \in \{a,b\},\ \EQAX,\ELEAX \vdash \tau \in \{a,b\}
		\end{align}
		となるので
		\begin{align}
			c \in \{a,b\},\ \EXTAX,\EQAX,\COMAX,\ELEAX \vdash a = \tau \vee b = \tau
		\end{align}
		が従い,代入原理(定理\ref{thm:the_principle_of_substitution})と
		(\refeq{fom:pair_members_are_exactly_the_given_two_2})より
		\begin{align}
			c \in \{a,b\},\ \EXTAX,\EQAX,\COMAX,\ELEAX \vdash a = c \vee b = c
		\end{align}
		が得られる.
		\QED
	\end{sketch}
	
	この逆,つまり
	\begin{align}
		a = c \vee b = c \rarrow c \in \{a,b\}
	\end{align}
	は一般には成立しない.実際$a,b$が共に真類であるときは
	\begin{align}
		\{a,b\} = \emptyset
	\end{align}
	となるためである(定理\ref{thm:pair_of_proper_classes_is_emptyset}).
	
	\begin{screen}
		\begin{thm}[表示の順番を入れ替えても対は等しい]
		\label{thm:commutative_law_of_pairs}
			$a$と$b$を類とするとき
			\begin{align}
				\EXTAX,\EQAX,\COMAX \vdash \{a,b\} = \{b,a\}.
			\end{align}
		\end{thm}
	\end{screen}
	
	\begin{sketch}
		いま
		\begin{align}
			\tau \defeq \varepsilon x \negation (\, x \in \{a,b\} \lrarrow x \in \{b,a\}\, )
		\end{align}
		とおく(必要に応じて$x \in \{a,b\} \lrarrow x \in \{b,a\}$は$\lang{\varepsilon}$の
		式に書き換える).
		定理\ref{thm:pair_members_are_exactly_the_given_two}より
		\begin{align}
			\EXTAX,\EQAX,\COMAX \vdash
			\tau \in \{a,b\} \rarrow a = \tau \vee b = \tau
		\end{align}
		が成り立つので,演繹定理の逆より
		\begin{align}
			\tau \in \{a,b\},\ \EXTAX,\EQAX,\COMAX \vdash a = \tau \vee b = \tau
		\end{align}
		となる.また論理和の可換律
		(推論法則\ref{logicalthm:commutative_law_of_disjunction})より
		\begin{align}
			\tau \in \{a,b\},\ \EXTAX,\EQAX,\COMAX \vdash b = \tau \vee a = \tau
		\end{align}
		が成り立ち,定理\ref{thm:pair_members_are_exactly_the_given_two}より
		\begin{align}
			\tau \in \{a,b\},\ \EXTAX,\EQAX,\COMAX \vdash \tau \in \{b,a\}
		\end{align}
		が従う.そして演繹定理より
		\begin{align}
			\EXTAX,\EQAX,\COMAX \vdash \tau \in \{a,b\} \rarrow \tau \in \{b,a\}
		\end{align}
		が得られる.$a$と$b$を入れ替えれば
		\begin{align}
			\EXTAX,\EQAX,\COMAX \vdash \tau \in \{b,a\} \rarrow \tau \in \{a,b\}
		\end{align}
		が得られるので,論理積の導入より
		\begin{align}
			\EXTAX,\EQAX,\COMAX \vdash \tau \in \{a,b\} \lrarrow \tau \in \{b,a\}
		\end{align}
		が成り立ち,全称の導出(推論法則\ref{logicalthm:derivation_of_universal_by_epsilon})より
		\begin{align}
			\EXTAX,\EQAX,\COMAX \vdash \forall x\, (\, x \in \{a,b\} \lrarrow x \in \{b,a\}\, )
		\end{align}
		となり,外延性公理より
		\begin{align}
			\EXTAX,\EQAX,\COMAX \vdash \{a,b\} = \{b,a\}
		\end{align}
		が従う.
		\QED
	\end{sketch}
		
	\begin{screen}
		\begin{axm}[対の公理] 次の式を$\PAIAX$により参照する:
			\begin{align}
				\forall x\, \forall y\, \exists p\, \forall z\, 
				(\, x = z \vee y = z \lrarrow z \in p\, ).
			\end{align}
		\end{axm}
	\end{screen}
	
	\begin{screen}
		\begin{thm}[集合の対は集合である]
		\label{thm:pair_of_sets_is_a_set}
			$a$と$b$を類とするとき
			\begin{align}
				\EXTAX,\EQAX,\COMAX,\PAIAX \vdash 
				\set{a} \wedge \set{b} \rarrow \set{\{a,b\}}.
			\end{align}
		\end{thm}
	\end{screen}
	
	\begin{sketch}\mbox{}
		\begin{description}
			\item[step1]
				論理積の除去より
				\begin{align}
					\set{a} \wedge \set{b} &\vdash \exists x\, (\, a = x\, ), \\
					\set{a} \wedge \set{b} &\vdash \exists x\, (\, b = x\, )
				\end{align}
				が成り立つので,
				\begin{align}
					\tau &\defeq \varepsilon x\, (\, a = x\, ), \\
					\sigma &\defeq \varepsilon x\, (\, b = x\, )
				\end{align}
				とおけば
				\begin{align}
					\set{a} \wedge \set{b} &\vdash a = \tau, 
					\label{fom:pair_of_sets_is_a_set_1} \\
					\set{a} \wedge \set{b} &\vdash b = \sigma
				\end{align}
				が成り立つ.対の公理より$\tau$と$\sigma$に対しては
				\begin{align}
					\PAIAX \vdash \exists p\, \forall z\, 
						(\, \tau = z \vee \sigma = z \lrarrow z \in p\, )
				\end{align}
				が成り立つので,
				\begin{align}
					\rho \defeq \varepsilon p\, \forall z\, 
						(\, \tau = z \vee \sigma = z \lrarrow z \in p\, )
				\end{align}
				とおけば
				\begin{align}
					\PAIAX \vdash \forall z\, (\, \tau = z \vee \sigma = z \lrarrow z \in \rho\, )
					\label{fom:pair_of_sets_is_a_set_2}
				\end{align}
				となる.
				
			\item[step2]
				次に
				\begin{align}
					\forall z\, (\, z \in \{a,b\} \lrarrow z \in \rho\, )
				\end{align}
				を示すために
				\begin{align}
					\zeta \defeq \varepsilon z \negation (\, z \in \{a,b\} \lrarrow z \in \rho\, )
				\end{align}
				とおく(当然$\lang{\varepsilon}$の式に書き換える).
				等号の推移律(定理\ref{thm:transitive_law_of_equality})より
				\begin{align}
					\EXTAX,\EQAX \vdash a = \tau \rarrow (\, a = \zeta \rarrow \tau = \zeta\, )
				\end{align}
				が成り立つので,(\refeq{fom:pair_of_sets_is_a_set_1})との三段論法より
				\begin{align}
					\set{a} \wedge \set{b},\ \EXTAX,\EQAX \vdash 
					a = \zeta \rarrow \tau = \zeta
				\end{align}
				が成り立ち,論理和の導入より
				\begin{align}
					\set{a} \wedge \set{b},\ \EXTAX,\EQAX \vdash 
					a = \zeta \rarrow \tau = \zeta \vee \sigma = \zeta
				\end{align}
				が従う.同様にして
				\begin{align}
					\set{a} \wedge \set{b},\ \EXTAX,\EQAX \vdash 
					b = \zeta \rarrow \tau = \zeta \vee \sigma = \zeta
				\end{align}
				も成り立つので,論理和の除去より
				\begin{align}
					\set{a} \wedge \set{b},\ \EXTAX,\EQAX \vdash 
					a = \zeta \vee b = \zeta \rarrow \tau = \zeta \vee \sigma = \zeta
				\end{align}
				が得られる.同様に
				\begin{align}
					\set{a} \wedge \set{b},\ \EXTAX,\EQAX \vdash 
					\tau = \zeta \vee \sigma = \zeta \rarrow a = \zeta \vee b = \zeta
				\end{align}
				も得られ,論理積の導入より
				\begin{align}
					\set{a} \wedge \set{b},\ \EXTAX,\EQAX \vdash 
					a = \zeta \vee b = \zeta \lrarrow \tau = \zeta \vee \sigma = \zeta
				\end{align}
				が従う.他方で定理\ref{thm:pair_members_are_exactly_the_given_two}より
				\begin{align}
					\EXTAX,\EQAX,\COMAX \vdash 
					\zeta \in \{a,b\} \lrarrow a = \zeta \vee b = \zeta
				\end{align}
				が成り立ち,また(\refeq{fom:pair_of_sets_is_a_set_2})より
				\begin{align}
					\PAIAX \vdash \tau = \zeta \vee \sigma = \zeta \lrarrow \zeta \in \rho
				\end{align}
				も成り立つので,同値記号の推移律
				(推論法則\ref{logicalthm:transitive_law_of_equivalence_symbol})より
				\begin{align}
					\set{a} \wedge \set{b},\ \EXTAX,\EQAX,\COMAX,\PAIAX \vdash 
					\zeta \in \{a,b\} \lrarrow \zeta \in \rho
				\end{align}
				が従う.そして全称の導出(推論法則\ref{logicalthm:derivation_of_universal_by_epsilon})より
				\begin{align}
					\set{a} \wedge \set{b},\ \EXTAX,\EQAX,\COMAX,\PAIAX \vdash 
					\forall z\, (\, z \in \{a,b\} \lrarrow z \in \rho\, )
				\end{align}
				が成り立ち,外延性公理より
				\begin{align}
					\set{a} \wedge \set{b},\ \EXTAX,\EQAX,\COMAX,\PAIAX \vdash 
					\{a,b\} = \rho
				\end{align}
				が従い,存在記号の推論公理より
				\begin{align}
					\set{a} \wedge \set{b},\ \EXTAX,\EQAX,\COMAX,\PAIAX \vdash 
					\exists p\, (\, \{a,b\} = p\, )
				\end{align}
				が成り立つ.
				\QED
		\end{description}
	\end{sketch}
	
	\begin{screen}
		\begin{thm}[集合は自分自身の対の要素である]
		\label{thm:set_is_an_element_of_its_pair}
			$a$と$b$を類とするとき
			\begin{align}
				\EXTAX,\EQAX,\COMAX &\vdash \set{a} \rarrow a \in \{a,b\}, \\
				\EXTAX,\EQAX,\COMAX &\vdash \set{b} \rarrow b \in \{a,b\}.
			\end{align}
		\end{thm}
	\end{screen}
	
	\begin{sketch}\mbox{}
		\begin{description}
			\item[step1]
				いま
				\begin{align}
					\tau \defeq \varepsilon x\, (\, a = x\, )
				\end{align}
				とおくと
				\begin{align}
					\set{a} \vdash a = \tau
					\label{fom:set_is_an_element_of_its_pair_1}
				\end{align}
				が成り立ち,論理和の導入より
				\begin{align}
					\set{a} \vdash a = \tau \vee b = \tau
				\end{align}
				も成り立つ.定理\ref{thm:pair_members_are_exactly_the_given_two}より
				\begin{align}
					\EXTAX,\EQAX,\COMAX \vdash 
					a = \tau \vee b = \tau \rarrow \tau \in \{a,b\}
				\end{align}
				が成り立つので三段論法より
				\begin{align}
					\set{a},\ \EXTAX,\EQAX,\COMAX \vdash \tau \in \{a,b\}
					\label{fom:set_is_an_element_of_its_pair_2}
				\end{align}
				が従う.また(\refeq{fom:set_is_an_element_of_its_pair_1})と相等性公理より
				\begin{align}
					\set{a},\ \EQAX \vdash \tau = a
				\end{align}
				となり
				\begin{align}
					\set{a},\ \EQAX \vdash \tau \in \{a,b\} \rarrow a \in \{a,b\}
				\end{align}
				となるので,(\refeq{fom:set_is_an_element_of_its_pair_2})と三段論法より
				\begin{align}
					\set{a},\ \EXTAX,\EQAX,\COMAX \vdash a \in \{a,b\}
				\end{align}
				が成立する.
			
			\item[step2]
				前段で$a$と$b$を入れ替えれば
				\begin{align}
					\set{b},\ \EXTAX,\EQAX,\COMAX \vdash b \in \{b,a\}
					\label{fom:set_is_an_element_of_its_pair_3}
				\end{align}
				が成立する.ところで対の対称性(定理\ref{thm:commutative_law_of_pairs})より
				\begin{align}
					\EXTAX,\EQAX,\COMAX \vdash \{b,a\} = \{a,b\}
				\end{align}
				が成立し,また相等性公理より
				\begin{align}
					\EQAX \vdash \{b,a\} = \{a,b\}
					\rarrow (\, b \in \{b,a\} \rarrow b \in \{a,b\}\, )
				\end{align}
				も成り立つので,三段論法より
				\begin{align}
					\EXTAX,\EQAX,\COMAX \vdash b \in \{b,a\} \rarrow b \in \{a,b\}
					\label{fom:set_is_an_element_of_its_pair_4}
				\end{align}
				が従う.(\refeq{fom:set_is_an_element_of_its_pair_3})と
				(\refeq{fom:set_is_an_element_of_its_pair_4})と三段論法より
				\begin{align}
					\set{b},\ \EXTAX,\EQAX,\COMAX \vdash b \in \{a,b\}
				\end{align}
				が得られる.
				\QED
		\end{description}
	\end{sketch}
	
	$a$を集合とすれば対の公理より$\{a\}$も集合となるので,
	定理\ref{thm:set_is_an_element_of_its_pair}より
	\begin{align}
		\EXTAX,\EQAX,\COMAX \vdash \set{a} \rarrow a \in \{a\}
	\end{align}
	が成立する.一方で$a$も$b$も真類であると$\{a,b\}$は空になる.
	
	\begin{screen}
		\begin{thm}[真類同士の対は空]\label{thm:pair_of_proper_classes_is_emptyset}
			$a$と$b$を類とするとき,
			\begin{align}
				\EXTAX,\EQAX,\COMAX \vdash\ 
				\negation \set{a} \wedge \negation \set{b} \rarrow \{a,b\} = \emptyset.
			\end{align}
		\end{thm}
	\end{screen}
	
	\begin{sketch}
		いま
		\begin{align}
			\tau \defeq \varepsilon x \negation (\, x \notin \{a,b\}\, )
		\end{align}
		とおく($x \notin \{a,b\}$は$\lang{\varepsilon}$の式に書き換える).
		\begin{align}
			\negation \set{a} \wedge\ \negation \set{b}
			\vdash\ \negation \exists x\, (\, a = x\, )
		\end{align}
		が成り立ち,De Morgan の法則
		(推論法則\ref{logicalthm:strong_De_Morgan_law_for_quantifiers_2})より
		\begin{align}
			\negation \set{a} \wedge \negation \set{b}
			\vdash \forall x\, (\, a \neq x\, )
		\end{align}
		が従い,全称記号の推論公理より
		\begin{align}
			\negation \set{a} \wedge \negation \set{b} \vdash a \neq \tau
		\end{align}
		となる.同様にして
		\begin{align}
			\negation \set{a} \wedge \negation \set{b} \vdash b \neq \tau
		\end{align}
		も成り立つので,論理積の導入より
		\begin{align}
			\negation \set{a} \wedge \negation \set{b} \vdash
			a \neq \tau \wedge b \neq \tau
		\end{align}
		が成立し,De Morgan の法則(推論法則\ref{logicalthm:weak_De_Morgan_law_1})より
		\begin{align}
			\negation \set{a} \wedge \negation \set{b} \vdash\ 
			\negation (\, a = \tau \vee b = \tau\, )
			\label{fom:pair_of_proper_classes_is_emptyset_1}
		\end{align}
		が従う.ところで定理\ref{thm:pair_members_are_exactly_the_given_two}より
		\begin{align}
			\EXTAX,\EQAX,\COMAX \vdash \tau \in \{a,b\} \rarrow a = \tau \vee b = \tau
		\end{align}
		が成り立つので,対偶を取って
		\begin{align}
			\EXTAX,\EQAX,\COMAX \vdash\ 
			\negation (\, a = \vee b = \tau\, ) \rarrow \tau \notin \{a,b\}
		\end{align}
		が成り立つ(推論法則\ref{logicalthm:introduction_of_contraposition}).
		そして(\refeq{fom:pair_of_proper_classes_is_emptyset_1})との三段論法より
		\begin{align}
			\negation \set{a} \wedge \negation \set{b},\ \EXTAX,\EQAX,\COMAX \vdash
			\tau \notin \{a,b\}
		\end{align}
		が従い,全称の導出(推論法則\ref{logicalthm:derivation_of_universal_by_epsilon})より
		\begin{align}
			\negation \set{a} \wedge \negation \set{b},\ \EXTAX,\EQAX,\COMAX \vdash
			\forall x\, (\, x \notin \{a,b\}\, )
		\end{align}
		が従う.要素を持たない類は空集合である(定理\ref{thm:uniqueness_of_emptyset})ので
		\begin{align}
			\negation \set{a} \wedge \negation \set{b},\ \EXTAX,\EQAX,\COMAX \vdash
			\{a,b\} = \emptyset
		\end{align}
		が得られる.
		\QED
	\end{sketch}
	
	上の定理とは逆に$\{a,b\}$が空ならば$a$も$b$も真類である.
	
	\begin{screen}
		\begin{thm}[空な対に表示されている類は集合ではない]
		\label{thm:classes_displayed_in_empty_pair_are_not_sets}
			$a$と$b$を類とするとき,
			\begin{align}
				\EXTAX,\EQAX,\COMAX \vdash \{a,b\} = \emptyset \rarrow\ \negation \set{a} \wedge \negation \set{b}.
			\end{align}
		\end{thm}
	\end{screen}
	
	\begin{sketch}
		いま
		\begin{align}
			\tau \defeq \varepsilon x\, (\, a = x\, )
		\end{align}
		とおけば
		\begin{align}
			\set{a} \vdash a = \tau
		\end{align}
		が成立し,また定理\ref{thm:set_is_an_element_of_its_pair}より
		\begin{align}
			\set{a},\ \EXTAX,\EQAX,\COMAX \vdash a \in \{a,b\}
		\end{align}
		が成り立つので相等性公理より
		\begin{align}
			\set{a},\ \EXTAX,\EQAX,\COMAX \vdash \tau \in \{a,b\}
		\end{align}
		が従い,存在記号の推論公理より
		\begin{align}
			\set{a},\ \EXTAX,\EQAX,\COMAX \vdash \exists x\, (\, x \in \{a,b\}\, )
		\end{align}
		が成り立つ.演繹定理より
		\begin{align}
			\EXTAX,\EQAX,\COMAX \vdash \set{a} \rarrow \exists x\, (\, x \in \{a,b\}\, )
		\end{align}
		となり,対偶を取れば
		\begin{align}
			\EXTAX,\EQAX,\COMAX \vdash\ \negation \exists x\, (\, x \in \{a,b\}\, )
			\rarrow\ \negation \set{a}
			\label{fom:classes_displayed_in_empty_pair_are_not_sets_1}
		\end{align}
		が得られる(推論法則\ref{logicalthm:introduction_of_contraposition}).
		他方で空の類は要素を持たない(定理\ref{thm:uniqueness_of_emptyset})ので
		\begin{align}
			\EXTAX,\EQAX,\COMAX \vdash \{a,b\} = \emptyset \rarrow \forall x\, (\, x \notin \{a,b\}\, )
			\label{fom:classes_displayed_in_empty_pair_are_not_sets_2}
		\end{align}
		が成り立ち,また De Morgan の法則
		(推論法則\ref{logicalthm:strong_De_Morgan_law_for_quantifiers_1})より
		\begin{align}
			\vdash \forall x\, (\, x \notin \{a,b\}\, ) \rarrow\ \negation \exists x\, (\, x \in \{a,b\}\, )
			\label{fom:classes_displayed_in_empty_pair_are_not_sets_3}
		\end{align}
		も成り立つので,(\refeq{fom:classes_displayed_in_empty_pair_are_not_sets_2})
		(\refeq{fom:classes_displayed_in_empty_pair_are_not_sets_3})
		(\refeq{fom:classes_displayed_in_empty_pair_are_not_sets_1})を併せて
		\begin{align}
			\EXTAX,\EQAX,\COMAX \vdash \{a,b\} = \emptyset \rarrow\ \negation \set{a}
		\end{align}
		が従う.同様にして
		\begin{align}
			\EXTAX,\EQAX,\COMAX \vdash \{a,b\} = \emptyset \rarrow\ \negation \set{b}
		\end{align}
		も成り立ち,論理積の導入より
		\begin{align}
			\EXTAX,\EQAX,\COMAX \vdash \{a,b\} = \emptyset \rarrow\ \negation \set{a} \wedge \negation \set{b}
		\end{align}
		が得られる.
		\QED
	\end{sketch}
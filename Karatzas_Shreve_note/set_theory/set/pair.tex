\section{対}
	$a$と$b$を類とするとき,$a$か$b$の少なくとも一方に等しい集合の全体,つまり
	\begin{align}
		a = x \vee b = x
	\end{align}
	を満たす全ての集合$x$を集めたものを$a$と$b$の対と呼び
	\begin{align}
		\{a,b\}
	\end{align}
	と書く.解釈としては``$a$と$b$のみを要素とする類''のことであり,当然$a$が集合であるならば
	\begin{align}
		a \in \{a,b\}
	\end{align}
	が成立する.しかし$a$と$b$が共に真類であるときは,いかなる集合も$a$にも$b$にも等しくないため
	\begin{align}
		\{a,b\} = \emptyset
	\end{align}
	となる.以上が大雑把な対の説明である.
	
	\begin{screen}
		\begin{dfn}[対]
			$x,y$を$\mathcal{L}$の項とし,$z$を$x$にも$y$にも自由に現れない変項とするとき,
			\begin{align}
				\{x,y\} \defeq \Set{z}{x = z \vee y = z}
			\end{align}
			で$\{x,y\}$を定義し,これを$x$と$y$の{\bf 対}\index{つい@対}{\bf (pair)}と呼ぶ.
			特に$\{x,x\}$を$\{x\}$と書く.
		\end{dfn}
	\end{screen}
	
	上の定義では省略したが,$x$や$y$が内包項である場合は$z = x \vee z = y$を
	$\lang{\varepsilon}$の式に書き換えてから$\{x,y\}$を定めるのである.つまり
	\begin{align}
		\varphi \defarrow x = z \vee y = z
	\end{align}
	とおけば,$\varphi$を$\lang{\varepsilon}$の式に書き換えた式$\hat{\varphi}$によって
	\begin{align}
		\{x,y\} \defeq \Set{z}{\hat{\varphi}(z)}
	\end{align}
	と定めるのである.たとえば$\varphi$に自由に現れる変項が$z$だけならば
	\begin{align}
		\COMAX \vdash \forall z\, (\, z \in \{x,y\} \lrarrow \hat{\varphi}(z)\, )
	\end{align}
	が成立するし,同時に定理\ref{thm:equivalent_formula_rewriting_1}と
	定理\ref{thm:equivalent_formula_rewriting_2}より
	\begin{align}
		\EXTAX,\EQAX,\COMAX \vdash 
		\forall z\, (\, \hat{\varphi}(z) \lrarrow \varphi(z)\, )
	\end{align}
	も成り立つので
	\begin{align}
		\EXTAX,\EQAX,\COMAX \vdash 
		\forall z\, (\, z \in \{x,y\} \lrarrow x = z \vee y = z\, )
	\end{align}
	が得られる.
	
	\begin{screen}
		\begin{thm}[対は表示されている要素しか持たない]
		\label{thm:pair_members_are_exactly_the_given_two}
			$a$と$b$を類とするとき次が成立する:
			\begin{align}
				\EXTAX,\EQAX,\COMAX \vdash 
				\forall x\, (\, x \in \{a,b\} \lrarrow a = x \vee b = x\, ).
			\end{align}
		\end{thm}
	\end{screen}
	
	\begin{sketch}
		大まかな流れは前述したので
	\end{sketch}
	
	\begin{screen}
		\begin{thm}[要素の表示の順番を入れ替えても対は等しい]\label{thm:commutative_law_of_pairs}
			$a$と$b$を類とするとき
			\begin{align}
				\EXTAX,\EQAX,\COMAX \vdash \{a,b\} = \{b,a\}.
			\end{align}
		\end{thm}
	\end{screen}
	
	\begin{sketch}
		いま
		\begin{align}
			\tau \defeq \varepsilon x \negation (\, x \in \{a,b\} \lrarrow x \in \{b,a\}\, )
		\end{align}
		とおく(必要に応じて$x \in \{a,b\} \lrarrow x \in \{b,a\}$は$\lang{\varepsilon}$の
		式に書き換える).
		定理\ref{thm:pair_members_are_exactly_the_given_two}より
		\begin{align}
			\EXTAX,\EQAX,\COMAX \vdash
			\tau \in \{a,b\} \rarrow a = \tau \vee b = \tau
		\end{align}
		が成り立つので,演繹法則の逆より
		\begin{align}
			\tau \in \{a,b\},\ \EXTAX,\EQAX,\COMAX \vdash a = \tau \vee b = \tau
		\end{align}
		となる.また論理和の可換律
		(推論法則\ref{logicalthm:commutative_law_of_disjunction})より
		\begin{align}
			\tau \in \{a,b\},\ \EXTAX,\EQAX,\COMAX \vdash b = \tau \vee a = \tau
		\end{align}
		が成り立ち,定理\ref{thm:pair_members_are_exactly_the_given_two}より
		\begin{align}
			\tau \in \{a,b\},\ \EXTAX,\EQAX,\COMAX \vdash \tau \in \{b,a\}
		\end{align}
		が従う.そして演繹規則より
		\begin{align}
			\EXTAX,\EQAX,\COMAX \vdash \tau \in \{a,b\} \rarrow \tau \in \{b,a\}
		\end{align}
		が得られる.$a$と$b$を入れ替えれば
		\begin{align}
			\EXTAX,\EQAX,\COMAX \vdash \tau \in \{b,a\} \rarrow \tau \in \{a,b\}
		\end{align}
		が得られるので,論理積の導入規則より
		\begin{align}
			\EXTAX,\EQAX,\COMAX \vdash \tau \in \{a,b\} \lrarrow \tau \in \{b,a\}
		\end{align}
		が成り立ち,全称記号の推論規則より
		\begin{align}
			\EXTAX,\EQAX,\COMAX \vdash \forall x\, (\, x \in \{a,b\} \lrarrow x \in \{b,a\}\, )
		\end{align}
		となり,外延性公理より
		\begin{align}
			\EXTAX,\EQAX,\COMAX \vdash \{a,b\} = \{b,a\}
		\end{align}
		が従う.
		\QED
	\end{sketch}
		
	\begin{screen}
		\begin{axm}[対の公理]
			\begin{align}
				\PAIAX \defarrow \forall x\, \forall y\, \exists p\, \forall z\, 
				(\, x = z \vee y = z \lrarrow z \in p\, ).
			\end{align}
		\end{axm}
	\end{screen}
	
	\begin{screen}
		\begin{thm}[集合の対は集合である]
		\label{thm:pair_of_sets_is_a_set}
			$a$と$b$を類とするとき
			\begin{align}
				\EXTAX,\EQAX,\COMAX,\PAIAX \vdash 
				\set{a} \wedge \set{b} \rarrow \set{\{a,b\}}.
			\end{align}
		\end{thm}
	\end{screen}
	
	\begin{sketch}\mbox{}
		\begin{description}
			\item[step1]
				論理積の除去規則より
				\begin{align}
					\set{a} \wedge \set{b} &\vdash \exists x\, (\, a = x\, ), \\
					\set{a} \wedge \set{b} &\vdash \exists x\, (\, b = x\, )
				\end{align}
				が成り立つので,
				\begin{align}
					\tau &\defeq \varepsilon x\, (\, a = x\, ), \\
					\sigma &\defeq \varepsilon x\, (\, b = x\, )
				\end{align}
				とおけば
				\begin{align}
					\set{a} \wedge \set{b} &\vdash a = \tau, 
					\label{fom:pair_of_sets_is_a_set_1} \\
					\set{a} \wedge \set{b} &\vdash b = \tau
				\end{align}
				が成り立つ.対の公理より$\tau$と$\sigma$に対しては
				\begin{align}
					\PAIAX \vdash \exists p\, \forall z\, 
						(\, \tau = z \vee \sigma = z \lrarrow z \in p\, )
				\end{align}
				が成り立つので,
				\begin{align}
					\rho \defeq \varepsilon p\, \forall z\, 
						(\, \tau = z \vee \sigma = z \lrarrow z \in p\, )
				\end{align}
				とおけば
				\begin{align}
					\PAIAX \vdash \forall z\, (\, \tau = z \vee \sigma = z \lrarrow z \in \rho\, )
					\label{fom:pair_of_sets_is_a_set_2}
				\end{align}
				となる.
				
			\item[step2]
				次に
				\begin{align}
					\forall z\, (\, z \in \{a,b\} \lrarrow z \in \rho\, )
				\end{align}
				を示すために
				\begin{align}
					\zeta \defeq \varepsilon z \negation (\, z \in \{a,b\} \lrarrow z \in \rho\, )
				\end{align}
				とおく(当然$\lang{\varepsilon}$の式に書き換える).
				等号の推移律(定理\ref{thm:transitive_law_of_equality})より
				\begin{align}
					\EXTAX,\EQAX \vdash a = \tau \rarrow (\, a = \zeta \rarrow \tau = \zeta\, )
				\end{align}
				が成り立つので,(\refeq{fom:pair_of_sets_is_a_set_1})との三段論法より
				\begin{align}
					\set{a} \wedge \set{b},\ \EXTAX,\EQAX \vdash 
					a = \zeta \rarrow \tau = \zeta
				\end{align}
				が成り立ち,論理和の導入規則より
				\begin{align}
					\set{a} \wedge \set{b},\ \EXTAX,\EQAX \vdash 
					a = \zeta \rarrow \tau = \zeta \vee \sigma = \zeta
				\end{align}
				が従う.同様にして
				\begin{align}
					\set{a} \wedge \set{b},\ \EXTAX,\EQAX \vdash 
					b = \zeta \rarrow \tau = \zeta \vee \sigma = \zeta
				\end{align}
				も成り立つので,論理和の除去規則より
				\begin{align}
					\set{a} \wedge \set{b},\ \EXTAX,\EQAX \vdash 
					a = \zeta \vee b = \zeta \rarrow \tau = \zeta \vee \sigma = \zeta
				\end{align}
				が得られる.同様に
				\begin{align}
					\set{a} \wedge \set{b},\ \EXTAX,\EQAX \vdash 
					\tau = \zeta \vee \sigma = \zeta \rarrow a = \zeta \vee b = \zeta
				\end{align}
				も得られ,論理積の導入規則より
				\begin{align}
					\set{a} \wedge \set{b},\ \EXTAX,\EQAX \vdash 
					a = \zeta \vee b = \zeta \lrarrow \tau = \zeta \vee \sigma = \zeta
				\end{align}
				が従う.他方で定理\ref{thm:pair_members_are_exactly_the_given_two}より
				\begin{align}
					\EXTAX,\EQAX,\COMAX \vdash 
					\zeta \in \{a,b\} \lrarrow a = \zeta \vee b = \zeta
				\end{align}
				が成り立ち,また(\refeq{fom:pair_of_sets_is_a_set_2})より
				\begin{align}
					\PAIAX \vdash \tau = \zeta \vee \sigma = \zeta \lrarrow \zeta \in \rho
				\end{align}
				も成り立つので,同値記号の推移律
				(推論法則\ref{logicalthm:transitive_law_of_equivalence_symbol})より
				\begin{align}
					\set{a} \wedge \set{b},\ \EXTAX,\EQAX,\COMAX,\PAIAX \vdash 
					\zeta \in \{a,b\} \lrarrow \zeta \in \rho
				\end{align}
				が従う.そして全称記号の推論規則より
				\begin{align}
					\set{a} \wedge \set{b},\ \EXTAX,\EQAX,\COMAX,\PAIAX \vdash 
					\forall z\, (\, z \in \{a,b\} \lrarrow z \in \rho\, )
				\end{align}
				が成り立ち,外延性公理より
				\begin{align}
					\set{a} \wedge \set{b},\ \EXTAX,\EQAX,\COMAX,\PAIAX \vdash 
					\{a,b\} = \rho
				\end{align}
				が従い,存在記号の推論規則より
				\begin{align}
					\set{a} \wedge \set{b},\ \EXTAX,\EQAX,\COMAX,\PAIAX \vdash 
					\exists p\, (\, \{a,b\} = p\, )
				\end{align}
				が成り立つ.
				\QED
		\end{description}
	\end{sketch}
	
	\begin{screen}
		\begin{logicalthm}[量化記号の性質(ロ)]\label{logicalthm:properties_of_quantifiers_2}
			$A,B$を$\mathcal{L}'$の式とし,$x$を$A,B$に現れる文字とするとき,$x$のみが$A,B$で量化されていないならば以下は定理である:
			\begin{description}
				\item[(a)] $\exists x ( A(x) \vee B(x) ) \lrarrow \exists x A(x) \vee \exists x B(x)$.
				
				\item[(b)] $\forall x ( A(x) \wedge B(x) ) \lrarrow \forall x A(x) \wedge \forall x B(x)$.
			\end{description}
		\end{logicalthm}
	\end{screen}
	
	\begin{prf}\mbox{}
		\begin{description}
			\item[(a)]
				いま$c(x) \overset{\mathrm{def}}{\lrarrow} A(x) \vee B(x)$とおけば,
				$\exists x ( A(x) \vee B(x) )$と$\exists x ( C(x) )$は同じ記号列であるから
				\begin{align}
					\exists x ( A(x) \vee B(x) ) \rarrow \exists x C(x)
					\label{eq:logicalthm_properties_of_quantifiers_1}
				\end{align}
				が成立する.また推論法則\ref{logicalthm:transitive_law_of_implication}より
				\begin{align}
					\exists x C(x) \rarrow C(\varepsilon x C(x))
					\label{eq:logicalthm_properties_of_quantifiers_2}
				\end{align}
				が成立する.$C(\varepsilon x C(x))$と$A(\varepsilon x C(x)) \vee B(\varepsilon x C(x))$
				は同じ記号列であるから
				\begin{align}
					C(\varepsilon x C(x)) \rarrow A(\varepsilon x C(x)) \vee B(\varepsilon x C(x))
					\label{eq:logicalthm_properties_of_quantifiers_3}
				\end{align}
				が成立する.ここで推論法則\ref{logicalthm:transitive_law_of_implication}と
				推論規則\ref{logicalaxm:fundamental_rules_of_inference}より
				\begin{align}
					A(\varepsilon x C(x)) &\rarrow \exists x A(x) \\
						&\rarrow \exists x A(x) \vee \exists x B(x), \\
					B(\varepsilon x C(x)) &\rarrow \exists x B(x) \\
						&\rarrow \exists x A(x) \vee \exists x B(x)
				\end{align}
				が成立するので,場合分け法則より
				\begin{align}
					A(\varepsilon x C(x)) \vee B(\varepsilon x C(x))
					\rarrow \exists x A(x) \vee \exists x B(x)
					\label{eq:logicalthm_properties_of_quantifiers_4}
				\end{align}
				が成り立つ.(\refeq{eq:logicalthm_properties_of_quantifiers_1})
				(\refeq{eq:logicalthm_properties_of_quantifiers_2})
				(\refeq{eq:logicalthm_properties_of_quantifiers_3})
				(\refeq{eq:logicalthm_properties_of_quantifiers_4})
				に推論法則\ref{logicalthm:transitive_law_of_implication}を順次適用すれば
				\begin{align}
					\exists x ( A(x) \vee B(x) ) \rarrow \exists x A(x) \vee \exists x B(x)
				\end{align}
				が得られる.他方,推論規則\ref{logicalaxm:rules_of_quantifiers}より
				\begin{align}
					\exists x A(x) &\rarrow A(\varepsilon x A(x)) \\
						&\rarrow A(\varepsilon x A(x)) \vee B(\varepsilon x A(x)) \\
						&\rarrow C(\varepsilon x A(x)) \\
						&\rarrow C(\varepsilon x C(x)) \\
						&\rarrow \exists x C(x) \\
						&\rarrow \exists x (A(x) \vee B(x))
				\end{align}
				が成立し,$A$を$B$に置き換えれば
				$\exists x B(x) \rarrow \exists x (A(x) \vee B(x))$も成り立つので,
				場合分け法則より
				\begin{align}
					\exists x A(x) \vee \exists x B(x) \rarrow \exists x (A(x) \vee B(x))
				\end{align}
				も得られる.
			
			\item[(b)]
				簡略して説明すれば
				\begin{align}
					\forall x \left( A(x) \wedge B(x) \right)
					&\lrarrow\ \negation \exists x \negation \left( A(x) \wedge B(x) \right) & (\mbox{推論法則\ref{logicalthm:De_Morgan_law_for_quantifiers}}) \\
					&\lrarrow\ \negation \exists x \left( \negation A(x) \vee \negation B(x) \right) & (\mbox{De Morganの法則}) \\
					&\lrarrow\ \negation \left( \exists x \negation A(x) \vee \exists x \negation B(x) \right) & (\mbox{前段の対偶}) \\
					&\lrarrow\ \negation \left( \negation \forall x A(x) \vee \negation \forall x B(x) \right) & (\mbox{推論法則\ref{logicalthm:De_Morgan_law_for_quantifiers}}) \\
					&\lrarrow\ \negation \negation \forall x A(x) \wedge \negation \negation \forall x B(x) & (\mbox{De Morganの法則}) \\
					&\lrarrow \forall x A(x) \wedge \forall x B(x) &(\mbox{二重否定の法則})
				\end{align}
				となる.
				\QED
		\end{description}
	\end{prf}
	
	\begin{screen}
		\begin{thm}[集合は対の要素たりうる]\label{thm:set_is_an_element_of_its_pair}
			$a,b$を類とするとき,
			\begin{align}
				\set{a} \rarrow a \in \{a,b\}.
			\end{align}
		\end{thm}
	\end{screen}
	
	\begin{sketch}
		いま
		\begin{align}
			\set{a}
		\end{align}
		が成り立っているとする.ここで
		\begin{align}
			\tau \defeq \varepsilon x\, (\, a = x\, )
		\end{align}
		とおけば,存在記号に関する規則より
		\begin{align}
			a = \tau
		\end{align}
		が成り立つ.ゆえに
		\begin{align}
			\tau = a \vee \tau = b
		\end{align}
		も成り立つ.ゆえに
		\begin{align}
			\tau \in \{a,b\}
		\end{align}
		が成り立ち,相当性の公理より
		\begin{align}
			a \in \{a,b\}
		\end{align}
		が従う.そして演繹法則より\
		\begin{align}
			\set{a} \rarrow a \in \{a,b\}
		\end{align}
		が得られる.
		\QED
	\end{sketch}
	
	$a$を集合とすれば対の公理より$\{a\}$も集合となり,定理\ref{thm:set_is_an_element_of_its_pair}より
	\begin{align}
		a \in \{a\}
	\end{align}
	が成立する.
	
	\begin{screen}
		\begin{thm}[真類同士の対は空]\label{thm:pair_of_proper_classes_is_emptyset}
			$a,b$を類とするとき,
			\begin{align}
				\negation \set{a} \wedge \negation \set{b} \lrarrow \{a,b\} = \emptyset.
			\end{align}
		\end{thm}
	\end{screen}
	
	\begin{sketch}
		いま
		\begin{align}
			\negation \set{a} \wedge \negation \set{b}
		\end{align}
		が成り立っているとする.このとき
		\begin{align}
			\forall x\, (\, a \neq x\, ) \wedge \forall x\, (\, b \neq x\, )
		\end{align}
		が成立し,推論法則\ref{logicalthm:properties_of_quantifiers_2}より
		\begin{align}
			\forall x\, (\, a \neq x \wedge b \neq x\, )
		\end{align}
		が成立する.すなわち$\chi$を$\mathcal{L}$の任意の対象とすれば
		\begin{align}
			a \neq \chi \wedge b \neq \chi
		\end{align}
		が成立する.他方で
		\begin{align}
			a \neq \chi \wedge b \neq \chi \lrarrow \chi \notin \{a,b\}
		\end{align}
		も満たされているので,三段論法より
		\begin{align}
			\chi \notin \{a,b\}
		\end{align}
		が成立する.$\chi$の任意性より
		\begin{align}
			\forall x\, (\, x \notin \{a,b\}\, )
		\end{align}
		が成立し,定理\ref{thm:uniqueness_of_emptyset}
		\begin{align}
			\{a,b\} = \emptyset
		\end{align}
		が従う.そして演繹法則を適用して
		\begin{align}
			\negation \set{a} \wedge \negation \set{b} \rarrow \{a,b\} = \emptyset
		\end{align}
		が得られる.逆に,定理\ref{thm:set_is_an_element_of_its_pair}から
		\begin{align}
			\set{a} \rarrow a \in \{a,b\}
		\end{align}
		が成り立ち,定理\ref{thm:emptyset_does_not_contain_any_class}から
		\begin{align}
			a \in \{a,b\} \rarrow \{a,b\} \neq \emptyset
		\end{align}
		が成り立つので,含意の推移律より
		\begin{align}
			\set{a} \rarrow \{a,b\} \neq \emptyset
		\end{align}
		が成立する.同様に
		\begin{align}
			\set{b} \rarrow \{a,b\} \neq \emptyset
		\end{align}
		も成り立つから,場合分け法則より
		\begin{align}
			\set{a} \vee \set{b} \rarrow \{a,b\} \neq \emptyset
		\end{align}
		が成立し,この対偶を取って
		\begin{align}
			\{a,b\} = \emptyset \rarrow\ \negation \set{a} \wedge \negation \set{b}
		\end{align}
		が得られる.
		\QED
	\end{sketch}
	
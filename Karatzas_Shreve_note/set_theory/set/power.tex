\section{冪}
	\begin{screen}
		\begin{dfn}[冪]
			$x$を$\mathcal{L}$の項とするとき,
			\begin{align}
				\power{x} \defeq \Set{y}{\forall z\, (\, z \in y \rarrow z \in x\, )}
			\end{align}
			で定める項(必要に応じて$z \in x$は$\lang{\varepsilon}$の式に書き換える)を
			$x$の{\bf 冪}\index{べき@冪}{\bf (power)}と呼ぶ.
		\end{dfn}
	\end{screen}
	
	$x$の冪とはすなわち「$x$の部分集合の全体」である:
	\begin{align}
		\power{x} = \Set{y}{y \subset x}.
	\end{align}
	
	\begin{screen}
		\begin{axm}[冪の公理]
			次の公理を$\POWAX$によって参照する:
			\begin{align}
				\forall x\, \exists p\, \forall y\, 
				(\, \forall z\, (\, z \in y \rarrow z \in x\, ) \lrarrow y \in p\, ).
			\end{align}
		\end{axm}
	\end{screen}
	
	\begin{screen}
		\begin{thm}[集合の冪は集合]
			$a$を類とするとき
			\begin{align}
				\set{a} \rarrow \set{\power{a}}.
			\end{align}
		\end{thm}
	\end{screen}
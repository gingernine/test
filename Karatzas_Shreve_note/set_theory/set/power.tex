\section{冪}
	\begin{screen}
		\begin{dfn}[冪]
			$x$を$\mathcal{L}$の項とするとき,
			\begin{align}
				\power{x} \defeq \Set{y}{\forall z\, (\, z \in y \rarrow z \in x\, )}
			\end{align}
			で定める項(必要に応じて$z \in x$は$\lang{\varepsilon}$の式に書き換える)を
			$x$の{\bf 冪}\index{べき@冪}{\bf (power)}と呼ぶ.
		\end{dfn}
	\end{screen}
	
	$x$の冪とはすなわち「$x$の部分集合の全体」である:
	\begin{align}
		\power{x} = \Set{y}{y \subset x}.
	\end{align}
	
	\begin{screen}
		\begin{axm}[冪の公理]
			次の公理を$\POWAX$によって参照する:
			\begin{align}
				\forall x\, \exists p\, \forall y\, 
				(\, \forall z\, (\, z \in y \rarrow z \in x\, ) \lrarrow y \in p\, ).
			\end{align}
		\end{axm}
	\end{screen}
	
	\begin{screen}
		\begin{thm}[集合の冪は集合]\label{thm:power_of_a_set_is_a_set}
			$a$をクラスとするとき
			\begin{align}
				\EXTAX,\EQAX,\COMAX,\POWAX \vdash \set{a} \rarrow \set{\power{a}}.
			\end{align}
		\end{thm}
	\end{screen}
	
	\begin{sketch}\mbox{}
		\begin{description}
			\item[step1]
				$a$が主要$\varepsilon$項であるとき,
				\begin{align}
					\POWAX \vdash \exists p\, \forall y\, (\, \forall z\, (\, z \in y \rarrow z \in a\, ) \lrarrow y \in p\, )
				\end{align}
				が成り立つので
				\begin{align}
					\rho \defeq \varepsilon p\, \forall y\, (\, \forall z\, (\, z \in y \rarrow z \in a\, ) \lrarrow y \in p\, )
				\end{align}
				とおけば
				\begin{align}
					\POWAX \vdash \forall y\, (\, \forall z\, (\, z \in y \rarrow z \in a\, ) \lrarrow y \in \rho\, )
					\label{fom:power_of_a_set_is_a_set_1}
				\end{align}
				となる.よって定理\ref{thm:equivalent_formula_rewriting_4}より
				\begin{align}
					\EXTAX,\COMAX,\POWAX \vdash 
					\Set{y}{\forall z\, (\, z \in y \rarrow z \in a\, )} = \rho
				\end{align}
				が従い,存在記号の論理的公理より
				\begin{align}
					\EXTAX,\COMAX,\POWAX \vdash \exists p\, 
					(\, \Set{y}{\forall z\, (\, z \in y \rarrow z \in a\, )} = p\, )
				\end{align}
				が得られる.
				
			\item[step2]
				$a$が$\Set{x}{\varphi(x)}$なる形の項であるとき
				($\varphi$は$\lang{\varepsilon}$の式),
				\begin{align}
					\tau &\defeq \varepsilon x\, (\, a = x\, ), \\
					\rho &\defeq \varepsilon p\, \forall y\, (\, \forall z\, (\, z \in y \rarrow z \in \tau\, ) \lrarrow y \in p\, )
				\end{align}
				とおけば,前段の(\refeq{fom:power_of_a_set_is_a_set_1})より
				\begin{align}
					\POWAX \vdash \forall y\, (\, \forall z\, (\, z \in y \rarrow z \in \tau\, ) \lrarrow y \in \rho\, )
					\label{fom:power_of_a_set_is_a_set_4}
				\end{align}
				が成立する.今の場合の$\power{a}$は
				\begin{align}
					\power{a} \defeq \Set{y}{\forall z\, (\, z \in y \rarrow \varphi(z)\, )}
				\end{align}
				によって定められているので,
				\begin{align}
					\forall y\, (\, \forall z\, (\, z \in y \rarrow \varphi(z)\, ) \lrarrow y \in \rho\, )
					\label{fom:power_of_a_set_is_a_set_2}
				\end{align}
				を導くために
				\begin{align}
					\eta \defeq \varepsilon y \negation (\, \forall z\, (\, z \in y \rarrow \varphi(z)\, ) \lrarrow y \in \rho\, )
				\end{align}
				とおき,まずは
				\begin{align}
					\set{a},\ \EQAX,\COMAX \vdash 
					\forall z\, (\, z \in \eta \rarrow \varphi(z)\, )
					\lrarrow \forall z\, (\, z \in \eta \rarrow z \in \tau\, )
					\label{fom:power_of_a_set_is_a_set_3}
				\end{align}
				を示す.
				\begin{itemize}
					\item (\refeq{fom:power_of_a_set_is_a_set_3})の$\rarrow$をしめす.
						\begin{align}
							\zeta \defeq \varepsilon z \negation
							(\, z \in \eta \rarrow z \in \tau\, )
						\end{align}
						とおけば
						\begin{align}
							\forall z\, (\, z \in \eta \rarrow \varphi(z)\, )
							\vdash \zeta \in \eta \rarrow \varphi(\zeta)
						\end{align}
						が成り立つが,
						\begin{align}
							\COMAX \vdash \varphi(\zeta) \rarrow \zeta \in a
						\end{align}
						と
						\begin{align}
							\set{a},\ \EQAX \vdash \zeta \in a \rarrow \zeta \in \tau
						\end{align}
						より
						\begin{align}
							\forall z\, (\, z \in \eta \rarrow \varphi(z)\, ),\ 
							\set{a},\ \EQAX,\COMAX
							\vdash \zeta \in \eta \rarrow \zeta \in \tau
						\end{align}
						が従い,全称の導出(論理的定理\ref{logicalthm:derivation_of_universal_by_epsilon})より
						\begin{align}
							\forall z\, (\, z \in \eta \rarrow \varphi(z)\, ),\ 
							\set{a},\ \EQAX,\COMAX \vdash 
							\forall z\, (\, z \in \eta \rarrow z \in \tau\, )
						\end{align}
						が得られる.
					
					\item (\refeq{fom:power_of_a_set_is_a_set_3})の逆を示す.
						\begin{align}
							\zeta \defeq \varepsilon z \negation
							(\, z \in \eta \rarrow \varphi(z)\, )
						\end{align}
						とおけば
						\begin{align}
							\forall z\, (\, z \in \eta \rarrow z \in \tau\, )
							\vdash \zeta \in \eta \rarrow \zeta \in \tau
						\end{align}
						が成り立つが,
						\begin{align}
							\COMAX \vdash \zeta \in a \rarrow \varphi(\zeta)
						\end{align}
						と
						\begin{align}
							\set{a},\ \EQAX \vdash \zeta \in \tau \rarrow \zeta \in a
						\end{align}
						より
						\begin{align}
							\forall z\, (\, z \in \eta \rarrow z \in \tau\, ),\ 
							\set{a},\ \EQAX,\COMAX
							\vdash \zeta \in \eta \rarrow \varphi(\zeta)
						\end{align}
						が従い,全称の導出(論理的定理\ref{logicalthm:derivation_of_universal_by_epsilon})より
						\begin{align}
							\forall z\, (\, z \in \eta \rarrow z \in \tau\, ),\ 
							\set{a},\ \EQAX,\COMAX \vdash 
							\forall z\, (\, z \in \eta \rarrow \varphi(z)\, )
						\end{align}
						が得られる.
				\end{itemize}
				(\refeq{fom:power_of_a_set_is_a_set_4})より
				\begin{align}
					\POWAX \vdash \forall z\, (\, z \in \eta \rarrow z \in \tau\, )
					\lrarrow \eta \in \rho
					\label{fom:power_of_a_set_is_a_set_5}
				\end{align}
				が成り立つので,(\refeq{fom:power_of_a_set_is_a_set_3})と
				(\refeq{fom:power_of_a_set_is_a_set_5})と同値記号の推移律
				(論理的定理\ref{logicalthm:transitive_law_of_equivalence_symbol})より
				\begin{align}
					\set{a},\ \EQAX,\COMAX,\POWAX \vdash 
					\forall z\, (\, z \in \eta \rarrow \varphi(z)\, ) \lrarrow \eta \in \rho
				\end{align}
				が得られ,全称の導出(論理的定理\ref{logicalthm:derivation_of_universal_by_epsilon})より
				\begin{align}
					\set{a},\ \EQAX,\COMAX,\POWAX \vdash 
					\forall y\, (\, \forall z\, (\, z \in y \rarrow \varphi(z)\, ) \lrarrow y \in \rho\, )
				\end{align}
				が従う.そして定理\ref{thm:equivalent_formula_rewriting_4}と併せて
				\begin{align}
					\set{a},\ \EXTAX,\EQAX,\COMAX,\POWAX \vdash 
					\Set{y}{\forall z\, (\, z \in y \rarrow \varphi(z)\, )} = \rho
				\end{align}
				が従い,存在記号の論理的公理より
				\begin{align}
					\set{a},\ \EXTAX,\EQAX,\COMAX,\POWAX \vdash \exists p\, 
					(\, \Set{y}{\forall z\, (\, z \in y \rarrow \varphi(z)\, )} = p\, )
				\end{align}
				が得られる.
				\QED
		\end{description}
	\end{sketch}
\section{アレフ数}
	有限基数を抜いた基数の全体を
	\begin{align}
		\InfCN \defeq \CN \backslash \Natural
	\end{align}
	とおく.`I'を付けたのは,これが無限濃度の全体を表しているからである.
	$\InfCN$は$\ON$の部分集合であるが,$\ON$から$\InfCN$への順序同型写像を取ることが出来る.
	それは
	\begin{align}
		\aleph
	\end{align}
	と書かれ,アレフ数と呼ばれる. 
	
	\begin{screen}
		\begin{dfn}[無限基数割り当て写像$\aleph$]
			$\Univ$上の写像$G$を
			\begin{align}
				G \defeq \Set{x}{\exists s\, \exists \alpha \in \ON\, 
				\left[\, x=(s,\alpha) \wedge \alpha \in \InfCN \backslash \ran{s} \wedge
				\forall \beta \in \ON\, \left(\, \beta \in \InfCN \backslash \ran{s}
				\Longrightarrow \beta \leq \alpha\, \right)\, \right]}
			\end{align}
			で定めるとき,超限帰納法による写像の構成から
			\begin{align}
				\forall \beta \in \ON\, 
				\left(\, \aleph(\beta) = \mu \alpha\, (\, \alpha \in \InfCN \backslash \aleph \ast \beta\, )\, \right)
			\end{align}
			を満たす$\ON$上の写像$\aleph$が取れる.$\alpha$を順序数とするとき,
			\begin{align}
				\aleph(\alpha)
			\end{align}
			のことを
			\begin{align}
				\aleph_\alpha
			\end{align}
			と書く.
		\end{dfn}
	\end{screen}
	
	次の定理は{\bf 異なる無限がいくらでも存在する}ことを主張している.
	
	\begin{screen}
		\begin{thm}[$\aleph$は$\ON$と順序同型]
			$\aleph$は$\ON$から$\InfCN$への順序同型となる.つまり
			\begin{align}
				\aleph:\ON \bij \InfCN \wedge \forall \gamma, \delta \in \ON\, \left(\, \gamma < \delta
				\Longrightarrow \aleph_\gamma < \aleph_\delta\, \right).
			\end{align}
		\end{thm}
	\end{screen}
	
	\begin{sketch}
		いま$\gamma,\delta$を$\ON$の要素として
		\begin{align}
			\gamma < \delta
		\end{align}
		であると仮定する.このとき
		\begin{align}
			F \ast \gamma \subset F \ast \delta
		\end{align}
		かつ
		\begin{align}
			F(\delta) \in \InfCN \backslash F \ast \delta
		\end{align}
		が満たされるので
		\begin{align}
			F(\delta) \in \InfCN \backslash F \ast \gamma
		\end{align}
		が成立する.従って
		\begin{align}
			F(\gamma) \leq F(\delta)
		\end{align}
		が成立する.一方で
		\begin{align}
			F(\gamma) \in F \ast \delta \wedge
			F(\delta) \in \InfCN \backslash F \ast \delta
		\end{align}
		から
		\begin{align}
			F(\gamma) \neq F(\delta)
		\end{align}
		も満たされるので
		\begin{align}
			F(\gamma) < F(\delta)
		\end{align}
		が従う.以上より
		\begin{align}
			\forall \gamma, \delta \in \ON\, (\, \gamma < \delta \Longrightarrow F(\gamma) < F(\delta)\, )
		\end{align}
		が得られる.またこの結果より$F$が単射であることも従う.
	\end{sketch}
	
	\begin{screen}
		\begin{thm}
			$\alpha$が極限数であるならば
			\begin{align}
				\bigcup_{\beta \in \alpha} \aleph_{\beta} = \aleph_{\alpha}.
			\end{align}
		\end{thm}
	\end{screen}
	
	\begin{sketch}
		$\gamma$を
		\begin{align}
			\Natural \leq \gamma < \aleph_{\alpha}
		\end{align}
		を満たす順序数とすると
		\begin{align}
			\card{\gamma} = \aleph_{\beta}
		\end{align}
		を満たす$\alpha$の要素$\beta$が取れる.このとき
		\begin{align}
			\card{\gamma} < \aleph_{\beta+1}
		\end{align}
		であるから
		\begin{align}
			\gamma < \aleph_{\beta+1}
		\end{align}
		が成り立つ.なぜならば,任意の基数$\delta$に対して
		\begin{align}
			\delta \leq \gamma \Longrightarrow 
			\delta = \card{\delta} \leq \card{\gamma}
		\end{align}
		が成り立つからである.以上より
		\begin{align}
			\gamma < \aleph_{\beta+1} < \aleph_{\alpha}
		\end{align}
		が従う.ゆえに
		\begin{align}
			\bigcup_{\beta \in \alpha} \aleph_{\beta} = \aleph_{\alpha}
		\end{align}
		を得た.
		\QED
	\end{sketch}
	
	\begin{screen}
		\begin{thm}[順序数がそのアレフ数を超えることはない]
		\label{thm:no_ordinal_number_is_bigger_than_its_aleph_number}
			任意の順序数$\alpha$に対して
			\begin{align}
				\alpha \leq \aleph_{\alpha}.
			\end{align}
		\end{thm}
	\end{screen}
	
	\begin{sketch}
		超限帰納法により示す.まず
		\begin{align}
			0 \leq \aleph_{0}.
		\end{align}
		次に$\alpha$を$0$でない順序数とし,$\alpha$の任意の要素$\beta$に対して
		\begin{align}
			\beta \leq \aleph_{\beta}
		\end{align}
		が成り立っているとする.
		\begin{align}
			\alpha = \beta + 1
		\end{align}
		を満たす順序数$\beta$が取れるとき,
		\begin{align}
			\beta < \aleph_{\alpha}
		\end{align}
		より
		\begin{align}
			\alpha = \beta + 1 \leq \aleph_{\alpha}
		\end{align}
		が従う.$\alpha$が極限数であるとき,
		\begin{align}
			\alpha = \bigcup_{\beta \in \alpha} \beta
			\leq \bigcup_{\beta \in \alpha} \aleph_{\beta}
			= \aleph_{\alpha}
		\end{align}
		が成立する.ゆえに超限帰納法より,任意の順序数$\alpha$に対して
		\begin{align}
			\alpha \leq \aleph_{\alpha}
		\end{align}
		が成立する.
		\QED
	\end{sketch}
	
	\begin{screen}
		\begin{thm}[無限順序数は後続数と濃度が等しい]
			\begin{align}
				\forall \alpha \in \ON\, (\,
				\Natural \leq \alpha \rarrow \card{\alpha} = \card{(\alpha + 1)}\, ).
			\end{align}
		\end{thm}
	\end{screen}
	
	\begin{sketch}
		$\alpha + 1$から$\alpha$への全単射を
		\begin{align}
			\alpha + 1 \ni x \longmapsto
			\begin{cases}
				0 & \mbox{if } x = \alpha, \\
				1 + x & \mbox{if } x < \Natural, \\
				x & \mbox{if } x \neq \alpha \wedge \Natural \leq x
			\end{cases}
		\end{align}
		によって定めることが出来るので
		\begin{align}
			\card{\alpha} = \card{(\alpha + 1)}
		\end{align}
		となる.
		\QED
	\end{sketch}
	
	\begin{screen}
		\begin{thm}[無限基数は極限数]
			\begin{align}
				\forall \alpha \in \ON\, \limo{\aleph_{\alpha}}.
			\end{align}
		\end{thm}
	\end{screen}
	
	\begin{sketch}
		$\beta$を$\aleph_{\alpha}$の要素とすれば,
		\begin{align}
			\beta < \Natural
		\end{align}
		ならば
		\begin{align}
			\beta < \beta + 1 < \Natural \leq \aleph_{\alpha}
		\end{align}
		となる.他方で
		\begin{align}
			\Natural \leq \beta
		\end{align}
		ならば
		\begin{align}
			\beta + 1 \leq \aleph_{\alpha}
		\end{align}
		かつ
		\begin{align}
			\card{\beta} = \card{(\beta + 1)} \leq \beta < \aleph_{\alpha}
		\end{align}
		となるので
		\begin{align}
			\beta + 1 < \aleph_{\alpha}
		\end{align}
		となる.以上で
		\begin{align}
			\forall \beta \in \aleph_{\alpha}\, (\, \beta + 1 < \aleph_{\alpha}\, )
		\end{align}
		が得られ
		\begin{align}
			\limo{\aleph_{\alpha}}
		\end{align}
		が出る.
		\QED
	\end{sketch}
\section{順序数}
	$0,1,2,\cdots$で表される数字は,集合論において
	\begin{align}
		0 &\defeq \emptyset, \\
		1 &\defeq \{0\} = \{\emptyset\}, \\
		2 &\defeq \{0,1\} = \{\emptyset,\{\emptyset\}\}, \\
		3 &\defeq \{0,1,2\} = \{\emptyset,\{\emptyset\},\{\emptyset,\{\emptyset\}\}\}
	\end{align}
	といった反復操作で定められる.上の操作を受け継いで``頑張れば手で書き出せる''類を自然数と呼ぶ.
	$0$は集合であり,集合の対は集合であるから$1$もまた集合である.
	更には集合の合併も集合であるから$2,3,4,\cdots$と続く自然数が全て集合であることがわかる.
	自然数の冪も自然数同士の集合演算もその結果は全て集合になるが,
	ここで「集合は$0$に集合演算を施しただけの素姓が明らかなものに限られるか」
	という疑問というか期待が自然に生まれてくる.実際それは正則性公理によって肯定されるわけだが,
	そこでキーになるのは順序数と呼ばれる概念である.
	
	\begin{screen}
		\begin{axm}[正則性公理]
			次の式を$\REGAX$で参照する:
			\begin{align}
				\forall r\, (\, \exists x\, (\, x \in r\, )
				\rarrow \exists y\, (\, y \in r \wedge \forall z\, 
				(\, z \in r \rarrow z \notin y\, )\, )\, ).
			\end{align}
		\end{axm}
	\end{screen}
	
	正則性公理の主張は「空でない集合は自分自身と交わらない要素を持つ」ということである.
	
	\begin{screen}
		\begin{thm}[類は自分自身を要素に持たない]
		\label{thm:no_class_contains_itself}
			$a$を類とするとき
			\begin{align}
				\EXTAX,\EQAX,\COMAX,\ELEAX,\PAIAX,\REGAX \vdash a \notin a.
			\end{align}
			ただし$a$が主要$\varepsilon$項であれば
			\begin{align}
				\EXTAX,\EQAX,\COMAX,\PAIAX,\REGAX \vdash a \notin a.
			\end{align}
		\end{thm}
	\end{screen}
	
	\begin{sketch}
		要素の公理の対偶より
		\begin{align}
			\ELEAX \vdash\ \negation \set{a} \rarrow a \notin a
		\end{align}
		が成り立つので,後は
		\begin{align}
			\EXTAX,\EQAX,\COMAX,\ELEAX,\PAIAX,\REGAX 
			\vdash \set{a} \rarrow a \notin a
		\end{align}
		を示せばよい.集合の対は集合である(定理\ref{thm:pair_of_sets_is_a_set})から
		\begin{align}
			\set{a},\ \EXTAX,\EQAX,\COMAX,\PAIAX \vdash \set{\{a\}}
		\end{align}
		が成り立つ.ここで
		\begin{align}
			\tau \defeq \varepsilon x\, (\, \{a\} = x\, )
		\end{align}
		とおけば,量化記号の論理的公理より
		\begin{align}
			\set{a},\ \EXTAX,\EQAX,\COMAX,\PAIAX \vdash \{a\} = \tau
			\label{fom:no_class_contains_itself_1}
		\end{align}
		と
		\begin{align}
			\REGAX \vdash \exists x\, (\, x \in \tau\, )
			\rarrow \exists y\, (\, y \in \tau \wedge \forall z\, (\, 
			z \in \tau \rarrow z \notin y\, )\, )
			\label{fom:no_class_contains_itself_2}
		\end{align}
		が成り立つ.集合は自分自身の対の要素である
		(定理\ref{thm:set_is_an_element_of_its_pair})から
		\begin{align}
			\set{a},\ \EXTAX,\EQAX,\COMAX \vdash a \in \{a\}
			\label{fom:no_class_contains_itself_3}
		\end{align}
		が成り立ち,(\refeq{fom:no_class_contains_itself_1})と併せて
		\begin{align}
			\set{a},\ \EXTAX,\EQAX,\COMAX,\PAIAX \vdash a \in \tau
		\end{align}
		が成り立つ.これによって
		\begin{align}
			\set{a},\ \EXTAX,\EQAX,\COMAX,\ELEAX,\PAIAX \vdash 
			\exists x\, (\, x \in \tau\, )
		\end{align}
		が従い(定理\ref{thm:emptyset_does_not_contain_any_class}によるが,
		$a$が主要$\varepsilon$項であれば$\ELEAX$は不要),
		(\refeq{fom:no_class_contains_itself_2})との三段論法より
		\begin{align}
			\set{a},\ \EXTAX,\EQAX,\COMAX,\ELEAX,\PAIAX,\REGAX \vdash
			\exists y\, (\, y \in \tau \wedge \forall z\, (\, 
			z \in \tau \rarrow z \notin y\, )\, )
		\end{align}
		となる.
		\begin{align}
			\zeta &\defeq \varepsilon x\, (\, a = x\, ), \\
			\eta &\defeq \varepsilon y\, (\, y \in \tau \wedge \forall z\, (\, 
			z \in \tau \rarrow z \notin y\, )\, )
		\end{align}
		とおけば存在記号の論理的公理より
		\begin{align}
			\set{a} &\vdash a = \zeta, \label{fom:no_class_contains_itself_4} \\
			\set{a},\ \EXTAX,\EQAX,\COMAX,\ELEAX,\PAIAX,\REGAX &\vdash 
			\eta \in \tau, \label{fom:no_class_contains_itself_5} \\
			\set{a},\ \EXTAX,\EQAX,\COMAX,\ELEAX,\PAIAX,\REGAX &\vdash 
			\forall z\, (\, z \in \tau \rarrow z \notin \eta\, )
			\label{fom:no_class_contains_itself_6}
		\end{align}
		が成り立つが,まず(\refeq{fom:no_class_contains_itself_3})と
		(\refeq{fom:no_class_contains_itself_1})と
		(\refeq{fom:no_class_contains_itself_4})より
		\begin{align}
			\set{a},\ \EXTAX,\EQAX,\COMAX \vdash \zeta \in \tau
		\end{align}
		が成り立つので,(\refeq{fom:no_class_contains_itself_6})より
		\begin{align}
			\set{a},\ \EXTAX,\EQAX,\COMAX,\ELEAX,\PAIAX,\REGAX \vdash 
			\zeta \notin \eta
			\label{fom:no_class_contains_itself_7}
		\end{align}
		が従う.また(\refeq{fom:no_class_contains_itself_5})と
		(\refeq{fom:no_class_contains_itself_1})より
		\begin{align}
			\set{a},\ \EXTAX,\EQAX,\COMAX,\ELEAX,\PAIAX,\REGAX \vdash \eta \in \{a\}
		\end{align}
		が成り立つので,定理\ref{thm:pair_members_are_exactly_the_given_two}より
		\begin{align}
			\set{a},\ \EXTAX,\EQAX,\COMAX,\ELEAX,\PAIAX,\REGAX \vdash \eta = a
		\end{align}
		が従い,(\refeq{fom:no_class_contains_itself_7})と
		(\refeq{fom:no_class_contains_itself_4})と併せて
		\begin{align}
			\set{a},\ \EXTAX,\EQAX,\COMAX,\ELEAX,\PAIAX,\REGAX \vdash a \notin a
		\end{align}
		が出る.
		\QED
	\end{sketch}
	
	\begin{screen}
		\begin{logicalthm}[選言三段論法]
		\label{logicalthm:disjunctive_syllogism}
			$A$と$B$を文とするとき
			\begin{align}
				\vdash A \vee B \rarrow (\, \negation A \rarrow B\, ).
			\end{align}
		\end{logicalthm}
	\end{screen}
	
	\begin{sketch}
		まず含意の導入より
		\begin{align}
			\vdash B \rarrow (\, \negation A \rarrow B\, )
		\end{align}
		が成り立つ.また矛盾の導入より
		\begin{align}
			A,\ \negation A \vdash \bot
		\end{align}
		が成り立つが,爆発律(論理的定理\ref{logicalthm:principle_of_explosion})より
		\begin{align}
			A,\ \negation A \vdash B
		\end{align}
		が従い,演繹定理より
		\begin{align}
			\vdash A \rarrow (\, \negation A \rarrow B\, )
		\end{align}
		も得られる.そして論理和の除去より
		\begin{align}
			\vdash A \vee B \rarrow (\, \negation A \rarrow B\, )
		\end{align}
		が出る.
		\QED
	\end{sketch}
	
	\begin{screen}
		\begin{thm}[集合のどの二組も所属関係で堂々巡りしない]
		\label{thm:no_pair_of_sets_go_round}
			\begin{align}
				\EXTAX,\EQAX,\COMAX,\PAIAX,\REGAX \vdash 
				\forall x\, \forall y\, (\, x \in y \rarrow y \notin x\, ).
			\end{align}
		\end{thm}
	\end{screen}
	
	\begin{sketch}
		いま
		\begin{align}
			\chi &\defeq \varepsilon x \negation \forall y\, (\, x \in y \rarrow y \notin x\, ), \\
			\eta &\defeq \varepsilon y \negation (\, \tau \in y \rarrow y \notin \tau\, )
		\end{align}
		とおく.$\chi$と$\eta$は主要$\varepsilon$項であるから
		定理\ref{thm:critical_epsilon_term_is_set}より
		\begin{align}
			\EXTAX &\vdash \set{\chi}, \\
			\EXTAX &\vdash \set{\eta}
		\end{align}
		となり,従って定理\ref{thm:pair_of_sets_is_a_set}より
		\begin{align}
			\EXTAX,\EQAX,\COMAX,\PAIAX \vdash \set{\{\chi,\eta\}}
		\end{align}
		となり,また定理\ref{thm:set_is_an_element_of_its_pair}より
		\begin{align}
			\EXTAX,\EQAX,\COMAX \vdash \chi \in \{\chi,\eta\}
			\label{fom:no_pair_of_sets_go_round_1}
		\end{align}
		となる.ここで
		\begin{align}
			\tau \defeq \varepsilon x\, (\, \{\chi,\eta\} = x\, )
		\end{align}
		とおけば
		\begin{align}
			\EXTAX,\EQAX,\COMAX,\PAIAX \vdash \{\chi,\eta\} = \tau
			\label{fom:no_pair_of_sets_go_round_2}
		\end{align}
		および
		\begin{align}
			\REGAX \vdash \exists x\, (\, x \in \tau\, ) 
			\rarrow \exists y\, (\, y \in \tau \wedge \forall z\, (\, z \in \tau 
			\rarrow z \notin y\, )\, )
			\label{fom:no_pair_of_sets_go_round_3}
		\end{align}
		が成り立つ.(\refeq{fom:no_pair_of_sets_go_round_1})と
		(\refeq{fom:no_pair_of_sets_go_round_2})より
		\begin{align}
			\EXTAX,\EQAX,\COMAX,\PAIAX \vdash \chi \in \tau
			\label{fom:no_pair_of_sets_go_round_6}
		\end{align}
		が成り立つので%定理\ref{thm:emptyset_does_not_contain_any_class}より
		\begin{align}
			\EXTAX,\EQAX,\COMAX,\PAIAX \vdash \exists x\, (\, x \in \tau\, )
		\end{align}
		となり,(\refeq{fom:no_pair_of_sets_go_round_3})より
		\begin{align}
			\EXTAX,\EQAX,\COMAX,\PAIAX,\REGAX \vdash
			\exists y\, (\, y \in \tau \wedge \forall z\, (\, z \in \tau 
			\rarrow z \notin y\, )\, )
		\end{align}
		が従う.ここで
		\begin{align}
			\gamma \defeq \varepsilon y\, (\, y \in \tau \wedge \forall z\, (\, z \in \tau \rarrow z \in y\, )\, )
		\end{align}
		とおけば
		\begin{align}
			\EXTAX,\EQAX,\COMAX,\PAIAX,\REGAX &\vdash \gamma \in \tau, 
			\label{fom:no_pair_of_sets_go_round_4} \\
			\EXTAX,\EQAX,\COMAX,\PAIAX,\REGAX &\vdash \forall z\, (\, z \in \tau \rarrow z \notin \gamma\, )
			\label{fom:no_pair_of_sets_go_round_5}
		\end{align}
		が成り立つ.(\refeq{fom:no_pair_of_sets_go_round_5})より
		\begin{align}
			\EXTAX,\EQAX,\COMAX,\PAIAX,\REGAX \vdash 
			\chi \in \tau \rarrow \chi \notin \gamma
		\end{align}
		となり,(\refeq{fom:no_pair_of_sets_go_round_6})との三段論法より
		\begin{align}
			\EXTAX,\EQAX,\COMAX,\PAIAX,\REGAX \vdash \chi \notin \gamma
			\label{fom:no_pair_of_sets_go_round_8}
		\end{align}
		が成り立つが,
		\begin{align}
			\EQAX \vdash \chi \notin \gamma \rarrow 
			(\, \eta = \gamma \rarrow \chi \notin \eta\, )
		\end{align}
		も成り立つので三段論法より
		\begin{align}
			\EXTAX,\EQAX,\COMAX,\PAIAX,\REGAX \vdash 
			\eta = \gamma \rarrow \chi \notin \eta
		\end{align}
		となり,対偶律2 (論理的定理\ref{logicalthm:contraposition_2})と演繹定理の逆より
		\begin{align}
			\chi \in \eta,\ \EXTAX,\EQAX,\COMAX,\PAIAX,\REGAX \vdash 
			\eta \neq \gamma
			\label{fom:no_pair_of_sets_go_round_7}
		\end{align}
		が従う.他方で(\refeq{fom:no_pair_of_sets_go_round_2})と
		(\refeq{fom:no_pair_of_sets_go_round_4})より
		\begin{align}
			\EXTAX,\EQAX,\COMAX,\PAIAX,\REGAX \vdash \gamma \in \{\chi,\eta\}
		\end{align}
		が成り立つので,定理\ref{thm:pair_members_are_exactly_the_given_two}より
		\begin{align}
			\EXTAX,\EQAX,\COMAX,\PAIAX,\REGAX \vdash \chi = \gamma \vee \eta = \gamma
		\end{align}
		が従う.これと(\refeq{fom:no_pair_of_sets_go_round_7})と
		選言三段論法(論理的定理\ref{logicalthm:disjunctive_syllogism})より
		\begin{align}
			\chi \in \eta,\ \EXTAX,\EQAX,\COMAX,\PAIAX,\REGAX \vdash 
			\chi = \gamma
			\label{fom:no_pair_of_sets_go_round_9}
		\end{align}
		が従う.(\refeq{fom:no_pair_of_sets_go_round_8})を導いたのと同様にして
		\begin{align}
			\EXTAX,\EQAX,\COMAX,\PAIAX,\REGAX \vdash \eta \notin \gamma
		\end{align}
		も成り立つので,(\refeq{fom:no_pair_of_sets_go_round_9})と相等性公理より
		\begin{align}
			\chi \in \eta,\ \EXTAX,\EQAX,\COMAX,\PAIAX,\REGAX \vdash 
			\eta \notin \chi
		\end{align}
		が従う.演繹定理より
		\begin{align}
			\EXTAX,\EQAX,\COMAX,\PAIAX,\REGAX \vdash 
			\chi \in \eta \rarrow \eta \notin \chi
		\end{align}
		が成り立ち,全称の導出
		(論理的定理\ref{logicalthm:derivation_of_universal_by_epsilon})より
		\begin{align}
			\EXTAX,\EQAX,\COMAX,\PAIAX,\REGAX \vdash 
			\forall x\, \forall y\, (\, x \in y \rarrow y \notin x\, )
		\end{align}
		が得られる.
		\QED
	\end{sketch}
	
	\begin{screen}
		\begin{logicalthm}[論理和の結合律]
		\label{logicalthm:associative_law_of_conjunctions}
			$A,B,C$を$\mathcal{L}$の文とするとき
			\begin{align}
				\vdash (\, A \vee B\, ) \vee C \rarrow A \vee (\, B \vee C\, ).
			\end{align}
		\end{logicalthm}
	\end{screen}
	
	\begin{sketch}
		論理和の導入より
		\begin{align}
			\vdash A \rarrow A \vee (\, B \vee C\, )
			\label{fom:associative_law_of_conjunctions_1}
		\end{align}
		となる.同じく論理和の導入より
		\begin{align}
			B \vdash B \vee C
		\end{align}
		となるが,再び論理和の導入より
		\begin{align}
			\vdash B \vee C \rarrow A \vee (\, B \vee C\, )
		\end{align}
		となるので,三段論法より
		\begin{align}
			B \vdash A \vee (\, B \vee C\, )
		\end{align}
		となり,演繹定理より
		\begin{align}
			\vdash B \rarrow A \vee (\, B \vee C\, )
			\label{fom:associative_law_of_conjunctions_2}
		\end{align}
		が従う.(\refeq{fom:associative_law_of_conjunctions_1})と
		(\refeq{fom:associative_law_of_conjunctions_2})と論理和の除去より
		\begin{align}
			\vdash A \vee B \rarrow A \vee (\, B \vee C\, )
		\end{align}
		が得られる.他方で(\refeq{fom:associative_law_of_conjunctions_2})の導出と同様にして
		\begin{align}
			\vdash C \rarrow A \vee (\, B \vee C\, )
		\end{align}
		も得られるから,再び論理和の除去により
		\begin{align}
			\vdash (\, A \vee B\, ) \vee C \rarrow A \vee (\, B \vee C\, )
		\end{align}
		が出る.
		\QED
	\end{sketch}
	
	\begin{screen}
		\begin{thm}[集合のどの三組も所属関係で堂々巡りしない]
		\label{thm:no_three_sets_go_round}
			\begin{align}
				\EXTAX,\EQAX,\COMAX,\PAIAX,\UNIAX,\REGAX \vdash 
				\forall x\, \forall y\, \forall z\, 
				(\, x \in y \wedge y \in z \rarrow z \notin x\, ).
			\end{align}
		\end{thm}
	\end{screen}
	
	\begin{sketch}
		いま
		\begin{align}
			\chi &\defeq \varepsilon x \negation \forall y\, \forall z\, 
				(\, x \in y \wedge y \in z \rarrow z \notin x\, ), \\
			\eta &\defeq \varepsilon y \negation \forall z\, 
				(\, \chi \in y \wedge y \in z \rarrow z \notin \chi\, ), \\
			\zeta &\defeq \varepsilon z \negation 
				(\, \chi \in \eta \wedge \eta \in z \rarrow z \notin \chi\, )
		\end{align}
		とおく.$\chi,\eta,\zeta$は主要$\varepsilon$項であるから,
		定理\ref{thm:critical_epsilon_term_is_set}より
		\begin{align}
			\EXTAX &\vdash \set{\chi}, \label{fom:no_three_sets_go_round_1} \\
			\EXTAX &\vdash \set{\eta}, \label{fom:no_three_sets_go_round_2} \\
			\EXTAX &\vdash \set{\zeta} \label{fom:no_three_sets_go_round_3}
		\end{align}
		が成り立ち,定理\ref{thm:pair_of_sets_is_a_set} (集合の対は集合)より
		\begin{align}
			\EXTAX,\EQAX,\COMAX,\PAIAX \vdash \set{\{\chi,\eta\}}
			\label{fom:no_three_sets_go_round_4}
		\end{align}
		および
		\begin{align}
			\EXTAX,\EQAX,\COMAX,\PAIAX \vdash \set{\{\zeta\}}
			\label{fom:no_three_sets_go_round_5}
		\end{align}
		が成り立ち,定理\ref{thm:set_is_an_element_of_its_pair}
		(集合は自分自身の対の要素)より
		\begin{align}
			\EXTAX,\EQAX,\COMAX \vdash \chi \in \{\chi,\eta\}
			\label{fom:no_three_sets_go_round_6}
		\end{align}
		が成り立つ.定理\ref{thm:union_of_pair_is_union_of_their_elements}より
		\begin{align}
			\EXTAX,\EQAX,\COMAX \vdash 
			\set{\{\chi,\eta\}} \rarrow \forall x\, (\, x \in \{\chi,\eta\}
			\rarrow x \in \{\chi,\eta,\zeta\}\, )
		\end{align}
		が成り立つので,(\refeq{fom:no_three_sets_go_round_4})と
		(\refeq{fom:no_three_sets_go_round_6})との三段論法より
		\begin{align}
			\EXTAX,\EQAX,\COMAX,\PAIAX \vdash \chi \in \{\chi,\eta,\zeta\}
			\label{fom:no_three_sets_go_round_7}
		\end{align}
		となる.同様にして
		\begin{align}
			\EXTAX,\EQAX,\COMAX,\PAIAX \vdash \eta \in \{\chi,\eta,\zeta\},
			\label{fom:no_three_sets_go_round_8}
		\end{align}
		も成り立つ.また(\refeq{fom:no_three_sets_go_round_4})
		と(\refeq{fom:no_three_sets_go_round_5})と
		定理\ref{thm:pair_of_sets_is_a_set} (集合の対は集合)より
		\begin{align}
			\EXTAX,\EQAX,\COMAX,\PAIAX \vdash \set{\{\{\chi,\eta\},\{\zeta\}\}}
		\end{align}
		となるので,定理\ref{thm:union_of_a_set_is_a_set} (集合の合併は集合)と併せて
		\begin{align}
			\EXTAX,\EQAX,\COMAX,\PAIAX,\UNIAX \vdash \set{\{\chi,\eta,\zeta\}}
		\end{align}
		が成立する.ここで
		\begin{align}
			\tau \defeq \varepsilon x\, (\, \{\chi,\eta,\zeta\} = x\, )
			\label{fom:no_three_sets_go_round_9}
		\end{align}
		とおけば
		\begin{align}
			\EXTAX,\EQAX,\COMAX,\PAIAX,\UNIAX \vdash \{\chi,\eta,\zeta\} = \tau
			\label{fom:no_three_sets_go_round_10}
		\end{align}
		が成り立つので,(\refeq{fom:no_three_sets_go_round_7})と相等性公理より
		\begin{align}
			\EXTAX,\EQAX,\COMAX,\PAIAX,\UNIAX \vdash \chi \in \tau
			\label{fom:no_three_sets_go_round_11}
		\end{align}
		が従い,
		\begin{align}
			\EXTAX,\EQAX,\COMAX,\PAIAX,\UNIAX \vdash \exists x\, (\, x \in \tau\, )
			\label{fom:no_three_sets_go_round_12}
		\end{align}
		が従う.
		\begin{align}
			\REGAX \vdash \exists x\, (\, x \in \tau\, )
			\rarrow \exists y\, (\, y \in \tau \wedge \forall z\, (\, z \in \tau \rarrow z \notin y\, )\, )
		\end{align}
		が成り立つので,(\refeq{fom:no_three_sets_go_round_12})との三段論法より
		\begin{align}
			\EXTAX,\EQAX,\COMAX,\PAIAX,\UNIAX,\REGAX \vdash 
			\exists y\, (\, y \in \tau \wedge \forall z\, (\, z \in \tau \rarrow z \notin y\, )\, )
		\end{align}
		が従う.
		\begin{align}
			\gamma \defeq \varepsilon y\, (\, y \in \tau \wedge \forall z\, (\, z \in \tau \rarrow z \notin y\, )\, )
		\end{align}
		とおけば
		\begin{align}
			\EXTAX,\EQAX,\COMAX,\PAIAX,\UNIAX,\REGAX &\vdash \gamma \in \tau, 
			\label{fom:no_three_sets_go_round_13} \\
			\EXTAX,\EQAX,\COMAX,\PAIAX,\UNIAX,\REGAX &\vdash \forall z\, (\, z \in \tau \rarrow z \notin \gamma\, )\, )
			\label{fom:no_three_sets_go_round_14}
		\end{align}
		が成り立つ.特に
		\begin{align}
			\EXTAX,\EQAX,\COMAX,\PAIAX,\UNIAX,\REGAX &\vdash \chi \in \tau \rarrow \chi \notin \gamma, \\
			\EXTAX,\EQAX,\COMAX,\PAIAX,\UNIAX,\REGAX &\vdash \eta \in \tau \rarrow \eta \notin \gamma
		\end{align}
		となるが,(\refeq{fom:no_three_sets_go_round_7})と
		(\refeq{fom:no_three_sets_go_round_8})および
		(\refeq{fom:no_three_sets_go_round_10})より
		\begin{align}
			\EXTAX,\EQAX,\COMAX,\PAIAX,\UNIAX &\vdash \chi \in \tau, \\
			\EXTAX,\EQAX,\COMAX,\PAIAX,\UNIAX &\vdash \eta \in \tau
		\end{align}
		が成り立つので
		\begin{align}
			\EXTAX,\EQAX,\COMAX,\PAIAX,\UNIAX,\REGAX &\vdash \chi \notin \gamma, 
			\label{fom:no_three_sets_go_round_15} \\
			\EXTAX,\EQAX,\COMAX,\PAIAX,\UNIAX,\REGAX &\vdash \eta \notin \gamma
			\label{fom:no_three_sets_go_round_16}
		\end{align}
		が従う.ところで
		\begin{align}
			\EQAX &\vdash \chi \notin \gamma \rarrow (\, \chi \in \eta \rarrow \eta \neq \gamma\, ), \\
			\EQAX &\vdash \eta \notin \gamma \rarrow (\, \eta \in \zeta \rarrow \zeta \neq \gamma\, )
		\end{align}
		が成り立つので,(\refeq{fom:no_three_sets_go_round_15})と
		(\refeq{fom:no_three_sets_go_round_16})から
		\begin{align}
			\EXTAX,\EQAX,\COMAX,\PAIAX,\UNIAX,\REGAX &\vdash \chi \in \eta \rarrow \eta \neq \gamma, 
			\label{fom:no_three_sets_go_round_17} \\
			\EXTAX,\EQAX,\COMAX,\PAIAX,\UNIAX,\REGAX &\vdash \eta \in \zeta \rarrow \zeta \neq \gamma
			\label{fom:no_three_sets_go_round_18}
		\end{align}
		が従い,
		\begin{align}
			\chi \in \eta \wedge \eta \in \zeta,\ 
			\EXTAX,\EQAX,\COMAX,\PAIAX,\UNIAX,\REGAX \vdash \eta \neq \gamma \wedge \zeta \neq \gamma
		\end{align}
		が従い,De Morgan の法則(論理的定理\ref{logicalthm:weak_De_Morgan_law_1})より
		\begin{align}
			\chi \in \eta \wedge \eta \in \zeta,\ 
			\EXTAX,\EQAX,\COMAX,\PAIAX,\UNIAX,\REGAX \vdash\ 
			\negation (\, \eta = \gamma \vee \zeta = \gamma\, )
			\label{fom:no_three_sets_go_round_19}
		\end{align}
		となる.他方で定理\ref{thm:union_of_pair_is_union_of_their_elements}より
		\begin{align}
			\EXTAX,\EQAX,\COMAX \vdash 
			\set{\{\zeta\}} \rarrow \forall x\, (\, x \in \{\zeta\}
			\rarrow x \in \{\zeta\} \cup \{\chi,\eta\}\, )
		\end{align}
		が成り立つので,(\refeq{fom:no_three_sets_go_round_5})と
		\begin{align}
			\EXTAX,\EQAX,\COMAX \vdash \zeta \in \{\zeta\}
		\end{align}
		(定理\ref{thm:set_is_an_element_of_its_pair})との三段論法より
		\begin{align}
			\EXTAX,\EQAX,\COMAX,\PAIAX \vdash \zeta \in \{\zeta\} \cup \{\chi,\eta\}
		\end{align}
		が従い,定理\ref{thm:symmetry_of_union_of_a_pair}
		(合併の対称性)より
		\begin{align}
			\EXTAX,\EQAX,\COMAX,\PAIAX \vdash \zeta \in \{\chi,\eta,\zeta\}
		\end{align}
		が成り立つ.従って(\refeq{fom:no_three_sets_go_round_10})より
		\begin{align}
			\EXTAX,\EQAX,\COMAX,\PAIAX,\UNIAX \vdash \zeta \in \tau
			\label{fom:no_three_sets_go_round_20}
		\end{align}
		となる.(\refeq{fom:no_three_sets_go_round_13})と
		(\refeq{fom:no_three_sets_go_round_10})より
		\begin{align}
			\EXTAX,\EQAX,\COMAX,\PAIAX,\UNIAX,\REGAX \vdash \gamma \in \{\chi,\eta,\zeta\}
		\end{align}
		が成り立つので,定理\ref{thm:triple_members_are_exactly_the_given_three}
		(表示されている要素しか持たない)より
		\begin{align}
			\EXTAX,\EQAX,\COMAX,\PAIAX,\UNIAX,\REGAX \vdash 
			(\, \chi = \gamma \vee \eta = \gamma\, ) \vee \zeta = \gamma
		\end{align}
		となるが,論理和の結合律
		(論理的定理\ref{logicalthm:associative_law_of_conjunctions})より
		\begin{align}
			\EXTAX,\EQAX,\COMAX,\PAIAX,\UNIAX,\REGAX \vdash 
			\chi = \gamma \vee (\, \eta = \gamma \vee \zeta = \gamma\, )
		\end{align}
		となる.
	\end{sketch}
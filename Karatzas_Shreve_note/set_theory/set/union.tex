\section{合併}
	$a$を空でないクラスとするとするとき,$a$の要素もまた空でなければ要素を持つ.
	$a$の要素の要素を全て集めたものを$a$の合併と呼び,その受け皿の意味を込めて
	\begin{align}
		\bigcup a
	\end{align}
	と書く.当然ながら,空の合併は空となる.
	
	\begin{screen}
		\begin{dfn}[合併]
			$x$を$\mathcal{L}$の項とするとき,
			$x$の{\bf 合併}\index{がっぺい@合併}{\bf (union)}を
			\begin{align}
				\bigcup x \defeq \Set{y}{\exists z \in x\, (\, y \in z\, )}
				\label{eq:definition_of_union_1}
			\end{align}
			で定める($x$が内包項なら$\exists z \in x\, (\, y \in z\, )$は
			$\lang{\varepsilon}$の式に書き換える).
		\end{dfn}
	\end{screen}
	
	\begin{description}
		\item[量化子が付いた式の略記法]
		上の定義で
		\begin{align}
			\exists z \in x\, (\, y \in z\, )
		\end{align}
		という式を書いたが,これは
		\begin{align}
			\exists z \in x\, (\, y \in z\, ) \defarrow 
			\exists z\, (\, z \in x \wedge y \in z\, )
		\end{align}
		により定義される省略形である.同様にして,$\varphi$を式とするとき
		\begin{align}
			\exists z\, \left(\, z \in x \wedge \varphi\, \right)
		\end{align}
		なる式を
		\begin{align}
			\exists z \in x\, \varphi
		\end{align}
		と略記する.また全称記号についても
		\begin{align}
			\forall z\, \left(\, z \in x \rarrow \varphi\, \right)
		\end{align}
		なる式を
		\begin{align}
			\forall z \in x\, \varphi
		\end{align}
		と略記する.
	\end{description}
	
	\begin{screen}
		\begin{thm}[合併の内包性]
		\label{thm:comprehension_of_unions}
			$a$をクラスとするとき
			\begin{align}
				\COMAX \vdash \forall y\, (\, y \in \bigcup a \lrarrow 
				\exists z\, (\, z \in a \wedge y \in z\, )\, ).
			\end{align}
		\end{thm}
	\end{screen}
	
	\begin{sketch}
		$a$が主要$\varepsilon$項である場合は内包性公理から直接
		\begin{align}
			\COMAX \vdash \forall y\, (\, y \in \bigcup a \lrarrow 
				\exists z\, (\, z \in a \wedge y \in z\, )\, )
		\end{align}
		が成立する.$a$が$\Set{x}{\varphi(x)}$なる内包項の場合は
		\begin{align}
			\bigcup a \defeq \Set{y}{\exists z\, (\, \varphi(z) \wedge y \in z\, )}
		\end{align}
		と定義されることになり,
		\begin{align}
			\COMAX \vdash \forall y\, (\, y \in \bigcup a \lrarrow 
				\exists z\, (\, \varphi(z) \wedge y \in z\, )\, )
			\label{fom:comprehension_of_unions_1}
		\end{align}
		が成立する.ここで
		\begin{align}
			\COMAX \vdash \forall y\, (\, y \in \bigcup a \lrarrow 
				\exists z\, (\, z \in a \wedge y \in z\, )\, )
		\end{align}
		を示すために
		\begin{align}
			\eta \defeq \varepsilon y \negation (\, y \in \bigcup a \lrarrow 
				\exists z\, (\, z \in a \wedge y \in z\, )\, )
		\end{align}
		とおく(右辺は$\lang{\varepsilon}$の式に直す).
		\begin{description}
			\item[step1]
				いま
				\begin{align}
					\zeta \defeq \varepsilon z\, (\, \varphi(z) \wedge \eta \in z\, )
				\end{align}
				とおけば,(\refeq{fom:comprehension_of_unions_1})より
				\begin{align}
					\eta \in \bigcup a,\ \COMAX \vdash
					\varphi(\zeta) \wedge \eta \in \zeta
				\end{align}
				が成立する.他方で
				\begin{align}
					\COMAX \vdash \varphi(\zeta) \rarrow \zeta \in a
				\end{align}
				も成り立つので
				\begin{align}
					\eta \in \bigcup a,\ \COMAX \vdash
					\zeta \in a \wedge \eta \in \zeta
				\end{align}
				が従う.ゆえに
				\begin{align}
					\COMAX \vdash \eta \in \bigcup a \rarrow
					\exists z\, (\, z \in a \wedge \eta \in z\, )
					\label{fom:comprehension_of_unions_2}
				\end{align}
				が得られる.
				
			\item[step2]
				$\zeta$を先と同じものにすれば
				\begin{align}
					\exists z\, (\, z \in a \wedge \eta \in z\, )
					\vdash \zeta \in a \wedge \eta \in \zeta
				\end{align}
				が成立する.また
				\begin{align}
					\COMAX \vdash \zeta \in a \rarrow \varphi(\zeta)
				\end{align}
				も成り立つので
				\begin{align}
					\exists z\, (\, z \in a \wedge \eta \in z\, ),\ \COMAX
					\vdash \varphi(\zeta) \wedge \eta \in \zeta
				\end{align}
				が従う.(\refeq{fom:comprehension_of_unions_1})より
				\begin{align}
					\COMAX \vdash \varphi(\zeta) \wedge \eta \in \zeta
					\rarrow \eta \in \bigcup a
				\end{align}
				も成り立つので
				\begin{align}
					\exists z\, (\, z \in a \wedge \eta \in z\, ),\ \COMAX \vdash 
					\eta \in \bigcup a
					\label{fom:comprehension_of_unions_3}
				\end{align}
				が得られる.
			
			\item[step3]
				(\refeq{fom:comprehension_of_unions_2})と
				(\refeq{fom:comprehension_of_unions_3})より
				\begin{align}
					\COMAX \vdash \eta \in \bigcup a 
					\lrarrow \exists z\, (\, z \in a \wedge \eta \in z\, )
				\end{align}
				が成立するので,全称の導出(論理的定理\ref{logicalthm:derivation_of_universal_by_epsilon})より
				\begin{align}
					\COMAX \vdash \forall y\, (\, y \in \bigcup a 
					\lrarrow \exists z\, (\, z \in a \wedge y \in z\, )\, )
				\end{align}
				が得られる.
				\QED
		\end{description}
	\end{sketch}
	
	\begin{screen}
		\begin{axm}[合併の公理]
			次の式を$\UNIAX$によって参照する:
			\begin{align}
				\forall x\, \exists u\, \forall y\, (\, \exists z\, (\, z \in x \wedge y \in z\, ) \lrarrow y \in u\, ).
			\end{align}
		\end{axm}
	\end{screen}
	
	\begin{screen}
		\begin{thm}[集合の合併は集合]
		\label{thm:union_of_a_set_is_a_set}
			$a$をクラスとするとき
			\begin{align}
				\EXTAX,\EQAX,\COMAX,\UNIAX \vdash \set{a} \rarrow \set{\bigcup a}.
			\end{align}
		\end{thm}
	\end{screen}
	
	\begin{sketch}\mbox{}
		\begin{description}
			\item[step1]
				まず
				\begin{align}
					\tau \defeq \varepsilon x\, (\, a = x\, )
				\end{align}
				とおけば(必要に応じて$a = x$を$\lang{\varepsilon}$の式に書き換える),
				\begin{align}
					\set{a} \vdash a = \tau
				\end{align}
				が成立する.$\tau$に対して
				\begin{align}
					\UNIAX \vdash \exists u\, \forall y\, (\, \exists z\, (\, z \in \tau \wedge y \in z\, ) \lrarrow y \in u\, )
				\end{align}
				が成り立つので,
				\begin{align}
					\upsilon \defeq \varepsilon u\, \forall y\, (\, \exists z\, (\, z \in \tau \wedge y \in z\, ) \lrarrow y \in u\, )
				\end{align}
				とおけば
				\begin{align}
					\UNIAX \vdash \forall y\, (\, \exists z\, (\, z \in \tau \wedge y \in z\, ) \lrarrow y \in \upsilon\, )
					\label{fom:union_of_a_set_is_a_set_1}
				\end{align}
				が成立する.次に$\tau$を$a$に置き換えた場合に
				\begin{align}
					\set{a},\ \EQAX,\UNIAX \vdash \forall y\, (\, \exists z\, (\, z \in a \wedge y \in z\, ) \lrarrow y \in \upsilon\, )
				\end{align}
				が成立することを示す.
				
			\item[step2]
				いま
				\begin{align}
					\eta \defeq \varepsilon y \negation  (\, \exists z\, (\, z \in a \wedge y \in z\, ) \lrarrow y \in \upsilon\, )
				\end{align}
				とおけば,(\refeq{fom:union_of_a_set_is_a_set_1})より
				\begin{align}
					\UNIAX \vdash \exists z\, (\, z \in \tau \wedge \eta \in z\, )
					\lrarrow \eta \in \upsilon
				\end{align}
				が成立する.
				\begin{align}
					\zeta \defeq \varepsilon z\, (\, z \in a \wedge \eta \in z\, )
				\end{align}
				とおけば
				\begin{align}
					\exists z\, (\, z \in a \wedge \eta \in z\, )
					\vdash \zeta \in a \wedge \eta \in \zeta
				\end{align}
				が成り立ち,
				\begin{align}
					\EQAX \vdash a = \tau \rarrow (\, \zeta \in a \rarrow \zeta \in \tau\, )
				\end{align}
				と併せて
				\begin{align}
					\exists z\, (\, z \in a \wedge \eta \in z\, ),\ \set{a},\ \EQAX
					\vdash \zeta \in \tau \wedge \eta \in \zeta
				\end{align}
				が成立する.また(\refeq{fom:union_of_a_set_is_a_set_1})より
				\begin{align}
					\UNIAX \vdash (\, \zeta \in \tau \wedge \eta \in \zeta\, )
					\rarrow \eta \in \upsilon
				\end{align}
				が成り立つので
				\begin{align}
					\exists z\, (\, z \in a \wedge \eta \in z\, ),\ \set{a},\ \EQAX,\UNIAX \vdash \eta \in \upsilon
				\end{align}
				が従う.ゆえに
				\begin{align}
					\set{a},\ \EQAX,\UNIAX \vdash 
					\exists z\, (\, z \in a \wedge \eta \in z\, ) \rarrow \eta \in \upsilon
					\label{fom:union_of_a_set_is_a_set_2}
				\end{align}
				が得られた.
				
			\item[step3]
				逆に(\refeq{fom:union_of_a_set_is_a_set_1})より
				\begin{align}
					\eta \in \upsilon,\ \UNIAX \vdash
					\exists z\, (\, z \in \tau \wedge \eta \in z\, )
				\end{align}
				が成り立つので
				\begin{align}
					\eta \in \upsilon,\ \UNIAX \vdash
					\zeta \in \tau \wedge \eta \in \zeta
				\end{align}
				が従い,
				\begin{align}
					\set{a},\ \EQAX \vdash
					\zeta \in \tau \rarrow \zeta \in a
				\end{align}
				と併せて
				\begin{align}
					\eta \in \upsilon,\ \set{a},\ \EQAX,\UNIAX \vdash
					\zeta \in a \wedge \eta \in \zeta
				\end{align}
				が従い,
				\begin{align}
					\eta \in \upsilon,\ \set{a},\ \EQAX,\UNIAX \vdash
					\exists z\, (\, z \in a \wedge \eta \in z\, )
				\end{align}
				が従う.そして演繹定理より
				\begin{align}
					\set{a},\ \EQAX,\UNIAX \vdash
					\eta \in \upsilon \rarrow \exists z\, (\, z \in a \wedge \eta \in z\, )
					\label{fom:union_of_a_set_is_a_set_3}
				\end{align}
				も得られる.
				
			\item[step4]
				(\refeq{fom:union_of_a_set_is_a_set_2})と
				(\refeq{fom:union_of_a_set_is_a_set_3})より
				\begin{align}
					\set{a},\ \EQAX,\UNIAX \vdash
					\exists z\, (\, z \in a \wedge \eta \in z\, ) \lrarrow \eta \in \upsilon
				\end{align}
				が得られ,全称の導出(論理的定理\ref{logicalthm:derivation_of_universal_by_epsilon})より
				\begin{align}
					\set{a},\ \EQAX,\UNIAX \vdash
					\forall y\, (\, \exists z\, (\, z \in a \wedge y \in z\, ) \lrarrow y \in \upsilon\, )
				\end{align}
				となり,定理\ref{thm:equivalent_formula_rewriting_4}より
				\begin{align}
					\set{a},\ \EXTAX,\EQAX,\COMAX,\UNIAX \vdash
					\Set{z}{\exists z\, (\, z \in a \wedge y \in z\, )} = \upsilon
				\end{align}
				が成り立つ.存在記号の論理的公理より
				\begin{align}
					\set{a},\ \EXTAX,\EQAX,\COMAX,\UNIAX \vdash
					\exists u\, (\, \Set{z}{\exists z\, (\, z \in a \wedge y \in z\, )} = u\, )
				\end{align}
				が成り立つので,定理が得られた.
				\QED
		\end{description}
	\end{sketch}
	
	\begin{screen}
		\begin{thm}[空集合の合併は空]\label{thm:the_union_of_the_emptyset_is_empty}
			\begin{align}
				\EXTAX,\COMAX \vdash \bigcup \emptyset = \emptyset.
			\end{align}
		\end{thm}
	\end{screen}
	
	\begin{sketch}
		いま
		\begin{align}
			\zeta &\defeq \varepsilon z \negation (\, z \notin \bigcup \emptyset\, ), \\
			\eta &\defeq \varepsilon y \negation \negation (\, y \in \emptyset \wedge \zeta \in y\, )
		\end{align}
		とおく.定理\ref{thm:emptyset_has_nothing}より
		\begin{align}
			\EXTAX,\COMAX \vdash \eta \notin \emptyset
		\end{align}
		が成り立つので
		\begin{align}
			\EXTAX,\COMAX \vdash \eta \notin \emptyset \vee \zeta \notin \eta
		\end{align}
		も成立し,De Morgan の法則(論理的定理\ref{logicalthm:strong_De_Morgan_law_1})より
		\begin{align}
			\EXTAX,\COMAX \vdash\ \negation (\, \eta \in \emptyset \wedge \zeta \in \eta\, )
		\end{align}
		が成立し,全称の導出(論理的定理\ref{logicalthm:derivation_of_universal_by_epsilon})より
		\begin{align}
			\EXTAX,\COMAX \vdash \forall y \negation (\, y \in \emptyset \wedge \zeta \in y\, )
		\end{align}
		が成立する.そして量化子の De Morgan の法則
		(論理的定理\ref{logicalthm:strong_De_Morgan_law_for_quantifiers_1})より
		\begin{align}
			\EXTAX,\COMAX \vdash\ \negation \exists y\, (\, y \in \emptyset \wedge \zeta \in y\, )
			\label{fom:the_union_of_the_emptyset_is_empty_1}
		\end{align}
		が得られる.他方で
		\begin{align}
			\COMAX \vdash \zeta \in \bigcup \emptyset
			\rarrow \exists y\, (\, y \in \emptyset \wedge \zeta \in y\, )
		\end{align}
		が成り立つので,対偶を取って
		\begin{align}
			\COMAX \vdash\ 
			\negation \exists y\, (\, y \in \emptyset \wedge \zeta \in y\, )
			\rarrow \zeta \notin \bigcup \emptyset
			\label{fom:the_union_of_the_emptyset_is_empty_2}
		\end{align}
		が得られる.(\refeq{fom:the_union_of_the_emptyset_is_empty_1})と
		(\refeq{fom:the_union_of_the_emptyset_is_empty_2})より
		\begin{align}
			\EXTAX,\COMAX \vdash \zeta \notin \bigcup \emptyset
		\end{align}
		が成り立つので,全称の導出(論理的定理\ref{logicalthm:derivation_of_universal_by_epsilon})より
		\begin{align}
			\EXTAX,\COMAX \vdash \forall z\, (\, z \notin \bigcup \emptyset\, )
		\end{align}
		が従い,定理\ref{thm:uniqueness_of_emptyset}より
		\begin{align}
			\EXTAX,\COMAX \vdash \bigcup \emptyset = \emptyset
		\end{align}
		が得られる.
		\QED
	\end{sketch}
	
	\begin{screen}
		\begin{thm}[等しいクラスの合併は等しい]\label{thm:unions_of_equal_classes_are_equal}
			$a$と$b$をクラスとするとき
			\begin{align}
				\EXTAX,\EQAX,\COMAX \vdash a = b \rarrow \bigcup a = \bigcup b.
			\end{align}
		\end{thm}
	\end{screen}
	
	\begin{sketch}
		いま
		\begin{align}
			\eta \defeq \varepsilon y \negation (\, y \in \bigcup a 
			\lrarrow y \in \bigcup b\, )
		\end{align}
		とおく.合併の内包性(定理\ref{thm:comprehension_of_unions})より
		\begin{align}
			\eta \in \bigcup a,\ \COMAX \vdash
			\exists z\, (\, z \in a \wedge \eta \in z\, )
		\end{align}
		が成り立つので,
		\begin{align}
			\zeta \defeq \varepsilon z\, (\, z \in a \wedge \eta \in z\, )
		\end{align}
		とおけば
		\begin{align}
			\eta \in \bigcup a,\ \COMAX \vdash
			\zeta \in a \wedge \eta \in \zeta
		\end{align}
		となる.ところで
		\begin{align}
			\EQAX \vdash a = b \rarrow (\, \zeta \in a \rarrow \zeta \in b\, )
		\end{align}
		が成り立つので
		\begin{align}
			a = b,\ \eta \in \bigcup a,\ \EQAX,\COMAX \vdash
			\zeta \in b \wedge \eta \in \zeta
		\end{align}
		が得られ,
		\begin{align}
			a = b,\ \eta \in \bigcup a,\ \EQAX,\COMAX \vdash
			\exists z\, (\, z \in b \wedge \eta \in z\, )
		\end{align}
		が従う.そして合併の内包性(定理\ref{thm:comprehension_of_unions})より
		\begin{align}
			a = b,\ \eta \in \bigcup a,\ \EQAX,\COMAX \vdash \eta \in \bigcup b
			\label{fom:unions_of_equal_classes_are_equal_1}
		\end{align}
		が得られる.$a$と$b$を入れ替えれば
		\begin{align}
			\EQAX,\COMAX \vdash 
			b = a \rarrow (\, \eta \in \bigcup b \rarrow \eta \in \bigcup a\, )
		\end{align}
		となるが,
		\begin{align}
			a = b,\ \EQAX \vdash b = a 
		\end{align}
		と併せて
		\begin{align}
			a = b,\ \EQAX,\COMAX \vdash 
			\eta \in \bigcup b \rarrow \eta \in \bigcup a
			\label{fom:unions_of_equal_classes_are_equal_2}
		\end{align}
		が得られる.(\refeq{fom:unions_of_equal_classes_are_equal_1})と
		(\refeq{fom:unions_of_equal_classes_are_equal_2})より
		\begin{align}
			a = b,\ \EQAX,\COMAX \vdash 
			\eta \in \bigcup a \lrarrow \eta \in \bigcup b
		\end{align}
		が従い,全称の導出(論理的定理\ref{logicalthm:derivation_of_universal_by_epsilon})より
		\begin{align}
			a = b,\ \EQAX,\COMAX \vdash 
			\forall y\, (\, y \in \bigcup a \lrarrow y \in \bigcup b\, )
		\end{align}
		が成立し,外延性公理と併せて
		\begin{align}
			a = b,\ \EXTAX,\EQAX,\COMAX \vdash 
			\bigcup a = \bigcup b
		\end{align}
		を得る.
		\QED
	\end{sketch}
	
	\begin{itembox}[l]{対の合併}
		$x$と$y$を$\mathcal{L}$の項とするとき,その対の合併を
		\begin{align}
			x \cup y \defeq \Set{z}{z \in x \vee z \in y}
		\end{align}
		と書く.
	\end{itembox}
	
	\begin{screen}
		\begin{thm}[二つのクラスの合併はそれぞれの要素を合わせたもの]
		\label{thm:elements_of_pair_union}
			$a$と$b$をクラスとするとき
			\begin{align}
				\EXTAX,\EQAX,\COMAX,\ELEAX \vdash \forall x\, (\, x \in a \cup b \lrarrow x \in a \vee x \in b\, ).
			\end{align}
			ただし$a$と$b$が共に主要$\varepsilon$項であれば
			\begin{align}
				\COMAX \vdash \forall x\, (\, x \in a \cup b \lrarrow x \in a \vee x \in b\, ).
			\end{align}
		\end{thm}
	\end{screen}
	
	\begin{sketch}
		前者の主張は注意\ref{rem:epsilon_terms_of_not_L_epsilon_formula}より,
		後者の主張は内包性公理の直接の適用で得られる.
		\QED
	\end{sketch}
	
	\begin{screen}
		\begin{thm}[対の合併の対称性]
		\label{thm:symmetry_of_union_of_a_pair}
			$a$と$b$をクラスとするとき
			\begin{align}
				\EXTAX,\EQAX,\COMAX,\ELEAX \vdash a \cup b = b \cup a.
			\end{align}
			ただし,$a$と$b$が共に主要$\varepsilon$項であれば
			\begin{align}
				\EXTAX,\COMAX \vdash a \cup b = b \cup a.
			\end{align}
		\end{thm}
	\end{screen}
	
	\begin{sketch}
		いま
		\begin{align}
			\zeta \defeq \varepsilon z \negation (\, z \in a \cup b \lrarrow z \in b \cup a\, )
		\end{align}
		とおく.定理\ref{thm:elements_of_pair_union}より
		\begin{align}
			\zeta \in a \cup b,\ \EXTAX,\EQAX,\COMAX,\ELEAX \vdash \zeta \in a \vee \zeta \in b
			\label{fom:symmetry_of_union_of_a_pair_1}
		\end{align}
		が成り立ち,論理和の可換律(論理的定理\ref{logicalthm:commutative_law_of_disjunction})より
		\begin{align}
			\zeta \in a \cup b,\ \EXTAX,\EQAX,\COMAX,\ELEAX \vdash \zeta \in b \vee \zeta \in a
		\end{align}
		が従い,再び定理\ref{thm:elements_of_pair_union}より
		\begin{align}
			\zeta \in a \cup b,\ \EXTAX,\EQAX,\COMAX,\ELEAX \vdash \zeta \in b \cup a
		\end{align}
		となり,演繹定理より
		\begin{align}
			\EXTAX,\EQAX,\COMAX,\ELEAX \vdash \zeta \in a \cup b \rarrow \zeta \in b \cup a
		\end{align}
		が得られる.同様にして
		\begin{align}
			\EXTAX,\EQAX,\COMAX,\ELEAX \vdash \zeta \in b \cup a \rarrow \zeta \in a \cup b
		\end{align}
		も得られるから,論理積の導入より
		\begin{align}
			\EXTAX,\EQAX,\COMAX,\ELEAX \vdash \zeta \in a \cup b \lrarrow \zeta \in b \cup a
		\end{align}
		が成り立ち,全称の導出(論理的定理\ref{logicalthm:derivation_of_universal_by_epsilon})より
		\begin{align}
			\EXTAX,\EQAX,\COMAX,\ELEAX \vdash \forall z\, (\, z \in a \cup b \lrarrow z \in b \cup a\, )
		\end{align}
		が従う.そして外延性公理より
		\begin{align}
			\EXTAX,\EQAX,\COMAX,\ELEAX \vdash a \cup b = b \cup a
			\label{fom:symmetry_of_union_of_a_pair_2}
		\end{align}
		が出る.$a$と$b$が主要$\varepsilon$項である場合は,(\refeq{fom:symmetry_of_union_of_a_pair_1})
		の前提のうち$\EXTAX,\EQAX,\ELEAX$が不要になり,(\refeq{fom:symmetry_of_union_of_a_pair_2})で$\EXTAX$が追加される.
		\QED
	\end{sketch}
	
	\begin{screen}
		\begin{thm}[二つの集合の合併は対の合併]
		\label{thm:union_of_two_sets_is_union_of_pair}
			$a$と$b$をクラスとするとき
			\begin{align}
				\set{a},\ \set{b},\ \EXTAX,\EQAX,\COMAX,\ELEAX \vdash a \cup b = \bigcup \{a,b\}.
			\end{align}
		\end{thm}
	\end{screen}
	
	\begin{sketch}
		いま
		\begin{align}
			\tau \defeq \varepsilon x \negation (\, x \in a \cup b \lrarrow x \in \bigcup \{a,b\}\, )
		\end{align}
		とおく.
		\begin{description}
			\item[step1] この段では
				\begin{align}
					\set{a},\ \set{b},\ \EXTAX,\EQAX,\COMAX,\ELEAX \vdash 
					\tau \in a \cup b \rarrow \tau \in \bigcup \{a,b\}
				\end{align}
				を示す.
				\begin{align}
					\chi \defeq \varepsilon x\, (\, a = x\, )
				\end{align}
				とおけば
				\begin{align}
					\set{a} \vdash a = \chi
				\end{align}
				が成り立つが,他方で定理\ref{thm:pair_members_are_exactly_the_given_two}
				(対は表示されている要素しか持たない)より
				\begin{align}
					\EXTAX,\EQAX,\COMAX \vdash a = \chi \vee b = \chi \rarrow \chi \in \{a,b\}
				\end{align}
				が成り立つので,
				\begin{align}
					\set{a},\ \EXTAX,\EQAX,\COMAX \vdash \chi \in \{a,b\}
				\end{align}
				となる.また相等性公理より
				\begin{align}
					\tau \in a,\ \set{a},\ \EQAX \vdash \tau \in \chi
				\end{align}
				も成り立つから,論理積の導入より
				\begin{align}
					\tau \in a,\ \set{a},\ \EXTAX,\EQAX,\COMAX \vdash \chi \in \{a,b\} \wedge \tau \in \chi
				\end{align}
				が従い,存在記号の論理的公理より
				\begin{align}
					\tau \in a,\ \set{a},\ \EXTAX,\EQAX,\COMAX \vdash \exists x\, (\, x \in \{a,b\} \wedge \tau \in x\, )
				\end{align}
				となる.合併の定義と注意\ref{rem:epsilon_terms_of_not_L_epsilon_formula}より
				\begin{align}
					\EXTAX,\EQAX,\COMAX,\ELEAX \vdash \exists x\, (\, x \in \{a,b\} \wedge \tau \in x\, )
					\rarrow \tau \in \bigcup \{a,b\}
				\end{align}
				が成り立つから,三段論法より
				\begin{align}
					\tau \in a,\ \set{a},\ \EXTAX,\EQAX,\COMAX,\ELEAX \vdash \tau \in \bigcup \{a,b\}
				\end{align}
				が従い,演繹定理より
				\begin{align}
					\set{a},\ \EXTAX,\EQAX,\COMAX,\ELEAX \vdash 
					\tau \in a \rarrow \tau \in \bigcup \{a,b\}
				\end{align}
				が得られる.同様にして
				\begin{align}
					\set{b},\ \EXTAX,\EQAX,\COMAX,\ELEAX \vdash \tau \in b \rarrow \tau \in \bigcup \{a,b\}
				\end{align}
				も得られるので,論理和の除去より
				\begin{align}
					\set{a},\ \set{b},\ \EXTAX,\EQAX,\COMAX,\ELEAX \vdash 
					\tau \in a \vee \tau \in b \rarrow \tau \in \bigcup \{a,b\}
				\end{align}
				が従う.ところで定理\ref{thm:elements_of_pair_union}より
				\begin{align}
					\tau \in a \cup b,\ \EXTAX,\EQAX,\COMAX,\ELEAX \vdash \tau \in a \vee \tau \in b
				\end{align}
				が成り立つので,三段論法と演繹定理より
				\begin{align}
					\set{a},\ \set{b},\ \EXTAX,\EQAX,\COMAX,\ELEAX \vdash 
					\tau \in a \cup b \rarrow \tau \in \bigcup \{a,b\}
					\label{fom:union_of_two_sets_is_union_of_pair_5}
				\end{align}
				が出る.
				
			\item[step2] この段では
				\begin{align}
					\EXTAX,\EQAX,\COMAX,\ELEAX \vdash 
					\tau \in \bigcup \{a,b\} \rarrow \tau \in a \cup b
				\end{align}
				を示す.合併の定義と注意\ref{rem:epsilon_terms_of_not_L_epsilon_formula}より
				\begin{align}
					\tau \in \bigcup \{a,b\},\ \EXTAX,\EQAX,\COMAX,\ELEAX \vdash 
					\exists x\, (\, x \in \{a,b\} \wedge \tau \in x\, )
				\end{align}
				が成り立つから,ここで
				\begin{align}
					\chi \defeq \varepsilon x\, (\, x \in \{a,b\} \wedge \tau \in x\, )
				\end{align}
				とおけば
				\begin{align}
					\tau \in \bigcup \{a,b\},\ \EXTAX,\EQAX,\COMAX,\ELEAX &\vdash \chi \in \{a,b\}, 
					\label{fom:union_of_two_sets_is_union_of_pair_1} \\
					\tau \in \bigcup \{a,b\},\ \EXTAX,\EQAX,\COMAX,\ELEAX &\vdash \tau \in \chi
					\label{fom:union_of_two_sets_is_union_of_pair_2}
				\end{align}
				が成り立つ.定理\ref{thm:pair_members_are_exactly_the_given_two}
				(対は表示されている要素しか持たない)より
				\begin{align}
					\EXTAX,\EQAX,\COMAX \vdash \chi \in \{a,b\} \rarrow a = \chi \vee b = \chi
				\end{align}
				が成り立ち,(\refeq{fom:union_of_two_sets_is_union_of_pair_1})との三段論法より
				\begin{align}
					\tau \in \bigcup \{a,b\},\ \EXTAX,\EQAX,\COMAX,\ELEAX \vdash a = \chi \vee b = \chi
					\label{fom:union_of_two_sets_is_union_of_pair_3}
				\end{align}
				が従う.他方で,相等性公理より
				\begin{align}
					a = \chi,\ \tau \in \chi,\ \EQAX \vdash \tau \in a, \\
					b = \chi,\ \tau \in \chi,\ \EQAX \vdash \tau \in b
				\end{align}
				が成り立つので,論理和の導入より
				\begin{align}
					a = \chi,\ \tau \in \chi,\ \EQAX \vdash \tau \in a \vee \tau \in b, \\
					b = \chi,\ \tau \in \chi,\ \EQAX \vdash \tau \in a \vee \tau \in b
				\end{align}
				となり,演繹定理と論理和の除去より
				\begin{align}
					\tau \in \chi,\ \EQAX \vdash a = \chi \vee b = \chi \rarrow \tau \in a \vee \tau \in b
				\end{align}
				が成り立つ.これと(\refeq{fom:union_of_two_sets_is_union_of_pair_3})との三段論法より
				\begin{align}
					\tau \in \chi,\ \tau \in \bigcup \{a,b\},\ \EXTAX,\EQAX,\COMAX,\ELEAX 
					\vdash \tau \in a \vee \tau \in b
				\end{align}
				が成り立ち,演繹定理より
				\begin{align}
					\tau \in \bigcup \{a,b\},\ \EXTAX,\EQAX,\COMAX,\ELEAX 
					\vdash \tau\ \in \chi \rarrow \tau \in a \vee \tau \in b
				\end{align}
				が従い,(\refeq{fom:union_of_two_sets_is_union_of_pair_2})との三段論法より
				\begin{align}
					\tau \in \bigcup \{a,b\},\ \EXTAX,\EQAX,\COMAX,\ELEAX 
					\vdash \tau \in a \vee \tau \in b
				\end{align}
				となる.定理\ref{thm:elements_of_pair_union}より
				\begin{align}
					\EXTAX,\EQAX,\COMAX,\ELEAX \vdash \tau \in a \vee \tau \in b \rarrow \tau \in a \cup b
				\end{align}
				が成り立つので,三段論法と演繹定理より
				\begin{align}
					\EXTAX,\EQAX,\COMAX,\ELEAX \vdash \tau \in \bigcup \{a,b\} \rarrow \tau \in a \cup b
					\label{fom:union_of_two_sets_is_union_of_pair_4}
				\end{align}
				が出る.
				
			\item[step3] (\refeq{fom:union_of_two_sets_is_union_of_pair_4})と
				(\refeq{fom:union_of_two_sets_is_union_of_pair_5})と論理積の導入より
				\begin{align}
					\set{a},\ \set{b},\ \EXTAX,\EQAX,\COMAX,\ELEAX \vdash 
					\tau \in a \cup b \lrarrow \tau \in \bigcup \{a,b\}
				\end{align}
				が成り立ち,全称の導出(論理的定理\ref{logicalthm:derivation_of_universal_by_epsilon})より
				\begin{align}
					\set{a},\ \set{b},\ \EXTAX,\EQAX,\COMAX,\ELEAX \vdash 
					\forall x\, (\, x \in a \cup b \lrarrow x \in \bigcup \{a,b\}\, )
				\end{align}
				となり,外延性公理より
				\begin{align}
					\set{a},\ \set{b},\ \EXTAX,\EQAX,\COMAX,\ELEAX \vdash a \cup b = \bigcup \{a,b\}
				\end{align}
				が得られる.
				\QED
		\end{description}
	\end{sketch}
	
	\begin{screen}
		\begin{thm}[二つの集合の合併はそれぞれの要素を合わせたもの]
		\label{thm:union_of_pair_is_union_of_their_elements}
			$a$と$b$をクラスとするとき
			\begin{align}
				\EXTAX,\EQAX,\COMAX &\vdash 
				\forall x\, (\, x \in a \cup b \rarrow x \in a \vee x \in b\, ), \\
				\EXTAX,\EQAX,\COMAX &\vdash 
				\set{a} \rarrow \forall x\, (\, x \in a \rarrow x \in a \cup b\, ).
			\end{align}
		\end{thm}
	\end{screen}
	
	定理の二つ目の主張で$a$と$b$を入れ替えれば
	\begin{align}
		\set{b},\ \EXTAX,\EQAX,\COMAX \vdash \forall x\, (\, x \in b \rarrow x \in b \cup a\, )
	\end{align}
	が成り立つが,対の合併の対称性(定理\ref{thm:symmetry_of_union_of_a_pair})より
	\begin{align}
		\EXTAX,\EQAX,\COMAX \vdash b \cup a = a \cup b
	\end{align}
	が成り立つので
	\begin{align}
		\set{b},\ \EXTAX,\EQAX,\COMAX \vdash \forall x\, (\, x \in b \rarrow x \in a \cup b\, )
	\end{align}
	が従う.ゆえに
	\begin{align}
		\set{a},\ \set{b},\ \EXTAX,\EQAX,\COMAX \vdash 
		\forall x\, (\, x \in a \vee x \in b \rarrow x \in a \cup b\, )
	\end{align}
	が成り立つ.そして一つ目の主張と併せれば
	\begin{align}
		\set{a},\ \set{b},\ \EXTAX,\EQAX,\COMAX \vdash 
		\forall x\, (\, x \in a \cup b \lrarrow x \in a \vee x \in b\, )
	\end{align}
	が得られる.つまり,{\bf ``二つの集合の合併は''それぞれの要素を合わせたものに等しいのである}.
	
	\begin{sketch}\mbox{}
		\begin{description}
			\item[step1]
				いま
				\begin{align}
					\tau \defeq \varepsilon x \negation 
					(\, x \in a \cup b \rarrow x \in a \vee x \in b\, )
				\end{align}
				とおくと,定理\ref{thm:comprehension_of_unions}より
				\begin{align}
					\tau \in a \cup b,\ \COMAX \vdash
					\exists z\, (\, z \in \{a,b\} \wedge \tau \in z\, )
				\end{align}
				が成り立つので,
				\begin{align}
					\zeta \defeq \varepsilon z\, 
					(\, z \in \{a,b\} \wedge \tau \in z\, )
				\end{align}
				とおけば
				\begin{align}
					\tau \in a \cup b,\ \COMAX &\vdash \zeta \in \{a,b\}, 
					\label{fom:union_of_pair_is_union_of_their_elements_1} \\
					\tau \in a \cup b,\ \COMAX &\vdash \tau \in \zeta
					\label{fom:union_of_pair_is_union_of_their_elements_2}
				\end{align}
				が成り立つ.(\refeq{fom:union_of_pair_is_union_of_their_elements_2})と
				\begin{align}
					a = \zeta,\ \EQAX &\vdash \zeta = a, \\
					a = \zeta,\ \EQAX &\vdash \zeta = a \rarrow
					(\, \tau \in \zeta \rarrow \tau \in a\, )
				\end{align}
				より
				\begin{align}
					\tau \in a \cup b,\ \EQAX,\COMAX \vdash
					a = \zeta \rarrow \tau \in a 
				\end{align}
				が従い,
				\begin{align}
					\tau \in a \cup b,\ \EQAX,\COMAX \vdash
					a = \zeta \rarrow \tau \in a \vee \tau \in b
				\end{align}
				が成り立つ.同様に
				\begin{align}
					\tau \in a \cup b,\ \EQAX,\COMAX \vdash
					b = \zeta \rarrow \tau \in a \vee \tau \in b
				\end{align}
				が成り立ち,論理和の除去より
				\begin{align}
					\tau \in a \cup b,\ \EQAX,\COMAX \vdash
					a = \zeta \vee b = \zeta \rarrow \tau \in a \vee \tau \in b
					\label{fom:union_of_pair_is_union_of_their_elements_3}
				\end{align}
				が成り立つ.他方で定理\ref{thm:pair_members_are_exactly_the_given_two}より
				\begin{align}
					\EXTAX,\EQAX,\COMAX \vdash
					\zeta \in \{a,b\} \rarrow a = \zeta \vee b = \zeta
				\end{align}
				が成り立つので,(\refeq{fom:union_of_pair_is_union_of_their_elements_1})
				と併せて
				\begin{align}
					\tau \in a \cup b,\ \EXTAX,\EQAX,\COMAX &\vdash 
					a = \zeta \vee b = \zeta
				\end{align}
				が成り立つ.よって(\refeq{fom:union_of_pair_is_union_of_their_elements_3})
				と併せて
				\begin{align}
					\tau \in a \cup b,\ \EXTAX,\EQAX,\COMAX \vdash 
					\tau \in a \vee \tau \in b
				\end{align}
				が従い,全称の導出(論理的定理\ref{logicalthm:derivation_of_universal_by_epsilon})より
				\begin{align}
					\EXTAX,\EQAX,\COMAX \vdash 
					\forall x\, (\, x \in a \cup b \rarrow x \in a \vee x \in b\, )
				\end{align}
				が得られる.
				
			\item[step2]
				いま
				\begin{align}
					\chi &\defeq \varepsilon x \negation (\, x \in a \rarrow x \in a \cup b\, ), \\
					\tau &\defeq \varepsilon s\, (\, a = s\, )
				\end{align}
				とおく(右辺は$\lang{\varepsilon}$の式に書き換える).まず存在記号の論理的公理より
				\begin{align}
					\set{a} \vdash a = \tau
				\end{align}
				が成り立つ.また集合は対の要素になれる
				(定理\ref{thm:set_is_an_element_of_its_pair})ので
				\begin{align}
					\set{a},\ \EXTAX,\EQAX,\COMAX \vdash a \in \{a,b\}
				\end{align}
				も成り立ち,相等性公理より
				\begin{align}
					\set{a},\ \EXTAX,\EQAX,\COMAX \vdash \tau \in \{a,b\}
				\end{align}
				が従う.同じく相等性公理より
				\begin{align}
					\chi \in a,\ \set{a},\ \EXTAX,\EQAX,\COMAX \vdash \chi \in \tau
				\end{align}
				も成立する.ゆえに
				\begin{align}
					\chi \in a,\ \set{a},\ \EXTAX,\EQAX,\COMAX \vdash 
					\tau \in \{a,b\} \wedge \chi \in \tau
				\end{align}
				が成立し,存在記号の論理的公理より
				\begin{align}
					\chi \in a,\ \set{a},\ \EXTAX,\EQAX,\COMAX \vdash 
					\exists z\, (\, z \in \{a,b\} \wedge \chi \in z\, )
				\end{align}
				が従う.合併の内包性(定理\ref{thm:comprehension_of_unions})より
				\begin{align}
					\chi \in a,\ \set{a},\ \EXTAX,\EQAX,\COMAX \vdash 
					\chi \in a \cup b
				\end{align}
				となり,演繹定理より
				\begin{align}
					\set{a},\ \EXTAX,\EQAX,\COMAX \vdash 
					\chi \in a \rarrow \chi \in a \cup b
				\end{align}
				となり,全称の導出(論理的定理\ref{logicalthm:derivation_of_universal_by_epsilon})より
				\begin{align}
					\set{a},\ \EXTAX,\EQAX,\COMAX \vdash 
					\forall x\, (\, x \in a \rarrow x \in a \cup b\, )
				\end{align}
				が得られる.
				\QED
		\end{description}
	\end{sketch}
	
	\begin{itembox}[l]{三つ組の``対''}
		$\mathcal{L}$の項$x,y,z$に対して
		\begin{align}
			\{x,y,z\} \defeq \{x,y\} \cup \{z\}
		\end{align}
		と定める.
	\end{itembox}
	
	\begin{screen}
		\begin{thm}[三つ組も表示されている要素しか持たない]
		\label{thm:triple_members_are_exactly_the_given_three}
			$a,b,c$をクラスとするとき
			\begin{align}
				\EXTAX,\EQAX,\COMAX \vdash 
				\forall x\, (\, x \in \{a,b,c\} \lrarrow 
				(\, a = x \vee b = x\, ) \vee c = x\, ).
			\end{align}
		\end{thm}
	\end{screen}
	
	\begin{sketch}
		いま
		\begin{align}
			\tau \defeq \varepsilon x \negation (\, x \in \{a,b,c\} \lrarrow 
			(\, a = x \vee b = x\, ) \vee c = x\, )
		\end{align}
		とおく.後は
		\begin{align}
			\EXTAX,\EQAX,\COMAX \vdash \tau \in \{a,b,c\} \lrarrow 
			(\, a = \tau \vee b = \tau\, ) \vee c = \tau
		\end{align}
		を示せば,全称の導出
		(論理的定理\ref{logicalthm:derivation_of_universal_by_epsilon})より
		\begin{align}
			\EXTAX,\EQAX,\COMAX \vdash 
			\forall x\, (\, x \in \{a,b,c\} \lrarrow 
			(\, a = x \vee b = x\, ) \vee c = x\, )
		\end{align}
		を得る.
		\begin{description}
			\item[step1]
				この段では
				\begin{align}
					\EXTAX,\EQAX,\COMAX \vdash \tau \in \{a,b,c\} \rarrow 
					(\, a = \tau \vee b = \tau\, ) \vee c = \tau
				\end{align}
				を示す.定理\ref{thm:union_of_pair_is_union_of_their_elements}より
				\begin{align}
					\tau \in \{a,b,c\},\ \EXTAX,\EQAX,\COMAX \vdash 
					\tau \in \{a,b\} \vee \tau \in \{c\}
				\end{align}
				が成り立つが,定理\ref{thm:pair_members_are_exactly_the_given_two}
				(対は表示されている要素しか持たない)より
				\begin{align}
					\EXTAX,\EQAX,\COMAX \vdash \tau \in \{a,b\} \rarrow 
					a = \tau \vee b = \tau
				\end{align}
				および
				\begin{align}
					\EXTAX,\EQAX,\COMAX \vdash \tau \in \{c\} \rarrow c = \tau
				\end{align}
				となり,論理和の導入より
				\begin{align}
					\EXTAX,\EQAX,\COMAX &\vdash \tau \in \{a,b\} \rarrow 
					(\, a = \tau \vee b = \tau\, ) \vee c = \tau, \\
					\EXTAX,\EQAX,\COMAX &\vdash \tau \in \{c\} \rarrow 
					(\, a = \tau \vee b = \tau\, ) \vee c = \tau
				\end{align}
				となり,論理和の除去より
				\begin{align}
					\tau \in \{a,b,c\},\ \EXTAX,\EQAX,\COMAX \vdash 
					(\, a = \tau \vee b = \tau\, ) \vee c = \tau
				\end{align}
				が従う.
				
			\item[step2]
				この段では
				\begin{align}
					\EXTAX,\EQAX,\COMAX \vdash 
					(\, a = \tau \vee b = \tau\, ) \vee c = \tau \rarrow 
					\tau \in \{a,b,c\}
				\end{align}
				を示す.定理\ref{thm:pair_members_are_exactly_the_given_two}
				(対は表示されている要素しか持たない)より
				\begin{align}
					\EXTAX,\EQAX,\COMAX \vdash a = \tau \vee b = \tau 
					\rarrow \tau \in \{a,b\}
				\end{align}
				および
				\begin{align}
					\EXTAX,\EQAX,\COMAX \vdash c = \tau
					\rarrow \tau \in \{c\}
				\end{align}
				となり,論理和の導入より
				\begin{align}
					\EXTAX,\EQAX,\COMAX &\vdash a = \tau \vee b = \tau 
					\rarrow \tau \in \{a,b\} \vee \tau \in \{c\}, \\
					\EXTAX,\EQAX,\COMAX &\vdash c = \tau 
					\rarrow \tau \in \{a,b\} \vee \tau \in \{c\}
				\end{align}
				となり,論理和の除去と演繹定理の逆より
				\begin{align}
					(\, a = \tau \vee b = \tau\, ) \vee c = \tau,\ 
					\EXTAX,\EQAX,\COMAX \vdash \tau \in \{a,b\} \vee \tau \in \{c\}
				\end{align}
				が従う.定理\ref{thm:union_of_pair_is_union_of_their_elements}より
				\begin{align}
					\EXTAX,\EQAX,\COMAX \vdash 
					\tau \in \{a,b\} \vee \tau \in \{c\} 
					\rarrow \tau \in \{a,b,c\}
				\end{align}
				が成り立つので
				\begin{align}
					(\, a = \tau \vee b = \tau\, ) \vee c = \tau,\ 
					\EXTAX,\EQAX,\COMAX \vdash \tau \in \{a,b,c\}
				\end{align}
				が得られる.
				\QED
		\end{description}
	\end{sketch}
	
	\begin{comment}
	\begin{screen}
		\begin{thm}[要素の部分集合は合併の部分集合]
		\label{thm:union_is_bigger_than_any_member}
			$a$をクラスとするとき
			\begin{align}
				\forall x\, \left[\, \exists t \in a\, (\, x \subset t\, ) \rarrow x \subset \bigcup a\, \right].
			\end{align}
		\end{thm}
	\end{screen}
	
	\begin{sketch}
		$\chi$を$\mathcal{L}$の任意の対象として
		\begin{align}
			\exists t \in a\, (\, x \subset t\, )
			\label{fom:thm_union_is_bigger_than_any_member_1}
		\end{align}
		であるとする.ここで
		\begin{align}
			\tau \defeq \varepsilon t\, (\, t \in a \wedge \chi \subset t\, )
		\end{align}
		とおく.$s$を$\mathcal{L}$の任意の対象として
		\begin{align}
			s \in \chi
		\end{align}
		であるとすると,
		\begin{align}
			\chi \subset \tau
		\end{align}
		より
		\begin{align}
			\tau \in a \wedge s \in \tau
		\end{align}
		が成立するので,存在記号の論理的公理より
		\begin{align}
			\exists t\, \left(\, t \in a \wedge s \in t\, \right)
		\end{align}
		が成り立ち
		\begin{align}
			s \in \bigcup a
		\end{align}
		が従う.$s$は任意に与えられていたので,(\refeq{fom:thm_union_is_bigger_than_any_member_1})の下で
		\begin{align}
			\forall s\, (\, s \in \chi \rarrow s \in \bigcup a\, )
		\end{align}
		すなわち
		\begin{align}
			\chi \subset \bigcup a
		\end{align}
		が成り立つ.ゆえに
		\begin{align}
			\exists t \in a\, \left(\, \chi \subset t\, \right) \rarrow \chi \subset \bigcup a
		\end{align}
		が従い,$\chi$も任意に与えられていたので
		\begin{align}
			\forall x\, \left[\, \exists t \in a\, (\, x \subset t\, ) \rarrow x \subset \bigcup a\, \right]
		\end{align}
		が得られる.
		\QED
	\end{sketch}
	
	\begin{screen}
		\begin{thm}[部分集合の合併は部分クラス]\label{thm:union_of_subsets_is_subclass}
			$a$と$b$をクラスとするとき
			\begin{align}
				\forall x \in a\, (\, x \subset b\, ) \rarrow \bigcup a \subset b.
			\end{align}
		\end{thm}
	\end{screen}
	
	\begin{sketch}
		いま
		\begin{align}
			\forall x \in a\, (\, x \subset b\, )
			\label{fom:thm_union_of_subsets_is_subclass_1}
		\end{align}
		が成り立っているとする.$\chi$を$\mathcal{L}$の任意の対象とし,
		\begin{align}
			\chi \in \bigcup a
		\end{align}
		であるとする.すると
		\begin{align}
			\exists t\, \left(\, t \in a \wedge \chi \in t\, \right)
		\end{align}
		が成り立つので,
		\begin{align}
			\tau \defeq \varepsilon t\, \left(\, t \in a \wedge \chi \in t\, \right)
		\end{align}
		とおけば
		\begin{align}
			\tau \in a \wedge \chi \in \tau
		\end{align}
		が成立する.ここで(\refeq{fom:thm_union_of_subsets_is_subclass_1})より
		\begin{align}
			\tau \subset b
		\end{align}
		となるから
		\begin{align}
			\chi \in b
		\end{align}
		が従い,演繹定理より(\refeq{fom:thm_union_of_subsets_is_subclass_1})の下で
		\begin{align}
			\chi \in \bigcup a \rarrow \chi \in b
		\end{align}
		が成立する.$\chi$の任意性ゆえに(\refeq{fom:thm_union_of_subsets_is_subclass_1})の下で
		\begin{align}
			\bigcup a \subset b
		\end{align}
		が成立し,演繹定理より
		\begin{align}
			\forall x \in a\, (\, x \subset b\, ) \rarrow \bigcup a \subset b
		\end{align}
		が得られる.
		\QED
	\end{sketch}
	\end{comment}
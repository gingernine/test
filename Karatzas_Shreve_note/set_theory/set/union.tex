\section{合併}
	$a$を空でない類とするとするとき,$a$の要素もまた空でなければ要素を持つ.
	$a$の要素の要素を全て集めたものを$a$の合併と呼び,その受け皿の意味を込めて
	\begin{align}
		\bigcup a
	\end{align}
	と書く.当然ながら,空の合併は空となる.
	
	\begin{screen}
		\begin{dfn}[合併]
			$x$を$\mathcal{L}$の項とするとき,
			$x$の{\bf 合併}\index{がっぺい@合併}{\bf (union)}を
			\begin{align}
				\bigcup x \defeq \Set{y}{\exists z \in x\, (\, y \in z\, )}
				\label{eq:definition_of_union_1}
			\end{align}
			で定める.
		\end{dfn}
	\end{screen}
	
	\begin{description}
		\item[量化子が付いた式の略記法]
		上の定義で
		\begin{align}
			\exists z \in x\, (\, y \in z\, )
		\end{align}
		という式を書いたが,これは
		\begin{align}
			\exists z \in x\, (\, y \in z\, ) \defarrow 
			\exists z\, (\, z \in x \wedge y \in z\, )
		\end{align}
		により定義される省略形である.同様にして,$\varphi$を式とするとき
		\begin{align}
			\exists z\, \left(\, z \in x \wedge \varphi\, \right)
		\end{align}
		なる式を
		\begin{align}
			\exists z \in x\, \varphi
		\end{align}
		と略記する.また全称記号についても
		\begin{align}
			\forall z\, \left(\, z \in x \rarrow \varphi\, \right)
		\end{align}
		なる式を
		\begin{align}
			\forall z \in x\, \varphi
		\end{align}
		と略記する.
	\end{description}
	
	\begin{screen}
		\begin{axm}[合併の公理]
			次の式を$\POWAX$によって参照する:
			\begin{align}
				\forall x\, \exists u\, \forall y\, (\, \exists z\, (\, z \in x \wedge y \in z\, ) \lrarrow y \in u\, ).
			\end{align}
		\end{axm}
	\end{screen}
	
	\begin{screen}
		\begin{thm}[集合の合併は集合]
		\label{thm:union_of_a_set_is_a_set}
			$a$を類とするとき
			\begin{align}
				\EXTAX,\EQAX,\COMAX,\POWAX \vdash \set{a} \rarrow \set{\bigcup a}.
			\end{align}
		\end{thm}
	\end{screen}
	
	\begin{sketch}\mbox{}
		\begin{description}
			\item[step1]
				まず
				\begin{align}
					\tau \defeq \varepsilon x\, (\, a = x\, )
				\end{align}
				とおけば(必要に応じて$a = x$を$\lang{\varepsilon}$の式に書き換える),
				\begin{align}
					\set{a} \vdash a = \tau
				\end{align}
				が成立する.$\tau$に対して
				\begin{align}
					\POWAX \vdash \exists u\, \forall y\, (\, \exists z\, (\, z \in \tau \wedge y \in z\, ) \lrarrow y \in u\, )
				\end{align}
				が成り立つので,
				\begin{align}
					\upsilon \defeq \varepsilon u\, \forall y\, (\, \exists z\, (\, z \in \tau \wedge y \in z\, ) \lrarrow y \in u\, )
				\end{align}
				とおけば
				\begin{align}
					\POWAX \vdash \forall y\, (\, \exists z\, (\, z \in \tau \wedge y \in z\, ) \lrarrow y \in \upsilon\, )
					\label{fom:union_of_a_set_is_a_set_1}
				\end{align}
				が成立する.次に$\tau$を$a$に置き換えた場合に
				\begin{align}
					\set{a},\ \EQAX,\POWAX \vdash \forall y\, (\, \exists z\, (\, z \in a \wedge y \in z\, ) \lrarrow y \in \upsilon\, )
				\end{align}
				が成立することを示す.
				
			\item[step2]
				いま
				\begin{align}
					\eta \defeq \varepsilon y \negation  (\, \exists z\, (\, z \in a \wedge y \in z\, ) \lrarrow y \in \upsilon\, )
				\end{align}
				とおけば,(\refeq{fom:union_of_a_set_is_a_set_1})より
				\begin{align}
					\POWAX \vdash \exists z\, (\, z \in \tau \wedge \eta \in z\, )
					\lrarrow \eta \in \upsilon
				\end{align}
				が成立する.
				\begin{align}
					\zeta \defeq \varepsilon z\, (\, z \in a \wedge \eta \in z\, )
				\end{align}
				とおけば
				\begin{align}
					\exists z\, (\, z \in a \wedge \eta \in z\, )
					\vdash \zeta \in a \wedge \eta \in \zeta
				\end{align}
				が成り立ち,
				\begin{align}
					\EQAX \vdash a = \tau \rarrow (\, \zeta \in a \rarrow \zeta \in \tau\, )
				\end{align}
				と併せて
				\begin{align}
					\exists z\, (\, z \in a \wedge \eta \in z\, ),\ \set{a},\ \EQAX
					\vdash \zeta \in \tau \wedge \eta \in \zeta
				\end{align}
				が成立する.また(\refeq{fom:union_of_a_set_is_a_set_1})より
				\begin{align}
					\POWAX \vdash (\, \zeta \in \tau \wedge \eta \in \zeta\, )
					\rarrow \eta \in \upsilon
				\end{align}
				が成り立つので
				\begin{align}
					\exists z\, (\, z \in a \wedge \eta \in z\, ),\ \set{a},\ \EQAX,\POWAX \vdash \eta \in \upsilon
				\end{align}
				が従う.ゆえに
				\begin{align}
					\set{a},\ \EQAX,\POWAX \vdash 
					\exists z\, (\, z \in a \wedge \eta \in z\, ) \rarrow \eta \in \upsilon
					\label{fom:union_of_a_set_is_a_set_2}
				\end{align}
				が得られた.
				
			\item[step3]
				逆に(\refeq{fom:union_of_a_set_is_a_set_1})より
				\begin{align}
					\eta \in \upsilon,\ \POWAX \vdash
					\exists z\, (\, z \in \tau \wedge \eta \in z\, )
				\end{align}
				が成り立つので
				\begin{align}
					\eta \in \upsilon,\ \POWAX \vdash
					\zeta \in \tau \wedge \eta \in \zeta
				\end{align}
				が従い,
				\begin{align}
					\set{a},\ \EQAX \vdash
					\zeta \in \tau \rarrow \zeta \in a
				\end{align}
				と併せて
				\begin{align}
					\eta \in \upsilon,\ \set{a},\ \EQAX,\POWAX \vdash
					\zeta \in a \wedge \eta \in \zeta
				\end{align}
				が従い,
				\begin{align}
					\eta \in \upsilon,\ \set{a},\ \EQAX,\POWAX \vdash
					\exists z\, (\, z \in a \wedge \eta \in z\, )
				\end{align}
				が従う.そして演繹規則より
				\begin{align}
					\set{a},\ \EQAX,\POWAX \vdash
					\eta \in \upsilon \rarrow \exists z\, (\, z \in a \wedge \eta \in z\, )
					\label{fom:union_of_a_set_is_a_set_3}
				\end{align}
				も得られる.
				
			\item[step4]
				(\refeq{fom:union_of_a_set_is_a_set_2})と
				(\refeq{fom:union_of_a_set_is_a_set_3})より
				\begin{align}
					\set{a},\ \EQAX,\POWAX \vdash
					\exists z\, (\, z \in a \wedge \eta \in z\, ) \lrarrow \eta \in \upsilon
				\end{align}
				が得られ,全称記号の推論規則より
				\begin{align}
					\set{a},\ \EQAX,\POWAX \vdash
					\forall y\, (\, \exists z\, (\, z \in a \wedge y \in z\, ) \lrarrow y \in \upsilon\, )
				\end{align}
				となり,定理\ref{thm:equivalent_formula_rewriting_4}より
				\begin{align}
					\set{a},\ \EXTAX,\EQAX,\COMAX,\POWAX \vdash
					\Set{z}{\exists z\, (\, z \in a \wedge y \in z\, )} = \upsilon
				\end{align}
				が成り立つ.存在記号の推論規則より
				\begin{align}
					\set{a},\ \EXTAX,\EQAX,\COMAX,\POWAX \vdash
					\exists u\, (\, \Set{z}{\exists z\, (\, z \in a \wedge y \in z\, )} = u\, )
				\end{align}
				が成り立つので,定理が得られた.
				\QED
		\end{description}
	\end{sketch}
	
	\begin{screen}
		\begin{thm}[空集合の合併は空]\label{thm:the_union_of_the_emptyset_is_empty}
			次が成立する:
			\begin{align}
				\bigcup \emptyset = \emptyset.
			\end{align}
		\end{thm}
	\end{screen}
	
	\begin{sketch}
		いま
		\begin{align}
			\zeta &\defeq \varepsilon z \negation (\, z \notin \bigcup \emptyset\, ), \\
			\eta &\defeq \varepsilon y \negation \negation (\, y \in \emptyset \wedge \zeta \in y\, )
		\end{align}
		とおく.定理\ref{thm:emptyset_has_nothing}より
		\begin{align}
			\EXTAX,\COMAX \vdash \eta \notin \emptyset
		\end{align}
		が成り立つので
		\begin{align}
			\EXTAX,\COMAX \vdash \eta \notin \emptyset \vee \zeta \notin \eta
		\end{align}
		も成立し,De Morgan の法則(推論法則\ref{logicalthm:strong_De_Morgan_law_1})より
		\begin{align}
			\EXTAX,\COMAX \vdash\ \negation (\, \eta \in \emptyset \wedge \zeta \in \eta\, )
		\end{align}
		が成立し,全称記号の推論規則より
		\begin{align}
			\EXTAX,\COMAX \vdash \forall y \negation (\, y \in \emptyset \wedge \zeta \in y\, )
		\end{align}
		が成立する.そして量化子の De Morgan の法則
		(推論法則\ref{logicalthm:strong_De_Morgan_law_for_quantifiers_1})より
		\begin{align}
			\EXTAX,\COMAX \vdash\ \negation \exists y\, (\, y \in \emptyset \wedge \zeta \in y\, )
			\label{fom:the_union_of_the_emptyset_is_empty_1}
		\end{align}
		が得られる.他方で
		\begin{align}
			\COMAX \vdash \zeta \in \bigcup \emptyset
			\rarrow \exists y\, (\, y \in \emptyset \wedge \zeta \in y\, )
		\end{align}
		が成り立つので,対偶を取って
		\begin{align}
			\COMAX \vdash\ 
			\negation \exists y\, (\, y \in \emptyset \wedge \zeta \in y\, )
			\rarrow \zeta \notin \bigcup \emptyset
			\label{fom:the_union_of_the_emptyset_is_empty_2}
		\end{align}
		が得られる.(\refeq{fom:the_union_of_the_emptyset_is_empty_1})と
		(\refeq{fom:the_union_of_the_emptyset_is_empty_2})より
		\begin{align}
			\EXTAX,\COMAX \vdash \zeta \notin \bigcup \emptyset
		\end{align}
		が成り立つので,全称記号の推論規則より
		\begin{align}
			\EXTAX,\COMAX \vdash \forall z\, (\, z \notin \bigcup \emptyset\, )
		\end{align}
		が従い,定理\ref{thm:uniqueness_of_emptyset}より
		\begin{align}
			\EXTAX,\COMAX \vdash \bigcup \emptyset = \emptyset
		\end{align}
		が得られる.
		\QED
	\end{sketch}
	
	\begin{screen}
		\begin{thm}[要素の部分集合は合併の部分集合]
		\label{thm:union_is_bigger_than_any_member}
			$a$を類とするとき
			\begin{align}
				\forall x\, \left[\, \exists t \in a\, (\, x \subset t\, ) \rarrow x \subset \bigcup a\, \right].
			\end{align}
		\end{thm}
	\end{screen}
	
	\begin{sketch}
		$\chi$を$\mathcal{L}$の任意の対象として
		\begin{align}
			\exists t \in a\, (\, x \subset t\, )
			\label{fom:thm_union_is_bigger_than_any_member_1}
		\end{align}
		であるとする.ここで
		\begin{align}
			\tau \defeq \varepsilon t\, (\, t \in a \wedge \chi \subset t\, )
		\end{align}
		とおく.$s$を$\mathcal{L}$の任意の対象として
		\begin{align}
			s \in \chi
		\end{align}
		であるとすると,
		\begin{align}
			\chi \subset \tau
		\end{align}
		より
		\begin{align}
			\tau \in a \wedge s \in \tau
		\end{align}
		が成立するので,存在記号の規則より
		\begin{align}
			\exists t\, \left(\, t \in a \wedge s \in t\, \right)
		\end{align}
		が成り立ち
		\begin{align}
			s \in \bigcup a
		\end{align}
		が従う.$s$は任意に与えられていたので,(\refeq{fom:thm_union_is_bigger_than_any_member_1})の下で
		\begin{align}
			\forall s\, (\, s \in \chi \rarrow s \in \bigcup a\, )
		\end{align}
		すなわち
		\begin{align}
			\chi \subset \bigcup a
		\end{align}
		が成り立つ.ゆえに
		\begin{align}
			\exists t \in a\, \left(\, \chi \subset t\, \right) \rarrow \chi \subset \bigcup a
		\end{align}
		が従い,$\chi$も任意に与えられていたので
		\begin{align}
			\forall x\, \left[\, \exists t \in a\, (\, x \subset t\, ) \rarrow x \subset \bigcup a\, \right]
		\end{align}
		が得られる.
		\QED
	\end{sketch}
	
	\begin{screen}
		\begin{thm}[部分集合の合併は部分類]\label{thm:union_of_subsets_is_subclass}
			$a$と$b$を類とするとき
			\begin{align}
				\forall x \in a\, (\, x \subset b\, ) \rarrow \bigcup a \subset b.
			\end{align}
		\end{thm}
	\end{screen}
	
	\begin{sketch}
		いま
		\begin{align}
			\forall x \in a\, (\, x \subset b\, )
			\label{fom:thm_union_of_subsets_is_subclass_1}
		\end{align}
		が成り立っているとする.$\chi$を$\mathcal{L}$の任意の対象とし,
		\begin{align}
			\chi \in \bigcup a
		\end{align}
		であるとする.すると
		\begin{align}
			\exists t\, \left(\, t \in a \wedge \chi \in t\, \right)
		\end{align}
		が成り立つので,
		\begin{align}
			\tau \defeq \varepsilon t\, \left(\, t \in a \wedge \chi \in t\, \right)
		\end{align}
		とおけば
		\begin{align}
			\tau \in a \wedge \chi \in \tau
		\end{align}
		が成立する.ここで(\refeq{fom:thm_union_of_subsets_is_subclass_1})より
		\begin{align}
			\tau \subset b
		\end{align}
		となるから
		\begin{align}
			\chi \in b
		\end{align}
		が従い,演繹法則より(\refeq{fom:thm_union_of_subsets_is_subclass_1})の下で
		\begin{align}
			\chi \in \bigcup a \rarrow \chi \in b
		\end{align}
		が成立する.$\chi$の任意性ゆえに(\refeq{fom:thm_union_of_subsets_is_subclass_1})の下で
		\begin{align}
			\bigcup a \subset b
		\end{align}
		が成立し,演繹法則より
		\begin{align}
			\forall x \in a\, (\, x \subset b\, ) \rarrow \bigcup a \subset b
		\end{align}
		が得られる.
		\QED
	\end{sketch}
	
	\begin{itembox}[l]{対の合併}
		$a,b$を類とするとき,その対の合併を
		\begin{align}
			a \cup b \defeq \bigcup \{a,b\}
		\end{align}
		と書く.
	\end{itembox}
	
	\begin{screen}
		\begin{thm}[対の合併はそれぞれの要素を合わせたもの]\label{thm:union_of_pair_is_union_of_their_elements}
			$a$と$b$を集合とするとき
			\begin{align}
				\forall x\, (\, x \in a \cup b \lrarrow x \in a \vee x \in b\, ).
			\end{align}
		\end{thm}
	\end{screen}
	
	\monologue{
		この定理の主張は,$a$と$b$を類とするとき
		\begin{align}
			\set{a} \wedge \set{b} \rarrow
			\forall x\, (\, x \in a \cup b \lrarrow x \in a \vee x \in b\, )
		\end{align}
		が成り立つということですが,式にまとめてしまうと見づらいのではじめから$a$と$b$を集合としています.
	}
	
	\begin{sketch}
		$\chi$を$\mathcal{L}$の任意の対象とする.
		\begin{align}
			\chi \in a \cup b
		\end{align}
		であるとき,
		\begin{align}
			\exists t\, \left(\, t \in \{a,b\} \wedge \chi \in t\, \right)
		\end{align}
		が成り立つので,
		\begin{align}
			\tau \defeq \varepsilon t\, \left(\, t \in \{a,b\} \wedge \chi \in t\, \right)
		\end{align}
		とおけば
		\begin{align}
			\tau \in \{a,b\} \wedge \chi \in \tau
		\end{align}
		が成立する.
		\begin{align}
			\tau \in \{a,b\}
		\end{align}
		が成り立つので,定理\ref{thm:pair_members_are_exactly_the_given_two}より
		\begin{align}
			\tau = a \vee \tau = b
			\label{fom:thm_union_of_pair_is_union_of_their_elements_1}
		\end{align}
		が従う.ここで相等性の公理より
		\begin{align}
			\tau = a \rarrow \chi \in a
		\end{align}
		が成り立ち,論理和の規則から
		\begin{align}
			\tau = a \rarrow \chi \in a \vee \chi \in b
		\end{align}
		も成り立つ.同様にして
		\begin{align}
			\tau = b \rarrow \chi \in a \vee \chi \in b
		\end{align}
		が成り立つので,場合分け法則より
		\begin{align}
			\tau = a \vee \tau = b \rarrow \chi \in a \vee \chi \in b
		\end{align}
		が成立し,(\refeq{fom:thm_union_of_pair_is_union_of_their_elements_1})と三段論法より
		\begin{align}
			\chi \in a \vee \chi \in b
		\end{align}
		が成立する.ゆえに演繹法則から
		\begin{align}
			\chi \in a \cup b \rarrow \chi \in a \vee \chi \in b
			\label{fom:thm_union_of_pair_is_union_of_their_elements_2}
		\end{align}
		が成立する.逆に
		\begin{align}
			\chi \in a
		\end{align}
		であるとすると,
		\begin{align}
			\tau_a \defeq \varepsilon x\, (\, a = x\, )
		\end{align}
		とおけば定理\ref{thm:set_is_an_element_of_its_pair}より
		\begin{align}
			\tau_a \in \{a,b\} \wedge \chi \in \tau_a
		\end{align}
		が成り立つので,
		\begin{align}
			\exists t\, \left(\, t \in \{a,b\} \wedge \chi \in t\, \right)
		\end{align}
		が成り立ち
		\begin{align}
			\chi \in a \cup b
		\end{align}
		が従う.これでまず
		\begin{align}
			\chi \in a \rarrow \chi \in a \cup b
		\end{align}
		が得られた.同様にして
		\begin{align}
			\chi \in b \rarrow \chi \in a \cup b
		\end{align}
		も得られ,場合分け法則より
		\begin{align}
			\chi \in a \vee \chi \in b \rarrow \chi \in a \cup b
			\label{fom:thm_union_of_pair_is_union_of_their_elements_3}
		\end{align}
		が成立する.以上(\refeq{fom:thm_union_of_pair_is_union_of_their_elements_2})と
		(\refeq{fom:thm_union_of_pair_is_union_of_their_elements_3})から
		\begin{align}
			\chi \in a \cup b \lrarrow \chi \in a \vee \chi \in b
		\end{align}
		が従い,$\chi$の任意性より
		\begin{align}
			\forall x\, (\, x \in a \cup b \lrarrow x \in a \vee x \in b\, ).
		\end{align}
		が出る.
		\QED
	\end{sketch}
	
	\begin{screen}
		\begin{thm}[等しい類の合併は等しい]\label{thm:unions_of_equal_classes_are_equal}
			$a$と$b$を類とするとき
			\begin{align}
				a = b \rarrow \bigcup a = \bigcup b.
			\end{align}
		\end{thm}
	\end{screen}
	
	\begin{sketch}
		いま
		\begin{align}
			a = b
			\label{fom:thm_unions_of_equal_classes_are_equal}
		\end{align}
		が成り立っているとする.$\chi$を$\mathcal{L}$の任意の対象として
		\begin{align}
			\chi \in \bigcup a
		\end{align}
		であるとすれば,
		\begin{align}
			\tau \in a \wedge \chi \in \tau
		\end{align}
		なる$\mathcal{L}$の対象$\tau$が取れる.このとき相等性の公理より
		\begin{align}
			\tau \in b
		\end{align}
		が成り立つから
		\begin{align}
			\tau \in b \wedge \chi \in \tau
		\end{align}
		が従い,ゆえに
		\begin{align}
			\chi \in \bigcup b
		\end{align}
		が従う.ゆえに(\refeq{fom:thm_unions_of_equal_classes_are_equal})の下で
		\begin{align}
			\chi \in \bigcup a \rarrow \chi \in \bigcup b
		\end{align}
		が得られたが,$a$と$b$を入れ替えれば
		\begin{align}
			\chi \in \bigcup b \rarrow \chi \in \bigcup a
		\end{align}
		も得られるので
		\begin{align}
			\chi \in \bigcup a \lrarrow \chi \in \bigcup b
		\end{align}
		が成立する.そして$\chi$の任意性と外延性の公理から
		\begin{align}
			\bigcup a = \bigcup b
		\end{align}
		が成立する.ゆえに演繹法則から
		\begin{align}
			a = b \rarrow \bigcup a = \bigcup b
		\end{align}
		が従う.
		\QED
	\end{sketch}
	
	\begin{screen}
		\begin{thm}[合併の可換律]
			$a$と$b$を類とするとき
			\begin{align}
				a \cup b = b \cup a.
			\end{align}
		\end{thm}
	\end{screen}
	
	\begin{sketch}
		定理\ref{thm:commutative_law_of_pairs}より
		\begin{align}
			\{a,b\} = \{b,a\}
		\end{align}
		が成り立つので,定理\ref{thm:unions_of_equal_classes_are_equal}から
		\begin{align}
			a \cup b = b \cup a
		\end{align}
		が従う.
		\QED
	\end{sketch}
\section{空集合}
	\begin{screen}
		\begin{logicalthm}[分配された論理積の簡約]
		\label{logicalthm:contraction_law_of_distributed_injunctions}
			$A,B,C$を$\mathcal{L}$の文とするとき,
			\begin{align}
				\vdash (A \wedge C) \wedge (B \wedge C) \rarrow A \wedge B.
			\end{align}
		\end{logicalthm}
	\end{screen}
	
	\begin{sketch}
		論理積の除去規則より
		\begin{align}
			(A \wedge C) \wedge (B \wedge C) \vdash A \wedge C
		\end{align}
		となり,また同じく論理積の除去規則より
		\begin{align}
			(A \wedge C) \wedge (B \wedge C) &\vdash A \wedge C \rarrow A
		\end{align}
		となるので,三段論法より
		\begin{align}
			(A \wedge C) \wedge (B \wedge C) &\vdash A, 
			\label{fom:logicalthm_contraction_law_of_injunctions_1}
		\end{align}
		が従う.同様にして
		\begin{align}
			(A \wedge C) \wedge (B \wedge C) \vdash B
			\label{fom:logicalthm_contraction_law_of_injunctions_2}
		\end{align}
		も得られる.ここで論理積の導入規則より
		\begin{align}
			(A \wedge C) \wedge (B \wedge C) \vdash A \rarrow (B \rarrow A \wedge B)
		\end{align}
		が成り立つので,(\refeq{fom:logicalthm_contraction_law_of_injunctions_1})と
		(\refeq{fom:logicalthm_contraction_law_of_injunctions_2})との三段論法より
		\begin{align}
			(A \wedge C) \wedge (B \wedge C) \vdash A \wedge B
		\end{align}
		が出る.
		\QED
	\end{sketch}
	
	\begin{screen}
		\begin{dfn}[空集合]
			$\emptyset \defeq \Set{x}{x \neq x}$で定める類$\emptyset$を{\bf 空集合}\index{くうしゅうごう@空集合}{\bf (empty set)}と呼ぶ.
		\end{dfn}
	\end{screen}
	
	$x$が集合であれば
	\begin{align}
		x = x
	\end{align}
	が成り立つので,$\emptyset$に入る集合など存在しない.
	つまり$\emptyset$は丸っきり``空っぽ''なのである.
	さて,$\emptyset$は集合であるか否か,という問題を考える.
	当然これが``大きすぎる集まり''であるはずはないし,
	そもそも名前に``集合''と付いているのだから
	$\emptyset$は集合であるべきだと思われるのだが,
	実際にこれが集合であることを示すには少し骨が折れる.
	まずは置換公理と分出定理を拵えなくてはならない.
	
	\begin{screen}
		\begin{axm}[置換公理]
			$\varphi$を$\mathcal{L}$の式とし,
			$s,t$を$\varphi$に自由に現れる変項とし,
			$\varphi$に自由に現れる項は$s,t$のみであるとし,
			$x$は$\varphi$で$s$への代入について自由であり,
			$y,z,v$は$\varphi$で$t$への代入について自由であるとするとき,
			\begin{align}
				\REPAX \defarrow \forall x\, \forall y\, \forall z\, 
				(\, \varphi(x,y) \wedge \varphi(x,z)
				\rarrow y = z\, )
				\rarrow \forall a\, \exists u\, \forall v\,
				(\, v \in u \lrarrow \exists x\, (\, x \in a \wedge 
				\varphi(x,v)\, )\, ).
			\end{align}
		\end{axm}
	\end{screen}
	
	$\Set{x}{\varphi(x)}$は集合であるとは限らないが,
	集合$a$に対して
	\begin{align}
		a \cap \Set{x}{\varphi(x)}
	\end{align}
	なる類は当然$a$より``小さい集まり''なのだから,集合であってほしいものである.
	置換公理によってそのこと保証され,分出定理として知られている.
	
	\begin{screen}
		\begin{thm}[分出定理]\label{thm:axiom_of_separation}
			$\varphi$を$\mathcal{L}$の式とし,$x$を$\varphi$に自由に現れる変項とし,
			$\varphi$に自由に現れる項は$x$のみであるとする.このとき
			\begin{align}
				\EXTAX,\EQAX,\EQAXEP,\REPAX \vdash 
				\forall a\, \exists s\, \forall x\,
				(\, x \in s \lrarrow x \in a \wedge \varphi(x)\, ).
				\label{fom:thm_axiom_of_separation_0}
			\end{align}
		\end{thm}
	\end{screen}
	
	\begin{sketch}
		$y$を,$\varphi$の$x$への代入について自由である変項とする.
		そして$x$と$y$が自由に現れる式$\psi(x,y)$を
		\begin{align}
			x = y \wedge \varphi(x)
		\end{align}
		と設定する.
		\begin{description}
			\item[step1]
				まず
				\begin{align}
					\EXTAX,\EQAX \vdash \forall x\, \forall y\, \forall z\, 
					(\, \psi(x,y) \wedge \psi(x,z) \rarrow y = z\, )
					\label{fom:thm_axiom_of_separation_1}
				\end{align}
				が成り立つことを示す.これを見越して
				\begin{align}
					\tau &\defeq \varepsilon x \negation \forall y\, \forall z\, 
					(\, \psi(x,y) \wedge \psi(x,z) \rarrow y = z\, ), \\
					\sigma &\defeq \varepsilon y \negation \forall z\, 
					(\, \psi(\tau,y) \wedge \psi(\tau,z) \rarrow y = z\, ), \\
					\rho &\defeq \varepsilon z \negation 
					(\, \psi(\tau,\sigma) \wedge \psi(\tau,z) \rarrow \sigma = z\, )
				\end{align}
				とおく.$\psi(\tau,\sigma) \wedge \psi(\tau,\rho)$は縮約可能であって(
				推論法則\ref{logicalthm:contraction_law_of_distributed_injunctions})
				\begin{align}
					\vdash (\, \tau = \sigma \wedge \varphi(\tau)\, )
						\wedge (\, \tau = \rho \wedge \varphi(\tau)\, )
						\rarrow \tau = \sigma \wedge \tau = \rho
				\end{align}
				が成り立つので
				\begin{align}
					\psi(\tau,\sigma) \wedge \psi(\tau,\rho) 
					\vdash \tau = \sigma \wedge \tau = \rho
				\end{align}
				がとなり,さらに論理積の除去法則より
				\begin{align}
					\psi(\tau,\sigma) \wedge \psi(\tau,\rho) &\vdash \tau = \sigma, \\
					\psi(\tau,\sigma) \wedge \psi(\tau,\rho) &\vdash \tau = \rho
				\end{align}
				が出る.ここで等号の推移律(定理\ref{thm:transitive_law_of_equality})より
				\begin{align}
					\EXTAX,\EQAX \vdash \tau = \sigma \rarrow 
						(\, \tau = \rho \rarrow \sigma = \rho\, )
				\end{align}
				が成り立つので,三段論法を二回用いれば
				\begin{align}
					\psi(\tau,\sigma) \wedge \psi(\tau,\rho),\ \EXTAX,\EQAX 
					\vdash \sigma = \rho
				\end{align}
				が得られる.ゆえに演繹法則より
				\begin{align}
					\EXTAX,\EQAX \vdash \psi(\tau,\sigma) \wedge \psi(\tau,\rho)
					\rarrow \sigma = \rho
				\end{align}
				となり,全称記号の推論規則より
				\begin{align}
					\EXTAX,\EQAX &\vdash \forall z\, 
						(\, \psi(\tau,\sigma) \wedge \psi(\tau,z) 
						\rarrow \sigma = z\, ), \\
					\EXTAX,\EQAX &\vdash \forall y\, \forall z\, 
						(\, \psi(\tau,y) \wedge \psi(\tau,z) \rarrow y = z\, ), \\
					\EXTAX,\EQAX &\vdash \forall x\, \forall y\, \forall z\, 
						(\, \psi(x,y) \wedge \psi(x,z) \rarrow y = z\, )
				\end{align}
				が従う.
			
			\item[step2]
				置換公理より
				\begin{align}
					\REPAX \vdash \forall x\, \forall y\, \forall z\, 
					(\, \psi(x,y) \wedge \psi(x,z)
					\rarrow y = z\, )
					\rarrow \forall a\, \exists u\, \forall v\,
					(\, v \in u \lrarrow \exists x\, (\, x \in a \wedge 
					\psi(x,v)\, )\, )
				\end{align}
				が成り立つので,(\refeq{fom:thm_axiom_of_separation_1})との三段論法より
				\begin{align}
					\EXTAX,\EQAX,\REPAX \vdash \forall a\, \exists u\, \forall v\,
					(\, v \in u \lrarrow \exists x\, (\, x \in a \wedge 
					\psi(x,v)\, )\, )
					\label{fom:thm_axiom_of_separation_2}
				\end{align}
				が成立する.(\refeq{fom:thm_axiom_of_separation_1})を示したいので
				\begin{align}
					\alpha \defeq \varepsilon a \negation \exists s\, \forall x\,
					(\, x \in s \lrarrow x \in a \wedge \varphi(x)\, )
				\end{align}
				とおくと,全称記号の推論規則より
				\begin{align}
					\EXTAX,\EQAX,\REPAX \vdash &\forall a\, \exists u\, \forall v\,
					(\, v \in u \lrarrow \exists x\, (\, x \in a \wedge 
					\psi(x,v)\, )\, ) \\
					&\rarrow \exists u\, \forall v\,
					(\, v \in u \lrarrow \exists x\, (\, x \in \alpha \wedge 
					\psi(x,v)\, )\, )
				\end{align}
				となるので,(\refeq{fom:thm_axiom_of_separation_2})との三段論法より
				\begin{align}
					\EXTAX,\EQAX,\REPAX \vdash \exists u\, \forall v\,
					(\, v \in u \lrarrow \exists x\, (\, x \in \alpha \wedge 
					\psi(x,v)\, )\, )
					\label{fom:thm_axiom_of_separation_3}
				\end{align}
				が従う.ここで
				\begin{align}
					\zeta \defeq \varepsilon u\, \forall v\,
					(\, v \in u \lrarrow \exists x\, (\, x \in \alpha \wedge 
					\psi(x,v)\, )\, )
				\end{align}
				とおけば,量化子の推論規則と(\refeq{fom:thm_axiom_of_separation_2})との
				三段論法により
				\begin{align}
					\EXTAX,\EQAX,\REPAX \vdash \forall v\,
					(\, v \in \zeta \lrarrow \exists x\, (\, x \in \alpha \wedge 
					\psi(x,v)\, )\, )
					\label{fom:thm_axiom_of_separation_4}
				\end{align}
				が成り立つ.
			
			\item[step3]
				最後に
				\begin{align}
					\EXTAX,\EQAX,\EQAXEP,\REPAX \vdash \forall x\,
					(\, x \in \zeta \lrarrow x \in \alpha \wedge \varphi(x)\, )
					\label{fom:thm_axiom_of_separation_8}
				\end{align}
				となることを示す.いま
				\begin{align}
					\tau \defeq \varepsilon x \negation
					(\, x \in \zeta \lrarrow x \in \alpha \wedge \varphi(x)\, )
				\end{align}
				とおけば,(\refeq{fom:thm_axiom_of_separation_4})と全称記号の推論規則より
				\begin{align}
					\EXTAX,\EQAX,\REPAX \vdash 
					\tau \in \zeta \lrarrow \exists x\, (\, x \in \alpha \wedge 
					\psi(x,\tau)\, )
					\label{fom:thm_axiom_of_separation_5}
				\end{align}
				が従う.ゆえに
				\begin{align}
					\tau \in \zeta,\ \EXTAX,\EQAX,\REPAX \vdash
					\exists x\, (\, x \in \alpha \wedge \psi(x,\tau)\, )
				\end{align}
				となる.ここで
				\begin{align}
					\sigma \defeq \varepsilon x\, (\, x \in \alpha \wedge
					\psi(x,\tau)\, )
				\end{align}
				とおけば
				\begin{align}
					\tau \in \zeta,\ \EXTAX,\EQAX,\REPAX \vdash
					\sigma \in \alpha \wedge \psi(\sigma,\tau)
				\end{align}
				となるので,
				\begin{align}
					\tau \in \zeta,\ \EXTAX,\EQAX,\REPAX &\vdash \sigma \in \alpha, \\
					\tau \in \zeta,\ \EXTAX,\EQAX,\REPAX &\vdash \sigma = \tau, \\
					\tau \in \zeta,\ \EXTAX,\EQAX,\REPAX &\vdash \varphi(\sigma)
				\end{align}
				が従う.ところで相等性公理と代入原理
				(定理\ref{thm:the_principle_of_substitution})より
				\begin{align}
					\tau \in \zeta,\ \EXTAX,\EQAX,\REPAX &\vdash 
						\sigma = \tau \rarrow (\, \sigma \in \alpha \rarrow
						\tau \in \alpha\, ), \\
					\tau \in \zeta,\ \EXTAX,\EQAX,\EQAXEP,\REPAX &\vdash
						\sigma = \tau \rarrow (\, \varphi(\sigma) \rarrow
						\varphi(\tau)\, ), \\
				\end{align}
				が成り立つので,三段論法より
				\begin{align}
					\tau \in \zeta,\ \EXTAX,\EQAX,\REPAX &\vdash \tau \in \alpha, \\
					\tau \in \zeta,\ \EXTAX,\EQAX,\EQAXEP,\REPAX &\vdash \varphi(\tau)
				\end{align}
				が従い
				\begin{align}
					\tau \in \zeta,\ \EXTAX,\EQAX,\EQAXEP,\REPAX \vdash
					\tau \in \alpha \wedge \varphi(\tau)
				\end{align}
				となる.以上で
				\begin{align}
					\EXTAX,\EQAX,\EQAXEP,\REPAX \vdash \tau \in \zeta \rarrow
					\tau \in \alpha \wedge \varphi(\tau)
					\label{fom:thm_axiom_of_separation_6}
				\end{align}
				が得られた.逆に定理\ref{thm:any_class_equals_to_itself}と併せて
				\begin{align}
					\tau \in \alpha \wedge \varphi(\tau),\ 
					\EXTAX,\EQAX,\EQAXEP,\REPAX \vdash
					\tau \in \alpha \wedge (\, \tau = \tau \wedge \varphi(\tau)\, )
				\end{align}
				が成り立つので,存在記号の推論規則より
				\begin{align}
					\tau \in \alpha \wedge \varphi(\tau),\ 
					\EXTAX,\EQAX,\EQAXEP,\REPAX \vdash
					\exists x\, (\, x \in \alpha \wedge \psi(x,\tau)\, )
				\end{align}
				となる.他方で(\refeq{fom:thm_axiom_of_separation_5})より
				\begin{align}
					\EXTAX,\EQAX,\REPAX \vdash 
					\exists x\, (\, x \in \alpha \wedge 
					\psi(x,\tau)\, ) \rarrow \tau \in \zeta
				\end{align}
				が成り立つので,三段論法より
				\begin{align}
					\tau \in \alpha \wedge \varphi(\tau),\ 
					\EXTAX,\EQAX,\EQAXEP,\REPAX \vdash \tau \in \zeta
				\end{align}
				が従う.以上で
				\begin{align}
					\EXTAX,\EQAX,\EQAXEP,\REPAX \vdash 
					\tau \in \alpha \wedge \varphi(\tau) \rarrow \tau \in \zeta
					\label{fom:thm_axiom_of_separation_7}
				\end{align}
				も得られた.(\refeq{fom:thm_axiom_of_separation_6})と
				(\refeq{fom:thm_axiom_of_separation_7})および存在記号の推論規則より
				(\refeq{fom:thm_axiom_of_separation_8})が出る.
				すると存在記号の推論規則より
				\begin{align}
					\EXTAX,\EQAX,\EQAXEP,\REPAX \vdash \exists s\, \forall x\,
					(\, x \in s \lrarrow x \in \alpha \wedge \varphi(x)\, )
				\end{align}
				となり,全称記号の推論規則より
				\begin{align}
					\EXTAX,\EQAX,\EQAXEP,\REPAX \vdash 
					\forall a\, \exists s\, \forall x\,
					(\, x \in s \lrarrow x \in a \wedge \varphi(x)\, )
				\end{align}
				が従う.
				\QED
		\end{description}
	\end{sketch}
	
	\begin{screen}
		\begin{thm}[$\emptyset$は集合]\label{thm:emptyset_is_a_set}
			$\emptyset$は集合である:
			\begin{align}
				\set{\emptyset}.
			\end{align}
		\end{thm}
	\end{screen}
	
	\begin{sketch}
		分出定理より
		\begin{align}
			\forall z\, \exists y\, \forall x\,
			(\, x \in y \lrarrow x \in z \wedge x \neq x\, )
			\label{fom:thm_emptyset_is_a_set_1}
		\end{align}
		が成立するが,この式から
		\begin{align}
			\exists y\, \forall x\, (\, x \in y \lrarrow x \neq x\, )
			\label{fom:thm_emptyset_is_a_set_2}
		\end{align}
		を示せる.これはすなわち$\emptyset$が集合であるということを示唆する.
		$\zeta$を勝手な$\varepsilon$項として,後々の便宜のために
		\begin{align}
			\sigma &\defeq \varepsilon y\, \forall x\,
			(\, x \in y \lrarrow x \in \zeta \wedge x \neq x\, ), \\
			\tau &\defeq \varepsilon x \negation
			(\, x \in \sigma \lrarrow x \neq x\, )
		\end{align}
		とおけば,(\refeq{fom:thm_emptyset_is_a_set_1})より
		\begin{align}
			\tau \in \sigma \lrarrow \tau \in \zeta \wedge \tau \neq \tau
		\end{align}
		が成立する.論理和の規則より
		\begin{align}
			\tau \in \zeta \wedge \tau \neq \tau \rarrow \tau \neq \tau
		\end{align}
		が満たされるので,まずは
		\begin{align}
			\tau \in \sigma \rarrow \tau \neq \tau
		\end{align}
		が得られる.また
		\begin{align}
			\tau = \tau
		\end{align}
		は正しいので,
		\begin{align}
			\tau = \tau \rarrow (\, \tau \notin \sigma \rarrow
			\tau = \tau\, )
		\end{align}
		と併せて
		\begin{align}
			\tau \notin \sigma \rarrow \tau = \tau
		\end{align}
		が成り立ち,対偶を取れば
		\begin{align}
			\tau \neq \tau \rarrow \tau \in \sigma
		\end{align}
		も得られる.ゆえに
		\begin{align}
			\forall x\, (\, x \in \sigma \lrarrow x \neq x\, )
		\end{align}
		が得られ,(\refeq{fom:thm_emptyset_is_a_set_2})が従う.
		\QED
	\end{sketch}
	
	\begin{screen}
		\begin{thm}[空集合は$\mathcal{L}$のいかなる対象も要素に持たない]\label{thm:emptyset_has_nothing}
			\begin{align}
				\forall x\, (\, x \notin \emptyset\, ).
			\end{align}
		\end{thm}
	\end{screen}
	
	\begin{sketch}
		$\tau$を$\mathscr{L}$の対象とするとき,類の公理より
		\begin{align}
			\tau \in \emptyset \rarrow \tau \neq \tau
		\end{align}
		が成り立つから,対偶を取れば
		\begin{align}
			\tau = \tau \rarrow \tau \notin \emptyset
		\end{align}
		が成り立つ(推論法則\ref{thm:contraposition_is_true}).定理\ref{thm:any_class_equals_to_itself}より
		\begin{align}
			\tau = \tau
		\end{align}
		は正しいので,三段論法より
		\begin{align}
			\tau \notin \emptyset
		\end{align}
		が成り立つ.そして$\tau$の任意性より
		\begin{align}
			\forall x\, (\, x \notin \emptyset\, )
		\end{align}
		が得られる.
		\QED
	\end{sketch}
	
	\begin{screen}
		\begin{thm}[$\mathcal{L}$のいかなる対象も要素に持たない類は空集合に等しい]
		\label{thm:uniqueness_of_emptyset}
			$a$を類とするとき次が成り立つ:
			\begin{align}
				\forall x\, (\, x \notin a\, ) \lrarrow a = \emptyset.
			\end{align}
		\end{thm}
	\end{screen}
	
	\begin{prf}
		$a$を類として$\forall x\, (\, x \notin a\, )$が成り立っていると仮定する.このとき
		$\tau$を$\mathcal{L}$の任意の対象とすれば
		\begin{align}
			\tau \notin a \vee \tau \in \emptyset
		\end{align}
		と
		\begin{align}
			\tau \notin \emptyset \vee \tau \in a
		\end{align}
		が共に成り立つので,推論法則\ref{logicalthm:rule_of_inference_3}より
		\begin{align}
			\tau \in a \rarrow \tau \in \emptyset
		\end{align}
		と
		\begin{align}
			\tau \in \emptyset \rarrow \tau \in a
		\end{align}
		が共に成り立つ.よって
		\begin{align}
			\tau \in a \lrarrow \tau \in \emptyset
		\end{align}
		が成立し,$\tau$の任意性と推論法則\ref{logicalthm:fundamental_law_of_universal_quantifier}から
		\begin{align}
			\forall x\, (\, x \in a \lrarrow x \in \emptyset\, )
		\end{align}
		が得られる.ゆえに外延性の公理より
		\begin{align}
			a = \emptyset
		\end{align}
		が成立し,演繹法則より
		\begin{align}
			\forall x\, (\, x \notin a\, ) \rarrow a = \emptyset
		\end{align}
		が得られる.逆に
		\begin{align}
			a = \emptyset
		\end{align}
		が成り立っていると仮定する.ここで$\chi$を$\mathcal{L}$の任意の対象とすれば,
		相等性の公理より
		\begin{align}
			\chi \in a \rarrow \chi \in \emptyset
		\end{align}
		が成立するので,対偶を取れば
		\begin{align}
			\chi \notin \emptyset \rarrow \chi \notin a
		\end{align}
		が成り立つ.定理\ref{thm:emptyset_has_nothing}より
		\begin{align}
			\chi \notin \emptyset
		\end{align}
		が満たされているので,三段論法より
		\begin{align}
			\chi \notin a
		\end{align}
		が成立し,$\chi$の任意性と推論法則\ref{logicalthm:fundamental_law_of_universal_quantifier}より
		\begin{align}
			\forall x\, (\, x \notin a\, )
		\end{align}
		が成立する.ここに演繹法則を適用して
		\begin{align}
			a = \emptyset \rarrow \forall x\, (\, x \notin a\, )
		\end{align}
		も得られる.
		\QED
	\end{prf}
	
	\begin{screen}
		\begin{thm}[空集合はいかなる類も要素に持たない]
		\label{thm:emptyset_does_not_contain_any_class}
			$a,b$を類とするとき次が成り立つ:
			\begin{align}
				b = \emptyset \rarrow a \notin b.
			\end{align}
		\end{thm}
	\end{screen}
	
	\begin{prf}
		いま$a \in b$が成り立っていると仮定する.このとき要素の公理と三段論法より
		\begin{align}
			\set{a}
		\end{align}
		が成立する.ここで
		\begin{align}
			\tau \defeq \varepsilon x\, (\, a = x\, )
		\end{align}
		とおけば,存在記号に関する規則から
		\begin{align}
			a = \tau
		\end{align}
		が成り立つので,相等性の公理より
		\begin{align}
			\tau \in b
		\end{align}
		が従い,存在記号に関する規則より
		\begin{align}
			\exists x\, (\, x \in b\, )
		\end{align}
		が成り立つ.よって演繹法則から
		\begin{align}
			a \in b \rarrow \exists x\, (\, x \in b\, )
		\end{align}
		が成り立つ.この対偶を取り推論法則\ref{logicalthm:De_Morgan_law_for_quantifiers}を適用すれば
		\begin{align}
			\forall x\, (\, x \notin b\, ) \rarrow a \notin b
		\end{align}
		が得られる.定理\ref{thm:uniqueness_of_emptyset}より
		\begin{align}
			b = \emptyset \rarrow \forall x\, (\, x \notin b\, )
		\end{align}
		も正しいので,含意の推移律から
		\begin{align}
			b = \emptyset \rarrow a \notin b
		\end{align}
		が得られる.
		\QED
	\end{prf}
	
	\begin{screen}
		\begin{dfn}[部分類]
			$a,b$を$\mathcal{L}'$の項とするとき,
			\begin{align}
				a \subset b \overset{\mathrm{def}}{\lrarrow}
				\forall x\ (\ x \in a \rarrow x \in b\ )
			\end{align}
			と定める.式$a \subset b$を``$a$は$b$の{\bf 部分類}\index{ぶぶんるい@部分類}{\bf (subclass)}である''
			と翻訳し,特に$a$が集合である場合は``$a$は$b$の{\bf 部分集合}\index{ぶぶんしゅうごう@部分集合}{\bf (subset)}である''と翻訳する.
			また次の記号も定める:
			\begin{align}
				a \subsetneq b \defarrow a \subset b \wedge a \neq b.
			\end{align}
		\end{dfn}
	\end{screen}
	
	空虚な真の一例として次の結果を得る.
	
	\begin{screen}
		\begin{thm}[空集合は全ての類に含まれる]\label{thm:emptyset_if_a_subset_of_every_class}
			$a$を類とするとき次が成り立つ:
			\begin{align}
				\emptyset \subset a.
			\end{align}
		\end{thm}
	\end{screen}
	
	\begin{prf}
		$a$を類とする.$\tau$を$\mathcal{L}$の任意の対象とすれば
		\begin{align}
			\tau \notin \emptyset
		\end{align}
		が成り立つから,推論規則\ref{logicalaxm:fundamental_rules_of_inference}を適用して
		\begin{align}
			\tau \notin \emptyset \vee \tau \in a
		\end{align}
		が成り立つ.従って
		\begin{align}
			\tau \in \emptyset \rarrow \tau \in a
		\end{align}
		が成り立ち,$\tau$の任意性と推論法則\ref{logicalthm:fundamental_law_of_universal_quantifier}より
		\begin{align}
			\forall x\, (\, x \in \emptyset \rarrow x \in a\, )
		\end{align}
		が成立する.
		\QED
	\end{prf}
	
	$a \subset b$とは$a$に属する全ての``$\mathcal{L}$の対象''は$b$に属するという定義であったが,
	要素となりうる類は集合であるという公理から,$a$に属する全ての``類''もまた$b$に属する.
	
	\begin{screen}
		\begin{thm}[類はその部分類に属する全ての類を要素に持つ]\label{thm:subclass_contains_all_elements}
			$a,b,c$を類とすれば次が成り立つ:
			\begin{align}
				a \subset b \rarrow (\, c \in a \rarrow c \in b\, ).
			\end{align}
		\end{thm}
	\end{screen}
	
	\begin{prf}	
		いま$a \subset b$が成り立っているとする.このとき
		\begin{align}
			c \in a
		\end{align}
		が成り立っていると仮定すれば,要素の公理より
		\begin{align}
			\set{c}
		\end{align}
		が成り立つ.ここで
		\begin{align}
			\tau \defeq \varepsilon x\, (\, c=x\, )
		\end{align}
		とおくと
		\begin{align}
			c = \tau
		\end{align}
		が成り立つので,相等性の公理より
		\begin{align}
			\tau \in a
		\end{align}
		が成り立ち,$a \subset b$と推論法則\ref{logicalthm:fundamental_law_of_universal_quantifier}から
		\begin{align}
			\tau \in b
		\end{align}
		が従う.再び相等性の公理を適用すれば
		\begin{align}
			c \in b
		\end{align}
		が成り立つので,演繹法則より,$a \subset b$が成り立っている下で
		\begin{align}
			c \in a \rarrow c \in b
		\end{align}
		が成立する.再び演繹法則を適用すれば定理の主張が得られる.
		\QED
	\end{prf}
	
	宇宙$\Univ$は類の一つであった.当然のようであるが,それは最大の類である.
	\begin{screen}
		\begin{thm}[$\Univ$は最大の類である]
			$a$を類とするとき次が成り立つ:
			\begin{align}
				a \subset \Univ.
			\end{align}
		\end{thm}
	\end{screen}
	
	\begin{prf}
		$\tau$を$\mathcal{L}$の任意の対象とすれば,定理\ref{thm:any_class_equals_to_itself}と類の公理より
		\begin{align}
			\tau \in \Univ
		\end{align}
		が成立するので,推論規則\ref{logicalaxm:fundamental_rules_of_inference}より
		\begin{align}
			\tau \notin a \vee \tau \in \Univ
		\end{align}
		が成立する.このとき推論法則\ref{logicalthm:rule_of_inference_3}より
		\begin{align}
			\tau \in a \rarrow \tau \in \Univ
		\end{align}
		が成立し,$\tau$の任意性と推論法則\ref{logicalthm:fundamental_law_of_universal_quantifier}から
		\begin{align}
			\forall x\, (\, x \in a \rarrow x \in \Univ\, )
		\end{align}
		が従う.
		\QED
	\end{prf}
	
	\begin{screen}
		\begin{thm}[互いに互いの部分類となる類同士は等しい]\label{thm:mutually_sub_classes_are_equivalent}
			$a,b$を類とするとき次が成り立つ:
			\begin{align}
				a \subset b \wedge b \subset a \lrarrow a = b.
			\end{align}
		\end{thm}
	\end{screen}
	
	\begin{sketch}
		$a \subset b \wedge b \subset a$が成り立っていると仮定する.
		このとき$\tau$を$\mathcal{L}$の任意の対象とすれば,
		$a \subset b$と推論法則\ref{logicalthm:fundamental_law_of_universal_quantifier}より
		\begin{align}
			\tau \in a \rarrow \tau \in b
		\end{align}
		が成立し,$b \subset a$と推論法則\ref{logicalthm:fundamental_law_of_universal_quantifier}より
		\begin{align}
			\tau \in b \rarrow \tau \in a
		\end{align}
		が成立するので,
		\begin{align}
			\tau \in a \lrarrow \tau \in b
		\end{align}
		が成り立つ.$\tau$の任意性と推論法則\ref{logicalthm:fundamental_law_of_universal_quantifier}および
		外延性の公理より
		\begin{align}
			a = b
		\end{align}
		が出るので,演繹法則より
		\begin{align}
			a \subset b \wedge b \subset a \rarrow a = b
		\end{align}
		が得られる.逆に$a = b$が満たされていると仮定するとき,$\tau$を$\mathcal{L}$の任意の対象とすれば
		\begin{align}
			\tau \in a \rarrow \tau \in b
		\end{align}
		と
		\begin{align}
			\tau \in b \rarrow \tau \in a
		\end{align}
		が共に成り立つ. よって推論法則\ref{logicalthm:fundamental_law_of_universal_quantifier}より
		\begin{align}
			a \subset b
		\end{align}
		と
		\begin{align}
			b \subset a
		\end{align}
		が共に従う.よって演繹法則より
		\begin{align}
			a = b \rarrow a \subset b \wedge b \subset a
		\end{align}
		も得られる.
		\QED
	\end{sketch}
	
	\monologue{
		定理\ref{thm:subclass_contains_all_elements}と定理\ref{thm:mutually_sub_classes_are_equivalent}より,
			類$a,b$が$a = b$を満たすならば,$a$と$b$は要素に持つ$\mathcal{L}$の対象のみならず,
			要素に持つ類までも一致するのですね.
	}
	
\section{順序型について}
	$(A,R)$を整列集合とするとき,
	\begin{align}
		x \longmapsto 
		\begin{cases}
			\min{A \backslash \ran{x}} & \mbox{if } \ran{x} \subsetneq A \\
			A & \mbox{o.w.} \\
		\end{cases}
	\end{align}
	なる写像$G$に対して
	\begin{align}
		\forall \alpha\, F(\alpha) = G(\rest{F}{\alpha})
	\end{align}
	なる写像$F$を取り
	\begin{align}
		\alpha \defeq \min{\Set{\alpha \in \ON}{F(\alpha) = A}}
	\end{align}
	とおけば,$\alpha$は$(A,R)$の順序型.
	
\section{超限再帰について}
	$\Univ$上の写像$G$が与えられたら,
	\begin{align}
		F \defeq \Set{(\alpha,x)}{\ord{\alpha} \wedge
		\exists f\, \left(\, f \fon \alpha \wedge
		\forall \beta \in \alpha\, \left(\, f(\beta) = G(\rest{f}{\beta})\, \right)
		\wedge x = G(f)\, \right)}
	\end{align}
	により$F$を定めれば
	\begin{align}
		\forall \alpha\, F(\alpha) = G(\rest{F}{\alpha})
	\end{align}
	が成立する.
	
	\begin{screen}
		任意の順序数$\alpha$および$\alpha$上の写像$f$と$g$に対して,
		\begin{align}
			\forall \beta \in \alpha\,
			\left(\, f(\beta) = G(\rest{f}{\beta})\, \right)
		\end{align}
		かつ
		\begin{align}
			\forall \beta \in \alpha\,
			\left(\, g(\beta) = G(\rest{g}{\beta})\, \right)
		\end{align}
		ならば$f = g$である.
	\end{screen}
	
	まず
	\begin{align}
		f(0) = G(\rest{f}{0}) = G(0) = G(\rest{g}{0}) = g(0)
	\end{align}
	が成り立つ.また
	\begin{align}
		\forall \delta \in \beta\, \left(\, 
		\delta \in \alpha \rarrow f(\delta) = g(\delta)\, \right)
	\end{align}
	ならば,$\beta \in \alpha$であるとき
	\begin{align}
		\rest{f}{\beta} = \rest{g}{\beta}
	\end{align}
	となるので
	\begin{align}
		\beta \in \alpha \rarrow f(\beta) = g(\beta)
	\end{align}
	が成り立つ.ゆえに
	\begin{align}
		f = g
	\end{align}
	が得られる.
	
	\begin{screen}
		任意の順序数$\alpha$に対して,$\alpha$上の写像$f$で
		\begin{align}
			\forall \beta \in \alpha\, \left(\, 
			f(\beta) = G(\rest{f}{\beta})\, \right)
		\end{align}
		を満たすものが取れる.
	\end{screen}
	
	$\alpha = 0$のとき$f \defeq 0$とすればよい.$\alpha$の任意の要素$\beta$に対して
	\begin{align}
		g \fon \beta \wedge \forall \gamma\in \beta\, \left(\, 
		g(\gamma) = G(\rest{g}{\gamma})\, \right)
	\end{align}
	なる$g$が存在するとき,
	\begin{align}
		f \defeq \Set{(\beta,x)}{\beta \in \alpha \wedge
		\exists g\, \left(\, g \fon \beta \wedge
		\forall \gamma \in \beta\, \left(\, g(\gamma) = G(\rest{g}{\gamma})\, \right)
		\wedge x = G(g)\, \right)}
	\end{align}
	と定めれば,$f$は$\alpha$上の写像であって
	\begin{align}
		\forall \beta \in \alpha\, \left(\, 
		f(\beta) = G(\rest{f}{\beta})\, \right)
	\end{align}
	を満たす.
	
	\begin{screen}
		任意の順序数$\alpha$に対して$F(\alpha) = G(\rest{F}{\alpha})$が成り立つ.
	\end{screen}
	
	$\alpha = 0$ならば,$0$上の写像は$0$のみなので
	\begin{align}
		F(0) = G(0) = G(\rest{F}{0})
	\end{align}
	である.
	\begin{align}
		\forall \beta \in \alpha\, F(\beta) = G(\rest{F}{\beta})
	\end{align}
	が成り立っているとき,
	\begin{align}
		\forall \beta \in \alpha\, f(\beta) = G(\rest{f}{\beta})
	\end{align}
	を満たす$\alpha$上の写像$f$を取れば,前の一意性より
	\begin{align}
		f = \rest{F}{\alpha}
	\end{align}
	が成立する.よって
	\begin{align}
		F(\alpha) = G(f) = G(\rest{F}{\alpha})
	\end{align}
	となる.
	\QED
	
\section{自然数の全体について}
	$\Natural$を
	\begin{align}
		\Natural \defeq \Set{\beta}{\mbox{$\alpha \leq \beta$である$\alpha$は
		$0$であるか後続型順序数}}
	\end{align}
	によって定めれば,無限公理より
	\begin{align}
		\set{\Natural}
	\end{align}
	である.また$\ord{\Natural}$と$\limo{\Natural}$も証明できるはず.
	$\Natural$が最小の極限数であることは$\Natural$を定義した論理式より従う.
\section{空集合}
	\begin{screen}
		\begin{logicalthm}[分配された論理積の簡約]
		\label{logicalthm:contraction_law_of_distributed_injunctions}
			$A,B,C$を$\mathcal{L}$の文とするとき,
			\begin{align}
				\vdash (A \wedge C) \wedge (B \wedge C) \rarrow A \wedge B.
			\end{align}
		\end{logicalthm}
	\end{screen}
	
	\begin{sketch}
		論理積の除去より
		\begin{align}
			(A \wedge C) \wedge (B \wedge C) \vdash A \wedge C
		\end{align}
		となり,また同じく論理積の除去より
		\begin{align}
			(A \wedge C) \wedge (B \wedge C) &\vdash A \wedge C \rarrow A
		\end{align}
		となるので,三段論法より
		\begin{align}
			(A \wedge C) \wedge (B \wedge C) &\vdash A, 
			\label{fom:logicalthm_contraction_law_of_injunctions_1}
		\end{align}
		が従う.同様にして
		\begin{align}
			(A \wedge C) \wedge (B \wedge C) \vdash B
			\label{fom:logicalthm_contraction_law_of_injunctions_2}
		\end{align}
		も得られる.ここで論理積の導入より
		\begin{align}
			(A \wedge C) \wedge (B \wedge C) \vdash A \rarrow (B \rarrow A \wedge B)
		\end{align}
		が成り立つので,(\refeq{fom:logicalthm_contraction_law_of_injunctions_1})と
		(\refeq{fom:logicalthm_contraction_law_of_injunctions_2})との三段論法より
		\begin{align}
			(A \wedge C) \wedge (B \wedge C) \vdash A \wedge B
		\end{align}
		が出る.
		\QED
	\end{sketch}
	
	\begin{screen}
		\begin{dfn}[空集合]
			$\emptyset \defeq \Set{x}{x \neq x}$で定めるクラス$\emptyset$を{\bf 空集合}\index{くうしゅうごう@空集合}{\bf (empty set)}と呼ぶ.
		\end{dfn}
	\end{screen}
	
	$x$が集合であれば
	\begin{align}
		x = x
	\end{align}
	が成り立つので,$\emptyset$に入る集合など存在しない.
	つまり$\emptyset$は丸っきり``空っぽ''なのである.
	さて,$\emptyset$は集合であるか否か,という問題を考える.
	当然これが``大きすぎる集まり''であるはずはないし,
	そもそも名前に``集合''と付いているのだから
	$\emptyset$は集合であるべきだと思われるのだが,
	実際にこれが集合であることを示すには少し骨が折れる.
	まずは置換公理と分出定理を拵えなくてはならない.
	
	\begin{screen}
		\begin{axm}[置換公理]
			$x,y,s,y$を変項とし,
			$\varphi$を$s,t$のみが自由に現れる$\mathcal{L}$の式とし,
			$x$は$\varphi$で$s$への代入について自由であり,
			$y,z$は$\varphi$で$t$への代入について自由であるとするとき,
			次の式を$\REPAX$により参照する:
			\begin{align}
				\forall x\, \forall y\, \forall z\, 
				(\, \varphi(x,y) \wedge \varphi(x,z)
				\rarrow y = z\, )
				\rarrow \forall a\, \exists u\, \forall v\,
				(\, v \in u \lrarrow \exists x\, (\, x \in a \wedge 
				\varphi(x,v)\, )\, ).
			\end{align}
		\end{axm}
	\end{screen}
	
	$\Set{x}{\varphi(x)}$は集合であるとは限らないが,
	集合$a$との交叉(後述)
	\begin{align}
		a \cap \Set{x}{\varphi(x)}
	\end{align}
	は当然$a$より``小さい集まり''なのだから,集合であってほしいものである.
	これを公理化した式は{\bf 分出公理}\index{ぶんしゅつこうり@分出公理}
	{\bf (axiom of separation)}と呼ばれるが,公理化せずとも置換公理によって導かれる.
	
	\begin{screen}
		\begin{thm}[分出定理]\label{thm:axiom_of_separation}
			$\varphi$を$\mathcal{L}$の式とし,$x$を変項とし,$\varphi$には
			$x$のみが自由に現れるとするとき,
			\begin{align}
				\EXTAX,\EQAX,\COMAX,\ELEAX,\REPAX \vdash 
				\forall a\, \exists s\, \forall x\,
				(\, x \in s \lrarrow x \in a \wedge \varphi(x)\, ).
				\label{fom:thm_axiom_of_separation_0}
			\end{align}
			が成り立つ.ただし$\varphi$が$\lang{\varepsilon}$の式であるときは
			\begin{align}
				\EXTAX,\EQAX,\REPAX \vdash 
				\forall a\, \exists s\, \forall x\,
				(\, x \in s \lrarrow x \in a \wedge \varphi(x)\, ).
			\end{align}
		\end{thm}
	\end{screen}
	
	\begin{sketch}
		$y$を,$\varphi$の$x$への代入について自由である変項とする.
		そして$x$と$y$が自由に現れる式$\psi(x,y)$を
		\begin{align}
			x = y \wedge \varphi(x)
		\end{align}
		と設定する.
		\begin{description}
			\item[step1]
				まず
				\begin{align}
					\EXTAX,\EQAX \vdash \forall x\, \forall y\, \forall z\, 
					(\, \psi(x,y) \wedge \psi(x,z) \rarrow y = z\, )
					\label{fom:thm_axiom_of_separation_1}
				\end{align}
				が成り立つことを示す.これを見越して
				\begin{align}
					\tau &\defeq \varepsilon x \negation \forall y\, \forall z\, 
					(\, \psi(x,y) \wedge \psi(x,z) \rarrow y = z\, ), \\
					\sigma &\defeq \varepsilon y \negation \forall z\, 
					(\, \psi(\tau,y) \wedge \psi(\tau,z) \rarrow y = z\, ), \\
					\rho &\defeq \varepsilon z \negation 
					(\, \psi(\tau,\sigma) \wedge \psi(\tau,z) \rarrow \sigma = z\, )
				\end{align}
				とおく.$\psi(\tau,\sigma) \wedge \psi(\tau,\rho)$は縮約可能であって(
				論理的定理\ref{logicalthm:contraction_law_of_distributed_injunctions})
				\begin{align}
					\vdash (\, \tau = \sigma \wedge \varphi(\tau)\, )
						\wedge (\, \tau = \rho \wedge \varphi(\tau)\, )
						\rarrow \tau = \sigma \wedge \tau = \rho
				\end{align}
				が成り立つので
				\begin{align}
					\psi(\tau,\sigma) \wedge \psi(\tau,\rho) 
					\vdash \tau = \sigma \wedge \tau = \rho
				\end{align}
				がとなり,さらに論理積の除去より
				\begin{align}
					\psi(\tau,\sigma) \wedge \psi(\tau,\rho) &\vdash \tau = \sigma, \\
					\psi(\tau,\sigma) \wedge \psi(\tau,\rho) &\vdash \tau = \rho
				\end{align}
				が出る.ここで等号の推移律(定理\ref{thm:transitive_law_of_equality})より
				\begin{align}
					\EXTAX,\EQAX \vdash \tau = \sigma \rarrow 
						(\, \tau = \rho \rarrow \sigma = \rho\, )
				\end{align}
				が成り立つので,三段論法を二回用いれば
				\begin{align}
					\psi(\tau,\sigma) \wedge \psi(\tau,\rho),\ \EXTAX,\EQAX 
					\vdash \sigma = \rho
				\end{align}
				が得られる.ゆえに演繹定理より
				\begin{align}
					\EXTAX,\EQAX \vdash \psi(\tau,\sigma) \wedge \psi(\tau,\rho)
					\rarrow \sigma = \rho
				\end{align}
				となり,全称の導出(論理的定理\ref{logicalthm:derivation_of_universal_by_epsilon})より
				\begin{align}
					\EXTAX,\EQAX &\vdash \forall z\, 
						(\, \psi(\tau,\sigma) \wedge \psi(\tau,z) 
						\rarrow \sigma = z\, ), \\
					\EXTAX,\EQAX &\vdash \forall y\, \forall z\, 
						(\, \psi(\tau,y) \wedge \psi(\tau,z) \rarrow y = z\, ), \\
					\EXTAX,\EQAX &\vdash \forall x\, \forall y\, \forall z\, 
						(\, \psi(x,y) \wedge \psi(x,z) \rarrow y = z\, )
				\end{align}
				が従う.
			
			\item[step2]
				置換公理より
				\begin{align}
					\REPAX \vdash \forall x\, \forall y\, \forall z\, 
					(\, \psi(x,y) \wedge \psi(x,z)
					\rarrow y = z\, )
					\rarrow \forall a\, \exists u\, \forall v\,
					(\, v \in u \lrarrow \exists x\, (\, x \in a \wedge 
					\psi(x,v)\, )\, )
				\end{align}
				が成り立つので,(\refeq{fom:thm_axiom_of_separation_1})との三段論法より
				\begin{align}
					\EXTAX,\EQAX,\REPAX \vdash \forall a\, \exists u\, \forall v\,
					(\, v \in u \lrarrow \exists x\, (\, x \in a \wedge 
					\psi(x,v)\, )\, )
					\label{fom:thm_axiom_of_separation_2}
				\end{align}
				が成立する.(\refeq{fom:thm_axiom_of_separation_0})を示したいので
				\begin{align}
					\alpha \defeq \varepsilon a \negation \exists s\, \forall x\,
					(\, x \in s \lrarrow x \in a \wedge \varphi(x)\, )
				\end{align}
				とおくと,全称記号の論理的公理より
				\begin{align}
					\EXTAX,\EQAX,\REPAX \vdash &\forall a\, \exists u\, \forall v\,
					(\, v \in u \lrarrow \exists x\, (\, x \in a \wedge 
					\psi(x,v)\, )\, ) \\
					&\rarrow \exists u\, \forall v\,
					(\, v \in u \lrarrow \exists x\, (\, x \in \alpha \wedge 
					\psi(x,v)\, )\, )
				\end{align}
				となるので,(\refeq{fom:thm_axiom_of_separation_2})との三段論法より
				\begin{align}
					\EXTAX,\EQAX,\REPAX \vdash \exists u\, \forall v\,
					(\, v \in u \lrarrow \exists x\, (\, x \in \alpha \wedge 
					\psi(x,v)\, )\, )
					\label{fom:thm_axiom_of_separation_3}
				\end{align}
				が従う.ここで
				\begin{align}
					\zeta \defeq \varepsilon u\, \forall v\,
					(\, v \in u \lrarrow \exists x\, (\, x \in \alpha \wedge 
					\psi(x,v)\, )\, )
				\end{align}
				とおけば,存在記号の論理的公理により
				\begin{align}
					\EXTAX,\EQAX,\REPAX \vdash \forall v\,
					(\, v \in \zeta \lrarrow \exists x\, (\, x \in \alpha \wedge 
					\psi(x,v)\, )\, )
					\label{fom:thm_axiom_of_separation_4}
				\end{align}
				が成り立つ.
			
			\item[step3]
				最後に
				\begin{align}
					\EXTAX,\EQAX,\COMAX,\ELEAX,\REPAX \vdash \forall x\,
					(\, x \in \zeta \lrarrow x \in \alpha \wedge \varphi(x)\, )
					\label{fom:thm_axiom_of_separation_8}
				\end{align}
				となることを示す.いま
				\begin{align}
					\tau \defeq \varepsilon x \negation
					(\, x \in \zeta \lrarrow x \in \alpha \wedge \varphi(x)\, )
				\end{align}
				とおけば,(\refeq{fom:thm_axiom_of_separation_4})と全称記号の論理的公理より
				\begin{align}
					\EXTAX,\EQAX,\REPAX \vdash 
					\tau \in \zeta \lrarrow \exists x\, (\, x \in \alpha \wedge 
					\psi(x,\tau)\, )
					\label{fom:thm_axiom_of_separation_5}
				\end{align}
				が従う.ゆえに
				\begin{align}
					\tau \in \zeta,\ \EXTAX,\EQAX,\REPAX \vdash
					\exists x\, (\, x \in \alpha \wedge \psi(x,\tau)\, )
				\end{align}
				となる.ここで
				\begin{align}
					\sigma \defeq \varepsilon x\, (\, x \in \alpha \wedge
					\psi(x,\tau)\, )
				\end{align}
				とおけば
				\begin{align}
					\tau \in \zeta,\ \EXTAX,\EQAX,\REPAX \vdash
					\sigma \in \alpha \wedge \psi(\sigma,\tau)
				\end{align}
				となるので,
				\begin{align}
					\tau \in \zeta,\ \EXTAX,\EQAX,\REPAX &\vdash \sigma \in \alpha, \\
					\tau \in \zeta,\ \EXTAX,\EQAX,\REPAX &\vdash \sigma = \tau, \\
					\tau \in \zeta,\ \EXTAX,\EQAX,\REPAX &\vdash \varphi(\sigma)
				\end{align}
				が従う.ところで相等性公理と代入原理
				(定理\ref{thm:the_principle_of_substitution})より
				\begin{align}
					\tau \in \zeta,\ \EXTAX,\EQAX,\REPAX &\vdash 
						\sigma = \tau \rarrow (\, \sigma \in \alpha \rarrow
						\tau \in \alpha\, ), \\
					\tau \in \zeta,\ \EXTAX,\EQAX,\COMAX,\ELEAX,\REPAX &\vdash
						\sigma = \tau \rarrow (\, \varphi(\sigma) \rarrow
						\varphi(\tau)\, ), 
						\label{fom:thm_axiom_of_separation_9} \\
				\end{align}
				が成り立つので,三段論法より
				\begin{align}
					\tau \in \zeta,\ \EXTAX,\EQAX,\REPAX &\vdash \tau \in \alpha, \\
					\tau \in \zeta,\ \EXTAX,\EQAX,\COMAX,\ELEAX,\REPAX &\vdash \varphi(\tau)
				\end{align}
				が従い
				\begin{align}
					\tau \in \zeta,\ \EXTAX,\EQAX,\COMAX,\ELEAX,\REPAX \vdash
					\tau \in \alpha \wedge \varphi(\tau)
				\end{align}
				となる.以上で
				\begin{align}
					\EXTAX,\EQAX,\COMAX,\ELEAX,\REPAX \vdash \tau \in \zeta \rarrow
					\tau \in \alpha \wedge \varphi(\tau)
					\label{fom:thm_axiom_of_separation_6}
				\end{align}
				が得られた.逆に定理\ref{thm:any_class_equals_to_itself}と併せて
				\begin{align}
					\tau \in \alpha \wedge \varphi(\tau),\ \EXTAX \vdash
					\tau \in \alpha \wedge (\, \tau = \tau \wedge \varphi(\tau)\, )
				\end{align}
				が成り立つので,存在記号の論理的公理より
				\begin{align}
					\tau \in \alpha \wedge \varphi(\tau),\ \EXTAX \vdash
					\exists x\, (\, x \in \alpha \wedge \psi(x,\tau)\, )
				\end{align}
				となる.他方で(\refeq{fom:thm_axiom_of_separation_5})より
				\begin{align}
					\EXTAX,\EQAX,\REPAX \vdash 
					\exists x\, (\, x \in \alpha \wedge 
					\psi(x,\tau)\, ) \rarrow \tau \in \zeta
				\end{align}
				が成り立つので,三段論法より
				\begin{align}
					\tau \in \alpha \wedge \varphi(\tau),\ 
					\EXTAX,\EQAX,\REPAX \vdash \tau \in \zeta
				\end{align}
				が従う.以上で
				\begin{align}
					\EXTAX,\EQAX,\REPAX \vdash 
					\tau \in \alpha \wedge \varphi(\tau) \rarrow \tau \in \zeta
					\label{fom:thm_axiom_of_separation_7}
				\end{align}
				も得られた.(\refeq{fom:thm_axiom_of_separation_6})と
				(\refeq{fom:thm_axiom_of_separation_7})および存在記号の論理的公理より
				(\refeq{fom:thm_axiom_of_separation_8})が出る.
				すると存在記号の論理的公理より
				\begin{align}
					\EXTAX,\EQAX,\COMAX,\ELEAX,\REPAX \vdash \exists s\, \forall x\,
					(\, x \in s \lrarrow x \in \alpha \wedge \varphi(x)\, )
				\end{align}
				となり,全称の導出(論理的定理\ref{logicalthm:derivation_of_universal_by_epsilon})より
				\begin{align}
					\EXTAX,\EQAX,\COMAX,\ELEAX,\REPAX \vdash 
					\forall a\, \exists s\, \forall x\,
					(\, x \in s \lrarrow x \in a \wedge \varphi(x)\, )
				\end{align}
				が従う.$\varphi$が$\lang{\varepsilon}$の式である場合は
				(\refeq{fom:thm_axiom_of_separation_9})で$\COMAX$と$\ELEAX$が
				追加されないので
				\begin{align}
					\EXTAX,\EQAX,\REPAX \vdash 
					\forall a\, \exists s\, \forall x\,
					(\, x \in s \lrarrow x \in a \wedge \varphi(x)\, )
				\end{align}
				が得られる.
				\QED
		\end{description}
	\end{sketch}
	
	%分出定理は$\EQAXEP$を前提としているが,ステートメントの$\varphi(x)$が
	%\begin{align}
	%	x \neq x
	%\end{align}
	%なる式の場合は
	%\begin{align}
	%	\EXTAX,\EQAX,\REPAX \vdash \forall a\, \exists s\, \forall x\,
	%		(\, x \in s \lrarrow x \in a \wedge x \neq x\, )
	%\end{align}
	%が成立する.そもそも分出定理の証明で$\EQAXEP$が必要になったのは
	%\begin{align}
	%	\sigma = \tau \rarrow (\, \varphi(\sigma) \rarrow \varphi(\tau)\, )
	%\end{align}
	%を用いるためであったが(\refeq{fom:thm_axiom_of_separation_9}),
	%$\varphi(x)$が$x \neq x$なる単純な式である場合は
	%\begin{align}
	%	\EXTAX \vdash \sigma = \tau \rarrow
	%	(\, \sigma \neq \sigma \rarrow \tau \neq \tau\, )
	%\end{align}
	%が成り立つのである.実際
	%\begin{align}
	%	\sigma = \tau,\ \tau = \tau,\ \EXTAX &\vdash \sigma = \sigma,
	%	&& (\mbox{定理\ref{thm:any_class_equals_to_itself}}) \\
	%	\sigma = \tau,\ \EXTAX &\vdash \tau = \tau \rarrow \sigma = \sigma,
	%	&& (\mbox{演繹定理(メタ定理\ref{metathm:deduction_theorem})}) \\
	%	\sigma = \tau,\ \EXTAX &\vdash \sigma \neq \sigma \rarrow \tau \neq \tau,
	%	&& (\mbox{対偶(論理的定理\ref{logicalthm:introduction_of_contraposition})})
	%\end{align}
	%となる.
	
	\begin{screen}
		\begin{thm}[$\emptyset$は集合]\label{thm:emptyset_is_a_set}
			\begin{align}
				\EXTAX,\EQAX,\COMAX,\REPAX \vdash \set{\emptyset}.
			\end{align}
		\end{thm}
	\end{screen}
	
	\begin{sketch}
		分出定理(\ref{thm:axiom_of_separation})より
		\begin{align}
			\EXTAX,\EQAX,\REPAX \vdash \forall a\, \exists s\, \forall x\,
			(\, x \in s \lrarrow x \in a \wedge x \neq x\, )
			\label{fom:thm_emptyset_is_a_set_1}
		\end{align}
		が成立する.$\alpha$を主要$\varepsilon$項とすれば,全称記号の論理的公理より
		\begin{align}
			\EXTAX,\EQAX,\REPAX \vdash \exists s\, \forall x\,
			(\, x \in s \lrarrow x \in \alpha \wedge x \neq x\, )
		\end{align}
		となり,また
		\begin{align}
			\sigma \defeq \varepsilon s\, \forall x\,
			(\, x \in s \lrarrow x \in \alpha \wedge x \neq x\, )
		\end{align}
		とおけば存在記号の論理的公理より
		\begin{align}
			\EXTAX,\EQAX,\REPAX \vdash \forall x\,
			(\, x \in \sigma \lrarrow x \in \alpha \wedge x \neq x\, )
			\label{fom:thm_emptyset_is_a_set_2}
		\end{align}
		が成立する.いま
		\begin{align}
			\tau \defeq \varepsilon x \negation (\, x \in s \lrarrow x \neq x\, )
		\end{align}
		とおけば,(\refeq{fom:thm_emptyset_is_a_set_2})と全称記号の論理的公理より
		\begin{align}
			\EXTAX,\EQAX,\REPAX \vdash 
			\tau \in \sigma \lrarrow \tau \in \alpha \wedge \tau \neq \tau
		\end{align}
		となるので
		\begin{align}
			\tau \in \sigma,\ \EXTAX,\EQAX,\REPAX \vdash
			\tau \in \alpha \wedge \tau \neq \tau
		\end{align}
		が従い,論理積の除去により
		\begin{align}
			\tau \in \sigma,\ \EXTAX,\EQAX,\REPAX \vdash \tau \neq \tau
		\end{align}
		が従う.以上で
		\begin{align}
			\EXTAX,\EQAX,\REPAX \vdash \tau \in \sigma \rarrow \tau \neq \tau
			\label{fom:thm_emptyset_is_a_set_3}
		\end{align}
		が得られた.逆に,定理\ref{thm:any_class_equals_to_itself}より
		\begin{align}
			\EXTAX \vdash \tau = \tau 
		\end{align}
		が成り立つので矛盾の導入と併せて
		\begin{align}
			\tau \neq \tau,\ \EXTAX,\EQAX,\REPAX \vdash \bot
		\end{align}
		となり,爆発律(論理的定理\ref{logicalthm:principle_of_explosion})より
		\begin{align}
			\tau \neq \tau,\ \EXTAX,\EQAX,\REPAX \vdash \tau \in \sigma
		\end{align}
		が従う.以上で
		\begin{align}
			\EXTAX,\EQAX,\REPAX \vdash \tau \neq \tau \rarrow \tau \in \sigma
			\label{fom:thm_emptyset_is_a_set_4}
		\end{align}
		も得られた.ゆえに(\refeq{fom:thm_emptyset_is_a_set_3})
		と(\refeq{fom:thm_emptyset_is_a_set_4})より
		\begin{align}
			\EXTAX,\EQAX,\REPAX \vdash \tau \in \sigma \lrarrow \tau \neq \tau
		\end{align}
		が成立し,全称の導出(論理的定理\ref{logicalthm:derivation_of_universal_by_epsilon})より
		\begin{align}
			\EXTAX,\EQAX,\REPAX \vdash 
			\forall x\, (\, x \in \sigma \lrarrow x \neq x\, )
			\label{fom:thm_emptyset_is_a_set_6}
		\end{align}
		が得られる.
		
		次に
		\begin{align}
			\EXTAX,\EQAX,\COMAX,\REPAX \vdash
			\forall x\, (\, x \in \sigma \lrarrow x \in \Set{x}{x \neq x}\, )
			\label{fom:thm_emptyset_is_a_set_5}
		\end{align}
		を示す.いま
		\begin{align}
			\chi \defeq x \negation (\, x \in \sigma \lrarrow x \in \Set{x}{x \neq x}\, )
		\end{align}
		とおけば,(\refeq{fom:thm_emptyset_is_a_set_6})と全称記号の論理的公理より
		\begin{align}
			\EXTAX,\EQAX,\REPAX \vdash \chi \in \sigma \lrarrow \chi \neq \chi
		\end{align}
		となり,他方で内包性公理より
		\begin{align}
			\COMAX \vdash \chi \neq \chi \lrarrow \chi \in \Set{x}{x \neq x}
		\end{align}
		が成り立つので,同値関係の推移律
		(論理的定理\ref{logicalthm:transitive_law_of_equivalence_symbol})より
		\begin{align}
			\EXTAX,\EQAX,\COMAX,\REPAX \vdash
			\chi \in \sigma \lrarrow \chi \in \Set{x}{x \neq x}
		\end{align}
		が従い,全称の導出(論理的定理\ref{logicalthm:derivation_of_universal_by_epsilon})より(\refeq{fom:thm_emptyset_is_a_set_5})が出る.
		ゆえに外延性公理より
		\begin{align}
			\EXTAX,\EQAX,\COMAX,\REPAX \vdash \sigma = \emptyset
		\end{align}
		となり,存在記号の論理的公理より
		\begin{align}
			\EXTAX,\EQAX,\COMAX,\REPAX \vdash \exists s\, (\, \emptyset = s\, )
		\end{align}
		が得られる.
		\QED
	\end{sketch}
	
	\begin{screen}
		\begin{thm}[空集合はいかなる集合も持たない]\label{thm:emptyset_has_nothing}
			\begin{align}
				\EXTAX,\COMAX \vdash \forall x\, (\, x \notin \emptyset\, ).
			\end{align}
		\end{thm}
	\end{screen}
	
	\begin{sketch}
		$\forall x\, (\, x \notin \emptyset\, )$とは
		\begin{align}
			\forall x \negation (\, x \in \emptyset\, )
		\end{align}
		の略記であり,これを$\lang{\varepsilon}$の式に書き換えると
		\begin{align}
			\forall x \negation (\, x \neq x\, )
		\end{align}
		となる.ここで
		\begin{align}
			\tau \defeq \varepsilon x \negation (\, \negation (\, x \neq x\, )\, )
		\end{align}
		とおけば,内包性公理と全称記号の論理的公理より
		\begin{align}
			\COMAX \vdash \tau \in \emptyset \rarrow \tau \neq \tau
		\end{align}
		が成り立つから,対偶を取れば
		\begin{align}
			\COMAX \vdash \tau = \tau \rarrow \tau \notin \emptyset
		\end{align}
		が成り立つ(論理的定理\ref{logicalthm:introduction_of_contraposition}).
		定理\ref{thm:any_class_equals_to_itself}より
		\begin{align}
			\EXTAX \vdash \tau = \tau
		\end{align}
		となるので,三段論法より
		\begin{align}
			\EXTAX,\COMAX \vdash \tau \notin \emptyset
		\end{align}
		が成り立つ.そして全称の導出(論理的定理\ref{logicalthm:derivation_of_universal_by_epsilon})より
		\begin{align}
			\EXTAX,\COMAX \vdash \forall x\, (\, x \notin \emptyset\, )
		\end{align}
		が得られる.
		\QED
	\end{sketch}
	
	\begin{screen}
		\begin{thm}[空のクラスは空集合に等しい]\label{thm:uniqueness_of_emptyset}
			$a$をクラスとするとき
			\begin{align}
				\EXTAX,\COMAX &\vdash \forall x\, (\, x \notin a\, ) \rarrow a = \emptyset, \\
				\EXTAX,\EQAX,\COMAX &\vdash a = \emptyset \rarrow \forall x\, (\, x \notin a\, ).
			\end{align}
		\end{thm}
	\end{screen}
	
	\begin{prf}\mbox{}
		\begin{description}
			\item[step1]
				いま
				\begin{align}
					\varepsilon x \negation (\, x \in a \lrarrow x \in \emptyset\, )
				\end{align}
				(実際には$x \in a \lrarrow x \in \emptyset$を
				$\lang{\varepsilon}$の式に書き換える)とすれば,
				\begin{align}
					\forall x\, (\, x \notin a\, ) \vdash \tau \notin a
				\end{align}
				と論理和の導入により
				\begin{align}
					\forall x\, (\, x \notin a\, ) \vdash 
					\tau \notin a \vee \tau \in \emptyset
				\end{align}
				となり,含意に書き換えれば(論理的定理
				\ref{logicalthm:disjunction_of_negation_rewritable_by_implication})
				\begin{align}
					\forall x\, (\, x \notin a\, ) \vdash 
					\tau \in a \rarrow \tau \in \emptyset
					\label{fom:uniqueness_of_emptyset_1}
				\end{align}
				が得られる.また定理\ref{thm:emptyset_has_nothing}より
				\begin{align}
					\EXTAX,\COMAX \vdash \tau \notin \emptyset
				\end{align}
				が成り立つので
				\begin{align}
					\EXTAX,\COMAX \vdash \tau \notin \emptyset \vee \tau \in a
				\end{align}
				となり,含意に書き換えれば
				\begin{align}
					\EXTAX,\COMAX \vdash \tau \in \emptyset \rarrow \tau \in a
					\label{fom:uniqueness_of_emptyset_2}
				\end{align}
				が得られる.(\refeq{fom:uniqueness_of_emptyset_1})と
				(\refeq{fom:uniqueness_of_emptyset_2})より
				\begin{align}
					\forall x\, (\, x \notin a\, ),\ \EXTAX,\COMAX \vdash
					\tau \in a \lrarrow \tau \in \emptyset
				\end{align}
				が従い,全称の導出(論理的定理\ref{logicalthm:derivation_of_universal_by_epsilon})より
				\begin{align}
					\forall x\, (\, x \notin a\, ),\ \EXTAX,\COMAX \vdash
					\forall x\, (\, x \in a \lrarrow x \in \emptyset\, )
				\end{align}
				が従い,外延性公理より
				\begin{align}
					\forall x\, (\, x \notin a\, ),\ \EXTAX,\COMAX \vdash
					a = \emptyset
				\end{align}
				が従い,演繹定理より
				\begin{align}
					\EXTAX,\COMAX \vdash
					\forall x\, (\, x \notin a\, ) \rarrow a = \emptyset
				\end{align}
				が得られる.
				
			\item[step2]
				外延性公理の逆(定理\ref{thm:inverse_of_axiom_of_extensionality})より
				\begin{align}
					a = \emptyset,\ \EQAX \vdash
					\forall x\, (\, x \in a \lrarrow x \in \emptyset\, )
					\label{fom:uniqueness_of_emptyset_3}
				\end{align}
				が成り立つ.ここで
				\begin{align}
					\tau \defeq \varepsilon x \negation (\, x \notin a\, )
				\end{align}
				とおけば(必要ならば$x \notin a$を$\lang{\varepsilon}$の式に書き換える),
				(\refeq{fom:uniqueness_of_emptyset_3})と全称記号の論理的公理より
				\begin{align}
					a = \emptyset,\ \EQAX \vdash \tau \in a \rarrow \tau \in \emptyset
				\end{align}
				が成立し,これの対偶を取れば
				\begin{align}
					a = \emptyset,\ \EQAX \vdash 
					\tau \notin \emptyset \rarrow \tau \notin a
					\label{fom:uniqueness_of_emptyset_4}
				\end{align}
				となる(論理的定理\ref{logicalthm:introduction_of_contraposition}).
				ところで定理\ref{thm:emptyset_has_nothing}より
				\begin{align}
					\EXTAX,\COMAX \vdash \tau \notin \emptyset
				\end{align}
				が成り立つので,(\refeq{fom:uniqueness_of_emptyset_4})との三段論法より
				\begin{align}
					a = \emptyset,\ \EXTAX,\EQAX,\COMAX \vdash \tau \notin a
				\end{align}
				が従う.全称の導出(論理的定理\ref{logicalthm:derivation_of_universal_by_epsilon})より
				\begin{align}
					a = \emptyset,\ \EXTAX,\EQAX,\COMAX \vdash 
					\forall x\, (\, x \notin a\, )
				\end{align}
				が成り立ち,演繹定理より
				\begin{align}
					\EXTAX,\EQAX,\COMAX \vdash 
					a = \emptyset \rarrow \forall x\, (\, x \notin a\, )
				\end{align}
				が得られる.
				\QED
		\end{description}
	\end{prf}
	
	\begin{screen}
		\begin{thm}[クラスを要素として持てば空ではない]
		\label{thm:emptyset_does_not_contain_any_class}
			$a,b$をクラスとするとき
			\begin{align}
				\EQAX,\ELEAX \vdash a \in b \rarrow \exists x\, (\, x \in b\, ).
			\end{align}
		\end{thm}
	\end{screen}
	
	\begin{prf}
		要素の公理より
		\begin{align}
			\ELEAX \vdash a \in b \rarrow \set{a}
		\end{align}
		が成立するので,
		\begin{align}
			\tau \defeq \varepsilon x\, (\, a = x\, )
		\end{align}
		とおけば(必要ならば$a = x$を$\lang{\varepsilon}$の式に書き換える),存在記号の論理的公理から
		\begin{align}
			a \in b,\ \ELEAX \vdash a = \tau
		\end{align}
		が成り立つ.相等性の公理より
		\begin{align}
			\EQAX \vdash a = \tau \rarrow (\, a \in b \rarrow \tau \in b\, )
		\end{align}
		となるので,三段論法より
		\begin{align}
			a \in b,\ \EQAX,\ELEAX \vdash \tau \in b
		\end{align}
		となる.存在記号の論理的公理より
		\begin{align}
			a \in b,\ \EQAX,\ELEAX \vdash \exists x\, (\, x \in b\, )
		\end{align}
		が成り立ち,演繹定理から
		\begin{align}
			\EQAX,\ELEAX \vdash a \in b \rarrow \exists x\, (\, x \in b\, )
		\end{align}
		が得られる.
		\QED
	\end{prf}
	
	\begin{screen}
		\begin{dfn}[部分クラス]
			$x,y$を$\mathcal{L}$の項とするとき,
			\begin{align}
				x \subset y \defarrow
				\forall z\, (\, z \in x \rarrow z \in y\, )
			\end{align}
			と定める.式$z \subset y$を「$x$は$y$の{\bf 部分クラス}\index{ぶぶんくらす@部分クラス}
			{\bf (subclass)}である」や「$x$は$y$に含まれる」などと翻訳し,特に$x$が集合である場合は
			「$x$は$y$の{\bf 部分集合}\index{ぶぶんしゅうごう@部分集合}{\bf (subset)}である」
			と翻訳する.また
			\begin{align}
				x \subsetneq y \defarrow x \subset y \wedge x \neq y
			\end{align}
			と定め,これを「$x$は$y$に{\bf 真に含まれる}」と翻訳する.
		\end{dfn}
	\end{screen}
	
	空虚な真の一例として次の結果を得る.
	
	\begin{screen}
		\begin{thm}[空集合は全てのクラスに含まれる]
		\label{thm:emptyset_if_a_subset_of_every_class}
			$a$をクラスとするとき
			\begin{align}
				\EXTAX,\COMAX \vdash \emptyset \subset a.
			\end{align}
		\end{thm}
	\end{screen}
	
	\begin{prf}
		いま
		\begin{align}
			\tau \defeq \varepsilon x \negation (\, x \in \emptyset \rarrow x \in a\, )
		\end{align}
		とおく(実際は$x \in \emptyset \rarrow x \in a$は$\lang{\varepsilon}$の式に書き換える).
		定理\ref{thm:emptyset_has_nothing}より
		\begin{align}
			\EXTAX,\COMAX \vdash \tau \notin \emptyset
		\end{align}
		が成り立つから,論理和の導入により
		\begin{align}
			\EXTAX,\COMAX \vdash \tau \notin \emptyset \vee \tau \in a
		\end{align}
		が成り立つ.これを含意の形になおせば
		\begin{align}
			\EXTAX,\COMAX \vdash \tau \in \emptyset \rarrow \tau \in a
		\end{align}
		が成り立ち
		(論理的定理\ref{logicalthm:disjunction_of_negation_rewritable_by_implication}),
		全称の導出(論理的定理\ref{logicalthm:derivation_of_universal_by_epsilon})より
		\begin{align}
			\EXTAX,\COMAX \vdash \forall x\, (\, x \in \emptyset \rarrow x \in a\, )
		\end{align}
		が従う.
		\QED
	\end{prf}
	
	$a \subset b$とは$a$に属する全ての主要$\varepsilon$項が$b$に属するという定義であったが,
	要素となりうるクラスは集合であるという公理から,$a$に属する全てのクラスもまた$b$に属する.
	
	\begin{screen}
		\begin{thm}[クラスはその部分クラスに属する全てのクラスを要素に持つ]
		\label{thm:subclass_contains_all_elements}
			$a,b,c$をクラスとするとき
			\begin{align}
				\EQAX,\ELEAX \vdash 
				a \subset b \rarrow (\, c \in a \rarrow c \in b\, ).
			\end{align}
		\end{thm}
	\end{screen}
	
	\begin{prf}	
		要素の公理より
		\begin{align}
			c \in a,\ \ELEAX \vdash \set{c}
		\end{align}
		が成り立つので,
		\begin{align}
			\tau \defeq \varepsilon x\, (\, c=x\, )
		\end{align}
		とおくと(必要ならば$c = x$を$\lang{\varepsilon}$の式に書き換える)
		存在記号の論理的公理より
		\begin{align}
			c \in a,\ \ELEAX \vdash c = \tau
			\label{fom:subclass_contains_all_elements_1}
		\end{align}
		となる.相等性公理より
		\begin{align}
			\EQAX \vdash c = \tau \rarrow (\, c \in a \rarrow \tau \in a\, )
		\end{align}
		が成り立つので,三段論法より
		\begin{align}
			c \in a,\ \EQAX,\ELEAX \vdash \tau \in a
		\end{align}
		が従う.ところで,$\subset$の定義と全称記号の論理的公理より
		\begin{align}
			a \subset b \vdash \tau \in a \rarrow \tau \in b
		\end{align}
		が成り立つので,三段論法より
		\begin{align}
			a \subset b,\ c \in a,\ \EQAX,\ELEAX \vdash \tau \in b
			\label{fom:subclass_contains_all_elements_2}
		\end{align}
		が従う.相等性公理より
		\begin{align}
			\EQAX \vdash c = \tau \rarrow \tau = c
		\end{align}
		が成り立つので,(\refeq{fom:subclass_contains_all_elements_1})との三段論法より
		\begin{align}
			a \subset b,\ c \in a,\ \EQAX,\ELEAX \vdash \tau = c
			\label{fom:subclass_contains_all_elements_3}
		\end{align}
		となり,相等性公理より
		\begin{align}
			\EQAX \vdash \tau = c \rarrow (\, \tau \in b \rarrow c \in b\, )
		\end{align}
		が成り立つので,(\refeq{fom:subclass_contains_all_elements_3})と
		(\refeq{fom:subclass_contains_all_elements_2})との三段論法より
		\begin{align}
			a \subset b,\ c \in a,\ \EQAX,\ELEAX \vdash c \in b
		\end{align}
		が従う.そして演繹定理より
		\begin{align}
			\EQAX,\ELEAX \vdash 
			a \subset b \rarrow (\, c \in a \rarrow c \in b\, )
		\end{align}
		が得られる.
		\QED
	\end{prf}
	
	宇宙$\Univ$はクラスの一つであった.当然のようであるが,それは最大のクラスである.
	\begin{screen}
		\begin{thm}[$\Univ$は最大のクラスである]
			$a$をクラスとするとき
			\begin{align}
				\EXTAX,\COMAX \vdash a \subset \Univ.
			\end{align}
		\end{thm}
	\end{screen}
	
	\begin{sketch}
		いま
		\begin{align}
			\tau \defeq \varepsilon x \negation (\, x \in a \rarrow a \in \Univ\, )
		\end{align}
		とおく.定理\ref{thm:any_class_equals_to_itself}より
		\begin{align}
			\EXTAX \vdash \tau = \tau
		\end{align}
		となり,他方で内包性公理より
		\begin{align}
			\COMAX \vdash \tau = \tau \rarrow \tau \in \Univ
		\end{align}
		が成り立つので,三段論法より
		\begin{align}
			\EXTAX,\COMAX \vdash \tau \in \Univ
		\end{align}
		が成立する.ゆえに
		\begin{align}
			\tau \in a,\ \EXTAX,\COMAX \vdash \tau \in \Univ
		\end{align}
		も成り立ち,演繹定理より
		\begin{align}
			\EXTAX,\COMAX \vdash \tau \in a \rarrow \tau \in \Univ
		\end{align}
		が従い,全称の導出(論理的定理\ref{logicalthm:derivation_of_universal_by_epsilon})より
		\begin{align}
			\EXTAX,\COMAX \vdash \forall x\, (\, x \in a \rarrow x \in \Univ\, )
		\end{align}
		が得られる.
		\QED
	\end{sketch}
	
	\begin{screen}
		\begin{thm}[等しいクラスは相手を包含する]
		\label{thm:equivalent_classes_includes_the_other}
			$a,b$をクラスとするとき
			\begin{align}
				\EQAX \vdash a = b \rarrow a \subset b \wedge b \subset a.
			\end{align}
		\end{thm}
	\end{screen}
	
	\begin{sketch}
		いま
		\begin{align}
			\tau \defeq \varepsilon x \negation (\, x \in a \rarrow x \in b\, )
		\end{align}
		とおけば,外延性公理の逆(定理\ref{thm:inverse_of_axiom_of_extensionality})と
		全称記号の論理的公理,および論理積の除去により
		\begin{align}
			a = b,\ \EQAX \vdash \tau \in a \rarrow \tau \in b
		\end{align}
		が成り立つので,全称の導出(論理的定理\ref{logicalthm:derivation_of_universal_by_epsilon})より
		\begin{align}
			a = b,\ \EQAX \vdash \forall x\, (\, x \in a \rarrow x \in b\, )
		\end{align}
		が成り立つ.また$\tau$を
		\begin{align}
			\tau \defeq \varepsilon x \negation (\, x \in b \rarrow x \in a\, )
		\end{align}
		として,先ほどの$a$と$b$を入れ替えれば
		\begin{align}
			\EQAX \vdash b = a \rarrow (\, \tau \in b \rarrow \tau \in a\, )
		\end{align}
		が得られるが,相等性公理より
		\begin{align}
			a = b,\ \EQAX \vdash b = a
		\end{align}
		が成り立つので三段論法より
		\begin{align}
			a = b,\ \EQAX \vdash \tau \in b \rarrow \tau \in a
		\end{align}
		が従う.ゆえに全称の導出(論理的定理\ref{logicalthm:derivation_of_universal_by_epsilon})より
		\begin{align}
			a = b,\ \EQAX \vdash b \subset a
		\end{align}
		も得られる.
		\QED
	\end{sketch}
	
	\begin{screen}
		\begin{thm}[互いに相手を包含するクラス同士は等しい]
		\label{thm:mutually_including_classes_are_equivalent}
			$a,b$をクラスとするとき
			\begin{align}
				\EXTAX \vdash a \subset b \wedge b \subset a \rarrow a = b.
			\end{align}
		\end{thm}
	\end{screen}
	
	\begin{sketch}
		いま
		\begin{align}
			\tau \defeq \varepsilon x \negation (\, x \in a \lrarrow x \in b\, )
		\end{align}
		とおく.論理積の除去により
		\begin{align}
			a \subset b \wedge b \subset a \vdash a \subset b
		\end{align}
		となり,全称記号の論理的公理より
		\begin{align}
			a \subset b \wedge b \subset a \vdash \tau \in a \rarrow \tau \in b
		\end{align}
		が成り立つ.同様にして
		\begin{align}
			a \subset b \wedge b \subset a \vdash \tau \in b \rarrow \tau \in a
		\end{align}
		も成り立つので,論理積の導入により
		\begin{align}
			a \subset b \wedge b \subset a \vdash \tau \in a \lrarrow \tau \in b
		\end{align}
		となり,全称の導出(論理的定理\ref{logicalthm:derivation_of_universal_by_epsilon})より
		\begin{align}
			a \subset b \wedge b \subset a \vdash 
			\forall x\, (\, x \in a \lrarrow x \in b\, )
		\end{align}
		が得られる.そして外延性公理により
		\begin{align}
			a \subset b \wedge b \subset a,\ \EXTAX \vdash a = b 
		\end{align}
		が得られる.
		\QED
	\end{sketch}
	
	%\begin{screen}
	%	\begin{thm}[集合の部分クラスは集合]
	%	\label{thm:subclass_of_set_is_set}
	%		$a$と$b$をクラスとするとき
	%		\begin{align}
	%			a \subset b \rarrow (\, \set{b} \rarrow \set{a}\, ).
	%		\end{align}
	%	\end{thm}
	%\end{screen}
	
	%\begin{sketch}\mbox{}
	%	\begin{description}
	%		\item[step1]
	%		\item[step2] $a$と$b$が主要$\varepsilon$項であるとき,分出定理
	%			(定理\ref{thm:axiom_of_separation})より
	%			\begin{align}
	%				\EXTAX,\EQAX,\REPAX \vdash \exists s\, \forall x\, (\, x \in s \lrarrow x \in a \wedge x \in b\, )
	%			\end{align}
	%			が成り立つ(分出定理の$\varphi(x)$は$x \in a$).ここで
	%			\begin{align}
	%				\sigma \defeq \varepsilon s\, \forall x\, (\, x \in s \lrarrow x \in a \wedge x \in b\, )
	%			\end{align}
	%			とおけば,存在記号の論理的公理より			
	%	\end{description}
	%\end{sketch}
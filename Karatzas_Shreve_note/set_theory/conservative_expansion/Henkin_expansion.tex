\section{Henkin拡大}
	第\ref{chap:inference}章の体系の証明は,$\lang{\varepsilon}$の文しか使われていない場合に
	{\bf HK}の証明に書き直せることを示す.そもそも第\ref{chap:inference}章における証明体系は
	次を公理とした証明体系であり,便宜上{\bf HE}と呼ぶことにする.
	{\bf HK}と違って{\bf HE}の証明に使われる式は全て文である.
	\begin{screen}
		\begin{logicalaxm}[{\bf HE}の公理(命題論理)]
			$\varphi$と$\psi$と$\xi$を$\lang{\varepsilon}$の文とするとき
			\begin{description}
				\item[(S)] $(\, \varphi \rarrow (\, \psi \rarrow \chi\, )\, ) 
					\rarrow (\, (\, \varphi \rarrow \psi\, )
					\rarrow (\, \varphi \rarrow \chi\, )\, ).$
				\item[(K)] $\varphi \rarrow (\, \psi \rarrow \varphi\, ).$
				\item[(CTD1)] $\varphi \rarrow (\, \negation \varphi \rarrow \bot\, ).$
				\item[(CTD2)] $\negation \varphi \rarrow (\, \varphi \rarrow \bot\, ).$
				\item[(DI)] $(\, \varphi \rarrow \bot\, ) \rarrow\ \negation \varphi.$
				\item[(DI1)] $\varphi \rarrow (\, \varphi \vee \psi\, ).$
				\item[(DI2)] $\psi \rarrow (\, \varphi \vee \psi\, ).$
				\item[(DE)] $(\, \varphi \rarrow \chi\, ) \rarrow 
					(\, (\, \psi \rarrow \chi\, ) 
					\rarrow (\, (\, \varphi \vee \psi) \rarrow \chi\, )\, ).$
				\item[(CI)] $\varphi \rarrow (\, \psi \rarrow (\, \varphi \wedge \psi\, )\, ).$
				\item[(CE1)] $(\, \varphi \wedge \psi\, ) \rarrow \varphi.$
				\item[(CE2)] $(\, \varphi \wedge \psi\, ) \rarrow \psi.$
				\item[(DNE)] $\negation \negation \varphi \rarrow \varphi$.
			\end{description}
		\end{logicalaxm}
	\end{screen}
	
	\begin{screen}
		\begin{logicalaxm}[{\bf HE}の公理(量化)]
			$\varphi$を$\lang{\varepsilon}の$式とし,$\tau$を主要$\varepsilon$項とし,
			$x$を変項とし,$\varphi$には$x$のみが自由に現れているとするとき
			\begin{description}
				\item[(DM)] $\negation \forall x \varphi
					\rarrow \exists x \negation \varphi.$
				
				\item[(UE)] $\forall x \varphi \rarrow \varphi(x/\tau).$
				
				\item[(EI)] $\varphi(x/\tau) \rarrow \exists x \varphi.$
				
				\item[(EE)] $\exists x \varphi \rarrow \varphi(x/\varepsilon x \varphi).$
			\end{description}
		\end{logicalaxm}
	\end{screen}
	
	第\ref{chap:inference}章での証明可能性の定義を列の概念を用いて書き直しておく.
	
	\begin{screen}
		\begin{metadfn}[{\bf HE}における証明]
			$\mathscr{S}$を$\lang{\varepsilon}$の文からなる公理系とする.
			このとき$\lang{\varepsilon}$の文の列$\varphi_{1},\varphi_{2},\cdots,
			\varphi_{n}$が$\mathscr{S}$から$\varphi_{n}$への{\bf HE}の証明であるとは,
			各$\varphi_{i}$に対して
			\begin{itemize}
				\item $\varphi_{i}$は{\bf HK}の公理である.
				\item $\varphi_{i}$は$\mathscr{S}$の公理である.
				\item $\varphi_{i}$は,これより前の式$\varphi_{j}$と$\varphi_{k}$の
					三段論法で得られる.
			\end{itemize}
			が満たされているということである.
		\end{metadfn}
	\end{screen}
	
	{\bf HE}の証明と第\ref{chap:inference}章の証明の違いは
	使われている文が$\lang{\varepsilon}$のものに限定されているか否かであるが,
	区別がつきやすいように$\lang{\varepsilon}$の文だけからなる証明が存在することを
	\begin{align}
		\provable{\mbox{{\bf HE}}}
	\end{align}
	と書くことにする.
	
	いま{\bf HK}に
	\begin{description}
		\item[存在記号の除去] 
			変項$x$のみが自由に現れる$\lang{\varepsilon}$の式$\varphi$に対して
			\begin{align}
				\exists x \varphi \rarrow \varphi(\varepsilon x \varphi)
			\end{align}
	\end{description}
	の形の公理を追加した証明体系を{\bf HK$\varepsilon$}とする.
	
	\begin{screen}
		\begin{metathm}[{\bf HE}で証明可能なら{\bf HK$\varepsilon$}でも証明可能]
		\label{metathm:Henkin_expansion_1}
			$\mathscr{S}$を$\lang{\in}$の文からなる公理系とし,
			$\psi$を$\lang{\in}$の文とするとき,$\mathscr{S} \provable{\mbox{{\bf HE}}} \psi$ならば
			$\mathscr{S} \provable{\mbox{{\bf HK$\varepsilon$}}} \psi$である.
		\end{metathm}
	\end{screen}
	
	\begin{metaprf}
		{\bf HE}の公理が{\bf HK$\varepsilon$}で証明可能であることを示せばよい.
		{\bf HE}の公理で{\bf HK$\varepsilon$}の公理でないものは
		\begin{align}
			\negation \forall x \varphi \rarrow \exists x \negation \varphi
		\end{align}
		だけであるが,これはDe Morgan の法則
		(定理\ref{classic:strong_De_Morgan_law_for_quantifier_1})より導かれる.
		\QED
	\end{metaprf}
	
	\begin{screen}
		\begin{metathm}[{\bf HK$\varepsilon$}で証明可能なら{\bf HK}でも証明可能]
		\label{metathm:Henkin_expansion_2}
			$\mathscr{S}$を$\lang{\in}$の文からなる公理系とし,$\psi$を$\lang{\in}$の文
			とするとき,$\mathscr{S} \provable{\mbox{{\bf HK$\varepsilon$}}} \psi$ならば
			$\mathscr{S} \provable{\mbox{{\bf HK}}} \psi$である.
		\end{metathm}
	\end{screen}
	
	\begin{sketch}
		$\psi$を$\lang{\in}$の文とし,
		\begin{align}
			\mathscr{S} \provable{\mbox{{\bf HK$\varepsilon$}}} \psi
		\end{align}
		であるとする.このときの証明を$\varphi_{1},\cdots,\varphi_{n}$とし,
		この中から「存在記号の除去」であるものを全て取り出して
		$\varphi_{i_{1}},\cdots\varphi_{i_{m}}$と並べる.すると,
		\begin{align}
			\mathscr{S} \cup \{\varphi_{i_{1}},\cdots,\varphi_{i_{m}}\} 
			\provable{\mbox{{\bf HK}}} \psi
		\end{align}
		ということになる.各$\varphi_{i_{j}}$は
		\begin{align}
			\exists x_{j} F_{j}(x_{j}) \rarrow F_{j}(\varepsilon x_{j} F_{j})
		\end{align}
		なる形をしているが,ここで$\varepsilon x_{m} F_{m}$は
		$\varepsilon x_{1} F_{1},\cdots,\varepsilon x_{m} F_{m}$の中で
		極大である(他の項の真部分項ではない)とする.すると
		\begin{align}
			\mathscr{S} \cup \{\varphi_{i_{1}},\cdots,\varphi_{i_{m-1}}\} 
			\provable{\mbox{{\bf HK}}} \psi
		\end{align}
		が示される.実際,
		\begin{align}
			\mathscr{S} \cup \{\varphi_{i_{1}},\cdots,\varphi_{i_{m}}\} 
			\provable{\mbox{{\bf HK}}} \psi
		\end{align}
		に対して演繹法則より
		\begin{align}
			\mathscr{S} \cup \{\varphi_{i_{1}},\cdots,\varphi_{i_{m-1}}\} 
			\provable{\mbox{{\bf HK}}} 
			(\, \exists x_{m} F_{m}(x_{m}) \rarrow F_{m}(\varepsilon x_{m} F_{m})\, ) \rarrow \psi
		\end{align}
		が成り立つので,$\mathscr{S} \cup \{\varphi_{i_{1}},\cdots,\varphi_{i_{m-1}}\}$から
		$(\, \exists x_{m} F_{m}(x_{m}) \rarrow F_{m}(\varepsilon x_{m} F_{m})\, ) \rarrow \psi$
		への証明に現れる$\varepsilon x_{m} F_{m}$を,
		この証明に使われていない変項$y$に置き換えれば
		\begin{align}
			\mathscr{S} \cup \{\varphi_{i_{1}},\cdots,\varphi_{i_{m-1}}\} 
			\provable{\mbox{{\bf HK}}} 
			(\, \exists x_{m} F_{m}(x_{m}) \rarrow F_{m}(y)\, ) \rarrow \psi
		\end{align}
		が成り立つ.すると汎化により
		\begin{align}
			\mathscr{S} \cup \{\varphi_{i_{1}},\cdots,\varphi_{i_{m-1}}\} 
			\provable{\mbox{{\bf HK}}} 
			\forall y\, (\, (\, \exists x_{m} F_{m}(x_{m}) \rarrow F_{m}(y)\, ) \rarrow \psi\, )
		\end{align}
		となり,量化規則(EE)により
		\begin{align}
			\mathscr{S} \cup \{\varphi_{i_{1}},\cdots,\varphi_{i_{m-1}}\} 
			\provable{\mbox{{\bf HK}}} 
			\exists y\, (\, \exists x_{m} F_{m}(x_{m}) \rarrow F_{m}(y)\, ) \rarrow \psi
		\end{align}
		が従う.定理\ref{classic:lemma_for_Henkin_expansion}より
		\begin{align}
			\provable{\mbox{{\bf HK}}} 
			\exists y\, (\, \exists x_{m} F_{m}(x_{m}) \rarrow F_{m}(y)\, )
		\end{align}
		が成り立つので,三段論法より
		\begin{align}
			\mathscr{S} \cup \{\varphi_{i_{1}},\cdots,\varphi_{i_{m-1}}\} 
			\provable{\mbox{{\bf HK}}} \psi
		\end{align}
		が得られる.以降も同様にして,極大な主要$\varepsilon$項が属する
		「存在記号の除去」を一本ずつ削除していけば
		\begin{align}
			\mathscr{S} \provable{\mbox{{\bf HK}}} \psi
		\end{align}
		が出る.
		\QED
	\end{sketch}
	
	第\ref{chap:inference}章の$\Sigma$の公理は$\lang{\varepsilon}$の文の集まりであったが,
	それらを$\lang{\in}$の文に直した公理体系を$\Gamma$と書く.$\Gamma$とは
	\begin{description}
		\item[集合の存在]
			\begin{align}
				\exists x\, (\, x = x\, ).
			\end{align}
		
		\item[外延性]
			\begin{align}
				\forall x\, \forall y\, (\, \forall z\, 
				(\, z \in x \lrarrow z \in y\, ) \rarrow x = y\, ).
			\end{align}
			
		\item[相等性] 
			\begin{align}
				&\forall x\, \forall y\, (\, x = y \rarrow y = x\, ), \\
				&\forall x\, \forall y\, \forall z\, 
				(\, x = y \rarrow (\, x \in z \rarrow y \in z\, )\, ), \\
				&\forall x\, \forall y\, \forall z\, 
				(\, x = y \rarrow (\, z \in x \rarrow z \in y\, )\, ).
			\end{align}
		
		\item[置換] $\varphi$を$\lang{\in}$の式とし,
			$s,t$を$\varphi$に自由に現れる変項とし,
			$x$は$\varphi$で$s$への代入について自由であり,
			$y,z$は$\varphi$で$t$への代入について自由であるとするとき,
			次の式の全称閉包\footnotemark は公理である:
			\begin{align}
				\forall x\, \forall y\, \forall z\, 
				(\, \varphi(x,y) \wedge \varphi(x,z)
				\rarrow y = z\, )
				\rarrow \forall a\, \exists z\, \forall y\,
				(\, y \in z \lrarrow \exists x\, (\, x \in a \wedge 
				\varphi(x,y)\, )\, ).
			\end{align}
			
		\item[対] 
			\begin{align}
				\forall x\, \forall y\, \exists p\, \forall z\, 
				(\, x = z \vee y = z \lrarrow z \in p\, ).
			\end{align}
			
		\item[合併] 
			\begin{align}
				\forall x\, \exists u\, \forall y\, (\, \exists z\, (\, z \in x \wedge y \in z\, ) \lrarrow y \in u\, ).
			\end{align}
			
		\item[冪] 
			\begin{align}
				\forall x\, \exists p\, \forall y\, 
				(\, \forall z\, (\, z \in y \rarrow z \in x\, ) \lrarrow y \in p\, ).
			\end{align}
			
		\item[正則性] 
			\begin{align}
				\forall r\, (\, \exists x\, (\, x \in r\, ) \rarrow
				\exists y\, (\, y \in r \wedge \forall z\, (\, z \in r \rarrow
				z \notin y\, )\, )\, ).
			\end{align}
			
		\item[無限] 
			\begin{align}
				\exists x\, (\, 
				\exists s\, (\, \forall t\, (\, t \notin s\, ) \wedge s \in x\, ) 
				\wedge \forall y\, (\, 
				y \in x \rarrow \exists u\, (\, 
				\forall v\, (\, v \in u \lrarrow v \in y \vee v = y\, )
				\wedge u \in x\, )\, )\, ).
			\end{align}
	\end{description}
	からなる.
	
	\footnotetext{
		$\varphi$を$\lang{\varepsilon}$の式とするとき,$\varphi$の{\bf 全称閉包}
		\index{ぜんしょうへいほう@全称閉包}{\bf (universal closure)}とは
		\begin{align}
			\forall x_{1}\, \cdots \forall x_{n} \varphi
		\end{align}
		なる形の文を指す.ただし$x_{1},\cdots,x_{n}$は$\varphi$に自由に現れる変項であって,
		また$\varphi$に自由に現れる変項はこれらのみであるとする.$\varphi$が文であるときは
		$\varphi$自身を全称閉包とする.
	}
	
	\begin{screen}
		\begin{metathm}[$\Sigma$の定理は$\Gamma$の定理]
		\label{metathm:Henkin_expansion_3}
			$\psi$を$\lang{\in}$の文とするとき,$\Sigma \provable{\mbox{{\bf HE}}} \psi$ならば
			$\Gamma \provable{\mbox{{\bf HE}}} \psi$である.
		\end{metathm}
	\end{screen}
	
	\begin{sketch}
		$\Sigma$の公理が$\lang{\varepsilon}$の文であるときに$\Gamma$から証明できることを示せばよい.
		$\Sigma$と$\Gamma$で形が違う公理は外延性,相等性,要素,置換である
		(内包性公理は$\lang{\varepsilon}$の式ではありえないので今回は対象外).
		\begin{description}
			\item[外延性]	$a$と$b$を主要$\varepsilon$項とするとき,{\bf HE}の公理(UE)によって
				\begin{align}
					\Gamma &\provable{\mbox{{\bf HE}}} \forall x\, \forall y\, (\, \forall z\, 
						(\, z \in x \lrarrow z \in y\, ) \rarrow x = y\, ), \\
					\Gamma &\provable{\mbox{{\bf HE}}} \forall y\, (\, \forall z\, 
						(\, z \in a \lrarrow z \in y\, ) \rarrow a = y\, ), \\
					\Gamma &\provable{\mbox{{\bf HE}}} \forall z\, 
						(\, z \in a \lrarrow z \in b\, ) \rarrow a = b
				\end{align}
				となる.$\Sigma$の相等性の公理も同様に導かれる.
				
			\item[要素] $a$と$b$を主要$\varepsilon$項とするとき,
				定理\ref{thm:any_class_equals_to_itself}と
				定理\ref{thm:critical_epsilon_term_is_set}の証明をもう一度おさらいすれば
				\begin{align}
					\sigma &\defeq \varepsilon z \negation (\, z \in a \lrarrow z \in a\, ), \\
					&\vdash \sigma \in a \lrarrow \sigma \in a, 
						&& \mbox{含意の反射律(推論法則\ref{logicalthm:reflective_law_of_implication})と論理積の導入} \\
					&\vdash \forall z\, (\, z \in a \lrarrow z \in a\, ), 
						&& \mbox{全称の導出(推論法則\ref{logicalthm:derivation_of_universal_by_epsilon})} \\
					\Gamma &\vdash \forall z\, (\, z \in a \lrarrow z \in a\, ) \rarrow a = a, 
						&& \mbox{前段の結果} \\
					\Gamma &\vdash a = a, 
						&& \mbox{三段論法} \\
					\Gamma &\vdash \exists x\, (\, a = x\, )
						&& \mbox{{\bf HE}の公理(EI)}
				\end{align}
				となる.含意の導入(K)より
				\begin{align}
					\vdash \exists x\, (\, a = x\, ) \rarrow
					(\, a \in b \rarrow \exists x\, (\, a = x\, )\, )
				\end{align}
				が成り立つので,三段論法より
				\begin{align}
					\Gamma \vdash a \in b \rarrow \exists x\, (\, a = x\, )
				\end{align}
				が従う.
				
			\item[置換]
		\end{description}
		他の公理も同様に$\mathscr{S}$から証明可能である.
		ただし内包性公理だけは$\lang{\varepsilon}$の文ではありえないので考慮する必要はない.
		\QED
	\end{sketch}
	
	\begin{screen}
		\begin{metathm}[{\bf HE}で証明可能なら{\bf HK$\varepsilon$}でも証明可能]
			$\psi$を$\lang{\in}$の文とするとき,$\Sigma \provable{\mbox{{\bf HE}}} \psi$ならば
			$\Gamma \provable{\mbox{{\bf HK}}} \psi$である.
		\end{metathm}
	\end{screen}
	
	\begin{sketch}
		$\Sigma \provable{\mbox{{\bf HE}}} \psi$ならば,
		メタ定理\ref{metathm:Henkin_expansion_3}より
		\begin{align}
			\Gamma \provable{\mbox{{\bf HE}}} \psi
		\end{align}
		となり,メタ定理\ref{metathm:Henkin_expansion_1}より
		\begin{align}
			\Gamma \provable{\mbox{{\bf HK$\varepsilon$}}} \psi
		\end{align}
		となり,メタ定理\ref{metathm:Henkin_expansion_2}より
		\begin{align}
			\Gamma \provable{\mbox{{\bf HK}}} \psi
		\end{align}
		となる.
		\QED
	\end{sketch}
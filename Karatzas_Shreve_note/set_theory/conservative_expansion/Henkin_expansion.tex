\section{Henkin拡大}
	第\ref{chap:inference}章における証明を{\bf HK}の証明に書き直す.
	第\ref{chap:inference}章における証明体系は,次を推論の公理とした証明体系{\bf HE}と
	実質的に同一である.もちろん{\bf HE}の推論規則は三段論法のみである.
	
	\begin{screen}
		\begin{logicalaxm}[{\bf HE}の公理(命題論理)]
			$\varphi$と$\psi$と$\xi$を$\lang{\varepsilon}$の文とするとき,
			次は{\bf HE}の公理である.
			\begin{description}
				\item[(S)] $(\, \varphi \rarrow (\, \psi \rarrow \chi\, )\, ) 
					\rarrow (\, (\, \varphi \rarrow \psi\, )
					\rarrow (\, \varphi \rarrow \chi\, )\, ).$
				\item[(K)] $\varphi \rarrow (\, \psi \rarrow \varphi\, ).$
				\item[(CTD1)] $\varphi \rarrow (\, \negation \varphi \rarrow \bot\, ).$
				\item[(CTD2)] $\negation \varphi \rarrow (\, \varphi \rarrow \bot\, ).$
				\item[(DI)] $(\, \varphi \rarrow \bot\, ) \rarrow\ \negation \varphi.$
				\item[(DI1)] $\varphi \rarrow (\, \varphi \vee \psi\, ).$
				\item[(DI2)] $\psi \rarrow (\, \varphi \vee \psi\, ).$
				\item[(DE)] $(\, \varphi \rarrow \chi\, ) \rarrow 
					(\, (\, \psi \rarrow \chi\, ) 
					\rarrow (\, (\, \varphi \vee \psi) \rarrow \chi\, )\, ).$
				\item[(CI)] $\varphi \rarrow (\, \psi \rarrow (\, \varphi \wedge \psi\, )\, ).$
				\item[(CE1)] $(\, \varphi \wedge \psi\, ) \rarrow \varphi.$
				\item[(CE2)] $(\, \varphi \wedge \psi\, ) \rarrow \psi.$
				\item[(DNE)] $\negation \negation \varphi \rarrow \varphi$.
			\end{description}
		\end{logicalaxm}
	\end{screen}
	
	\begin{screen}
		\begin{logicalaxm}[{\bf HE}の公理(量化)]
			$\varphi$を$\lang{\varepsilon}の$式とし,$t$を主要$\varepsilon$項とし,
			$x$を変項とし,$\varphi$には$x$のみが自由に現れているとする.
			このとき次は{\bf HE}の公理である.
			\begin{description}
				\item[(UI)] $\varphi(x/\varepsilon x \negation \varphi)
					\rarrow \forall x \varphi.$
				
				\item[(UE)] $\forall x \varphi \rarrow \varphi(x/t).$
				
				\item[(EI)] $\varphi(x/t) \rarrow \exists x \varphi.$
				
				\item[(EE)] $\exists x \varphi \rarrow \varphi(x/\varepsilon x \varphi).$
			\end{description}
		\end{logicalaxm}
	\end{screen}
	
	証明を書き直す手順は次のとおりである.
	
	\begin{description}
		\item[step1]
			まずは{\bf HK}の公理に
			\begin{description}
				\item[存在記号の除去] 
					変項$x$のみが自由に現れる$\lang{\varepsilon}$の式$\varphi$に対して
					\begin{align}
						\exists x \varphi \rarrow \varphi(\varepsilon x \varphi)
					\end{align}
			\end{description}
			を追加した証明体系を{\bf HK$\varepsilon$}とし,第\ref{chap:inference}章における
			証明を{\bf HK$\varepsilon$}の証明に書き直す.
			
		\item[step2]
			{\bf HK}${}_\varepsilon$の証明から{\bf HK}の証明を構成する.
			
		\item[step3]
			step2で得られた証明に残っている主要$\varepsilon$項を何らかの変項に置き換えれば,
			$\lang{\in}$の式による{\bf HK}の証明が得られる.
	\end{description}
	
	第\ref{chap:inference}章の$\Sigma$の公理は$\lang{\varepsilon}$の文の集まりであったが,
	それらを$\lang{\in}$の文に直した公理体系を$\mathscr{S}$と書く.$\mathscr{S}$とは
	\begin{description}
		\item[相等性] 
			\begin{align}
				&\forall x\, \forall y\, \forall z\, 
				(\, x = y \rarrow (\, x \in z \rarrow y \in z\, )\, ), \\
				&\forall x\, \forall y\, \forall z\, 
				(\, x = y \rarrow (\, z \in x \rarrow z \in y\, )\, ).
			\end{align}
			
		\item[外延性] $\forall x\, \forall y\, (\, \forall z\, 
			(\, z \in x \lrarrow z \in y\, ) \rarrow x = y\, ).$
			
		\item[対] $\forall x\, \forall y\, \exists p\, \forall z\, 
			(\, x = z \vee y = z \lrarrow z \in p\, ).$
			
		\item[合併] $\forall x\, \exists u\, \forall y\, (\, \exists z\, (\, z \in x \wedge y \in z\, ) \lrarrow y \in u\, ).$
			
		\item[冪] $\forall x\, \exists p\, \forall y\, 
			(\, \forall z\, (\, z \in y \rarrow z \in x\, ) \lrarrow y \in p\, ).$
			
		\item[置換] $\varphi$を$\lang{\in}$の式とし,
			$s,t$を$\varphi$に自由に現れる変項とし,
			$\varphi$に自由に現れる項は$s,t$のみであるとし,
			$x$は$\varphi$で$s$への代入について自由であり,
			$y,z$は$\varphi$で$t$への代入について自由であるとするとき,
			\begin{align}
				\forall x\, \forall y\, \forall z\, 
				(\, \varphi(x,y) \wedge \varphi(x,z)
				\rarrow y = z\, )
				\rarrow \forall a\, \exists z\, \forall y\,
				(\, y \in z \lrarrow \exists x\, (\, x \in a \wedge 
				\varphi(x,y)\, )\, ).
			\end{align}
			
		\item[正則性] $\forall a\, (\, \exists x\, (\, x \in a\, ) \rarrow
				\exists y\, (\, y \in a \wedge \forall z\, (\, z \in y \rarrow
				z \notin a\, )\, )\, ).$
			
		\item[無限] $\exists x\, (\, 
				\exists s\, (\, \forall t\, (\, t \notin s\, ) \wedge s \in x\, ) 
				\wedge \forall y\, (\, 
				y \in x \rarrow \exists u\, (\, 
				\forall v\, (\, v \in u \lrarrow v \in y \vee v = y\, )
				\wedge u \in x\, )\, )\, ).$
			
		\item[選択]
			
	\end{description}
	である.
	
	\begin{description}
		\item[step0]
			step1に入る前に公理系を$\mathscr{S}$の形式に揃える.
			いま$\psi$を$\lang{\in}$の文とし,
			\begin{align}
				\Sigma \vdash \psi
			\end{align}
			であるとする.このとき
			\begin{align}
				\mathscr{S} \vdash \psi
			\end{align}
			である.これには$\Sigma$の公理が$\lang{\varepsilon}$の文であれば
			$\mathscr{S}$から証明可能であることを示せばよい.
			たとえば外延性公理については,どんな主要$\varepsilon$項$a,b$に対しても
			全称記号の推論規則によって
			\begin{align}
				\mathscr{S} &\vdash \forall x\, \forall y\, (\, \forall z\, 
					(\, z \in x \lrarrow z \in y\, ) \rarrow x = y\, ), \\
				\mathscr{S} &\vdash \forall y\, (\, \forall z\, 
					(\, z \in a \lrarrow z \in y\, ) \rarrow a = y\, ), \\
				\mathscr{S} &\vdash \forall z\, 
					(\, z \in a \lrarrow z \in b\, ) \rarrow a = b
			\end{align}
			となる.他の公理も同様に$\mathscr{S}$から証明可能である.
			ただし内包性公理だけは$\lang{\varepsilon}$の文ではありえないので考慮する必要はない.
			
		\item[step1]
			$\psi$を$\lang{\in}$の文とし,
			\begin{align}
				\mathscr{S} \vdash \psi
			\end{align}
			であるとする.このとき
			\begin{align}
				\mathscr{S} \provable{\mbox{{\bf HK$\varepsilon$}}} \psi
			\end{align}
			であることを示す.第\ref{chap:inference}章の推論規則であって
			{\bf HK$\varepsilon$}の公理でないものは
			\begin{align}
				\varphi(\varepsilon x \negation \varphi) \rarrow \forall x \varphi
			\end{align}
			の形の文のみであるから,これが{\bf HK$\varepsilon$}から導かれることを示せばよい.
			いま$\varphi$を$\lang{\varepsilon}$の式とし,$\varphi$には変項$x$のみが
			自由に現れているとする.
			このとき
			\begin{align}
				\provable{\mbox{{\bf HK$\varepsilon$}}}
				\ \negation \forall x \varphi \rarrow \exists x \negation \varphi
			\end{align}
			と
			\begin{align}
				\provable{\mbox{{\bf HK$\varepsilon$}}}
				\exists x \negation \varphi \rarrow\ 
				\negation \varphi(\varepsilon x \negation \varphi)
			\end{align}
			が成り立つので
			\begin{align}
				\provable{\mbox{{\bf HK$\varepsilon$}}}
				\ \negation \forall x \varphi \rarrow\ 
				\negation \varphi(\varepsilon x \negation \varphi)
			\end{align}
			が従い,対偶論法より
			\begin{align}
				\provable{\mbox{{\bf HK$\varepsilon$}}}
				\varphi(\varepsilon x \negation \varphi) \rarrow \forall x \varphi
			\end{align}
			が得られる.
			
		\item[step2]
			$\psi$を$\lang{\in}$の文とし,
			\begin{align}
				\mathscr{S} \provable{\mbox{{\bf HK$\varepsilon$}}} \psi
			\end{align}
			であるとする.このときの証明を$\varphi_{1},\cdots,\varphi_{n}$とし,
			この中から「存在記号の除去」であるものを全て取り出して
			$\varphi_{i_{1}},\cdots\varphi_{i_{m}}$と並べる.すると,
			\begin{align}
				\mathscr{S} \cup \{\varphi_{i_{1}},\cdots,\varphi_{i_{m}}\} 
				\provable{\mbox{{\bf HK}}} \psi
			\end{align}
			ということになる.各$\varphi_{i_{j}}$は
			\begin{align}
				\exists x_{j} F_{j}(x_{j}) \rarrow F_{j}(\varepsilon x_{j} F_{j})
			\end{align}
			なる形をしているが,ここで$\varepsilon x_{m} F_{m}$は
			$\varepsilon x_{1} F_{1},\cdots,\varepsilon x_{m} F_{m}$の中で
			極大である(他の項の真部分項ではない)とする.すると
			\begin{align}
				\mathscr{S} \cup \{\varphi_{i_{1}},\cdots,\varphi_{i_{m-1}}\} 
				\provable{\mbox{{\bf HK}}} \psi
			\end{align}
			が示される.実際,
			\begin{align}
				\mathscr{S} \cup \{\varphi_{i_{1}},\cdots,\varphi_{i_{m}}\} 
				\provable{\mbox{{\bf HK}}} \psi
			\end{align}
			に対して演繹法則より
			\begin{align}
				\mathscr{S} \cup \{\varphi_{i_{1}},\cdots,\varphi_{i_{m-1}}\} 
				\provable{\mbox{{\bf HK}}} 
				(\, \exists x_{m} F_{m}(x_{m}) \rarrow F_{m}(\varepsilon x_{m} F_{m})\, ) \rarrow \psi
			\end{align}
			が成り立つので,$\mathscr{S} \cup \{\varphi_{i_{1}},\cdots,\varphi_{i_{m-1}}\}$から
			$(\, \exists x_{m} F_{m}(x_{m}) \rarrow F_{m}(\varepsilon x_{m} F_{m})\, ) \rarrow \psi$
			への証明に現れる$\varepsilon x_{m} F_{m}$を,
			この証明に使われていない変項$y$に置き換えれば
			\begin{align}
				\mathscr{S} \cup \{\varphi_{i_{1}},\cdots,\varphi_{i_{m-1}}\} 
				\provable{\mbox{{\bf HK}}} 
				(\, \exists x_{m} F_{m}(x_{m}) \rarrow F_{m}(y)\, ) \rarrow \psi
			\end{align}
			が成り立つ.すると汎化により
			\begin{align}
				\mathscr{S} \cup \{\varphi_{i_{1}},\cdots,\varphi_{i_{m-1}}\} 
				\provable{\mbox{{\bf HK}}} 
				\forall y\, (\, (\, \exists x_{m} F_{m}(x_{m}) \rarrow F_{m}(y)\, ) \rarrow \psi\, )
			\end{align}
			となり,量化規則(EE)により
			\begin{align}
				\mathscr{S} \cup \{\varphi_{i_{1}},\cdots,\varphi_{i_{m-1}}\} 
				\provable{\mbox{{\bf HK}}} 
				\exists y\, (\, \exists x_{m} F_{m}(x_{m}) \rarrow F_{m}(y)\, ) \rarrow \psi
			\end{align}
			が従う.他方で
			\begin{align}
				\provable{\mbox{{\bf HK}}} \exists x_{m} F_{m}(x_{m}) 
				\rarrow \exists y F_{m}(y)
			\end{align}
			と
			\begin{align}
				\provable{\mbox{{\bf HK}}} 
				(\, \exists x_{m} F_{m}(x_{m}) \rarrow \exists y F_{m}(y)\, )
				\rarrow \exists y\, (\, \exists x_{m} F_{m}(x_{m}) \rarrow F_{m}(y)\, )
			\end{align}
			が成り立つので,三段論法より
			\begin{align}
				\mathscr{S} \cup \{\varphi_{i_{1}},\cdots,\varphi_{i_{m-1}}\} 
				\provable{\mbox{{\bf HK}}} \psi
			\end{align}
			が得られる.以降も同様にして,極大な主要$\varepsilon$項が属する
			「存在記号の除去」を一本ずつ削除していけば
			\begin{align}
				\mathscr{S} \provable{\mbox{{\bf HK}}} \psi
			\end{align}
			が出る.
			\QED
	\end{description}
\section{Henkin拡大}
\label{sec:Henkin_expansion}
	この節では「$\Sigma$から$\psi$への{\bf HE}の証明で$\lang{\varepsilon}$の文の列
	であるものが取れる」ならば「$\Gamma$から$\psi$への{\bf HK}の証明で$\lang{\in}$の
	式の列であるものが取れる」ことを示す.{\bf HE}の証明に使われる式は全て\underline{文}である.
	
	\begin{screen}
		\begin{logicalaxm}[{\bf HE}の公理(命題論理)]
			$\varphi$と$\psi$と$\xi$を文とするとき
			\begin{description}
				\item[(S)] $(\, \varphi \rarrow (\, \psi \rarrow \chi\, )\, ) 
					\rarrow (\, (\, \varphi \rarrow \psi\, )
					\rarrow (\, \varphi \rarrow \chi\, )\, ).$
				\item[(K)] $\varphi \rarrow (\, \psi \rarrow \varphi\, ).$
				\item[(CTD1)] $\varphi \rarrow (\, \negation \varphi \rarrow \bot\, ).$
				\item[(CTD2)] $\negation \varphi \rarrow (\, \varphi \rarrow \bot\, ).$
				\item[(DI)] $(\, \varphi \rarrow \bot\, ) \rarrow\ \negation \varphi.$
				\item[(DI1)] $\varphi \rarrow (\, \varphi \vee \psi\, ).$
				\item[(DI2)] $\psi \rarrow (\, \varphi \vee \psi\, ).$
				\item[(DE)] $(\, \varphi \rarrow \chi\, ) \rarrow 
					(\, (\, \psi \rarrow \chi\, ) 
					\rarrow (\, (\, \varphi \vee \psi) \rarrow \chi\, )\, ).$
				\item[(CI)] $\varphi \rarrow (\, \psi \rarrow (\, \varphi \wedge \psi\, )\, ).$
				\item[(CE1)] $(\, \varphi \wedge \psi\, ) \rarrow \varphi.$
				\item[(CE2)] $(\, \varphi \wedge \psi\, ) \rarrow \psi.$
				\item[(DNE)] $\negation \negation \varphi \rarrow \varphi$.
			\end{description}
		\end{logicalaxm}
	\end{screen}
	
	\begin{screen}
		\begin{logicalaxm}[{\bf HE}の公理(量化)]
			$\varphi$を式とし,$\tau$を主要$\varepsilon$項とし,
			$x$を変項とし,$\varphi$には$x$のみが自由に現れているとするとき
			\begin{description}
				\item[(DM)] $\negation \forall x \varphi
					\rarrow \exists x \negation \varphi.$
				
				\item[(UE)] $\forall x \varphi \rarrow \varphi(x/\tau).$
				
				\item[(EI)] $\varphi(x/\tau) \rarrow \exists x \varphi.$
				
				\item[(EE)] $\hat{\varphi}$を,$\varphi$が$\lang{\varepsilon}$の式でないときは
					$\varphi$を$\lang{\varepsilon}$の式に書き直したものとし,
					$\varphi$が$\lang{\varepsilon}$の式であるときは$\varphi$そのものとする.このとき
					\begin{align}
						\exists x \varphi \rarrow \varphi(x/\varepsilon x \hat{\varphi}).
					\end{align}
			\end{description}
		\end{logicalaxm}
	\end{screen}
	
	第\ref{chap:inference}章での証明可能性の定義を列の概念を用いて書き直しておく.
	
	\begin{screen}
		\begin{metadfn}[{\bf HE}における証明]
			$\mathscr{S}$を文からなる公理系とする.このとき文の列$\varphi_{1},\varphi_{2},\cdots,
			\varphi_{n}$が$\mathscr{S}$から$\varphi_{n}$への{\bf HE}の証明であるとは,
			各$\varphi_{i}$が次のいずれかであるということである:
			\begin{itemize}
				\item $\varphi_{i}$は{\bf HE}の公理である.
				\item $\varphi_{i}$は$\mathscr{S}$の公理である.
				\item $\varphi_{i}$は,これより前の式$\varphi_{j}$と$\varphi_{k}$の
					三段論法で得られる.
			\end{itemize}
		\end{metadfn}
	\end{screen}
	
	$\psi$を文とし,$\mathscr{S}$を公理系とするとき,$\mathscr{S}$から$\psi$への
	{\bf HE}の証明で$\lang{\varepsilon}$の文の列であるものが取れることを
	\begin{align}
		\mathscr{S} \provable{\mbox{{\bf HE}},\lang{\varepsilon}} \psi
	\end{align}
	と書く.他方で$\mathscr{S}$から$\psi$への{\bf HE}の証明で$\mathcal{L}$の文の列であるものが
	取れることは,第\ref{chap:inference}章の証明可能性と同義であるから
	\begin{align}
		\mathscr{S} \vdash \psi
	\end{align}
	と書く.
	
	いま{\bf HK}の公理に{\bf HE}の(EE)を追加した証明体系を{\bf HK$\varepsilon$}とする.
	$\mathscr{T}$を公理系とするとき,$\mathscr{T}$に
	{\bf HE}の(EE)を追加した公理系を$\mathscr{T}$の{\bf Henkin拡大}
	\index{Henkinかくだい@Henkin拡大}{\bf (Henkin extension)}と呼ぶが,
	今回は{\bf HK}の公理に追加しているのでHenkin拡大の一種と見ることが出来る.
	Henkin拡大とはすなわち,全ての存在文に証人を付けるための拡大である.
	公理系$\mathscr{S}$から文$\psi$への{\bf HK$\varepsilon$}の証明で
	$\lang{\varepsilon}$の式の列であるものが取れることを
	\begin{align}
		\mathscr{S} \provable{\mbox{{\bf HK}$\varepsilon$},\lang{\varepsilon}} \psi
	\end{align}
	と書く.
	
	\begin{screen}
		\begin{metathm}[{\bf HE}で証明可能なら{\bf HK$\varepsilon$}でも証明可能]
		\label{metathm:Henkin_expansion_1}
			$\mathscr{S}$を$\lang{\in}$の文からなる公理系とし,$\psi$を$\lang{\in}$の文とする.
			このとき$\mathscr{S} \provable{\mbox{{\bf HE}},\lang{\varepsilon}} \psi$ならば
			$\mathscr{S} \provable{\mbox{{\bf HK}$\varepsilon$},\lang{\varepsilon}} \psi$
			である.
		\end{metathm}
	\end{screen}
	
	\begin{metaprf}
		{\bf HE}の公理で{\bf HK$\varepsilon$}の公理でないものは
		\begin{align}
			\negation \forall x \varphi \rarrow \exists x \negation \varphi
		\end{align}
		だけであるが,これはDe Morgan の法則
		(定理\ref{classic:weak_De_Morgan_law_for_quantifier_1})より導かれる.
		従って,証明の中に{\bf HE}の公理(DM)があれば,それより前の列に(DM)への
		{\bf HK}の証明を挿入すれば,$\mathscr{S}$から$\psi$への{\bf HK$\varepsilon$}の証明になる.
		\QED
	\end{metaprf}
	
	\begin{screen}
		\begin{metathm}[$\varepsilon$項を変項に取り替えても証明]
		\label{metathm:Henkin_expansion_2_lemma}
			$\mathscr{S}$を$\lang{\in}$の文からなる公理系とし,$\psi$を$\lang{\in}$の文とし,
			$\varphi_{1},\cdots,\varphi_{n}$を$\mathscr{S}$から$\psi$への
			{\bf HK}の証明で$\lang{\varepsilon}$の式の列であるものとする.
			また$e$をこの列の式に現れる主要$\varepsilon$項とし,$y$を
			この証明に現れない変項とする.そして各$\varphi_{i}$に対して
			そこに現れる$e$を全て$y$に取り替えた式を$\hat{\varphi}_{i}$と書く.
			ただし取り替えるのは他の項の真部分項になっていない$e$のみであり,
			$\varphi_{i}$に$e$が現れなければ$\hat{\varphi}_{i}$は$\varphi_{i}$とする.
			このとき
			\begin{align}
				\hat{\varphi}_{1},\ \cdots,\ \hat{\varphi}_{n}
			\end{align}
			は$\mathscr{S}$から$\psi$への{\bf HK}の証明である.
		\end{metathm}
	\end{screen}
	
	\begin{metaprf}\mbox{}
		\begin{description}
			\item[case1] $\varphi_{i}$が{\bf HK}の命題論理の公理であるとき,
				たとえば$\varphi_{i}$が
				\begin{align}
					\varphi \rarrow (\, \chi \rarrow \varphi\, )
				\end{align}
				なる式なら,$\hat{\varphi}_{i}$も
				\begin{align}
					\hat{\varphi} \rarrow (\, \hat{\chi} \rarrow \hat{\varphi}\, )
				\end{align}
				なる形の式となるので{\bf HK}の公理である.他の場合も同様である.
				
			\item[case2] $\varphi_{i}$が{\bf HK}の量化公理であるとき,つまり
				\begin{align}
					&\forall z\, (\, \chi \rarrow \varphi(x/z)\, ) 
						\rarrow (\, \chi \rarrow \forall x \varphi\, ), \\
					&\forall x \varphi \rarrow \varphi(x/t), \\
					&\varphi(x/t) \rarrow \exists x \varphi, \\
					&\forall z\, (\, \varphi(x/z) \rarrow \chi\, )
						\rarrow (\, \exists x \varphi \rarrow \chi\, )
				\end{align}
				のいずれかであるとき,$\hat{\varphi}_{i}$は
				\begin{align}
					&\forall z\, (\, \hat{\chi} \rarrow \hat{\varphi}(x/z)\, ) 
						\rarrow (\, \hat{\chi} \rarrow \forall x \hat{\varphi}\, ), \\
					&\forall x \hat{\varphi} \rarrow \hat{\varphi}(x/t), \\
					&\hat{\varphi}(x/t) \rarrow \exists x \hat{\varphi}, \\
					&\forall z\, (\, \hat{\varphi}(x/z) \rarrow \hat{\chi}\, )
						\rarrow (\, \exists x \hat{\varphi} \rarrow \hat{\chi}\, )
				\end{align}
				なる形の式となり,{\bf HK}の公理であるための変項の条件も満たされる.
				ここで注意しておくと,$e$が$\varphi(x/t)$に現れる場合,
				$\varphi$で$x$に代入された$t$を含むようには$e$は現れない.もし
				そのような$t$が$e$に現れたら,$e$のその$t$を$x$に置き換えた$\varepsilon$項$e'$は,
				$\varphi$すなわち$\exists x \varphi$の中に現れることになるが,$e'$は
				主要$\varepsilon$項ではないので第\ref{sec:restriction_of_formulas}節の
				約束に違反してしまう.従って$\varphi(x/t)$に現れる$e$を$y$に置き換えた式は
				$\hat{\varphi}(x/t)$なる形で書けるのである.
				
			\item[case3] $\varphi_{i}$が$\mathscr{S}$の公理であるときは
				$\lang{\in}$の文なので,$\hat{\varphi}_{i}$は$\varphi_{i}$である.
				
			\item[case4] $\varphi_{i}$が前の式$\varphi_{j},\varphi_{k}$から
				三段論法で得られているとき,$\varphi_{k}$が$\varphi_{j} \rarrow \varphi_{i}$
				なる式なら$\hat{\varphi}_{k}$は
				\begin{align}
					\hat{\varphi}_{j} \rarrow \hat{\varphi}_{i}
				\end{align}
				なる式であるから,$\hat{\varphi}_{i}$は$\hat{\varphi}_{j}$と
				$\hat{\varphi}_{k}$から三段論法で得られる.
				
			\item[case5] $\varphi_{i}$が前の式$\varphi_{j}$から
				汎化で得られているとき,つまり変項$a,x$と$x$が自由に現れる式$\chi$が取れて,
				$\varphi_{j}$が$\chi(x/a)$で$\varphi_{i}$が$\forall x \chi$であるとき,
				$\hat{\varphi}_{j}$は
				\begin{align}
					\hat{\chi}(x/a)
				\end{align}
				なる式で($e$は$x$に代入された$a$を含むようには現れない)
				$\hat{\varphi}_{i}$は
				\begin{align}
					\forall x \hat{\chi}
				\end{align}
				なる式であるから,$\hat{\varphi}_{i}$は$\hat{\varphi}_{j}$から汎化で得られる.
				\QED
		\end{description}
	\end{metaprf}
	
	\begin{screen}
		\begin{metathm}[{\bf HK$\varepsilon$}で証明可能なら{\bf HK}でも証明可能]
		\label{metathm:Henkin_expansion_2}
			$\mathscr{S}$を$\lang{\in}$の文からなる公理系とし,$\psi$を$\lang{\in}$の文とするとき,
			$\mathscr{S} \provable{\mbox{{\bf HK}$\varepsilon$},\lang{\varepsilon}} \psi$
			ならば$\mathscr{S} \provable{\mbox{{\bf HK}},\lang{\varepsilon}} \psi$である.
		\end{metathm}
	\end{screen}
	
	\begin{sketch}
		$\varphi_{1},\cdots,\varphi_{n}$の中から{\bf HE}の(EE)であるものを全て取り出して
		$\varphi_{i_{1}},\cdots\varphi_{i_{m}}$と並べれば,
		$\varphi_{1},\cdots,\varphi_{n}$は公理系
		$\varphi_{i_{1}},\cdots,\varphi_{i_{m}},\mathscr{S}$から$\psi$への
		{\bf HK}の証明となる.各$\varphi_{i_{j}}$は
		\begin{align}
			\exists x_{j} F_{j}(x_{j}) \rarrow F_{j}(\varepsilon x_{j} F_{j})
		\end{align}
		なる形をしている.ここで$\varepsilon x_{m} F_{m}$は$F_{1},\cdots,F_{m-1}$の中には
		現れないとすると
		\footnote{
			このような項$\varepsilon x_{m} F_{m}$は必ず取れる.たとえば
			$\varepsilon x_{i} F_{i}$が$F_{j}$に現れたら,
			$\varepsilon x_{j} F_{j}$は$F_{i}$には現れない.
			実際$F_{i}$に現れたら$\varepsilon x_{i} F_{i}$が$F_{i}$に現れることになるが,
			$F_{i}$より長い$\varepsilon x_{i} F_{i}$が$F_{i}$に現れることなどあり得ない.
			同様に,$\varepsilon x_{j} F_{j}$が$F_{k}$に現れたら,
			$\varepsilon x_{k} F_{k}$は$F_{i}$と$F_{j}$には現れない.この確認を繰り返せばよい.
		}
		\begin{align}
			\varphi_{i_{1}},\cdots,\varphi_{i_{m-1}},\mathscr{S} 
			\provable{\mbox{{\bf HK}}} \psi
		\end{align}
		が示される.実際,{\bf HK}の演繹定理より
		\begin{align}
			\varphi_{i_{1}},\cdots,\varphi_{i_{m-1}},\mathscr{S} 
			\provable{\mbox{{\bf HK}}} 
			(\, \exists x_{m} F_{m}(x_{m}) \rarrow F_{m}(\varepsilon x_{m} F_{m})\, ) \rarrow \psi
		\end{align}
		が成り立つが,このときの$\varphi_{i_{1}},\cdots,\varphi_{i_{m-1}},\mathscr{S}$から
		$(\, \exists x_{m} F_{m}(x_{m}) \rarrow F_{m}(\varepsilon x_{m} F_{m})\, ) 
		\rarrow \psi$への証明に現れる$\varepsilon x_{m} F_{m}$を,
		その証明に使われていない変項$y$に置き換えれば
		\footnote{
			置き換える$\varepsilon x_{m} F_{m}$は他の項の真部分項になっていない
			箇所のものだけである.また$\varepsilon x_{m} F_{m}$は
			$F_{1},\cdots,F_{m-1}$の中には現れないので
			$\varphi_{i_{1}},\cdots,\varphi_{i_{m-1}}$の中にも現れない.ここで注意しておくと,
			たとえば$\varepsilon x_{m} F_{m}$が$F_{1}(\varepsilon x_{1} F_{1})$に
			現れたとすると,その$\varepsilon x_{m} F_{m}$の中の$\varepsilon x_{1} F_{1}$を
			$x_{1}$に置き換えた$\varepsilon$項が$F_{1}$に現れることになる.しかし
			その$\varepsilon$項は主要$\varepsilon$項ではなく,
			第\ref{sec:restriction_of_formulas}節の約束に違反してしまう.
			ゆえに$\varepsilon x_{m} F_{m}$は$\varphi_{i_{1}},\cdots,\varphi_{i_{m-1}}$
			の中に現れず,これらの式は置換による影響を受けない.
			$\mathscr{S}$の公理も$\lang{\in}$の文であるから置換による影響を受けない.
		},それで得られる式の列は$\varphi_{i_{1}},\cdots,\varphi_{i_{m-1}},\mathscr{S}$
		から$(\, \exists x_{m} F_{m}(x_{m}) \rarrow F_{m}(y)\, ) \rarrow \psi$への
		{\bf HK}の証明となる(メタ定理\ref{metathm:Henkin_expansion_2_lemma}).つまり
		\begin{align}
			\varphi_{i_{1}},\cdots,\varphi_{i_{m-1}},\mathscr{S} 
			\provable{\mbox{{\bf HK}}} 
			(\, \exists x_{m} F_{m}(x_{m}) \rarrow F_{m}(y)\, ) \rarrow \psi
		\end{align}
		が成り立つ.すると汎化により
		\begin{align}
			\varphi_{i_{1}},\cdots,\varphi_{i_{m-1}},\mathscr{S} 
			\provable{\mbox{{\bf HK}}} 
			\forall y\, (\, (\, \exists x_{m} F_{m}(x_{m}) \rarrow F_{m}(y)\, ) \rarrow \psi\, )
		\end{align}
		となり,{\bf HK}の公理(EE)により
		\begin{align}
			\varphi_{i_{1}},\cdots,\varphi_{i_{m-1}},\mathscr{S} 
			\provable{\mbox{{\bf HK}}} 
			\exists y\, (\, \exists x_{m} F_{m}(x_{m}) \rarrow F_{m}(y)\, ) \rarrow \psi
		\end{align}
		が従う.定理\ref{classic:lemma_for_Henkin_expansion}より
		\begin{align}
			\provable{\mbox{{\bf HK}}} 
			\exists y\, (\, \exists x_{m} F_{m}(x_{m}) \rarrow F_{m}(y)\, )
		\end{align}
		が成り立つので,三段論法より
		\begin{align}
			\varphi_{i_{1}},\cdots,\varphi_{i_{m-1}},\mathscr{S} 
			\provable{\mbox{{\bf HK}}} \psi
		\end{align}
		が従う.以降も同様にして{\bf HK}の公理(EE)を一本ずつ削除していけば,
		$\mathscr{S}$から$\psi$への{\bf HK}の証明で
		$\lang{\varepsilon}$の式の列であるものが得られる.
		\QED
	\end{sketch}
	
	第\ref{chap:inference}章の$\Sigma$の公理は$\lang{\varepsilon}$の文の集まりであったが,
	それらを$\lang{\in}$の文に直した公理体系を$\Gamma$と書く.$\Gamma$は次の文からなる.
	\begin{description}
	\label{axioms_of_Gamma}
		\item[集合の存在]
			\begin{align}
				\exists x\, (\, x = x\, ).
			\end{align}
		
		\item[外延性]
			\begin{align}
				\forall x\, \forall y\, (\, \forall z\, 
				(\, z \in x \lrarrow z \in y\, ) \rarrow x = y\, ).
			\end{align}
			
		\item[相等性] 
			\begin{align}
				&\forall x\, \forall y\, (\, x = y \rarrow y = x\, ), \\
				&\forall x\, \forall y\, \forall z\, 
				(\, x = y \rarrow (\, x \in z \rarrow y \in z\, )\, ), \\
				&\forall x\, \forall y\, \forall z\, 
				(\, x = y \rarrow (\, z \in x \rarrow z \in y\, )\, ).
			\end{align}
		
		\item[置換] $\varphi$を$\lang{\in}$の式とし,
			$s,t$を$\varphi$に自由に現れる変項とし,
			$x$は$\varphi$で$s$への代入について自由であり,
			$y,z$は$\varphi$で$t$への代入について自由であるとするとき,
			次の式の全称閉包\footnotemark は公理である:
			\begin{align}
				\forall x\, \forall y\, \forall z\, 
				(\, \varphi(x,y) \wedge \varphi(x,z)
				\rarrow y = z\, )
				\rarrow \forall a\, \exists z\, \forall y\,
				(\, y \in z \lrarrow \exists x\, (\, x \in a \wedge 
				\varphi(x,y)\, )\, ).
			\end{align}
			
		\item[対] 
			\begin{align}
				\forall x\, \forall y\, \exists p\, \forall z\, 
				(\, x = z \vee y = z \lrarrow z \in p\, ).
			\end{align}
			
		\item[合併] 
			\begin{align}
				\forall x\, \exists u\, \forall y\, (\, \exists z\, (\, z \in x \wedge y \in z\, ) \lrarrow y \in u\, ).
			\end{align}
			
		\item[冪] 
			\begin{align}
				\forall x\, \exists p\, \forall y\, 
				(\, \forall z\, (\, z \in y \rarrow z \in x\, ) \lrarrow y \in p\, ).
			\end{align}
			
		\item[正則性] 
			\begin{align}
				\forall r\, (\, \exists x\, (\, x \in r\, ) \rarrow
				\exists y\, (\, y \in r \wedge \forall z\, (\, z \in r \rarrow
				z \notin y\, )\, )\, ).
			\end{align}
			
		\item[無限] 
			\begin{align}
				\exists x\, (\, 
				\exists s\, (\, \forall t\, (\, t \notin s\, ) \wedge s \in x\, ) 
				\wedge \forall y\, (\, 
				y \in x \rarrow \exists u\, (\, 
				\forall v\, (\, v \in u \lrarrow v \in y \vee v = y\, )
				\wedge u \in x\, )\, )\, ).
			\end{align}
	\end{description}
	
	\footnotetext{
		$\varphi$を$\lang{\varepsilon}$の式とするとき,$\varphi$の{\bf 全称閉包}
		\index{ぜんしょうへいほう@全称閉包}{\bf (universal closure)}とは
		\begin{align}
			\forall x_{1}\, \cdots \forall x_{n} \varphi
		\end{align}
		なる形の文を指す.ただし$x_{1},\cdots,x_{n}$は$\varphi$に自由に現れる変項であって,
		また$\varphi$に自由に現れる変項はこれらのみであるとする.$\varphi$が文であるときは
		$\varphi$自身を全称閉包とする.
	}
	
	\begin{screen}
		\begin{metathm}[$\Sigma$の定理は$\Gamma$の定理]
		\label{metathm:Henkin_expansion_3}
			$\psi$を$\lang{\in}$の文とするとき,
			$\Sigma \provable{\mbox{{\bf HE}},\lang{\varepsilon}} \psi$ならば
			$\Gamma \provable{\mbox{{\bf HE}},\lang{\varepsilon}} \psi$である.
		\end{metathm}
	\end{screen}
	
	\begin{sketch}
		$\Sigma$の公理が$\lang{\varepsilon}$の文であるときに$\Gamma$から証明できることを示せばよい.
		$\Sigma$と$\Gamma$で形が違う公理は外延性,相等性,要素,置換である
		(内包性公理は$\lang{\varepsilon}$の式ではありえないので今回は対象外).
		\begin{description}
			\item[外延性]	$a$と$b$を主要$\varepsilon$項とするとき,{\bf HE}の公理(UE)によって
				\begin{align}
					\Gamma &\provable{\mbox{{\bf HE}},\lang{\varepsilon}} \forall x\, \forall y\, (\, \forall z\, 
						(\, z \in x \lrarrow z \in y\, ) \rarrow x = y\, ), \\
					\Gamma &\provable{\mbox{{\bf HE}},\lang{\varepsilon}} \forall y\, (\, \forall z\, 
						(\, z \in a \lrarrow z \in y\, ) \rarrow a = y\, ), \\
					\Gamma &\provable{\mbox{{\bf HE}},\lang{\varepsilon}} \forall z\, 
						(\, z \in a \lrarrow z \in b\, ) \rarrow a = b
				\end{align}
				となる.$\Sigma$の相等性の公理も同様に導かれる.
				
			\item[要素] $a$と$b$を主要$\varepsilon$項とするとき,
				定理\ref{thm:any_class_equals_to_itself}と
				定理\ref{thm:critical_epsilon_term_is_set}の証明をもう一度おさらいすれば
				\begin{align}
					\sigma &\defeq \varepsilon z \negation (\, z \in a \lrarrow z \in a\, ), \\
					&\provable{\mbox{{\bf HE}},\lang{\varepsilon}} \sigma \in a \lrarrow \sigma \in a, 
						&& \mbox{含意の反射律(推論法則\ref{logicalthm:reflective_law_of_implication})と論理積の導入} \\
					&\provable{\mbox{{\bf HE}},\lang{\varepsilon}} \forall z\, (\, z \in a \lrarrow z \in a\, ), 
						&& \mbox{全称の導出(推論法則\ref{logicalthm:derivation_of_universal_by_epsilon})} \\
					\Gamma &\provable{\mbox{{\bf HE}},\lang{\varepsilon}} \forall z\, (\, z \in a \lrarrow z \in a\, ) \rarrow a = a, 
						&& \mbox{前段の結果} \\
					\Gamma &\provable{\mbox{{\bf HE}},\lang{\varepsilon}} a = a, 
						&& \mbox{三段論法} \\
					\Gamma &\provable{\mbox{{\bf HE}},\lang{\varepsilon}} \exists x\, (\, a = x\, )
						&& \mbox{{\bf HE}の公理(EI)}
				\end{align}
				となる.含意の導入(K)より
				\begin{align}
					\provable{\mbox{{\bf HE}},\lang{\varepsilon}} \exists x\, (\, a = x\, ) \rarrow
					(\, a \in b \rarrow \exists x\, (\, a = x\, )\, )
				\end{align}
				が成り立つので,三段論法より
				\begin{align}
					\Gamma \provable{\mbox{{\bf HE}},\lang{\varepsilon}} a \in b \rarrow \exists x\, (\, a = x\, )
				\end{align}
				が従う.
				
			\item[置換] $\psi$が
				\begin{align}
					\forall x\, \forall y\, \forall z\, 
					(\, \varphi(x,y) \wedge \varphi(x,z)
					\rarrow y = z\, )
					\rarrow \forall a\, \exists z\, \forall y\,
					(\, y \in z \lrarrow \exists x\, (\, x \in a \wedge 
					\varphi(x,y)\, )\, )
				\end{align}
				なる文であるとき,$\varphi$に現れる主要$\varepsilon$項で
				$\varphi$の中で極大であるもの(他の項の真部分項になっていないもの)
				を$\psi$に現れない変項で置き換える.その際主要$\varepsilon$項ごとに
				違う変項を用いるが,同じ主要$\varepsilon$項は同じ変項で置き換える.
				そうして得られた式を$\tilde{\psi}$とし,新しく追加した変項を
				$x_{1},\cdots,x_{n}$とすれば
				\begin{align}
					\forall x_{1} \cdots \forall x_{n} \tilde{\psi}(x_{1},\cdots,x_{n})
				\end{align}
				は$\Gamma$の置換公理となる.$x_{1},\cdots,x_{n}$によって置き換えられた
				主要$\varepsilon$項を$\tau_{1},\cdots,\tau_{n}$とすれば,
				{\bf HE}の公理(UE)より
				\begin{align}
					\Gamma &\provable{\mbox{{\bf HE}},\lang{\varepsilon}} \forall x_{1} \cdots \forall x_{n} \tilde{\psi}(x_{1},\cdots,x_{n}), \\
					\Gamma &\provable{\mbox{{\bf HE}},\lang{\varepsilon}} \forall x_{2} \cdots \forall x_{n} \tilde{\psi}(\tau_{1},x_{2},\cdots,x_{n}), \\
					\Gamma &\provable{\mbox{{\bf HE}},\lang{\varepsilon}} \forall x_{3} \cdots \forall x_{n} \tilde{\psi}(\tau_{1},\tau_{2},x_{3},\cdots,x_{n}), \\
					&\vdots \\
					\Gamma &\provable{\mbox{{\bf HE}},\lang{\varepsilon}} \tilde{\psi}(\tau_{1},\cdots,\tau_{n}), \\
				\end{align}
				が得られる.$\tilde{\psi}(\tau_{1},\cdots,\tau_{n})$とは$\psi$のことであるから
				$\psi$は$\Gamma$の定理である.
				\QED
		\end{description}
	\end{sketch}
	
	\begin{screen}
		\begin{metathm}[{\bf HE}で証明可能なら{\bf HK}でも証明可能]
		\label{metathm:Henkin_expansion_HE_to_HK}
			$\psi$を$\lang{\in}$の文とするとき,$\Sigma \provable{\mbox{{\bf HE}},\lang{\varepsilon}} \psi$ならば
			$\Gamma \provable{\mbox{{\bf HK}},\lang{\in}} \psi$である.
		\end{metathm}
	\end{screen}
	
	\begin{sketch}
		$\Sigma \provable{\mbox{{\bf HE}},\lang{\varepsilon}} \psi$ならば,
		メタ定理\ref{metathm:Henkin_expansion_3}より
		\begin{align}
			\Gamma \provable{\mbox{{\bf HE}},\lang{\varepsilon}} \psi
		\end{align}
		となり,メタ定理\ref{metathm:Henkin_expansion_1}より
		\begin{align}
			\Gamma \provable{\mbox{{\bf HK$\varepsilon$}},\lang{\varepsilon}} \psi
		\end{align}
		となり,メタ定理\ref{metathm:Henkin_expansion_2}より
		\begin{align}
			\Gamma \provable{\mbox{{\bf HK}},\lang{\varepsilon}} \psi
		\end{align}
		となる.最後の{\bf HK}の証明を$\varphi_{1},\cdots,\varphi_{n}$とする.
		$\varphi_{1},\cdots,\varphi_{n}$には主要$\varepsilon$項が残っている場合,
		$\varphi_{1},\cdots,\varphi_{n}$の中で極大に現れる(他の項の真部分項ではない)
		主要$\varepsilon$項が$e_{1},\cdots,e_{m}$で全てであるなら,
		$\varphi_{1},\cdots,\varphi_{n}$に現れない相異なる変項$y_{1},\cdots,y_{m}$
		を用意して,$e_{1}$を$y_{1}$に,$e_{2}$を$y_{2}$に,…,$e_{m}$を$y_{m}$に置き換える.
		全て置き換え終わった後の式の列はメタ定理\ref{metathm:Henkin_expansion_2_lemma}より
		{\bf HK}の証明であるし,式自体は$\lang{\in}$のものとなる.
		\QED
	\end{sketch}
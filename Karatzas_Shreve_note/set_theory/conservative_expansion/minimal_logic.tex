\subsection{最小論理}
	{\bf HK}の公理から二重否定除去(DNE)を抜いた体系を{\bf 最小論理}
	\index{さいしょうろんり@最小論理}{\bf (minimal logic)}と呼ぶ.
	この体系では背理法が成り立たないので「~と仮定すると矛盾するので…」といった論法は使えない.
	
	\begin{screen}
		\begin{thm}[対偶律$1$]\label{classic:contraposition_1}
			$\varphi$と$\psi$を式とするとき
			\begin{align}
				\provable{\mbox{{\bf HK}}} (\, \varphi \rarrow \psi\, )
				\rarrow (\, \negation \psi \rarrow\ \negation \varphi\, ).
			\end{align}
		\end{thm}
	\end{screen}
	
	\begin{sketch}
		$\varphi$と$\varphi \rarrow \psi$の三段論法から
		\begin{align}
			\varphi,\ \negation \psi,\ \varphi \rarrow \psi
			\provable{\mbox{{\bf HK}}} \psi
		\end{align}
		が成り立ち,
		\begin{align}
			\varphi,\ \negation \psi,\ \varphi \rarrow \psi
			\provable{\mbox{{\bf HK}}}\ \negation \psi
		\end{align}
		も成り立つので,矛盾の規則(DTC1)より
		\begin{align}
			\varphi,\ \negation \psi,\ \varphi \rarrow \psi
			\provable{\mbox{{\bf HK}}} \bot
		\end{align}
		が従う.演繹定理より
		\begin{align}
			\negation \psi,\ \varphi \rarrow \psi
			\provable{\mbox{{\bf HK}}} \varphi \rarrow \bot
		\end{align}
		となり,否定の導入(NI)より
		\begin{align}
			\negation \psi,\ \varphi \rarrow \psi
			\provable{\mbox{{\bf HK}}}\ \negation \varphi
		\end{align}
		が従う.そして演繹定理より
		\begin{align}
			\varphi \rarrow \psi
			\provable{\mbox{{\bf HK}}}\ \negation \psi \rarrow\ \negation \varphi
		\end{align}
		が得られる.
		\QED
	\end{sketch}
	
	\begin{screen}
		\begin{thm}[弱 De Morgan の法則$1$]
		\label{classic:weak_De_Morgan_law_for_quantifier_1}
			$\varphi$を式とし,
			変項$x$が$\varphi$に自由に現れるとするとき,
			\begin{align}
				\provable{\mbox{{\bf HK}}}
				\ \negation \exists x \varphi \rarrow \forall x \negation \varphi.
			\end{align}
		\end{thm}
	\end{screen}
	
	\begin{sketch}
		$y$を$\varphi$には現れない変項とすると,存在記号の導入規則より
		\begin{align}
			\provable{\mbox{{\bf HK}}} \varphi(x/y) \rarrow \exists x \varphi
		\end{align}
		が成り立ち,対偶律$1$ (定理\ref{classic:contraposition_1})より
		\begin{align}
			\provable{\mbox{{\bf HK}}}\ 
			\negation \exists x \varphi \rarrow\ \negation \varphi(x/y) 
		\end{align}
		となる.汎化により
		\begin{align}
			\provable{\mbox{{\bf HK}}} \forall y\, (\, \negation \exists x \varphi \rarrow\ \negation \varphi(x/y) \, ) 
		\end{align}
		が成り立つので,量化の公理(UI)との三段論法より
		\begin{align}
			\provable{\mbox{{\bf HK}}}\ 
			\negation \exists x \varphi \rarrow \forall x \negation \varphi 
		\end{align}
		が得られる.
		\QED
	\end{sketch}
	
	\begin{screen}
		\begin{thm}[強 De Morgan の法則$1$]
		\label{classic:strong_De_Morgan_law_for_quantifier_1}
			$\varphi$を式とし,
			変項$x$が$\varphi$に自由に現れるとするとき,
			\begin{align}
				\provable{\mbox{{\bf HK}}}
				\exists x \negation \varphi \rarrow\ \negation \forall x \varphi.
			\end{align}
		\end{thm}
	\end{screen}
	
	\begin{sketch}
		$y$を$\varphi$に現れない変項とすれば,量化の公理(UE)より
		\begin{align}
			\provable{\mbox{{\bf HK}}} \forall x \varphi \rarrow \varphi(x/y)
		\end{align}
		が成り立ち,対偶律1 (定理\ref{classic:contraposition_1})より
		\begin{align}
			\provable{\mbox{{\bf HK}}}\ \negation \varphi(x/y) \rarrow\ \negation \forall x \varphi
		\end{align}
		となる.汎化によって
		\begin{align}
			\provable{\mbox{{\bf HK}}} \forall y\, (\, \negation \varphi(x/y) \rarrow\ \negation \forall x \varphi\, )
		\end{align}
		が成り立ち,量化の公理(EE)より
		\begin{align}
			\provable{\mbox{{\bf HK}}} \exists x \negation \varphi \rarrow\ \negation \forall x \varphi
		\end{align}
		が得られる.
		\QED
	\end{sketch}
	
	\begin{screen}
		\begin{thm}[二重否定の導入]
		\label{classic:introduction_of_double_negation}
			$\varphi$を式とするとき
			\begin{align}
				\provable{\mbox{{\bf HK}}} \varphi \rarrow\ \negation \negation \varphi.
			\end{align}
		\end{thm}
	\end{screen}
	
	\begin{sketch}
		矛盾の導入(CTD1)より
		\begin{align}
			\varphi \provable{\mbox{{\bf HK}}}\ \negation \varphi \rarrow \bot
		\end{align}
		が成り立ち,否定の導入(NI)より
		\begin{align}
			\varphi \provable{\mbox{{\bf HK}}}\ \negation \negation \varphi
		\end{align}
		が従う.
		\QED
	\end{sketch}
	
	\begin{screen}
		\begin{thm}[対偶律$2$]\label{classic:contraposition_2}
			$\varphi$と$\psi$を式とするとき
			\begin{align}
				\provable{\mbox{{\bf HK}}} (\, \varphi \rarrow\ \negation \psi\, )
				\rarrow (\, \psi \rarrow\ \negation \varphi\, ).
			\end{align}
		\end{thm}
	\end{screen}
	
	\begin{sketch}
		対偶律$1$ (定理\ref{classic:contraposition_1})より
		\begin{align}
			\varphi \rarrow\ \negation \psi \provable{\mbox{{\bf HK}}}\ 
			\negation \negation \psi \rarrow\ \negation \varphi
		\end{align}
		が成り立ち,他方で二重否定の導入(定理\ref{classic:introduction_of_double_negation})より
		\begin{align}
			\psi \provable{\mbox{{\bf HK}}}\ \negation \negation \psi
		\end{align}
		が成り立つので,三段論法より
		\begin{align}
			\psi,\ \varphi \rarrow\ \negation \psi \provable{\mbox{{\bf HK}}}\ 
			\negation \varphi
		\end{align}
		が従い,演繹定理より
		\begin{align}
			\varphi \rarrow\ \negation \psi \provable{\mbox{{\bf HK}}}
			\psi \rarrow\ \negation \varphi
		\end{align}
		が得られる.
		\QED
	\end{sketch}
	
	\begin{screen}
		\begin{thm}[弱 De Morgan の法則$2$]
		\label{classic:weak_De_Morgan_law_for_quantifier_2}
			$\varphi$を式とし,
			変項$x$が$\varphi$に自由に現れるとするとき,
			\begin{align}
				\provable{\mbox{{\bf HK}}} \forall x \negation \varphi
				\rarrow\ \negation \exists x \varphi.
			\end{align}
		\end{thm}
	\end{screen}
	
	\begin{sketch}
		$y$を$\varphi$に現れない変項とすれば,量化の公理(UE)より
		\begin{align}
			\provable{\mbox{{\bf HK}}} \forall x \negation \varphi \rarrow\ \negation \varphi(x/y)
		\end{align}
		となるので,対偶律$2$ (定理\ref{classic:contraposition_2})より
		\begin{align}
			\provable{\mbox{{\bf HK}}} \varphi(x/y) \rarrow\ \negation \forall x \negation \varphi
		\end{align}
		となる.汎化によって
		\begin{align}
			\provable{\mbox{{\bf HK}}}\ \forall y\, (\, \varphi(x/y) \rarrow\ \negation \forall x \negation \varphi\, )
		\end{align}
		が成り立ち,量化の公理(EE)によって
		\begin{align}
			\provable{\mbox{{\bf HK}}}\ \exists x \negation \varphi \rarrow\ \negation \forall x \negation \varphi
		\end{align}
		が従い,再び対偶律$2$ (定理\ref{classic:contraposition_2})より
		\begin{align}
			\provable{\mbox{{\bf HK}}} \forall x \negation \varphi \rarrow\ \negation \exists x \varphi
		\end{align}
		が得られる.
		\QED
	\end{sketch}
	
	\begin{screen}
		\begin{thm}[De Morgan の法則1]
		\label{classic:De_Morgan_law_1}
			$\varphi$と$\psi$を式とするとき
			\begin{align}
				\provable{\mbox{{\bf HK}}} (\, \negation \varphi \vee \psi\, ) 
				\rarrow\ \negation (\, \varphi \wedge \negation \psi\, ).
			\end{align}
		\end{thm}
	\end{screen}
	
	\begin{sketch}
		論理積の除去(CE1)(CE2)より
		\begin{align}
			\varphi \wedge \negation \psi &\provable{\mbox{{\bf HK}}}\ \negation \varphi, \\
			\varphi \wedge \negation \psi &\provable{\mbox{{\bf HK}}} \psi
		\end{align}
		が成り立つので,矛盾の導入(CTD1)(CTD2)より
		\begin{align}
			\varphi \wedge \negation \psi &\provable{\mbox{{\bf HK}}}
			\varphi \rarrow \bot, \\
			\varphi \wedge \negation \psi &\provable{\mbox{{\bf HK}}}\ 
			\negation \psi \rarrow \bot
		\end{align}
		となり,論理和の除去(DE)より
		\begin{align}
			\varphi \wedge \negation \psi &\provable{\mbox{{\bf HK}}}
			\varphi \vee \negation \psi \rarrow \bot
		\end{align}
		が従い,否定の導入(NI)より
		\begin{align}
			\varphi \wedge \negation \psi &\provable{\mbox{{\bf HK}}}\ 
			\negation (\, \varphi \vee \negation \psi\, )
		\end{align}
		が得られる.
		\QED
	\end{sketch}
	
	\begin{screen}
		\begin{thm}[論理和の対称律]
		\label{classic:symmetry_of_disjunction}
			$\varphi$と$\psi$を式とするとき,
			\begin{align}
				\provable{\mbox{{\bf HK}}} 
				\varphi \vee \psi \rarrow \psi \vee \varphi.
			\end{align}
		\end{thm}
	\end{screen}
	
	\begin{sketch}
		論理和の導入(DI1)(DI2)より
		\begin{align}
			&\provable{\mbox{{\bf HK}}} \varphi \rarrow \psi \vee \varphi, \\
			&\provable{\mbox{{\bf HK}}} \psi \rarrow \psi \vee \varphi
		\end{align}
		が成り立つので,論理和の除去(DE)より
		\begin{align}
			\provable{\mbox{{\bf HK}}} \varphi \vee \psi \rarrow \psi \vee \varphi
		\end{align}
		が従う.
		\QED
	\end{sketch}
	
	\begin{screen}
		\begin{thm}[含意の論理和への遺伝性]
		\label{classic:heredity_of_implication_to_disjunction}
			$\varphi$と$\psi$と$\chi$を式とするとき,
			\begin{align}
				\provable{\mbox{{\bf HK}}} (\, \varphi \rarrow \psi\, )
				\rarrow (\, \varphi \wedge \chi \rarrow \psi \wedge \chi\, ).
			\end{align}
		\end{thm}
	\end{screen}
	
	\begin{sketch}
		三段論法より
		\begin{align}
			\varphi,\ \varphi \rarrow \psi \provable{\mbox{{\bf HK}}} \psi
		\end{align}
		が成り立ち,論理和の導入(DI1)より
		\begin{align}
			\varphi,\ \varphi \rarrow \psi \provable{\mbox{{\bf HK}}} \psi \vee \chi
		\end{align}
		が従い,演繹定理より
		\begin{align}
			\varphi \rarrow \psi \provable{\mbox{{\bf HK}}} 
			\varphi \rarrow \psi \vee \chi
		\end{align}
		が得られる.他方で論理和の導入(DI2)より
		\begin{align}
			\varphi \rarrow \psi \provable{\mbox{{\bf HK}}} 
			\chi \rarrow \psi \vee \chi
		\end{align}
		も成り立つので,論理和の除去(DE)より
		\begin{align}
			\varphi \rarrow \psi \provable{\mbox{{\bf HK}}} 
			\varphi \vee \chi \rarrow \psi \vee \chi
		\end{align}
		が得られる.
		\QED
	\end{sketch}
	
	\begin{screen}
		\begin{thm}[含意の論理積への遺伝性]
		\label{classic:heredity_of_implication_to_conjunction}
			$\varphi$と$\psi$と$\chi$を式とするとき,
			\begin{align}
				\provable{\mbox{{\bf HK}}} (\, \psi \rarrow \chi\, )
				\rarrow (\, \varphi \wedge \psi \rarrow \varphi \wedge \chi\, ).
			\end{align}
		\end{thm}
	\end{screen}
	
	\begin{sketch}
		論理積の除去(CE1)(CE2)及び三段論法より
		\begin{align}
			\psi \rarrow \chi,\ \varphi \wedge \psi &\provable{\mbox{{\bf HK}}} \varphi, \\
			\psi \rarrow \chi,\ \varphi \wedge \psi &\provable{\mbox{{\bf HK}}} \psi,
		\end{align}
		そして
		\begin{align}
			\psi \rarrow \chi,\ \varphi \wedge \psi \provable{\mbox{{\bf HK}}} \chi
		\end{align}
		が成り立つので,論理積の導入(CI)より
		\begin{align}
			\psi \rarrow \chi,\ \varphi \wedge \psi \provable{\mbox{{\bf HK}}} \varphi \wedge \chi
		\end{align}
		が従う.
		\QED
	\end{sketch}
	
	\begin{screen}
		\begin{thm}[論理積と全称の交換]
		\label{classic:commutation_of_conjunction_and_universal_quantifier}
			$\varphi$と$\psi$を式とし,
			$\psi$には変項$x$が自由に現れるとするとき,
			\begin{align}
				\provable{\mbox{{\bf HK}}} \forall x\, (\, \varphi \wedge \psi\, )
				\rarrow \varphi \wedge \forall x \psi.
			\end{align}
		\end{thm}
	\end{screen}
	
	\begin{sketch}
		量化の公理(UE)より
		\begin{align}
			\forall x\, (\, \varphi \wedge \psi\, ) \provable{\mbox{{\bf HK}}} 
			\varphi \wedge \psi
		\end{align}
		が成り立ち,論理積の除去(CE1)(CE2)より
		\begin{align}
			\forall x\, (\, \varphi \wedge \psi\, ) &\provable{\mbox{{\bf HK}}} \varphi, \\
			\forall x\, (\, \varphi \wedge \psi\, ) &\provable{\mbox{{\bf HK}}} \psi
		\end{align}
		となる.汎化によって
		\begin{align}
			\forall x\, (\, \varphi \wedge \psi\, ) \provable{\mbox{{\bf HK}}} 
			\forall x \psi
		\end{align}
		が成り立ち,論理積の導入(CI)によって
		\begin{align}
			\forall x\, (\, \varphi \wedge \psi\, ) \provable{\mbox{{\bf HK}}} 
			\varphi \wedge \forall x \psi
		\end{align}
		が得られる.
		\QED
	\end{sketch}
	
	\begin{screen}
		\begin{thm}
		\label{classic:no_description_1}
			$\varphi$と$\psi$に変項$x$が自由に現れるとき,
			\begin{align}
				\provable{\mbox{{\bf HK}}} \forall x\, (\, \varphi \rarrow \psi\, )
				\rarrow (\, \exists x \varphi \rarrow \exists x \psi\, ).
			\end{align}
		\end{thm}
	\end{screen}
	
	\begin{sketch}
		量化の公理(UE)より
		\begin{align}
			\forall x\, (\, \varphi \rarrow \psi\, ) \provable{\mbox{{\bf HK}}}
			\varphi \rarrow \psi
		\end{align}
		となるので,演繹定理より
		\begin{align}
			\varphi,\ \forall x\, (\, \varphi \rarrow \psi\, ) 
			\provable{\mbox{{\bf HK}}} \psi
		\end{align}
		が成り立つ.量化の公理(EI)より
		\begin{align}
			\varphi,\ \forall x\, (\, \varphi \rarrow \psi\, ) 
			\provable{\mbox{{\bf HK}}} \exists x \psi
		\end{align}
		が成り立ち,演繹定理より
		\begin{align}
			\forall x\, (\, \varphi \rarrow \psi\, ) 
			\provable{\mbox{{\bf HK}}} \varphi \rarrow \exists x \psi
		\end{align}
		が従う.汎化によって
		\begin{align}
			\forall x\, (\, \varphi \rarrow \psi\, ) \provable{\mbox{{\bf HK}}} 
			\forall x\, (\, \varphi \rarrow \exists x \psi\, )
		\end{align}
		となり,量化の公理(EE)より
		\begin{align}
			\forall x\, (\, \varphi \rarrow \psi\, ) \provable{\mbox{{\bf HK}}} 
			\exists x \varphi \rarrow \exists x \psi
		\end{align}
		が従う.
		\QED
	\end{sketch}
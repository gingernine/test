\section{$\mathcal{L}$の証明の変換}
\label{sec:L_proof_to_L_epsilon_proof}
	この節では 「$\Sigma$から$\psi$への{\bf HE}の証明で$\mathcal{L}$の文の列
	であるものが取れる」ならば 「$\Sigma$から$\psi$への{\bf HE}の証明で
	$\lang{\varepsilon}$の文の列であるものが取れる」ことを示す
	
	\begin{screen}
		\begin{metathm}[$\mathcal{L}$の文の証明は$\lang{\varepsilon}$の文の証明に直せる]
			$\psi$を$\lang{\in}$の文とするとき,
			$\Sigma \provable{\mbox{{\bf HE}},\mathcal{L}} \psi$ならば
			$\Sigma \provable{\mbox{{\bf HE}},\lang{\varepsilon}} \psi$である.
		\end{metathm}
	\end{screen}
	
	\begin{metaprf}
		$\varphi_{1},\cdots,\varphi_{n}$を$\Sigma$から$\psi$への証明で$\mathcal{L}$の
		文の列であるものとし,これらを$\lang{\varepsilon}$の文に書き換えたものを
		$\widehat{\varphi}_{1},\cdots,\widehat{\varphi}_{n}$とする.
		ただし,同じ原子式の書き換えは証明全体で一致するようにしておく.このとき
		メタ定理\ref{metathm:rewritten_formulas_are_of_L_epsilon}と
		メタ定理\ref{metathm:variables_unchanged_after_rewriting}より
		$\widehat{\varphi}_{1},\cdots,\widehat{\varphi}_{n}$はいずれも
		$\lang{\varepsilon}$の文であり,また各$\widehat{\varphi}_{i}$について次が言える:
		\begin{description}
			\item[(1)] $\varphi_{i}$が{\bf HE}の命題論理の公理ならば$\widehat{\varphi}_{i}$も
				{\bf HE}の公理である.
				
			\item[(2)] $\varphi_{i}$が{\bf HE}の量化公理か$\Sigma$の公理ならば$\Sigma 
				\provable{\mbox{{\bf HE}},\lang{\varepsilon}} \widehat{\varphi}_{i}$である.
				
			\item[(3)] $\varphi_{i}$が前の文$\varphi_{j},\varphi_{k}$から三段論法で
				得られている場合は,$\widehat{\varphi}_{i}$は$\widehat{\varphi}_{j}$と
				$\widehat{\varphi}_{k}$から三段論法で得られる.
		\end{description}
		
		(1)については,書き換えによって式の結合形式が変わらないため.
		
		(3)については,$\varphi_{k}$を$\varphi_{j} \rarrow \varphi_{i}$なる文とすれば,
		同じ原子式の書き換えは証明全体で一致しているので$\widehat{\varphi_{k}}$は
		$\widehat{\varphi}_{j} \rarrow \widehat{\varphi}_{i}$なる文であり,
		$\widehat{\varphi}_{i}$は$\widehat{\varphi}_{j}$と$\widehat{\varphi}_{k}$から
		三段論法で得られるのである.
		
		(2)について,内包項を含みうる$\Sigma$の公理は外延性,相等性,内包性,要素であるから
		これらについて一つずつ示していく.
		\begin{description}
			\item[case1] $\varphi_{i}$が
				\begin{align}
					\exists x \varphi \rarrow \varphi(\varepsilon x \check{\varphi})
				\end{align}
				なる公理であれば($\check{\varphi}$は$\varphi$を$\lang{\varepsilon}$の式に
				書き直したもの),$\widehat{\varphi}_{i}$は
				\begin{align}
					\exists x \widehat{\varphi} \rarrow \widetilde{\varphi}(\varepsilon x \check{\varphi})
				\end{align}
				なる形の式である($\widetilde{\varphi}$は$\varphi$を$\lang{\varepsilon}$の式に
				書き直したもの).これは
				\begin{align}
					\exists x \widehat{\varphi} &\provable{\mbox{{\bf HE}},\lang{\varepsilon}} \widehat{\varphi}(\varepsilon x \widehat{\varphi}), \\
					\Sigma &\provable{\mbox{{\bf HE}},\lang{\varepsilon}} \widehat{\varphi}(\varepsilon x \widehat{\varphi}) \rarrow \check{\varphi}(\varepsilon x \widehat{\varphi}), \\
					\exists x \widehat{\varphi},\ \Sigma &\provable{\mbox{{\bf HE}},\lang{\varepsilon}} \check{\varphi}(\varepsilon x \widehat{\varphi}), \\
					&\provable{\mbox{{\bf HE}},\lang{\varepsilon}} \check{\varphi}(\varepsilon x \widehat{\varphi}) \rarrow \exists x \check{\varphi}, \\
					\exists x \widehat{\varphi},\ \Sigma &\provable{\mbox{{\bf HE}},\lang{\varepsilon}} \exists x \check{\varphi}, \\
					&\provable{\mbox{{\bf HE}},\lang{\varepsilon}} \exists x \check{\varphi} \rarrow \check{\varphi}(\varepsilon x \check{\varphi}), \\
					\exists x \widehat{\varphi},\ \Sigma &\provable{\mbox{{\bf HE}},\lang{\varepsilon}} \check{\varphi}(\varepsilon x \check{\varphi}), \\
					\Sigma &\provable{\mbox{{\bf HE}},\lang{\varepsilon}} \check{\varphi}(\varepsilon x \check{\varphi}) \rarrow  \widetilde{\varphi}(\varepsilon x \check{\varphi}), \\
					\exists x \widehat{\varphi},\ \Sigma &\provable{\mbox{{\bf HE}},\lang{\varepsilon}} \widetilde{\varphi}(\varepsilon x \check{\varphi})
				\end{align}
				から従う.
				
			\item[case2] $\varphi_{i}$が
				\begin{align}
					\varphi(\tau) \rarrow \exists x \varphi
				\end{align}
				なる公理であれば,$\widehat{\varphi}_{i}$は
				\begin{align}
					\check{\varphi}(\tau) \rarrow \exists x \widetilde{\varphi}
				\end{align}
				なる式となる.これは{\bf HE}から演繹可能である.実際,定理\ref{logicalthm:equivalence_by_replacing_bound_variables}より
				\begin{align}
					\provable{\mbox{{\bf HE}},\lang{\varepsilon}} \check{\varphi}(\tau) \rarrow \widetilde{\varphi}(\tau)
				\end{align}
				が成り立ち,他方で{\bf HE}の量化公理より
				\begin{align}
					\provable{\mbox{{\bf HE}},\lang{\varepsilon}} \widetilde{\varphi}(\tau) \rarrow \exists x \widetilde{\varphi}
				\end{align}
				が成り立つので,
				\begin{align}
					\provable{\mbox{{\bf HE}},\lang{\varepsilon}} \check{\varphi}(\tau) \rarrow \exists x \widetilde{\varphi}
				\end{align}
				が従う.
				
			\item[case2] $\varphi_{i}$が
				\begin{align}
					\forall x \varphi \rarrow \varphi(\tau)
				\end{align}
				なる公理であれば,$\widehat{\varphi}_{i}$は
				\begin{align}
					\forall x \widetilde{\varphi} \rarrow \check{\varphi}(\tau)
				\end{align}
				なる式となる.これは{\bf HE}から演繹可能である.実際,{\bf HE}の量化公理より
				\begin{align}
					\provable{\mbox{{\bf HE}},\lang{\varepsilon}} \forall x \widetilde{\varphi} \rarrow \widetilde{\varphi}(\tau)
				\end{align}
				が成り立ち,他方で定理\ref{logicalthm:equivalence_by_replacing_bound_variables}より
				\begin{align}
					\provable{\mbox{{\bf HE}},\lang{\varepsilon}} \widetilde{\varphi}(\tau) \rarrow \check{\varphi}(\tau)
				\end{align}
				が成り立つので,
				\begin{align}
					\provable{\mbox{{\bf HE}},\lang{\varepsilon}} \forall x \widetilde{\varphi} \rarrow \check{\varphi}(\tau)
				\end{align}
				が従う.
				
			\item[case1] $\varphi_{i}$が外延性公理
				\begin{align}
					\forall x\, (\, x \in a \lrarrow x \in b\, ) \rarrow a = b
				\end{align}
				であるとき,$a,b$が共に主要$\varepsilon$項ならばこれは$\lang{\varepsilon}$
				の文である.
				$a$が$\Set{y}{\varphi(y)}$なる項で$b$が
				主要$\varepsilon$項であるときは,$\widehat{\varphi}_{i}$は
				\begin{align}
					\forall x\, (\, \varphi(x) \lrarrow x \in b\, ) \rarrow 
					\forall z\, (\, \varphi(z) \lrarrow z \in b\, )
				\end{align}
				なる形の文となり,これは${\bf HE}$で証明可能である.実際
				\begin{align}
					\zeta \defeq \varepsilon z \negation (\, \varphi(z) \lrarrow z \in b\, )
				\end{align}
				とおけば
				\begin{align}
					\forall x\, (\, \varphi(x) \lrarrow x \in b\, ) 
					\provable{\mbox{{\bf HE}},\lang{\varepsilon}} \varphi(\zeta) \lrarrow \zeta \in b
				\end{align}
				が成り立つので,全称の導出(論理的定理\ref{logicalthm:derivation_of_universal_by_epsilon})より
				\begin{align}
					\forall x\, (\, \varphi(x) \lrarrow x \in b\, ) 
					\provable{\mbox{{\bf HE}},\lang{\varepsilon}}
					\forall z\, (\, \varphi(z) \lrarrow z \in b\, )
				\end{align}
				となる.$a$が$\Set{y}{\varphi(y)}$なる項で$b$が$\Set{z}{\psi(z)}$なる
				項のときは,$\widehat{\varphi}_{i}$は
				\begin{align}
					\forall x\, (\, \varphi(x) \lrarrow \psi(x)\, ) \rarrow 
					\forall u\, (\, \varphi(u) \lrarrow \psi(u)\, )
				\end{align}
				なる形の文となり,これも${\bf HE}$で証明可能である.
				$a$が主要$\varepsilon$項で$b$が$\Set{z}{\psi(z)}$なる項のときも
				同様に$\widehat{\varphi}_{i}$は{\bf HE}で証明可能である.
			
			\item[case2] $\varphi_{i}$が内包性公理
				\begin{align}
					\forall x\, (\, x \in \Set{y}{\varphi(y)} \lrarrow \varphi(x)\, )
				\end{align}
				なる式であるとき,$\widehat{\varphi}_{i}$は
				\begin{align}
					\forall x\, (\, x \in \varphi(x) \lrarrow \varphi(x)\, )
				\end{align}
				なる式であり,これは{\bf HE}から証明可能である.実際
				\begin{align}
					\tau \defeq \varepsilon x \negation (\, \varphi(x) \lrarrow \varphi(x)\, )
				\end{align}
				とおけば,含意の反射律(論理的定理\ref{logicalthm:reflective_law_of_implication})と論理積の導入より
				\begin{align}
					\provable{\mbox{{\bf HE}},\lang{\varepsilon}} \varphi(\tau) \lrarrow \varphi(\tau)
				\end{align}
				が成り立つので,全称の導出(論理的定理\ref{logicalthm:derivation_of_universal_by_epsilon})より
				\begin{align}
					\provable{\mbox{{\bf HE}},\lang{\varepsilon}} \forall x\, (\, x \in \varphi(x) \lrarrow \varphi(x)\, )
				\end{align}
				となる.
			
			\item[case3] $\varphi_{i}$が要素の公理
				\begin{align}
					a \in b \rarrow \exists x\, (\, a = x\, )
				\end{align}
				なる式であるとき,$a$も$b$も主要$\varepsilon$項ならば
				\begin{align}
					\Sigma \provable{\mbox{{\bf HE}},\lang{\varepsilon}} \exists x\, (\, a = x\, )
				\end{align}
				(定理\ref{thm:critical_epsilon_term_is_set})と含意の導入
				\begin{align}
					\provable{\mbox{{\bf HE}},\lang{\varepsilon}} \exists x\, (\, a = x\, )
					\rarrow (\, a \in b \rarrow \exists x\, (\, a = x\, )\, )
				\end{align}
				から
				\begin{align}
					\Sigma \provable{\mbox{{\bf HE}},\lang{\varepsilon}} a \in b \rarrow \exists x\, (\, a = x\, )
				\end{align}
				が従う.$a$が主要$\varepsilon$項で$b$が$\Set{z}{\psi(z)}$なる項であるとき,
				$\widehat{\varphi}_{i}$は
				\begin{align}
					\psi(a) \rarrow \exists x\, (\, a = x\, )
				\end{align}
				となるが,上と同様にして{\bf HE}で証明できる.
				$a$が$\Set{y}{\varphi(y)}$なる項で$b$が主要$\varepsilon$であるとき,
				$\widehat{\varphi}_{i}$は
				\begin{align}
					\exists s\, (\, \forall u\, (\, \varphi(u) \lrarrow u \in s\, )
					\wedge s \in b\, ) \rarrow \exists x\, \forall v\, (\, \varphi(v) \lrarrow v \in x\, )
				\end{align}
				となるが,これも{\bf HE}で証明可能で,実際
				\begin{align}
					\sigma &\defeq \varepsilon s\, (\, \forall u\, (\, \varphi(u) \lrarrow u \in s\, ), \\
					\tau &\defeq \varepsilon v \negation (\, \varphi(v) \lrarrow v \in \sigma\, )
				\end{align}
				とおけば
				\begin{align}
					\exists s\, (\, \forall u\, (\, \varphi(u) \lrarrow u \in s\, )
					\wedge s \in b\, )
					&\provable{\mbox{{\bf HE}},\lang{\varepsilon}} 
					\forall u\, (\, \varphi(u) \lrarrow u \in \sigma\, )
					\wedge \sigma \in b, \\
					\exists s\, (\, \forall u\, (\, \varphi(u) \lrarrow u \in s\, )
					\wedge s \in b\, )
					&\provable{\mbox{{\bf HE}},\lang{\varepsilon}} 
					\forall u\, (\, \varphi(u) \lrarrow u \in \sigma\, ), \\
					\exists s\, (\, \forall u\, (\, \varphi(u) \lrarrow u \in s\, )
					\wedge s \in b\, )
					&\provable{\mbox{{\bf HE}},\lang{\varepsilon}} \varphi(\tau) \lrarrow \tau \in \sigma, \\
					&\provable{\mbox{{\bf HE}},\lang{\varepsilon}} \forall v\, (\, \varphi(v) \lrarrow v \in \sigma\, ), \\
					&\provable{\mbox{{\bf HE}},\lang{\varepsilon}} \exists x\, \forall v\, (\, \varphi(v) \lrarrow v \in x\, )
				\end{align}
				が成り立つ.$a$が$\Set{y}{\varphi(y)}$なる項で$b$が$\Set{z}{\psi(z)}$なる項
				であるとき,$\widehat{\varphi}_{i}$は
				\begin{align}
					\exists s\, (\, \forall u\, (\, \varphi(u) \lrarrow u \in s\, )
					\wedge \psi(s)\, ) \rarrow \exists x\, \forall v\, (\, \varphi(v) \lrarrow v \in x\, )
				\end{align}
				となるが,これも同様に{\bf HE}で証明可能である.
				
			\item[case4] $\varphi_{i}$が相等性公理
				\begin{align}
					a = b \rarrow b = a
				\end{align}
				なる式である場合,たとえば$a$が$\Set{y}{\varphi(y)}$なる項で
				$b$が主要$\varepsilon$項であれば,$\widehat{\varphi}_{i}$は
				\begin{align}
					\forall u\, (\, \varphi(u) \lrarrow u \in b\, ) 
					\rarrow \forall v\, (\, v \in b \lrarrow \varphi(v)\, ) 
				\end{align}
				となるが,これは{\bf HE}で証明可能であって,実際
				\begin{align}
					\tau \defeq \varepsilon v \negation (\, v \in b \lrarrow \varphi(v)\, )
				\end{align}
				とおけば
				\begin{align}
					\forall u\, (\, \varphi(u) \lrarrow u \in b\, ) 
					&\provable{\mbox{{\bf HE}},\lang{\varepsilon}} \varphi(\tau) \lrarrow \tau \in b, \\
					\forall u\, (\, \varphi(u) \lrarrow u \in b\, ) 
					&\provable{\mbox{{\bf HE}},\lang{\varepsilon}} \tau \in b \lrarrow \varphi(\tau), \\
					&\provable{\mbox{{\bf HE}},\lang{\varepsilon}} \forall v\, (\, v \in b \lrarrow \varphi(v)\, )
				\end{align}
				が成り立つ.$a$も$b$も内包項である場合や,$a$が主要$\varepsilon$項で
				$b$が内包項である場合も同様のことが言える.
			
			\item[case5] $\varphi_{i}$が相等性公理
				\begin{align}
					a = b \rarrow (\, a \in c \rarrow b \in c\, )
				\end{align}
				なる式である場合,
		\end{description}
	\end{metaprf}
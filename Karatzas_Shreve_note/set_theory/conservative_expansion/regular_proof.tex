\section{正則証明}
\label{sec:regular_proof}
	この節では「$\Gamma$から$\psi$への{\bf HK}の証明で$\lang{\in}$の
	式の列であるものが取れる」ならば「$\Sigma$から$\psi$への{\bf HE}の証明で
	$\lang{\varepsilon}$の文の列であるものが取れる」ことを示す
	{\bf HK}の証明の中で汎化が使われている場合,その固有変項は
	適当な主要$\varepsilon$項に置き換えることになる.たとえば
	\begin{align}
		\psi(x/a)
	\end{align}
	から($\psi$は$x$のみ自由に現れる式とする)
	\begin{align}
		\forall x \psi
	\end{align}
	が汎化で導かれる場合,$a$を$\varepsilon x \negation \psi$に置き換えれば
	\begin{align}
		\psi(x/\varepsilon x \negation \psi), 
		\quad \psi(x/\varepsilon x \negation \psi) \rarrow \forall x \psi
	\end{align}
	から三段論法で$\forall x \psi$が出てくる.ここで注意しておくと,汎化の固有変項の条件より
	$a$は$\forall x \psi$に自由に現れないので,$a$は$\psi$にも自由に現れず,
	\begin{align}
		\psi(x/a)(a/\varepsilon x \negation \psi)
	\end{align}
	と
	\begin{align}
		\psi(x/\varepsilon x \negation \psi)
	\end{align}
	は一致しているのである.固有変項の置き換えは証明全体で一斉に行うので,
	二つの汎化に対して同じ固有変項が使われている場合は
	代入する主要$\varepsilon$項をうまく選ぶことが出来ない.
	従って,どの固有変項も一度の汎化にしか用いられないように証明を直す必要がある.
	
	\begin{screen}
		\begin{metadfn}[正則証明]
			{\bf 正則証明}\index{せいそくしょうめい@正則証明}{\bf (regular proof)}とは
			次を満たす証明$\varphi_{1},\cdots,\varphi_{n}$である.
			第一に,証明の中に現れるどの固有変項も一度の汎化にしか用いられない.
			第二に,$a$が$\varphi_{i}$から$\varphi_{j}$への汎化の固有変項ならば,
			$a$は$\varphi_{i+1}$以降の式には自由に現れない.
		\end{metadfn}
	\end{screen}
	
	{\bf HK}の任意の証明は正則なものに変換することが出来る.
	
	\begin{screen}
		\begin{metathm}[証明に現れる変項に代入しても証明]
		\label{metathm:substitute_HK_proof}
			$\lang{\varepsilon}$の式の列$\varphi_{1},\cdots,\varphi_{n}$を
			$\Gamma$から$\varphi_{n}$への{\bf HK}の証明とし,$a$をこの証明に自由に現れる変項とし,
			$b$をこの証明に現れない$\lang{\varepsilon}$の項とする.
			\begin{description}
				\item[(1)] $b$が変項である場合,
					$\varphi_{1}(a/b),\cdots,\varphi_{n}(a/b)$は
					$\Gamma$からの{\bf HK}の証明となる.
				
				\item[(2)] $b$が変項でない場合\footnotemark,$a$がこの証明の固有変項でなければ
					$\varphi_{1}(a/b),\cdots,\varphi_{n}(a/b)$は
					$\Gamma$からの{\bf HK}の証明となる.
			\end{description}
			(1)の場合も(2)の場合も$\varphi_{i}$が$\varphi_{j}$の汎化で得られているなら
			$\varphi_{i}(a/b)$も$\varphi_{j}(a/b)$の汎化で得られる.
		\end{metathm}
	\end{screen}
	
	\footnotetext{
		第\ref{sec:restriction_of_formulas}節の約束によって,
		この場合$b$は主要$\varepsilon$項である.
	}
	
	\begin{metaprf}
		$b$が変項であるか否かが関係するのは case5 である.
		式の列が証明であるための条件に照合していく.各$\varphi_{i}$に対して
		\begin{description}
			\item[case1] $\varphi_{i}$が{\bf HK}の命題論理の公理である場合,
				たとえば$\varphi_{i}$が
				\begin{align}
					\varphi \rarrow (\, \psi \rarrow \varphi\, )
				\end{align}
				なる形の公理ならば,$\varphi_{i}(a/b)$は
				\begin{align}
					\varphi(a/b) \rarrow (\, \psi(a/b) \rarrow \varphi(a/b)\, )
				\end{align}
				なる式であるから{\bf HK}の公理である.
				
			\item[case2] $\varphi_{i}$が{\bf HK}の量化公理である場合,
				\begin{itemize}
					\item たとえば$\varphi_{i}$が{\bf HK}の(UI)
						\begin{align}
							\forall y\, (\, \psi \rarrow \varphi(x/y)\, )
							\rarrow (\, \psi \rarrow \forall x \varphi\, )
						\end{align}
						であるとする.このとき,$a$が$x$であれば$\varphi_{i}(a/b)$は
						\begin{align}
							\forall y\, (\, \psi(a/b) \rarrow \varphi(x/y)\, )
							\rarrow (\, \psi(a/b) \rarrow \forall x \varphi\, )
						\end{align}
						なる式となる.$a$が$y$であれば$a$は$\varphi_{i}$には自由に現れないので
						$\varphi_{i}(a/b)$と$\varphi_{i}$は一致する.$a$が$x$とも$y$とも
						違うとき,$\varphi(x/y)(a/b)$と$\varphi(a/b)(x/y)$は一致するので
						$\varphi_{i}(a/b)$は
						\begin{align}
							\forall y\, (\, \psi(a/b) \rarrow \varphi(a/b)(x/y)\, )
							\rarrow (\, \psi(a/b) \rarrow \forall x \varphi(a/b)\, )
						\end{align}
						なる式となる.ゆえにいずれの場合も$\varphi_{i}(a/b)$は{\bf HK}の(UI)となる.
						同様に$\varphi_{i}$が{\bf HK}の(EE)であるときも
						$\varphi_{i}(a/b)$は{\bf HK}の(EE)となる.
				
					\item たとえば$\varphi_{i}$が{\bf HK}の(EI)
						\begin{align}
							\varphi(x/t) \rarrow \exists x \varphi
						\end{align}
						であるとする.このとき,$a$が$x$であれば,$x$と$t$が違えば$\varphi_{i}$に
						$a$は自由に現れないので$\varphi_{i}(a/b)$は$\varphi_{i}$に一致する.
						$x$と$t$が同じであれば$\varphi_{i}(a/b)$は
						\begin{align}
							\varphi(x/b) \rarrow \exists x \varphi
						\end{align}
						なる式となる.$a$が$t$であれば($t$と$x$は違うとする),
						$\varphi(x/t)(a/b)$と$\varphi(a/b)(x/b)$は一致するので
						$\varphi_{i}(a/b)$は
						\begin{align}
							\varphi(a/b)(x/b) \rarrow \exists x \varphi(a/b)
						\end{align}
						なる式となる.$a$が$x$とも$t$とも違うとき,$\varphi(x/t)(a/b)$と
						$\varphi(a/b)(x/t)$は一致するので$\varphi_{i}(a/b)$は
						\begin{align}
							\varphi(a/b)(x/t) \rarrow \exists x \varphi(a/b)
						\end{align}
						なる式となる.ゆえにいずれの場合も$\varphi_{i}(a/b)$は{\bf HK}の(EI)となる.
						同様に$\varphi_{i}$が{\bf HK}の(UE)であるときも
						$\varphi_{i}(a/b)$は{\bf HK}の(UE)となる.
				\end{itemize}
				
			\item[case3] $\varphi_{i}$が$\Gamma$の公理である場合,
				$\varphi_{i}$は文なので$\varphi_{i}(a/b)$は$\varphi_{i}$である.
			
			\item[case4] $\varphi_{i}$が前の式$\varphi_{j},\varphi_{k}$から
				三段論法で得られるとき,$\varphi_{k}$が$\varphi_{j} \rarrow \varphi_{i}$
				なる形の式ならば$\varphi_{k}(a/b)$は
				\begin{align}
					\varphi_{j}(a/b) \rarrow \varphi_{i}(a/b)
				\end{align}
				なる式となる.つまり$\varphi_{i}(a/b)$は$\varphi_{j}(a/b)$と
				$\varphi_{k}(a/b)$から三段論法で得られる.
				
			\item[case5] $\varphi_{i}$が前の式$\varphi_{j}$から汎化で得られるとき,
				変項$e,x$と$x$が自由に現れる式$\psi$が取れて,$\varphi_{j}$は$\psi(x/e)$,
				$\varphi_{i}$は$\forall x \psi$なる式である.また$e$は$\psi$の中で$x$への代入
				について自由であり,$e$は$\forall x \psi$に自由に現れない.このとき
				
				\begin{description}
					\item[(1)] $b$が変項である場合,
						$a$が$x$であれば,$x$と$e$が同じなら$\varphi_{j}(a/b)$は
						$\psi(x/b)$となり,$\varphi_{i}(a/b)$は$\forall x \psi$のままである.
						$x$と$e$が違うなら$\psi(x/e)$に$a$は自由に現れないので,
						$\varphi_{j}(a/b)$は$\psi(x/e)$のままであり,
						$\varphi_{i}(a/b)$も$\forall x \psi$のままである.
						
						$a$が$e$であれば,$\psi$に$e$は自由に現れないので$\varphi_{j}(a/b)$は
						$\psi(x/b)$となり,$\varphi_{i}(a/b)$は$\forall x \psi$のままである.
				
						$a$が$x$とも$e$とも違うとき,$\varphi_{j}(a/b)$は$\psi(a/b)(x/e)$に
						一致し,$\varphi_{i}(a/b)$は$\forall x \psi(a/b)$となる.
						
					\item[(2)] $b$が変項でない場合,
						$a$は固有変項ではないので$a$と$e$は違う変項である.
						$a$が$x$であれば,$\psi(x/e)$に$a$は自由に現れないので
						$\varphi_{j}(a/b)$は$\psi(x/e)$のままであり,
						$\varphi_{i}(a/b)$も$\forall x \psi$のままである.
				
						$a$が$x$と違うとき,$\varphi_{j}(a/b)$は$\psi(a/b)(x/e)$に一致し,
						$\varphi_{i}(a/b)$は$\forall x \psi(a/b)$となる.
				\end{description}
				
				ゆえに,いずれの場合も$\varphi_{i}(a/b)$は$\varphi_{j}(a/b)$から汎化で得られる.
				\QED
		\end{description}
	\end{metaprf}
	
	\begin{screen}
		\begin{metathm}[どんな証明も正則化できる]
			$\varphi_{1},\cdots,\varphi_{n}$を$\Gamma$から$\varphi_{n}$への{\bf HK}の
			証明とするとき,$\Gamma$から$\varphi_{n}$への{\bf HK}の正則証明が得られる.
		\end{metathm}
	\end{screen}
	
	\begin{metaprf}
		$\varphi_{1},\cdots,\varphi_{n}$の中で汎化が使われていなければ
		これ自体が正則証明である.汎化が使われている場合,
		使われている箇所を
		\begin{align}
			\varphi_{i_{1}} \quad &\mbox{から} \quad \varphi_{j_{1}}, \\
			\varphi_{i_{2}} \quad &\mbox{から} \quad \varphi_{j_{2}}, \\
			&\vdots \\
			\varphi_{i_{\ell}} \quad &\mbox{から} \quad \varphi_{j_{\ell}}
		\end{align}
		とすべて列挙し($i_{1} < i_{2} < \cdots < i_{\ell}$),
		$a_{1},\cdots,a_{\ell}$をそれぞれの固有変項とする.
		ただし,もしかすると$a_{2}$と$a_{5}$は同じ文字$b$であるかもしれない.
		いま想定しているのはそのような状況であり,これから正則証明を構成するのである.
		
		$\varphi_{i_{1}}$の直後に$\varphi_{j_{1}}$を移動し,
		$\varphi_{i_{2}}$の直後に$\varphi_{j_{2}}$に移動し,…,
		$\varphi_{i_{\ell}}$の直後に$\varphi_{j_{\ell}}$を移動することによって
		$\varphi_{1},\cdots,\varphi_{n}$を並べ替えたものを
		\begin{align}
			\psi_{1},\cdots,\psi_{n}
		\end{align}
		と書けば,これもまた{\bf HK}の証明となっている.なぜならこの並び替えでは
		三段論法と汎化の順番が崩れないからである.並び替えによって
		$\varphi_{i_{1}},\cdots,\varphi_{i_{\ell}}$の位置も変動しうるが,
		これらが動いた先をそれぞれ$\psi_{k_{1}},\cdots,\psi_{k_{\ell}}$とする.
		また$b_{1},\cdots,b_{\ell}$を$\varphi_{i_{1}},\cdots,\varphi_{i_{\ell}}$に
		現れない相異なる変項とする.このとき
		\begin{align}
			&\psi_{1}(a_{1}/b_{1}),\ \cdots,\ \textcolor{red}{\psi_{k_{1}+1}(a_{1}/b_{1})}, \\
			&\psi_{1}(a_{2}/b_{2}),\ \cdots,\ \psi_{k_{1}+1}(a_{2}/b_{2}),\ \cdots,\ \textcolor{red}{\psi_{k_{2}+1}(a_{2}/b_{2})}, \\
			&\psi_{1}(a_{3}/b_{3}),\ \cdots,\ \psi_{k_{1}+2}(a_{3}/b_{3}),\ \cdots,\ \psi_{k_{2}+1}(a_{3}/b_{3}),\ \cdots,\ \textcolor{red}{\psi_{k_{3}+1}(a_{3}/b_{3})}, \\
			&\vdots \\
			&\psi_{1}(a_{\ell}/b_{\ell}),\ \cdots,\ \psi_{k_{1}+1}(a_{\ell}/b_{\ell}),\ \cdots,\ \psi_{k_{2}+1}(a_{\ell}/b_{\ell}),\ \cdots,\ 
			\psi_{k_{3}+1}(a_{\ell}/b_{\ell}), \cdots,\ \textcolor{red}{\psi_{k_{\ell}+1}(a_{\ell}/b_{\ell})}, \\
			&\psi_{k_{\ell}+2}(a_{\ell}/b_{\ell}),\ \cdots,\ \psi_{n}(a_{\ell}/b_{\ell})
		\end{align}
		は{\bf HK}の証明となっている.ところで,固有変項の条件より$a_{1}$は$\psi_{k_{1}+1}$に
		自由に現れないので$\psi_{k_{1}+1}(a_{1}/b_{1})$は$\psi_{k_{1}+1}$に一致する.
		同様に,赤字の$\psi_{k_{2}+1}(a_{2}/b_{2}),\ \psi_{k_{3}+1}(a_{3}/a_{3}),\ 
		\cdots,\ \psi_{k_{\ell}+1}(a_{\ell}/b_{\ell})$はそれぞれ
		$\psi_{k_{2}+1},\ \psi_{k_{3}+1},\ \cdots,\ \psi_{k_{\ell}+1}$と同じ式である.つまり
		\begin{align}
			&\psi_{1}(a_{1}/b_{1}),\ \cdots,\ \textcolor{red}{\psi_{k_{1}+1}}, \\
			&\psi_{1}(a_{2}/b_{2}),\ \cdots,\ \psi_{k_{1}+1}(a_{2}/b_{2}),\ \cdots,\ \textcolor{red}{\psi_{k_{2}+1}}, \\
			&\psi_{1}(a_{3}/b_{3}),\ \cdots,\ \psi_{k_{1}+2}(a_{3}/b_{3}),\ \cdots,\ \psi_{k_{2}+1}(a_{3}/b_{3}),\ \cdots,\ \textcolor{red}{\psi_{k_{3}+1}}, \\
			&\vdots \\
			&\psi_{1}(a_{\ell}/b_{\ell}),\ \cdots,\ \psi_{k_{1}+1}(a_{\ell}/b_{\ell}),\ \cdots,\ \psi_{k_{2}+1}(a_{\ell}/b_{\ell}),\ \cdots,\ 
			\psi_{k_{3}+1}(a_{\ell}/b_{\ell}), \cdots,\ \textcolor{red}{\psi_{k_{\ell}+1}}, \\
			&\psi_{k_{\ell}+2}(a_{\ell}/b_{\ell}),\ \cdots,\ \psi_{n}(a_{\ell}/b_{\ell})
		\end{align}
		は{\bf HK}の証明である.
		
		$\varphi$を$\lang{\in}$の文とし,$\varphi_{1},\cdots,\varphi_{n}$を
		$\lang{\in}$の式からなる$\varphi$への{\bf HK}の証明とする.
		$\varphi_{i}$から$\varphi_{j}$にかけて汎化が用いられ(固有変項$a$),
		$\varphi_{k}$から$\varphi_{\ell}$にかけて汎化が用いられているとき(固有変項$a$),
		$\varphi_{1},\cdots,\varphi_{n}$に自由に現れる$a$を$b$に置き換えたものを
		$\hat{\varphi}_{1},\cdots,\hat{\varphi}_{n}$と書けば,
		\begin{align}
			\varphi_{1},\cdots,\varphi_{j},
			\hat{\varphi}_{1},\cdots,\hat{\varphi}_{j-1},\hat{\varphi}_{j+1},
			\cdots,\hat{\varphi}_{n}
		\end{align}
		は$\varphi$への正則証明になっている.
		\QED
	\end{metaprf}
	
	\begin{screen}
		\begin{metathm}[{\bf HK}の定理は{\bf HE}の定理]
		\label{metathm:theorems_in_HK_provable_in_HE}
			$\mathscr{S}$を$\lang{\varepsilon}$の文からなる公理系とし,
			$\psi$を$\lang{\in}$の文とするとき,
			$\mathscr{S} \provable{\mbox{{\bf HK}},\lang{\in}} \psi$ならば
			$\mathscr{S} \provable{\mbox{{\bf HE}},\lang{\varepsilon}} \psi$である.
		\end{metathm}
	\end{screen}
	
	\begin{metaprf}
		$\lang{\in}$の式の列$\varphi_{1},\cdots,\varphi_{n}$を
		$\mathscr{S}$から$\psi$への{\bf HK}の正則な証明とし,また
		\begin{align}
			a_{1},\cdots,a_{m}
		\end{align}
		をこの証明に使われる固有変項とし,$a_{1},a_{2},\cdots$の順番に汎化に用いられるとする.
	
		\begin{description}
			\item[step1]
				$\varphi_{1},\cdots,\varphi_{n}$の中に
				自由に現れる変項のうち,$a_{1},\cdots,a_{m}$以外の全てを
				$x_{1},\cdots,x_{k}$とする.これらに対し
				相異なる主要$\varepsilon$項$\tau_{1},\cdots,\tau_{k}$を用意して
				(これらは$\varphi_{1},\cdots,\varphi_{n}$に現れないものとする),
				自由に現れる全ての$x_{i}$に$\tau_{i}$を代入する($1 \leq i \leq n$).
				そして得られる式の列を$\tilde{\varphi}_{1},\cdots,\tilde{\varphi}_{n}$とする.
				この列は{\bf HK}の証明である(メタ定理\ref{metathm:substitute_HK_proof}).
				
			\item[step2]
				次に$a_{m},a_{m-1},\cdots$の順に固有変項を置き換える.$a_{m}$が
				\begin{align}
					\varphi(x/a_{m})
				\end{align}
				から
				\begin{align}
					\forall x \varphi
				\end{align}
				への汎化に使われているなら,$\tilde{\varphi}_{1},\cdots,
				\tilde{\varphi}_{n}$に自由に現れる$a_{m}$を全て
				$\varepsilon x \negation \varphi$に置き換えて,列の$\forall x \varphi$の前に
				\begin{align}
					\varphi(x/\varepsilon x \negation \varphi) 
					\rarrow \forall x \varphi
				\end{align}
				への{\bf HE}の証明を挿入する
				(論理的定理\ref{logicalthm:derivation_of_universal_by_epsilon}).
				同じ要領で$a_{m-1},\cdots,a_{1}$も主要$\varepsilon$項に置き換えていく.
				
			\item[step3]
				step2の終了後に得られる式の列を$\hat{\varphi}_{1},\cdots,\hat{\varphi}_{r}$
				とする.これらは全て$\lang{\varepsilon}$の文であり,
				そして各$\hat{\varphi}_{i}$は次のいずれかである:
				\begin{itemize}
					\item $\varphi_{i}$が(UI)と(EE)以外の{\bf HK}の公理ならば
						$\hat{\varphi}_{i}$は{\bf HE}の公理である.
						
					\item $\varphi_{i}$が(UI)か(EE)ならば後述.
					
					\item $\varphi_{i}$が$\mathscr{S}$の公理ならば,
						$\varphi_{i}$は変項の置換による影響を受けないので
						$\hat{\varphi}_{i}$は$\varphi_{i}$と同一である.
					
					\item $\varphi_{i}$が前の式$\varphi_{j},\varphi_{k}$から
						三段論法で得られているならば,$\hat{\varphi}_{i}$も
						$\hat{\varphi}_{j},\hat{\varphi}_{k}$から三段論法で得られる.
					
					\item $\varphi_{i}$が前の式から汎化で得られているならば,
						$\hat{\varphi}_{i}$は前の式から三段論法で得られる.
				\end{itemize}
			
				この列の中に{\bf HK}の公理(UI)と(EE)の形の式が残っている場合は
				まだ{\bf HE}の証明ではない.とはいえ下で示す通り(UI)と(EE)は{\bf HE}で証明できるから,
				$\hat{\varphi}_{1},\cdots,\hat{\varphi}_{m}$の中で
				(UI)または(EE)の形の式があれば,その式の前の列にその式への{\bf HE}の
				証明を挿入すればよい.
				
				\begin{description}
					\item[(UI)の証明]
						$\forall y\, (\, \psi \rarrow \varphi(x/y)\, ) 
						\rarrow (\, \psi \rarrow \forall x \varphi\, )$を示す.
						{\bf HE}の公理(UE)より
						\begin{align}
							\forall y\, (\, \psi \rarrow \varphi(x/y)\, ) \provable{\mbox{{\bf HE}},\lang{\varepsilon}} 
							\psi \rarrow \varphi(x/\varepsilon x \negation \varphi)
						\end{align}
						が成り立つので
						\begin{align}
							\psi,\ \forall y\, (\, \psi \rarrow \varphi(x/y)\, ) \provable{\mbox{{\bf HE}},\lang{\varepsilon}} 
							\varphi(x/\varepsilon x \negation \varphi)
						\end{align}
						となり,全称の導出
						(論理的定理\ref{logicalthm:derivation_of_universal_by_epsilon})
						\begin{align}
							\provable{\mbox{{\bf HE}},\lang{\varepsilon}} \varphi(x/\varepsilon x \negation \varphi)
							\rarrow \forall x \varphi
						\end{align}
						との三段論法より
						\begin{align}
							\psi,\ \forall y\, (\, \psi \rarrow \varphi(x/y)\, ) \provable{\mbox{{\bf HE}},\lang{\varepsilon}}
							\forall x \varphi
						\end{align}
						が従う.よって演繹定理より
						\begin{align}
							\provable{\mbox{{\bf HE}},\lang{\varepsilon}} \forall y\, (\, \psi \rarrow \varphi(x/y)\, )
							\rarrow (\, \psi \rarrow \forall x \varphi\, )
						\end{align}
						が得られる.
					
					\item[(EE)の証明]
						$\forall y\, (\, \varphi(x/y) \rarrow \psi\, ) 
						\rarrow (\, \exists x \varphi \rarrow \psi\, )$を示す.
						{\bf HE}の公理(UE)より
						\begin{align}
							\forall y\, (\, \varphi(x/y) \rarrow \psi\, ) \provable{\mbox{{\bf HE}},\lang{\varepsilon}}
							\varphi(x/\varepsilon x \varphi) \rarrow \psi
						\end{align}
						が成り立ち,他方で{\bf HE}の公理(EE)より
						\begin{align}
							\exists x \varphi \provable{\mbox{{\bf HE}},\lang{\varepsilon}} \varphi(x/\varepsilon x \varphi)
						\end{align}
						も成り立つので,三段論法より
						\begin{align}
							\exists x \varphi,\ \forall y\, (\, \varphi(x/y) \rarrow \psi\, ) \provable{\mbox{{\bf HE}},\lang{\varepsilon}} \psi
						\end{align}
						が成り立つ.よって演繹定理より
						\begin{align}
							\provable{\mbox{{\bf HE}},\lang{\varepsilon}} \forall y\, (\, \varphi(x/y) \rarrow \psi\, ) 
							\rarrow (\, \exists x \varphi \rarrow \psi\, )
						\end{align}
						が得られる.
						\QED
				\end{description}
		\end{description}
	\end{metaprf}
	
	\begin{screen}
		\begin{metathm}[$\Gamma$の定理は$\Sigma$の定理]
			$\psi$を$\lang{\in}$の文とするとき,
			$\Gamma \provable{\mbox{{\bf HK}},\lang{\in}} \psi$ならば
			$\Sigma \provable{\mbox{{\bf HE}},\lang{\varepsilon}} \psi$である.
		\end{metathm}
	\end{screen}
	
	\begin{metaprf}
		メタ定理\ref{metathm:theorems_in_HK_provable_in_HE}より
		$\Gamma \provable{\mbox{{\bf HK}},\lang{\in}} \psi$ならば
		$\Gamma \provable{\mbox{{\bf HE}},\lang{\varepsilon}} \psi$であるから,あとは
		$\Gamma$の公理が$\Sigma$から証明可能であることを示せばよい.
		$\Sigma$のものと違う$\Gamma$の公理は外延性,相等性,置換であるが,
		たとえば外延性
		\begin{align}
			\forall x\, \forall y\, (\, \forall z\, 
			(\, z \in x \lrarrow z \in y\, ) \rarrow x = y\, )
		\end{align}
		については
		\begin{align}
			a &\defeq \varepsilon x \negation \forall y\, (\, \forall z\, 
			(\, z \in x \lrarrow z \in y\, ) \rarrow x = y\, ), \\
			b &\defeq \varepsilon y \negation (\, \forall z\, 
			(\, z \in a \lrarrow z \in y\, ) \rarrow a = y\, ),
		\end{align}
		とおけば
		\begin{align}
			\Sigma \provable{\mbox{{\bf HE}},\lang{\varepsilon}} \forall z\, (\, z \in a \lrarrow z \in b\, ) \rarrow a = b
		\end{align}
		が成り立つので,全称の導出
		(論理的定理\ref{logicalthm:derivation_of_universal_by_epsilon})より
		\begin{align}
			\Sigma &\provable{\mbox{{\bf HE}},\lang{\varepsilon}} \forall y\, (\, \forall z\, 
			(\, z \in a \lrarrow z \in y\, ) \rarrow a = y\, ), \\
			\Sigma &\provable{\mbox{{\bf HE}},\lang{\varepsilon}} \forall x\, \forall y\, (\, \forall z\, 
			(\, z \in x \lrarrow z \in y\, ) \rarrow x = y\, )
		\end{align}
		が従う.相等性と置換の公理も同様にして導かれる.
		\QED
	\end{metaprf}
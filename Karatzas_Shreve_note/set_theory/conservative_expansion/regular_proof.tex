\section{正則証明}
\label{sec:regular_proof}
	この節では「$\Gamma$から$\psi$への{\bf HK}の証明で$\lang{\in}$の
	式の列であるものが取れる」ならば「$\Sigma$から$\psi$への{\bf HE}の証明で
	$\lang{\varepsilon}$の文の列であるものが取れる」ことを示す
	{\bf HK}の証明の中で汎化が使われている場合,その固有変項は
	適当な主要$\varepsilon$項に置き換えることになる.たとえば
	\begin{align}
		\psi(x/a)
	\end{align}
	から($\psi$は$x$のみ自由に現れる式とする)
	\begin{align}
		\forall x \psi
	\end{align}
	が汎化で導かれる場合,$a$を$\varepsilon x \negation \psi$に置き換えれば
	\begin{align}
		\psi(x/\varepsilon x \negation \psi), 
		\quad \psi(x/\varepsilon x \negation \psi) \rarrow \forall x \psi
	\end{align}
	から三段論法で$\forall x \psi$が出てくる.ここで注意しておくと,汎化の固有変項の条件より
	$a$は$\forall x \psi$に自由に現れないので,$a$は$\psi$にも自由に現れず,
	\begin{align}
		\psi(x/a)(a/\varepsilon x \negation \psi)
	\end{align}
	と
	\begin{align}
		\psi(x/\varepsilon x \negation \psi)
	\end{align}
	は一致しているのである.固有変項の置き換えは証明全体で一斉に行うので,
	二つの汎化に対して同じ固有変項が使われている場合は
	代入する主要$\varepsilon$項をうまく選ぶことが出来ない.
	従って,どの固有変項も一度の汎化にしか用いられないように証明を直す必要がある.
	
	\begin{screen}
		\begin{metadfn}[正則証明]
			{\bf 正則証明}\index{せいそくしょうめい@正則証明}{\bf (regular proof)}とは
			次を満たす{\bf HK}の証明$\varphi_{1},\cdots,\varphi_{n}$である.
			第一に,証明の中に現れるどの固有変項も一度の汎化にしか用いられない.
			第二に,$a$が$\varphi_{m}$から$\varphi_{k}$への汎化の固有変項ならば,
			$a$は$\varphi_{m+1}$以降の式には自由に現れない.
		\end{metadfn}
	\end{screen}
	
	{\bf HK}の任意の証明は正則なものに変換出来る
	(メタ定理\ref{metathm:regularization_of_HK_proof}).
	
	\begin{screen}
		\begin{metathm}[証明に現れる変項に代入しても証明]
		\label{metathm:substitute_HK_proof}
			$\mathscr{S}$を$\lang{\varepsilon}$の文からなる公理系とし,
			$\lang{\varepsilon}$の式の列$\varphi_{1},\cdots,\varphi_{n}$を
			$\mathscr{S}$から$\varphi_{n}$への{\bf HK}の証明とし,
			$a$をこの証明に自由に現れる\footnotemark
			変項とし,
			$b$をこの証明に現れない$\lang{\varepsilon}$の項とする.このとき
			\begin{description}
				\item[(1)] $b$が変項である場合,$\varphi_{1}(a/b),\cdots,\varphi_{n}(a/b)$
					は$\mathscr{S}$からの{\bf HK}の証明となる.
					
				\item[(2)] $b$が変項ではない場合\footnotemark,$a$がこの証明の固有変項でないなら
					$\varphi_{1}(a/b),\cdots,\varphi_{n}(a/b)$は
					$\mathscr{S}$からの{\bf HK}の証明となる.
					
				\item[(3)] (1)の場合も(2)の場合も,
					$\varphi_{i}$が{\bf HK}の公理なら$\varphi_{i}(a/b)$も{\bf HK}の
					公理であり,$\varphi_{i}$が$\mathscr{S}$の公理なら$\varphi_{i}(a/b)$も
					$\mathscr{S}$の公理であり,$\varphi_{i}$が$\varphi_{j}$と$\varphi_{k}$
					から三段論法で得られているなら$\varphi_{i}(a/b)$も$\varphi_{j}(a/b)$と
					$\varphi_{k}(a/b)$から三段論法で得られる.
					
				\item[(4)] $\varphi_{i}$が$\varphi_{j}$の汎化で得られていて,その固有変項が
					$e$であるとするき,(1)の場合も(2)の場合も$\varphi_{i}(a/b)$は
					$\varphi_{j}(a/b)$の汎化で得られ,固有変項は,
					$a$と$e$が同じなら$b$となり,$a$と$e$が違うなら$e$のままである.
					ただし(2)の場合は$a$は固有変項ではないとする.
			\end{description}
		\end{metathm}
	\end{screen}
	
	\footnotetext{
		「証明に自由に現れる」とは,$\varphi_{1},\cdots,\varphi_{n}$のうち少なくとも一本に
		$a$が自由に現れているという意味である.
	}
	\footnotetext{
		第\ref{sec:restriction_of_formulas}節の約束によって,
		この場合$b$は主要$\varepsilon$項である.
	}
	
	\begin{metaprf}
		$b$が変項であるか否かが関係するのは case5 である.
		式の列が証明であるための条件に照合していく.各$\varphi_{i}$に対して
		\begin{description}
			\item[case1] $\varphi_{i}$が{\bf HK}の命題論理の公理である場合,
				たとえば$\varphi_{i}$が
				\begin{align}
					\varphi \rarrow (\, \psi \rarrow \varphi\, )
				\end{align}
				なる形の公理ならば,$\varphi_{i}(a/b)$は
				\begin{align}
					\varphi(a/b) \rarrow (\, \psi(a/b) \rarrow \varphi(a/b)\, )
				\end{align}
				なる式であるから{\bf HK}の公理である.他の式も同様に結合形式は崩れない.
				
			\item[case2] $\varphi_{i}$が{\bf HK}の量化公理である場合,
				\begin{itemize}
					\item たとえば$\varphi_{i}$が{\bf HK}の(UI)
						\begin{align}
							\forall y\, (\, \psi \rarrow \varphi(x/y)\, )
							\rarrow (\, \psi \rarrow \forall x \varphi\, )
						\end{align}
						であるとする.このとき,
						\begin{itemize}
							\item $a$が$y$であれば$a$は$\varphi_{i}$には自由に
								現れないので$\varphi_{i}(a/b)$と$\varphi_{i}$は一致する.
							\item $a$が$y$と違うとき,$a$が$x$であれば$\varphi_{i}(a/b)$は
								\begin{align}
									\forall y\, (\, \psi(a/b) \rarrow \varphi(x/y)\, )
									\rarrow (\, \psi(a/b) \rarrow \forall x \varphi\, )
								\end{align}
								なる式となる.$b$は新しい変項であるから$\psi(a/b)$に$y$は
								自由に現れない.
							\item $a$が$y$と違うとき,$a$が$x$とも違うなら,
								$(\forall x \varphi)(a/b)$と
								$\forall x \varphi(a/b)$,および$\varphi(x/y)(a/b)$と
								$\varphi(a/b)(x/y)$は一致するので,$\varphi_{i}(a/b)$は
								\begin{align}
									\forall y\, (\, \psi(a/b) \rarrow \varphi(a/b)(x/y)\, )
									\rarrow (\, \psi(a/b) \rarrow \forall x \varphi(a/b)\, )
								\end{align}
								なる式となる.
						\end{itemize}
						ゆえにいずれの場合も$\varphi_{i}(a/b)$は{\bf HK}の(UI)
						となる.同様に$\varphi_{i}$が{\bf HK}の(EE)であるときも
						$\varphi_{i}(a/b)$は{\bf HK}の(EE)となる.
				
					\item たとえば$\varphi_{i}$が{\bf HK}の(EI)
						\begin{align}
							\varphi(x/t) \rarrow \exists x \varphi
						\end{align}
						であるとする.このとき,
						\begin{itemize}
							\item $a$と$x$が同じで,$a$と$t$が違えば,$\varphi_{i}$に
								$a$は自由に現れないので$\varphi_{i}(a/b)$は$\varphi_{i}$
								に一致する.
								
							\item $a$と$x$が同じで,$a$と$t$も同じであれば,
								$\varphi_{i}(a/b)$は
								\begin{align}
									\varphi(x/b) \rarrow \exists x \varphi
								\end{align}
								なる式となる($b$は新しい変項なので$\varphi$の中で$x$への代入
								について自由である).
								
							\item $a$が$x$と違うとき,$a$と$t$が同じであれば,
								$(\exists x \varphi)(a/b)$と$\exists x \varphi(a/b)$,
								および$\varphi(x/t)(a/b)$と$\varphi(a/b)(x/b)$は一致する
								ので,$\varphi_{i}(a/b)$は
								\begin{align}
									\varphi(a/b)(x/b) \rarrow \exists x \varphi(a/b)
								\end{align}
								なる式となる.
							
							\item $a$が$x$とも$t$とも違うとき,
								$(\exists x \varphi)(a/b)$と$\exists x \varphi(a/b)$,
								および$\varphi(x/t)(a/b)$と$\varphi(a/b)(x/t)$は一致する
								ので,$\varphi_{i}(a/b)$は
								\begin{align}
									\varphi(a/b)(x/t) \rarrow \exists x \varphi(a/b)
								\end{align}
								なる式となる.
						\end{itemize}		
						ゆえにいずれの場合も$\varphi_{i}(a/b)$は{\bf HK}の(EI)となる.
						同様に$\varphi_{i}$が{\bf HK}の(UE)であるときも
						$\varphi_{i}(a/b)$は{\bf HK}の(UE)となる.
				\end{itemize}
				
			\item[case3] $\varphi_{i}$が$\mathscr{S}$の公理である場合,
				$\varphi_{i}$は文なので$\varphi_{i}(a/b)$は$\varphi_{i}$である.
			
			\item[case4] $\varphi_{i}$が前の式$\varphi_{j},\varphi_{k}$から
				三段論法で得られるとき,$\varphi_{k}$が$\varphi_{j} \rarrow \varphi_{i}$
				なる形の式ならば$\varphi_{k}(a/b)$は
				\begin{align}
					\varphi_{j}(a/b) \rarrow \varphi_{i}(a/b)
				\end{align}
				なる式となる.つまり$\varphi_{i}(a/b)$は$\varphi_{j}(a/b)$と
				$\varphi_{k}(a/b)$から三段論法で得られる.
				
			\item[case5] $\varphi_{i}$が前の式$\varphi_{j}$から汎化で得られるとき
				(固有変項$e$),変項$x$と$x$が自由に現れる式$\psi$が取れて,
				$e$は$\psi$の中で$x$への代入について自由であり,$\varphi_{j}$は$\psi(x/e)$,
				$\varphi_{i}$は$\forall x \psi$なる式である.また$e$は$\psi$の中で$x$への代入
				について自由であり,$e$は$\forall x \psi$に自由に現れない.このとき
				
				\begin{description}
					\item[(1)] $b$が変項である場合,
						\begin{itemize}
							\item $a$と$e$が同じとき,$a$と$x$も同じなら,$\varphi_{j}(a/b)$は
								$\psi(x/b)$となり,$\forall x \psi$に$e$は自由に現れないので
								$\varphi_{i}(a/b)$は$\forall x \psi$のままである.
								
							\item $a$と$e$が同じとき,$a$が$x$が違うなら,
								$\psi$に$e$は自由に現れないので$\varphi_{j}(a/b)$は
								$\psi(x/b)$となり,$\forall x \psi$に$e$は自由に現れないので
								$\varphi_{i}(a/b)$は$\forall x \psi$のままである.
							
							\item $a$が$e$が違うとき,$a$と$x$が同じなら,
								$\varphi_{j}(a/b)$は$\psi(x/e)(x/b)$,つまり$\psi(x/e)$
								のままであり,$\varphi_{i}(a/b)$も$\forall x \psi$のままである.
							
							\item $a$が$e$が違うとき,$a$が$x$とも違うなら,
								$\varphi_{j}(a/b)$は$\psi(a/b)(x/e)$に一致し,
								$\varphi_{i}(a/b)$は$\forall x \psi(a/b)$となる.
								このとき$e$は$\psi(a/b)$の中で$x$への代入について自由であり,
								$\forall x \psi(a/b)$には自由に現れない.
						\end{itemize}
						
					\item[(2)] $b$が変項でない場合,$a$と$e$は違う変項とする.
						\begin{itemize}
							\item $a$が$x$であれば,$\psi(x/e)$に$a$は自由に現れないので
								$\varphi_{j}(a/b)$は$\psi(x/e)$のままであり,
								$\varphi_{i}(a/b)$も$\forall x \psi$のままである.
							
							\item $a$が$x$と違うとき,$\varphi_{j}(a/b)$は
								$\psi(a/b)(x/e)$に一致し,$\varphi_{i}(a/b)$は
								$\forall x \psi(a/b)$となる.
						\end{itemize}
				\end{description}
				ゆえに,いずれの場合も$\varphi_{i}(a/b)$は$\varphi_{j}(a/b)$の汎化で得られる.
				\QED
		\end{description}
	\end{metaprf}
	
	上のメタ定理において,$\varphi_{1},\cdots,\varphi_{n}$が$\lang{\in}$の式の列であるとき,
	$b$が変項ならば$\varphi_{1}(a/b),\cdots,\varphi_{n}(a/b)$もまた
	{\bf HK}の証明となるが,各式は$\lang{\in}$の式であるからこれは$\lang{\in}$の式からなる
	{\bf HK}の証明である.
	
	\begin{screen}
		\begin{metathm}[正則証明に現れる変項に代入しても正則]
		\label{metathm:substitute_regular_HK_proof}
			$\mathscr{S}$を$\lang{\varepsilon}$の文からなる公理系とし,
			$\lang{\varepsilon}$の式の列$\varphi_{1},\cdots,\varphi_{n}$を
			$\mathscr{S}$から$\varphi_{n}$への{\bf HK}の正則証明とし,
			$a$をこの証明に自由に現れる変項とし,
			$b$をこの証明に現れない変項とする.このとき
			$\varphi_{1}(a/b),\cdots,\varphi_{n}(a/b)$
			は$\mathscr{S}$から$\varphi_{n}(a/b)$への{\bf HK}の正則証明である.
			特に$a$が$\varphi_{1},\cdots,\varphi_{n}$の固有変項なら,
			$\varphi_{1}(a/b),\cdots,\varphi_{n}(a/b)$は
			$\mathscr{S}$から$\varphi_{n}$への{\bf HK}の正則証明である.
		\end{metathm}
	\end{screen}
	
	\begin{metaprf}
		メタ定理\ref{metathm:substitute_HK_proof}より
		$\varphi_{1}(a/b),\cdots,\varphi_{n}(a/b)$は$\mathscr{S}$から$\varphi_{n}(a/b)$への
		{\bf HK}の証明であるから,あとは汎化について見ればよい.
		\begin{description}
			\item[case1] $a$がいずれかの固有変項と一致しているとき,つまり
				$a$が$\varphi_{m}$から$\varphi_{k}$を導く汎化の固有変項であるとき,
				正則証明の条件より$a$が固有変項として使われる汎化はここだけであるが,
				$\varphi_{m}(a/b)$から$\varphi_{k}(a/b)$への汎化の固有変項は$b$となる.
				他の汎化の固有変項は据え置かれる(メタ定理\ref{metathm:substitute_HK_proof}より
				証明の中での汎化の適用の位置は不変).
				また$a$は$\varphi_{m+1}$以後の式には自由に現れないので,
				$b$は$\varphi_{m+1}(a/b)$以後の式には現れない.
				特に$\varphi_{m+1}(a/b),\cdots,\varphi_{n}(a/b)$は
				それぞれ$\varphi_{m+1},\cdots,\varphi_{n}$に一致する.
				ゆえに
				\begin{align}
					\varphi_{1}(a/b),\cdots,\varphi_{n}(a/b)
				\end{align}
				は$\mathscr{S}$から$\varphi_{n}$への{\bf HK}の正則証明である.
				
			\item[case2] $a$がどの固有変項とも違うとき,
				メタ定理\ref{metathm:substitute_HK_proof}よりどの汎化も
				固有変数は変わらないし,証明の中での汎化の適用の位置は不変である.ゆえに
				どの固有変項も一度の汎化にしか使われないし,汎化に使われた後の式には自由に現れない.
				\QED
		\end{description}
	\end{metaprf}
	
	\begin{screen}
		\begin{metathm}[どんな証明も正則化できる]
		\label{metathm:regularization_of_HK_proof}
			$\mathscr{S}$を$\lang{\in}$の文からなる公理系とし,
			$\varphi_{1},\cdots,\varphi_{n}$を$\mathscr{S}$から
			$\varphi_{n}$への{\bf HK}の証明で$\lang{\in}$の式の列であるものとするとき,
			$\mathscr{S}$から$\varphi_{n}$への
			{\bf HK}の正則証明で$\lang{\in}$の式の列であるものが取れる.
		\end{metathm}
	\end{screen}
	
	\begin{metaprf}
		証明内の汎化の数に関する帰納法で示す.
		\begin{description}
			\item[step1] 証明の中で汎化が一度も使われていなければその証明自体が正則証明である.
				
			\item[step2] $N$を任意の自然数として次を仮定する:
				\begin{itembox}[l]{IH (帰納法の仮定)}
					$\psi_{1},\cdots,\psi_{m}$を$\mathscr{S}$から
					$\psi_{m}$への{\bf HK}の証明で$\lang{\in}$の式の列であるものとするとき,
					証明の中の汎化の適用が$N$回以下ならば,$\mathscr{S}$から$\psi_{m}$への
					{\bf HK}の正則証明で$\lang{\in}$の式の列であるものが取れる.
				\end{itembox}
				
				$\varphi_{1},\cdots,\varphi_{n}$の中で$N+1$回汎化が使われているとし,
				最後の汎化が$\varphi_{i}$から$\varphi_{j}$の導出に使われているとし
				\footnote{
					$\varphi_{1},\cdots,\varphi_{n}$の中で汎化が使われている箇所を
					\begin{align}
						\varphi_{i_{1}} \quad &\mbox{から} \quad \varphi_{j_{1}}, \\
						\varphi_{i_{2}} \quad &\mbox{から} \quad \varphi_{j_{2}}, \\
						&\vdots \\
						\varphi_{i_{N+1}} \quad &\mbox{から} \quad \varphi_{j_{N+1}},
					\end{align}
					ただし$i_{1}<i_{2}<\cdots$,としたときの$i_{N+1}$が
					$i$である.ちなみに$j_{1},j_{2},\cdots,j_{N+1}$の
					大小は添数順になっているとは限らない.
				},
				この汎化の固有変項を$a$とする.任意の$k$ $(1 \leq k \leq i)$に対して
				\begin{align}
					\varphi_{1},\ \varphi_{2},\ \cdots,\ \varphi_{k}
				\end{align}
				は$\varphi_{k}$への{\bf HK}の証明であるが,汎化の適用は$N$回以下なので
				(IH)より$\varphi_{k}$への正則証明$P_{k}$が取れる.
				$P_{1},\cdots,P_{i}$に現れる固有変項には重複があるかもしれないから,
				固有変項を取り替えて重複を取らなくてはいけない.$P_{k}$に現れる固有変項が全てで
				\begin{align}
					a_{k}^{1},\ a_{k}^{2},\ \cdots,\ a_{k}^{e_{k}}
				\end{align}
				であるとして,$a_{1}^{1},\cdots,a_{i}^{e_{i}}$のそれぞれに対して
				新しい変項$b_{1}^{1},\cdots,b_{i}^{e_{i}}$を用意する.
				ただし$b_{1}^{1},\cdots,b_{i}^{e_{i}}$としては$P_{1},\cdots,P_{k}$および
				$\varphi_{1},\cdots,\varphi_{n}$の中に全く現れない変項を取る.
				このとき,$P_{k}$が
				\begin{align}
					\varphi_{k}^{1},\ \varphi_{k}^{2},\ \cdots,\ \varphi_{k}^{u_{k}}
				\end{align}
				である式の列として($\varphi_{k}^{u_{k}}$は$\varphi_{k}$である),$P_{k}'$を
				\begin{align}
					&\varphi_{k}^{1}(a_{k}^{1}/b_{k}^{1})\cdots(a_{k}^{e_{k}}/b_{1}^{e_{k}}), \\
					&\varphi_{k}^{2}(a_{k}^{1}/b_{k}^{1})\cdots(a_{k}^{e_{k}}/b_{1}^{e_{k}}), \\
					&\vdots \\ 
					&\varphi_{k}^{u_{k}}(a_{k}^{1}/b_{k}^{1})\cdots(a_{k}^{e_{k}}/b_{1}^{e_{k}})
				\end{align}
				とすると,メタ定理\ref{metathm:substitute_regular_HK_proof}より$P_{k}'$は
				$\varphi_{k}$への{\bf HK}の正則証明である($P_{k}$で汎化が使われていなければ
				$P_{k}'$は$P_{k}$とすればよい).
				
				ここで$b$を$P_{1}',\cdots,P_{i}'$および$\varphi_{i+1},\cdots,\varphi_{n}$の
				中に全く現れない変項とする.メタ定理\ref{metathm:substitute_HK_proof}より
				\begin{align}
					\varphi_{1}(a/b),\ \varphi_{2}(a/b),\ \cdots,\ \varphi_{i}(a/b)
				\end{align}
				は$\lang{\in}$の式からなる{\bf HK}の証明であり,また
				代入後も証明の中での汎化の位置は変わらないので,この証明の汎化は$N$回以下である.
				従って(IH)より$\varphi_{i}(a/b)$への正則証明が取れるから,これを$Q$とする.
				$Q$に現れる固有変項も$P_{1}',\cdots,P_{i}'$のものと被らないように
				取り替えて,そして得られる証明を$Q'$とする($Q$で汎化が使われていなければ
				$Q'$は$Q$とすればよい).$Q$の最終式$\varphi_{i}(a/b)$に$b$が自由に現れているので,
				$b$は$Q$の固有変項ではない.
				ゆえにメタ定理\ref{metathm:substitute_regular_HK_proof}より$Q'$もまた
				$\varphi_{i}(a/b)$への正則証明である.このとき
				\begin{align}
					P_{1}',P_{2}',\ \cdots,P_{i}',\ Q',\ \varphi_{i+1},\ 
					\cdots,\ \varphi_{j},\ \cdots,\ \varphi_{n}
					\label{seq:regularization_of_HK_proof_1}
				\end{align}
				は証明となる.実際,$P_{1}',\cdots,P_{i}',Q'$はそれぞれ証明であるし,
				またこの列には$\varphi_{1},\cdots,\varphi_{i}$が部分的に現れるので
				元の証明
				\begin{align}
					\varphi_{1},\cdots,\varphi_{n}
				\end{align}
				の構造も崩さない.従って,(\refeq{seq:regularization_of_HK_proof_1})の列のどの
				式も{\bf HK}の公理であるか$\mathscr{S}$の公理であるか前の式から
				三段論法で得られるか前の式から汎化で得られる,
				ちなみに$\varphi_{j}$は$\varphi_{i}(a/b)$から汎化で得られるし,
				$b$は$\varphi_{i+1},\cdots,\varphi_{n}$には現れない変項である.
				つまりこの証明の固有変数はいずれも汎化に使われた後は二度と自由に現れないので
				正則証明となっている.
				\QED
		\end{description}
	\end{metaprf}
	
	\begin{screen}
		\begin{metathm}[{\bf HK}の定理は{\bf HE}の定理]
		\label{metathm:theorems_in_HK_provable_in_HE}
			$\mathscr{S}$を$\lang{\in}$の文からなる公理系とし,
			$\psi$を$\lang{\in}$の文とするとき,
			$\mathscr{S} \provable{\mbox{{\bf HK}},\lang{\in}} \psi$ならば
			$\mathscr{S} \provable{\mbox{{\bf HE}},\lang{\varepsilon}} \psi$である.
		\end{metathm}
	\end{screen}
	
	\begin{metaprf}
		$\lang{\in}$の式の列$\varphi_{1},\cdots,\varphi_{n}$を
		$\mathscr{S}$から$\psi$への{\bf HK}の正則な証明とし
		(メタ定理\ref{metathm:regularization_of_HK_proof}),また
		\begin{align}
			a_{1},\cdots,a_{m}
		\end{align}
		をこの証明に使われる固有変項とし,$a_{1},a_{2},\cdots$の順番に汎化に用いられるとする.
		つまり,$\varphi_{1},\cdots,\varphi_{n}$の中で汎化が使われている箇所を
		\begin{align}
			\varphi_{i_{1}} \quad &\mbox{から} \quad \varphi_{j_{1}}, \\
			\varphi_{i_{2}} \quad &\mbox{から} \quad \varphi_{j_{2}}, \\
			&\vdots \\
			\varphi_{i_{m}} \quad &\mbox{から} \quad \varphi_{j_{m}},
		\end{align}
		ただし$i_{1}<i_{2}<\cdots$,としたとき,$a_{p}$は$\varphi_{i_{p}}$から$\varphi_{j_{p}}$
		への汎化の固有変項である.
	
		\begin{description}
			\item[step1]
				$\varphi_{1},\cdots,\varphi_{n}$の中に
				自由に現れる変項のうち,$a_{1},\cdots,a_{m}$以外の全てを
				$x_{1},\cdots,x_{k}$とする.これらに対し
				相異なる主要$\varepsilon$項$\tau_{1},\cdots,\tau_{k}$を用意して
				(これらは$\varphi_{1},\cdots,\varphi_{n}$に現れないものとする),
				自由に現れる全ての$x_{i}$に$\tau_{i}$を代入する($1 \leq i \leq n$).
				そして得られる式の列を$\widetilde{\varphi}_{1},\cdots,\widetilde{\varphi}_{n}$とする.
				この列は{\bf HK}の証明である(メタ定理\ref{metathm:substitute_HK_proof}).
				$\psi$は$\lang{\in}$の文なので$\widetilde{\varphi}_{n}$は$\psi$である.
				
			\item[step2]
				次に$a_{m},a_{m-1},\cdots$の順に固有変項を置き換える.$a_{m}$が
				\begin{align}
					\varphi(x/a_{m})
				\end{align}
				から
				\begin{align}
					\forall x \varphi
				\end{align}
				への汎化に使われているなら,$\widetilde{\varphi}_{1},\cdots,
				\widetilde{\varphi}_{n}$に自由に現れる$a_{m}$を全て
				$\varepsilon x \negation \varphi$に置き換える.
				ちなみにこのとき,$\varepsilon x \negation \varphi$を$\tau$とおけば
				\begin{align}
					\widetilde{\varphi}_{1}(a_{m}/\tau),\ 
					\widetilde{\varphi}_{2}(a_{m}/\tau),\ \cdots,\ 
					\widetilde{\varphi}_{i_{m}}(a_{m}/\tau)
				\end{align}
				は{\bf HK}の証明となっている(メタ定理\ref{metathm:substitute_HK_proof}).
				あとは,$\widetilde{\varphi}_{i_{m}}(a_{m}/\tau)$と$\forall x \varphi$の
				間に
				\begin{align}
					\varphi(x/\varepsilon x \negation \varphi) 
					\rarrow \forall x \varphi
				\end{align}
				の{\bf HE}の証明
				(論理的定理\ref{logicalthm:derivation_of_universal_by_epsilon})
				を挿入すれば,$\widetilde{\varphi}_{n}(a_{m}/\tau)$への,つまり$\psi$への
				{\bf HK}の証明が得られる.同じ要領で他の固有変項も
				主要$\varepsilon$項に置き換えていく.
				
			\item[step3]
				step2の終了後に得られる式の列を$\xi_{1},\cdots,\xi_{r}$
				とする.これらは全て$\lang{\varepsilon}$の文であり,
				そして各$\xi_{i}$は次のいずれかである:
				\begin{itemize}
					\item {\bf HE}の公理である.
					
					\item $\mathscr{S}$の公理である.
					
					\item 前の式$\xi_{j},\xi_{k}$から三段論法で得られる.
				\end{itemize}
			
				この列の中に{\bf HK}の公理(UI)と(EE)の形の式が残っている場合は
				まだ{\bf HE}の証明ではない.とはいえ下で示す通り(UI)と(EE)は{\bf HE}で証明できるから,
				$\xi_{1},\cdots,\xi_{r}$にある(UI)(EE)の前にその式への{\bf HE}の
				証明を挿入すればよい.
				
				\begin{description}
					\item[(UI)の証明]
						$\forall y\, (\, \psi \rarrow \varphi(x/y)\, ) 
						\rarrow (\, \psi \rarrow \forall x \varphi\, )$を示す.
						{\bf HE}の公理(UE)より
						\begin{align}
							\forall y\, (\, \psi \rarrow \varphi(x/y)\, ) \provable{\mbox{{\bf HE}},\lang{\varepsilon}} 
							\psi \rarrow \varphi(x/\varepsilon x \negation \varphi)
						\end{align}
						が成り立つので
						\begin{align}
							\psi,\ \forall y\, (\, \psi \rarrow \varphi(x/y)\, ) \provable{\mbox{{\bf HE}},\lang{\varepsilon}} 
							\varphi(x/\varepsilon x \negation \varphi)
						\end{align}
						となり,全称の導出
						(論理的定理\ref{logicalthm:derivation_of_universal_by_epsilon})
						\begin{align}
							\provable{\mbox{{\bf HE}},\lang{\varepsilon}} \varphi(x/\varepsilon x \negation \varphi)
							\rarrow \forall x \varphi
						\end{align}
						との三段論法より
						\begin{align}
							\psi,\ \forall y\, (\, \psi \rarrow \varphi(x/y)\, ) \provable{\mbox{{\bf HE}},\lang{\varepsilon}}
							\forall x \varphi
						\end{align}
						が従う.よって演繹定理より
						\begin{align}
							\provable{\mbox{{\bf HE}},\lang{\varepsilon}} \forall y\, (\, \psi \rarrow \varphi(x/y)\, )
							\rarrow (\, \psi \rarrow \forall x \varphi\, )
						\end{align}
						が得られる.
					
					\item[(EE)の証明]
						$\forall y\, (\, \varphi(x/y) \rarrow \psi\, ) 
						\rarrow (\, \exists x \varphi \rarrow \psi\, )$を示す.
						{\bf HE}の公理(UE)より
						\begin{align}
							\forall y\, (\, \varphi(x/y) \rarrow \psi\, ) \provable{\mbox{{\bf HE}},\lang{\varepsilon}}
							\varphi(x/\varepsilon x \varphi) \rarrow \psi
						\end{align}
						が成り立ち,他方で{\bf HE}の公理(EE)より
						\begin{align}
							\exists x \varphi \provable{\mbox{{\bf HE}},\lang{\varepsilon}} \varphi(x/\varepsilon x \varphi)
						\end{align}
						も成り立つので,三段論法より
						\begin{align}
							\exists x \varphi,\ \forall y\, (\, \varphi(x/y) \rarrow \psi\, ) \provable{\mbox{{\bf HE}},\lang{\varepsilon}} \psi
						\end{align}
						が成り立つ.よって演繹定理より
						\begin{align}
							\provable{\mbox{{\bf HE}},\lang{\varepsilon}} \forall y\, (\, \varphi(x/y) \rarrow \psi\, ) 
							\rarrow (\, \exists x \varphi \rarrow \psi\, )
						\end{align}
						が得られる.
						\QED
				\end{description}
		\end{description}
	\end{metaprf}
	
	\begin{screen}
		\begin{metathm}[$\Gamma$の定理は$\Sigma$の定理]
			$\psi$を$\lang{\in}$の文とするとき,
			$\Gamma \provable{\mbox{{\bf HK}},\lang{\in}} \psi$ならば
			$\Sigma \provable{\mbox{{\bf HE}},\lang{\varepsilon}} \psi$である.
		\end{metathm}
	\end{screen}
	
	\begin{metaprf}
		メタ定理\ref{metathm:theorems_in_HK_provable_in_HE}より
		$\Gamma \provable{\mbox{{\bf HK}},\lang{\in}} \psi$ならば
		$\Gamma \provable{\mbox{{\bf HE}},\lang{\varepsilon}} \psi$であるから,あとは
		$\Gamma$の公理が$\Sigma$から証明可能であることを示せばよい.
		$\Sigma$のものと違う$\Gamma$の公理は外延性,相等性,置換であるが,
		たとえば外延性
		\begin{align}
			\forall x\, \forall y\, (\, \forall z\, 
			(\, z \in x \lrarrow z \in y\, ) \rarrow x = y\, )
		\end{align}
		については
		\begin{align}
			a &\defeq \varepsilon x \negation \forall y\, (\, \forall z\, 
			(\, z \in x \lrarrow z \in y\, ) \rarrow x = y\, ), \\
			b &\defeq \varepsilon y \negation (\, \forall z\, 
			(\, z \in a \lrarrow z \in y\, ) \rarrow a = y\, ),
		\end{align}
		とおけば
		\begin{align}
			\Sigma \provable{\mbox{{\bf HE}},\lang{\varepsilon}} \forall z\, (\, z \in a \lrarrow z \in b\, ) \rarrow a = b
		\end{align}
		が成り立つので,全称の導出
		(論理的定理\ref{logicalthm:derivation_of_universal_by_epsilon})より
		\begin{align}
			\Sigma &\provable{\mbox{{\bf HE}},\lang{\varepsilon}} \forall y\, (\, \forall z\, 
			(\, z \in a \lrarrow z \in y\, ) \rarrow a = y\, ), \\
			\Sigma &\provable{\mbox{{\bf HE}},\lang{\varepsilon}} \forall x\, \forall y\, (\, \forall z\, 
			(\, z \in x \lrarrow z \in y\, ) \rarrow x = y\, )
		\end{align}
		が従う.相等性と置換の公理も同様にして導かれる.
		\QED
	\end{metaprf}
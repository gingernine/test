\section{正則証明}
	今度は逆に,$\lang{\in}$の式による{\bf HK}の証明から
	$\lang{\varepsilon}$の文による第\ref{chap:inference}章の証明を構成する.
	{\bf HK}の証明の中で汎化が使われている場合,その固有変項を
	適当な主要$\varepsilon$項に置き換えることになる.たとえば
	\begin{align}
		\psi(x/a)
	\end{align}
	から($\psi$は$x$のみ自由に現れる式とする)
	\begin{align}
		\forall x \psi
	\end{align}
	が汎化で導かれる場合,$a$を$\varepsilon x \negation \psi$とすれば
	\begin{align}
		\psi(x/\varepsilon x \negation \psi), 
		\quad \psi(x/\varepsilon x \negation \psi) \rarrow \forall x \psi
	\end{align}
	から三段論法で$\forall x \psi$が出てくる.
	しかし二つの汎化に対して同じ固有変項が使われている場合は,
	その固有変項を主要$\varepsilon$項に置き換えることが出来ない.
	ゆえに,一つの固有変項が一つの汎化にしか用いられないように証明を直す必要がある.
	
	\begin{screen}
		\begin{metadfn}[正則証明]
			{\bf 正則証明}\index{せいそくしょうめい@正則証明}{\bf (regular proof)}とは,
			その証明の中に現れる全ての固有変項について,
			一つの固有変項はただ一つの汎化にしか対応していないものである.
		\end{metadfn}
	\end{screen}
	
	\begin{screen}
		\begin{metathm}[どんな証明も正則化できる]
			
		\end{metathm}
	\end{screen}
	
	$\lang{\in}$の正則証明は$\lang{\varepsilon}$の証明に書き直せる.
	$\varphi$を$\lang{\in}$の文とし,$\varphi_{1},\cdots,\varphi_{n}$を
	$\varphi$への$\lang{\in}$の正則証明とする.そして
	\begin{align}
		a_{1},\cdots,a_{m}
	\end{align}
	をこの証明に使われる固有変項とする.
	\begin{description}
		\item[step1]
			まず$\varphi_{1},\cdots,\varphi_{n}$の中に
			自由に現れる変項のうち,$a_{1},\cdots,a_{m}$以外のものをすべて相異なる
			主要$\varepsilon$項に置き換える.たとえば$x$が自由に現れているなら,
			主要$\varepsilon$項$\tau$を取ってきて,$\varphi_{1},\cdots,\varphi_{n}$
			に自由に現れている$x$をいっせいに$\tau$に置き換えるといった要領である.
			$a_{1},\cdots,a_{m}$以外の自由な変項を全て置き換え終わった式の列を
			\begin{align}
				\hat{\varphi}_{1}, \cdots, \hat{\varphi}_{n}
			\end{align}
			と書く.$\varphi_{i}$が前の式から三段論法で得られているならば
			$\hat{\varphi}_{i}$も前の式から三段論法で得られる.
			
		\item[step2]
			次に固有変数を置き換える.$a_{i}$が
			\begin{align}
				\hat{\varphi}_{j} \equiv F \rarrow G(a_{i})
			\end{align}
			から
			\begin{align}
				\hat{\varphi}_{k} \equiv F \rarrow \forall x G
			\end{align}
			への全称汎化に使われている固有変数ならば,証明を通して自由に現れる全て$a_{i}$を
			$\varepsilon x \negation G$に置き換えて,
			置き換えた後の式$\tilde{\varphi}_{j}$と$\tilde{\varphi}_{k}$の間に
			\begin{align}
				&G(\varepsilon x \negation G) \rarrow \forall x G, \\
				&(\, G(\varepsilon x \negation G) \rarrow \forall x G\, )
				\rarrow (\, F \rarrow (\, G(\varepsilon x \negation G) \rarrow \forall x G\, )\, ), \\
				&F \rarrow (\, G(\varepsilon x \negation G) \rarrow \forall x G\, ), \\
				&(\, F \rarrow (\, G(\varepsilon x \negation G) \rarrow \forall x G\, )\, ) \rarrow
				(\, (\, F \rarrow G(\varepsilon x \negation G)\, ) \rarrow
				(\, F \rarrow \forall x G\, )\, ), \\
				&(\, F \rarrow G(\varepsilon x \negation G)\, ) \rarrow
				(\, F \rarrow \forall x G\, )
			\end{align}
			を差し入れる.
	\end{description}
	以上で$\lang{\varepsilon}$の文による$\varphi$への証明が得られる.
	
\section{$\mathcal{L}$の証明の変換}
	$\lang{\varepsilon}$の証明は$\mathcal{L}$の証明でもあるが,逆に
	$\mathcal{L}$の証明を$\lang{\varepsilon}$の証明にっ変換することも出来る.
	
	いま$\varphi$を$\lang{\varepsilon}$の文とし,$\varphi_{1},\cdots,\varphi_{n}$を
	$\varphi$への$\mathcal{L}$の証明とする.
	そして$\varphi_{i}$を$\lang{\varepsilon}$の式に書き直し,$\hat{\varphi}_{i}$と書く.
	一般に式の書き換えは新しく用意する変項の違いで一意性を欠くが,
	同じ式を書き換える際に変項を揃えれば解決できる.
	たとえば,$\mathcal{L}$の文の列
	\begin{align}
		\varphi,\quad \varphi \rarrow \psi,\quad \psi
	\end{align}
	を$\lang{\varepsilon}$の文に書き換えるときは,
	左の$\varphi$を$\hat{\varphi}$に書き換えたならば,
	真ん中の$\varphi \rarrow \psi$は$\hat{\varphi} \rarrow \tilde{\psi}$に書き換えて,
	右の$\psi$は$\tilde{\psi}$に書き換えればよい.また
	\begin{align}
		\exists x G(x) \rarrow G(\varepsilon x \hat{G}(x))
	\end{align}
	なる$\mathcal{L}$の文については,
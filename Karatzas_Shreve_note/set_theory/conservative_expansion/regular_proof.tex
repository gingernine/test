\section{正則証明}
	今度は逆に,$\lang{\in}$の式による{\bf HK}の証明から
	$\lang{\varepsilon}$の文による第\ref{chap:inference}章の証明を構成する.
	{\bf HK}の証明の中で汎化が使われている場合,その固有変項を
	適当な主要$\varepsilon$項に置き換えることになる.たとえば
	\begin{align}
		\psi(x/a)
	\end{align}
	から($\psi$は$x$のみ自由に現れる式とする)
	\begin{align}
		\forall x \psi
	\end{align}
	が汎化で導かれる場合,$a$を$\varepsilon x \negation \psi$に置き換えれば
	\begin{align}
		\psi(x/\varepsilon x \negation \psi), 
		\quad \psi(x/\varepsilon x \negation \psi) \rarrow \forall x \psi
	\end{align}
	から三段論法で$\forall x \psi$が出てくる.
	しかし二つの汎化に対して同じ固有変項が使われている場合は,
	その固有変項をどういった主要$\varepsilon$項に置き換えれば良いのかわからない.
	ゆえに,一つの固有変項が一つの汎化にしか用いられないように証明を直す必要がある.
	
	\begin{screen}
		\begin{metadfn}[正則証明]
			{\bf 正則証明}\index{せいそくしょうめい@正則証明}{\bf (regular proof)}とは,
			その証明の中に現れるどの固有変項も一度しか汎化に用いられていないものである.
		\end{metadfn}
	\end{screen}
	
	\begin{screen}
		\begin{metathm}[どんな証明も正則化できる]
			
		\end{metathm}
	\end{screen}
	
	\begin{metaprf}
		$\varphi$を$\lang{\in}$の文とし,$\varphi_{1},\cdots,\varphi_{n}$を
		$\lang{\in}$の式からなる$\varphi$への{\bf HK}の証明とする.
		$\varphi_{i}$から$\varphi_{j}$にかけて汎化が用いられ(固有変項$a$),
		$\varphi_{k}$から$\varphi_{\ell}$にかけて汎化が用いられているとき(固有変項$a$),
		$\varphi_{1},\cdots,\varphi_{n}$に自由に現れる$a$を$b$に置き換えたものを
		$\hat{\varphi}_{1},\cdots,\hat{\varphi}_{n}$と書けば,
		\begin{align}
			\varphi_{1},\cdots,\varphi_{j},
			\hat{\varphi}_{1},\cdots,\hat{\varphi}_{j-1},\hat{\varphi}_{j+1},
			\cdots,\hat{\varphi}_{n}
		\end{align}
		は$\varphi$への正則証明になっている.
		\QED
	\end{metaprf}
	
	$\lang{\in}$の正則証明は$\lang{\varepsilon}$の証明に書き直せる.
	$\psi$を$\lang{\in}$の文とし,
	\begin{align}
		\mathscr{S} \provable{\mbox{{\bf HK$\varepsilon$}}} \psi
	\end{align}
	$\varphi_{1},\cdots,\varphi_{n}$を$\psi$への$\lang{\in}$の正則証明とする.そして
	\begin{align}
		a_{1},\cdots,a_{m}
	\end{align}
	をこの証明に使われる固有変項とし,$a_{1},a_{2},\cdots$の順に汎化に用いられるとする.
	
	\begin{description}
		\item[step1]
			まず$\varphi_{1},\cdots,\varphi_{n}$の中に
			自由に現れる変項のうち,$a_{1},\cdots,a_{m}$以外のものをすべて相異なる
			主要$\varepsilon$項に置き換える.たとえば$x$が$\varphi_{1},\cdots,\varphi_{n}$
			のいずれかの式の中に自由に現れているなら,主要$\varepsilon$項$\tau$を取ってきて,
			$\varphi_{1},\cdots,\varphi_{n}$
			に自由に現れている$x$を一斉に$\tau$に置き換えるといった要領である.
			$a_{1},\cdots,a_{m}$以外の自由な変項を全て主要$\varepsilon$項に
			置き換え終わった式の列を
			\begin{align}
				\hat{\varphi}_{1}, \cdots, \hat{\varphi}_{n}
			\end{align}
			と書く.このとき,
			\begin{itemize}
				\item $\varphi_{i}$が(UI)と(EE)以外の{\bf HK}の公理ならば
					$\hat{\varphi}_{i}$は第\ref{chap:inference}章の推論法則である.
				\item $\varphi_{i}$が(UI)か(EE)ならば
					$\hat{\varphi}_{i}$は第\ref{chap:inference}章の体系から証明可能である(step3).
				\item $\varphi_{i}$が$\mathscr{S}$の公理ならば,
					$\varphi_{i}$は変項の置換による影響を受けないので
					$\hat{\varphi}_{i}$は$\varphi_{i}$と同一である.
				\item $\varphi_{i}$が前の式$\varphi_{j},\varphi_{k}$から
					三段論法で得られているならば,$\hat{\varphi}_{i}$も
					$\hat{\varphi}_{j},\hat{\varphi}_{k}$から三段論法で得られる.
				\item $\varphi_{i}$が前の式$\varphi_{j}$から
					汎化で得られているならば,$\hat{\varphi}_{i}$も
					$\hat{\varphi}_{j}$から汎化で得られる.
			\end{itemize}
			
		\item[step2]
			次に$a_{m},a_{m-1},\cdots$の順に固有変項を置き換える.$a_{i}$が
			\begin{align}
				F(a_{i})
			\end{align}
			から
			\begin{align}
				\forall x F
			\end{align}
			への汎化に使われているならば,$\hat{\varphi}_{1}, \cdots, \hat{\varphi}_{n}$に
			自由に現れる$a_{i}$を全て$\varepsilon x \negation F$に置き換えて,
			$\forall x F$の前の列に
			\begin{align}
				F(\varepsilon x \negation F) \rarrow \forall x F
			\end{align}
			を挿入すればよい.
		
		\item[step3]
			step2の終了後に得られる式の列を
			$\tilde{\varphi}_{1},\cdots,\tilde{\varphi}_{m}$とする.
			これらは全て$\lang{\varepsilon}$の文であるが,
			この中には{\bf HK}の公理(UI)と(EE)の形の式が残っている場合があるので,
			これはまだ第\ref{chap:inference}章における証明とはなっていない.
			(UI)と(EE)は第\ref{chap:inference}章の体系で証明可能であるから,
			$\tilde{\varphi}_{1},\cdots,\tilde{\varphi}_{m}$の中で
			(UI)または(EE)の形の式があれば,それより前の列に
			第\ref{chap:inference}章の体系からその式への証明を挿入すればよい.
			
			最後に(UI)と(EE)が第\ref{chap:inference}章の体系で証明可能であることを示す.
			
			\begin{description}
				\item[(UI)]
					$\forall y\, (\, \psi \rarrow \varphi(x/y)\, ) 
					\rarrow (\, \psi \rarrow \forall x \varphi\, )$を示す.
					全称記号に関する規則より
					\begin{align}
						\forall y\, (\, \psi \rarrow \varphi(x/y)\, ) \vdash 
						\psi \rarrow \varphi(x/\varepsilon x \negation \varphi)
					\end{align}
					が成り立つので
					\begin{align}
						\psi,\ \forall y\, (\, \psi \rarrow \varphi(x/y)\, ) \vdash 
						\varphi(x/\varepsilon x \negation \varphi)
					\end{align}
					が成り立ち,
					\begin{align}
						\vdash \varphi(x/\varepsilon x \negation \varphi)
						\rarrow \forall x \varphi
					\end{align}
					との三段論法より
					\begin{align}
						\psi,\ \forall y\, (\, \psi \rarrow \varphi(x/y)\, ) \vdash
						\forall x \varphi
					\end{align}
					が従う.よって演繹規則より
					\begin{align}
						\vdash \forall y\, (\, \psi \rarrow \varphi(x/y)\, )
						\rarrow (\, \psi \rarrow \forall x \varphi\, )
					\end{align}
					が得られる.
					\QED
					
				\item[(EE)]
					$\forall y\, (\, \varphi(x/y) \rarrow \psi\, ) 
					\rarrow (\, \exists x \varphi \rarrow \psi\, )$を示す.
					全称記号に関する規則より
					\begin{align}
						\forall y\, (\, \varphi(x/y) \rarrow \psi\, ) \vdash
						\varphi(x/\varepsilon x \varphi) \rarrow \psi
					\end{align}
					が成り立ち,他方で
					\begin{align}
						\exists x \varphi \vdash \varphi(x/\varepsilon x \varphi)
					\end{align}
					も成り立つので,三段論法より
					\begin{align}
						\exists x \varphi,\ \forall y\, (\, \varphi(x/y) \rarrow \psi\, ) \vdash \psi
					\end{align}
					が成り立つ.よって演繹規則より
					\begin{align}
						\vdash \forall y\, (\, \varphi(x/y) \rarrow \psi\, ) 
						\rarrow (\, \exists x \varphi \rarrow \psi\, )
					\end{align}
					が得られる.
					\QED
			\end{description}
	\end{description}
	以上で$\lang{\varepsilon}$の文による$\varphi$への証明が得られる.
	
\section{$\mathcal{L}$の証明の変換}
	$\lang{\varepsilon}$の証明は$\mathcal{L}$の証明でもあるが,逆に
	$\mathcal{L}$の証明を$\lang{\varepsilon}$の証明にっ変換することも出来る.
	
	いま$\varphi$を$\lang{\varepsilon}$の文とし,$\varphi_{1},\cdots,\varphi_{n}$を
	$\varphi$への$\mathcal{L}$の証明とする.
	そして$\varphi_{i}$を$\lang{\varepsilon}$の式に書き直し,$\hat{\varphi}_{i}$と書く.
	一般に式の書き換えは新しく用意する変項の違いで一意性を欠くが,
	同じ式を書き換える際に変項を揃えれば解決できる.
	たとえば,$\mathcal{L}$の文の列
	\begin{align}
		\varphi,\quad \varphi \rarrow \psi,\quad \psi
	\end{align}
	を$\lang{\varepsilon}$の文に書き換えるときは,
	左の$\varphi$を$\hat{\varphi}$に書き換えたならば,
	真ん中の$\varphi \rarrow \psi$は$\hat{\varphi} \rarrow \tilde{\psi}$に書き換えて,
	右の$\psi$は$\tilde{\psi}$に書き換えればよい.また
	\begin{align}
		\exists x G(x) \rarrow G(\varepsilon x \hat{G}(x))
	\end{align}
	なる$\mathcal{L}$の文については,
\section{正則証明}
	今度は逆に,$\lang{\in}$の式からなる{\bf HK}の証明から
	$\lang{\varepsilon}$の文からなる{\bf HE}の証明を構成する.
	{\bf HK}の証明の中で汎化が使われている場合,その固有変項は
	適当な主要$\varepsilon$項に置き換えることになる.たとえば
	\begin{align}
		\psi(x/a)
	\end{align}
	から($\psi$は$x$のみ自由に現れる式とする)
	\begin{align}
		\forall x \psi
	\end{align}
	が汎化で導かれる場合,$a$を$\varepsilon x \negation \psi$に置き換えれば
	\begin{align}
		\psi(x/\varepsilon x \negation \psi), 
		\quad \psi(x/\varepsilon x \negation \psi) \rarrow \forall x \psi
	\end{align}
	から三段論法で$\forall x \psi$が出てくる.ここで注意しておくと,汎化の固有変項の条件より
	$a$は$\forall x \psi$に自由に現れないので,$a$は$\psi$にも自由に現れず,
	\begin{align}
		\psi(x/a)(a/\varepsilon x \negation \psi)
	\end{align}
	と
	\begin{align}
		\psi(x/\varepsilon x \negation \psi)
	\end{align}
	は一致しているのである.固有変項の置き換えは証明全体で一斉に行うので,
	二つの汎化に対して同じ固有変項が使われている場合は
	代入する主要$\varepsilon$項をうまく選ぶことが出来ない.
	従って,どの固有変項も一度の汎化にしか用いられないように証明を直す必要がある.
	
	\begin{screen}
		\begin{metadfn}[正則証明]
			{\bf 正則証明}\index{せいそくしょうめい@正則証明}{\bf (regular proof)}とは,
			その証明の中に現れるどの固有変項も一度の汎化にしか用いられていないものである.
		\end{metadfn}
	\end{screen}
	
	\begin{screen}
		\begin{metathm}[証明の変項は取り替えても可能]
			$\varphi_{1},\cdots,\varphi_{n}$を$\Gamma$から$\varphi_{n}$への{\bf HK}の
			証明とし,$a$をこの証明に自由に現れる変項とし,$b$をこの証明に現れない変項とする.このとき
			\begin{align}
				\varphi_{1}(a/b),\ \varphi_{2}(a/b),\ \cdots,\ \varphi_{n}(a/b)
			\end{align}
			は$\Gamma$から$\varphi_{n}(a/b)$への{\bf HK}の証明となる.
		\end{metathm}
	\end{screen}
	
	\begin{metaprf}
		式の列が証明であるための条件に照合していく.各$\varphi_{i}$に対して
		\begin{description}
			\item[case1] $\varphi_{i}$が{\bf HK}の公理である場合,たとえば$\varphi_{i}$が
				\begin{align}
					\varphi \rarrow (\, \psi \rarrow \varphi\, )
				\end{align}
				なる形の公理ならば,$\varphi_{i}(a/b)$は
				\begin{align}
					\varphi(a/b) \rarrow (\, \psi(a/b) \rarrow \varphi(a/b)\, )
				\end{align}
				なる式であるから{\bf HK}の公理である.$\varphi_{i}$が
				\begin{align}
					\forall y\, (\, \psi \rarrow \varphi(x/y)\, )
					\rarrow (\, \psi \rarrow \forall x \varphi\, )
				\end{align}
				なる形の公理ならば,$\varphi_{i}(a/b)$は
				\begin{align}
					\forall y\, (\, \psi \rarrow \varphi(x/y)\, )
					\rarrow (\, \psi \rarrow \forall x \varphi\, )
				\end{align}
				
			\item[case2] $\varphi_{i}$が$\Gamma$の公理である場合,
				$\varphi_{i}$は文なので$\varphi_{i}(a/b)$は$\varphi_{i}$である.
			
			\item[case3] $\varphi_{i}$が前の式$\varphi_{j},\varphi_{k}$から
				三段論法で得られるとき,$\varphi_{k}$が$\varphi_{j} \rarrow \varphi_{i}$
				なる形の式であるとすれば,$\varphi_{k}(a/b)$は
				\begin{align}
					\varphi_{j}(a/b) \rarrow \varphi_{i}(a/b)
				\end{align}
				となる.つまり$\varphi_{i}(a/b)$は$\varphi_{j}(a/b)$と$\varphi_{k}(a/b)$から
				三段論法で得られる.
				
			\item[case4] $\varphi_{i}$が前の式$\varphi_{j}$から汎化で得られるとき,
				変項$e,x$と$x$が自由に現れる式$\psi$が取れて,$\varphi_{j}$は$\psi(x/e)$,
				$\varphi_{i}$は$\forall x \psi$なる式である.また$e$は$\psi$の中で$x$への代入
				について自由であり,$e$は$\forall x \psi$に自由に現れない.
				このとき$\varphi_{j}(a/b)$は
				\begin{align}
					\psi(a/b)(x/e)
				\end{align}
				となり,$\varphi_{i}(a/b)$は
				\begin{align}
					\forall x \psi(a/b)
				\end{align}
				となるので,$\varphi_{i}(a/b)$は$\varphi_{j}(a/b)$から汎化で得られる.
				\QED
		\end{description}
	\end{metaprf}
	
	\begin{screen}
		\begin{metathm}[どんな証明も正則化できる]
			$\varphi_{1},\cdots,\varphi_{n}$を$\Gamma$から$\varphi_{n}$への{\bf HK}の
			証明とするとき,$\Gamma$から$\varphi_{n}$への{\bf HK}の正則証明が得られる.
		\end{metathm}
	\end{screen}
	
	\begin{metaprf}
		$\varphi_{1},\cdots,\varphi_{n}$の中で汎化が使われていなければ
		これ自体が正則証明である.汎化が使われている場合,
		使われている箇所を
		\begin{align}
			\varphi_{i_{1}} \quad &\mbox{から} \quad \varphi_{j_{1}}, \\
			\varphi_{i_{2}} \quad &\mbox{から} \quad \varphi_{j_{2}}, \\
			&\vdots \\
			\varphi_{i_{\ell}} \quad &\mbox{から} \quad \varphi_{j_{\ell}}
		\end{align}
		とすべて列挙し(ただし$i_{1} < i_{2} < \cdots < i_{\ell}$),
		$a_{1},\cdots,a_{\ell}$をそれぞれの固有変項とする.
		ただし,もしかすると$a_{2}$と$a_{5}$は同じ文字$x$であるかもしれない.
		いま想定しているのはこのような状況であり,これから正則証明を構成するのである.
		
		$\varphi_{i_{1}}$の直後に$\varphi_{j_{1}}$を移動し,
		$\varphi_{i_{2}}$の直後に$\varphi_{j_{2}}$に移動し,…,
		$\varphi_{i_{\ell}}$の直後に$\varphi_{j_{\ell}}$を移動することによって
		$\varphi_{1},\cdots,\varphi_{n}$を並べ替えたものを
		\begin{align}
			\psi_{1},\cdots,\psi_{n}
		\end{align}
		と書けば,これもまた{\bf HK}の証明となっている.なぜならこの並び替えでは
		三段論法と汎化の順番が崩れないためである.並び替えによって
		$\varphi_{i_{1}},\cdots,\varphi_{i_{\ell}}$の位置も変動しうるが,
		動いた先をそれぞれ$\psi_{k_{1}},\cdots,\psi_{k_{\ell}}$とする.
		そして$b_{1},\cdots,b_{\ell}$を$\varphi_{i_{1}},\cdots,\varphi_{i_{\ell}}$に
		現れない相異なる変項とする.このとき
		\begin{align}
			&\psi_{1}(a_{1}/b_{1}),\ \cdots,\ \textcolor{red}{\psi_{k_{1}+1}(a_{1}/b_{1})}, \\
			&\psi_{1}(a_{2}/b_{2}),\ \cdots,\ \psi_{k_{1}+1}(a_{2}/b_{2}),\ \cdots,\ \textcolor{red}{\psi_{k_{2}+1}(a_{2}/b_{2})}, \\
			&\psi_{1}(a_{3}/b_{3}),\ \cdots,\ \psi_{k_{1}+2}(a_{3}/b_{3}),\ \cdots,\ \psi_{k_{2}+1}(a_{3}/b_{3}),\ \cdots,\ \textcolor{red}{\psi_{k_{3}+1}(a_{3}/b_{3})}, \\
			&\vdots \\
			&\psi_{1}(a_{\ell}/b_{\ell}),\ \cdots,\ \psi_{k_{1}+1}(a_{\ell}/b_{\ell}),\ \cdots,\ \psi_{k_{2}+1}(a_{\ell}/b_{\ell}),\ \cdots,\ 
			\psi_{k_{3}+1}(a_{\ell}/b_{\ell}), \cdots,\ \textcolor{red}{\psi_{k_{\ell}+1}(a_{\ell}/b_{\ell})}, \\
			&\psi_{k_{\ell}+2}(a_{\ell}/b_{\ell}),\ \cdots,\ \psi_{n}(a_{\ell}/b_{\ell})
		\end{align}
		は{\bf HK}の証明となっている.ところで,固有変項の条件より$a_{1}$は$\psi_{k_{1}+1}$に
		自由に現れないので$\psi_{k_{1}+1}(a_{1}/b_{1})$は$\psi_{k_{1}+1}$に一致する.
		同様に,赤字の$\psi_{k_{2}+1}(a_{2}/b_{2}),\ \psi_{k_{3}+1}(a_{3}/a_{3}),\ 
		\cdots,\ \psi_{k_{\ell}+1}(a_{\ell}/b_{\ell})$はそれぞれ
		$\psi_{k_{2}+1},\ \psi_{k_{3}+1},\ \cdots,\ \psi_{k_{\ell}+1}$と同じ式である.つまり
		\begin{align}
			&\psi_{1}(a_{1}/b_{1}),\ \cdots,\ \textcolor{red}{\psi_{k_{1}+1}}, \\
			&\psi_{1}(a_{2}/b_{2}),\ \cdots,\ \psi_{k_{1}+1}(a_{2}/b_{2}),\ \cdots,\ \textcolor{red}{\psi_{k_{2}+1}}, \\
			&\psi_{1}(a_{3}/b_{3}),\ \cdots,\ \psi_{k_{1}+2}(a_{3}/b_{3}),\ \cdots,\ \psi_{k_{2}+1}(a_{3}/b_{3}),\ \cdots,\ \textcolor{red}{\psi_{k_{3}+1}}, \\
			&\vdots \\
			&\psi_{1}(a_{\ell}/b_{\ell}),\ \cdots,\ \psi_{k_{1}+1}(a_{\ell}/b_{\ell}),\ \cdots,\ \psi_{k_{2}+1}(a_{\ell}/b_{\ell}),\ \cdots,\ 
			\psi_{k_{3}+1}(a_{\ell}/b_{\ell}), \cdots,\ \textcolor{red}{\psi_{k_{\ell}+1}}, \\
			&\psi_{k_{\ell}+2}(a_{\ell}/b_{\ell}),\ \cdots,\ \psi_{n}(a_{\ell}/b_{\ell})
		\end{align}
		は{\bf HK}の証明である.
		
		$\varphi$を$\lang{\in}$の文とし,$\varphi_{1},\cdots,\varphi_{n}$を
		$\lang{\in}$の式からなる$\varphi$への{\bf HK}の証明とする.
		$\varphi_{i}$から$\varphi_{j}$にかけて汎化が用いられ(固有変項$a$),
		$\varphi_{k}$から$\varphi_{\ell}$にかけて汎化が用いられているとき(固有変項$a$),
		$\varphi_{1},\cdots,\varphi_{n}$に自由に現れる$a$を$b$に置き換えたものを
		$\hat{\varphi}_{1},\cdots,\hat{\varphi}_{n}$と書けば,
		\begin{align}
			\varphi_{1},\cdots,\varphi_{j},
			\hat{\varphi}_{1},\cdots,\hat{\varphi}_{j-1},\hat{\varphi}_{j+1},
			\cdots,\hat{\varphi}_{n}
		\end{align}
		は$\varphi$への正則証明になっている.
		\QED
	\end{metaprf}
	
	\begin{screen}
		\begin{metathm}[{\bf HK}の定理は{\bf HE}の定理]
		\label{metathm:theorems_in_HK_provable_in_HE}
			$\mathscr{S}$を$\lang{\varepsilon}$の文からなる公理系とし,
			$\psi$を$\lang{\in}$の文とするとき,
			$\mathscr{S} \provable{\mbox{{\bf HK}}} \psi$ならば
			$\mathscr{S} \provable{\mbox{{\bf HE}}} \psi$である.
			ただし$\psi$への{\bf HK}の証明は$\lang{\in}$の式だけからなるものとする.
		\end{metathm}
	\end{screen}
	
	\begin{sketch}
		$\lang{\in}$の式の列$\varphi_{1},\cdots,\varphi_{n}$を
		$\mathscr{S}$から$\psi$への{\bf HK}の正則な証明とし,また
		\begin{align}
			a_{1},\cdots,a_{m}
		\end{align}
		をこの証明に使われる固有変項とし,$a_{1},a_{2},\cdots$の順番に汎化に用いられるとする.
	
		\begin{description}
			\item[step1]
				まず$\varphi_{1},\cdots,\varphi_{n}$の中に
				自由に現れる変項のうち,$a_{1},\cdots,a_{m}$以外のものをすべて相異なる
				主要$\varepsilon$項に置き換える.たとえば$x$が$\varphi_{1},\cdots,\varphi_{n}$
				のいずれかの式の中に自由に現れているなら,主要$\varepsilon$項$\tau$を取ってきて,
				$\varphi_{1},\cdots,\varphi_{n}$
				に自由に現れている$x$を一斉に$\tau$に置き換えるといった要領である.
				$a_{1},\cdots,a_{m}$以外の自由な変項を全て主要$\varepsilon$項に
				置き換え終わった式の列を
				\begin{align}
					\hat{\varphi}_{1}, \cdots, \hat{\varphi}_{n}
				\end{align}
				と書く.このとき,
				\begin{itemize}
					\item $\varphi_{i}$が(UI)と(EE)以外の{\bf HK}の公理ならば
						$\hat{\varphi}_{i}$は第\ref{chap:inference}章の推論法則である.
					\item $\varphi_{i}$が(UI)か(EE)ならば
						$\hat{\varphi}_{i}$は{\bf HE}から証明可能である(step3).
					\item $\varphi_{i}$が$\mathscr{S}$の公理ならば,
						$\varphi_{i}$は変項の置換による影響を受けないので
						$\hat{\varphi}_{i}$は$\varphi_{i}$と同一である.
					\item $\varphi_{i}$が前の式$\varphi_{j},\varphi_{k}$から
						三段論法で得られているならば,$\hat{\varphi}_{i}$も
						$\hat{\varphi}_{j},\hat{\varphi}_{k}$から三段論法で得られる.
					\item $\varphi_{i}$が前の式$\varphi_{j}$から
						汎化で得られているならば,$\hat{\varphi}_{i}$も
						$\hat{\varphi}_{j}$から汎化で得られる.
				\end{itemize}
			
			\item[step2]
				次に$a_{m},a_{m-1},\cdots$の順に固有変項を置き換える.$a_{m}$が
				\begin{align}
					F(x/a_{m})
				\end{align}
				から
				\begin{align}
					\forall x F
				\end{align}
				への汎化に使われているならば,$\hat{\varphi}_{1}, \cdots, \hat{\varphi}_{n}$に
				自由に現れる$a_{m}$を全て$\varepsilon x \negation F$に置き換えて,
				$\forall x F$の前の列に
				\begin{align}
					F(x/\varepsilon x \negation F) \rarrow \forall x F
				\end{align}
				への{\bf HE}の証明を挿入すればよい
				(推論法則\ref{logicalthm:derivation_of_universal_by_epsilon}).
				同様にして$a_{m-1},\cdots,a_{1}$も主要$\varepsilon$項に置き換えていく.
				
			\item[step3]
				step2の終了後に得られる式の列を
				$\tilde{\varphi}_{1},\cdots,\tilde{\varphi}_{m}$とする.
				これらは全て$\lang{\varepsilon}$の文であるが,
				この列の中に{\bf HK}の公理(UI)と(EE)の形の式が残っている場合は
				まだ{\bf HE}の証明ではない.とはいえ下で示す通り(UI)と(EE)は{\bf HE}で証明できるから,
				$\tilde{\varphi}_{1},\cdots,\tilde{\varphi}_{m}$の中で
				(UI)または(EE)の形の式があれば,その式の前の列にその式への{\bf HE}の
				証明を挿入すればよい.
			
				最後に(UI)と(EE)が{\bf HE}で証明可能であることを示す.
				\begin{description}
					\item[(UI)]
						$\forall y\, (\, \psi \rarrow \varphi(x/y)\, ) 
						\rarrow (\, \psi \rarrow \forall x \varphi\, )$を示す.
						{\bf HE}の公理(UE)より
						\begin{align}
							\forall y\, (\, \psi \rarrow \varphi(x/y)\, ) \provable{\mbox{{\bf HE}}} 
							\psi \rarrow \varphi(x/\varepsilon x \negation \varphi)
						\end{align}
						が成り立つので
						\begin{align}
							\psi,\ \forall y\, (\, \psi \rarrow \varphi(x/y)\, ) \provable{\mbox{{\bf HE}}} 
							\varphi(x/\varepsilon x \negation \varphi)
						\end{align}
						となり,全称の導出
						(推論法則\ref{logicalthm:derivation_of_universal_by_epsilon})
						\begin{align}
							\provable{\mbox{{\bf HE}}} \varphi(x/\varepsilon x \negation \varphi)
							\rarrow \forall x \varphi
						\end{align}
						との三段論法より
						\begin{align}
							\psi,\ \forall y\, (\, \psi \rarrow \varphi(x/y)\, ) \provable{\mbox{{\bf HE}}}
							\forall x \varphi
						\end{align}
						が従う.よって演繹定理より
						\begin{align}
							\provable{\mbox{{\bf HE}}} \forall y\, (\, \psi \rarrow \varphi(x/y)\, )
							\rarrow (\, \psi \rarrow \forall x \varphi\, )
						\end{align}
						が得られる.
					
					\item[(EE)]
						$\forall y\, (\, \varphi(x/y) \rarrow \psi\, ) 
						\rarrow (\, \exists x \varphi \rarrow \psi\, )$を示す.
						{\bf HE}の公理(UE)より
						\begin{align}
							\forall y\, (\, \varphi(x/y) \rarrow \psi\, ) \provable{\mbox{{\bf HE}}}
							\varphi(x/\varepsilon x \varphi) \rarrow \psi
						\end{align}
						が成り立ち,他方で{\bf HE}の公理(EE)より
						\begin{align}
							\exists x \varphi \provable{\mbox{{\bf HE}}} \varphi(x/\varepsilon x \varphi)
						\end{align}
						も成り立つので,三段論法より
						\begin{align}
							\exists x \varphi,\ \forall y\, (\, \varphi(x/y) \rarrow \psi\, ) \provable{\mbox{{\bf HE}}} \psi
						\end{align}
						が成り立つ.よって演繹定理より
						\begin{align}
							\provable{\mbox{{\bf HE}}} \forall y\, (\, \varphi(x/y) \rarrow \psi\, ) 
							\rarrow (\, \exists x \varphi \rarrow \psi\, )
						\end{align}
						が得られる.
						\QED
				\end{description}
		\end{description}
	\end{sketch}
	
	\begin{screen}
		\begin{metathm}[$\Gamma$の定理は$\Sigma$の定理]
			$\psi$を$\lang{\in}$の文とするとき,
			$\Gamma \provable{\mbox{{\bf HK}}} \psi$ならば
			$\Sigma \provable{\mbox{{\bf HE}}} \psi$である.
			ただし$\psi$への{\bf HK}の証明は$\lang{\in}$の式だけからなるものとする.
		\end{metathm}
	\end{screen}
	
	\begin{sketch}
		$\Gamma \provable{\mbox{{\bf HK}}} \psi$ならば
		メタ定理\ref{metathm:theorems_in_HK_provable_in_HE}より
		$\Gamma \provable{\mbox{{\bf HE}}} \psi$であるから,あとは
		$\Gamma$の公理が$\Sigma$から証明可能であることを示せばよい.
		$\Sigma$のものと違う$\Gamma$の公理は外延性,相等性,置換であるが,
		たとえば外延性公理
		\begin{align}
			\forall x\, \forall y\, (\, \forall z\, 
			(\, z \in x \lrarrow z \in y\, ) \rarrow x = y\, )
		\end{align}
		については
		\begin{align}
			a &\defeq \varepsilon x \negation \forall y\, (\, \forall z\, 
			(\, z \in x \lrarrow z \in y\, ) \rarrow x = y\, ), \\
			b &\defeq \varepsilon y \negation (\, \forall z\, 
			(\, z \in a \lrarrow z \in y\, ) \rarrow a = y\, ),
		\end{align}
		とおけば
		\begin{align}
			\Sigma \vdash \forall z\, (\, z \in a \lrarrow z \in b\, ) \rarrow a = b
		\end{align}
		が成り立つので,全称の導出
		(推論法則\ref{logicalthm:derivation_of_universal_by_epsilon})より
		\begin{align}
			\Sigma &\vdash \forall y\, (\, \forall z\, 
			(\, z \in a \lrarrow z \in y\, ) \rarrow a = y\, ), \\
			\Sigma &\vdash \forall x\, \forall y\, (\, \forall z\, 
			(\, z \in x \lrarrow z \in y\, ) \rarrow x = y\, )
		\end{align}
		が従う.相等性と置換の公理も同様にして導かれる.
		\QED
	\end{sketch}
	
\section{$\mathcal{L}$の証明の変換}
	$\lang{\varepsilon}$の証明は$\mathcal{L}$の証明でもあるが,逆に
	$\mathcal{L}$の証明を$\lang{\varepsilon}$の証明にっ変換することも出来る.
	
	いま$\varphi$を$\lang{\varepsilon}$の文とし,$\varphi_{1},\cdots,\varphi_{n}$を
	$\varphi$への$\mathcal{L}$の証明とする.
	そして$\varphi_{i}$を$\lang{\varepsilon}$の式に書き直し,$\hat{\varphi}_{i}$と書く.
	一般に式の書き換えは新しく用意する変項の違いで一意性を欠くが,
	同じ式を書き換える際に変項を揃えれば解決できる.
	たとえば,$\mathcal{L}$の文の列
	\begin{align}
		\varphi,\quad \varphi \rarrow \psi,\quad \psi
	\end{align}
	を$\lang{\varepsilon}$の文に書き換えるときは,
	左の$\varphi$を$\hat{\varphi}$に書き換えたならば,
	真ん中の$\varphi \rarrow \psi$は$\hat{\varphi} \rarrow \tilde{\psi}$に書き換えて,
	右の$\psi$は$\tilde{\psi}$に書き換えればよい.また
	\begin{align}
		\exists x G(x) \rarrow G(\varepsilon x \hat{G}(x))
	\end{align}
	なる$\mathcal{L}$の文については,
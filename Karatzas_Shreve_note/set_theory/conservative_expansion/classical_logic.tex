\section{古典論理}
	\begin{screen}
		\begin{logicalaxm}[{\bf HK}の公理(命題論理)]
			$\varphi$と$\psi$と$\xi$を式とするとき,次は{\bf HK}の公理である.
			\begin{description}
				\item[(S)] $(\, \varphi \rarrow (\, \psi \rarrow \chi\, )\, ) 
					\rarrow (\, (\, \varphi \rarrow \psi\, )
					\rarrow (\, \varphi \rarrow \chi\, )\, ).$
				\item[(K)] $\varphi \rarrow (\, \psi \rarrow \varphi\, ).$
				\item[(CTD1)] $\varphi \rarrow (\, \negation \varphi \rarrow \bot\, ).$
				\item[(CTD2)] $\negation \varphi \rarrow (\, \varphi \rarrow \bot\, ).$
				\item[(NI)] $(\, \varphi \rarrow \bot\, ) \rarrow\ \negation \varphi.$
				\item[(DI1)] $\varphi \rarrow \varphi \vee \psi.$
				\item[(DI2)] $\psi \rarrow \varphi \vee \psi.$
				\item[(DE)] $(\, \varphi \rarrow \chi\, ) \rarrow 
					(\, (\, \psi \rarrow \chi\, ) 
					\rarrow (\, \varphi \vee \psi \rarrow \chi\, )\, ).$
				\item[(CI)] $\varphi \rarrow (\, \psi \rarrow (\, \varphi \wedge \psi\, )\, ).$
				\item[(CE1)] $\varphi \wedge \psi \rarrow \varphi.$
				\item[(CE2)] $\varphi \wedge \psi \rarrow \psi.$
				\item[(DNE)] $\negation \negation \varphi \rarrow \varphi$.
			\end{description}
		\end{logicalaxm}
	\end{screen}
	
	\begin{screen}
		\begin{logicalaxm}[{\bf HK}の公理(量化)]
			$\varphi$と$\psi$と$\xi$を式とし,$x$と$y$を変項とし,$t$を項とする.また
			$y$は$\psi$には自由に現れず,$\varphi$には$x$が自由に現れ,
			$y$と$t$は$\varphi$の中で$x$への代入について自由であるとする.このとき
			次は{\bf HK}の公理である.
			\begin{description}
				\item[(UI)] $\forall y\, (\, \psi \rarrow \varphi(x/y)\, ) 
					\rarrow (\, \psi \rarrow \forall x \varphi\, ).$
				
				\item[(UE)] $\forall x \varphi \rarrow \varphi(x/t).$
				
				\item[(EI)] $\varphi(x/t) \rarrow \exists x \varphi.$
				
				\item[(EE)] $\forall y\, (\, \varphi(x/y) \rarrow \psi\, )
						\rarrow (\, \exists x \varphi \rarrow \psi\, ).$
			\end{description}
		\end{logicalaxm}
	\end{screen}
	
	古典論理で証明可能なことを$\provable{\mbox{{\bf HK}}}$と書く.
	
	\begin{screen}
		\begin{metadfn}[{\bf HK}における証明可能性]
			式$\varphi$が公理系$\mathscr{S}$から
			{\bf 証明された}だとか{\bf 証明可能である}\index{しょうめいかのう@証明可能}
			{\bf (provable)}ということは,
			\begin{itemize}
				\item $\varphi$は$\mathscr{S}$の公理である.
				\item $\varphi$は{\bf HK}の公理である.
				\item 式$\psi$で,$\psi$と$\psi \rightarrow \varphi$が$\mathscr{S}$から
				証明されているものが取れる({\bf 三段論法}\index{さんだんろんぽう@三段論法}
				{\bf (Modus Pones)}).
				\item 式$\psi$と変項$a$が取れて,$\psi$には$x$が自由に現れていて,
				$a$は$\varphi$の中で$x$への代入について自由であり,
				また$\mathscr{S}$のどの公理の中にも$a$は自由に現れないとする.
				そして$\mathscr{S}$から$\psi(x/a)$が証明されていて,
				$\varphi$とは$\forall x \psi$なる形の式である
				({\bf 汎化}\index{はんか@汎化}{\bf (generalization)}).
			\end{itemize}
			のいずれかが満たされているということである.
		\end{metadfn}
	\end{screen}
	
	\begin{screen}
		\begin{thm}[対偶律$1$]\label{classic:contraposition_1}
			$\varphi$と$\psi$を$\lang{\varepsilon}$の式とするとき
			\begin{align}
				\provable{\mbox{{\bf HK}}} (\, \varphi \rarrow \psi\, )
				\rarrow (\, \negation \psi \rarrow\ \negation \varphi\, ).
			\end{align}
		\end{thm}
	\end{screen}
	
	\begin{sketch}
		$\varphi$と$\varphi \rarrow \psi$の三段論法から
		\begin{align}
			\varphi,\ \negation \psi,\ \varphi \rarrow \psi
			\provable{\mbox{{\bf HK}}} \psi
		\end{align}
		が成り立ち,
		\begin{align}
			\varphi,\ \negation \psi,\ \varphi \rarrow \psi
			\provable{\mbox{{\bf HK}}}\ \negation \psi
		\end{align}
		も成り立つので,矛盾の規則(DTC1)より
		\begin{align}
			\varphi,\ \negation \psi,\ \varphi \rarrow \psi
			\provable{\mbox{{\bf HK}}} \bot
		\end{align}
		が従う.演繹定理より
		\begin{align}
			\negation \psi,\ \varphi \rarrow \psi
			\provable{\mbox{{\bf HK}}} \varphi \rarrow \bot
		\end{align}
		となり,否定の導入(NI)より
		\begin{align}
			\negation \psi,\ \varphi \rarrow \psi
			\provable{\mbox{{\bf HK}}}\ \negation \varphi
		\end{align}
		が従う.そして演繹定理より
		\begin{align}
			\varphi \rarrow \psi
			\provable{\mbox{{\bf HK}}}\ \negation \psi \rarrow\ \negation \varphi
		\end{align}
		が得られる.
		\QED
	\end{sketch}
	
	\begin{screen}
		\begin{thm}[弱 De Morgan の法則$1$]
		\label{classic:weak_De_Morgan_law_1}
			$\varphi$を$\lang{\varepsilon}$の式とし,
			変項$x$が$\varphi$に自由に現れるとするとき,
			\begin{align}
				\provable{\mbox{{\bf HK}}}
				\ \negation \exists x \varphi \rarrow \forall x \negation \varphi.
			\end{align}
		\end{thm}
	\end{screen}
	
	\begin{sketch}
		$y$を$\varphi$には現れない変項とすると,存在記号の導入規則より
		\begin{align}
			\provable{\mbox{{\bf HK}}} \varphi(x/y) \rarrow \exists x \varphi
		\end{align}
		が成り立ち,対偶律$1$ (定理\ref{classic:contraposition_1})より
		\begin{align}
			\provable{\mbox{{\bf HK}}}\ 
			\negation \exists x \varphi \rarrow\ \negation \varphi(x/y) 
		\end{align}
		となる.汎化により
		\begin{align}
			\provable{\mbox{{\bf HK}}} \forall y\, (\, \negation \exists x \varphi \rarrow\ \negation \varphi(x/y) \, ) 
		\end{align}
		が成り立つので,量化の公理(UI)との三段論法より
		\begin{align}
			\provable{\mbox{{\bf HK}}}\ 
			\negation \exists x \varphi \rarrow \forall x \negation \varphi 
		\end{align}
		が得られる.
		\QED
	\end{sketch}
	
	\begin{screen}
		\begin{thm}[強 De Morgan の法則$1$]
		\label{classic:strong_De_Morgan_law_1}
			$\varphi$を$\lang{\varepsilon}$の式とし,
			変項$x$が$\varphi$に自由に現れるとするとき,
			\begin{align}
				\provable{\mbox{{\bf HK}}}
				\exists x \negation \varphi \rarrow\ \negation \forall x \varphi.
			\end{align}
		\end{thm}
	\end{screen}
	
	\begin{sketch}
		$y$を$\varphi$に現れない変項とすれば,量化の公理(UE)より
		\begin{align}
			\provable{\mbox{{\bf HK}}} \forall x \varphi \rarrow \varphi(x/y)
		\end{align}
		が成り立ち,対偶律1 (定理\ref{classic:contraposition_1})より
		\begin{align}
			\provable{\mbox{{\bf HK}}}\ \negation \varphi(x/y) \rarrow\ \negation \forall x \varphi
		\end{align}
		となる.汎化によって
		\begin{align}
			\provable{\mbox{{\bf HK}}} \forall y\, (\, \negation \varphi(x/y) \rarrow\ \negation \forall x \varphi\, )
		\end{align}
		が成り立ち,量化の公理(EE)より
		\begin{align}
			\provable{\mbox{{\bf HK}}} \exists x \negation \varphi \rarrow\ \negation \forall x \varphi
		\end{align}
		が得られる.
		\QED
	\end{sketch}
	
	\begin{screen}
		\begin{thm}[二重否定の導入]
		\label{classic:introduction_of_double_negation}
			$\varphi$を$\lang{\varepsilon}$の式とするとき
			\begin{align}
				\provable{\mbox{{\bf HK}}} \varphi \rarrow\ \negation \negation \varphi.
			\end{align}
		\end{thm}
	\end{screen}
	
	\begin{sketch}
		矛盾の導入(CTD1)より
		\begin{align}
			\varphi \provable{\mbox{{\bf HK}}}\ \negation \varphi \rarrow \bot
		\end{align}
		が成り立ち,否定の導入(NI)より
		\begin{align}
			\varphi \provable{\mbox{{\bf HK}}}\ \negation \negation \varphi
		\end{align}
		が従う.
		\QED
	\end{sketch}
	
	\begin{screen}
		\begin{thm}[対偶律$2$]\label{classic:contraposition_2}
			$\varphi$と$\psi$を$\lang{\varepsilon}$の式とするとき
			\begin{align}
				\provable{\mbox{{\bf HK}}} (\, \varphi \rarrow\ \negation \psi\, )
				\rarrow (\, \psi \rarrow\ \negation \varphi\, ).
			\end{align}
		\end{thm}
	\end{screen}
	
	\begin{sketch}
		対偶律$1$ (定理\ref{classic:contraposition_1})より
		\begin{align}
			\varphi \rarrow\ \negation \psi \provable{\mbox{{\bf HK}}}\ 
			\negation \negation \psi \rarrow\ \negation \varphi
		\end{align}
		が成り立ち,他方で二重否定の導入(定理\ref{classic:introduction_of_double_negation})より
		\begin{align}
			\psi \provable{\mbox{{\bf HK}}}\ \negation \negation \psi
		\end{align}
		が成り立つので,三段論法より
		\begin{align}
			\psi,\ \varphi \rarrow\ \negation \psi \provable{\mbox{{\bf HK}}}\ 
			\negation \varphi
		\end{align}
		が従い,演繹定理より
		\begin{align}
			\varphi \rarrow\ \negation \psi \provable{\mbox{{\bf HK}}}
			\psi \rarrow \negation \varphi
		\end{align}
		が得られる.
		\QED
	\end{sketch}
	
	\begin{screen}
		\begin{thm}[弱 De Morgan の法則$2$]
		\label{classic:weak_De_Morgan_law_2}
			$\varphi$を$\lang{\varepsilon}$の式とし,
			変項$x$が$\varphi$に自由に現れるとするとき,
			\begin{align}
				\provable{\mbox{{\bf HK}}} \forall x \negation \varphi
				\rarrow\ \negation \exists x \varphi.
			\end{align}
		\end{thm}
	\end{screen}
	
	\begin{sketch}
		$y$を$\varphi$に現れない変項とすれば,量化の公理(UE)より
		\begin{align}
			\provable{\mbox{{\bf HK}}} \forall x \negation \varphi \rarrow\ \negation \varphi(x/y)
		\end{align}
		となるので,対偶律$2$ (定理\ref{classic:contraposition_2})より
		\begin{align}
			\provable{\mbox{{\bf HK}}} \varphi(x/y) \rarrow\ \negation \forall x \negation \varphi
		\end{align}
		となる.汎化によって
		\begin{align}
			\provable{\mbox{{\bf HK}}}\ \forall y\, (\, \varphi(x/y) \rarrow\ \negation \forall x \negation \varphi\, )
		\end{align}
		が成り立ち,量化の公理(EE)によって
		\begin{align}
			\provable{\mbox{{\bf HK}}}\ \exists x \negation \varphi \rarrow\ \negation \forall x \negation \varphi
		\end{align}
		が従い,再び対偶律$2$ (定理\ref{classic:contraposition_2})より
		\begin{align}
			\provable{\mbox{{\bf HK}}} \forall x \negation \varphi \rarrow\ \negation \exists x \varphi
		\end{align}
		が得られる.
		\QED
	\end{sketch}
	
	\begin{screen}
		\begin{thm}[強 De Morgan の法則$2$]
		\label{classic:strong_De_Morgan_law_2}
			$\varphi$を$\lang{\varepsilon}$の式とし,
			変項$x$が$\varphi$に自由に現れるとするとき,
			\begin{align}
				\provable{\mbox{{\bf HK}}}\ \negation \forall x \varphi \rarrow 
				\exists x \negation \varphi.
			\end{align}
		\end{thm}
	\end{screen}
	
	\begin{sketch}
		$y$を$\varphi$に現れない変項とすれば,量化の公理(EI)より
		\begin{align}
			\provable{\mbox{{\bf HK}}}\ \negation \varphi(x/y) \rarrow \exists x \negation \varphi
		\end{align}
		となり,対偶律3 (定理\ref{classic:contraposition_3})より
		\begin{align}
			\provable{\mbox{{\bf HK}}}\ 
			\negation \exists x \varphi \rarrow \varphi(x/y)
		\end{align}
		が成り立つ.汎化によって
		\begin{align}
			\provable{\mbox{{\bf HK}}}
			\forall y\, (\, \negation \exists x \varphi \rarrow \varphi(x/y)\, )
		\end{align}
		となり,量化の公理(UI)より
		\begin{align}
			\provable{\mbox{{\bf HK}}}\ 
			\negation \exists x \varphi \rarrow \forall x \varphi
		\end{align}
		が従い,再び対偶律3 (定理\ref{classic:contraposition_3})より
		\begin{align}
			\provable{\mbox{{\bf HK}}}\ \negation \forall x \varphi \rarrow \exists x \negation \varphi
		\end{align}
		が得られる.
		\QED
	\end{sketch}
	
	\begin{screen}
		\begin{thm}
			$\varphi$と$\psi$に変項$x$が自由に現れるとき,
			\begin{align}
				\provable{\mbox{{\bf HK}}} \forall x\, (\, \varphi \rarrow \psi\, )
				\rarrow (\, \exists x \varphi \rarrow \exists x \psi\, ).
			\end{align}
		\end{thm}
	\end{screen}
	
	\begin{sketch}
		全称記号の除去より
		\begin{align}
			\forall x\, (\, \varphi \rarrow \psi\, ) \provable{\mbox{{\bf HK}}}
			\varphi \rarrow \psi
		\end{align}
		となるので,
		\begin{align}
			\varphi,\ \forall x\, (\, \varphi \rarrow \psi\, ) 
			\provable{\mbox{{\bf HK}}} \psi
		\end{align}
		が成り立ち,存在記号の導入より
		\begin{align}
			\varphi,\ \forall x\, (\, \varphi \rarrow \psi\, ) 
			\provable{\mbox{{\bf HK}}} \exists x \psi
		\end{align}
		が成り立ち,演繹法則より
		\begin{align}
			\forall x\, (\, \varphi \rarrow \psi\, ) 
			\provable{\mbox{{\bf HK}}} \varphi \rarrow \exists x \psi
		\end{align}
		が従う.汎化によって
		\begin{align}
			\forall x\, (\, \varphi \rarrow \psi\, ) \provable{\mbox{{\bf HK}}} 
			\forall x\, (\, \varphi \rarrow \exists x \psi\, )
		\end{align}
		となり,存在記号の除去より
		\begin{align}
			\forall x\, (\, \varphi \rarrow \psi\, ) \provable{\mbox{{\bf HK}}} 
			\exists x \varphi \rarrow \exists x \psi
		\end{align}
		が従う.
	\end{sketch}
	
	\begin{screen}
		\begin{thm}
			$\varphi$と$\psi$を$\lang{\varepsilon}$の式とし,
			$\psi$には$x$が自由に現れて,$\varphi$には$x$が自由に現れないとき,
			\begin{align}
				\provable{\mbox{{\bf HK}}} (\, \varphi \rarrow \exists x \psi\, ) 
				\rarrow \exists x\, (\, \varphi \rarrow \psi\, ).
			\end{align}
		\end{thm}
	\end{screen}
	
	\begin{sketch}
		\begin{align}
			(\, \varphi \rarrow \exists x \psi\, ) &\rarrow 
				(\, \negation \varphi \vee \exists x \psi\, ), \\
			(\, \negation \varphi \vee \exists x \psi\, ) &\rarrow
				\ \negation (\, \varphi \wedge \negation \exists x \psi\, ), \\
			\negation (\, \varphi \wedge \negation \exists x \psi\, ) &\rarrow 
				\ \negation (\, \varphi \wedge \forall x \negation \psi\, ), \\
			\negation (\, \varphi \wedge \forall x \negation \psi\, ) &\rarrow 
				\ \negation \forall x\, (\, \varphi \wedge \negation \psi\, ), \\
			\negation \forall x\, (\, \varphi \wedge \negation \psi\, ) &\rarrow 
				\exists x \negation (\, \varphi \wedge \negation \psi\, ), \\
			\exists x \negation (\, \varphi \wedge \negation \psi\, ) &\rarrow 
				\exists x\, (\, \negation \varphi \vee \psi\, ), \\
			\exists x\, (\, \negation \varphi \vee \psi\, ) &\rarrow 
				\exists x\, (\, \varphi \rarrow \psi\, )
		\end{align}
	\end{sketch}
	
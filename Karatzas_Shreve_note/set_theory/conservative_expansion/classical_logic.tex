\section{古典論理}
	\begin{screen}
		\begin{logicalaxm}[{\bf HK}の公理(命題論理)]
			$\varphi$と$\psi$と$\xi$を式とするとき,次は{\bf HK}の公理である.
			\begin{description}
				\item[(S)] $(\, \varphi \rarrow (\, \psi \rarrow \chi\, )\, ) 
					\rarrow (\, (\, \varphi \rarrow \psi\, )
					\rarrow (\, \varphi \rarrow \chi\, )\, ).$
				\item[(K)] $\varphi \rarrow (\, \psi \rarrow \varphi\, ).$
				\item[(CTD1)] $\varphi \rarrow (\, \negation \varphi \rarrow \bot\, ).$
				\item[(CTD2)] $\negation \varphi \rarrow (\, \varphi \rarrow \bot\, ).$
				\item[(NI)] $(\, \varphi \rarrow \bot\, ) \rarrow\ \negation \varphi.$
				\item[(DI1)] $\varphi \rarrow \varphi \vee \psi.$
				\item[(DI2)] $\psi \rarrow \varphi \vee \psi.$
				\item[(DE)] $(\, \varphi \rarrow \chi\, ) \rarrow 
					(\, (\, \psi \rarrow \chi\, ) 
					\rarrow (\, \varphi \vee \psi \rarrow \chi\, )\, ).$
				\item[(CI)] $\varphi \rarrow (\, \psi \rarrow (\, \varphi \wedge \psi\, )\, ).$
				\item[(CE1)] $\varphi \wedge \psi \rarrow \varphi.$
				\item[(CE2)] $\varphi \wedge \psi \rarrow \psi.$
				\item[(DNE)] $\negation \negation \varphi \rarrow \varphi$.
			\end{description}
		\end{logicalaxm}
	\end{screen}
	
	\begin{screen}
		\begin{logicalaxm}[{\bf HK}の公理(量化)]
			$\varphi$と$\psi$と$\xi$を式とし,$x$と$y$を変項とし,$t$を項とする.また
			$y$は$\psi$には自由に現れず,$\varphi$には$x$が自由に現れ,
			$y$と$t$は$\varphi$の中で$x$への代入について自由であるとする.このとき
			次は{\bf HK}の公理である.
			\begin{description}
				\item[(UI)] $\forall y\, (\, \psi \rarrow \varphi(x/y)\, ) 
					\rarrow (\, \psi \rarrow \forall x \varphi\, ).$
				
				\item[(UE)] $\forall x \varphi \rarrow \varphi(x/t).$
				
				\item[(EI)] $\varphi(x/t) \rarrow \exists x \varphi.$
				
				\item[(EE)] $\forall y\, (\, \varphi(x/y) \rarrow \psi\, )
						\rarrow (\, \exists x \varphi \rarrow \psi\, ).$
			\end{description}
		\end{logicalaxm}
	\end{screen}
	
	古典論理で証明可能なことを$\provable{\mbox{{\bf HK}}}$と書く.
	
	\begin{screen}
		\begin{metadfn}[{\bf HK}における証明可能性]
			式$\varphi$が公理系$\mathscr{S}$から
			{\bf 証明された}だとか{\bf 証明可能である}\index{しょうめいかのう@証明可能}
			{\bf (provable)}ということは,
			\begin{itemize}
				\item $\varphi$は$\mathscr{S}$の公理である.
				\item $\varphi$は{\bf HK}の公理である.
				\item 式$\psi$で,$\psi$と$\psi \rightarrow \varphi$が$\mathscr{S}$から
				証明されているものが取れる({\bf 三段論法}\index{さんだんろんぽう@三段論法}
				{\bf (Modus Pones)}).
				\item 式$\psi$と変項$a$が取れて,$\psi$には$x$が自由に現れていて,
				$a$は$\varphi$の中で$x$への代入について自由であり,
				また$\mathscr{S}$のどの公理の中にも$a$は自由に現れないとする.
				そして$\mathscr{S}$から$\psi(x/a)$が証明されていて,
				$\varphi$とは$\forall x \psi$なる形の式である
				({\bf 汎化}\index{はんか@汎化}{\bf (generalization)}).
			\end{itemize}
			のいずれかが満たされているということである.
		\end{metadfn}
	\end{screen}
	
\subsection{最小論理}
	\begin{screen}
		\begin{thm}[対偶律$1$]\label{classic:contraposition_1}
			$\varphi$と$\psi$を$\lang{\varepsilon}$の式とするとき
			\begin{align}
				\provable{\mbox{{\bf HK}}} (\, \varphi \rarrow \psi\, )
				\rarrow (\, \negation \psi \rarrow\ \negation \varphi\, ).
			\end{align}
		\end{thm}
	\end{screen}
	
	\begin{sketch}
		$\varphi$と$\varphi \rarrow \psi$の三段論法から
		\begin{align}
			\varphi,\ \negation \psi,\ \varphi \rarrow \psi
			\provable{\mbox{{\bf HK}}} \psi
		\end{align}
		が成り立ち,
		\begin{align}
			\varphi,\ \negation \psi,\ \varphi \rarrow \psi
			\provable{\mbox{{\bf HK}}}\ \negation \psi
		\end{align}
		も成り立つので,矛盾の規則(DTC1)より
		\begin{align}
			\varphi,\ \negation \psi,\ \varphi \rarrow \psi
			\provable{\mbox{{\bf HK}}} \bot
		\end{align}
		が従う.演繹定理より
		\begin{align}
			\negation \psi,\ \varphi \rarrow \psi
			\provable{\mbox{{\bf HK}}} \varphi \rarrow \bot
		\end{align}
		となり,否定の導入(NI)より
		\begin{align}
			\negation \psi,\ \varphi \rarrow \psi
			\provable{\mbox{{\bf HK}}}\ \negation \varphi
		\end{align}
		が従う.そして演繹定理より
		\begin{align}
			\varphi \rarrow \psi
			\provable{\mbox{{\bf HK}}}\ \negation \psi \rarrow\ \negation \varphi
		\end{align}
		が得られる.
		\QED
	\end{sketch}
	
	\begin{screen}
		\begin{thm}[弱 De Morgan の法則$1$]
		\label{classic:weak_De_Morgan_law_for_quantifier_1}
			$\varphi$を$\lang{\varepsilon}$の式とし,
			変項$x$が$\varphi$に自由に現れるとするとき,
			\begin{align}
				\provable{\mbox{{\bf HK}}}
				\ \negation \exists x \varphi \rarrow \forall x \negation \varphi.
			\end{align}
		\end{thm}
	\end{screen}
	
	\begin{sketch}
		$y$を$\varphi$には現れない変項とすると,存在記号の導入規則より
		\begin{align}
			\provable{\mbox{{\bf HK}}} \varphi(x/y) \rarrow \exists x \varphi
		\end{align}
		が成り立ち,対偶律$1$ (定理\ref{classic:contraposition_1})より
		\begin{align}
			\provable{\mbox{{\bf HK}}}\ 
			\negation \exists x \varphi \rarrow\ \negation \varphi(x/y) 
		\end{align}
		となる.汎化により
		\begin{align}
			\provable{\mbox{{\bf HK}}} \forall y\, (\, \negation \exists x \varphi \rarrow\ \negation \varphi(x/y) \, ) 
		\end{align}
		が成り立つので,量化の公理(UI)との三段論法より
		\begin{align}
			\provable{\mbox{{\bf HK}}}\ 
			\negation \exists x \varphi \rarrow \forall x \negation \varphi 
		\end{align}
		が得られる.
		\QED
	\end{sketch}
	
	\begin{screen}
		\begin{thm}[強 De Morgan の法則$1$]
		\label{classic:strong_De_Morgan_law_for_quantifier_1}
			$\varphi$を$\lang{\varepsilon}$の式とし,
			変項$x$が$\varphi$に自由に現れるとするとき,
			\begin{align}
				\provable{\mbox{{\bf HK}}}
				\exists x \negation \varphi \rarrow\ \negation \forall x \varphi.
			\end{align}
		\end{thm}
	\end{screen}
	
	\begin{sketch}
		$y$を$\varphi$に現れない変項とすれば,量化の公理(UE)より
		\begin{align}
			\provable{\mbox{{\bf HK}}} \forall x \varphi \rarrow \varphi(x/y)
		\end{align}
		が成り立ち,対偶律1 (定理\ref{classic:contraposition_1})より
		\begin{align}
			\provable{\mbox{{\bf HK}}}\ \negation \varphi(x/y) \rarrow\ \negation \forall x \varphi
		\end{align}
		となる.汎化によって
		\begin{align}
			\provable{\mbox{{\bf HK}}} \forall y\, (\, \negation \varphi(x/y) \rarrow\ \negation \forall x \varphi\, )
		\end{align}
		が成り立ち,量化の公理(EE)より
		\begin{align}
			\provable{\mbox{{\bf HK}}} \exists x \negation \varphi \rarrow\ \negation \forall x \varphi
		\end{align}
		が得られる.
		\QED
	\end{sketch}
	
	\begin{screen}
		\begin{thm}[二重否定の導入]
		\label{classic:introduction_of_double_negation}
			$\varphi$を$\lang{\varepsilon}$の式とするとき
			\begin{align}
				\provable{\mbox{{\bf HK}}} \varphi \rarrow\ \negation \negation \varphi.
			\end{align}
		\end{thm}
	\end{screen}
	
	\begin{sketch}
		矛盾の導入(CTD1)より
		\begin{align}
			\varphi \provable{\mbox{{\bf HK}}}\ \negation \varphi \rarrow \bot
		\end{align}
		が成り立ち,否定の導入(NI)より
		\begin{align}
			\varphi \provable{\mbox{{\bf HK}}}\ \negation \negation \varphi
		\end{align}
		が従う.
		\QED
	\end{sketch}
	
	\begin{screen}
		\begin{thm}[対偶律$2$]\label{classic:contraposition_2}
			$\varphi$と$\psi$を$\lang{\varepsilon}$の式とするとき
			\begin{align}
				\provable{\mbox{{\bf HK}}} (\, \varphi \rarrow\ \negation \psi\, )
				\rarrow (\, \psi \rarrow\ \negation \varphi\, ).
			\end{align}
		\end{thm}
	\end{screen}
	
	\begin{sketch}
		対偶律$1$ (定理\ref{classic:contraposition_1})より
		\begin{align}
			\varphi \rarrow\ \negation \psi \provable{\mbox{{\bf HK}}}\ 
			\negation \negation \psi \rarrow\ \negation \varphi
		\end{align}
		が成り立ち,他方で二重否定の導入(定理\ref{classic:introduction_of_double_negation})より
		\begin{align}
			\psi \provable{\mbox{{\bf HK}}}\ \negation \negation \psi
		\end{align}
		が成り立つので,三段論法より
		\begin{align}
			\psi,\ \varphi \rarrow\ \negation \psi \provable{\mbox{{\bf HK}}}\ 
			\negation \varphi
		\end{align}
		が従い,演繹定理より
		\begin{align}
			\varphi \rarrow\ \negation \psi \provable{\mbox{{\bf HK}}}
			\psi \rarrow\ \negation \varphi
		\end{align}
		が得られる.
		\QED
	\end{sketch}
	
	\begin{screen}
		\begin{thm}[弱 De Morgan の法則$2$]
		\label{classic:weak_De_Morgan_law_for_quantifier_2}
			$\varphi$を$\lang{\varepsilon}$の式とし,
			変項$x$が$\varphi$に自由に現れるとするとき,
			\begin{align}
				\provable{\mbox{{\bf HK}}} \forall x \negation \varphi
				\rarrow\ \negation \exists x \varphi.
			\end{align}
		\end{thm}
	\end{screen}
	
	\begin{sketch}
		$y$を$\varphi$に現れない変項とすれば,量化の公理(UE)より
		\begin{align}
			\provable{\mbox{{\bf HK}}} \forall x \negation \varphi \rarrow\ \negation \varphi(x/y)
		\end{align}
		となるので,対偶律$2$ (定理\ref{classic:contraposition_2})より
		\begin{align}
			\provable{\mbox{{\bf HK}}} \varphi(x/y) \rarrow\ \negation \forall x \negation \varphi
		\end{align}
		となる.汎化によって
		\begin{align}
			\provable{\mbox{{\bf HK}}}\ \forall y\, (\, \varphi(x/y) \rarrow\ \negation \forall x \negation \varphi\, )
		\end{align}
		が成り立ち,量化の公理(EE)によって
		\begin{align}
			\provable{\mbox{{\bf HK}}}\ \exists x \negation \varphi \rarrow\ \negation \forall x \negation \varphi
		\end{align}
		が従い,再び対偶律$2$ (定理\ref{classic:contraposition_2})より
		\begin{align}
			\provable{\mbox{{\bf HK}}} \forall x \negation \varphi \rarrow\ \negation \exists x \varphi
		\end{align}
		が得られる.
		\QED
	\end{sketch}
	
	\begin{screen}
		\begin{thm}[De Morgan の法則1]
		\label{classic:De_Morgan_law_1}
			$\varphi$と$\psi$を$\lang{\varepsilon}$の式とするとき
			\begin{align}
				\provable{\mbox{{\bf HK}}} (\, \negation \varphi \vee \psi\, ) 
				\rarrow\ \negation (\, \varphi \wedge \negation \psi\, ).
			\end{align}
		\end{thm}
	\end{screen}
	
	\begin{sketch}
		論理積の除去(CE1)(CE2)より
		\begin{align}
			\varphi \wedge \negation \psi &\provable{\mbox{{\bf HK}}}\ \negation \varphi, \\
			\varphi \wedge \negation \psi &\provable{\mbox{{\bf HK}}} \psi
		\end{align}
		が成り立つので,矛盾の導入(CTD1)(CTD2)より
		\begin{align}
			\varphi \wedge \negation \psi &\provable{\mbox{{\bf HK}}}
			\varphi \rarrow \bot, \\
			\varphi \wedge \negation \psi &\provable{\mbox{{\bf HK}}}\ 
			\negation \psi \rarrow \bot
		\end{align}
		となり,論理和の除去(DE)より
		\begin{align}
			\varphi \wedge \negation \psi &\provable{\mbox{{\bf HK}}}
			\varphi \vee \negation \psi \rarrow \bot
		\end{align}
		が従い,否定の導入(NI)より
		\begin{align}
			\varphi \wedge \negation \psi &\provable{\mbox{{\bf HK}}}\ 
			\negation (\, \varphi \vee \negation \psi\, )
		\end{align}
		が得られる.
		\QED
	\end{sketch}
	
	\begin{screen}
		\begin{thm}[論理和の対称律]
		\label{classic:symmetry_of_disjunction}
			$\varphi$と$\psi$を$\lang{\varepsilon}$の式とするとき,
			\begin{align}
				\provable{\mbox{{\bf HK}}} 
				\varphi \vee \psi \rarrow \psi \vee \varphi.
			\end{align}
		\end{thm}
	\end{screen}
	
	\begin{sketch}
		論理和の導入(DI1)(DI2)より
		\begin{align}
			&\provable{\mbox{{\bf HK}}} \varphi \rarrow \psi \vee \varphi, \\
			&\provable{\mbox{{\bf HK}}} \psi \rarrow \psi \vee \varphi
		\end{align}
		が成り立つので,論理和の除去(DE)より
		\begin{align}
			\provable{\mbox{{\bf HK}}} \varphi \vee \psi \rarrow \psi \vee \varphi
		\end{align}
		が従う.
		\QED
	\end{sketch}
	
	\begin{screen}
		\begin{thm}[含意の論理和への遺伝性]
		\label{classic:heredity_of_implication_to_disjunction}
			$\varphi$と$\psi$と$\chi$を$\lang{\varepsilon}$の式とするとき,
			\begin{align}
				\provable{\mbox{{\bf HK}}} (\, \varphi \rarrow \psi\, )
				\rarrow (\, \varphi \wedge \chi \rarrow \psi \wedge \chi\, ).
			\end{align}
		\end{thm}
	\end{screen}
	
	\begin{sketch}
		三段論法より
		\begin{align}
			\varphi,\ \varphi \rarrow \psi \provable{\mbox{{\bf HK}}} \psi
		\end{align}
		が成り立ち,論理和の導入(DI1)より
		\begin{align}
			\varphi,\ \varphi \rarrow \psi \provable{\mbox{{\bf HK}}} \psi \vee \chi
		\end{align}
		が従い,演繹定理より
		\begin{align}
			\varphi \rarrow \psi \provable{\mbox{{\bf HK}}} 
			\varphi \rarrow \psi \vee \chi
		\end{align}
		が得られる.他方で論理和の導入(DI2)より
		\begin{align}
			\varphi \rarrow \psi \provable{\mbox{{\bf HK}}} 
			\chi \rarrow \psi \vee \chi
		\end{align}
		も成り立つので,論理和の除去(DE)より
		\begin{align}
			\varphi \rarrow \psi \provable{\mbox{{\bf HK}}} 
			\varphi \vee \chi \rarrow \psi \vee \chi
		\end{align}
		が得られる.
		\QED
	\end{sketch}
	
	\begin{screen}
		\begin{thm}[含意の論理積への遺伝性]
		\label{classic:heredity_of_implication_to_conjunction}
			$\varphi$と$\psi$と$\chi$を$\lang{\varepsilon}$の式とするとき,
			\begin{align}
				\provable{\mbox{{\bf HK}}} (\, \psi \rarrow \chi\, )
				\rarrow (\, \varphi \wedge \psi \rarrow \varphi \wedge \chi\, ).
			\end{align}
		\end{thm}
	\end{screen}
	
	\begin{sketch}
		論理積の除去(CE1)(CE2)及び三段論法より
		\begin{align}
			\psi \rarrow \chi,\ \varphi \wedge \psi &\provable{\mbox{{\bf HK}}} \varphi, \\
			\psi \rarrow \chi,\ \varphi \wedge \psi &\provable{\mbox{{\bf HK}}} \psi,
		\end{align}
		そして
		\begin{align}
			\psi \rarrow \chi,\ \varphi \wedge \psi \provable{\mbox{{\bf HK}}} \chi
		\end{align}
		が成り立つので,論理積の導入(CI)より
		\begin{align}
			\psi \rarrow \chi,\ \varphi \wedge \psi \provable{\mbox{{\bf HK}}} \varphi \wedge \chi
		\end{align}
		が従う.
		\QED
	\end{sketch}
	
	\begin{screen}
		\begin{thm}[論理積と全称の交換]
		\label{classic:commutation_of_conjunction_and_universal_quantifier}
			$\varphi$と$\psi$を$\lang{\varepsilon}$の式とし,
			$\psi$には変項$x$が自由に現れるとするとき,
			\begin{align}
				\provable{\mbox{{\bf HK}}} \forall x\, (\, \varphi \wedge \psi\, )
				\rarrow \varphi \wedge \forall x \psi.
			\end{align}
		\end{thm}
	\end{screen}
	
	\begin{sketch}
		量化の公理(UE)より
		\begin{align}
			\forall x\, (\, \varphi \wedge \psi\, ) \provable{\mbox{{\bf HK}}} 
			\varphi \wedge \psi
		\end{align}
		が成り立ち,論理積の除去(CE1)(CE2)より
		\begin{align}
			\forall x\, (\, \varphi \wedge \psi\, ) &\provable{\mbox{{\bf HK}}} \varphi, \\
			\forall x\, (\, \varphi \wedge \psi\, ) &\provable{\mbox{{\bf HK}}} \psi
		\end{align}
		となる.汎化によって
		\begin{align}
			\forall x\, (\, \varphi \wedge \psi\, ) \provable{\mbox{{\bf HK}}} 
			\forall x \psi
		\end{align}
		が成り立ち,論理積の導入(CI)によって
		\begin{align}
			\forall x\, (\, \varphi \wedge \psi\, ) \provable{\mbox{{\bf HK}}} 
			\varphi \wedge \forall x \psi
		\end{align}
		が得られる.
		\QED
	\end{sketch}
	
	\begin{screen}
		\begin{thm}
		\label{classic:no_description_1}
			$\varphi$と$\psi$に変項$x$が自由に現れるとき,
			\begin{align}
				\provable{\mbox{{\bf HK}}} \forall x\, (\, \varphi \rarrow \psi\, )
				\rarrow (\, \exists x \varphi \rarrow \exists x \psi\, ).
			\end{align}
		\end{thm}
	\end{screen}
	
	\begin{sketch}
		量化の公理(UE)より
		\begin{align}
			\forall x\, (\, \varphi \rarrow \psi\, ) \provable{\mbox{{\bf HK}}}
			\varphi \rarrow \psi
		\end{align}
		となるので,演繹定理より
		\begin{align}
			\varphi,\ \forall x\, (\, \varphi \rarrow \psi\, ) 
			\provable{\mbox{{\bf HK}}} \psi
		\end{align}
		が成り立つ.量化の公理(EI)より
		\begin{align}
			\varphi,\ \forall x\, (\, \varphi \rarrow \psi\, ) 
			\provable{\mbox{{\bf HK}}} \exists x \psi
		\end{align}
		が成り立ち,演繹定理より
		\begin{align}
			\forall x\, (\, \varphi \rarrow \psi\, ) 
			\provable{\mbox{{\bf HK}}} \varphi \rarrow \exists x \psi
		\end{align}
		が従う.汎化によって
		\begin{align}
			\forall x\, (\, \varphi \rarrow \psi\, ) \provable{\mbox{{\bf HK}}} 
			\forall x\, (\, \varphi \rarrow \exists x \psi\, )
		\end{align}
		となり,量化の公理(EE)より
		\begin{align}
			\forall x\, (\, \varphi \rarrow \psi\, ) \provable{\mbox{{\bf HK}}} 
			\exists x \varphi \rarrow \exists x \psi
		\end{align}
		が従う.
		\QED
	\end{sketch}

\subsection{二重否定の除去}
%以下(DNE)を使う.
	\begin{screen}
		\begin{thm}[対偶律3]\label{classic:contraposition_3}
			$\varphi$と$\psi$を$\lang{\varepsilon}$の式とするとき
			\begin{align}
				\provable{\mbox{{\bf HK}}} (\, \negation \varphi \rarrow \psi\, )
				\rarrow (\, \negation \psi \rarrow \varphi\, ).
			\end{align}
		\end{thm}
	\end{screen}
	
	\begin{sketch}
		対偶律1 (定理\ref{classic:contraposition_1})より
		\begin{align}
			\negation \varphi \rarrow \psi \provable{\mbox{{\bf HK}}}\ 
			\negation \psi \rarrow\ \negation \negation \varphi
		\end{align}
		が成り立つので,演繹定理より
		\begin{align}
			\negation \psi,\ \negation \varphi \rarrow \psi 
			\provable{\mbox{{\bf HK}}}\ \negation \negation \varphi
		\end{align}
		となり,二重否定の除去(DNE)より
		\begin{align}
			\negation \psi,\ \negation \varphi \rarrow \psi 
			\provable{\mbox{{\bf HK}}} \varphi
		\end{align}
		が従う.そして演繹定理より
		\begin{align}
			\negation \varphi \rarrow \psi 
			\provable{\mbox{{\bf HK}}}\ \negation \psi \rarrow \varphi
		\end{align}
		が得られる.
		\QED
	\end{sketch}
	
	\begin{screen}
		\begin{thm}[強 De Morgan の法則$2$]
		\label{classic:strong_De_Morgan_law_for_quantifier_2}
			$\varphi$を$\lang{\varepsilon}$の式とし,
			変項$x$が$\varphi$に自由に現れるとするとき,
			\begin{align}
				\provable{\mbox{{\bf HK}}}\ \negation \forall x \varphi \rarrow 
				\exists x \negation \varphi.
			\end{align}
		\end{thm}
	\end{screen}
	
	\begin{sketch}
		$y$を$\varphi$に現れない変項とすれば,量化の公理(EI)より
		\begin{align}
			\provable{\mbox{{\bf HK}}}\ \negation \varphi(x/y) \rarrow \exists x \negation \varphi
		\end{align}
		となり,対偶律3 (定理\ref{classic:contraposition_3})より
		\begin{align}
			\provable{\mbox{{\bf HK}}}\ 
			\negation \exists x \varphi \rarrow \varphi(x/y)
		\end{align}
		が成り立つ.汎化によって
		\begin{align}
			\provable{\mbox{{\bf HK}}}
			\forall y\, (\, \negation \exists x \varphi \rarrow \varphi(x/y)\, )
		\end{align}
		となり,量化の公理(UI)より
		\begin{align}
			\provable{\mbox{{\bf HK}}}\ 
			\negation \exists x \varphi \rarrow \forall x \varphi
		\end{align}
		が従い,再び対偶律3 (定理\ref{classic:contraposition_3})より
		\begin{align}
			\provable{\mbox{{\bf HK}}}\ \negation \forall x \varphi \rarrow \exists x \negation \varphi
		\end{align}
		が得られる.
		\QED
	\end{sketch}
	
	\begin{screen}
		\begin{thm}[対偶律4]\label{classic:contraposition_4}
			$\varphi$と$\psi$を$\lang{\varepsilon}$の式とするとき
			\begin{align}
				\provable{\mbox{{\bf HK}}} (\, \negation \varphi \rarrow\ \negation \psi\, )
				\rarrow (\, \psi \rarrow \varphi\, ).
			\end{align}
		\end{thm}
	\end{screen}
	
	\begin{sketch}
		対偶律3 (定理\ref{classic:contraposition_3})と演繹定理より
		\begin{align}
			\negation \varphi \rarrow\ \negation \psi \provable{\mbox{{\bf HK}}}\ 
			\negation \negation \psi \rarrow \varphi
		\end{align}
		が成り立つ.二重否定の導入(定理\ref{classic:introduction_of_double_negation})より
		\begin{align}
			\psi \provable{\mbox{{\bf HK}}}\ \negation \negation \psi
		\end{align}
		が成り立つので,三段論法より
		\begin{align}
			\psi,\ \negation \varphi \rarrow\ \negation \psi \provable{\mbox{{\bf HK}}} \varphi
		\end{align}
		が従い,演繹定理より
		\begin{align}
			\negation \varphi \rarrow\ \negation \psi \provable{\mbox{{\bf HK}}} 
			\psi \rarrow \varphi
		\end{align}
		が得られる.
		\QED
	\end{sketch}
	
	\begin{screen}
		\begin{thm}[背理法の原理]
		\label{classic:proof_by_contradiction}
			$\varphi$を$\lang{\varepsilon}$の式とするとき
			\begin{align}
				\provable{\mbox{{\bf HK}}} (\, \negation \varphi \rarrow \bot\, )
				\rarrow \varphi.
			\end{align}
		\end{thm}
	\end{screen}
	
	\begin{sketch}
		否定の導入(NI)と演繹定理より
		\begin{align}
			\negation \varphi \rarrow \bot \provable{\mbox{{\bf HK}}}\ 
			\negation \negation \varphi
		\end{align}
		が成り立ち,二重否定の除去(DNE)より
		\begin{align}
			\negation \varphi \rarrow \bot \provable{\mbox{{\bf HK}}} \varphi
		\end{align}
		が従う.
		\QED
	\end{sketch}
	
	\begin{screen}
		\begin{thm}[爆発律]
		\label{classic:principle_of_explosion}
			$\varphi$を$\lang{\varepsilon}$の式とするとき
			\begin{align}
				\provable{\mbox{{\bf HK}}} \bot \rarrow \varphi.
			\end{align}
		\end{thm}
	\end{screen}
	
	\begin{sketch}
		含意の導入(K)と演繹定理より
		\begin{align}
			\bot \provable{\mbox{{\bf HK}}}\ \negation \varphi \rarrow \bot
		\end{align}
		が成り立ち,背理法の原理(定理\ref{classic:proof_by_contradiction})より
		\begin{align}
			\bot \provable{\mbox{{\bf HK}}} \varphi
		\end{align}
		が従う.
		\QED
	\end{sketch}
	
	\begin{screen}
		\begin{thm}[否定の論理和は含意で表せる]
		\label{classic:disjunction_of_negation_rewritable_by_implication}
			$\varphi$と$\psi$を$\lang{\varepsilon}$の式とするとき
			\begin{align}
				\provable{\mbox{{\bf HK}}} (\, \negation \varphi \vee \psi\, )
				\rarrow (\, \varphi \rarrow \psi\, ).
			\end{align}
		\end{thm}
	\end{screen}
	
	\begin{sketch}
		矛盾の導入(CTD1)と演繹定理より
		\begin{align}
			\negation \varphi,\ \varphi \provable{\mbox{{\bf HK}}} \bot 
		\end{align}
		となり,爆発律(定理\ref{classic:principle_of_explosion})より
		\begin{align}
			\negation \varphi,\ \varphi \provable{\mbox{{\bf HK}}} \psi
		\end{align}
		となり,演繹定理より
		\begin{align}
			\provable{\mbox{{\bf HK}}}\ \negation \varphi \rarrow 
			(\, \varphi \rarrow \psi\, )
		\end{align}
		が得られる.他方で含意の導入(K)より
		\begin{align}
			\provable{\mbox{{\bf HK}}} \psi \rarrow (\, \varphi \rarrow \psi\, )
		\end{align}
		も成り立つので,論理和の除去(DE)より
		\begin{align}
			\provable{\mbox{{\bf HK}}} (\, \negation \varphi \vee \psi\, ) 
			\rarrow (\, \varphi \rarrow \psi\, )
		\end{align}
		が従う.
		\QED
	\end{sketch}
	
	\begin{screen}
		\begin{thm}[驚嘆すべき帰結]\label{classic:consequentia_mirabilis}
			$\varphi$を$\lang{\varepsilon}$の式とするとき
			\begin{align}
				\provable{\mbox{{\bf HK}}} 
				(\, \negation \varphi \rarrow \varphi\, ) \rarrow \varphi.
			\end{align}
		\end{thm}
	\end{screen}
	
	\begin{sketch}
		三段論法より
		\begin{align}
			\negation \varphi,\ \negation \varphi \rarrow \varphi
			\provable{\mbox{{\bf HK}}} \varphi
		\end{align}
		が成り立ち,矛盾の導入(CTD1)より
		\begin{align}
			\negation \varphi,\ \negation \varphi \rarrow \varphi
			\provable{\mbox{{\bf HK}}}\ \negation \varphi \rarrow \bot
		\end{align}
		が成り立ち,三段論法より
		\begin{align}
			\negation \varphi,\ \negation \varphi \rarrow \varphi
			\provable{\mbox{{\bf HK}}} \bot
		\end{align}
		が従う.演繹定理より
		\begin{align}
			\negation \varphi \rarrow \varphi
			\provable{\mbox{{\bf HK}}}\ \negation \varphi \rarrow \bot
		\end{align}
		となり,背理法の原理(定理\ref{classic:proof_by_contradiction})より
		\begin{align}
			\negation \varphi \rarrow \varphi
			\provable{\mbox{{\bf HK}}} \varphi
		\end{align}
		が従う.
		\QED
	\end{sketch}
	
	\begin{screen}
		\begin{thm}[排中律]\label{classic:law_of_excluded_middle}
			$\varphi$を$\lang{\varepsilon}$の式とするとき
			\begin{align}
				\provable{\mbox{{\bf HK}}} \varphi \vee \negation \varphi.
			\end{align}
		\end{thm}
	\end{screen}
	
	\begin{sketch}
		論理和の導入(DI1)と演繹定理より
		\begin{align}
			\varphi \provable{\mbox{{\bf HK}}} \varphi \vee \negation \varphi
		\end{align}
		が成り立ち,矛盾の導入(CTD1)より
		\begin{align}
			\varphi \provable{\mbox{{\bf HK}}}\ 
			\negation (\, \varphi \vee \negation \varphi\, ) \rarrow \bot
		\end{align}
		が成り立ち,演繹定理より
		\begin{align}
			\varphi,\ \negation (\, \varphi \vee \negation \varphi\, )
			\provable{\mbox{{\bf HK}}} \bot
		\end{align}
		となり,再び演繹定理より
		\begin{align}
			\negation (\, \varphi \vee \negation \varphi\, )
			\provable{\mbox{{\bf HK}}} \varphi \rarrow \bot
		\end{align}
		となり,否定の導入(NI)より
		\begin{align}
			\negation (\, \varphi \vee \negation \varphi\, )
			\provable{\mbox{{\bf HK}}}\ \negation \varphi
		\end{align}
		が成り立つ.論理和の導入(DI2)より
		\begin{align}
			\negation (\, \varphi \vee \negation \varphi\, )
			\provable{\mbox{{\bf HK}}} \varphi \vee \negation \varphi
		\end{align}
		が従い,演繹定理より
		\begin{align}
			\provable{\mbox{{\bf HK}}}\ 
			\negation (\, \varphi \vee \negation \varphi\, )
			\rarrow \varphi \vee \negation \varphi
		\end{align}
		が成り立ち,驚嘆すべき帰結(定理\ref{classic:consequentia_mirabilis})との三段論法より
		\begin{align}
			\provable{\mbox{{\bf HK}}} \varphi \vee \negation \varphi
		\end{align}
		が得られる.
		\QED
	\end{sketch}
	
	\begin{screen}
		\begin{thm}[含意は否定と論理和で表せる]
		\label{classic:implication_rewritable_by_disjunction_of_negation}
			$\varphi$と$\psi$を$\lang{\varepsilon}$の式とするとき
			\begin{align}
				\provable{\mbox{{\bf HK}}} (\, \varphi \rarrow \psi\, )
				\rarrow (\, \negation \varphi \vee \psi\, ).
			\end{align}
		\end{thm}
	\end{screen}
	
	\begin{sketch}
		含意の論理和への遺伝性(定理\ref{classic:heredity_of_implication_to_disjunction})
		と演繹定理より
		\begin{align}
			\varphi \rarrow \psi \provable{\mbox{{\bf HK}}}
			\varphi \vee \negation \varphi \rarrow \psi \vee \negation \varphi
		\end{align}
		が成り立ち,排中律(定理\ref{classic:law_of_excluded_middle})との三段論法より
		\begin{align}
			\varphi \rarrow \psi \provable{\mbox{{\bf HK}}}
			\psi \vee \negation \varphi
		\end{align}
		が従い,論理和の対称律(定理\ref{classic:symmetry_of_disjunction})より
		\begin{align}
			\varphi \rarrow \psi \provable{\mbox{{\bf HK}}}\ 
			\negation \varphi \vee \psi
		\end{align}
		が得られる.
		\QED
	\end{sketch}
	
	\begin{screen}
		\begin{thm}[De Morganの法則2]
		\label{classic:De_Morgan_law_2}
			$\varphi$と$\psi$を$\lang{\varepsilon}$の式とするとき
			\begin{align}
				\provable{\mbox{{\bf HK}}}\ 
				\negation (\, \varphi \wedge \negation \psi\, )
				\rarrow (\, \negation \varphi \vee \psi\, ).
			\end{align}
		\end{thm}
	\end{screen}
	
	\begin{sketch}
		論理積の導入(CI2)より
		\begin{align}
			\varphi \provable{\mbox{{\bf HK}}}\ \negation \psi
			\rarrow \varphi \wedge \negation \psi
		\end{align}
		が成り立つので,対偶律3 (定理\ref{classic:contraposition_3})より
		\begin{align}
			\varphi \provable{\mbox{{\bf HK}}}\ 
			\negation (\, \varphi \wedge \negation \psi\, ) \rarrow \psi
		\end{align}
		が成り立つ.演繹定理より
		\begin{align}
			\varphi,\ \negation (\, \varphi \wedge \negation \psi\, ) 
			&\provable{\mbox{{\bf HK}}} \psi, \\
			\negation (\, \varphi \wedge \negation \psi\, ) 
			&\provable{\mbox{{\bf HK}}} \varphi \rarrow \psi
		\end{align}
		が従い,右辺を否定と論理和に書き換えれば
		\begin{align}
			\negation (\, \varphi \wedge \negation \psi\, ) 
			\provable{\mbox{{\bf HK}}}\ \negation \varphi \vee \psi
		\end{align}
		が得られる(定理\ref{classic:implication_rewritable_by_disjunction_of_negation}).
		\QED
	\end{sketch}
	
	\begin{screen}
		\begin{thm}
		\label{classic:no_description_2}
			$\varphi$と$\psi$を$\lang{\varepsilon}$の式とし,
			$\psi$には$x$が自由に現れて,$\varphi$に$x$が自由に現れないとするとき,
			\begin{align}
				\provable{\mbox{{\bf HK}}} (\, \varphi \rarrow \exists x \psi\, )
				\rarrow \exists x\, (\, \varphi \rarrow \psi\, ).
			\end{align}
		\end{thm}
	\end{screen}
	
	\begin{sketch}
		含意は否定と論理和で表せる
		(定理\ref{classic:implication_rewritable_by_disjunction_of_negation})ので
		\begin{align}
			\varphi \rarrow \exists x \psi \provable{\mbox{{\bf HK}}}\ 
			\negation \varphi \vee \exists x \psi
		\end{align}
		となり,De Morgan の法則(定理\ref{classic:De_Morgan_law_1})より
		\begin{align}
			\varphi \rarrow \exists x \psi \provable{\mbox{{\bf HK}}}\ 
			\negation (\, \varphi \wedge \negation \exists x \psi\, )
			\label{fom:classic_no_description_2_1}
		\end{align}
		となる.ところで量化記号についてのDe Morganの法則
		(定理\ref{classic:weak_De_Morgan_law_for_quantifier_2})より
		\begin{align}
			\provable{\mbox{{\bf HK}}} \forall x \negation \psi
			\rarrow\ \negation \exists x \psi
		\end{align}
		が成り立つので,含意の論理積への遺伝性
		(定理\ref{classic:heredity_of_implication_to_conjunction})より
		\begin{align}
			\provable{\mbox{{\bf HK}}} 
			\varphi \wedge \forall x \negation \psi
			\rarrow \varphi \wedge \negation \exists x \psi
		\end{align}
		が得られ,対偶律4 (定理\ref{classic:contraposition_4})より
		\begin{align}
			\provable{\mbox{{\bf HK}}}\ 
			\negation (\, \varphi \wedge \negation \exists x \psi\, )
			\rarrow\ \negation (\, \varphi \wedge \forall x \negation \psi\, )
		\end{align}
		となり,(\refeq{fom:classic_no_description_2_1})との三段論法より
		\begin{align}
			\varphi \rarrow \exists x \psi \provable{\mbox{{\bf HK}}}\ 
			\negation (\, \varphi \wedge \forall x \negation \psi\, )
			\label{fom:classic_no_description_2_2}
		\end{align}
		が従う.また論理積と全称の交換
		(定理\ref{classic:commutation_of_conjunction_and_universal_quantifier})
		と対偶律4 (定理\ref{classic:contraposition_4})より
		\begin{align}
			\provable{\mbox{{\bf HK}}}\ 
			\negation (\, \varphi \wedge \forall x \negation \psi\, ) 
			\rarrow\ \negation \forall x\, (\, \varphi \wedge \negation \psi\, )
		\end{align}
		が成り立つので,(\refeq{fom:classic_no_description_2_2})との三段論法より
		\begin{align}
			\varphi \rarrow \exists x \psi \provable{\mbox{{\bf HK}}}\ 
			\negation \forall x\, (\, \varphi \wedge \negation \psi\, )
		\end{align}
		が従い,量化記号についてのDe Morganの法則
		(定理\ref{classic:strong_De_Morgan_law_for_quantifier_2})より
		\begin{align}
			\varphi \rarrow \exists x \psi \provable{\mbox{{\bf HK}}}\ 
			\exists x \negation (\, \varphi \wedge \negation \psi\, )
			\label{fom:classic_no_description_2_3}
		\end{align}
		が従う.ここでDe Morganの法則(定理\ref{classic:De_Morgan_law_2})と汎化により
		\begin{align}
			\provable{\mbox{{\bf HK}}} \forall x\, 
			(\, \negation (\, \varphi \wedge \negation \psi\, )
			\rarrow (\, \negation \varphi \vee \psi\, )\, )
		\end{align}
		が成り立つので,定理\ref{classic:no_description_1}より
		\begin{align}
			\provable{\mbox{{\bf HK}}} \exists x \negation (\, \varphi \wedge \negation \psi\, )
			\rarrow \exists x\, (\, \negation \varphi \vee \psi\, )
		\end{align}
		が得られ,(\refeq{fom:classic_no_description_2_3})との三段論法より
		\begin{align}
			\varphi \rarrow \exists x \psi \provable{\mbox{{\bf HK}}}
			\exists x\, (\, \negation \varphi \vee \psi\, )
			\label{fom:classic_no_description_2_4}
		\end{align}
		が従う.同様に
		定理\ref{classic:disjunction_of_negation_rewritable_by_implication}と汎化により
		\begin{align}
			\provable{\mbox{{\bf HK}}} \forall x\, 
			(\, (\, \negation \varphi \vee \psi\, )
			\rarrow (\, \varphi \rarrow \psi\, )\, )
		\end{align}
		が成り立つので,定理\ref{classic:no_description_1}より
		\begin{align}
			\provable{\mbox{{\bf HK}}} \exists x\, (\, \negation \varphi \vee \psi\, )
			\rarrow \exists x\, (\, \varphi \rarrow \psi\, )
		\end{align}
		が得られ,(\refeq{fom:classic_no_description_2_4})との三段論法より
		\begin{align}
			\varphi \rarrow \exists x \psi \provable{\mbox{{\bf HK}}}
			\exists x\, (\, \varphi \rarrow \psi\, )
		\end{align}
		が従う.
		\QED
	\end{sketch}
	
	\begin{screen}
		\begin{thm}
			$\varphi$と$\lang{\varepsilon}$の式とし,$x$と$y$を変項とし,$\varphi$には$x$が
			自由に現れて,$y$は$\varphi$の中で$x$への代入について自由であるとするとき,
			\begin{align}
				\provable{\mbox{{\bf HK}}} \exists x \varphi \rarrow \exists y \varphi(x/y).
			\end{align}
		\end{thm}
	\end{screen}
	
	\begin{sketch}
		量化の公理(EI)より
		\begin{align}
			\provable{\mbox{{\bf HK}}} \varphi \rarrow \exists y \varphi(x/y)
		\end{align}
		が成り立ち,汎化によって
		\begin{align}
			\provable{\mbox{{\bf HK}}} \forall x\, (\, \varphi \rarrow \exists y \varphi(x/y)\, )
		\end{align}
		となり,量化の公理(EE)によって
		\begin{align}
			\provable{\mbox{{\bf HK}}} \exists x \varphi \rarrow \exists y \varphi(x/y).
		\end{align}
		が得られる.
		\QED
	\end{sketch}
	
	\begin{screen}
		\begin{thm}
			$\varphi$と$\lang{\varepsilon}$の式とし,$x$と$y$を変項とし,
			$\varphi$には$x$が自由に現れて,$y$は$\varphi$の中で$x$への代入について自由であるとき,
			\begin{align}
				\provable{\mbox{{\bf HK}}} \exists y\, (\, \exists x \varphi \rarrow \varphi(x/y)\, ).
			\end{align}
		\end{thm}
	\end{screen}
	
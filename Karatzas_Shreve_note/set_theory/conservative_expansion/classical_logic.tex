	この章では,$\lang{\in}$の任意の文$\varphi$に対して
	\begin{align}
		\Sigma \vdash \varphi
	\end{align}
	と
	\begin{align}
		\mathscr{S} \provable{\mbox{{\bf HK}}} \varphi
	\end{align}
	が同値であるということを示す.ここで$\mathscr{S}$とは
	
	始めの二節では扱う式は全て$\lang{\varepsilon}$の式であるとする.
	第\ref{sec:restriction_of_formulas}節で決めた通り,
	扱う式は全て,そこに現れる$\varepsilon$項は全て主要$\varepsilon$項であり,
	現れる内包項は全て正則内包項である.
	
\section{古典論理}
	{\bf HK}とは{\bf 古典論理}\index{こてんろんり@古典論理}{\bf (classical logic)}と呼ばれる
	(Hilbert流)証明体系である.これの公理と推論規則は以下のように与えられる.
	
	\begin{screen}
		\begin{logicalaxm}[{\bf HK}の公理(命題論理)]
			$\varphi$と$\psi$と$\xi$を式とするとき,次は{\bf HK}の公理である.
			\begin{description}
				\item[(S)] $(\, \varphi \rarrow (\, \psi \rarrow \chi\, )\, ) 
					\rarrow (\, (\, \varphi \rarrow \psi\, )
					\rarrow (\, \varphi \rarrow \chi\, )\, ).$
				\item[(K)] $\varphi \rarrow (\, \psi \rarrow \varphi\, ).$
				\item[(CTD1)] $\varphi \rarrow (\, \negation \varphi \rarrow \bot\, ).$
				\item[(CTD2)] $\negation \varphi \rarrow (\, \varphi \rarrow \bot\, ).$
				\item[(NI)] $(\, \varphi \rarrow \bot\, ) \rarrow\ \negation \varphi.$
				\item[(DI1)] $\varphi \rarrow \varphi \vee \psi.$
				\item[(DI2)] $\psi \rarrow \varphi \vee \psi.$
				\item[(DE)] $(\, \varphi \rarrow \chi\, ) \rarrow 
					(\, (\, \psi \rarrow \chi\, ) 
					\rarrow (\, \varphi \vee \psi \rarrow \chi\, )\, ).$
				\item[(CI)] $\varphi \rarrow (\, \psi \rarrow (\, \varphi \wedge \psi\, )\, ).$
				\item[(CE1)] $\varphi \wedge \psi \rarrow \varphi.$
				\item[(CE2)] $\varphi \wedge \psi \rarrow \psi.$
				\item[(DNE)] $\negation \negation \varphi \rarrow \varphi$.
			\end{description}
		\end{logicalaxm}
	\end{screen}
	
	\begin{screen}
		\begin{logicalaxm}[{\bf HK}の公理(量化)]
			$\varphi$と$\psi$と$\xi$を式とし,$x$と$y$を変項とし,$t$を項とする.また
			$y$は$\psi,\forall x \varphi, \exists x \varphi$には自由に現れず,
			$\varphi$には$x$が自由に現れ,$y$と$t$は$\varphi$の中で$x$への代入について
			自由であるとする.このとき次は{\bf HK}の公理である.
			\begin{description}
				\item[(UI)] $\forall y\, (\, \psi \rarrow \varphi(x/y)\, ) 
					\rarrow (\, \psi \rarrow \forall x \varphi\, ).$
				
				\item[(UE)] $\forall x \varphi \rarrow \varphi(x/t).$
				
				\item[(EI)] $\varphi(x/t) \rarrow \exists x \varphi.$
				
				\item[(EE)] $\forall y\, (\, \varphi(x/y) \rarrow \psi\, )
						\rarrow (\, \exists x \varphi \rarrow \psi\, ).$
			\end{description}
		\end{logicalaxm}
	\end{screen}
	
	第\ref{chap:inference}章では明記しなかったが,
	\begin{align}
		\mathscr{S} \vdash \varphi
	\end{align}
	ということは$\mathcal{L}$の文の列$\varphi_{1},\varphi_{2},\cdots,\varphi_{n}$で
	\footnote{
		ここで添え字に数字が使われているが,これらは集合論の中で定義される数字ではなく
		生活の中にありふれた数字である.足し算や引き算,大小関係も日常的な感覚で使えるものとする.
	}
	\begin{description}
		\item[(1)] 各$\varphi_{i}$に対して
			\begin{itemize}
				\item $\varphi_{i}$は推論公理である.
				\item $\varphi_{i}$は$\mathscr{S}$の公理である.
				\item $\varphi_{i}$は,これより前の文$\varphi_{j}$と$\varphi_{k}$の
					三段論法で得られる.つまり$\varphi_{k}$は
					$\varphi_{j} \rarrow \varphi_{i}$なる式である.
			\end{itemize}
		
		\item[(2)] $\varphi_{n}$は$\varphi$である.
	\end{description}
	を満たすものが取れるということである.そしてこの列$\varphi_{1},\varphi_{2},\cdots,\varphi_{n}$
	を$\mathscr{S}$から$\varphi$への{\bf 証明}\index{しょうめい@証明}{\bf (proof)}と呼ぶ.
	第\ref{chap:inference}章で規定した証明とは若干違ってしまっているが,
	異なる証明体系を比較するには証明は列であると考える方が都合が良い.
	ただし,集合論を実演した章で見られる通り実際の証明でこのような列を構成することは殆どなく,
	現実的には第\ref{chap:inference}章で規定した証明の方が合っているであろう.
	
	第\ref{chap:inference}章では証明に使われる式は全て文であるとしたが,
	\underline{{\bf HK}の証明では文に限らず一般の式も使用する}.
	これは$\lang{\in}$の式による証明を得ることを予定しているためである.
	
	\begin{screen}
		\begin{metadfn}[{\bf HK}における証明]
			$\mathscr{S}$を$\lang{\varepsilon}$の式からなる公理系とする.
			このとき$\lang{\varepsilon}$の式の列$\varphi_{1},\varphi_{2},\cdots,
			\varphi_{n}$が$\mathscr{S}$から$\varphi_{n}$への{\bf HK}の証明であるとは,
			各$\varphi_{i}$に対して
			\begin{itemize}
				\item $\varphi_{i}$は{\bf HK}の公理である.
				\item $\varphi_{i}$は$\mathscr{S}$の公理である.
				\item $\varphi_{i}$は,これより前の式$\varphi_{j}$と$\varphi_{k}$の
					三段論法で得られる.
				\item $\varphi_{i}$は,これより前の式$\varphi_{j}$から
					{\bf 汎化}\index{はんか@汎化}{\bf (generalization)}で得られる.
					これは,変項$a,x$と$x$が自由に現れる式$\psi$が取れて,
					$\varphi_{j}$は$\psi(x/a)$,$\varphi_{i}$は$\forall x \psi$
					であるということである.ただし$a$は$\forall x \psi$にも
					$\mathscr{S}$のどの公理にも自由に現れず,
					また$\psi$の中で$x$への代入について自由である.
					$a$をこの汎化の{\bf 固有変項}\index{こゆうへんこう@固有変項}
					{\bf (eigenvariable)}と呼ぶ.
			\end{itemize}
			が満たされているということである.
		\end{metadfn}
	\end{screen}
	
	$\lang{\varepsilon}$の式からなる公理系$\mathscr{S}$から
	$\lang{\varepsilon}$の式$\varphi$を終点にした{\bf HK}の証明が取れることを
	\begin{align}
		\mathscr{S} \provable{\mbox{{\bf HK}}} \varphi
	\end{align}
	と書く.
	
\subsection{演繹定理}
	\begin{screen}
		\begin{metathm}[{\bf HK}の演繹定理]
		\label{metathm:deduction_theorem_of_HK}
			$\mathscr{S}$を$\lang{\varepsilon}$の式からなる公理系とし,
			$\varphi$と$\psi$を$\lang{\varepsilon}$の式とするとき
			\begin{description}
				\item[(1)] $\mathscr{S} \provable{\mbox{{\bf HK}}} \varphi \rarrow \psi$ならば$\varphi,\ \mathscr{S} \provable{\mbox{{\bf HK}}} \psi$.
				\item[(2)] $\varphi,\ \mathscr{S} \provable{\mbox{{\bf HK}}} \psi$ならば$\mathscr{S} \provable{\mbox{{\bf HK}}} \varphi \rarrow \psi$.
			\end{description}
		\end{metathm}
	\end{screen}
	
	\begin{metaprf}\mbox{}
		\begin{description}
			\item[(1)] $\mathscr{S} \provable{\mbox{{\bf HK}}} \varphi \rarrow \psi$
				であるとき,$\mathscr{S}$から$\varphi \rarrow \psi$への{\bf HK}の証明を
				$\varphi_{1},\varphi_{2}\cdots,\varphi_{n}$とすれば
				\begin{align}
					\varphi_{1},\ \varphi_{2},\ \cdots,\ \varphi_{n},\ \varphi,\ \psi
				\end{align}
				は$\varphi,\ \mathscr{S}$から$\psi$への{\bf HK}の証明である.
				
			\item[(2)] $\varphi,\ \mathscr{S} \provable{\mbox{{\bf HK}}} \psi$
				であるとき,$\varphi,\ \mathscr{S}$から$\psi$への{\bf HK}の証明を
				$\varphi_{1},\varphi_{2}\cdots,\varphi_{n}$とし,
				以下の要領で$\varphi_{1}$から順番に,式を削除して別の式で置き換えたり
				式を新しく追加したりしていく.
				\begin{description}
					\item[case1] $\varphi_{i}$が{\bf HK}の公理または$\mathscr{S}$の公理
						であるとき,$\varphi_{i}$と$\varphi_{i+1}$との間に
						\begin{align}
							&\varphi_{i} \rarrow (\, \varphi \rarrow \varphi_{i}\, ), \\
							&\varphi \rarrow \varphi_{i}
						\end{align}
						を追加する.上の式は公理(K)の形の式であり,$\varphi_{i}$との三段論法で
						$\varphi \rarrow \varphi_{i}$が出るという図になる.
						
					\item[case2] $\varphi_{i}$が$\varphi$であるとき
						$\varphi_{i}$を証明列から削除し,その位置は
						\begin{align}
							&(\, \varphi \rarrow (\, (\, \varphi \rarrow \varphi\, )
								\rarrow \varphi\, )\, )
								\rarrow (\, (\, \varphi \rarrow 
								(\, \varphi \rarrow \varphi\, )\, )
								\rarrow (\, \varphi \rarrow \varphi\, )\, ), \\
							&\varphi \rarrow (\, (\, \varphi \rarrow \varphi\, )
								\rarrow \varphi\, ), \\
							&(\, \varphi \rarrow 
								(\, \varphi \rarrow \varphi\, )\, )
								\rarrow (\, \varphi \rarrow \varphi\, ), \\
							&\varphi \rarrow 
								(\, \varphi \rarrow \varphi\, ), \\
							&\varphi \rarrow \varphi
						\end{align}
						で置き換える.上の式は公理(S)(K)の形の式の三段論法で
						$\varphi \rarrow \varphi$が出るという図になる.
						
					\item[case3] $\varphi_{i}$が前の式$\varphi_{j}$と$\varphi_{k}$の
						三段論法で得られているとする.ここで$\varphi_{k}$は
						$\varphi_{j} \rarrow \varphi_{i}$なる形の式とする.このとき
						$\varphi_{i}$を証明列から削除し,その位置は
						\begin{align}
							&(\, \varphi \rarrow (\, \varphi_{j} \rarrow \varphi_{i}\, )\, )
								\rarrow (\, (\, \varphi \rarrow \varphi_{j}\, )
								\rarrow (\, \varphi \rarrow \varphi_{i}\, )\, ), \\
							&(\, \varphi \rarrow \varphi_{j}\, )
								\rarrow (\, \varphi \rarrow \varphi_{i}\, ), \\
							&\varphi \rarrow \varphi_{i}
						\end{align}
						で置き換える.上の式は,$\varphi \rarrow \varphi_{j}$と
						$\varphi \rarrow \varphi_{k}$に至る式の列が得られていれば
						公理(S)(K)の形の式との三段論法で$\varphi \rarrow \varphi_{i}$
						が出るという図になる.
						
					\item[case4] $\varphi_{i}$が前の式$\varphi_{j}$から汎化で
						得られているとする.つまり変項$a,x$と$x$が自由に現れる式$\psi$が取れて,
						$\varphi_{j}$は$\psi(x/a)$,$\varphi_{i}$は$\forall x \psi$
						である.ただし$a$は$\forall x \psi$にも$\varphi$にも
						$\mathscr{S}$のどの公理にも自由に現れず,
						また$\psi$の中で$x$への代入について自由である.このとき
						$\varphi_{i}$を証明列から削除し,その位置は
						\begin{align}
							&\forall a\, (\, \varphi \rarrow \psi(x/a)\, ), \\
							&\forall a\, (\, \varphi \rarrow \psi(x/a)\, )
							\rarrow (\, \varphi \rarrow \forall x \psi\, ), \\
							&\varphi \rarrow \forall x \psi
						\end{align}
						で置き換える.上の式は,$\varphi \rarrow \varphi_{j}$に至る式の列
						が得られていれば汎化および公理(UI)の形の式との三段論法で
						$\varphi \rarrow \varphi_{i}$が出るという図になる.
				\end{description}
				以上の追加と置換の操作を$\varphi_{1}$から順に$\varphi_{n}$まで施していけば,
				最終的に得る式の列は$\varphi \rarrow \psi$への{\bf HK}の証明になっている.
				\QED
		\end{description}
	\end{metaprf}
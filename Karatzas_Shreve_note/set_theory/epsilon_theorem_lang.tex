\section{言語}
	\begin{description}
	\item[{\bf EC}]
	{\bf EC}(Elementary calculus)の言語を$L(EC)$と書く.$L(EC)$の構成要素は
	\begin{description}
		\item[矛盾記号] $\bot$
		\item[論理記号] $\rightharpoondown$, $\vee$, $\wedge$, $\rightarrow$
		\item[述語記号] $=$, $\in$
		\item[変項] $x_{0},x_{1},x_{2},\cdots$
	\end{description}
	とする.変項は$L(EC)$の項であって,また$L(EC)$の項は変項だけである.
	$L(EC)$の式は
	\begin{itemize}
		\item 項$s$と式$t$に対して$\in st$と$= st$は式である.
		\item 式$\varphi$と式$\psi$に対して$\rightharpoondown \varphi,
			\vee \varphi \psi,\ \wedge \varphi \psi, \rightarrow \varphi \psi$
			は式である.
		\item 以上のみが$L(EC)$の式である.
	\end{itemize}
	
	\item[{\bf PC}]
	{\bf PC}(Predicate calculus)の言語を$L(PC)$と書く.$L(PC)$の構成要素は
	\begin{description}
		\item[矛盾記号] $\bot$
		\item[論理記号] $\rightharpoondown$, $\vee$, $\wedge$, $\Longrightarrow$
		\item[量化子] $\forall$, $\exists$
		\item[述語記号] $=$, $\in$
		\item[変項] $x_{0},x_{1},x_{2},\cdots$
	\end{description}
	とする.変項は$L(PC)$の項であって,また$L(PC)$の項は変項だけである.
	$L(PC)$の式は
	\begin{itemize}
		\item 項$s$と式$t$に対して$\in st$と$= st$は式である.
		\item 式$\varphi$と式$\psi$に対して$\rightharpoondown \varphi,
			\vee \varphi \psi,\ \wedge \varphi \psi, \rightarrow \varphi \psi$
			は式である.
		\item 式$\varphi$と変項$x$に対して,$\forall x \varphi$と$\exists x \varphi$は式である.
		\item 以上のみが$L(PC)$の式である.
	\end{itemize}
	\end{description}
	
	$L(EC)$と$L(PC)$に$\varepsilon$項を追加した言語をそれぞれ$L(EC_{\varepsilon}),
	L(PC_{\varepsilon})$とする.
	
	\begin{description}
	\item[{\bf EC${}_{\varepsilon}$}]
	言語$L(EC_{\varepsilon})$の構成要素は
	\begin{description}
		\item[矛盾記号] $\bot$
		\item[論理記号] $\rightharpoondown$, $\vee$, $\wedge$, $\rightarrow$
		\item[述語記号] $=$, $\in$
		\item[変項] $x_{0},x_{1},x_{2},\cdots$
		\item[$\varepsilon$記号] $\varepsilon$
	\end{description}
	とする.$L(EC_{\varepsilon})$の項と式は
	\begin{itemize}
		\item 変項は項である.
		\item 項$s$と式$t$に対して$\in st$と$= st$は式である.
		\item 式$\varphi$と式$\psi$に対して$\rightharpoondown \varphi,
			\vee \varphi \psi,\ \wedge \varphi \psi, \rightarrow \varphi \psi$
			は式である.
		\item 式$\varphi$と変項$x$に対して,$\epsilon x \varphi$は項である.
		\item 以上のみが$L(EC_{\varepsilon})$の項と式である.
	\end{itemize}
	
	\item[{\bf PC${}_{\varepsilon}$}]
	言語$L(PC_{\varepsilon})$の構成要素は
	\begin{description}
		\item[矛盾記号] $\bot$
		\item[論理記号] $\rightharpoondown$, $\vee$, $\wedge$, $\Longrightarrow$
		\item[量化子] $\forall$, $\exists$
		\item[述語記号] $=$, $\in$
		\item[変項] $x_{0},x_{1},x_{2},\cdots$
		\item[$\varepsilon$記号] $\varepsilon$
	\end{description}
	とする.$L(PC_{\varepsilon})$の項と式は
	\begin{itemize}
		\item 変項は項である.
		\item 項$s$と式$t$に対して$\in st$と$= st$は式である.
		\item 式$\varphi$と式$\psi$に対して$\rightharpoondown \varphi,
			\vee \varphi \psi,\ \wedge \varphi \psi, \rightarrow \varphi \psi$
			は式である.
		\item 式$\varphi$と変項$x$に対して,$\forall x \varphi$と$\exists x \varphi$は式である.
		\item 式$\varphi$と変項$x$に対して,$\epsilon x \varphi$は項である.
		\item 以上のみが$L(PC_{\varepsilon})$の項と式である.
	\end{itemize}
	\end{description}
	
\section{証明}
	\begin{description}
	\item[{\bf EC}]\mbox{}
	
	\begin{itembox}[l]{{\bf EC}の公理}
		$\varphi$と$\psi$と$\xi$を$L(EC)$の式とするとき,次は{\bf EC}の公理である.
		\begin{description}
			\item[(S)] $(\varphi \rightarrow (\psi \rightarrow \chi)) 
				\rightarrow ((\varphi \rightarrow \psi)
				\rightarrow (\varphi \rightarrow \chi)).$
			\item[(K)] $\varphi \rightarrow (\psi \rightarrow \varphi).$
			\item[(DI1)] $\varphi \rightarrow (\varphi \vee \psi).$
			\item[(DI2)] $\psi \rightarrow (\varphi \vee \psi).$
			\item[(DE)] $(\varphi \rightarrow \chi) \rightarrow 
				((\psi \rightarrow \chi) \rightarrow ((\varphi \vee \psi) \rightarrow \chi)).$
			\item[(CI)] $\varphi \rightarrow (\psi \rightarrow (\varphi \wedge \psi)).$
			\item[(CE1)] $(\varphi \wedge \psi) \rightarrow \varphi.$
			\item[(CE2)] $(\varphi \wedge \psi) \rightarrow \psi.$
				
			\item[(CTI)] $\varphi \rightarrow (\rightharpoondown \varphi \rightarrow \bot).$
			
			\item[(NI)] $(\varphi \rightarrow \bot) \rightarrow\ \rightharpoondown \varphi.$
			\item[(DNE)] $\rightharpoondown \rightharpoondown \varphi \rightarrow \varphi.$
		\end{description}
	\end{itembox}
	
	$\Gamma$を公理系という場合は,$\Gamma$は$L(EC)$の式の集合である.$\Gamma$が空である場合もある.
	$L(EC)$の式$\chi$に対して$\Gamma$から{\bf EC}の証明が存在する(証明可能である)ことを
	\begin{align}
		\Gamma \provable{\mbox{{\bf EC}}} \chi
	\end{align}
	と書くが,{\bf EC}における$\Gamma$から$\chi$への証明とは,
	$L(EC)$の式の列$\varphi_{1},\varphi_{2},
	\cdots,\varphi_{n}$であって,$\varphi_{n}$は$\chi$であり,
	各$i \in \{1,2,\cdots,n\}$に対して
	\begin{itemize}
		\item $\varphi_{i}$は{\bf EC}の公理である.
		\item $\varphi_{i}$は$\Gamma$の公理である.
		\item $\varphi_{i}$は前の式から推論規則を用いて得られる式である.{\bf EC}の推論規則とは,
			\begin{description}
			\item[三段論法]
				$j,k < i$なる$k,j$が取れて,$\varphi_{k}$は
				$\varphi_{j} \rightarrow \varphi_{i}$である.
		\end{description} 
	\end{itemize}
	が満たされているものである.
	
	\item[{\bf PC}]
	$\varphi$をいずれかの言語の式とし,$x$を変項とする.
	$\varphi$に$x$が自由に現れているとき,$\varphi$に自由に現れている
	$x$を変項$t$で置き換えた式を
	\begin{align}
		\varphi(t/x)
	\end{align}
	とする.ただし$t$は$\varphi(t/x)$で{\bf $x$に置き換わった位置で束縛されない}とする.
	このことを{\bf $t$は$\varphi$の中で$x$への代入について自由である}とも言う.
	
	\begin{itembox}[l]{{\bf PC}の公理}
		$\varphi$と$\psi$と$\xi$を$L(PC)$の式とし,$x$と$t$を変項とするとき,
		次は{\bf PC}の公理である.
		\begin{description}
			\item[(S)] $(\varphi \rightarrow (\psi \rightarrow \chi)) 
				\rightarrow ((\varphi \rightarrow \psi)
				\rightarrow (\varphi \rightarrow \chi)).$
			\item[(K)] $\varphi \rightarrow (\psi \rightarrow \varphi).$
			\item[(DI1)] $\varphi \rightarrow (\varphi \vee \psi).$
			\item[(DI2)] $\psi \rightarrow (\varphi \vee \psi).$
			\item[(DE)] $(\varphi \rightarrow \chi) \rightarrow 
				((\psi \rightarrow \chi) \rightarrow ((\varphi \vee \psi) \rightarrow \chi)).$
			\item[(CI)] $\varphi \rightarrow (\psi \rightarrow (\varphi \wedge \psi)).$
			\item[(CE1)] $(\varphi \wedge \psi) \rightarrow \varphi.$
			\item[(CE2)] $(\varphi \wedge \psi) \rightarrow \psi.$
			
			\item[(UE)] $\forall x \varphi \rightarrow \varphi(\tau/x).$
				\\ \textcolor{red}{ただし,$\varphi$には$x$が自由に現れて,
				$t$は$\varphi$の中で$x$への代入について自由である.}
				
			\item[(EI)] $\varphi(\tau/x) \rightarrow \exists x \varphi.$
				\\ \textcolor{red}{ただし,$\varphi$には$x$が自由に現れて,
				$t$は$\varphi$の中で$x$への代入について自由である.}
				
			\item[(CTI)] $\varphi \rightarrow (\rightharpoondown \varphi \rightarrow \bot).$
			
			\item[(NI)] $(\varphi \rightarrow \bot) \rightarrow\ \rightharpoondown \varphi.$
			\item[(DNE)] $\rightharpoondown \rightharpoondown \varphi \rightarrow \varphi.$
		\end{description}
	\end{itembox}
	
	$\Gamma$を公理系という場合は,$\Gamma$は$L(PC)$の文の集合である.$\Gamma$が空である場合もある.
	$L(PC)$の式$\chi$に対して$\Gamma$から{\bf PC}の証明が存在する(証明可能である)ことを
	\begin{align}
		\Gamma \provable{\mbox{{\bf PC}}} \chi
	\end{align}
	と書くが,{\bf PC}における$\Gamma$から$\chi$への証明とは,
	$L(PC)$の式の列$\varphi_{1},\varphi_{2},
	\cdots,\varphi_{n}$であって,$\varphi_{n}$は$\chi$であり,
	各$i \in \{1,2,\cdots,n\}$に対して
	\begin{itemize}
		\item $\varphi_{i}$は{\bf PC}の公理である.
		\item $\varphi_{i}$は$\Gamma$の公理である.
		\item $\varphi_{i}$は前の式から推論規則を用いて得られる式である.{\bf PC}の推論規則とは,
			\begin{description}
			\item[三段論法]
				$j,k < i$なる$k,j$が取れて,$\varphi_{k}$は
				$\varphi_{j} \rightarrow \varphi_{i}$である.
			 	
			\item[存在汎化] 
				$j < i$なる$j$が取れて,$\varphi_{j}$は
				$\varphi(t/x) \rightarrow \psi$なる式であって,
				$\varphi_{i}$は$\exists x \varphi \rightarrow \psi$なる式である.
				\\ \textcolor{red}{ただし,$\varphi$には$x$が自由に現れ,
				$t$は$\varphi$の中で$x$への代入について自由である.
				また$t$は$\varphi$と$\psi$には自由に現れない.}
			
			\item[全称汎化] 
				$j < i$なる$j$が取れて,$\varphi_{j}$は
				$\psi \rightarrow \varphi(t/x)$なる式であって,
				$\varphi_{i}$は$\psi \rightarrow \forall x \varphi$なる式である.
				\\ \textcolor{red}{ただし,$\varphi$には$x$が自由に現れ,
				$t$は$\varphi$の中で$x$への代入について自由である.
				また$t$は$\varphi$と$\psi$には自由に現れない.}
		\end{description} 
	\end{itemize}
	が満たされているものである.
	
	\item[主要論理式]
	{\bf EC}${}_{\varepsilon}$と{\bf PC}${}_{\varepsilon}$の公理には
	\begin{align}
		\varphi(t/x) \rightarrow \varphi(\varepsilon x \varphi/x)
	\end{align}
	なる形の式が追加される.ただし$x$は$\varphi$に自由に現れて,
	$t$は$\varphi$の中で$x$への代入について自由である.
	この形の式を{\bf 主要論理式}{\bf (principal formula)}と呼ぶ.
	
	\item[{\bf EC}${}_{\varepsilon}$]\mbox{}
	
	\begin{itembox}[l]{{\bf EC}${}_{\varepsilon}$の公理}
		$\varphi$と$\psi$と$\xi$を$L(EC_{\varepsilon})$の式とするとき,
		次は{\bf EC}${}_{\varepsilon}$の公理である.
		\begin{description}
			\item[(S)] $(\varphi \rightarrow (\psi \rightarrow \chi)) 
				\rightarrow ((\varphi \rightarrow \psi)
				\rightarrow (\varphi \rightarrow \chi)).$
			\item[(K)] $\varphi \rightarrow (\psi \rightarrow \varphi).$
			\item[(DI1)] $\varphi \rightarrow (\varphi \vee \psi).$
			\item[(DI2)] $\psi \rightarrow (\varphi \vee \psi).$
			\item[(DE)] $(\varphi \rightarrow \chi) \rightarrow 
				((\psi \rightarrow \chi) \rightarrow ((\varphi \vee \psi) \rightarrow \chi)).$
			\item[(CI)] $\varphi \rightarrow (\psi \rightarrow (\varphi \wedge \psi)).$
			\item[(CE1)] $(\varphi \wedge \psi) \rightarrow \varphi.$
			\item[(CE2)] $(\varphi \wedge \psi) \rightarrow \psi.$
				
			\item[(CTI)] $\varphi \rightarrow (\rightharpoondown \varphi \rightarrow \bot).$
			
			\item[(NI)] $(\varphi \rightarrow \bot) \rightarrow\ \rightharpoondown \varphi.$
			\item[(DNE)] $\rightharpoondown \rightharpoondown \varphi \rightarrow \varphi.$
			\item[(PF)] $\varphi(t/x) \rightarrow \varphi(\varepsilon x \varphi/x).$
				\\ \textcolor{red}{ただし,$\varphi$には$x$が自由に現れて,
				$t$は$\varphi$の中で$x$への代入について自由である.}
		\end{description}
	\end{itembox}
	
	$\Gamma$を公理系とする.$L(EC_{\varepsilon})$の式$\chi$に対して$\Gamma$から
	{\bf EC}${}_{\varepsilon}$の証明が存在する(証明可能である)ことを
	\begin{align}
		\Gamma \provable{\mbox{{\bf EC}${}_{\varepsilon}$}} \chi
	\end{align}
	と書くが,{\bf EC}${}_{\varepsilon}$における$\Gamma$から$\chi$への証明とは,
	$L(EC_{\varepsilon})$の式の列$\varphi_{1},\varphi_{2},
	\cdots,\varphi_{n}$であって,$\varphi_{n}$は$\chi$であり,
	各$i \in \{1,2,\cdots,n\}$に対して
	\begin{itemize}
		\item $\varphi_{i}$は{\bf EC}${}_{\varepsilon}$の公理である.
		\item $\varphi_{i}$は$\Gamma$の公理である.
		\item $\varphi_{i}$は前の式から推論規則を用いて得られる式である.
			{\bf EC}${}_{\varepsilon}$の推論規則とは,
			\begin{description}
			\item[三段論法]
				$j,k < i$なる$k,j$が取れて,$\varphi_{k}$は
				$\varphi_{j} \rightarrow \varphi_{i}$である.
		\end{description} 
	\end{itemize}
	が満たされているものである.
	
	\item[{\bf PC}${}_{\varepsilon}$]\mbox{}
	
	\begin{itembox}[l]{{\bf PC}${}_{\varepsilon}$の公理}
		$\varphi$と$\psi$と$\xi$を$L(PC_{\varepsilon})$の式とし,$x$と$t$を変項とするとき,
		次は{\bf PC}${}_{\varepsilon}$の公理である.
		\begin{description}
			\item[(S)] $(\varphi \rightarrow (\psi \rightarrow \chi)) 
				\rightarrow ((\varphi \rightarrow \psi)
				\rightarrow (\varphi \rightarrow \chi)).$
			\item[(K)] $\varphi \rightarrow (\psi \rightarrow \varphi).$
			\item[(DI1)] $\varphi \rightarrow (\varphi \vee \psi).$
			\item[(DI2)] $\psi \rightarrow (\varphi \vee \psi).$
			\item[(DE)] $(\varphi \rightarrow \chi) \rightarrow 
				((\psi \rightarrow \chi) \rightarrow ((\varphi \vee \psi) \rightarrow \chi)).$
			\item[(CI)] $\varphi \rightarrow (\psi \rightarrow (\varphi \wedge \psi)).$
			\item[(CE1)] $(\varphi \wedge \psi) \rightarrow \varphi.$
			\item[(CE2)] $(\varphi \wedge \psi) \rightarrow \psi.$
			
			\item[(UE)] $\forall x \varphi \rightarrow \varphi(\tau/x).$
				\\ \textcolor{red}{ただし,$\varphi$には$x$が自由に現れて,
				$t$は$\varphi$の中で$x$への代入について自由である.}
				
			\item[(EI)] $\varphi(\tau/x) \rightarrow \exists x \varphi.$
				\\ \textcolor{red}{ただし,$\varphi$には$x$が自由に現れて,
				$t$は$\varphi$の中で$x$への代入について自由である.}
				
			\item[(CTI)] $\varphi \rightarrow (\rightharpoondown \varphi \rightarrow \bot).$
			
			\item[(NI)] $(\varphi \rightarrow \bot) \rightarrow\ \rightharpoondown \varphi.$
			\item[(DNE)] $\rightharpoondown \rightharpoondown \varphi \rightarrow \varphi.$
			\item[(PF)] $\varphi(t/x) \rightarrow \varphi(\varepsilon x \varphi/x).$
				\\ \textcolor{red}{ただし,$\varphi$には$x$が自由に現れて,
				$t$は$\varphi$の中で$x$への代入について自由である.}
		\end{description}
	\end{itembox}
	
	$\Gamma$を公理系とする.$L(PC_{\varepsilon})$の式$\chi$に対して
	$\Gamma$から{\bf PC}${}_{\varepsilon}$の証明が存在する(証明可能である)ことを
	\begin{align}
		\Gamma \provable{\mbox{{\bf PC}${}_{\varepsilon}$}} \chi
	\end{align}
	と書くが,{\bf PC}${}_{\varepsilon}$における$\Gamma$から$\chi$への証明とは,
	$L(PC_{\varepsilon})$の式の列$\varphi_{1},\varphi_{2},
	\cdots,\varphi_{n}$であって,$\varphi_{n}$は$\chi$であり,
	各$i \in \{1,2,\cdots,n\}$に対して
	\begin{itemize}
		\item $\varphi_{i}$は{\bf PC}${}_{\varepsilon}$の公理である.
		\item $\varphi_{i}$は$\Gamma$の公理である.
		\item $\varphi_{i}$は前の式から推論規則を用いて得られる式である.
			{\bf PC}${}_{\varepsilon}$の推論規則とは,
			\begin{description}
			\item[三段論法]
				$j,k < i$なる$k,j$が取れて,$\varphi_{k}$は
				$\varphi_{j} \rightarrow \varphi_{i}$である.
			 	
			\item[存在汎化] 
				$j < i$なる$j$が取れて,$\varphi_{j}$は
				$\varphi(t/x) \rightarrow \psi$なる式であって,
				$\varphi_{i}$は$\exists x \varphi \rightarrow \psi$なる式である.
				\\ \textcolor{red}{ただし,$\varphi$には$x$が自由に現れ,
				$t$は$\varphi$の中で$x$への代入について自由である.
				また$t$は$\varphi$と$\psi$には自由に現れない.}
			
			\item[全称汎化] 
				$j < i$なる$j$が取れて,$\varphi_{j}$は
				$\psi \rightarrow \varphi(t/x)$なる式であって,
				$\varphi_{i}$は$\psi \rightarrow \forall x \varphi$なる式である.
				\\ \textcolor{red}{ただし,$\varphi$には$x$が自由に現れ,
				$t$は$\varphi$の中で$x$への代入について自由である.
				また$t$は$\varphi$と$\psi$には自由に現れない.}
		\end{description} 
	\end{itemize}
	が満たされているものである.
	\end{description}
	第\ref{sec:restriction_of_formulas}節で決めた通り,
	扱う式は全て,そこに現れる$\varepsilon$項は全て主要$\varepsilon$項であり,
	現れる内包項は全て正則内包項であるとする.
	
\section{証明}
	本節では{\bf 推論規則}\index{すいろんきそく@推論規則}{\bf (rule of inference)}を導入し,
	基本的な{\bf 推論法則}\index{すいろんほうそく@推論法則}を導出する.
	推論法則とは,他の本ではそれが用いている証明体系の定理などと呼ばれるが,
	本稿では集合論特有の定理と区別するために推論法則と呼ぶ.
	以下では
	\begin{align}
		\vdash
	\end{align}
	なる記号を用いて,
	\begin{align}
		\varphi \vdash \psi
	\end{align}
	などと書く.$\vdash$の左右にあるのは必ず($\mathcal{L}$の)文であって,
	右側に置かれる文は必ず一本だけであるが,左側には文がいくつあっても良いし,全く無くても良い.
	特に
	\begin{align}
		\vdash \psi
	\end{align}
	を満たす文$\psi$を推論法則と呼ぶ.
	{\bf ``$\vdash$の右の文は,$\vdash$の左の文から証明できる''},と読むが,
	証明とはどのようにされるのだとか,$\vdash \psi$を満たすとは
	どういう意味なのか,とかいったことは後に回して,とりあえず記号のパズルゲームと見立てて
	$\vdash$のルールを定める.
	
	\begin{screen}
		\begin{logicalrule}[演繹規則]\label{logicalrule:deduction_rule}
			$A,B,C,D$を文とするとき,
			\begin{description}
				\item[(a)] $A \vdash D$ならば$\vdash A \rarrow D$が成り立つ.
				\item[(b)] $A,B \vdash D$ならば
					\begin{align}
						B \vdash A \rarrow D,\quad
						A \vdash B \rarrow D
					\end{align}
					が成り立つ.
				\item[(c)] $A,B,C \vdash D$ならば
					\begin{align}
						B,C \vdash A \rarrow D,\quad
						A,C \vdash B \rarrow D,\quad
						A,B \vdash C \rarrow D
					\end{align}
					のいずれも成り立つ.
			\end{description}
		\end{logicalrule}
	\end{screen}
	
	演繹規則においては$\vdash$の左側にせいぜい三つの文しかないのだが,
	実は$\vdash$の左側に不特定多数の文を持ってきても
	演繹規則じみたことが成立する(後述の演繹法則).
	
	では{\bf 証明}\index{しょうめい@証明}{\bf (proof)}とは何かを規定する.
	
	自由な変項が現れない($\mathcal{L}$の)式を{\bf 文}\index{ぶん@文}{\bf (sentence)}や
	{\bf 閉式}\index{へいしき@閉式}{\bf (closed formula)}と呼ぶ.
	証明される式や証明の過程で出てくる式は全て文である.本稿では証明された文を
	{\bf 真な}\index{しん@真}{\bf (true)}文と呼ぶことにするが,
	``証明された''や``真である''という状態は議論が立脚している前提に依存する.
	ここでいう前提とは,推論規則や言語ではなくて
	{\bf 公理系}\index{こうりけい@公理系}{\bf (axioms)}と呼ばれるものを指している.
	公理系とは文の集まりである.$\mathscr{S}$を公理系とするとき,
	$\mathscr{S}$に集められた文を$\mathscr{S}$の{\bf 公理}\index{こうり@公理}{\bf (axiom)}
	と呼ぶ.以下では本稿の集合論が立脚する公理系を$\Sigma$と書くが,
	$\Sigma$に属する文は単に公理と呼んだりもする.
	
	$\Sigma$とは以下の文からなる:
	\begin{description}
		\item[相等性] $a,b,c$を類とするとき
			\begin{align}
				&a = b \rarrow (\, a \in c \rarrow b \in c\, ), \\
				&a = b \rarrow (\, c \in a \rarrow c \in b\, ).
			\end{align}
			
		\item[外延性] $a$と$b$を類とするとき
			\begin{align}
				\forall x\, (\, x \in a \lrarrow x \in b\, ) \rarrow a = b.
			\end{align}
			
		\item[内包性] $\varphi$を$\mathcal{L}$の式とし,$y$を$\varphi$に自由に現れる変項とし,
			$\varphi$に自由に現れる項は$y$のみであるとし,
			$x$は$\varphi$で$y$への代入について自由であるとするとき,
			\begin{align}
				\forall x\, \left(\, x \in \Set{y}{\varphi(y)} \lrarrow \varphi(x)\, \right).
			\end{align}
		
		\item[要素] $a,b$を類とするとき
			\begin{align}
				a \in b \rarrow \exists x\, (\, x = a\, ).
			\end{align}
			
		\item[対] $\forall x\, \forall y\, \exists p\, \forall z\, 
			(\, x = z \vee y = z \lrarrow z \in p\, ).$
			
		\item[合併] $\forall x\, \exists u\, \forall y\, (\, \exists z\, (\, z \in x \wedge y \in z\, ) \lrarrow y \in u\, ).$
			
		\item[冪] $\forall x\, \exists p\, \forall y\, 
			(\, \forall z\, (\, z \in y \rarrow z \in x\, ) \lrarrow y \in p\, ).$
			
		\item[置換] $\varphi$を$\mathcal{L}$の式とし,
			$s,t$を$\varphi$に自由に現れる変項とし,
			$\varphi$に自由に現れる項は$s,t$のみであるとし,
			$x$は$\varphi$で$s$への代入について自由であり,
			$y,z$は$\varphi$で$t$への代入について自由であるとするとき,
			\begin{align}
				\forall x\, \forall y\, \forall z\, 
				(\, \varphi(x,y) \wedge \varphi(x,z)
				\rarrow y = z\, )
				\rarrow \forall a\, \exists z\, \forall y\,
				(\, y \in z \lrarrow \exists x\, (\, x \in a \wedge 
				\varphi(x,y)\, )\, ).
			\end{align}
			
		\item[正則性] $a$を類とするとき,
			\begin{align}
				\exists x\, (\, x \in a\, ) \rarrow
				\exists y\, (\, y \in a \wedge \forall z\, (\, z \in y \rarrow
				z \notin a\, )\, ).
			\end{align}
			
		\item[無限] $\exists x\, (\, 
				\exists s\, (\, \forall t\, (\, t \notin s\, ) \wedge s \in x\, ) 
				\wedge \forall y\, (\, 
				y \in x \rarrow \exists u\, (\, 
				\forall v\, (\, v \in u \lrarrow v \in y \vee v = y\, )
				\wedge u \in x\, )\, )\, ).$
			
		\item[選択]
			
	\end{description}
	
	\begin{screen}
		\begin{metadfn}[証明可能]
			文$\varphi$が公理系$\mathscr{S}$から
			{\bf 証明された}だとか{\bf 証明可能である}\index{しょうめいかのう@証明可能}
			{\bf (provable)}ということは,
			\begin{itemize}
				\item $\varphi$は$\mathscr{S}$の公理である.
				\item $\vdash \varphi$である.
				\item 文$\psi$で,$\psi$と$\psi \rightarrow \varphi$が$\mathscr{S}$から
				証明されているものが取れる({\bf 三段論法}\index{さんだんろんぽう@三段論法}
				{\bf (Modus Pones)}).
			\end{itemize}
			のいずれかが満たされているということである.
		\end{metadfn}
	\end{screen}		
	
	$\varphi$が$\mathscr{S}$から証明可能であることを
	\begin{align}
		\mathscr{S} \vdash \varphi
	\end{align}
	と書く.ただし,{\bf 公理系に変項が生じた場合の証明可能性には
	演繹規則や後述の演繹法則,およびその逆の結果を適用することが出来る}.
	
	たとえばどんな文$\varphi$に対しても
	\begin{align}
		\varphi \vdash \varphi
	\end{align}
	となるし,どんな文$\psi$を追加しても
	\begin{align}
		\varphi,\psi \vdash \varphi
	\end{align}
	となる.これらは最も単純なケースであり,大抵の定理は数多くの複雑なステップを踏まなくては得られない.
	$\mathscr{S}$から証明済みの$\varphi$を起点にして$\mathscr{S} \vdash \psi$であると判明すれば,
	$\varphi$から始めて$\psi$が真であることに辿り着くまでの一連の作業は$\psi$の$\mathscr{S}$からの
	{\bf 証明}\index{しょうめい@証明}{\bf (proof)}と呼ばれ,
	$\psi$は$\mathscr{S}$の{\bf 定理}\index{ていり@定理}{\bf (theorem)}と呼ばれる.
	
	$A,B \vdash \varphi$とは
	$A$と$B$の二つの文のみを公理とした体系において$\varphi$が証明可能であることを表している.
	特に{\bf 推論法則とは公理の無い体系で推論規則だけから導かれる定理}のことである.
	
	ではさっそく演繹法則の証明に進む.ところで,後で見るとおり演繹法則とは
	証明が持つ性質に対する言明であって,つまりメタ視点での定理ということになるので,
	演繹法則の``証明''とは言っても上で規定した証明とは意味が違う.
	メタ定理の``証明''は,本稿では{\bf メタ証明}と呼んで区別する.
	演繹法則を示す前に推論法則を三本用意しなくてはならない.
	
	\begin{screen}
		\begin{logicalthm}[含意の反射律]\label{logicalthm:reflective_law_of_implication}
			$A$を文とするとき
			\begin{align}
				\vdash A \rarrow A.
			\end{align}
		\end{logicalthm}
	\end{screen}
	
	上の言明は``どんな文でも持ってくれば,その式に対して反射律が成立する''という意味である.
	このように無数に存在し得る定理を一括して表す式は{\bf 公理図式}\index{こうりずしき@公理図式}{\bf (schema)}と呼ばれる.
	
	\begin{prf}
		$A \vdash A$であるから,演繹規則より$\vdash A \rarrow A$となる.
		\QED
	\end{prf}
	
	\begin{screen}
		\begin{logicalthm}[含意の導入]
		\label{logicalthm:introduction_of_implication}
			$A,B$を文とするとき
			\begin{align}
				\vdash B \rarrow (\, A \rarrow B\, ).
			\end{align}
		\end{logicalthm}
	\end{screen}
	
	\begin{prf}
		\begin{align}
			A,B \vdash B
		\end{align}
		より演繹規則から
		\begin{align}
			B \vdash A \rarrow B
		\end{align}
		となり,再び演繹規則より
		\begin{align}
			\vdash B \rarrow (\, A \rarrow B\, )
		\end{align}
		が得られる.
		\QED
	\end{prf}
	
	演繹法則を示すための推論法則の導出は次で最後である.
	
	\begin{screen}
		\begin{logicalthm}[含意の分配則]
		\label{logicalthm:distributive_law_of_implication}
			$A,B,C$を文とするとき
			\begin{align}
				\vdash (\, A \rarrow (\, B \rarrow C\, )\, ) 
				\rarrow (\, (\, A \rarrow B\, ) \rarrow (\, A \rarrow C\, )\, ).
			\end{align}
		\end{logicalthm}
	\end{screen}
	
	\begin{prf}
		証明可能性の規則より
		\begin{align}
			A \rarrow (\, B \rarrow C\, ),\ A \rarrow B,\ A
			&\vdash A, \\
			A \rarrow (\, B \rarrow C\, ),\ A \rarrow B,\ A
			&\vdash A \rarrow B
		\end{align}
		となるので
		\begin{align}
			A \rarrow (\, B \rarrow C\, ),\ A \rarrow B,\ A
			\vdash B
		\end{align}
		が成り立つし,同じように
		\begin{align}
			A \rarrow (\, B \rarrow C\, ),\ A \rarrow B,\ A
			&\vdash A, \\
			A \rarrow (\, B \rarrow C\, ),\ A \rarrow B,\ A
			&\vdash A \rarrow (\, B \rarrow C\, )
		\end{align}
		であるから
		\begin{align}
			A \rarrow (\, B \rarrow C\, ),\ A \rarrow B,\ A
			\vdash B \rarrow C
		\end{align}
		も成り立つ.これによって
		\begin{align}
			A \rarrow (\, B \rarrow C\, ),\ A \rarrow B,\ A \vdash C
		\end{align}
		も成り立つから,あとは演繹規則を順次適用すれば
		\begin{align}
			A \rarrow (\, B \rarrow C\, ),\ A \rarrow B
			&\vdash A \rarrow C, \\
			A \rarrow (\, B \rarrow C\, )
			&\vdash (\, A \rarrow B\, ) \rarrow (\, A \rarrow C\, ), \\
			&\vdash (\, A \rarrow (\, B \rarrow C\, )\, ) 
				\rarrow (\, (\, A \rarrow B\, ) \rarrow (\, A \rarrow C\, )\, )
		\end{align}
		となる.
		\QED
	\end{prf}
	
	\begin{screen}
		\begin{metaaxm}[証明に対する構造的帰納法]
			$\mathscr{S}$を公理系とし,Xを文に対する何らかの言明とするとき,
			\begin{itemize}
				\item $\mathscr{S}$の公理に対してXが言える.
				\item 推論法則に対してXが言える.
				\item $\varphi$と$\varphi \rarrow \psi$が$\mathscr{S}$の
					定理であるような文$\varphi$と文$\psi$が取れたとき,
					$\varphi$と$\varphi \rarrow \psi$に対して
					Xが言えるならば,$\psi$に対してXが言える.
			\end{itemize}
			のすべてが満たされていれば,$\mathscr{S}$から証明可能なあらゆる文に対してXが言える.
		\end{metaaxm}
	\end{screen}
	
	公理系$\mathscr{S}$に文$A$を追加した公理系を
	\begin{align}
		A,\ \mathscr{S}
	\end{align}
	や
	\begin{align}
		\mathscr{S},\ A
	\end{align}
	と書く.$A$が既に$\mathscr{S}$の公理であってもこのように表記するが,
	その場合は$\mathscr{S}, A$や$A,\mathscr{S}$とは$\mathscr{S}$そのものである.
	
	\begin{screen}
		\begin{metathm}[演繹法則]\label{metathm:deduction_theorem}
			$\mathscr{S}$を公理系とし,$A$を文とするとき,
			$\mathscr{S}, A$の任意の定理$B$に対して
			\begin{align}
				\mathscr{S} \vdash A \rarrow B
			\end{align}
			が成り立つ.
		\end{metathm}
	\end{screen}
	
	\begin{metaprf}\mbox{}
		\begin{description}
			\item[第一段]
				$B$を$\mathscr{S},A$の公理か或いは推論法則とする.
				$B$が$A$ならば含意の反射律
				(推論法則\ref{logicalthm:reflective_law_of_implication})より
				\begin{align}
					\vdash A \rarrow B
				\end{align}
				が成り立つので
				\begin{align}
					\mathscr{S} \vdash A \rarrow B
				\end{align}
				となる.$B$が$\mathscr{S}$の公理又は推論法則であるとき,まず
				\begin{align}
					\mathscr{S} \vdash B
				\end{align}
				が成り立つが,他方で含意の導入
				(推論法則\ref{logicalthm:introduction_of_implication})より
				\begin{align}
					\mathscr{S} \vdash B \rarrow (\, A \rarrow B\, ) 
				\end{align}
				も成り立つので,証明可能性の定義より
				\begin{align}
					\mathscr{S} \vdash A \rarrow B
				\end{align}
				が従う.
				
			\item[第二段]
				$C$及び$C \rarrow B$が$\mathscr{S}$の定理であるような
				文$C$と文$B$が取れた場合,
				\begin{align}
					\mathscr{S} \vdash A \rarrow (\, C \rarrow B\, )
				\end{align}
				かつ
				\begin{align}
					\mathscr{S} \vdash A \rarrow C
				\end{align}
				であると仮定する.含意の分配則
				(\ref{logicalthm:distributive_law_of_implication})より
				\begin{align}
					\mathscr{S} \vdash 
					(\, A \rarrow (\, C \rarrow B\, )\, ) 
					\rarrow (\, (\, A \rarrow C\, ) \rarrow (\, A \rarrow B\, )\, )
				\end{align}
				が満たされるので,証明可能性の定義の通りに
				\begin{align}
					\mathscr{S} \vdash (\, A \rarrow C\, ) 
					\rarrow (\, A \rarrow B\, )
				\end{align}
				が従い,
				\begin{align}
					\mathscr{S} \vdash A \rarrow B
				\end{align}
				が従う.以上と構造的帰納法より,$\mathscr{S},A$の任意の定理$B$に対して
				\begin{align}
					\mathscr{S} \vdash A \rarrow B
				\end{align}
				が言える.
				\QED
		\end{description}
	\end{metaprf}
	
	演繹法則の逆も得られる.つまり,
	$\mathscr{S}$を公理系とし,$A$と$B$を文とするとき,
	\begin{align}
		\mathscr{S} \vdash A \rarrow B
	\end{align}
	であれば
	\begin{align}
		A,\ \mathscr{S} \vdash B
	\end{align}
	が成り立つ.実際
	\begin{align}
		A,\ \mathscr{S} \vdash A
	\end{align}
	が成り立つのは証明の定義の通りであるし,
	$A \rarrow B$が$\mathscr{S}$の定理ならば
	\begin{align}
		A,\ \mathscr{S} \vdash A \rarrow B
		\label{fom:inversion_of_deduction_theorem}
	\end{align}
	が成り立つので,併せて
	\begin{align}
		A,\ \mathscr{S} \vdash B
	\end{align}
	が従う.ただし(\refeq{fom:inversion_of_deduction_theorem})に
	関しては次のメタ定理を示さなくてはいけない.
	
	\begin{screen}
		\begin{metathm}[公理が増えても証明可能]
			$\mathscr{S}$を公理系とし,$A$を文とするとき,
			$\mathscr{S}$の任意の定理$B$に対して
			\begin{align}
				A,\ \mathscr{S} \vdash B
			\end{align}
			が成り立つ.
		\end{metathm}
	\end{screen}
	
	\begin{metaprf}
		$B$が$\mathscr{S}$の公理であるか推論規則であれば
		\begin{align}
			A,\ \mathscr{S} \vdash B
		\end{align}
		は言える.また
		\begin{align}
			\mathscr{S} &\vdash C, \\
			\mathscr{S} &\vdash C \rarrow B
		\end{align}
		を満たす文$C$が取れるとき,
		\begin{align}
			A,\ \mathscr{S} &\vdash C, \\
			A,\ \mathscr{S} &\vdash C \rarrow B
		\end{align}
		と仮定すれば
		\begin{align}
			A,\ \mathscr{S} \vdash B
		\end{align}
		となる.以上と構造的帰納法より$\mathscr{S}$の任意の定理$B$に対して
		\begin{align}
			A,\ \mathscr{S} \vdash B
		\end{align}
		が成り立つ.
		\QED
	\end{metaprf}
	
	\begin{screen}
		\begin{metathm}[演繹法則の逆]
		\label{metathm:inverse_of_deduction_theorem}
			$\mathscr{S}$を公理系とし,$A$と$B$を文とするとき,
			\begin{align}
				\mathscr{S} \vdash A \rarrow B
			\end{align}
			であれば
			\begin{align}
				A,\ \mathscr{S} \vdash B
			\end{align}
			が成り立つ.
		\end{metathm}
	\end{screen}
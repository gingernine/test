\section{推論}
	この節では「類は集合であるか真類であるかのいずれかには定まる」と
	「集合であり真類でもある類は存在しない」の二つの言明を得ることを主軸に
	基本的な推論法則を導出する.
	
	ここで論理記号の名称を書いておく.
	\begin{itemize}
		\item $\vee$を{\bf 論理和}\index{ろんりわ@論理和}{\bf (logical disjunction)}
			や{\bf 選言}と呼ぶ.
		\item $\wedge$を{\bf 論理積}\index{ろんりせき@論理積}{\bf (logical conjunction)}
			や{\bf 連言}と呼ぶ.
		\item $\rarrow$を{\bf 含意}\index{がんい@含意}{\bf (implication)}と呼ぶ.
		\item $\negation$を{\bf 否定}\index{ひてい@否定}{\bf (negation)}と呼ぶ.
	\end{itemize}
	
	\begin{screen}
		\begin{logicalrule}[否定と矛盾に関する規則]
		\label{logicalrule:rules_of_negation_and_contradiction}
			$A$を文とするとき以下が成り立つ:
			\begin{description}
				\item[矛盾の導入] 否定が共に成り立つとき矛盾が起きる:
					\begin{align}
						A,\ \negation A \vdash \bot.
					\end{align}
				\item[否定の導入] 矛盾が導かれるとき否定が成り立つ:
					\begin{align}
						A \rarrow \bot \vdash\ \negation A.
					\end{align}
			\end{description}
		\end{logicalrule}
	\end{screen}
	
	矛盾の導入規則に演繹規則を適用すれば
	\begin{align}
		\vdash A \rarrow (\, \negation A \rarrow \bot\, )
	\end{align}
	が得られる.証明可能性の定義では推論法則を直接用いることはが許されているので,
	実際の証明の最中にこの規則を``適用する''ときは,上ように推論法則に直したものを使う.
	たとえば公理系$\mathscr{S}$の下で
	\begin{align}
		\mathscr{S} \vdash A
		\label{fom:introduction_of_contradiction_1}
	\end{align}
	と
	\begin{align}
		\mathscr{S} \vdash\ \negation A
		\label{fom:introduction_of_contradiction_2}
	\end{align}
	が導かれたとすれば,矛盾の導入規則から得られた推論法則は
	\begin{align}
		\mathscr{S} \vdash A \rarrow (\, \negation A \rarrow \bot\, )
	\end{align}
	を満たすので,(\refeq{fom:introduction_of_contradiction_1})との三段論法より
	\begin{align}
		\mathscr{S} \vdash\ \negation A \rarrow \bot
	\end{align}
	となり,(\refeq{fom:introduction_of_contradiction_2})との三段論法より
	\begin{align}
		\mathscr{S} \vdash \bot
	\end{align}
	が従う,といった要領である.これを始めと結論だけ見て直感的に
	\begin{prooftree}
		\AxiomC{$\mathscr{S} \vdash A$}
		\AxiomC{$\mathscr{S} \vdash\ \negation A$}
		\BinaryInfC{$\mathscr{S} \vdash \bot$}
	\end{prooftree}
	と書いてみれば,こちらは矛盾の導入規則そのままの形に似ているので,
	あたかも矛盾の導入規則を``直接''適用したように見える.他の推論規則も
	実際の証明では推論法則に直したものを用いるのであるが,
	始めと結論だけ見れば
	\begin{prooftree}
		\AxiomC{$\mathscr{S} \vdash A \rarrow \bot$}
		\UnaryInfC{$\mathscr{S} \vdash\ \negation A$}
	\end{prooftree}
	であったり
	\begin{prooftree}
		\AxiomC{$\mathscr{S} \vdash A$}
		\UnaryInfC{$\mathscr{S} \vdash A \vee B$}
	\end{prooftree}
	であったり
	\begin{prooftree}
		\AxiomC{$\mathscr{S} \vdash A(\tau)$}
		\UnaryInfC{$\mathscr{S} \vdash \exists x A(x)$}
	\end{prooftree}
	であったりが``成り立つ''のである.そしてこれらの水平線が入った規則もどきは
	他の証明体系では正式な推論規則であったりする.
	どうせすぐに推論法則に直してしまうものを,どうしてわざわざ推論規則として導入したのかというと
	(推論法則の形で推論の公理として導入しても同じである),
	他の証明体系の規則との対応を意識しているのと,
	$\vdash$によって前後に分割してある方が若干見やすいからである.
	
	\begin{screen}
		\begin{dfn}[対偶]
			$\varphi \rarrow \psi$なる式に対して
			\begin{align}
				\negation \psi \rarrow\ \negation \varphi
			\end{align}
			を$\varphi \rarrow \psi$の{\bf 対偶}\index{たいぐう@対偶}
			{\bf (contraposition)}と呼ぶ.
		\end{dfn}
	\end{screen}
	
	\begin{screen}
		\begin{logicalthm}[対偶命題が導かれる]
		\label{logicalthm:introduction_of_contraposition}
			$A$と$B$を文とするとき
			\begin{align}
				\vdash (\, A \rarrow B\, ) 
				\rarrow (\, \negation B \rarrow\ \negation A\, ).
			\end{align}
		\end{logicalthm}
	\end{screen}
	
	\begin{prf}
		証明可能性の定義より
		\begin{align}
			A,\ \negation B,\ A \rarrow B &\vdash A, \\
			A,\ \negation B,\ A \rarrow B &\vdash A \rarrow B
		\end{align}
		となるので,三段論法より
		\begin{align}
			A,\ \negation B,\ A \rarrow B \vdash B
			\label{fom:introduction_of_contraposition_1}
		\end{align}
		が従う.同じく証明可能性の定義より
		\begin{align}
			A,\ \negation B,\ A \rarrow B \vdash\ \negation B
			\label{fom:introduction_of_contraposition_2}
		\end{align}
		も成り立つ.ところで矛盾の導入規則より
		\begin{align}
			\vdash B \rarrow (\, \negation B \rarrow \bot\, )
		\end{align}
		が成り立つので,証明可能性の定義より
		\begin{align}
			A,\ \negation B,\ A \rarrow B \vdash
			B \rarrow (\, \negation B \rarrow \bot\, )
		\end{align}
		となる.これと(\refeq{fom:introduction_of_contraposition_1})との三段論法より
		\begin{align}
			A,\ \negation B,\ A \rarrow B \vdash\ \negation B \rarrow \bot
		\end{align}
		が従い,これと(\refeq{fom:introduction_of_contraposition_2})との三段論法より
		\begin{align}
			A,\ \negation B,\ A \rarrow B \vdash \bot
		\end{align}
		が従う.演繹規則より
		\begin{align}
			\negation B,\ A \rarrow B \vdash A \rarrow \bot
		\end{align}
		となるが,今度は否定の導入規則より
		\begin{align}
			\negation B,\ A \rarrow B \vdash 
			(\, A \rarrow \bot\, ) \rarrow\ \negation A
		\end{align}
		が満たされるので,三段論法より
		\begin{align}
			\negation B,\ A \rarrow B \vdash\ \negation A
		\end{align}
		が出る.そして演繹規則より
		\begin{align}
			A \rarrow B \vdash\ \negation B \rarrow\ \negation A
		\end{align}
		が得られ,再び演繹規則より
		\begin{align}
			\vdash (\, A \rarrow B\, ) \rarrow
			(\, \negation B \rarrow\ \negation A\, )
		\end{align}
		が得られる.
		\QED
	\end{prf}
	
	公理系$\mathscr{S}$の下で$A \rarrow B$が導かれたとすれば,上の推論法則より
	\begin{align}
		\mathscr{S} \vdash (\, A \rarrow B\, ) 
		\rarrow (\, \negation B \rarrow\ \negation A\, )
	\end{align}
	が成り立つので三段論法より
	\begin{align}
		\mathscr{S} \vdash\ \negation B \rarrow\ \negation A
	\end{align}
	が従う.以下では,$\mathscr{S} \vdash A \rarrow B$であるときに
	{\bf 「対偶を取る」}と宣言して$\mathscr{S} \vdash\ \negation B \rarrow\ \negation A$
	に繋げることもある.
	
	\begin{screen}
		\begin{dfn}[二重否定]
			式$\varphi$に対して,$\negation$を二つ連結させた式
			\begin{align}
				\negation \negation \varphi
			\end{align}
			を$\varphi$の{\bf 二重否定}\index{にじゅうひてい@二重否定}
			{\bf (double negation)}と呼ぶ.
		\end{dfn}
	\end{screen}
		
	\begin{screen}
		\begin{logicalthm}[二重否定の導入]
		\label{logicalthm:introduction_of_double_negation}
			$A$を文とするとき
			\begin{align}
				\vdash A \rarrow \negation \negation A.
			\end{align}
		\end{logicalthm}
	\end{screen}
	
	\begin{prf}
		矛盾の導入規則より
		\begin{align}
			A,\ \negation A \vdash \bot
		\end{align}
		となるので,演繹規則より
		\begin{align}
			A \vdash\ \negation A \rarrow \bot
			\label{fom:introduction_of_double_negation}
		\end{align}
		が従う.また否定の導入規則より
		\begin{align}
			\vdash (\, \negation A \rarrow \bot\, ) \rarrow \negation \negation A
		\end{align}
		が成り立つので,証明可能性の定義より
		\begin{align}
			A \vdash (\, \negation A \rarrow \bot\, ) \rarrow \negation \negation A
		\end{align}
		も成り立ち,(\refeq{fom:introduction_of_double_negation})との三段論法より
		\begin{align}
			A \vdash\ \negation \negation A
		\end{align}
		が従う.そして演繹規則より
		\begin{align}
			\vdash A \rarrow \negation \negation A
		\end{align}
		が得られる.
		\QED
	\end{prf}
	
	\begin{screen}
		\begin{logicalrule}[論理積の除去]
		\label{logicalrule:elimination_of_conjunction}
			$A$と$B$を文とするとき
			\begin{align}
				A &\wedge B \vdash A, \\
				A &\wedge B \vdash B.
			\end{align}
		\end{logicalrule}
	\end{screen}
	
	肯定と否定は両立しない.
	
	\begin{screen}
		\begin{logicalthm}[無矛盾律]
		\label{logicalthm:law_of_noncontradiction}
			$A$を文とするとき
			\begin{align}
				\vdash\ \negation (\, A \wedge \negation A\, ).
			\end{align}
		\end{logicalthm}
	\end{screen}
	
	\begin{prf}
		論理積の除去規則より
		\begin{align}
			A \wedge \negation A &\vdash A, \\
			A \wedge \negation A &\vdash\ \negation A
		\end{align}
		が成り立ち,また矛盾の導入規則より
		\begin{align}
			A \wedge \negation A \vdash A \rarrow (\, \negation A \rarrow \bot\, )
		\end{align}
		が成り立つので,三段論法より
		\begin{align}
			A \wedge \negation A \vdash \bot
		\end{align}
		が従う.ゆえに演繹規則より
		\begin{align}
			\vdash (\, A \wedge \negation A\, ) \rarrow \bot
		\end{align}
		となり,否定の導入規則
		\begin{align}
			\vdash (\, (\, A \wedge \negation A\, ) \rarrow \bot\, )
			\rarrow \negation (\, A \wedge \negation A\, )
		\end{align}
		との三段論法より
		\begin{align}
			\vdash\ \negation (\, A \wedge \negation A\, )
		\end{align}
		が得られる.
		\QED
	\end{prf}
	
	ここで新しい論理記号$\lrarrow$を定めるが,そのときに$\defarrow$なる記号を用いる.
	これは{\bf 定義記号}\index{ていぎきごう@定義記号}と呼ばれ,
	\begin{align}
		P \defarrow \varphi
	\end{align}
	と書けば「式$\varphi$を記号$P$で置き換えて良い」という意味で略記法を導入できる.
	
	\begin{screen}
		\begin{dfn}[同値記号]
			$A$と$B$を$\mathcal{L}$の式とするとき,
			\begin{align}
				A \lrarrow B  \defarrow
				(\, A \rarrow B\, ) \wedge (\, B \rarrow A\, )
			\end{align}
			により$\lrarrow$を定め,式`$A \lrarrow B$'を
			「$A$と$B$は{\bf 同値である}\index{どうち@同値}{\bf (equivalent)}」と読む.
		\end{dfn}
	\end{screen}
	
	以降で{\bf De Morgan の法則}\index{De Morgan の法則}{\bf (De Morgan's laws)}
	\begin{align}
		&\vdash\ \negation (\, \varphi \vee \psi\, ) \lrarrow\ \negation \varphi
		\wedge \negation \psi, \\
		&\vdash\ \negation (\, \varphi \wedge \psi\, ) \lrarrow\ \negation \varphi
		\vee \negation \psi
	\end{align}
	を順番に示していくが,区別するために前者を{\bf 弱 De Morgan の法則}と呼び,
	後者を{\bf 強 De Morgan の法則}と呼ぶ.
	
	\begin{screen}
		\begin{logicalrule}[論理和の除去]
		\label{logicalrule:elimination_of_disjunction}
			$A$と$B$と$C$を文とするとき
			\begin{align}
				A \rarrow C,\ B \rarrow C \vdash A \vee B \rarrow C.
			\end{align}
		\end{logicalrule}
	\end{screen}
	
	{\bf 論理和の除去}とは{\bf 場合分け}\index{ばあいわけ@場合分け}{\bf (proof by case)}
	とも呼ばれる.また場合分け規則に演繹規則を二回適用すれば
	\begin{align}
		\vdash (\, A \rarrow C\, ) 
		\rarrow (\, (\, B \rarrow C\, ) \rarrow (\, A \vee B \rarrow C\, )\, )
	\end{align}
	なる推論法則が得られる.場合分け規則を実際に用いる際には主にこちらの推論法則を使う.
	
	\begin{screen}
		\begin{logicalthm}[弱 De Morgan の法則(1)]
		\label{logicalthm:weak_De_Morgan_law_1}
			$A$と$B$を文とするとき
			\begin{align}
				\vdash\ \negation A \wedge \negation B
				\rarrow\ \negation (\, A \vee B\, ).
			\end{align}
		\end{logicalthm}
	\end{screen}
	
	\begin{prf}
		論理積の除去規則より
		\begin{align}
			\negation A \wedge \negation B \vdash\ \negation A
			\label{fom:weak_De_Morgan_law_1_1}
		\end{align}
		となり,また矛盾の導入規則より
		\begin{align}
			\vdash\ \negation A \rarrow (\, A \rarrow \bot\, )
		\end{align}
		が成り立つので
		\begin{align}
			\negation A \wedge \negation B 
			\vdash\ \negation A \rarrow (\, A \rarrow \bot\, )
		\end{align}
		も成り立ち,(\refeq{fom:weak_De_Morgan_law_1_1})との三段論法より
		\begin{align}
			\negation A \wedge \negation B &\vdash A \rarrow \bot
			\label{fom:weak_De_Morgan_law_1_2}
		\end{align}
		が従う.同様に
		\begin{align}
			\negation A \wedge \negation B \vdash B \rarrow \bot
			\label{fom:weak_De_Morgan_law_1_3}
		\end{align}
		も得られる.ところで場合分け規則より
		\begin{align}
			\vdash (\, A \rarrow \bot\, ) \rarrow (\, (\, B \rarrow \bot\, )
			\rarrow (\, A \vee B \rarrow \bot\, )\, )
		\end{align}
		が成り立つので,(\refeq{fom:weak_De_Morgan_law_1_2})と
		(\refeq{fom:weak_De_Morgan_law_1_3})との三段論法より
		\begin{align}
			\negation A \wedge \negation B &\vdash (\, A \rarrow \bot\, ) 
			\rarrow (\, (\, B \rarrow \bot\, ) 
			\rarrow (\, A \vee B \rarrow \bot\, )\, ), \\
			\negation A \wedge \negation B &\vdash (\, B \rarrow \bot\, ) 
			\rarrow (\, A \vee B \rarrow \bot\, ), \\
			\negation A \wedge \negation B &\vdash A \vee B \rarrow \bot
		\end{align}
		となり,否定の導入規則から得られる推論法則
		\begin{align}
			\vdash (\, A \vee B \rarrow \bot\, ) \rarrow\ \negation (\, A \vee B\, )
		\end{align}
		との三段論法より
		\begin{align}
			\negation A \wedge \negation B \vdash\ \negation (\, A \vee B\, ) 
		\end{align}
		が得られる.そして演繹規則より
		\begin{align}
			\vdash\ \negation A \wedge \negation B
				\rarrow\ \negation (\, A \vee B\, )
		\end{align}
		が出る.
		\QED
	\end{prf}
	
	\begin{screen}
		\begin{logicalthm}[強 De Morgan の法則(1)]
		\label{logicalthm:strong_De_Morgan_law_1}
			$A$と$B$を文とするとき
			\begin{align}
				\vdash\ \negation A \vee \negation B
				\rarrow\ \negation (\, A \wedge B\, ).
			\end{align}
		\end{logicalthm}
	\end{screen}
	
	\begin{prf}
		論理積の除去より
		\begin{align}
			\vdash (\, A \wedge B\, ) \rarrow A
		\end{align}
		が成り立つので,対偶を取れば
		\begin{align}
			\vdash\ \negation A \rarrow\ \negation (\, A \wedge B\, )
			\label{fom:strong_De_Morgan_law_1_1}
		\end{align}
		が成り立つ(推論法則\ref{logicalthm:introduction_of_contraposition}).
		同様に
		\begin{align}
			\vdash\ \negation B \rarrow\ \negation (\, A \wedge B\, )
			\label{fom:strong_De_Morgan_law_1_2}
		\end{align}
		も得られる.また論理和の除去規則より
		\begin{align}
			\vdash (\, \negation A \rarrow\ \negation (\, A \wedge B\, )\, )
			\rarrow (\, (\, \negation B \rarrow\ \negation (\, A \wedge B\, )\, )
			\rarrow (\, \negation A \vee \negation B 
			\rarrow\ \negation (\, A \wedge B\, )\, )\, )
		\end{align}
		が成り立つので,(\refeq{fom:strong_De_Morgan_law_1_1})との三段論法より
		\begin{align}
			\vdash (\, \negation B \rarrow\ \negation (\, A \wedge B\, )\, )
			\rarrow (\, \negation A \vee \negation B 
			\rarrow\ \negation (\, A \wedge B\, )\, )
		\end{align}
		が従い,(\refeq{fom:strong_De_Morgan_law_1_2})との三段論法より
		\begin{align}
			\vdash\ \negation A \vee \negation B 
			\rarrow\ \negation (\, A \wedge B\, )
		\end{align}
		が得られる.
		\QED
	\end{prf}
	
	\begin{screen}
		\begin{logicalrule}[論理和の導入]
		\label{logicalrule:introduction_of_disjunction}
			$A$と$B$を文とするとき
			\begin{align}
				A &\vdash A \vee B, \\
				B &\vdash A \vee B.
			\end{align}
		\end{logicalrule}
	\end{screen}
	
	\begin{screen}
		\begin{logicalthm}[論理和の可換律]
		\label{logicalthm:commutative_law_of_disjunction}
			$A,B$を文とするとき
			\begin{align}
				\vdash A \vee B \rarrow B \vee A.
			\end{align}
		\end{logicalthm}
	\end{screen}
	
	\begin{prf}
		論理和の導入規則と演繹規則により
		\begin{align}
			\vdash A \rarrow B \vee A
			\label{fom:logicalthm_commutative_law_of_disjunction_1}
		\end{align}
		と
		\begin{align}
			\vdash B \rarrow B \vee A
			\label{fom:logicalthm_commutative_law_of_disjunction_2}
		\end{align}
		が成り立つ.また場合分け規則より
		\begin{align}
			\vdash (\, A \rarrow B \vee A\, ) \rarrow (\, (\, B \rarrow B \vee A\, ) 
			\rarrow (\, A \vee B \rarrow B \vee A\, )\, )
		\end{align}
		が成り立つので,(\refeq{fom:logicalthm_commutative_law_of_disjunction_1})と
		三段論法より
		\begin{align}
			\vdash (\, B \rarrow B \vee A\, ) 
			\rarrow (\, A \vee B \rarrow B \vee A\, )
		\end{align}
		となり,(\refeq{fom:logicalthm_commutative_law_of_disjunction_2})と
		三段論法より
		\begin{align}
			\vdash A \vee B \rarrow B \vee A
		\end{align}
		となる.
		\QED
	\end{prf}
	
	\begin{screen}
		\begin{logicalrule}[論理積の導入]
		\label{logicalrule:introduction_of_conjunction}
			$A$と$B$を文とするとき
			\begin{align}
				A,\ B \vdash A \wedge B.
			\end{align}
		\end{logicalrule}
	\end{screen}
	
	論理積の導入に演繹規則を適用すれば
	\begin{align}
		A,B &\vdash A \wedge B, \\
		A &\vdash B \rarrow A \wedge B, \\
		&\vdash A \rarrow (\, B \rarrow A \wedge B\, )
	\end{align}
	となる.これで
	\begin{align}
		\vdash A \rarrow (\, B \rarrow A \wedge B\, )
	\end{align}
	なる推論法則が得られた.
	
	\begin{screen}
		\begin{logicalthm}[弱 De Morgan の法則(2)]
		\label{logicalthm:weak_De_Morgan_law_2}
			$A$と$B$を文とするとき
			\begin{align}
				\vdash\ \negation (\, A \vee B\, ) 
				\rarrow\ \negation A \wedge \negation B.
			\end{align}
		\end{logicalthm}
	\end{screen}
	
	\begin{prf}
		論理和の導入規則より
		\begin{align}
			\vdash A \rarrow A \vee B
		\end{align}
		が成り立つが,対偶を取れば
		\begin{align}
			\vdash\ \negation (\, A \vee B\, ) \rarrow\ \negation A
			\label{fom:weak_De_Morgan_law_2_1}
		\end{align}
		となる(推論法則\ref{logicalthm:introduction_of_contraposition}).
		同じく論理和の導入規則より
		\begin{align}
			\vdash B \rarrow A \vee B
		\end{align}
		が成り立つので
		\begin{align}
			\vdash\ \negation (\, A \vee B\, ) \rarrow\ \negation B
			\label{fom:weak_De_Morgan_law_2_2}
		\end{align}
		も得られる.ここで(\refeq{fom:weak_De_Morgan_law_2_1})と
		(\refeq{fom:weak_De_Morgan_law_2_2})と演繹法則の逆より
		\begin{align}
			\negation (\, A \vee B\, ) &\vdash\ \negation A, 
			\label{fom:weak_De_Morgan_law_2_3} \\
			\negation (\, A \vee B\, ) &\vdash\ \negation B
			\label{fom:weak_De_Morgan_law_2_4}
		\end{align}
		が従う.ところで論理積の導入規則より
		\begin{align}
			\vdash\ \negation A \rarrow (\, \negation B \rarrow\
			\negation A \wedge \negation B\, )
		\end{align}
		が成り立つので,(\refeq{fom:weak_De_Morgan_law_2_3})と
		(\refeq{fom:weak_De_Morgan_law_2_4})との三段論法より
		\begin{align}
			\negation (\, A \vee B\, ) &\vdash\ \negation A \rarrow 
				(\, \negation B \rarrow\ \negation A \wedge \negation B\, ), \\
			\negation (\, A \vee B\, ) &\vdash\ 
				\negation B \rarrow\ \negation A \wedge \negation B, \\
			\negation (\, A \vee B\, ) &\vdash\ \negation A \wedge \negation B
		\end{align}
		が従い,演繹規則より
		\begin{align}
			\vdash\ \negation (\, A \vee B\, ) 
			\rarrow\ \negation A \wedge \negation B
		\end{align}
		が得られる.
		\QED
	\end{prf}
	
	\begin{screen}
		\begin{logicalthm}[論理積の可換律]
		\label{logicalthm:commutative_law_of_conjunction}
			$A,B$を文とするとき
			\begin{align}
				\vdash A \wedge B \rarrow B \wedge A.
			\end{align}
		\end{logicalthm}
	\end{screen}
	
	\begin{prf}
		論理積の除去と演繹規則より
		\begin{align}
			A \wedge B \vdash A
			\label{fom:logicalthm_commutative_law_of_conjunction_1}
		\end{align}
		と
		\begin{align}
			A \wedge B \vdash B
			\label{fom:logicalthm_commutative_law_of_conjunction_2}
		\end{align}
		が成り立つ.また論理積の導入により
		\begin{align}
			\vdash B \rarrow (\, A \rarrow B \wedge A\, )
		\end{align}
		となるので
		\begin{align}
			A \wedge B \vdash B \rarrow (\, A \rarrow B \wedge A\, )
		\end{align}
		も成り立ち,(\refeq{fom:logicalthm_commutative_law_of_conjunction_2})との三段論法より
		\begin{align}
			A \wedge B \vdash A \rarrow B \wedge A
		\end{align}
		となり,(\refeq{fom:logicalthm_commutative_law_of_conjunction_1})との三段論法より
		\begin{align}
			A \wedge B \vdash B \wedge A
		\end{align}
		となり,演繹規則より
		\begin{align}
			\vdash A \wedge B \rarrow B \wedge A
		\end{align}
		が得られる.
		\QED
	\end{prf}
	\begin{screen}
		\begin{logicalaxm}[量化記号に関する規則]\label{logicalaxm:rules_of_quantifiers}
			$A$を$\mathcal{L}$の式とし,$x$を$A$に自由に現れる変項とし,
			$A$に自由に現れる項が$x$のみであるとする.
			また$\tau$を任意の$\varepsilon$項とする.このとき以下を推論規則とする.
			\begin{align}
				A(\tau) &\vdash \exists x A(x), \\
				\exists x A(x) &\vdash A(\varepsilon x A(x)), \\
				\forall x A(x) &\vdash A(\tau), \\
				A(\varepsilon x \negation A(x)) &\vdash \forall x A(x).
			\end{align}
		\end{logicalaxm}
	\end{screen}
	
	どれでも一つ,$A$を成り立たせるような$\varepsilon$項$\tau$が取れれば
	$\exists x A(x)$が成り立つのだし,逆に$\exists x A(x)$が成り立つならば
	$\varepsilon x A(x)$なる$\epsilon$項が$A$を満たすのであるから,
	$\exists x A(x)$が成り立つということと$A$を満たす$\varepsilon$項が取れるということは
	同じ意味になる.
	
	$\forall x A(x)$が成り立つならばいかなる$\varepsilon$項も$A$を満たすし,
	逆にいかなる$\varepsilon$項も$A$を満たすならば,特に$\varepsilon x \negation A(x)$
	なる$\varepsilon$項も$A$を満たすのだから,$\forall x A(x)$が成立する.
	つまり,$\forall x A(x)$が成り立つということと,任意の$\varepsilon$項が$A$を満たすということは
	同じ意味になる.
	
	後述することであるが,$\varepsilon$項はどれも集合であって,また集合である類は
	いずれかの$\varepsilon$項と等しい.ゆえに,量化子の亘る範囲は集合に制限されるのである.
	
	\begin{screen}
		\begin{logicalthm}[量化記号の性質(イ)]\label{logicalthm:properties_of_quantifiers}
			$A,B$を$\mathcal{L}'$の式とし,$x$を$A,B$に現れる文字とし,$x$のみが$A,B$で量化されていないとする.
			$\mathcal{L}$の任意の対象$\tau$に対して
			\begin{align}
				A(\tau) \lrarrow B(\tau)
			\end{align}
			が成り立っているとき,
			\begin{align}
				\exists x A(x) \lrarrow \exists x B(x)
			\end{align}
			および
			\begin{align}
				\forall x A(x) \lrarrow \forall x B(x)
			\end{align}
			が成り立つ.
		\end{logicalthm}
	\end{screen}
	
	\begin{prf}
		いま,$\mathcal{L}$の任意の対象$\tau$に対して
		\begin{align}
			A(\tau) \lrarrow B(\tau)
			\label{logicalthm:properties_of_quantifiers_1}
		\end{align}
		が成り立っているとする.
		ここで
		\begin{align}
			\exists x A(x)
		\end{align}
		が成り立っていると仮定すると,
		\begin{align}
			\tau \defeq \varepsilon x A(x)
		\end{align}
		とおけば存在記号に関する規則より
		\begin{align}
			A(\tau)
		\end{align}
		が成立し,(\refeq{logicalthm:properties_of_quantifiers_1})と併せて
		\begin{align}
			B(\tau)
		\end{align}
		が成立する.再び存在記号に関する規則より
		\begin{align}
			\exists x B(x)
		\end{align}
		が成り立つので,演繹法則から
		\begin{align}
			\exists x A(x) \rarrow \exists x B(x)
		\end{align}
		が得られる.$A$と$B$の立場を入れ替えれば
		\begin{align}
			\exists x B(x) \rarrow \exists x A(x)
		\end{align}
		も得られる.今度は
		\begin{align}
			\forall x A(x)
		\end{align}
		が成り立っていると仮定すると,
		推論法則\ref{logicalthm:fundamental_law_of_universal_quantifier}より
		$\mathcal{L}$の任意の対象$\tau$に対して
		\begin{align}
			A(\tau)
		\end{align}
		が成立し,(\refeq{logicalthm:properties_of_quantifiers_1})と併せて
		\begin{align}
			B(\tau)
		\end{align}
		が成立する.$\tau$の任意性と推論法則\ref{logicalthm:fundamental_law_of_universal_quantifier}より
		\begin{align}
			\forall x B(x)
		\end{align}
		が成り立つので,演繹法則から
		\begin{align}
			\forall x A(x) \rarrow \forall x B(x)
		\end{align}
		が得られる.$A$と$B$の立場を入れ替えれば
		\begin{align}
			\forall x B(x) \rarrow \forall x A(x)
		\end{align}
		も得られる.
		\QED
	\end{prf}
	
	\begin{screen}
		\begin{logicalthm}[量化記号に対する De Morgan の法則]\label{logicalthm:De_Morgan_law_for_quantifiers}
			$A$を$\mathcal{L}'$の式とし,$x$を$A$に現れる文字とし,$x$のみが$A$で量化されていないとする.このとき
			\begin{align}
				\exists x \negation A(x) \lrarrow \negation \forall x A(x)
			\end{align}
			および
			\begin{align}
				\forall x \negation A(x) \lrarrow \negation \exists x A(x)
			\end{align}
			が成り立つ.
		\end{logicalthm}
	\end{screen}
	
	\begin{sketch}
		推論規則\ref{logicalaxm:rules_of_quantifiers}より
		\begin{align}
			\exists x \negation A(x) \lrarrow 
			\negation A(\varepsilon x \negation A(x))
		\end{align}
		は定理である.他方で推論規則\ref{logicalaxm:rules_of_quantifiers}より
		\begin{align}
			A(\varepsilon x \negation A(x)) \lrarrow \forall x A(x) 
		\end{align}
		もまた定理であり,この対偶を取れば
		\begin{align}
			\negation A(\varepsilon x \negation A(x)) \lrarrow 
			\negation \forall x A(x)
		\end{align}
		が成り立つ.ゆえに
		\begin{align}
			\exists x \negation A(x) \lrarrow \negation \forall x A(x)
		\end{align}
		が従う.$A$を$\negation A$に置き換えれば
		\begin{align}
			\forall x \negation A(x) \lrarrow 
			\negation \exists x \negation \negation A(x)
		\end{align}
		が成り立ち,また$\mathcal{L}$の任意の対象$\tau$に対して
		\begin{align}
			A(\tau) \lrarrow \negation \negation A(\tau)
		\end{align}
		が成り立つので,推論法則\ref{logicalthm:properties_of_quantifiers}より
		\begin{align}
			\exists x \negation \negation A(x)
			\lrarrow \exists x A(x)
		\end{align}
		も成り立つ.ゆえに
		\begin{align}
			\forall x \negation A(x) \lrarrow 
			\negation \exists x A(x)
		\end{align}
		が従う.
		\QED
	\end{sketch}

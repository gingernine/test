	\begin{screen}
		\begin{logicalrule}[量化記号に関する規則]
		\label{logicalrule:rules_of_quantifiers}
			$A$を$\mathcal{L}$の式とし,$x$を$A$に自由に現れる変項とし,
			$A$に自由に現れる項が$x$のみであるとする.
			また$\tau$を主要$\varepsilon$項とする.このとき以下を推論規則とする.
			\begin{align}
				A(\tau) &\vdash \exists x A(x), \\
				\exists x A(x) &\vdash A(\varepsilon x \hat{A}(x)), \\
				\forall x A(x) &\vdash A(\tau), \\
				\negation \forall x A(x) &\vdash \exists x \negation A(x).
			\end{align}
			ただし$\hat{A}$とは必要に応じて$A$を$\lang{\varepsilon}$の式に書き直したものである.
		\end{logicalrule}
	\end{screen}
	
	存在記号の推論規則より
	\begin{align}
		\vdash \exists x \negation A \rarrow\ \negation A(\varepsilon x \widehat{\negation A})
	\end{align}
	が成り立つが,ここで$\widehat{\negation A}$と$\negation \hat{A}$は同一の記号列なので
	(もとより$A$が$\lang{\varepsilon}$の式ならばどちらも$\negation A$である)
	\begin{align}
		\vdash \exists x \negation A \rarrow\ \negation A(\varepsilon x \negation \hat{A})
	\end{align}
	となり,量化の四番目の推論規則との三段論法で
	\begin{align}
		\negation \forall x A \vdash\ \negation A(\varepsilon x \negation \hat{A})
	\end{align}
	が従う.そして対偶論法(推論法則\ref{logicalthm:proof_by_contraposition})より
	\begin{align}
		\vdash A(\varepsilon x \negation \hat{A}) \rarrow \forall x A
	\end{align}
	が得られる.これは非常に有用な結果であるから一つの定理として述べておく.
	
	\begin{screen}
		\begin{logicalthm}[$\varepsilon$項による全称の導出]
		\label{logicalthm:derivation_of_universal_by_epsilon}
			$A$を$\mathcal{L}$の式とし,$x$を$A$に自由に現れる変項とし,
			$A$に自由に現れる項が$x$のみであるとする.このとき
			\begin{align}
				\vdash A(\varepsilon x \negation \hat{A}(x)) \rarrow \forall x A(x).
			\end{align}
			ただし$\hat{A}$とは必要に応じて$A$を$\lang{\varepsilon}$の式に書き直したものである.
		\end{logicalthm}
	\end{screen}
	
	どれでも一つ,$A(\tau)$を成り立たせるような主要$\varepsilon$項$\tau$が取れれば
	$\exists x A(x)$が成り立つのだし,逆に$\exists x A(x)$が成り立つならば
	$\varepsilon x A(x)$なる$\epsilon$項が$A(\varepsilon x A(x))$を満たすのである.
	そして主要$\varepsilon$項は集合であるから(定理\ref{thm:critical_epsilon_term_is_set}),
	「$A(x)$を満たす集合$x$が存在する」ということと
	「$A(x)$を満たす集合$x$が{\bf ``実際に取れる''}」ということが同じ意味になる.
	
	$\forall x A(x)$が成り立つならばいかなる主要$\varepsilon$項$\tau$も$A(\tau)$を満たすし,
	逆にいかなる主要$\varepsilon$項$\tau$も$A(\tau)$を満たすならば,
	特に$\varepsilon x \negation A(x)$なる$\varepsilon$項も
	$A(\varepsilon x \negation A(x))$を満たすのだから$\forall x A(x)$が成立する.
	つまり,「$\forall x A(x)$が成り立つ」ということと
	「任意の主要$\varepsilon$項$\tau$が$A(\tau)$を満たす」ということは同じ意味になる.
	
	後述することであるが,主要$\varepsilon$項はどれも集合であって
	(定理\ref{thm:critical_epsilon_term_is_set}),また集合である類は
	いずれかの主要$\varepsilon$項と等しい
	(定理\ref{thm:if_a_class_is_a_set_then_equal_to_some_epsilon_term}).
	ゆえに,{\bf 量化子の亘る範囲は集合に制限される}のである.
	
	量化記号についても De Morgan の法則があり,それを
	\begin{description}
		\item[弱 De Morgan の法則] $\exists x \negation A(x) \lrarrow\ \negation \forall x A(x)$,
		\item[強 De Morgan の法則] $\forall x \negation A(x) \lrarrow\ \negation \exists x A(x)$,
	\end{description}
	と呼ぶことにする.
	
	\begin{screen}
		\begin{logicalthm}[量化記号に対する弱 De Morgan の法則(1)]
		\label{logicalthm:weak_De_Morgan_law_for_quantifiers_1}
			$A$を$\mathcal{L}$の式とし,$x$を$A$に自由に現れる変項とし,
			また$A$に自由に現れる変項は$x$のみであるとする.このとき
			\begin{align}
				\vdash \exists x \negation A(x) \rarrow\ \negation \forall x A(x).
			\end{align}
		\end{logicalthm}
	\end{screen}
	
	\begin{sketch}
		必要に応じて$A$を$\lang{\varepsilon}$の式に書き換えたものを$\hat{A}$とする.
		存在記号の推論規則より
		\begin{align}
			\exists x \negation A(x) \vdash\ \negation A(\varepsilon x \negation \hat{A}(x))
			\label{fom:weak_De_Morgan_law_for_quantifiers_1_1}
		\end{align}
		となる.また全称記号の推論規則より
		\begin{align}
			\vdash \forall x A(x) \rarrow A(\varepsilon x \negation \hat{A}(x))
		\end{align}
		が成り立つので,対偶を取って
		\begin{align}
			\vdash\ \negation A(\varepsilon x \negation \hat{A}(x)) \rarrow\ \negation \forall x A(x)
			\label{fom:weak_De_Morgan_law_for_quantifiers_1_2}
		\end{align}
		となる(推論法則\ref{logicalthm:introduction_of_contraposition}).
		(\refeq{fom:weak_De_Morgan_law_for_quantifiers_1_1})と
		(\refeq{fom:weak_De_Morgan_law_for_quantifiers_1_2})の三段論法より
		\begin{align}
			\exists x \negation A(x) \vdash\ \negation \forall x A(x)
		\end{align}
		が従い,演繹規則より
		\begin{align}
			\vdash \exists x \negation A(x) \rarrow\ \negation \forall x A(x)
		\end{align}
		が得られる.
		\QED
	\end{sketch}
	
	\begin{screen}
		\begin{logicalthm}[量化記号に対する弱 De Morgan の法則(2)]
		\label{logicalthm:weak_De_Morgan_law_for_quantifiers_2}
			$A$を$\mathcal{L}$の式とし,$x$を$A$に自由に現れる変項とし,
			また$A$に自由に現れる変項は$x$のみであるとする.このとき
			\begin{align}
				\vdash\ \negation \forall x A(x) \rarrow \exists x \negation A(x).
			\end{align}
		\end{logicalthm}
	\end{screen}
	
	\begin{sketch}
		推論規則
		\begin{align}
			\negation \forall x A(x) \vdash \exists x \negation A(x)
		\end{align}
		に演繹規則を適用して得られる.
		\QED
	\end{sketch}
	
	\begin{comment}
		必要に応じて$A$を$\lang{\varepsilon}$の式に書き換えたものを$\hat{A}$とする.
		存在記号の推論規則より
		\begin{align}
			\vdash\ \negation A(\varepsilon x \negation \hat{A}(x)) \rarrow \exists x \negation A(x)
		\end{align}
		が成り立つので,対偶を取って
		\begin{align}
			\vdash\ \negation \exists x \negation A(x) \rarrow\ \negation\negation A(\varepsilon x \negation \hat{A}(x))
		\end{align}
		が従い(推論法則\ref{logicalthm:introduction_of_contraposition}),演繹法則の逆より
		\begin{align}
			\negation \exists x \negation A(x) \vdash\ \negation\negation A(\varepsilon x \negation \hat{A}(x))
		\end{align}
		となる.そして二重否定の除去規則より
		\begin{align}
			\negation \exists x \negation A(x) \vdash A(\varepsilon x \negation \hat{A}(x))
		\end{align}
		が成立し,全称の導出(推論法則\ref{derivation_of_universal_by_epsilon})より
		\begin{align}
			\negation \exists x \negation A(x) \vdash \forall x A(x)
		\end{align}
		が従う.演繹規則より
		\begin{align}
			\vdash\ \negation \exists x \negation A(x) \rarrow \forall x A(x)
		\end{align}
		となり,対偶を取れば
		\begin{align}
			\vdash\ \negation \forall x A(x) \rarrow\ \negation\negation \exists x \negation A(x)
		\end{align}
		となるが,先と同様に二重否定の除去によって
		\begin{align}
			\vdash\ \negation \forall x A(x) \rarrow \exists x \negation A(x)
		\end{align}
		が得られる.
		\QED
	\end{comment}
	
	\begin{screen}
		\begin{logicalthm}[量化記号に対する強 De Morgan の法則(1)]
		\label{logicalthm:strong_De_Morgan_law_for_quantifiers_1}
			$A$を$\mathcal{L}$の式とし,$x$を$A$に自由に現れる変項とし,
			また$A$に自由に現れる変項は$x$のみであるとする.このとき
			\begin{align}
				\vdash \forall x \negation A(x) \rarrow\ \negation \exists x A(x).
			\end{align}
		\end{logicalthm}
	\end{screen}
	
	\begin{sketch}
		必要に応じて$A$を$\lang{\varepsilon}$の式に書き換えたものを$\hat{A}$とする.
		まず存在記号の推論規則より
		\begin{align}
			\vdash \exists x A(x) \rarrow A(\varepsilon x \hat{A}(x))
		\end{align}
		が成り立つので,対偶を取って
		\begin{align}
			\vdash\ \negation A(\varepsilon x \hat{A}(x)) 
			\rarrow\ \negation \exists x A(x)
		\end{align}
		が成り立つ(推論法則\ref{logicalthm:introduction_of_contraposition}).また
		全称記号の推論規則より
		\begin{align}
			\forall x \negation A(x) \vdash\ \negation A(\varepsilon x \hat{A}(x))
		\end{align}
		が成り立つので,三段論法より
		\begin{align}
			\forall x \negation A(x) \vdash\ \negation \exists x A(x)
		\end{align}
		が従い,演繹規則より
		\begin{align}
			\vdash \forall x \negation A(x) \rarrow\ \negation \exists x A(x)
		\end{align}
		が得られる.
		\QED
	\end{sketch}
	
	\begin{screen}
		\begin{logicalthm}[量化記号に対する強 De Morgan の法則(2)]
		\label{logicalthm:strong_De_Morgan_law_for_quantifiers_2}
			$A$を$\mathcal{L}$の式とし,$x$を$A$に自由に現れる変項とし,
			また$A$に自由に現れる変項は$x$のみであるとする.このとき
			\begin{align}
				\vdash\ \negation \exists x A(x) \rarrow \forall x \negation A(x).
			\end{align}
		\end{logicalthm}
	\end{screen}
	
	\begin{sketch}
		必要に応じて$A$を$\lang{\varepsilon}$の式に書き換えたものを$\hat{A}$とする.
		まず存在記号の推論規則より
		\begin{align}
			\vdash A(\varepsilon x \negation \negation \hat{A}(x))
			\rarrow \exists x A(x)
		\end{align}
		が成り立つので,対偶を取って
		\begin{align}
			\vdash\ \negation \exists x A(x) 
			\rarrow\ \negation A(\varepsilon x \negation \negation \hat{A}(x))
		\end{align}
		が成り立ち(推論法則\ref{logicalthm:introduction_of_contraposition}),
		演繹法則の逆より
		\begin{align}
			\negation \exists x A(x) \vdash\ \negation A(\varepsilon x \negation \negation \hat{A}(x))
		\end{align}
		が従う.また全称の導出(推論法則\ref{logicalthm:derivation_of_universal_by_epsilon})より
		\begin{align}
			\vdash \negation A(\varepsilon x \negation \negation \hat{A}(x))
			\rarrow \forall x \negation A(x)
		\end{align}
		が成り立つので,三段論法より
		\begin{align}
			\negation \exists x A(x) \vdash \forall x \negation A(x)
		\end{align}
		が従い,演繹規則より
		\begin{align}
			\vdash\ \negation \exists x A(x) \rarrow \forall x \negation A(x)
		\end{align}
		が得られる.
		\QED
	\end{sketch}
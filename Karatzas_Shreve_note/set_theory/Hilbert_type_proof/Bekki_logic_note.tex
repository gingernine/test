	\begin{flushleft}
		参考文献: 戸次大介「数理論理学」
	\end{flushleft}
	
	この章に出てくる式と項は言語$\mathcal{L}_{\in}$のものとする.
	証明論の奇妙なところは,扱う式が文とは限らないところである.
	式に自由な変項が残ったままであるとその式の意味は定まらない.
	逆に変更が全て束縛されている文は,それが表す意味は非常に判然としている.
	奇妙なのは,$\varphi$が文であって,これが証明されたとしても,その証明の過程には
	文でない式が出現し得る点である.意味が不明瞭な式をもって意味がはっきり
	定まった式を導こうというところが腑に落ちない.と今まで思っていたが,
	どうも勉強しているうちにそれほど不自然には感じなくなってきた.
	
	%たとえば,$\psi$と$\psi \rightarrow \varphi$が証明されれば通常は三段論法から
	%$\varphi$が導かれるわけだが,$\psi$も$\varphi$も文であるとは限らない.

\section{{\bf HK}}
	証明体系には様々な流派があるが,流派の一つHilbert流と呼ばれる証明体系のうちで最も標準的なものが
	{\bf HK}であると,と理解している.
	証明のシステムを概括するために,抽象的に{\bf H}をHilbert流の証明体系とする.
	はじめに,{\bf H}の(論理的)公理と推論規則と言われるものが与えられる.
	また証明の前には公理系と呼ばれる式の集合も与えられる.公理系に属する式をその公理系の公理と呼ぶが,
	公理は意味のはっきりした式であるべきだと思うので{\bf 全て文とする}.
	いま公理系を$\Gamma$とすれば,$\varphi$が$\Gamma$の公理であるか{\bf H}の公理であることを
	\begin{align}
		\Gamma \provable{\mbox{{\bf H}}} \varphi
	\end{align}
	と書く.通常は公理が全くない場合も考察対象であり,その場合は$\varphi$が{\bf H}の公理であることを
	\begin{align}
		\provable{\mbox{{\bf H}}} \varphi
	\end{align}
	と書くのである.$\varphi$が一般の式である場合は,
	$\Gamma \provable{\mbox{{\bf H}}} \varphi$
	なることを「$\varphi$は$\Gamma$の下での定理である」といった趣旨の言い方をする.
	つまり$\Gamma$の公理や{\bf H}の公理は$\Gamma$の下での定理であるわけであるが,
	他の式については,それがすでに既に定理とされた式から{\bf H}の推論規則によって得られている
	ときに限り定理となる.
	
	まずは一番弱い体系の{\bf SK}から始める.以下で$\Gamma$と書いたらそれは公理系を表す.
	
\subsection{{\bf SK}}
	\begin{itembox}[l]{{\bf SK}の公理}
		$\varphi$と$\psi$と$\xi$を式とするとき,次は{\bf SK}の公理である.
		\begin{description}
			\item[(S)] $(\varphi \rightarrow (\psi \rightarrow \chi)) 
				\rightarrow ((\varphi \rightarrow \psi)
				\rightarrow (\varphi \rightarrow \chi)).$
			
			\item[(K)] $\varphi \rightarrow (\psi \rightarrow \varphi).$
		\end{description}
	\end{itembox}
	
	そして
	\begin{align}
		\provable{\mbox{{\bf SK}}} (\varphi \rightarrow (\psi \rightarrow \chi)) 
			\rightarrow ((\varphi \rightarrow \psi)
			\rightarrow (\varphi \rightarrow \chi))
	\end{align}
	および
	\begin{align}
		\provable{\mbox{{\bf SK}}} \varphi \rightarrow (\psi \rightarrow \varphi)
	\end{align}
	と書く.
	
	\begin{itembox}[l]{{\bf SK}の推論規則}
		$\varphi$と$\psi$を式とするとき,次は{\bf SK}の推論規則である.
		\begin{description}
			\item[三段論法] $\Gamma \provable{\mbox{{\bf SK}}} \psi$かつ
				$\Gamma \provable{\mbox{{\bf SK}}} \psi \rightarrow \varphi$ならば
				$\Gamma \provable{\mbox{{\bf SK}}} \varphi$である.
		\end{description}
	\end{itembox}
	
	{\bf SK}から証明可能な式
	\begin{description}
		\item[(I)] $\varphi \rightarrow \varphi$
		\item[(B)] $(\psi \rightarrow \chi) \rightarrow ((\varphi \rightarrow \psi) \rightarrow (\varphi \rightarrow \chi)).$
		\item[(C)] $(\varphi \rightarrow (\psi \rightarrow \chi)) \rightarrow (\psi \rightarrow (\varphi \rightarrow \chi)).$
		\item[(W)] $(\varphi \rightarrow (\varphi \rightarrow \psi)) \rightarrow (\varphi \rightarrow \psi).$
		\item[(B')] $(\varphi \rightarrow \psi) \rightarrow ((\psi \rightarrow \chi) \rightarrow (\varphi \rightarrow \chi)).$
		\item[(C$\ast$)] $\varphi \rightarrow ((\varphi \rightarrow \psi) \rightarrow \psi)$
	\end{description}
	
\subsection{否定}
	{\bf SK}の公理に否定の公理を追加し,推論規則はそのまま据え置いた証明体系を{\bf SK'}とする.
	
	\begin{itembox}[l]{{\bf SK'}で追加された公理}
		$\varphi$と$\psi$と$\xi$を式とするとき,{\bf SK'}の公理は(S)(K)に以下の式を加えたものである.
		\begin{description}
			\item[(CTI1)] $\varphi \rightarrow (\rightharpoondown \varphi \rightarrow \bot).$
			
			\item[(CTI2)] $\rightharpoondown \varphi \rightarrow (\varphi \rightarrow \bot).$
			
			\item[(NI)] $(\varphi \rightarrow \bot) \rightarrow\ \rightharpoondown \varphi.$
		\end{description}
	\end{itembox}
	
	このとき証明可能な式
	\begin{description}
		\item[(DNI)] $\varphi \rightarrow\ \rightharpoondown \rightharpoondown \varphi.$
		\item[(CON1)] $(\varphi \rightarrow \psi) \rightarrow (\rightharpoondown \psi \rightarrow\ \rightharpoondown \varphi).$
		\item[(CON2)] $(\varphi \rightarrow\ \rightharpoondown \psi) \rightarrow (\psi \rightarrow\ \rightharpoondown \varphi).$
	\end{description}
	
\subsection{{\bf HM}}
	{\bf HM}とは最小論理と呼ばれる証明体系である.{\bf HK}においては$\bot$が示されると
	あらゆる式が導かれることになるが(爆発律),{\bf HM}ではそれが起こらないので矛盾許容論理と言われる.
	また背理法が成立しないので「$\negation A$と仮定すると矛盾するので$A$」という論法は使えず,
	証明は矛盾に頼らないという意味で``構成的''になる.
	
	\begin{align}
		\varphi(t/x)
	\end{align}
	とは,式$\varphi$に{\bf 自由に}現れる変項$x$を項$t$で置き換えた式を表す.
	ただし$t$は$\varphi$の中で$x$への代入について自由である.
	
	\begin{itembox}[l]{{\bf HM}の公理}
		$\varphi$と$\psi$と$\xi$を式とし,$x$と$t$を変項とし,$\tau$を項とするとき,
		次は{\bf HM}の公理である.
		\begin{description}
			\item[(S)] $(\varphi \rightarrow (\psi \rightarrow \chi)) 
				\rightarrow ((\varphi \rightarrow \psi)
				\rightarrow (\varphi \rightarrow \chi)).$
			\item[(K)] $\varphi \rightarrow (\psi \rightarrow \varphi).$
			\item[(DI1)] $\varphi \rightarrow (\varphi \vee \psi).$
			\item[(DI2)] $\psi \rightarrow (\varphi \vee \psi).$
			\item[(DE)] $(\varphi \rightarrow \chi) \rightarrow 
				((\psi \rightarrow \chi) \rightarrow ((\varphi \vee \psi) \rightarrow \chi)).$
			\item[(CI)] $\varphi \rightarrow (\psi \rightarrow (\varphi \wedge \psi)).$
			\item[(CE1)] $(\varphi \wedge \psi) \rightarrow \varphi.$
			\item[(CE2)] $(\varphi \wedge \psi) \rightarrow \psi.$
			\item[(UI)] $\forall t (\psi \rightarrow \varphi(t/x)) 
				\rightarrow (\psi \rightarrow \forall x \varphi).$
			\item[(UE)] $\forall x \varphi \rightarrow \varphi(\tau/x).$
			\item[(EI)] $\varphi(\tau/x) \rightarrow \exists x \varphi.$
			\item[(EE)] $\forall t (\varphi(t/x) \rightarrow \psi)
				\rightarrow (\exists x \varphi \rightarrow \psi).$
		\end{description}
	\end{itembox}
	
	{\bf SK}と同様に,上の{\bf HM}の公理は全て$\provable{\mbox{{\bf HM}}} ...$と書かれる.
	
	\begin{itembox}[l]{{\bf HM}の推論規則}
		$\varphi$と$\psi$を式とするとき,次は{\bf HM}の推論規則である.
		\begin{description}
			\item[三段論法] $\Gamma \provable{\mbox{{\bf HM}}} \psi$かつ
				$\Gamma \provable{\mbox{{\bf HM}}} \psi \rightarrow \varphi$ならば
				$\Gamma \provable{\mbox{{\bf HM}}} \varphi$である.
			\item[汎化] $\psi$に変項$x$が自由に現れているとき,
				$\Gamma \provable{\mbox{{\bf HM}}} \psi(y/x)$ならば
				$\Gamma \provable{\mbox{{\bf HM}}} \forall x \psi$である.
				ただし$y$は$\forall x \psi$に自由には現れない変項とする.
		\end{description}
	\end{itembox}
	
	{\bf HM}から証明可能な式
	\begin{description}
		\item[(LNC)] $\rightharpoondown (\varphi \wedge \rightharpoondown \varphi).$
		\item[(DIST$\wedge$)] $\varphi \vee (\psi \wedge \chi) 
			\leftrightarrow (\varphi \vee \psi) \wedge (\varphi \vee \chi).$
		\item[(DIST$\vee$)] $\varphi \wedge (\psi \vee \chi) 
			\leftrightarrow (\varphi \wedge \psi) \vee (\varphi \wedge \chi).$
		\item[(DM$\vee$)] $\rightharpoondown (\varphi \vee \psi) \leftrightarrow
			\ \rightharpoondown \varphi \wedge \rightharpoondown \psi.$
	\end{description}
	
	\begin{sketch}[LNC]
		\begin{align}
			\varphi \wedge \rightharpoondown \varphi &\provable{\mbox{{\bf HM}}} \varphi, \\
			\varphi \wedge \rightharpoondown \varphi &\provable{\mbox{{\bf HM}}}\ \rightharpoondown \varphi, \\
			\varphi \wedge \rightharpoondown \varphi &\provable{\mbox{{\bf HM}}}
				\varphi \rightarrow (\rightharpoondown \varphi \rightarrow \bot), \\
			\varphi \wedge \rightharpoondown \varphi &\provable{\mbox{{\bf HM}}}\ \rightharpoondown \varphi \rightarrow \bot, \\
			\varphi \wedge \rightharpoondown \varphi &\provable{\mbox{{\bf HM}}} \bot, \\
			&\provable{\mbox{{\bf HM}}} (\varphi \wedge \rightharpoondown \varphi) \rightarrow \bot, \\
			&\provable{\mbox{{\bf HM}}} ((\varphi \wedge \rightharpoondown \varphi) \rightarrow \bot)
				\rightarrow\ \rightharpoondown (\varphi \wedge \rightharpoondown \varphi), \\
			&\provable{\mbox{{\bf HM}}}\ \rightharpoondown (\varphi \wedge \rightharpoondown \varphi).
		\end{align}
		\QED
	\end{sketch}
	
	\begin{sketch}[DM$\vee$]
		\begin{align}
			&\provable{\mbox{{\bf HM}}} \varphi \rightarrow (\varphi \vee \psi), && \mbox{(DI1)}\\
			&\provable{\mbox{{\bf HM}}} (\varphi \rightarrow (\varphi \vee \psi))
				\rightarrow (\rightharpoondown (\varphi \vee \psi) \rightarrow\ \rightharpoondown \varphi), 
				&& \mbox{(CON1)}\\
			&\provable{\mbox{{\bf HM}}}\ \rightharpoondown (\varphi \vee \psi) \rightarrow\ \rightharpoondown \varphi, 
				&& \mbox{(MP)}\\
			\rightharpoondown (\varphi \vee \psi) &\provable{\mbox{{\bf HM}}}\ \rightharpoondown \varphi.
				&& \mbox{(DR)}
		\end{align}
		同様に
		\begin{align}
			\rightharpoondown (\varphi \vee \psi) \provable{\mbox{{\bf HM}}}\ \rightharpoondown \psi
		\end{align}
		となり,
		\begin{align}
			\rightharpoondown (\varphi \vee \psi) &\provable{\mbox{{\bf HM}}}\ \rightharpoondown \varphi
				\rightarrow (\rightharpoondown \psi \rightarrow 
				(\rightharpoondown \varphi \wedge \rightharpoondown \psi)), && \mbox{(CI)}\\
			\rightharpoondown (\varphi \vee \psi) &\provable{\mbox{{\bf HM}}}\ 
				\rightharpoondown \psi \rightarrow (\rightharpoondown \varphi \wedge \rightharpoondown \psi), 
				&& \mbox{(MP)}\\
			\rightharpoondown (\varphi \vee \psi) &\provable{\mbox{{\bf HM}}}\ 
				\rightharpoondown \varphi \wedge \rightharpoondown \psi && \mbox{(MP)}
		\end{align}
		が得られる.逆に
		\begin{align}
			\rightharpoondown \varphi \wedge \rightharpoondown \psi &\provable{\mbox{{\bf HM}}}\ \rightharpoondown \varphi, 
				&& \mbox{(CE1)}\\
			\rightharpoondown \varphi \wedge \rightharpoondown \psi &\provable{\mbox{{\bf HM}}}\ 
			\rightharpoondown \varphi \rightarrow (\varphi \rightarrow \bot), && \mbox{(CTI2)}\\
			\rightharpoondown \varphi \wedge \rightharpoondown \psi &\provable{\mbox{{\bf HM}}} \varphi \rightarrow \bot
				&& \mbox{(MP)}
		\end{align}
		となり,同様に
		\begin{align}
			\rightharpoondown \varphi \wedge \rightharpoondown \psi \provable{\mbox{{\bf HM}}} \psi \rightarrow \bot
		\end{align}
		も成り立つ.よって
		\begin{align}
			\rightharpoondown \varphi \wedge \rightharpoondown \psi &\provable{\mbox{{\bf HM}}} 
				(\varphi \rightarrow \bot) \rightarrow ((\psi \rightarrow \bot) 
				\rightarrow ((\varphi \vee \psi) \rightarrow \bot)), && \mbox{(DE)}\\
			\rightharpoondown \varphi \wedge \rightharpoondown \psi &\provable{\mbox{{\bf HM}}} 
				(\psi \rightarrow \bot) \rightarrow ((\varphi \vee \psi) \rightarrow \bot), && \mbox{(MP)}\\
			\rightharpoondown \varphi \wedge \rightharpoondown \psi &\provable{\mbox{{\bf HM}}} 
				(\varphi \vee \psi) \rightarrow \bot, && \mbox{(MP)}\\
			\rightharpoondown \varphi \wedge \rightharpoondown \psi &\provable{\mbox{{\bf HM}}} 
				((\varphi \vee \psi) \rightarrow \bot) \rightarrow\ \rightharpoondown (\varphi \vee \psi), && \mbox{(NI)}\\
			\rightharpoondown \varphi \wedge \rightharpoondown \psi &\provable{\mbox{{\bf HM}}} 
				\ \rightharpoondown (\varphi \vee \psi) && \mbox{(MP)}
		\end{align}
		が得られる.
		\QED
	\end{sketch}
	
\subsection{{\bf HK}}
	{\bf HM}の推論規則はそのままに,公理に{\bf 二重否定除去}を追加すると
	古典論理の証明体系{\bf HK}となる.
	
	\begin{itembox}[l]{{\bf HK}の公理}
		{\bf HM}の公理に次を追加:
		\begin{description}
			\item[(DNE)] $\rightharpoondown \rightharpoondown \varphi \rightarrow \varphi.$
		\end{description}
	\end{itembox}
	
	{\bf HK}から証明可能な式
	\begin{description}
		\item[(CON3)] $(\rightharpoondown \varphi \rightarrow \psi) \rightarrow (\rightharpoondown \psi \rightarrow \varphi).$
		\item[(CON4)] $(\rightharpoondown \varphi \rightarrow\ \rightharpoondown \psi) 
			\rightarrow (\psi \rightarrow \varphi).$
		\item[(RAA)] $(\rightharpoondown \varphi \rightarrow \bot) \rightarrow \varphi.$
		\item[(EFQ)] $\bot \rightarrow \varphi.$
	\end{description}
	
	\begin{sketch}[CON3]
		\begin{align}
			\rightharpoondown \varphi \rightarrow \psi,\ \rightharpoondown \psi &\provable{\mbox{{\bf HK}}}
				\ \rightharpoondown \psi \rightarrow\ \rightharpoondown \rightharpoondown \varphi,
				&& \mbox{(CON1)} \\
			\rightharpoondown \varphi \rightarrow \psi,\ \rightharpoondown \psi &\provable{\mbox{{\bf HK}}}
				\ \rightharpoondown \psi, \\
			\rightharpoondown \varphi \rightarrow \psi,\ \rightharpoondown \psi &\provable{\mbox{{\bf HK}}}
				\ \rightharpoondown \rightharpoondown \varphi, && \mbox{(MP)} \\
			\rightharpoondown \varphi \rightarrow \psi,\ \rightharpoondown \psi &\provable{\mbox{{\bf HK}}}
				\ \rightharpoondown \rightharpoondown \varphi \rightarrow \varphi, && \mbox{(DNE)} \\
			\rightharpoondown \varphi \rightarrow \psi,\ \rightharpoondown \psi &\provable{\mbox{{\bf HK}}}
				\varphi, && \mbox{(MP)} \\
			\rightharpoondown \varphi \rightarrow \psi &\provable{\mbox{{\bf HK}}}
				\ \rightharpoondown \psi \rightarrow \varphi. && \mbox{(DR)}
		\end{align}
		\QED
	\end{sketch}
	
	\begin{sketch}[CON4]
		\begin{align}
			\rightharpoondown \varphi \rightarrow\ \rightharpoondown \psi,\ \psi &\provable{\mbox{{\bf HK}}} \psi, \\
			\rightharpoondown \varphi \rightarrow\ \rightharpoondown \psi,\ \psi &\provable{\mbox{{\bf HK}}} 
				\psi \rightarrow\ \rightharpoondown \rightharpoondown \psi, && \mbox{(DNI)} \\
			\rightharpoondown \varphi \rightarrow\ \rightharpoondown \psi,\ \psi &\provable{\mbox{{\bf HK}}} 
				\ \rightharpoondown \rightharpoondown \psi. && \mbox{(MP)}
		\end{align}
		及び,(CON3)より
		\begin{align}
			\rightharpoondown \varphi \rightarrow\ \rightharpoondown \psi,\ \psi \provable{\mbox{{\bf HK}}} 
				\ \rightharpoondown \rightharpoondown \psi \rightarrow \varphi
		\end{align}
		となるので,(MP)より
		\begin{align}
			\rightharpoondown \varphi \rightarrow\ \rightharpoondown \psi,\ \psi \provable{\mbox{{\bf HK}}} \varphi
		\end{align}
		が成り立つ.よって演繹法則より
		\begin{align}
			\rightharpoondown \varphi \rightarrow\ \rightharpoondown \psi \provable{\mbox{{\bf HK}}} \psi \rightarrow \varphi
		\end{align}
		が得られる.
		\QED
	\end{sketch}
	
	\begin{sketch}[RAA]
		\begin{align}
			&\provable{\mbox{{\bf HK}}} (\rightharpoondown \varphi \rightarrow \bot) \rightarrow\ 
				\rightharpoondown \rightharpoondown \varphi, && \mbox{(NI)} \\
			\rightharpoondown \varphi \rightarrow \bot &\provable{\mbox{{\bf HK}}}\ 
				\rightharpoondown \rightharpoondown \varphi, && \mbox{(DR)} \\
			\rightharpoondown \varphi \rightarrow \bot &\provable{\mbox{{\bf HK}}}\ 
				\rightharpoondown \rightharpoondown \varphi \rightarrow \varphi, && \mbox{(DNE)} \\
			\rightharpoondown \varphi \rightarrow \bot &\provable{\mbox{{\bf HK}}} \varphi, && \mbox{(MP)} \\
			&\provable{\mbox{{\bf HK}}} (\rightharpoondown \varphi \rightarrow \bot) \rightarrow \varphi. && \mbox{(DR)}
		\end{align}
		\QED
	\end{sketch}
	
	\begin{sketch}[EFQ]
		\begin{align}
			&\provable{\mbox{{\bf HK}}} \bot \rightarrow (\rightharpoondown \varphi \rightarrow \bot), && \mbox{(K)} \\
			\bot &\provable{\mbox{{\bf HK}}}\ \rightharpoondown \varphi \rightarrow \bot, && \mbox{(DR)} \\
			\bot &\provable{\mbox{{\bf HK}}} (\rightharpoondown \varphi \rightarrow \bot) \rightarrow \varphi,
				&& \mbox{(RAA)} \\
			\bot &\provable{\mbox{{\bf HK}}} \varphi, && \mbox{(MP)} \\
			&\provable{\mbox{{\bf HK}}} \bot \rightarrow \varphi. && \mbox{(DR)}
		\end{align}
		\QED
	\end{sketch}
	
\section{{\bf HK'}}
	{\bf HK}の量化公理から(UI)と(EE)を取り除き,三段論法に加え
	{\bf 存在汎化}と{\bf 全称汎化}といった推論規則を用いた証明体系を{\bf HK'}とする.
	つまり,
	
	\begin{itembox}[l]{{\bf HK'}の公理}
		$\varphi$と$\psi$と$\xi$を式とし,$x$を変項とし,$\tau$を項とするとき,
		次は{\bf HK'}の公理である.
		\begin{description}
			\item[(S)] $(\varphi \rightarrow (\psi \rightarrow \chi)) 
				\rightarrow ((\varphi \rightarrow \psi)
				\rightarrow (\varphi \rightarrow \chi)).$
			\item[(K)] $\varphi \rightarrow (\psi \rightarrow \varphi).$
			\item[(DI1)] $\varphi \rightarrow (\varphi \vee \psi).$
			\item[(DI2)] $\psi \rightarrow (\varphi \vee \psi).$
			\item[(DE)] $(\varphi \rightarrow \chi) \rightarrow 
				((\psi \rightarrow \chi) \rightarrow ((\varphi \vee \psi) \rightarrow \chi)).$
			\item[(CI)] $\varphi \rightarrow (\psi \rightarrow (\varphi \wedge \psi)).$
			\item[(CE1)] $(\varphi \wedge \psi) \rightarrow \varphi.$
			\item[(CE2)] $(\varphi \wedge \psi) \rightarrow \psi.$
			
			\item[(UE)] $\forall x \varphi \rightarrow \varphi(\tau/x).$
			\item[(EI)] $\varphi(\tau/x) \rightarrow \exists x \varphi.$
			
			\item[(CTI1)] $\varphi \rightarrow (\rightharpoondown \varphi \rightarrow \bot).$
			
			\item[(CTI2)] $\rightharpoondown \varphi \rightarrow (\varphi \rightarrow \bot).$
			
			\item[(NI)] $(\varphi \rightarrow \bot) \rightarrow\ \rightharpoondown \varphi.$
			\item[(DNE)] $\rightharpoondown \rightharpoondown \varphi \rightarrow \varphi.$
		\end{description}
	\end{itembox}
	
	とし,推論規則には,三段論法に加えて
	
	\begin{itembox}[l]{{\bf HK'}の汎化規則}
		$\varphi$と$\psi$を式とし,$x$と$t$を変項とし,$\varphi$に$x$が自由に現れるとし,
		また{\bf $\psi$と$\exists x \varphi$に$t$は自由に現れないとする}.
		\begin{description}
			\item[存在汎化] 
				$\Gamma \provable{\mbox{{\bf HK'}}} \varphi(t/x) \rightarrow \psi$ならば
				$\Gamma \provable{\mbox{{\bf HK'}}} \exists x \varphi \rightarrow \psi$となる.
			
			\item[全称汎化] 
				$\Gamma \provable{\mbox{{\bf HK'}}} \psi \rightarrow \varphi(t/x)$ならば
				$\Gamma \provable{\mbox{{\bf HK'}}} \psi \rightarrow \forall x \varphi$となる.
		\end{description}
	\end{itembox}
	を用いる.規則の前提で太字で強調した文言は{\bf 固有変項条件}と呼ばれる.
	
	\begin{screen}
		\begin{metathm}[{\bf HK}と{\bf HK'}は同値]
			任意の式$\varphi$に対して,$\Gamma \provable{\mbox{{\bf HK}}} \varphi$ならば
			$\Gamma \provable{\mbox{{\bf HK'}}} \varphi$であり,その逆もまた然り.
		\end{metathm}
	\end{screen}
	
	\begin{metaprf}\mbox{}
		\begin{description}
			\item[{\bf HK'}から示されたら{\bf HK}からも証明可能]
			いま
			\begin{align}
				\Gamma \provable{\mbox{{\bf HK'}}} \varphi
			\end{align}
			であるとする.$\varphi$が{\bf HK'}の公理であれば{\bf HK}の公理でもあるし,
			$\Gamma$の公理であれば言わずもがな,これらの場合は
			\begin{align}
				\Gamma \provable{\mbox{{\bf HK}}} \varphi
			\end{align}
			となる.$\varphi$が存在汎化によって得られているとき,つまり,$\varphi$とは
			\begin{align}
				\exists x \psi \rightarrow \chi
			\end{align}
			なる形の式であって,$\psi(t/x) \rightarrow \chi$から存在汎化で得られている場合,
			ここで$t$は$\chi$と$\exists x \psi$に自由に現れない変項であるが,このとき,
			\begin{align}
				\Gamma \provable{\mbox{{\bf HK}}} \psi(t/x) \rightarrow \chi
			\end{align}
			であると仮定すれば,汎化と量化公理(EE)によって
			\begin{align}
				\Gamma &\provable{\mbox{{\bf HK}}} \forall t(\psi(t/x) \rightarrow \chi), \\
				\Gamma &\provable{\mbox{{\bf HK}}} \forall t(\psi(t/x) \rightarrow \chi)
					\rightarrow (\exists x \psi \rightarrow \chi), \\
				\Gamma &\provable{\mbox{{\bf HK}}} \exists x \psi \rightarrow \chi
			\end{align}
			となる.また$\varphi$が全称汎化によって得られているとき,つまり,$\varphi$とは
			\begin{align}
				\chi \rightarrow \forall x \psi
			\end{align}
			なる形の式であって,$\chi \rightarrow \psi(t/x)$から存在汎化で得られている場合,
			ここで$t$は$\chi$と$\forall x \psi$に自由に現れない変項であるが,このとき,
			\begin{align}
				\Gamma \provable{\mbox{{\bf HK}}} \chi \rightarrow \psi(t/x)
			\end{align}
			であると仮定すれば,汎化と量化公理(UI)によって
			\begin{align}
				\Gamma &\provable{\mbox{{\bf HK}}} \forall t(\chi \rightarrow \psi(t/x)), \\
				\Gamma &\provable{\mbox{{\bf HK}}} \forall t(\chi \rightarrow \psi(t/x))
					\rightarrow (\chi \rightarrow \forall x \psi), \\
				\Gamma &\provable{\mbox{{\bf HK}}} \chi \rightarrow \forall x \psi
			\end{align}
			となる.$\varphi$が三段論法で得られている場合,つまり$\psi$と$\psi \rightarrow \varphi$
			なる形の式が{\bf HK'}から示されている場合であるが,
			\begin{align}
				\Gamma &\provable{\mbox{{\bf HK}}} \psi, \\
				\Gamma &\provable{\mbox{{\bf HK}}} \psi \rightarrow \varphi
			\end{align}
			と仮定すれば$\Gamma \provable{\mbox{{\bf HK}}} \varphi$も従う.
	
		\item[{\bf HK}から示されたら{\bf HK'}からも証明可能]
			いま
			\begin{align}
				\Gamma \provable{\mbox{{\bf HK}}} \varphi
			\end{align}
			とする.$\varphi$が量化公理(UI)(EE)以外の{\bf HK}の公理か,$\Gamma$の公理であれば
			\begin{align}
				\Gamma \provable{\mbox{{\bf HK'}}} \varphi
			\end{align}
			である.$\varphi$が
			\begin{align}
				\forall t (\chi \rightarrow \psi(t/x)) 
				\rightarrow (\chi \rightarrow \forall x \psi)
			\end{align}
			なる形の公理であるとき($t$は$\chi$と$\forall x \psi$には自由に現れない),
			\begin{align}
				\Gamma &\provable{\mbox{{\bf HK'}}} 
					\forall t (\chi \rightarrow \psi(t/x)) 
					\rightarrow (\chi \rightarrow \psi(t/x)), \\
				\color{red}\chi \rightarrow \psi(t/x),\ \Gamma &
				\color{red}\provable{\mbox{{\bf HK'}}}
					\chi \rightarrow \psi(t/x), \\
				\color{red}\chi \rightarrow \psi(t/x),\ \Gamma &
				\color{red}\provable{\mbox{{\bf HK'}}}
					\chi \rightarrow \forall x \psi, \\
				\Gamma &\provable{\mbox{{\bf HK'}}} (\chi \rightarrow \psi(t/x)) 
					\rightarrow (\chi \rightarrow \forall x \psi), \\
				\forall t (\chi \rightarrow \psi(t/x)),\ \Gamma
					&\provable{\mbox{{\bf HK'}}} \forall t (\chi \rightarrow \psi(t/x)), \\
				\forall t (\chi \rightarrow \psi(t/x)),\ \Gamma
					&\provable{\mbox{{\bf HK'}}} \forall t (\chi \rightarrow \psi(t/x)) 
					\rightarrow (\chi \rightarrow \psi(t/x)), \\
				\forall t (\chi \rightarrow \psi(t/x)),\ \Gamma
					&\provable{\mbox{{\bf HK'}}} \chi \rightarrow \psi(t/x), \\
				\forall t (\chi \rightarrow \psi(t/x)),\ \Gamma
					&\provable{\mbox{{\bf HK'}}} (\chi \rightarrow \psi(t/x)) 
					\rightarrow (\chi \rightarrow \forall x \psi), \\
				\forall t (\chi \rightarrow \psi(t/x)),\ \Gamma
					&\provable{\mbox{{\bf HK'}}} \chi \rightarrow \forall x \psi, \\
				\Gamma &\provable{\mbox{{\bf HK'}}} \forall t (\chi \rightarrow \psi(t/x)) 
					\rightarrow (\chi \rightarrow \forall x \psi)
			\end{align}
			となる(赤字で{\bf HK'}の全称汎化規則を用いた箇所を示している).
			つまり(UI)は{\bf HK'}の定理である.同様に(EE)も{\bf HK'}の定理である.実際,
			\begin{align}
				\Gamma &\provable{\mbox{{\bf HK'}}} 
					\forall t (\psi(t/x) \rightarrow \chi) 
					\rightarrow (\psi(t/x) \rightarrow \chi), \\
				\color{red}\psi(t/x) \rightarrow \chi,\ \Gamma &
				\color{red}\provable{\mbox{{\bf HK'}}}
					\psi(t/x) \rightarrow \chi, \\
				\color{red}\psi(t/x) \rightarrow \chi,\ \Gamma &
				\color{red}\provable{\mbox{{\bf HK'}}}
					\exists x \psi \rightarrow \chi, \\
				\Gamma &\provable{\mbox{{\bf HK'}}} (\psi(t/x) \rightarrow \chi) 
					\rightarrow (\exists x \psi \rightarrow \chi), \\
				\forall t (\psi(t/x) \rightarrow \chi),\ \Gamma
					&\provable{\mbox{{\bf HK'}}} \forall t (\psi(t/x) \rightarrow \chi), \\
				\forall t (\psi(t/x) \rightarrow \chi),\ \Gamma
					&\provable{\mbox{{\bf HK'}}} \forall t (\psi(t/x) \rightarrow \chi)
					\rightarrow (\psi(t/x) \rightarrow \chi), \\
				\forall t (\psi(t/x) \rightarrow \chi),\ \Gamma
					&\provable{\mbox{{\bf HK'}}} \psi(t/x) \rightarrow \chi, \\
				\forall t (\psi(t/x) \rightarrow \chi),\ \Gamma
					&\provable{\mbox{{\bf HK'}}} (\psi(t/x) \rightarrow \chi) 
					\rightarrow (\exists x \psi \rightarrow \chi), \\
				\forall t (\psi(t/x) \rightarrow \chi),\ \Gamma
					&\provable{\mbox{{\bf HK'}}} \exists x \psi \rightarrow \chi, \\
				\Gamma &\provable{\mbox{{\bf HK'}}} \forall t (\psi(t/x) \rightarrow \chi) 
					\rightarrow (\exists x \psi \rightarrow \chi)
			\end{align}
			となる.$\varphi$が汎化によって導かれているとき,つまり$\varphi$は
			\begin{align}
				\forall x \psi
			\end{align}
			なる式であって,先に$\psi(t/x)$なる式が{\bf HK}から証明されているとき
			($x$は$\psi$に自由に現れ,$t$は$\forall x \psi$に自由に現れない変項である),
			\begin{align}
				\Gamma \provable{\mbox{{\bf HK'}}} \psi(t/x)
			\end{align}
			と仮定したら
			\begin{align}
				\Gamma \provable{\mbox{{\bf HK'}}} \forall x \psi
			\end{align}
			が成り立つ.実際,$\varphi$を$\chi \rightarrow \chi$といった文とすれば
			\begin{align}
				\Gamma \provable{\mbox{{\bf HK'}}} \varphi \rightarrow \psi(t/x)
			\end{align}
			が成り立つので,全称汎化より
			\begin{align}
				\Gamma \provable{\mbox{{\bf HK'}}} \varphi \rightarrow \forall x \psi
			\end{align}
			となり,
			\begin{align}
				\Gamma \provable{\mbox{{\bf HK'}}} \varphi
			\end{align}
			と併せて
			\begin{align}
				\Gamma \provable{\mbox{{\bf HK'}}} \forall x \psi
			\end{align}
			が従う.
			\QED
		\end{description}
	\end{metaprf}
	
\begin{comment}	
\subsection{汎化規則は{\bf HK'}から導かれる}
	{\bf HK}の量化公理に
	\begin{align}
		\forall x (\psi \rightarrow \varphi) \rightarrow
		(\forall x \psi \rightarrow \forall x \varphi),
		\quad \mbox{ただし$\varphi$と$\psi$には$x$が自由に現れる}
	\end{align}
	を追加すれば,一般化規則は導かれる.
	
	\begin{screen}
		$x$と$t$を変項とし,$x$は$\psi$に自由に現れるとし,
		$t$は$\forall x \psi$に自由に現れないとする..
		このとき
		\begin{align}
			\Gamma \provable{\mbox{{\bf HK'}}} \psi(t/x)
		\end{align}
		ならば
		\begin{align}
			\Gamma \provable{\mbox{{\bf HK'}}} \forall x \psi
		\end{align}
		である.
	\end{screen}
	
	\begin{sketch}
		$\provable{\mbox{{\bf HK'}}} \psi(t/x)$であるときは
		\begin{align}
			\provable{\mbox{{\bf HK}}} \forall x \psi
		\end{align}
		が成立する.$\varphi$が三段論法によって示されるとき,つまり式$\psi$で
		\begin{align}
			\Gamma &\provable{\mbox{{\bf HK}}} \psi, \\
			\Sigma &\provable{\mbox{{\bf HK}}} \psi \rightarrow \varphi
		\end{align}
		を満たすものが取れるとき,$\psi$に$x$が自由に現れているかいないかで
		\begin{description}
			\item[case1] $\psi$に$x$が自由に現れていないとき,
				\begin{align}
					\Gamma \provable{\mbox{{\bf HK}}} \psi
				\end{align}
				かつ
				\begin{align}
					\Gamma \provable{\mbox{{\bf HK}}} \forall x (\psi \rightarrow \varphi)
				\end{align}
				
			\item[case2] $\psi$に$x$が自由に現れているとき,
				\begin{align}
					\Gamma \provable{\mbox{{\bf HK}}} \forall x \psi
				\end{align}
				かつ
				\begin{align}
					\Gamma \provable{\mbox{{\bf HK}}} \forall x (\psi \rightarrow \varphi)
				\end{align}
		\end{description}
		のいずれかのケースを一つ仮定する.
		\begin{description}
			\item[case1] 量化公理(UI)より
				\begin{align}
					\Gamma &\provable{\mbox{{\bf HK}}} \forall x (\psi \rightarrow \varphi), \\
					\Gamma &\provable{\mbox{{\bf HK}}} \forall x (\psi \rightarrow \varphi) \rightarrow (\psi \rightarrow \forall x \varphi), \\
					\Gamma &\provable{\mbox{{\bf HK}}} \psi \rightarrow \forall x \varphi
				\end{align}
				が成り立つので,$\Gamma \provable{\mbox{{\bf HK}}} \psi$の仮定と併せて
				\begin{align}
					\Gamma \provable{\mbox{{\bf HK}}} \forall x \varphi
				\end{align}
				が得られる.
				
			\item[case2]
				新しく追加した量化公理を用いれば,
				\begin{align}
					\Gamma &\provable{\mbox{{\bf HK}}} \forall x (\psi \rightarrow \varphi), \\
					\Gamma &\provable{\mbox{{\bf HK}}} \forall x (\psi \rightarrow \varphi) \rightarrow (\forall x \psi \rightarrow \forall x \varphi), \\
					\Gamma &\provable{\mbox{{\bf HK}}} \forall x \psi \rightarrow \forall x \varphi
				\end{align}
				が得られるので,$\Gamma \provable{\mbox{{\bf HK}}} \forall x \psi$の仮定と併せて
				\begin{align}
					\Gamma \provable{\mbox{{\bf HK}}} \forall x \varphi
				\end{align}
				が従う.
				\QED
		\end{description}
	\end{sketch}
	
	逆に,新しく追加した量化公理は一般化規則から導かれる.実際,量化規則(UE)より
	\begin{align}
		\forall x (\psi \rightarrow \varphi),\ \forall x \psi &\provable{\mbox{{\bf HK}}} \forall x (\psi \rightarrow \varphi), \\
		\forall x (\psi \rightarrow \varphi),\ \forall x \psi &\provable{\mbox{{\bf HK}}} \forall x (\psi \rightarrow \varphi)
		\rightarrow (\psi(t/x) \rightarrow \varphi(t/x)), \\
		\forall x (\psi \rightarrow \varphi),\ \forall x \psi &\provable{\mbox{{\bf HK}}} \psi(t/x) \rightarrow \varphi(t/x)
	\end{align}
	となり(ただし$t$は$\psi$と$\varphi$に現れない変項とする),また一方で
	\begin{align}
		\forall x (\psi \rightarrow \varphi),\ \forall x \psi &\provable{\mbox{{\bf HK}}} \forall x \psi, \\
		\forall x (\psi \rightarrow \varphi),\ \forall x \psi &\provable{\mbox{{\bf HK}}} \forall x \psi \rightarrow \psi(t/x), \\
		\forall x (\psi \rightarrow \varphi),\ \forall x \psi &\provable{\mbox{{\bf HK}}} \psi(t/x)
	\end{align}
	も成り立つから,前と併せて三段論法より
	\begin{align}
		\forall x (\psi \rightarrow \varphi),\ \forall x \psi \provable{\mbox{{\bf HK}}} \varphi(t/x)
	\end{align}
	となる.演繹定理より
	\begin{align}
		\forall x (\psi \rightarrow \varphi) \provable{\mbox{{\bf HK}}} \forall x \psi  \rightarrow \varphi(t/x)
	\end{align}
	となり,一般化規則より
	\begin{align}
		\forall x (\psi \rightarrow \varphi) \provable{\mbox{{\bf HK}}} \forall t ( \forall x \psi  \rightarrow \varphi(t/x))
	\end{align}
	となる.量化規則(UI)より
	\begin{align}
		\forall x (\psi \rightarrow \varphi) \provable{\mbox{{\bf HK}}} \forall t ( \forall x \psi  \rightarrow \varphi(t/x)) \rightarrow
		(\forall x \psi \rightarrow \forall x \varphi)
	\end{align}
	が成り立つので,
	\begin{align}
		\forall x (\psi \rightarrow \varphi) \provable{\mbox{{\bf HK}}}
		\forall x \psi \rightarrow \forall x \varphi
	\end{align}
	となる.演繹定理より
	\begin{align}
		\provable{\mbox{{\bf HK}}} \forall x (\psi \rightarrow \varphi)
		\rightarrow (\forall x \psi \rightarrow \forall x \varphi)
	\end{align}
	が得られる.
	
\end{comment}

\section{直観主義と古典論理}
	任意の式$\varphi$に対してその否定翻訳を$\varphi^{N}$と書く.
	$\provable{\mbox{{\bf HM}}} \varphi^{N}$ならば
	$\provable{\mbox{{\bf HK}}} \varphi^{N}$は当たり前.
	逆に$\provable{\mbox{{\bf HK}}} \varphi^{N}$ならば
	$\provable{\mbox{{\bf HK}}} \varphi$を経由して
	$\provable{\mbox{{\bf HM}}} \varphi^{N}$となる.
	{\bf 式を否定翻訳に制限すれば直観主義と古典論理は変わらない.}
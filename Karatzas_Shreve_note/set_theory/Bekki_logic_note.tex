\section{Hilbert流証明論メモ}
	参考文献: 戸次大介「数理論理学」
	
	\begin{itembox}[l]{{\bf SK}の公理}
		\begin{description}
			\item[(S)] $(\varphi \rightarrow (\psi \rightarrow \chi)) 
				\rightarrow ((\varphi \rightarrow \psi)
				\rightarrow (\varphi \rightarrow \chi)).$
			
			\item[(K)] $\varphi \rightarrow (\psi \rightarrow \varphi).$
		\end{description}
	\end{itembox}
	
	{\bf SK}から証明可能な式
	\begin{description}
		\item[(I)] $\varphi \rightarrow \varphi$
		\item[(B)] $(\psi \rightarrow \chi) \rightarrow ((\varphi \rightarrow \psi) \rightarrow (\varphi \rightarrow \chi)).$
		\item[(C)] $(\varphi \rightarrow (\psi \rightarrow \chi)) \rightarrow (\psi \rightarrow (\varphi \rightarrow \chi)).$
		\item[(W)] $(\varphi \rightarrow (\varphi \rightarrow \psi)) \rightarrow (\varphi \rightarrow \psi).$
		\item[(B')] $(\varphi \rightarrow \psi) \rightarrow ((\psi \rightarrow \chi) \rightarrow (\varphi \rightarrow \chi)).$
		\item[(C$\ast$)] $\varphi \rightarrow ((\varphi \rightarrow \psi) \rightarrow \psi)$
	\end{description}
	
	\begin{itembox}[l]{否定の追加}
		\begin{description}
			\item[(CTI1)] $\varphi \rightarrow (\rightharpoondown \varphi \rightarrow \bot).$
			
			\item[(CTI2)] $\rightharpoondown \varphi \rightarrow (\varphi \rightarrow \bot).$
			
			\item[(NI)] $(\varphi \rightarrow \bot) \rightarrow\ \rightharpoondown \varphi.$
		\end{description}
	\end{itembox}
	
	このとき証明可能な式
	\begin{description}
		\item[(DNI)] $\varphi \rightarrow\ \rightharpoondown \rightharpoondown \varphi.$
		\item[(CON1)] $(\varphi \rightarrow \psi) \rightarrow (\rightharpoondown \psi \rightarrow\ \rightharpoondown \varphi).$
		\item[(CON2)] $(\varphi \rightarrow\ \rightharpoondown \psi) \rightarrow (\psi \rightarrow\ \rightharpoondown \varphi).$
	\end{description}
	
	\begin{itembox}[l]{{\bf HM}の公理}
		\begin{description}
			\item[(S)] $(\varphi \rightarrow (\psi \rightarrow \chi)) 
				\rightarrow ((\varphi \rightarrow \psi)
				\rightarrow (\varphi \rightarrow \chi)).$
			\item[(K)] $\varphi \rightarrow (\psi \rightarrow \varphi).$
			\item[(DI1)] $\varphi \rightarrow (\varphi \vee \psi).$
			\item[(DI2)] $\psi \rightarrow (\varphi \vee \psi).$
			\item[(DE)] $(\varphi \rightarrow \chi) \rightarrow 
				((\psi \rightarrow \chi) \rightarrow ((\varphi \vee \psi) \rightarrow \chi)).$
			\item[(CI)] $\varphi \rightarrow (\psi \rightarrow (\varphi \wedge \psi)).$
			\item[(CE1)] $(\varphi \wedge \psi) \rightarrow \varphi.$
			\item[(CE2)] $(\varphi \wedge \psi) \rightarrow \psi.$
			\item[(UI)] $\forall \zeta (\psi \rightarrow \varphi[\zeta/\xi]) 
				\rightarrow (\psi \rightarrow \forall \xi \varphi).$
			\item[(UE)] $\forall \xi \varphi \rightarrow \varphi[\tau/\xi].$
			\item[(EI)] $\varphi[\tau/\xi] \rightarrow \exists \xi \varphi.$
			\item[(EE)] $\forall \zeta (\varphi[\zeta/\xi] \rightarrow \psi)
				\rightarrow (\exists \xi \varphi \rightarrow \psi).$
		\end{description}
	\end{itembox}
	
	{\bf HM}から証明可能な式
	\begin{description}
		\item[LNC] $\rightharpoondown (\varphi \wedge \rightharpoondown \varphi).$
		\item[(DIST$\wedge$)] $\varphi \vee (\psi \wedge \chi) 
			\leftrightarrow (\varphi \vee \psi) \wedge (\varphi \vee \chi).$
		\item[(DIST$\vee$)] $\varphi \wedge (\psi \vee \chi) 
			\leftrightarrow (\varphi \wedge \psi) \vee (\varphi \wedge \chi).$
		\item[(DM$\vee$)] $\rightharpoondown (\varphi \vee \psi) \leftrightarrow
			\ \rightharpoondown \varphi \wedge \rightharpoondown \psi.$
	\end{description}
	
	\begin{sketch}[LNC]
		\begin{align}
			\varphi \wedge \rightharpoondown \varphi &\provable{\mbox{{\bf HM}}} \varphi, \\
			\varphi \wedge \rightharpoondown \varphi &\provable{\mbox{{\bf HM}}}\ \rightharpoondown \varphi, \\
			\varphi \wedge \rightharpoondown \varphi &\provable{\mbox{{\bf HM}}}
				\varphi \rightarrow (\rightharpoondown \varphi \rightarrow \bot), \\
			\varphi \wedge \rightharpoondown \varphi &\provable{\mbox{{\bf HM}}}\ \rightharpoondown \varphi \rightarrow \bot, \\
			\varphi \wedge \rightharpoondown \varphi &\provable{\mbox{{\bf HM}}} \bot, \\
			&\provable{\mbox{{\bf HM}}} (\varphi \wedge \rightharpoondown \varphi) \rightarrow \bot, \\
			&\provable{\mbox{{\bf HM}}} ((\varphi \wedge \rightharpoondown \varphi) \rightarrow \bot)
				\rightarrow\ \rightharpoondown (\varphi \wedge \rightharpoondown \varphi), \\
			&\provable{\mbox{{\bf HM}}}\ \rightharpoondown (\varphi \wedge \rightharpoondown \varphi).
		\end{align}
		\QED
	\end{sketch}
	
	\begin{sketch}[DM$\vee$]
		\begin{align}
			&\provable{\mbox{{\bf HM}}} \varphi \rightarrow (\varphi \vee \psi), && \mbox{(DI1)}\\
			&\provable{\mbox{{\bf HM}}} (\varphi \rightarrow (\varphi \vee \psi))
				\rightarrow (\rightharpoondown (\varphi \vee \psi) \rightarrow\ \rightharpoondown \varphi), 
				&& \mbox{(CON1)}\\
			&\provable{\mbox{{\bf HM}}}\ \rightharpoondown (\varphi \vee \psi) \rightarrow\ \rightharpoondown \varphi, 
				&& \mbox{(MP)}\\
			\rightharpoondown (\varphi \vee \psi) &\provable{\mbox{{\bf HM}}}\ \rightharpoondown \varphi.
				&& \mbox{(DR)}
		\end{align}
		同様に
		\begin{align}
			\rightharpoondown (\varphi \vee \psi) \provable{\mbox{{\bf HM}}}\ \rightharpoondown \psi
		\end{align}
		となり,
		\begin{align}
			\rightharpoondown (\varphi \vee \psi) &\provable{\mbox{{\bf HM}}}\ \rightharpoondown \varphi
				\rightarrow (\rightharpoondown \psi \rightarrow 
				(\rightharpoondown \varphi \wedge \rightharpoondown \psi)), && \mbox{(CI)}\\
			\rightharpoondown (\varphi \vee \psi) &\provable{\mbox{{\bf HM}}}\ 
				\rightharpoondown \psi \rightarrow (\rightharpoondown \varphi \wedge \rightharpoondown \psi), 
				&& \mbox{(MP)}\\
			\rightharpoondown (\varphi \vee \psi) &\provable{\mbox{{\bf HM}}}\ 
				\rightharpoondown \varphi \wedge \rightharpoondown \psi && \mbox{(MP)}
		\end{align}
		が得られる.逆に
		\begin{align}
			\rightharpoondown \varphi \wedge \rightharpoondown \psi &\provable{\mbox{{\bf HM}}}\ \rightharpoondown \varphi, 
				&& \mbox{(CE1)}\\
			\rightharpoondown \varphi \wedge \rightharpoondown \psi &\provable{\mbox{{\bf HM}}}\ 
			\rightharpoondown \varphi \rightarrow (\varphi \rightarrow \bot), && \mbox{(CTI2)}\\
			\rightharpoondown \varphi \wedge \rightharpoondown \psi &\provable{\mbox{{\bf HM}}} \varphi \rightarrow \bot
				&& \mbox{(MP)}
		\end{align}
		となり,同様に
		\begin{align}
			\rightharpoondown \varphi \wedge \rightharpoondown \psi \provable{\mbox{{\bf HM}}} \psi \rightarrow \bot
		\end{align}
		も成り立つ.よって
		\begin{align}
			\rightharpoondown \varphi \wedge \rightharpoondown \psi &\provable{\mbox{{\bf HM}}} 
				(\varphi \rightarrow \bot) \rightarrow ((\psi \rightarrow \bot) 
				\rightarrow ((\varphi \vee \psi) \rightarrow \bot)), && \mbox{(DE)}\\
			\rightharpoondown \varphi \wedge \rightharpoondown \psi &\provable{\mbox{{\bf HM}}} 
				(\psi \rightarrow \bot) \rightarrow ((\varphi \vee \psi) \rightarrow \bot), && \mbox{(MP)}\\
			\rightharpoondown \varphi \wedge \rightharpoondown \psi &\provable{\mbox{{\bf HM}}} 
				(\varphi \vee \psi) \rightarrow \bot, && \mbox{(MP)}\\
			\rightharpoondown \varphi \wedge \rightharpoondown \psi &\provable{\mbox{{\bf HM}}} 
				((\varphi \vee \psi) \rightarrow \bot) \rightarrow\ \rightharpoondown (\varphi \vee \psi), && \mbox{(NI)}\\
			\rightharpoondown \varphi \wedge \rightharpoondown \psi &\provable{\mbox{{\bf HM}}} 
				\ \rightharpoondown (\varphi \vee \psi) && \mbox{(MP)}
		\end{align}
		が得られる.
		\QED
	\end{sketch}
	
	\begin{itembox}[l]{{\bf HK}の公理}
		{\bf HM}の公理に次を追加:
		\begin{description}
			\item[(DNE)] $\rightharpoondown \rightharpoondown \varphi \rightarrow \varphi.$
		\end{description}
	\end{itembox}
	
	{\bf HK}から証明可能な式
	\begin{description}
		\item[(CON3)] $(\rightharpoondown \varphi \rightarrow \psi) \rightarrow (\rightharpoondown \psi \rightarrow \varphi).$
		\item[(CON4)] $(\rightharpoondown \varphi \rightarrow\ \rightharpoondown \psi) 
			\rightarrow (\psi \rightarrow \varphi).$
		\item[(RAA)] $(\rightharpoondown \varphi \rightarrow \bot) \rightarrow \varphi.$
		\item[(EFQ)] $\bot \rightarrow \varphi.$
	\end{description}
	
	\begin{sketch}[CON3]
		\begin{align}
			\rightharpoondown \varphi \rightarrow \psi,\ \rightharpoondown \psi &\provable{\mbox{{\bf HK}}}
				\ \rightharpoondown \psi \rightarrow\ \rightharpoondown \rightharpoondown \varphi,
				&& \mbox{(CON1)} \\
			\rightharpoondown \varphi \rightarrow \psi,\ \rightharpoondown \psi &\provable{\mbox{{\bf HK}}}
				\ \rightharpoondown \psi, \\
			\rightharpoondown \varphi \rightarrow \psi,\ \rightharpoondown \psi &\provable{\mbox{{\bf HK}}}
				\ \rightharpoondown \rightharpoondown \varphi, && \mbox{(MP)} \\
			\rightharpoondown \varphi \rightarrow \psi,\ \rightharpoondown \psi &\provable{\mbox{{\bf HK}}}
				\ \rightharpoondown \rightharpoondown \varphi \rightarrow \varphi, && \mbox{(DNE)} \\
			\rightharpoondown \varphi \rightarrow \psi,\ \rightharpoondown \psi &\provable{\mbox{{\bf HK}}}
				\varphi, && \mbox{(MP)} \\
			\rightharpoondown \varphi \rightarrow \psi &\provable{\mbox{{\bf HK}}}
				\ \rightharpoondown \psi \rightarrow \varphi. && \mbox{(DR)}
		\end{align}
		\QED
	\end{sketch}
	
	\begin{sketch}[CON4]
		\begin{align}
			\rightharpoondown \varphi \rightarrow\ \rightharpoondown \psi,\ \psi &\provable{\mbox{{\bf HK}}} \psi, \\
			\rightharpoondown \varphi \rightarrow\ \rightharpoondown \psi,\ \psi &\provable{\mbox{{\bf HK}}} 
				\psi \rightarrow\ \rightharpoondown \rightharpoondown \psi, && \mbox{(DNI)} \\
			\rightharpoondown \varphi \rightarrow\ \rightharpoondown \psi,\ \psi &\provable{\mbox{{\bf HK}}} 
				\ \rightharpoondown \rightharpoondown \psi. && \mbox{(MP)}
		\end{align}
		及び,(CON3)より
		\begin{align}
			\rightharpoondown \varphi \rightarrow\ \rightharpoondown \psi,\ \psi \provable{\mbox{{\bf HK}}} 
				\ \rightharpoondown \rightharpoondown \psi \rightarrow \varphi
		\end{align}
		となるので,(MP)より
		\begin{align}
			\rightharpoondown \varphi \rightarrow\ \rightharpoondown \psi,\ \psi \provable{\mbox{{\bf HK}}} \varphi
		\end{align}
		が成り立つ.よって演繹法則より
		\begin{align}
			\rightharpoondown \varphi \rightarrow\ \rightharpoondown \psi \provable{\mbox{{\bf HK}}} \psi \rightarrow \varphi
		\end{align}
		が得られる.
		\QED
	\end{sketch}
	
	\begin{sketch}[RAA]
		\begin{align}
			&\provable{\mbox{{\bf HK}}} (\rightharpoondown \varphi \rightarrow \bot) \rightarrow\ 
				\rightharpoondown \rightharpoondown \varphi, && \mbox{(NI)} \\
			\rightharpoondown \varphi \rightarrow \bot &\provable{\mbox{{\bf HK}}}\ 
				\rightharpoondown \rightharpoondown \varphi, && \mbox{(DR)} \\
			\rightharpoondown \varphi \rightarrow \bot &\provable{\mbox{{\bf HK}}}\ 
				\rightharpoondown \rightharpoondown \varphi \rightarrow \varphi, && \mbox{(DNE)} \\
			\rightharpoondown \varphi \rightarrow \bot &\provable{\mbox{{\bf HK}}} \varphi, && \mbox{(MP)} \\
			&\provable{\mbox{{\bf HK}}} (\rightharpoondown \varphi \rightarrow \bot) \rightarrow \varphi. && \mbox{(DR)}
		\end{align}
		\QED
	\end{sketch}
	
	\begin{sketch}[EFQ]
		\begin{align}
			&\provable{\mbox{{\bf HK}}} \bot \rightarrow (\rightharpoondown \varphi \rightarrow \bot), && \mbox{(K)} \\
			\bot &\provable{\mbox{{\bf HK}}}\ \rightharpoondown \varphi \rightarrow \bot, && \mbox{(DR)} \\
			\bot &\provable{\mbox{{\bf HK}}} (\rightharpoondown \varphi \rightarrow \bot) \rightarrow \varphi,
				&& \mbox{(RAA)} \\
			\bot &\provable{\mbox{{\bf HK}}} \varphi, && \mbox{(MP)} \\
			&\provable{\mbox{{\bf HK}}} \bot \rightarrow \varphi. && \mbox{(DR)}
		\end{align}
		\QED
	\end{sketch}
	
	{\bf HK}の定理の証明とは,式の列$\varphi_{1},\cdots,\varphi_{n}$であって,
	各$i$について以下が満たされるものである.
	\begin{itemize}
		\item $\varphi_{i}$は{\bf HK}の公理である.
		\item $j,k < i$なる$j,k$が取れて,$\varphi_{k}$は$\varphi_{j} \rightarrow \varphi_{i}$に一致する.
		\item $j < i$なる$j$が取れて,$\varphi_{i}$は$\forall x \varphi_{j}$なる形式である.
	\end{itemize}
	
\subsection{量化公理の取り替え}
	{\bf HK'}の公理を
	\begin{screen}
		\begin{description}
			\item[(S)] $(\varphi \rightarrow (\psi \rightarrow \chi)) 
				\rightarrow ((\varphi \rightarrow \psi)
				\rightarrow (\varphi \rightarrow \chi)).$
			\item[(K)] $\varphi \rightarrow (\psi \rightarrow \varphi).$
			\item[(DI1)] $\varphi \rightarrow (\varphi \vee \psi).$
			\item[(DI2)] $\psi \rightarrow (\varphi \vee \psi).$
			\item[(DE)] $(\varphi \rightarrow \chi) \rightarrow 
				((\psi \rightarrow \chi) \rightarrow ((\varphi \vee \psi) \rightarrow \chi)).$
			\item[(CI)] $\varphi \rightarrow (\psi \rightarrow (\varphi \wedge \psi)).$
			\item[(CE1)] $(\varphi \wedge \psi) \rightarrow \varphi.$
			\item[(CE2)] $(\varphi \wedge \psi) \rightarrow \psi.$
			
			\item[(UE)] $\forall \xi \varphi \rightarrow \varphi[\tau/\xi].$
			\item[(EI)] $\varphi[\tau/\xi] \rightarrow \exists \xi \varphi.$
			
			\item[(CTI1)] $\varphi \rightarrow (\rightharpoondown \varphi \rightarrow \bot).$
			
			\item[(CTI2)] $\rightharpoondown \varphi \rightarrow (\varphi \rightarrow \bot).$
			
			\item[(NI)] $(\varphi \rightarrow \bot) \rightarrow\ \rightharpoondown \varphi.$
			\item[(DNE)] $\rightharpoondown \rightharpoondown \varphi \rightarrow \varphi.$
		\end{description}
	\end{screen}
	とするとき,任意の式$\varphi$に対して
	\begin{align}
		\provable{\mbox{{\bf HK}}} \varphi
		\Longleftrightarrow \provable{\mbox{{\bf HK'}}} \varphi
	\end{align}
	が成り立つ.
	\begin{align}
		\provable{\mbox{{\bf HK}}} \varphi
	\end{align}
	であるとして,$\varphi_{1},\varphi_{2},\cdots,\varphi_{n}$を$\varphi$への証明とする.
	$\varphi_{i}$が量化公理(UI)(EE)以外の公理であれば
	\begin{align}
		\provable{\mbox{{\bf HK'}}} \varphi_{i}
	\end{align}
	である.$\varphi_{i}$が
	\begin{align}
		\forall \zeta (\psi \rightarrow \varphi[\zeta/\xi]) 
		\rightarrow (\psi \rightarrow \forall \xi \varphi)
	\end{align}
	なる形の公理であるとき,
	\begin{align}
		&\provable{\mbox{{\bf HK'}}} 
			\forall \zeta (\psi \rightarrow \varphi[\zeta/\xi]) 
			\rightarrow (\psi \rightarrow \varphi[\zeta/\xi]), \\
		\color{red}\psi \rightarrow \varphi[\zeta/\xi] &
		\color{red}\provable{\mbox{{\bf HK'}}}
			\psi \rightarrow \varphi[\zeta/\xi], \\
		\color{red}\psi \rightarrow \varphi[\zeta/\xi] &
		\color{red}\provable{\mbox{{\bf HK'}}}
			\psi \rightarrow \forall \xi \varphi, \\
		&\provable{\mbox{{\bf HK'}}} (\psi \rightarrow \varphi[\zeta/\xi]) 
			\rightarrow (\psi \rightarrow \forall \xi \varphi), \\
		\forall \zeta (\psi \rightarrow \varphi[\zeta/\xi])
			&\provable{\mbox{{\bf HK'}}} \forall \zeta (\psi \rightarrow \varphi[\zeta/\xi]), \\
		\forall \zeta (\psi \rightarrow \varphi[\zeta/\xi])
			&\provable{\mbox{{\bf HK'}}} \forall \zeta (\psi \rightarrow \varphi[\zeta/\xi]) 
			\rightarrow (\psi \rightarrow \varphi[\zeta/\xi]), \\
		\forall \zeta (\psi \rightarrow \varphi[\zeta/\xi])
			&\provable{\mbox{{\bf HK'}}} \psi \rightarrow \varphi[\zeta/\xi], \\
		\forall \zeta (\psi \rightarrow \varphi[\zeta/\xi])
			&\provable{\mbox{{\bf HK'}}} (\psi \rightarrow \varphi[\zeta/\xi]) 
			\rightarrow (\psi \rightarrow \forall \xi \varphi), \\
		\forall \zeta (\psi \rightarrow \varphi[\zeta/\xi])
			&\provable{\mbox{{\bf HK'}}} \psi \rightarrow \forall \xi \varphi, \\
		&\provable{\mbox{{\bf HK'}}} \forall \zeta (\psi \rightarrow \varphi[\zeta/\xi]) 
		\rightarrow (\psi \rightarrow \forall \xi \varphi)
	\end{align}
	となる(赤字で{\bf HK'}の推論規則を用いた箇所を示している).
	つまり(UI)は{\bf HK'}の定理である.同様に(EE)も{\bf HK'}の定理である.実際,
	\begin{align}
		&\provable{\mbox{{\bf HK'}}} 
			\forall \zeta (\varphi[\zeta/\xi] \rightarrow \psi) 
			\rightarrow (\varphi[\zeta/\xi] \rightarrow \psi), \\
		\color{red}\varphi[\zeta/\xi] \rightarrow \psi &
		\color{red}\provable{\mbox{{\bf HK'}}}
			\varphi[\zeta/\xi] \rightarrow \psi, \\
		\color{red}\varphi[\zeta/\xi] \rightarrow \psi &
		\color{red}\provable{\mbox{{\bf HK'}}}
			\exists \xi \varphi \rightarrow \psi, \\
		&\provable{\mbox{{\bf HK'}}} (\varphi[\zeta/\xi] \rightarrow \psi) 
			\rightarrow (\exists \xi \varphi \rightarrow \psi), \\
		\forall \zeta (\varphi[\zeta/\xi] \rightarrow \psi)
			&\provable{\mbox{{\bf HK'}}} \forall \zeta (\varphi[\zeta/\xi] \rightarrow \psi), \\
		\forall \zeta (\varphi[\zeta/\xi] \rightarrow \psi)
			&\provable{\mbox{{\bf HK'}}} \forall \zeta (\varphi[\zeta/\xi]) \rightarrow \psi 
			\rightarrow (\varphi[\zeta/\xi] \rightarrow \psi), \\
		\forall \zeta (\varphi[\zeta/\xi] \rightarrow \psi)
			&\provable{\mbox{{\bf HK'}}} \varphi[\zeta/\xi] \rightarrow \psi, \\
		\forall \zeta (\varphi[\zeta/\xi] \rightarrow \psi)
			&\provable{\mbox{{\bf HK'}}} (\varphi[\zeta/\xi] \rightarrow \psi) 
			\rightarrow (\exists \xi \varphi \rightarrow \psi), \\
		\forall \zeta (\varphi[\zeta/\xi] \rightarrow \psi)
			&\provable{\mbox{{\bf HK'}}} \exists \xi \varphi \rightarrow \psi, \\
		&\provable{\mbox{{\bf HK'}}} \forall \zeta (\varphi[\zeta/\xi] \rightarrow \psi) 
		\rightarrow (\exists \xi \varphi \rightarrow \psi)
	\end{align}
	となる.
	
\subsection{一般化規則は導かれる}
	{\bf HK}の量化公理に
	\begin{align}
		\forall x (\psi \rightarrow \varphi) \rightarrow
		(\forall x \psi \rightarrow \forall x \varphi)
	\end{align}
	を追加すれば,一般化規則は導かれる.
	
	\begin{screen}
		$\Gamma$を公理系(文の集合)とし,$\varphi$を式とし,$x$を$\varphi$に自由に現れる変項とする.
		このとき
		\begin{align}
			\Gamma \provable{\mbox{{\bf HK}}} \varphi
		\end{align}
		ならば
		\begin{align}
			\Gamma \provable{\mbox{{\bf HK}}} \forall x \varphi
		\end{align}
		である.
	\end{screen}
	
	\begin{sketch}
		$\varphi$が$\provable{\mbox{{\bf HK}}} \varphi$であるときは
		\begin{align}
			\provable{\mbox{{\bf HK}}} \forall x \varphi
		\end{align}
		が成立する.$\varphi$が三段論法によって示されるとき,つまり式$\psi$で
		\begin{align}
			\Gamma &\provable{\mbox{{\bf HK}}} \psi, \\
			\Sigma &\provable{\mbox{{\bf HK}}} \psi \rightarrow \varphi
		\end{align}
		を満たすものが取れるとき,$\psi$に$x$が自由に現れているかいないかで
		\begin{description}
			\item[case1] $\psi$に$x$が自由に現れていないとき,
				\begin{align}
					\Gamma \provable{\mbox{{\bf HK}}} \psi
				\end{align}
				かつ
				\begin{align}
					\Gamma \provable{\mbox{{\bf HK}}} \forall x (\psi \rightarrow \varphi)
				\end{align}
				
			\item[case2] $\psi$に$x$が自由に現れているとき,
				\begin{align}
					\Gamma \provable{\mbox{{\bf HK}}} \forall x \psi
				\end{align}
				かつ
				\begin{align}
					\Gamma \provable{\mbox{{\bf HK}}} \forall x (\psi \rightarrow \varphi)
				\end{align}
		\end{description}
		のいずれかのケースを一つ仮定する.
		\begin{description}
			\item[case1] 量化公理(UI)より
				\begin{align}
					\Gamma &\provable{\mbox{{\bf HK}}} \forall x (\psi \rightarrow \varphi), \\
					\Gamma &\provable{\mbox{{\bf HK}}} \forall x (\psi \rightarrow \varphi) \rightarrow (\psi \rightarrow \forall x \varphi), \\
					\Gamma &\provable{\mbox{{\bf HK}}} \psi \rightarrow \forall x \varphi
				\end{align}
				が成り立つので,$\Gamma \provable{\mbox{{\bf HK}}} \psi$の仮定と併せて
				\begin{align}
					\Gamma \provable{\mbox{{\bf HK}}} \forall x \varphi
				\end{align}
				が得られる.
				
			\item[case2]
				新しく追加した量化公理を用いれば,
				\begin{align}
					\Gamma &\provable{\mbox{{\bf HK}}} \forall x (\psi \rightarrow \varphi), \\
					\Gamma &\provable{\mbox{{\bf HK}}} \forall x (\psi \rightarrow \varphi) \rightarrow (\forall x \psi \rightarrow \forall x \varphi), \\
					\Gamma &\provable{\mbox{{\bf HK}}} \forall x \psi \rightarrow \forall x \varphi
				\end{align}
				が得られるので,$\Gamma \provable{\mbox{{\bf HK}}} \forall x \psi$の仮定と併せて
				\begin{align}
					\Gamma \provable{\mbox{{\bf HK}}} \forall x \varphi
				\end{align}
				が従う.
				\QED
		\end{description}
	\end{sketch}
	
	逆に,新しく追加した量化公理は一般化規則から導かれる.実際,量化規則(UE)より
	\begin{align}
		\forall x (\psi \rightarrow \varphi),\ \forall x \psi &\provable{\mbox{{\bf HK}}} \forall x (\psi \rightarrow \varphi), \\
		\forall x (\psi \rightarrow \varphi),\ \forall x \psi &\provable{\mbox{{\bf HK}}} \forall x (\psi \rightarrow \varphi)
		\rightarrow (\psi(t/x) \rightarrow \varphi(t/x)), \\
		\forall x (\psi \rightarrow \varphi),\ \forall x \psi &\provable{\mbox{{\bf HK}}} \psi(t/x) \rightarrow \varphi(t/x)
	\end{align}
	となり(ただし$t$は$\psi$と$\varphi$に現れない変項とする),また一方で
	\begin{align}
		\forall x (\psi \rightarrow \varphi),\ \forall x \psi &\provable{\mbox{{\bf HK}}} \forall x \psi, \\
		\forall x (\psi \rightarrow \varphi),\ \forall x \psi &\provable{\mbox{{\bf HK}}} \forall x \psi \rightarrow \psi(t/x), \\
		\forall x (\psi \rightarrow \varphi),\ \forall x \psi &\provable{\mbox{{\bf HK}}} \psi(t/x)
	\end{align}
	も成り立つから,併せて三段論法より
	\begin{align}
		\forall x (\psi \rightarrow \varphi),\ \forall x \psi \provable{\mbox{{\bf HK}}} \varphi(t/x)
	\end{align}
	となる.演繹定理より
	\begin{align}
		\forall x (\psi \rightarrow \varphi) \provable{\mbox{{\bf HK}}} \forall x \psi  \rightarrow \varphi(t/x)
	\end{align}
	となり,一般化規則より
	\begin{align}
		\forall x (\psi \rightarrow \varphi) \provable{\mbox{{\bf HK}}} \forall t ( \forall x \psi  \rightarrow \varphi(t/x))
	\end{align}
	となる.量化規則(UI)より
	\begin{align}
		\forall x (\psi \rightarrow \varphi) \provable{\mbox{{\bf HK}}} \forall t ( \forall x \psi  \rightarrow \varphi(t/x)) \rightarrow
		(\forall x \psi \rightarrow \forall x \varphi)
	\end{align}
	が成り立つので,
	\begin{align}
		\forall x (\psi \rightarrow \varphi) \provable{\mbox{{\bf HK}}}
		\forall x \psi \rightarrow \forall x \varphi
	\end{align}
	となる.演繹定理より
	\begin{align}
		\provable{\mbox{{\bf HK}}} \forall x (\psi \rightarrow \varphi)
		\rightarrow (\forall x \psi \rightarrow \forall x \varphi)
	\end{align}
	が得られる.
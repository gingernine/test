\section{第一イプシロン定理メモ}
	
	言語$L(EC)$及び$L(EC_{\varepsilon})$を高橋先生の資料と同じものとする.
	{\bf 主要論理式}\index{しゅようろんりしき@主要論理式}{\bf (principal formula)}とは
	\begin{align}
		A(t) \Longrightarrow A(\varepsilon x A)
	\end{align}
	なる形の$L(EC)$の式を指す.ここで$A$とは$L(EC)$の式であって,変項$x$が$A$に自由に現れていて,
	また$A$に自由に出現するのは$x$のみである.$A(t)$とは$A$における$x$の自由な出現を全て閉項$t$に置き換えた式であり,
	$A(\varepsilon x A)$とは$A$における$x$の自由な出現を全て項$\varepsilon x A$に置き換えた式である.
	このとき$\varepsilon x A$は$A(t) \Longrightarrow A(\varepsilon x A)$に{\bf 属している}という.
	
	$EC$の公理とはトートロジーだけである.トートロジーは$EC_{\varepsilon}$の公理でもあるが,
	これに加えて主要論理式も$EC_{\varepsilon}$の公理である.
	
	$\pi = (\varphi_{0},\varphi_{1},\cdots,\varphi_{n})$を$EC_{\varepsilon}$の文の列とするとき,
	{\bf $\pi$の主要論理式}や{\bf $\pi$に現れる主要論理式}とは主要論理式である$\varphi_{i}$を指す.
	また$\pi$の主要論理式に属している$\varepsilon$項を{\bf $\pi$の主要$\varepsilon$項}と呼ぶ.
	
\subsection{埋め込み定理}
	$A$を$L(PC_{\varepsilon})$の式とするとき,$A$を$L(EC_{\varepsilon})$の式に書き換える.
	\begin{align}
		x^{\varepsilon} &\rightarrow x \\
		(\in \tau \sigma)^{\varepsilon} &\rightarrow \in \tau^{\varepsilon} \sigma^{\varepsilon} \\
		(= \tau \sigma)^{\varepsilon} &\rightarrow = \tau^{\varepsilon} \sigma^{\varepsilon} \\
		(\rightharpoondown \varphi)^{\varepsilon} &\rightarrow \rightharpoondown \varphi^{\varepsilon} \\
		(\vee \varphi \psi)^{\varepsilon} &\rightarrow \vee \varphi^{\varepsilon} \psi^{\varepsilon} \\
		(\wedge \varphi \psi)^{\varepsilon} &\rightarrow \wedge \varphi^{\varepsilon} \psi^{\varepsilon} \\
		(\Longrightarrow \varphi \psi)^{\varepsilon} &\rightarrow \Longrightarrow \varphi^{\varepsilon} \psi^{\varepsilon} \\
		(\exists x \varphi)^{\varepsilon} &\rightarrow \varphi^{\varepsilon}(\varepsilon x \varphi^{\varepsilon}) \\
		(\forall x \varphi)^{\varepsilon} &\rightarrow \varphi^{\varepsilon}(\varepsilon x \rightharpoondown \varphi^{\varepsilon}) \\
		(\varepsilon x \psi)^{\varepsilon} &\rightarrow \varepsilon x \varphi^{\varepsilon}
	\end{align}
	
	$A$が$L(PC_{\varepsilon})$の式で,$x$が$A$に自由に現れて,
	かつ$A$に自由に現れているのが$x$のみであるとき,
	$A^{\varepsilon}$にも$x$が自由に現れて,かつ$A^{\varepsilon}$に
	自由に現れているのは$x$のみである.
	
	\begin{align}
		(\varphi[x/\tau])^{\varepsilon} \rightarrow \varphi^{\varepsilon}
		(\varphi^{\varepsilon}[x/\tau^{\varepsilon}]). \\
	\end{align}
	
	\begin{itembox}[c]{$PC_{\varepsilon}$の証明を$EC_{\varepsilon}$の証明に埋め込む}
		$A$を$L(PC_{\varepsilon})$の文とし,$PC_{\varepsilon} \vdash A$であるとする.
		このとき$EC_{\varepsilon} \vdash A^{\varepsilon}$である.
	\end{itembox}
	
	示すべきことは
	\begin{itemize}
		\item $A \in Ax(PC_{\varepsilon})$ならば$\vdash A^{\varepsilon}$であること.
			\begin{itemize}
				\item $\vdash A$ならば$\vdash A^{\varepsilon}$であること.
				\item $A$に$x$が自由に現れて,かつ自由に現れているのが$x$のみであるとき,
					\begin{align}
						\vdash A^{\varepsilon}(t^{\varepsilon}) \Longrightarrow A^{\varepsilon}(\varepsilon x A^{\varepsilon})
					\end{align}
					であること.
				\item $A$に$x$が自由に現れて,かつ自由に現れているのが$x$のみであるとき,
					\begin{align}
						\vdash A^{\varepsilon}(\varepsilon x \rightharpoondown A^{\varepsilon}) \Longrightarrow A^{\varepsilon}(t^{\varepsilon})
					\end{align}
					であること.
			\end{itemize}
		
		\item $PC_{\varepsilon} \vdash B$かつ$PC_{\varepsilon} \vdash B \Longrightarrow A$である$B$が取れるとき,
			$(B \Longrightarrow A)^{\varepsilon}$は$B^{\varepsilon} \Longrightarrow A^{\varepsilon}$なので
			$EC_{\varepsilon} \vdash B^{\varepsilon}$ならば
			$EC_{\varepsilon} \vdash A^{\varepsilon}$となる.
	\end{itemize}

\subsection{階数}
	$B(x,y,z)$を,変項$x,y,z$が,そしてこれらのみが自由に現れる
	$L(EC)$の式とする.このとき
	\begin{align}
		\exists x\, \exists y\, \exists z\, B(x,y,z) \label{fom:rank_of_epsilon_term_3}
	\end{align}
	に対して,$z$から順に$\varepsilon$項に変換していくと
	\begin{align}
		&\exists x\, \exists y\, B(x,y,\varepsilon z B(x,y,z)), \label{fom:rank_of_epsilon_term_1} \\
		&\exists x\, \, B(x,\varepsilon y B(x,y,\varepsilon z B(x,y,z)),\varepsilon z B(x,\varepsilon y B(x,y,\varepsilon z B(x,y,z)),z)) \label{fom:rank_of_epsilon_term_2}
	\end{align}
	となるが,最後に$\exists x$を無くすと式が長くなりすぎるので一旦止めておく.
	さて$z$に注目すれば,$B$に自由に現れていた$z$はまず
	\begin{align}
		\varepsilon z B(x,y,z)
	\end{align}
	に置き換えられる(\refeq{fom:rank_of_epsilon_term_1}).この時点では$x$と$y$は自由なままであるから,この$\varepsilon$項を
	\begin{align}
		e_{1}[x,y]
	\end{align}
	と略記する.次に$y$は
	\begin{align}
		\varepsilon y B(x,y,\varepsilon z B(x,y,z))
	\end{align}
	に置き換えられる(\refeq{fom:rank_of_epsilon_term_2})が,$e_{1}[x,y]$を使えば
	\begin{align}
		\varepsilon y B(x,y,e_{1}[x,y])
	\end{align}
	と書ける.この$\varepsilon$項でも$x$は自由なままであるから
	\begin{align}
		e_{2}[x]
	\end{align}
	と略記する.$e_{1}$と$e_{2}$を用いれば(\refeq{fom:rank_of_epsilon_term_2})の式は
	\begin{align}
		\exists x\, B\left(x,e_{2}[x],e_{1}[x,e_{2}[x]]\right)
	\end{align}
	と見やすく書き直せる.残る$\exists$を除去するには$x$を
	\begin{align}
		\varepsilon x B\left(x,e_{2}[x],e_{1}[x,e_{2}[x]]\right)
	\end{align}
	に置き換えれば良い.この$\varepsilon$項を$e_{3}$と書く.以上で
	(\refeq{fom:rank_of_epsilon_term_3})の式は$L(EC_{\varepsilon})$の式
	\begin{align}
		B\left(e_{3},e_{2}[e_{3}],e_{1}[e_{3},e_{2}[e_{3}]]\right)
	\end{align}
	に変換されたわけである.それはさておき,ここで考察するのは{\bf 項間の主従関係}である.
	$e_{2}[x]$は$x$のみによってコントロールされているのだから,
	$x$を司る$e_{3}$を親分だと思えば$e_{2}[x]$は$e_{3}$の直属の子分である.
	$e_{1}[x,y]$は$y$によってもコントロールされているので,
	$e_{1}[x,y]$とは$e_{2}[x]$の子分であり,すなわち$e_{3}$の子分の子分であって,
	この例において一番身分が低いわけである.
	
	$\varepsilon$項を構文解析して,それが何重の子分を従えているかを測った指標を
	{\bf 階数}{\bf (rank)}と呼ぶ.とはいえ直属の子分が複数いることもあり得るので,
	子分の子分の子分の子分…と次々に枝分かれしていく従属関係の中で,最も
	深いものを辿って階数を定めることにする.
	
	\begin{screen}
		\begin{metadfn}[従属]
			$\varepsilon x A$を$L(EC_{\varepsilon})$の$\varepsilon$項とし,
			$e$を$A$に現れる$L(EC_{\varepsilon})$の$\varepsilon$項とするとき,
			$x$が$e$に自由に現れているなら$e$は{\bf $\varepsilon x A$に従属している}
			\index{じゅうぞく@従属}{\bf (subordinate to $\varepsilon x A$)}という.
		\end{metadfn}
	\end{screen}
	
	はじめの例では,$e_{2}[x]$と$e_{1}[x,e_{2}[x]]$は共に$e_{3}$に従属しているし,
	$e_{1}[x,y]$は$e_{2}[x]$に従属している.
	$e_{1}[x,e_{2}[x]]$に従属している$\varepsilon$項は無いし,
	$e_{1}[x,y]$に従属している$\varepsilon$項も無い.
	
	\begin{screen}
		\begin{metadfn}[階数]
			$e$を$L(EC_{\varepsilon})$の$\varepsilon$項とするとき,
			$e$の{\bf 階数}\index{かいすう@階数}{\bf (rank)}を
			以下の要領で定義する.
			\begin{enumerate}
				\item $e$に従属する$\varepsilon$項が無いならば,$e$の階数を$1$とする.
				\item $e$に従属する$\varepsilon$項があるならば,$e$に従属する
					$\varepsilon$項の階数の最大値に$1$を足したものを$e$の階数とする.
			\end{enumerate}
			また$e$の階数を$rk(e)$と書く.
		\end{metadfn}
	\end{screen}
	
	実際に$L(EC_{\varepsilon})$の全ての$\varepsilon$項に対して階数が定まっている.
	(構造的帰納法について準備不足だが,直感的に次の説明は妥当である...)
	\begin{description}
		\item[step1] $e$が$L(EC)$の式で作られた$\varepsilon$項ならば$e$の階数は$1$である.
		
		\begin{comment}
		\item[step3] 項$\tau_{1},\cdots,\tau_{n}$のそれぞれに対して,
			その全ての部分$\varepsilon$項に階数が定まっていれば,
			$f$を$n$項関数として,$f\tau_{1}\cdots\tau_{n}$の階数は
			$rk(\tau_{1}),\cdots,rk(\tau_{n})$の中の最大値である.
			というのも,$f\tau_{1}\cdots\tau_{n}$に現れる$\varepsilon$項は
			$\tau_{1},\cdots,\tau_{n}$のいずれかの部分項になっているためである.
			
		\item[step4] 項$\tau_{1},\cdots,\tau_{n}$のそれぞれに対して,
			その全ての部分$\varepsilon$項に階数が定まっていれば,
			$p$を$n$項述語として,$p\tau_{1}\cdots\tau_{n}$の階数は
			$rk(\tau_{1}),\cdots,rk(\tau_{n})$の中の最大値である.
		
		\item[step5] 式$\varphi$と$\psi$のそれぞれに対して,
			その全ての部分$\varepsilon$項に階数が定まっていれば,
			\begin{align}
				rk(\rightharpoondown \varphi) &\coloneqq rk(\varphi), \\
				rk(\vee \varphi \psi) &\coloneqq \max\{rk(\varphi),rk(\psi)\}, \\
				rk(\wedge \varphi \psi) &\coloneqq \max\{rk(\varphi),rk(\psi)\}, \\
				rk(\Longrightarrow \varphi \psi) 
				&\coloneqq \max\{rk(\varphi),rk(\psi)\}, \\
			\end{align}
			である.というのも,左辺の式に現れる$\varepsilon$項は
			$\varphi$か$\psi$の少なくとも一方に現れているからである.
		\end{comment}
		
		\item[step2] $e$に従属している全ての$\varepsilon$項に対して階数が定まっているならば,
			$e$の階数は定義通りに定めることが出来る.
	\end{description}
	
	\begin{screen}
		\begin{metathm}[階数定理]
			$e$を$L(EC_{\varepsilon})$の$\varepsilon$項とし,
			$s$と$t$を,$e$の中で束縛されている変項がどれも自由に現れない$L(EC_{\varepsilon})$の項とする.
			このとき,$e$に現れる$s$の一つを$t$に置き換えた式を$e^{t}$とすれば
			\begin{align}
				rk(e) = rk(e^{t})
			\end{align}
			が成り立つ.$e$に$s$が現れなければ$e^{t}$は$e$とする.
		\end{metathm}
	\end{screen}
	
	\begin{metaprf}\mbox{}
		\begin{description}
			\item[step1]
				$e$の中に$\varepsilon$項$u$が現れているとして,
				$u$に(もし現れているなら)現れる$s$が$t$に置き換わった項を$u^{t}$と書く.
				$u$が$e$に従属していないとき,$u^{t}$は$e$に従属しない.実際,$e$を
				\begin{align}
					\varepsilon x A
				\end{align}
				なる$\varepsilon$項だとして,$x$は$t$に自由に現れないので
				$u^{t}$にも$x$は自由に現れない.
				
				$u$が$e$に従属している場合,$u^{t}$の階数は
				$u^{t}$に従属する$\varepsilon$項によって定まるのだから,
				$u^{t}$の階数の如何は$u$に従属する$\varepsilon$項の階数が
				置換によって変動するか否かにかかっている.
				
				$u$に従属する$\varepsilon$項の階数についても,それに従属する
				$\varepsilon$項の階数が置換によって変動するか否かで決まる.
				
				従属する$\varepsilon$項を辿っていけば,いずれは従属する$\varepsilon$項
				を持たない$\varepsilon$項に行きつくのであるから,
				$e$には従属する$\varepsilon$項が現れないとして
				$e$と$e^{t}$の階数が等しいことを示せば良い.それは次段で示す.
				
			\item[step2]
				$e$に従属する$\varepsilon$項が無ければ,
				$e^{t}$に従属する$\varepsilon$項も無い.
				なぜならば,$s$にも$t$にも$x$は自由に現れていないのであり,
				仮に$e$に現れる$\varepsilon$項の中に$s$があったとしても,
				それが$t$へ置き換わったところで$e$には従属しないからである.
				\QED
		\end{description}
	\end{metaprf}
	
	$\varepsilon x A(x)$と$\varepsilon y A(y)$のランクは同じ?
	
	\begin{screen}
		\begin{metathm}[置換定理]
			$\pi$を$L(EC_{\varepsilon})$の証明とし,
			$e$を,$\pi$の主要$\varepsilon$項の中で階数が最大であって,かつ
			階数が最大の$\pi$の主要$\varepsilon$項の中で極大であるものとする.また
			$B(s) \Longrightarrow B(\varepsilon y B)$を$\pi$の主要論理式とし,
			$e$と$\varepsilon y B$は別物であるとする.そして,$B(s) \Longrightarrow B(\varepsilon y B)$に現れる
			$e$を全て閉項$t$に置き換えた式を$C$とする.このとき,
			\begin{description}
				\item[(1)] $C$は主要論理式である.$C$に属する$\varepsilon$項を$e'$と書く.
				\item[(2)] $rk(\varepsilon y B) = rk(e')$が成り立つ.
				\item[(3)] $rk(\varepsilon y B) = rk(e)$ならば$\varepsilon y B$と$e'$は一致する.
			\end{description}
		\end{metathm}
	\end{screen}
	
	\begin{metaprf}\mbox{}
		\begin{description}
			\item[step1]
				$B(s)$ (或いは$B(\varepsilon y B)$)とは,
				$B$で自由に現れる$y$を$s$ (或いは$\varepsilon y B$)で置き換えた式である.
				$y$から代わった$s$ (或いは$\varepsilon y B$)の少なくとも一つを部分項として含む形で
				$e$が$B(s)$ (或いは$B(\varepsilon y B)$)に出現しているとする.
				
				実はこれは起こり得ない.もし起きたとすると,$e$に現れる$s$ (或いは$\varepsilon y B$)
				を元の$y$に戻した項を$e'$とすれば,
				$e'$には$y$が自由に現れるので(そうでないと$y$は$s$
				(或いは$\varepsilon y B$)に置き換えられない),$e'$は
				$y$とは別の変項$x$と適当な式$A$によって
				\begin{align}
					\varepsilon x A
				\end{align}
				なる形をしている.つまり$e'$は$\varepsilon y B$に従属していることになり
				\footnote{
					$e'$が$\varepsilon$項であって$B$に現れることの証明.
				}
				,階数定理と併せて
				\begin{align}
					rk(e) = rk(e') < rk(\varepsilon y B)
				\end{align}
				が成り立ってしまう.しかしこれは$rk(e)$が最大であることに矛盾する.
				
			\item[step2] $rk(\varepsilon y B) = rk(\pi)$ならば$B$に$e$は現れない.なぜならば,
				$e$は階数が$rk(\pi)$である$\pi$の主要$\varepsilon$項の中で極大であるからである.
				$\varepsilon y B$にも$e$は現れず,前段の結果より$B(\varepsilon y B)$に$e$が現れることもない.
				ゆえに,$s$に(もし現れるなら)現れる$e$を$t$に置換した項を$s'$とすれば,$C$は
				\begin{align}
					B(s') \Longrightarrow B(\varepsilon y B)
				\end{align}
				となる.
			
			\item[step3]
				$rk(\varepsilon y B) < rk(\pi)$である場合
				\begin{align}
					rk(\varepsilon y B) = rk(e')
				\end{align}
				が成り立つことを示す.$B$に$e$が現れないならば$e'$は$\varepsilon y B$に一致する.
				$B$に$e$が現れる場合,$B$に現れる$e$を$t$に置き換えた式を$B^{t}$とする.
				このとき階数定理より
				\begin{align}
					rk(B) = rk(B^{t})
				\end{align}
				となる.ゆえに
				\begin{align}
					rk(\varepsilon y B) = rk(B) + 1 = rk(B^{t}) + 1 = rk(\varepsilon y B^{t})
				\end{align}
				となる.
				\QED
		\end{description}
	\end{metaprf}
	
\subsection{アイデア}
	$L(EC)$の公理系を$AX(EC)$と書く.
	$L(EC_{\varepsilon})$の公理系を$AX(EC_{\varepsilon})$と書く.
	$L(PC_{\varepsilon})$の公理系を$AX(PC_{\varepsilon})$と書く.
	
	証明の中から$A[t] \rightarrow A[e]$なる主要論理式は排除されるが,
	$A[u] \rightarrow A[e]$なる主要論理式は残っている(かもしれない).
	しかし,$e$が属する主要論理式はたしかに減る.
	別の,最高階数の$\varepsilon$項が属する主要論理式は増えたりしている場合があるっけ?
	でも証明に現れる主要論理式は増えても,それらに属している最高階数の$\varepsilon$項の個数は増えない.
	だから,$e$が属する主要論理式をまず排除すれば最高階数の$\varepsilon$項の個数は一つ減る.
	次に,別の最高階数の$\varepsilon$項を一つ選んで,それが属する主要論理式を一本ずつ排除していく.
	
	\begin{itembox}[l]{第一イプシロン定理の流れ}
		\begin{itemize}
			\item $B$を$EC$の式とし,$B$が$AX(PC_{\varepsilon})$から証明可能であるとする.
			\item このとき$AX(EC_{\varepsilon})から$$B$への証明$\pi$が得られる.
			\item $e$を,$\pi$の主要$\varepsilon$項のうち階数が最大であって,かつ
				その階数を持つ$\pi$の主要$\varepsilon$項の中で次数が最大であるものとする.
			\item $e$が属する$\pi$の主要論理式の一つ$A(t) \Longrightarrow A(e)$を取る.
			\item $\pi$をベースにして,$A(t) \Longrightarrow A(e)$を用いずに
				$AX(EC_{\varepsilon})$から$B$への証明$\pi'$を構成する.このとき以下が満たされる.
				\begin{enumerate}
					\item $A(t) \Longrightarrow A(e)$以外の主要論理式は,$\pi'$に残っているし,
						またそれらが階数不変で生まれ変わったもの(置換定理による)は$\pi'$の主要論理式となる.
						$A(t) \Longrightarrow A(e)$のみ消える.
						他に主要論理式は現れない.
					
					\item 特に,階数$rk(\pi)$の主要$\varepsilon$項は増えない.
						
					\item 特に,$e$が属する主要論理式は$A(t) \Longrightarrow A(e)$の分だけ消えて,
						新しく増えることはない.
				\end{enumerate}
				
			\item 証明$\pi$の主要$\varepsilon$項の階数の最大値を$rk(\pi)$とする.
				また主要論理式に属する$\varepsilon$項の階数を,その主要論理式の階数と呼ぶことにする.
				前段の操作を続けていけば,まずは階数$rk(\pi)$の主要論理式を全く用いない
				$B$への証明$\pi_{1}$が得られる.このとき$rk(\pi_{1})$は
				$rk(\pi)$よりも小さい.同様にして階数$rk(\pi_{1})$の主要論理式を全く用いない
				$B$への証明$\pi_{2}$が得られる.もちろん$rk(\pi_{2})$は
				$rk(\pi_{1})$よりも小さい.これを繰り返していけば,いずれは主要論理式を全く用いない
				$B$への証明$\pi^{\ast}$が得られる.$\pi^{\ast}$にはトートロジーか
				モーダスポンネスで導かれる式しかない.
				あとは,$\pi^{\ast}$に現れる$\varepsilon$項を$EC$の項に置き換えれば,その式の列は
				$EC$から$B$への証明となっている.
		\end{itemize}
	\end{itembox}
	
	$\pi$を$\varphi_{0},\varphi_{1},\cdots,\varphi_{n}$とし,
	$\varphi_{0},\varphi_{1},\cdots,\varphi_{n}$に現れる$e$を$t$に置き換えた式を
	\begin{align}
		\tilde{\varphi}_{0},\ \tilde{\varphi}_{1},\cdots, \tilde{\varphi}_{n}
	\end{align}
	と書く($e$は,どれかの項の部分項であるときも置き換える).
	このとき,任意の$0 \leq i \leq n$で
	\begin{enumerate}
		\item $\varphi_{i}$がトートロジーなら$\tilde{\varphi}_{i}$もトートロジーである.
		\item $\varphi_{i}$が主要論理式で,$e$が$\varphi_{i}$の主要項であるならば,
			$\tilde{\varphi}_{i}$は$A(u) \Longrightarrow A(t)$なる形の式である
			\footnotemark.
		\item $\varphi_{i}$が主要論理式で,$e$が$\varphi_{i}$の主要項ではないならば,
			$\tilde{\varphi}_{i}$も主要論理式である.
	\end{enumerate}
	
	\footnotetext{
		$\varepsilon x A$と$\varepsilon y B$が記号列として一致すれば,
		$x$と$y$は一致するし,式$A$と式$B$も一致するので
		$A(\varepsilon x A)$と$B(\varepsilon y B)$も記号列として一致する.
	}
	
	$\varphi$が$A(t) \Longrightarrow A(e)$でない$EC_{\varepsilon}$の公理ならば,
	$\tilde{\varphi}_{i}$と$\tilde{\varphi}_{i+1}$の間に
	\begin{align}
		&\tilde{\varphi}_{i} \Longrightarrow 
		\left( A(t) \Longrightarrow \tilde{\varphi}_{i} \right), \\
		&A(t) \Longrightarrow \tilde{\varphi}_{i}
	\end{align}
	を挿入する.$\varphi_{i}$が$\varphi_{j}$と$\varphi_{k}$からモーダスポンネスで得られる場合は,
	$\tilde{\varphi}_{i}$を
	\begin{align}
		&\left( A(t) \Longrightarrow \tilde{\varphi}_{j} \right)
		\Longrightarrow \left[ \left( A(t) \Longrightarrow 
		\left( \tilde{\varphi}_{j}\Longrightarrow \tilde{\varphi}_{i} \right) \right)
		\Longrightarrow \left( A(t) \Longrightarrow \tilde{\varphi}_{i} \right) \right], \\
		&\left( A(t) \Longrightarrow 
		\left( \tilde{\varphi}_{j}\Longrightarrow \tilde{\varphi}_{i} \right) \right)
		\Longrightarrow \left( A(t) \Longrightarrow \tilde{\varphi}_{i} \right), \\
		&A(t) \Longrightarrow \tilde{\varphi}_{i}
	\end{align}
	で置き換える.すると,$A(t) \Longrightarrow A(e)$を使わない
	$EC_{\varepsilon}$から$A(t) \Longrightarrow B$への証明が得られる.
	$\varphi_{i}$が$e$が属する主要論理式$A(s) \Longrightarrow A(e)$であるときは,
	$\tilde{\varphi}_{i}$とは
	\begin{align}
		A(s') \Longrightarrow A(t)
	\end{align}
	なる形の式であるが
	\footnote{
		$x$を$A$に現れている自由な変項とすれば,$e$とは$\varepsilon x A$のことであるし,
		$A(\varepsilon x A)$とは$A$に自由に現れる$x$を$\varepsilon x A$に置換した式である.
		$A$には$\varepsilon x A$は現れていないので,というのも$\varepsilon x A$が登場するのは
		$A$が作られた後であるからだが,$A(e)$に現れる$e$を$t$に変換した式は
		$A(t)$になる.同様に,$A(s)$に$e$が現れるとすれば,その$e$は$y$に代入された$s$の
		部分項でしかありえない.すなわち,$A(s)$に現れる$e$を$t$で置換した式は,
		$s'$を$s$に現れる$e$を$t$に変換した項として ($s$に$e$が現れなければ$s'$は$s$である)
		$A(s')$となるわけである.
	},$\tilde{\varphi}_{i}$を
	\begin{align}
		&A(t) \Longrightarrow (A(s') \Longrightarrow A(t)), \\
		&A(s') \Longrightarrow A(t)
	\end{align}
	で置き換える.
	
	同様に$A(t) \Longrightarrow A(e)$を使わない$EC_{\varepsilon})$から
	$\rightharpoondown A(t) \Longrightarrow B$への証明を構成する.
	今度は$\pi$に現れる$e$を$t$に置き換える必要はない.
	$\varphi_{i}$が$A(t) \Longrightarrow A(e)$でない$EC_{\varepsilon}$の公理ならば,
	$\varphi_{i}$と$\varphi_{i+1}$の間に
	\begin{align}
		&\varphi_{i} \Longrightarrow (\rightharpoondown A(t) \Longrightarrow \varphi_{i}), \\
		&\rightharpoondown A(t) \Longrightarrow \varphi_{i}
	\end{align}
	を挿入する.$\varphi_{i}$が$\varphi_{j}$と$\varphi_{k}$からモーダスポンネスで得られる場合は,
	$\varphi_{i}$を
	\begin{align}
		&(\rightharpoondown A(t) \Longrightarrow \varphi_{j}) \Longrightarrow
		[(\rightharpoondown A(t) \Longrightarrow 
		(\varphi_{j}\Longrightarrow \varphi_{i}))
		\Longrightarrow (\rightharpoondown A(t) \Longrightarrow \varphi_{i})], \\
		&(\rightharpoondown A(t) \Longrightarrow 
		(\varphi_{j} \Longrightarrow \varphi_{i}))
		\Longrightarrow (\rightharpoondown A(t) \Longrightarrow \varphi_{i}), \\
		&\rightharpoondown A(t) \Longrightarrow \varphi_{i}
	\end{align}
	で置き換える.$\varphi_{i}$が$A(t) \Longrightarrow A(e)$であるときは,$\varphi_{i}$を
	\begin{align}
		\rightharpoondown A(t) \Longrightarrow (A(t) \Longrightarrow A(e))
	\end{align}
	で置き換える.
	
	以上で$A(t) \Longrightarrow B$と$\rightharpoondown A(t) \Longrightarrow B$に対して
	$A(t) \Longrightarrow A(e)$を用いない$EC_{\varepsilon}$からの証明が得られた.後はこれに
	\begin{align}
		&(A(t) \Longrightarrow B) \Longrightarrow
		((\rightharpoondown A(t) \Longrightarrow B) \Longrightarrow
		((A(t) \Longrightarrow B) \wedge (\rightharpoondown A(t) \Longrightarrow B))), \\
		&(\rightharpoondown A(t) \Longrightarrow B) \Longrightarrow
		((A(t) \Longrightarrow B) \wedge (\rightharpoondown A(t) \Longrightarrow B)), \\
		&(A(t) \Longrightarrow B) \wedge (\rightharpoondown A(t) \Longrightarrow B), \\
		&((A(t) \Longrightarrow B) \wedge (\rightharpoondown A(t) \Longrightarrow B))
		\Longrightarrow ((A(t) \vee \rightharpoondown A(t)) \Longrightarrow B), \\
		&(A(t) \vee \rightharpoondown A(t)) \Longrightarrow B, \\
		&A(t) \vee \rightharpoondown A(t), \\
		&B
	\end{align}
	を追加すれば,$A(t) \Longrightarrow A(e)$を用いない$EC_{\varepsilon}$から$B$への証明となる.
	
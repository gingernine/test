\section{第一イプシロン定理}
	$A$を$L(PC_{\varepsilon})$の式とするとき,$A$を$L(EC_{\varepsilon})$の式に書き換える.
	\begin{align}
		x^{\varepsilon} &\rightarrow x \\
		(\in \tau \sigma)^{\varepsilon} &\rightarrow \in \tau^{\varepsilon} \sigma^{\varepsilon} \\
		(= \tau \sigma)^{\varepsilon} &\rightarrow = \tau^{\varepsilon} \sigma^{\varepsilon} \\
		(\rightharpoondown \varphi)^{\varepsilon} &\rightarrow \rightharpoondown \varphi^{\varepsilon} \\
		(\vee \varphi \psi)^{\varepsilon} &\rightarrow \vee \varphi^{\varepsilon} \psi^{\varepsilon} \\
		(\wedge \varphi \psi)^{\varepsilon} &\rightarrow \wedge \varphi^{\varepsilon} \psi^{\varepsilon} \\
		(\Longrightarrow \varphi \psi)^{\varepsilon} &\rightarrow \Longrightarrow \varphi^{\varepsilon} \psi^{\varepsilon} \\
		(\exists x \varphi)^{\varepsilon} &\rightarrow \varphi^{\varepsilon}(\varepsilon x \varphi^{\varepsilon}) \\
		(\forall x \varphi)^{\varepsilon} &\rightarrow \varphi^{\varepsilon}(\varepsilon x \rightharpoondown \varphi^{\varepsilon}) \\
		(\varepsilon x \psi)^{\varepsilon} &\rightarrow \varepsilon x \varphi^{\varepsilon}
	\end{align}
	
	$A$が$L(PC_{\varepsilon})$の式で,$x$が$A$に自由に現れて,
	かつ$A$に自由に現れているのが$x$のみであるとき,
	$A^{\varepsilon}$にも$x$が自由に現れて,かつ$A^{\varepsilon}$に
	自由に現れているのは$x$のみである.
	
	\begin{align}
		(\varphi[x/\tau])^{\varepsilon} \rightarrow \varphi^{\varepsilon}
		(\varphi^{\varepsilon}[x/\tau^{\varepsilon}]). \\
	\end{align}
	
	\begin{itembox}[c]{$PC_{\varepsilon}$の証明を$EC_{\varepsilon}$の証明に埋め込む}
		$A$を$L(PC_{\varepsilon})$の文とし,$PC_{\varepsilon} \vdash A$であるとする.
		このとき$EC_{\varepsilon} \vdash A^{\varepsilon}$である.
	\end{itembox}
	
	示すべきことは
	\begin{itemize}
		\item $A \in Ax(PC_{\varepsilon})$ならば$\vdash A^{\varepsilon}$であること.
			\begin{itemize}
				\item $\vdash A$ならば$\vdash A^{\varepsilon}$であること.
				\item $A$に$x$が自由に現れて,かつ自由に現れているのが$x$のみであるとき,
					\begin{align}
						\vdash A^{\varepsilon}(t^{\varepsilon}) \Longrightarrow A^{\varepsilon}(\varepsilon x A^{\varepsilon})
					\end{align}
					であること.
				\item $A$に$x$が自由に現れて,かつ自由に現れているのが$x$のみであるとき,
					\begin{align}
						\vdash A^{\varepsilon}(\varepsilon x \rightharpoondown A^{\varepsilon}) \Longrightarrow A^{\varepsilon}(t^{\varepsilon})
					\end{align}
					であること.
			\end{itemize}
		
		\item $PC_{\varepsilon} \vdash B$かつ$PC_{\varepsilon} \vdash B \Longrightarrow A$である$B$が取れるとき,
			$(B \Longrightarrow A)^{\varepsilon}$は$B^{\varepsilon} \Longrightarrow A^{\varepsilon}$なので
			$EC_{\varepsilon} \vdash B^{\varepsilon}$ならば
			$EC_{\varepsilon} \vdash A^{\varepsilon}$となる.
	\end{itemize}
	
	\begin{screen}
		\begin{thm}[置換補題]
			高橋先生のと同じ.
		\end{thm}
	\end{screen}
	
	\begin{sketch}\mbox{}
		\begin{description}
			\item[Case1]
				$B(s)$ (或いは$B(\varepsilon y B)$)とは,
				$B$で自由に現れる$y$を$s$ (或いは$\varepsilon y B$)で置き換えた式である.
				$y$から代わった$s$ (或いは$\varepsilon y B$)の少なくとも一つを部分項として含む形で
				$e$が$B(s)$ (或いは$B(\varepsilon y B)$)に出現しているとする.
				
				実はこれは起こり得ない.もし起きたとすると,$e$に現れる$s$ (或いは$\varepsilon y B$)
				を元の$y$に戻した項を$e'$とすれば,$e'$は$B$に現れる$\varepsilon$項であって
				\footnotemark,$e'$には$y$が自由に現れるので,
				\begin{align}
					rk(e) = rk(e') < rk(\varepsilon y B)
				\end{align}
				が成り立ってしまう.しかしこれは$rk(e)$が最大であることに矛盾する.
				
			\item[case2] $rk(\varepsilon y B) = rk(\pi)$ならば$B$に$e$は現れない.なぜならば,
				$e$は階数が$rk(\pi)$である$\pi$の主要$\varepsilon$項の中で極大であるからである.
				$\varepsilon y B$にも$e$は現れず,前段の結果より$B(\varepsilon y B)$に$e$が現れることもない.
				ゆえに,$s$に現れる$e$を$t$に置換した項を$s'$とすれば,$C$は
				\begin{align}
					B(s') \Longrightarrow B(\varepsilon y B)
				\end{align}
				となる.
			
			\item[Case3]
				$rk(\varepsilon y B) < rk(\pi)$である場合
				\begin{align}
					rk(\varepsilon y B) = rk(e')
				\end{align}
				が成り立つことを示す.$B$に$e$が現れないならば$e'$は$\varepsilon y B$に一致する.
				$B$に$e$が現れる場合,$B$に現れる$e$を$t$に置き換えた式を$B^{t}$とする.
				$B$に現れる$\varepsilon$項で,その中に$y$が自由に現れているものは
				$B^{t}$にもそのまま残っているから
				\begin{align}
					rk(B) = rk(B^{t})
				\end{align}
				となる.ゆえに
				\begin{align}
					rk(\varepsilon y B) = rk(B) + 1 = rk(B^{t}) + 1 = rk(\varepsilon y B^{t})
				\end{align}
				となる.
				\QED
		\end{description}
	\end{sketch}
	
	\footnotetext{
		$e'$が$\varepsilon$項であって$B$に現れることの証明.
	}
	
\subsection{アイデア}
	言語$L(EC)$及び$L(EC_{\varepsilon})$を高橋先生の資料と同じものとする.
	{\bf 主要論理式}\index{しゅようろんりしき@主要論理式}{\bf (principal formula)}とは
	\begin{align}
		A(t) \Longrightarrow A(\varepsilon x A)
	\end{align}
	なる形の$L(EC)$の式のこと.ここで$A$とは$L(EC)$の式であって,変項$x$が$A$に自由に現れていて,
	$A(t)$とは$A$における$x$の自由な出現を全て項$t$に置き換えた式,
	$A(\varepsilon x A)$とは$A$における$x$の自由な出現を全て項$\varepsilon x A$に置き換えた式である.
	また$\varepsilon x A$は$A(t) \Longrightarrow A(\varepsilon x A)$に{\bf 属している}という.
	
	$EC$の公理とはトートロジーだけである.トートロジーは$EC_{\varepsilon}$の公理でもあるが,
	これに加えて主要論理式も$EC_{\varepsilon}$の公理である.
	
	$\pi = (\varphi_{0},\varphi_{1},\cdots,\varphi_{n})$を$EC_{\varepsilon}$における証明とするとき,
	{\bf $\pi$の主要論理式}や{\bf $\pi$に現れる主要論理式}とは主要論理式であるいずれかの$\varphi_{i}$を指す.
	また$\pi$の主要論理式に属している$\varepsilon$項を{\bf $\pi$の主要$\varepsilon$項}と呼ぶ.
	
	\begin{itembox}[l]{第一イプシロン定理の流れ}
		\begin{itemize}
			\item $B$を$EC$の式とし,$B$が$PC_{\varepsilon}$から証明可能であるとする.
			\item このとき$EC_{\varepsilon}から$$B$への証明$\pi$が得られる.
			\item $e$を,$\pi$の主要$\varepsilon$項のうち階数が最大であって,かつ
				その階数を持つ$\pi$の主要$\varepsilon$項の中で次数が最大であるものとする.
			\item $e$が属する$\pi$の主要論理式の一つ$A(t) \Longrightarrow A(e)$を取る.
			\item $\pi$をベースにして,$A(t) \Longrightarrow A(e)$を用いずに
				$EC_{\varepsilon}$から$B$への証明$\pi'$を構成する.このとき以下が満たされる.
				\begin{enumerate}
					\item $A(t) \Longrightarrow A(e)$を除く$\pi$の主要論理式は
						$\pi'$の主要論理式である.
					\item また$e$が属する主要論理式については,それが$\pi'$の主要論理式であるならば$\pi$の主要論理式
						でもある.つまり,直感的に書けば
						\begin{align}
							&\Set{\varphi}{\mbox{$\varphi$は$e$が属する$\pi'$の主要論理式}} \\
							&= \Set{\varphi}{\mbox{$\varphi$は$e$が属する$\pi$の主要論理式}} \backslash \{A(t) \Longrightarrow A(e)\}
						\end{align}
						が成り立つということであって,$e$が属する主要論理式は減る一方である.
						$\pi$の主要論理式で$e$が属しているものが$A(t) \Longrightarrow A(e)$のみ
						であるならば,$\pi'$には$e$が属する主要論理式は現れない.
						
					\item $e$が属する主要論理式が$\pi'$にも残っている場合,
						$e$は$\pi'$の主要$\varepsilon$項の中も階数が最大であって,
						かつその階数を持つ$\pi'$の主要$\varepsilon$項の中で次数も最大である.
					
					\item $\pi'$の主要$\varepsilon$項のうち,$e$と階数が同じであるものは
						$\pi$の主要$\varepsilon$項でもあった.
				\end{enumerate}
				
			\item 上の操作を続けていけば,まずは階数$r_{0} = rk(\pi)$の主要論理式を全て
				$Ax(EC_{\varepsilon})$から除外した公理系$Ax(EC_{\varepsilon})^{r_{0}}$
				から$B$への証明$\pi_{1}$が得られる.このとき
				\begin{align}
					r_{1} \defeq rk(\pi_{1}) < r_{0}
				\end{align}
				である.同様に階数$r_{1}$の主要論理式を全て
				$Ax(EC_{\varepsilon})^{r_{0}}$から除外した公理系
				$Ax(EC_{\varepsilon})^{r_{1}}$から$B$への証明$\pi_{2}$が得られる.
				繰り返せば,いずれは階数$1$以上の主要論理式を全て
				$Ax(EC_{\varepsilon})$から除外した公理系$Ax(EC_{\varepsilon})^{\ast}$
				から$B$への証明$\pi^{\ast}$が得られるが,$Ax(EC_{\varepsilon})^{\ast}$とは
				$Ax(EC_{\varepsilon})$から主要論理式を全て取り除いた公理系であるから,
				$\pi^{\ast}$はトートロジーとモーダスポンネスだけからなる証明である.
				あとは,そこに現れる$\varepsilon$項を$EC$の項に置き換えれば,その式の列は
				$Ax(EC)$から$B$への証明となっている.
		\end{itemize}
	\end{itembox}
	
	$\pi$を$\varphi_{0},\varphi_{1},\cdots,\varphi_{n}$とし,
	$\varphi_{0},\varphi_{1},\cdots,\varphi_{n}$に現れる$e$を$t$に置き換えた式を
	\begin{align}
		\tilde{\varphi}_{0},\ \tilde{\varphi}_{1},\cdots, \tilde{\varphi}_{n}
	\end{align}
	と書く($e$は,どれかの項の部分項であるときも置き換える?).
	このとき,任意の$0 \leq i \leq n$で
	\begin{enumerate}
		\item $\varphi_{i}$がトートロジーなら$\tilde{\varphi}_{i}$もトートロジーである.
		\item $\varphi_{i}$が主要論理式で,$e$が$\varphi_{i}$の主要項であるならば,
			$\tilde{\varphi}_{i}$は$A(u) \Longrightarrow A(t)$なる形の式である
			\footnotemark.
		\item $\varphi_{i}$が主要論理式で,$e$が$\varphi_{i}$の主要項ではないならば,
			$\tilde{\varphi}_{i}$も主要論理式である.
	\end{enumerate}
	
	\footnotetext{
		$A$と$B$を式とし,$A$には$x$が自由に現れ,$B$には$y$が自由に現れ,
		かつ$A$に自由に現れるのは$x$のみで,$B$に自由に現れるのは$y$のみであるとする.
		このとき,$\varepsilon x A$と$\varepsilon y B$が記号列として一致すれば,
		$A(\varepsilon x A)$と$B(\varepsilon y B)$も記号列として一致する.
	}
	
	$\varphi$が$A(t) \Longrightarrow A(e)$でない$EC_{\varepsilon}$の公理ならば,
	$\tilde{\varphi}_{i}$と$\tilde{\varphi}_{i+1}$の間に
	\begin{align}
		&\tilde{\varphi}_{i} \Longrightarrow 
		\left( A(t) \Longrightarrow \tilde{\varphi}_{i} \right), \\
		&A(t) \Longrightarrow \tilde{\varphi}_{i}
	\end{align}
	を挿入する.$\varphi_{i}$が$\varphi_{j}$と$\varphi_{k}$からモーダスポンネスで得られる場合は,
	$\tilde{\varphi}_{i}$を
	\begin{align}
		&\left( A(t) \Longrightarrow \tilde{\varphi}_{j} \right)
		\Longrightarrow \left[ \left( A(t) \Longrightarrow 
		\left( \tilde{\varphi}_{j}\Longrightarrow \tilde{\varphi}_{i} \right) \right)
		\Longrightarrow \left( A(t) \Longrightarrow \tilde{\varphi}_{i} \right) \right], \\
		&\left( A(t) \Longrightarrow 
		\left( \tilde{\varphi}_{j}\Longrightarrow \tilde{\varphi}_{i} \right) \right)
		\Longrightarrow \left( A(t) \Longrightarrow \tilde{\varphi}_{i} \right), \\
		&A(t) \Longrightarrow \tilde{\varphi}_{i}
	\end{align}
	で置き換える.すると,$A(t) \Longrightarrow A(e)$を使わない
	$EC_{\varepsilon}$から$A(t) \Longrightarrow B$への証明が得られる.
	$\varphi_{i}$が$e$が属する主要論理式$A(s) \Longrightarrow A(e)$であるときは,
	$\tilde{\varphi}_{i}$とは
	\begin{align}
		A(s') \Longrightarrow A(t)
	\end{align}
	なる形の式であるが
	\footnote{
		$x$を$A$に現れている自由な変項とすれば,$e$とは$\varepsilon x A$のことであるし,
		$A(\varepsilon x A)$とは$A$に自由に現れる$x$を$\varepsilon x A$に置換した式である.
		$A$には$\varepsilon x A$は現れていないので,というのも$\varepsilon x A$が登場するのは
		$A$が作られた後であるからだが,$A(e)$に現れる$e$を$t$に変換した式は
		$A(t)$になる.同様に,$A(s)$に$e$が現れるとすれば,その$e$は$y$に代入された$s$の
		部分項でしかありえない.すなわち,$A(s)$に現れる$e$を$t$で置換した式は,
		$s'$を$s$に現れる$e$を$t$に変換した項として ($s$に$e$が現れなければ$s'$は$s$である)
		$A(s')$となるわけである.
	},$\tilde{\varphi}_{i}$を
	\begin{align}
		&A(t) \Longrightarrow (A(s') \Longrightarrow A(t)), \\
		&A(s') \Longrightarrow A(t)
	\end{align}
	で置き換える.
	
	同様に$A(t) \Longrightarrow A(e)$を使わない$EC_{\varepsilon})$から
	$\rightharpoondown A(t) \Longrightarrow B$への証明を構成する.
	今度は$\pi$に現れる$e$を$t$に置き換える必要はない.
	$\varphi_{i}$が$A(t) \Longrightarrow A(e)$でない$EC_{\varepsilon}$の公理ならば,
	$\varphi_{i}$と$\varphi_{i+1}$の間に
	\begin{align}
		&\varphi_{i} \Longrightarrow (\rightharpoondown A(t) \Longrightarrow \varphi_{i}), \\
		&\rightharpoondown A(t) \Longrightarrow \varphi_{i}
	\end{align}
	を挿入する.$\varphi_{i}$が$\varphi_{j}$と$\varphi_{k}$からモーダスポンネスで得られる場合は,
	$\varphi_{i}$を
	\begin{align}
		&(\rightharpoondown A(t) \Longrightarrow \varphi_{j}) \Longrightarrow
		[(\rightharpoondown A(t) \Longrightarrow 
		(\varphi_{j}\Longrightarrow \varphi_{i}))
		\Longrightarrow (\rightharpoondown A(t) \Longrightarrow \varphi_{i})], \\
		&(\rightharpoondown A(t) \Longrightarrow 
		(\varphi_{j} \Longrightarrow \varphi_{i}))
		\Longrightarrow (\rightharpoondown A(t) \Longrightarrow \varphi_{i}), \\
		&\rightharpoondown A(t) \Longrightarrow \varphi_{i}
	\end{align}
	で置き換える.$\varphi_{i}$が$A(t) \Longrightarrow A(e)$であるときは,$\varphi_{i}$を
	\begin{align}
		\rightharpoondown A(t) \Longrightarrow (A(t) \Longrightarrow A(e))
	\end{align}
	で置き換える.
	
	以上で$A(t) \Longrightarrow B$と$\rightharpoondown A(t) \Longrightarrow B$に対して
	$A(t) \Longrightarrow A(e)$を用いない$EC_{\varepsilon}$からの証明が得られた.後はこれに
	\begin{align}
		&(A(t) \Longrightarrow B) \Longrightarrow
		((\rightharpoondown A(t) \Longrightarrow B) \Longrightarrow
		((A(t) \Longrightarrow B) \wedge (\rightharpoondown A(t) \Longrightarrow B))), \\
		&(\rightharpoondown A(t) \Longrightarrow B) \Longrightarrow
		((A(t) \Longrightarrow B) \wedge (\rightharpoondown A(t) \Longrightarrow B)), \\
		&(A(t) \Longrightarrow B) \wedge (\rightharpoondown A(t) \Longrightarrow B), \\
		&((A(t) \Longrightarrow B) \wedge (\rightharpoondown A(t) \Longrightarrow B))
		\Longrightarrow ((A(t) \vee \rightharpoondown A(t)) \Longrightarrow B), \\
		&(A(t) \vee \rightharpoondown A(t)) \Longrightarrow B, \\
		&A(t) \vee \rightharpoondown A(t), \\
		&B
	\end{align}
	を追加すれば,$A(t) \Longrightarrow A(e)$を用いない$EC_{\varepsilon}$から$B$への証明となる.
	
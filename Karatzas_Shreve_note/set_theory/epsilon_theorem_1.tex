\section{第一イプシロン定理}
	$A$を$L(PC_{\varepsilon})$の式とするとき,$A$を$L(EC_{\varepsilon})$の式に書き換える.
	\begin{align}
		x^{\varepsilon} &\rightarrow x \\
		(\in \tau \sigma)^{\varepsilon} &\rightarrow \in \tau^{\varepsilon} \sigma^{\varepsilon} \\
		(= \tau \sigma)^{\varepsilon} &\rightarrow = \tau^{\varepsilon} \sigma^{\varepsilon} \\
		(\rightharpoondown \varphi)^{\varepsilon} &\rightarrow \rightharpoondown \varphi^{\varepsilon} \\
		(\vee \varphi \psi)^{\varepsilon} &\rightarrow \vee \varphi^{\varepsilon} \psi^{\varepsilon} \\
		(\wedge \varphi \psi)^{\varepsilon} &\rightarrow \wedge \varphi^{\varepsilon} \psi^{\varepsilon} \\
		(\Longrightarrow \varphi \psi)^{\varepsilon} &\rightarrow \Longrightarrow \varphi^{\varepsilon} \psi^{\varepsilon} \\
		(\exists x \varphi)^{\varepsilon} &\rightarrow \varphi^{\varepsilon}(\varepsilon x \varphi^{\varepsilon}) \\
		(\forall x \varphi)^{\varepsilon} &\rightarrow \varphi^{\varepsilon}(\varepsilon x \rightharpoondown \varphi^{\varepsilon}) \\
		(\varepsilon x \psi)^{\varepsilon} &\rightarrow \varepsilon x \varphi^{\varepsilon}
	\end{align}
	
	$A$が$L(PC_{\varepsilon})$の式で,$x$が$A$に自由に現れて,
	かつ$A$に自由に現れているのが$x$のみであるとき,
	$A^{\varepsilon}$にも$x$が自由に現れて,かつ$A^{\varepsilon}$に
	自由に現れているのは$x$のみである.
	
	\begin{align}
		(\varphi[x/\tau])^{\varepsilon} \rightarrow \varphi^{\varepsilon}
		(\varphi^{\varepsilon}[x/\tau^{\varepsilon}]). \\
	\end{align}
	
	\begin{itembox}[c]{$PC_{\varepsilon}$の証明を$EC_{\varepsilon}$の証明に埋め込む}
		$A$を$L(PC_{\varepsilon})$の文とし,$PC_{\varepsilon} \vdash A$であるとする.
		このとき$EC_{\varepsilon} \vdash A^{\varepsilon}$である.
	\end{itembox}
	
	示すべきことは
	\begin{itemize}
		\item $A \in Ax(PC_{\varepsilon})$ならば$\vdash A^{\varepsilon}$であること.
			\begin{itemize}
				\item $\vdash A$ならば$\vdash A^{\varepsilon}$であること.
				\item $A$に$x$が自由に現れて,かつ自由に現れているのが$x$のみであるとき,
					\begin{align}
						\vdash A^{\varepsilon}(t^{\varepsilon}) \Longrightarrow A^{\varepsilon}(\varepsilon x A^{\varepsilon})
					\end{align}
					であること.
				\item $A$に$x$が自由に現れて,かつ自由に現れているのが$x$のみであるとき,
					\begin{align}
						\vdash A^{\varepsilon}(\varepsilon x \rightharpoondown A^{\varepsilon}) \Longrightarrow A^{\varepsilon}(t^{\varepsilon})
					\end{align}
					であること.
			\end{itemize}
		
		\item $PC_{\varepsilon} \vdash B$かつ$PC_{\varepsilon} \vdash B \Longrightarrow A$である$B$が取れるとき,
			$(B \Longrightarrow A)^{\varepsilon}$は$B^{\varepsilon} \Longrightarrow A^{\varepsilon}$なので
			$EC_{\varepsilon} \vdash B^{\varepsilon}$ならば
			$EC_{\varepsilon} \vdash A^{\varepsilon}$となる.
	\end{itemize}
	
	\begin{screen}
		\begin{thm}[置換補題]
		\end{thm}
	\end{screen}
	
	\begin{sketch}\mbox{}
		\begin{description}
			\item[Case1]
				$e$が$B$に現れていない場合,$B[s]$または$B[\varepsilon y B]$が$e$の中に現れることはない.
				仮に$B[s]$に$e$が現れているとすれば,$B$には$e[s/y]$が現れることになる.
				$e[s/y]$に$y$が自由に現れているなら
				\begin{align}
					rk(e[s/y]) + 1 \leq rk(\varepsilon y B)
				\end{align}
				であるし,またこのとき
				\begin{align}
					rk(e) = rk(e[s/y])
				\end{align}
				であるから($s$は閉じているので階数補題適用可),
				\begin{align}
					rk(\pi) < rk(\varepsilon y B)
				\end{align}
				となってしまい矛盾.$y$が$e[s/y]$で束縛されているなら,
				
				
			\item[Case2]
				$B$の中に$e$は現れないが,$s$の中には$e$が現れる場合.$\theta$を項または式とするとき,
				$\theta$に現れる$e$を$t$に置き換えた式を$\theta^{t}$と書くと,
				\begin{align}
					B^{t} &\equiv B, \\
					(B[s])^{t} &\equiv B^{t}[s^{t}] \\
					&\equiv B[s^{t}], \\
					(B[\varepsilon y B])^{t} &\equiv B^{t}[(\varepsilon y B)^{t}] \\
					&\equiv B[\varepsilon y B^{t}] \\
					&\equiv B[\varepsilon y B]
				\end{align}
				より,$C$は
				\begin{align}
					B[s^{t}] \rightarrow B[\varepsilon y B]
				\end{align}
				である.
				
			\item[Case3]
				$B$に$e$が現れる場合.
				\begin{enumerate}
					\item $e$の中に$y$の自由な出現は無い.なぜならば,$y$が$e$の中に自由に現れていると
						\begin{align}
							rk(e) + 1 \leq rk(\varepsilon y B)
						\end{align}
						となるから.つまり$e$は$\varepsilon y B$には従属していない.ゆえに
						\begin{align}
							deg(e) < deg(\varepsilon y B)
						\end{align}
						である.$deg(e)$は階数が$rk(\pi)$である$\varepsilon$項の次数の最大値であるから
						\begin{align}
							rk(\varepsilon y B) < rk(\pi)
						\end{align}
						が満たされている筈である.
					
					\item $\theta$を項または式とするとき,$\theta$に現れる$e$を$t$に置き換えた式を
						$\theta^{t}$と書くと,$C$は
						\begin{align}
							B^{t}[s^{t}] \rightarrow B^{t}[\varepsilon y B^{t}]
						\end{align}
						となる.実際,
						\begin{align}
							(B[s])^{t} \equiv B^{t}[s^{t}]
						\end{align}
						及び
						\begin{align}
							(B[\varepsilon y B])^{t} \equiv
							B^{t}[(\varepsilon y B)^{t}] \equiv
							B^{t}[\varepsilon y B^{t}]
						\end{align}
						なので.また$B$に現れる$\varepsilon$項で,その中に$y$が自由に現れているものは
						$B^{t}$にもそのまま残っているから
						\begin{align}
							rk(B) = rk(B^{t})
						\end{align}
						となる.ゆえに
						\begin{align}
							rk(\varepsilon y B) = rk(B) + 1 = rk(B^{t}) + 1 = rk(\varepsilon y B^{t})
						\end{align}
						となる.
				\end{enumerate}
		\end{description}
	\end{sketch}
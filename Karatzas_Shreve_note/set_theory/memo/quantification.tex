\section{量化再考}
	\begin{screen}
		\begin{logicalaxm}[量化の公理]\mbox{}
			\begin{enumerate}
				\item $\forall y\, \left(\, \forall x \varphi(x) \Longrightarrow \varphi(y)\, \right)$
				\item $\forall x\, \left(\, \varphi(x) \Longrightarrow \exists y \varphi(y)\, \right)$
				\item $\forall y\, \left(\, \varphi \Longrightarrow \psi(y)\, \right)
					\Longrightarrow \left(\, \varphi \Longrightarrow \forall y \psi(y)\, \right)$
				\item $\forall x\, \left(\, \varphi(x) \Longrightarrow \psi\, \right)
					\Longrightarrow \left(\, \exists x \varphi(x) \Longrightarrow \psi\, \right)$
				\item $\forall x\,  \left(\, \varphi(x) \Longrightarrow \psi(x)\, \right)
					\Longrightarrow \left(\, \forall x \varphi(x) \Longrightarrow \forall x \psi(x)\, \right)$
			
				\item $\forall x \varphi(x) \Longrightarrow \exists x \varphi(x)$	
			\end{enumerate}
		\end{logicalaxm}
	\end{screen}
	
	\begin{screen}
		$\forall x \varphi(x) \Longrightarrow \forall y \varphi(y).$
	\end{screen}
	
	\begin{align}
		&\forall y\, \left(\, \forall x \varphi(x) \Longrightarrow \varphi(y)\, \right)
		&& \mbox{(公理1)} \\
		&\forall y\, \left(\, \forall x \varphi(x) \Longrightarrow \varphi(y)\, \right)
		\Longrightarrow \left(\, \forall x \varphi(x) \Longrightarrow \forall y \varphi(y)\, \right),
		&& \mbox{(公理3)} \\
		&\forall x \varphi(x) \Longrightarrow \forall y \varphi(y).
		&& \mbox{(MP)}
	\end{align}
	
	\begin{screen}
		$\exists x \varphi(x) \Longrightarrow \exists y \varphi(y).$
	\end{screen}
	
	\begin{align}
		&\forall x\, \left(\, \varphi(x) \Longrightarrow \exists y \varphi(y)\, \right)
		&& \mbox{(公理2)} \\
		&\forall x\, \left(\, \varphi(x) \Longrightarrow \exists y \varphi(y)\, \right)
		\Longrightarrow \left(\, \exists x \varphi(x) \Longrightarrow \exists y \varphi(y)\, \right),
		&& \mbox{(公理4)} \\
		&\exists x \varphi(x) \Longrightarrow \exists y \varphi(y).
		&& \mbox{(MP)}
	\end{align}
	
	\begin{screen}
		$\forall x\,  \left(\, \varphi(x) \Longrightarrow \psi\, \right)
		\Longrightarrow \left(\, \forall x \varphi(x) \Longrightarrow \psi\, \right).$
	\end{screen}
	
	$\forall x\,  \left(\, \varphi(x) \Longrightarrow \psi\, \right)$
	と$\forall x \varphi(x)$からなる文の集合を$\Gamma$とすると,
	\begin{align}
		\Gamma &\vdash \forall x\,  \left(\, \varphi(x) \Longrightarrow \psi\, \right) \\
		\Gamma &\vdash \forall x\,  \left(\, \varphi(x) \Longrightarrow \psi\, \right) \Longrightarrow \left(\, \exists x \varphi(x) \Longrightarrow \psi\, \right) \\
		\Gamma &\vdash \exists x \varphi(x) \Longrightarrow \psi \\
		& \\
		\Gamma &\vdash \forall x \varphi(x) \\
		\Gamma &\vdash \forall x \varphi(x) \Longrightarrow \exists x \varphi \\
		\Gamma &\vdash \exists x \varphi(x) \\
		& \\
		\Gamma &\vdash \psi
	\end{align}
	が成り立つので,演繹法則より
	\begin{align}
		\vdash \forall x\,  \left(\, \varphi(x) \Longrightarrow \psi\, \right)
		\Longrightarrow \left(\, \forall x \varphi(x) \Longrightarrow \psi\, \right)
	\end{align}
	が得られる.
	
	\begin{screen}
		$\tau$を$\mathcal{L}_{\in}$には無い定数記号として,
		$\mathcal{L}_{\in}' = \mathcal{L}_{\in} \cup \{\tau\}$とおく.
		$\varphi$を$\mathcal{L}_{\in}'$の式とし,
		\begin{align}
			\Sigma \vdash_{\mathcal{L}_{\in}'} \varphi
		\end{align}
		であるとする.項$x$を,もし$\tau$が$\varphi$に現れるならば
		$\varphi$の中の$\tau$の出現位置で束縛されない変項とする.
		このとき,$\tau$が$\varphi$に現れるならば
		\begin{align}
			\Sigma \vdash_{\mathcal{L}_{\in}} \forall x \varphi(x/\tau)
		\end{align}
		が成り立つ.$\tau$が$\varphi$に現れなければ
		\begin{align}
			\Sigma \vdash_{\mathcal{L}_{\in}} \varphi
		\end{align}
		が成り立つ.
	\end{screen}
	
	\begin{sketch}
		$\varphi$が$\Sigma$の公理であるときは
		\begin{align}
			\Sigma \vdash_{\mathcal{L}_{\in}} \varphi
		\end{align}
		となるし,$\varphi$が推論法則であるときは,$\varphi$に$\tau$が現れなければ
		\begin{align}
			\Sigma \vdash_{\mathcal{L}_{\in}} \varphi
		\end{align}
		となるし,$\varphi$に$\tau$が現れても
		\begin{align}
			\Sigma \vdash_{\mathcal{L}_{\in}} \forall x \varphi(x/\tau)
		\end{align}
		が成立する.$\varphi$が三段論法によって示されるとき,つまり$\mathcal{L}_{\in}'$の文$\psi$で
		\begin{align}
			\Sigma &\vdash_{\mathcal{L}_{\in}'} \psi, \\
			\Sigma &\vdash_{\mathcal{L}_{\in}'} \psi \Longrightarrow \varphi
		\end{align}
		を満たすものが取れるとき,$y$を$\psi$にも$\varphi$にも表れない変項とする.
		また$\varphi$と$\psi$に$\tau$が現れているかいないかで
		\begin{description}
			\item[case1] $\varphi$にも$\psi$にも$\tau$が現れていないとき,
				\begin{align}
					\Sigma \vdash_{\mathcal{L}_{\in}} \psi
				\end{align}
				かつ
				\begin{align}
					\Sigma \vdash_{\mathcal{L}_{\in}} \psi \Longrightarrow \varphi
				\end{align}
				
			\item[case2] $\varphi$には$\tau$が現れているが,$\psi$には$\tau$が現れていないとき,
				\begin{align}
					\Sigma \vdash_{\mathcal{L}_{\in}} \psi
				\end{align}
				かつ
				\begin{align}
					\Sigma \vdash_{\mathcal{L}_{\in}} 
					\forall y\, (\, \psi \Longrightarrow \varphi(y/\tau)\, )
				\end{align}
				
			\item[case3] $\varphi$には$\tau$が現れていないが,$\psi$には$\tau$が現れているとき,
				\begin{align}
					\Sigma \vdash_{\mathcal{L}_{\in}} \forall y \psi(y/\tau)
				\end{align}
				かつ
				\begin{align}
					\Sigma \vdash_{\mathcal{L}_{\in}} 
					\forall y\, (\, \psi(y/\tau) \Longrightarrow \varphi\, )
				\end{align}
				
			\item[case4] $\varphi$にも$\psi$にも$\tau$が現れているとき,
				\begin{align}
					\Sigma \vdash_{\mathcal{L}_{\in}} \forall y \psi(y/\tau)
				\end{align}
				かつ
				\begin{align}
					\Sigma \vdash_{\mathcal{L}_{\in}} \forall y\, (\, \psi(y/\tau) \Longrightarrow \varphi(y/\tau)\, )
				\end{align}
				
		\end{description}
		のいずれかのケースを一つ仮定する.
		\begin{description}
			\item[case1] 証明可能性の定義より
				\begin{align}
					\Sigma \vdash_{\mathcal{L}_{\in}} \varphi
				\end{align}
				が成り立つ.
				
			\item[case2]
				公理2と併せて
				\begin{align}
					\Sigma \vdash_{\mathcal{L}_{\in}} \psi \Longrightarrow \forall y \varphi(y/\tau)
				\end{align}
				が成り立つので,証明可能性の定義より
				\begin{align}
					\Sigma \vdash_{\mathcal{L}_{\in}} \forall y \varphi(y/\tau)
				\end{align}
				となる.そして
				\begin{align}
					\Sigma \vdash_{\mathcal{L}_{\in}} \forall x \varphi(x/\tau)
				\end{align}
				も成り立つ.
				
			\item[case3]
				\begin{align}
					\Sigma \vdash_{\mathcal{L}_{\in}} \forall y \psi(y/\tau) \Longrightarrow \varphi
				\end{align}
				が成り立つので,,証明可能性の定義より
				\begin{align}
					\Sigma \vdash_{\mathcal{L}_{\in}} \varphi
				\end{align}
				となる.
				
			\item[case4]
				公理5より
				\begin{align}
					\Sigma \vdash_{\mathcal{L}_{\in}} \forall y \psi(y/\tau) \Longrightarrow \forall y \varphi(y/\tau)
				\end{align}
				が成り立つので,証明可能性の定義より
				\begin{align}
					\Sigma \vdash_{\mathcal{L}_{\in}} \forall y \varphi(y/\tau)
				\end{align}
				となる.そして
				\begin{align}
					\Sigma \vdash_{\mathcal{L}_{\in}} \forall x \varphi(x/\tau)
				\end{align}
				も成り立つ.
				\QED
		\end{description}
	\end{sketch}
	
	\begin{screen}
		$\forall x \varphi(x),\ \forall x\, (\, \varphi(x) \Longrightarrow \psi(x)\, )
		\vdash \forall x \psi(x).$
	\end{screen}
	
	公理5より
	\begin{align}
		\forall x\, (\, \varphi(x) \Longrightarrow \psi(x)\, )
		\vdash \forall x \varphi(x) \Longrightarrow \forall x \psi(x)
	\end{align}
	が成り立つので,三段論法より
	\begin{align}
		\forall x \varphi(x),\ \forall x\, (\, \varphi(x) \Longrightarrow \psi(x)\, )
		\vdash \forall x \psi(x)
	\end{align}
	が従う.
	
	\begin{screen}
		$\tau$を定項とし,$\mathcal{L}_{\in}' = \mathcal{L}_{\in} \cup \{\tau\}$とする.
		また$\varphi$を$\mathcal{L}_{\in}$の式とし,項$x$が$\varphi$に自由に現れて,
		また$\varphi$で自由に現れる項は$x$のみであるとする.このとき
		$\vdash_{\mathcal{L}_{\in}'} \varphi(\tau)$なら
		$\vdash_{\mathcal{L}_{\in}} \forall x \varphi(x)$.
	\end{screen}
	
	\begin{screen}
		\begin{logicalthm}[De Morgan 1]
			$\vdash_{\mathcal{L}_{\in}} \forall x \rightharpoondown \varphi(x) 
			\Longrightarrow\ \rightharpoondown \exists x \varphi(x).$
		\end{logicalthm}
	\end{screen}
	
	公理4より
	\begin{align}
		\vdash_{\mathcal{L}_{\in}} \forall x\, (\, \varphi(x) \Longrightarrow\ 
		\rightharpoondown \forall x \rightharpoondown \varphi(x)\, )
		\Longrightarrow (\, \exists x \varphi(x)
		\Longrightarrow\ \rightharpoondown \forall x \rightharpoondown \varphi(x)\, )
	\end{align}
	が成り立ち,また公理1より
	\begin{align}
		\vdash_{\mathcal{L}_{\in}}
		\forall x\, (\, \forall x \rightharpoondown \varphi(x)
		\Longrightarrow \varphi(x)\, )
	\end{align}
	が成り立つので
	\begin{align}
		\vdash_{\mathcal{L}_{\in}}
		\forall x\, (\, \rightharpoondown \varphi(x)
		\Longrightarrow\ \rightharpoondown \forall x \rightharpoondown \varphi(x)\, )
	\end{align}
	も成り立つ.そして三段論法より
	\begin{align}
		\vdash_{\mathcal{L}_{\in}} \exists x \varphi(x)
		\Longrightarrow\ \rightharpoondown \forall x \rightharpoondown \varphi(x)
	\end{align}
	が従う.ゆえに
	\begin{align}
		\vdash_{\mathcal{L}_{\in}} \forall x \rightharpoondown \varphi(x) 
		\Longrightarrow\ \rightharpoondown \exists x \varphi(x)
	\end{align}
	となる.
	
	\begin{screen}
		\begin{logicalthm}[De Morgan 2]
			$\vdash_{\mathcal{L}_{\in}} \rightharpoondown \exists x \varphi(x)
			\Longrightarrow \forall x \rightharpoondown \varphi(x).$
		\end{logicalthm}
	\end{screen}
	
	\begin{align}
		&\forall y\, (\, \varphi(y) \Longrightarrow \exists x \varphi(x)\, )
		&& \mbox{(公理2)} \\
		&\forall y\, (\, \rightharpoondown \exists x \varphi(x) 
		\Longrightarrow\ \rightharpoondown \varphi(y)\, )
		&& \mbox{()} \\
		&\forall y\, (\, \rightharpoondown \exists x \varphi(x) 
		\Longrightarrow\ \rightharpoondown \varphi(y)\, )
		\Longrightarrow (\, \rightharpoondown \exists x \varphi(x) 
		\Longrightarrow \forall y \rightharpoondown \varphi(y)\, )
		&& \mbox{(公理3)} \\
		&\rightharpoondown \exists x \varphi(x)
		\Longrightarrow \forall y \rightharpoondown \varphi(y)
		&& \mbox{(MP)} \\
		&\forall y \rightharpoondown \varphi(y)
		\Longrightarrow \forall x \rightharpoondown \varphi(x)
		&& \mbox{()} \\
		&\rightharpoondown \exists x \varphi(x)
		\Longrightarrow \forall x \rightharpoondown \varphi(x)
		&& \mbox{()}
	\end{align}
	より.
	
	\begin{screen}
		$\rightharpoondown \forall x \varphi(x) \Longrightarrow
		\exists x \rightharpoondown \varphi(x).$
	\end{screen}
	
	公理7より
	\begin{align}
		\vdash \rightharpoondown \forall x 
		\rightharpoondown \rightharpoondown \varphi(x)
		\Longrightarrow \exists x \rightharpoondown \varphi(x)
	\end{align}
	が成り立つ.また
	\begin{align}
		\vdash \forall x\, (\, \rightharpoondown \rightharpoondown \varphi(x)
		\Longrightarrow \varphi(x)\, )
	\end{align}
	と公理5より
	\begin{align}
		\vdash \forall x \rightharpoondown \rightharpoondown \varphi(x)
		\Longrightarrow \forall x \varphi(x)
	\end{align}
	が成り立つので,対偶を取って
	\begin{align}
		\vdash\ \rightharpoondown \forall x \varphi(x) \Longrightarrow\ 
		\rightharpoondown \forall x \rightharpoondown \rightharpoondown \varphi(x)
	\end{align}
	が得られる.よって
	\begin{align}
		\vdash\ \rightharpoondown \forall x \varphi(x) \Longrightarrow
		\exists x \rightharpoondown \varphi(x)
	\end{align}
	となる.
	
	\begin{screen}
		$\vdash \exists x \rightharpoondown \varphi(x) 
		\Longrightarrow\ \rightharpoondown \forall x \varphi(x).$
	\end{screen}
	
	$\vdash_{\mathcal{L}_{\in}'} \varphi(\tau) \Longrightarrow\ 
	\rightharpoondown \rightharpoondown \varphi(\tau)$より
	\begin{align}
		\vdash \forall x \varphi(x) \Longrightarrow
		\forall x \rightharpoondown \rightharpoondown \varphi(x)
	\end{align}
	が成り立ち,さらに公理7より
	\begin{align}
		\vdash \forall x \rightharpoondown \rightharpoondown \varphi(x)
		\Longrightarrow\ \rightharpoondown \exists x \rightharpoondown \varphi(x)
	\end{align}
	が成り立つので,
	\begin{align}
		\vdash \forall x \varphi(x) \Longrightarrow
		\Longrightarrow\ \rightharpoondown \exists x \rightharpoondown \varphi(x)
	\end{align}
	が得られる.
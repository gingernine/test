\section{徒然なるままに支離滅裂}
わからないわからないわからない

基礎論における証明は大抵が直感に頼っているように見えますが,ではその直感が正しいとは誰が保証するのでしょうか.
手元にあるどの本でも保証されていません.もしかしたら神様という超然的な存在を暗黙の裡に認めていて,
直感とは神様が用意した論理であるとして無断で使っているだけなのかもしれませんが,
残念ながら読者はテレパシーを使えないので,筆者の暗黙の了解を推察するなんて困難です.

しかしながら,暗黙の了解を排除しようとすると,その分だけ日本語による明示的な約束が必要になります.
すると新たな問題が生じます.それは日本語で書かれた言明をどこまで信用するか,という問題です.
基礎論の難しさは,その表面上のややこしさよりも日本語に対する認識を揃えることにあるのでしょうか.

論理構造を集合論の結果を用いて解明しようというのならまだしも(こちらは数理論理学と呼ばれる分野で,本来は数学基礎論とは別物だそうです),
集合論を構築することが目的である場合,その土台となる基礎論を集合論の上に展開すると理論が循環することになるでしょう.
基礎論が基礎にしている集合論は「メタ理論」と呼ばれるらしいですが,
その「メタ理論」がどう構成されたのかという点には誰も全く言及していないのですから,
「メタ理論」という言葉は単なる逃げ口上にしか聞こえず,理論の循環を解消できません.
私の考えでは,メタ理論の代わりに絶対的な原理が与えられたとして数学を構築すれば良いのです.
まあ言い方を変えて印象を良く?しようというだけの下らない事情であって,
もったいぶって思想的な立場を主張しても集合論には関係のないことなのですが.

前提:我々は数の概念を持っている.個数の概念を持っている.物の数を数えることが出来る.
数の概念とは?個数の概念とは?
ここで言う数は数学的に構成する数ではなくて,神が用意した概念としての数.
そこまで踏み込むときりがない.

排中律と無矛盾性の違い:
排中律から$\negation (A \wedge \negation A)$が導かれるが,
$A \wedge \negation A$が導かれることを否定しているわけではない.

目的:いかに自然で人工的な世界を作るか.

数学について注意深く考え込んでいると,うっかりとんでもない落とし穴にはまってしまうかもしれません.
そのとき,きっと次の事柄に悩まされます.
前提がわからない.明らかなものと明らかでないものとの線引きがわからない.
日本語をどこまで信用してよいのかもわからない.
突き詰めると何も見えなくなる.数学の立脚地は永遠に届かない...
このノートがそういった受難を乗り越える役に立てますように.

\begin{description}
	\item[前提その一] はじめに素朴な数字の概念は持っている.それによってモノを数えることもできる.
		モノの数え方は慣習どおり.
		
	\item[前提その二] 当たり前のことが当たり前であるためには,言葉でそれを保証しなければならない.
	
	\item[] $ZF$集合論では存在という言葉が具体的な意味を持っていない.
		存在したらこうなるであろうという推論規則によってしか存在という概念を表現し得ない.
		$\varepsilon$項は集合である.それも,存在したら``取れる''集合である.
		$\varepsilon$項によって集合を具体的に扱える.
		さらに内包項によって閉じた世界を作ることが出来る.
		$ZF$集合論では集合の宇宙は閉じていないので,存在したらそれに名前を付けて言語を保存拡大するという手法を取る.
		内包項を導入すれば,言語を拡大する必要はなくなる.
		公理とは,既に作られた世界においてどれが集合でどれが集合でないかを選り分ける用に使われる.
\end{description}

\begin{itemize}
	\item $\lang{\varepsilon}$の公理は$\lang{\in}$の公理と同じで良い.
	\item $\mathcal{L}$で示せることは$\lang{\in}$で示せる.実際,
		$\varphi_{1},\cdots,\varphi_{n}$を$\mathcal{L}$の文でできた証明とすれば,
		それを$\lang{\in}$の式に書き換えた式(変項の名前替えは要るかもしれない)の列
		$\psi_{1},\cdots,\psi_{n}$は$\lang{\in}$の文でできた証明となる.
\end{itemize}
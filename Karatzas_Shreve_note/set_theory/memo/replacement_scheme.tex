\section{置換公理}
	置換公理の二つの形式の同値性をざっくりと.
	\begin{description}
		\item[(T)] $\sing{f} \Longrightarrow \forall a\, \set{f \ast a}.$
		\item[(K)] $\forall a\, \left[\, \forall x \in a\, \exists!y \varphi(x,y)
				\Longrightarrow \exists z\, \forall y\,
				(\, y \in z \Longleftrightarrow \exists x\, (\, x \in a \wedge 
				\varphi(x,y)\, )\, )\, \right].$
	\end{description}
	
	ただし
	\begin{align}
		\sing{f} &\defarrow \forall x,y,z\, (\, (x,y) \in f \wedge (x,z) \in f
		\Longrightarrow y = z\, ), \\
		f \ast a &\defeq \Set{y}{\exists x \in a\, (\, (x,y) \in f\, )}, \\
		\set{s} \defarrow \exists x\, (\, s = x\, )
	\end{align}
	であるし,$\varphi$に自由に現れているのは二つの変項のみで,それらが$s$と$t$とおけば,
	$\varphi$に自由に現れている$s$を全て$x$に,
	$\varphi$に自由に現れている$t$を全て$y$に置き換えた式が
	\begin{align}
		\varphi(x,y)
	\end{align}
	である.またこのとき$x$も$y$も$\varphi(x,y)$で束縛されていないものとする
	($x$と$y$はそのように選ばれた変項であるということである).
	
	\begin{description}
		\item[(T)$\Longrightarrow$(K)]
			$a$を任意の集合とし,
			\begin{align}
				\forall x \in a \exists!y \varphi(x,y)
			\end{align}
			であるとする.
			\begin{align}
				f \defeq \Set{(x,y)}{x \in a \wedge \varphi(x,y)}
			\end{align}
			とおけば$f$は$a$上の写像であって,(T)より
			\begin{align}
				\exists z\, (\, z = f \ast a\, )
			\end{align}
			となる.ところで$f \ast a$とは
			\begin{align}
				\Set{y}{\exists x \in a\, (\, (x,y) \in f\, )}
			\end{align}
			なので
			\begin{align}
				f \ast a = \Set{y}{\exists x \in a \varphi(x,y)}.
			\end{align}
			ゆえに
			\begin{align}
				\exists z\, \forall y\, (\, y \in z \Longleftrightarrow
				\exists x \in a \varphi(x,y)\, )
			\end{align}
			が成り立つ.
			
		\item[(K)$\Longrightarrow$(T)]
			$\sing{f}$とし,$a$を集合とする.
			\begin{align}
				b \defeq a \cap \dom{f}
			\end{align}
			とおけば,(K)からは分出公理が示せるので$b$は集合である.そして
			\begin{align}
				\forall x \in b\, \exists!y\, (\, (x,y) \in f\, )
			\end{align}
			が成り立つのだから,(K)より
			\begin{align}
				z = \Set{y}{\exists x \in b\, (\, (x,y) \in f\, )}
			\end{align}
			が従う.ここで
			\begin{align}
				\Set{y}{\exists x \in b\, (\, (x,y) \in f\, )}
				= f \ast b
				= f \ast a
			\end{align}
			であるから(T)が得られる.
			\QED
	\end{description}
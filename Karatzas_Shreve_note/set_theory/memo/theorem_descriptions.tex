\chapter{定理参照メモ}

\section{証明}
	\begin{screen}
		\begin{logicalaxm}[演繹規則]\ref{logicalaxm:deduction_rule}
			$A,B,C,D$を文とするとき,
			\begin{description}
				\item[(a)] $A \vdash D$ならば$\vdash A \rarrow D$が成り立つ.
				\item[(b)] $A,B \vdash D$ならば
					\begin{align}
						B \vdash A \rarrow D,\quad
						A \vdash B \rarrow D
					\end{align}
					が成り立つ.
				\item[(c)] $A,B,C \vdash D$ならば
					\begin{align}
						B,C \vdash A \rarrow D,\quad
						A,C \vdash B \rarrow D,\quad
						A,B \vdash C \rarrow D
					\end{align}
					のいずれも成り立つ.
			\end{description}
		\end{logicalaxm}
	\end{screen}
	
	\begin{screen}
		\begin{metadfn}[証明可能]
			文$\varphi$が公理系$\mathscr{S}$から
			{\bf 証明された}だとか{\bf 証明可能である}\index{しょうめいかのう@証明可能}
			{\bf (provable)}ということは,
			\begin{itemize}
				\item $\varphi$は$\mathscr{S}$の公理である.
				\item $\vdash \varphi$である.
				\item 文$\psi$で,$\psi$と$\psi \rightarrow \varphi$が$\mathscr{S}$から
				証明されているものが取れる({\bf 三段論法}\index{さんだんろんぽう@三段論法}
				{\bf (Modus Pones)}).
			\end{itemize}
			のいずれかが満たされているということであり,$\varphi$が$\mathscr{S}$から証明可能であることを
			\begin{align}
				\mathscr{S} \vdash \varphi
			\end{align}
			と書く.ただし,公理系に変項が生じた場合の証明可能性には
			演繹規則や後述の演繹法則,およびその逆の結果を適用することが出来る.
		\end{metadfn}
	\end{screen}
	
	\begin{screen}
		\begin{logicalthm}[含意の反射律]
		\ref{logicalthm:reflective_law_of_implication}
			$A$を文とするとき
			\begin{align}
				\vdash A \rarrow A.
			\end{align}
		\end{logicalthm}
	\end{screen}
	
	\begin{screen}
		\begin{logicalthm}[含意の導入]
		\ref{logicalthm:introduction_of_implication}
			$A,B$を文とするとき
			\begin{align}
				\vdash B \rarrow (A \rarrow B).
			\end{align}
		\end{logicalthm}
	\end{screen}
	
	\begin{screen}
		\begin{logicalthm}[含意の分配則]
		\ref{logicalthm:distributive_law_of_implication}
			$A,B,C$を文とするとき
			\begin{align}
				\vdash (A \rarrow (B \rarrow C)) \rarrow ((A \rarrow B) \rarrow (A \rarrow C)).
			\end{align}
		\end{logicalthm}
	\end{screen}
	
	\begin{screen}
		\begin{metaaxm}[証明に対する構造的帰納法]
			$\mathscr{S}$を公理系とし,Xを文に対する何らかの言明とするとき,
			\begin{itemize}
				\item $\mathscr{S}$の公理に対してXが言える.
				\item 推論法則に対してXが言える.
				\item $\varphi$と$\varphi \rarrow \psi$が$\mathscr{S}$の
					定理であるような文$\varphi$と文$\psi$が取れたとき,
					$\varphi$と$\varphi \rarrow \psi$に対して
					Xが言えるならば,$\psi$に対してXが言える.
			\end{itemize}
			のすべてが満たされていれば,$\mathscr{S}$から証明可能なあらゆる文に対してXが言える.
		\end{metaaxm}
	\end{screen}
	
	\begin{screen}
		\begin{metathm}[演繹法則]\ref{metathm:deduction_theorem}
			$\mathscr{S}$を公理系とし,$A$を文とするとき,
			$\mathscr{S}, A$の任意の定理$B$に対して
			\begin{align}
				\mathscr{S} \vdash A \rarrow B
			\end{align}
			が成り立つ.
		\end{metathm}
	\end{screen}
	
	\begin{screen}
		\begin{metathm}[公理が増えても証明可能]
			$\mathscr{S}$を公理系とし,$A$を文とするとき,
			$\mathscr{S}$の任意の定理$B$に対して
			\begin{align}
				\mathscr{S}, A \vdash B
			\end{align}
			が成り立つ.
		\end{metathm}
	\end{screen}
	
	\begin{screen}
		\begin{metathm}[演繹法則の逆]
		\ref{metathm:inverse_of_deduction_theorem}
			$\mathscr{S}$を公理系とし,$A$と$B$を文とするとき,
			\begin{align}
				\mathscr{S} \vdash A \rarrow B
			\end{align}
			であれば
			\begin{align}
				A,\ \mathscr{S} \vdash B
			\end{align}
			が成り立つ.
		\end{metathm}
	\end{screen}
	
\section{推論}
	\begin{screen}
		\begin{logicalaxm}[否定と矛盾に関する規則]
		\ref{logicalaxm:rules_of_negation_and_contradiction}
			$A$を文とするとき以下が成り立つ:
			\begin{description}
				\item[矛盾の導入] 否定が共に成り立つとき矛盾が起きる:
					\begin{align}
						A,\ \negation A \vdash \bot.
					\end{align}
				\item[否定の導入] 矛盾が導かれるとき否定が成り立つ:
					\begin{align}
						A \rarrow \bot \vdash\ \negation A.
					\end{align}
			\end{description}
		\end{logicalaxm}
	\end{screen}
	
	\begin{screen}
		\begin{dfn}[対偶]
			$\varphi \rarrow \psi$なる式に対して
			\begin{align}
				\negation \psi \rarrow\ \negation \varphi
			\end{align}
			を$\varphi \rarrow \psi$の{\bf 対偶}\index{たいぐう@対偶}
			{\bf (contraposition)}と呼ぶ.
		\end{dfn}
	\end{screen}
	
	\begin{screen}
		\begin{logicalthm}[対偶命題が導かれる]
		\ref{logicalthm:introduction_of_contraposition}
			$A$と$B$を文とするとき
			\begin{align}
				\vdash (\, A \rarrow B\, ) 
				\rarrow (\, \negation B \rarrow \negation A\, ).
			\end{align}
		\end{logicalthm}
	\end{screen}
	
	\begin{screen}
		\begin{dfn}[二重否定]
			式$\varphi$に対して,$\negation$を二つ連結させた式
			\begin{align}
				\negation \negation \varphi
			\end{align}
			を$\varphi$の{\bf 二重否定}\index{にじゅうひてい@二重否定}
			{\bf (double negation)}と呼ぶ.
		\end{dfn}
	\end{screen}
	
	\begin{screen}
		\begin{logicalthm}[二重否定の導入]
		\ref{logicalthm:introduction_of_double_negation}
			$A$を文とするとき
			\begin{align}
				\vdash A \rarrow \negation \negation A.
			\end{align}
		\end{logicalthm}
	\end{screen}
	
	\begin{screen}
		\begin{logicalaxm}[論理積の除去]
		\ref{logicalaxm:elimination_of_conjunction}
			$A$と$B$を文とするとき
			\begin{align}
				A &\wedge B \vdash A, \\
				A &\wedge B \vdash B.
			\end{align}
		\end{logicalaxm}
	\end{screen}
	
	\begin{screen}
		\begin{logicalthm}[無矛盾律]
		\ref{logicalthm:law_of_noncontradiction}
			$A$を文とするとき
			\begin{align}
				\vdash\ \negation (\, A \wedge \negation A\, ).
			\end{align}
		\end{logicalthm}
	\end{screen}
	
	\begin{screen}
		\begin{dfn}[同値記号]
			$A$と$B$を$\mathcal{L}$の式とするとき,
			\begin{align}
				A \lrarrow B  \defarrow
				(\, A \rarrow B\, ) \wedge (\, B \rarrow A\, )
			\end{align}
			により$\lrarrow$を定め,式`$A \lrarrow B$'を
			「$A$と$B$は{\bf 同値である}\index{どうち@同値}{\bf (equivalent)}」と読む.
		\end{dfn}
	\end{screen}
	
	\begin{screen}
		\begin{logicalaxm}[場合分け規則]
		\ref{logicalaxm:elimination_of_disjunction}
			$A$と$B$と$C$を文とするとき
			\begin{align}
				A \rarrow C,\ B \rarrow C \vdash A \vee B \rarrow C.
			\end{align}
		\end{logicalaxm}
	\end{screen}
	
	\begin{screen}
		\begin{logicalthm}[弱 De Morgan の法則(1)]
		\ref{logicalthm:weak_De_Morgan_law_1}
			$A$と$B$を文とするとき
			\begin{align}
				\vdash\ \negation A \wedge \negation B
				\rarrow\ \negation (\, A \vee B\, ).
			\end{align}
		\end{logicalthm}
	\end{screen}
	
	\begin{screen}
		\begin{logicalthm}[強 De Morgan の法則(1)]
		\ref{logicalthm:strong_De_Morgan_law_1}
			$A$と$B$を文とするとき
			\begin{align}
				\vdash\ \negation A \vee \negation B
				\rarrow\ \negation (\, A \wedge B\, ).
			\end{align}
		\end{logicalthm}
	\end{screen}
	
	\begin{screen}
		\begin{logicalaxm}[論理和の導入]
		\ref{logicalaxm:introduction_of_disjunction}
			$A$と$B$を文とするとき
			\begin{align}
				A &\vdash A \vee B, \\
				B &\vdash A \vee B.
			\end{align}
		\end{logicalaxm}
	\end{screen}
	
	\begin{screen}
		\begin{logicalthm}[論理和の可換律]
		\ref{logicalthm:commutative_law_of_disjunction}
			$A,B$を文とするとき
			\begin{align}
				\vdash A \vee B \rarrow B \vee A.
			\end{align}
		\end{logicalthm}
	\end{screen}
	
	\begin{screen}
		\begin{logicalaxm}[論理積の導入]
		\ref{logicalaxm:introduction_of_conjunction}
			$A$と$B$を文とするとき
			\begin{align}
				A,\ B \vdash A \wedge B.
			\end{align}
		\end{logicalaxm}
	\end{screen}
	
	\begin{screen}
		\begin{logicalthm}[弱 De Morgan の法則(2)]
		\ref{logicalthm:weak_De_Morgan_law_2}
			$A$と$B$を文とするとき
			\begin{align}
				\vdash\ \negation (\, A \vee B\, ) 
				\rarrow\ \negation A \wedge \negation B.
			\end{align}
		\end{logicalthm}
	\end{screen}
	
	\begin{screen}
		\begin{logicalthm}[論理積の可換律]
		\ref{logicalthm:commutative_law_of_conjunction}
			$A,B$を文とするとき
			\begin{align}
				\vdash A \wedge B \rarrow B \wedge A.
			\end{align}
		\end{logicalthm}
	\end{screen}
	
	\begin{screen}
		\begin{logicalaxm}[二重否定の除去]
		\ref{logicalaxm:elimination_of_double_negation}
			$A$を文とするとき以下が成り立つ:
			\begin{align}
				\negation \negation A \vdash A.
			\end{align}
		\end{logicalaxm}
	\end{screen}
	
	\begin{screen}
		\begin{logicalthm}[対偶論法の原理]
		\ref{logicalthm:proof_by_contraposition}
			$A$と$B$を文とするとき
			\begin{align}
				\vdash (\, \negation B \rarrow\ \negation A\, )
				\rarrow (\, A \rarrow B\, ).
			\end{align}
		\end{logicalthm}
	\end{screen}
	
	\begin{screen}
		\begin{logicalthm}[背理法の原理]
		\ref{logicalthm:proof_by_contradiction}
			$A$を文とするとき
			\begin{align}
				\vdash (\, \negation A \rarrow \bot\, ) \rarrow A.
			\end{align}
		\end{logicalthm}
	\end{screen}
	
	\begin{screen}
		\begin{logicalthm}[爆発律]
		\ref{logicalthm:principle_of_explosion}
			$A$を文とするとき
			\begin{align}
				\vdash \bot \rarrow A.
			\end{align}
		\end{logicalthm}
	\end{screen}
	
	\begin{screen}
		\begin{logicalthm}[否定の論理和は含意で書ける]
		\ref{logicalthm:disjunction_of_negation_rewritable_by_implication}
			$A$と$B$を文とするとき
			\begin{align}
				\vdash\ \negation A \vee B \rarrow (\, A \rarrow B\, ).
			\end{align}
		\end{logicalthm}
	\end{screen}
	
	\begin{screen}
		\begin{logicalthm}[排中律]\ref{logicalthm:law_of_excluded_middle}
			$A$を文とするとき
			\begin{align}
				\vdash A \vee \negation A.
			\end{align}
		\end{logicalthm}
	\end{screen}
	
	\begin{screen}
		\begin{logicalthm}[含意の論理和への遺伝性]
		\ref{logicalthm:heredity_of_implication_to_disjunction}
			$A,B,C$を文とするとき
			\begin{align}
				\vdash (\, A \rarrow B\, ) \rarrow (\, A \vee C \rarrow B \vee C\, ).
			\end{align}
		\end{logicalthm}
	\end{screen}
	
	\begin{screen}
		\begin{logicalthm}[含意は否定と論理和で表せる]
		\ref{logicalthm:implication_rewritable_by_disjunction_of_negation}
			$A$と$B$を文とするとき
			\begin{align}
				\vdash (\, A \rarrow B\, ) \rarrow (\, \negation A \vee B\, ).
			\end{align}
		\end{logicalthm}
	\end{screen}
	
	\begin{screen}
		\begin{logicalthm}[強 De Morgan の法則(2)]
		\ref{logicalthm:strong_De_Morgan_law_2}
			$A$と$B$を文とするとき
			\begin{align}
				\vdash\ \negation (\, A \wedge B\, )
				\rarrow\ \negation A \vee \negation B.
			\end{align}
		\end{logicalthm}
	\end{screen}
	
	\begin{screen}
		\begin{logicalaxm}[量化記号に関する規則]
		\ref{logicalaxm:rules_of_quantifiers}
			$A$を$\mathcal{L}$の式とし,$x$を$A$に自由に現れる変項とし,
			$A$に自由に現れる項が$x$のみであるとする.
			また$\tau$を任意の$\varepsilon$項とする.このとき以下を推論規則とする.
			\begin{align}
				A(\tau) &\vdash \exists x A(x), \\
				\exists x A(x) &\vdash A(\varepsilon x A(x)), \\
				\forall x A(x) &\vdash A(\tau), \\
				A(\varepsilon x \negation A(x)) &\vdash \forall x A(x).
			\end{align}
		\end{logicalaxm}
	\end{screen}
	
	\begin{screen}
		\begin{logicalthm}[量化記号に対する弱 De Morgan の法則(1)]
		\label{logicalthm:weak_De_Morgan_law_for_quantifiers_1}
			$A$を$\mathcal{L}$の式とし,$x$を$A$に自由に現れる変項とし,
			また$A$に自由に現れる変項は$x$のみであるとする.このとき
			\begin{align}
				\vdash \exists x \negation A(x) \rarrow\ \negation \forall x A(x).
			\end{align}
		\end{logicalthm}
	\end{screen}
	
	\begin{screen}
		\begin{logicalthm}[量化記号に対する弱 De Morgan の法則(2)]
		\label{logicalthm:weak_De_Morgan_law_for_quantifiers_2}
			$A$を$\mathcal{L}$の式とし,$x$を$A$に自由に現れる変項とし,
			また$A$に自由に現れる変項は$x$のみであるとする.このとき
			\begin{align}
				\vdash\ \negation \forall x A(x) \rarrow \exists x \negation A(x).
			\end{align}
		\end{logicalthm}
	\end{screen}
	
	\begin{screen}
		\begin{logicalthm}[量化記号に対する強 De Morgan の法則(1)]
		\label{logicalthm:strong_De_Morgan_law_for_quantifiers_1}
			$A$を$\mathcal{L}$の式とし,$x$を$A$に自由に現れる変項とし,
			また$A$に自由に現れる変項は$x$のみであるとする.このとき
			\begin{align}
				\vdash \forall x \negation A(x) \rarrow\ \negation \exists x A(x).
			\end{align}
		\end{logicalthm}
	\end{screen}
	
	\begin{screen}
		\begin{logicalthm}[量化記号に対する強 De Morgan の法則(2)]
		\label{logicalthm:strong_De_Morgan_law_for_quantifiers_2}
			$A$を$\mathcal{L}$の式とし,$x$を$A$に自由に現れる変項とし,
			また$A$に自由に現れる変項は$x$のみであるとする.このとき
			\begin{align}
				\vdash\ \negation \exists x A(x) \rarrow \forall x \negation A(x).
			\end{align}
		\end{logicalthm}
	\end{screen}
	
\section{その他の推論法則}
	\begin{screen}
		\begin{logicalthm}[含意の推移律]
		\ref{logicalthm:transitive_law_of_implication}
			$A,B,C$を文とするとき
			\begin{align}
				\vdash (A \rarrow B) \rarrow ((B \rarrow C) \rarrow (A \rarrow C)).
			\end{align}
		\end{logicalthm}
	\end{screen}
	
	\begin{screen}
		\begin{logicalthm}[二式が同時に導かれるならその論理積が導かれる]
		\ref{logicalthm:conjunction_of_consequences}
			$A,B,C$を文とするとき
			\begin{align}
				\vdash (A \rarrow B) \rarrow ((A \rarrow C) 
				\rarrow (A \rarrow B \wedge C)).
			\end{align}
		\end{logicalthm}
	\end{screen}
	
	\begin{screen}
		\begin{logicalthm}[含意は遺伝する]
		\ref{logicalthm:rule_of_inference_1}
			$A,B,C$を$\mathcal{L}'$の閉式とするとき以下が成り立つ:
			\begin{description}
				\item[(a)] $(A \rarrow B) \rarrow ( (A \vee C) \rarrow (B \vee C) )$.
				
				\item[(b)] $(A \rarrow B) \rarrow ( (A \wedge C) \rarrow (B \wedge C) )$.
				
				\item[(c)] $(A \rarrow B) \rarrow ( (B \rarrow C) \rarrow (A \rarrow C) )$.
				
				\item[(c)] $(A \rarrow B) \rarrow ( (C \rarrow A) \rarrow (C \rarrow B) )$.
			\end{description}
		\end{logicalthm}
	\end{screen}
	
	\begin{screen}
		\begin{logicalthm}[同値記号の遺伝性質]
		\ref{logicalthm:hereditary_of_equivalence}
			$A,B,C$を$\mathcal{L}'$の閉式とするとき以下の式が成り立つ:
			\begin{description}
				\item[(a)] $(A \lrarrow B) \rarrow ((A \vee C) \lrarrow (B \vee C))$.
				\item[(b)] $(A \lrarrow B) \rarrow ((A \wedge C) \lrarrow (B \wedge C))$.
				\item[(c)] $(A \lrarrow B) \rarrow ((B \rarrow C) \lrarrow (A \rarrow C))$.
				
				\item[(d)] $(A \lrarrow B) \rarrow ((C \rarrow A) \lrarrow (C \rarrow B))$.
			\end{description}
		\end{logicalthm}
	\end{screen}
	
	\begin{screen}
		\begin{logicalthm}[偽な式は矛盾を導く]
		\ref{logicalthm:false_and_negation_are_equivalent}
			$A$を文とするとき
			\begin{align}
				\vdash\ \negation A \rarrow (A \rarrow \bot).
			\end{align}
		\end{logicalthm}
	\end{screen}
	
	\begin{screen}
		\begin{thm}[類は集合であるか真類であるかのいずれかに定まる]
			$a$を類とするとき
			\begin{align}
				\vdash \set{a} \vee \negation \set{a}.
			\end{align}
		\end{thm}
	\end{screen}
	
	\begin{screen}
		\begin{logicalthm}[矛盾を導く式はあらゆる式を導く]
		\ref{logicalthm:formula_leading_to_contradiction_derives_any_formula}
			$A,B$を文とするとき
			\begin{align}
				\vdash (A \rarrow \bot) \rarrow (A \rarrow B).
			\end{align}
		\end{logicalthm}
	\end{screen}
	
	\begin{screen}
		\begin{logicalthm}[含意は否定と論理和で表せる]
		\ref{logicalthm:rule_of_inference_3}
			$A,B$を文とするとき
			\begin{align}
				\vdash (A \rarrow B) \lrarrow (\negation A \vee B).
			\end{align}
		\end{logicalthm}
	\end{screen}
	
	\begin{screen}
		\begin{logicalthm}[二重否定の法則の逆が成り立つ]
		\ref{logicalthm:converse_of_law_of_double_negative}
			$A$を文とするとき
			\begin{align}
				\vdash A \rarrow \negation \negation A.
			\end{align}
		\end{logicalthm}
	\end{screen}
	
	\begin{screen}
		\begin{logicalthm}[対偶命題は同値]\ref{thm:contraposition_is_true}
			$A,B$を文とするとき
			\begin{align}
				\vdash (A \rarrow B) \lrarrow (\negation B \rarrow \negation A).
			\end{align}
		\end{logicalthm}
	\end{screen}
	
	\begin{screen}
		\begin{logicalthm}[De Morganの法則]
			$A,B$を文とするとき
			\begin{itemize}
				\item $\vdash\ \negation (A \vee B) \lrarrow \negation A \wedge \negation B$.
			
				\item $\vdash\ \negation (A \wedge B) \lrarrow \negation A \vee \negation B$.
			\end{itemize}
		\end{logicalthm}
	\end{screen}
	
	\begin{screen}
		\begin{thm}[集合であり真類でもある類は存在しない]
			$a$を類とするとき
			\begin{align}
				\vdash\ \negation (\ \set{a} \wedge \negation \set{a}\ ).
			\end{align}
		\end{thm}
	\end{screen}
	
	\begin{screen}
		\begin{logicalthm}[量化記号の性質(ロ)]
		\ref{logicalthm:properties_of_quantifiers_2}
			$A,B$を$\mathcal{L}'$の式とし,$x$を$A,B$に現れる文字とするとき,$x$のみが$A,B$で量化されていないならば以下は定理である:
			\begin{description}
				\item[(a)] $\exists x ( A(x) \vee B(x) ) \lrarrow \exists x A(x) \vee \exists x B(x)$.
				
				\item[(b)] $\forall x ( A(x) \wedge B(x) ) \lrarrow \forall x A(x) \wedge \forall x B(x)$.
			\end{description}
		\end{logicalthm}
	\end{screen}
	
	\begin{screen}
		\begin{logicalthm}[量化記号の性質(イ)]
		\ref{logicalthm:properties_of_quantifiers}
			$A,B$を$\mathcal{L}'$の式とし,$x$を$A,B$に現れる文字とし,$x$のみが$A,B$で量化されていないとする.
			$\mathcal{L}$の任意の対象$\tau$に対して
			\begin{align}
				A(\tau) \lrarrow B(\tau)
			\end{align}
			が成り立っているとき,
			\begin{align}
				\exists x A(x) \lrarrow \exists x B(x)
			\end{align}
			および
			\begin{align}
				\forall x A(x) \lrarrow \forall x B(x)
			\end{align}
			が成り立つ.
		\end{logicalthm}
	\end{screen}
	
	\begin{screen}
		\begin{logicalthm}[量化記号に対する De Morgan の法則]
		\ref{logicalthm:De_Morgan_law_for_quantifiers}
			$A$を$\mathcal{L}'$の式とし,$x$を$A$に現れる文字とし,$x$のみが$A$で量化されていないとする.このとき
			\begin{align}
				\exists x \negation A(x) \lrarrow \negation \forall x A(x)
			\end{align}
			および
			\begin{align}
				\forall x \negation A(x) \lrarrow \negation \exists x A(x)
			\end{align}
			が成り立つ.
		\end{logicalthm}
	\end{screen}

\section{集合}
	\begin{screen}
		\begin{dfn}[集合]
			$a$を類とするとき,$a$が集合であるという言明を
			\begin{align}
				\set{a} \defarrow \exists x\, (\, a = x\, )
			\end{align}
			で定める.$\Sigma \vdash \set{a}$を満たす類$a$を
			{\bf 集合}\index{しゅうごう@集合}{\bf (set)}と呼び,
			$\Sigma \vdash\ \negation \set{a}$を満たす類$a$を
			{\bf 真類}\index{しんるい@真類}{\bf (proper class)}と呼ぶ.
		\end{dfn}
	\end{screen}
	
	\begin{screen}
		\begin{thm}[集合である内包項は$\varepsilon$項で書ける]
			$\varphi$を$\mathcal{L}$の式とし,$x$を$\varphi$に自由に現れる変項とし,
			$x$のみが$\varphi$で自由であるとする.このとき
			\begin{align}
				\set{\Set{x}{\varphi(x)}} \vdash \Set{x}{\varphi(x)} 
				= \varepsilon y\, \forall x\, (\, \varphi(x) \lrarrow x \in y\, ).
			\end{align}
		\end{thm}
	\end{screen}
	
\section{相等性}
	\begin{screen}
		\begin{axm}[外延性の公理 (Extensionality)]
			任意の類$a,b$に対して
			\begin{align}
				\EXTAX \defarrow \forall x\, (\, x \in a \lrarrow x \in b\, ) 
				\rarrow a=b.
			\end{align}
		\end{axm}
	\end{screen}
	
	\begin{screen}
		\begin{thm}[任意の類は自分自身と等しい]\ref{thm:any_class_equals_to_itself}
			任意の類$\tau$に対して
			\begin{align}
				\EXTAX \vdash \tau = \tau.
			\end{align}
		\end{thm}
	\end{screen}
	
	\begin{screen}
		\begin{thm}[類である$\varepsilon$項は集合である]
			$\tau$を類である$\varepsilon$項とするとき
			\begin{align}
				\EXTAX \vdash \set{\tau}.
			\end{align}
		\end{thm}
	\end{screen}
	
	\begin{screen}
		\begin{axm}[相等性公理]
			$a,b,c$を類とするとき
			\begin{align}
				\EQAX \defarrow
				\begin{cases}
					a = b \rarrow b = a, & \\
					a = b \rarrow (\, a \in c \rarrow b \in c\, ), & \\
					a = b \rarrow (\, c \in a \rarrow c \in b\, ). & 
				\end{cases}
			\end{align}
		\end{axm}
	\end{screen}
	
	\begin{screen}
		\begin{thm}[外延性の公理の逆も成り立つ]
		\ref{thm:inverse_of_axiom_of_extensionality}
			$a$と$b$を類とするとき
			\begin{align}
				\EQAX \vdash 
				a = b \rarrow \forall x\, (\, x \in a  \lrarrow x \in b\, ).
			\end{align}
		\end{thm}
	\end{screen}
	
	\begin{screen}
		\begin{axm}[内包性公理] 
			$\varphi$を$\mathcal{L}$の式とし,$y$を$\varphi$に自由に現れる変項とし,
			$\varphi$に自由に現れる項は$y$のみであるとし,
			$x$は$\varphi$で$y$への代入について自由であるとするとき,
			\begin{align}
				\COMAX \defarrow \forall x\, (\, x \in \Set{y}{\varphi(y)} \lrarrow \varphi(x)\, ).
			\end{align}
		\end{axm}
	\end{screen}
	
	\begin{screen}
		\begin{thm}[条件を満たす集合は要素である]\ref{thm:satisfactory_set_is_an_element}
			$\varphi$を$\mathcal{L}$の式とし,$x$を$\varphi$に自由に現れる変項とし,
			$x$のみが$\varphi$で束縛されていないとする.このとき,任意の類$a$に対して
			\begin{align}
				\EQAX,\COMAX \vdash \varphi(a) \rarrow 
				\left(\, \set{a} \rarrow a \in \Set{x}{\varphi(x)}\, \right).
			\end{align}
		\end{thm}
	\end{screen}
	
	\begin{screen}
		\begin{dfn}[宇宙]
			$\Univ \defeq \Set{x}{x=x}$で定める類$\Univ$を{\bf 宇宙}\index{うちゅう@宇宙}
			{\bf (Universe)}と呼ぶ.
		\end{dfn}
	\end{screen}
	
	\begin{screen}
		\begin{axm}[要素の公理]
			要素となりうる類は集合である.つまり,$a,b$を類とするとき
			\begin{align}
				\ELEAX \defarrow a \in b \rarrow \set{a}.
			\end{align}
		\end{axm}
	\end{screen}
	
	\begin{screen}
		\begin{thm}[$\Univ$は集合の全体である]
		\ref{thm:V_is_the_whole_of_sets}
			$a$を類とするとき次が成り立つ:
			\begin{align}
				\EXTAX,\EQAX,\ELEAX,\COMAX \vdash \set{a} \lrarrow a \in \Univ.
			\end{align}
		\end{thm}
	\end{screen}
	
	\begin{screen}
		\begin{logicalthm}[同値関係の可換律]
		\ref{logicalthm:commutative_law_of_equivalence_symbol}
			$A,B$を$\mathcal{L}$の文とするとき
			\begin{align}
				\vdash (A \lrarrow B) \rarrow (B \lrarrow A).
			\end{align}
		\end{logicalthm}
	\end{screen}
	
	\begin{screen}
		\begin{logicalthm}[同値関係の推移律]
		\ref{logicalthm:transitive_law_of_equivalence_symbol}
			$A,B,C$を$\mathcal{L}$の文とするとき
			\begin{align}
				\vdash (A \lrarrow B) \rarrow ((B \lrarrow C) \rarrow 
				(A \lrarrow C)).
			\end{align}
		\end{logicalthm}
	\end{screen}
	
	\begin{screen}
		\begin{thm}[等号の推移律]\ref{thm:transitive_law_of_equality}
			$a,b,c$を類とするとき
			\begin{align}
				\EXTAX,\EQAX \vdash a = b \rarrow (\, a = c \rarrow b = c\, ).
			\end{align}
		\end{thm}
	\end{screen}
	
\section{代入原理}
	\begin{screen}
		\begin{axm}[$\varepsilon$項に対する相等性公理]
			$a,b$を類とし,$\varphi$を$\lang{\varepsilon}$の式とし,$\varphi$には変項$x,y$が
			自由に現れ,また$\varphi$に自由に現れる変項はこれらのみであるとする.このとき
			\begin{align}
				\EQAXEP \defarrow
				a = b \rarrow \varepsilon x \varphi(x,a) = \varepsilon x \varphi(x,b).
			\end{align}
		\end{axm}
	\end{screen}
	
	\begin{screen}
		\begin{thm}[代入原理]\ref{thm:the_principle_of_substitution}
			$a,b$を類とし,$\varphi$を$\mathcal{L}$の式とし,$x$を$\varphi$に自由に現れる変項
			とし,$\varphi$に自由に現れる変項は$x$のみであるとする.このとき
			\begin{align}
				\EXTAX,\EQAX,\EQAXEP \vdash a = b \rarrow 
				(\, \varphi(a) \lrarrow \varphi(b)\, ).
			\end{align}
		\end{thm}
	\end{screen}

\section{空集合}
	\begin{screen}
		\begin{logicalthm}[分配された論理積の簡約]
		\ref{logicalthm:contraction_law_of_distributed_injunctions}
			$A,B,C$を$\mathcal{L}$の文とするとき,
			\begin{align}
				\vdash (A \wedge C) \wedge (B \wedge C) \rarrow A \wedge B.
			\end{align}
		\end{logicalthm}
	\end{screen}
	
	\begin{screen}
		\begin{dfn}[空集合]
			$\emptyset \defeq \Set{x}{x \neq x}$で定める類$\emptyset$を{\bf 空集合}\index{くうしゅうごう@空集合}{\bf (empty set)}と呼ぶ.
		\end{dfn}
	\end{screen}
	
	\begin{screen}
		\begin{axm}[置換公理]
			$\varphi$を$\mathcal{L}$の式とし,
			$s,t$を$\varphi$に自由に現れる変項とし,
			$\varphi$に自由に現れる項は$s,t$のみであるとし,
			$x$は$\varphi$で$s$への代入について自由であり,
			$y,z,v$は$\varphi$で$t$への代入について自由であるとするとき,
			\begin{align}
				\REPAX \defarrow \forall x\, \forall y\, \forall z\, 
				(\, \varphi(x,y) \wedge \varphi(x,z)
				\rarrow y = z\, )
				\rarrow \forall a\, \exists u\, \forall v\,
				(\, v \in u \lrarrow \exists x\, (\, x \in a \wedge 
				\varphi(x,v)\, )\, ).
			\end{align}
		\end{axm}
	\end{screen}
	
	\begin{screen}
		\begin{thm}[分出定理]\ref{thm:axiom_of_separation}
			$\varphi$を$\mathcal{L}$の式とし,$x$を$\varphi$に自由に現れる変項とし,
			$\varphi$に自由に現れる項は$x$のみであるとする.このとき
			\begin{align}
				\EXTAX,\EQAX,\EQAXEP,\REPAX \vdash 
				\forall a\, \exists s\, \forall x\,
				(\, x \in s \lrarrow x \in a \wedge \varphi(x)\, ).
			\end{align}
		\end{thm}
	\end{screen}
	
	\begin{screen}
		\begin{thm}[$\emptyset$は集合]\ref{thm:emptyset_is_a_set}
			\begin{align}
				\EXTAX,\EQAX,\COMAX,\REPAX \vdash \set{\emptyset}.
			\end{align}
		\end{thm}
	\end{screen}
	
	\begin{screen}
		\begin{thm}[空集合はいかなる集合も持たない]\ref{thm:emptyset_has_nothing}
			\begin{align}
				\EXTAX,\COMAX \vdash \forall x\, (\, x \notin \emptyset\, ).
			\end{align}
		\end{thm}
	\end{screen}
	
	\begin{screen}
		\begin{thm}[空の類は空集合に等しい]\ref{thm:uniqueness_of_emptyset}
			$a$を類とするとき
			\begin{align}
				\EXTAX,\COMAX &\vdash \forall x\, (\, x \notin a\, ) \rarrow a = \emptyset, \\
				\EXTAX,\EQAX,\COMAX &\vdash a = \emptyset \rarrow \forall x\, (\, x \notin a\, ).
			\end{align}
		\end{thm}
	\end{screen}
	
	\begin{screen}
		\begin{thm}[類を要素として持てば空ではない]
		\ref{thm:emptyset_does_not_contain_any_class}
			$a,b$を類とするとき
			\begin{align}
				\EQAX,\ELEAX \vdash a \in b \rarrow \exists x\, (\, x \in b\, ).
			\end{align}
		\end{thm}
	\end{screen}
	
	\begin{screen}
		\begin{dfn}[部分類]
			$x,y$を$\mathcal{L}$の項とするとき,
			\begin{align}
				x \subset y \defarrow
				\forall z\, (\, z \in x \rarrow z \in y\, )
			\end{align}
			と定める.式$z \subset y$を「$x$は$y$の{\bf 部分類}\index{ぶぶんるい@部分類}
			{\bf (subclass)}である」や「$x$は$y$に含まれる」などと翻訳し,特に$x$が集合である場合は
			「$x$は$y$の{\bf 部分集合}\index{ぶぶんしゅうごう@部分集合}{\bf (subset)}である」
			と翻訳する.また
			\begin{align}
				x \subsetneq y \defarrow x \subset y \wedge x \neq y
			\end{align}
			と定め,これを「$x$は$y$に{\bf 真に含まれる}」と翻訳する.
		\end{dfn}
	\end{screen}
	
	\begin{screen}
		\begin{thm}[空集合は全ての類に含まれる]
		\ref{thm:emptyset_if_a_subset_of_every_class}
			$a$を類とするとき
			\begin{align}
				\EXTAX,\COMAX \vdash \emptyset \subset a.
			\end{align}
		\end{thm}
	\end{screen}
	
	\begin{screen}
		\begin{thm}[類はその部分類に属する全ての類を要素に持つ]
		\ref{thm:subclass_contains_all_elements}
			$a,b,c$を類とするとき
			\begin{align}
				\EQAX,\ELEAX \vdash 
				a \subset b \rarrow (\, c \in a \rarrow c \in b\, ).
			\end{align}
		\end{thm}
	\end{screen}
	
	\begin{screen}
		\begin{thm}[$\Univ$は最大の類である]
			$a$を類とするとき
			\begin{align}
				\EXTAX,\COMAX \vdash a \subset \Univ.
			\end{align}
		\end{thm}
	\end{screen}
	
	\begin{screen}
		\begin{thm}[等しい類は相手を包含する]
		\ref{thm:equivalent_classes_includes_the_other}
			$a,b$を類とするとき
			\begin{align}
				\EQAX \vdash a = b \rarrow a \subset b \wedge b \subset a.
			\end{align}
		\end{thm}
	\end{screen}
	
	\begin{screen}
		\begin{thm}[互いに相手を包含する類同士は等しい]
		\ref{thm:mutually_including_classes_are_equivalent}
			$a,b$を類とするとき
			\begin{align}
				\EXTAX \vdash a \subset b \wedge b \subset a \rarrow a = b.
			\end{align}
		\end{thm}
	\end{screen}

\section{変換の同値性}
	\begin{screen}
		\begin{logicalthm}[同値記号の対称律]
		\ref{logicalthm:symmetry_of_equivalence_arrows}
			$A,B$を$\mathcal{L}$の文とするとき
			\begin{align}
				\vdash (A \lrarrow B) \rarrow (B \lrarrow A).
			\end{align}
		\end{logicalthm}
	\end{screen}
	
	\begin{screen}
		\begin{thm}
		\ref{thm:equivalent_formula_rewriting_1}
			$a$を主要$\varepsilon$項とし,$\psi$を$\lang{\varepsilon}$の式とし,
			$z$を$\psi$に自由に現れる変項とし,$\psi$に自由に現れる変項は$z$のみであるとする.このとき
			\begin{align}
				\EQAX,\COMAX \vdash a = \Set{z}{\psi(z)} 
				\rarrow \forall v\, (\, v \in a \lrarrow \psi(v)\, ).
			\end{align}
		\end{thm}
	\end{screen}
	
	\begin{screen}
		\begin{thm}
		\ref{thm:equivalent_formula_rewriting_2}
			$a$を主要$\varepsilon$項とし,$\psi$を$\lang{\varepsilon}$の式とし,
			$z$を$\psi$に自由に現れる変項とし,$\psi$に自由に現れる変項は$z$のみであるとする.このとき
			\begin{align}
				\EXTAX,\COMAX \vdash \forall v\, (\, v \in a \lrarrow \psi(v)\, )
				\rarrow a = \Set{z}{\psi(z)}.
			\end{align}
		\end{thm}
	\end{screen}
	
	\begin{screen}
		\begin{thm}
		\ref{thm:equivalent_formula_rewriting_3}
			$b$を主要$\varepsilon$項とし,$\varphi$を$\lang{\varepsilon}$の式とし,
			$y$を$\varphi$に自由に現れる変項とし,$\varphi$に自由に現れる変項は$y$のみ
			であるとする.このとき
			\begin{align}
				\EQAX,\COMAX \vdash \Set{y}{\varphi(y)} = b 
				\rarrow \forall u\, (\, \varphi(u) \lrarrow u \in b\, ).
			\end{align}
		\end{thm}
	\end{screen}
	
	\begin{screen}
		\begin{thm}
		\ref{thm:equivalent_formula_rewriting_4}
			$b$を主要$\varepsilon$項とし,$\varphi$を$\lang{\varepsilon}$の式とし,
			$y$を$\varphi$に自由に現れる変項とし,$\varphi$に自由に現れる変項は$y$のみ
			であるとする.このとき
			\begin{align}
				\EXTAX,\COMAX \vdash \forall u\, (\, \varphi(u) \lrarrow u \in b\, )
				\rarrow \Set{y}{\varphi(y)} = b.
			\end{align}
		\end{thm}
	\end{screen}
	
	\begin{screen}
		\begin{thm}
		\ref{thm:equivalent_formula_rewriting_5}
			$\varphi$と$\psi$を$\lang{\varepsilon}$の式とし,
			$y$を$\varphi$に自由に現れる変項とし,
			$z$を$\psi$に自由に現れる変項とし,
			$\varphi$に自由に現れる変項は$y$のみであるとし,
			$\psi$に自由に現れる変項は$z$のみであるとし,する.このとき
			\begin{align}
				\EQAX,\COMAX \vdash \Set{y}{\varphi(y)} = \Set{z}{\psi(z)}
				\rarrow \forall u\, (\, \varphi(u) \lrarrow \psi(u)\, ).
			\end{align}
		\end{thm}
	\end{screen}
	
	\begin{screen}
		\begin{thm}
		\ref{thm:equivalent_formula_rewriting_6}
			$\varphi$と$\psi$を$\lang{\varepsilon}$の式とし,
			$y$を$\varphi$に自由に現れる変項とし,
			$z$を$\psi$に自由に現れる変項とし,
			$\varphi$に自由に現れる変項は$y$のみであるとし,
			$\psi$に自由に現れる変項は$z$のみであるとし,する.このとき
			\begin{align}
				\EXTAX,\COMAX \vdash \forall u\, (\, \varphi(u) \lrarrow \psi(u)\, )
				\rarrow \Set{y}{\varphi(y)} = \Set{z}{\psi(z)}.
			\end{align}
		\end{thm}
	\end{screen}
	
	\begin{screen}
		\begin{thm}
		\ref{thm:equivalent_formula_rewriting_7}
			$a$を主要$\varepsilon$項とし,$\psi$を$\lang{\varepsilon}$の式とし,
			$z$を$\psi$に自由に現れる変項とし,$\psi$に自由に現れる変項は$z$のみであるとする.このとき
			\begin{align}
				\COMAX \vdash a \in \Set{z}{\psi(z)} \rarrow \psi(a).
			\end{align}
		\end{thm}
	\end{screen}
	
	\begin{screen}
		\begin{thm}
		\ref{thm:equivalent_formula_rewriting_8}
			$a$を主要$\varepsilon$項とし,$\psi$を$\lang{\varepsilon}$の式とし,
			$z$を$\psi$に自由に現れる変項とし,$\psi$に自由に現れる変項は$z$のみであるとする.このとき
			\begin{align}
				\COMAX \vdash \psi(a) \rarrow a \in \Set{z}{\psi(z)}.
			\end{align}
		\end{thm}
	\end{screen}
	
	\begin{screen}
		\begin{thm}
		\ref{thm:equivalent_formula_rewriting_9}
			$b$を主要$\varepsilon$項とし,$\varphi$を$\lang{\varepsilon}$の式とし,
			$y$を$\varphi$に自由に現れる変項とし,
			$\varphi$に自由に現れる変項は$y$のみであるとする.このとき
			\begin{align}
				\EQAX,\COMAX,\ELEAX \vdash \Set{y}{\varphi(y)} \in b
				\rarrow \exists s\, (\, 
				\forall u\, (\, \varphi(u) \lrarrow u \in s\, )
				\wedge s \in b\, ).
			\end{align}
		\end{thm}
	\end{screen}
	
	\begin{screen}
		\begin{thm}
		\ref{thm:equivalent_formula_rewriting_10}
			$b$を主要$\varepsilon$項とし,$\varphi$を$\lang{\varepsilon}$の式とし,
			$y$を$\varphi$に自由に現れる変項とし,
			$\varphi$に自由に現れる変項は$y$のみであるとする.このとき
			\begin{align}
				\EXTAX,\EQAX,\COMAX \vdash \exists s\, (\, \forall u\, (\, \varphi(u) \lrarrow u \in s\, ) \wedge s \in b\, ) \rarrow \Set{y}{\varphi(y)} \in b.
			\end{align}
		\end{thm}
	\end{screen}
	
	\begin{screen}
		\begin{thm}
		\ref{thm:equivalent_formula_rewriting_11}
			$\varphi$と$\psi$を$\lang{\varepsilon}$の式とし,
			$y$を$\varphi$に自由に現れる変項とし,
			$z$を$\psi$に自由に現れる変項とし,
			$\varphi$に自由に現れる変項は$y$のみであるとし,
			$\psi$に自由に現れる変項は$z$のみであるとし,する.このとき
			\begin{align}
				\EQAX,\COMAX,\ELEAX \vdash \Set{y}{\varphi(y)} \in \Set{z}{\psi(z)}
				\rarrow \exists s\, (\, 
				\forall u\, (\, \varphi(u) \lrarrow u \in s\, )
				\wedge \psi(s)\, ).
			\end{align}
		\end{thm}
	\end{screen}
	
	\begin{screen}
		\begin{thm}
		\ref{thm:equivalent_formula_rewriting_12}
			$\varphi$と$\psi$を$\lang{\varepsilon}$の式とし,
			$y$を$\varphi$に自由に現れる変項とし,
			$z$を$\psi$に自由に現れる変項とし,
			$\varphi$に自由に現れる変項は$y$のみであるとし,
			$\psi$に自由に現れる変項は$z$のみであるとし,する.このとき
			\begin{align}
				\EXTAX,\EQAX,\COMAX \vdash \exists s\, (\, \forall u\, (\, \varphi(u) \lrarrow u \in s\, ) \wedge \psi(s)\, ) \rarrow \Set{y}{\varphi(y)} \in \Set{z}{\psi(z)}.
			\end{align}
		\end{thm}
	\end{screen}
	
\section{対}
	\begin{screen}
		\begin{dfn}[対]
			$x,y$を$\mathcal{L}$の項とし,$z$を$x$にも$y$にも自由に現れない変項とするとき,
			\begin{align}
				\{x,y\} \defeq \Set{z}{x = z \vee y = z}
			\end{align}
			で$\{x,y\}$を定義し,これを$x$と$y$の{\bf 対}\index{つい@対}{\bf (pair)}と呼ぶ.
			特に$\{x,x\}$を$\{x\}$と書く.
		\end{dfn}
	\end{screen}
	
	\begin{screen}
		\begin{thm}[対は表示されている要素しか持たない]
		\ref{thm:pair_members_are_exactly_the_given_two}
			$a$と$b$を類とするとき次が成立する:
			\begin{align}
				\EXTAX,\EQAX,\COMAX \vdash 
				\forall x\, (\, x \in \{a,b\} \lrarrow a = x \vee b = x\, ).
			\end{align}
		\end{thm}
	\end{screen}
	
	\begin{screen}
		\begin{thm}[対の対称性]
		\ref{thm:commutative_law_of_pairs}
			$a$と$b$を類とするとき
			\begin{align}
				\EXTAX,\EQAX,\COMAX \vdash \{a,b\} = \{b,a\}.
			\end{align}
		\end{thm}
	\end{screen}
	
	\begin{screen}
		\begin{axm}[対の公理]
			\begin{align}
				\PAIAX \defarrow \forall x\, \forall y\, \exists p\, \forall z\, 
				(\, x = z \vee y = z \lrarrow z \in p\, ).
			\end{align}
		\end{axm}
	\end{screen}
	
	\begin{screen}
		\begin{thm}[集合の対は集合である]
		\ref{thm:pair_of_sets_is_a_set}
			$a$と$b$を類とするとき
			\begin{align}
				\EXTAX,\EQAX,\COMAX,\PAIAX \vdash 
				\set{a} \wedge \set{b} \rarrow \set{\{a,b\}}.
			\end{align}
		\end{thm}
	\end{screen}
	
	\begin{screen}
		\begin{thm}[集合は対の要素たりうる]\ref{thm:set_is_an_element_of_its_pair}
			$a$と$b$を類とするとき
			\begin{align}
				\EXTAX,\EQAX,\COMAX \vdash \set{a} \rarrow a \in \{a,b\}.
			\end{align}
		\end{thm}
	\end{screen}
	
	\begin{screen}
		\begin{thm}[真類同士の対は空]\ref{thm:pair_of_proper_classes_is_emptyset}
			$a$と$b$を類とするとき,
			\begin{align}
				\EXTAX,\EQAX,\COMAX \vdash\ 
				\negation \set{a} \wedge \negation \set{b} \rarrow \{a,b\} = \emptyset.
			\end{align}
		\end{thm}
	\end{screen}
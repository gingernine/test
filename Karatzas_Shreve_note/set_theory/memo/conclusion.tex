\chapter{結論}
	本論文ではクラス(漢字で「類」とも書かれる)を扱うための{\bf ZF}集合論の一つの拡張を提示した.
	それは主要$\varepsilon$項を集合の基準系とし,内包的記法によって書かれた記号列を
	正式に項として採用することによって為されたし,拡張に合わせて集合論の公理や
	証明の規則を変形してもその証明力が{\bf ZF}集合論の保存拡大になっていることも示した.
	$\varepsilon$項の中でも主要$\varepsilon$項のみを抽出して他は切り捨てるという点が
	Hilbertの$\varepsilon$計算との大きな違いであるが,そもそも根本的に違っているのは$\varepsilon$を用いた動機である.
	Hilbertのオリジナルの$\varepsilon$は式から量化子を抹消する(これを整約という)ための方便であり,
	この整約によって述語計算をそれより弱い命題計算の体系に翻訳することが出来る.
	これが無矛盾性の証明指針となるとHilbertは考えたわけであるが,本論文ではそのような経緯とは無関係に,
	あくまでも「存在」の「実在」化,さらに言えば量化の亘る範囲の具体化に必要な分だけを取った.
	オリジナルの精神を若干無視する形となってしまったが,Henkin拡大を具体的に構成するためには,
	式から直接項を形成するという$\varepsilon$の特質が非常に大きな助けとなった.
	Hilbertの$\varepsilon$のアイデアが無ければ素姓のわからない記号を用意するほか思いつかなかったであろうし,
	量化の規則を単純化することもできなかった筈である.
	他方の内包項に関しては,クラスを``直接的''かつ``具体的''に導入する竹内\cite{TakeuchiSet}の方法を援用した.
	このアイデアよって,一階述語論理の範疇でクラスオブジェクトを扱えるようにはなったが,
	式の書き換えの節で見たように構文的考察に関する複雑度は跳ね上がった.
	この構文的な難所はもう少し見通し良く書き直す必要がある.
	
	集合の``個数''についてもう一度言っておくと,本論文の集合論では集合もクラスも全て記号列で書けるモノであるし,
	使う記号は論理記号,文字,及び$\varepsilon$や$\natural$など数え切れる程しかないので,
	序論でも述べた通り集合もクラスも可算個しかない.これでは様々な無限を扱う集合論には不釣合に思えるが,
	G$\ddot{\mbox{o}}$delの完全性定理によると{\bf ZF}集合論が無矛盾であれば
	可算集合のモデルが作れるので別におかしい話ではない.
	
	しかし乍ら,実際に集合論を組み立てていく中で,本論文の集合論には証明が冗長になるという難点があると判明した.
	{\bf ZF}集合論の証明では変項がむき出しのまま使えるところを,
	本論文では証明は全て文で行うことにしてしまったために,一々主要$\varepsilon$項を用意する必要が出来て,
	それらの項に代用するための文字が増えてしまうという悪性のインフレが起きる.また汎化の代わりに
	$\varphi(\varepsilon x \negation \varphi(x)) \rarrow \forall x \varphi(x)$
	を用いるため,証明の初めには然るべき主要$\varepsilon$を一々宣言しなくてはならない.
	とはいえ若干冗長さを取り除く論法があって,たとえば$\forall x \varphi(x)$を示したいのなら,
	証明の始めに「任意の集合$\tau$に対して」と書けばよい.なぜなら
	$\tau$が集合であればそれは何らかの主要$\varepsilon$項に等しいわけだし,
	また任意の$\tau$で言えることは$\varepsilon x \negation \varphi(x)$に対しても言えるからである.
	同様に$\exists x \psi(x)$が導かれたら「$\psi(\sigma)$を満たす集合$\sigma$が取れる」と書けば十分である.
	
	最後に参考文献について書いておく.第\ref{chap:languages}章の構文論はKunen\cite{Kunen}に,
	第\ref{chap:inference}章と第\ref{chap:conservative_extension}章
	の論理的定理の証明及び順番は前原\cite{Maehara}と戸次\cite{Bekki}に準じている.
	第\ref{chap:set_theory}章は主に竹内\cite{TakeuchiSet}に準じているが,
	要素の公理はG$\ddot{o}$del\cite{Godel}の引用である.
	第\ref{chap:conservative_extension}章の定理\ref{metathm:Henkin_expansion_2}は
	菊池\cite{Kikuchi}を,正則証明の構成は竹内\cite{TakeuchiProof}を参考にした.
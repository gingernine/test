\chapter{結論}
	本論文ではクラス(漢字で「類」とも書かれる)を扱うための{\bf ZF}集合論の一つの拡張を提示した.
	それは主要$\varepsilon$項を集合の基準系とし,内包的記法によって書かれた記号列を
	正式に項として採用することによって為されたし,拡張に合わせて集合論の公理や
	証明の規則を変形してもその証明力が{\bf ZF}集合論の保存拡大になっていることも示した.
	$\varepsilon$項の中でも主要$\varepsilon$項のみを抽出して他は切り捨てるという点が
	Hilbertの$\varepsilon$計算との大きな違いであるが,それは目的が違うから成し得たことであり,
	本論文ではあくまでも集合の実在化,さらに言えば量化の亘る範囲の具体化に必要な分だけを取ったのである.
	本論文の集合論では集合もクラスも全て記号列で書けるモノであるし,
	使う記号は論理記号,文字,及び$\varepsilon$や$\natural$など数え切れる程しかないので,
	序論でも述べた通り集合もクラスも可算個しかない.これでは様々な無限を扱う集合論には不釣合に思えるが,
	G$\ddot{\mbox{o}}$delの完全性定理によると{\bf ZF}集合論が無矛盾であれば
	可算集合のモデルが作れるので別におかしい話ではない.
	
	しかし乍ら,実際に集合論を組み立てていく中で,本論文の集合論には大きな難点があると判明した.
	証明が冗長になる点である.{\bf ZF}集合論の証明では変項がむき出しであるまま使えるところを,
	本論文では証明は全て文で行うことにしてしまったために,一々主要$\varepsilon$項を用意して
	その項の代用のための文字が増えてしまうし,また汎化の代わりに
	\begin{align}
		\varphi(\varepsilon x \negation \varphi(x)) \rarrow \forall x \varphi(x)
	\end{align}
	を用いるため,証明の初めには然るべき主要$\varepsilon$を一々宣言しなくてはならない.
	とはいえ若干冗長さを取り除く論法があって,たとえば$\forall x \varphi(x)$を示したいのなら,
	証明の始めに「$\tau$を$\varepsilon x \negation \varphi(x)$とする」と書くのではなく
	「任意の集合$\tau$に対して」と書けばよい.なぜなら
	$\tau$が集合であればそれは何らかの主要$\varepsilon$項に等しいわけだし,
	また任意の$\tau$で言えることは$\varepsilon x \negation \varphi(x)$に対しても言えるからである.
	同様に$\exists x \psi(x)$が導かれたら,「$\sigma$を$\varepsilon x \psi(x)$とおけば
	$\psi(\sigma)$が成り立つ」と書く代わりに「$\psi(\sigma)$を満たす集合$\sigma$が取れる」と
	書けばよい.
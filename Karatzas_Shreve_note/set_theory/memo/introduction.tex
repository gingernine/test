\section{導入}
	Hilbert\cite{Hilbert}の$\varepsilon$計算とは項を形成するオペレーター$\varepsilon$を用いた
	述語計算の拡張である.
	$\varepsilon$は式$\varphi(x)$から項$\varepsilon x \varphi(x)$を作るものであり,
	この項は次の$\varepsilon$-論理式によって制御される:
	\begin{align}
		\varphi(t) \rarrow \varphi(\varepsilon x \varphi(x)).
		\label{fom:introduction_1}
	\end{align}
	Hilbertが$\varepsilon$を導入したのは述語計算を
	命題計算に埋め込むためであり,その際には$\exists$や$\forall$の付いた式を
	\begin{align}
		\varphi(\varepsilon x \varphi(x)) &\defarrow \exists x \varphi(x), \\
		\varphi(\varepsilon x \negation \varphi(x)) &\defarrow \forall x \varphi(x)
	\end{align}
	と変換する.
	
	この変換は本稿において最も重要な公理の基となるが,
	ただし本稿において$\varepsilon$を導入したのは述語計算を埋め込むためではなく,
	集合の「存在」と「実在」を同義とするためである.
	本稿で実践しているのはHilbertの$\varepsilon$計算ではなく
	一種のHenkin拡大であり,%$\varepsilon$は具体的な形でHenkin拡大を提供してくれる.
	先述の主要論理式は本稿では全く不要であって,代わりに
	\begin{align}
		\exists x \varphi(x) \rarrow \varphi(\varepsilon x \varphi(x))
		\label{fom:introduction_2}
	\end{align}
	が主要な公理となる.この式の意図するところは「$\varphi$である集合が存在すれば,
	そのような集合の一つは$\varepsilon x \varphi(x)$である」ということであり,
	$\varepsilon$項によって集合の「存在」が「実在」化されることの裏付けとなる.
	そもそも実在化の背景には,{\bf ZF (Zermelo-Fraenkel)}集合論の世界では集合が一つも
	実在していないという事実がある.なぜなら,{\bf ZF}集合論は一階述語論理の言語$\{\in\}$
	で記述されるが,この言語には集合というモノが用意されていないからである.たとえば
	\begin{align}
		\exists x\, \forall y\, (\, y \notin x\, )
	\end{align}
	は「空集合は存在する」という定理を表しているが,存在するはずの空集合を実際に取ってくることは
	出来ない.にもかかわらず多くの場面では空集合$\emptyset$は実在するモノとして扱われているが,
	これにはどういう正当性があるのかというと,一つには$\forall y\, (\, y \notin \emptyset\, )$を
	$\emptyset$の定義式として$\emptyset$を語彙に追加する{\bf 定義による拡大}
	\index{ていぎによるかくだい@定義による拡大}{\bf (extension by definition)}という方法がある.
	ところが$\varepsilon$を使えば単に
	\begin{align}
		\varepsilon x\, \forall y\, (\, y \notin x\, )
	\end{align}
	と書くだけで空集合が手に入る.
	%「定義による拡大」による実在の確保と$\varepsilon$項
	%による実在の確保の決定的な違いは,前者では存在の「唯一性」が前提であるのに対し,
	%後者は存在さえすればその一つを取れるという点である.顕現的に項を形成することで
	%この意味の選択原理を内蔵しているというのが$\varepsilon$項の特徴である.
	
	$\varepsilon$項の中でも特にその導入の意図に適っているものを本稿では
	{\bf 主要${\boldsymbol \varepsilon}$項}{\bf (critical epsilon term)}と呼ぶ.
	この用語自体はMoser$\&$Zach\cite{Moser_Zach}やMiyamoto$\&$Moser\cite{Miyamoto_Moser}でも登場し,
	本来は(\refeq{fom:introduction_1})型の主要論理式が付属するものであるが,代わりに
	(\refeq{fom:introduction_2})型の式に結びつけたのが本稿での主要$\varepsilon$項である.
	適切な公理と集合の定義によって,主要$\varepsilon$項及び主要$\varepsilon$項に
	等しいモノが全て集合であり,かつ集合はこれらに限られるという理論を構築できる.
	ちなみにこの集合の概念は定義による拡大で作られうる集合も全て網羅しているが,
	$\varepsilon$項の奔放な組み合わせがいくらでも可能であるために,
	定義による拡大で得られる集合の全てよりも遥かに多くの集合の存在を示唆する.
	この理論で第一に特徴的なのは,集合が記号列で書き尽くせるということである.
	となると,記号列とは有限個の記号を有限個並べたものでしかないから集合は可算個しかないことになるが,
	集合の中には実数が存在する筈なので,集合が可算個しかないというのは致命的な矛盾であるように思える.
	しかし濃度の話はあくまで式の翻訳の話であって,理論上は実在する集合が可算個あれば
	通常の数学が展開出来てしまうのである.実数を扱う場合だって実数を漏れなく用意
	しなければ議論が破綻するというわけはなく,「実数である」という性質を持つ集合から何が導かれるかが
	考察対象になっているのである.ただし,この理論の集合の概念を受け入れることに抵抗があるならば,
	空想的な「集合の宇宙」を仮定するのが自然なのかもしれない.
	他の特徴としては,「存在」が「実在」で補完できるために直感的な証明が組み立てやすくなったり,
	証明で用いる推論規則が三段論法のみで済むといった点がある.
	
	Bourbaki\cite{Bourbaki}や島内\cite{Shimauchi}でも$\varepsilon$計算を使った集合論を展開している
	(Bourbaki\cite{Bourbaki}では$\varepsilon$ではなく$\tau$が使われている).
	ところで,本稿では$\varepsilon$項だけではなく,「$\varphi$である集合の全体」を表す
	\begin{align}
		\Set{x}{\varphi(x)}
	\end{align}
	なるモノも取り入れる.Bourbaki\cite{Bourbaki}や島内\cite{Shimauchi}では
	\begin{align}
		\Set{x}{\varphi(x)} \defeq \varepsilon y\, \forall x\, 
		(\, \varphi(x) \lrarrow x \in y\, )
	\end{align}
	と定めるが,たしかに$\varphi(x)$がどのような式であっても$\Set{x}{\varphi(x)}$は定義されるものの,
	\begin{align}
		\exists y\, \forall x\, (\, \varphi(x) \lrarrow x \in y\, )
	\end{align}
	が成立しない場合には,見た目に反して「$\varphi$である集合の全体」という意味を持たないばかりか,
	どのようなモノであるかさえも把握できなくなってしまう.
	{\bf ZF}集合論でも定義による拡大によって$\Set{x}{\varphi(x)}$なる項を
	追加することはできるが,これも同じく
	$\exists y\, \forall x\, (\, \varphi(x) \lrarrow x \in y\, )$
	の成立が条件となる.その一方で,たとえば%$\varphi(x)$として$x = x$といったごく単純な式を取っても
	\begin{align}
		\exists y\, \forall x\, (\, x = x \lrarrow x \in y\, )
	\end{align}
	は不成立であるにもかかわらず$\Set{x}{x = x}$なるモノは非公式に項として扱われることが多い.
	存在式の不成立によって項の正体が不明になったり,存在式の不成立によって項が定義できなかったり,
	不成立にもかかわらず非公式に項を定義したり,といったいささか厄介な問題が起こるのは
	「集合しか扱わない」という立場にいることが原因であるが,これらを一気に払拭して,
	しかもその意図するところを損なわずに$\Set{x}{\varphi(x)}$なる項を
	導入できる簡単で具体的な方法がある.それは竹内\cite{TakeuchiSet}にあるように
	$\varphi$から直接$\Set{x}{\varphi(x)}$を作ればよいのである.
	
	$\Set{x}{\varphi(x)}$なる項は「モノの集まり」という観点からはまさしく「集合」
	なのだが,たとえばRussellのパラドックスが示す通り
	\begin{align}
		\Set{x}{x \notin x}
	\end{align}
	は数学の世界での集合であってはならず,「モノの集まり」を数学の世界の集合であるものと
	そうでないものとに分類しなくてはならない.数学の世界では単なる「モノの集まり」は類(class)と呼ばれ,
	集合でない類は真類(proper class)と呼ばれる.$\varepsilon$項を採用している本稿では
	\begin{align}
		\varepsilon x \varphi(x),\quad \Set{x}{\varphi(x)}
	\end{align}
	の形の項を類と定義し,「類$a$が集合である」ということは,竹内\cite{TakeuchiSet}に倣って
	\begin{align}
		\exists x\, (\, a = x\, )
	\end{align}
	が成り立つことであるとする.また$\Set{x}{\varphi(x)}$に対して
	「$\varphi$である集合の全体」の意味を実質的に与えるために,
	\begin{align}
		\forall u\, (\, u \in \Set{x}{\varphi(x)} \lrarrow \varphi(u)\, )
	\end{align}
	と
	\begin{align}
		a \in b \rarrow \exists s\, (\, a = s\, )
	\end{align}
	を集合論の公理とする.前者の公理によって$\Set{x}{\varphi(x)}$は
	「$\varphi$であるモノの全体」となり,後者の公理によって「そのモノは全て集合である」ということになる.
	
	なお,類を公式に扱う集合論には他にも{\bf BG (Bernays-G$\ddot{\mbox{o}}$del)}集合論
	(G$\ddot{\mbox{o}}$del\cite{Godel})や{\bf MK (Morse-Kelly)}集合論(Morse\cite{Morse})
	や{\bf NF (New Foundations)}集合論(Quine\cite{Quine})がある.
	{\bf BG}集合論と{\bf MK}集合論は二階述語論理に基づいて設計されており,
	つまり変項が二種類あって,それぞれ類に対するものと集合に対するものに使い分けているのだが,
	本稿では変項は一種類のみ,集合に対するものしか扱わない.本稿は一階述語論理の範疇で
	{\bf ZF}集合論の語彙を拡張しているだけであり,これらの集合論が類を全称量化するところは
	{\bf 図式}\index{ずしき@図式}{\bf (schema)}で対応する.
	また{\bf NF}とは式の階層化という概念を使用した集合論であるが,本稿ではそのようなことはしていない.
	
	本稿では{\bf ZF}集合論に上記の項や公理を追加していくわけであるが,
	この集合論の拡張は妥当である.妥当であるとは本稿の集合論が現代数学で受容可能であるということであり,
	それは{\bf ZF}集合論のどの命題に対しても「{\bf ZF}集合論で証明可能」ならば
	「本稿の集合論で証明可能」であり,逆に「本稿の集合論で証明可能」ならば「{\bf ZF}集合論で証明可能」であるという意味である.
	このような妥当な拡張のことを{\bf 保存拡大}\index{ほぞんかくだい@保存拡大}
	{\bf (conservative extension)}と呼ぶ.
	
\section{章立て}
	第\ref{chap:languages}章では集合論の言語というものを導入し,また構文論的な性質についていくつか述べる.
	その際3つの言語が登場するが,{\bf ZF}集合論の言語は$\lang{\in}$と書き,
	それに$\varepsilon$項を追加した言語を$\lang{\varepsilon}$と書き,
	最後に$\Set{x}{\varphi(x)}$なる形の項を追加した言語を$\mathcal{L}$と書く.
	第\ref{chap:inference}章では証明とは何かを規定する.本稿の証明体系は主に古典論理に準じているが
	\footnote{
		論文タイトルに「直観」と入れたが,これは{\bf 直観主義論理}
		\index{ちょっかんしゅぎろんり@直観主義論理}{\bf (intuitionistic logic)}
		の意味ではなく日常的な感覚としての意味である.
		直観主義論理は最小論理に爆発律のみを追加した論理体系であるが,
		本稿は古典論理に準じているので爆発律よりも強い二重否定の除去が公理となる.
		通ずるものがあるとすれば{\bf 構成的}な証明に出来るだけ拘っているという点である.
		%{\bf 真偽}についての観点であって,本稿では「真である」ということは「証明できること」であるとする.
		%また証明も{\bf 構成的}であることにこだわり,
		つまり「…と仮定すると矛盾する」といった{\bf 背理法}はなるべく用いない.
		ただし,その代わりに{\bf 対偶法}はよく用いる.対偶法は実際は最小論理の下で背理法と同値なのだが,
		矛盾に頼っていないように見えるという点で対偶法による証明は``構成的''になる.
	}
	,
	$\varepsilon$項を利用するために若干変更を施す.
	第\ref{chap:set_theory}章では$\mathcal{L}$と第\ref{chap:inference}章の証明体系で集合論が展開できることを実演する.
	第\ref{chap:conservative_extension}章では,本稿の集合論が{\bf ZF}集合論の保存拡大になっていることを示す.
	
	%Hilbertの$\varepsilon$計算は,項を形成するオペレーター$\varepsilon$と
	%そのような項を含む initial formula による初等的,或いは述語計算の拡張である.
	%$\varepsilon$計算の基本的な結果は$\varepsilon$定理と呼ばれ,それらは
	%$\varepsilon$除去法によって証明される.$\varepsilon$除去法とは
	%$\varepsilon$計算での証明を初等的または述語計算の証明に変換する手法であり,
	%具体的には initial formula を除去するのである.主要な結果の一つで,
	%BernaysとHilbertにより示されたHerbrandの定理は,拡張$\varepsilon$定理の系として出てくる.
	
	%Hilbertの$\varepsilon$計算は$\varepsilon$-オペレーターを用いた述語計算の拡張であり,
	%$\varepsilon$は式$A(x)$から項$\varepsilon_{x}A(x)$を作るものである.
	%このオペレーターは次の initial formula によって制御される.一つは
	%\begin{align}
	%	A(t) \rarrow A(\varepsilon_{x}A(x))
	%\end{align}
	%といった形の主要論理式である.ここで$t$は任意の項である.もう一つは
	%$\varepsilon$-等号論理式
	%\begin{align}
	%	\vec{u} = \vec{v} \rarrow 
	%	\varepsilon_{x}B(x,\vec{u}) = \varepsilon_{x}B(x,\vec{v})
	%\end{align}
	%である.ここで$\vec{u}$と$\vec{v}$は項の列$u_{0},u_{1},\cdots,u_{n-1}$と
	%$v_{0},v_{1},\cdots,v_{n-1}$であり,$\vec{u} = \vec{v}$とは
	%$u_{0} = v_{0},\ u_{1} = v_{1},\ \cdots,$及び$u_{n-1} = v_{n-1}$のことである.
	%また$\varepsilon_{x}B(x,\vec{a})$の真部分項は$\vec{a}$のみである.
	%純粋な$\varepsilon$計算は$\varepsilon$オペレーターと主要論理式による初等計算の拡張である.
	%$\varepsilon$オペレーターによって存在と全称をエンコード可能である,
	%$\exists x A(x) \defeq A(\varepsilon_{x}A(x))$や
	%$\forall x A(x) \defeq A(\varepsilon_{x}\negation A(x))$と定義ですれば
	%$\varepsilon$計算に埋め込める.
	
	%$\varepsilon$計算はHilbertプログラムの文脈で開発された.Gentzen以前の黎明期の証明論は
	%$\varepsilon$計算に集中され,$\varepsilon$-除去法,$\varepsilon$-代入法,それから
	%それらの業績はBernaysやAckermann,Von Neumannによってもたらされた.
	%$\varepsilon$計算を使ったHerbrandの定理の正しい証明は[Bus94]にある.
	%通常,定理はオリジナルのものより若干一般性を欠いて以下のように述べられる.
	%存在式の冠頭標準形$\exists \vec{x} A(\vec{x})$に対して,
	%初等計算における項$\vec{t}_{0},\vec{t}_{1},\cdots,\vec{t}_{k-1}$が取れて
	%初等計算で$A(\vec{t}_{0}) \vee A(\vec{t}_{1}) \vee \cdots \vee A(\vec{t}_{k-1})$
	%が証明される.しかし$\varepsilon$計算は独立で永続的に惹かれる,また計算機科学や
	%証明論的観点でとりわけ価値がある.
	
	%$\varepsilon$定理やHerbrandの定理を証明する流れの中で,$\varepsilon$-除去法は,
	%$\varepsilon$計算での証明を上で述べた initial formula を用いない証明に証明論的に変形する.
	%$\varepsilon$計算において$A(\vec{t})$への証明があったとすると,
	%ここで$\vec{t}$とは$\varepsilon$項が現れうる項を含んだ有限列である,
	%$\varepsilon$-除去法によって
	%$A(\vec{s}_{0}) \vee A(\vec{s}_{1}) \vee \cdots \vee A(\vec{s}_{k-1})$
	%への初等的証明が得られる.ここで$\vec{s}_{0},\vec{s}_{1},\cdots,\vec{s}_{k-1}$
	%とは$\varepsilon$が無い項である.
	%この選言は式$A(\vec{t})$のHerbrand選言と呼ばれるものであり,
	%この論文の目的はHerbrand複雑度の解析であり,それは元の式の最短のHerbrand選言の長さ$k$のことである.
	
	%Hilbertの$\varepsilon$計算の大元は形式主義にあり,我々は古典的一階論理に焦点を絞る.
	
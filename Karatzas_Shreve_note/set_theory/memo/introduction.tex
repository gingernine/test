\section{導入}
	Hilbertの$\varepsilon$計算とは項を形成するオペレーター$\varepsilon$を用いた
	述語計算の拡張である.$\varepsilon$は式$\varphi(x)$から
	項$\varepsilon x \varphi(x)$を作るものであり,
	この項は次の主要論理式によって制御される:
	\begin{align}
		\varphi(t) \rarrow \varphi(\varepsilon x \varphi(x)).
	\end{align}
	Hilbertが$\varepsilon$を導入したのは述語計算を
	命題計算に埋め込むためであり,その際には$\exists$や$\forall$の付いた式を
	\begin{align}
		\varphi(\varepsilon x \varphi(x)) &\defarrow \exists x \varphi(x), \\
		\varphi(\varepsilon x \negation \varphi(x)) &\defarrow \forall x \varphi(x)
	\end{align}
	と変換する.
	
	この変換は本稿において最も重要な公理の基となるが,
	ただし本稿において$\varepsilon$を導入したのは述語計算を埋め込むためではなく,
	集合を「具体化」するためである.本稿で実践しているのはHilbertの$\varepsilon$計算ではなく
	一種のHenkin拡大であり,先述の主要論理式は本稿では全く不要であって,代わりに
	\begin{align}
		\exists x \varphi(x) \rarrow \varphi(\varepsilon x \varphi(x))
	\end{align}
	が主要な公理となる.「具体化」に対する問題意識の源は,
	通常の公理的集合論においては集合が無定義であるという不可解さである.
	純粋に一階述語論理の言語から構築される集合論を
	``生の''集合論と呼ぶことにすれば,``生の''集合論の言語では
	集合というオブジェクトが用意されていないため「存在」は「実在」を意味しない.たとえば
	\begin{align}
		\exists x\, \forall y\, (\, y \notin x\, )
	\end{align}
	は「空集合は存在する」という定理を表しているが,存在するはずの空集合を実際に取ってくることは
	出来ないのである.それなのに集合論において$\emptyset$が恰も実在するオブジェクトとして扱われているのは,
	一つには$\forall y\, (\, y \notin \emptyset\, )$を
	$\emptyset$の定義式として$\emptyset$を言語に追加している(定義による拡張)のであろうが,
	$\varepsilon$を使えば単に
	\begin{align}
		\varepsilon x\, \forall y\, (\, y \notin x\, )
	\end{align}
	と書くだけで「存在」を「実在」に格上げする効果がある.
	適切な公理と集合の定義によって,
	$\varepsilon$項及び$\varepsilon$項に等しいオブジェクトが全て集合であり,
	かつ集合はこれらに限られるといった体系を構築できるので,
	この意味で$\varepsilon$項によって集合の「具体化」が実現する.
	その他にも$\varepsilon$項を導入することで得られるメリットとして,
	直感的な証明が組み立てやすくなったり,証明で用いる推論規則が三段論法のみで済むといった点がある.
	
	ブルバキ\cite{key5}や島内\cite{key6}でも$\varepsilon$計算を使った集合論を展開している
	(ブルバキ\cite{key5}では$\varepsilon$ではなく$\tau$が使われている).
	ところで,本稿では$\varepsilon$項だけではなく,
	「$\varphi(x)$を満たす集合$x$の全体」の役割を期して
	\begin{align}
		\Set{x}{\varphi(x)}
	\end{align}
	というオブジェクトも取り入れる.ブルバキ\cite{key5}や島内\cite{key6}では
	\begin{align}
		\Set{x}{\varphi(x)} \defeq \varepsilon x\, \forall u\, 
		(\, \varphi(u) \lrarrow u \in x\, )
	\end{align}
	と定めるが,これは欠点がある.
	\begin{align}
		\exists x\, \forall u\, (\, \varphi(u) \lrarrow u \in x\, )
	\end{align}
	が成立しない場合は「$\varphi(x)$を満たす集合$x$の全体」という意味を持たないためである.
	この欠点を解消するには,竹内\cite{key4}に倣って$\varphi$から直接$\Set{x}{\varphi(x)}$
	の形のオブジェクトを作ればよい.
	
	$\Set{x}{\varphi(x)}$なる項は「モノの集まり」という観点からはまさしく「集合」
	なのだが,たとえばRussellのパラドックスが示す通り
	\begin{align}
		\Set{x}{x \notin x}
	\end{align}
	は数学の世界での集合であってはならず,パラドックスを回避するためには
	「モノの集まり」を数学の世界の集合であるものとそうでないものとに分類しなくてはならない.
	数学の世界では単なる「モノの集まり」は類(class)と呼ばれ,
	集合でない類は真類(proper class)と呼ばれる.
	$\varepsilon$項を採用している本稿では
	\begin{align}
		\varepsilon x \varphi(x),\quad \Set{x}{\varphi(x)}
	\end{align}
	の形の項を類と定義し,類$a$が集合であることの判断基準は,竹内\cite{key4}に倣って
	\begin{align}
		\exists x\, (\, a = x\, )
	\end{align}
	が成り立つことであるとする.また$\Set{x}{\varphi(x)}$に対して
	「$\varphi(x)$を満たす集合$x$の全体」の意味を実質的に与えるために,
	\begin{align}
		\forall u\, (\, u \in \Set{x}{\varphi(x)} \lrarrow \varphi(u)\, )
	\end{align}
	と
	\begin{align}
		a \in \Set{x}{\varphi(x)} \rarrow \exists s\, (\, a = s\, )
	\end{align}
	を集合論の公理とする.前者の公理によって$\Set{x}{\varphi(x)}$は
	「$\varphi(x)$を満たす$x$の全体」となり,後者の公理によって
	「$\Set{x}{\varphi(x)}$の要素は全て集合である」ということになる.
	
	本稿で行う集合論の拡張は妥当なものである.
	妥当であるとは本稿の集合論が現代数学で受容可能であるということであり,
	それは``生の''集合論のどの命題に対しても「``生の''集合論で証明可能である」ことと
	「本稿の集合論で証明可能である」ことが同じであるという意味である.
	このような妥当な拡張のことを{\bf 保存拡大}\index{ほぞんかくだい@保存拡大}
	{\bf (conservative extension)}と呼ぶ.
	
\section{章立て}
	第2章では集合論の言語というものを導入し,また構文論的な性質についていくつか述べる.
	その際3つの言語が登場するが,``生の''集合論の言語は$\lang{\in}$と書き,
	それに$\varepsilon$項を追加した言語を$\lang{\varepsilon}$と書き,
	最後に$\Set{x}{\varphi(x)}$なる形の項を追加した言語を$\mathcal{L}$と書く.
	第3章では証明とは何かを規定する.本稿の証明体系は主に古典論理に準じているが,
	$\varepsilon$項を利用するために若干変更を施す.
	第4章では言語$\mathcal{L}$と第3章の証明体系で集合論が展開できることを実演する.
	第5章では,本稿の集合論が``生の''集合論の保存拡大になっていることを示す.
	
	メタ定理とは式や項の形状的な性質に対する主張であって,
	メタ証明はメタ定理の妥当性を日本語によって検証するものである.
	またメタ証明に必要な直感的真理をメタ公理として提示する.
	
	\begin{comment}
	Hilbertの$\varepsilon$計算は,項を形成するオペレーター$\varepsilon$と
	そのような項を含む initial formula による初等的,或いは述語計算の拡張である.
	$\varepsilon$計算の基本的な結果は$\varepsilon$定理と呼ばれ,それらは
	$\varepsilon$除去法によって証明される.$\varepsilon$除去法とは
	$\varepsilon$計算での証明を初等的または述語計算の証明に変換する手法であり,
	具体的には initial formula を除去するのである.主要な結果の一つで,
	BernaysとHilbertにより示されたHerbrandの定理は,拡張$\varepsilon$定理の系として出てくる.
	
	Hilbertの$\varepsilon$計算は$\varepsilon$-オペレーターを用いた述語計算の拡張であり,
	$\varepsilon$は式$A(x)$から項$\varepsilon_{x}A(x)$を作るものである.
	このオペレーターは次の initial formula によって制御される.一つは
	\begin{align}
		A(t) \rarrow A(\varepsilon_{x}A(x))
	\end{align}
	といった形の主要論理式である.ここで$t$は任意の項である.もう一つは
	$\varepsilon$-等号論理式
	\begin{align}
		\vec{u} = \vec{v} \rarrow 
		\varepsilon_{x}B(x,\vec{u}) = \varepsilon_{x}B(x,\vec{v})
	\end{align}
	である.ここで$\vec{u}$と$\vec{v}$は項の列$u_{0},u_{1},\cdots,u_{n-1}$と
	$v_{0},v_{1},\cdots,v_{n-1}$であり,$\vec{u} = \vec{v}$とは
	$u_{0} = v_{0},\ u_{1} = v_{1},\ \cdots,$及び$u_{n-1} = v_{n-1}$のことである.
	また$\varepsilon_{x}B(x,\vec{a})$の真部分項は$\vec{a}$のみである.
	純粋な$\varepsilon$計算は$\varepsilon$オペレーターと主要論理式による初等計算の拡張である.
	$\varepsilon$オペレーターによって存在と全称をエンコード可能である,
	$\exists x A(x) \defeq A(\varepsilon_{x}A(x))$や
	$\forall x A(x) \defeq A(\varepsilon_{x}\negation A(x))$と定義ですれば
	$\varepsilon$計算に埋め込める.
	
	$\varepsilon$計算はHilbertプログラムの文脈で開発された.Gentzen以前の黎明期の証明論は
	$\varepsilon$計算に集中され,$\varepsilon$-除去法,$\varepsilon$-代入法,それから
	それらの業績はBernaysやAckermann,Von Neumannによってもたらされた.
	$\varepsilon$計算を使ったHerbrandの定理の正しい証明は[Bus94]にある.
	通常,定理はオリジナルのものより若干一般性を欠いて以下のように述べられる.
	存在式の冠頭標準形$\exists \vec{x} A(\vec{x})$に対して,
	初等計算における項$\vec{t}_{0},\vec{t}_{1},\cdots,\vec{t}_{k-1}$が取れて
	初等計算で$A(\vec{t}_{0}) \vee A(\vec{t}_{1}) \vee \cdots \vee A(\vec{t}_{k-1})$
	が証明される.しかし$\varepsilon$計算は独立で永続的に惹かれる,また計算機科学や
	証明論的観点でとりわけ価値がある.
	
	$\varepsilon$定理やHerbrandの定理を証明する流れの中で,$\varepsilon$-除去法は,
	$\varepsilon$計算での証明を上で述べた initial formula を用いない証明に証明論的に変形する.
	$\varepsilon$計算において$A(\vec{t})$への証明があったとすると,
	ここで$\vec{t}$とは$\varepsilon$項が現れうる項を含んだ有限列である,
	$\varepsilon$-除去法によって
	$A(\vec{s}_{0}) \vee A(\vec{s}_{1}) \vee \cdots \vee A(\vec{s}_{k-1})$
	への初等的証明が得られる.ここで$\vec{s}_{0},\vec{s}_{1},\cdots,\vec{s}_{k-1}$
	とは$\varepsilon$が無い項である.
	この選言は式$A(\vec{t})$のHerbrand選言と呼ばれるものであり,
	この論文の目的はHerbrand複雑度の解析であり,それは元の式の最短のHerbrand選言の長さ$k$のことである.
	
	Hilbertの$\varepsilon$計算の大元は形式主義にあり,我々は古典的一階論理に焦点を絞る.
	\end{comment}
	
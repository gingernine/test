\section{導入}
	Hilbertの$\varepsilon$計算は,項を形成するオペレーター$\varepsilon$と
	そのような項を含む initial formula による初等的,或いは述語計算の拡張である.
	$\varepsilon$計算の基本的な結果は$\varepsilon$定理と呼ばれ,それらは
	$\varepsilon$除去法によって証明される.$\varepsilon$除去法とは
	$\varepsilon$計算での証明を初等的または述語計算の証明に変換する手法であり,
	具体的には initial formula を除去するのである.主要な結果の一つで,
	BernaysとHilbertにより示されたHerbrandの定理は,拡張$\varepsilon$定理の系として出てくる.
	
	Hilbertの$\varepsilon$計算は$\varepsilon$-オペレーターを用いた述語計算の拡張であり,
	$\varepsilon$は式$A(x)$から項$\varepsilon_{x}A(x)$を作るものである.
	このオペレーターは次の initial formula によって制御される.一つは
	\begin{align}
		A(t) \rarrow A(\varepsilon_{x}A(x))
	\end{align}
	といった形の主要論理式である.ここで$t$は任意の項である.もう一つは
	$\varepsilon$-等号論理式
	\begin{align}
		\vec{u} = \vec{v} \rarrow 
		\varepsilon_{x}B(x,\vec{u}) = \varepsilon_{x}B(x,\vec{v})
	\end{align}
	である.ここで$\vec{u}$と$\vec{v}$は項の列$u_{0},u_{1},\cdots,u_{n-1}$と
	$v_{0},v_{1},\cdots,v_{n-1}$であり,$\vec{u} = \vec{v}$とは
	$u_{0} = v_{0},\ u_{1} = v_{1},\ \cdots,$及び$u_{n-1} = v_{n-1}$のことである.
	また$\varepsilon_{x}B(x,\vec{a})$の真部分項は$\vec{a}$のみである.
	純粋な$\varepsilon$計算は$\varepsilon$オペレーターと主要論理式による初等計算の拡張である.
	$\varepsilon$オペレーターによって存在と全称をエンコード可能である,
	$\exists x A(x) \defeq A(\varepsilon_{x}A(x))$や
	$\forall x A(x) \defeq A(\varepsilon_{x}\negation A(x))$と定義ですれば
	$\varepsilon$計算に埋め込める.
	
	$\varepsilon$計算はHilbertプログラムの文脈で開発された.Gentzen以前の黎明期の証明論は
	$\varepsilon$計算に集中され,$\varepsilon$-除去法,$\varepsilon$-代入法,それから
	それらの業績はBernaysやAckermann,Von Neumannによってもたらされた.
	$\varepsilon$計算を使ったHerbrandの定理の正しい証明は[Bus94]にある.
	通常,定理はオリジナルのものより若干一般性を欠いて以下のように述べられる.
	存在式の冠頭標準形$\exists \vec{x} A(\vec{x})$に対して,
	初等計算における項$\vec{t}_{0},\vec{t}_{1},\cdots,\vec{t}_{k-1}$が取れて
	初等計算で$A(\vec{t}_{0}) \vee A(\vec{t}_{1}) \vee \cdots \vee A(\vec{t}_{k-1})$
	が証明される.しかし$\varepsilon$計算は独立で永続的に惹かれる,また計算機科学や
	証明論的観点でとりわけ価値がある.
	
	$\varepsilon$定理やHerbrandの定理を証明する流れの中で,$\varepsilon$-除去法は,
	$\varepsilon$計算での証明を上で述べた initial formula を用いない証明に証明論的に変形する.
	$\varepsilon$計算において$A(\vec{t})$への証明があったとすると,
	ここで$\vec{t}$とは$\varepsilon$項が現れうる項を含んだ有限列である,
	$\varepsilon$-除去法によって
	$A(\vec{s}_{0}) \vee A(\vec{s}_{1}) \vee \cdots \vee A(\vec{s}_{k-1})$
	への初等的証明が得られる.ここで$\vec{s}_{0},\vec{s}_{1},\cdots,\vec{s}_{k-1}$
	とは$\varepsilon$が無い項である.
	この選言は式$A(\vec{t})$のHerbrand選言と呼ばれるものであり,
	この論文の目的はHerbrand複雑度の解析であり,それは元の式の最短のHerbrand選言の長さ$k$のことである.
	
	Hilbertの$\varepsilon$計算の大元は形式主義にあり,我々は古典的一階論理に焦点を絞る.
	
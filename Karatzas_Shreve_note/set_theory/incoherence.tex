\section{徒然なるままに支離滅裂}
わからないわからないわからない

基礎論における証明は大抵が直感に頼っているように見えますが,ではその直感が正しいとは誰が保証するのでしょうか.
手元にあるどの本でも保証されていません.もしかしたら神様という超然的な存在を暗黙の裡に認めていて,
直感とは神様が用意した論理であるとして無断で使っているだけなのかもしれませんが,
残念ながら読者はテレパシーを使えないので,筆者の暗黙の了解を推察するなんて困難です.

しかしながら,暗黙の了解を排除しようとすると,その分だけ日本語による明示的な約束が必要になります.
すると新たな問題が生じます.それは日本語で書かれた言明をどこまで信用するか,という問題です.
基礎論の難しさは,その表面上のややこしさよりも日本語に対する認識を揃えることにあるのでしょうか.

論理構造を集合論の結果を用いて解明しようというのならまだしも(こちらは数理論理学と呼ばれる分野で,本来は数学基礎論とは別物だそうです),
集合論を構築することが目的である場合,その土台となる基礎論を集合論の上に展開すると理論が循環することになるでしょう.
基礎論が基礎にしている集合論は「メタ理論」と呼ばれるらしいですが,
その「メタ理論」がどう構成されたのかという点には誰も全く言及していないのですから,
「メタ理論」という言葉は単なる逃げ口上にしか聞こえず,理論の循環を解消できません.
私の考えでは,メタ理論の代わりに絶対的な原理が与えられたとして数学を構築すれば良いのです.
まあ言い方を変えて印象を良く?しようというだけの下らない事情であって,
もったいぶって思想的な立場を主張しても集合論には関係のないことなのですが.

前提:我々は数の概念を持っている.個数の概念を持っている.物の数を数えることが出来る.
数の概念とは?個数の概念とは?
ここで言う数は数学的に構成する数ではなくて,神が用意した概念としての数.
そこまで踏み込むときりがない.

排中律と無矛盾性の違い:
排中律から$\rightharpoondown (A \wedge \rightharpoondown A)$が導かれるが,
$A \wedge \rightharpoondown A$が導かれることを否定しているわけではない.

目的:いかに自然で人工的な世界を作るか.
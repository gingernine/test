\section{証明}
	本節では,「集合でも真類でもない類は存在しない」と「集合であり真類でもある類は存在しない」の二つの言明の正否の決定を主軸にして
	{\bf 推論規則}\index{すいろんきそく@推論規則}{\bf (rule of inference)}を導入し,基本的な
	{\bf 推論法則}\index{すいろんほうそく@推論法則}を導出する.
	推論法則とは他の本で{\bf 恒真式}\index{こうしんしき@恒真式}{\bf (tautology)}と呼ばれるものであるが,
	それらの本では真理表を前提にしているのに対し.本稿では真理表は用いずに推論規則から形式的に演繹していく.
	\begin{screen}
		\begin{logicalaxm}[演繹規則]
			$A,B,C,D$を文とするとき,
			\begin{description}
				\item[(a)] $A \vdash D$ならば$\vdash A \Longrightarrow D$が成り立つ.
				\item[(b)] $A,B \vdash D$ならば
					\begin{align}
						B \vdash A \Longrightarrow D,\quad
						A \vdash B \Longrightarrow D
					\end{align}
					が成り立つ.
				\item[(c)] $A,B,C \vdash D$ならば
					\begin{align}
						B,C \vdash A \Longrightarrow D,\quad
						A,C \vdash B \Longrightarrow D,\quad
						A,B \vdash C \Longrightarrow D
					\end{align}
					のいずれも成り立つ.
			\end{description}
		\end{logicalaxm}
	\end{screen}
	
	演繹規則より,たとえば
	\begin{align}
		A,A \Longrightarrow B \vdash B
	\end{align}
	であれば,
	\begin{align}
		A \vdash (A \Longrightarrow B) \Longrightarrow B
	\end{align}
	および
	\begin{align}
		\vdash A \Longrightarrow ((A \Longrightarrow B) \Longrightarrow B)
	\end{align}
	となる.これは{\bf 三段論法}\index{さんだんろんぽう@三段論法}{\bf (modus pones)}
	と呼ばれる推論規則を推論法則に直したものである.
	
	\begin{screen}
		\begin{logicalaxm}[三段論法]
			$A$と$B$を文とするとき
			\begin{align}
				A,A \Longrightarrow B \vdash B.
			\end{align}
		\end{logicalaxm}
	\end{screen}
	
	三段論法と演繹規則から
	\begin{align}
		\vdash A \Longrightarrow ((A \Longrightarrow B) \Longrightarrow B)
	\end{align}
	が得られたが,これは証明の過程で最も基本的な推論法則となる.
	ここで{\bf 証明}\index{しょうめい@証明}{\bf (proof)}とは何かを規定してしまおう.
	
	自由な変項が現れない($\mathcal{L}$の)式を{\bf 文}\index{ぶん@文}{\bf (sentence)}や
	{\bf 閉式}\index{へいしき@閉式}{\bf (closed formula)}と呼ぶ.
	証明される式や証明の過程で出てくる式は全て文である.本稿では証明された文を
	{\bf 真な}\index{しん@真}{\bf (true)}文と呼ぶことにするが,
	``証明された''や``真である''という状態は議論が立脚している前提に依存する.
	ここでいう前提とは,推論規則や言語ではなくて
	{\bf 公理系}\index{こうりけい@公理系}{\bf (axioms)}と呼ばれるものを指している.
	公理系とは文の集まりである.そしてそこに集められた文のことを``その公理系に所属する文''と呼ぶことにする.
	
	$\mathscr{S}$を公理系とするとき,$\mathscr{S}$に所属する文を$\mathscr{S}$の{\bf 公理}\index{こうり@公理}{\bf (axiom)}
	と呼ぶ.以下では本稿の集合論が立脚する公理系を$\Sigma$と書くが,$\Sigma$に属する文は単に公理と呼んだりもする.
	
	$\Sigma$とは以下の文からなる:
	\begin{description}
		\item[相等性]
		\item[外延性] $a$と$b$を類とするとき
			\begin{align}
				\forall x\, (\, x \in a \Longleftrightarrow x \in b\, )
				\Longrightarrow a = b.
			\end{align}
			
		\item[内包性] 
			\begin{align}
				\forall x\, \left(\, x \in \Set{y}{B(y)} \Longleftrightarrow B(x)\, \right).
			\end{align}
			
		\item[合併]
		\item[対]
		\item[冪]
		\item[置換]
		\item[正則性]
		\item[無限]
		\item[選択]
	\end{description}
	
	$\mathscr{S}$を公理系とするときに,文$\varphi$が$\mathscr{S}$から
	{\bf 証明された}だとか{\bf 証明可能である}\index{しょうめいかのう@証明可能}{\bf (provable)}ということは,
	\begin{itemize}
		\item $\varphi$は$\mathscr{S}$の公理である.
		\item $\vdash \varphi$である.
		\item 文$\psi$で,$\psi$と$\psi \rightarrow \varphi$が$\mathscr{S}$から証明されている
			ものが取れる.
	\end{itemize}
	
	のいずれかが満たされているということであり,$\varphi$が$\mathscr{S}$から証明可能であることを
	\begin{align}
		\mathscr{S} \vdash \varphi
	\end{align}
	と書く.
	
	$\mathscr{S}$から証明済みの$\varphi$を起点にして$\mathscr{S} \vdash \psi$であると判明すれば,
	$\varphi$から始めて$\psi$が真であることに辿り着くまでの一連の作業は$\psi$の$\mathscr{S}$からの
	{\bf 証明}\index{しょうめい@証明}{\bf (proof)}と呼ばれ,
	$\psi$は$\mathscr{S}$の{\bf 定理}\index{ていり@定理}{\bf (theorem)}と呼ばれる.
	
	$A,B \vdash \varphi$とは
	$A$と$B$の二つの文のみを公理とした体系において$\varphi$が証明可能であることを表している.
	特に{\bf 推論法則とは公理の無い体系で推論規則だけから導かれる定理}のことである.
	
	いま$A,B,C$を文とするとき,次の推論法則が示される:
	\begin{align}
		\vdash (A \Longrightarrow B) \Longrightarrow [(A \Longrightarrow (
		B \Longrightarrow C)) \Longrightarrow (A \Longrightarrow C)].
		\label{lemma_for_the_deduction_theorem}
	\end{align}
	実際,
	\begin{align}
		A \Longrightarrow B,\ A \Longrightarrow (B \Longrightarrow C),\ A
		&\vdash A, \\
		A \Longrightarrow B,\ A \Longrightarrow (B \Longrightarrow C),\ A
		&\vdash A \Longrightarrow B
	\end{align}
	より
	\begin{align}
		A \Longrightarrow B,\ A \Longrightarrow (B \Longrightarrow C),\ A
		\vdash B
	\end{align}
	が成り立つし,
	\begin{align}
		A \Longrightarrow B,\ A \Longrightarrow (B \Longrightarrow C),\ A
		&\vdash A, \\
		A \Longrightarrow B,\ A \Longrightarrow (B \Longrightarrow C),\ A
		&\vdash A \Longrightarrow (B \Longrightarrow C)
	\end{align}
	より
	\begin{align}
		A \Longrightarrow B,\ A \Longrightarrow (B \Longrightarrow C),\ A
		\vdash B \Longrightarrow C
	\end{align}
	も成り立つ.これによって
	\begin{align}
		A \Longrightarrow B,\ A \Longrightarrow (B \Longrightarrow C),\ A
		\vdash C
	\end{align}
	も成り立つから,あとは演繹規則を順次適用すれば
	\begin{align}
		A \Longrightarrow B,\ A \Longrightarrow (B \Longrightarrow C)
		&\vdash A \Longrightarrow C, \\
		A \Longrightarrow B
		&\vdash (A \Longrightarrow (B \Longrightarrow C)) \Longrightarrow
		(A \Longrightarrow C), \\
		&\vdash (A \Longrightarrow B) \Longrightarrow [(A \Longrightarrow (
		B \Longrightarrow C)) \Longrightarrow (A \Longrightarrow C)]
	\end{align}
	となる.
	
	\begin{screen}
		\begin{logicalthm}[含意の反射律]\label{logicalthm:reflective_law_of_implication}
			$A$を文とするとき
			\begin{align}
				\vdash A \Longrightarrow A.
			\end{align}
		\end{logicalthm}
	\end{screen}
	
	上の言明は``どんな文でも持ってくれば,その式に対して反射律が成立する''という意味である.
	このように無数に存在し得る定理を一括して表す式は{\bf 公理図式}\index{こうりずしき@公理図式}{\bf (schema)}と呼ばれる.
	
	\begin{prf}
		$A \vdash A$であるから,演繹法則より$\vdash A \Longrightarrow A$となる.
		\QED
	\end{prf}
	
	\begin{screen}
		\begin{logicalthm}[正しい式は仮定を選ばない]\label{logicalthm:rule_of_inference_2}
			$A,B$を文とするとき
			\begin{align}
				\vdash B \Longrightarrow (A \Longrightarrow B).
			\end{align}
		\end{logicalthm}
	\end{screen}
	
	\begin{prf}
		\begin{align}
			A,B \vdash B
		\end{align}
		より演繹法則から
		\begin{align}
			B \vdash A \Longrightarrow B
		\end{align}
		となり,再び演繹法則より
		\begin{align}
			\vdash B \Longrightarrow (A \Longrightarrow B)
		\end{align}
		が得られる.
		\QED
	\end{prf}
	
	\begin{screen}
		\begin{metaaxm}[証明に対する構造的帰納法]
			$\mathscr{S}$を公理系とする理論において,
			\begin{itemize}
				\item $\mathscr{S}$の公理に対して(...)である.
				\item 推論法則に対して(...)である.
				\item $\varphi$と$\varphi \Longrightarrow \psi$に対して
					(...)ならば$\psi$に対して(...)である.
			\end{itemize}
			ならば,$\psi$の証明過程に現れるあらゆる文に対して(...)である.
		\end{metaaxm}
	\end{screen}
	
	\begin{screen}
		\begin{metathm}[演繹法則]
			$\mathscr{S}$を任意の公理系とし,$A$と$B$を任意の文とする.このとき
			\begin{align}
				\mathscr{S},\ A \vdash B
			\end{align}
			ならば
			\begin{align}
				\mathscr{S} \vdash A \Longrightarrow B
			\end{align}
			が成り立つ.
		\end{metathm}
	\end{screen}
	
	\begin{metaprf}\mbox{}
		\begin{description}
			\item[第一段]
				$B$が$A$であるとき,含意の反射律
				(推論法則\ref{logicalthm:reflective_law_of_implication})より
				\begin{align}
					\vdash A \Longrightarrow B
				\end{align}
				が成り立つので
				\begin{align}
					\mathscr{S} \vdash A \Longrightarrow B
				\end{align}
				となる.
				
			\item[第二段]
				$B$が$\mathscr{S}$の公理であるとき,まず
				\begin{align}
					\mathscr{S} \vdash B
				\end{align}
				が成り立つが,他方で推論法則\ref{logicalthm:rule_of_inference_2}より
				\begin{align}
					\mathscr{S} \vdash B \Longrightarrow (A \Longrightarrow B) 
				\end{align}
				も成り立つので
				\begin{align}
					\mathscr{S} \vdash A \Longrightarrow B
				\end{align}
				が従う.
				
			\item[第三段]
				$C$及び$C \Longrightarrow B$が$\mathscr{S}$から証明可能な
				文$C$が取れる場合,
				\begin{align}
					\mathscr{S} \vdash A \Longrightarrow C
				\end{align}
				かつ
				\begin{align}
					\mathscr{S} \vdash A \Longrightarrow (C \Longrightarrow B)
				\end{align}
				であると仮定すれば,(\refeq{lemma_for_the_deduction_theorem})より
				\begin{align}
					\mathscr{S} \vdash 
					(A \Longrightarrow C) \Longrightarrow [(A \Longrightarrow (
					C \Longrightarrow B)) \Longrightarrow (A \Longrightarrow B)]
				\end{align}
				が満たされるので,順番に
				\begin{align}
					\mathscr{S} \vdash (A \Longrightarrow (
					C \Longrightarrow B)) \Longrightarrow (A \Longrightarrow B)
				\end{align}
				が従い,
				\begin{align}
					\mathscr{S} \vdash A \Longrightarrow B
				\end{align}
				が従う.
				\QED
		\end{description}
	\end{metaprf}
	
	演繹法則での留意点は以下のことである.
	\begin{screen}
		\begin{itemize}
			\item $B$が$A$であるとき,$\mathscr{S}$から$A \Longrightarrow B$への証明過程
				で使われる文は
				\begin{align}
					A \Longrightarrow B
				\end{align}
				だけで十分である.
				
			\item $B$が$\mathscr{S}$の公理であるとき,$\mathscr{S}$から
				$A \Longrightarrow B$への証明過程で使われる文は
				\begin{align}
					B,\quad B \Longrightarrow (A \Longrightarrow B),\quad A \Longrightarrow B
				\end{align}
				だけで十分である.
				
			\item $C$及び$C \Longrightarrow B$が$\mathscr{S}$から証明可能な
				文$C$が取れる場合,$\mathscr{S}$から$A \Longrightarrow B$への証明過程
				で使われる文は,$\mathscr{S}$から$A \Longrightarrow C$への証明に使われた文と,
				$\mathscr{S}$から$A \Longrightarrow (C \Longrightarrow B)$への証明に使われた文に,
				\begin{align}
					&(A \Longrightarrow C) \Longrightarrow [(A \Longrightarrow (
					C \Longrightarrow B)) \Longrightarrow (A \Longrightarrow B)], \\
					&(A \Longrightarrow (
					C \Longrightarrow B)) \Longrightarrow (A \Longrightarrow B), \\
					&A \Longrightarrow B
				\end{align}
				を加えたものである.
		\end{itemize}
	\end{screen}
	
	これによって演繹規則の逆が得られる.つまり,
	$\mathscr{T}$を文の集合とし,$\psi$と$\varphi$を文とするとき,
	\begin{align}
		\mathscr{T} \vdash \psi \Longrightarrow \varphi
	\end{align}
	であれば
	\begin{align}
		\mathscr{T},\psi \vdash \varphi
	\end{align}
	が成り立つ.実際,
	\begin{align}
		\mathscr{T},\psi \vdash \psi
	\end{align}
	かつ
	\begin{align}
		\mathscr{T},\psi \vdash \psi \Longrightarrow \varphi
	\end{align}
	であるから,
	\begin{align}
		&\psi \\
		&\psi \Longrightarrow \varphi \\
		&\psi \Longrightarrow ((\psi \Longrightarrow \varphi) \Longrightarrow \varphi) \\
		&(\psi \Longrightarrow \varphi) \Longrightarrow \varphi \\
		&\varphi
	\end{align}
	が$\mathscr{T},\psi$から$\varphi$への証明となっている.
	
	\begin{description}
		\item[$\vee$の導入] $A$と$B$を文とするとき
			\begin{align}
				A &\vdash A \vee B, \\
				B &\vdash A \vee B.
			\end{align}
		
		\item[$\wedge$の導入] $A$と$B$を文とするとき
			\begin{align}
				A,B \vdash A \wedge B.
			\end{align}
		
		\item[$\wedge$の除去] $A$と$B$を文とするとき
			\begin{align}
				A &\wedge B \vdash A, \\
				A &\wedge B \vdash B.
			\end{align}
			
		\item[場合分け法則] $A$と$B$と$C$を文とするとき
			\begin{align}
				A \vee B, A \Longrightarrow C, B \Longrightarrow C \vdash C.
			\end{align}
	\end{description}
	
	例えばいま
	\begin{align}
		\mathscr{S} \vdash A
	\end{align}
	かつ
	\begin{align}
		\mathscr{S} \vdash B
	\end{align}
	であるとすれば
	\begin{align}
		\mathscr{S} \vdash A \wedge B
	\end{align}
	が成り立つ.実際,$\wedge$の導入に演繹規則を二度適用すれば
	\begin{align}
		\vdash A \Longrightarrow (B \Longrightarrow (A \wedge B))
	\end{align}
	が成り立つのであるから,
	$\mathscr{S}$からの$A$への証明に$A \Longrightarrow (B \Longrightarrow (A \wedge B))$と
	$B \Longrightarrow (A \wedge B)$を追加した文の列は$\mathscr{S}$からの
	$B \Longrightarrow (A \wedge B)$への証明となり,ここに$\mathscr{S}$からの$B$への証明を追加して
	最後に$A \wedge B$を載せれば,その文の列は$\mathscr{S}$からの$A \wedge B$への証明となっている.
	
	
	
	\begin{align}
		A(\varepsilon x \rightharpoondown A(x)) \vdash \forall x A(x)
	\end{align}
	によって
	\begin{align}
		\Set{A(\tau)}{\tau:term} \vdash \forall x A(x)
	\end{align}
	となる.
	\begin{align}
		\vdash A(\varepsilon x \rightharpoondown A(x)) \Longrightarrow \forall x A(x)
	\end{align}
	と
	\begin{align}
		\Set{A(\tau)}{\tau:term} \vdash A(\varepsilon x \rightharpoondown A(x))
	\end{align}
	より.
	\begin{align}
		\vdash\ \rightharpoondown A(\tau) \Longrightarrow \rightharpoondown A(\varepsilon x \rightharpoondown A(x))
	\end{align}
	より
	\begin{align}
		\vdash A(\varepsilon x \rightharpoondown A(x)) \Longrightarrow A(\tau).
	\end{align}
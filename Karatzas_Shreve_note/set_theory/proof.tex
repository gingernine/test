\section{証明}
	
	\begin{itemize}
		\item $\Sigma$の閉式は真である.
		\item $A$と$\rightarrow AB$が真であると判明しているならば,$B$は真である.
		\item $\rightarrow \wedge ABA$と$\rightarrow \wedge ABB$は真である.
		\item $A$と$B$が真であると判明しているならば$\wedge AB$と$\wedge BA$は真である.
		\item $\rightarrow A\vee AB$と$\rightarrow B \vee AB$は真である.
		\item $\rightarrow AC$と$\rightarrow BC$が真であると判明しているならば
			$\rightarrow \vee ABC$は真である.
		\item $\rightarrow\wedge A \rightharpoondown A \bot$は真である.
		\item $\rightarrow \rightarrow A \bot \rightharpoondown A$は真である.
		\item $\rightarrow \rightharpoondown\rightharpoondown AA$は真である.
	\end{itemize}
	
	真であると判明している式$\varphi$を起点にして,
	上の推論規則を駆使して閉式$\psi$が真であると判明すれば,
	$\varphi$から始めて$\psi$が真であることに辿り着くまでの手続きは$\psi$の証明と呼ばれ,
	$\psi$は定理と呼ばれる.
	
	証明には真であると判明している式が必要であり,その根本として選ばれた式が$\Sigma$の文である.
	$\Sigma$の文は証明なしに真であると決められているのであり,これらを公理と呼び定理と区別する.
	
	$\mathscr{S}$を文の集合とするとき,$\mathscr{S}$に属する文は$\mathscr{S}$の定理である.
	また以下の推論規則によって帰納的に$\mathscr{S}$の定理が定まっていく.
	
	\begin{description}
		\item[演繹法則] $\mathscr{T}$を文の集合とし,$\psi$を文とするとき,任意の文$\varphi$に対して
			\begin{align}
				\mathscr{T} \cup \{\psi\} \vdash \varphi
			\end{align}
			ならば
			\begin{align}
				\mathscr{T} \vdash \psi \rightarrow \varphi
			\end{align}
			
		\item[三段論法] $\psi$と$\varphi$を文とするとき,
			$\psi$と$\psi \rightarrow \varphi$が共に$\mathscr{S}$の定理ならば$\varphi$は$\mathscr{S}$の定理である.
		
		\item[$\vee$の導入] $\psi$と$\varphi$を文とするとき,
			$\psi$が$\mathscr{S}$の定理ならば$\vee \varphi \psi$と$\vee \psi \varphi$は$\mathscr{S}$の定理である.
		
		\item[$\wedge$の導入] $\psi$と$\varphi$を文とするとき,
			$\psi$と$\varphi$が$\mathscr{S}$の定理ならば$\wedge \psi \varphi$は$\mathscr{S}$の定理である.
		
		\item[$\wedge$の除去] $\psi$と$\varphi$を文とするとき,
			$\wedge \psi \varphi$が$\mathscr{S}$の定理ならば$\psi$と$\varphi$は$\mathscr{S}$の定理である.
			
		\item[場合分け法則] $A$と$B$と$C$を文とするとき,
			$\vee A B$と$A \rightarrow C$と$B \rightarrow C$が$\mathscr{S}$の定理ならば$C$は$\mathscr{S}$の定理である.
	\end{description}
	
	与えられた閉式$\varphi$が証明可能であるとは,
	\begin{itemize}
		\item 閉式$\psi$で,$\psi$と$\psi \rightarrow \varphi$が真であると判明している者が得られる.
		\item 真であると判明している閉式$\psi$と$\xi$が得られて,$\varphi$は$\psi \wedge \xi$である.
		\item 閉式$\psi$と$\xi$で,$\psi \vee \xi$と$\psi \rightarrow \varphi$と$\xi \rightarrow \varphi$が真であると判明しているものが得られる.
	\end{itemize}
	
	のいずれかの場合であり,
	\begin{align}
		\vdash \varphi
	\end{align}
	と書く.
	
	証明された式が真なる式である.では真なる式は
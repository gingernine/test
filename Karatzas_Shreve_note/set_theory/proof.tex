\section{証明}
	閉式には,「真」であるか,「偽」であるか,のどちらかのラベルが付けられる.
	「真である」という言明は,「正しい」や「成り立つ」などとも言い換えられる.
	式が真であるか偽であるかは,次の手順に従って発見的に判明していく.
	
	\begin{itemize}
		\item $\Sigma$の閉式は真である.
		\item $A$と$\rightarrow AB$が真であると判明しているならば,$B$は真である.
		\item $\rightarrow \wedge ABA$と$\rightarrow \wedge ABB$は真である.
		\item $A$と$B$が真であると判明しているならば$\wedge AB$と$\wedge BA$は真である.
		\item $\rightarrow A\vee AB$と$\rightarrow B \vee AB$は真である.
		\item $\rightarrow AC$と$\rightarrow BC$が真であると判明しているならば
			$\rightarrow \vee ABC$は真である.
		\item $\rightarrow\wedge A \rightharpoondown A \bot$は真である.
		\item $\rightarrow \rightarrow A \bot \rightharpoondown A$は真である.
		\item $\rightarrow \rightharpoondown\rightharpoondown AA$は真である.
	\end{itemize}
	
	真であると判明している式$\varphi$を起点にして,
	上の推論規則を駆使して閉式$\psi$が真であると判明すれば,
	$\varphi$から始めて$\psi$が真であることに辿り着くまでの手続きは$\psi$の証明と呼ばれ,
	$\psi$は定理と呼ばれる.
	
	証明には真であると判明している式が必要であり,その大元として選ばれた式が$\Sigma$の式である.
	$\Sigma$の式は証明なしに真であると決められているのであり,これらを公理と呼び定理と区別する.
	
	与えられた閉式$\varphi$が証明可能であるとは,
	\begin{itemize}
		\item 閉式$\psi$で,$\psi$と$\psi \rightarrow \varphi$が真であると判明している者が得られる.
		\item 真であると判明している閉式$\psi$と$\xi$が得られて,$\varphi$は$\psi \wedge \xi$である.
		\item 閉式$\psi$と$\xi$で,$\psi \vee \xi$と$\psi \rightarrow \varphi$と$\xi \rightarrow \varphi$が真であると判明しているものが得られる.
	\end{itemize}
	
	のいずれかの場合であり,
	\begin{align}
		\vdash \varphi
	\end{align}
	と書く.
	
	証明された式が真なる式である.では真なる式は
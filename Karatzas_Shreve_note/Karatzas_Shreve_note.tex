\documentclass[a4j,10.5pt,oneside,openany]{jsbook}
%
\usepackage{amsmath,amssymb}
\usepackage{amsthm}
\usepackage{makeidx}
\usepackage{txfonts}
\usepackage{mathrsfs} %花文字
\usepackage{mathtools} %参照式のみ式番号表示
\usepackage{latexsym} %qed
\usepackage{ascmac}
\usepackage{color}
\usepackage{relsize}
\usepackage{comment}
\usepackage{url}
\newtheoremstyle{mystyle}% % Name
	{20pt}%                      % Space above
	{20pt}%                      % Space below
	{\rm}%           % Body font
	{}%                      % Indent amount
	{\gt}%             % Theorem head font
	{.}%                      % Punctuation after theorem head
	{10pt}%                     % Space after theorem head, ' ', or \newline
	{}%                      % Theorem head spec (can be left empty, meaning `normal')
\theoremstyle{mystyle}

\allowdisplaybreaks[1]
\newcommand{\bhline}[1]{\noalign {\hrule height #1}} %表の罫線を太くする.
\newcommand{\bvline}[1]{\vrule width #1} %表の罫線を太くする.
\newtheorem{Prop}{$Proposition.$}
\newtheorem{Proof}{$Proof.$}
\newcommand{\QED}{% %証明終了
	\relax\ifmmode
		\eqno{%
		\setlength{\fboxsep}{2pt}\setlength{\fboxrule}{0.3pt}
		\fcolorbox{black}{black}{\rule[2pt]{0pt}{1ex}}}
	\else
		\begingroup
		\setlength{\fboxsep}{2pt}\setlength{\fboxrule}{0.3pt}
		\hfill\fcolorbox{black}{black}{\rule[2pt]{0pt}{1ex}}
		\endgroup
	\fi}
\newtheorem{thm}{定理}[section]
\newtheorem{dfn}[thm]{定義}
\newtheorem{prp}[thm]{命題}
\newtheorem{cor}[thm]{系}
\newtheorem{lem}[thm]{補題}
\newtheorem*{prf}{証明}
\newtheorem{rem}[thm]{注意}
\newtheorem{e.g.}[thm]{例}
\newcommand{\defunc}{\mbox{1}\hspace{-0.25em}\mbox{l}} %定義関数
\newcommand*{\sgn}[1]{\operatorname{sgn}\left( #1 \right)} %signal関数
\def\supp#1{\operatorname{supp} #1 } %support
\def\Box#1{$(\mbox{#1})$} %丸括弧つきコメント
\def\Ddot#1{$\ddot{\mathrm{#1}}$} %文中ddot
\def\Set#1#2{\left\{\ #1\ \, ; \quad #2\ \right\}} %集合の書き方
\def\eqqcolon{=\mathrel{\mathop:}} %定義=:
\def\oparrow{\overset{\mathrm{op}}{\rightarrow}} %作用素の矢印
\newcommand{\wlim}{\mbox{w-}\lim} %弱収束
\newcommand{\wstarlim}{\ast \mbox{w-}\lim} %弱収束
\def\max#1#2{\operatorname*{max}_{#1} #2 } %最大
\def\min#1#2{\operatorname*{min}_{#1} #2 } %最小
\def\sin#1#2{\operatorname{sin}^{#2} #1} %sin
\def\cos#1#2{\operatorname{cos}^{#2} #1} %cos
\def\tan#1#2{\operatorname{tan}^{#2} #1} %tan
\def\inprod<#1>{\left\langle #1 \right\rangle} %内積
\def\sup#1#2{\operatorname*{sup}_{#1} #2 } %上限
\def\inf#1#2{\operatorname*{inf}_{#1} #2 } %下限
\def\esssup#1#2{\operatorname*{ess\mbox{.}sup}_{#1} #2 } %本質的上限
\def\Vector#1{\mbox{\boldmath $#1$}} %ベクトルを太字表示
\def\Norm#1#2{\left\|\, #1\, \right\|_{#2} } %ノルム
\def\Log#1{\operatorname{log} #1} %log
\def\Rank#1{\operatorname{rank} #1} %階数
\def\Det#1{\operatorname{det} (#1)} %行列式
\def\Diag#1{\operatorname{diag} \left(#1\right)} %行列の対角成分
\def\Tmat#1{#1^\mathrm{T}} %転置行列
\def\Dim#1{\operatorname{dim} #1} %次元
\def\Codim#1{\operatorname{codim} #1} %余次元
\def\Dom#1 {\mathcal{D}\left(#1\right)} %作用素の定義域
\def\Graph#1 {\mathcal{G}\left(#1\right)} %作用素のグラフ
\def\Ran#1{\mathcal{R}\left(#1\right)} %作用素の値域
\def\Ker#1{\mathcal{N}\left(#1\right)} %作用素の核
\def\Bop#1#2{\mathrm{B}(#1,#2)} %有界作用素の空間
\def\selfBop#1{\mathrm{B}(#1)} %有界作用素の空間[始集合=終集合]
\def\Cop#1#2{\mathrm{B}_c(#1,#2)} %コンパクト作用素の空間
\def\selfCop#1{\mathrm{B}_c(#1)} %コンパクト作用素の空間[始集合=終集合]
\def\Res#1{\mathfrak{\rho} (#1)} %レゾルベント集合
\def\Spctr#1{\mathfrak{\sigma} (#1)} %スペクトル集合
\def\pSpctr#1{\mathfrak{\sigma}_p (#1)} %点スペクトル集合
\def\Oproj#1{\mathrm{Proj}( #1 )} %直交射影(orthogonal projection)
\def\Exp#1{\operatorname{E} \left[ #1 \right]} %期待値
\def\Var#1{\operatorname{V} \left[ #1 \right]} %分散
\def\Cov#1#2{\operatorname{Cov} \left[ #1,\ #2 \right]} %共分散
\def\LH#1{\mathrm{L.h.}\left[\, #1\, \right] } %線型包(linear hull)
\def\exp#1{e^{#1}} %指数関数
\def\N{\mathbb{N}} %自然数全体
\def\Q{\mathbb{Q}} %有理数全体
\def\R{\mathbb{R}} %実数全体
\def\Z{\mathbb{Z}} %整数全体
\def\C{\mathbb{C}} %複素数全体
\def\card#1{\mathrm{card}\ #1} %濃度
\def\K{\mathbb{K}} %係数体K
\def\im{\mathrm{i}} %虚数単位
\def\Re#1{\mathrm{Re} #1} %実部
\def\Im#1{\mathrm{Im} #1} %虚部
\def\conj#1{\overline{#1}} %共役複素数
\def\borel#1{\mathscr{B}(#1)} %Borel集合族
\def\open#1{\mathscr{O}(#1)} %位相空間 #1 の位相
\def\close#1{\mathscr{A}(#1)} %%位相空間 #1 の閉集合系
\def\nbh#1{\mathscr{V}_{#1}} %位相空間の点 #1 の近傍全体
\def\fnbh#1{\mathscr{B}_{#1}} %基本近傍系 fundamental neighborhood
\newcommand*{\interior}[1]{{\kern0pt#1}^{\mathrm{o}}} %内核
\def\ball#1#2{\operatorname{B} \left(#1\, ;\, #2 \right)} %開球
\def\closure#1{\overline{#1}} %閉包
\def\rapid#1{\mathcal{S}(#1)} %急減少空間
\def\tempdist#1{\mathcal{S}'(#1)} %緩増加超関数空間 tempered distribution
\def\comtempdist#1{\mathcal{E}'(#1)} %コンパクト台を持つ緩増加超関数空間
\def\ft#1{\mathscr{F}[#1]} %Fourier変換
\def\cft#1{\overline{\mathscr{F}}[#1]} %逆Fourier変換
\def\c#1{C(#1)} %連続関数
\def\cbound#1{C_b (#1)} %有界連続関数
\def\supp#1{\operatorname{supp} #1} %関数の台
\def\ckon#1{C_c (#1)} %コンパクトな台を持つ連続関数
\def\cvan#1{C_0 (#1)} %遠方で0になる連続関数
\def\Lp#1#2{\operatorname{L}^{#1} \left(#2\right)} %L^p
\def\cn#1#2{C^{#2} (#1)} %n回連続微分可能関数
\def\cinf#1{C^{\infty} (#1)} %無限回連続微分可能関数
\def\Test#1{\mathscr{D}(#1)} %テスト関数の空間
\def\sgmalg#1{\sigma \left[#1\right]} %#1が生成するσ加法族
\def\prob#1{\operatorname{P} \left(#1\right)} %確率
\def\cprob#1#2{\operatorname{P} \left(\left\{ #1 \ \middle|\ #2 \right\}\right)} %条件付確率
\def\cexp#1#2{E\left( #1 \, \middle|\, #2 \right)} %条件付期待値
%\renewcommand{\contentsname}{\bm Index}
%
\makeindex
%
\setlength{\textwidth}{\fullwidth}
\setlength{\textheight}{40\baselineskip}
\addtolength{\textheight}{\topskip}
%\setlength{\voffset}{-0.55in}
%
%
\title{Karatzas-Shreve solutions}
\author{}
\date{\today}

\begin{document}
%
%

\mathtoolsset{showonlyrefs = true}
\maketitle
%
%
\tableofcontents
\frontmatter
%
\mainmatter
%
%本文
\chapter{Martingales, Stopping Times, and Filtrations}
\section{Stochastic Processes and $\sigma$-Fields}
\begin{itembox}[l]{Problem 1.5 修正}
	Let $Y$ be a modification of $X$, and suppose that \textcolor{red}{every 
	sample path of both processes are right-continuous sample paths.} 
	Then $X$ and $Y$ are indistinguishable.
\end{itembox}

\begin{prf}
	$X,Y$のパスの右連続性より
	\begin{align}
		\left\{ X_t = Y_t,\ \forall t \geq 0 \right\}
		= \bigcap_{r \in \Q \cap [0,\infty)} \left\{ X_r = Y_r \right\}
	\end{align}
	が成立するから,$P(X_r = Y_r) = 1\   (\forall r \geq 0)$より
	\begin{align}
		P(X_t = Y_t,\ \forall t \geq 0)
		= P \biggl( \bigcap_{r \in \Q \cap [0,\infty)} \left\{ X_r = Y_r \right\} \biggr)
		= 1
	\end{align}
	が従う.
	\QED
\end{prf}

\begin{itembox}[l]{Problem 1.7}
		Let $X$ be a process with every sample path RCLL. 
		Let $A$ be the event that $X$ is continuous on $[0,t_0)$. 
		Show that $A \in \mathscr{F}^X_{t_0}$.
\end{itembox}

\begin{prf}[参照元:\cite{key2}]
	$[0,t_0)$に属する有理数の全体を$\Q^* \coloneqq \Q \cap [0,t_0)$と表すとき,
	\begin{align}
		A = \bigcap_{m \geq 1} \bigcup_{n \geq 1} \bigcap_{\substack{p,q \in \Q^* \\ |p-q| < 1/n}}
		\Set{\omega \in \Omega}{\left|X_p(\omega) - X_q(\omega) \right| < \frac{1}{m}}
	\end{align}
	が成立することを示せばよい.これが示されれば,$\omega \longmapsto \left(X_p(\omega), X_q(\omega) \right)$の
	$\mathscr{F}^X_{t_0}/\borel{\R^2}$-可測性と
	\begin{align}
		\Phi:\R \times \R \ni (x,y) \longmapsto |x-y| \in \R
	\end{align}
	の$\borel{\R^2}/\borel{\R}$-可測性より
	\begin{align}
		\Set{\omega \in \Omega}{\left|X_p(\omega) - X_q(\omega) \right| < \frac{1}{m}}
		= \Set{\omega \in \Omega}{\left(X_p(\omega), X_q(\omega) \right) \in 
		\Phi^{-1}\left(B_{1/m}(0)\right)}
		\in \mathscr{F}^X_{t_0}
	\end{align}
	が得られ$A \in \mathscr{F}^X_{t_0}$が従う.$\left(B_{1/m}(0) = \Set{x \in \R}{|x| < 1/m}.\right)$
	\begin{description}
		\item[第一段]
			$\omega \in A^c$を任意にとる.このとき
			或る$s \in (0,t_0)$が存在して,$t \longmapsto X_t(\omega)$は
			$t = s$において左側不連続である.従って或る$m \geq 1$については,
			任意の$n \geq 1$に対し$0< s-u < 1/3n$を満たす
			$u$が存在して
			\begin{align}
				\left|X_u(\omega) - X_s(\omega) \right| \geq \frac{1}{m}
			\end{align}
			を満たす.一方でパスの右連続性より
			$0 < p - s,\ q - u < 1/3n$を満たす$p,q \in \Q^*$が存在して
			\begin{align}
				\left|X_p(\omega) - X_s(\omega) \right| < \frac{1}{4m},
				\quad \left|X_q(\omega) - X_u(\omega) \right| < \frac{1}{4m}
			\end{align}
			が成立する.このとき$0 < |p - q| < 1/n$かつ
			\begin{align}
				\left|X_p(\omega) - X_q(\omega) \right|
				\geq \left|X_p(\omega) - X_s(\omega) \right|
					- \left|X_s(\omega) - X_u(\omega) \right|
					- \left|X_q(\omega) - X_u(\omega) \right|
				\geq \frac{1}{2m}
			\end{align}
			が従い
			\begin{align}
				\omega \in \bigcup_{m \geq 1} \bigcap_{n \geq 1} \bigcup_{\substack{p,q \in \Q^* \\ |p-q| < 1/n}}
		\Set{\omega \in \Omega}{\left|X_p(\omega) - X_q(\omega) \right| \geq \frac{1}{m}}
			\end{align}
			を得る.
		
		\item[第二段]
			任意に$\omega \in A$を取る.各点で有限な左極限が存在するという仮定から,
			\begin{align}
				X_{t_0}(\omega) \coloneqq \lim_{t \uparrow t_0}X_t(\omega)
			\end{align}
			と定めることにより
			\footnote{
				実際$X_{t_0}(\omega)$は所与のものであるが,いまは$[0,t_0]$上での連続性を考えればよいから
				便宜上値を取り替える.
			}
			$t \longmapsto X_t(\omega)$は$[0,t_0]$上で一様連続となる.
			従って
			\begin{align}
				\omega \in \bigcap_{m \geq 1} \bigcup_{n \geq 1} \bigcap_{\substack{p,q \in \Q^* \\ |p-q| < 1/n}}
		\Set{\omega \in \Omega}{\left|X_p(\omega) - X_q(\omega) \right| < \frac{1}{m}}
			\end{align}
			を得る.
			\QED
	\end{description}
\end{prf}

\begin{itembox}[l]{Lemma2 for Exercise 1.8}
	$T = \{1,2,3,\cdots\}$を高々可算集合とし,
	$S_i$を第二可算公理を満たす位相空間,$X_i$を
	確率空間$(\Omega,\mathscr{F},P)$上の$S_i$-値確率変数とする$(i \in T)$.
	このとき,任意の並び替え$\pi:T \longrightarrow T$
	に対して$S \coloneqq \prod_{i \in T} S_{\pi(i)}$とおけば次が成立する:
	\begin{align}
		\sigma(X_i;\ i \in T) = \Set{\left\{ (X_{\pi(1)},X_{\pi(2)},\cdots) \in A \right\}}{A \in \borel{S}}.
		\label{eq:lem2_for_chap_1_exercise_1_8_1}
	\end{align}
\end{itembox}

\begin{prf}\mbox{}
	\begin{description}
		\item[第一段]
			射影$S \longrightarrow S_{\pi(n)}$を$p_n$で表す.
			任意に$t_i \in T$を取り$n \coloneqq \pi^{-1}(i)$とおけば,
			任意の$B \in \borel{S_n}$に対して
			\begin{align}
				X_i^{-1}(B) = \left\{ (\cdots, X_{\pi(n)},\cdots) \in p_n^{-1}(B) \right\} \in \Set{\left\{ (X_{\pi(1)}, X_{\pi(2)},\cdots) \in A \right\}}{A \in \borel{S}}
			\end{align}
			が成り立つから$\sigma(X_i;\ i \in T) \subset 
			\Set{\left\{ (X_{\pi(1)}, X_{\pi(2)},\cdots) \in A \right\}}{A \in \borel{S}}$が従う.
		
		\item[第二段]
			任意の有限部分集合$j \in T$と$B_j \in \borel{S_{\pi(j)}}$に対し
			\begin{align}
				\left\{ (X_{\pi(1)}, X_{\pi(2)},\cdots) \in p_j^{-1}(B_j) \right\}
				= X_{\pi(j)}^{-1}(B_j)
				\in \sigma(X_i;\ i \in T)
			\end{align}
			が成立するから
			\begin{align}
				\Set{p_i^{-1}(B_i)}{B_i \in \borel{S_{\pi(i)}},\ i \in T}
				\subset \Set{A \in \borel{S}}{\left\{ (X_{\pi(1)}, X_{\pi(2)},\cdots) \in A \right\} \in \sigma(X_i;\ i \in T)}
			\end{align}
			が従う.右辺は$\sigma$-加法族であり,定理\ref{thm:Borel_algebra_of_products_of_second_countable_spaces}より
			左辺は$\borel{S}$を生成するから前段と併せて(\refeq{eq:lem2_for_chap_1_exercise_1_8_1})を得る.
			\QED
	\end{description}
\end{prf}

\begin{itembox}[l]{Lemma3 for Exercise 1.8}
	$X = \Set{X_t}{0 \leq t < \infty}$を確率空間$(\Omega,\mathscr{F},P)$上の$\R^d$-値確率過程とする.
	任意の空でない$S \subset [0,\infty)$に対し
	\begin{align}
		\mathcal{F}^X_S \coloneqq \sigma(X_s;\ s \in S)
	\end{align}
	とおくとき,任意の空でない$T \subset [0,\infty)$に対して次が成立する:
	\begin{align}
		\mathcal{F}^X_T \coloneqq \bigcup_{S \subset T:at\ most\ countable} \mathcal{F}^X_S.
		\label{eq:lem3_for_chap_1_exercise_1_8_1}
	\end{align}
\end{itembox}

\begin{prf}
	便宜上
	\begin{align}
		\mathcal{F} \coloneqq \bigcup_{S \subset T:at\ most\ countable} \mathcal{F}^X_S
	\end{align}
	とおく.まず,任意の$S \subset T$に対し$\mathcal{F}^X_S \subset \mathcal{F}^X_T$が成り立つから
	\begin{align}
		\mathcal{F} \subset \mathcal{F}^X_T
	\end{align}
	が従う.また$\sigma(X_t) = \mathcal{F}^X_{\{t\}},\ (\forall t \in T)$より
	\begin{align}
		\bigcup_{t \in T} \sigma(X_t) \subset \mathcal{F}
	\end{align}
	が成り立つから,あとは$\mathcal{F}$が$\sigma$-加法族であることを示せばよい.実際,
	$\mathcal{F}$は$\sigma$-加法族の合併であるから$\Omega$を含みかつ補演算で閉じる.
	また$B_n \in \mathcal{F},\ n=1,2,\cdots$に対しては,$B_n \in \mathcal{F}^X_{S_n}$を満たす
	高々可算集合$S_n \subset T$が対応して
	\begin{align}
		\bigcup_{n=1}^\infty \mathcal{F}^X_{S_n}
		= \bigcup_{n=1}^\infty \sigma(X_s;\ s \in S_n)
		\subset \sigma\biggl(X_s;\ s \in \bigcup_{n=1}^\infty S_n \biggr)
	\end{align}
	が成り立つから,
	\begin{align}
		\bigcup_{n=1}^\infty B_n \in \sigma\biggl(X_s;\ s \in \bigcup_{n=1}^\infty S_n \biggr)
		\subset \mathcal{F}
	\end{align}
	が従う.ゆえに$\mathcal{F}$は$\sigma$-加法族であり
	(\refeq{eq:lem3_for_chap_1_exercise_1_8_1})を得る.
	\QED
\end{prf}

\begin{itembox}[l]{Exercise 1.8}
	Let $X$ be a process whose sample paths are RCLL almost surely, 
	and let $A$ be the event that $X$ is continuous on $[0,t_0)$. Show 
	that $A$ can fail to be in $\mathscr{F}^X_{t_0}$, but if $\Set{\mathscr{F}_t}{t \geq 0}$ is 
	a fitration satisfying $\mathscr{F}^X_t \subset \mathscr{F}_t,\ t \geq 0$, and 
	$\mathscr{F}^X_{t_0}$ contains all $P$-null sets of $\mathscr{F}$, then $A \in \mathscr{F}_{t_0}$.
\end{itembox}

\begin{prf}\mbox{}
	\begin{description}
		\item[第一段]
			高々可算な集合$S = \{t_1,t_2,\cdots\} \subset [0,t_0]$に対し,昇順に並び替えたものを
			$t_{\pi(1)} < t_{\pi(2)} < \cdots$と表し
			\begin{align}
				\mathcal{F}^X_S \coloneqq 
				\Set{\left\{(X_{t_{\pi(1)}},X_{t_{\pi(2)}},\cdots) \in B \right\}}{B \in \borel{(\R^d)^{\# S}}}
			\end{align}
			とおく.ただし$S$が可算無限の場合は$(\R^d)^{\# S} = \R^\infty$である.
			このとき(\refeq{eq:lem2_for_chap_1_exercise_1_8_1})より
			\begin{align}
				\sigma(X_s;\ s \in S) = \mathcal{F}^X_S
			\end{align}
			が成り立ち,(\refeq{eq:lem3_for_chap_1_exercise_1_8_1})より
			\begin{align}
				\mathscr{F}^X_{t_0}
				= \sigma(X_t;\ 0 \leq t \leq t_0)
				= \bigcup_{S \subset [0,t_0]:at\ most\ countable} \mathcal{F}^X_S
			\end{align}
			が満たされる.すなわち,$\mathscr{F}^X_{t_0}$の任意の元は
			$\left\{(X_{t_1},X_{t_2},\cdots) \in B \right\},\ (t_1 < t_2 < \cdots)$の形で表される.
			
		\item[第二段]
	\end{description}
\end{prf}

\begin{itembox}[l]{Problem 1.10 unsolved}
		Let $X$ be a process with every sample path LCRL, and 
		let A be the event that $X$ is continuous on $[0,x_0]$.
		Let $X$ be adapted to a right-continuous filtration 
		$(\mathscr{F}_t)_{t \geq 0}$. Show that $A \in \mathscr{F}_{t_0}$.
\end{itembox}

\begin{prf}\mbox{}
	\begin{description}
		\item[第一段]
			$\Q^* \coloneqq \Q \cap [0,t_0]$とおく.
			いま,任意の$n \geq 1$と$r \in \Q^*$に対し
			\begin{align}
				B_n (r) \coloneqq
				\bigcup_{m \geq 1} \bigcap_{k \leq m} 
				\Set{\omega \in \Omega}{\left| X_r(\omega)-X_{r+\frac{1}{k}}(\omega) \right| \leq \frac{1}{n}}
			\end{align}
			と定めるとき,
			\begin{align}
				A = \bigcap_{r \in \Q^*} \bigcap_{n \geq 1} B_n(r)
			\end{align}
			が成立することを示す.これが示されれば,
			\begin{align}
				\Set{\omega \in \Omega}{\left| X_r(\omega)-X_{r+\frac{1}{k}}(\omega) \right| \leq \frac{1}{n}}
				\in \mathscr{F}_{r+\frac{1}{k}},
				\quad (\forall r \in \Q^*,\ k \geq 1)
			\end{align}
			とフィルトレーションの右連続性から
			\begin{align}
				B_n (r) \in \bigcap_{k \geq m} \mathscr{F}_{r+\frac{1}{k}} = \mathscr{F}_{r+} = \mathscr{F}_{r}
			\end{align}
			が従い$A \in \mathscr{F}_{t_0}$が出る.
		
		\item[第二段]
			
	\end{description}
\end{prf}

\begin{itembox}[l]{Problem 1.16}
	If the process $X$ is measurable and the random time $T$ is finite, 
	then the function $X_T$ is a random variable.
\end{itembox}

\begin{prf}
	\begin{align}
		\tau:\Omega \ni \omega \longmapsto (T(\omega),\omega) \in [0,\infty) \times \Omega
	\end{align}
	とおけば,
	任意の$A \in \borel{[0,\infty)},\ B \in \mathscr{F}$に対して
	\begin{align}
		\tau^{-1}(A \times B) = \Set{\omega \in \Omega}{(T(\omega),\omega) \in A \times B}
		= T^{-1}(A) \cap B \in \mathscr{F}
	\end{align}
	が満たされる
	\begin{align}
		\Set{A \times B}{A \in \borel{[0,\infty)},\ B \in \mathscr{F}}
		\subset \Set{E \in \borel{[0,\infty)} \otimes \mathscr{F}}{\tau^{-1}(E) \in \mathscr{F}}
	\end{align}
	が従い$\tau$の$\mathscr{F}/\borel{[0,\infty)} \otimes \mathscr{F}$-可測性が出る.
	$X_T = X \circ \tau$より$X_T$は可測$\mathscr{F}/\borel{\R^d}$である.
	\QED
\end{prf}

\begin{itembox}[l]{Problem 1.17}
	Let $X$ be a measurable process and $T$ a random time. Show that 
	the collection of all sets of the form $\{X_T \in A\}$ and 
	$\{X_T \in A\} \cup \{T = \infty\};A \in \borel{\R}$, forms a 
	sub-$\sigma$-field of $\mathscr{F}$.
\end{itembox}

\begin{prf}
	$X_T$の定義域は$\{T<\infty\}$であるから,
	\begin{align}
		\mathscr{G} \coloneqq \Set{\{T < \infty\} \cap E}{E \in \mathscr{F}}
	\end{align}
	とおけば,前問の結果より$X_T$は可測$\mathscr{G}/\borel{\R}$である.
	$\mathscr{G} \subset \mathscr{F}$より
	\begin{align}
		\mathscr{H} \coloneqq \Set{\{X_T \in A\},\ \{X_T \in A\} \cup \{T = \infty\}}{A \in \borel{\R}}
	\end{align}
	に対して$\mathscr{H} \subset \mathscr{F}$が成立する.
	あとは$\mathscr{H}$が$\sigma$-加法族であることを示せばよい.
	実際,$A = \R$のとき
	\begin{align}
		\{X_T \in A\} \cup \{T = \infty\} = \{T < \infty\} \cup \{T = \infty\} = \Omega
	\end{align}
	となり$\Omega \in \mathscr{H}$が従い,また
	\begin{align}
		&\{X_T \in A\}^c = \{X_T \in A^c\} \cup \{T = \infty\}, \\
		&\left( \{X_T \in A\} \cup \{T = \infty\} \right)^c
		=  \{X_T \in A^c\} \cap \{T < \infty\}
		= \{X_T \in A^c\}
	\end{align}
	より$\mathscr{H}$は補演算で閉じる.更に$B_n \in \mathscr{H}\ (n=1,2,\cdots)$を取れば,
	\begin{align}
		\bigcup_{n=1}^{\infty} B_n = \left\{X_T \in \bigcup_{n=1}^{\infty} A_n \right\}
	\end{align}
	或は
	\begin{align}
		\bigcup_{n=1}^{\infty} B_n = \left\{X_T \in \bigcup_{n=1}^{\infty} A_n \right\} \cup \{T = \infty\}
	\end{align}
	が成立し$\bigcup_{n=1}^\infty B_n \in \mathscr{H}$を得る.
	\QED
\end{prf}
\section{Stopping Times}
	\begin{itembox}[l]{$[0,\infty]$の位相}
		$[0,\infty]$の位相は拡張実数$[-\infty,\infty]$の相対位相である.
		$O \subset [-\infty,\infty]$が開集合であるとは,
		任意の$x \in O$に対し,
		\begin{description}
			\item[(O1)] $x \in \R$なら或る$\epsilon > 0$が存在して
				$B_\epsilon(x) \subset O$が満たされる,
			
			\item[(O2)] $x = \infty$なら或る$a \in \R$が存在して
				$(a,\infty] \subset O$が満たされる,
			
			\item[(O3)] $x = -\infty$なら或る$a \in \R$が存在して
				$[-\infty,a) \subset O$が満たされる,
		\end{description}
		で定義される.この性質を満たす$O$の全体に$\emptyset$を加えたものが
		$[-\infty,\infty]$の位相であり,
		\begin{align}
			[-\infty,r),\quad (r,r'), \quad (r,\infty],
			\quad (r,r' \in \Q)
		\end{align}
		の全体が可算開基となる.従って$[0,\infty]$の位相の可算開基は
		\begin{align}
			[0,r),\quad (r,r'), \quad (r,\infty],
			\quad (r,r' \in \Q \cap [0,\infty])
		\end{align}
		の全体であり,写像$\tau:\Omega \longrightarrow [0,\infty]$が
		$\mathscr{F}/\borel{[0,\infty]}$-可測性を持つかどうかを調べるには
		\begin{align}
			\{\tau < a\} = \tau^{-1}([0,a)) \in \mathscr{F},
			\quad (\forall a \in (0,\infty))
		\end{align}
		が満たされているかどうかを確認すれば十分である.
	\end{itembox}
	
	\begin{itembox}[l]{Problem 2.2}
		Let $X$ be a stochastic process and $T$ a stopping time of 
		$\left\{ \mathscr{F}^X_t \right\}$. Suppose that for some pair $\omega,\omega' \in \Omega$, 
		we have $X_t(\omega) = X_t(\omega')$ for all $t \in [0,T(\omega)] \cap [0,\infty)$. 
		Show that $T(\omega) = T(\omega')$. 
	\end{itembox}
	
	\begin{prf}[参照元:\cite{key3}]
		$\omega,\omega'$を分離しない集合族$\mathscr{H}$を
		\begin{align}
			\mathscr{H} \coloneqq \Set{A \subset \Omega}{\{\omega,\omega'\} \subset A,\ or\ \{\omega,\omega'\} \subset \Omega \backslash A}
		\end{align}
		により定めれば,$\mathscr{H}$は$\sigma$-加法族である.このとき,
		$\{T = T(\omega)\} \in \mathscr{H}$を示せばよい.
		\begin{description}
			\item[case1]
				$T(\omega) = \infty$の場合,
				任意の$A \in \borel{\R^d}$及び$0 \leq t < \infty$に対して,
				仮定より
				\begin{align}
					\omega \in X_t^{-1}(A) \quad \Leftrightarrow \quad
					\omega' \in X_t^{-1}(A)
				\end{align}
				が成り立ち
				\begin{align}
					\sigma(X_t;\ 0 \leq t < \infty) \subset \mathscr{H}
				\end{align}
				となる.任意の$t \geq 0$に対し$\{T \leq t\} \in \mathscr{F}^X_t \subset 
				\sigma(X_t;\ 0 \leq t < \infty)$が満たされるから
				\begin{align}
					\{T = \infty\} = \bigcap_{n=1}^\infty \{T \leq n\}^c
					\in \sigma(X_t;\ 0 \leq t < \infty) \subset \mathscr{H}
				\end{align}
				が成立し,$\omega \in \{T = \infty\}$より$\omega' \in \{T = \infty\}$が従い
				$T(\omega) = T(\omega')$を得る.
				
			\item[case2]
				$T(\omega) < \infty$の場合,
				case1と同様に任意の$0 \leq t \leq T(\omega)$に対し
				$\sigma(X_t) \subset \mathscr{H}$が満たされるから
				\begin{align}
					\mathscr{F}^X_{T(\omega)} \subset \mathscr{H}
				\end{align}
				が成り立つ.$\{T = T(\omega)\} \in \mathscr{F}^X_{T(\omega)}$より
				$\omega' \in \{T = T(\omega)\}$が従い$T(\omega) = T(\omega')$を得る.
				\QED
		\end{description}
	\end{prf}
	
	\begin{itembox}[l]{Lemma for Proposition 2.3}
		$(\mathscr{F}_t)_{t \geq 0}$を可測空間
		$(\Omega,\mathscr{F})$のフィルトレーションとするとき,
		任意の$t \geq 0$及び任意の点列$s_1  > s_2 > \cdots > t, (s_n \downarrow t)$
		に対して次が成立する:
		\begin{align}
			\bigcap_{s>t} \mathscr{F}_s = \bigcap_{n=1}^\infty \mathscr{F}_{s_n}.
		\end{align}
	\end{itembox}
	
	\begin{prf}
		先ず任意の$n \geq 1$に対して
		\begin{align}
			\bigcap_{s > t} \mathscr{F}_s \subset \mathscr{F}_{s_n}
		\end{align}
		が成り立つから
		\begin{align}
			\bigcap_{s > t} \mathscr{F}_s \subset \bigcap_{n=1}^\infty \mathscr{F}_{s_n}
		\end{align}
		を得る.一方で,任意の$s > t$に対し$s \geq s_n$を満たす$n$が存在するから,
		\begin{align}
			\mathscr{F}_s \supset  \mathscr{F}_{s_n}
			\supset \bigcap_{n=1}^\infty \mathscr{F}_{s_n}
		\end{align}
		が成立し
		\begin{align}
			\bigcap_{s > t} \mathscr{F}_s \supset \bigcap_{n=1}^\infty \mathscr{F}_{s_n}
		\end{align}
		が従う.
		\QED
	\end{prf}
	
	$(\mathscr{F}_{t+})_{t \geq 0}$は右連続である.実際,任意の$t \geq 0$で
	\begin{align}
		\bigcap_{s > t} \mathscr{F}_{s+} = \bigcap_{s > t} \bigcap_{u > s} \mathscr{F}_u
		= \bigcap_{s > t} \mathscr{F}_s
		= \mathscr{F}_{t+}
	\end{align}
	が成立する.
	
	\begin{itembox}[l]{Corollary 2.4}\label{chapter_1_Corollary_2_4}
		$T$ is an optional time of the filtration $\{\mathscr{F}_t\}$ if and only if 
		it is a stopping time of the (right-continuous!) filtration $\{\mathscr{F}_{t+}\}$.
	\end{itembox}
	言い換えれば,確率時刻$T$に対し
	\begin{align}
		\{T < t\} \in \mathscr{F}_t,\ \forall t \geq 0
		\quad \Leftrightarrow \quad
		\{T \leq t\} \in \mathscr{F}_{t+},\ \forall t \geq 0
	\end{align}
	が成り立つことを主張している.
	\begin{prf}
		$T$が$(\mathscr{F}_{t+})$-停止時刻であるとき,
		任意の$n \geq 1$に対して$\{T \leq t - 1/n\} \in \mathscr{F}_{(t-1/n)+} \subset \mathscr{F}_t$
		が満たされるから
		\begin{align}
			\{T < t\} = \bigcup_{n=1}^\infty \left\{T \leq t - \frac{1}{n}\right\} \in \mathscr{F}_t
		\end{align}
		が従う.逆に$T$が$(\mathscr{F}_t)$-弱停止時刻
		\footnote{
			optional time の訳語がわからないので弱停止時刻と呼ぶ.
		}
		のとき,任意の$m \geq 1$に対し
		\begin{align}
			\{T \leq t\} = \bigcap_{n=m}^\infty \left\{T < t+\frac{1}{n} \right\}
			\in \mathscr{F}_{t + 1/m}
		\end{align}
		が成立するから
		\begin{align}
			\{T \leq t\} \in \bigcap_{n=1}^\infty \mathscr{F}_{t + 1/n} = \mathscr{F}_{t+}
		\end{align}
		を得る.
		\QED
	\end{prf}
	
	\begin{itembox}[l]{Problem 2.6}
		If the set $\Gamma$ in Example 2.5 is open, show that $H_\Gamma$ is 
		an optional time.
	\end{itembox}
	
	\begin{prf}
		$\{H_\Gamma < 0\}=\emptyset$であるから,以下$t > 0$とする.
		$H_\Gamma(\omega) < t \Leftrightarrow \exists s < t,\ X_s(\omega) \in \Gamma$より
		\begin{align}
			\{H_\Gamma < t\} = \bigcup_{0 \leq s < t} \{X_s \in \Gamma\}
		\end{align}
		となる.また全てのパスが右連続であることと$\Gamma$が開集合であることにより
		\begin{align}
			\bigcup_{0 \leq s < t} \{X_s \in \Gamma\}
			= \bigcup_{\substack{0 \leq r < t \\ r \in \Q}} \{X_r \in \Gamma\}
		\end{align}
		が成り立ち$\{H_\Gamma < t\} \in \mathscr{F}_t$が従う.
		\QED
	\end{prf}
	
	\begin{itembox}[l]{Problem 2.7}
		If the set $\Gamma$ in Example 2.5 is closed and the sample paths of the 
		process $X$ are continuous, then $H_\Gamma$ is a stopping time.
	\end{itembox}
	
	\begin{prf}\mbox{}
		\begin{description}
			\item[第一段]
				$\R^d$上のEuclid距離を$\rho$で表し,
				\begin{align}
					\rho(x,\Gamma) \coloneqq \inf{y \in \Gamma}{\rho(x, y)},
					\quad \Gamma_n \coloneqq \Set{x \in \R^d}{\rho(x,\Gamma) < \frac{1}{n}},
					\quad (x \in \R^d,\ n=1,2,\cdots)
				\end{align}
				とおく.$\R^d \ni x \longmapsto \rho(x,\Gamma)$の連続性より$\Gamma_n$は開集合であるから,
				Problem 2.6の結果より$T_n \coloneqq H_{\Gamma_n}$で定める$T_n,\ n=1,2,\cdots$は
				$(\mathscr{F}_t)$-弱停止時刻であり,
				また$H \coloneqq H_\Gamma$とおけば次の(1)と(2)が成立する:
				\begin{description}
					\setlength{\leftskip}{3.0cm}
					\item[(1)] $\{H = 0\} = \{X_0 \in \Gamma\}$,
					
					\setlength{\leftskip}{3.0cm}
					\item[(2)] $H(\omega) \leq t 
					\quad \Leftrightarrow \quad 
					T_n(\omega) < t,\ \forall n=1,2,\cdots,
					\quad (\forall \omega \in \{H>0\},\ \forall t>0)$.
				\end{description}
				(1)と(2)及び$T_n,\ n=1,2,\cdots$が$(\mathscr{F}_t)$-弱停止時刻であることにより
				\begin{align}
					\{H \leq t\}
					= \{H \leq t\} \cap \{H > 0\} + \{H = 0\}
					= \left\{ \bigcap_{n=1}^\infty \{T_n < t\} \right\} \cap \{H > 0\} + \{H = 0\}
					\in \mathscr{F}_t,
					\quad (\forall t \geq 0)
				\end{align}
				が成立するから$H$は$(\mathscr{F}_t)$-停止時刻である.
			
			\item[第二段]
				(1)を示す.実際,
				$X_0(\omega) \in \Gamma$なら$H(\omega) = 0$であり,
				$X_0(\omega) \notin \Gamma$なら,$\Gamma$が閉であることと
				パスの連続性より
				\begin{align}
					X_t(\omega) \notin \Gamma,
					\quad (0 \leq t \leq h)
				\end{align}
				を満たす$h > 0$が存在して$H(\omega) \geq h > 0$となる.
		
			\item[第三段]
				$\omega \in \{H>0\},\ t > 0$として(2)を示す.まずパスの連続性より
				\begin{align}
					T_n(\omega) < t \quad \Leftrightarrow \quad
					\exists s \leq t, \quad X_s(\omega) \in \Gamma_n
				\end{align}
				が成り立つ.$H(\omega) \leq t$の場合,
				$\beta \coloneqq H(\omega)$とおけば,$\Gamma$が閉であることと
				パスの連続性より
				\begin{align}
					X_\beta(\omega) \in \Gamma \subset \Gamma_n,
					\quad (\forall n=1,2,\cdots)
				\end{align}
				が満たされ$T_n(\omega) < t\ (\forall n \geq 1)$が従う.
				逆に,$H(\omega) > t$のとき
				\begin{align}
					X_s(\omega) \notin \Gamma,
					\quad (\forall s \in [0,t])
				\end{align}
				が満たされ,パスの連続性と$\rho$の連続性より
				$[0,t] \ni s \longmapsto \rho(X_s(\omega),\Gamma)$
				は連続であるから,
				\begin{align}
					d \coloneqq \min{s \in [0,t]}{\rho(X_s(\omega),\Gamma)} > 0
				\end{align}
				が定まる.このとき$1/n < d/2$を満たす$n \geq 1$を一つ取れば
				\begin{align}
					X_s(\omega) \notin \Gamma_n,
					\quad (\forall s \in [0,t])
				\end{align}
				が成立する.実際,任意の$s \in [0,t],\ x \in \Gamma_n$に対し
				\begin{align}
					\rho(X_s(\omega),x)
					\geq \rho(X_s(\omega),\Gamma) - \rho(x,\Gamma)
					\geq d - \frac{d}{2}
					= \frac{d}{2}
					> \frac{1}{n}
				\end{align}
				が満たされる.従って$T_n(\omega) \geq t$となる.
				\QED
		\end{description}
	\end{prf}
	
	\begin{itembox}[l]{Lemma 2.9 の式変形について}
		第一の式変形は
		\begin{align}
			\{T + S > t\}
			&= \{T = 0,\ T+S > t\} + \{0 < T < t,\ T+S > t\} + \{T \geq t,\ T+S > t\} \\
			&= \{T = 0,\ T+S > t\} + \{0 < T < t,\ T+S > t\} + \{T \geq t,\ T+S > t,\ S = 0\} \\
				&\quad+ \{T \geq t,\ T+S > t,\ S > 0\} \\
			&= \{T = 0,\ S > t\} + \{0 < T < t,\ T+S > t\} + \{T > t,\ S = 0\}
				+ \{T \geq t,\ S > 0\}
		\end{align}
		である.
	\end{itembox}
	
	\begin{itembox}[l]{Problem 2.10}
		Let $T,S$ be optional times; then $T + S$ is optional. 
		It is a stopping time, if one of the following conditions holds:
		\begin{description}
			\item[(i)] $T > 0,\ S > 0$;
			\item[(ii)] $T > 0,$ $T$ is a stopping time.
		\end{description}
	\end{itembox}
	
	\begin{prf}
		$T,S$が$(\mathscr{F}_t)$-弱停止時刻であるとすれば,
		任意の$t > 0$に対し
		\begin{align}
			\{T + S < t\}
			&= \{T = 0,\ T + S < t\} + \{0 < T < t,\ T + S < t\} \\
			&= \{T = 0,\ S < t\} + \bigcup_{\substack{0 < r < t \\ r \in \Q}} \{0 < T < r,\ S < t-r\} \\
			&\in \mathscr{F}_t
		\end{align}
		が成り立つから$T + S$も$(\mathscr{F}_t)$-弱停止時刻である.
		\begin{description}
			\item[(i)] この場合$\{T + S \leq 0\} = \emptyset$である.また$t > 0$なら
				\begin{align}
					\{T + S > t\} = \{0 < T < t,\ T + S > t\} + \{T \geq t,\ T + S > t \}
					= \bigcup_{\substack{0 < r < t \\ r \in \Q}} \{r < T < t,\ S > t-r\} + \{T \geq t\} \in \mathscr{F}_t
				\end{align}
				が成立する.
				
			\item[(ii)]
				この場合も$\{T + S \leq 0\} = \emptyset$であり,また$t > 0$のとき
				\begin{align}
					\{T + S > t\} &= \{0 < T < t,\ T + S > t\} + \{T \geq t,\ T + S > t \} \\
					&= \{0 < T < t,\ T + S > t\} + \{T \geq t,\ T + S > t,\ S=0 \} + \{T \geq t,\ T + S > t,\ S>0 \} \\
					&= \{0 < T < t,\ T + S > t\} + \{T > t,\ S=0 \} + \{T \geq t,\ S>0 \} \\
					&\in \mathscr{F}_t
				\end{align}
				が成立する.
				\QED
		\end{description}
	\end{prf}
	
	\begin{itembox}[l]{Problem 2.13}
		Verify that $\mathscr{F}_T$ is actually a $\sigma$-field and $T$ is 
		$\mathscr{F}_T$-measurable. Show that if $T(\omega) = t$ for some constant 
		$t \geq 0$ and every $\omega \in \Omega$, then $\mathscr{F}_T = \mathscr{F}_t$.
	\end{itembox}
	
	\begin{prf}\mbox{}
		\begin{description}
			\item[第一段]
				$\mathscr{F}_T$が$\sigma$-加法族であることを示す.実際,
				$\Omega \cap \{T \leq t\} = \{T \leq t\} \in \mathscr{F}_t,\ (\forall t \geq 0)$
				より$\Omega \in \mathscr{F}_T$が従い,また
				\begin{align}
					A^c \cap \{T \leq t\} = \{T \leq t\} - A \cap \{T \leq t\},
					\quad \left\{ \bigcup_{n=1}^\infty A_n \right\} \cap \{T \leq t\}
					= \bigcup_{n=1}^\infty \left( A_n \cap \{T \leq t\} \right)
				\end{align}
				より$\mathscr{F}_T$は補演算と可算和で閉じる.
				
			\item[第二段]
				任意の$\alpha \geq 0$に対し
				\begin{align}
					\{T \leq \alpha \} \cap \{T \leq t\}
					= \{T \leq \alpha \wedge t\}
					\in \mathscr{F}_{\alpha \wedge t} \subset \mathscr{F}_t
				\end{align}
				が成立し$T$の$\mathscr{F}_T/\borel{[0,\infty]}$-可測性が出る.
				
			\item[第三段]
				$A \in \mathscr{F}_T$なら$A = A \cap \{T \leq t\} \in \mathscr{F}_t$となり,
				$A \in \mathscr{F}_t$については,任意の$s \geq 0$に対し
				$s \geq t$なら
				\begin{align}
					A \cap \{T \leq s\} = A \in \mathscr{F} \subset \mathscr{F}_s,
				\end{align}
				$s < t$なら
				\begin{align}
					A \cap \{T \leq s\} = \emptyset \in \mathscr{F}_s
				\end{align}
				が成り立ち$A \in \mathscr{F}_T$が従う.
				\QED
		\end{description}
	\end{prf}
	
	\begin{itembox}[l]{Exercise 2.14}
		Let $T$ be a stopping time and $S$ a random time such that $S \geq T$ 
		on $\Omega$. If $S$ is $\mathscr{F}_T$-measurable, then it is also a stopping time.
	\end{itembox}
	
	\begin{prf}
		任意の$t \geq 0$に対し
		\begin{align}
			\{S \leq t\} = \{S \leq t\} \cap \{T \leq t\} \in \mathscr{F}_t
		\end{align}
		が成立する.
		\QED
	\end{prf}
	
	\begin{itembox}[l]{Problem 2.17 修正}\label{chapter_1_Problem_2_17}
		Let $T,S$ be stopping times and $Z$ an $\mathscr{F}/\borel{\R}$-measurable, 
		integrable random variable. Then
		\begin{align}
			A \in \mathscr{F}_T \quad \Rightarrow \quad A \cap \{T \leq S\}, A \cap \{T < S\} \in \mathscr{F}_{S \wedge T},
		\end{align}
		and we have
		\begin{description}
			\item[(i)] $\defunc_{\{T \leq S\}} \cexp{Z}{\mathscr{F}_T} = \defunc_{\{T \leq S\}} \cexp{Z}{\mathscr{F}_{S \wedge T}},\ \mbox{$P$-a.s.}$
			\item[(ii)] $\defunc_{\{T < S\}} \cexp{Z}{\mathscr{F}_T} = \defunc_{\{T < S\}} \cexp{Z}{\mathscr{F}_{S \wedge T}},\ \mbox{$P$-a.s.}$
			\item[(iii)] $\cexp{\cexp{Z}{\mathscr{F}_T}}{\mathscr{F}_S} = \cexp{Z}{\mathscr{F}_{S \wedge T}},\ \mbox{$P$-a.s.}$
		\end{description}
	\end{itembox}
	
	\begin{prf}\mbox{}
		\begin{description}
			\item[第一段]
				任意の$A \in \mathscr{F}_T$に対し$A \cap \{T \leq S\} \in \mathscr{F}_{S \wedge T}$
				が成り立つ.実際,
				\begin{align}
					A \cap \{T \leq S\} \cap \{S \wedge T \leq t\}
					= \biggl[ A \cap \{T \leq t\} \biggr] \cap \{T \leq S\} \cap \{S \wedge T \leq t\}
					\in \mathscr{F}_t,
					\quad (\forall t \geq 0)
				\end{align}
				が成立する.同様に$A \cap \{T < S\} \in \mathscr{F}_{S \wedge T}$も得られる.
				
			\item[第二段]
				任意の$A \in \mathscr{F}_T$に対し,前段の結果より
				\begin{align}
					\int_{A \cap \{T \leq S\}} Z\ dP
					= \int_{A \cap \{T \leq S\}} \cexp{Z}{\mathscr{F}_{S \wedge T}}\ dP
				\end{align}
				が従う.$\defunc_{\{T \leq S\}} \cexp{Z}{\mathscr{F}_{S \wedge T}}$
				も$\mathscr{F}_T/\borel{\R}$-可測であるから(i)が得られ,同様に(ii)も出る.
			
			\item[第三段]
				任意の$B \in \mathscr{F}_S$に対し,第一段と第二段の結果により
				\begin{align}
					\int_B \cexp{\cexp{Z}{\mathscr{F}_T}}{\mathscr{F}_S}\ dP
					&= \int_B \cexp{Z}{\mathscr{F}_T}\ dP
					= \int_{B\cap\{S < T\}} \cexp{Z}{\mathscr{F}_T}\ dP
						+ \int_{B\cap\{T \leq S\}} \cexp{Z}{\mathscr{F}_T}\ dP \\
					&= \int_{B \cap \{S < T\}} Z\ dP
						+ \int_{B\cap\{T \leq S\}} \cexp{Z}{\mathscr{F}_{S \wedge T}}\ dP \\
					&= \int_{B \cap \{S < T\}} \cexp{Z}{\mathscr{F}_{S \wedge T}}\ dP
						+ \int_{B\cap\{T \leq S\}} \cexp{Z}{\mathscr{F}_{S \wedge T}}\ dP \\
					&= \int_B \cexp{Z}{\mathscr{F}_{S \wedge T}}\ dP
				\end{align}
				が成り立つ.$\cexp{Z}{\mathscr{F}_{S \wedge T}}$も$\mathscr{F}_S/\borel{\R}$-可測
				であるから(iii)を得る.
				\QED
		\end{description}
	\end{prf}
	
	\begin{itembox}[l]{Proposition 2.18}\label{chapter_1_Problem_2_18}
		Let $X = \Set{X_t,\mathscr{F}_t}{0 \leq t < \infty}$ be a progressively measurable 
		process, and let $T$ be a stopping time of the filtration $\{\mathscr{F}_t\}$. 
		Then the random variable $X_T$ of Definition 1.15, defined on the set 
		$\{T < \infty\} \in \mathscr{F}_T$, is $\mathscr{F}_T$-measurable, and
		the ``stopped process'' $\Set{X_{T \wedge t},\mathscr{F}_t}{0 \leq t < \infty}$
		is progressively measurable.
	\end{itembox}
	
	\begin{prf}\mbox{}
		\begin{description}
			\item[第一段]
				停止過程の発展的可測性を示す.$t \geq 0$を固定する.
				このとき,全ての$\omega \in \Omega$に対して
				$[0,t] \ni s \longmapsto T(\omega) \wedge s$は連続であり,かつ
				全ての$s \in [0,t]$に対し$\Omega \ni \omega \longmapsto T(\omega) \wedge s$は
				$\mathscr{F}_t/\borel{[0,t]}$-可測であるから,
				$[0,t] \times \Omega \ni (s,\omega) \longmapsto T(\omega) \wedge s$
				は$\borel{[0,t]} \otimes \mathscr{F}_t/\borel{[0,t]}$-可測である.
				従って,任意の$A \in \borel{[0,t]}$と$B \in \mathscr{F}_t$に対し
				\begin{align}
					\Set{(s,\omega) \in [0,t] \times \Omega}{(T(\omega) \wedge s,\omega) \in A \times B}
					&= \Set{(s,\omega) \in [0,t] \times \Omega}{T(\omega) \wedge s \in A}
					\cap ([0,t] \times B) \\
					&\in \borel{[0,t]} \otimes \mathscr{F}_t
				\end{align}
				が成り立つから,任意の$E \in \borel{[0,t]} \otimes \mathscr{F}_t$
				に対して
				\begin{align}
					\Set{(s,\omega) \in [0,t] \times \Omega}{(T(\omega) \wedge s,\omega) \in E} 
					\in \borel{[0,t]} \otimes \mathscr{F}_t
				\end{align}
				が満たされ$(s,\omega) \longmapsto (T(\omega) \wedge s,\omega)$の
				$\borel{[0,t]} \otimes \mathscr{F}_t/\borel{[0,t]} \otimes \mathscr{F}_t$-可測性を得る.
				\begin{align}
					X(s,\omega) = X|_{[0,t] \times \Omega}(s,\omega),
					\quad (\forall (s,\omega) \in [0,t] \times \Omega)
				\end{align}
				かつ$X|_{[0,t] \times \Omega}$は$\borel{[0,t]} \otimes \mathscr{F}_t/\borel{\R^d}$-可測であるから,
				$[0,t] \times \Omega \ni (s,\omega) \longmapsto X(T(\omega) \wedge s,\omega) 
				= X|_{[0,t] \times \Omega}(T(\omega) \wedge s,\omega)$の
				$\borel{[0,t]} \otimes \mathscr{F}_t/\borel{\R^d}$-可測性が出る.
				
			\item[第二段]
				定理\ref{lem:Fubini_lemma_1} (P. \pageref{lem:Fubini_lemma_1})より
				$\omega \longmapsto X(T(\omega) \wedge t,\omega)$
				は$\mathscr{F}_t/\borel{\R^d}$であるから,
				任意の$B \in \borel{\R^d}$に対し
				\begin{align}
					\left\{ X_T \defunc_{\{T < \infty\}} \in B \right\} \cap \{T \leq t\}
					= \left\{ X_{T \wedge t} \in B \right\} \cap \{T \leq t\}
					\in \mathscr{F}_t,
					\quad (\forall t \geq 0)
				\end{align}
				が成立し$X_T \defunc_{\{T < \infty\}}$の$\mathscr{F}_T/\borel{\R^d}$-可測性を得る.
				\QED
		\end{description}
	\end{prf}
	
	\begin{itembox}[l]{Problem 2.19}
		Under the same assumption as in Proposition 2.18, and with 
		$f(t,x);[0,\infty) \times \R^d \longrightarrow \R$ a bounded,
		$\borel{[0,\infty)} \otimes \borel{\R^d}$-measurable function,
		show that the process $Y_t = \int_0^t f(s,X_s)\ ds;\ t \geq 0$ is
		progressively measurable with respect to $\{\mathscr{F}_t\}$, 
		and  $Y_T$ is an $\mathscr{F}_T$-measurable random variable.
	\end{itembox}
	
	\begin{prf}
		$[0,t] \times \Omega \ni (s,\omega) \longmapsto f(s,X_s(\omega))$
		が$\borel{[0,t]} \otimes \mathscr{F}_t/\borel{\R}$-可測であれば,
		Fuiniの定理より$\Set{Y_t,\mathscr{F}_t}{0 \leq t < \infty}$は
		適合過程となり,可積分性より$t \longmapsto Y_t(\omega),\ (\forall \omega \in \Omega)$が
		連続であるから$Y$の発展的可測性が従う.実際,
		\begin{align}
			[0,t] \times \Omega \ni (s,\omega)
			\longmapsto \left( s,X_s(\omega)\right) 
			= \left( s,X|_{[0,t] \times \Omega}(s,\omega) \right)
		\end{align}
		による$A \times B,\ (A \in \borel{[0,\infty)},\ B \in \borel{\R^d})$の引き戻しは
		\begin{align}
			\left\{ ([0,t] \cap A) \times \Omega \right\} \cap
			X|_{[0,t] \times \Omega}^{-1}(B)
			\in \borel{[0,t]} \otimes \mathscr{F}_t
		\end{align}
		となるから,$[0,t] \times \Omega \ni (s,\omega) \longmapsto f(s,X_s(\omega))$
		は$\borel{[0,t]} \otimes \mathscr{F}_t/\borel{\R}$-可測である.
		\QED
	\end{prf}
	
	\begin{itembox}[l]{Problem 2.21}
		Verify that the class $\mathscr{F}_{T+}$ is indeed a $\sigma$-field
		with respect to which $T$ is measurable, that it coincides with
		$\Set{A \in \mathscr{F}}{A \cap \{T < t\} \in \mathscr{F}_t,\ \forall t \geq 0}$,
		and that if $T$ is a stopping time (so that both $\mathscr{F}_T,\mathscr{F}_{T+}$
		are defined), then $\mathscr{F}_T \subset \mathscr{F}_{T+}$.
	\end{itembox}
	
	\begin{prf}\mbox{}
		\begin{description}
			\item[第一段]
				$\Omega \cap \{T \leq t\} = \{T \leq t\} \in \mathscr{F}_t,\ (\forall t \geq 0)$
				より$\Omega \in \mathscr{F}_{T+}$が従い,また
				\begin{align}
					A^c \cap \{T \leq t\} = \{T \leq t\} - A \cap \{T \leq t\},
					\quad \left\{ \bigcup_{n=1}^\infty A_n \right\} \cap \{T \leq t\}
					= \bigcup_{n=1}^\infty \left( A_n \cap \{T \leq t\} \right)
				\end{align}
				より$\mathscr{F}_{T+}$は補演算と可算和で閉じるから
				$\mathscr{F}_{T+}$は$\sigma$-加法族である.また,
				\begin{align}
					\{T < \alpha\} \cap \{T \leq t\}
					= \begin{cases}
						\{T < \alpha\}, & (\alpha \leq t), \\
						\{T \leq t\}, & (\alpha > t),
					\end{cases}
					\in \mathscr{F}_{t+},
					\quad (\forall t \geq 0)
				\end{align}
				より$(\mathscr{F}_t)$-弱停止時刻$T$は
				$\mathscr{F}_{T+}/\borel{[0,\infty]}$-可測である.
				
			\item[第二段]
				任意の$t \geq 0$に対し
				\begin{align}
					A \cap \{T < t\} = \bigcup_{n=1}^\infty A \cap \left\{T \leq t - \frac{1}{n}\right\},
					\quad A \cap \{T \leq t\} = \bigcap_{n=1}^\infty A \cap \left\{T < t + \frac{1}{n}\right\}
				\end{align}
				が成り立ち$\mathscr{F}_{T+} = \Set{A \in \mathscr{F}}{A \cap \{T < t\} \in \mathscr{F}_t,\ \forall t \geq 0}$
				が従う.
			
			\item[第三段]
				$T$が$(\mathscr{F}_t)$-停止時刻であるとき,
				任意の$A \in \mathscr{F}_T$に対し
				\begin{align}
					A \cap \{T \leq t\} \in \mathscr{F}_t \subset \mathscr{F}_{t+},
					\quad (\forall t \geq 0)
				\end{align}
				となり$\mathscr{F}_T \subset \mathscr{F}_{T+}$が成り立つ.
				\QED
		\end{description}
	\end{prf}
	
	\begin{itembox}[l]{Lemma: 弱停止時刻の可測性}
		$T$を$(\mathscr{F}_t)$-弱停止時刻とすれば,任意の
		$t \geq 0$に対し$T \wedge t$は$\mathscr{F}_t/\borel{[0,\infty)}$-可測である.
	\end{itembox}
	
	\begin{prf}
		任意の$\alpha \geq 0$に対し
		\begin{align}
			\{T \wedge t \leq \alpha\} = 
			\begin{cases}
			\Omega, & (t \leq \alpha), \\
			\{T \leq \alpha \}, & (t > \alpha),
			\end{cases}
			 \in \mathscr{F}_t
		\end{align}
		が成立する.
		\QED
	\end{prf}
	
	\begin{itembox}[l]{Probelem 2.22}
		Verify that analogues of Lemmas 2.15 and 2.16 hold if $T$ and
		$S$ are assumed to be optional and $\mathscr{F}_T,\ \mathscr{F}_S$
		and $\mathscr{F}_{T \wedge S}$ are replaced by $\mathscr{F}_{T+},\ \mathscr{F}_{S+}$
		and $\mathscr{F}_{(T \wedge S)+}$, respectively. Prove that if $S$ is 
		an optional time and $T$ is a positive stopping time with $S \leq T$,
		and $S < T$ on $\{S < \infty\}$, then $\mathscr{F}_{S+} \subset \mathscr{F}_T$.
	\end{itembox}
	
	\begin{prf}\mbox{}
		\begin{description}
			\item[第一段]
				$T \wedge t,\ S \wedge t$は
				$\mathscr{F}_t/\borel{[0,\infty)}$-可測であるから、
				任意の$A \in \mathscr{F}_{S+}$に対して
				\begin{align}
					A \cap \{S \leq T\} \cap \{T \leq t\}
					= (A \cap \{S \leq t\}) \cap \{S \wedge t \leq T \wedge t\} \cap \{T \leq t\}
					\in \mathscr{F}_{t+},
					\quad (\forall t \geq 0)
				\end{align}
				となり$A \cap \{S \leq T\} \in \mathscr{F}_{T+}$が成立する.
				特に,$\Omega$上で$S \leq T$なら$\mathscr{F}_{S+} \subset \mathscr{F}_{T+}$が従う.
				
			\item[第二段]
				前段の結果より$\mathscr{F}_{(T \wedge S)+} \subset \mathscr{F}_{T+} \cap \mathscr{F}_{S+}$
				が満たされる.一方で,任意の$A \in \mathscr{F}_{T+} \cap \mathscr{F}_{S+}$に対し
				\begin{align}
					A \cap \{T \wedge S \leq t\}
					= \left( A \cap \{T \leq t\} \right) \cup \left( A \cap \{S \leq t\} \right)
					\in \mathscr{F}_{t+},
					\quad (\forall t \geq 0)
				\end{align}
				が成り立ち$\mathscr{F}_{(T \wedge S)+} = \mathscr{F}_{T+} \cap \mathscr{F}_{S+}$を得る.
				また
				\begin{align}
					\{S < T\} \cap \{T \wedge S \leq t\}
					= \Biggl( \bigcup_{\substack{0 \leq r \leq t \\ r \in \Q \cup \{t\}}} \{S \leq r\} \cap \{r < T\} \Biggr) 
					\cap \{S \leq t\}
					\in \mathscr{F}_{t+},
					\quad (\forall t \geq 0)
				\end{align}
				により$\{S < T\} \in \mathscr{F}_{(T \wedge S)+}$及び
				$\{T < S\} \in \mathscr{F}_{(T \wedge S)+}$となり,
				$\{T \leq S\},\{S \leq T\},\{T = S\} \in \mathscr{F}_{(T \wedge S)+}$が従う.
			
			\item[第三段]
				$T$が停止時刻で$\{T < \infty\}$上で$S < T$
				が満たされているとき.任意の$A \in \mathscr{F}_{S+}$に対し
				\begin{align}
					A \cap \{T \leq t\}
					= A \cap \{S < t\} \cap \{T \leq t\}
					\in \mathscr{F}_t,
					\quad (\forall t \geq 0)
				\end{align}
				が成り立り$\mathscr{F}_{S+} \subset \mathscr{F}_T$となる.
				\QED
		\end{description}
	\end{prf}
	
	\begin{itembox}[l]{Problem 2.23}
		Show that if $\{T_n\}_{n=1}^\infty$ is a sequence of optional times
		and $T = \inf{n \geq 1}{T_n}$, then $\mathscr{F}_{T+} = \bigcap_{n=1}^\infty \mathscr{F}_{T_n+}$.
		Besides, if each $T_n$ is a positive stopping time and $T < T_n$ on
		$\{T < \infty\}$, then we have $\mathscr{F}_{T+} = \bigcap_{n=1}^\infty \mathscr{F}_{T_n}$.
	\end{itembox}
	
	\begin{prf}
		$T \leq T_n,\ (\forall n \geq 1)$より
		$\mathscr{F}_{T+} \subset \bigcap_{n=1}^\infty \mathscr{F}_{T_n+}$
		が成り立つ.一方で$A \in \bigcap_{n=1}^\infty \mathscr{F}_{T_n+}$に対し
		\begin{align}
			A \cap \{T < t\}
			= \bigcup_{n=1}^\infty A \cap \{T_n < t\}
			\in \mathscr{F}_t,
			\quad (\forall t > 0)
			\label{eq:chapter_1_problem_2_23}
		\end{align}
		が成り立つから,Problem 2.21より$A \in \mathscr{F}_{T+}$が従う.
		また$\{T < \infty\}$上で$T < T_n,\ (\forall n \geq 1)$であるとき,
		Problem 2.22より$\mathscr{F}_{T+} \subset \bigcap_{n=1}^\infty \mathscr{F}_{T_n}$
		が従い,また$T_n,\ n \geq 1$が停止時刻の場合も(\refeq{eq:chapter_1_problem_2_23})は成立するので
		$\mathscr{F}_{T+} = \bigcap_{n=1}^\infty \mathscr{F}_{T_n}$が出る.
		\QED
	\end{prf}
	
	\begin{itembox}[l]{Problem 2.24 修正}\label{chapter_1_Problem_2_24}
		Given an optional time $T$ of the filtration $\{\mathscr{F}_t\}$,
		consider the sequence $\{T_n\}_{n=1}^\infty$ of random times given by
		\begin{align}
			T_n(\omega) = 
			\begin{cases}
				+\infty; & \mbox{on $\Set{\omega}{T(\omega) \geq n}$} \\
				\displaystyle\frac{k}{2^n}; & \mbox{on $\Set{\omega}{\frac{k-1}{2^n} \leq T(\omega) < \frac{k}{2^n}}$ for $k=1,\cdots,n2^n$},
			\end{cases}
		\end{align}
		for $n \geq 1$. Obviously $T_n \geq T_{n+1} \geq T$,
		for every $n \geq 1$. Show that each $T_n$ is a stopping time,
		that $\lim_{n \to \infty} T_n = T$, and that for every $A \in \mathscr{F}_{T+}$
		we have $A \cap \left\{ T_n = (k/2^n) \right\} \in \mathscr{F}_{k/2^n};\ n \geq 1, 1 \leq k \leq n2^n$.
	\end{itembox}
	
	\begin{prf}\mbox{}
		\begin{description}
			\item[第一段]
				$T_n(\omega)<\infty$を満たす$\omega \in \Omega$に対し,
				或る$1 \leq j \leq (n+1)2^{n+1},\ 1\leq k \leq n2^n$が存在して
				\begin{align}
					\frac{j-1}{2^{n+1}} \leq T(\omega) < \frac{j}{2^{n+1}},
					\quad \frac{k-1}{2^n} \leq T(\omega) < \frac{k}{2^n}
				\end{align}
				となる.このとき
				\begin{align}
					\frac{2k-2}{2^{n+1}} \leq T(\omega) < \frac{2k-1}{2^{n+1}}
				\end{align}
				または
				\begin{align}
					\frac{2k-1}{2^{n+1}} \leq T(\omega) < \frac{2k}{2^{n+1}}
				\end{align}
				のどちらかであるから,すなわち$j=2k-1$或は$j=2k$であり
				\begin{align}
					T(\omega) < \frac{j}{2^{n+1}} = T_{n+1}(\omega)
					\leq \frac{2k}{2^{n+1}} = T_n(\omega)
				\end{align}
				が成立する.$T_n(\omega) = \infty$の場合も併せて$T_n \geq T_{n+1} \geq T\ (\forall n \geq 1)$を得る.
			
			\item[第二段]
				任意の$t \geq 0$に対して
				\begin{align}
					\{T_n \leq t\}
					= \bigcup_{k/2^n \leq n \wedge t} \Set{\omega}{\frac{k-1}{2^n} \leq T(\omega) < \frac{k}{2^n}}
					\in \mathscr{F}_t,
					\quad (\forall t \geq 0)
				\end{align}
				が成り立つから$T_n$は$(\mathscr{F}_t)$-停止時刻である.また$\{T < \infty\}$上では
				$T(\omega) < n$のとき
				\begin{align}
					0 < T_n(\omega) - T(\omega) \leq \frac{1}{2^n} \longrightarrow 0
					\quad (n \longrightarrow \infty)
				\end{align}
				となる.
			
			\item[第三段]
				任意の$A \in \mathscr{F}_{T+}$に対して,Problem 2.21より
				\begin{align}
					A \cap \left\{T_n = \frac{k}{2^n}\right\}
					= A \cap \left\{T < \frac{k}{2^n}\right\}
					- A \cap \left\{T < \frac{k-1}{2^n}\right\}
					\in \mathscr{F}_{k/2^n}
				\end{align}
				が成り立つ.
				\QED
		\end{description}
	\end{prf}
\input{thms/chapter_1_3_A}
\subsection{Convergence Results}
	\begin{itembox}[l]{Problem 3.16}
		Let $\Set{X_t,\mathscr{F}_t}{0 \leq t < \infty}$ be a right-continuous, nonnegative
		supermatingale; then $X_\infty(\omega) = \lim_{t \to \infty} X_t(\omega)$ exists for
		$P$-a.e. $\omega \in \Omega$, and $\Set{X_t,\mathscr{F}_t}{0 \leq t \leq \infty}$ is a supermartingale.
	\end{itembox}
	
	\begin{prf}
		$\Set{-X_t,\mathscr{F}_t}{0 \leq t < \infty}$は右連続な$(\mathscr{F}_t)$-劣マルチンゲールとなり
		\begin{align}
			\sup{t \geq 0}{E(-X_t)^+} = 0
		\end{align}
		が満たされるから,劣マルチンゲール収束定理により或る$P$-零集合$A$が存在して
		\begin{align}
			Z_\infty \coloneqq \lim_{t \to \infty} (-X_t)\defunc_{\Omega \backslash A}
		\end{align}
		により$\mathscr{F}_\infty/\borel{\R}$-可測な可積分関数$Z_\infty$が定まる.
		すなわち
		\begin{align}
			X_\infty \coloneqq \lim_{t \to \infty} X_t\defunc_{\Omega \backslash A}
		\end{align}
		により$\mathscr{F}_\infty/\borel{\R}$-可測関数が定まり,
		かつ$X_\infty = -Z_\infty$より$X_\infty$は可積分である.
		またFatouの補題により任意の$t \geq 0$及び$A \in \mathscr{F}_t$に対し
		\begin{align}
			\int_A X_\infty\ dP \leq \liminf_{\substack{n \to \infty \\ n > t}} \int_A X_n\ dP \leq \int_A X_t\ dP
		\end{align}
		が成立するから$\Set{X_t,\mathscr{F}_t}{0 \leq t \leq \infty}$は優マルチンゲールである.
		\QED
	\end{prf}
	
	\begin{itembox}[l]{Exercise 3.18}
		Suppose that the filtration $\{\mathscr{F}_t\}$ satisfies the usual conditions.
		Then every right-continuous, uniformly integrable supermartingale $\Set{X_t,\mathscr{F}_t}{0 \leq t < \infty}$
		admits the Riesz decomposition $X_t = M_t + Z_t,\ \mbox{a.s. $P$}$, as the sum
		of a right-continuous, uniformly integrable martingale $\Set{M_t,\mathscr{F}_t}{0 \leq t < \infty}$
		and a potential $\Set{Z_t,\mathscr{F}_t}{0 \leq t < \infty}$.
 	\end{itembox}
 	条件を満たす二つの分解$X_t = M_t + Z_t = M'_t + Z'_t\ \mbox{a.s. $P$}, (\forall t \geq 0)$が存在する場合,
 	次の意味で分解は一意である:
 	\begin{align}
 		P \left( M_t = M'_t,\ Z_t = Z'_t,\ \forall t \geq 0 \right) = 1.
 		\label{eq:chapter_1_Exercise_3_18_4}
 	\end{align}
 	
 	\begin{prf}\mbox{}
		\begin{description}
			\item[第一段] $M$を構成する.いま,$t \geq 0$を固定する.
				$n > t$を満たす$n \in \N$と任意の$A \in \mathscr{F}_t$に対し
 				\begin{align}
 					\int_A \cexp{X_{n+1}}{\mathscr{F}_t}\ dP
 					&= \int_A X_{n+1}\ dP
 					= \int_A \cexp{X_{n+1}}{\mathscr{F}_n}\ dP \\
		 			&\leq \int_A X_n\ dP
 					= \int_A \cexp{X_n}{\mathscr{F}_t}\ dP
 				\end{align}
 				が成り立つから
 				\begin{align}
 					E \coloneqq \bigcup_{n > t}\Set{\omega \in \Omega}{\cexp{X_n}{\mathscr{F}_t}(\omega) < \cexp{X_{n+1}}{\mathscr{F}_t}(\omega)}
 				\end{align}
 				として$P$-零集合が定まる.また,同様に優マルチンゲール性より
 				\begin{align}
 					F \coloneqq \bigcup_{n > t}\Set{\omega \in \Omega}{\cexp{X_n}{\mathscr{F}_t}(\omega) > X_t(\omega)}
 				\end{align}
 				も$P$-零集合である.このとき,単調減少性より
 				\begin{align}
 					X^*_t \coloneqq \lim_{n \to \infty} \cexp{X_n}{\mathscr{F}_t} \defunc_{\Omega \backslash (E \cup F)}
 				\end{align}
 				が$-\infty$まで込めて確定し,$X^*_t$は$\mathscr{F}_t/\borel{[-\infty,\infty]}$-可測であり
 				\begin{align}
 					X_t(\omega) \geq X^*_t(\omega), \quad (\forall \omega \in \Omega \backslash (E \cup F))
 				\end{align}
 				を満たす.単調収束定理と$\sup{n \geq 1}{E|X_n|} < \infty$ (一様可積分性)より
 				\begin{align}
 					E\left( X_t - X^*_t \right)
 					= \int_{\Omega \backslash (E \cup F)} \lim_{n \to \infty} \left( X_t - \cexp{X_n}{\mathscr{F}_t} \right)\ dP
 					= \lim_{n \to \infty} \int_{\Omega \backslash (E \cup F)} X_t - \cexp{X_n}{\mathscr{F}_t}\ dP
 					= E X_t - \lim_{n \to \infty} EX_n < \infty
 				\end{align}
 				が成立するから$X^*_t$は可積分性であり$P$-a.s.に$|X^*_t| <\infty$となる.ここで
 				\begin{align}
 					X^{**}_t \coloneqq X^*_t \defunc_{|X^*_t| < \infty}
 				\end{align}
 				により$\mathscr{F}_t/\borel{\R}$-可測な可積分関数を定めれば,
 				単調収束定理より
 				\begin{align}
 					E X^{**}_t
 					= \lim_{n \to \infty} \int_\Omega \cexp{X_n}{\mathscr{F}_t}\ dP
 					= \lim_{n \to \infty} E X_n
 					\label{eq:chapter_1_Exercise_3_18_1}
 				\end{align}
 				となる.任意の$t \geq 0$に対し$X^{**}_t$を定めれば,任意の$0 \leq s < t$及び$A \in \mathscr{F}_s$に対して
 				\begin{align}
 					\int_A X^{**}_t\ dP
 					= \lim_{n \to \infty} \int_A \cexp{X_n}{\mathscr{F}_t}\ dP 
 					= \lim_{n \to \infty} \int_A \cexp{X_n}{\mathscr{F}_s}\ dP
 					= \int_A X^{**}_s\ dP
 					\label{eq:chapter_1_Exercise_3_18_3}
 				\end{align}
 				が成り立つから$\Set{X^{**}_t,\mathscr{F}_t}{0 \leq t < \infty}$はマルチンゲールである.
 				マルチンゲール性より$[0,\infty) \ni t \longmapsto EX^{**}_t$は定数であるから
 				Theorem 3.13により右連続な修正$\Set{M_t,\mathscr{F}_t}{0 \leq t < \infty}$が存在する.
 		
 			\item[第二段]
 				まず$\lim_{t \to \infty} EX_t$が存在することを示す.
 				任意の単調増大列$(t_k)_{k=1}^\infty,\ t_k \uparrow \infty$に対し優マルチンゲール性より
	 			\begin{align}
	 				\lim_{k \to \infty} EX_{t_k} = \inf{k \geq 1}{EX_{t_k}}
	 			\end{align}
	 			が確定し,任意の$n \in \N$に対し$n < t_k$を満たす$k$が存在するから
	 			\begin{align}
	 				\inf{n \geq 1}{EX_n} \geq \inf{k \geq 1}{EX_{t_k}}
	 			\end{align}
	 			が従う.逆に任意の$t_k$に対し$t_k < n$を満たす$n$が存在するから
	 			\begin{align}
	 				\lim_{n \to \infty} EX_n = \inf{n \geq 1}{EX_n} 
	 				= \inf{k \geq 1}{EX_{t_k}} = \lim_{k \to \infty} EX_{t_k}
	 			\end{align}
	 			が成立し,$(t_k)_{k=1}^\infty$の任意性から$\lim_{t \to \infty} EX_t$が存在して
	 			\begin{align}
	 				\lim_{t \to \infty} EX_t = \lim_{n \to \infty} EX_n
	 				\label{eq:chapter_1_Exercise_3_18_2}
	 			\end{align}
	 			となる.右連続な優マルチンゲール$\Set{Z_t,\mathscr{F}_t}{0 \leq t < \infty}$を
	 			\begin{align}
 					Z_t \coloneqq X_t - M_t,
 					\quad (\forall t \geq 0)
 				\end{align}
 				により定めれば,
 				(\refeq{eq:chapter_1_Exercise_3_18_1})より任意の$t \geq 0$に対し
 				\begin{align}
 					E(X_t - M_t)
 					= E X_t - E M_t
 					= EX_t - \lim_{n \to \infty} E X_n
 				\end{align}
 				が成り立ち,(\refeq{eq:chapter_1_Exercise_3_18_2})より
 				\begin{align}
 					\lim_{t \to \infty} E(X_t - M_t)
 					= \lim_{t \to \infty} E X_t - \lim_{n \to \infty} E X_n
 					= 0
 				\end{align}
 				が満たされるから$\Set{Z_t,\mathscr{F}_t}{0 \leq t < \infty}$はポテンシャルである.
			
			\item[第三段]
				分解の一意性を示す.任意の$t \geq 0$及び$A \in \mathscr{F}_t$に対し,
				(\refeq{eq:chapter_1_Exercise_3_18_3})と$M'$のマルチンゲール性より
				\begin{align}
					\int_A M_t\ dP
					= \lim_{\substack{n \to \infty \\ n > t}} \int_A X_n\ dP
					= \lim_{\substack{n \to \infty \\ n > t}} \left\{ \int_A M'_n - Z'_n\ dP \right\}
					= \lim_{\substack{n \to \infty \\ n > t}} \left\{ \int_A M'_t\ dP - \int_A Z'_n\ dP \right\}
				\end{align}
				が成立する.またポテンシャルは非負であるから
				\begin{align}
					0 \leq \int_A Z'_n\ dP \leq \int_\Omega Z'_n\ dP \longrightarrow 0
					\quad (n \longrightarrow \infty)
				\end{align}
				が成り立ち,$M_t = M'_t\ \mbox{$P$-a.s.}$及び$Z_t = Z'_t\ \mbox{$P$-a.s.}$が従う.パスの右連続性より
				(\refeq{eq:chapter_1_Exercise_3_18_4})が出る.
 				\QED
 		\end{description}
 	\end{prf}
	
	\begin{itembox}[l]{Problem 3.19}
		Assume that $\mathscr{F}_0$ contains all the $P$-negligible events in $\mathscr{F}$ \footnotemark.
		Then the following three conditions are equivalent for a nonnegative, right-continuous 
		submartingale $\Set{X_t,\mathscr{F}_t}{0 \leq t < \infty}$:
		\begin{description}
			\item[(a)] it is a uniformly integrable family of random variables;
			\item[(b)] is converges in $L^1$, as $t \to \infty$;
			\item[(c)] it converges $P$ a.s. (as $t \to \infty$) to an integrable random variable $X_\infty$,
			such that $\Set{X_t,\mathscr{F}_t}{0 \leq t \leq \infty}$ is a submartingale.
		\end{description}
		Observe that the implications (a) $\Rightarrow$ (b) $\Rightarrow$ (c) hold without the assumption of nonnegativity. 
	\end{itembox}
	\footnotetext{
		証明の第二段で出てくる$E$が$\mathscr{F}_\infty$に属していなければならない.
	}
	\begin{prf}\mbox{}
		\begin{description}
			\item[第一段]
				(a) $\Rightarrow$ (b)を示す.実際,一様可積分性の同値条件の補題より
				\begin{align}
					\sup{t \geq 0}{EX_t^+} \leq \sup{t \geq 0}{E|X_t|} < \infty
				\end{align}
				となるから,劣マルチンゲール収束定理より或る$\mathscr{F}_\infty/\borel{\R}$-可測な
				\footnote{
					Theorem 3.15における$X_\infty$は$\pm \infty$も取るが,可積分性より
					$P$-a.s.に$\R$値であるから$X_\infty \defunc_{|X_\infty|<\infty}$を$X_\infty$に置き換えればよい.
				}
				可積分関数$X_\infty$が存在して
				\begin{align}
					\lim_{t \to \infty} X_t = X_\infty
					\quad \mbox{$P$-a.s.}
				\end{align}
				が満たされる.一様可積分性と平均収束の補題より,$t_n \uparrow \infty$となる任意の単調増大列$(t_n)_{n=1}^\infty$に対して
				\begin{align}
					E|X_{t_n} - X_\infty| \longrightarrow 0
					\quad (n \longrightarrow \infty)
				\end{align}
				が成立するから
				\begin{align}
					E|X_t - X_\infty| \longrightarrow 0
					\quad (t \longrightarrow \infty)
				\end{align}
				が従う.
			
			\item[第二段]
				(b) $\Rightarrow$ (c)を示す.(b)の下で,或る可積分関数$X_*$が存在して
				\begin{align}
					E|X_n - X_*| \longrightarrow 0
					\quad (n \longrightarrow \infty)
				\end{align}
				が満たされるから,或る部分列$\left( X_{n_k} \right)_{k=1}^\infty$と$P$-零集合$E$が存在して
				\begin{align}
					\lim_{k \to \infty} X_{n_k}(\omega) = X_*(\omega),
					\quad (\forall \omega \in \Omega \backslash E)
				\end{align}
				となる.$X_{n_k}\defunc_{\Omega \backslash E}$は全て$\mathscr{F}_\infty/\borel{\R}$-可測であるから,
				\begin{align}
					X_\infty \coloneqq \lim_{k \to \infty} X_{n_k} \defunc_{\Omega \backslash E}
				\end{align}
				とおけば$X_\infty$は$\mathscr{F}_\infty/\borel{\R}$-可測,
				かつ$X_\infty = X^*\ \mbox{$P$-a.s.}$より可積分であり
				\begin{align}
					E|X_n - X_\infty| = E|X_n - X_*| \longrightarrow 0
					\quad (n \longrightarrow \infty)
					\label{eq:chapter_1_Problem_3_19_3}
				\end{align}
				を満たす.任意の$t \geq 0$及び$A \in \mathscr{F}_t$に対し
				\begin{align}
					\int_A X_t\ dP \leq \int_A X_n\ dP,
					\quad (\forall n > t)
					\label{eq:chapter_1_Problem_3_19_1}
				\end{align}
				が成り立つから,(\refeq{eq:chapter_1_Problem_3_19_3})より
				\begin{align}
					\int_A X_t\ dP \leq \int_A X_\infty\ dP
					\label{eq:chapter_1_Problem_3_19_2}
				\end{align}
				が出る.
				
			\item[第三段]
				$X_t \geq 0\ (\forall t \geq 0)$を仮定して(c) $\Rightarrow$ (a)を示す.実際,
				劣マルチンゲール性より
				\begin{align}
					\int_{|X_t| > \lambda} |x_t|\ dP
					= \int_{X_t > \lambda} X_t\ dP
					\leq \int_{X_t > \lambda} X_\infty\ dP
				\end{align}
				かつ
				\begin{align}
					P\left( X_t > \lambda \right)
					\leq \frac{1}{\lambda} EX_t
					\leq \frac{1}{\lambda} EX_\infty
				\end{align}
				が成り立ち,$X_\infty$の可積分性より
				\begin{align}
					\sup{t \geq 0}{\int_{|X_t| > \lambda} |x_t|\ dP} 
					\longrightarrow 0
					\quad (\lambda \longrightarrow \infty)
				\end{align}
				となる.
				\QED
		\end{description}
	\end{prf}
	
	\begin{itembox}[l]{Problem 3.20}
		Assume that $\mathscr{F}_0$ contains all the $P$-negligible events in $\mathscr{F}$.
		Then the following four conditions are equivalent for a right-continuous martingale
		$\Set{X_t,\mathscr{F}_t}{0 \leq t < \infty}$:
		\begin{description}
			\item[(a),(b)] as in Problem 3.19;
			\item[(c)] it converges $P$ a.s. (as $t \to \infty$) to an integrable random variable $X_\infty$,
				such that $\Set{X_t,\mathscr{F}_t}{0 \leq t \leq \infty}$ is a martingale;
			\item[(d)] there exists an integrable random variable $Y$, such that $X_t = \cexp{Y}{\mathscr{F}_t}$ a.s. $P$,
				for every $t \geq 0$.
		\end{description}
		Besides, if (d) holds and $X_\infty$ is the random variable in (c), then
		\begin{align}
			\cexp{Y}{\mathscr{F}_\infty} = X_\infty
			\quad \mbox{a.s. $P$}.
			\label{eq:chapter_1_Problem_3_20_1}
		\end{align}
	\end{itembox}
	
	\begin{prf}\mbox{}
		\begin{description}
			\item[第一段] マルチンゲールは劣マルチンゲールであるから,Problem 3.19より(a) $\Rightarrow$ (b)が従う.
				また今の仮定の下では
				(\refeq{eq:chapter_1_Problem_3_19_1})と(\refeq{eq:chapter_1_Problem_3_19_2})
				の不等号が等号に代わり(b) $\Rightarrow$ (c)となる.$Y \coloneqq X_\infty$として(c) $\Rightarrow$ (d)が得られ,
				一様可積分性と条件付き期待値に関する補題(P. \pageref{lem:uniformly_integrability_and_conditional_expectations})
				より(d) $\Rightarrow$ (a)が出る.
				
			\item[第二段]
				(\refeq{eq:chapter_1_Problem_3_20_1})を示す.
				いま,任意の$t \geq 0$及び$A \in \mathscr{F}_t$に対し
				\begin{align}
					\int_A Y\ dP = \int_A X_t\ dP = \int_A X_\infty\ dP
				\end{align}
				が成立するから
				\begin{align}
					\int_A Y\ dP = \int_A X_\infty\ dP,
					\quad (\forall A \in \bigcup_{t \geq 0} \mathscr{F}_t)
				\end{align}
				が従う.$Y$と$X_\infty$の可積分性より
				\begin{align}
					\mathscr{D} \coloneqq
					\Set{A \in \mathscr{F}_\infty}{\int_A Y\ dP = \int_A X_\infty\ dP}
				\end{align}
				はDynkin族をなし乗法族$\bigcup_{t \geq 0} \mathscr{F}_t$を含むから,
				Dynkin族定理より
				\begin{align}
					\int_A Y\ dP = \int_A X_\infty\ dP,
					\quad (\forall A \in \mathscr{F}_\infty)
				\end{align}
				が成立する.
				\QED
		\end{description}
	\end{prf}
\subsection{The Optional Sampling Theorem}
	\begin{itembox}[l]{Lemma: 離散時間の任意抽出定理}
		$0 = t_0 < t_1 < \cdots < t_n < \infty$とし,
		$\Set{X_{t_n},\mathscr{F}_{t_n}}{n=0,\cdots,n}$を劣マルチンゲール,
		$S,T:\Omega \longrightarrow \{t_0,t_1,\cdots,t_n,\infty\}$を$(\mathscr{F}_{t_n})$-停止時刻とする.
		また或る$\mathscr{F}/\borel{\R}$-可測関数$Y$が存在して
		\begin{align}
			X_T(\omega) \coloneqq Y(\omega)\ (\forall \omega \in \{T=\infty\}),
			\quad X_S(\omega) \coloneqq Y(\omega)\ (\forall \omega \in \{S=\infty\})
		\end{align}
		を満たしているとする.このとき,
		\begin{description}
			\item[(a)] $S,T < \infty$.
			\item[(b)] $Y$が可積分で$\cexp{Y}{\mathscr{F}_{t_n}} \geq X_{t_n}\ \mbox{a.s. $P$},\ (n=0,\cdots,n)$を満たす.
		\end{description}
		のいずれかの場合次が成り立つ:
		\begin{align}
			\cexp{X_T}{\mathscr{F}_S} \geq X_{S \wedge T}
			\quad \mbox{a.s. $P$}.
			\label{eq:lem_optional_sampling_theorem_1}
		\end{align}
	\end{itembox}
	
	\begin{prf}\mbox{}
		\begin{description}
			\item[第一段]	
				$S \leq T$と仮定して(\refeq{eq:lem_optional_sampling_theorem_1})を示す.先ず
				\begin{align}
					\int_\Omega |X_S|\ dP
					= \sum_{i=0}^n \int_{\{S=t_i\}} |X_{t_i}|\ dP
						+ \int_{\{S=\infty\}} |Y|\ dP
				\end{align}
				より(a),(b)いずれの場合も$X_S,X_T$は可積分である.
				また,劣マルチンゲール性より任意の$A \in \mathscr{F}_S$に対して
				\begin{align}
					\int_{A \cap \{S=t_i\}} X_{t_i}\ dP
					&= \int_{A \cap \{S=t_i\} \cap \{T=t_i\}} X_{t_i}\ dP
						+ \int_{A \cap \{S=t_i\} \cap \{T>t_i\}} X_{t_i}\ dP \\
					&\leq \int_{A \cap \{S=t_i\} \cap \{T=t_i\}} X_T\ dP
						+ \int_{A \cap \{S=t_i\} \cap \{T>t_i\}} X_{t_{i+1}}\ dP \\
					&= \int_{A \cap \{S=t_i\} \cap \{T=t_i\}} X_T\ dP
						+ \int_{A \cap \{S=t_i\} \cap \{T=t_{i+1}\}} X_T\ dP
						+ \int_{A \cap \{S=t_i\} \cap \{T>t_{i+1}\}} X_{t_{i+1}}\ dP \\
					&\cdots \\
					&\leq \sum_{j=i}^n \int_{A \cap \{S=t_i\} \cap \{T=t_j\}} X_T\ dP
						+ \int_{A \cap \{S=t_i\} \cap \{T>t_n\}} X_{t_n}\ dP
				\end{align}
				及び
				\begin{align}
					\int_{A \cap \{S=\infty\}} X_S\ dP
					= \int_{A \cap \{S=\infty\}} Y\ dP
					= \int_{A \cap \{S=\infty\}} X_T\ dP
				\end{align}
				が成り立つから,(a)の場合は
				\begin{align}
					\int_{A \cap \{S=t_i\}} X_{t_i}\ dP \leq
					\sum_{j=i}^n \int_{A \cap \{S=t_i\} \cap \{T=t_j\}} X_T\ dP
					= \int_{A \cap \{S=t_i\}} X_T\ dP,
				\end{align}
				(b)の場合は
				\begin{align}
					\int_{A \cap \{S=t_i\}} X_{t_i}\ dP
					&\leq \sum_{j=i}^n \int_{A \cap \{S=t_i\} \cap \{T=t_j\}} X_T\ dP
						+ \int_{A \cap \{S=t_i\} \cap \{T>t_n\}} X_{t_n}\ dP \\
					&\leq \sum_{j=i}^n \int_{A \cap \{S=t_i\} \cap \{T=t_j\}} X_T\ dP
						+ \int_{A \cap \{S=t_i\} \cap \{T>t_n\}} Y\ dP \\
					&= \int_{A \cap \{S=t_i\}} X_T\ dP
				\end{align}
				となり,いずれの場合も
				\begin{align}
					\int_A X_S\ dP
					= \sum_{i=0}^n \int_{A \cap \{S=t_i\}} X_{t_i}\ dP
						+ \int_{A \cap \{S=\infty\}} X_S\ dP
					\leq \sum_{i=0}^n \int_{A \cap \{S=t_i\}} X_T\ dP + \int_{A \cap \{S=\infty\}} X_T\ dP
					= \int_A X_T\ dP
				\end{align}
				が成立する.
			
			\item[第二段]
				一般の$S,T$に対して(\refeq{eq:lem_optional_sampling_theorem_1})を示す.
				任意の$A \in \mathscr{F}_S$に対し,Problem 2.17 (P. \pageref{chapter_1_Problem_2_17})
				と前段の結果より
				\begin{align}
					\int_A \cexp{X_T}{\mathscr{F}_S}\ dP
					&= \int_{A \cap \{S \leq T\}} \cexp{X_T}{\mathscr{F}_S}\ dP
						+ \int_{A \cap \{S > T\}} \cexp{X_T}{\mathscr{F}_S}\ dP \\
					&= \int_{A \cap \{S \leq T\}} \cexp{X_T}{\mathscr{F}_{S \wedge T}}\ dP
						+ \int_{A \cap \{S > T\}} X_T\ dP \\
					&\geq \int_{A \cap \{S \leq T\}} X_{S \wedge T}\ dP
					 	+ \int_{A \cap \{S > T\}} X_{S \wedge T}\ dP \\
					&= \int_A X_{S \wedge T}\ dP
				\end{align}
				となる.
		\end{description}
	\end{prf}
\section{The Doob-Meyer Decomposition}
	\begin{itembox}[l]{martingale transform}
		If $A = \Set{A_n,\mathscr{F}_n}{n=0,1,\cdots}$ is predictable with $E|A_n|<\infty$ for every $n$,
		and if $\Set{M_n,\mathscr{F}_n}{n=0,1,\cdots}$ is bounded martingale, then the martingale transform of $A$
		by $M$ defined by
		\begin{align}
			Y_0 = 0 \quad \mbox{and} \quad
			Y_n = \sum_{k=1}^n A_k (M_k - M_{k-1});
			\quad n \geq 1, 
		\end{align}
		is itself a martingale.
	\end{itembox}
	
	\begin{prf}
		$A_k(M_k - M_{k-1})\ (k \leq n)$は$\mathscr{F}_n/\borel{\R}$-可測であるから
		$(Y_n)_{n=1}^\infty$は$(\mathscr{F}_n)$-適合である.また
		\begin{align}
			E|Y_n| = E\left| \sum_{k=1}^n A_k (M_k - M_{k-1}) \right|
			\leq \sum_{k=1}^n \left\{\esssup{\omega \in \Omega}{\left(|M_k(\omega)|+|M_{k-1}(\omega)|\right)}\right\} E|A_k| < \infty
		\end{align}
		が成り立つ.更に任意の$n \geq 0$に対し
		\begin{align}
			\cexp{Y_{n+1} - Y_n}{\mathscr{F}_n}
			= \cexp{A_{n+1}(M_{n+1} - M_n)}{\mathscr{F}_n}
			= A_{n+1}\cexp{M_{n+1} - M_n}{\mathscr{F}_n}
			= 0,
			\quad \mbox{a.s. $P$}
		\end{align}
		が満たされる.
		\QED
	\end{prf}
	
	\begin{itembox}[l]{Doob's decomposition}\label{lem:Doob_decomposition}
		Any submartingale $\Set{X_n,\mathscr{F}_n}{n=0,1,\cdots}$ admits the unique decomposition
		$X_n = M_n + A_n$ as the summation of a martingale $\{M_n,\mathscr{F}_n\}$ and an 
		predictable and increasing sequence $\{A_n,\mathscr{F}_n\}$, where
		\begin{align}
			A_n = \sum_{k=0}^{n-1}\cexp{X_{k+1}-X_k}{\mathscr{F}_k},
			\quad \mbox{a.s. $P$},\ n \geq 1.
		\end{align}
	\end{itembox}
	
	\begin{prf}\mbox{}
		\begin{description}
			\item[第一段]
				Doob分解が存在するとして,分解の一意性を示す.
				実際,分解が存在すれば
				\begin{align}
					A_{n+1} - A_n = \cexp{A_{n+1}-A_n}{\mathscr{F}_{n}}
					= \cexp{X_{n+1}-X_n}{\mathscr{F}_{n}} - \cexp{M_{n+1}-M_n}{\mathscr{F}_{n}}
					= \cexp{X_{n+1}-X_n}{\mathscr{F}_{n}},
					\quad \mbox{a.s. $P$}
				\end{align}
				が成立し,$A_n\ (n \geq 1)$は
				\begin{align}
					A_n = \sum_{k=0}^{n-1} \cexp{X_{k+1}-X_k}{\mathscr{F}_{k}},
					\quad \mbox{a.s. $P$}
				\end{align}
				を満たすことになり分解の一意性が出る.
				
			\item[第二段]
				分解可能性を示す.
				\begin{align}
					A_0 \coloneqq 0,
					\quad A_n \coloneqq \sum_{k=0}^{n-1} \cexp{X_{k+1}-X_k}{\mathscr{F}_{k}},
					\quad (n=1,2,\cdots)
				\end{align}
				と定めれば$(A_n)$は可予測かつ可積分であり,
				\begin{align}
					A_{n+1} - A_n = \cexp{X_{k+1}-X_k}{\mathscr{F}_{k}} \geq 0,
					\quad \mbox{a.s. $P$}
					\ (\forall n \geq 1)
				\end{align}
				より増大過程である.また$M_n \coloneqq X_n - A_n$により$(\mathscr{F}_n)$-適合かつ可積分な過程を定めれば,
				\begin{align}
					\cexp{M_{n+1} - M_n}{\mathscr{F}_n}
					&= \cexp{(X_{n+1} - X_n)-(A_{n+1}-A_n)}{\mathscr{F}_n} \\
					&= \cexp{X_{n+1} - X_n}{\mathscr{F}_n} - \cexp{\cexp{X_{n+1} - X_n}{\mathscr{F}_n}}{\mathscr{F}_n}
					= 0,
					\quad \mbox{a.s. $P$}
				\end{align}
				が成り立つから$\{M_n,\mathscr{F}_n\}$はマルチンゲールである.
				\QED
		\end{description}
	\end{prf}
	
	\begin{itembox}[l]{Proposition 4.3 修正}
		An increasing random sequence $A$ \textcolor{red}{has a predictable modification}
		if and only if it is natural.
	\end{itembox}
	
	\begin{prf}
		$A$が可予測な修正$\tilde{A}$を持つとき,任意の有界マルチンゲール$M$に対して
		\begin{align}
			\tilde{Y}_0 \coloneqq 0,
			\quad \tilde{Y}_n \coloneqq \sum_{k=1}^n \tilde{A}_k(M_k - M_{k-1}); \quad n \geq 1
		\end{align}
		は$(\mathscr{F}_n)$-マルチンゲールとなる.
		このとき$M_n \tilde{A}_n$と$\sum_{k=1}^n M_{k-1}(\tilde{A}_k - \tilde{A}_{k-1})$は可積分であり
		\begin{align}
			0 = E \tilde{Y}_n = E\left[ M_n \tilde{A}_n - \sum_{k=1}^n M_{k-1}(\tilde{A}_k - \tilde{A}_{k-1}) \right]
			= E(M_n A_n) - E\sum_{k=1}^n M_{k-1}(A_k - A_{k-1}),
			\quad (\forall n \geq 1)
		\end{align}
		が成り立つから$A$はナチュラルである.逆に$A$がナチュラルであるとき,
		有界マルチンゲール$M$に対して
		\begin{align}
			0 &= E\left[ M_n A_n - \sum_{k=1}^n M_{k-1}(A_k - A_{k-1}) \right] \\
			&= E\left[ A_n(M_n-M_{n-1}) \right] - E\left[ M_{n-1} A_{n-1} - \sum_{k=1}^{n-1} M_{k-1}(A_k - A_{k-1}) \right] \\
			&= E\left[ A_n(M_n-M_{n-1}) \right],
			\quad (\forall n \geq 1)
		\end{align}
		が成り立つ.一方で
		\begin{align}
			E\left[ M_{n-1}(A_n-\cexp{A_n}{\mathscr{F}_{n-1}}) \right]
			&= E\left[ \cexp{M_{n-1} (A_n-\cexp{A_n}{\mathscr{F}_{n-1}})}{\mathscr{F}_{n-1}} \right] \\
			&= E\left[ M_{n-1} \cexp{A_n-\cexp{A_n}{\mathscr{F}_{n-1}}}{\mathscr{F}_{n-1}} \right]
			= 0,
			\quad (\forall n \geq 1)
		\end{align}
		及び
		\begin{align}
			E\left[ \cexp{A_n}{\mathscr{F}_{n-1}}(M_n-M_{n-1}) \right]
			&= E\left[ \cexp{ \cexp{A_n}{\mathscr{F}_{n-1}}(M_n-M_{n-1})}{\mathscr{F}_{n-1}} \right] \\
			&= E\left[ \cexp{A_n}{\mathscr{F}_{n-1}}\cexp{M_n-M_{n-1}}{\mathscr{F}_{n-1}} \right]
			= 0,
			\quad (\forall n \geq 1)
		\end{align}
		となるから
		\begin{align}
			E\left[ M_n(A_n - \cexp{A_n}{\mathscr{F}_{n-1}}) \right]
			&= E\left[ A_n(M_n-M_{n-1}) \right] \\
			&\quad	+ E\left[ M_{n-1}(A_n-\cexp{A_n}{\mathscr{F}_{n-1}}) \right] \\
			&\quad	- E\left[ \cexp{A_n}{\mathscr{F}_{n-1}}(M_n-M_{n-1}) \right] \\
			&= 0,
			\quad (\forall n \geq 1)
		\end{align}
		が従う.ここで各$n \geq 1$に対し,
		$\borel{\R}/\borel{\R}$-可測関数$\operatorname{sgn} = \defunc_{(0,\infty)} - \defunc_{(-\infty,0)}$を用いて
		\begin{align}
			M^{(n)}_k \coloneqq 
			\begin{cases}
				\sgn{A_n - \cexp{A_n}{\mathscr{F}_{n-1}}}, & (k \geq n), \\
				\cexp{\sgn{A_n - \cexp{A_n}{\mathscr{F}_{n-1}}}}{\mathscr{F}_k}, & (0 \leq k < n)
			\end{cases}
		\end{align}
		により有界マルチンゲール$M^{(n)} = \Set{M^{(n)}_k,\mathscr{F}_k}{k=0,1,\cdots}$を定めれば,
		\begin{align}
			0 = E\left[ M^{(n)}_n(A_n - \cexp{A_n}{\mathscr{F}_{n-1}}) \right] 
			= E\left| A_n - \cexp{A_n}{\mathscr{F}_{n-1}} \right|,
			\quad (\forall n \geq 1)
		\end{align}
		が得られ
		\begin{align}
			\tilde{A}_0 \coloneqq 0,
			\quad \tilde{A}_n \coloneqq \cexp{A_n}{\mathscr{F}_{n-1}}; \quad n \geq 1
		\end{align}
		は$A$の可予測な修正となる.
		\QED
	\end{prf}
	
	\begin{itembox}[l]{区別不能性によるパスの同値類}
		区間\footnotemark $I \subset [0,\infty)$
		の上で右連続な確率過程の全体を$RCSP(I)$と書く.また$RCSP([0,\infty))$は$RCSP$と書く.
		任意の$M = \Set{M_t}{t \in I},N = \Set{N_t}{t \in I} \in RCSP(I)$に対し,
		\begin{align}
			\{M_t = N_t,\ \forall t \in I\} = 
			\begin{cases}
				\displaystyle \bigcap_{r \in (I \cap \Q) \cup \{\sup{}{I}\}}\{M_r = N_r\}, & (\sup{}{I} \in I), \\
				\displaystyle \bigcap_{r \in I \cap \Q}\{M_r = N_r\}, & (\sup{}{I} \notin I)
			\end{cases}
		\end{align}
		となるから$\{M_t = N_t,\ \forall t \in I\}$は可測であり,
		このとき,
		\begin{align}
			M \sim N \quad \overset{\mathrm{def}}{\Longleftrightarrow} \quad 
			P(M_t = N_t,\ \forall t \in I) = 1
			\label{eq:equivalence_with_respect_to_path}
		\end{align}
		により同値関係$\sim$が定まる.
	\end{itembox}
	\footnotetext{
		この場合区間は$[a,b],(a,b),[a,b),(a,b],[a,\infty),(a,\infty),\ (0 \leq a < b < \infty)$のいずれかと考える.
	}
	
	\begin{itembox}[l]{Definition 4.4 修正}
		\textcolor{red}{Let $I \subset [0,\infty)$ be an interval.}
		An adapted process \textcolor{red}{$A = \Set{A_t,\mathscr{F}_t}{t \in I}$} 
		is called increasing if \textcolor{red}{for all $\omega \in \Omega$} we have
		\begin{description}
			\item[(a)] $A_0(\omega) = 0$
			\item[(b)] $t \longmapsto A_t(\omega)$ is nondecreasing, right-continuous function,
		\end{description}
		and $E(A_t) < \infty$ holds for every \textcolor{red}{$t \in I$}.
		An increasing process is called integrable if \textcolor{red}{$E\left(A_{\infty}\right) < \infty$,
		where $A_{\infty} = \lim_{t \to \sup{}{I}} A_t$.
		Since $A$ is nondecreasing, $A_{\infty} = A_{(\sup{}{I})-}$ if $\sup{}{I} \in I$.}
	\end{itembox}
	
	\begin{itembox}[l]{Definition 4.5 修正}
		\textcolor{red}{Let $I \subset [0,\infty)$ be an interval and $\alpha \coloneqq \inf{}{I}$.}
		An increasing processs \textcolor{red}{$A = \Set{A_t,\mathscr{F}_t}{t \in I}$} 
		is called natural if for every bounded, 
		\textcolor{red}{$RCLL$ martingale $\Set{M_t,\mathscr{F}_t}{t \in I}$} we have
		\begin{align}
			E \int_{(\alpha,t]} M_s\ dA_s = E \int_{(\alpha,t]} M_{s-}\ dA_s,
			\quad \mbox{for every $t \in (\alpha,\infty) \cap I$}.
		\end{align}
		\textcolor{red}{Let us denote the subset of $RCSP(I)$ as
		\begin{align}
			NAT(I) \coloneqq
			\Set{A \in RCSP(I)}{\mbox{natural}},
			\quad NAT \coloneqq NAT([0,\infty))
		\end{align}
		and the equivalent class of $A \in NAT$
		in the meaning of (\refeq{eq:equivalence_with_respect_to_path}) as $[A]_{NAT}\ ( \subset NAT)$.}
	\end{itembox}
	
	プロセスが$RCLL$とは全てのパスが$RCLL$であるということである.Theorem 3.8によれば
	右連続な劣マルチンゲールはa.e.のパスが$RCLL$であるから,
	(\refeq{eq:equivalence_with_respect_to_path})の意味で同値である.
	$A$も全てのパスが右連続かつ単調非減少であるから,
	全ての$\omega \in \Omega$に対し$\int_{(0,t]} M_s(\omega)\ dA_s(\omega)$と
	$\int_{(0,t]} M_{s-}(\omega)\ dA_s(\omega)$が定義される.
	たぶん余計な煩雑さを回避できる.
		
	\begin{itembox}[l]{$RCLL$なパスの不連続点は高々可算個}
		$(S,d)$を距離空間とする.写像$f:[0,\infty) \longrightarrow S$について
		各点$t \in [0,\infty)$で右連続かつ各点$t \in (0,\infty)$で左極限が存在するとき,
		$f$の不連続点は存在しても高々可算個である.
	\end{itembox}
	
	\begin{prf}
		各点$t > 0$における$f$の左極限を$f(t-)$と書けば
		\begin{align}
			\mbox{$f$が$t \in (0,\infty)$で不連続}
			\quad \Leftrightarrow \quad
			\mbox{$d(f(t),f(t-)) > 0$}
		\end{align}
		が成立するから,任意に$T > 0$を選び固定して
		\begin{align}
			D(n) \coloneqq \Set{t \in (0,T]}{\frac{1}{n+1} \leq d(f(t),f(t-)) < \frac{1}{n}},
			\quad E(n) \coloneqq \Set{t \in (0,T]}{n \leq d(f(t),f(t-)) < n+1}
		\end{align}
		とおけば
		\begin{align}
			D_T \coloneqq \Set{t \in (0,T]}{\mbox{$f$が$t \in (0,\infty)$で不連続}}
			= \bigcup_{n=1}^\infty D(n) \cup E(n)
		\end{align}
		となる.このとき$D(n),E(n)$は全て有限集合である.実際,或る$n$に対し$D(n)$が無限集合なら
		\begin{align}
			\left\{ t_k \right\}_{k=1}^\infty \subset D(n),
			\quad t_k \neq t_j\ (k \neq j)
		\end{align}
		を満たす可算集合が存在し,$[0,T]$のコンパクト性より
		或る部分列$\left( t_{k_m} \right)_{m=1}^\infty$は
		或る$y \in [0,T]$に収束する.
		$y=0$の場合,右連続の仮定より$1/2(n+1) > \epsilon > 0$に対し或る$\delta > 0$が存在して
		\begin{align}
			d(f(0),f(t)) < \epsilon, \quad (\forall 0 < t < \delta)
		\end{align}
		が成り立つが,一方で$0 < t_{k_m} < \delta$を満たす$t_{k_m}$が存在して
		\begin{align}
			\frac{1}{n+1} - \epsilon < d(f(t_{k_m}),f(t_{k_m}-)) - d(f(0),f(t_{k_m}-))
			\leq d(f(0),f(t_{k_m})) < \epsilon 
		\end{align}
		となり矛盾が生じる.
		$y > 0$の場合も,$1/2(n+1) > \epsilon > 0$に対し或る$\delta > 0$が存在して
		\begin{align}
			d(f(y-),f(t)) < \epsilon, \quad (\forall t \in (y-\delta,y))
		\end{align}
		となるが,$f$が$y$で右連続であるから(或は$y=T$のとき) $y-\delta < t_{k_m} \leq y$を満たす$t_{k_m}$が存在して
		\begin{align}
			\frac{1}{n+1} - \epsilon < 
			d(f(t_{k_m}-),f(t_{k_m})) - d(f(t_{k_m}-),f(y-)) \leq d(f(y-),f(t_{k_m})) < \epsilon
		\end{align}
		が従い矛盾が生じる.よって任意の$n \geq 1$に対して$D(n)$は有限集合であり,同様に
		$E(n)$も有限集合であるから$D_T$は高々可算集合である.
		$f$の不連続点の全体は$\bigcup_{T=1}^\infty D_T$に一致するから高々可算個である.
		\QED
	\end{prf}
	
	\begin{itembox}[l]{Remarks 4.6 (i) 修正}
		If $A$ is an increasing and $X$ a measurable process, then with $\omega \in \Omega$ fixed,
		the sample path $\Set{X_t(\omega)}{0 \leq t < \infty}$ is a measurable function from $[0,\infty)$
		into $\R$. It follows that the Lebesgue-Stieltjes integrals
		\begin{align}
			I^{\pm}_t(\omega) \coloneqq
			\int_{(0,t]} X^\pm_s(\omega)\ dA_s(\omega)
		\end{align}
		are well defined. \textcolor{red}{If $X$ is bounded, right-continuous and adapted
		to the filtration $(\mathscr{F}_t)$, then $I$ is finite, right-continuous and 
		$(\mathscr{F}_t)$-progressively measurable.}
	\end{itembox}
	
	\begin{prf}
		$X$が$\borel{[0,\infty)} \otimes \mathscr{F}/\borel{\R}$-可測なら,
		補題\ref{lem:Fubini_lemma_1} (P. \pageref{lem:Fubini_lemma_1})より
		$[0,\infty) \ni t \longmapsto X_t(\omega)$は
		$\borel{[0,\infty)}/\borel{\R}$-可測である.
		また全ての$\omega \in \Omega$に対し$t \longmapsto A_t(\omega)$は右連続非減少であるから
		\begin{align}
			\mu_\omega((a,b]) = A_b(\omega) - A_a(\omega),
			\quad (\forall (a,b] \subset [0,\infty)),
			\quad \mu_\omega(\{0\}) = 0
		\end{align}
		を満たす$\left([0,\infty),\borel{[0,\infty)}\right)$上の$\sigma$-有限測度が唯一つ存在して
		\begin{align}
			I^\pm_t(\omega) = \int_{(0,t]} X^\pm_s(\omega)\ dA_s(\omega)
			\coloneqq \int_{(0,t]} X^\pm_s(\omega)\ \mu_\omega(ds),
			\quad (0 < t < \infty)
		\end{align}
		及び$I_t \coloneqq I^+_t - I^-_t$が定義される.
		特に$\sup{s \in (0,t]}{|X^\pm_s|} \leq B < \infty$なら
		\begin{align}
			\left|I^\pm_t\right| \leq B A_t
		\end{align}
		となるから$I^\pm_t$は有限確定する.$X$が有界かつ右連続$(\mathscr{F}_t)$-適合であるとき,
		$t>0$を固定し$t^{(n)}_j \coloneqq tj/2^n$として
		\begin{align}
			X^{(n)\pm}_s \coloneqq X_0 \defunc_{\{0\}}(s) + 
				\sum_{j=0}^{2^n-1} X_{t^{(n)}_{j+1}} 
				\defunc_{\left(t^{(n)}_j,t^{(n)}_{j+1}\right]}(s)
		\end{align}
		とおけば右連続性より$X^{(n)\pm}_s \longrightarrow X^\pm_s,\ (\forall s \in [0,t])$が成立し,かつ
		\begin{align}
			I^{(n)\pm}_t \coloneqq \int_{(0,t]} X^{(n)\pm}_s\ dA_s
			= \sum_{j=0}^{2^n-1} X_{t^{(n)}_{j+1}} \left(A_{t^{(n)}_j} - A_{t^{(n)}_{j+1}}\right)
		\end{align}
		となり$I^{(n)\pm}_t$の$\mathscr{F}_t/\borel{\R}$-可測性が得られる.
		$X$が有界であるからLebesgueの収束定理より
		\begin{align}
			I^{\pm}_t = \lim_{n \to \infty} \int_{(0,t]} X^{(n)\pm}_s\ dA_s
			= \lim_{n \to \infty} I^{(n)\pm}_t
		\end{align}
		が成り立ち,定理\ref{lem:measurability_metric_space}より
		$I^{\pm}_t$の$\mathscr{F}_t/\borel{\R}$-可測性が従う.
		また$t<T$及び$\{t_n\}_{n=1}^\infty \subset (t,T],\ t_n \downarrow t$に対して,Lebesgueの収束定理より
		\begin{align}
			\lim_{n \to \infty} I^\pm_{t_n}
			= \lim_{n \to \infty} \int_{(0,T]} \defunc_{(0,t_n]}(s)X^\pm_s\ dA_s
			= \int_{(0,T]} \defunc_{(0,t]}(s)X^\pm_s\ dA_s
			= I^\pm_t
		\end{align}
		が成立し$t \longmapsto I_t(\omega)$の右連続性が出る.$I$は右連続$(\mathscr{F}_t)$-適合過程であるから
		$(\mathscr{F}_t)$-発展的可測である.
		\QED
	\end{prf}
	
	\begin{itembox}[l]{Remark 4.6 (ii) 修正}
		Every continuous, increasing process is natural. Indeed then, for \textcolor{red}{every} $\omega \in \Omega$
		we have
		\begin{align}
			\int_{(0,t]} (M_s(\omega)-M_{s-}(\omega))\ dA_s(\omega) = 0
			\quad \mbox{for every $0 < t < \infty$},
		\end{align}
		because every path $\Set{M_s(\omega)}{0 \leq s < \infty}$ has only countably many discontinuities
		(Theorem 3.8(v)).
	\end{itembox}
	
	\begin{prf}
		$RCLL$なパスの不連続点は高々可算個であり,
		連続な$A$で作る測度に対し一点集合は零集合となる.
		\QED
	\end{prf}
	
	\begin{itembox}[l]{Lemma 4.7 修正}
		If $A$ is an increasing process and $\Set{M_t,\mathscr{F}_t}{0 \leq t < \infty}$ is a bounded,
		\textcolor{red}{$RCLL$} martingale, then
		\begin{align}
			E(M_t A_t) = E \int_{(0,t]} M_s\ dA_s, \quad (\forall t > 0).
			\label{eq:chapter_1_lemma_4_7}
		\end{align}
	\end{itembox}
	
	\begin{prf}
		$t_j^{(n)} \coloneqq jt/2^n,\ (j=0,1,\cdots,2^n)$として
		\begin{align}
			M^{(n)}_s \coloneqq \sum_{j=1}^{2^n} \defunc_{\left(t^{(n)}_{j-1},t^{(n)}_j\right]}(s) M_{t^{(n)}_j},
			\quad (\forall s \in (0,t])
		\end{align}
		とおけば,$M$のパスの右連続性より任意の$s \in (0,t]$で$\lim_{n \to \infty} M^{(n)}_s = M_s$となる.また
		\begin{align}
			E \left[ A_{t^{(n)}_{j-1}} \left( M_{t^{(n)}_j} - M_{t^{(n)}_{j-1}} \right) \right]
			= E \left[ A_{t^{(n)}_{j-1}} \cexp{M_{t^{(n)}_j} - M_{t^{(n)}_{j-1}}}{\mathscr{F}_{t_{j-1}}} \right]
			= 0,
			\quad (j=1,\cdots,2^n)
		\end{align}
		が満たされるから任意の$n \geq 1$で
		\begin{align}
			E\int_{(0,t]} M^{(n)}_s\ dA_s
			&= E \sum_{j=1}^{2^n} M_{t^{(n)}_j} \left( A_{t^{(n)}_j} - A_{t^{(n)}_{j-1}} \right) \\
			&= E(M_t A_t) - \sum_{j=1}^{2^n} E \left[ A_{t^{(n)}_{j-1}} \left( M_{t^{(n)}_j} - M_{t^{(n)}_{j-1}} \right) \right] \\
			&= E(M_t A_t)
		\end{align}
		が成立する.仮定より$\sup{s \geq 0}{|M_s|} \leq b < \infty$を満たす$b$が存在して
		\begin{align}
			\left| \int_{(0,t]} M^{(n)}_s\ dA_s \right| \leq b (A_t - A_0) = b A_t,
			\quad (\forall n \geq 1)
		\end{align}
		となり,$A_t$の可積分性とLebesgueの収束定理より
		\begin{align}
			\lim_{n \to \infty} E \int_{(0,t]} M^{(n)}_s\ dA_s = E \lim_{n \to \infty} \int_{(0,t]} M^{(n)}_s\ dA_s
			= E \int_{(0,t]} M_s\ dA_s 
		\end{align}
		が従い(\refeq{eq:chapter_1_lemma_4_7})を得る.
		\QED
	\end{prf}
	
	\begin{itembox}[l]{Definition 4.8 修正}
		Let us consider the class $\mathscr{S}(\mathscr{S}_a)$ such as
		\begin{align}
			\mathscr{S} \coloneqq \Set{T:\mbox{stopping time of $(\mathscr{F}_t)$}}{\textcolor{red}{T < \infty}},
			\quad \mathscr{S}_a \coloneqq \Set{T:\mbox{stopping time of $(\mathscr{F}_t)$}}{\textcolor{red}{T \leq a}},\ (a > 0).
		\end{align}
		The right-continuous process $\Set{X_t,\mathscr{F}_t}{0 \leq t < \infty}$ is said to be 
		of class $D$, if the family $\{X_T\}_{T \in \mathscr{S}}$ is uniformly integrable;
		of class $DL$, if the family $\{X_T\}_{T \in \mathscr{S}_a}$ is uniformly integrable,
		for every $0 < a < \infty$.
	\end{itembox}
	$T \in \mathscr{S}(\mbox{resp. } \mathscr{S}_a)$, then 
	$T(\omega) < \infty\ (\mbox{resp. } \leq a)$
	for all $\omega \in \Omega$, not $P$-a.s. $\omega$.
	
	\begin{itembox}[l]{Problem 4.9 修正}
		$X = \Set{X_t,\mathscr{F}_t}{0 \leq t < \infty}$ is a right-continuous submartingale.
		Show that under any one of the following conditions, $X$ is of class $DL$.
		\begin{description}
			\item[(a)] $X_t \geq 0$ a.s. for every $t \geq 0$.
			\item[(b)] $X$ has the special form
				\begin{align}
					X_t = M_t + A_t, \quad 0 \leq t < \infty
				\end{align}
				suggested by the Doob-Meyer decomposition, where $\Set{M_t,\mathscr{F}_t}{0 \leq t < \infty}$
				is a martingale and $\Set{A_t,\mathscr{F}_t}{0 \leq t < \infty}$ is an increasing process.
		\end{description}
		Show also that if \textcolor{red}{$\mathscr{F}_0$ contains all the $P$-negligible events in $\mathscr{F}$} and
		$X$ is a uniformly integrable martingale, then it is of class $D$.
	\end{itembox}
	
	\begin{prf}\mbox{}
		\begin{description}
			\item[(a)]
				任意の$T \in \mathscr{S}_a$に対して
				$X_T$は$\mathscr{F}_T/\borel{\R}$-可測であるから
				(Proposition 2.18 修正),任意抽出定理より
				\begin{align}
					\int_{\{X_T > \lambda\}} X_T\ dP
					\leq \int_{\{X_T > \lambda\}} X_a\ dP,
					\quad (\forall \lambda > 0)
				\end{align}
				及び
				\begin{align}
					P\left( X_T > \lambda \right)
					\leq \frac{EX_T}{\lambda}
					\leq \frac{EX_a}{\lambda},
					\quad (\forall \lambda > 0)
				\end{align}
				が成立する.$X_a$が可積分であるから
				\begin{align}
					\sup{T \in \mathscr{S}_a}{\int_{\{X_T > \lambda\}} X_T\ dP}
					\longrightarrow 0
					\quad (\lambda \longrightarrow \infty)
				\end{align}
				となり,$(X_T)_{T \in \mathscr{S}_a}$の一様可積分性が得られる.
				
			\item[(b)]
				$a > 0$とすれば,任意抽出定理より
				\begin{align}
					M_T = \cexp{M_a}{\mathscr{F}_T},\ \mbox{a.s. $P$,}
					\quad (\forall T \in \mathscr{S}_a)
				\end{align}
				が成り立つから,定理\ref{lem:uniformly_integrability_and_conditional_expectations}
				(P. \pageref{lem:uniformly_integrability_and_conditional_expectations})より
				$(M_T)_{T \in \mathscr{S}_a}$は一様可積分である.このとき
				\begin{align}
					\int_{\{|X_T| > \lambda\}} |X_T|\ dP
					&\leq 2\int_{\{|M_T| > \lambda/2\}} |M_T|\ dP + 2\int_{\{|A_T| > \lambda/2\}} |A_T|\ dP \\
					&\leq 2\sup{T \in \mathscr{S}_a}{\int_{\{|M_T| > \lambda/2\}} |M_T|\ dP} + 2\int_{\{A_a > \lambda/2\}} A_a\ dP \\
					&\longrightarrow 0 \quad (\lambda \longrightarrow \infty)
				\end{align}
				が従い$(X_T)_{T \in \mathscr{S}_a}$の一様可積分性が出る.
		\end{description}
		$X$が一様可積分なマルチンゲールであるとき,Problem 3.20より
		\begin{align}
			X_t = \cexp{X_\infty}{\mathscr{F}_t},\ \mbox{a.s. $P$},
			\quad (\forall t \geq 0)
		\end{align}
		を満たす$\mathscr{F}_\infty/\borel{\R}$-可測可積分関数$X_\infty$が存在し,任意抽出定理より
		\begin{align}
			X_T = \cexp{X_\infty}{\mathscr{F}_T},\ \mbox{a.s. $P$},
			\quad (\forall T \in \mathscr{S})
		\end{align}
		が成り立つから$X$はクラス$DL$に属する.
		\QED
	\end{prf}
	
	\begin{itembox}[l]{Problem 4.11 修正}
		Let $(X,\mathscr{F},\mu)$ be a measure space and  
		$\left\{f_n\right\}_{n=1}^\infty$ be a sequence of integrable complex functions on $(X,\mathscr{F},\mu)$
		which converges weakly in $L^1$ to an integrable complex function $f$.
		Then for each $\sigma$-field $\mathscr{G} \subset \mathscr{F}$
		where $(X,\mathscr{G},\left.\mu\right|_{\mathscr{G}})$ is $\sigma$-finite,
		the sequence $\cexp{f_n}{\mathscr{G}}$ converges to $\cexp{f}{\mathscr{G}}$ weakly in $L^1$.
	\end{itembox}
	
	\begin{prf}
		$\nu \coloneqq \left.\mu\right|_{\mathscr{G}}$とおく.
		定理\ref{thm:properties_of_conditional_expectations}より
		任意の$g \in L^\infty(\mu)$と$F \in L^1(\mu)$に対して
		\begin{align}
			\int_X g\cexp{F}{\mathscr{G}}\ d\mu
			&= \int_X \cexp{g\cexp{F}{\mathscr{G}}}{\mathscr{G}}\ d\nu \\
			&= \int_X \cexp{g}{\mathscr{G}}\cexp{F}{\mathscr{G}}\ d\nu \\
			&= \int_X \cexp{\cexp{g}{\mathscr{G}}F}{\mathscr{G}}\ d\nu \\
			&= \int_X \cexp{g}{\mathscr{G}}F\ d\mu
		\end{align}
		と$\Norm{\cexp{g}{\mathscr{G}}}{L^\infty(\nu)} \leq \Norm{g}{L^\infty(\mu)}$が成り立ち
		\begin{align}
			\lim_{n \to \infty} \int_X g\cexp{f_n}{\mathscr{G}}\ d\mu
			= \lim_{n \to \infty} \int_X \cexp{g}{\mathscr{G}}f_n\ d\mu
			= \int_X \cexp{g}{\mathscr{G}}f\ d\mu
			= \int_X g\cexp{f}{\mathscr{G}}\ d\mu
		\end{align}
		となるから$\cexp{f_n}{\mathscr{G}}$は$\cexp{f}{\mathscr{G}}$に$L^1(\mu)$で弱収束する.
		\QED
	\end{prf}
	
	\begin{itembox}[l]{Lemma for theorem 4.10}\label{lem:uniqueness_of_Doob_Meyer_decomposition}
		Let $I \subset [0,\infty)$ be an interval and 
		$\Set{M_t,\mathscr{F}_t}{t \in I}$ be a right-continuous martingale,
		where the filtration $(\mathscr{F}_t)_{t \in I}$ is usual.
		If $M$ is a difference of two natural processes 
		$\Set{A_t,\mathscr{F}_t}{t \in I}$
		and $\Set{B_t,\mathscr{F}_t}{t \in I}$, namely
		\begin{align}
			M_t = A_t - B_t; \quad \forall t \in I,
		\end{align}
		then $P\Set{M_t = 0}{\forall t \in I} = 1$.
	\end{itembox}
	
	\begin{prf}
		$a_0 \coloneqq \inf{}{I}$として任意に$a \in I \cap (a_0,\infty)$を取り,
		\begin{align}
			t^{(n)}_j \coloneqq a_0 + \frac{j}{2^n}(a-a_0), \quad (j=0,1,\cdots,2^n)
		\end{align}
		とおく.任意の有界かつ$RCLL$なマルチンゲール$\xi = \Set{\xi_t,\mathscr{F}_t}{t \in I}$に対し
		\begin{align}
			\xi^{(n)}_t \coloneqq \sum_{j=1}^{2^n} \defunc_{\left(t_{j-1}^{(n)},t_j^{(n)}\right]}(t)\ \xi_{t^{(n)}_{j-1}},
			\quad (\forall t \in (a_0,a])
		\end{align}
		とおけば,任意の$\omega \in \Omega$と$t \in (a_0,a]$で
		\begin{align}
			\lim_{n \to \infty} \xi^{(n)}_t(\omega) = \xi_{t-}(\omega)
		\end{align}
		が満たされるからLebesgueの収束定理より
		\begin{align}
			&\lim_{n \to \infty} \int_{(a_0,a]} \xi^{(n)}_t(\omega)\ dA_t(\omega) 
				= \int_{(a_0,a]} \xi_{t-}(\omega)\ dA_t(\omega), \\
			&\lim_{n \to \infty} \int_{(a_0,a]} \xi^{(n)}_t(\omega)\ dB_t(\omega) 
				= \int_{(a_0,a]} \xi_{t-}(\omega)\ dB_t(\omega)
		\end{align}
		が成立する.また$A_a,B_a$の可積性と$\xi$の有界性により,再びLebesgueの収束定理を適用すれば
		\begin{align}
			E\left[ \xi_a\left( A_a - B_a \right) \right]
			&= E\left[ \xi_a A_a \right] -  E\left[ \xi_a B_a \right]
			= E \int_{(a_0,a]} \xi_{t-}\ dA_t - E\int_{(a_0,a]} \xi_{t-}\ dB_t \\
			&= E \left[ \lim_{n \to \infty} \int_{(a_0,a]} \xi^{(n)}_t\ dA_t \right]
				- E \left[ \lim_{n \to \infty} \int_{(a_0,a]} \xi^{(n)}_t\ dB_t \right] \\
			&= \lim_{n \to \infty} E\left[ \sum_{j=1}^{2^n}\xi_{t^{(n)}_{j-1}}\left( A_{t^{(n)}_j} - A_{t^{(n)}_{j-1}} \right) \right]
				-  \lim_{n \to \infty} E \left[ \sum_{j=1}^{2^n}\xi_{t^{(n)}_{j-1}}\left( B_{t^{(n)}_j} - B_{t^{(n)}_{j-1}} \right) \right] \\
			&= \lim_{n \to \infty} E \left[ \sum_{j=1}^{2^n}\xi_{t^{(n)}_{j-1}}\left( M_{t^{(n)}_j} - M_{t^{(n)}_{j-1}} \right) \right]
		\end{align}
		が従い,このとき右辺は$M$のマルチンゲール性より
		\begin{align}
			E\xi_{t^{(n)}_{j-1}}\left( M_{t^{(n)}_j} - M_{t^{(n)}_{j-1}} \right)
			= E \left[\cexp{\xi_{t^{(n)}_{j-1}}\left( M_{t^{(n)}_j} - M_{t^{(n)}_{j-1}} \right)}{\mathscr{F}_{t^{(n)}_{j-1}}} \right]
			= E \left[ \xi_{t^{(n)}_{j-1}}\cexp{M_{t^{(n)}_j} - M_{t^{(n)}_{j-1}}}{\mathscr{F}_{t^{(n)}_{j-1}}} \right]
			= 0 
		\end{align}
		となるから
		\begin{align}
			E\left[ \xi_a\left( A_a - B_a \right) \right] = 0
		\end{align}
		が得られる.$\xi$を有界マルチンゲール
		$\Set{\cexp{\sgn{A_a - B_a}}{\mathscr{F}_t},\mathscr{F}_t}{t \in I}$
		の$RCLL$な修正とすれば(usual条件よりTheorem 3.13を適用)
		\begin{align}
			0 = E\left[ \xi_a\left( A_a - B_a \right) \right]
			= E\left[ \sgn{A_a - B_a}\left( A_a - B_a \right) \right]
			= E\left| A_a - B_a \right|
		\end{align}
		が成り立ち,$a > 0$の任意性及び$A,B$のパスの右連続性より
		\begin{align}
			P\left[ \Set{A_t = B_t}{t \in I}\right] =
			\begin{cases}
				\displaystyle P\Biggl( \bigcap_{r \in (I \cap \Q) \cup \{\sup{}{I}\}}\{A_r = B_r\} \Biggr) = 1, 
					& (\sup{}{I} \in I), \\
				\displaystyle P\Biggl( \bigcap_{r \in I \cap \Q}\{A_r = B_r\} \Biggr) = 1, & (\sup{}{I} \notin I)
			\end{cases}
		\end{align}
		が出る.
		\QED
	\end{prf}
	
	\begin{itembox}[l]{Theorem 4.10 (Doob-Meyer Decomposition) 修正}
		Let $\{\mathscr{F}_t\}$ satisfy the usual conditions. If the right-continuous
		submartingale $X = \Set{X_t,\mathscr{F}_t}{0 \leq t < \infty}$ is of class $DL$, then
		\textcolor{red}{there exists a unique $[A]_{NAT}$ where $X - A'$ is right-continuous martingale
		for every $A' \in [A]_{NAT}$.}
		Further, if $X$ is of class $D$, then $M$ is a uniformly integrable martingale 
		and $A$ is integrable.	
	\end{itembox}
	
	\begin{prf}\mbox{}
		\begin{description}
			\item[第一段]
				$[A]_{NAT}$の一意性を示す.二つの右連続マルチンゲール$M,M'$とナチュラルな$A,A'$により
				\begin{align}
					X_t = M_t + A_t = M'_t + A'_t,
					\quad \forall t \geq 0
				\end{align}
				と書けるとき,
				\begin{align}
					B=\Set{B_t \coloneqq A_t - A'_t = M'_t - M_t,\mathscr{F}_t}{0 \leq t < \infty}
				\end{align}
				はLemmaの仮定を満たすマルチンゲールとなるから$[A]_{NAT} = [A']_{NAT}$が従う.
				
			\item[第二段]
				任意の区間$[0,a]$上で分解の存在を示せば
				$[0,\infty)$での分解が得られる.
				実際任意の$n \geq 1$に対し
				\begin{align}
					X_t = M^n_t + A^n_t, \quad (t \in [0,n])
				\end{align}
				と分解されるなら,$m > n$に対して
				\begin{align}
					M^n_t + A^n_t = X_t = M^m_t + A^m_t, \quad (t \in [0,n])
				\end{align}
				となり,Lemmaより或る$P$-零集合$E_{n,m}$が存在して,任意の$\omega \in \Omega \backslash E_{n,m}$で
				\begin{align}
					A^n_t(\omega) = A^m_t(\omega), \quad (\forall t \in [0,n])
				\end{align}
				が成立し,かつ$[0,n) \ni t \longmapsto A^n_t(\omega)$が右連続非減少となる.ここで
				\begin{align}
					E \coloneqq \bigcup_{\substack{n,m \in \N \\ n<m}} E_{n,m}
				\end{align}
				により$P$-零集合を定めれば,任意の$\omega \in \Omega \backslash E$及び$t \geq 0$に対して
				\begin{align}
					A^n_t(\omega) = A^m_t(\omega), \quad (\forall m > n > t)
				\end{align}
				となり$\lim_{n \to \infty} A^n_t(\omega)$が確定する.
				usual条件より$E \in \mathscr{F}_0$だから$A^n_t \defunc_{\Omega \backslash E}\ (n > t)$は
				$\mathscr{F}_t/\borel{\R}$-可測であり,
				\begin{align}
					A_t \coloneqq  \lim_{n \to \infty} A^n_t \defunc_{\Omega \backslash E},
					\quad (\forall t \geq 0)
				\end{align}
				で$A_t$を定めれば$A_t$は$\mathscr{F}_t/\borel{\R}$-可測となる.また
				任意の$n \geq 1$で
				\begin{align}
					A_t = A^n_t \defunc_{\Omega \backslash E}, \quad (\forall t \in [0,n))
				\end{align}
				が成り立つから$A_t$は可積分であり,$[0,\infty) \ni t \longmapsto A_t(\omega)$は右連続かつ非減少である.
				$\Set{\xi_t,\mathscr{F}_t}{0 \leq t < \infty}$を有界$RCLL$マルチンゲールとすれば
				任意の$t > 0$で
				\begin{align}
					E \int_{(0,t]} \xi_s\ dA_s = E \int_{(0,t]} \xi_s\ dA^n_s 
					= E \int_{(0,t]} \xi_{s-}\ dA^n_s = E \int_{(0,t]} \xi_{s-}\ dA_s,
					\quad (t < n)
				\end{align}
				が成立する.
				\begin{align}
					M \coloneqq X - A
				\end{align}
				とおけば$(M_t)_{t \geq 0}$は$(\mathscr{F}_t)$-適合かつ可積分であり,
				任意の$0 \leq s < t$及び$t < n$に対して
				\begin{align}
					M_t = X_t - A^n_t \defunc_{\Omega \backslash E} = M^n_t,
					\quad M_s = X_s - A^n_s \defunc_{\Omega \backslash E} = M^n_s,
					\quad \mbox{a.s. $P$}
				\end{align}
				となるから$\cexp{M_t}{\mathscr{F}_s} = M_s\ \mbox{a.s. $P$}$が満たされる.
				次段以降で$[0,a]$上で分解の存在を示す.
			
			\item[第三段]
				%\footnote{
				%	$X_\infty$が定義され
				%	$\Set{X_t,\mathscr{F}_t}{0 \leq t \leq \infty}$が劣マルチンゲールの場合に$a=\infty$とする.
				%}
				$\Set{Z_t,\mathscr{F}_t}{0 \leq t < \infty}$
				を$\Set{\cexp{X_a}{\mathscr{F}_t},\mathscr{F}_t}{0 \leq t < \infty}$の
				右連続な修正として(Theorem 3.13),
				\begin{align}
					Y_t \coloneqq X_t - Z_t,
					\quad (t \in [0,a])
				\end{align}
				により非正値の劣マルチンゲール$\Set{Y_t,\mathscr{F}_t}{0 \leq t \leq a}$を定め
				\begin{align}
					\Set{Y_{t^{(n)}_j},\mathscr{F}_{t^{(n)}_j}}{t^{(n)}_j = \frac{j}{2^n}a,\ j=0,1,\cdots,2^n},
					\quad n=1,2,\cdots,
					%\quad (\mbox{$a = \infty$の場合は$t^{(n)}_j = j/2^n$},\ j \in \N_0)
				\end{align}
				で離散化すれば,離散時のDoob分解 (P. \pageref{lem:Doob_decomposition})より
				\begin{align}
					&A^{(n)}_0 \coloneqq 0,
					\quad A^{(n)}_{t^{(n)}_j} \coloneqq \sum_{k=0}^{j-1} \cexp{Y_{t^{(n)}_{k+1}} - Y_{t^{(n)}_k}}{\mathscr{F}_{t^{(n)}_k}}; \\
					%\ A^{(n)}_\infty \coloneqq \sum_{k=0}^\infty \cexp{Y_{t^{(n)}_{k+1}} - Y_{t^{(n)}_k}}{\mathscr{F}_{t^{(n)}_k}}; \\
					&M^{(n)}_{t^{(n)}_j} \coloneqq Y_{t^{(n)}_j} - A^{(n)}_{t^{(n)}_j}
				\end{align}
				により可予測な増大過程$A^{(n)}$とマルチンゲール$M^{(n)}$に分解され,
				$Y_a = 0\ \mbox{a.s. $P$}$であるから
				\begin{align}
					Y_{t^{(n)}_j} = A^{(n)}_{t^{(n)}_j} +  M^{(n)}_{t^{(n)}_j}
					= A^{(n)}_{t^{(n)}_j} + \cexp{M^{(n)}_a}{\mathscr{F}_{t^{(n)}_j}}
					= A^{(n)}_{t^{(n)}_j} - \cexp{A^{(n)}_a}{\mathscr{F}_{t^{(n)}_j}},
					\quad \mbox{a.s. $P$},
					\quad j=0,1,\cdots,2^n
				\end{align}
				となる.

			\item[第四段]
				$(Y_T)_{T \in \mathscr{S}_a}$が一様可積分であることを示す.
				先ず任意の$T \in \mathscr{S}_a$に対し
				\begin{align}
					Z_T = \cexp{X_a}{\mathscr{F}_T},\quad \mbox{a.s. $P$}
					\label{eq:chapter_1_theorem_4_10_2}
				\end{align}
				が成立する.実際,任意抽出定理より
				\begin{align}
					\int_A Z_T\ dP = \int_A Z_a\ dP
					= \int_A X_a\ dP
					= \int_A \cexp{X_a}{\mathscr{F}_T}\ dP,
					\quad (\forall A \in \mathscr{F}_T)
				\end{align}
				が従い(\refeq{eq:chapter_1_theorem_4_10_2})が得られる.
				$\left(\cexp{X_a}{\mathscr{F}_T}\right)_{T \in \mathscr{S}_a}$は
				定理\ref{lem:uniformly_integrability_and_conditional_expectations}より一様可積分であるから
				$\left(Z_T\right)_{T \in \mathscr{S}_a}$も一様可積分であり,
				また$X$がクラス$DL$に属しているので$(Y_T)_{T \in \mathscr{S}_a}$の一様可積分性が従う.
				
			\item[第五段]
				$\left( A^{(n)}_a \right)_{n=1}^\infty$が一様可積分であることを示す.任意に$\lambda > 0$を取り
				\begin{align}
					T_\lambda^{(n)} \coloneqq
					a \wedge \min{}{\Set{t^{(n)}_{j-1}}{A^{(n)}_{t^{(n)}_j} > \lambda \mbox{ for some } j,\ 1 \leq j \leq 2^n}}
				\end{align}
				とおけば,$A^{(n)}$の可予測性より任意の$t \geq 0$で
				\begin{align}
					\left\{ T_\lambda^{(n)} \leq t \right\}
					= \bigcup_{j\, :\, t^{(n)}_{j-1} \leq t} \left\{ T_\lambda^{(n)} = t^{(n)}_{j-1} \right\}
					= \bigcup_{j\, :\, t^{(n)}_{j-1} \leq t}
						\left[ \bigcap_{k=1}^{j-1} \left\{ A^{(n)}_{t^{(n)}_k} \leq \lambda \right\} \right] 
						\cap \left\{ A^{(n)}_{t^{(n)}_j} > \lambda \right\}
					\in \mathscr{F}_t
				\end{align}
				が成り立つから$T_\lambda^{(n)} \in \mathscr{S}_a$が満たされ,また
				\begin{align}
					\mu < \lambda
					\quad \Longrightarrow \quad
					\left\{T^{(n)}_\lambda < a\right\} \subset \left\{T^{(n)}_\mu < a\right\}
					\label{eq:chapter_1_theorem_4_10_6}
				\end{align}
				及び
				\begin{align}
					T^{(n)}_\lambda(\omega) < a
					\quad \Longrightarrow \quad
					A^{(n)}_{T^{(n)}_\lambda}(\omega) \leq \lambda
					\label{eq:chapter_1_theorem_4_10_3}
				\end{align}
				も満たされる.
				\begin{align}
					N \coloneqq \bigcup_{k=1}^{2^n} \left\{ \cexp{Y_{t^{(n)}_k} - Y_{t^{(n)}_{k-1}}}{\mathscr{F}_{t^{(n)}_{k-1}}} < 0 \right\}
				\end{align}
				により$P$-零集合を定めれば,$\Omega \backslash N$の上で
				$A^{(n)}_0 \leq A^{(n)}_{t^{(n)}_1} \leq \cdots \leq A^{(n)}_a$となるから
				\begin{align}
					\left\{T^{(n)}_\lambda < a\right\} \cap (\Omega \backslash N)
					= \left\{A^{(n)}_a > \lambda\right\} \cap (\Omega \backslash N)
					\label{eq:chapter_1_theorem_4_10_1}
				\end{align}
				が従う.任意に$\Lambda \in \mathscr{F}_{T^{(n)}_\lambda}$を取れば,
				$\Lambda \cap \left\{T^{(n)}_\lambda=t^{(n)}_{j-1}\right\} \in \mathscr{F}_{t^{(n)}_{j-1}},
				\ (j=1,\cdots,2^n)$より
				\begin{align}
					\int_\Lambda Y_{T^{(n)}_\lambda}\ dP = 
					\sum_{j=1}^{2^n} \int_{\Lambda \cap \left\{T^{(n)}_\lambda=t^{(n)}_{j-1}\right\}} Y_{t^{(n)}_{j-1}}\ dP
					&= \sum_{j=1}^{2^n} \int_{\Lambda \cap \left\{T^{(n)}_\lambda=t^{(n)}_{j-1}\right\}} 
						A^{(n)}_{t^{(n)}_{j-1}} - \cexp{A^{(n)}_a}{\mathscr{F}_{t^{(n)}_{j-1}}}\ dP \\
					&= \sum_{j=1}^{2^n} \int_{\Lambda \cap \left\{T^{(n)}_\lambda=t^{(n)}_{j-1}\right\}} 
						A^{(n)}_{T^{(n)}_\lambda} - A^{(n)}_a\ dP \\
					&= \int_\Lambda A^{(n)}_{T^{(n)}_\lambda} - A^{(n)}_a\ dP
					\label{eq:chapter_1_theorem_4_10_5}
				\end{align}
				が成立するから,(\refeq{eq:chapter_1_theorem_4_10_3})と(\refeq{eq:chapter_1_theorem_4_10_1})と併せて
				\begin{align}
					\int_{\left\{A^{(n)}_a > \lambda\right\}} A^{(n)}_a\ dP
					= \int_{\left\{T^{(n)}_\lambda < a\right\}} A^{(n)}_{T^{(n)}_\lambda}\ dP
						- \int_{\left\{T^{(n)}_\lambda < a\right\}} Y_{T^{(n)}_\lambda}\ dP
					\leq \lambda P\left(T^{(n)}_\lambda < a\right) 
						- \int_{\left\{T^{(n)}_\lambda < a\right\}} Y_{T^{(n)}_\lambda}\ dP
				\end{align}
				となる.一方で(\refeq{eq:chapter_1_theorem_4_10_6}),(\refeq{eq:chapter_1_theorem_4_10_3}),
				(\refeq{eq:chapter_1_theorem_4_10_1}),(\refeq{eq:chapter_1_theorem_4_10_5})より
				\begin{align}
					\int_{\left\{T^{(n)}_{\lambda/2} < a\right\}} Y_{T^{(n)}_{\lambda/2}}\ dP
					&= \int_{\left\{T^{(n)}_{\lambda/2} < a\right\}} A^{(n)}_{T^{(n)}_{\lambda/2}} - A^{(n)}_a\ dP \\
					&\leq \int_{\left\{T^{(n)}_{\lambda} < a\right\}} A^{(n)}_{T^{(n)}_{\lambda/2}} - A^{(n)}_a\ dP \\
					&\leq -\frac{\lambda}{2} P\left(T^{(n)}_{\lambda} < a\right)
				\end{align}
				が成立するから
				\begin{align}
					\int_{\left\{A^{(n)}_a > \lambda\right\}} A^{(n)}_a\ dP
					\leq -2 \int_{\left\{T^{(n)}_{\lambda/2} < a\right\}} Y_{T^{(n)}_{\lambda/2}}\ dP
						- \int_{\left\{T^{(n)}_\lambda < a\right\}} Y_{T^{(n)}_\lambda}\ dP
				\end{align}
				となる.ここで
				\begin{align}
					P\left(T^{(n)}_{\lambda} < a\right)
					= P\left(A^{(n)}_a > \lambda\right)
					\leq \frac{E A^{(n)}_a}{\lambda}
					= \frac{- E M^{(n)}_a}{\lambda}
					= \frac{- E M^{(n)}_0}{\lambda}
					= \frac{- E Y_0}{\lambda}
				\end{align}
				より$P\left(T^{(n)}_{\lambda} < a\right)$は$\lambda$のみに依存して
				0に収束し,定理\ref{thm:appendix_uniform_integrability_equivalence}と$(Y_T)_{T \in \mathscr{S}_a}$の
				一様可積分性により
				\begin{align}
					\sup{n \in \N}{\int_{\left\{A^{(n)}_a > \lambda\right\}} A^{(n)}_a\ dP}
					\leq 2 \sup{n \in \N}{\int_{\left\{T^{(n)}_{\lambda/2} < a\right\}} \left|Y_{T^{(n)}_{\lambda/2}}\right|\ dP}
					+ \sup{n \in \N}{\int_{\left\{T^{(n)}_{\lambda} < a\right\}} \left|Y_{T^{(n)}_{\lambda}}\right|\ dP}
					\longrightarrow 0
					\quad (\lambda \longrightarrow \infty)
				\end{align}
				が従い$\left( A^{(n)}_a \right)_{n=1}^\infty$が一様可積分性が出る.
				
			\item[第六段]
				Dunford-Pettisの定理より$\left( A^{(n)}_a \right)_{n=1}^\infty$の或る部分列
				$\left( A^{(n_k)}_a \right)_{k=1}^\infty$は$L^1(P)$で弱収束する.つまり
				或る$A_a \in L^1(P)$が存在して
				任意の$\xi \in L^\infty(P)$に対し
				\begin{align}
					E \left( \xi A^{(n_k)}_a \right) \longrightarrow E (\xi A_a)
					\quad (k \longrightarrow \infty)
				\end{align}
				が成立する.
				\begin{align}
					\Pi_n \coloneqq \Set{t^{(n)}_j}{t^{(n)}_j = \frac{j}{2^n}a,\ j=0,1,\cdots,2^n},
					\quad \Pi \coloneqq \bigcup_{n=1}^\infty \Pi_n
				\end{align}
				とすれば,任意の$t \in \Pi$に対し或る$K \geq 1$が存在して
				$t \in \Pi_{n_k}\ (\forall k > K)$となり,Problem 4.11より
				\begin{align}
					E \left( \xi A^{(n_k)}_t \right)
					= E \xi\left\{ Y_t + \cexp{A^{(n_k)}_a}{\mathscr{F}_t} \right\}
					\longrightarrow E \xi\left\{ Y_t + \cexp{A_a}{\mathscr{F}_t} \right\}
					\quad (k > K,\ k \longrightarrow \infty)
					\label{eq:chapter_1_theorem_4_10_7}
				\end{align}
				が成り立つから$A^{(n_k)}_t$は$Y_t + \cexp{A_a}{\mathscr{F}_t}$に弱収束する.
				ここで
				\begin{align}
					\tilde{A}_t \coloneqq Y_t + \cexp{A_a}{\mathscr{F}_t},
					\quad (t \in [0,a])
				\end{align}
				と定めれば$\Set{\tilde{A}_t,\mathscr{F}_t}{0 \leq t \leq a}$は
				劣マルチンゲールとなり,$\Set{X_t,\mathscr{F}_t}{0 \leq t <\infty}$の右連続性より
				\begin{align}
					[0,a] \ni t \longmapsto E\left[ Y_t + \cexp{A_a}{\mathscr{F}_t} \right]
					= E X_t - E X_a + E A_a
				\end{align}
				は右連続であるから(Theorem 3.13),$\tilde{A}$の右連続な修正$\Set{A_t,\mathscr{F}_t}{0 \leq t \leq a}$
				が得られる.
			
			\item[第七段]
				$t \longmapsto A_t(\omega)$がa.s.に0出発かつ非減少であることを示す.
				実際,$\xi = \sgn{A_0}$として,(\refeq{eq:chapter_1_theorem_4_10_7})より
				\begin{align}
					E |A_0| = E \xi A_0 = E \xi \tilde{A}_0 = \lim_{k \to \infty} E \xi A^{(n_k)}_0 = 0
				\end{align}
				が成り立つから$A_0 = 0\ \mbox{a.s. $P$}$が従う.また任意に$s,t \in \Pi,\ (s<t)$を取れば
				或る$K \geq 1$が存在して$s,t \in \Pi_{n_k}\ (\forall k > K)$が満たされ,
				$A^{(n_k)}$は増大過程であるから$\xi = \defunc_{\{A_s > A_t\}}$として
				\begin{align}
					E \xi (A_t - A_s) = E \xi \left( \tilde{A}_t - \tilde{A}_s \right)
					= \lim_{k \to \infty} E \xi \left( A^{(n_k)}_t - A^{(n_k)}_s \right) \geq 0 
				\end{align}
				となり$P(A_s > A_t) = 0$が成り立つ.$t \longmapsto A_t$が右連続性であるから,$P$-零集合を
				\begin{align}
					N \coloneqq \Biggl(\bigcup_{\substack{s,t \in \Pi \\ s < t}} \{A_s > A_t\}\Biggr) \cup \{A_0 \neq 0\}
				\end{align}
				で定めれば$\Omega \backslash N$上で$t \longmapsto A_t$は0出発非減少となり,
				$N$上で$A \equiv 0$と修正すれば$A$は増大過程となる.
				
			\item[第八段]
				$A$がナチュラルであることを示す.$\xi = \Set{\xi_t,\mathscr{F}_t}{0 \leq t \leq a}$を有界な$RCLL$マルチンゲールとすれば
				\begin{align}
					E \xi_a A^{(n_k)}_a 
					&= E\left[ \sum_{j=1}^{2^{n_k}}\xi_{t^{(n_k)}_{j-1}} \left( A^{(n_k)}_{t^{(n_k)}_j} - A^{(n_k)}_{t^{(n_k)}_{j-1}} \right) \right] \\
					&= E\left[ \sum_{j=1}^{2^{n_k}}\xi_{t^{(n_k)}_{j-1}} \left( Y_{t^{(n_k)}_j} - Y_{t^{(n_k)}_{j-1}} \right) \right]
						+ E\left[ \sum_{j=1}^{2^{n_k}}\xi_{t^{(n_k)}_{j-1}} \left( \cexp{A^{(n_k)}_a}{\mathscr{F}_{t^{(n_k)}_j}} - \cexp{A^{(n_k)}_a}{\mathscr{F}_{t^{(n_k)}_{j-1}}} \right) \right] \\
					&= E\left[ \sum_{j=1}^{2^{n_k}}\xi_{t^{(n_k)}_{j-1}} \left( A_{t^{(n_k)}_j} - A_{t^{(n_k)}_{j-1}} \right) \right]
				\end{align}
				が任意の$k \geq 1$で成り立ち(Proposition 4.3),$k \longrightarrow \infty$として
				\begin{align}
					E \xi_a A_a = E \int_{(0,a]} \xi_{s-}\ dA_s
				\end{align}
				が得られる.任意の$t \in (0,a]$に対し
				$\xi^{(t)} = \Set{\xi^{(t)}_s \coloneqq \xi_{t \wedge s},\mathscr{F}_s}{0 \leq s \leq a}$
				も$RCLL$マルチンゲールであり
				\begin{align}
					\xi^{(t)}_{s-} &= \xi_{s-},\quad (\forall s \in (0,t]), \\
					\xi^{(t)}_{s-} &= \xi_t, \quad (\forall s \in (t,a])
				\end{align}
				より
				\begin{align}
					E \xi_t A_t + E \xi_t(A_a - A_t) = E \xi^{(t)}_a A_a 
					= E \int_{(0,a]} \xi^{(t)}_{s-}\ dA_s
					= E \int_{(0,t]} \xi_{s-}\ dA_s + E \xi_t (A_a - A_t)
				\end{align}
				となり
				\begin{align}
					E \xi_t A_t = E \int_{(0,t]} \xi_{s-}\ dA_s,
					\quad (\forall t \in (0,a])
				\end{align}
				が成立する.よって$A$はナチュラルである.
				
			\item[第九段]
				$\Set{M_t \coloneqq X_t - A_t, \mathscr{F}_t}{0 \leq t \leq a}$がマルチンゲールであることを示す.
				$M$の適合性と可積分性は$X,A$のそれより従い,また任意に
				$0 \leq s \leq t \leq a$を取れば,任意の$A \in \mathscr{F}_s$で
				\begin{align}
					\int_A M_s\ dP = \int_A X_s - A_s\ dP
					&= \int_A X_s - \left(Y_s - \cexp{A_a}{\mathscr{F}_s}\right)\ dP \\
					&= \int_A X_s - \left(X_s - Z_s - \cexp{A_a}{\mathscr{F}_s}\right)\ dP \\
					&= \int_A Z_t + \cexp{A_a}{\mathscr{F}_t}\ dP \\
					&= \int_A X_t - \left(X_t - Z_t - \cexp{A_a}{\mathscr{F}_t}\right)\ dP \\
					&= \int_A M_t\ dP
				\end{align}
				が成立する.
				\QED
		\end{description}
	\end{prf}
	
	\begin{itembox}[l]{Problem 4.13}
		Verify that a continuous, nonnegative submartingale is regular. 
	\end{itembox}
	
	\begin{prf}
		Problem 4.9 より$(X_{T_n})_{n=1}^\infty$は一様可積分であり,またパスの連続性より
		$X_{T_n} \longrightarrow X_T\ (n \longrightarrow \infty)$
		となるから,定理\ref{lem:uniformly_integrable_and_convergence_in_mean}より
		$\lim_{n \to \infty} EX_{T_n} = EX_T$が成立する.
		\QED
	\end{prf}
	
	\begin{itembox}[l]{Theorem 4.14 修正}
		Suppose that $X = \Set{X_t}{0 \leq t < \infty}$ is a right-continuous submartingale
		of class $DL$ with respect to the filtration $\{\mathscr{F}_t\}$, which
		satisfies the usual conitions, and 
		\textcolor{red}{let $[A]_{NAT}$ be of the Doob-Meyer decomposition of $X$.
		There exists a continuous version of $A$ in $[A]_{NAT}$ if and only if $X$ is regular.}
	\end{itembox}
	
	\begin{prf}\mbox{}
		\begin{description}
			\item[第一段] $A$が連続であるとき,
				増大列$\{T_n\}_{n=1}^\infty \subset \mathscr{S}_a$と
				$T \coloneqq \lim_{n \to \infty} \in T_n \mathscr{S}_a$に対し
				単調収束定理より
				\begin{align}
					\lim_{n \to \infty} EA_{T_n}
					= E \lim_{n \to \infty} A_{T_n}
					= EA_T
				\end{align}
				が成立する.また任意抽出定理より
				\begin{align}
					E(X_{T_n} - A_{T_n}) = E(X_{T} - A_{T}),
					\quad (\forall n \geq 1)
				\end{align}
				となるから
				\begin{align}
					\lim_{n \to \infty} EX_{T_n} 
					= \lim_{n \to \infty} E(X_{T_n} - A_{T_n}) + \lim_{n \to \infty} EA_{T_n} 
					= E(X_{T} - A_{T}) + EA_T
					= EX_T
				\end{align}
				が従う.
				
			\item[第二段]
				以降$X$がレギュラーであるとする.このとき任意の有界な停止時刻の増大列
				$(T_n)$と$T \coloneqq \lim T_n$に対し,$X-A$のマルチンゲール性と任意抽出定理,
				及び$X$のレギュラリティより
				\begin{align}
					EA_{T_n} &= EX_{T_n} - E(X_{T_n} - A_{T_n})
					= EX_{T_n} - E(X_T - A_T) \\
					&\qquad \longrightarrow EX_T - E(X_T - A_T)
					= EA_T
					\quad (n \longrightarrow \infty)
					\label{eq:chapter_1_theorem_4_14_1}
				\end{align}
				が得られる.いま,任意に$a \in \N$を取り
				\begin{align}
					\Pi_n \coloneqq 
					\Set{t^{(n)}_j}{t^{(n)}_j = \frac{j}{2^n}a,\ j=0,1,\cdots,2^n},
					\quad \Pi \coloneqq \bigcup_{n=1}^\infty \Pi_n
				\end{align}
				とおく.また任意に$\lambda \in \N$を取り,各$j = 0,1,\cdots,2^n$に対し
				\begin{align}
					Y^{(n),j}_t \coloneqq
					\cexp{\lambda \wedge A_{t^{(n)}_{j+1}}}{\mathscr{F}_t},
					\quad (\forall t \geq 0)
				\end{align}
				によりマルチンゲール$\Set{Y^{(n),j}_t,\mathscr{F}_t}{0 \leq t < \infty}$を定めれば,
				\begin{align}
					[0,\infty) \ni t \longmapsto EY^{(n),j}_t 
					= E\left(\lambda \wedge A_{t^{(n)}_{j+1}}\right)
				\end{align}
				と Theorem 3.13 より$RCLL$な修正$\tilde{Y}^{(n),j}$が存在する.このとき
				各$t \geq 0$で
				\begin{align}
					\int_A \tilde{Y}^{(n),j}_t\ dP 
					= \int_A \lambda \wedge A_{t^{(n)}_{j+1}}\ dP
					\leq \lambda P(A),
					\quad (\forall A \in \mathscr{F}_t)
				\end{align}
				となり,一方で各$t \in \left[t^{(n)}_j, t^{(n)}_{j+1} \right)$で
				\begin{align}
					\int_A \tilde{Y}^{(n),j}_t\ dP
					= \int_A \lambda \wedge A_{t^{(n)}_{j+1}}\ dP
					\geq \int_A \lambda \wedge A_t\ dP,
					\quad (\forall A \in \mathscr{F}_t) 
				\end{align}
				となるから,各$j$で
				\begin{align}
					E_j &\coloneqq \Set{\tilde{Y}^{(n),j}_t > \lambda}{\exists t \geq 0} 
						\cup \Set{\tilde{Y}^{(n),j}_t < \lambda \wedge A_t}{\exists t \in \left[t^{(n)}_j, t^{(n)}_{j+1} \right)} \\
					&= \left[ \bigcup_{r \in [0,\infty)\cap\Q}\left\{\tilde{Y}^{(n),j}_r > \lambda\right\} \right]
					\bigcup \left[ \bigcup_{r \in \left[t^{(n)}_j, t^{(n)}_{j+1} \right)\cap\Q}\left\{\tilde{Y}^{(n),j}_r < \lambda \wedge A_r\right\} \right]
				\end{align}
				とおけば$P$-零集合$E \coloneqq \bigcup_{j=0}^{2^n} E_j$が定まる.usual条件より$E \in \mathscr{F}_0$であるから
				\begin{align}
					\Set{Z^{(n),j}_t \coloneqq \tilde{Y}^{(n),j}_t \defunc_{\Omega \backslash E},
					\mathscr{F}_t}{0 \leq t < \infty}
				\end{align}
				で定める$Y^{(n),j}$のバージョン$Z^{(n),j}$は
				\begin{align}
					\omega \in \Omega \backslash E
					\quad \Longrightarrow \quad
					\begin{cases}
						Z^{(n),j}_t(\omega) \leq \lambda, & \forall t \geq 0, \\
						Z^{(n),j}_t(\omega) \geq \lambda \wedge A_t(\omega), & \forall t \in \left[t^{(n)}_j, t^{(n)}_{j+1} \right)
					\end{cases}
				\end{align}
				を満たす$RCLL$かつ有界なマルチンゲールとなり,
				\begin{align}
					\eta^{(n)}_t \coloneqq
					\sum_{j=0}^{2^n-1} Z^{(n),j}_t \defunc_{\left[t^{(n)}_j,t^{(n)}_{j+1}\right)}(t)
						+ (\lambda \wedge A_a) \defunc_{[a,\infty)}(t),
					\quad (t \geq 0)
				\end{align}
				とおけば
				\begin{align}
					\omega \in \Omega \backslash E
					\quad \Longrightarrow \quad
					\begin{cases}
						\eta^{(n)}_t(\omega) \leq \lambda, & (\forall t \geq 0), \\
						\eta^{(n)}_t(\omega) \geq \lambda \wedge A_t(\omega), & (\forall t \in [0,a])
					\end{cases}
					\label{eq:chapter_1_theorem_4_14_4}
				\end{align}
				が成り立つ.また$\eta^{(n)}$の右連続性,Corollary2.4,Problem2.5 及びusual条件より
				\begin{align}
					T^{(n)}_\epsilon \coloneqq
					a \wedge \inf{}{\Set{t \geq 0}{\eta^{(n)}_t - (\lambda \wedge A_t)  > \epsilon}}
				\end{align}
				は$\mathscr{S}_a$に属する停止時刻となり,このとき
				\begin{align}
					\varphi_n(t) \coloneqq 
					\begin{cases}
						t^{(n)}_{j+1}, & t^{(n)}_j \leq t < t^{(n)}_{j+1},\ j=0,1,\cdots,2^n-1 \\
						a, & t = a
					\end{cases}
				\end{align}
				を用いれば,任意抽出定理より
				\begin{align}
					E\left( \eta_{T^{(n)}_\epsilon} \right)
					&= \sum_{j=0}^{2^n-1} \int_{\left\{t^{(n)}_j \leq T^{(n)}_\epsilon < t^{(n)}_{j+1}\right\}} Z^{(n),j}_{T^{(n)}_\epsilon}\ dP
						+ \int_{\left\{T^{(n)}_\epsilon = a\right\}} \lambda \wedge A_a\ dP \\
					&= \sum_{j=0}^{2^n-1} \int_{\left\{t^{(n)}_j \leq T^{(n)}_\epsilon < t^{(n)}_{j+1}\right\}} \cexp{Z^{(n),j}_{t^{(n)}_{j+1}}}{\mathscr{F}_{T^{(n)}_\epsilon}}\ dP
						+ \int_{\left\{T^{(n)}_\epsilon = a\right\}} \lambda \wedge A_a\ dP \\
					&= \sum_{j=0}^{2^n-1} \int_{\left\{t^{(n)}_j \leq T^{(n)}_\epsilon < t^{(n)}_{j+1}\right\}} Z^{(n),j}_{t^{(n)}_{j+1}}\ dP
						+ \int_{\left\{T^{(n)}_\epsilon = a\right\}} \lambda \wedge A_a\ dP \\
					&= \sum_{j=0}^{2^n-1} \int_{\left\{t^{(n)}_j \leq T^{(n)}_\epsilon < t^{(n)}_{j+1}\right\}} \lambda \wedge A_{t^{(n)}_{j+1}}\ dP
						+ \int_{\left\{T^{(n)}_\epsilon = a\right\}} \lambda \wedge A_a\ dP \\
					&= \sum_{j=0}^{2^n-1} \int_{\left\{t^{(n)}_j \leq T^{(n)}_\epsilon < t^{(n)}_{j+1}\right\}} \lambda \wedge A_{\varphi_n\left(T^{(n)}_\epsilon\right)}\ dP
						+ \int_{\left\{T^{(n)}_\epsilon = a\right\}} \lambda \wedge A_{\varphi_n\left(T^{(n)}_\epsilon\right)}\ dP \\
					&= E\left(\lambda \wedge A_{\varphi_n\left(T^{(n)}_\epsilon\right)}\right)
				\end{align}
				が従う.また$t \longmapsto \eta^{(n)}_t - (\lambda \wedge A_t)$の右連続性より
				\begin{align}
					T^{(n)}_\epsilon(\omega) < a
					\quad \Longrightarrow
					\quad \eta^{(n)}_{T^{(n)}_\epsilon}(\omega) - \left(\lambda \wedge A_{T^{(n)}_\epsilon}(\omega)\right)
						\geq \epsilon
				\end{align}
				となるから
				\begin{align}
					E\left(\lambda \wedge A_{\varphi_n\left(T^{(n)}_\epsilon\right)}
						- \lambda \wedge A_{T^{(n)}_\epsilon} \right)
					&= E\left(\eta^{(n)}_{T^{(n)}_\epsilon}
						- \lambda \wedge A_{T^{(n)}_\epsilon} \right) \\
					&= E\defunc_{\left\{T^{(n)}_\epsilon < a\right\}}\left(\eta^{(n)}_{T^{(n)}_\epsilon}
						- \lambda \wedge A_{T^{(n)}_\epsilon} \right)
					\geq \epsilon P\left(T^{(n)}_\epsilon < a\right)
					\label{eq:chapter_1_theorem_4_14_5}
				\end{align}
				が成立する.
				
			\item[第三段]
				$\left( \eta^{(n)} \right)_{n=1}^\infty$は$n$に関して
				$P$-a.s. に減少していく.実際,任意の$t \in [0,a)$に対し
				\begin{align}
					t \in \left[t^{(n)}_j, t^{(n)}_{j+1}\right)
				\end{align}
				を満たす$0 \leq j \leq 2^n-1$を取れば
				$t \in \left[t^{(n+1)}_{2j}, t^{(n+1)}_{2j+1}\right)$或は
				$t \in \left[t^{(n+1)}_{2j+1}, t^{(n+1)}_{2j+2}\right)$となるから,
				任意の$A \in \mathscr{F}_t$で
				\begin{align}
					\int_A \eta^{(n)}_t\ dP
					= \int_A \lambda \wedge A_{t^{(n)}_{j+1}}\ dP
					\begin{cases}
						\displaystyle= \int_A \lambda \wedge A_{t^{(n+1)}_{2j+2}}\ dP \\
						\displaystyle\geq \int_A \lambda \wedge A_{t^{(n+1)}_{2j+1}}\ dP
					\end{cases}
					= \int_A \eta^{(n+1)}_t\ dP
				\end{align}
				が成り立ち$\eta^{(n)}_t \geq \eta^{(n+1)}_t,\ \mbox{a.s. $P$}$が従う.
				$\eta^{(n)},\eta^{(n+1)}$のパスは右連続であるから
				\begin{align}
					F_n \coloneqq \Set{\eta^{(n)}_t < \eta^{(n+1)}_t}{\exists t \in [0,a)}
					= \bigcup_{r \in [0,a) \cap \Q} \left\{\eta^{(n)}_r < \eta^{(n+1)}_r\right\}
				\end{align}
				で$P$-零集合が定まり,$F \coloneqq \bigcup_{n=1}^\infty F_n$とおけば
				任意の$\omega \in \Omega \backslash F$と$t \in [0,a]$で
				$\left( \eta^{(n)}_t(\omega) \right)_{n=1}^\infty$は減少し
				\begin{align}
					T^{(1)}_\epsilon \defunc_{\Omega \backslash F} 
					\leq T^{(2)}_\epsilon \defunc_{\Omega \backslash F} \leq \cdots \leq a
					\label{eq:chapter_1_theorem_4_14_2}
				\end{align}
				となる.usual条件より$F \in \mathscr{F}_0$であるから
				\begin{align}
					\left\{ T^{(n)}_\epsilon \defunc_{\Omega \backslash F} \leq t \right\}
					= \left\{ T^{(n)}_\epsilon \leq t \right\} \cap (\Omega \backslash F) + F
					\in \mathscr{F}_t,\quad (\forall t \geq 0)
				\end{align}
				が成り立つので$T^{(n)}_\epsilon \defunc_{\Omega \backslash F} \in \mathscr{S}_a$となり,
				単調増大性より
				\begin{align}
					T_\epsilon \coloneqq \lim_{n \to \infty} T^{(n)}_\epsilon \defunc_{\Omega \backslash F}
				\end{align}
				と定めれば$T_\epsilon \in \mathscr{S}_a$も満たされる.
				一方$\varphi_n\left(T^{(n)}_\epsilon\right)$についても
				\begin{align}
					\left\{\varphi_n\left(T^{(n)}_\epsilon\right) \leq t\right\}
					= \bigcup_{j\, :\, t^{(n)}_{j+1} \leq t} \left\{t^{(n)}_j \leq T^{(n)}_\epsilon < t^{(n)}_{j+1}\right\}
					\in \mathscr{F}_t,
					\quad (\forall t \geq 0)
				\end{align}
				より$\varphi_n\left(T^{(n)}_\epsilon\right) \in \mathscr{S}_a$が従い,
				また$\varphi_n(t) \geq t$と$t \longmapsto \varphi_n(t)$の増大性より
				\begin{align}
					T^{(n)}_\epsilon(\omega)
					\leq \varphi_n\left(T^{(n)}_\epsilon(\omega)\right)
					\leq \varphi_n\left(T_\epsilon(\omega)\right),
					\quad (\forall \omega \in \Omega \backslash F)
				\end{align}
				が成立し,$A$のパスの増大性と併せて
				\begin{align}
					E \left( \lambda \wedge A_{T^{(n)}_\epsilon} \right)
					\leq E \left( \lambda \wedge A_{\varphi_n\left(T^{(n)}_\epsilon\right)} \right)
					\leq E \left( \lambda \wedge A_{\varphi_n\left(T_\epsilon\right)} \right)
				\end{align}
				が満たされる.このとき(\refeq{eq:chapter_1_theorem_4_14_1})より
				\begin{align}
					\lim_{n \to \infty} E \left( \lambda \wedge A_{T^{(n)}_\epsilon} \right)
					= E \left( \lambda \wedge A_{T_\epsilon} \right)
				\end{align}
				が成り立ち,右辺も$\varphi_n(t) \downarrow t$と$A$のパスの右連続性
				及びLebesgueの収束定理より$E \left( \lambda \wedge A_{T_\epsilon} \right)$に収束するから
				\begin{align}
					\lim_{n \to \infty} E \left( \lambda \wedge A_{\varphi_n\left(T^{(n)}_\epsilon\right)} \right) 
					= E \left( \lambda \wedge A_{T_\epsilon} \right)
				\end{align}
				が得られる.
			
			\item[第五段]
				任意の$\omega \in \Omega$と$n \geq 1$に対し
				\begin{align}
					T^{(n)}_\epsilon(\omega) < a
					\quad \Longleftrightarrow \quad
					\sup{0 \leq t \leq a}{\left\{(\lambda \wedge A_t(\omega)) - \eta^{(n)}_t(\omega)\right\}} > \epsilon
				\end{align}
				が満たされ,また(\refeq{eq:chapter_1_theorem_4_14_4})より
				$\Omega \backslash E$の上で$\eta^{(n)}_t - (\lambda \wedge A_t) \geq 0,\ (\forall t \in [0,a])$だから,
				(\refeq{eq:chapter_1_theorem_4_14_5})と前段の結果と併せて
				\begin{align}
					&P\left(\sup{0 \leq t \leq a}{\left|\eta^{(n)}_t - (\lambda \wedge A_t)\right|} > \epsilon\right)
					= P\left(T^{(n)}_\epsilon < a\right) \\
					&\qquad \leq \frac{1}{\epsilon} E\left(\lambda \wedge A_{\varphi_n\left(T^{(n)}_\epsilon\right)} 
						- \lambda \wedge A_{T^{(n)}_\epsilon} \right)
					\longrightarrow \frac{1}{\epsilon} E\left(\lambda \wedge A_{T_\epsilon}
						- \lambda \wedge A_{T_\epsilon} \right) = 0 \quad (n \longrightarrow \infty)
				\end{align}
				が得られる.従って定理\ref{thm:convergence_in_measure_then_convergence_almost_everywhere}より
				或る部分列$(n_k)_{k=1}^\infty$と$P$-零集合$G$が存在して
				\begin{align}
					\sup{0 \leq t \leq a}{\left|\eta^{(n_k)}_t(\omega) - (\lambda \wedge A_t(\omega))\right|}
					\longrightarrow 0
					\quad (k \longrightarrow \infty),
					\quad (\forall \omega \in \Omega \backslash G)
					\label{eq:chapter_1_theorem_4_14_6}
				\end{align}
				が成立する.
				
			\item[第六段]
				$A$はナチュラルであり,$Z^{(n),j}$は有界かつ$RCLL$なマルチンゲールであるから
				\begin{align}
					E\int_{\left(t^{(n)}_j,t^{(n)}_{j+1}\right]} Z^{(n),j}_s\ dA_s
					&= E\int_{\left(0,t^{(n)}_{j+1}\right]} Z^{(n),j}_s\ dA_s
						- E\int_{\left(0,t^{(n)}_j\right]} Z^{(n),j}_s\ dA_s \\
					&= E\int_{\left(0,t^{(n)}_{j+1}\right]} Z^{(n),j}_{s-}\ dA_s
						- E\int_{\left(0,t^{(n)}_j\right]} Z^{(n),j}_{s-}\ dA_s \\
					&= E\int_{\left(t^{(n)}_j,t^{(n)}_{j+1}\right]} Z^{(n),j}_{s-}\ dA_s
				\end{align}
				が成立する.従って
				\begin{align}
					\xi^{(n)}_t \coloneqq
					\sum_{j=0}^{2^n-1} Z^{(n),j}_t \defunc_{\left(t^{(n)}_j,t^{(n)}_{j+1}\right]}(t),
					\quad (t \geq 0)
				\end{align}
				とおけば任意の$t \in (0,a]$で$\xi^{(n)}_{t-}$が存在し
				\begin{align}
					E\int_{(0,a]} \xi^{(n)}_s\ dA_s
					= \sum_{j=0}^{2^n-1} E\int_{\left(t^{(n)}_j,t^{(n)}_{j+1}\right]} Z^{(n),j}_s\ dA_s
					= \sum_{j=0}^{2^n-1} E\int_{\left(t^{(n)}_j,t^{(n)}_{j+1}\right]} Z^{(n),j}_{s-}\ dA_s
					= E\int_{(0,a]} \xi^{(n)}_{s-}\ dA_s
				\end{align}
				が成立する.一方で$t \notin \Pi$で$\xi^{(n)}_t = \eta^{(n)}_t,\ (\forall n \geq 1)$
				であるから(\refeq{eq:chapter_1_theorem_4_14_6})より
				\begin{align}
					\sup{t \in (0,a]\backslash\Pi}{\left|\xi^{(n_k)}_t(\omega) - \lambda \wedge A_t(\omega)\right|}
					\longrightarrow 0 \quad (k \longrightarrow \infty),
					\quad (\forall \omega \in \Omega \backslash G)
				\end{align}
				が従い,これにより
				\begin{align}
					\sup{t \in (0,a]}{\left|\xi^{(n_k)}_{t-}(\omega) - \lambda \wedge A_{t-}(\omega)\right|}
					\longrightarrow 0 \quad (k \longrightarrow \infty),
					\quad (\forall \omega \in \Omega \backslash G)
				\end{align}
				も出る.実際,$\omega \in \Omega \backslash G$を固定すれば,
				任意の$\epsilon > 0$に対し或る$K = K(\omega,\epsilon) \geq 1$が存在して
				\begin{align}
					\sup{t \in (0,a]\backslash\Pi}{\left|\xi^{(n_k)}_t(\omega) - \lambda \wedge A_t(\omega)\right|} 
					< \epsilon,\quad (\forall k \geq K)
				\end{align}
				となり,このとき任意の$t \in (0,a]$と$k \geq K$で
				\begin{align}
					&\left|\xi^{(n_k)}_{t-}(\omega) - \lambda \wedge A_{t-}(\omega)\right| \\
					&\quad \leq \left|\xi^{(n_k)}_{t-}(\omega) - \xi^{(n_k)}_{s}(\omega)\right|
						+ \left|\xi^{(n_k)}_{s}(\omega) - \lambda \wedge A_{s}(\omega)\right|
						+ \left|\lambda \wedge A_{s}(\omega) - \lambda \wedge A_{t-}(\omega)\right| \\
					&\quad < \epsilon
				\end{align}
				を満たす$s = s(t,k) \in (0,a]\backslash\Pi,\ (s < t)$が取れるから
				\begin{align}
					\sup{t \in (0,a]}{\left|\xi^{(n_k)}_{t-}(\omega) - \lambda \wedge A_{t-}(\omega)\right|} 
					\leq \epsilon,\quad (\forall k \geq K)
				\end{align}
				が成立する.$t \in \Pi$なら或る$N = N(t)$で$t \in \Pi_N$となるから
				$\xi^{(n)}_t = \lambda \wedge A_t,\ \mbox{$P$-a.s.},\ (\forall n \geq N)$となり
				\begin{align}
					H_t \coloneqq \bigcup_{n \geq N} \left\{\xi^{(n)}_t \neq \lambda \wedge A_t\right\},
					\quad H \coloneqq \bigcup_{t \in \Pi} H_t
				\end{align}
				により$P$-零集合$H$を定めれば任意の$t \in [0,a]$で
				\begin{align}
					\lim_{k \to \infty} \xi^{(n_k)}_t(\omega) = \lambda \wedge A_t(\omega),
					\quad (\forall \omega \in \Omega \backslash (G \cup H))
				\end{align}
				となる.Lebesgueの収束定理より
				\begin{align}
					E\int_{(0,a]} \lambda \wedge A_t\ dA_t
					= E\int_{(0,a]} \lambda \wedge A_{t-}\ dA_t
				\end{align}
				が得られ,$A$の単調非減少性より$A_{t-} \leq A_t$であるから
				或る$P$-零集合$U_a$が存在し,任意の$\omega \in \Omega \backslash U_a$で
				\begin{align}
					\int_{(0,a]} (\lambda \wedge A_t(\omega)) 
					- (\lambda \wedge A_{t-}(\omega))\ dA_t(\omega) = 0
				\end{align}
				が成立し$(0,a] \ni t \longmapsto \lambda \wedge A_t(\omega)$
				の連続性が出る.$a$の任意性より
				$V_\lambda \coloneqq \bigcup_{a=1}^\infty U_a$とおけば
				\begin{align}
					(0,\infty) \ni t \longmapsto \lambda \wedge A_t(\omega),
					 \quad (\forall \omega \in \Omega \backslash V_\lambda)
				\end{align}
				は連続となり,$\lambda$も任意であるから
				$V \coloneqq \bigcup_{\lambda=1}^\infty V_\lambda$として
				\begin{align}
					(0,\infty) \ni t \longmapsto A_t(\omega),
					\quad (\forall \omega \in \Omega \backslash V)
				\end{align}
				は連続となる.$\tilde{A} \coloneqq A \defunc_{\Omega \backslash V} \in [A]_{NAT}$
				が求める$A$のバージョンである.
				\QED
		\end{description}
	\end{prf}
	
	\begin{itembox}[l]{Problem 4.15}
		Let $X = \Set{X_t,\mathscr{F}_t}{0 \leq t < \infty}$ be a continuous, nonnegative process
		with $X_0 = 0$ a.s., and $A = \Set{A_t,\mathscr{F}_t}{0 \leq t < \infty}$ any continuous,
		increasing process for which
		\begin{align}
			E(X_T) \leq E(A_T)
		\end{align}
		holds for every bounded stopping time $T$ of $\{\mathscr{F}_t\}$. Introduce the process
		$V_t \coloneqq \max{0 \leq s \leq t}{X_s}$, consider a continuous, increasing function $F$
		on $[0,\infty)$ with $F(0) = 0$, and define \textcolor{red}{$G(x) \coloneqq 2F(x) + x\int_{(x,\infty)} u^{-1}\ dF(u);
		\ 0 < x < \infty$.} Establish the inequalities
		\begin{description}
			\item[(4.14)] $\displaystyle P[V_T \geq \epsilon] \leq \frac{E(A_T)}{\epsilon};\quad \forall \epsilon > 0$
			\textcolor{red}{\item[(4.15) (Lenglart inequality)] $\displaystyle P[V_T \geq \epsilon] 
				\leq \frac{E(\delta \wedge A_T)}{\epsilon} + P[A_T \geq \delta];\quad \forall \epsilon > 0,\ \delta > 0$}
			\item[(4.16)] $EF(V_T) \leq EG(A_T)$
		\end{description}
		for any stopping time $T$ of $\{\mathscr{F}_t\}$.
	\end{itembox}
	
	\begin{prf}\mbox{}
		\begin{description}
			\item[(1)] $X$のパスの連続性とProblem 2.7より
				\begin{align}
					H_\epsilon \coloneqq \inf{}{\Set{t \geq 0}{X_t \geq \epsilon}}
				\end{align}
				で$(\mathscr{F}_t)$-停止時刻が定まる.このとき
				\begin{align}
					V_T(\omega) \geq \epsilon 
					&\quad \Longrightarrow \quad
					X_t(\omega) \geq \epsilon, \quad \exists t \in [0,T(\omega)] \\
					&\quad \Longrightarrow \quad
					H_\epsilon(\omega) \leq t \leq T(\omega)
				\end{align}
				が成立するから,$\{X_0 = 0\} \cap \{V_T \geq \epsilon\}$上で
				$\epsilon = X_{H_\epsilon} = X_{T \wedge H_\epsilon}$となり
				\begin{align}
					\epsilon P(V_T \geq \epsilon)
					= \int_{\{V_T \geq \epsilon\}} X_{T \wedge H_\epsilon}\ dP
					\leq EX_{T \wedge H_\epsilon}
					\leq EA_{T \wedge H_\epsilon}
					\leq EA_T
				\end{align}
				が得られる.
				
			\item[(2)] $S_\delta \coloneqq \inf{}{\Set{t \geq 0}{A_t \geq \delta}}$により$(\mathscr{F}_t)$-停止時刻を定めれば,
				$A_{S_\delta} = \delta$と$t \longmapsto A_t(\omega)$の増大性より
				\begin{align}
					A_T(\omega) < \delta \quad \Longleftrightarrow \quad
					T(\omega) < S_\delta(\omega)
				\end{align}
				となるから
				\begin{align}
					P\left( V_T \geq \epsilon,\ A_T < \delta \right)
					&= P\left( V_{T \wedge S_\delta} \geq \epsilon,\ A_T < \delta \right)
					\leq P\left( V_{T \wedge S_\delta} \geq \epsilon \right) \\
					&\leq \frac{E(A_{S_\delta \wedge T})}{\epsilon}
					= \frac{E(A_{S_\delta} \wedge A_T)}{\epsilon}
					= \frac{E(\delta \wedge A_T)}{\epsilon}
				\end{align}
				が成立し,両辺に$P\left( V_T \geq \epsilon,\ A_T \geq \delta \right)$を加えて
				Lenglartの不等式を得る.
				
			\item[(3)] $F$は連続かつ非減少であるからLebesgue-Stieltjes積分が構成され,任意の$x \in [0,\infty)$に対し
				\begin{align}
					F(x) = \int_{[0,\infty)} \defunc_{(0,x]}(u)\ dF(u)
				\end{align}
				が満たされる.
				\begin{align}
					(\omega,u) \longmapsto \defunc_{(0,V_T(\omega)]}(u)
				\end{align}
				は,$u$の関数として左連続であり,また$\omega$の関数としては$\mathscr{F}/\borel{\R}$-可測であるから(Problem 1.16),
				二変数関数として$\mathscr{F} \otimes \borel{[0,\infty)}/\borel{\R}$-可測であり,このときFubiniの定理より
				\begin{align}
					E F(V_T) &= \int_{[0,\infty)} E\left( \defunc_{[u,\infty)}(V_T) \right)\ dF(u) \\
					&= \int_{[0,\infty)} P(V_T \geq u)\ dF(u) \\
					&\leq \int_{[0,\infty)} \frac{E(u \wedge A_T)}{u} + P(A_T \geq u)\ dF(u) \\
					&= \int_{[0,\infty)} \frac{E(u \wedge A_T \defunc_{\{A_T \geq u\}})}{u} + 
						\frac{E(u \wedge A_T \defunc_{\{A_T < u\}})}{u} + P(A_T \geq u)\ dF(u) \\
					&= \int_{[0,\infty)} 2 P(A_T \geq u) + \frac{E(A_T \defunc_{\{A_T < u\}})}{u}\ dF(u) \\
					&=  E\left(2F(A_T)\right) 
						+ E\left[A_T \int_{[0,\infty)} \frac{1}{u} \defunc_{(A_T,\infty)}(u)\ dF(u)\right] \\
					&= EG(A_T)
				\end{align}
				が得られる.
				\QED
		\end{description}
	\end{prf}

\chapter{Brownian Motion}
\begin{itembox}[l]{Dynkin system theorem}
		Let $\mathscr{C}$ be a collection of subsets of $\Omega$ 
		which is closed under pairwise intersection. If $\mathscr{D}$ is 
		a Dynkin system containing $\mathscr{C}$, then $\mathscr{D}$ also 
		contains the $\sigma$-field $\sigma(\mathscr{C})$ generated by $\mathscr{C}$.
\end{itembox}

\begin{prf}
	定理\ref{thm:Dynkin_system_theorem}より
	$\sigma(\mathscr{C}) = \delta(\mathscr{C}) \subset \mathscr{D}$となる.
	\QED
\end{prf}

\begin{itembox}[l]{Problem 1.4}
\label{thm:application_dynkin_system_theorem_to_independence}
		Let $X = \Set{X_t}{0 \leq t < \infty}$ be a stochastic process 
		for which $X_0,X_{t_1} - X_{t_0}, \cdots, X_{t_n} - X_{t_{n-1}}$ are 
		independent random variables, for every integer $n \geq 1$ and indices 
		$0 = t_0 < t_1 < \cdots < t_n < \infty$. Then for any fixed $0 \leq s < t < \infty$, 
		the increment $X_t - X_s$ is independent of $\mathscr{F}^X_s$.
\end{itembox}
この主張の逆も成立する:
\begin{prf}
	先ず任意の$s \leq t \leq r$に対し$\sigma(X_t - X_s) \subset \mathscr{F}^X_r$が成り立つ.実際,
	\begin{align}
		\Phi:\R^d \times \R^d \ni (x,y) \longmapsto x - y
	\end{align}
	の連続性と$\borel{\R^d \times \R^d} = \borel{\R^d} \otimes \borel{\R^d}$より,
	任意の$E \in \borel{\R^d}$に対して
	\begin{align}
		(X_t - X_s)^{-1}(E) 
		= \left\{ \left( X_t,X_s \right) \in \Phi^{-1}(E) \right\}
		\in \sigma(X_s,X_t) \subset \mathscr{F}^X_r
		\label{eq:thm_application_dynkin_system_theorem_to_independence_1}
	\end{align}
	が満たされる.よって任意に$A_0 \in \sigma(X_0),\ A_i \in \sigma(X_{t_i} - X_{t_{i-1}})$を取れば,
	$X_{t_n} - X_{t_{n-1}}$が$\mathscr{F}^X_{t_{n-1}}$と独立であるから
	\begin{align}
		P(A_0 \cap A_1 \cap \cdots \cap A_n)
		= P(A_0 \cap A_1 \cap \cdots \cap A_{n-1}) P(A_n)
	\end{align}
	が成立する.帰納的に
	\begin{align}
		P(A_0 \cap A_1 \cap \cdots \cap A_n)
		= P(A_0) P(A_1) \cdots P(A_n)
	\end{align}
	が従い$X_0,X_{t_1} - X_{t_0}, \cdots, X_{t_n} - X_{t_{n-1}}$の独立性を得る.
	\QED
\end{prf}

\begin{prf}[Problem 1.4]\mbox{}
	\begin{description}
		\item[第一段]
			Dynkin族を次で定める:
			\begin{align}
				\mathscr{D} \coloneqq
				\Set{A \in \mathscr{F}}{P(A \cap B) = P(A)P(B),\ \forall B \in \sigma(X_t - X_s)}.
			\end{align}
			いま,任意に$0 = s_0 < \cdots < s_n = s$を取り固定し
			\begin{align}
				\mathscr{A}_{s_0, \cdots, s_n} \coloneqq
				\Set{\bigcap_{i=0}^n A_i}{A_0 \in \sigma(X_0),\ A_i \in \sigma(X_{s_i} - X_{s_j}),\ i=1,\cdots,n}
			\end{align}
			により乗法族を定めれば,仮定より$\sigma(X_{s_i} - X_{s_{i-1}})$と$\sigma(X_t - X_s)$が独立であるから
			\begin{align}
				\mathscr{A}_{s_0, \cdots, s_n}
				\subset \mathscr{D}
			\end{align}
			が成立し,Dynkin族定理により
			\begin{align}
				\sigma(X_{s_0},X_{s_1}-X_{s_0},\cdots,X_{s_n} - X_{s_{n-1}})
				= \sgmalg{\mathscr{A}_{s_0, \cdots, s_n}}
				\subset \mathscr{D}
				\label{eq:thm_application_dynkin_system_theorem_to_independence_2}
			\end{align}
			が従う.
		
		\item[第二段]
			$\sigma(X_{s_0},X_{s_1}-X_{s_0},\cdots,X_{s_n} - X_{s_{n-1}})$の全体が
			$\mathscr{F}^X_s$を生成することを示す.先ず,
			(\refeq{eq:thm_application_dynkin_system_theorem_to_independence_1})より
			\begin{align}
				\bigcup_{\substack{n \geq 1 \\ s_0 < \cdots < s_n}} 
				\sigma(X_{s_0},X_{s_1}-X_{s_0},\cdots,X_{s_n} - X_{s_{n-1}})
				\subset \mathscr{F}^X_s
				\label{eq:thm_application_dynkin_system_theorem_to_independence_3}
			\end{align}
			が成立する.一方で,任意の
			$X_r^{-1}(E)\ (\forall E \in \borel{\R^d},\ 0 < r \leq s)$について,
			\begin{align}
				\Psi:\R^d \times \R^d \ni (x,y) \longmapsto x + y
			\end{align}
			で定める連続写像を用いれば
			\begin{align}
				X_r^{-1}(E)
				= \left( X_r - X_0 + X_0 \right)^{-1}(E)
				= \left\{\left( X_r - X_0, X_0\right) \in \Psi^{-1}(E) \right\}
			\end{align}
			となり,$X_r^{-1}(E) \in \sigma(X_0, X_r - X_0)$が満たされ
			\begin{align}
				\sigma(X_r) \subset \sigma(X_0, X_r - X_0)
				\subset \sigma(X_0, X_r - X_0,X_s - X_r)
				\label{eq:thm_application_dynkin_system_theorem_to_independence_4}
			\end{align}
			が出る.
			$\sigma(X_0) \subset \sigma(X_0,X_s - X_0)$
			も成り立ち
			\begin{align}
				\bigcup_{0 \leq r \leq s} \sigma(X_r) \subset 
				\bigcup_{\substack{n \geq 1 \\ s_0 < \cdots < s_n}} \sigma(X_{s_0},X_{s_1}-X_{s_0},\cdots,X_{s_n} - X_{s_{n-1}})
			\end{align}
			が従うから,(\refeq{eq:thm_application_dynkin_system_theorem_to_independence_3})
			と併せて
			\begin{align}
				\mathscr{F}^X_s
				= \sgmalg{\bigcup_{\substack{n \geq 1 \\ s_0 < \cdots < s_n}} \sigma(X_{s_0},X_{s_1}-X_{s_0},\cdots,X_{s_n} - X_{s_{n-1}})}
				\label{eq:thm_application_dynkin_system_theorem_to_independence_5}
			\end{align}
			が得られる.
		
		\item[第三段]
			任意の$0 = s_0 < s_1 < \cdots < s_n = s$に対し,
			(\refeq{eq:thm_application_dynkin_system_theorem_to_independence_1})と
			(\refeq{eq:thm_application_dynkin_system_theorem_to_independence_4})より
			\begin{align}
				\sigma(X_{s_0},X_{s_1}-X_{s_0},\cdots,X_{s_n} - X_{s_{n-1}})
				= \sigma(X_{s_0},X_{s_1},\cdots,X_{s_n})
				\label{eq:thm_application_dynkin_system_theorem_to_independence_6}
			\end{align}
			が成り立つ.
		
		\item[第四段]
			二つの節点$0 = s_0 < \cdots < s_n = s$と$0 = r_0 < \cdots < r_m = s$
			の合併を$0 = u_0 < \cdots < u_k = s$と書けば
			\begin{align}
				\sigma(X_{s_0},\cdots,X_{s_n})
				\cup \sigma(X_{r_0},\cdots,X_{r_m})
				\subset \sigma(X_{u_0},\cdots,X_{u_k})
			\end{align}
			が成り立つから
			\begin{align}
				\bigcup_{\substack{n \geq 1 \\ s_0 < \cdots < s_n}} \sigma(X_{s_0},X_{s_1},\cdots,X_{s_n})
			\end{align}
			は交演算で閉じている.従って
			(\refeq{eq:thm_application_dynkin_system_theorem_to_independence_2}),
			(\refeq{eq:thm_application_dynkin_system_theorem_to_independence_5}),
			(\refeq{eq:thm_application_dynkin_system_theorem_to_independence_6})及び
			Dynkin族定理により
			\begin{align}
				\mathscr{F}^X_s 
				= \sgmalg{\bigcup_{\substack{n \geq 1 \\ s_0 < \cdots < s_n}} \sigma(X_{s_0},X_{s_1}-X_{s_0},\cdots,X_{s_n} - X_{s_{n-1}})}
				= \sgmalg{\bigcup_{\substack{n \geq 1 \\ s_0 < \cdots < s_n}} \sigma(X_{s_0},X_{s_1},\cdots,X_{s_n})}
				\subset \mathscr{D}
			\end{align}
			が従い定理の主張を得る.
			\QED
	\end{description}
\end{prf}
\section{The Consistency Theorem}
	Karatzas-Shreve より Bogachev の Measure Theory に載っている
	Kolmogorovの拡張定理の方が洗練された簡潔な証明になっているので
	頭に入りやすい.
	
	
\section{The Kolmogorov-\v{C}entsov Theorem}
	\begin{itembox}[l]{Exercise 2.7}
		The only $\borel{(\R^d)^{[0,\infty)}}$-measurable set contained 
		in $C[0,\infty)^d$ is the empty set.
	\end{itembox}
	
	\begin{prf}\mbox{}
		\begin{description}
			\item[第一段]
				$\borel{(\R^d)^{[0,\infty)}} = \sigma(B_t;\ 0 \leq t < \infty)$
				が成り立つことを示す.先ず,任意の$C \in \mathscr{C}$は
				\begin{align}
					C &= \Set{\omega \in (\R^d)^{[0,\infty)}}{(\omega(t_1),\cdots,\omega(t_n)) \in A} \\
					&=  \Set{\omega \in (\R^d)^{[0,\infty)}}{(B_{t_1}(\omega),\cdots,B_{t_n}(\omega)) \in A},
					\quad (A \in \borel{(\R^d)^n})
				\end{align}
				の形で表されるから$\mathscr{C} \subset \sigma(B_t;\ 0 \leq t < \infty)$
				が従い$\borel{(\R^d)^{[0,\infty)}} \subset \sigma(B_t;\ 0 \leq t < \infty)$
				を得る.逆に
				\begin{align}
					\sigma(B_t) \subset \mathscr{C},
					\quad (\forall t \geq 0)
				\end{align}
				より$\borel{(\R^d)^{[0,\infty)}} \supset \sigma(B_t;\ 0 \leq t < \infty)$
				も成立し$\borel{(\R^d)^{[0,\infty)}} = \sigma(B_t;\ 0 \leq t < \infty)$
				が出る.
				
			\item[第二段]
				高々可算集合$S = \{t_1,t_2,\cdots\} \subset [0,\infty)$に対して
				\begin{align}
					\mathcal{E}_S \coloneqq \Set{\Set{\omega \in (\R^d)^{[0,\infty)}}{(\omega(t_1),\omega(t_2),\cdots) \in A}}{A \in \borel{(\R^d)^{\# S}}}
				\end{align}
				とおけば
				\footnote{
					$S$が可算無限なら$(\R^d)^{\# S} = \R^\infty$.
				},座標過程$B$は
				$(\omega(t_1),\omega(t_2),\cdots) = (B_{t_1}(\omega),B_{t_2}(\omega),\cdots)$
				を満たすから
				\begin{align}
					\mathcal{E}_S = \Set{\left\{(B_{t_1},B_{t_2},\cdots) \in A\right\}}{A \in \borel{(\R^d)^{\# S}}} \eqqcolon \mathcal{F}^B_S
				\end{align}
				が成立する.従って第一章のLemma3 for Exercise 1.8と前段の結果より
				\begin{align}
					\borel{(\R^d)^{[0,\infty)}}
					&= \sigma(B_t;\ 0 \leq t < \infty)
					= \mathcal{F}^B_{[0,\infty)}
					= \bigcup_{S \subset [0,\infty):at\ most\ countable} \mathcal{F}^B_S\\
					&= \bigcup_{S \subset [0,\infty):at\ most\ countable} \mathcal{E}_S
				\end{align}
				を得る.すなわち,$\borel{(\R^d)^{[0,\infty)}}$の任意の元は
				$\Set{\omega \in (\R^d)^{[0,\infty)}}{(\omega(t_1),\omega(t_2),\cdots) \in A}$
				の形で表現され,$A \neq \emptyset$ならば
				$\Set{\omega \in (\R^d)^{[0,\infty)}}{(\omega(t_1),\omega(t_2),\cdots) \in A} \not\subset C[0,\infty)^d$となり主張が従う.
				\QED
		\end{description}
	\end{prf}
	
	\begin{itembox}[l]{Theorem 2.8 and Problem 2.9}
		Suppose that a process $X = \Set{X_t}{t \in  [0,T]^d}$ ($d \geq 1$)
		on a probability space $(\Omega,\mathscr{F},P)$ satisfies the condition
		\begin{align}
			E|X_t - X_s|^\alpha \leq C\Norm{t-s}{}^{d + \beta},
			\quad \mbox{where} \Norm{t-s}{} = \operatorname*{max}_{1 \leq i \leq d}|t_i - s_i|,
		\end{align}
		for some positive constants $\alpha,\beta$, and $C$. Then there exists a 
		continuous modification $\tilde{X} = \Set{\tilde{X}_t}{t \in [0,T]^d}$ of $X$, 
		which is locally H\Ddot{o}lder-continuous with exponent $\gamma$ for every 
		$\gamma \in (0,\beta/\alpha)$. \textcolor{red}{More precisely, for every $\gamma \in (0,\beta/\alpha)$,
		\begin{align}
			\forall \omega \in \Omega^*, \quad \sup{\substack{0 < \Norm{t-s}{} < h(\omega) \\ s,t \in [0,T]^d}}{\frac{\left| \tilde{X}_t(\omega) - \tilde{X}_s(\omega) \right|}{\Norm{t-s}{}^\gamma}} \leq \frac{2}{1-2^{-\gamma}}
		\end{align}
		for some $\Omega^* \in \mathscr{F}$ with $P(\Omega^*)=1$ and 
		positive random variable $h$, where $\Omega^*$ and $h$ depend on $\gamma$.}
	\end{itembox}
	
	\begin{prf}\mbox{}
		\begin{description}
			\item[第一段]
				$[0,T]^d$における順序$\prec$を
				\begin{align}
					s \prec t \Longleftrightarrow \forall i \in \{1,2,\cdots,d\}[ s_i \leq t_i] \wedge \exists i \in \{1,2,\cdots,d\}[ s_i < t_i]
				\end{align}
				で定める.$\N$の任意の要素$n$に対して
				\begin{align}
					L_n = \Set{\frac{kT}{2^n}}{k=0,1,2,\cdots,2^n-1}
				\end{align}
				として$L = \bigcup_{n \in \N} L_n$とおく.$L$は$[0,T]^d$において稠密である.
				$L_n$の要素$s$に対して
				\begin{align}
					R_n(s) = \Set{t \in L_n}{s \prec t \wedge \Norm{t-s}{} = T2^{-n}}
				\end{align}
				とおく.つまり,$R_n(s)$とは$s$の各成分を最大$T2^{-n}$だけ動かした順序対の集合である.
				いま,$L_n$の要素数は$2^{nd}$,$L_n$の各要素$s$に対して
				$R_n(s)$の要素数は$2^d$である.
				Chebyshevの不等式より,任意の正数$\epsilon$に対して
				\begin{align}
					P\left(|X_t-X_s|\geq\epsilon\right)
					\leq \epsilon^{-\alpha}E|X_t-X_s|^\alpha
					\leq C\epsilon^{-\alpha}\Norm{t-s}{}^{d+\beta}
				\end{align}
				となり,特に$\epsilon = 2^{-\gamma n}$かつ
				$\Norm{t-s}{} = T2^{-n}$の場合は
				\begin{align}
					P\left(|X_t-X_s|\geq2^{-\gamma n}\right)
					\leq C 2^{-n(d+\beta - \alpha \gamma)}
				\end{align}
				が成り立つから,
				\begin{align}
					P\left(\operatorname*{max}_{s \in L_n \wedge t \in R_n(s)}
					|X_t-X_s|\geq2^{-\gamma n}\right)
					= P\left(\bigcup_{s \in L_n} \bigcup_{t \in R_n(s)}
					\{|X_t-X_s|\geq2^{-\gamma n}\}\right)
					\leq 2^d C T^{d+\beta} 2^{-n(\beta - \alpha \gamma)}
				\end{align}
				が成り立つ.$A_n = \left\{\operatorname*{max}_{s \in L_n \wedge t \in R_n(s)}|X_t-X_s|\geq2^{-\gamma n}\right\}$とおけば,Borel-Cantelliの補題より
				\begin{align}
					N = \bigcap_{n \in \N} \bigcup_{k \geq n} A_k
				\end{align}
				は$P$-零集合となり,
				\begin{align}
					\forall \omega \in \Omega \backslash N,\
					\exists N \in \N,\
					\forall n \in \N,\quad
					N \leq n \Longrightarrow \operatorname*{max}_{s \in L_n \wedge t \in R_n(s)}
					|X_t(\omega) - X_s(\omega)| < 2^{-\gamma n}
					\label{eq:chapter_2_theorem_2_8_1}
				\end{align}
				が満たされる.
				
			\item[第二段]
				$\Omega \backslash N$の要素$\omega$に対して,
				(\refeq{eq:chapter_2_theorem_2_8_1})を満たす自然数$N$のうち
				最小なものを$n^*(\omega)$と定める(自然数の整列性).つまり$n^*$は
		\begin{align}
			n^* = \Biggl\{\ (\omega,n)\ \, : \quad &\omega \in \Omega \wedge n \in \N \wedge \Biggr.\\
			&\forall m \in \N\left[n \leq m \Longrightarrow \operatorname*{max}_{s \in L_m \wedge t \in R_m(s)}
					|X_t(\omega) - X_s(\omega)| < 2^{-\gamma m}\right] \wedge \\
			&\forall N \in \N
			\Biggl. \left[\forall m \in \N\left[N \leq m \Longrightarrow \operatorname*{max}_{s \in L_m \wedge t \in R_m(s)}
					|X_t(\omega) - X_s(\omega)| < 2^{-\gamma m}\right] \Longrightarrow n \leq N\right]\ \Biggr\}
		\end{align}
		で与えられる写像である.写像$n^*$は$\mathscr{F}/\borel{\R}$-可測性を持つ.実際,
		任意の自然数$\ell$に対して
		\begin{align}
			{n^*}^{-1}(\ell) = \left\{ \bigcap_{n = \ell}^\infty A_n^c \right\} \cap \left\{ \bigcap_{1 \leq j \leq \ell-1} \bigcap_{n = j}^\infty A_n \right\}
		\end{align}
		を満たす.
		\end{description}
	\end{prf}
	
	
	
	確率変数$h$について,厳密には
	\begin{align}
		h(\omega) \coloneqq 
		\begin{cases}
			2^{-n^*(\omega)}, & (\omega \in \Omega^*), \\
			0, & (\omega \in \Omega \backslash \Omega^*)
		\end{cases}
	\end{align}
	とおけばよい.
	
	\begin{itembox}[l]{Corollary to Theorem 2.8}
		There is a probability measure $P$ on $(\R^{[0,\infty)},\mathscr{B}(\R^{[0,\infty)}))$,
		and a stochastic process $W = \Set{W_t,\mathscr{F}_t^W}{t \geq 0}$ on the same space,
		such that under $P$, $W$ is a Brownian motion.
	\end{itembox}
	
	\begin{prf}\mbox{}
		\begin{description}
			\item[第一段]
				Corollary to Theorem 2.2より,$(\R^{[0,\infty)},\mathscr{B}(\R^{[0,\infty)}))$にただ一つの確率測度$P$が存在して,
				\begin{align}
					B = \Set{(x,y)}{\exists t \in [0,\infty) \exists \omega \in \R^{[0,\infty)}
					\left( x=(t,\omega) \wedge y=\omega(t)-\omega(0) \right)}
				\end{align}
				で定める写像$B$が$P$の下で
				\begin{itemize}
					\item $\R^{[0,\infty)}$の任意の要素$\omega$に対して
						$B_0(\omega) = 0$,
					\item 任意の実数$s,t$に対し,$0 \leq s < t$ならば
						$B_t - B_s$は$\mathscr{F}_s$と独立,
					\item 任意の実数$s,t$に対し,$0 \leq s < t$ならば
						$P(B_t - B_s)^{-1}$は平均0で分散が$t-s$の正規分布
				\end{itemize}
				となる.Theorem2.8 と Problem2.10 により,1以上の任意の自然数$N$に対し,
				$[0,N]$上で$B$の修正$W^N$が存在する.
				\begin{align}
					\Omega_N &= \Set{\omega \in \R^{[0,\infty)}}{\forall t \in [0,N] \cap \Q,
					\quad W_t^N(\omega) = B_t(\omega)} \\
					&= \bigcap_{t \in [0,N] \cap \Q} \Set{\omega \in \R^{[0,\infty)}}{W_t^N(\omega) = B_t(\omega)}
				\end{align}
				とおけば,$W^N$は$B$の修正であるから$P(\Omega_N)=1$.ここで
				$\tilde{\Omega} = \bigcap_{N \in \N}\Omega_N$とおく.
				0以上の実数$t$と$\tilde{\Omega}$の要素$\omega$が任意に与えられたとき,
				$t < N$を満たす自然数$N$を取れば,$N$以上の任意の自然数$n$で
				\begin{align}
					\forall s \in [0,N] \cap \Q, \quad
					B_s(\omega) = W^N_s(\omega) \wedge B_s(\omega) = W^n_s(\omega)
				\end{align}
				となり,$W^N(\omega)$と$W^n(\omega)$の連続性と定理
				\ref{thm:equivalence_set_of_two_mappings_into_Hausdorff_space_is_closed}より
				$W^N(\omega)$と$W^n(\omega)$は$[0,N]$上で一致する.すなわち
				\begin{align}
					\forall n \in \N,\quad N \leq n \Longrightarrow W^n_t(\omega)
					= W^N_t(\omega)
				\end{align}
				が成り立つから,このとき$\lim_{n \to \infty} W^n_t(\omega)$が確定する.
				\begin{align}
					W_t(\omega) = 
					\begin{cases}
						\lim_{n \to \infty} W^n_t(\omega), & (\omega \in \tilde{\Omega}), \\
						0, & (\omega \in \R^{[0,\infty)} \backslash \tilde{\Omega})
					\end{cases}
				\end{align}
				で$W$を定めれば,$W$は$B$の修正となる.実際,0以上の任意の実数$t$に対し,
				$t < N$を満たす自然数$N$を取れば
				\begin{align}
					\forall \omega \in \tilde{\Omega},\quad 
					W_t(\omega) = W^N_t(\omega)
				\end{align}
				となり,$W^N$が$B$の修正であるから
				\begin{align}
					P(W_t \neq B_t) \leq P(W_t \neq W^N_t) + P(W^N_t \neq B_t) = 0
				\end{align}
				が成立する.またこの$t$において,$W^N(\omega)$の連続性から$W(\omega)$の$t$での連続性が従う.
				
			\item[第二段]
				前段で定めた$W$が$(\R^{[0,\infty)},\mathscr{B}(\R^{[0,\infty)}),P)$の上の
				Brown運動であることを示す.まず$P$-a.s.に$W_0 = B_0$である.また
				$0 \leq s < t$を満たす任意の実数$s,t$に対し,
				\begin{align}
					\Omega' = \Set{\omega \in \R^{[0,\infty)}}{W_s(\omega) \neq B_s(\omega) \wedge W_t(\omega) \neq B_t(\omega)}
				\end{align}
				とおく.$\borel{\R}$の要素$E,F$が任意に与えられたとして,
				\begin{align}
					W_s^{-1}(F) \cap \Omega' = B_s^{-1}(F) \cap \Omega',
					\quad
					(W_t-W_s)^{-1}(E) \cap \Omega' = (B_t-B_s)^{-1}(E) \cap \Omega'
				\end{align}
				が成り立ち,かつ$P(\Omega') = 1$であるから
				\begin{align}
					P\left( W_s^{-1}(F) \right)
					= P\left( W_s^{-1}(F) \cap \Omega' \right)
					&= P\left( B_s^{-1}(F) \cap \Omega' \right)
					= P\left( B_s^{-1}(F) \right), \\
					P\left( (W_t-W_s)^{-1}(E) \right) &= P\left( (B_t-B_s)^{-1}(E) \right), \label{eq:chapter_2_Corollary_to_Theorem_2_8} \\
					P\left( W_s^{-1}(F) \cap (W_t-W_s)^{-1}(E) \right)
					&= P\left( B_s^{-1}(F) \cap (B_t-B_s)^{-1}(E) \right)
				\end{align}
				が従い,$B$の独立増分性と併せて
				\begin{align}
					P\left( W_s^{-1}(F) \cap (W_t-W_s)^{-1}(E) \right)
					&= P\left( B_s^{-1}(F) \cap (B_t-B_s)^{-1}(E) \right) \\
					&= P\left( B_s^{-1}(F) \right) P\left( (B_t-B_s)^{-1}(E) \right) \\
					&= P\left( W_s^{-1}(F) \right) P\left( (W_t-W_s)^{-1}(E) \right) \\
				\end{align}
				となる.以上で$W$の独立増分性が示された.また
				(\refeq{eq:chapter_2_Corollary_to_Theorem_2_8})から
				$W_t-W_s$の分布は$B_t-B_s$の分布に一致する.
				\QED
		\end{description}
	\end{prf}
\section{The Space $C[0,\infty)$, Weak Convergence, and the Wiener Measure}
	\begin{itembox}[l]{Problem 4.1}
		Show that $\rho$ defined by (4.1) is a metric on $C[0,\infty)^d$ and, under $\rho$, 
		$C[0,\infty)^d$ is a complete, separable metric space.
	\end{itembox}
	以下,$C[0,\infty)^d$には$\rho$により広義一様収束位相を導入する.
	
	\begin{prf}
		付録の定理\ref{thm:appendix_complete_separability_of_spaces_of_continuous_functions}により従う.
		\QED
	\end{prf}

\begin{itembox}[l]{Problem 4.2}\label{chapter_2_problem_4_2}
	Let $\mathscr{C}(\mathscr{C}_t)$ be the collection of finite-dimensional cylinder sets of the form (2.1); i.e.,
	\begin{align}
		C = \Set{\omega \in C[0,\infty)^d}{(\omega(t_1),\cdots,\omega(t_n)) \in A};
		\quad n \geq 1,\ A \in \borel{(\R^d)^n},
	\end{align}
	where, for all $i=1,\cdots,n,\ t_i \in [0,\infty)$ (respectively, $t_i \in [0,t]$).
	Denote by $\mathscr{G}(\mathscr{G}_t)$ the smallest $\sigma$-field containing $\mathscr{C}(\mathscr{C}_t)$.
	Show that $\mathscr{G} = \borel{C[0,\infty)^d}$, the Borel $\sigma$-field generated by the open sets in
	$C[0,\infty)^d$, and that $\mathscr{G}_t = \varphi_t^{-1}\left( \borel{C[0,\infty)^d} \right) \eqqcolon
	\mathscr{B}_t\left( C[0,\infty)^d \right)$, where $\varphi_t:C[0,\infty)^d \longrightarrow C[0,\infty)^d$ is the
	mapping $(\varphi_t\omega)(s) = \omega(t \wedge s);\ 0 \leq s < \infty$.
\end{itembox}

\begin{prf}\mbox{}
	\begin{description}
		\item[第一段]
			$w_0 \in C[0,\infty)^d$とする.任意に$w \in C[0,\infty)^d$を取れば,$w$の連続性により$d(w_0,w)$の各項について
			\begin{align}
				\sup{t \leq n}{|w_0(t) - w(t)|} = \sup{r \in [0,n]\cap\Q}{|w_0(r) - w(r)|} \quad (n = 1,2,\cdots)
			\end{align}
			とできる.いま,任意に実数$\alpha \in \R$を取れば
			\begin{align}
				\Set{w \in C[0,\infty)^d}{\sup{r \in [0,n]\cap\Q}{|w_0(r) - w(r)|} \leq \alpha}
				= \bigcap_{r \in [0,n]\cap\Q} \Set{w \in C[0,\infty)^d}{|w_0(r) - w(r)| \leq \alpha}
			\end{align}
			が成立し,右辺の各集合は
			$\mathscr{C}$に属するから$\mbox{左辺} \in \sgmalg{\mathscr{C}}$となる.従って
			\begin{align}
				\psi_n : C[0,\infty)^d \ni w \longmapsto \sup{r \in [0,n]\cap\Q}{|w_0(r) - w(r)|} \in \R, \quad (n = 1,2,\cdots)
			\end{align}
			で定める$\psi_n$は可測$\sgmalg{\mathscr{C}}/\borel{\R}$である.
			$x \longmapsto x \wedge 1$の連続性より$\psi_n \wedge 1$
			も$\sgmalg{\mathscr{C}}/\borel{\R}$-可測性を持ち,
			\begin{align}
				d(w_0,w) = \sum_{n=1}^{\infty}\frac{1}{2^n} \left( \psi_n(w) \wedge 1 \right)
			\end{align}
			により$C[0,\infty)^d \ni w \longmapsto d(w_0,w) \in \R$の
			$\sgmalg{\mathscr{C}}/\borel{\R}$-可測性が出るから,任意の$\epsilon > 0$に対する球について
			\begin{align}
				\Set{w \in C[0,\infty)^d}{d(w_0,w) < \epsilon} \in \sgmalg{\mathscr{C}}
			\end{align}
			が成り立つ.$C[0,\infty)^d$は第二可算公理を満たし,可算基底は上式の形の球で構成されるから,
			$\open{C[0,\infty)^d} \subset \sgmalg{\mathscr{C}}$が従い$\borel{C[0,\infty)^d} \subset \sgmalg{\mathscr{C}}$を得る.
			次に逆の包含関係を示す.いま任意に$n \in \Z_+$と$t_1 < \cdots < t_n$を選んで
			\begin{align}
				\phi : C[0,\infty)^d \ni w \longmapsto (w(t_1),\cdots, w(t_n)) \in (\R^d)^n
			\end{align}
			で定める写像は連続である.
			実際,任意の一点$w_0$での連続性を考えると,任意の$\epsilon > 0$に対して$t_n \leq N$を満たす$N \in \N$を取れば,
			$d(w_0,w) < \epsilon/(n2^N)$ならば
			$\sum_{i=1}^{n}|w_0(t_i) - w(t_i)| < \epsilon$が成り立つ.よって
			$\phi$は$w_0$で連続であり
			\begin{align}
				\borel{(\R^d)^n} \subset \Set{A \in \borel{(\R^d)^n}}{\phi^{-1}(A) \in \borel{C[0,\infty)^d}}
			\end{align}
			が出る.任意の$C \in \mathscr{C}$は,$n \in \N$と時点$t_1 < \cdots < t_n$によって決まる写像$\phi$によって
			$C = \phi^{-1}(B)\ (\exists B \in \borel{(\R^d)^n})$と表現できるから,
			$\mathscr{C} \subset \borel{C[0,\infty)^d}$が成り立ち
			$\sgmalg{\mathscr{C}} \subset \borel{C[0,\infty)^d}$が得られる.
			
		\item[第二段]
			$t \geq 0$とする.$C[0,\infty)^d$の位相を$\open{C[0,\infty)^d}$と書けば
			\begin{align}
				\varphi_t^{-1}\left( \borel{C[0,\infty)^d} \right)
				= \sgmalg{\Set{\varphi_t^{-1}(O)}{O \in \open{C[0,\infty)^d}}}
			\end{align}
			が成り立つ.任意の$\alpha \in \R$と$r \geq 0$に対して
			\begin{align}
				&\Set{w \in C[0,\infty)^d}{|w_0(r) - (\varphi_t w)(r)| \leq \alpha} \\
				&\qquad = \begin{cases}
					\Set{w \in C[0,\infty)^d}{|w_0(r) - (\varphi_t w)(r)| \leq \alpha}, & (r \leq t), \\
					\Set{w \in C[0,\infty)^d}{|w_0(r) - (\varphi_t w)(t)| \leq \alpha}, & (r > t),
				\end{cases}
				\quad \in \mathscr{C}_t
			\end{align}
			となるから
			\begin{align}
				\psi^t_n : C[0,\infty)^d \ni w \longmapsto \sup{r \in [0,n]\cap\Q}{|w_0(r) - (\varphi_t w)(r)|} \in \R, \quad (n = 1,2,\cdots)
			\end{align}
			で定める$\psi^t_n$は可測$\sgmalg{\mathscr{C}_t}/\borel{\R}$である.
			$x \longmapsto x \wedge 1$の連続性より$\psi^t_n \wedge 1$
			も$\sgmalg{\mathscr{C}_t}/\borel{\R}$-可測性を持ち,
			\begin{align}
				d(w_0,\varphi_t w) = \sum_{n=1}^{\infty}\frac{1}{2^n} \left( \psi^t_n(w) \wedge 1 \right)
			\end{align}
			により$C[0,\infty)^d \ni w \longmapsto d(w_0,\varphi_t w) \in \R$の
			$\sgmalg{\mathscr{C}_t}/\borel{\R}$-可測性が出るから,任意の$\epsilon > 0$に対する球について
			\begin{align}
				\Set{w \in C[0,\infty)^d}{d(w_0,\varphi_t w) < \epsilon} \in \sgmalg{\mathscr{C}_t}
			\end{align}
			が成り立つ.特に
			\begin{align}
				\varphi_t^{-1}\left( \Set{w \in C[0,\infty)^d}{d(w_0,w) < \epsilon} \right)
				= \Set{w \in C[0,\infty)^d}{d(w_0,\varphi_t w) < \epsilon}
			\end{align}
			が満たされ,$C[0,\infty)^d$の第二可算性より
			\begin{align}
				\varphi_t^{-1}(O) \in \sgmalg{\mathscr{C}_t},
				\quad (\forall O \in \open{C[0,\infty)^d})
			\end{align}
			が従う.ゆえに$\varphi_t^{-1}\left( \borel{C[0,\infty)^d} \right) \subset \sgmalg{\mathscr{C}_t}$となる.
			\QED
	\end{description}
\end{prf}

\begin{comment}
次の事柄は後の定理の証明で使うからここで証明しておく.
\begin{screen}
	\begin{thm}[$\mathscr{C}$は乗法族である]
		$\mathscr{C}$は交演算について閉じている.
	\end{thm}
\end{screen}
\begin{prf}
	任意に$A_1, A_2 \in \mathscr{C}$を取れば,$A_1,\ A_2$それぞれに対し
	$n_1,n_2 \in \N,\ C_1 \in \borel{(\R^d)^{n_1}},\ C_2 \in \borel{(\R^d)^{n_2}},\ t_1<\cdots<t_{n_1}$それから
	$s_1<\cdots<s_{n_2}$が決まっていて,
	\begin{align}
		A_1 &= \left\{\ w \in C[0,\infty)^d\quad |\quad (w(t_1), \cdots, w(t_{n_1})) \in C_1\ \right\} \\
		A_2 &= \left\{\ w \in C[0,\infty)^d\quad |\quad (w(s_1), \cdots, w(s_{n_2})) \in C_2\ \right\}
	\end{align}
	と表されている.$A_1,A_2$の時点に重複があるかないかで場合分けして示す.
	\begin{description}
	\item[時点に重複がない場合]
		集合を次のように同値な表記に直す:
		\begin{align}
			A_1 &= \left\{\ w \in C[0,\infty)^d\quad |\quad (w(t_1), \cdots, w(t_{n_1}),w(s_1), \cdots, w(s_{n_2})) \in C_1 \times (\R^d)^{n_2}\ \right\} \\
			A_2 &= \left\{\ w \in C[0,\infty)^d\quad |\quad (w(t_1), \cdots, w(t_{n_1}),w(s_1), \cdots, w(s_{n_2})) \in (\R^d)^{n_1} \times C_2\ \right\}
		\end{align}
		表現を変えれば乗法を考えやすくなり,上の場合は
		\begin{align}
			A_1 \cap A_2 = \left\{\ w \in C[0,\infty)^d\quad |\quad (w(t_1), \cdots, w(t_{n_1}),w(s_1), \cdots, w(s_{n_2})) \in C_1 \times C_2\ \right\}
		\end{align}
		と表現できる.$t_1,\cdots,s_{n_2}$の並びが気になるなら,この時点の並びを昇順に変換する$(dn_1 + dn_2) \times (dn_1 + dn_2)$行列$J_1$
		を用いて($J_1$は連続,線型,全単射),
		\begin{align}
			A_1 \cap A_2 &= \left\{\ w \in C[0,\infty)^d\quad |\quad J_1\Vector{w} \in J_1(C_1 \times C_2)\ \right\} \\
			\left(\Vector{w} \right.&= {}^{T}(w(t_1), \cdots, \left.w(t_{n_1}),w(s_1), \cdots, w(s_{n_2}))\right)
		\end{align}
		とすれば,$J(C_1 \times C_2) \in \borel{(\R^d)^{n_1 + n_2}}$であるから,$A_1 \cap A_2 \in \mathscr{C}$であることが明確になる.
	
	\item[時点に重複がある場合]
		$(r_{k_1},\cdots,r_{k_l}) \subset (t_1,\cdots,t_{n_1})$が重複時点であるとき,$A_1,A_2$の同値な表記は次のようにすればよい:
		\begin{align}
			A_1 &= \left\{\ w \in C[0,\infty)^d\quad |\quad (w(t_1),.,w(r_{k_1}),.,w(r_{k_l}),.,w(t_{n_1}),\mbox{\scriptsize($s_1, \cdots, s_{n_2}$から$r_{k_1},\cdots,r_{k_l}$を抜いたものを並べる)}) \in C_1 \times (\R^d)^{n_2 - l}\ \right\} \\
			A_2 &= \left\{\ w \in C[0,\infty)^d\quad |\quad (w(s_1),.,w(r_{k_1}),.,w(r_{k_l}),.,w(s_{n_2}),\mbox{\scriptsize($t_1, \cdots, t_{n_1}$から$r_{k_1},\cdots,r_{k_l}$を抜いたものを並べる)}) \in C_2 \times (\R^d)^{n_1 - l}\ \right\}
		\end{align}
		$A_2$について,条件中の時点の並びを変換し$A_1$の条件の順番に合わせる行列$J_2$(連続,線型,全単射)を用いて
		\begin{align}
			A_2 = \left\{\ w \in C[0,\infty)^d\quad |\quad (w(t_1),.,w(r_{k_1}),.,w(r_{k_l}),.,w(t_{n_1}),\mbox{\scriptsize($s_1, \cdots, s_{n_2}$から$r_{k_1},\cdots,r_{k_l}$を抜いたものを並べる)}) \in J_2(C_2 \times (\R^d)^{n_1 - l})\ \right\}
		\end{align}
		と書き直せば,$A_1 \cap A_2$は前段の様に表現可能であり,前段と同様に最後に時点を昇順に変換する行列を用いることで$A_1 \cap A_2 \in \mathscr{C}$となることが明確に判る.
	\end{description}
	\QED
\end{prf}
\end{comment}
\section{Weak Convergence}
	いま,$X$を局所コンパクトHausdorff空間として
	\begin{align}
		C_0(X) \coloneqq \Set{f:X \longrightarrow \C}{\mbox{連続かつ,任意の$\epsilon > 0$に対し
		$\closure{\Set{x \in X}{|f(x)| \geq \epsilon}}$がコンパクト}} 
	\end{align}
	とおく.この$C_0(X)$はノルム$\Norm{f}{C_0(X)} \coloneqq \sup{x \in X}{|f(x)|}$
	により複素Banach空間となる.また
	$(X,\borel{X})$上の複素測度$\mu$について,その総変動$|\mu|$が正則測度であるとき
	$\mu$は正則であるという.$X$上の正則複素測度の全体を$RM(X)$と書き,
	総変動ノルム$\Norm{\mu}{RM(X)} \coloneqq |\mu|(X)$によりノルム位相を導入する.
	任意の複素測度$\mu$に対し
	\begin{align}
		\Phi_\mu(f) \coloneqq \int_X f(x)\ \mu(dx)
	\end{align}
	により$C_0(X)$上の有界線型汎関数$\Phi_\mu$が定まる.
	
	\begin{screen}
		\begin{thm}[Rieszの表現定理]
			$X$を局所コンパクトHausdorff空間とする.
			$C_0(X)$に$\Norm{\cdot}{C_0(X)}$で位相を入れるとき,
			共役空間$C_0(X)^*$と書く.
			このとき$C_0(X)^*$と$RM(X)$は
			\begin{align}
				\Phi:RM(X) \ni \mu \longrightarrow \Phi_\mu \in C_0(X)
			\end{align}
			で定める対応関係$\Phi$によりBanach空間として等長同型となる.
		\end{thm}
	\end{screen}
	
	$C_0(X)^*$に汎弱位相を入れるとき,
	汎関数列$\left( \Phi_{\mu_n} \right)_{n=1}^\infty$が
	$\Phi_\mu$に汎弱収束することと
	\begin{align}
		\Phi_{\mu_n}(f) \longrightarrow \Phi_\mu(f)\ (n \longrightarrow \infty),
		\quad (\forall f \in C_0(X))
	\end{align}
	は同値になる.$C_0(X)^*$の汎弱位相の$\Phi$による逆像位相を
	$RM(X)$の弱位相と定めれば,$\Phi$は弱位相に関して位相同型となる.
	このとき,$(\mu_n)_{n=1}^\infty$が$\mu$に弱収束することは
	$\left( \Phi_{\mu_n} \right)_{n=1}^\infty$が$\Phi_\mu$に汎弱収束することと同値になり,
	すなわち
	\begin{align}
		\int_X f(x)\ \mu_n(dx) \longrightarrow \int_X f(x)\ \mu(dx)\ (n \longrightarrow \infty),
		\quad (\forall f \in C_0(X))
	\end{align}
	と同値になる.$X$上の正則な確率測度の全体を$\mathscr{P}(X)$と書けば
	$\mathscr{P}(X) \subset RM(X)$となり,正則確率測度の列$(P_n)_{n=1}^\infty$が
	$P \in \mathscr{P}(X)$に弱収束することは
	\begin{align}
		\int_X f(x)\ P_n(dx) \longrightarrow \int_X f(x)\ P(dx)\ (n \longrightarrow \infty),
		\quad (\forall f \in C_0(X))
	\end{align}
	と同値になる.
	
	\begin{itembox}[l]{Definition 4.3}
		It follows, in particular, that the weak limit $P$ is a probability measure, 
		and that it is unique.
	\end{itembox}
	
	\begin{prf}
		$f \equiv 1$として
		\begin{align}
			P(S) = \lim_{n \to \infty} P_n(S) = 1
		\end{align}
		が従うから$P$は確率測度である.また任意の有界連続関数$f:S \longrightarrow \R$に対し
		\begin{align}
			\int_S f\ dP = \int_S f\ dQ
		\end{align}
		が成り立つとき,任意の閉集合$A \subset S$に対して
		\begin{align}
			f_k(s) \coloneqq \frac{1}{1 + k d(s,A)},
			\quad (k=1,2,\cdots)
		\end{align}
		と定めれば$\lim_{k \to \infty} f_k = \defunc_A$(各点収束)が満たされるから,Lebesgueの収束定理より
		\begin{align}
			P(A) = \lim_{k \to \infty} \int_S f_k\ dP
			= \lim_{k \to \infty} \int_S f_k\ dQ
			= Q(A)
		\end{align}
		となり,測度の一致の定理より$P = Q$が得られる.すなわち弱極限は一意である.
		\QED
	\end{prf}
	
	\begin{itembox}[l]{lemma: change of variables for expectation}
		$(\Omega,\mathscr{F},P)$を確率空間,
		$(S,\mathscr{S})$を可測空間とする.
		このとき任意の
		有界$\mathscr{S}/\borel{\R}$-可測関数$f$
		と$\mathscr{F}/\mathscr{S}$-可測写像$X$に対して
		\begin{align}
			\int_\Omega f(X)\ dP = \int_S f\ dPX^{-1}
		\end{align}
		が成立する.
	\end{itembox}
	
	\begin{prf}
		任意の$A \in \mathscr{S}$に対して
		\begin{align}
			\int_S \defunc_A dPX^{-1}
			= P(X^{-1}(A))
			= \int_\Omega \defunc_{X^{-1}(A)}\ dP
			= \int_\Omega \defunc_{A}(X)\ dP
		\end{align}
		が成り立つから,任意の$\mathscr{S}/\borel{\R}$-可測単関数$g$に対し
		\begin{align}
			\int_\Omega g(X)\ dP = \int_S g\ dPX^{-1}
		\end{align}
		となる.$f$が有界なら一様有界な単関数で近似できるので,Lebesgueの収束定理より
		\begin{align}
			\int_\Omega f(X)\ dP = \int_S f\ dPX^{-1}
		\end{align}
		が出る.
		\QED
	\end{prf}
	
	\begin{itembox}[l]{Definition 4.4}
		Equivalently, $X_n \overset{\mathscr{D}}{\longrightarrow} X$ if and only if
		\begin{align}
			\lim_{n \to \infty} E_n f(X_n) = E f(X)
		\end{align} 
		for every bounded, continuous real-valued function $f$ on $S$, 
		where $E_n$ and $E$ denote expectations with respect to $P_n$ and $P$, respectively.
	\end{itembox}
	
	\begin{prf}
		任意の有界実連続関数$f:S \longrightarrow \R$に対し
		\begin{align}
			\int_\Omega f(X_n)\ dP_n = \int_S f\ dP_nX_n^{-1},
			\quad \int_\Omega f(X)\ dP = \int_S f\ dPX^{-1},
		\end{align}
		が成り立つから,$P_nX_n^{-1}$が$PX^{-1}$に弱収束することと
		$\lim_{n \to \infty} E_n f(X_n) = E f(X)$は同値である.
		\QED
	\end{prf}
	
	\begin{itembox}[l]{Problem 4.5}
		Suppose $\{X_n\}_{n=1}^\infty$ is a sequence of random variables taking values 
		in a metric space $(S_1,\rho_1)$ and converging in distribution to $X$. Suppose 
		$(S_2,\rho_2)$ is another metric space, and $\varphi:S_1 \longrightarrow S_2$ 
		is continuous. Show that $Y_n \coloneqq \varphi(X_n)$ converges in distribution 
		to $Y \coloneqq \varphi(X)$.
	\end{itembox}
	
	\begin{prf}
		任意の有界実連続関数$f:S_2 \longrightarrow \R$に対し
		$f \circ \varphi$は$S_1$上の有界実連続関数であるから
		\begin{align}
			\int_{S_2} f\ dPY_n^{-1}
			&= \int_{\Omega} f(Y_n)\ dP
			= \int_{\Omega} f(\varphi(X_n))\ dP
			= \int_{S_1} f \circ \varphi\ dPX_n^{-1} \\
			& \longrightarrow 
			\int_{S_1} f \circ \varphi\ dPX^{-1}
			= \int_{S_2} f\ dPY^{-1}
			\quad (n \longrightarrow \infty)
		\end{align}
		が成立する.
		\QED
	\end{prf}
	
\section{Tightness}
	テキスト本文において$m^T(\omega,\delta)$は
	\begin{align}
		m^T(\omega,\delta) \coloneqq \operatorname*{max}_{\substack{|s-t| \leq \delta \\ 0 \leq s,t \leq T}}|\omega(s) - \omega(t)|
	\end{align}
	で定められるが,$\operatorname{max}$と書いて妥当であることを確認しておく.
	まず
	\begin{align}
		D \coloneqq \Set{(s,t) \in \R \times \R}{|s-t| \leq \delta \wedge 0 \leq s,t \leq T}
	\end{align}
	で定められる集合は$\R \times \R$のコンパクト集合である.そして$\omega$は連続写像であるから
	\begin{align}
		\R \times \R \ni (s,t) \longmapsto \omega(s), 
		\quad \R \times \R \ni (s,t) \longmapsto \omega(t)
	\end{align}
	は共に実連続写像である.引き算は連続,絶対値も連続であるから
	\begin{align}
		\R \times \R \ni (s,t) \longmapsto |\omega(s) - \omega(t)|
	\end{align}
	は$\R \times \R$から$\R$への連続写像であり,$D$のコンパクト性から$D$上で最大値を取る.
	
	\begin{itembox}[l]{Problem 4.8}
		Show that $m^T(\omega,\delta)$ is continuous in $\omega \in C[0,\infty)$ under the metric
		$\rho$ of (4.1), is nondecreasing in $\delta$, and 
		$\lim_{\delta \downarrow 0}m^T(\omega,\delta) = 0$ for each $\omega \in C[0,\infty)$.
	\end{itembox}
	
	\begin{sketch}\mbox{}
		\begin{description}
			\item[第一段]
				$m^T(\omega,\delta)$が$\omega$に関して連続であることを示す.まず大雑把に,
				\begin{align}
					\left|\, \operatorname*{max}_{x} |f(x)| - \operatorname*{max}_{x} |g(x)|\, \right|
					\leq \operatorname*{max}_{x} |f(x) - g(x)|
				\end{align}
				が成立する.実際,
				\begin{align}
					\operatorname*{max}_{x} |f(x)| - \operatorname*{max}_{x} |g(x)|
					\leq \operatorname*{max}_{x} |f(x) - g(x)|
				\end{align}
				が成り立つことを確認するには
				\begin{align}
					|f(x_1)| = \operatorname*{max}_{x} |f(x)|
				\end{align}
				なる$x_1$を取り,
				\begin{align}
					\operatorname*{max}_{x} |f(x)| - \operatorname*{max}_{x} |g(x)|
					&= |f(x_1)| - \operatorname*{max}_{x} |g(x)| \\
					&\leq |f(x_1)| - |g(x_1)| \\
					&\leq |f(x_1) - g(x_1)| \\
					&\leq \operatorname*{max}_{x} |f(x) - g(x)|
				\end{align}
				となることを見ればよい.$f,g$を入れ替えれば
				\begin{align}
					\operatorname*{max}_{x} |g(x)| - \operatorname*{max}_{x} |f(x)|
					\leq \operatorname*{max}_{x} |f(x) - g(x)|
				\end{align}
				も成り立つから当初の主張を得る.よって$\omega_1,\omega_2$を$C[0,\infty)$の要素とすれば
				\begin{align}
					\left| m^T(\omega_1,\delta) - m^T(\omega_2,\delta) \right|
					\leq \operatorname*{max}_{\substack{|s-t| \leq \delta \\ 0 \leq s,t \leq T}}
					|(\omega_1(s) - \omega_1(t)) - (\omega_2(s) - \omega_2(t))|
				\end{align}
				が成立する.ところで,いま$\epsilon$を任意に与えられた正数とし,
				\begin{align}
					T \leq n
				\end{align}
				を満たす自然数$n$を取り
				\begin{align}
					\rho(\omega_1,\omega_2) < 2^{-n} \epsilon
				\end{align}
				が満たされていると仮定すれば,
				\begin{align}
					\operatorname*{sup}_{0 \leq t \leq n}|\omega_1(t) - \omega_2(t)|
					< \epsilon
				\end{align}
				となるから
				\begin{align}
					0 \leq t \leq T \Longrightarrow |\omega_1(t) - \omega_2(t)| < \epsilon
				\end{align}
				が満たされる.このとき
				\begin{align}
					0 \leq s,t \leq T \Longrightarrow 
					&|(\omega_1(s) - \omega_1(t)) - (\omega_2(s) - \omega_2(t))| \\
					&\leq |\omega_1(s) - \omega_2(s)| + |\omega_1(t) - \omega_2(t)| \\
					&< 2\epsilon
				\end{align}
				が成り立つので
				\begin{align}
					\left| m^T(\omega_1,\delta) - m^T(\omega_2,\delta) \right| < 2\epsilon
				\end{align}
				が従い,$m^T(\omega,\delta)$の$\omega$に関する連続性が得られた.
			
			\item[第二段]
				$\delta$に関して非減少であることを示す.いま$0 < \delta \leq \delta'$とする.
				\begin{align}
					(s,t) \longmapsto |\omega(s) - \omega(t)|
				\end{align}
				は
				\begin{align}
					\Set{(s,t)}{|s-t| \leq \delta \wedge 0 \leq s,t \leq T}
				\end{align}
				の上で最大値を取るのであるから,
				\begin{align}
					|\tilde{s} - \tilde{t}| \leq \delta \wedge
					0 \leq \tilde{s}, \tilde{t} \leq T
				\end{align}
				かつ
				\begin{align}
					|\omega(\tilde{s}) - \omega(\tilde{t})| = m^T(\omega,\delta)
				\end{align}
				を満たす$\tilde{s},\tilde{t}$を取ることが出来るが,
				\begin{align}
					|\tilde{s} - \tilde{t}| \leq \delta'
				\end{align}
				も満たされるので
				\begin{align}
					|\omega(\tilde{s}) - \omega(\tilde{t})|
					\in \Set{|\omega(s) - \omega(t)|}{|s - t| \leq \delta \wedge 0 \leq s, t \leq T}
				\end{align}
				となり
				\begin{align}
					|\omega(\tilde{s}) - \omega(\tilde{t})| \leq m^T(\omega,\delta')
				\end{align}
				が従う.よって
				\begin{align}
					\delta \leq \delta' \Longrightarrow m^T(\omega,\delta) \leq m^T(\omega,\delta')
				\end{align}
				が示された.
			
			\item[第三段]
				$\lim_{\delta \downarrow 0}m^T(\omega,\delta) = 0$が成り立つことを示す.
				$\epsilon$を任意に与えられた正数とする.$\omega$は$[0,T]$上で一様連続となるので
				\begin{align}
					|s-t| \leq \delta \Longrightarrow |\omega(s) - \omega(t)| < \epsilon
				\end{align}
				を満たす正数$\delta$が取れるが,このとき
				\begin{align}
					\delta' \leq \delta
				\end{align}
				を満たす任意の正数$\delta'$に対しても
				\begin{align}
					|s-t| \leq \delta' \Longrightarrow |\omega(s) - \omega(t)| < \epsilon
				\end{align}
				となるから
				\begin{align}
					\lim_{\delta \downarrow 0}m^T(\omega,\delta) = 0
				\end{align}
				が得られる.
				\QED
		\end{description}
	\end{sketch}
	
	\begin{itembox}[l]{Theorem 4.10}
	\end{itembox}
	
	\begin{sketch}\mbox{}
		\begin{description}
			\item[第一段]
				$\eta$を任意に与えられた正数とする.
				$\{P_n\}_{n=1}^\infty$は緊密なので,$C[0,\infty)$の或るコンパクト部分集合$K$が存在して
				\begin{align}
					\forall n \in \N\, (\, 1 - \eta \leq P_n(K)\, )
				\end{align}
				が満たされる.他方で十分大きな正数$\lambda$を取れば
				\begin{align}
					\forall \omega \in K\, (\, |\omega(0)| \leq \lambda\, )
				\end{align}
				となる.これはすなわち
				\begin{align}
					K \subset \Set{\omega}{|\omega(0)| \leq \lambda}
				\end{align}
				を表し,
				\begin{align}
					\forall n \in \N\, \left(\, P_n\Set{\omega}{\lambda < |\omega(0)|}
					\leq P_n(C[0,\infty) \backslash K) \leq \eta\, \right)
				\end{align}
				が従う.また$T,\epsilon$を任意に与えられた正数とすれば,或る正数$\delta_0$が存在して
				\begin{align}
					0 < \delta \leq \delta_0
					\Longrightarrow \forall \omega \in K\, \left(\, m^T(\omega,\delta) \leq \epsilon\, \right)
				\end{align}
				が成立する.つまり
				\begin{align}
					0 < \delta \leq \delta_0
					\Longrightarrow K \subset \Set{\omega}{m^T(\omega,\delta) \leq \epsilon}
				\end{align}
				が成り立つので,
				\begin{align}
					0 < \delta \leq \delta_0
					\Longrightarrow \forall n \in \N\, \left(\, P_n\Set{\omega}{\epsilon < m^T(\omega,\delta)}
					\leq P_n(C[0,\infty) \backslash K) \leq \eta\, \right)
				\end{align}
				が満たされる.
				
			\item[第二段]
		\end{description}
	\end{sketch}
	
	\begin{itembox}[l]{Problem 4.12}
		Suppose $\{P_n\}_{n=1}^\infty$ is a sequence of probability measures on
		$\left( C[0,\infty),\borel{C[0,\infty)} \right)$ which converges weakly to a probability
		measure $P$. Suppose, in addition, that $\{f_n\}_{n=1}^\infty$ is a uniformly bounded sequence
		of real-valued, continuous functions on $C[0,\infty)$ converging to a continuous function $f$,
		the convergence being uniform on compact subsets of $C[0,\infty)$. Then
		\begin{align}
			\lim_{n \to \infty} \int_{C[0,\infty)} f_n(\omega)\ dP_n(\omega)
			= \int_{C[0,\infty)} f(\omega)\ dP(\omega).
		\end{align}
	\end{itembox}
	
	\begin{sketch}\mbox{}
		\begin{description}
			\item[第一段]
				$\{f_n\}_{n=1}^\infty$は一様有界なので
				\begin{align}
					\forall b \in \N\, \forall \omega \in C[0,\infty)\,
					\left(\, |f_n(\omega)| < b\, \right)
				\end{align}
				を満たす正数$b$が存在する.$C[0,\infty)$の各点$\omega$で
				\begin{align}
					f_n(\omega) \longrightarrow f(\omega)\quad (n \longrightarrow \infty)
				\end{align}
				となるから
				\begin{align}
					\forall \omega \in C[0,\infty)\, (\, |f(\omega)| < b\, )
				\end{align}
				が満たされる.すなわち$f$は有界連続であり,$(P_n)_{n=1}^\infty$が$P$に弱収束するので
				\begin{align}
					\lim_{n \to \infty} \int_{C[0,\infty)} f\ dP_n
					= \int_{C[0,\infty)} f\ dP
				\end{align}
				が成立する.
				
			\item[第二段]
				前段の結果より
				\begin{align}
					\left| \int_{C[0,\infty)} f\ dP_n
					- \int_{C[0,\infty)} f\ dP\right|
					\longrightarrow 0 \quad (n \longrightarrow \infty)
				\end{align}
				が成り立つから,
				\begin{align}
					\left|\int_{C[0,\infty)} f_n\ dP_n
					- \int_{C[0,\infty)} f\ dP_n\right|
					\longrightarrow 0 \quad (n \longrightarrow \infty)
					\label{eq:chapter_2_Problem_4_12}
				\end{align}
				が成り立つことを示せば定理の主張が得られる.
				$\{P_n\}_{n=1}^\infty$は相対コンパクトであるからProhorovの定理より緊密である.
				いま$\epsilon$を任意に与えられた正数とすると,$C[0,\infty)$の或るコンパクト部分集合$K$が存在して
				\begin{align}
					\forall n \in \N\, \left(\, P_n(C[0,\infty) \backslash K) < \epsilon\, \right)
				\end{align}
				となる.他方で$K$上で$(f_n)_{n=1}^\infty$は$f$に一様収束するので,
				或る自然数$N$を取れば
				\begin{align}
					\forall n \in \N\, \left(\, N \leq n
					\Longrightarrow \forall \omega \in K\, (\, |f_n(\omega) - f(\omega)| < \epsilon\, )\, \right)
				\end{align}
				が満たされる.このとき
				\begin{align}
					N \leq n \Longrightarrow
					&\left| \int_{C[0,\infty)} f_n\ dP_n
					- \int_{C[0,\infty)} f\ dP_n\right| \\
					&\leq \int_{C[0,\infty)} |f_n - f|\ dP_n \\
					&\leq \int_K |f_n - f|\ dP_n + \int_{C[0,\infty) \backslash K} |f_n - f|\ dP_n \\ \\
					&< \epsilon P_n(K) + 2b P_n(C[0,\infty) \backslash K) \\
					&< (1+2b) \epsilon
				\end{align}
				が成り立つので(\refeq{eq:chapter_2_Problem_4_12})が示された.
				\QED
		\end{description}
	\end{sketch}

\appendix
\chapter{}
\section{集合か位相}
\subsection{Dynkin族定理}
	\begin{screen}
		\begin{dfn}[乗法族・Dynkin族]\label{def:Dynkin_system_theorem}
			集合$X$の部分集合の族$\mathscr{A}$が
			任意の$A,B \in \mathscr{A}$に対し$A \cap B \in \mathscr{A}$を満たすとき
			$\mathscr{A}$を$X$上の乗法族($\pi$-system)という.
			$X$の部分集合の族$\mathscr{D}$が
			\begin{description}
				\item[(D1)] $X \in \mathscr{D}$,
				\item[(D2)] $A,B \in \mathscr{D},
					\ A \subset B \quad \Longrightarrow \quad B \backslash A \in \mathscr{D}$,
				\item[(D3)] $\{A_n\}_{n=1}^\infty \subset \mathscr{D},
					\ A_n \cap A_m = \emptyset\ (n \neq m)
					\quad \Longrightarrow \quad \sum_{n=1}^\infty A_n \in \mathscr{D}$,
			\end{description}
			を満たすとき,$\mathscr{D}$を$X$上のDynkin族(Dynkin system)という.
		\end{dfn}
	\end{screen}
	
	\begin{screen}
		\begin{dfn}[Dynkin族定理]\label{thm:Dynkin_system_theorem}
			集合$X$上の乗法族$\mathscr{A}$に対し,
			$\mathscr{A}$を含む最小のDynkin族を$\delta(\mathscr{A})$と書くとき,
			\begin{align}
				\delta(\mathscr{A}) = \sigma(\mathscr{A}).
			\end{align}
		\end{dfn}
	\end{screen}
	
	\begin{prf}\mbox{}
		\begin{description}
			\item[第一段]
				$\delta(\mathscr{C})$が交演算で閉じていれば
				$\delta(\mathscr{C})$は$\sigma$-加法族となる.実際任意の$A \in \delta(\mathscr{A})$に対し
				\begin{align}
					A^c = X \backslash A \in \delta(\mathscr{A})
				\end{align}
				となるから,$\delta(\mathscr{C})$が交演算で閉じていれば任意の
				$A_n \in \delta(\mathscr{C})\ (n=1,2,\cdots)$に対し
				\begin{align}
					\bigcup_{n=1}^{\infty} A_n
					= \sum_{n=1}^{\infty} A_1^c \cap A_2^c \cap \cdots \cap A_{n-1}^c \cap A_n
					\in \delta(\mathscr{C})
				\end{align}
				が従う.$\sigma$-加法族はDynkin族であるから
				$\sigma(\mathscr{C}) \subset \delta(\mathscr{C})$も成り立ち
				$\sigma(\mathscr{C}) = \delta(\mathscr{C})$が得られる.
			
			\item[第二段]
				$\delta(\mathscr{C})$が交演算について閉じていることを示す.いま,
				\begin{align}
					\mathscr{D}_1 \coloneqq
					\Set{B \in \delta(\mathscr{C})}{ A \cap B \in \delta(\mathscr{C}),\ 
					\forall A \in \mathscr{C}}
				\end{align}
				により定める$\mathscr{D}_1$はDynkin族であり$\mathscr{C}$を含むから
				\begin{align}
					\delta(\mathscr{C}) \subset \mathscr{D}_1
				\end{align}
				が成立する.従って
				\begin{align}
					\mathscr{D}_2 \coloneqq
					\Set{B \in \delta(\mathscr{C})}{ A \cap B \in \delta(\mathscr{C}),\ 
					\forall A \in \delta(\mathscr{C})}
				\end{align}
				によりDynkin族$\mathscr{D}_2$を定めれば,$\mathscr{C} \subset \mathscr{D}_2$が満たされ
				\begin{align}
					\delta(\mathscr{C}) \subset \mathscr{D}_2
				\end{align}
				が得られる.よって$\delta(\mathscr{C})$は交演算について閉じている.
				\QED
		\end{description}
	\end{prf}
	
	\begin{screen}
		\begin{thm}
			集合$X$の部分集合族$\mathscr{D}$が
			の定義\ref{def:Dynkin_system_theorem}の(D1),(D2)を満たしているとき,
			$\mathscr{D}$が(D3)を満たすことと
			$\mathscr{D}$が増大列の可算和で閉じることは同値である.
		\end{thm}
	\end{screen}
	
	\begin{prf}
		$\mathscr{D}$が可算直和について閉じているとする.このとき
		単調増大列$A_1 \subset A_2 \subset \cdots$を取り
		\begin{align}
			B_1 \coloneqq A_1,
			\quad B_n \coloneqq A_n \backslash A_{n-1},
			\quad (n \geq 2)
		\end{align}
		とおけば(D2)より$B_n \in \mathscr{D},\ (\forall n \geq 1)$が満たされ
		\begin{align}
			\bigcup_{n=1}^{\infty} A_n = \sum_{n=1}^{\infty} B_n \in \mathscr{D} 
		\end{align}
		が成立する.逆に$\mathscr{D}$が増大列の可算和で閉じているとする.
		(D1)(D2)より互いに素な$A,B \in \mathscr{D}$に対し
		$A^c \in \mathscr{D}$及び$A^c \cap B^c = A^c \backslash B\in \mathscr{D}$が成り立つから,
		$\mathscr{D}$の互いに素な集合列$(B_n)_{n=1}^{\infty}$を取れば
		\begin{align}
			B_1^c \cap B_2^c \cap \cdots \cap B_n^c
			= \left( \cdots \left( \left( B_1^c \cap B_2^c \right) \cap B_3^c \right) \cap \cdots \cap B_{n-1}^c \right) \cap B_n^c
			\in \mathscr{D},
			\quad (n=1,2,\cdots)
		\end{align}
		が得られる.よって
		\begin{align}
			D_n \coloneqq \bigcup_{i=1}^n B_i = X \backslash \Biggl( \bigcap_{i=1}^n B_i^c \Biggr),
			\quad (n=1,2,\cdots)
		\end{align}
		により$\mathscr{D}$の単調増大列$(D_n)_{n=1}^{\infty}$を定めれば
		\begin{align}
			\sum_{n=1}^{\infty} B_n = \bigcup_{n=1}^{\infty} D_n \in \mathscr{D}
		\end{align}
		が成立する.
		\QED
	\end{prf}

\subsection{上限下限}
	\begin{screen}
		\begin{thm}[上限の冪と冪の上限]\label{thm:exponentiation_of_supremum_supremum_of_exponentiation}
			任意の空でない$S \subset [0,\infty)$と$t > 0$に対し次が成立する:
			\begin{align}
				(\sup{}{S})^t = \sup{}{\Set{s^t}{s \in S}}.
			\end{align}
		\end{thm}
	\end{screen}
	
	\begin{prf}
		$S=\{0\}$なら両辺0で一致するので,$S$は$\{0\}$より真に大きいとする.このとき
		任意の$s \in S$に対し$s^t \leq (\sup{}{S})^t$となるから$\sup{}{\Set{s^t}{s \in S}} \leq (\sup{}{S})^t$が従う.
		また任意の$(\sup{}{S})^t > \alpha > 0$に対し$s > \alpha^{1/t}$を満たす$s \in S$が存在し
		$(\sup{}{S})^t \geq s^t > \alpha$となるから$\sup{}{\Set{s^t}{s \in S}} = (\sup{}{S})^t$が得られる.
		\QED
	\end{prf}

\subsection{位相}
	\begin{screen}
		\begin{dfn}[位相]
			集合$S$の部分集合の族$\mathscr{O}$が
			以下を満たすとき,$\mathscr{O}$を$S$の位相(topology)と呼ぶ:
			\begin{description}
				\item[(O1)] $\emptyset, S \in \mathscr{O}$,
				\item[(O2)] $O_1,O_2 \in \mathscr{O} 
					\quad \Longrightarrow \quad O_1 \cap O_2 \in \mathscr{O}$,
				\item[(O3)] $\displaystyle\mathscr{U} \subset \mathscr{O}
					\quad \Longrightarrow \quad \bigcup \mathscr{U} = 
					\bigcup_{U \in \mathscr{U}} U \in \mathscr{O}$.
			\end{description}
		\end{dfn}
	\end{screen}
	\begin{screen}
		\begin{dfn}[近傍・基本近傍系]
			空でない位相空間$S$において,$x \in S$と$U \subset S$に対し
			\begin{align}
				x \in U^{\mathrm{o}}
			\end{align}
			が満たされるとき$U$は$x$の近傍(neighborhood)であるという.
			同様に$A \subset S$と$V \subset S$に対し
			\begin{align}
				A \subset V^{\mathrm{o}}
			\end{align}
			が満たされるとき,$V$は$A$の近傍であるという.
			点$x$の近傍全体を$\mathscr{V}(x)$と書くとき,
			$S$は$x$の最大の近傍であるから$\mathscr{V}(x)$は空ではない.
			また$\mathscr{V}(x)$の空でない部分集合$\mathscr{U}(x)$が
			\begin{align}
				\forall V \in \mathscr{V}(x),
				\quad \exists U \in \mathscr{U}(x),
				\quad U \subset V
			\end{align}
			を満たすとき,$\mathscr{U}(x)$を$x$の基本近傍系(local base of a point $x$)と呼ぶ.
		\end{dfn}
	\end{screen}
	
	\begin{screen}
		\begin{thm}[基本近傍系は開集合を決定する]\label{thm:local_base_defines_open_sets}
			$S$を空でない位相空間,
			$\mathscr{U}(x)$を点$x$の基本近傍系とすれば
			\begin{align}
				\mbox{$O$が$S$の開集合} \quad \Longleftrightarrow \quad 
				\mbox{$O = \emptyset$,或は任意の$x \in O$に対し
				$U \subset O$を満たす$U \in \mathscr{U}(x)$が存在する}
			\end{align}
			が成立する.すなわち,$\{\mathscr{U}(x)\}_{x \in S}$を基本近傍系とする$S$の位相は唯一つである.
		\end{thm}
	\end{screen}
	
	\begin{prf}
		$O$が開集合なら任意の$x \in O$に対し$O$は$x$の近傍となるから,
		或る$U \in \mathscr{U}(x)$が存在して$U \subset O$を満たす.
		逆に任意の$x \in O$に対し$U \subset O$を満たす$U \in \mathscr{U}(x)$が存在するとき,
		\begin{align}
			x \in U^{\mathrm{o}} \subset O^{\mathrm{o}}
		\end{align}
		となり$O = O^{\mathrm{o}}$が成立するから$O$は開集合である.
		\QED
	\end{prf}
	
	\begin{screen}
		\begin{thm}[基本近傍系は位相を復元する]\mbox{}
			\begin{description}
				\item[(1)] 
					$(S,\mathscr{O})$を空でない位相空間とし,各点
					$x \in S$に対し$\mathscr{U}(x)$を基本近傍系とすれば以下が成り立つ:
					\begin{description}
						\item[(LB1)] $\mathscr{U}(x)$は空ではなく,また任意の$U \in \mathscr{U}(x)$は$x \in U$を満たす.
						\item[(LB2)] 任意の$U,V \in \mathscr{U}(x)$に対し或る$W \in \mathscr{U}(x)$
							が存在して$W \subset U \cap V$を満たす.
						\item[(LB3)] 任意の$U \in \mathscr{U}(x)$に対し或る$V \in \mathscr{U}(x)$が存在し,
							任意の$y \in V$に対し$W_y \subset V$を満たす$W_y \in \mathscr{U}(y)$が取れる.
					\end{description}
				\item[(2)]
					空でない集合$S$の各点$x$に対し(LB1)(LB2)(LB3)を満たす部分集合族$\mathscr{U}(x)$が与えられれば,
					\begin{align}
						\mathscr{O} \coloneqq
						\Set{O \subset S}{\mbox{$O = \emptyset$,或は任意の$x \in O$に対し
						$U \subset O$を満たす$U \in \mathscr{U}(x)$が存在する}}
					\end{align}
					により$S$に位相が定まり,$\{\mathscr{U}(x)\}_{x \in S}$は
					$(S,\mathscr{O})$において基本近傍系となる.
				\item[(3)] 空でない位相空間$(S,\mathscr{O})$から基本近傍系
					$\{\mathscr{U}(x)\}_{x \in S}$を得れば,
					$\{\mathscr{U}(x)\}_{x \in S}$を基本近傍系とする位相
					を(2)の手続きで構成することにより$\mathscr{O}$を復元できる.
			\end{description}
		\end{thm}
	\end{screen}
	
	\begin{prf}\mbox{}
		\begin{description}
			\item[(1)] 任意の$U \in \mathscr{U}(x)$は$x$の近傍であるから
				$(LB1)$が満たされる.また$U,V \in \mathscr{U}(x)$に対し
				\begin{align}
					x \in U^{\mathrm{o}} \cap V^{\mathrm{o}} = (U \cap V)^{\mathrm{o}}
				\end{align}
				となるから$U \cap V$は$x$の近傍であり(LB2)も従う.
				任意の$U \in \mathscr{U}(x)$に対し$V \coloneqq U^{\mathrm{o}}$とおけば,
				$V$は任意の$y \in V$の開近傍となるから(LB3)も得られる.
			
			\item[(2)] 
				$\mathscr{U}(x)$は空ではないから$S \in \mathscr{O}$となる.
				また$O_1,O_2 \in \mathscr{O}$を取れば,
				任意の$x \in O_1 \cap O_2$に対し
				\begin{align}
					x \in U_1 \subset O_1,
					\quad x \in U_2 \subset O_2
				\end{align}
				を満たす$U_1,U_2 \in \mathscr{U}(x)$が存在し,
				(LB2)より或る$U_3 \in \mathscr{U}(x)$に対して
				\begin{align}
					U_3 \subset U_1 \cap U_2 \subset O_1 \cap O_2
				\end{align}
				が成り立つから$O_1 \cap O_2 \in \mathscr{O}$となる.
				任意に$\mathscr{G} \subset \mathscr{O}$を取れば
				任意の$x \in \bigcup \mathscr{G}$は或る$G \in \mathscr{G}$の点であるから,
				\begin{align}
					U \subset G \subset \bigcup \mathscr{G}
				\end{align}
				を満たす$U \in \mathscr{U}(x)$が存在し$\bigcup \mathscr{G} \in \mathscr{O}$が従う.
				よって$\mathscr{O}$は位相である.
				また(LB3)の$V$は$\mathscr{O}$の元であり
				\begin{align}
					x \in V \subset U^{\mathrm{o}}
				\end{align}
				が成り立つから任意の$U \in \mathscr{U}(x)$は$x$の近傍である.
				そして$W$を$x$の任意の近傍とすれば,$\mathscr{O}$の定め方より或る$U \in \mathscr{U}(x)$が
				$U \subset W^{\mathrm{o}}$を満たすから$\mathscr{U}(x)$は$x$の基本近傍系である.
			
			\item[(3)] 
				定理\ref{thm:local_base_defines_open_sets}より
				$\{\mathscr{U}(x)\}_{x \in S}$を基本近傍系とする位相は唯一つであるから
				主張が従う.
				\QED
		\end{description}
	\end{prf}
	
	\begin{screen}
		\begin{dfn}[相対位相]
			$(S,\mathscr{O})$を位相空間,$M \subset S$を部分集合,
			$i:M \longrightarrow S$を恒等写像とするとき,
			\begin{align}
				\mathscr{O}_M \coloneqq 
				\Set{i^{-1}(O) = O \cap M}{O \in \mathscr{O}}
			\end{align}
			で定まる$i$による$\mathscr{O}$の引き戻しを$M$の相対位相(relative topology)と呼ぶ.
		\end{dfn}
	\end{screen}
	
	\begin{screen}
		\begin{thm}[位相の生成]
			$S$を集合,$\mathcal{P}(S)$を冪集合として
			任意に$M \subset \mathcal{P}(S)$を取り
			\begin{align}
				\mathscr{A} \coloneqq
				\Set{\bigcap_{i=1}^n I_i}{I_i \in M,\ n = 1,2,\cdots}
			\end{align}
			とおくとき,$M$を含む最小の位相は
			\begin{align}
				\mathscr{O} \coloneqq
				\Set{\bigcup \Lambda}{\Lambda \subset \mathscr{A}}
				\cup \{S\}
			\end{align}
			で与えられる.この$\mathscr{O}$を$M$が生成する$S$の位相と呼ぶ.
		\end{thm}
	\end{screen}
	
	\begin{prf}
		$\mathscr{O}$は定め方より$S$と$\emptyset$を含む.また
		任意の$O_1 = \bigcup \Lambda_1,\ O_2=\bigcup \Lambda_2 \in \mathscr{O},\ 
		(\Lambda_1,\Lambda_2 \subset \mathscr{A})$に対し
		\begin{align}
			\Set{I \cap J}{I \in \Lambda_1,\ J \in \Lambda_2} \subset \mathscr{A}
		\end{align}
		となるから
		\begin{align}
			O_1 \cap O_2 = \bigcup_{I \in \Lambda_1,\ J \in \Lambda_2} I \cap J \in \mathscr{O}
		\end{align}
		が成立する.任意に$\emptyset \neq \mathscr{U} \subset \mathscr{O}$を取れば,
		各$U \in \mathscr{U}$に$U = \bigcup \Lambda_U$を満たす
		$\Lambda_U \subset \mathscr{A}$が対応し,このとき
		\begin{align}
			\bigcup_{U \in \mathscr{U}} \Lambda_U \subset \mathscr{A}
		\end{align}
		となるから
		\begin{align}
			\bigcup \mathscr{U} = \bigcup \Biggl(\bigcup_{U \in \mathscr{U}} \Lambda_U\Biggr)
			\in \mathscr{O}
		\end{align}
		が従う.$M$を含む任意の位相は$\mathscr{A}$を含みかつその任意和で閉じるから$\mathscr{O}$を含む.
		\QED
	\end{prf}
	
	\begin{screen}
		\begin{dfn}[始位相]
			$f \in \mathscr{F}$を集合$S$から位相空間$(T_f,\mathscr{O}_f)$への写像とするとき,
			全ての$f \in \mathscr{F}$を連続にする最弱の位相を$S$の$\mathscr{F}$-始位相
			(initial topology)と呼ぶ.$\mathscr{F}$-始位相は次が生成する位相である:
			\begin{align}
				\bigcup_{f \in \mathscr{F}} \Set{f^{-1}(O)}{O \in \mathscr{O}_f}.
			\end{align}
		\end{dfn}
	\end{screen}
	
\subsection{分離公理}
	\begin{screen}
		\begin{dfn}[位相的に識別可能・分離]
			$S$を位相空間とする.
			\begin{itemize}
				\item $x,y \in S$に対し$x \notin \overline{\{y\}}$
					或は$y \notin \overline{\{x\}}$が満たされるとき,
					$x$と$y$は位相的に識別可能(topologically distinguishable)であるという.
				\item $A,B \subset S$に対し$\overline{A} \cap B = \emptyset$
					或は$A \cap \overline{B} = \emptyset$が満たされるとき,
					$A$と$B$は分離される(separeted)という.点と点,点と集合の分離は一点集合を考える.
				\item $A,B \subset S$が近傍で分離される(separated by neighborhoods)とは,
					$A,B$が互いに交わらない近傍を持つことをいう.
				\item 閉集合$A,B \subset S$が関数で分離される(separated by a function)とは,
					或る連続関数$f:S \longrightarrow [0,1]$によって$f(A) = \{0\},\ f(B) = \{1\}$
					が満たされることをいう.
				\item 閉集合$A,B \subset S$が関数でちょうど分離される
					(precisely separated by a function)とは,
					或る連続関数$f:S \longrightarrow [0,1]$によって
					$A = f^{-1}(\{0\}),\ B = f^{-1}(\{1\})$が満たされることをいう.
			\end{itemize}
		\end{dfn}
	\end{screen}
	
	\begin{screen}
		\begin{thm}[位相的に識別可能な二点は相異なる]
			$S$を位相空間とするとき,任意の$x,y \in S$に対し
			\begin{align}
				\mbox{$x$と$y$が位相的に識別可能} \quad \Longrightarrow \quad
				x \neq y .
			\end{align}
		\end{thm}
	\end{screen}
	
	\begin{prf}
		$x = y$なら$\overline{\{x\}} = \overline{\{y\}}$となる.
		後述の$T_0$空間とは,この逆が満たされる位相空間である.
		\QED
	\end{prf}
	
	\begin{screen}
		\begin{thm}[分離される集合は他方を含まない近傍を持つ]
		\label{thm:the_equivalent_condition_of_separatedness}
			位相空間$S$において,$A,B \subset S$が分離されることと
			\begin{align}
				A \subset U,\quad B \subset V,\quad 
				A \cap V = \emptyset,
				\quad B \cap U = \emptyset
				\label{eq:thm_the_equivalent_condition_of_separatedness}
			\end{align}
			を満たす開集合$U,V$が存在することは同値である.
		\end{thm}
	\end{screen}
	
	\begin{prf}
		$A,B \subset S$が分離されるとき,$U \coloneqq \overline{B}^c,\ V \coloneqq \overline{A}^c$
		とおけば(\refeq{eq:thm_the_equivalent_condition_of_separatedness})が成立する.
		逆に$A,B$に対し(\refeq{eq:thm_the_equivalent_condition_of_separatedness})を満たす
		開集合$U,V$が存在するとき,$\closure{A} \subset V^c \subset B^c$及び
		$\closure{B} \subset U^c \subset A^c$となるから$A,B$は分離される.
		\QED
	\end{prf}
	
	\begin{screen}
		\begin{dfn}[分離公理]\mbox{}
			\begin{itemize}
				\item 任意の二点が位相的に識別可能である位相空間を$T_0$空間,或はKolmogorov空間という.
				\item 任意の二点が分離される位相空間を$T_1$空間という.
				\item 任意の二点が互いに交わらない近傍を持つ位相区間を$T_2$空間,或はHausdorff空間という.
				\item 任意の交わらない点(一点集合)と閉集合が近傍で分離される位相空間を
					正則(regular)空間という.
				\item $T_0$かつ正則な位相空間を$T_3$空間,或は正則Hausdorff空間という.
				\item 任意の交わらない二つの閉集合が近傍で分離される位相空間を正規(normal)空間という.
				\item $T_1$かつ正規な位相空間を$T_4$空間,或は正規Hausdorff空間という.
				\item 任意の部分位相空間が正規である位相空間は全部分正規(completely normal)であるという.
				\item $T_1$かつ全部分正規な位相空間を$T_5$空間,或は全部分正規Hausdorff空間という.
				\item 任意の交わらない二つの閉集合が関数でちょうど分離される位相空間は完全正規(perfectly normal)であるという.
				\item $T_1$かつ完全正規な位相空間を$T_6$空間,或は完全正規Hausdorff空間という.
			\end{itemize}
		\end{dfn}
	\end{screen}
	
	\begin{screen}
		\begin{thm}[$T_1$空間とは一点集合が閉である空間]
			位相空間$S$に対し,
			\begin{align}
				\mbox{$S$が$T_1$}
				&\quad \Longleftrightarrow \quad \mbox{$S$は$T_0$かつ位相的に識別可能な任意の二点が分離される} \\
				&\quad \Longleftrightarrow \quad \mbox{$S$の任意の一点集合は閉} \\
				&\quad \Longleftrightarrow \quad \mbox{$x \in S$が$A \subset S$の集積点であることと$x$の任意の近傍が$A$と交わることは同値}.
			\end{align}
		\end{thm}
	\end{screen}
	
	\begin{screen}
		\begin{thm}[Hausdorff空間のコンパクト部分集合は閉]
			Hausdorff空間のコンパクト部分集合は閉である.
		\end{thm}
	\end{screen}
	
	\begin{prf}
		$S$をHausdorff空間,$K \subset S$をコンパクト部分集合とするとき,
		任意に$x \in S \backslash K,\ y \in K$を取れば
		\begin{align}
			x \in U_y,\quad y \in V_y, \quad U_y \cap V_y = \emptyset
		\end{align}
		を満たす開集合$U_y,V_y$が取れる.或る$\{y_i\}_{i=1}^n \subset K$に対し
		$K \subset \bigcup_{i=1}^n V_{y_i}$となるから,
		$U \coloneqq \bigcap_{i=1}^n U_{y_i}$とおけば
		\begin{align}
			x \in U,\quad U \subset \bigcap_{i=1}^n \left(S\backslash V_{y_i}\right)
			\subset S \backslash K
		\end{align}
		が成立する.従って$S \backslash K$は開集合であり,$K$は閉集合である.
		\QED
	\end{prf}
	
	\begin{screen}
		\begin{thm}[Hausdorff空間においてコンパクト集合の閉部分集合はコンパクト]
			$S$をHausdorff空間,$K \subset S$をコンパクト部分集合,$F \subset S$を閉集合とするとき,
			$K \cap F$はコンパクトである.
		\end{thm}
	\end{screen}
	
	\begin{prf}
		$K \cap F$の任意の開被覆に$S \backslash F$を加えれば
		$K$の開被覆となるから,そのうち$K$の有限被覆を取ることができる.
		$S \backslash F$を除けば$K \cap F$の有限被覆が残り
		$K \cap F$のコンパクト性が出る.
		\QED
	\end{prf}
	
	\begin{screen}
		\begin{thm}[Hausdorff空間とは交わらない二つのコンパクト集合が近傍で分離される空間]
		\label{thm:Hausdorff_space_two_disjoint_compact_sets_are_separated_by_nbh}
			位相空間において,Hausdorff性と,交わらない二つのコンパクト集合が近傍で分離されることは同値である.
		\end{thm}
	\end{screen}
	
	\begin{prf}
		$A,B$をHausdorff空間の交わらないコンパクト集合とするとき,
		任意の$p \in A$に対し
		\begin{align}
			p \in U_p,\quad B \subset V_p,\quad U_p \cap V_p = \emptyset
			\label{eq:thm_Hausdorff_space_two_disjoint_compact_sets_are_separated_by_nbh_1}
		\end{align}
		を満たす開集合$U_p,V_p$が存在する.実際
		任意の$q \in B$に対し
		\begin{align}
			p \in U_p(q),\quad q \in V_p(q),\quad U_p(q) \cap U_p(q) = \emptyset
		\end{align}
		を満たす開集合$U_p(q), U_p(q)$が取れ,$B$のコンパクト性より
		或る$\{q_i\}_{i=1}^n \subset B$で$B \subset \bigcup_{i=1}^n U_p(q_i)$となるから,
		\begin{align}
			U_p \coloneqq \bigcap_{i=1}^n U_p(q_i),
			\quad V_p \coloneqq \bigcup_{i=1}^n V_p(q_i)
		\end{align}
		とおけば(\refeq{eq:thm_Hausdorff_space_two_disjoint_compact_sets_are_separated_by_nbh_1})
		が成立する.$A$のコンパクト性より或る$\{p_j\}_{j=1}^m \subset A$で
		$A \subset \bigcup_{j=1}^m U_{p_j}$となるから,
		\begin{align}
			U \coloneqq \bigcup_{j=1}^m U_{p_j},
			\quad V \coloneqq \bigcap_{j=1}^m V_{p_j}
		\end{align}
		とおけば$A$と$B$は$U,V$により分離される.
		逆の主張は一点集合がコンパクトであることより従う.
		\QED
	\end{prf}
	
	\begin{screen}
		\begin{thm}[Hausdorff空間値連続写像の等価域は閉]
			$S$を位相空間,$T$をHausdorff空間,$f,g$を
			$S$から$T$への連続写像とするとき,$E \coloneqq \Set{x \in S}{f(x) = g(x)}$は$S$で閉じている.
			特に,$E$が$X$で稠密なら$f=g$となる.
		\end{thm}
	\end{screen}
	
	\begin{prf}
		任意に$x \in \Set{x \in S}{f(x) \neq g(x)}$を取れば,Hausdorff性より
		\begin{align}
			f(x) \in A,\quad g(x) \in B,\quad A \cap B = \emptyset
		\end{align}
		を満たす$T$の開集合$A,B$が存在する.
		$f^{-1}(A) \cap g^{-1}(B)$は$x$の開近傍であり,
		\begin{align}
			f^{-1}(A) \cap g^{-1}(B) \subset \Set{x \in S}{f(x) \neq g(x)}
		\end{align}
		となるから$\Set{x \in S}{f(x) \neq g(x)}$は$S$の開集合である.
		従って$E$は閉である.
		\QED
	\end{prf}
	
	\begin{screen}
		\begin{thm}[正則空間とは交わらないコンパクト集合と閉集合が近傍で分離できる空間]
		\label{thm:each_point_in_regular_space_has_closesd_local_base}\mbox{}
			\begin{description}
				\item[(1)] 位相空間において,正則性と,交わらないコンパクト集合と閉集合が近傍で分離されることは同値である.
					
				\item[(2)]
					$K,W$をそれぞれ局所コンパクトな正則空間のコンパクト集合,開集合とするとき,
					閉包がコンパクトな開集合$U$が存在して次を満たす:
					\begin{align}
						K \subset U \subset \overline{U} \subset W.
						\label{eq:thm_each_point_in_regular_space_has_closesd_local_base}
					\end{align}
			\end{description}
		\end{thm}
	\end{screen}
	
	\begin{prf}\mbox{}
		\begin{description}
			\item[(1)]
				$K,F$を正則空間のコンパクト集合,閉集合とするとき,
				$K \cap F = \emptyset$なら任意の点$x \in K$に対して
				\begin{align}
					x \in U_x,\ \quad F \subset V_x,
					\quad U_x \cap V_x = \emptyset
				\end{align}
				を満たす開集合$U_x,V_x$が取れる.
				$K$はコンパクトであるから或る$\{x_i\}_{i=1}^n \subset K$で
				$K \subset \bigcup_{i=1}^n U_{x_i}$となり
				\begin{align}
					K \subset U \coloneqq \bigcup_{i=1}^n U_{x_i},
					\quad F \subset V \coloneqq \bigcap_{i=1}^n V_{x_i},
					\quad U \cap V = \emptyset
				\end{align}
				が成立する.逆の主張は一点集合がコンパクトであることにより従う.
			\item[(2)]
				任意の$x \in K$に対し,$\overline{U_x} \subset W$
				となる開近傍$U_x$と閉包がコンパクトな開近傍$C_x$が存在するから,
				\begin{align}
					K \subset (C_{y_1} \cap U_{y_1}) \cup \cdots \cup (C_{y_m} \cap U_{y_m})
				\end{align}
				を満たす$\{y_i\}_{i=1}^m \subset K$に対し
				$U \coloneqq \bigcup_{i=1}^m C_{y_i} \cap U_{y_i}$
				とおけば,$\overline{U}$はコンパクトであり
				(\refeq{eq:thm_each_point_in_regular_space_has_closesd_local_base})を満たす.
				\QED
		\end{description}
	\end{prf}
	
	\begin{screen}
		\begin{thm}[局所コンパクトなら$T_2$と$T_3$は同値]
		\label{thm:T_2_equals_to_T_3_in_locally_compact_spaces}
			局所コンパクト位相空間において,$T_2 \Longleftrightarrow T_3$である.
		\end{thm}
	\end{screen}
	
	\begin{prf}
		$T_3$ならば$T_2$であるから$\Longleftarrow$を得る.
		逆に$S$を局所コンパクトHausdorff空間とし,点$x$と閉集合$F$が$x \notin F$を満たしているとする.
		$x$のコンパクトな近傍$K$を取れば,Hausdorff性より$K \cap F$はコンパクトであるから
		\begin{align}
			U_0 \cap V_0 = \emptyset, \quad x \in U_0,  \quad K \cap F \subset V_0
		\end{align}
		を満たす開集合$U_0,V_0$が存在する.このとき,
		\begin{align}
			U \coloneqq U_0 \cap K^{\mathrm{o}},
			\quad V \coloneqq V_0 \cup (S \backslash K)
		\end{align}
		により開集合$U,V$を定めれば
		\begin{align}
			U \cap V = \emptyset,
			\quad x \in U,
			\quad F \subset V
		\end{align}
		が成立し,$S$の正則性が出る.$S$は$T_0$空間でもあるから$T_3$である.
		\QED
	\end{prf}
	
	\begin{screen}
		\begin{thm}[正規空間とは交わらない二つの閉集合が関数で分離される空間(Urysohnの補題)]
		\label{thm:Urysohn_lemma}
			位相空間において,正規性と,任意の交わらない二つの閉集合が関数で分離されることは同値である.
		\end{thm}
	\end{screen}
	
	\begin{screen}
		\begin{dfn}[$G_\delta$集合・$F_\sigma$集合]
			位相空間の部分集合で,開集合の可算交叉で表されるものを$G_\delta$集合,
			閉集合の可算和で表されるものを$F_\sigma$集合と呼ぶ.
			特に,任意の閉集合が$G_\delta$である空間では任意の開集合が$F_\sigma$となる.
		\end{dfn}
	\end{screen}
	
	\begin{screen}
		\begin{thm}[完全正規空間とは正規かつ閉集合が全て$G_\delta$である空間]
		\label{thm:perfectly_normal_Hausdorff_is_normal_and_closed_is_G_delta}\mbox{}
			\begin{description}
				\item[(1)]
					$F$を完全正規空間の閉集合とすれば,次を満たす閉集合系$(F_n)_{n=1}^\infty$が存在する:
					\begin{align}
						F = \bigcap_{n=1}^\infty F_n,
						\quad F_n^{\mathrm{o}} \supset F_{n+1}. 
					\end{align}
					
				\item[(2)]
					位相空間において,完全正規であることと,正規かつ任意の閉集合が$G_\delta$であることは同値である.
			\end{description}
		\end{thm}
	\end{screen}
	
	\begin{prf}
		$S$を完全正規空間,$A,B$を互いに交わらない$S$の閉集合とすれば,
		$A=f^{-1}(\{0\}),\ B = f^{-1}(\{1\})$を満たす連続関数
		$f:S \longrightarrow \R$が存在する.このとき
		$U \coloneqq f^{-1}([0,1/2)),\ V \coloneqq f^{-1}((1/2,1])$
		で開集合$U,V$を定めれば
		\begin{align}
			A \subset U,\quad B \subset V,\quad U \cap V = \emptyset
		\end{align}
		となるから$S$は正規である.また$F$を閉集合とすれば
		或る連続関数$g:S \longrightarrow \R,\ (\emptyset = g^{-1}(\{1\}))$により
		\begin{align}
			F = g^{-1}(\{0\}) 
			= g^{-1}\Biggl(\bigcap_{n=1}^\infty\left[0,n^{-1}\right)\Biggr)
			= \bigcap_{n=1}^\infty g^{-1}\left(\left[0,n^{-1}\right)\right)
		\end{align}
		が成立するから$F$は$G_\delta$である.特に,このとき
		$F_n \coloneqq g^{-1}\left(\left[0,n^{-1}\right]\right)$とおけば
		\begin{align}
			F = \bigcap_{n=1}^\infty g^{-1}\left(\left[0,n^{-1}\right]\right)
			= \bigcap_{n=1}^\infty F_n,
			\quad F_n^{\mathrm{o}} \supset g^{-1}\left(\left[0,n^{-1}\right)\right)
			\supset g^{-1}\left(\left[0,(n+1)^{-1}\right]\right)
			= F_{n+1}
		\end{align}
		となり(1)の主張が得られる.逆に$S$が正規かつ
		閉集合が全て$G_\delta$であるとき,任意の交わらない閉集合$A,B$に対し
		$A = \bigcap_{n=1}^\infty U_n,\ B = \bigcap_{n=1}^\infty V_n$
		を満たす開集合系$(U_n)_{n=1}^\infty,\ (V_n)_{n=1}^\infty$が取れて,
		定理\ref{thm:Urysohn_lemma}より各$n \geq 1$で
		\begin{align}
			f_n(A) = \{0\},\quad f_n(S \backslash U_n) = \{1\},
			\quad g_n(B) = \{0\},\quad g_n(S \backslash V_n) = \{1\}
		\end{align}
		を満たす連続写像$f_n,g_n:S \longrightarrow [0,1]$が存在する.
		ここで連続写像を$f \coloneqq \sum_{n=1}^\infty 2^{-n} f_n,\ 
		g \coloneqq \sum_{n=1}^\infty 2^{-n} g_n$で定めれば
		\begin{align}
			\begin{cases}
				f(x) = 0, & (x \in A), \\
				f(x) > 0, & (x \notin A),
			\end{cases}
			\quad \begin{cases}
				g(x) = 0, & (x \in B), \\
				g(x) > 0, & (x \notin B),
			\end{cases}
		\end{align}
		となり,$h \coloneqq f/(f+g)$とおけば$A = h^{-1}(\{0\}),\ B = h^{-1}(\{1\})$が成立する.
		従って$S$は完全正規である.
		\QED
	\end{prf}
	
	\begin{screen}
		\begin{thm}[連続な単射の引き戻しによる分離性の遺伝]
			$S,T$を位相空間とする.$S$から$T$への連続単射が存在するとき,
			$T$が$T_k$-空間$(k=0,1,\cdots,6)$なら
			$S$もまた$T_k$-空間となる.
		\end{thm}
	\end{screen}
	
	\begin{prf}
		任意に異なる二点$s_1,s_2 \in S$を取れば単射性より$f(s_1) \neq f(s_2)$となる.
		$T$の分離性より
	\end{prf}
	
\subsection{可算公理}
	\begin{screen}
		\begin{thm}[可算コンパクト性の同値条件]
		\end{thm}
	\end{screen}
	
	\begin{screen}
		\begin{thm}[第二可算空間の任意の基底は可算基を内包する]\label{thm:countable_base_of_second_countable_space}
			$\mathscr{B}$を第二可算空間$S$の任意の基底とするとき,或る可算部分集合
			$\mathscr{B}_0 \subset \mathscr{B}$もまた$S$の基底となる.
			すなわち第二可算空間はLindel\Ddot{o}f性を持つ.
		\end{thm}
	\end{screen}
	
	\begin{prf}
		$\mathscr{D}$を$S$の可算基とする.
		任意の開集合$U$に対し或る$\mathscr{B}_U \subset \mathscr{B}$が存在して
		$U = \bigcup_{V \in \mathscr{B}_U}V$を満たすから,
		\begin{align}
			\mathscr{D}_U \coloneqq
			\Set{W \in \mathscr{D}}{W \subset V,\ V \in \mathscr{B}_U}
			\label{eq:thm_countable_base_of_second_countable_space_1}
		\end{align}
		とおけば$U = \bigcup_{V \in \mathscr{B}_U} V
			= \bigcup_{V \in \mathscr{B}_U} \bigcup_{\substack{W \in \mathscr{D}_U \\ W \subset V}} W
			\subset \bigcup_{W \in \mathscr{D}_U} W
			\subset U$より
		\begin{align}
			U = \bigcup_{W \in \mathscr{D}_U} W
			\label{eq:thm_countable_base_of_second_countable_space_2}
		\end{align}
		が成り立つ.ここで(\refeq{eq:thm_countable_base_of_second_countable_space_1})より
		任意の$W \in \mathscr{D}_U$に対して
		$\Set{V \in \mathscr{B}}{W \subset V} \neq \emptyset$であるから
		\begin{align}
			\Phi_U \in \prod_{W \in \mathscr{D}_U} \Set{V \in \mathscr{B}}{W \subset V}
		\end{align}
		が取れる.$\mathscr{B}_U' \coloneqq \Set{\Phi_U(W)}{W \in \mathscr{D}_U}$とすれば
		$U = \bigcup_{W \in \mathscr{D}_U} W \subset \bigcup_{W \in \mathscr{D}_U} \Phi(W)
		\subset \bigcup_{V \in \mathscr{B}_U'} V \subset U$より
		\begin{align}
			U = \bigcup_{V \in \mathscr{B}_U'} V
			\label{eq:thm_countable_base_of_second_countable_space_3}
		\end{align}
		が満たされ,
		\begin{align}
			\mathscr{B}_0 \coloneqq \bigcup_{W \in \mathscr{D}} \mathscr{B}_W'
		\end{align}
		と定めれば$\mathscr{B}_0$は求める$S$の可算基となる.実際,任意の開集合$U$に対し
		(\refeq{eq:thm_countable_base_of_second_countable_space_2})と
		(\refeq{eq:thm_countable_base_of_second_countable_space_3})より
		\begin{align}
			U = \bigcup_{W \in \mathscr{D}_U} W
			= \bigcup_{W \in \mathscr{D}_U} \bigcup_{V \in \mathscr{B}_W'} V
		\end{align}
		となる.
		\QED
	\end{prf}
	
	\begin{screen}
		\begin{thm}[局所コンパクトHausdorff空間が第二可算なら$\sigma$-コンパクト]\label{thm:second_countable_Hausdorff_sigma_compact}
			$S$が第二可算性をもつ局所コンパクトHausdorff空間なら,
			次を満たすコンパクト部分集合の列$(K_n)_{n=1}^\infty$が存在する:
			\begin{align}
				K_n \subset K_{n+1}^{\mathrm{o}},
				\quad S = \bigcup_{n=1}^\infty K_n.
			\end{align}
		\end{thm}
	\end{screen}
	
	\begin{prf}
		任意の$x \in S$に対して閉包がコンパクトな開近傍$U_x$を取っておく.
		$\mathscr{O}$を$S$の開集合系として
		\begin{align}
			\mathscr{B} \coloneqq
			\Set{U \in \mathscr{O}}{\mbox{$\overline{U}$がコンパクト}}
		\end{align}
		とおけば,$\mathscr{B}$は$\mathscr{O}$の基底となる.実際,
		任意の$O \in \mathscr{O}$に対し$O \cap U_x \in \mathscr{B}$かつ
		\begin{align}
			O = \bigcup_{x \in O} O \cap U_x
		\end{align}
		となる.従って定理\ref{thm:countable_base_of_second_countable_space}より
		或る可算部分集合$\{U_n\}_{n=1}^\infty \subset \mathscr{B}$が
		$\mathscr{O}$の基底となる.いま,$K_1 \coloneqq \overline{U_1}$として,
		またコンパクト集合$K_n$が選ばれたとして,
		$K_n$の有限被覆$\mathscr{U}_n \subset \mathscr{B}_0$を取り
		\begin{align}
			K_{n+1} \coloneqq \overline{U_{n+1}} \cup \bigcup_{V \in \mathscr{U}_n} \overline{V}
		\end{align}
		とすれば,$K_{n+1}$はコンパクトであり$K_n \subset K_{n+1}^{\mathrm{o}}$を満たす.
		この操作で$(K_n)_{n=1}^\infty$を構成すれば
		\begin{align}
			S = \bigcup_{n=1}^\infty U_n \subset \bigcup_{n=1}^\infty K_n \subset S
		\end{align}
		が成立する.
		\QED
	\end{prf}
	
\subsection{距離空間}
	\begin{screen}
		\begin{thm}[距離関数の連続性]
			$(x,y) \longmapsto d(x,y)$は直積位相に関し連続である.
			$x \longmapsto d(x,A)$は連続である.
		\end{thm}
	\end{screen}
	
	\begin{screen}
		\begin{thm}[距離空間の完全正規性]
			任意の距離空間は,その距離で導入する位相により$T_6$空間となる.
		\end{thm}
	\end{screen}
	
	\begin{prf}
		$(S,d)$を距離空間とし距離位相を導入すれば,$S$はHausdorffとなる.
		実際相異なる二点$x,y$に対し
		\begin{align}
			B_\epsilon(x) \coloneqq \Set{s \in S}{d(s,x) < \frac{\epsilon}{2}},
			\quad B_\epsilon(y) \coloneqq \Set{s \in S}{d(s,y) < \frac{\epsilon}{2}},
			\quad (\epsilon \coloneqq d(x,y))
		\end{align}
		で交わらない開球を定めれば,$x,y$は
		これらで分離される.また$A,B$を交わらない閉集合として
		\begin{align}
			f(x) \coloneqq \frac{d(x,A)}{d(x,A) + d(x,B)},
			\quad (\forall x \in S)
		\end{align}
		により連続写像$f:S \longrightarrow \R$を定めれば,
		$A = f^{-1}(\{0\}),\ B = f^{-1}(\{1\})$となるから$S$は完全正規である.
		\QED
	\end{prf}
	
\subsection{範疇定理}
	\begin{screen}
		\begin{thm}[Cantorの共通部分定理]\label{thm:Cantor_intersection_theorem}
			$S$をHausdorff空間とし,
			$(K_n)_{n=1}^\infty$をコンパクト部分集合の列とする.
			このとき,任意の$n \geq 1$に対して$\bigcap_{i=1}^n K_i \neq \emptyset$なら
			$\bigcap_{i=1}^\infty K_i \neq \emptyset$が成り立つ.
		\end{thm}
	\end{screen}
	
	\begin{prf}
		$\bigcap_{i=1}^\infty K_i = \emptyset$と仮定すれば,
		$K_1 \subset \bigcup_{n=1}^\infty K_n^c = S$と$K_1$のコンパクト性より
		\begin{align}
			K_1 \subset \bigcup_{n=1}^N K_n^c = \Biggl( \bigcap_{n=1}^N K_n \Biggr)^c
		\end{align}
		を満たす$N \geq 1$が存在し,$\bigcap_{n=1}^N K_n \subset K_1$より$\bigcap_{n=1}^N K_n = \emptyset$が従う.
		\QED
	\end{prf}
	
	\begin{screen}
		\begin{dfn}[疎集合・第一類集合・第二類集合]
			位相空間$S$の部分集合$A$が疎である(nowhere dense)とは
			$A$の閉包の内核が$\overline{A}^{\mathrm{o}} = \emptyset$を満たすことをいう.
			$S$が可算個の疎集合の合併で表せるとき$S$を第一類集合(the set of the first category)と呼び,
			そうでない場合はこれを第二類集合と呼ぶ.
		\end{dfn}
	\end{screen}
	
	\begin{screen}
		\begin{thm}[Baireの範疇定理]\label{thm:Baire_category_theorem}
			空でない完備距離空間と局所コンパクトHausdorff空間は第二類集合である.
		\end{thm}
	\end{screen}
	
	\begin{prf} $S \neq \emptyset$を完備距離空間,或は局所コンパクトHausdorff空間とする.\mbox{}
		\begin{description}
			\item[第一段]
				$(V_n)_{n=1}^\infty$を$S$で稠密な開集合系とするとき
				\begin{align}
					\overline{\bigcap_{n=1}^\infty V_n} = S,
					\label{eq:thm_Baire_category_theorem_1}
				\end{align}
				となることを示す.実際(\refeq{eq:thm_Baire_category_theorem_1})が満たされていれば,
				任意の疎集合系$(E_n)_{n=1}^\infty$に対して
				\begin{align}
					V_n \coloneqq \overline{E_n}^c,
					\quad n=1,2,\cdots
				\end{align}
				で開集合系$(V_n)$を定めると定理\ref{thm:topology_note_closure_interior}より
				\begin{align}
					\overline{V_n} = \overline{E_n}^{ca} = \overline{E_n}^{ic} = \emptyset^c = S
				\end{align}
				となるから,$\bigcap_{n=1}^\infty V_n \neq \emptyset$が従い
				$S \neq \bigcup_{n=1}^\infty \overline{E_n} \supset \bigcup_{n=1}^\infty E_n$
				が成り立つ.従って$S$は第二類である.
				
			\item[第二段]
				任意の空でない開集合$B_0$に対し$B_0 \cap \left( \bigcap_{n=1}^\infty V_n \right) \neq \emptyset$
				となることを示せば(\refeq{eq:thm_Baire_category_theorem_1})が従う.
				$V_1$は稠密であるから$B_0 \cap V_1 \neq \emptyset$となり,
				点$x_1 \in B_0 \cap V_1$を取れば,
				$S$が距離空間なら或る半径$<1$の開球$B_1$が存在して
				\begin{align}
					x_1 \in B_1 \subset \overline{B_1} \subset B_0 \cap V_1
					\label{eq:thm_Baire_category_theorem_2}
				\end{align}
				を満たす.$S$が局所コンパクトHausdorffの場合も,
				定理\ref{thm:each_point_in_regular_space_has_closesd_local_base}と
				定理\ref{thm:T_2_equals_to_T_3_in_locally_compact_spaces}より
				(\refeq{eq:thm_Baire_category_theorem_2})を満たす
				相対コンパクトな開集合$B_1$が取れる.
				同様に半径$<1/n$の開球,或は相対コンパクトな開集合$B_n$と$x_n \in S$で
				\begin{align}
					x_n \in B_n \subset \overline{B_n} \subset B_{n-1} \cap V_n
				\end{align}
				を満たすものが存在する.このとき$S$が完備距離空間なら$(x_n)_{n=1}^\infty$は
				Cauchy列をなし,その極限点$x_\infty$は
				\begin{align}
					x_\infty \in \bigcap_{n=1}^\infty \overline{B_n}
				\end{align}
				を満たす.$S$が局所コンパクトHausdorff空間なら定理\ref{thm:Cantor_intersection_theorem}より
				\begin{align}
					\bigcap_{n=1}^\infty \overline{B_n} \neq \emptyset
				\end{align}
				となるから,いずれの場合も
				\begin{align}
					\emptyset \neq \bigcap_{n=1}^\infty \overline{B_n} 
					\subset B_0 \cap \Biggl( \bigcap_{n=1}^\infty V_n \Biggr)
				\end{align}
				が従い定理の主張が得られる.
				\QED
		\end{description}
	\end{prf}
	
	\begin{screen}
		\begin{thm}[閉包・内核]\label{thm:topology_note_closure_interior}
			$S$を位相空間,$h:S \longrightarrow S$を同相,$A$を$S$の部分集合とするとき次が成り立つ.
			\begin{description}
				\item[(1)] $A^{ic} = A^{ca}$.
				\item[(2)] $h(A^a) = h(A)^a$.
				\item[(3)] $h(A^i) = h(A)^i$.
			\end{description}
		\end{thm}
	\end{screen}
	
	\begin{prf}\mbox{}
		\begin{description}
			\item[(1)]
				$A^i \subset A$より$A^{ic} \supset A^c$が従い,
				$A^{ic}$が閉であるから$A^{ic} \supset A^{ca}$となる.
				一方で$A^c \subset A^{ca}$より$A \supset A^{cac}$が従い,
				$A^{cac}$は開であるから$A^i \supset A^{cac}$すなわち
				$A^{ic} \subset A^{ca}$となる.
			
			\item[(2)]
				$h(A) \subset h(A^a)$かつ$h(A^a)$は閉であるから$h(A)^a \subset h(A^a)$が従う.一方で
				任意の$x \in h(A^a)$に対し$x = h(y)$を満たす
				$y \in A^a$と$x$の任意の近傍$V$を取れば,
				$h^{-1}(V) \cap A \neq \emptyset$より
				$V \cap h(A) \neq \emptyset$が成り立ち
				$x \in h(A)^a$となる.
				
			\item[(3)]
				$h(A^i) \subset h(A)$かつ$h(A^i)$は開であるから
				$h(A^i) \subset h(A)^i$が従う.一方で
				任意の開集合$O \subset h(A)$に対し
				$h^{-1}(O) \subset A$より
				$h^{-1}(O) \subset A^i$となり,
				$O \subset h(A^i)$が成り立つから
				$h(A)^i \subset h(A^i)$が得られる.
				\QED
		\end{description}
	\end{prf}
	
	\begin{screen}
		\begin{thm}[第一類集合の性質]
			$S$を位相空間とする.
			\begin{description}
				\item[(a)] $A \subset B \subset S$に対し$B$が第一類なら$A$も第一類である.
				\item[(b)] 第一類集合の可算和も第一類である.
				\item[(c)] 内核が空である閉集合は第一類である.
				\item[(d)] $S$から$S$への位相同型$h$と$E \subset S$に対し次が成り立つ:
					\begin{align}
						\mbox{$E$が第一類} \quad \Longleftrightarrow \quad
						\mbox{$h(E)$が第一類}.
					\end{align}
			\end{description}
		\end{thm}
	\end{screen}
	
	\begin{prf}\mbox{}
		\begin{description}
			\item[(a)] $B = \bigcup_{n=1}^\infty E_n$
				を満たす疎集合系$(E_n)_{n=1}^\infty$に対し
				$A \cap E_n$は疎であり$A = \bigcup_{n=1}^\infty (A \cap E_n)$となる.
			\item[(b)] $A_n \subset S,\ (n=1,2,\cdots)$が第一類集合とし
				$(E_{n,i})_{i=1}^\infty$を$A_n = \bigcup_{i=1}^\infty E_{n,i}$
				を満たす疎集合系とすれば
				\begin{align}
					\bigcup_{n=1}^\infty A_n
					= \bigcup_{n,i=1}^\infty E_{n,i}
				\end{align}
				が成り立つ.
				
			\item[(c)] 内核が空である閉集合はそれ自身が疎であり,自身の可算和に一致する.
			\item[(d)] $E$が第一類のとき,$E = \bigcup_{i=1}^\infty E_i$を満たす
				疎集合系$(E_i)_{i=1}^\infty$に対し定理\ref{thm:topology_note_closure_interior}より
				\begin{align}
					\emptyset = h(E_i^{ai})
					= h(E_i^a)^i
					= h(E_i)^{ai}
				\end{align}
				が成り立つから$h(E_i)$は疎であり,
				\begin{align}
					h(E) = \bigcup_{i=1}^\infty h(E_i)
				\end{align}
				となるから$h(E)$も第一類である.$h(E)$が第一類なら$E = h^{-1}(h(E))$も第一類である.
				\QED
		\end{description}
	\end{prf}
	
\subsection{有向点族}
\section{測度}
	\subsection{Lebesgue拡大}
		\begin{screen}
			\begin{dfn}[Lebesgue拡大]
				$(X,\mathcal{B},\mu)$を測度空間とするとき,
				\begin{align}
					\overline{\mathcal{B}} &\coloneqq
					\Set{B \subset X}{\exists A_1,A_2 \in \mathcal{B},\ \mbox{s.t.}\quad A_1 \subset B \subset A_2,\ \mu(A_2 - A_1)=0 }, \\
					\overline{\mu}(B) &\coloneqq \mu(A_1) \quad (\forall B \in \overline{\mathcal{B}},\ \mbox{$A_1$ as in above})
				\end{align}
				により得られる完備測度空間$(X,\overline{\mathcal{B}},\overline{\mu})$を
				$(X,\mathcal{B},\mu)$のLebesgue拡大と呼ぶ.
			\end{dfn}
		\end{screen}
		$\overline{\mu}$はwell-definedである.実際,$B \subset X$に対し
		$A_1,A_2,B_1,B_2 \in \mathcal{B}$が
		\begin{align}
			&A_1 \subset B \subset A_2, \quad \mu(A_2 - A_1) = 0, \\
			&B_1 \subset B \subset B_2, \quad \mu(B_2 - B_1) = 0,
		\end{align}
		を満たすとき,$A_1 \cup B_1 \subset B \subset A_2 \cap B_2$となるが,
		\begin{align}
			(A_2 \cap B_2) \cap (A_1 \cup B_1)^c
			\subset A_2 \backslash A_1
		\end{align}
		より$\mu(A_1 \cup B_1) = \mu(A_2 \cap B_2)$が従い
		\begin{align}
			\mu(A_2) &= \mu(A_1) \leq \mu(A_1 \cup B_1) = \mu(A_2 \cap B_2) \leq \mu(B_2), \\
			\mu(B_2) &= \mu(B_1) \leq \mu(A_1 \cup B_1) = \mu(A_2 \cap B_2) \leq \mu(A_2)
		\end{align}
		が成り立つから$\mu(A_2) = \mu(B_2)$が出る.
		また,任意の$B \subset X$について
		\begin{align}
			\overline{\mathcal{B}}
			= \Set{B \subset X}{\exists A,N \in \mathcal{B},\ \mbox{s.t.}\quad \mu(N)=0,
			\ B \cap A^c, A \cap B^c \subset N}
			\label{eq:appendix_Lebesgue_expansion_note_1}
		\end{align}
		が成立する.実際,$B \in \overline{\mathcal{B}}$なら
		$A_1 \subset B \subset A_2$かつ$\mu(A_2 - A_1) = 0$を満たす$A_1,A_2 \in \mathcal{B}$が存在するから
		\begin{align}
			A = A_2, \quad N = A_2 - A_1
		\end{align}
		として$(\subset)$を得る.逆に右辺を満たす$A,N$が存在するとき,
		\begin{align}
			A \cap N^c &\subset A \cap B \subset B 
			\subset A \cup (A^c \cap B)
			\subset A \cup N
		\end{align}
		より$A_1 = A\cap N^c,\ A_2 = A \cup N$として$(\supset)$を得る.
	
		\begin{screen}
			\begin{thm}[完備化前後の可測関数の関係]
				$(X,\mathcal{B},\mu)$を測度空間,そのLebesgue拡大を
				$(X,\overline{\mathcal{B}},\overline{\mu})$と書き,
				$f:X \longrightarrow [-\infty,\infty]$とする.
				このとき次は同値である:
				\begin{description}
					\item[(a)] 或る$\mathcal{B}/\borel{[-\infty,\infty]}$-可測関数$g$が存在して
						$f = g\quad \mbox{$\overline{\mu}$-a.e.}$を満たす.
					\item[(b)] 或る$\mathcal{B}/\borel{[-\infty,\infty]}$-可測関数$g_1,g_2$が存在して
						$g_1(x) \leq f(x) \leq g_2(x)\ (\forall x \in X)$かつ$g_1 = g_2\quad \mbox{$\overline{\mu}$-a.e.}$を満たす.
					\item[(c)] $f$は$\overline{\mathcal{B}}/\borel{[-\infty,\infty]}$-可測である.
				\end{description}
			\end{thm}
		\end{screen}
		
		\begin{prf}\mbox{}
			\begin{description}
				\item[第一段]
					$B \subset X$に対して$f = \defunc_B$と表せるとき,
					\begin{align}
						\mbox{$f$が$\overline{\mathcal{B}}/\borel{[-\infty,\infty]}$-可測}
						&\Leftrightarrow B = f^{-1}(\{1\}) \in \overline{\mathcal{B}} \\
						&\Leftrightarrow \exists A_1,A_2 \in \mathcal{B},\ \mbox{s.t.}\quad A_1 \subset B \subset A_2,\ \mu(A_2 - A_1)=0 \\
						&\Rightarrow \defunc_{A_1} \leq f \leq \defunc_{A_2},\quad \defunc_{A_1}=\defunc_{A_2}\ \mbox{$\overline{\mu}$-a.e.} \\
						&\Rightarrow (b) \\
						&\Rightarrow (c)
					\end{align}
					となる.また$(c)$が満たされているとき,
					\begin{align}
						f(x) = g(x) \quad (\forall x \in X \backslash N),
						\label{eq:appendix_Lebesgue_expansion_note_2}
					\end{align}
					を満たす$\mu$-零集合$N \in \mathcal{B}$が存在して
					\begin{align}
						f^{-1}(E) \cap \left( g^{-1}(E) \right)^c \subset N,
						\quad g^{-1}(E) \cap \left( f^{-1}(E) \right)^c \subset N,
						\quad (\forall E \in \borel{[-\infty,\infty]})
					\end{align}
					が成り立つから,(\refeq{eq:appendix_Lebesgue_expansion_note_1})より
					$f^{-1}(E) \in \overline{\mathcal{B}}$が従い$(c) \Rightarrow (a)$が出る.
				
				\item[第二段]
					$0 \leq f \leq \infty$かつ単関数$f = \sum_{n=0}^N \alpha_n \defunc_{B_n}\ 
					(\alpha_0 = 0,\ i \neq j \Rightarrow \alpha_i \neq \alpha_j)$
					として表されるとき,
					\begin{align}
						\mbox{$f$が$\overline{\mathcal{B}}/\borel{[-\infty,\infty]}$-可測}
						&\Leftrightarrow B_n = f^{-1}(\{\alpha_n\}) \in \overline{\mathcal{B}},\ (n=0,1,\cdots,N) \\
						&\Leftrightarrow \exists g_{1,n},g_{2,n}:\ \mbox{$\mathcal{B}/\borel{[-\infty,\infty]}$-measurable }, \\
							&\qquad \mbox{s.t.}\quad g_{1,n} \leq \defunc_{B_n} \leq g_{2,n},
							\ \mu(g_{1,n} \neq g_{1,n})=0,\ (n=0,1,\cdots,N) \\
						&\Rightarrow g_1 \coloneqq \sum_{n=0}^N \alpha_n g_{1,n},
							\quad g_2 \coloneqq \sum_{n=0}^N \alpha_n g_{2,n}, \\
							&\qquad g_1 \leq f \leq g_2,\quad \mu(g_1 \neq g_2) \leq \mu\Biggl( \bigcup_{n=0}^N \left\{g_{1,n} \neq g_{2,n}\right\} \Biggr)=0 \\
						&\Rightarrow (b) \\
						&\Rightarrow (c)
					\end{align}
					が成り立つ.また前段と同じ理由で$(c) \Rightarrow (a)$が出る.
					
				\item[第三段]
					$0 \leq f \leq \infty$のとき,
					$f$が$\overline{\mathcal{B}}/\borel{[-\infty,\infty]}$-可測なら
					$f_n(x) \uparrow f(x)\ (\forall x \in X)$を満たす
					非負$\overline{\mathcal{B}}$-可測単関数列$(f_n)_{n=1}^\infty$が存在し,
					第二段の結果より各$f_n$に対して
					\begin{align}
						g_{1,n} \leq f_n \leq g_{2,n},
						\quad \mu\left( g_{1,n} \neq g_{2,n} \right)=0
					\end{align}
					を満たす$\mathcal{B}/\borel{[-\infty,\infty]}$-可測写像$g_{1,n},g_{2,n}$が存在する.
					\begin{align}
						g_1 \coloneqq \liminf_{n \to \infty} g_{1,n},
						\quad g_2 \coloneqq \limsup_{n \to \infty} g_{2,n}
					\end{align}
					とおけば
					\begin{align}
						g_{1,n}(x) = g_{2,n}(x)\ (\forall n \geq 1)
						\quad \Rightarrow \quad g_1(x) = \lim_{n \to \infty} f_n(x) = g_2(x)
					\end{align}
					が成り立ち
					\begin{align}
						\mu(g_1 \neq g_2)
						\leq \mu\Biggl( \bigcup_{n=1}^\infty \left\{g_{1,n} \neq g_{2,n}\right\} \Biggr)=0 \\
					\end{align}
					が従うから$(a) \Rightarrow (b)$及び$(b) \Rightarrow (c)$が得られる.
					第一段と同じ理由で$(c) \Rightarrow (a)$も成立する.
					
				\item[第四段]
					一般の$f:X \longrightarrow [-\infty,\infty]$に対し
					$f^+ \coloneqq f \defunc_{\{f \geq 0\}},\ f^- \coloneqq -f \defunc_{\{f < 0\}}$とおけば,
					$f$が$\overline{\mathcal{B}}/\borel{[-\infty,\infty]}$-可測なら
					$f^+,f^-$も$\overline{\mathcal{B}}/\borel{[-\infty,\infty]}$-可測である.従って
					\begin{align}
						g_1^\pm \leq f^\pm \leq g_2^\pm, \quad \mu\left( g_1^\pm \neq g_2^\pm \right) = 0,
						\quad \mbox{(複合同順)}
					\end{align}
					を満たす$\mathcal{B}/\borel{[-\infty,\infty]}$-可測写像$g_1^{\pm},g_2^{\pm}$が存在する.
					ここで
					\begin{align}
						g_1 \coloneqq g_1^+ - g_2^-,
						\quad g_2 \coloneqq g_2^+ - g_1^+
					\end{align}
					とおけば$(a) \Rightarrow (b)$成り立ち,前段と同様に$(b) \Rightarrow (c) \Rightarrow (a)$も得られる.
					\QED
			\end{description}
		\end{prf}
		
		\begin{screen}
			\begin{cor}
				$(X,\mathcal{B},\mu)$を測度空間,そのLebesgue拡大を
				$(X,\overline{\mathcal{B}},\overline{\mu})$と書き,
				$f:X \longrightarrow \C$とする.このとき次は同値である:
				\begin{description}
					\item[(a)] 或る$\mathcal{B}/\borel{\C}$-可測関数$g$が存在して
						$f = g\quad \mbox{$\overline{\mu}$-a.e.}$を満たす.
					\item[(b)] $f$は$\overline{\mathcal{B}}/\borel{\C}$-可測である.
				\end{description}
			\end{cor}
		\end{screen}
	
	\subsection{測度の構成1: 外測度による方法}
	\subsection{測度の構成2: Riesz-Markov-角谷の定理による方法}
	\subsection{測度の構成1と2の関係}
	\subsection{有限加法的測度の拡張}
		\begin{screen}
			\begin{thm}[Kolmogorov-Hopf]
				$(X,\mathcal{B},\mu_0)$を有限加法的測度空間($\mathcal{B}$は有限加法族,$\mu_0$は有限加法的)とし,
				\begin{align}
					\mu^*(A) \coloneqq \inf{}{}\Set{\sum_{n=1}^\infty \mu_0(B_n)}{B_n \in \mathcal{B},\ A \subset \bigcup_{n=1}^\infty B_n},
					\quad (\forall A \subset X)
				\end{align}
				により$X$上に外測度を定め,$\mu^*$-可測集合を$\mathcal{B}^*$と書く.このとき,
				\begin{description}
					\item[(1)] $\sigma[\mathcal{B}] \subset \mathcal{B}^*$が成り立つ.
						ここで$\mu' \coloneqq\left.\mu^*\right|_{\mathcal{B}^*},
						\ \mu \coloneqq \left.\mu^*\right|_{\sigma[\mathcal{B}]}$とおく.
					\item[(2)] $\mu_0$が$\mathcal{B}$上で$\sigma$-加法的なら
						\begin{align}
							\mu_0(B) = \mu(B),\quad (\forall B \in \mathcal{B})
							\label{eq:appendix_finite_additive_measure_expansion_1}
						\end{align}
						となる.つまり$\mu$は$\mu_0$の拡張である.
						
					\item[(3)] $\mu_0$が$\mathcal{B}$上で$\sigma$-有限的であるとき,
						$\left( X,\sigma[\mathcal{B}] \right)$上の測度$\mu_1,\mu_2$が
						(\refeq{eq:appendix_finite_additive_measure_expansion_1})を満たせば
						$\mu_1 = \mu_2$となる.
					
					\item[(4)] $\mu_0$が$\mathcal{B}$上で$\sigma$-加法的かつ$\sigma$-有限的ならば,
						$\mu$は$\mu_0$の$\left( X,\sigma[\mathcal{B}] \right)$への唯一つの拡張測度であり,
						$\left( X,\mathcal{B}^*,\mu' \right)$は$(X,\sigma[\mathcal{B}],\mu)$の
						Lebesgue拡大に一致する:
						\begin{align}
							\left( X,\mathcal{B}^*,\mu' \right) 
							= \left( X,\overline{\sigma[\mathcal{B}]},\overline{\mu} \right).
						\end{align}
				\end{description}
			\end{thm}
		\end{screen}
		
		\begin{prf}\mbox{}
			\begin{description}
				\item[(1)の証明]
					任意の$B \in \mathcal{B}$が$\mu^*$-可測であること,つまり任意の$A \subset X$に対し
					\begin{align}
						\mu^*(A) \geq \mu^*(A \cap B) + \mu^*(A \cap B^c)
						\label{eq:appendix_finite_additive_measure_expansion_2}
					\end{align}
					となることを示せば,$\mathcal{B} \subset \mathcal{B}^*$すなわち
					$\sigma[\mathcal{B}] \subset \mathcal{B}^*$が従う.
					任意の$A \subset X,\ \epsilon > 0$に対し
					\begin{align}
						A \subset \bigcup_{n=1}^\infty B_n,
						\quad \sum_{n=1}^\infty \mu_0(B_n) < \mu^*(A) + \epsilon
					\end{align}
					を満たす$\{B_n\}_{n=1}^\infty \subset \mathcal{B}$が存在する.
					このとき$A \cap B \subset \bigcup_{n=1}^\infty (B_n \cap B)
					,\ A \cap B^c \subset \bigcup_{n=1}^\infty (B_n \cap B^c)$より
					\begin{align}
						\mu^*(A \cap B) \leq \sum_{n=1}^\infty \mu_0(B_n \cap B),
						\quad \mu^*(A \cap B^c) \leq \sum_{n=1}^\infty \mu_0(B_n \cap B^c)
					\end{align}
					となるから
					\begin{align}
						\mu^*(A) + \epsilon
						&\geq \sum_{n=1}^\infty \mu_0(B_n)
						= \sum_{n=1}^\infty \left\{ \mu_0(B_n \cap B) + \mu_0(B_n \cap B^c) \right\} \\
						&= \sum_{n=1}^\infty \mu_0(B_n \cap B) + \sum_{n=1}^\infty \mu_0(B_n \cap B^c) \\
						&\geq \mu^*(A \cap B) + \mu^*(A \cap B^c)
					\end{align}
					が成り立つ.$\epsilon$の任意性より
					(\refeq{eq:appendix_finite_additive_measure_expansion_2})が出る.
				
				\item[(2)の証明]
					任意に$B \in \mathcal{B}$を取る.まず,
					$B \subset B \cup \emptyset \cup \emptyset \cup \cdots$より
					\begin{align}
						\mu^*(B) \leq \mu_0(B)
					\end{align}
					が成り立つ.一方で
					$B \subset \bigcup_{n=1}^\infty B_n$を満たす$\{B_n\}_{n=1}^\infty \subset \mathcal{B}$に対し
					\begin{align}
						B = \sum_{n=1}^\infty \Biggl( B \cap \Biggl( B_n \backslash \bigcup_{k=1}^{n-1}B_k \Biggr) \Biggr)
					\end{align}
					かつ$B \cap \left( B_n \backslash \bigcup_{k=1}^{n-1}B_k \right) \in \mathcal{B}$が満たされるから,
					$\mu_0$の$\sigma$-加法性より
					\begin{align}
						\mu_0(B) = \sum_{n=1}^\infty \mu_0\Biggl( B \cap \Biggl( B_n \backslash \bigcup_{k=1}^{n-1}B_k \Biggr) \Biggr)
						\leq \sum_{n=1}^\infty \mu_0(B_n)
					\end{align}
					が成り立ち$\mu_0(B) \leq \mu^*(B)$が従う.よって$\mu_0(B) = \mu^*(B) = \mu(B)$が得られる.
				
				\item[(3)の証明]
					$\sigma$-有限の仮定より,或る増大列$X_1 \subset X_2 \subset \cdots
					,\ \{X_n\}_{n=1}^\infty \subset \mathcal{B}$が存在して
					\begin{align}
						\mu_0 (X_n) < \infty \quad \bigcup_{n=1}^\infty X_n = X
						\label{eq:appendix_finite_additive_measure_expansion_3}
					\end{align}
					を満たす.このとき
					\begin{align}
						\mathscr{D}_n \coloneqq \Set{B \in \sigma[\mathcal{B}]}{\mu_1(B \cap X_n) = \mu_2(B \cap X_n)},
						\quad (n=1,2,\cdots)
					\end{align}
					とおけば,(\refeq{eq:appendix_finite_additive_measure_expansion_1})より
					$\mathscr{D}_n$は$\mathcal{B}$を含むDynkin族である.従ってDynkin族定理より
					\begin{align}
						\mathscr{D}_n = \sigma[\mathcal{B}],
						\quad (\forall n \geq 1)
					\end{align}
					が成り立ち
					\begin{align}
						\mu_1(B) = \lim_{n \to \infty} \mu_1(B \cap X_n)
						= \lim_{n \to \infty} \mu_2(B \cap X_n) = \mu_2(B),
						\quad (\forall B \in \sigma[\mathcal{B}])
					\end{align}
					が出る.
					
				\item[(4)の証明]
					(2)と(3)の結果より$\mu$は$\mu_0$の唯一つの拡張測度である.次に
					\begin{align}
						\mathcal{B}^* = \overline{\sigma[\mathcal{B}]}
					\end{align}
					を示す.$E \in \overline{\sigma[\mathcal{B}]}$なら
					或る$B_1,B_2 \in \sigma[\mathcal{B}]$が存在して
					\begin{align}
						B_1 \subset E \subset B_2, \quad \mu(B_2 - B_1) = 0
					\end{align}
					を満たす.このとき(1)より
					$\mu'(B_2 - B_1) = 0$であり,$\left( X,\mathcal{B}^*,\mu' \right)$の完備性より
					$E \backslash B_1 \in \mathcal{B}^*$が満たされ
					\begin{align}
						E = B_1 + E \backslash B_1 \in \mathcal{B}^*
					\end{align}
					が従う.いま,(\refeq{eq:appendix_finite_additive_measure_expansion_3})を満たす
					$\{X_n\}_{n=1}^\infty \subset \mathcal{B}$を取り,
					$E \in \mathcal{B}^*$に対して$E_n \coloneqq E \cap X_n$とおく.このとき
					\begin{align}
						\mu'(E_n) \leq \mu'(X_n) = \mu_0(X_n) < \infty
					\end{align}
					となり,任意の$k = 1,2,\cdots$に対して
					\begin{align}
						E_n \subset \bigcup_{j=1}^\infty B^n_{k,j},
						\quad
						\sum_{j=1}^\infty \mu_0\left( B^n_{k,j} \right)
						< \mu'(E_n) + \frac{1}{k}
					\end{align}
					を満たす$\left\{B^n_{k,j}\right\}_{j=1}^\infty \subset \mathcal{B}$が存在する.
					\begin{align}
						B_{2,n} \coloneqq \bigcap_{k=1}^\infty \bigcup_{j=1}^\infty B^n_{k,j}
					\end{align}
					とおけば$E_n \subset B_{2,n} \in \sigma[\mathcal{B}]$であり,
					任意の$k = 1,2,\cdots$に対して
					\begin{align}
						&\mu'(B_{2,n} - E_n) = \mu'(B_{2,n}) - \mu'(E_n)
						\leq \mu'\Biggl( \bigcup_{j=1}^\infty B^n_{k,j} \Biggr) - \mu'(E_n) \\
						&\qquad \leq \sum_{j=1}^\infty \mu'\left( B^n_{k,j} \right) - \mu'(E_n)
						< \mu'(E_n) + \frac{1}{k} - \mu'(E_n)
						= \frac{1}{k}
					\end{align}
					が成り立つから$\mu'(B_{2,n} - E_n) = 0$となる.
			\end{description}
		\end{prf}
\section{積分}
\subsection{積分}
	\begin{screen}
		\begin{thm}[複素数値可測$\Longleftrightarrow$実部虚部が可測]\label{thm:measurability_of_complex_measurable_functions}
			$(X,\mathscr{F})$を可測空間,$f:X \longrightarrow \C$とするとき,
			$f$が$\mathscr{F}/\borel{\C}$-可測であることと
			$f$の実部$u$と虚部$v$がそれぞれ$\mathscr{F}/\borel{\R}$-可測であることは同値である.
		\end{thm}
	\end{screen}
	
	\begin{prf}
		$z \in \C$に対し$x,y \in \C$の組が唯一つ対応し$z = x + i y$を満たす.この対応関係により定める写像
		\begin{align}
			\varphi:\C \ni z \longmapsto (x,y) \in \R^2
		\end{align}
		は位相同型である.射影を$p_1:\R^2 \ni (x,y) \longmapsto x,
		\ p_2:\R^2 \ni (x,y) \longmapsto y$とすれば
		\begin{align}
			u = p_1 \circ \varphi \circ f,
			\quad v = p_2 \circ \varphi \circ f
		\end{align}
		となるから,$f$が$\mathscr{F}/\borel{\C}$-可測であるなら
		$p_1,p_2,\varphi$の連続性より
		\begin{align}
			u^{-1}(A) = f^{-1} \circ \varphi^{-1} \circ p_1^{-1}(A) \in \mathscr{F},
			\quad v^{-1}(A) = f^{-1} \circ \varphi^{-1} \circ p_2^{-1}(A) \in \mathscr{F},
			\quad (\forall A \in \borel{\R})
		\end{align}
		が成り立ち$u,v$の$\mathscr{F}/\borel{\R}$-可測性が従う.逆に$u,v$が$\mathscr{F}/\borel{\R}$-可測であるとき,
		\begin{align}
			f^{-1}(B) = \Set{x \in X}{(u(x),v(x)) \in \varphi(B)} \in \mathscr{F},
			\quad (\forall B \in \borel{\C})
		\end{align}
		が成り立ち$f$の$\mathscr{F}/\borel{\C}$-可測性が出る.
		\QED
	\end{prf}
	
	\begin{screen}
		\begin{thm}[和・積・商の可測性]
			
		\end{thm}
	\end{screen}
	
	\begin{screen}
		\begin{thm}[相対位相のBorel集合族]\label{thm:Borel_algebra_of_relative_topology}
			$(S,\mathscr{O})$を位相空間とする.部分集合$A \subset S$に対して
			\begin{align}
				\borel{A} \coloneqq \sigma\left[ \Set{A \cap O}{O \in \mathscr{O}} \right]
			\end{align}
			とおくとき次が成り立つ:
			\begin{align}
				\borel{A} = \Set{A \cap E}{E \in \borel{S}}.
			\end{align}
			また$A \in \borel{S}$なら$\borel{A} \subset \borel{S}$となる.
		\end{thm}
	\end{screen}
	
	$\R$-値可測関数は$\C$-値可測関数でもある.
	
	\begin{screen}
		\begin{thm}[単関数近似列の存在]
			$(X,\mathscr{F})$を可測空間とする.
			\begin{description} 
				\item[(1)] 任意の$\mathscr{F}/\borel{[0,\infty]}$-可測写像$f$に対し
					\begin{align}
						0 \leq f_1 \leq f_2 \leq \cdots \leq f;
						\quad \lim_{n \to \infty} f_n(x) = f(x),\ (\forall x \in X)
					\end{align}
					を満たす$\mathscr{F}/\borel{[0,\infty)}$-可測単関数列$(f_n)_{n=1}^\infty$が存在する.
					
				\item[(2)]
					 任意の$\mathscr{F}/\borel{\C}$-可測写像$f$に対し
					\begin{align}
						0 \leq |f_1| \leq |f_2| \leq \cdots \leq |f|;
						\quad \lim_{n \to \infty} f_n(x) = f(x),\ (\forall x \in X)
					\end{align}
					を満たす$\mathscr{F}/\borel{\C}$-可測単関数列$(f_n)_{n=1}^\infty$が存在する.
				
				\item[(3)] (1)または(2)において,$f$が$E \in \mathscr{F}$上で有界なら
					$f_n \defunc_E$は一様に$f \defunc_E$を近似する:
					\begin{align}
						\sup{x \in E}{\left| f_n(x) - f(x) \right|} \longrightarrow 0
						\quad (n \longrightarrow \infty).
					\end{align}
			\end{description}
		\end{thm}
	\end{screen}
	
	\begin{screen}
		\begin{dfn}[複素数値可測関数の正値測度に関する積分]
			$(X,\mathscr{F},\mu)$を正値測度空間,
			$f$を$\mathscr{F}/\borel{\C}$-可測関数とする.
			$u \coloneqq \Re{f},\ v \coloneqq \Im{f}$とおけば
			$|u|,|v| \leq |f| \leq |u| + |v|$より
			\begin{align}
				\mbox{$|f|$が可積分} \quad \Longleftrightarrow \quad
				\mbox{$u,v$が共に可積分}
			\end{align}
			が成り立つ.$|f|$が可積分のとき,$f$は可積分であるといい$f$の$\mu$に関する積分を次で定める:
			\begin{align}
				\int_X f\ d\mu
				\coloneqq \int_X u\ d\mu + i \int_X v\ d\mu.
			\end{align}
		\end{dfn}
	\end{screen}
	
	\begin{screen}
		\begin{thm}[Lebesgueの収束定理]
			$(X,\mathscr{F},\mu)$を正値測度空間,
			$f,\ f_n\ (n=1,2,\cdots)$を$\mathscr{F}/\borel{\C}$-可測な可積分関数とする.
			このとき,$f = \lim_{n \to \infty} f_n\ \mbox{$\mu$-a.e.}$かつ
			\begin{align}
				|f_n| \leq g, \quad \mbox{$\mu$-a.e.}
			\end{align}
			を満たす可積分関数$g$が存在するとき
			\begin{align}
				\int_X |f - f_n|\ d\mu \longrightarrow 0
				\quad (n \longrightarrow \infty).
			\end{align}
		\end{thm}
	\end{screen}
	
	\begin{screen}
		\begin{thm}[積分の線形性・積分作用素の有界性]
			$(X,\mathscr{F})$を可測空間とし,$\mu$を$\mathscr{F}$上の正値測度とする.
			\begin{description}
				\item[(1)] 任意の$\mathscr{F}/\borel{\C}$-可測可積分関数$f,g$と
					$\alpha,\beta \in \C$に対して次が成り立つ:
					\begin{align}
						\int_X \alpha f + \beta g\ d\mu
						= \alpha \int_X f\ d\mu + \beta \int_X g\ d\mu.
					\end{align}
					
				\item[(2)] 任意の$\mathscr{F}/\borel{\C}$-可測可積分関数$f$に対して次が成り立つ:
					\begin{align}
						\left| \int_X f\ d\mu \right| \leq \int_X |f|\ d\mu.
					\end{align}
			\end{description}	
		\end{thm}
	\end{screen}
	
	\begin{prf}\mbox{}
		\begin{description}
			\item[(1)] 
			
			\item[(2)]
				$\alpha \coloneqq \int_X f\ d\mu$とおけば,$\alpha \neq 0$の場合
				\begin{align}
					|\alpha|
					= \frac{\overline{\alpha}}{|\alpha|} \int_X f\ d\mu
					= \int_X \frac{\overline{\alpha}}{|\alpha|} f\ d\mu
				\end{align}
				が成り立ち
				\begin{align}
					|\alpha| = \Re{|\alpha|}
					= \Re{\int_X \frac{\overline{\alpha}}{|\alpha|} f\ d\mu}
					= \int_X \Re{\frac{\overline{\alpha}}{|\alpha|} f}\ d\mu
					\leq \int_X |f|\ d\mu
				\end{align}
				が従う.$\alpha = 0$の場合も不等式は成立する.
				\QED
		\end{description}
	\end{prf}
	
	\begin{screen}
		\begin{lem}
			$S$を実数の集合とする.$-S \coloneqq \Set{-s}{s \in S}$とおくとき次が成り立つ:
			\begin{align}
				\inf{}{S} = -\sup{}{(-S)},
				\quad \sup{}{S} = -\inf{}{(-S)}.
			\end{align}
		\end{lem}
	\end{screen}
	
	\begin{prf}
		任意の$s \in S$に対して$-s \leq \sup{}{(-S)}$より
		$\inf{}{S} \geq -\sup{}{(-S)}$となる.一方で任意の$s \in S$に対し
		$\inf{}{S} \leq s$より$-s \leq -\inf{}{S}$となり
		$\sup{}{(-S)} \leq -\inf{}{S}$が従うから
		$-\sup{}{(-S)} \geq \inf{}{S}$も成り立ち
		$\inf{}{S} = -\sup{}{(-S)}$が出る.
		\QED
	\end{prf}
	
	\begin{screen}
		\begin{thm}[写像の値域は積分の平均値の範囲を出ない]\label{thm:mean_value_of_integral_and_closed_set}
			$(X,\mathscr{F},\mu)$を$\sigma$-有限測度空間,
			$f:X \longrightarrow \C$を$\mathscr{F}/\borel{\C}$-可測かつ$\mu$-可積分な関数,
			$C \subset \C$を閉集合とする.このとき
			\begin{align}
				\frac{1}{\mu(E)}\int_E f\ d\mu \in C,
				\quad (\forall E \in \mathscr{F},\ 0 < \mu(E) < \infty)
				\label{eq:thm_mean_value_of_integral_and_closed_set}
			\end{align}
			なら次が成り立つ:
			\begin{align}
				f(x) \in C \quad \mbox{$\mu$-a.e.}x \in X.
			\end{align}
		\end{thm}
	\end{screen}
	$C=\R$なら$f$は殆ど至る所$\R$値であり,
	$C=\{0\}$なら殆ど至る所$f=0$である.
	\begin{prf}
		$\sigma$-有限の仮定より次を満たす$\{X_n\}_{n=1}^\infty \subset \mathscr{F}$が存在する:
		\begin{align}
			\mu(X_n) < \infty,\ (\forall n \geq 1);
			\quad X = \bigcup_{n=1}^\infty X_n.
		\end{align}
		$C = \C$なら$f(x) \in C\ (\forall x \in X)$である.
		$C \neq \C$の場合,任意の$\alpha \in \C \backslash C$に対し
		或る$r > 0$が存在して
		\begin{align}
			B_r(\alpha) \coloneqq \Set{z \in \C}{|z - \alpha| \leq r} \subset \C \backslash C
		\end{align}
		を満たす.ここで
		\begin{align}
			E \coloneqq f^{-1}\left( B_r(\alpha) \right),
			\quad E_n \coloneqq E \cap X_n
		\end{align}
		とおけば,任意の$n \geq 1$について$\mu(E_n) > 0$なら
		\begin{align}
			\left| \frac{1}{\mu(E_n)}\int_{E_n} f\ d\mu - \alpha \right|
			= \left| \frac{1}{\mu(E_n)}\int_{E_n} f - \alpha\ d\mu \right|
			\leq \frac{1}{\mu(E_n)}\int_{E_n} |f - \alpha|\ d\mu
			\leq r
		\end{align}
		となり(\refeq{eq:thm_mean_value_of_integral_and_closed_set})に反するから,
		$\mu(E_n) = 0\ (\forall n \geq 1)$及び
		\begin{align}
			\mu(E) = \mu\Biggl( \bigcup_{n=1}^\infty E_n \Biggr) 
			\leq \sum_{n=1}^\infty \mu(E_n) = 0
		\end{align}
		が従う.$\C \backslash C$は開集合であり$B_r(\alpha)$の形の集合の可算和で表せるから
		\begin{align}
			\mu\left( f^{-1}\left( \C \backslash C \right) \right) = 0
		\end{align}
		が成り立ち主張が得られる.
		\QED
	\end{prf}
	
	\begin{screen}
		\begin{thm}[可積分なら積分値を一様に小さくできる]\label{thm:integrable_intvalue_uniformly_shrinking}
			$(X,\mathscr{F},\mu)$を正値測度空間,$f:X \longrightarrow \C$
			を$\mathscr{F}/\borel{\C}$-可測関数とするとき,
			$f$が可積分なら,任意の$\epsilon > 0$に対して或る$\delta > 0$が存在し次を満たす:
			\begin{align}
				\mu(E) < \delta \quad \Longrightarrow \quad \int_E |f|\ d\mu < \epsilon.
			\end{align}
		\end{thm}
	\end{screen}
	
	\begin{prf}
		$X_n \coloneqq \{|f| \leq n\}$により増大列$(X_n)_{n=1}^\infty$を定めれば
		単調収束定理より
		\begin{align}
			\int_X |f|\ d\mu = \lim_{n \to \infty} \int_{X_n} |f|\ d\mu
		\end{align}
		となるから,任意の$\epsilon > 0$に対し或る$n_0 \geq 1$が存在して
		\begin{align}
			\int_{X \backslash X_{n_0}} |f|\ d\mu < \epsilon
		\end{align}
		が成り立つ.このとき$\mu(E) < \delta \coloneqq \epsilon/n_0$なら
		\begin{align}
			\int_E |f|\ d\mu
			= \int_{E \cap X_{n_0}} |f|\ d\mu + \int_{E \cap (X \backslash X_{n_0})} |f|\ d\mu
			\leq n_0 \mu(E) + \int_{X \backslash X_{n_0}} |f|\ d\mu
			< 2\epsilon
		\end{align}
		が従う.
		\QED
	\end{prf}
	
	\subsection{関数列の収束}
		\begin{screen}
			\begin{dfn}[概収束すれば測度収束する]
				$(X,\mathscr{F},\mu)$を正値有限測度空間とする.
				$(f_n)_{n=1}^\infty,f$を全て$\mathscr{F}/\borel{\C}$-可測関数とするとき,
				$\lim_{n \to \infty} f_n = f,\ \mbox{$\mu$-a.e.}$なら
				$(f_n)_{n=1}^\infty$は$f$に測度収束する.
			\end{dfn}
		\end{screen}
		
		\begin{prf}
			任意の$\epsilon > 0$に対し
			\begin{align}
				A^n_\epsilon \coloneqq \left\{ |f_n - f| > \epsilon \right\}
			\end{align}
			とおけば,Lebesgueの収束定理より任意の$k \geq 1$で
			\begin{align}
				\epsilon \mu\left(A^n_\epsilon\right)
				\leq \int_{A^n_\epsilon} |f_n - f| \wedge \epsilon\ d\mu
				\leq \int_{X} |f_n - f| \wedge \epsilon\ d\mu
				\longrightarrow 0
				\quad (n \longrightarrow \infty)
			\end{align}
			が成立する.
			\QED
		\end{prf}
		
		上の定理で有限性を外すときの反例を示す.
		$X = \R$,$\mu$を一次元Lebesgue測度とするとき,
		\begin{align}
			f_n \coloneqq \defunc_{\R \backslash (-n,n)}
		\end{align}
		で定める関数列$(f_n)_{n=1}^\infty$は零写像に各点収束するが,$0 < \epsilon < 1$に対し
		\begin{align}
			\mu\left( f_n > \epsilon \right) = \mu((-\infty,-n] \cup [n,\infty)) = \infty,
			\quad (\forall n \geq 1)
		\end{align}
		を満たすから測度収束しない.
		
		\begin{screen}
			\begin{thm}[測度収束列の概収束部分列]\label{thm:convergence_in_measure_then_convergence_almost_everywhere}
				$(X,\mathscr{F},\mu)$を正値測度空間,
				$(f_n)_{n=1}^\infty,f$を全て$\mathscr{F}/\borel{\C}$-可測関数とするとき,
				$(f_n)_{n=1}^\infty$が$f$に測度収束するなら
				或る部分列$(f_{n_k})_{k=1}^\infty$は$f$に概収束する.
			\end{thm}
		\end{screen}
		
		\begin{prf}
			$(f_n)_{n=1}^\infty$が$f$に測度収束するとき,任意の$k \geq 1$に対し
			\begin{align}
				\mu\left( |f_{n_k} - f| > \frac{1}{2^k}\right) < \frac{1}{2^k}
			\end{align}
			を満たす添数列$n_1 < n_2 < n_3 < \cdots$が取れる.
			\begin{align}
				A_k \coloneqq \left\{|f_{n_k} - f| > \frac{1}{2^k}\right\},
				\quad A \coloneqq \bigcup_{k\geq1} \bigcap_{j>k} A_j^c
			\end{align}
			とおけば,$\mu(A^c) \leq \mu\left(\bigcup_{j>k} A_j\right),\ (\forall k \geq 1)$かつ
			\begin{align}
				\mu\Biggl(\bigcup_{j>k} A_j\Biggr) \leq \sum_{j>k} \frac{1}{2^j}
				= \frac{1}{2^k}
			\end{align}
			より$\mu(A^c) = 0$が従い,$x \in A$なら或る$k = k(x)$が存在して
			\begin{align}
				|f_{n_j}(x) - f(x)| \leq \frac{1}{2^j}, \quad (\forall j > k)
			\end{align}
			となるから$\lim_{k \to \infty} f_{n_k}(x) = f(x)$が満たされる.
			\QED
		\end{prf}
		
		\begin{screen}
			\begin{thm}[平均収束すれば測度収束する]
				$p \in (0,\infty)$,$(X,\mathscr{F},\mu)$を正値測度空間,
				$(f_n)_{n=1}^\infty,f$を全て$\mathscr{F}/\borel{\C}$-可測関数とするとき,
				\begin{align}
					\int_X |f_n - f|^p\ d\mu \longrightarrow 0
					\quad (n \longrightarrow \infty)
				\end{align}
				なら$(f_n)_{n=1}^\infty$は$f$に測度収束する.
			\end{thm}
		\end{screen}
		
		\begin{prf}
			任意の$\epsilon > 0$に対し
			\begin{align}
				\epsilon^p \mu\left(|f_n - f| > \epsilon\right)
				\leq \int_{\left\{|f_n - f| > \epsilon\right\}} |f_n-f|^p\ d\mu
				\leq \int_X |f_n - f|^p\ d\mu \longrightarrow 0
				\quad (n \longrightarrow \infty)
			\end{align}
			が成立する.
			\QED
		\end{prf}
		
		\begin{screen}
			\begin{thm}[Egorov]
			\end{thm}
		\end{screen}
		
	\subsection{Radon測度}
		\begin{screen}
			\begin{thm}[Riesz-Markov-Kakutaniの表現定理]
			\end{thm}
		\end{screen}
		
		\begin{screen}
			\begin{thm}[正値Borel測度の正則性定理]
				
			\end{thm}
		\end{screen}
	
\section{Stieltjes積分}
	$\R$の左半開区間とは
	$(a,b],\ (-\infty \leq a \leq b \leq \infty)$を指す.ただし
	\begin{align}
		(a,b] =
		\begin{cases}
			\emptyset, & a=b, \\
			(-\infty,b], & a=-\infty,\ b < \infty, \\
			(a,\infty), & -\infty < a,\ b = \infty, \\
			(-\infty,\infty), & a=-\infty,\ b = \infty, \\
		\end{cases}
	\end{align}
	と考える.ここで$d \geq 1$に対し$\left(a_1,b_1\right] \times \left(a_2,b_2\right] \times
	\cdots \times \left(a_d,b_d\right]$の形の集合を$\R^d$の左半開区間として
	\begin{align}
		\mathfrak{F} \coloneqq \Set{\sum_{i=1}^n I_i}{I_i \subset \R^d:\mbox{左半開区間},\ n=1,2,\cdots}
	\end{align}
	とおけば,定理\ref{thm:forming_finitely_additive_class}より$\mathfrak{F}$は$\R^d$の上の加法族となる.
	$f_\lambda:\R \longrightarrow \R,\ (\lambda = 1,\cdots,d)$を単調非減少関数として,
	任意の左半開区間$I = I^1 \times \cdots \times I^d \subset \R^d$($I^\lambda$は$\R$の左半開区間)に対し
	\begin{align}
		m_0(I) \coloneqq \prod_{\lambda=1}^d 
		\sup{}{\Set{f_\lambda(\beta_\lambda) - f_\lambda(\alpha_\lambda)}{\left(\alpha_\lambda,\beta_\lambda\right] \subset I^\lambda}}
	\end{align}
	とおき,また$I = \emptyset$なら$m_0(I) \coloneqq 0$として
	\begin{align}
		\mu_0(F) \coloneqq \sum_{i=1}^n m_0(I_i),
		\quad (\forall F = I_1 + I_2 + \cdots + I_n \in \mathfrak{F})
	\end{align}
	により$\mu_0$を定める.この$\mu_0$はwell-definedであり,有限劣加法的かつ有限加法的な$m_0$の拡張である.実際,
	\begin{align}
		I_1 + I_2 + \cdots + I_n = J_1 + J_2 + \cdots + J_m
	\end{align}
	に対し,$\sum_{i=1}^n I_i = \sum_{i=1}^n \sum_{j=1}^m I_i \cap J_j = \sum_{j=1}^m J_i$かつ$I_i \cap J_j$は区間であるから
	\begin{align}
		\mu_0\Biggl(\sum_{i=1}^n I_i\Biggr)
		= \sum_{i=1}^n \sum_{j=1}^m m_0(I_i \cap J_j)
		= \mu_0\Biggl(\sum_{j=1}^m J_j\Biggr)
	\end{align}
	が成り立ち,また有限加法性は$\mu_0$の定義より従う.
	
	\begin{screen}
		\begin{thm}[右連続性と完全加法性]
			単調非減少関数$f_\lambda:\R \longrightarrow \R,\ (\lambda=1,\cdots,d)$を用いて定める$\mu_0$について,
			全ての$f_\lambda$が右連続であることと$\mu_0$が$\mathfrak{F}$の上で完全加法的であることは同値である.
		\end{thm}
	\end{screen}
	
	$\mathfrak{F}$は$\borel{\R^d}$を生成する.
	任意の$n \geq 1$に対して
	\begin{align}
		\mu_0((-n,n] \times \cdots \times (-n,n]) 
		= \prod_{\lambda=1}^d \left\{f_\lambda(n) - f_\lambda(-n)\right\} < \infty
	\end{align}
	であるから
	$\mu_0$は$\mathfrak{F}$上で$\sigma$-有限的であり,$f_\lambda$が右連続であれば
	定理\ref{thm:appendix_Kolmogorov_Hopf}より$\mu_0$の拡張測度$\mu$が唯一つ存在する.
	
	$\mathfrak{F}$が加法族であるから,空でない任意の区間$I \neq \emptyset$に対して
	\begin{align}
		\mathfrak{F}_I \coloneqq \Set{I \cap F}{F \in \mathfrak{F}}
	\end{align}
	は加法族をなす.$I$上右連続単調非減少な,
	ただし$I$が有界なら$I$上で有界な関数$f_I$に対し
	\begin{align}
		a_0 \coloneqq \inf{}{\Set{f(x)}{\inf{}{I} < x < \sup{}{I}}},
		\quad b_0 \coloneqq \sup{}{\Set{f(x)}{\inf{}{I} < x < \sup{}{I}}}
	\end{align}
	とすれば,$\inf{}{I} \in I$なら$a_0 = f(\inf{}{I})$,
	$\sup{}{I} \in I$なら$b_0 = f(\sup{}{I})$であり,
	\begin{align}
		f(x) \coloneqq 
		\begin{cases}
			a_0 & -\infty < x \leq \inf{}{I} \\
			f_I(x) & \inf{}{I} < x < \sup{}{I} \\
			b_0 & \sup{}{I} \leq x < \infty
		\end{cases}
	\end{align}
	により$f_I$を$f$に拡張して$\mu_0$を定めるとき,
	\begin{align}
		\mu_{0,I}(I \cap F) \coloneqq \mu_0(I \cap F)
	\end{align}
	は$\mathfrak{F}_I$上で完全加法的となる.
	$\mu_{0,I}$の拡張測度を$\mu_I$と書き,これを$f_I$のStieltjes測度と呼ぶ.
	$\mu_I$のLebesgue拡大をLebesgue-Stieltjes測度と呼び,
	特に$I = \R,\ f(x) = x$のときLebesgue測度と呼ぶ.
	
	\begin{screen}
		\begin{thm}[左半開区間のStiletjes測度]
			$(\alpha,\beta] \subset I,\ (-\infty < \alpha < \beta < \infty)$に対して
			\begin{align}
				\mu((\alpha,\beta]) = f(\beta) - f(\alpha).
			\end{align}
		\end{thm}
	\end{screen}
	
	\begin{screen}
		\begin{thm}[Riemann-Stieltjes積分との関係]
			$F:I \longrightarrow \C$が右連続或は左連続なら
		\end{thm}
	\end{screen}
	
	\begin{screen}
		\begin{thm}[時間変更]
			
		\end{thm}
	\end{screen}
\section{Fubiniの定理}
	$(X,\mathcal{M}),(Y,\mathcal{N})$を可測空間とするとき,
	任意の$x \in X$に対し
	\begin{align}
		p_x:Y \ni y \longmapsto (x,y) \in X \times Y
	\end{align}
	で定める$p_x$は$\mathcal{N}/\mathcal{M} \otimes \mathcal{N}$-可測である.
	実際,$A \in \mathcal{M},\ B \in \mathcal{N}$に対しては
	\begin{align}
		p_x^{-1}(A \times B) = 
		\begin{cases}
			\emptyset, & (x \notin A), \\
			B, & (x \in A),
		\end{cases}
		\in \mathcal{N}
	\end{align}
	となるから,
	\begin{align}
		\Set{A \times B}{A \in \mathcal{M},\ B \in \mathcal{N}}
		\subset \Set{E \in \mathcal{M} \otimes \mathcal{N}}{p_x^{-1}(E) \in \mathcal{N}}
	\end{align}
	が従い$p_x$の$\mathcal{N}/\mathcal{M} \otimes \mathcal{N}$-可測性が出る.
	同様に任意の$y \in Y$に対し
	\begin{align}
		q_y:X \ni x \longmapsto (x,y) \in X \times Y
	\end{align}
	で定める$q_y$は$\mathcal{M}/\mathcal{M} \otimes \mathcal{N}$-可測である.
	
	\begin{screen}
		\begin{lem}[二変数可測写像は片変数で可測]\label{lem:Fubini_lemma_1}
			$(X,\mathcal{M}),(Y,\mathcal{N}),(Z,\mathcal{L})$を可測空間とするとき,
			写像$f: X \times Y \longmapsto Z$が
			$\mathcal{M}\otimes \mathcal{N}/ \mathcal{L}$-可測であれば,
			任意の$x_0 \in X,\ y_0 \in Y$に対し
			\begin{align}
				X \ni x \longmapsto f(x,y_0),
				\quad Y \ni y \longmapsto f(x_0,y)
			\end{align}
			はそれぞれ$\mathcal{M}/\mathcal{L}$-可測,
			$\mathcal{N}/\mathcal{L}$-可測である.
		\end{lem}
	\end{screen}
	
	\begin{prf}
		$X \ni x \longmapsto f(x,y_0)$は$f$と$q_{y_0}$の合成$f \circ q_{y_0}$であり,
		$Y \ni y \longmapsto f(x_0,y)$は$f \circ p_{x_0}$である.
		\QED
	\end{prf}
	
	\begin{screen}
		\begin{lem}\label{lem:Fubini_theorem}
			$(X,\mathcal{M},\mu),(Y,\mathcal{N},\nu)$を$\sigma$-有限な測度空間とするとき,
			任意の$Q \in \mathcal{M} \otimes \mathcal{N}$に対し
			\begin{align}
				\varphi_Q: X \ni x \longmapsto \int_Y \defunc_{Q}\circ p_x\ d\nu,
				\quad \psi_Q: Y \ni y \longmapsto \int_X \defunc_{Q} \circ q_y\ d\mu,
			\end{align}
			はそれぞれ$\mathcal{M}/\borel{[0,\infty]}$-可測,
			$\mathcal{N}/\borel{[0,\infty]}$-可測であり
			\begin{align}
				\int_X \varphi_Q\ d\mu
				= (\mu \otimes \nu)(Q)
				= \int_Y \psi_Q\ d\nu
				\label{eq:lem_Fubini_theorem_1}
			\end{align}
			が成立する.
		\end{lem}
	\end{screen}
	
	\begin{prf}\mbox{}
		\begin{description}
			\item[第一段]
				$\sigma$-有限の仮定より,
				\begin{align}
					\bigcup_{n=1}^\infty X_n = X,
					\quad \bigcup_{n=1}^\infty Y_n = Y,
					\quad \mu(X_n),\ \nu(Y_n) < \infty;
					\ n = 1,2,\cdots
				\end{align}
				を満たす増大列$\{X_n\}_{n=1}^\infty \subset \mathcal{M}$と
				$\{Y_n\}_{n=1}^\infty \subset \mathcal{N}$が存在する.ここで
				\begin{align}
					\mathcal{M}_n \coloneqq \Set{A \cap X_n}{A \in \mathcal{M}},
					\quad \mathcal{N}_n \coloneqq \Set{B \cap Y_n}{B \in \mathcal{N}}
				\end{align}
				により$X_n,Y_n$上の$\sigma$-加法族を定めて
				\begin{align}
					\mathcal{D}_n \coloneqq
					\Set{Q_n \in \mathcal{M}_n \otimes \mathcal{N}_n}{
					\substack{
					\displaystyle \varphi_{Q_n}: X \ni x \longmapsto \int_Y \defunc_{Q_n} \circ p_x\ d\nu \mbox{ が$\mathcal{M}/\borel{[0,\infty]}$-可測},\\
					\displaystyle \psi_{Q_n}: Y \ni y \longmapsto \int_X \defunc_{Q_n} \circ q_y\ d\mu \mbox{ が$\mathcal{N}/\borel{[0,\infty]}$-可測},\\
					\displaystyle \int_X \varphi_{Q_n}\ d\mu
					= (\mu \otimes \nu)(Q_n)
					= \int_Y \psi_{Q_n}\ d\nu}} 
				\end{align}
				とおけば,$\mathcal{D}_n$は$X_n \times Y_n$上のDynkin族であり
				\begin{align}
					\Set{A \times B}{A \in \mathcal{M}_n,\ B \in \mathcal{N}_n}
					\subset \mathcal{D}_n
				\end{align}
				を満たすから$\mathcal{M}_n \otimes \mathcal{N}_n = \mathcal{D}_n$が従う.
			
			\item[第二段]
				$\mathcal{M}_n \otimes \mathcal{N}_n = \Set{Q \cap (X_n \times Y_n)}{Q \in \mathcal{M} \otimes \mathcal{N}}$より,任意の$Q \in \mathcal{M} \otimes \mathcal{N}$に対して
				\begin{align}
					Q_n \coloneqq Q \cap (X_n \times Y_n) \in \mathcal{D}_n,
					\ (\forall n \geq 1),
					\quad Q_1 \subset Q_2 \subset \cdots \longrightarrow Q
				\end{align}
				が従い,単調収束定理より
				\begin{align}
					\varphi_Q(x) = \int_Y \defunc_Q \circ p_x\ d\nu
					= \lim_{n \to \infty} \int_Y \defunc_{Q_n} \circ p_x\ d\nu
					= \lim_{n \to \infty} \varphi_{Q_n}(x),
					\quad (\forall x \in X)
				\end{align}
				となるから$\varphi_Q$の$\mathcal{M}/\borel{[0,\infty]}$-可測性が出る.
				また,
				\begin{align}
					\varphi_{Q_n}(x) = \int_Y \defunc_{Q_n} \circ p_x\ d\nu
					\leq \int_Y \defunc_{Q_{n+1}} \circ p_x\ d\nu
					= \varphi_{Q_{n+1}}(x),
					\quad (n=1,2,\cdots)
				\end{align}
				が満たされているから,再び単調収束定理により
				\begin{align}
					\int_X \varphi_Q\ d\mu
					= \lim_{n \to \infty} \int_X \varphi_{Q_n}\ d\mu
					= \lim_{n \to \infty} (\mu \otimes \nu)(Q_n)
					= (\mu \otimes \nu)(Q)
				\end{align}
				が得られる.同様に$\psi_Q$は$\mathcal{N}/\borel{[0,\infty]}$-可測であり
				(\refeq{eq:lem_Fubini_theorem_1})を満たす.
				\QED
		\end{description}
	\end{prf}
	
	\begin{screen}
		\begin{thm}[Fubini]
			$(X,\mathcal{M},\mu),(Y,\mathcal{N},\nu)$を$\sigma$-有限な測度空間とする.
			\begin{description}
				\item[(1)]
					$f:X \times Y \longrightarrow [0,\infty]$を
					$\mathcal{M} \otimes \mathcal{N}/\borel{[0,\infty]}$-可測写像とするとき,
					\begin{align}
						\varphi: X \ni x \longmapsto \int_Y f \circ p_x\ d\nu,
						\quad \psi: Y \ni y \longmapsto \int_X f \circ q_y\ d\mu
					\end{align}
					により定める$\varphi,\psi$はそれぞれ$\mathcal{M}/\borel{[0,\infty]}$-可測,
					$\mathcal{N}/\borel{[0,\infty]}$-可測であり,
					\begin{align}
						\int_X \varphi\ d\mu
						= \int_{X \times Y} f\ d(\mu \otimes \nu)
						= \int_Y \psi\ d\nu
					\end{align}
					が成立する.
					
				\item[(2)]
					$f:X \times Y \longrightarrow \C$を
					$\mathcal{M} \otimes \mathcal{N}/\borel{\C}$-可測な
					可積分関数とするとき,
			\end{description}
		\end{thm}
	\end{screen}
	
	\begin{screen}
		\begin{thm}[$n$変数関数のFubiniの定理]
			$\left((X_i,\mathcal{M}_i,\mu_i)\right)_{i=1}^n,\ (n \geq 3)$を
			$\sigma$-有限な測度空間の族とし,
			\begin{align}
				\{i_1,\cdots,i_k\} \cup \{j_1,\cdots,j_h\} = \{1,2,\cdots,n\},
				\quad \{i_1,\cdots,i_k\} \cap \{j_1,\cdots,j_h\} = \emptyset
			\end{align}
			を満たす添数列$i_1, \cdots, i_k$と$j_1, \cdots, j_h,\ (1 \leq k,h \leq n-1)$を任意に取り
			\begin{align}
				&Y \coloneqq \prod_{i=1}^n X_i,
				\quad Y_1 \coloneqq \prod_{\ell=1}^k X_{i_\ell},
				\quad Y_2 \coloneqq \prod_{\ell=1}^h X_{j_\ell}, \\
				&\mathcal{N} \coloneqq \bigotimes_{i=1}^n \mathcal{M}_i,
				\quad \mathcal{N}_1 \coloneqq \bigotimes_{\ell=1}^k \mathcal{M}_{i_\ell},
				\quad \mathcal{N}_2 \coloneqq \bigotimes_{\ell=1}^h \mathcal{M}_{j_\ell}, \\
				&\mu \coloneqq \bigotimes_{i=1}^n \mu_i,
				\quad \nu_1 \coloneqq \bigotimes_{\ell=1}^k \mu_{i_\ell},
				\quad \nu_2 \coloneqq \bigotimes_{\ell=1}^h \mu_{j_\ell}
			\end{align}
			とおく.また
			\begin{align}
				p_{y_1}:Y_2 \ni y_2 \longmapsto (y_1,y_2),\ (\forall y_1 \in Y_1),
				\quad q_{y_2}:Y_1 \ni y_1 \longmapsto (y_1,y_2),\ (\forall y_2 \in Y_2)
			\end{align}
			とする.このとき,射影$\pi_1:Y \longrightarrow Y_1,\ \pi_2:Y \longrightarrow Y_2$に対し
			\begin{align}
				\varphi: Y_1 \times Y_2 \ni (y_1,y_2) \longmapsto \pi_1^{-1}(y_1) \cap \pi_2^{-1}(y_2)
			\end{align}
			により$\varphi:Y_1 \times Y_2 \longrightarrow Y$を定めれば
			$\varphi$は$\mathcal{N}_1 \otimes \mathcal{N}_2/\mathcal{N}$-可測であり,
			更に以下が成立する:
			\begin{description}
				\item[(1)] $f:Y \longrightarrow [0,\infty]$が$\mathcal{N}/\borel{[0,\infty]}$-可測なら次が成り立つ:
					\begin{align}
						\int_Y f\ d\mu
						= \int_{Y_1} \int_{Y_2} f \left(\varphi\left(p_{y_1}(y_2)\right)\right)\ \nu_2(dy_2)\ \nu_1(dy_1)
						= \int_{Y_2} \int_{Y_1} f \left(\varphi\left(q_{y_2}(y_1)\right)\right)\ \nu_1(dy_1)\ \nu_2(dy_2).
					\end{align}
			\end{description}
		\end{thm}
	\end{screen}
	
	\begin{prf}\mbox{}
		\begin{description}
			\item[第一段]
				$\varphi$の$\mathcal{N}_1 \otimes \mathcal{N}_2/\mathcal{N}$-可測性を示す.
				実際,$\varphi:Y_1 \times Y_2 \longrightarrow Y$が全単射であることより
				\begin{align}
					\varphi^{-1}(E_1 \times \cdots \times E_n)
					= \prod_{\ell=1}^k E_{i_\ell} \times \prod_{\ell=1}^h E_{j_\ell}
					\in \mathcal{N}_1 \otimes \mathcal{N}_2,
					\quad (\forall E_i \in \mathcal{M}_i,\ i=1,\cdots,n)
					\label{eq:Fubini_theorem_n_variables_1}
				\end{align}
				が成り立つから
				\begin{align}
					\Set{E_1 \times \cdots \times E_n}{E_i \in \mathcal{M}_i,\ i=1,\cdots,n}
					\subset \Set{E \in \mathcal{N}}{\varphi^{-1}(E) \in \mathcal{N}_1 \otimes \mathcal{N}_2}
				\end{align}
				となり,左辺は$\mathcal{N}$を生成するから$\varphi$は$\mathcal{N}_1 \otimes \mathcal{N}_2/\mathcal{N}$-可測である.
				
			\item[第二段]
				$f = \defunc_E\ (E \in \mathcal{N})$に対し
				\begin{align}
					\int_Y f\ d\mu 
					= \int_{Y_1 \times Y_2} f \circ \varphi\ d(\nu_1 \otimes \nu_2)
				\end{align}
				となることを示す.実際,(\refeq{eq:Fubini_theorem_n_variables_1})より
				\begin{align}
					\Set{E_1 \times \cdots \times E_n}{E_i \in \mathcal{M}_i,\ i=1,\cdots,n}
					\subset \Set{E \in \mathcal{N}}{\mu(E) = \nu_1 \otimes \nu_2\left( \varphi^{-1}(E) \right)}
				\end{align}
				となるから,Dinkin族定理より任意の$E \in \mathcal{N}$に対して
				$\mu(E) = \nu_1 \otimes \nu_2\left( \varphi^{-1}(E) \right)$が成立し
				\begin{align}
					\int_Y f\ d\mu
					= \mu(E)
					= \nu_1 \otimes \nu_2\left( \varphi^{-1}(E) \right)
					= \int_{Y_1 \times Y_2} f \circ \varphi\ d(\nu_1 \otimes \nu_2)
				\end{align}
				が従う.
		\end{description}
	\end{prf}
\section{$L^p$空間}

測度空間を$(X,\mathscr{F},\mu)$とする.$\mathscr{F}/\borel{\C}$-可測関数$f$に対して
\begin{align}
	\Norm{f}{\mathscr{L}^p} \coloneqq
	\begin{cases}
		\inf{}{\Set{r \in \C}{|f(x)| \leq r\quad \mbox{$\mu$-a.e.}x \in X}} & (p = \infty) \\
		\displaystyle\left( \int_{X} |f(x)|^p\ \mu(dx) \right)^{1/p} & (0 < p < \infty)
	\end{cases}
\end{align}
により$\Norm{\cdot}{\mathscr{L}^p}$を定め,
\begin{align}
	\mathscr{L}^p(X,\mathscr{F},\mu) \coloneqq \Set{f:X \rightarrow \C}{f:\mbox{可測}\mathscr{F}/\borel{\C},\ \Norm{f}{\mathscr{L}^p} < \infty} \quad (1 \leq p \leq \infty)
\end{align}
で空間$\mathscr{L}^p(X,\mathscr{F},\mu)$を定義する.$\mathscr{L}^p(\mu)$とも略記する.

\begin{screen}
	\begin{lem}\label{lem:holder_inequality}
		任意の$f \in \mathscr{L}^\infty(X,\mathscr{F},\mu)$に対して次が成り立つ:
		\begin{align}
			|f| \leq \Norm{f}{\mathscr{L}^\infty} \quad \mbox{$\mu$-a.e.}
		\end{align}
	\end{lem}
\end{screen}

\begin{prf}
	$\mathscr{L}^\infty(X,\mathscr{F},\mu)$の定義より任意の実数$\alpha > \Norm{f}{\mathscr{L}^\infty}$に対して
	\begin{align}
		\mu\left( \Set{x \in X}{|f(x)| > \alpha} \right) = 0
	\end{align}
	が成り立つから,
	\begin{align}
		\Set{x \in X}{|f(x)| > \Norm{f}{\mathscr{L}^\infty}} = \bigcup_{n =1}^{\infty} \Set{x \in X}{|f(x)| > \Norm{f}{\mathscr{L}^\infty} + \frac{1}{n}}
	\end{align}
	の右辺は$\mu$-零集合であり主張が従う.
	\QED
\end{prf}

\begin{screen}
	\begin{thm}[H\Ddot{o}lderの不等式]\label{thm:holder_inequality}
		$1 \leq p, q \leq \infty$,$p + q = pq\ (p = \infty$なら$q = 1)$とする.このとき
		任意の$\mathscr{F}/\borel{\C}$-可測関数$f,g$に対して次が成り立つ:
		\begin{align}
			\int_{X} |fg|\ d\mu \leq \Norm{f}{\mathscr{L}^p} \Norm{g}{\mathscr{L}^q}. \label{ineq:holder}
		\end{align}
	\end{thm}
\end{screen}

\begin{prf}
	$\Norm{f}{\mathscr{L}^p} = \infty$又は$\Norm{g}{\mathscr{L}^q} = \infty$なら(\refeq{ineq:holder})
		は成り立つから,$\Norm{f}{\mathscr{L}^p} < \infty$かつ$\Norm{g}{\mathscr{L}^q} < \infty$とする.
	\begin{description}
		\item[$p = \infty,\ q = 1$の場合]
			補題\ref{lem:holder_inequality}により或る零集合$A$が存在して
			\begin{align}
				|f(x)g(x)| \leq \Norm{f}{\mathscr{L}^\infty}|g(x)| \quad (\forall x \in X \backslash A).
			\end{align}
			が成り立つから,
			\begin{align}
				\int_{X} |fg|\ d\mu = \int_{X \backslash A} |fg|\ d\mu
				\leq \Norm{f}{\mathscr{L}^\infty} \int_{X \backslash A} |g|\ d\mu 
				= \Norm{f}{\mathscr{L}^\infty} \Norm{g}{\mathscr{L}^1}
			\end{align}
			が従い不等式(\refeq{ineq:holder})を得る.
		
		\item[$1 < p,q < \infty$の場合]
			$\Norm{f}{\mathscr{L}^p} = 0$のとき
			\begin{align}
				B \coloneqq \Set{x \in X}{|f(x)| > 0}
			\end{align}
			は零集合であるから,
			\begin{align}
				\int_{X} |fg|\ d\mu = \int_{X \backslash B} |fg|\ d\mu = 0
			\end{align}
			となり(\refeq{ineq:holder})を得る.$\Norm{g}{\mathscr{L}^q} = 0$の場合も同じである.
			次に$0 < \Norm{f}{\mathscr{L}^p},\Norm{g}{\mathscr{L}^q} < \infty$の場合を示す.
			実数値対数関数$(0,\infty) \ni t \longmapsto -\Log{t}$は凸であるから,$1/p + 1/q = 1$に対して
			\begin{align}
				-\Log{\left( \frac{s}{p} + \frac{t}{q} \right)} \leq \frac{1}{p}(-\Log{s}) + \frac{1}{q}(-\Log{t}) \quad (\forall s,t > 0)
			\end{align}
			を満たし
			\begin{align}
				s^{1/p}t^{1/q} \leq \frac{s}{p} + \frac{t}{q} \quad (\forall s,t > 0)
			\end{align}
			が従う.ここで
			\begin{align}
				F \coloneqq \frac{|f|^p}{\Norm{f}{\mathscr{L}^p}^p},
				\quad G \coloneqq \frac{|g|^q}{\Norm{g}{\mathscr{L}^q}^q}
			\end{align}
			により可積分関数$F,G$を定めれば,
			\begin{align}
				F(x)^{1/p}G(x)^{1/q} \leq \frac{1}{p}F(x) + \frac{1}{q}G(x) \quad (\forall x \in X)
			\end{align}
			が成り立つから
			\begin{align}
				\frac{1}{\Norm{f}{\mathscr{L}^p}\Norm{g}{\mathscr{L}^q}}\int_{X} |fg|\ d\mu
				= \int_{X} F^{1/p}G^{1/q}\ d\mu
				\leq \frac{1}{p} \int_{X} F\ d\mu + \frac{1}{q} \int_{X} G\ d\mu
				= \frac{1}{p} + \frac{1}{q} = 1
			\end{align}
			が従い,$\Norm{f}{\mathscr{L}^p}\Norm{g}{\mathscr{L}^q}$を移項して不等式(\refeq{ineq:holder})を得る.
			\QED
	\end{description}
\end{prf}

\begin{screen}
	\begin{thm}[Minkowskiの不等式]\label{thm:minkowski_inequality}
		$1 \leq p \leq \infty$のとき,
		任意の$\mathscr{F}/\borel{\C}$-可測関数$f,g$に対して次が成り立つ:
		\begin{align}
			\Norm{f+g}{\mathscr{L}^p} \leq \Norm{f}{\mathscr{L}^p} + \Norm{g}{\mathscr{L}^p}. \label{ineq:minkowski}
		\end{align}
	\end{thm}
\end{screen}

\begin{prf}
	$\Norm{f+g}{\mathscr{L}^p} = 0,\ \Norm{f}{\mathscr{L}^p} = \infty,\ \Norm{g}{\mathscr{L}^p} = \infty$
	のいずれかが満たされていれば(\refeq{ineq:minkowski})は成り立つから,
	$\Norm{f+g}{\mathscr{L}^p} > 0$かつ$\Norm{f}{\mathscr{L}^p} < \infty$かつ$\Norm{g}{\mathscr{L}^p} < \infty$
	の場合を考える.
	\begin{description}
		\item[$p = \infty$の場合]
			補題\ref{lem:holder_inequality}により
			\begin{align}
				C \coloneqq \Set{x \in X}{|f(x)| > \Norm{f}{\mathscr{L}^\infty}} \cup \Set{x \in X}{|g(x)| > \Norm{g}{\mathscr{L}^\infty}}
			\end{align}
			は零集合であり,
			\begin{align}
				|f(x) + g(x)| \leq |f(x)| + |g(x)| \leq \Norm{f}{\mathscr{L}^\infty} + \Norm{g}{\mathscr{L}^\infty} \quad (\forall x \in X \backslash C)
			\end{align}
			が成り立ち(\refeq{ineq:minkowski})が従う.
		
		\item[$p = 1$の場合]
			\begin{align}
				\int_X |f + g|\ d\mu \leq \int_X |f| + |g|\ d\mu = \Norm{f}{\mathscr{L}^1} + \Norm{g}{\mathscr{L}^1}
			\end{align}
			より(\refeq{ineq:minkowski})が従う.
		
		\item[$1 < p < \infty$の場合]
			$q$を$p$の共役指数とする.
			\begin{align}
				|f+g|^p = |f+g||f+g|^{p-1} \leq |f||f+g|^{p-1} + |g||f+g|^{p-1}
			\end{align}
			が成り立つから,H\Ddot{o}lderの不等式より
			\begin{align}
				\Norm{f+g}{\mathscr{L}^p}^p &= \int_{X} |f+ g|^p\ d\mu \\
				&\leq \int_{X} |f||f+g|^{p-1}\ d\mu + \int_{X} |g||f+g|^{p-1}\ d\mu \\
				&\leq \Norm{f}{\mathscr{L}^p}\Norm{f+g}{\mathscr{L}^p}^{p-1} + \Norm{g}{\mathscr{L}^p}\Norm{f+g}{\mathscr{L}^p}^{p-1}
				\label{Minkowski_1}
			\end{align}
			が得られる.また$|f|^p,|g|^p$の可積性と
			\begin{align}
				|f + g|^p \leq 2^p \left( |f|^p + |g|^p \right)
			\end{align}
			により$\Norm{f+g}{\mathscr{L}^p} < \infty$が従うから,
			(\refeq{Minkowski_1})の両辺を$\Norm{f+g}{\mathscr{L}^p}^{p-1}$で割って(\refeq{ineq:minkowski})を得る.
			\QED
	\end{description}
\end{prf}

以上の結果より$\mathscr{L}^p(X,\mathscr{F},\mu)$は線形空間となる.実際線型演算は
\begin{align}
	(f+g)(x) \coloneqq f(x) + g(x), \quad (\alpha f)(x) \coloneqq \alpha f(x),
	\quad (\forall x \in X,\ f,g \in \mathscr{L}^p(\mu),\ \alpha \in \C)
\end{align}
により定義され,Minkowskiの不等式により加法について閉じている.

\begin{screen}
	\begin{lem}
		$1 \leq p \leq \infty$に対し,$\Norm{\cdot}{\mathscr{L}^p}$は線形空間$\mathscr{L}^p(X,\mathscr{F},\mu)$のセミノルムである.
	\end{lem}
\end{screen}

\begin{prf}\mbox{}
	\begin{description}
	\item[半正値性] $\Norm{\cdot}{\mathscr{L}^p}$が正値であることは定義による.
		一方で,$E \neq \emptyset$を満たす$\mu$-零集合$E$が存在するとき,
		\begin{align}
			f(x) \coloneqq
			\begin{cases}
				1 & (x \in E) \\
				0 & (x \in \Omega \backslash E)
			\end{cases}
		\end{align}
		で定める$f$は零写像ではないが$\Norm{f}{\mathscr{L}^p} = 0$となる.
		
	\item[同次性] 
		任意に$\alpha \in \C,\ f \in \mathscr{L}^p(\mu)$を取る.
		$1 \leq p < \infty$の場合は
		\begin{align}
			\left( \int_{X} |\alpha f|^p\ d\mu \right)^{1/p} = \left( |\alpha|^p \int_{X} |f|^p\ d\mu \right)^{1/p} 
			= |\alpha| \left( \int_{X} |f|^p\ d\mu \right)^{1/p}
		\end{align}
		により,$p = \infty$の場合は
		\begin{align}
			\inf{}{\Set{r \in \R}{|\alpha f(x)| \leq r \quad \mbox{$\mu$-a.e.}x \in X}} = |\alpha|\inf{}{\Set{r \in \R}{|f(x)|  \leq r \quad \mbox{$\mu$-a.e.}x \in X}}
		\end{align}
		により$\Norm{\alpha f}{\mathscr{L}^p} = |\alpha|\Norm{f}{\mathscr{L}^p}$が成り立つ.
		
	\item[三角不等式] Minkowskiの不等式より従う.
	\QED
	\end{description}
\end{prf}

$\mathscr{L}^p$はノルム空間ではないが,同値類でまとめることによりノルム空間となる.
\begin{description}
	\item[可測関数全体の商集合]
		$\mathscr{F}/\borel{\C}$-可測関数全体の集合を
		\begin{align}
			\mathscr{L}^0(X,\mathscr{F},\mu) \coloneqq \Set{f:X \rightarrow \C}{f:\mbox{可測}\mathscr{F}/\borel{\C}}
		\end{align}
		とおく.$f,g \in \mathscr{L}^0(X,\mathscr{F},\mu)$に対し
		\begin{align}
			 f \sim g \quad \overset{\mathrm{def}}{\Longleftrightarrow} \quad f = g \quad \mbox{$\mu$-a.e.}
		\end{align}
		により定める$\sim$は同値関係であり,$\sim$による$\mathscr{L}^0(X,\mathscr{F},\mu)$の商集合を
		$L^0(X,\mathscr{F},\mu)$と表す.
	
	\item[商集合における算法]
		$L^0(\mu)$の元である関数類(同値類)を$[f]\ $($f$は関数類の代表)と表せば,$L^0(\mu)$は
		\begin{align}
			[f] + [g] \coloneqq [f+g],
			\quad \alpha [f] \coloneqq [\alpha f], \quad ([f],[g] \in L^0(\mu),\ \alpha \in \C).
		\end{align}
		を線型演算として$\C$上の線形空間となる.また
		\begin{align}
			[f][g] \coloneqq [fg] \quad \left([f],[g] \in L^{0}(\mu) \right).
		\end{align}
		を乗法として$L^0(\mu)$は環となる.$L^0(\mu)$の零元は零写像の関数類でありこれを[0]と書く.また
		単位元は恒等的に$1$を取る関数の関数類でありこれを[1]と書く.
		減法は
		\begin{align}
			[f] - [g] \coloneqq [f] + (-[g]) = [f] + [-g] = [f - g]
		\end{align}
		により定める.
	
	\item[関数類の順序]
		$[f],[g] \in L^0(\mu)$に対して次の関係$<(>)$を定める:
		\begin{align}
			[f] < [g]\ \left( [g] > [f] \right) \quad
			\overset{\mathrm{def}}{\Longleftrightarrow}
			\quad f < g \quad \mbox{$\mu$-a.s.} \label{dfn:equiv_class_order}
		\end{align}
		この定義はwell-definedである.実際任意の$f' \in [f],g' \in [g]$に対して
		\begin{align}
			\left\{ f' \geq g' \right\} \subset \left\{ f \neq f' \right\} \cup \left\{ f \geq g \right\} \cup \left\{ g \neq g' \right\}
		\end{align}
		の右辺は零集合であるから
		\begin{align}
			[f] < [g] \Leftrightarrow [f'] < [g']
		\end{align}
		が従う.$<(>)$または$=$であることを$\leq(\geq)$と書くとき,任意の$[f],[g],[h] \in L^0(\mu)$に対し,
		\begin{itemize}
			\item $[f] \leq [f]$が成り立つ.
			\item $[f] \leq [g]$かつ$[g] \leq [f]$ならば$[f] = [g]$が成り立つ.
			\item $[f] \leq [g],\ [g] \leq [h]$ならば$[f] \leq [h]$が成り立つ.
		\end{itemize}
		が満たされるから$\leq$は$L^0(\mu)$における順序となる.
\end{description}

\begin{screen}
	\begin{dfn}[商空間におけるノルムの定義]
		\begin{align}
			\Norm{[f]}{L^p} \coloneqq \Norm{f}{\mathscr{L}^p} 
			\quad (f \in \mathscr{L}^p(X,\mathscr{F},\mu),\ 1 \leq p \leq \infty)
		\end{align}
		により定める$\Norm{\cdot}{L^p}:L^0(X,\mathscr{F},\mu) \rightarrow \R$は関数類の代表に依らずに値が確定する.
		そして
		\begin{align}
			L^p(X,\mathscr{F},\mu) \coloneqq \Set{[f] \in L^0(X,\mathscr{F},\mu)}{\Norm{[f]}{L^p} < \infty} \quad (1 \leq p \leq \infty)
		\end{align}
		として定める空間は$\Norm{\cdot}{L^p}$をノルムとしてノルム空間となる.
	\end{dfn}
\end{screen}

\begin{screen}
	\begin{thm}[$L^p$はBanach空間]\label{thm:Lp_banach}
		ノルム空間$L^p(X,\mathscr{F},\mu)\ (1 \leq p \leq \infty)$の任意のCauchy列$\left( [f_n] \right)_{n=1}^\infty$
		に対してノルム収束極限$[f] \in L^p(\mu)$が存在する.
		また,このとき或る部分列$\left( \left[f_{n_k}\right] \right)_{k=1}^\infty$の代表は
		$f$に概収束する:
		\begin{align}
			\lim_{k \to \infty} f_{n_k} = f, \quad \mbox{$\mu$-a.e.}
		\end{align}
	\end{thm}
\end{screen}

\begin{prf}
	任意にCauchy列$[f_n] \in L^p(\mu)\ (n=1,2,3,\cdots)$を取れば,
	或る$N_1 \in \N$が存在して
	\begin{align}
		\Norm{[f_n]-[f_m]}{L^p} < \frac{1}{2}
		\quad (\forall n > m \geq N_1)
	\end{align}
	を満たす.ここで$m > N_1$を一つ選び$n_1$とおく.
	同様に$N_2 > N_1$を満たす$N_2 \in \N$が存在して
	\begin{align}
		\Norm{[f_n]-[f_m]}{L^p} < \frac{1}{2^2}
		\quad (\forall n > m \geq N_2)
	\end{align}
	を満たすから,$m > N_2$を一つ選び$n_2$とおけば
	\begin{align}
		\Norm{\left[f_{n_1}\right] - \left[f_{n_2}\right]}{L^p} < \frac{1}{2}
	\end{align}
	が成り立つ.同様の操作を繰り返して
	\begin{align}
		\Norm{\left[f_{n_k}\right] - \left[f_{n_{k+1}}\right]}{L^p} < \frac{1}{2^k} 
		\quad (n_k < n_{k+1},\ k=1,2,3,\cdots) \label{ineq:Lp_banach_2}
	\end{align}
	を満たす部分添数列$(n_k)_{k=1}^{\infty}$を構成する.
	\begin{description}
		\item[$p = \infty$の場合]
			$\left[f_{n_k}\right]$の代表$f_{n_k}\ (k=1,2,\cdots)$に対して
			\begin{align}
				A_k &\coloneqq \Set{x \in X}{\left| f_{n_k}(x) \right| > \Norm{f_{n_k}}{\mathscr{L}^\infty}}, \\
				A^k &\coloneqq \Set{x \in X}{\left| f_{n_k}(x) - f_{n_{k+1}}(x) \right| > \Norm{f_{n_k} - f_{n_{k+1}}}{\mathscr{L}^\infty}}
			\end{align}
			とおけば,補題\ref{lem:holder_inequality}より$\mu(A_k) = \mu(A^k) = 0\ (k=1,2,\cdots)$が成り立つ.
			\begin{align}
				A_\circ \coloneqq \bigcup_{k=1}^{\infty} A_k,
				\quad A^\circ \coloneqq \bigcup_{k=1}^{\infty}A^k,
				\quad A \coloneqq A_\circ \cup A^\circ
			\end{align}
			として$\mu$-零集合$A$を定めて
			\begin{align}
				\hat{f}_{n_k} \coloneqq f_{n_k} \defunc_{X \backslash A}
				\quad (\forall k=1,2,\cdots)
			\end{align}
			とおけば
			各$\hat{f}_{n_k}$は$\left[\hat{f}_{n_k}\right] = \left[f_{n_k}\right]$を満たす有界可測関数であり,
			(\refeq{ineq:Lp_banach_2})より
			\begin{align}
				\sup{x \in X}{\left|\hat{f}_{n_k}(x) - \hat{f}_{n_{k+1}}(x)\right|}
				\leq \Norm{\hat{f}_{n_k} - \hat{f}_{n_{k+1}}}{\mathscr{L}^\infty} < \frac{1}{2^k} \quad (k=1,2,3,\cdots) 
				\label{ineq:Lp_banach_1}
			\end{align}
			が成り立つ.
			このとき任意の$\epsilon > 0$に対し$1/2^N < \epsilon$を満たす$N \in \N$を取れば,$\ell > k > N$なら
			\begin{align}
				\left|\hat{f}_{n_k}(x) - \hat{f}_{n_{\ell}}(x)\right| 
				\leq \sum_{j=k}^{\ell-1}\left|\hat{f}_{n_j}(x) - \hat{f}_{n_{j+1}}(x)\right| 
				< \sum_{k > N} \frac{1}{2^k} = \frac{1}{2^N} < \epsilon
				\quad (\forall x \in X)
			\end{align}
			となるから,各点$x \in X$で$\left( \hat{f}_{n_k}(x) \right)_{k=1}^{\infty}$は$\C$のCauchy列となり収束する.
			\begin{align}
				\hat{f}(x) \coloneqq \lim_{k \to \infty} \hat{f}_{n_k}(x)
				\quad (\forall x \in X)
			\end{align}
			として$\hat{f}$を定めれば,$\hat{f}$は可測$\mathscr{F}/\borel{\C}$であり,且つ任意に$k \in \N$を取れば
			\begin{align}
				\sup{x \in X}{|\hat{f}_{n_k}(x) - \hat{f}(x)|} \leq \frac{1}{2^{k-1}} \label{ineq:Lp_banach_3}
			\end{align}
			を満たす.実際或る$y \in X$で$\alpha \coloneqq |\hat{f}_{n_k}(y) - \hat{f}(y)| > 1/2^{k-1}$が成り立つと仮定すれば,
			\begin{align}
				\left| \hat{f}_{n_k}(y) - \hat{f}_{n_\ell}(y) \right|
				\leq \sum_{j=k}^{\ell-1} \sup{x \in X}{\left|\hat{f}_{n_j}(x) - \hat{f}_{n_{j+1}}(x)\right|}
				< \sum_{j=k}^{\infty} \frac{1}{2^j}
				= \frac{1}{2^{k-1}}
				\quad (\forall \ell > k)
			\end{align}
			より
			\begin{align}
				0 < \alpha - \frac{1}{2^{k-1}} < \left| \hat{f}_{n_k}(y) - \hat{f}(y) \right| - \left| \hat{f}_{n_k}(y) - \hat{f}_{n_\ell}(y) \right|
				\leq \left| \hat{f}(y) - \hat{f}_{n_\ell}(y) \right|
				\quad (\forall \ell > k)
			\end{align}
			が従い各点収束に反する.不等式(\refeq{ineq:Lp_banach_3})により
			\begin{align}
				\sup{x \in X}{\left| \hat{f}(x) \right|} 
				< \sup{x \in X}{\left| \hat{f}(x) - \hat{f}_{n_k}(x) \right|} + \sup{x \in X}{\left| \hat{f}_{n_k}(x) \right|} 
				\leq \frac{1}{2^{k-1}} + \Norm{\hat{f}_{n_k}}{\mathscr{L}^\infty}
			\end{align}
			が成り立つから$\left[\hat{f}\right] \in L^\infty(\mu)$が従い,
			\begin{align}
				\Norm{\left[f_{n_k}\right] - \left[\hat{f}\right]}{L^\infty}
				= \Norm{\left[\hat{f}_{n_k}\right] - \left[\hat{f}\right]}{L^\infty}
				\leq \sup{x \in X}{|\hat{f}_{n_k}(x) - \hat{f}(x)|}
				\longrightarrow 0 \quad (k \longrightarrow \infty)
			\end{align}
			により部分列$\left( \left[f_{n_k}\right] \right)_{k=1}^{\infty}$が$\left[\hat{f}\right]$に収束するから
			元のCauchy列も$\left[\hat{f}\right]$に収束する.
			
		\item[$1 \leq p < \infty$の場合]
			$\left[f_{n_k}\right]$の代表$f_{n_k}\ (k=1,2,\cdots)$は
			\begin{align}	
				f_{n_k}(x) = f_{n_1}(x) + \sum_{j=1}^{k}\left( f_{n_j}(x) - f_{n_{j-1}}(x) \right) \quad (\forall x \in X)
				\label{eq:Lp_banach_3}
			\end{align}
			を満たし,これに対して
			\begin{align}
				g_k(x) &\coloneqq \left| f_{n_1}(x) \right| + \sum_{j=1}^{k} \left| f_{n_j}(x) - f_{n_{j-1}}(x) \right|
				\quad (\forall x \in X,\ k=1,2,\cdots)
			\end{align}
			により単調非減少な可測関数列$(g_k)_{k=1}^{\infty}$を定めれば,Minkowskiの不等式と(\refeq{ineq:Lp_banach_2})により
			\begin{align}
				\Norm{g_k}{\mathscr{L}^p} \leq \Norm{f_{n_1}}{\mathscr{L}^p} + \sum_{j=1}^{k}\Norm{f_{n_j} - f_{n_{j-1}}}{\mathscr{L}^p}
				< \Norm{f_{n_1}}{\mathscr{L}^p} + 1 < \infty
				\quad (k = 1,2,\cdots)
				\label{eq:thm_Lp_banach_1}
			\end{align}
			が成り立つ.ここで
			\begin{align}
				B_N \coloneqq \bigcap_{k=1}^{\infty} \Set{x \in X}{g_k(x) \leq N},
				\quad B \coloneqq \bigcup_{N=1}^{\infty} B_N
			\end{align}
			とおけば$(g_k)_{k=1}^{\infty}$は$B$上で各点収束し$X \backslash B$上では発散するが,
			$X \backslash B$は零集合である.実際
			\begin{align}
				\int_X g_k^p\ d\mu
				= \int_B g_k^p\ d\mu + \int_{X \backslash B} g_k^p\ d\mu
				\leq \left( \Norm{f_{n_1}}{\mathscr{L}^p} + 1 \right)^p,
				\quad (k=1,2,\cdots)
			\end{align}
			が満たされているから,単調収束定理より
			\begin{align}
				\int_B \lim_{k \to \infty} g_k^p\ d\mu + \int_{X \backslash B} \lim_{k \to \infty} g_k^p\ d\mu
				\leq \left( \Norm{f_{n_1}}{\mathscr{L}^p} + 1 \right)^p
			\end{align}
			が成り立ち$\mu(X \backslash B) = 0$が従う.$\mathscr{F}/\borel{\C}$-可測関数$g,f$を
			\begin{align}
				g \coloneqq \lim_{k \to \infty} g_k \defunc_B,
				\quad f \coloneqq \lim_{k \to \infty} f_{n_k} \defunc_B
			\end{align}
			で定義すれば,$|f| \leq g$と
			$g^p$の可積分性により$\left[f\right] \in L^p(\mu)$が成り立つ.
			また$\left|f_{n_k} - f\right|^p \leq 2^p g^p\ (\forall k=1,2,\cdots)$が満たされているから,
			Lebesgueの収束定理により
			\begin{align}
				\lim_{k \to \infty}\Norm{\left[f_{n_k}\right] - \left[f\right]}{L^p}^p
				= \lim_{k \to \infty} \int_X \left| f_{n_k} - f \right|^p\ d\mu = 0
			\end{align}
			が従い,部分列の収束により元のCauchy列も$\left[f\right]$に収束する.
			\QED
	\end{description}
\end{prf}

\section{複素測度}
	\begin{screen}
		\begin{dfn}[複素測度]
			$(X,\mathscr{F})$を可測空間とする.
			$\lambda: \mathscr{F} \longrightarrow \C$が
			任意の互いに素な列$(E_i)_{i=1}^{\infty} \subset \mathscr{F}$に対し
			\begin{align}
				\lambda\biggl( \sum_{i=1}^{\infty} E_i \biggr) = \sum_{i=1}^{\infty} \lambda(E_i)
				\label{eq:dfn_complex_measure}
			\end{align}
			を満たすとき,$\lambda$を複素測度(complex measure)という.
		\end{dfn}
	\end{screen}
	
	任意の全単射$\sigma:\N \rightarrow \N$に対し
	\begin{align}
		(E \coloneqq)\ \sum_{i=1}^{\infty}E_i = \sum_{i=1}^{\infty}E_{\sigma(i)}
	\end{align}
	が成り立つから
	\begin{align}
		\sum_{i=1}^{\infty} \lambda(E_i) = \lambda(E) = \sum_{i=1}^{\infty} \lambda(E_{\sigma(i)})
	\end{align}
	が従い,Riemannの級数定理より
	$\sum_{i=1}^{\infty} \lambda(E_i)$は絶対収束する.
	ここで,
	\begin{align}
		|\lambda(E)| \leq \mu(E) \quad (\forall E \in \mathscr{F})
		\label{radon_nikodym_1}
	\end{align}
	を満たすような或る$(X,\mathscr{F})$上の測度$\mu$が存在すると考える.
	このとき$\mu$は
	\begin{align}
		\sum_{i=1}^{\infty} |\lambda(E_i)| \leq \sum_{i=1}^{\infty} \mu(E_i) 
		= \mu\Biggl(\sum_{i=1}^{\infty} E_i\Biggr)
	\end{align}
	を満たすから
	\begin{align}
		\sup{}{\Set{\sum_{i=1}^{\infty} |\lambda(A_i)|}{E = \sum_{i=1}^\infty A_i,\ \{A_i\}_{i=1}^\infty \subset \mathscr{F}}} 
		\leq \mu(E),
		\quad (\forall E \in \mathscr{F})
		\label{radon_nikodym_2}
	\end{align}
	が成立する.実は,
	\begin{align}
		|\lambda|(E) \coloneqq 
		\sup{}{\Set{\sum_{i=1}^{\infty} |\lambda(A_i)|}{E = \sum_{i=1}^\infty A_i,\ \{A_i\}_{i=1}^\infty \subset \mathscr{F}}},
		\quad (\forall E \in \mathscr{F})
		\label{radon_nikodym_3}
	\end{align}
	で定める$|\lambda|$は(\refeq{radon_nikodym_1})を満たす最小の有限測度となる
	(定理\ref{thm:total_variation_measure},定理\ref{thm:total_variation_measure_bounded}).
	
	\begin{screen}
		\begin{dfn}[総変動・総変動測度]
			可測空間$(X,\mathscr{F})$上の複素測度$\lambda$に対し,(\refeq{radon_nikodym_3})で定める
			$|\lambda|$を$\lambda$の総変動測度(total variation measure)といい,$|\lambda|(X)$を
			$\lambda$の総変動(total variation)という.
		\end{dfn}
	\end{screen}
	特に$\lambda$が正値有限測度である場合は$\lambda = |\lambda|$が成り立つ.実際,任意の$E \in \mathscr{F}$に対して
	\begin{align}
		|\lambda|(E) = \sup{}{\Set{\sum_{i=1}^{\infty} |\lambda(A_i)|}{E = \sum_{i=1}^\infty A_i,\ \{A_i\}_{i=1}^\infty \subset \mathscr{F}}}
		= \lambda(E)
	\end{align}
	が成立する.

	\begin{screen}
		\begin{thm}[$|\lambda|$は測度]
			可測空間$(X,\mathscr{F})$上の複素測度$\lambda$に対して,
			(\refeq{radon_nikodym_3})で定める$|\lambda|$は正値測度である.
			\label{thm:total_variation_measure}
		\end{thm}
	\end{screen}
	
	\begin{prf}
		$|\lambda|$の正値性は(\refeq{radon_nikodym_3})より従うから,
		$|\lambda|$の完全加法性を示す.
		いま,互いに素な集合列$E_i \in \mathscr{F}\ (i=1,2,\cdots)$を取り
		$E \coloneqq \sum_{i=1}^{\infty} E_i$とおく.
		このとき,任意の$\epsilon > 0$に対して
		$E_i$の或る分割$(A_{ij})_{j=1}^{\infty} \subset \mathscr{F}$が存在して
		\begin{align}
			|\lambda|(E_i) \geq \sum_{j=1}^{\infty} |\lambda(A_{ij})| 
			> |\lambda|(E_i) - \frac{\epsilon}{2^i}
		\end{align}
		を満たすから,$E = \sum_{i,j=1}^{\infty} A_{ij}$と併せて
		\begin{align}
			|\lambda|(E) \geq \sum_{i,j=1}^{\infty} |\lambda(A_{ij})| \geq \sum_{i=1}^{\infty}\sum_{j=1}^{\infty} |\lambda(A_{ij})| > \sum_{i=1}^{\infty} |\lambda|(E_i) - \epsilon
		\end{align}
		となり,$\epsilon > 0$の任意性より
		\begin{align}
			|\lambda|(E) \geq \sum_{j=1}^{\infty} |\lambda|(E_j)
		\end{align}
		が従う.一方で$E$の任意の分割$(A_j)_{j=1}^{\infty} \subset \mathscr{F}$に対し
		\begin{align}
			\sum_{j=1}^{\infty} |\lambda(A_j)| 
			= \sum_{j=1}^{\infty} \left| \sum_{i=1}^{\infty} \lambda(A_j \cap E_i) \right|
			\leq \sum_{j=1}^{\infty} \sum_{i=1}^{\infty} |\lambda|(A_j \cap E_i)
			\leq \sum_{i=1}^{\infty} |\lambda|(E_i)
		\end{align}
		が成り立つから,$E$の分割について上限を取って
		\begin{align}
			|\lambda|(E) \leq \sum_{i=1} |\lambda|(E_i)
		\end{align}
		を得る.
		\QED
	\end{prf}
	
	\begin{screen}
		\begin{lem}\label{lem:total_variation_measure_bounded}
			$z_1,\cdots,z_N$を複素数とする.このとき,次を満たす或る部分集合$S \subset \{1,\cdots,N\}$が存在する:
			\begin{align}
				\left| \sum_{k \in S} z_k \right| \geq \frac{1}{2\pi} \sum_{k=1}^{N} |z_k|.
			\end{align}
		\end{lem}
	\end{screen}
	
	\begin{prf}
		$i = \sqrt{-1}$として,
		$z_k = |z_k|\exp{i \alpha_k}\ (-\pi \leq \alpha_k < \pi,\ k=1,\cdots,N)$を満たす$\alpha_1,\cdots,\alpha_N$を取り
		\begin{align}
			S(\theta) \coloneqq \Set{k \in \{1,\cdots,N\}}{\cos{(\alpha_k - \theta)}{} > 0},
			\quad (-\pi \leq \theta \leq \pi)
		\end{align}
		とおく.このとき,$\cos{x}{+} \coloneqq 0 \vee \cos{x}{}\ (x \in \R)$とすれば
		\begin{align}
			\left| \sum_{k \in S(\theta)} z_k \right| &= |\exp{-i\theta}|\left| \sum_{k \in S(\theta)} z_k \right| = \left| \sum_{k \in S(\theta)} |z_k|\exp{i(\alpha_k - \theta)} \right| \\
			&\geq \Re{\sum_{k \in S(\theta)} |z_k|\exp{i(\alpha_k - \theta)}} = \sum_{k \in S(\theta)} |z_k| \cos{(\alpha_k - \theta)}{} = \sum_{k=1}^{N} |z_k| \cos{(\alpha_k - \theta)}{+}
		\end{align}
		が成り立ち,最右辺は$\theta$に関して連続であるから最大値を達成する$\theta_0 \in [-\pi,\pi]$が存在する.
		$S \coloneqq S(\theta_0)$として
		\begin{align}
			\left| \sum_{k \in S} z_k \right| \geq \sum_{k=1}^{N} |z_k| \cos{(\alpha_k - \theta_0)}{+} \geq \sum_{k=1}^{N} |z_k| \cos{(\alpha_k - \theta)}{+}
			\quad (\forall \theta \in [-\pi, \pi])
		\end{align}
		となり,積分して
		\begin{align}
			\left| \sum_{k \in S} z_k \right| 
			&\geq \sum_{k=1}^{N} |z_k| \frac{1}{2\pi} \int_{[-\pi,\pi]} \cos{(\alpha_k - \theta)}{+}\ d\theta \\
			&= \sum_{k=1}^{N} |z_k| \frac{1}{2\pi} \int_{[-\pi,\pi]} \cos{\theta}{+}\ d\theta
			= \frac{1}{2\pi} \sum_{k=1}^{N} |z_k|
		\end{align}
		が得られる.
		\QED
	\end{prf}
	
	\begin{screen}
		\begin{thm}[総変動は有限]\label{thm:total_variation_measure_bounded}
			可測空間$(X,\mathscr{F})$上の複素測度$\lambda$の総変動測度$|\lambda|$について次が成り立つ:
			\begin{align}
				|\lambda|(X) < \infty.
			\end{align}
			特に,複素測度は有界である.
		\end{thm}
	\end{screen}

	\begin{prf} $|\lambda|(X) = \infty$と仮定して背理法により定理を導く.
		\begin{description}
		\item[第一段]
			或る$E \in \mathscr{F}$に対し$|\lambda|(E) = \infty$が成り立っているなら,
			\begin{align}
				|\lambda(A)| > 1, \quad |\lambda(B)| > 1, \quad E = A + B
			\end{align}
			を満たす$A,B \in \mathscr{F}$が存在することを示す.いま,$t \coloneqq 2\pi(1 + |\lambda(E)|)$とおけば
			\begin{align}
				\sum_{i=1}^{\infty} |\lambda(E_i)| > t
			\end{align}
			を満たす$E$の分割$(E_i)_{i=1}^{\infty}$が存在する.従って或る$N \in \N$に対し
			\begin{align}
				\sum_{i=1}^{N} |\lambda(E_i)| > t
			\end{align}
			が成り立ち,補題\ref{lem:total_variation_measure_bounded}より
			\begin{align}
				\left| \sum_{k \in S} \lambda(E_k) \right| \geq \frac{1}{2\pi} \sum_{k=1}^{N} |\lambda(E_k)| > \frac{t}{2\pi} > 1
			\end{align}
			を満たす$S \subset \{1,\cdots,N\}$が取れる.ここで$A \coloneqq \sum_{k \in S} E_k,\ B \coloneqq E - A$とおけば,
			$|\lambda(A)| > 1$かつ
			\begin{align}
				|\lambda(B)| = |\lambda(E)-\lambda(A)| \geq |\lambda(A)| - |\lambda(E)| > \frac{t}{2\pi} - |\lambda(E)| = 1
			\end{align}
			が成り立つ.また,
			\begin{align}
				|\lambda|(E) = |\lambda|(A) + |\lambda|(B)
			\end{align}
			より$|\lambda|(A),\ |\lambda|(B)$の少なくとも一方は$\infty$となる.
		
		\item[第二段]
			いま,$|\lambda|(X) = \infty$と仮定すると,前段の結果より
			\begin{align}
				|\lambda|(B_1) = \infty, \quad |\lambda(A_1)| > 1, \quad |\lambda(B_1)| > 1,
				\quad X = A_1 + B_1
			\end{align}
			を満たす$A_1,B_1 \in \mathscr{F}$が存在する.同様に$B_1$に対しても
			\begin{align}
				|\lambda|(B_2) = \infty, \quad |\lambda(A_2)| > 1, \quad |\lambda(B_2)| > 1,
				\quad B_1 = A_2 + B_2
			\end{align}
			を満たす$A_2,B_2 \in \mathscr{F}$が存在する.
			繰り返せば$|\lambda(A_j)| > 1\ (j=1,2,\cdots)$
			を満たす互いに素な集合列$(A_j)_{j=1}^{\infty}$が構成され,
			このとき$\sum_{j=1}^{\infty} |\lambda(A_j)| = \infty$となる.
			一方でRiemannの級数定理より$\sum_{j=1}^{\infty} |\lambda(A_j)| < \infty$
			が成り立つから矛盾が生じ,$|\lambda|(X) < \infty$が出る.
			\QED
		\end{description}
	\end{prf}
	
	\begin{screen}
		\begin{thm}[複素測度全体は線型空間・総変動ノルム]
			可測空間$(X,\mathscr{F})$上の複素測度の全体を$CM(X,\mathscr{F})$と書く.
			\begin{align}
				&(\lambda + \mu)(E) \coloneqq \lambda(E) + \mu(E), \\
				&(c\lambda)(E) \coloneqq c\lambda(E)
				\label{complex_measure_linear}
			\end{align}
			を線型演算として$CM(X,\mathscr{F})$は線形空間となり,また
			\begin{align}
				\Norm{\lambda}{TV} \coloneqq |\lambda|(X) \quad (\lambda \in CM(X,\mathscr{F}))
			\end{align}
			により$CM(X,\mathscr{F})$に総変動ノルム$\Norm{\cdot}{TV}$が定まる.
		\end{thm}
	\end{screen}
	
	\begin{prf}\mbox{}
		$\Norm{\cdot}{TV}$がノルムであることを示す.
		\begin{description}
			\item[第一段]
				$\lambda = 0$なら$\Norm{\lambda}{TV} = |\lambda|(X) = 0$となる.また
				$|\lambda(E)| \leq |\lambda|(E) \leq \Norm{\lambda}{TV}$より
				$\Norm{\lambda}{TV} = 0$なら$\lambda=0$が従う.
			
			\item[第二段]
				任意の$\lambda \in CM(X,\mathscr{F})$と$c \in \C$に対し
				\begin{align}
					\Norm{c\lambda}{TV} = \sup{}{\sum_{i}|(c\lambda)(E_i)|} = \sup{}{\sum_{i}|c\lambda(E_i)|} = |c|\sup{}{\sum_{i}|\lambda(E_i)|} = |c|\Norm{\lambda}{TV}
				\end{align}
				が成り立ち同次性が得られる.
			
			\item[第三段]
				$\lambda,\mu \in CM(X,\mathscr{F})$を任意に取る.このとき,
				$X$の任意の分割$X = \sum_{i=1}^\infty E_i\ (E_i \in \mathscr{F})$に対して
				\begin{align}
					\sum_{i=1}^\infty |(\lambda + \mu)(E_i)| 
					= \sum_{i=1}^\infty |\lambda(E_i) + \mu(E_i)| 
					\leq \sum_{i=1}^\infty |\lambda(E_i)| + \sum_{i=1}^\infty |\mu(E_i)| \leq \Norm{\lambda}{TV} + \Norm{\mu}{TV}
				\end{align}
				が成り立つから$\Norm{\lambda + \mu}{TV} \leq \Norm{\lambda}{TV} + \Norm{\mu}{TV}$が従う.
				\QED
		\end{description}
	\end{prf}
	
	可測空間$(X,\mathscr{F})$において,$\R$にしか値を取らない複素測度を符号付き測度(signed measure)という.
	\begin{screen}
		\begin{dfn}[正変動と負変動・Jordanの分解]
			$(X,\mathscr{F})$を可測空間とする.$(X,\mathscr{F})$上の符号付き測度$\mu$に対し
			\begin{align}
				\mu^+ \coloneqq \frac{1}{2}(|\mu| + \mu) , \quad \mu^- \coloneqq \frac{1}{2}(|\mu| - \mu)
			\end{align}
			として正値有限測度$\mu^+,\mu^-$を定める.
			$\mu^+\ (\mu^-)$を$\mu$の正(負)変動(positive (negative) variation)と呼び,
			\begin{align}
				\mu = \mu^+ - \mu^-
			\end{align}
			を符号付き測度$\mu$のJordan分解(Jordan decomposition)という.同時に$|\mu| = \mu^+ + \mu^-$も成り立つ.
		\end{dfn}
	\end{screen}
	
	\begin{screen}
		\begin{dfn}[絶対連続・特異]
			$(X,\mathscr{F})$を可測空間,
			$\mu$を$\mathscr{F}$上の正値測度
			,$\lambda,\lambda_1,\lambda_2$を$\mathscr{F}$上の任意の測度とする.
			\begin{itemize}
				\item $\mu(E)=0$ならば$\lambda(E)=0$となるとき,
					$\lambda$は$\mu$に関して絶対連続である(absolutely continuous)といい
					\begin{align}
						\lambda \ll \mu
					\end{align}
					と書く.
				
				\item 或る$A \in \mathscr{F}$が存在して
					\begin{align}
						\lambda(E) = \lambda(A \cap E),\quad (\forall E \in \mathscr{F})
					\end{align}
					が成り立つとき,$\lambda$は$A$に集中している(concentrated on A)という.
					$\lambda_1$が$A_1$に,$\lambda_2$が$A_2$に集中し,かつ
					$A_1 \cap A_2 = \emptyset$であるとき,
					$\lambda_1,\lambda_2$は互いに特異である(mutually singular)といい
					\begin{align}
						\lambda_1 \perp \lambda_2
					\end{align}
					と書く.
			\end{itemize}
		\end{dfn}
	\end{screen}
	
	\begin{screen}
		\begin{thm}[絶対連続性の同値条件]\label{thm:equivalent_condition_of_absolute_continuity}
			$\lambda,\mu$をそれぞれ可測空間$(X,\mathscr{F})$上の複素測度,正値測度とするとき,
			次は同値である:
			\begin{description}
				\item[(1)] $\lambda \ll \mu$,
				\item[(2)] $|\lambda| \ll \mu$
				\item[(3)] 任意の$\epsilon > 0$に対し或る$\delta > 0$が存在して
					$\mu(E) < \delta$なら$|\lambda|(E) < \epsilon$となる.
			\end{description}
		\end{thm}
	\end{screen}
	
	\begin{prf}\mbox{}
		\begin{description}
			\item[第一段]
				$(1) \Leftrightarrow (2)$を示す.
				任意の$E \in \mathscr{F}$に対し$|\lambda(E)| \leq |\lambda|(E)$より
				$(2) \Rightarrow (1)$が従う.また$\lambda \ll \mu$のとき,
				\begin{align}
					|\lambda|(E) =
					\sup{}{\Set{\sum_{i=1}^{\infty} |\lambda(A_i)|}{E = \sum_{i=1}^\infty A_i,\ \{A_i\}_{i=1}^\infty \subset \mathscr{F}}},
					\quad (\forall E \in \mathscr{F})
				\end{align}
				より$\mu(E) = 0$なら$\mu(A_i) = 0$となり
				$\lambda(A_i) = 0\ (\forall i \geq 1)$が満たされ
				$(1) \Rightarrow (2)$が従う.
				
			\item[第二段]
				$(2) \Leftrightarrow (3)$を示す.
				実際(3)が満たされているとき,$\mu(E) = 0$なら任意の$\delta > 0$に対し
				$\mu(E) < \delta$となるから$|\lambda|(E) < \epsilon\ (\forall \epsilon > 0)$
				となり$|\mu|(E) = 0$が出る.逆に(3)が満たされていないとき,或る$\epsilon > 0$に対して
				\begin{align}
					\mu(E_n) < \frac{1}{2^{n+1}}, \quad |\lambda|(E_n) \geq \epsilon,
					\quad (n=1,2,\cdots)
				\end{align}
				を満たす$\{E_n\}_{n=1}^\infty \subset \mathscr{F}$が存在する.このとき
				\begin{align}
					A_n \coloneqq \bigcup_{i=n}^\infty E_i,
					\quad A \coloneqq \bigcap_{n=1}^\infty A_n
				\end{align}
				とおけば
				\begin{align}
					\mu(A) = \lim_{n \to \infty} \mu(A_n) 
					\leq \lim_{n \to \infty} \frac{1}{2^n} = 0
				\end{align}
				かつ
				\begin{align}
					|\lambda|(A) = \lim_{n \to \infty} |\lambda|(A_n) 
					\geq \lim_{n \to \infty} |\lambda|(E_n) \geq \epsilon 
				\end{align}
				が成り立ち,対偶を取れば$(2) \Rightarrow (3)$が従う.
				\QED
		\end{description}
	\end{prf}
	
	\begin{screen}
		\begin{lem}\label{lem:Lebesgue_Radon_Nikodym}
			$(X,\mathscr{F},\mu)$を$\sigma$-有限測度空間とするとき,
			$0 < w < 1$を満たす可積分関数$w$が存在する.
		\end{lem}
	\end{screen}
	
	\begin{prf}
		$\mu(X) = 0$なら$w \equiv 1/2$とすればよい.$\mu(X) > 0$の場合,$\sigma$-有限の仮定より
		\begin{align}
			0 < \mu(X_n) < \infty,\ (\forall n \geq 1),
			\quad X = \bigcup_{n=1}^\infty X_n
		\end{align}
		を満たす$\{X_n\}_{n=1}^\infty \subset \mathscr{F}$が存在する.ここで
		\begin{align}
			w_n(x) \coloneqq
			\begin{cases}
				\displaystyle\frac{1}{2^n\left(1+\mu(X_n)\right)}, & x \in X_n, \\
				0, & x \in X \backslash X_n,
			\end{cases}
			\quad n=1,2,\cdots
		\end{align}
		に対して
		\begin{align}
			w \coloneqq \sum_{n=1}^\infty w_n
		\end{align}
		と定めれば,任意の$x \in X$は或る$X_n$に属するから
		\begin{align}
			0 < w_n(x) \leq w(x)
		\end{align}
		が成り立ち,かつ
		\begin{align}
			w(x) = w_1(x) + \sum_{n=2}^\infty w_n(x)
			\leq \frac{1}{2\left(1+\mu(X_1)\right)} + \frac{1}{2}
			< 1,
			\quad (\forall x \in X)
		\end{align}
		が満たされる.また単調収束定理より
		\begin{align}
			\int_X w\ d\mu \leq \sum_{n=1}^\infty \int_X w_n\ d\mu
			\leq \sum_{n=1}^\infty \frac{\mu(X_n)}{2^n\left(1+\mu(X_n)\right)}
			\leq 1
		\end{align}
		となり$w$の可積分性が出る.
		\QED
	\end{prf}
	
	\begin{screen}
		\begin{thm}[Lebesgue-Radon-Nikodym]
			$(X,\mathscr{F})$を可測空間,$\lambda$を$(X,\mathscr{F})$上の複素測度,
			$\mu$を$(X,\mathscr{F})$上の$\sigma$-有限正値測度$(\mu(X)>0)$とするとき,以下が成立する:
			\begin{description}
				\item[Lebesgue分解]
					$\lambda$は$\mu$に関して絶対連続な$\lambda_a$及び$\mu$と互いに特異な
					$\lambda_s$に一意に分解される:
					\begin{align}
						\lambda = \lambda_a + \lambda_s,
						\quad \lambda_a \ll \mu,
						\quad \lambda_s \perp \mu.
					\end{align}
				
				\item[密度関数の存在]
					$\lambda_a$に対し或る$g \in L^1(\mu) = L^1(X,\mathscr{F},\mu)$が唯一つ存在して次を満たす:
					\begin{align}
						\lambda_a(E) = \int_E g\ d\mu,
						\quad (\forall E \in \mathscr{F}).
					\end{align}
			\end{description}
		\end{thm}
	\end{screen}
	
	\begin{prf}\mbox{}
		\begin{description}
			\item[第一段] Lebesgueの分解の一意性を示す.
				$\lambda'_a \ll \mu$と$\lambda'_s \perp \mu$により
				\begin{align}
					\lambda_a + \lambda_s = \lambda'_a + \lambda'_s
				\end{align}
				が成り立つとき,
				\begin{align}
					\Lambda \coloneqq \lambda_a - \lambda'_a = \lambda'_s - \lambda_s,
					\quad \Lambda \ll \mu,
					\quad \Lambda \perp \mu
				\end{align}
				となり$\Lambda = 0$が従い分解の一意性が出る.
			
			\item[第二段] 密度関数の一意性を示す.実際,可積分関数$f$に対して
				\begin{align}
					\int_E f\ d\mu = 0, \quad (\forall E \in \mathscr{F})
				\end{align}
				が成り立つとき,定理\ref{thm:mean_value_of_integral_and_closed_set}より
				$f = 0,\ \mbox{$\mu$-a.e.}$が成り立つ.
				
			\item[第三段] Lebesgueの分解と密度関数の存在を示す.
				
		\end{description}
	\end{prf}
	
	\begin{screen}
		\begin{thm}[Vitali-Hahn-Saks]\label{thm;Vitali_Hahn_Saks}
			$(X,\mathscr{F})$を可測空間,$(\lambda_n)_{n=1}^\infty$をこの上の複素測度の列とするとき,
			\begin{align}
				\lambda(E) \coloneqq \lim_{n \to \infty} \lambda_n(E),
				\quad (\forall E \in \mathscr{F})
				\label{eq:thm_Vitali_Hahn_Saks}
			\end{align}
			が存在すれば$\lambda$もまた$(X,\mathscr{F})$上の複素測度となる.
			つまり$(CM(X,\mathscr{F}),\Norm{\cdot}{TV})$はBanach空間である.
		\end{thm}
	\end{screen}
	
	\begin{prf}$\lambda_n \equiv 0\ (\forall n \geq 1)$なら$\lambda \equiv 0$で複素測度となるから,
		或る$n$と$E \in \mathscr{F}$に対し$\lambda_n(E) \neq 0$と仮定する.
		\begin{description}
			\item[第一段] $(X,\mathscr{F})$上の有限測度を
				\begin{align}
					\mu \coloneqq \sum_{n=1}^\infty \frac{1}{2^n(1 + \Norm{\lambda_n}{TV})} |\lambda_n|
				\end{align}
				により定めるとき,%定理\ref{thm:equivalent_condition_of_absolute_continuity}より
				任意の$\epsilon > 0$に対して或る$\delta > 0$が存在し
				\begin{align}
					\mu(E) < \delta \quad \Rightarrow \quad |\lambda_n|(E) < \epsilon\ (\forall n \geq 1)
					\label{eq:thm_Vitali_Hahn_Saks_2}
				\end{align}
				となることを示す.
				任意の$n \geq 1$に対して$\lambda_n \ll \mu$であるから
				Lebesgue-Radon-Nikodymの定理より
				\begin{align}
					\lambda_n(E) = \int_E g_n\ d\mu,
					\quad (\forall E \in \mathscr{F})
				\end{align}
				を満たす$g_n \in L^1(\mu)$が存在し,このとき
				\begin{align}
					\left| \int_E g_n\ d\mu \right|
					\leq |\lambda_n|(E)
					\leq 2^n(1+\Norm{\lambda_n}{TV})\mu(E),
					\quad (\forall E \in \mathscr{F})
				\end{align}
				が成立するから定理\ref{thm:mean_value_of_integral_and_closed_set}より
				\begin{align}
					\Norm{g_n}{L^\infty(\mu)} \leq 2^n(1+\Norm{\lambda_n}{TV})
				\end{align}
				が従う.いま,任意の$E \in \mathscr{F}$に対し$f_E \coloneqq [\defunc_E]$として
				\begin{align}
					L \coloneqq \Set{f_E}{E \in \mathscr{F}}
				\end{align}
				とおけば,$\mu(X) < \infty$より$L \subset L^1(\mu)$となり,また
				\begin{align}
					d(f_E,f_{E'}) \coloneqq \Norm{f_E - f_{E'}}{L^1(\mu)}
				\end{align}
				で定める距離$d$により$L$は完備距離空間となる.実際,定理\ref{thm:Lp_banach}より
				$L$の任意のCauchy列$\left(f_{E_n}\right)_{n=1}^\infty$に対し極限$f \in L^1(\mu)$が存在し,
				或る部分列$\left(\defunc_{E_{n_k}}\right)_{k=1}^\infty$は或る$\mu$-零集合$A$を除いて各点収束するから
				\begin{align}
					\varphi \coloneqq \lim_{k \to \infty} \defunc_{E_{n_k}} \defunc_{X \backslash A}
				\end{align}
				に対し$E \coloneqq \{\varphi = 1\}$とおけば$f = [\defunc_E] \in L$が満たされる.ここで
				\begin{align}
					\Phi_n: L \ni f_E \longmapsto \int_X |g_n| f_E\ d\mu
				\end{align}
				とおけば,任意の$E \in \mathscr{F}$に対し$|\lambda_n|(E) \leq \Phi_n(f_E)$が満たされ,
				またH\Ddot{o}lderの不等式より
				\begin{align}
					\left| \Phi_n(f_E) - \Phi_n(f_{E'}) \right|
					\leq \int_X |g_n| |f_E - f_{E'}|\ d\mu
					\leq \Norm{g_n}{L^\infty(\mu)} d(f_E,f_{E'}),
					\quad (\forall f_E,f_{E'} \in L)
				\end{align}
				がとなるから$\Phi_n$は$L$上の連続写像である.いま$\epsilon > 0$を任意に取れば,
				$\eta \coloneqq \epsilon/4$に対して
				\begin{align}
					F_n(\eta) 
					\coloneqq \Set{f_E \in L}{\sup{k \geq 1}{\left| \Phi_n(f_E)-\Phi_{n+k}(f_E) \right|} \leq \eta}
					= \bigcap_{k \geq 1} \Set{f_E \in L}{\left| \Phi_n(f_E)-\Phi_{n+k}(f_E) \right| \leq \eta}
				\end{align}
				により定める$F_n(\delta)$は閉集合であり,任意の$f_E \in L$は
				\begin{align}
					\sup{k \geq 1}{\left| \Phi_n(f_E)-\Phi_{n+k}(f_E) \right|}
					&\leq \left| \Phi_n(f_E)-\lambda(E) \right|
						+ \sup{k \geq 1}{\left| \lambda(E)-\Phi_{n+k}(f_E) \right|} \\
					&= \left| \lambda_n(E)-\lambda(E) \right|
						+ \sup{k \geq 1}{\left| \lambda(E)-\lambda_{n+k}(E) \right|} \\
					&\longrightarrow 0 \quad (n \longrightarrow \infty)
				\end{align}
				を満たすから
				\begin{align}
					L = \bigcup_{n=1}^\infty F_n(\eta)
				\end{align}
				が成り立ち,Baireの範疇定理より或る$F_{n_0}(\eta)$は内点$f_{E_0}$を持つ.
				つまり或る$\delta_0 > 0$が存在して
				\begin{align}
					d(f_{E_0},f_E) < \delta_0
					\quad \Rightarrow \quad
					\sup{k \geq 1}{\left| \Phi_n(f_E)-\Phi_{n+k}(f_E) \right|} \leq \eta
				\end{align}
				となる.$\mu(E) < \delta_0$ならば,
				\begin{align}
					E_1 \coloneqq E \cup E_0,
					\quad E_2 \coloneqq E_0 \backslash (E \cap E_0)
				\end{align}
				とすれば$f_E = [\defunc_E] = [\defunc_{E_1} - \defunc_{E_2}]
				= [\defunc_{E_1}] - [\defunc_{E_2}] = f_{E_1} - f_{E_2}$かつ
				\begin{align}
					d(f_{E_0},f_{E_1}) = \mu(E \backslash E_0) < \delta_0,
					\quad d(f_{E_0},f_{E_2}) = \mu(E \cap E_0) < \delta_0
				\end{align}
				が満たされるから,$n > n_0$なら
				\begin{align}
					\left|\Phi_n(f_E)\right| 
					&\leq \left|\Phi_{n_0}(f_E)\right| + \left|\Phi_n(f_E) - \Phi_{n_0}(f_E)\right| \\
					&\leq \left|\Phi_{n_0}(f_E)\right| + \left|\Phi_n(f_{E_1}) - \Phi_{n_0}(f_{E_1})\right|
						+ \left|\Phi_n(f_{E_2}) - \Phi_{n_0}(f_{E_2})\right| \\
					&\leq \left|\Phi_{n_0}(f_E)\right| + 2\eta
				\end{align}
				が従い,一方で$n=1,2,\cdots,n_0$に対しては,
				定理\ref{thm:integrable_intvalue_uniformly_shrinking}より
				或る$\delta_n > 0$が存在して
				\begin{align}
					\mu(E) < \delta_n \Longrightarrow \Phi_n(f_E) = \int_E |g_n|\ d\mu < \frac{\epsilon}{2}
				\end{align}
				が成立し,$\delta \coloneqq \min{}{\{\delta_0,\delta_1,\cdots,\delta_{n_0}\}}$として
				\begin{align}
					\mu(E) < \delta_n \Longrightarrow |\lambda_n|(E) \leq \Phi_n(f_E) < \epsilon,\ (\forall n \geq 1)
				\end{align}
				が得られる.
				
			\item[第二段] $\lambda$の可算加法性を示す.任意の互いに素な$A,B \in \mathscr{F}$を取れば
				\begin{align}
					\lambda(A + B) = \lim_{n \to \infty} \lambda_n(A + B)
					= \lim_{n \to \infty} \lambda_n(A) + \lim_{n \to \infty} \lambda_n(B)
					= \lambda(A) + \lambda(B)
				\end{align}
				となるから$\lambda$は有限加法的であり,このとき任意の互いに素な列$\{E_i\}_{i=1}^\infty \subset \mathscr{F}$に対し
				\begin{align}
					\lambda\Biggl( \sum_{i=1}^\infty E_i \Biggr)
					= \lambda\Biggl( \sum_{i=1}^N E_i \Biggr) + \lambda\Biggl( \sum_{i=N+1}^\infty E_i \Biggr)
					= \sum_{i=1}^N \lambda(E_i) + \lambda\Biggl( \sum_{i=N+1}^\infty E_i \Biggr)
				\end{align}
				が任意の$N \geq 1$について満たされるが,
				\begin{align}
					\mu\Biggl( \sum_{i=N+1}^\infty E_i \Biggr) \longrightarrow 0 \quad (N \longrightarrow \infty)
				\end{align}
				と(\refeq{eq:thm_Vitali_Hahn_Saks_2})より
				\begin{align}
					\lambda\Biggl( \sum_{i=N+1}^\infty E_i \Biggr) \longrightarrow 0 \quad (N \longrightarrow \infty)
				\end{align}
				が従い
				\begin{align}
					\lambda\Biggl( \sum_{i=1}^\infty E_i \Biggr) = \sum_{i=1}^\infty \lambda(E_i)
				\end{align}
				が得られる.よって$\lambda$は複素測度である.
				\QED
		\end{description}
	\end{prf}
	
	\begin{screen}
		\begin{thm}[$L^p$の共役空間]\label{thm:dual_space_of_L_p}
			$1 \leq p < \infty$,$q$を$p$の共役指数とし,また$(X,\mathscr{F},\mu)$を$\sigma$-有限な測度空間とするとき,
			$g \in L^q(\mu)$に対して
			\begin{align}
				\Phi_g: L^p(\mu) \ni f \longmapsto \int_X fg\ d\mu
				\label{eq:thm_dual_space_of_L_p_1}
			\end{align}
			は有界線形作用素となる.また
			\begin{align}
				\Phi: L^q(\mu) \ni g \longmapsto \Phi_g \in \left( L^p(\mu) \right)^*
			\end{align}
			で定める$\Phi$は$\left( L^p(\mu) \right)^*$から$L^q(\mu)$への線型同型であり,
			次の意味で等長である:
			\begin{align}
				\Norm{g}{L^q(\mu)} = \Norm{\Phi_g}{\left( L^p(\mu) \right)^*}.
				\label{eq:thm_dual_space_of_L_p_asseretion_2}
			\end{align}
			$p=\infty$の場合,$\mu(X) < \infty$かつ$\varphi \in \left( L^\infty(\mu) \right)^*$に対し
			$\mathscr{F} \ni A \longmapsto \varphi(\defunc_A)$が可算加法的ならば,
			$\varphi$に対し或る$g \in L^1(\mu)$が唯一つ存在して
			$\varphi = \Phi_g$と(\refeq{eq:thm_dual_space_of_L_p_asseretion_2})を満たす.
		\end{thm}
	\end{screen}
	
	\begin{prf}\mbox{}
		\begin{description}
			\item[第一段]
				$\Phi_g$が(\refeq{eq:thm_dual_space_of_L_p_1})で与えられていれば,H\Ddot{o}lderの不等式より
				\begin{align}
					\left|\Phi_g(f)\right| \leq \Norm{g}{L^q(\mu)}\Norm{f}{L^p(\mu)}
				\end{align}
				が成り立つから
				\begin{align}
					\Norm{\Phi_g}{\left( L^p(\mu) \right)^*} \leq \Norm{g}{L^q(\mu)}
					\label{eq:thm_dual_space_of_L_p_3}
				\end{align}
				が従う.よって$\Phi_g \in \left( L^p(\mu) \right)^*$となる.
			
			\item[第二段]
				$\varphi \in \left( L^p(\mu) \right)^*$に対して
				$\Phi(g) = \varphi$を満たす$g \in L^q(\mu)$が存在するとき,
				$g$が$\varphi$に対して一意に決まることを示す.$\sigma$-有限の仮定より
				\begin{align}
					\mu(X_n) < \infty,\ (\forall n \geq 1);
					\quad X = \bigcup_{n=1}^\infty X_n
					\label{eq:thm_dual_space_of_L_p_6}
				\end{align}
				を満たす$\{X_n\}_{n=1}^\infty \subset \mathscr{F}$が存在する.
				いま,$g,g' \in L^q(\mu)$に対して
				\begin{align}
					\int_X fg\ d\mu = \int_X fg'\ d\mu,
					\quad (\forall f \in L^p(\mu))
				\end{align}
				が成り立っているとすれば,任意の$E \in \mathscr{F}$に対して
				$\defunc_{E \cap X_n} \in L^p(\mu)$であるから
				\begin{align}
					\int_{E \cap X_n} g-g'\ d\mu = 0,
					\quad (\forall n \geq 1)
				\end{align}
				となり,Lebesgueの収束定理より
				\begin{align}
					\int_E g-g'\ d\mu = 0
				\end{align}
				が従い$L^q(\mu)$で$g = g'$が成立する.
				
			\item[第三段]
				$1 \leq p < \infty$の場合,$\mu(X) < \infty$なら
				任意の$\varphi \in \left( L^p(\mu) \right)^*$に対して
				$\Phi(g) = \varphi$を満たす$g \in L^q(\mu)$が存在することを示す.
				\begin{align}
					\lambda(E) \coloneqq \varphi(\defunc_E)
					\label{eq:thm_dual_space_of_L_p_7}
				\end{align}
				により$\lambda$を定めれば
				\begin{align}
					\lambda(A + B) = \varphi(\defunc_{A+B}) = \varphi(\defunc_A + \defunc_B)
					= \varphi(\defunc_A) + \varphi(\defunc_B)
					= \lambda(A) + \lambda(B)
				\end{align}
				となり$\lambda$の加法性が出る.また
				任意の互いに素な$\{E_n\}_{n=1}^\infty \in \mathscr{F}$に対して
				\begin{align}
					A_k \coloneqq \sum_{n=1}^k E_n,
					\quad A \coloneqq \sum_{n=1}^\infty E_n
				\end{align}
				とおけば
				\begin{align}
					\left| \lambda(A) - \sum_{n=1}^k \lambda(E_n) \right|
					&= \left| \lambda(A) - \lambda(A_k) \right|
					= \left| \varphi(\defunc_A - \defunc_{A_k}) \right| \\
					&\leq \Norm{\varphi}{\left( L^p(\mu) \right)^*} \Norm{\defunc_A - \defunc_{A_k}}{L^p(\mu)}
					= \Norm{\varphi}{\left( L^p(\mu) \right)^*} \mu(A - A_k)^{1/p}
					\longrightarrow 0
					\quad (k \longrightarrow \infty)
				\end{align}
				が成り立つから$\lambda$は複素測度である.また
				\begin{align}
					|\lambda(E)| \leq \Norm{\varphi}{\left( L^p(\mu) \right)^*} \mu(E)^{1/p}
				\end{align}
				より$\lambda \ll \mu$となるから,Lebesgue-Radon-Nikodymの定理より
				\begin{align}
					\varphi(\defunc_E) = \lambda(E) = \int_X \defunc_E g\ d\mu,
					\quad (\forall E \in \mathscr{F})
					\label{eq:thm_dual_space_of_L_p_8}
				\end{align}
				を満たす$g \in L^1(\mu)$が存在する.$\varphi$の線型性より
				任意の単関数の同値類$f$に対して
				\begin{align}
					\varphi(f) = \int_X fg\ d\mu
					\label{eq:thm_dual_space_of_L_p_2}
				\end{align}
				が成立し,特に$f \in L^\infty(\mu)$に対しては
				\begin{align}
					B \coloneqq \Set{x \in X}{|f(x)| > \Norm{f}{L^\infty(\mu)}}
				\end{align}
				とおけば$\mu(B) = 0$となり,有界可測関数$f \defunc_{X \backslash B}$を
				一様に近似する単関数列$(f_n)_{n=1}^\infty$が存在して
				\begin{align}
					\left| \varphi(f) - \int_X fg\ d\mu \right|
					&\leq \left| \varphi(f) - \varphi(f_n) \right| + \left| \int_X f_ng\ d\mu - \int_X fg\ d\mu \right| \\
					&\leq \Norm{\varphi}{\left( L^p(\mu) \right)^*} \Norm{f - f_n}{L^p(\mu)}
						+ \int_X |f_n - f||g|\ d\mu \\
					&\longrightarrow 0 \quad (n \longrightarrow \infty)
				\end{align}
				となるから(\refeq{eq:thm_dual_space_of_L_p_2})が成立する.
			
			\item[第四段]
				$p = \infty,\ \mu(X) < \infty$の場合,
				$\varphi \in \left( L^p(\mu) \right)^*$に対して
				$\mathscr{F} \ni A \longmapsto \varphi(\defunc_A)$が可算加法的ならば
				(\refeq{eq:thm_dual_space_of_L_p_7})で定める$\lambda$は複素測度となり,
				前段と同じ理由で(\refeq{eq:thm_dual_space_of_L_p_8})を満たす$g \in L^1(\mu)$が存在し
				\begin{align}
					\varphi(f) = \int_X fg\ d\mu,
					\quad (\forall f \in L^\infty(\mu))
				\end{align}
				が成立する.すなわち$\varphi = \Phi_g$であり,
				このとき$f \coloneqq \defunc_{\{g \neq 0\}}\overline{g}/g \in L^\infty(\mu)$
				に対して
				\begin{align}
					\Norm{g}{L^1(\mu)} = \int_X fg\ d\mu = \varphi(f) 
					\leq \Norm{\varphi}{\left( L^\infty(\mu) \right)^*} 
				\end{align}
				となるから,(\refeq{eq:thm_dual_space_of_L_p_3})と併せて
				(\refeq{eq:thm_dual_space_of_L_p_asseretion_2})が満たされる.
				以降は$p < \infty$とする.
				
			\item[第五段]
				$g \in L^q(\mu)$であることを示す.$p = 1$の場合,
				任意の$E \in \mathscr{F}$に対して$f = \defunc_E$とすれば,
				(\refeq{eq:thm_dual_space_of_L_p_2})より
				\begin{align}
					\left| \int_E g\ d\mu \right| = \left| \varphi(\defunc_E) \right|
					\leq \Norm{\varphi}{\left( L^p(\mu) \right)^*} \mu(E)
				\end{align}
				が成立し
				\begin{align}
					\Norm{g}{L^q(\mu)} \leq \Norm{\varphi}{\left( L^p(\mu) \right)^*}
					\label{eq:thm_dual_space_of_L_p_4}
				\end{align}
				が従う.$1 < p < \infty$の場合は
				$\alpha \coloneqq \defunc_{\{g \neq 0\}}\overline{g}/g$と
				\begin{align}
					E_n \coloneqq \Set{x \in X}{|g(x)| \leq n},
					\quad (n=1,2,\cdots)
				\end{align}
				に対して$f \coloneqq \defunc_{E_n} |g|^{q-1} \alpha$とおけば,
				\begin{align}
					fg = \defunc_{E_n} |g|^q = |f|^p
				\end{align}
				が成り立ち$|f|^p \in L^\infty(\mu)$となるから(\refeq{eq:thm_dual_space_of_L_p_2})より
				\begin{align}
					\int_X \defunc_{E_n} |g|^q\ d\mu
					= \int_X fg\ d\mu
					= \varphi(f)
					\leq \Norm{\varphi}{\left( L^p(\mu) \right)^*} \Norm{f}{L^p(\mu)}
					= \Norm{\varphi}{\left( L^p(\mu) \right)^*} \left\{ \int_X \defunc_{E_n} |g|^q\ d\mu \right\}^{1/p}
				\end{align}
				が従い
				\begin{align}
					\left\{ \int_X \defunc_{E_n} |g|^q\ d\mu \right\}^{1/q} \leq \Norm{\varphi}{\left( L^p(\mu) \right)^*}
				\end{align}
				が得られ,単調収束定理より
				\begin{align}
					\Norm{g}{L^q(\mu)} \leq \Norm{\varphi}{\left( L^p(\mu) \right)^*}
					\label{eq:thm_dual_space_of_L_p_5}
				\end{align}
				が出る.
				
			\item[第六段]
				任意の$f \in L^p(\mu)$に対して,単関数近似列$(f_n)_{n=1}^\infty$は(\refeq{eq:thm_dual_space_of_L_p_2})を満たすから,
				H\Ddot{o}lderの不等式とLebesgueの収束定理より
				\begin{align}
					\left| \varphi(f) - \int_X fg\ d\mu \right|
					&\leq \left| \varphi(f) - \varphi(f_n) \right| + \left| \int_X f_ng\ d\mu - \int_X fg\ d\mu \right| \\
					&\leq \Norm{\varphi}{\left( L^p(\mu) \right)^*} \Norm{f - f_n}{L^p(\mu)}
						+ \Norm{f - f_n}{L^p(\mu)}\Norm{g}{L^q(\mu)} \\
					&\longrightarrow 0 \quad (n \longrightarrow \infty)
				\end{align}
				となり
				\begin{align}
					\varphi = \Phi(g)
				\end{align}
				が成り立つ.また,このとき(\refeq{eq:thm_dual_space_of_L_p_3})と(\refeq{eq:thm_dual_space_of_L_p_4})或は
				(\refeq{eq:thm_dual_space_of_L_p_5})より
				\begin{align}
					\Norm{g}{L^q(\mu)} = \Norm{\varphi}{\left( L^p(\mu) \right)^*}
				\end{align}
				が満たされる.
				
			\item[第七段]
				$\mu(X) = \infty$の場合,補題\ref{lem:Lebesgue_Radon_Nikodym}の関数$w$を用いて
				\begin{align}
					\tilde{\mu}(E) \coloneqq \int_E w\ d\mu,
					\quad (\forall E \in \mathscr{F})
				\end{align}
				により有限測度$\tilde{\mu}$を定める.このとき
				任意の$f \in L^p(\mu)$に対して
				\begin{align}
					F \coloneqq w^{-1/p} f
				\end{align}
				とおけば
				\begin{align}
					\int_X |F|^p\ d\tilde{\mu} = \int_X |F|^p w\ d\mu = \int_X |f|^p\ d\mu
					\label{eq:thm_dual_space_of_L_p_6}
				\end{align}
				が成立し,
				\begin{align}
					L^p \ni f \longmapsto w^{-1/p} f \in L^p(\tilde{\mu})
				\end{align}
				は等長な線型同型となる.ここで任意の$\varphi \in \left( L^p(\mu) \right)^*$に対して
				\begin{align}
					\Psi(F) \coloneqq \varphi\left( w^{1/p} F \right),
					\quad (\forall F \in L^p(\tilde{\mu}))
				\end{align}
				で線形作用素$\Psi$を定めれば
				\begin{align}
					\left| \Psi(F) \right| = \left| \varphi\left( w^{1/p} F \right) \right|
					\leq \Norm{\varphi}{\left( L^p(\mu) \right)^*}\Norm{w^{1/p} F}{L^p(\mu)}
					= \Norm{\varphi}{\left( L^p(\mu) \right)^*}\Norm{F}{L^p(\tilde{\mu})}
				\end{align}
				より$\Psi \in \left( L^p(\tilde{\mu}) \right)^*$が満たされ,かつ
				任意の$f \in L^p(\mu)$に対して
				\begin{align}
					\left| \varphi(f) \right| = \left| \Psi\left( w^{-1/p} f \right) \right|
					\leq \Norm{\Psi}{\left( L^p(\mu) \right)^*}\Norm{w^{-1/p} f}{L^p(\tilde{\mu})}
					= \Norm{\Psi}{\left( L^p(\tilde{\mu}) \right)^*}\Norm{f}{L^p(\mu)}
				\end{align}
				も成り立ち
				\begin{align}
					\Norm{\varphi}{\left( L^p(\mu) \right)^*} = \Norm{\Psi}{\left( L^p(\tilde{\mu}) \right)^*}
				\end{align}
				が得られる.前段までの結果より$\Psi$に対し或る$G \in L^q(\tilde{\mu})$が存在して
				\begin{align}
					\Psi(F) = \int_X FG\ d\tilde{\mu}
				\end{align}
				が成立するから,任意の$f \in L^p(\mu)$に対して
				\begin{align}
					\varphi(f) = \Psi\left( w^{-1/p} f \right)
					= \int_X w^{-1/p} f G w\ d\mu
					= \begin{cases}
						\displaystyle\int_X f G\ d\mu, & (p = 1), \\
						\displaystyle\int_X f w^{1/q} G\ d\mu, & (1 < p < \infty)
					\end{cases}
				\end{align}
				が従い,
				\begin{align}
					g \coloneqq
					\begin{cases}
						G, & (p = 1), \\
						w^{1/q} G, & (1 < p < \infty)
					\end{cases}
				\end{align}
				とおけば(\refeq{eq:thm_dual_space_of_L_p_6})より$g \in L^q(\mu)$となり,
				$\varphi = \Phi(g)$かつ
				\begin{align}
					\Norm{\varphi}{\left( L^p(\mu) \right)^*} = \Norm{\Psi}{\left( L^p(\tilde{\mu}) \right)^*}
					= \Norm{G}{L^q(\tilde{\mu})}
					= \Norm{g}{L^q(\mu)}
				\end{align}
				が満たされる.
				\QED
		\end{description}
	\end{prf}
\section{複素測度に関する積分}
	\begin{screen}
		\begin{thm}[複素測度の極分解]\label{thm:polar_decomposition_of_complex_measures}
			可測空間$(X,\mathscr{F})$上の任意の複素測度$\mu$に対し,次の意味での極分解
			\begin{align}
				\quad \mu(E) = \int_E e^{i\theta}\ d|\mu|,
				\quad (\forall E \in \mathscr{F})
			\end{align}
			を満たす$\mathscr{F}/\borel{\C}$-可測関数$\theta$が存在する.
			$\lambda \not\equiv 0$なら$e^{i \theta}$は
			$L^1(|\mu|)$の元として唯一つに決まる.
		\end{thm}
	\end{screen}
	
	\begin{prf} $\mu \equiv 0$なら$|\mu| \equiv 0$より$\theta \equiv \pi$でよい.
		$\mu \not\equiv 0$の場合,
		Lebesgue-Radon-Nikodymの定理より
		\begin{align}
			\mu(E) = \int_E h\ d|\mu|,
			\quad (\forall E \in \mathscr{F})
		\end{align}
		を満たす$[h] \in L^1(|\mu|)$が唯一つ存在する.このとき$|\mu|(E) > 0$なら
		\begin{align}
			\frac{1}{|\mu|(E)} \left| \int_E h\ d|\mu| \right|
			= \frac{|\mu(E)|}{|\mu|(E)} \leq 1
		\end{align}
		となるから,定理\refeq{thm:mean_value_of_integral_and_closed_set}より
		$|\mu|$-a.e.に$|h| \leq 1$となる.また
		\begin{align}
			E_r \coloneqq \{|h| \leq r\}
		\end{align}
		とおき$\{A_n\}_{n=1}^\infty \subset \mathscr{F}$を$E_r$の任意の分割とすれば,
		\begin{align}
			\sum_{n=1}^\infty |\mu(A_n)|
			= \sum_{n=1}^\infty \left|\int_{A_n} h\ d|\mu|\right|
			\leq \sum_{n=1}^\infty \int_{A_n} |h|\ d|\mu|
			\leq r \sum_{n=1}^\infty |\mu|(A_n)
			= r |\mu|(E_r)
		\end{align}
		が成り立つから$r < 1$なら$|\mu|(E_r) = 0$となり
		\begin{align}
			|\mu|\left(|h|< 1 \right)
			= |\mu| \Biggl(\bigcap_{n=1}^\infty E_{1-1/n} \Biggr)
			= 0
		\end{align}
		が従う.よって$|\mu|$-a.e.に$|h|=1$となる.ここで
		\begin{align}
			\theta(x) \coloneqq
			\begin{cases}
				0, & h(x) = 1, \\
				\pi, & h(x) \neq 1
			\end{cases}
		\end{align}
		と定めれば$[h] = [e^{i \theta}]$が成立する.
		\QED
	\end{prf}
	
	\begin{screen}
		\begin{dfn}[複素測度に関する積分]
			$(X,\mathscr{F})$を可測空間,$\mu$を$(X,\mathscr{F})$上の複素測度,
			$f$を$\mathscr{F}/\borel{\C}$-可測関数とする.
			$f$が$|\mu|$-可積分であるとき,極分解$d\mu = e^{i\theta}\ d|\mu|$を用いて
			\begin{align}
				\int_X f\ d\mu \coloneqq \int_X f e^{i \theta}\ d|\mu|
			\end{align}
			により$f$の$\mu$に関する積分を定める.
		\end{dfn}
	\end{screen}
	
	$\mu \not\equiv 0$なら極分解は定理\ref{thm:polar_decomposition_of_complex_measures}
	の意味で一意であるから$\mu$に関する積分はwell-definedである.
	$\mu \equiv 0$なら$|\mu| \equiv 0$であるから任意の可測写像は$|\mu|$について可積分となり,
	$\mu$に関する積分値は0で確定する(well-defined).
	
	\begin{screen}
		\begin{thm}[総変動測度の積分表現]
			$(X,\mathscr{F},\mu)$を正値測度空間,
			$f$を$\mathscr{F}/\borel{\C}$-可測な$\mu$-可積分関数とするとき,
			\begin{align}
				\lambda(E) \coloneqq \int_E f\ d\mu, \quad (\forall E \in \mathscr{F})
			\end{align}
			で複素測度$\lambda$を定めれば次が成り立つ:
			\begin{align}
				|\lambda|(E) = \int_E |f|\ d\mu, \quad (\forall E \in \mathscr{F}).
			\end{align}
		\end{thm}
	\end{screen}
	
	\begin{screen}
		\begin{thm}[積分の測度に関する線型性]\label{thm:linearity_of_integral_respect_to_complex_measure}
			$(X,\mathscr{F})$を可測空間,$\mu,\nu$をこの上の複素測度とする.$f:X \rightarrow \C$が$|\mu|$と$|\nu|$について可積分であるなら,
			$\alpha,\beta \in \C$に対し$|\alpha \mu + \beta \nu|$についても可積分であり,更に次が成り立つ:
			\begin{align}
				\int_X f\ d(\alpha\mu + \beta\nu) = \alpha \int_X f\ d\mu + \beta \int_X f\ d\nu.
			\end{align}
		\end{thm}
	\end{screen}
	
	\begin{prf}
		\begin{description}
			\item[第一段]
				$f$が可測単関数の場合について証明する.
				$a_i \in \C,\ A_i \in \mathcal{M}\ (i=1,\cdots,n,\ \sum_{i=1}^{n} A_i = X)$を用いて
				\begin{align}
					f = \sum_{i=1}^{n} a_i \defunc_{A_i}
				\end{align}
				と表されている場合,
				\begin{align}
					&\int_X f(x)\ (\alpha\mu + \beta\nu)(dx)
					= \sum_{i=1}^{n} a_i (\alpha\mu + \beta\nu)(A_i) \\
					&\qquad = \alpha \sum_{i=1}^{n} a_i \mu(A_i) + \beta \sum_{i=1}^{n} a_i \nu(A_i)
					= \alpha \int_X f(x)\ \mu(dx) + \beta \int_X f(x)\ \nu(dx)
				\end{align}
				が成り立つ.
				
			\item[第二段]
			$f$が一般の可測関数の場合について証明する.任意の$A \in \mathcal{M}$に対して
			\begin{align}
				\left| (\alpha \mu + \beta \nu)(A) \right| \leq |\alpha||\mu(A)| + |\beta||\nu(A)| \leq |\alpha||\mu|(A) + |\beta||\nu|(A)
 			\end{align}
 			が成り立つから,左辺で$A$を任意に分割しても右辺との大小関係は変わらず
 			\begin{align}
 				|\alpha \mu + \beta \nu|(A) \leq |\alpha||\mu|(A) + |\beta||\nu|(A)
 			\end{align}
 			となる.従って$f$が$|\mu|$と$|\nu|$について可積分であるなら
 			\begin{align}
 				\int_X |f(x)|\ |\alpha \mu + \beta \nu|(dx) \leq |\alpha| \int_X |f(x)|\ |\mu|(dx) + |\beta| \int_X |f(x)|\ |\nu|(dx) < \infty
 			\end{align}
 			が成り立ち前半の主張を得る.$f$の単関数近似列$(f_n)_{n=1}^{\infty}$を取れば,前段の結果と積分の定義より
 			\begin{align}
 				&\left| \int_X f(x)\ (\alpha\mu + \beta\nu)(dx) - \alpha \int_X f(x)\ \mu(dx) - \beta \int_X f(x)\ \nu(dx) \right| \\
 					&\qquad \leq \left| \int_X f(x)\ (\alpha\mu + \beta\nu)(dx) - \int_X f_n(x)\ (\alpha\mu + \beta\nu)(dx) \right| \\
 					&\qquad \quad + |\alpha| \left| \int_X f(x)\ \mu(dx) - \int_X f_n(x)\ \mu(dx) \right|
 					+ |\beta| \left| \int_X f(x)\ \nu(dx) - \int_X f_n(x)\ \nu(dx) \right| \\
 				&\qquad \longrightarrow 0 \quad (n \longrightarrow \infty)
 			\end{align}
 			が成り立ち後半の主張が従う.
 			\QED
		\end{description}
	\end{prf}
	
	\begin{screen}
		\begin{thm}[積分の複素共役]
			$(X,\mathscr{F})$を可測空間,$\mu$を複素測度,
			$f:X \rightarrow \C$を$|\mu|$について可積分な$\mathscr{F}/\borel{\C}$-可測関数とするとき
			次が成り立つ:
			\begin{align}
				\int_X f\ d\overline{\mu}
				= \overline{\int_X \overline{f}\ d\mu}.
			\end{align}
		\end{thm}
	\end{screen}
	
	\begin{prf}
		$u = \Re{f},\ v = \Im{f},\ \gamma = \Re{\mu},\ \theta = \Im{\mu}$とすれば,
		定理\refeq{thm:linearity_of_integral_respect_to_complex_measure}より
		\begin{align}
			\int_X f\ d\overline{\mu} &= \int_X f\ d\gamma - i \int_X f\ d\theta \\
			&= \int_X u\ d\gamma + i \int_X v\ d\gamma - i \int_X u\ d\theta + \int_X v\ d\theta \\
			&= \overline{\int_X u\ d\gamma - i \int_X v\ d\gamma + i \int_X u\ d\theta + \int_X v\ d\theta} \\
			&= \overline{\int_X \overline{f}\ d\gamma + i \int_X \overline{f}\ d\theta} \\
			&= \overline{\int_X \overline{f}\ d\mu}
		\end{align}
		が成立する.
		\QED
	\end{prf}
	
	\begin{screen}
		\begin{thm}[Rieszの表現定理(複素測度)]
		\end{thm}
	\end{screen}
\input{thms/Radon_Nikodym_theorem}
\section{条件付き期待値}
	\begin{screen}
		\begin{lem}
			$(X,\mathscr{F},\mu)$を$\sigma$-有限測度空間 $(\mu(X) > 0)$とするとき,
			$0 < w < 1$を満たす可積分関数$w$が存在する.
		\end{lem}
	\end{screen}
	
	\begin{prf}
		$\sigma$-有限の仮定より
		\begin{align}
			0 < \mu(X_n) < \infty,\ (\forall n \geq 1),
			\quad X = \bigcup_{n=1}^\infty X_n
		\end{align}
		を満たす$\{X_n\}_{n=1}^\infty \subset \mathscr{F}$が存在する.ここで
		\begin{align}
			w_n(x) \coloneqq
			\begin{cases}
				\displaystyle\frac{1}{2^n\left(1+\mu(X_n)\right)}, & x \in X_n, \\
				0, & x \in X \backslash X_n,
			\end{cases}
			\quad n=1,2,\cdots
		\end{align}
		に対して
		\begin{align}
			w \coloneqq \sum_{n=1}^\infty w_n
		\end{align}
		と定めれば,任意の$x \in X$は或る$X_n$に属するから
		\begin{align}
			0 < w_n(x) \leq w(x)
		\end{align}
		が成り立ち,かつ
		\begin{align}
			w(x) = w_1(x) + \sum_{n=2}^\infty w_n(x)
			\leq \frac{1}{2\left(1+\mu(X_1)\right)} + \frac{1}{2}
			< 1,
			\quad (\forall x \in X)
		\end{align}
		が満たされる.また単調収束定理より
		\begin{align}
			\int_X w\ d\mu \leq \sum_{n=1}^\infty \int_X w_n\ d\mu
			\leq \sum_{n=1}^\infty \frac{\mu(X_n)}{2^n\left(1+\mu(X_n)\right)}
			\leq 1
		\end{align}
		となり$w$の可積分性が出る.
		\QED
	\end{prf}
	
	\begin{screen}
		\begin{thm}[Lebesgue-Radon-Nikodym]
			$(X,\mathscr{F})$を可測空間,$\lambda$を$(X,\mathscr{F})$上の複素測度,
			$\mu$を$(X,\mathscr{F})$上の$\sigma$-有限正値測度とするとき,以下が成立する:
			\begin{description}
				\item[Lebesgue分解]
					$\lambda$は$\mu$に関して絶対連続な$\lambda_a$及び$\mu$と互いに特異な
					$\lambda_s$に一意に分解される:
					\begin{align}
						\lambda = \lambda_a + \lambda_s,
						\quad \lambda_a \ll \mu,
						\quad \lambda_s \perp \mu.
					\end{align}
				
				\item[密度関数の存在]
					$\lambda_a$に対し或る$g \in L^1(\mu) = L^1(X,\mathscr{F},\mu)$が唯一つ存在して次を満たす:
					\begin{align}
						\lambda_a(E) = \int_E g\ d\mu,
						\quad (\forall E \in \mathscr{F}).
					\end{align}
			\end{description}
		\end{thm}
	\end{screen}
	
	\begin{prf}\mbox{}
		\begin{description}
			\item[第一段] Lebesgueの分解の一意性を示す.
				$\lambda'_a \ll \mu$と$\lambda'_s \perp \mu$により
				\begin{align}
					\lambda_a + \lambda_s = \lambda'_a + \lambda'_s
				\end{align}
				が成り立つとき,
				\begin{align}
					\Lambda \coloneqq \lambda_a - \lambda'_a = \lambda'_s - \lambda_s,
					\quad \Lambda \ll \mu,
					\quad \Lambda \perp \mu
				\end{align}
				となり$\Lambda = 0$が従い分解の一意性が出る.
			
			\item[第二段] 密度関数の一意性は
			\item[第三段] Lebesgueの分解と密度関数の存在を示す.
		\end{description}
	\end{prf}
	
	\begin{screen}
		\begin{dfn}[条件付き期待値]
			$(X,\mathscr{F},\mu)$を測度空間,$f \in L^1(\mu)$とする.
			部分$\sigma$-加法族$\mathscr{G} \subset \mathscr{F}$に対し
			$\nu \coloneqq \left. \mu \right|_{\mathscr{G}}$が$\sigma$-有限であるとき,
			\begin{align}
				\lambda(A) \coloneqq \int_A f\ d\mu,
				\quad (\forall A \in \mathscr{G})
			\end{align}
			により$(X,\mathscr{G})$上に複素測度$\lambda$が定まり,$\lambda \ll \nu$であるから
			Lebesgue-Radon-Nikodymの定理より
			\begin{align}
				\lambda(A) = \int_A g\ d\nu,
				\quad (\forall A \in \mathscr{G})
			\end{align}
			を満たす$g \in L^1(\nu) = L^1\left(X,\mathscr{G},\nu\right)$
			が存在する.この$g$を$\mathscr{G}$で条件付けた$f$の条件付き期待値と呼び
			\begin{align}
				g = \cexp{f}{\mathscr{G}}
			\end{align}
			と書く.
		\end{dfn}
	\end{screen}
	
	\begin{screen}
		\begin{thm}
			\begin{description}
				\item[(1)]
					$X_n \leq X_{n+1}$
					$X_n \longrightarrow X\ a.s.P$
					$\cexp{X_n}{\mathscr{G}} \longrightarrow \cexp{X}{\mathscr{G}}\ a.s.P$
				\item[(2)]
					$X_n \geq 0$
					$\cexp{\liminf X_n}{\mathscr{G}} \leq \liminf \cexp{X_n}{\mathscr{G}}$
				\item[(3)]
					$|X_n| \leq Y$ $X_n \longrightarrow X\ a.s.P$
					$\cexp{X_n}{\mathscr{G}} \longrightarrow \cexp{X}{\mathscr{G}}\ a.s.P$
			\end{description}
		\end{thm}
	\end{screen}
	
	\begin{screen}
	\begin{lem}[凸関数の片側微係数の存在]
		任意の凸関数$\varphi:\R \longrightarrow \R$には
		各点で左右の微係数が存在する.特に,凸関数は連続であり,すなわちBorel可測である.
	\end{lem}
	\end{screen}
	
	\begin{prf}
		凸性より任意の$x < y < z$に対して
		\begin{align}
			\frac{\varphi(y) - \varphi(x)}{y - x} 
			\leq \frac{\varphi(z) - \varphi(x)}{z - x}
			\leq \frac{\varphi(z) - \varphi(y)}{z - y}
			\label{ineq:lem:convex_function_measurability_1}
		\end{align}
		が満たされる.従って,$x$を固定すれば,$x$に単調減少に近づく任意の点列$(x_n)_{n=1}^{\infty}$に対し
		 \begin{align}
		 	\left(\frac{f(x_n)-f(x)}{x_n-x}\right)_{n=1}^{\infty} 
		 	\label{seq:lem:convex_function_measurability_2}
		 \end{align}
		 は下に有界な単調減少列となり下限が存在する.$x$に単調減少に近づく別の点列$(y_k)_{k=1}^{\infty}$を取れば
		 \begin{align}
		 	\inf{k \in \N}{\frac{f(y_k)-f(x)}{y_k-x}} \leq \frac{f(x_n)-f(x)}{x_n-x} \quad (n=1,2,\cdots)
		 \end{align}
		 より
		 \begin{align}
		 	\inf{k \in \N}{\frac{f(y_k)-f(x)}{y_k-x}} \leq \inf{n \in \N}{\frac{f(x_n)-f(x)}{x_n-x}}
		 \end{align}
		 が成立し,$(x_n),(y_k)$の立場を変えれば逆向きの不等号も得られる.
		 すなわち極限は点列に依らず確定し,$\varphi$は$x$で右側微係数を持つ.
		 同様に左側微係数も存在し,特に$\varphi$の連続性及びBorel可測性が従う.
		 \QED
	\end{prf}
	
	\begin{screen}
	\begin{thm}[Jensenの不等式]
		$(X,\mathscr{F},\mu)$を測度空間,
		$\mathscr{G} \subset \mathscr{F}$を部分$\sigma$-加法族とし,
		$\left. \mu \right|_{\mathscr{G}}$が$\sigma$-有限であるとする.
		このとき,任意の可積分関数
		$f:X \longrightarrow \R$と
		凸関数$\varphi:\R \longrightarrow \R$に対し,
		$\varphi(f)$が可積分なら次が成立する:
		\begin{align}
			\varphi\left(\cexp{f}{\mathscr{G}} \right)
			\leq \cexp{\varphi(f)}{\mathscr{G}},
			\quad \mbox{$\mu$-a.e.}
		\end{align}
	\end{thm}
	\end{screen}
	
	\begin{prf}
			$\varphi$は各点$x \in \R$で右側接線を持つから,
			それを$\R \ni t \longmapsto a_x t + b_x$と表せば,
			\begin{align}
				\varphi(t) = \sup{r \in \Q}{\left\{ a_r t + b_r \right\}} \quad (\forall t \in \R)
				\label{eq:prp_properties_of_expanded_conditional_expectation_1}
			\end{align}
			が成立する.
			よって任意の$r \in \Q$に対して
			\begin{align}
				\varphi(f(x)) \geq a_r f(x) + b_r
			\end{align}
			が満たされるから
			\begin{align}
				\cexp{\varphi(f)}{\mathscr{G}}
				\geq a_r \cexp{f}{\mathscr{G}} + b_r 
				\quad \mbox{$\mu$-a.e.},
				\quad \forall r \in \Q 
			\end{align}
			が従い,各$r \in \Q$に対し
			\begin{align}
				N_r \coloneqq \Set{x \in X}{\cexp{\varphi(f)}{\mathscr{G}}(x)
				< a_r \cexp{f}{\mathscr{G}}(x) + b_r}
			\end{align}
			とおけば$\mu(N_r) = 0$かつ
			\begin{align}
				\cexp{\varphi(f)}{\mathscr{G}}(x)
				\geq a_r \cexp{f}{\mathscr{G}}(x) + b_r, 
				\quad \forall r \in \Q,\ x \notin \bigcup_{r \in \Q} N_r
			\end{align}
			となる.$r$の任意性と(\refeq{eq:prp_properties_of_expanded_conditional_expectation_1})より
			\begin{align}
				\cexp{\varphi(f)}{\mathscr{G}} \geq \varphi\left( \cexp{f}{\mathscr{G}} \right),
				\quad \mbox{$\mu$-a.e.}
			\end{align}
			が得られる.
			\QED
	\end{prf}
\section{一様可積分性}
	\begin{screen}
		\begin{dfn}[一様可積分]
			$(X,\mathscr{F},\mu)$を測度空間とし,$\mathscr{U}$を$\mathscr{L}^1(X,\mathscr{F},\mu)$の部分集合とする.
			\begin{itemize}
				\item $\mathscr{U}$が$\mathscr{L}^1(X,\mathscr{F},\mu)$で有界である:
					\begin{align}
						\sup{f \in \mathscr{U}}\int_X|f|\ d\mu < \infty.
					\end{align}
				
				\item $\epsilon$を任意に与えられた正数とすると,次を満たす正数$\delta$が取れる:
					\begin{align}
						\forall f \in \mathscr{U}\, \forall B \in \mathscr{F}\, \left(\, \mu(B) < \delta
						\Longrightarrow \int_B |f|\ d\mu < \epsilon\, \right).
					\end{align}
			\end{itemize}
			が満たされているとき,$\mathscr{U}$は{\bf 一様可積分}\index{いちようかせきぶん@一様可積分}
			{\bf (uniformly integrable)}であるという.
		\end{dfn}
	\end{screen}
	
	一様可積分な集合の部分集合もまた一様可積分である.
	
	\begin{screen}
	\begin{thm}[一様可積分性の同値条件]\label{thm:appendix_uniform_integrability_equivalence}
		$(X,\mathscr{F},\mu)$を測度空間とし,$\mathscr{U}$を$\mathscr{L}^1(X,\mathscr{F},\mu)$の部分集合とする.
		このとき次の(1)と(2)が成り立つ:
		\begin{description}
			\item[(1)] $\mathscr{U}$が一様可積分であるとき,$\epsilon$を任意に与えられた正数とすると,次を満たす正数$a$が取れる:
				\begin{align}
					\forall f \in \mathscr{U}\, \forall \lambda \in \R_+\,
					\left(\, a < \lambda \Longrightarrow \int_{\{|f| > \lambda\}} |f|\ d\mu < \epsilon\, \right).
				\end{align}
			
			\item[(2)] $\mu(X) < \infty$の場合(1)の逆が成立する.つまり,
				\begin{align}
					\forall \epsilon \in \R_+\, \exists a \in \R_+\, 
					\forall f \in \mathscr{U}\, \forall \lambda \in \R_+\,
					\left(\, a < \lambda \Longrightarrow \int_{\{|f| > \lambda\}} |f|\ d\mu < \epsilon\, \right)
				\end{align}
				が成り立つとき$\mathscr{U}$は一様可積分である.
		\end{description}
	\end{thm}
	\end{screen}
	
	\begin{sketch}\mbox{}
		\begin{description}
			\item[(1)]
				$\mathscr{U}$が一様可積分であるとする.
				いま$\epsilon$を任意に与えられた正数とする.このとき
				\begin{align}
					\forall f \in \mathscr{U}\, \forall B \in \mathscr{F}\, \left(\, \mu(B) < \delta
					\Longrightarrow \int_B |f|\ d\mu < \epsilon\, \right)
				\end{align}
				を満たす$\delta$が取れる.ここで
				\begin{align}
					\frac{1}{a}\sup{f \in \mathscr{U}}{\int_X|f|\ d\mu} < \delta
				\end{align}
				を満たす正の実数$a$を取れば,$a < \lambda$なる正数$\lambda$と$\mathscr{U}$の任意の要素$f$に対して
				\begin{align}
					\mu(|f| > \lambda) \leq \frac{1}{\lambda} \int_X |f|\ d\mu < \delta
				\end{align}
				となるので
				\begin{align}
					\forall f \in \mathscr{U}\, \forall \lambda \in \R_+\,
					\left(\, a < \lambda \Longrightarrow \int_{\{|f| > \lambda\}} |f|\ d\mu < \epsilon\, \right)
				\end{align}
				が成立する.
			
			\item[(2)]
				いま$\epsilon$を任意に与えられた正数とする.このとき
				\begin{align}
					\forall f \in \mathscr{U}\, 
					\left(\, \int_{\{|f| > a\}} |f|\ d\mu < \frac{\epsilon}{2}\, \right)
				\end{align}
				を満たす正数$a$が取れる.$f$を$\mathscr{U}$の任意の要素とし,$B$を$\mathscr{F}$の任意の要素とすれば
				\begin{align}
					\int_B |f|\ d\mu
					= \int_{\{|f|>a\} \cap B} |f|\ d\mu
						+ \int_{\{|f| \leq a\} \cap B} |f|\ d\mu
					\leq \frac{\epsilon}{2} + a\mu(B)
				\end{align}				
				が成り立つから,
				\begin{align}
					\sup{f \in \mathscr{U}}\int_X|f|\ d\mu < \infty
				\end{align}
				及び
				\begin{align}
					\forall B \in \mathscr{F}\, \left(\, \mu(B) < \frac{\epsilon}{2a}
					\Longrightarrow \int_B |f|\ d\mu < \epsilon\, \right)
				\end{align}
				が成立する.すなわち$\mathscr{U}$は一様可積分である.
				\QED
		\end{description}
	\end{sketch}
	
	\begin{screen}
	\begin{thm}[一様可積分性と平均収束]\label{lem:uniformly_integrable_and_convergence_in_mean}
		$(X,\mathscr{F},\mu)$を正値測度空間とし,
		\begin{align}
			\mu(X) < \infty
		\end{align}
		とする.また$\{f_n\}_{n=1}^\infty$を$\mathscr{L}^1(X,\mathscr{F},\mu)$の部分集合とし,
		$A$を$\mu$-零集合とし,$X \backslash A$の各点$x$で$(f_n(x))_{n=1}^\infty$が
		$\C$で収束するとする.このとき次の(1)と(2)は同値である:
		\begin{description}
			\item[(1)] $\{f_n\}_{n=1}^\infty$が一様可積分.
			\item[(2)] $f \defeq \lim_{n \to \infty} f_n \defunc_A$と$f$を定めると,$f$は可積分で
				\begin{align}
					\int_X |f - f_n|\ d\mu 
					\longrightarrow 0
					\quad (n \longrightarrow \infty).
				\end{align}
		\end{description}
	\end{thm}
	\end{screen}
	
	\begin{sketch}
		
	\end{sketch}
	
	\begin{screen}
	\begin{thm}[一様可積分性と条件付き期待値]\label{lem:uniformly_integrability_and_conditional_expectations}
		$(X,\mathscr{F},\mu)$を測度空間とし,$\mu(X) < \infty$とし,
		$f$を$\mathscr{L}^1(X,\mathscr{F},\mu)$の要素とする.
		また$\mathscr{S}$を$X$上の$\sigma$-加法族であり$\mathscr{F}$の部分集合であるものの全体とする.
		このとき$\left\{ \cexp{f}{\mathscr{G}} \right\}_{\mathscr{G} \in \mathscr{S}}$は一様可積分である.
	\end{thm}
	\end{screen}
	
	\begin{prf}
		定理\ref{thm:properties_of_conditional_expectations}より
				\begin{align}
					\int_{\left| \cexp{f}{\mathscr{G}} \right| > \lambda} \left| \cexp{f}{\mathscr{G}} \right|\ d\mu
					\leq \int_{\cexp{|f|}{\mathscr{G}} > \lambda} \cexp{|f|}{\mathscr{G}}\ d\mu
					= \int_{\cexp{|f|}{\mathscr{G}} > \lambda} |f|\ d\mu
				\end{align}
				が成り立つ.また$X$の可積分性より,任意の$\epsilon > 0$に対して
				或る$\delta > 0$が存在し
				\begin{align}
					\mu(B) < \delta \Rightarrow \int_B |f|\ d\mu < \epsilon
				\end{align}
				が満たされる.いま,Chebyshevの不等式より
				\begin{align}
					\mu\left( \cexp{|f|}{\mathscr{G}} > \lambda \right)
					\leq \frac{1}{\lambda} \int_X \cexp{|f|}{\mathscr{G}}\ d\mu
					= \frac{1}{\lambda} \int_X |f|\ d\mu
				\end{align}
				となるから,$\epsilon > 0$に対し或る$\lambda_0 > 0$が存在して
				\begin{align}
					\sup{\mathscr{G} \in \mathscr{S}}{\mu\left( \cexp{|f|}{\mathscr{G}} > \lambda \right)}
					< \delta,
					\quad (\forall \lambda > \lambda_0)
				\end{align}
				が満たされ
				\begin{align}
					\sup{\mathscr{G} \in \mathscr{S}}{\int_{\cexp{|f|}{\mathscr{G}} > \lambda}|f|\ d\mu}
					< \epsilon,
					\quad (\forall \lambda > \lambda_0)
				\end{align}
				が従う.
		\QED
	\end{prf}
\section{連続写像の空間の位相}
	$(X,d_X),(Y,d_Y)$を距離空間とし,
	\begin{align}
		C(X,Y) \coloneqq \Set{f:X \longrightarrow Y}{\mbox{$f$は連続写像}}
	\end{align}
	とおく.$X$が$\sigma$-コンパクトであるとき,つまり
	\begin{align}
		K_1 \subset K_2 \subset K_3 \subset \cdots,
		\quad \bigcup_{n=1}^\infty K_n = X 
	\end{align}
	を満たすコンパクト部分集合の列$(K_n)_{n=1}^\infty$が存在するとき,
	\begin{align}
		\rho(f,g) \coloneqq \sum_{n=1}^\infty 2^{-n} \left( 1 \wedge \sup{x \in K_n}{d_Y(f(x),g(x))} \right),
		\quad (f,g \in C(X,Y))
	\end{align}
	により定める$\rho$は$C(X,Y)$上の距離関数となる.
	実際,$f \in C(X,Y)$に対し$f(K_n)$はコンパクトであるから
	$\operatorname{diam}(f(K_n)) < \infty$
	\bddddegin{align}
		d_Y(f(x),g(x)) \leq d_Y(f(x),f(x_0)) + d_Y(f(x_0),g(x_0)) + d_Y(g(x_0),g(x))
		\leq \operatorname{diam}(f(K_n)) + d_Y(f(x_0),g(x_0)) + \operatorname{diam}(g(K_n))
	\end{align}
	
	\begin{screen}
		\begin{thm}
			$X$を$\sigma$-コンパクトな距離空間,$Y$を距離空間とするとき$C(X,Y)$は可分距離空間である.
		\end{thm}
	\end{screen}
	
\begin{thebibliography}{数字}
	\bibitem{key1} Moser, G. and Zach, R., ``The Epsilon Calculus and Herbrand Complexity'',
		Studia Logica 82, 133-155 (2006)
	
	\bibitem{key2} 高橋優太, ``1階述語論理に対する$\varepsilon$計算'', \\
		http://www2.kobe-u.ac.jp/~mkikuchi/ss2018files/takahashi1.pdf 
		
	\bibitem{key3} キューネン数学基礎論講義
	
	\bibitem{key5} ブルバキ, 数学原論 集合論 1, 
	
	\bibitem{key4} 竹内外史, 現代集合論入門, 増強版第5刷, 日本評論社, 2016, pp. 138-183, ISBN 978-4-535-60116-1
	
	\bibitem{key6} 島内剛一, 数学の基礎, 第1版第10刷, 日本評論社, 2016, ISBN 978-4-535-60106-2
	
	\bibitem{key7} 戸次大介, 数理論理学, 第2刷, 東京大学出版会, 2016, pp. 148-166, ISBN 978-4-13-062915-7
	
	\bibitem{key8} K. G$\ddot{\mbox{o}}$del, $The\ Consistency\ of\ the\ Continuum\ Hypothesis$, 8th printing, Princeton University Press 1970, p. 3, ISBN 0-691-07927-7.
	
	\bibitem{key9} 菊地誠, 不完全性定理, 初版3刷, 共立出版株式会社, 2017, pp. 86-91, ISBN 978-4-320-11096-0
	
	\bibitem{key10} 前原昭二, 記号論理入門, 新装版第8刷, 日本評論社, 2018, pp. 106-115, ISBN 4-535-60144-5
	
	\bibitem{key11} Kenji Miyamoto and Georg Moser, The Epsilon Calculus with Equality and Herbrand Complexity
\end{thebibliography}
\newpage
\printindex
%
%
\end{document}
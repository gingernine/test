\documentclass[a4j,10.5pt,oneside,openany]{jsbook}
%
\usepackage{amsmath,amssymb}
\usepackage{amsthm}
\usepackage{makeidx}
\makeindex
\usepackage{newpxmath,newpxtext}
\usepackage{mathrsfs} %花文字
\usepackage{mathtools} %参照式のみ式番号表示
\usepackage{latexsym} %qed
\usepackage{ascmac}
\usepackage{centernot} %\centernot\arrow
\usepackage{color}
\usepackage{relsize}
\usepackage{comment}
\usepackage{url}
\setcounter{tocdepth}{3} %table of contents subsection表示
\newtheoremstyle{mystyle}% % Name
	{20pt}%                      % Space above
	{20pt}%                      % Space below
	{\rm}%           % Body font
	{}%                      % Indent amount
	{\gt}%             % Theorem head font
	{.}%                      % Punctuation after theorem head
	{10pt}%                     % Space after theorem head, ' ', or \newline
	{}%                      % Theorem head spec (can be left empty, meaning `normal')
\theoremstyle{mystyle}

\allowdisplaybreaks[1]
\newcommand{\bhline}[1]{\noalign {\hrule height #1}} %表の罫線を太くする.
\newcommand{\bvline}[1]{\vrule width #1} %表の罫線を太くする.
\newtheorem{Prop}{$Proposition.$}
\newtheorem{Proof}{$Proof.$}
\newcommand{\QED}{% %証明終了
	\relax\ifmmode
		\eqno{%
		\setlength{\fboxsep}{2pt}\setlength{\fboxrule}{0.3pt}
		\fcolorbox{black}{black}{\rule[2pt]{0pt}{1ex}}}
	\else
		\begingroup
		\setlength{\fboxsep}{2pt}\setlength{\fboxrule}{0.3pt}
		\hfill\fcolorbox{black}{black}{\rule[2pt]{0pt}{1ex}}
		\endgroup
	\fi}

\definecolor{DarkMidnightBlue}{rgb}{0.0, 0.2, 0.4}
\definecolor{PakistanGreen}{rgb}{0.0, 0.4, 0.0}
\definecolor{Mahogany}{rgb}{0.65,0.10,0.10}
\definecolor{darkgray}{rgb}{0.21, 0.21, 0.21}
\definecolor{CarolinaBlue}{rgb}{0.6, 0.73, 0.89}

\newtheorem{thm}{\color{DarkMidnightBlue}{定理}}[section]
\newtheorem{dfn}[thm]{\color{PakistanGreen}{定義}}
\newtheorem{axm}[thm]{\color{Mahogany}{公理}}
\newtheorem{schema}[thm]{{公理図式}}
\newtheorem{metaaxm}[thm]{\color{Mahogany}{推論規則}}
\newtheorem{metathm}[thm]{{推論法則}}
\newtheorem{prp}[thm]{命題}
\newtheorem{cor}[thm]{系}
\newtheorem{lem}[thm]{補題}
\newtheorem*{prf}{証明}
\newtheorem{rem}[thm]{注意}
\newtheorem{e.g.}[thm]{例}
\newcommand{\defunc}{\mbox{1}\hspace{-0.25em}\mbox{l}} %定義関数
\newcommand*{\sgn}[1]{\operatorname{sgn}\left( #1 \right)} %signal関数
\newcommand{\monologue}[1]{
	%{\color{CarolinaBlue} \begin{itembox}[l]{院生室にて} #1 \end{itembox}}
	{\color{CarolinaBlue}\hspace{-10.5pt}\mask{\hspace{21pt}\vbox{
		\hsize 445pt
		\normalcolor{\vskip 7pt \noindent #1 \vskip 7pt}
	}\hspace{21pt}}{E}}
}

\def\Ddot#1{$\ddot{\mathrm{#1}}$} %文中ddot
\def\Set#1#2{\left\{\, #1 \mid \quad #2\, \right\}} %集合の書き方
\newcommand{\dom}[1]{\operatorname*{dom}(#1)} %類の定義域
\newcommand{\ran}[1]{\operatorname*{ran}(#1)} %類の値域
\newcommand{\sing}[1]{\operatorname*{Sing}(#1)} %single-valuedの定義式
\newcommand{\fnc}[1]{\operatorname*{Fnc}(#1)} %写像の定義式
\newcommand{\bij}{\underset{\mathrm{onto}}{\overset{\mathrm{1:1}}{\longrightarrow}}} %全単射
\newcommand{\tran}[1]{\operatorname*{Tran}(#1)} %推移的類の定義式
\newcommand{\ord}[1]{\operatorname*{Ord}(#1)} %順序数の定義式
\def\inprod<#1>{\left\langle #1 \right\rangle} %pairing
\def\sup#1#2{\operatorname*{sup}_{#1} #2 } %上限
\def\inf#1#2{\operatorname*{inf}_{#1} #2 } %下限
\def\esssup#1#2{\operatorname*{ess\mbox{.}sup}_{#1} #2 } %本質的上限
\def\Norm#1#2{\left\|\, #1\, \right\|_{#2} } %ノルム
\def\Log#1{\operatorname{log} #1} %log
\newcommand{\Univ}{\mathbf{V}} %宇宙
\newcommand{\ON}{\mathrm{On}} %順序数全体
\newcommand{\CN}{\mathrm{Cn}} %基数全体
\newcommand{\omg}{{\bf \omega}} %自然数全体omega
\def\N{\mathbf{N}} %自然数全体
\def\Q{\mathbf{Q}} %有理数全体
\def\R{\mathbf{R}} %実数全体
\def\Z{\mathbf{Z}} %整数全体
\def\C{\mathbf{C}} %複素数全体
\def\card#1{\mathrm{card}\ #1} %濃度
\def\Re#1{\operatorname{Re} #1} %実部
\def\Im#1{\operatorname{Im} #1} %虚部
\def\borel#1{\mathscr{B}(#1)} %Borel集合族
\def\open#1{\mathscr{O}(#1)} %位相空間 #1 の位相
\newcommand*{\interior}[1]{{\kern0pt#1}^{\mathrm{o}}} %内核
\def\closure#1{\overline{#1}} %閉包
\newcommand{\sgmalg}[1]{\sigma \left(#1\right)} %#1が生成するσ加法族
\newcommand{\cmeas}[3]{#1 \left( #2 \, \middle|\, #3 \right)} %条件付測度
\newcommand{\cexp}[2]{E\left( #1 \, \middle|\, #2 \right)}  %条件付期待値
%
%
\setlength{\textwidth}{\fullwidth}
\setlength{\textheight}{40\baselineskip}
\addtolength{\textheight}{\topskip}
%\setlength{\voffset}{-0.55in}
%
%
\title{Karatzas-Shreve solutions}
\author{}
\date{\today}

\begin{document}
%
%
\mathtoolsset{showonlyrefs = true}
\maketitle
%
%
\tableofcontents
\frontmatter
%
\mainmatter
%
%本文
\chapter{Martingales, Stopping Times, and Filtrations}
\section{Stochastic Processes and $\sigma$-Fields}
\begin{itembox}[l]{Problem 1.5 修正}
	Let $Y$ be a modification of $X$, and suppose that \textcolor{red}{every 
	sample path of both processes are right-continuous sample paths.} 
	Then $X$ and $Y$ are indistinguishable.
\end{itembox}

\begin{prf}
	$X,Y$のパスの右連続性より
	\begin{align}
		\left\{ X_t = Y_t,\ \forall t \geq 0 \right\}
		= \bigcap_{r \in \Q \cap [0,\infty)} \left\{ X_r = Y_r \right\}
	\end{align}
	が成立するから,$P(X_r = Y_r) = 1\   (\forall r \geq 0)$より
	\begin{align}
		P(X_t = Y_t,\ \forall t \geq 0)
		= P \biggl( \bigcap_{r \in \Q \cap [0,\infty)} \left\{ X_r = Y_r \right\} \biggr)
		= 1
	\end{align}
	が従う.
	\QED
\end{prf}

\begin{itembox}[l]{Problem 1.7}
		Let $X$ be a process with every sample path RCLL. 
		Let $A$ be the event that $X$ is continuous on $[0,t_0)$. 
		Show that $A \in \mathscr{F}^X_{t_0}$.
\end{itembox}

\begin{prf}[参照元:\cite{key2}]
	$[0,t_0)$に属する有理数の全体を$\Q^* \coloneqq \Q \cap [0,t_0)$と表すとき,
	\begin{align}
		A = \bigcap_{m \geq 1} \bigcup_{n \geq 1} \bigcap_{\substack{p,q \in \Q^* \\ |p-q| < 1/n}}
		\Set{\omega \in \Omega}{\left|X_p(\omega) - X_q(\omega) \right| < \frac{1}{m}}
	\end{align}
	が成立することを示せばよい.これが示されれば,$\omega \longmapsto \left(X_p(\omega), X_q(\omega) \right)$の
	$\mathscr{F}^X_{t_0}/\borel{\R^2}$-可測性と
	\begin{align}
		\Phi:\R \times \R \ni (x,y) \longmapsto |x-y| \in \R
	\end{align}
	の$\borel{\R^2}/\borel{\R}$-可測性より
	\begin{align}
		\Set{\omega \in \Omega}{\left|X_p(\omega) - X_q(\omega) \right| < \frac{1}{m}}
		= \Set{\omega \in \Omega}{\left(X_p(\omega), X_q(\omega) \right) \in 
		\Phi^{-1}\left(B_{1/m}(0)\right)}
		\in \mathscr{F}^X_{t_0}
	\end{align}
	が得られ$A \in \mathscr{F}^X_{t_0}$が従う.$\left(B_{1/m}(0) = \Set{x \in \R}{|x| < 1/m}.\right)$
	\begin{description}
		\item[第一段]
			$\omega \in A^c$を任意にとる.このとき
			或る$s \in (0,t_0)$が存在して,$t \longmapsto X_t(\omega)$は
			$t = s$において左側不連続である.従って或る$m \geq 1$については,
			任意の$n \geq 1$に対し$0< s-u < 1/3n$を満たす
			$u$が存在して
			\begin{align}
				\left|X_u(\omega) - X_s(\omega) \right| \geq \frac{1}{m}
			\end{align}
			を満たす.一方でパスの右連続性より
			$0 < p - s,\ q - u < 1/3n$を満たす$p,q \in \Q^*$が存在して
			\begin{align}
				\left|X_p(\omega) - X_s(\omega) \right| < \frac{1}{4m},
				\quad \left|X_q(\omega) - X_u(\omega) \right| < \frac{1}{4m}
			\end{align}
			が成立する.このとき$0 < |p - q| < 1/n$かつ
			\begin{align}
				\left|X_p(\omega) - X_q(\omega) \right|
				\geq \left|X_p(\omega) - X_s(\omega) \right|
					- \left|X_s(\omega) - X_u(\omega) \right|
					- \left|X_q(\omega) - X_u(\omega) \right|
				\geq \frac{1}{2m}
			\end{align}
			が従い
			\begin{align}
				\omega \in \bigcup_{m \geq 1} \bigcap_{n \geq 1} \bigcup_{\substack{p,q \in \Q^* \\ |p-q| < 1/n}}
		\Set{\omega \in \Omega}{\left|X_p(\omega) - X_q(\omega) \right| \geq \frac{1}{m}}
			\end{align}
			を得る.
		
		\item[第二段]
			任意に$\omega \in A$を取る.各点で有限な左極限が存在するという仮定から,
			\begin{align}
				X_{t_0}(\omega) \coloneqq \lim_{t \uparrow t_0}X_t(\omega)
			\end{align}
			と定めることにより
			\footnote{
				実際$X_{t_0}(\omega)$は所与のものであるが,いまは$[0,t_0]$上での連続性を考えればよいから
				便宜上値を取り替える.
			}
			$t \longmapsto X_t(\omega)$は$[0,t_0]$上で一様連続となる.
			従って
			\begin{align}
				\omega \in \bigcap_{m \geq 1} \bigcup_{n \geq 1} \bigcap_{\substack{p,q \in \Q^* \\ |p-q| < 1/n}}
		\Set{\omega \in \Omega}{\left|X_p(\omega) - X_q(\omega) \right| < \frac{1}{m}}
			\end{align}
			を得る.
			\QED
	\end{description}
\end{prf}

\begin{itembox}[l]{Lemma2 for Exercise 1.8}
	$T = \{1,2,3,\cdots\}$を高々可算集合とし,
	$S_i$を第二可算公理を満たす位相空間,$X_i$を
	確率空間$(\Omega,\mathscr{F},P)$上の$S_i$-値確率変数とする$(i \in T)$.
	このとき,任意の並び替え$\pi:T \longrightarrow T$
	に対して$S \coloneqq \prod_{i \in T} S_{\pi(i)}$とおけば次が成立する:
	\begin{align}
		\sigma(X_i;\ i \in T) = \Set{\left\{ (X_{\pi(1)},X_{\pi(2)},\cdots) \in A \right\}}{A \in \borel{S}}.
		\label{eq:lem2_for_chap_1_exercise_1_8_1}
	\end{align}
\end{itembox}

\begin{prf}\mbox{}
	\begin{description}
		\item[第一段]
			射影$S \longrightarrow S_{\pi(n)}$を$p_n$で表す.
			任意に$t_i \in T$を取り$n \coloneqq \pi^{-1}(i)$とおけば,
			任意の$B \in \borel{S_n}$に対して
			\begin{align}
				X_i^{-1}(B) = \left\{ (\cdots, X_{\pi(n)},\cdots) \in p_n^{-1}(B) \right\} \in \Set{\left\{ (X_{\pi(1)}, X_{\pi(2)},\cdots) \in A \right\}}{A \in \borel{S}}
			\end{align}
			が成り立つから$\sigma(X_i;\ i \in T) \subset 
			\Set{\left\{ (X_{\pi(1)}, X_{\pi(2)},\cdots) \in A \right\}}{A \in \borel{S}}$が従う.
		
		\item[第二段]
			任意の有限部分集合$j \in T$と$B_j \in \borel{S_{\pi(j)}}$に対し
			\begin{align}
				\left\{ (X_{\pi(1)}, X_{\pi(2)},\cdots) \in p_j^{-1}(B_j) \right\}
				= X_{\pi(j)}^{-1}(B_j)
				\in \sigma(X_i;\ i \in T)
			\end{align}
			が成立するから
			\begin{align}
				\Set{p_i^{-1}(B_i)}{B_i \in \borel{S_{\pi(i)}},\ i \in T}
				\subset \Set{A \in \borel{S}}{\left\{ (X_{\pi(1)}, X_{\pi(2)},\cdots) \in A \right\} \in \sigma(X_i;\ i \in T)}
			\end{align}
			が従う.右辺は$\sigma$-加法族であり,定理\ref{thm:Borel_algebra_of_products_of_second_countable_spaces}より
			左辺は$\borel{S}$を生成するから前段と併せて(\refeq{eq:lem2_for_chap_1_exercise_1_8_1})を得る.
			\QED
	\end{description}
\end{prf}

\begin{itembox}[l]{Lemma3 for Exercise 1.8}
	$X = \Set{X_t}{0 \leq t < \infty}$を確率空間$(\Omega,\mathscr{F},P)$上の$\R^d$-値確率過程とする.
	任意の空でない$S \subset [0,\infty)$に対し
	\begin{align}
		\mathcal{F}^X_S \coloneqq \sigma(X_s;\ s \in S)
	\end{align}
	とおくとき,任意の空でない$T \subset [0,\infty)$に対して次が成立する:
	\begin{align}
		\mathcal{F}^X_T \coloneqq \bigcup_{S \subset T:at\ most\ countable} \mathcal{F}^X_S.
		\label{eq:lem3_for_chap_1_exercise_1_8_1}
	\end{align}
\end{itembox}

\begin{prf}
	便宜上
	\begin{align}
		\mathcal{F} \coloneqq \bigcup_{S \subset T:at\ most\ countable} \mathcal{F}^X_S
	\end{align}
	とおく.まず,任意の$S \subset T$に対し$\mathcal{F}^X_S \subset \mathcal{F}^X_T$が成り立つから
	\begin{align}
		\mathcal{F} \subset \mathcal{F}^X_T
	\end{align}
	が従う.また$\sigma(X_t) = \mathcal{F}^X_{\{t\}},\ (\forall t \in T)$より
	\begin{align}
		\bigcup_{t \in T} \sigma(X_t) \subset \mathcal{F}
	\end{align}
	が成り立つから,あとは$\mathcal{F}$が$\sigma$-加法族であることを示せばよい.実際,
	$\mathcal{F}$は$\sigma$-加法族の合併であるから$\Omega$を含みかつ補演算で閉じる.
	また$B_n \in \mathcal{F},\ n=1,2,\cdots$に対しては,$B_n \in \mathcal{F}^X_{S_n}$を満たす
	高々可算集合$S_n \subset T$が対応して
	\begin{align}
		\bigcup_{n=1}^\infty \mathcal{F}^X_{S_n}
		= \bigcup_{n=1}^\infty \sigma(X_s;\ s \in S_n)
		\subset \sigma\biggl(X_s;\ s \in \bigcup_{n=1}^\infty S_n \biggr)
	\end{align}
	が成り立つから,
	\begin{align}
		\bigcup_{n=1}^\infty B_n \in \sigma\biggl(X_s;\ s \in \bigcup_{n=1}^\infty S_n \biggr)
		\subset \mathcal{F}
	\end{align}
	が従う.ゆえに$\mathcal{F}$は$\sigma$-加法族であり
	(\refeq{eq:lem3_for_chap_1_exercise_1_8_1})を得る.
	\QED
\end{prf}

\begin{itembox}[l]{Exercise 1.8}
	Let $X$ be a process whose sample paths are RCLL almost surely, 
	and let $A$ be the event that $X$ is continuous on $[0,t_0)$. Show 
	that $A$ can fail to be in $\mathscr{F}^X_{t_0}$, but if $\Set{\mathscr{F}_t}{t \geq 0}$ is 
	a fitration satisfying $\mathscr{F}^X_t \subset \mathscr{F}_t,\ t \geq 0$, and 
	$\mathscr{F}^X_{t_0}$ contains all $P$-null sets of $\mathscr{F}$, then $A \in \mathscr{F}_{t_0}$.
\end{itembox}

\begin{prf}\mbox{}
	\begin{description}
		\item[第一段]
			高々可算な集合$S = \{t_1,t_2,\cdots\} \subset [0,t_0]$に対し,昇順に並び替えたものを
			$t_{\pi(1)} < t_{\pi(2)} < \cdots$と表し
			\begin{align}
				\mathcal{F}^X_S \coloneqq 
				\Set{\left\{(X_{t_{\pi(1)}},X_{t_{\pi(2)}},\cdots) \in B \right\}}{B \in \borel{(\R^d)^{\# S}}}
			\end{align}
			とおく.ただし$S$が可算無限の場合は$(\R^d)^{\# S} = \R^\infty$である.
			このとき(\refeq{eq:lem2_for_chap_1_exercise_1_8_1})より
			\begin{align}
				\sigma(X_s;\ s \in S) = \mathcal{F}^X_S
			\end{align}
			が成り立ち,(\refeq{eq:lem3_for_chap_1_exercise_1_8_1})より
			\begin{align}
				\mathscr{F}^X_{t_0}
				= \sigma(X_t;\ 0 \leq t \leq t_0)
				= \bigcup_{S \subset [0,t_0]:at\ most\ countable} \mathcal{F}^X_S
			\end{align}
			が満たされる.すなわち,$\mathscr{F}^X_{t_0}$の任意の元は
			$\left\{(X_{t_1},X_{t_2},\cdots) \in B \right\},\ (t_1 < t_2 < \cdots)$の形で表される.
			
		\item[第二段]
	\end{description}
\end{prf}

\begin{itembox}[l]{Problem 1.10 unsolved}
		Let $X$ be a process with every sample path LCRL, and 
		let A be the event that $X$ is continuous on $[0,x_0]$.
		Let $X$ be adapted to a right-continuous filtration 
		$(\mathscr{F}_t)_{t \geq 0}$. Show that $A \in \mathscr{F}_{t_0}$.
\end{itembox}

\begin{prf}\mbox{}
	\begin{description}
		\item[第一段]
			$\Q^* \coloneqq \Q \cap [0,t_0]$とおく.
			いま,任意の$n \geq 1$と$r \in \Q^*$に対し
			\begin{align}
				B_n (r) \coloneqq
				\bigcup_{m \geq 1} \bigcap_{k \leq m} 
				\Set{\omega \in \Omega}{\left| X_r(\omega)-X_{r+\frac{1}{k}}(\omega) \right| \leq \frac{1}{n}}
			\end{align}
			と定めるとき,
			\begin{align}
				A = \bigcap_{r \in \Q^*} \bigcap_{n \geq 1} B_n(r)
			\end{align}
			が成立することを示す.これが示されれば,
			\begin{align}
				\Set{\omega \in \Omega}{\left| X_r(\omega)-X_{r+\frac{1}{k}}(\omega) \right| \leq \frac{1}{n}}
				\in \mathscr{F}_{r+\frac{1}{k}},
				\quad (\forall r \in \Q^*,\ k \geq 1)
			\end{align}
			とフィルトレーションの右連続性から
			\begin{align}
				B_n (r) \in \bigcap_{k \geq m} \mathscr{F}_{r+\frac{1}{k}} = \mathscr{F}_{r+} = \mathscr{F}_{r}
			\end{align}
			が従い$A \in \mathscr{F}_{t_0}$が出る.
		
		\item[第二段]
			
	\end{description}
\end{prf}

\begin{itembox}[l]{Problem 1.16}
	If the process $X$ is measurable and the random time $T$ is finite, 
	then the function $X_T$ is a random variable.
\end{itembox}

\begin{prf}
	\begin{align}
		\tau:\Omega \ni \omega \longmapsto (T(\omega),\omega) \in [0,\infty) \times \Omega
	\end{align}
	とおけば,
	任意の$A \in \borel{[0,\infty)},\ B \in \mathscr{F}$に対して
	\begin{align}
		\tau^{-1}(A \times B) = \Set{\omega \in \Omega}{(T(\omega),\omega) \in A \times B}
		= T^{-1}(A) \cap B \in \mathscr{F}
	\end{align}
	が満たされる
	\begin{align}
		\Set{A \times B}{A \in \borel{[0,\infty)},\ B \in \mathscr{F}}
		\subset \Set{E \in \borel{[0,\infty)} \otimes \mathscr{F}}{\tau^{-1}(E) \in \mathscr{F}}
	\end{align}
	が従い$\tau$の$\mathscr{F}/\borel{[0,\infty)} \otimes \mathscr{F}$-可測性が出る.
	$X_T = X \circ \tau$より$X_T$は可測$\mathscr{F}/\borel{\R^d}$である.
	\QED
\end{prf}

\begin{itembox}[l]{Problem 1.17}
	Let $X$ be a measurable process and $T$ a random time. Show that 
	the collection of all sets of the form $\{X_T \in A\}$ and 
	$\{X_T \in A\} \cup \{T = \infty\};A \in \borel{\R}$, forms a 
	sub-$\sigma$-field of $\mathscr{F}$.
\end{itembox}

\begin{prf}
	$X_T$の定義域は$\{T<\infty\}$であるから,
	\begin{align}
		\mathscr{G} \coloneqq \Set{\{T < \infty\} \cap E}{E \in \mathscr{F}}
	\end{align}
	とおけば,前問の結果より$X_T$は可測$\mathscr{G}/\borel{\R}$である.
	$\mathscr{G} \subset \mathscr{F}$より
	\begin{align}
		\mathscr{H} \coloneqq \Set{\{X_T \in A\},\ \{X_T \in A\} \cup \{T = \infty\}}{A \in \borel{\R}}
	\end{align}
	に対して$\mathscr{H} \subset \mathscr{F}$が成立する.
	あとは$\mathscr{H}$が$\sigma$-加法族であることを示せばよい.
	実際,$A = \R$のとき
	\begin{align}
		\{X_T \in A\} \cup \{T = \infty\} = \{T < \infty\} \cup \{T = \infty\} = \Omega
	\end{align}
	となり$\Omega \in \mathscr{H}$が従い,また
	\begin{align}
		&\{X_T \in A\}^c = \{X_T \in A^c\} \cup \{T = \infty\}, \\
		&\left( \{X_T \in A\} \cup \{T = \infty\} \right)^c
		=  \{X_T \in A^c\} \cap \{T < \infty\}
		= \{X_T \in A^c\}
	\end{align}
	より$\mathscr{H}$は補演算で閉じる.更に$B_n \in \mathscr{H}\ (n=1,2,\cdots)$を取れば,
	\begin{align}
		\bigcup_{n=1}^{\infty} B_n = \left\{X_T \in \bigcup_{n=1}^{\infty} A_n \right\}
	\end{align}
	或は
	\begin{align}
		\bigcup_{n=1}^{\infty} B_n = \left\{X_T \in \bigcup_{n=1}^{\infty} A_n \right\} \cup \{T = \infty\}
	\end{align}
	が成立し$\bigcup_{n=1}^\infty B_n \in \mathscr{H}$を得る.
	\QED
\end{prf}
\section{Stopping Times}
	\begin{itembox}[l]{$[0,\infty]$の位相}
		$[0,\infty]$の位相は拡張実数$[-\infty,\infty]$の相対位相である.
		$O \subset [-\infty,\infty]$が開集合であるとは,
		任意の$x \in O$に対し,
		\begin{description}
			\item[(O1)] $x \in \R$なら或る$\epsilon > 0$が存在して
				$B_\epsilon(x) \subset O$が満たされる,
			
			\item[(O2)] $x = \infty$なら或る$a \in \R$が存在して
				$(a,\infty] \subset O$が満たされる,
			
			\item[(O3)] $x = -\infty$なら或る$a \in \R$が存在して
				$[-\infty,a) \subset O$が満たされる,
		\end{description}
		で定義される.この性質を満たす$O$の全体に$\emptyset$を加えたものが
		$[-\infty,\infty]$の位相であり,
		\begin{align}
			[-\infty,r),\quad (r,r'), \quad (r,\infty],
			\quad (r,r' \in \Q)
		\end{align}
		の全体が可算開基となる.従って$[0,\infty]$の位相の可算開基は
		\begin{align}
			[0,r),\quad (r,r'), \quad (r,\infty],
			\quad (r,r' \in \Q \cap [0,\infty])
		\end{align}
		の全体であり,写像$\tau:\Omega \longrightarrow [0,\infty]$が
		$\mathscr{F}/\borel{[0,\infty]}$-可測性を持つかどうかを調べるには
		\begin{align}
			\{\tau < a\} = \tau^{-1}([0,a)) \in \mathscr{F},
			\quad (\forall a \in (0,\infty))
		\end{align}
		が満たされているかどうかを確認すれば十分である.
	\end{itembox}
	
	\begin{itembox}[l]{Problem 2.2}
		Let $X$ be a stochastic process and $T$ a stopping time of 
		$\left\{ \mathscr{F}^X_t \right\}$. Suppose that for some pair $\omega,\omega' \in \Omega$, 
		we have $X_t(\omega) = X_t(\omega')$ for all $t \in [0,T(\omega)] \cap [0,\infty)$. 
		Show that $T(\omega) = T(\omega')$. 
	\end{itembox}
	
	\begin{prf}[参照元:\cite{key3}]
		$\omega,\omega'$を分離しない集合族$\mathscr{H}$を
		\begin{align}
			\mathscr{H} \coloneqq \Set{A \subset \Omega}{\{\omega,\omega'\} \subset A,\ or\ \{\omega,\omega'\} \subset \Omega \backslash A}
		\end{align}
		により定めれば,$\mathscr{H}$は$\sigma$-加法族である.このとき,
		$\{T = T(\omega)\} \in \mathscr{H}$を示せばよい.
		\begin{description}
			\item[case1]
				$T(\omega) = \infty$の場合,
				任意の$A \in \borel{\R^d}$及び$0 \leq t < \infty$に対して,
				仮定より
				\begin{align}
					\omega \in X_t^{-1}(A) \quad \Leftrightarrow \quad
					\omega' \in X_t^{-1}(A)
				\end{align}
				が成り立ち
				\begin{align}
					\sigma(X_t;\ 0 \leq t < \infty) \subset \mathscr{H}
				\end{align}
				となる.任意の$t \geq 0$に対し$\{T \leq t\} \in \mathscr{F}^X_t \subset 
				\sigma(X_t;\ 0 \leq t < \infty)$が満たされるから
				\begin{align}
					\{T = \infty\} = \bigcap_{n=1}^\infty \{T \leq n\}^c
					\in \sigma(X_t;\ 0 \leq t < \infty) \subset \mathscr{H}
				\end{align}
				が成立し,$\omega \in \{T = \infty\}$より$\omega' \in \{T = \infty\}$が従い
				$T(\omega) = T(\omega')$を得る.
				
			\item[case2]
				$T(\omega) < \infty$の場合,
				case1と同様に任意の$0 \leq t \leq T(\omega)$に対し
				$\sigma(X_t) \subset \mathscr{H}$が満たされるから
				\begin{align}
					\mathscr{F}^X_{T(\omega)} \subset \mathscr{H}
				\end{align}
				が成り立つ.$\{T = T(\omega)\} \in \mathscr{F}^X_{T(\omega)}$より
				$\omega' \in \{T = T(\omega)\}$が従い$T(\omega) = T(\omega')$を得る.
				\QED
		\end{description}
	\end{prf}
	
	\begin{itembox}[l]{Lemma for Proposition 2.3}
		$(\mathscr{F}_t)_{t \geq 0}$を可測空間
		$(\Omega,\mathscr{F})$のフィルトレーションとするとき,
		任意の$t \geq 0$及び任意の点列$s_1  > s_2 > \cdots > t, (s_n \downarrow t)$
		に対して次が成立する:
		\begin{align}
			\bigcap_{s>t} \mathscr{F}_s = \bigcap_{n=1}^\infty \mathscr{F}_{s_n}.
		\end{align}
	\end{itembox}
	
	\begin{prf}
		先ず任意の$n \geq 1$に対して
		\begin{align}
			\bigcap_{s > t} \mathscr{F}_s \subset \mathscr{F}_{s_n}
		\end{align}
		が成り立つから
		\begin{align}
			\bigcap_{s > t} \mathscr{F}_s \subset \bigcap_{n=1}^\infty \mathscr{F}_{s_n}
		\end{align}
		を得る.一方で,任意の$s > t$に対し$s \geq s_n$を満たす$n$が存在するから,
		\begin{align}
			\mathscr{F}_s \supset  \mathscr{F}_{s_n}
			\supset \bigcap_{n=1}^\infty \mathscr{F}_{s_n}
		\end{align}
		が成立し
		\begin{align}
			\bigcap_{s > t} \mathscr{F}_s \supset \bigcap_{n=1}^\infty \mathscr{F}_{s_n}
		\end{align}
		が従う.
		\QED
	\end{prf}
	
	$(\mathscr{F}_{t+})_{t \geq 0}$は右連続である.実際,任意の$t \geq 0$で
	\begin{align}
		\bigcap_{s > t} \mathscr{F}_{s+} = \bigcap_{s > t} \bigcap_{u > s} \mathscr{F}_u
		= \bigcap_{s > t} \mathscr{F}_s
		= \mathscr{F}_{t+}
	\end{align}
	が成立する.
	
	\begin{itembox}[l]{Corollary 2.4}\label{chapter_1_Corollary_2_4}
		$T$ is an optional time of the filtration $\{\mathscr{F}_t\}$ if and only if 
		it is a stopping time of the (right-continuous!) filtration $\{\mathscr{F}_{t+}\}$.
	\end{itembox}
	言い換えれば,確率時刻$T$に対し
	\begin{align}
		\{T < t\} \in \mathscr{F}_t,\ \forall t \geq 0
		\quad \Leftrightarrow \quad
		\{T \leq t\} \in \mathscr{F}_{t+},\ \forall t \geq 0
	\end{align}
	が成り立つことを主張している.
	\begin{prf}
		$T$が$(\mathscr{F}_{t+})$-停止時刻であるとき,
		任意の$n \geq 1$に対して$\{T \leq t - 1/n\} \in \mathscr{F}_{(t-1/n)+} \subset \mathscr{F}_t$
		が満たされるから
		\begin{align}
			\{T < t\} = \bigcup_{n=1}^\infty \left\{T \leq t - \frac{1}{n}\right\} \in \mathscr{F}_t
		\end{align}
		が従う.逆に$T$が$(\mathscr{F}_t)$-弱停止時刻
		\footnote{
			optional time の訳語がわからないので弱停止時刻と呼ぶ.
		}
		のとき,任意の$m \geq 1$に対し
		\begin{align}
			\{T \leq t\} = \bigcap_{n=m}^\infty \left\{T < t+\frac{1}{n} \right\}
			\in \mathscr{F}_{t + 1/m}
		\end{align}
		が成立するから
		\begin{align}
			\{T \leq t\} \in \bigcap_{n=1}^\infty \mathscr{F}_{t + 1/n} = \mathscr{F}_{t+}
		\end{align}
		を得る.
		\QED
	\end{prf}
	
	\begin{itembox}[l]{Problem 2.6}
		If the set $\Gamma$ in Example 2.5 is open, show that $H_\Gamma$ is 
		an optional time.
	\end{itembox}
	
	\begin{prf}
		$\{H_\Gamma < 0\}=\emptyset$であるから,以下$t > 0$とする.
		$H_\Gamma(\omega) < t \Leftrightarrow \exists s < t,\ X_s(\omega) \in \Gamma$より
		\begin{align}
			\{H_\Gamma < t\} = \bigcup_{0 \leq s < t} \{X_s \in \Gamma\}
		\end{align}
		となる.また全てのパスが右連続であることと$\Gamma$が開集合であることにより
		\begin{align}
			\bigcup_{0 \leq s < t} \{X_s \in \Gamma\}
			= \bigcup_{\substack{0 \leq r < t \\ r \in \Q}} \{X_r \in \Gamma\}
		\end{align}
		が成り立ち$\{H_\Gamma < t\} \in \mathscr{F}_t$が従う.
		\QED
	\end{prf}
	
	\begin{itembox}[l]{Problem 2.7}
		If the set $\Gamma$ in Example 2.5 is closed and the sample paths of the 
		process $X$ are continuous, then $H_\Gamma$ is a stopping time.
	\end{itembox}
	
	\begin{prf}\mbox{}
		\begin{description}
			\item[第一段]
				$\R^d$上のEuclid距離を$\rho$で表し,
				\begin{align}
					\rho(x,\Gamma) \coloneqq \inf{y \in \Gamma}{\rho(x, y)},
					\quad \Gamma_n \coloneqq \Set{x \in \R^d}{\rho(x,\Gamma) < \frac{1}{n}},
					\quad (x \in \R^d,\ n=1,2,\cdots)
				\end{align}
				とおく.$\R^d \ni x \longmapsto \rho(x,\Gamma)$の連続性より$\Gamma_n$は開集合であるから,
				Problem 2.6の結果より$T_n \coloneqq H_{\Gamma_n}$で定める$T_n,\ n=1,2,\cdots$は
				$(\mathscr{F}_t)$-弱停止時刻であり,
				また$H \coloneqq H_\Gamma$とおけば次の(1)と(2)が成立する:
				\begin{description}
					\setlength{\leftskip}{3.0cm}
					\item[(1)] $\{H = 0\} = \{X_0 \in \Gamma\}$,
					
					\setlength{\leftskip}{3.0cm}
					\item[(2)] $H(\omega) \leq t 
					\quad \Leftrightarrow \quad 
					T_n(\omega) < t,\ \forall n=1,2,\cdots,
					\quad (\forall \omega \in \{H>0\},\ \forall t>0)$.
				\end{description}
				(1)と(2)及び$T_n,\ n=1,2,\cdots$が$(\mathscr{F}_t)$-弱停止時刻であることにより
				\begin{align}
					\{H \leq t\}
					= \{H \leq t\} \cap \{H > 0\} + \{H = 0\}
					= \left\{ \bigcap_{n=1}^\infty \{T_n < t\} \right\} \cap \{H > 0\} + \{H = 0\}
					\in \mathscr{F}_t,
					\quad (\forall t \geq 0)
				\end{align}
				が成立するから$H$は$(\mathscr{F}_t)$-停止時刻である.
			
			\item[第二段]
				(1)を示す.実際,
				$X_0(\omega) \in \Gamma$なら$H(\omega) = 0$であり,
				$X_0(\omega) \notin \Gamma$なら,$\Gamma$が閉であることと
				パスの連続性より
				\begin{align}
					X_t(\omega) \notin \Gamma,
					\quad (0 \leq t \leq h)
				\end{align}
				を満たす$h > 0$が存在して$H(\omega) \geq h > 0$となる.
		
			\item[第三段]
				$\omega \in \{H>0\},\ t > 0$として(2)を示す.まずパスの連続性より
				\begin{align}
					T_n(\omega) < t \quad \Leftrightarrow \quad
					\exists s \leq t, \quad X_s(\omega) \in \Gamma_n
				\end{align}
				が成り立つ.$H(\omega) \leq t$の場合,
				$\beta \coloneqq H(\omega)$とおけば,$\Gamma$が閉であることと
				パスの連続性より
				\begin{align}
					X_\beta(\omega) \in \Gamma \subset \Gamma_n,
					\quad (\forall n=1,2,\cdots)
				\end{align}
				が満たされ$T_n(\omega) < t\ (\forall n \geq 1)$が従う.
				逆に,$H(\omega) > t$のとき
				\begin{align}
					X_s(\omega) \notin \Gamma,
					\quad (\forall s \in [0,t])
				\end{align}
				が満たされ,パスの連続性と$\rho$の連続性より
				$[0,t] \ni s \longmapsto \rho(X_s(\omega),\Gamma)$
				は連続であるから,
				\begin{align}
					d \coloneqq \min{s \in [0,t]}{\rho(X_s(\omega),\Gamma)} > 0
				\end{align}
				が定まる.このとき$1/n < d/2$を満たす$n \geq 1$を一つ取れば
				\begin{align}
					X_s(\omega) \notin \Gamma_n,
					\quad (\forall s \in [0,t])
				\end{align}
				が成立する.実際,任意の$s \in [0,t],\ x \in \Gamma_n$に対し
				\begin{align}
					\rho(X_s(\omega),x)
					\geq \rho(X_s(\omega),\Gamma) - \rho(x,\Gamma)
					\geq d - \frac{d}{2}
					= \frac{d}{2}
					> \frac{1}{n}
				\end{align}
				が満たされる.従って$T_n(\omega) \geq t$となる.
				\QED
		\end{description}
	\end{prf}
	
	\begin{itembox}[l]{Lemma 2.9 の式変形について}
		第一の式変形は
		\begin{align}
			\{T + S > t\}
			&= \{T = 0,\ T+S > t\} + \{0 < T < t,\ T+S > t\} + \{T \geq t,\ T+S > t\} \\
			&= \{T = 0,\ T+S > t\} + \{0 < T < t,\ T+S > t\} + \{T \geq t,\ T+S > t,\ S = 0\} \\
				&\quad+ \{T \geq t,\ T+S > t,\ S > 0\} \\
			&= \{T = 0,\ S > t\} + \{0 < T < t,\ T+S > t\} + \{T > t,\ S = 0\}
				+ \{T \geq t,\ S > 0\}
		\end{align}
		である.
	\end{itembox}
	
	\begin{itembox}[l]{Problem 2.10}
		Let $T,S$ be optional times; then $T + S$ is optional. 
		It is a stopping time, if one of the following conditions holds:
		\begin{description}
			\item[(i)] $T > 0,\ S > 0$;
			\item[(ii)] $T > 0,$ $T$ is a stopping time.
		\end{description}
	\end{itembox}
	
	\begin{prf}
		$T,S$が$(\mathscr{F}_t)$-弱停止時刻であるとすれば,
		任意の$t > 0$に対し
		\begin{align}
			\{T + S < t\}
			&= \{T = 0,\ T + S < t\} + \{0 < T < t,\ T + S < t\} \\
			&= \{T = 0,\ S < t\} + \bigcup_{\substack{0 < r < t \\ r \in \Q}} \{0 < T < r,\ S < t-r\} \\
			&\in \mathscr{F}_t
		\end{align}
		が成り立つから$T + S$も$(\mathscr{F}_t)$-弱停止時刻である.
		\begin{description}
			\item[(i)] この場合$\{T + S \leq 0\} = \emptyset$である.また$t > 0$なら
				\begin{align}
					\{T + S > t\} = \{0 < T < t,\ T + S > t\} + \{T \geq t,\ T + S > t \}
					= \bigcup_{\substack{0 < r < t \\ r \in \Q}} \{r < T < t,\ S > t-r\} + \{T \geq t\} \in \mathscr{F}_t
				\end{align}
				が成立する.
				
			\item[(ii)]
				この場合も$\{T + S \leq 0\} = \emptyset$であり,また$t > 0$のとき
				\begin{align}
					\{T + S > t\} &= \{0 < T < t,\ T + S > t\} + \{T \geq t,\ T + S > t \} \\
					&= \{0 < T < t,\ T + S > t\} + \{T \geq t,\ T + S > t,\ S=0 \} + \{T \geq t,\ T + S > t,\ S>0 \} \\
					&= \{0 < T < t,\ T + S > t\} + \{T > t,\ S=0 \} + \{T \geq t,\ S>0 \} \\
					&\in \mathscr{F}_t
				\end{align}
				が成立する.
				\QED
		\end{description}
	\end{prf}
	
	\begin{itembox}[l]{Problem 2.13}
		Verify that $\mathscr{F}_T$ is actually a $\sigma$-field and $T$ is 
		$\mathscr{F}_T$-measurable. Show that if $T(\omega) = t$ for some constant 
		$t \geq 0$ and every $\omega \in \Omega$, then $\mathscr{F}_T = \mathscr{F}_t$.
	\end{itembox}
	
	\begin{prf}\mbox{}
		\begin{description}
			\item[第一段]
				$\mathscr{F}_T$が$\sigma$-加法族であることを示す.実際,
				$\Omega \cap \{T \leq t\} = \{T \leq t\} \in \mathscr{F}_t,\ (\forall t \geq 0)$
				より$\Omega \in \mathscr{F}_T$が従い,また
				\begin{align}
					A^c \cap \{T \leq t\} = \{T \leq t\} - A \cap \{T \leq t\},
					\quad \left\{ \bigcup_{n=1}^\infty A_n \right\} \cap \{T \leq t\}
					= \bigcup_{n=1}^\infty \left( A_n \cap \{T \leq t\} \right)
				\end{align}
				より$\mathscr{F}_T$は補演算と可算和で閉じる.
				
			\item[第二段]
				任意の$\alpha \geq 0$に対し
				\begin{align}
					\{T \leq \alpha \} \cap \{T \leq t\}
					= \{T \leq \alpha \wedge t\}
					\in \mathscr{F}_{\alpha \wedge t} \subset \mathscr{F}_t
				\end{align}
				が成立し$T$の$\mathscr{F}_T/\borel{[0,\infty]}$-可測性が出る.
				
			\item[第三段]
				$A \in \mathscr{F}_T$なら$A = A \cap \{T \leq t\} \in \mathscr{F}_t$となり,
				$A \in \mathscr{F}_t$については,任意の$s \geq 0$に対し
				$s \geq t$なら
				\begin{align}
					A \cap \{T \leq s\} = A \in \mathscr{F} \subset \mathscr{F}_s,
				\end{align}
				$s < t$なら
				\begin{align}
					A \cap \{T \leq s\} = \emptyset \in \mathscr{F}_s
				\end{align}
				が成り立ち$A \in \mathscr{F}_T$が従う.
				\QED
		\end{description}
	\end{prf}
	
	\begin{itembox}[l]{Exercise 2.14}
		Let $T$ be a stopping time and $S$ a random time such that $S \geq T$ 
		on $\Omega$. If $S$ is $\mathscr{F}_T$-measurable, then it is also a stopping time.
	\end{itembox}
	
	\begin{prf}
		任意の$t \geq 0$に対し
		\begin{align}
			\{S \leq t\} = \{S \leq t\} \cap \{T \leq t\} \in \mathscr{F}_t
		\end{align}
		が成立する.
		\QED
	\end{prf}
	
	\begin{itembox}[l]{Problem 2.17 修正}\label{chapter_1_Problem_2_17}
		Let $T,S$ be stopping times and $Z$ an $\mathscr{F}/\borel{\R}$-measurable, 
		integrable random variable. Then
		\begin{align}
			A \in \mathscr{F}_T \quad \Rightarrow \quad A \cap \{T \leq S\}, A \cap \{T < S\} \in \mathscr{F}_{S \wedge T},
		\end{align}
		and we have
		\begin{description}
			\item[(i)] $\defunc_{\{T \leq S\}} \cexp{Z}{\mathscr{F}_T} = \defunc_{\{T \leq S\}} \cexp{Z}{\mathscr{F}_{S \wedge T}},\ \mbox{$P$-a.s.}$
			\item[(ii)] $\defunc_{\{T < S\}} \cexp{Z}{\mathscr{F}_T} = \defunc_{\{T < S\}} \cexp{Z}{\mathscr{F}_{S \wedge T}},\ \mbox{$P$-a.s.}$
			\item[(iii)] $\cexp{\cexp{Z}{\mathscr{F}_T}}{\mathscr{F}_S} = \cexp{Z}{\mathscr{F}_{S \wedge T}},\ \mbox{$P$-a.s.}$
		\end{description}
	\end{itembox}
	
	\begin{prf}\mbox{}
		\begin{description}
			\item[第一段]
				任意の$A \in \mathscr{F}_T$に対し$A \cap \{T \leq S\} \in \mathscr{F}_{S \wedge T}$
				が成り立つ.実際,
				\begin{align}
					A \cap \{T \leq S\} \cap \{S \wedge T \leq t\}
					= \biggl[ A \cap \{T \leq t\} \biggr] \cap \{T \leq S\} \cap \{S \wedge T \leq t\}
					\in \mathscr{F}_t,
					\quad (\forall t \geq 0)
				\end{align}
				が成立する.同様に$A \cap \{T < S\} \in \mathscr{F}_{S \wedge T}$も得られる.
				
			\item[第二段]
				任意の$A \in \mathscr{F}_T$に対し,前段の結果より
				\begin{align}
					\int_{A \cap \{T \leq S\}} Z\ dP
					= \int_{A \cap \{T \leq S\}} \cexp{Z}{\mathscr{F}_{S \wedge T}}\ dP
				\end{align}
				が従う.$\defunc_{\{T \leq S\}} \cexp{Z}{\mathscr{F}_{S \wedge T}}$
				も$\mathscr{F}_T/\borel{\R}$-可測であるから(i)が得られ,同様に(ii)も出る.
			
			\item[第三段]
				任意の$B \in \mathscr{F}_S$に対し,第一段と第二段の結果により
				\begin{align}
					\int_B \cexp{\cexp{Z}{\mathscr{F}_T}}{\mathscr{F}_S}\ dP
					&= \int_B \cexp{Z}{\mathscr{F}_T}\ dP
					= \int_{B\cap\{S < T\}} \cexp{Z}{\mathscr{F}_T}\ dP
						+ \int_{B\cap\{T \leq S\}} \cexp{Z}{\mathscr{F}_T}\ dP \\
					&= \int_{B \cap \{S < T\}} Z\ dP
						+ \int_{B\cap\{T \leq S\}} \cexp{Z}{\mathscr{F}_{S \wedge T}}\ dP \\
					&= \int_{B \cap \{S < T\}} \cexp{Z}{\mathscr{F}_{S \wedge T}}\ dP
						+ \int_{B\cap\{T \leq S\}} \cexp{Z}{\mathscr{F}_{S \wedge T}}\ dP \\
					&= \int_B \cexp{Z}{\mathscr{F}_{S \wedge T}}\ dP
				\end{align}
				が成り立つ.$\cexp{Z}{\mathscr{F}_{S \wedge T}}$も$\mathscr{F}_S/\borel{\R}$-可測
				であるから(iii)を得る.
				\QED
		\end{description}
	\end{prf}
	
	\begin{itembox}[l]{Proposition 2.18}\label{chapter_1_Problem_2_18}
		Let $X = \Set{X_t,\mathscr{F}_t}{0 \leq t < \infty}$ be a progressively measurable 
		process, and let $T$ be a stopping time of the filtration $\{\mathscr{F}_t\}$. 
		Then the random variable $X_T$ of Definition 1.15, defined on the set 
		$\{T < \infty\} \in \mathscr{F}_T$, is $\mathscr{F}_T$-measurable, and
		the ``stopped process'' $\Set{X_{T \wedge t},\mathscr{F}_t}{0 \leq t < \infty}$
		is progressively measurable.
	\end{itembox}
	
	\begin{prf}\mbox{}
		\begin{description}
			\item[第一段]
				停止過程の発展的可測性を示す.$t \geq 0$を固定する.
				このとき,全ての$\omega \in \Omega$に対して
				$[0,t] \ni s \longmapsto T(\omega) \wedge s$は連続であり,かつ
				全ての$s \in [0,t]$に対し$\Omega \ni \omega \longmapsto T(\omega) \wedge s$は
				$\mathscr{F}_t/\borel{[0,t]}$-可測であるから,
				$[0,t] \times \Omega \ni (s,\omega) \longmapsto T(\omega) \wedge s$
				は$\borel{[0,t]} \otimes \mathscr{F}_t/\borel{[0,t]}$-可測である.
				従って,任意の$A \in \borel{[0,t]}$と$B \in \mathscr{F}_t$に対し
				\begin{align}
					\Set{(s,\omega) \in [0,t] \times \Omega}{(T(\omega) \wedge s,\omega) \in A \times B}
					&= \Set{(s,\omega) \in [0,t] \times \Omega}{T(\omega) \wedge s \in A}
					\cap ([0,t] \times B) \\
					&\in \borel{[0,t]} \otimes \mathscr{F}_t
				\end{align}
				が成り立つから,任意の$E \in \borel{[0,t]} \otimes \mathscr{F}_t$
				に対して
				\begin{align}
					\Set{(s,\omega) \in [0,t] \times \Omega}{(T(\omega) \wedge s,\omega) \in E} 
					\in \borel{[0,t]} \otimes \mathscr{F}_t
				\end{align}
				が満たされ$(s,\omega) \longmapsto (T(\omega) \wedge s,\omega)$の
				$\borel{[0,t]} \otimes \mathscr{F}_t/\borel{[0,t]} \otimes \mathscr{F}_t$-可測性を得る.
				\begin{align}
					X(s,\omega) = X|_{[0,t] \times \Omega}(s,\omega),
					\quad (\forall (s,\omega) \in [0,t] \times \Omega)
				\end{align}
				かつ$X|_{[0,t] \times \Omega}$は$\borel{[0,t]} \otimes \mathscr{F}_t/\borel{\R^d}$-可測であるから,
				$[0,t] \times \Omega \ni (s,\omega) \longmapsto X(T(\omega) \wedge s,\omega) 
				= X|_{[0,t] \times \Omega}(T(\omega) \wedge s,\omega)$の
				$\borel{[0,t]} \otimes \mathscr{F}_t/\borel{\R^d}$-可測性が出る.
				
			\item[第二段]
				定理\ref{lem:Fubini_lemma_1} (P. \pageref{lem:Fubini_lemma_1})より
				$\omega \longmapsto X(T(\omega) \wedge t,\omega)$
				は$\mathscr{F}_t/\borel{\R^d}$であるから,
				任意の$B \in \borel{\R^d}$に対し
				\begin{align}
					\left\{ X_T \defunc_{\{T < \infty\}} \in B \right\} \cap \{T \leq t\}
					= \left\{ X_{T \wedge t} \in B \right\} \cap \{T \leq t\}
					\in \mathscr{F}_t,
					\quad (\forall t \geq 0)
				\end{align}
				が成立し$X_T \defunc_{\{T < \infty\}}$の$\mathscr{F}_T/\borel{\R^d}$-可測性を得る.
				\QED
		\end{description}
	\end{prf}
	
	\begin{itembox}[l]{Problem 2.19}
		Under the same assumption as in Proposition 2.18, and with 
		$f(t,x);[0,\infty) \times \R^d \longrightarrow \R$ a bounded,
		$\borel{[0,\infty)} \otimes \borel{\R^d}$-measurable function,
		show that the process $Y_t = \int_0^t f(s,X_s)\ ds;\ t \geq 0$ is
		progressively measurable with respect to $\{\mathscr{F}_t\}$, 
		and  $Y_T$ is an $\mathscr{F}_T$-measurable random variable.
	\end{itembox}
	
	\begin{prf}
		$[0,t] \times \Omega \ni (s,\omega) \longmapsto f(s,X_s(\omega))$
		が$\borel{[0,t]} \otimes \mathscr{F}_t/\borel{\R}$-可測であれば,
		Fuiniの定理より$\Set{Y_t,\mathscr{F}_t}{0 \leq t < \infty}$は
		適合過程となり,可積分性より$t \longmapsto Y_t(\omega),\ (\forall \omega \in \Omega)$が
		連続であるから$Y$の発展的可測性が従う.実際,
		\begin{align}
			[0,t] \times \Omega \ni (s,\omega)
			\longmapsto \left( s,X_s(\omega)\right) 
			= \left( s,X|_{[0,t] \times \Omega}(s,\omega) \right)
		\end{align}
		による$A \times B,\ (A \in \borel{[0,\infty)},\ B \in \borel{\R^d})$の引き戻しは
		\begin{align}
			\left\{ ([0,t] \cap A) \times \Omega \right\} \cap
			X|_{[0,t] \times \Omega}^{-1}(B)
			\in \borel{[0,t]} \otimes \mathscr{F}_t
		\end{align}
		となるから,$[0,t] \times \Omega \ni (s,\omega) \longmapsto f(s,X_s(\omega))$
		は$\borel{[0,t]} \otimes \mathscr{F}_t/\borel{\R}$-可測である.
		\QED
	\end{prf}
	
	\begin{itembox}[l]{Problem 2.21}
		Verify that the class $\mathscr{F}_{T+}$ is indeed a $\sigma$-field
		with respect to which $T$ is measurable, that it coincides with
		$\Set{A \in \mathscr{F}}{A \cap \{T < t\} \in \mathscr{F}_t,\ \forall t \geq 0}$,
		and that if $T$ is a stopping time (so that both $\mathscr{F}_T,\mathscr{F}_{T+}$
		are defined), then $\mathscr{F}_T \subset \mathscr{F}_{T+}$.
	\end{itembox}
	
	\begin{prf}\mbox{}
		\begin{description}
			\item[第一段]
				$\Omega \cap \{T \leq t\} = \{T \leq t\} \in \mathscr{F}_t,\ (\forall t \geq 0)$
				より$\Omega \in \mathscr{F}_{T+}$が従い,また
				\begin{align}
					A^c \cap \{T \leq t\} = \{T \leq t\} - A \cap \{T \leq t\},
					\quad \left\{ \bigcup_{n=1}^\infty A_n \right\} \cap \{T \leq t\}
					= \bigcup_{n=1}^\infty \left( A_n \cap \{T \leq t\} \right)
				\end{align}
				より$\mathscr{F}_{T+}$は補演算と可算和で閉じるから
				$\mathscr{F}_{T+}$は$\sigma$-加法族である.また,
				\begin{align}
					\{T < \alpha\} \cap \{T \leq t\}
					= \begin{cases}
						\{T < \alpha\}, & (\alpha \leq t), \\
						\{T \leq t\}, & (\alpha > t),
					\end{cases}
					\in \mathscr{F}_{t+},
					\quad (\forall t \geq 0)
				\end{align}
				より$(\mathscr{F}_t)$-弱停止時刻$T$は
				$\mathscr{F}_{T+}/\borel{[0,\infty]}$-可測である.
				
			\item[第二段]
				任意の$t \geq 0$に対し
				\begin{align}
					A \cap \{T < t\} = \bigcup_{n=1}^\infty A \cap \left\{T \leq t - \frac{1}{n}\right\},
					\quad A \cap \{T \leq t\} = \bigcap_{n=1}^\infty A \cap \left\{T < t + \frac{1}{n}\right\}
				\end{align}
				が成り立ち$\mathscr{F}_{T+} = \Set{A \in \mathscr{F}}{A \cap \{T < t\} \in \mathscr{F}_t,\ \forall t \geq 0}$
				が従う.
			
			\item[第三段]
				$T$が$(\mathscr{F}_t)$-停止時刻であるとき,
				任意の$A \in \mathscr{F}_T$に対し
				\begin{align}
					A \cap \{T \leq t\} \in \mathscr{F}_t \subset \mathscr{F}_{t+},
					\quad (\forall t \geq 0)
				\end{align}
				となり$\mathscr{F}_T \subset \mathscr{F}_{T+}$が成り立つ.
				\QED
		\end{description}
	\end{prf}
	
	\begin{itembox}[l]{Lemma: 弱停止時刻の可測性}
		$T$を$(\mathscr{F}_t)$-弱停止時刻とすれば,任意の
		$t \geq 0$に対し$T \wedge t$は$\mathscr{F}_t/\borel{[0,\infty)}$-可測である.
	\end{itembox}
	
	\begin{prf}
		任意の$\alpha \geq 0$に対し
		\begin{align}
			\{T \wedge t \leq \alpha\} = 
			\begin{cases}
			\Omega, & (t \leq \alpha), \\
			\{T \leq \alpha \}, & (t > \alpha),
			\end{cases}
			 \in \mathscr{F}_t
		\end{align}
		が成立する.
		\QED
	\end{prf}
	
	\begin{itembox}[l]{Probelem 2.22}
		Verify that analogues of Lemmas 2.15 and 2.16 hold if $T$ and
		$S$ are assumed to be optional and $\mathscr{F}_T,\ \mathscr{F}_S$
		and $\mathscr{F}_{T \wedge S}$ are replaced by $\mathscr{F}_{T+},\ \mathscr{F}_{S+}$
		and $\mathscr{F}_{(T \wedge S)+}$, respectively. Prove that if $S$ is 
		an optional time and $T$ is a positive stopping time with $S \leq T$,
		and $S < T$ on $\{S < \infty\}$, then $\mathscr{F}_{S+} \subset \mathscr{F}_T$.
	\end{itembox}
	
	\begin{prf}\mbox{}
		\begin{description}
			\item[第一段]
				$T \wedge t,\ S \wedge t$は
				$\mathscr{F}_t/\borel{[0,\infty)}$-可測であるから、
				任意の$A \in \mathscr{F}_{S+}$に対して
				\begin{align}
					A \cap \{S \leq T\} \cap \{T \leq t\}
					= (A \cap \{S \leq t\}) \cap \{S \wedge t \leq T \wedge t\} \cap \{T \leq t\}
					\in \mathscr{F}_{t+},
					\quad (\forall t \geq 0)
				\end{align}
				となり$A \cap \{S \leq T\} \in \mathscr{F}_{T+}$が成立する.
				特に,$\Omega$上で$S \leq T$なら$\mathscr{F}_{S+} \subset \mathscr{F}_{T+}$が従う.
				
			\item[第二段]
				前段の結果より$\mathscr{F}_{(T \wedge S)+} \subset \mathscr{F}_{T+} \cap \mathscr{F}_{S+}$
				が満たされる.一方で,任意の$A \in \mathscr{F}_{T+} \cap \mathscr{F}_{S+}$に対し
				\begin{align}
					A \cap \{T \wedge S \leq t\}
					= \left( A \cap \{T \leq t\} \right) \cup \left( A \cap \{S \leq t\} \right)
					\in \mathscr{F}_{t+},
					\quad (\forall t \geq 0)
				\end{align}
				が成り立ち$\mathscr{F}_{(T \wedge S)+} = \mathscr{F}_{T+} \cap \mathscr{F}_{S+}$を得る.
				また
				\begin{align}
					\{S < T\} \cap \{T \wedge S \leq t\}
					= \Biggl( \bigcup_{\substack{0 \leq r \leq t \\ r \in \Q \cup \{t\}}} \{S \leq r\} \cap \{r < T\} \Biggr) 
					\cap \{S \leq t\}
					\in \mathscr{F}_{t+},
					\quad (\forall t \geq 0)
				\end{align}
				により$\{S < T\} \in \mathscr{F}_{(T \wedge S)+}$及び
				$\{T < S\} \in \mathscr{F}_{(T \wedge S)+}$となり,
				$\{T \leq S\},\{S \leq T\},\{T = S\} \in \mathscr{F}_{(T \wedge S)+}$が従う.
			
			\item[第三段]
				$T$が停止時刻で$\{T < \infty\}$上で$S < T$
				が満たされているとき.任意の$A \in \mathscr{F}_{S+}$に対し
				\begin{align}
					A \cap \{T \leq t\}
					= A \cap \{S < t\} \cap \{T \leq t\}
					\in \mathscr{F}_t,
					\quad (\forall t \geq 0)
				\end{align}
				が成り立り$\mathscr{F}_{S+} \subset \mathscr{F}_T$となる.
				\QED
		\end{description}
	\end{prf}
	
	\begin{itembox}[l]{Problem 2.23}
		Show that if $\{T_n\}_{n=1}^\infty$ is a sequence of optional times
		and $T = \inf{n \geq 1}{T_n}$, then $\mathscr{F}_{T+} = \bigcap_{n=1}^\infty \mathscr{F}_{T_n+}$.
		Besides, if each $T_n$ is a positive stopping time and $T < T_n$ on
		$\{T < \infty\}$, then we have $\mathscr{F}_{T+} = \bigcap_{n=1}^\infty \mathscr{F}_{T_n}$.
	\end{itembox}
	
	\begin{prf}
		$T \leq T_n,\ (\forall n \geq 1)$より
		$\mathscr{F}_{T+} \subset \bigcap_{n=1}^\infty \mathscr{F}_{T_n+}$
		が成り立つ.一方で$A \in \bigcap_{n=1}^\infty \mathscr{F}_{T_n+}$に対し
		\begin{align}
			A \cap \{T < t\}
			= \bigcup_{n=1}^\infty A \cap \{T_n < t\}
			\in \mathscr{F}_t,
			\quad (\forall t > 0)
			\label{eq:chapter_1_problem_2_23}
		\end{align}
		が成り立つから,Problem 2.21より$A \in \mathscr{F}_{T+}$が従う.
		また$\{T < \infty\}$上で$T < T_n,\ (\forall n \geq 1)$であるとき,
		Problem 2.22より$\mathscr{F}_{T+} \subset \bigcap_{n=1}^\infty \mathscr{F}_{T_n}$
		が従い,また$T_n,\ n \geq 1$が停止時刻の場合も(\refeq{eq:chapter_1_problem_2_23})は成立するので
		$\mathscr{F}_{T+} = \bigcap_{n=1}^\infty \mathscr{F}_{T_n}$が出る.
		\QED
	\end{prf}
	
	\begin{itembox}[l]{Problem 2.24 修正}\label{chapter_1_Problem_2_24}
		Given an optional time $T$ of the filtration $\{\mathscr{F}_t\}$,
		consider the sequence $\{T_n\}_{n=1}^\infty$ of random times given by
		\begin{align}
			T_n(\omega) = 
			\begin{cases}
				+\infty; & \mbox{on $\Set{\omega}{T(\omega) \geq n}$} \\
				\displaystyle\frac{k}{2^n}; & \mbox{on $\Set{\omega}{\frac{k-1}{2^n} \leq T(\omega) < \frac{k}{2^n}}$ for $k=1,\cdots,n2^n$},
			\end{cases}
		\end{align}
		for $n \geq 1$. Obviously $T_n \geq T_{n+1} \geq T$,
		for every $n \geq 1$. Show that each $T_n$ is a stopping time,
		that $\lim_{n \to \infty} T_n = T$, and that for every $A \in \mathscr{F}_{T+}$
		we have $A \cap \left\{ T_n = (k/2^n) \right\} \in \mathscr{F}_{k/2^n};\ n \geq 1, 1 \leq k \leq n2^n$.
	\end{itembox}
	
	\begin{prf}\mbox{}
		\begin{description}
			\item[第一段]
				$T_n(\omega)<\infty$を満たす$\omega \in \Omega$に対し,
				或る$1 \leq j \leq (n+1)2^{n+1},\ 1\leq k \leq n2^n$が存在して
				\begin{align}
					\frac{j-1}{2^{n+1}} \leq T(\omega) < \frac{j}{2^{n+1}},
					\quad \frac{k-1}{2^n} \leq T(\omega) < \frac{k}{2^n}
				\end{align}
				となる.このとき
				\begin{align}
					\frac{2k-2}{2^{n+1}} \leq T(\omega) < \frac{2k-1}{2^{n+1}}
				\end{align}
				または
				\begin{align}
					\frac{2k-1}{2^{n+1}} \leq T(\omega) < \frac{2k}{2^{n+1}}
				\end{align}
				のどちらかであるから,すなわち$j=2k-1$或は$j=2k$であり
				\begin{align}
					T(\omega) < \frac{j}{2^{n+1}} = T_{n+1}(\omega)
					\leq \frac{2k}{2^{n+1}} = T_n(\omega)
				\end{align}
				が成立する.$T_n(\omega) = \infty$の場合も併せて$T_n \geq T_{n+1} \geq T\ (\forall n \geq 1)$を得る.
			
			\item[第二段]
				任意の$t \geq 0$に対して
				\begin{align}
					\{T_n \leq t\}
					= \bigcup_{k/2^n \leq n \wedge t} \Set{\omega}{\frac{k-1}{2^n} \leq T(\omega) < \frac{k}{2^n}}
					\in \mathscr{F}_t,
					\quad (\forall t \geq 0)
				\end{align}
				が成り立つから$T_n$は$(\mathscr{F}_t)$-停止時刻である.また$\{T < \infty\}$上では
				$T(\omega) < n$のとき
				\begin{align}
					0 < T_n(\omega) - T(\omega) \leq \frac{1}{2^n} \longrightarrow 0
					\quad (n \longrightarrow \infty)
				\end{align}
				となる.
			
			\item[第三段]
				任意の$A \in \mathscr{F}_{T+}$に対して,Problem 2.21より
				\begin{align}
					A \cap \left\{T_n = \frac{k}{2^n}\right\}
					= A \cap \left\{T < \frac{k}{2^n}\right\}
					- A \cap \left\{T < \frac{k-1}{2^n}\right\}
					\in \mathscr{F}_{k/2^n}
				\end{align}
				が成り立つ.
				\QED
		\end{description}
	\end{prf}
\input{thms/chapter_1_3_A}
\subsection{Convergence Results}
	\begin{itembox}[l]{Problem 3.16}
		Let $\Set{X_t,\mathscr{F}_t}{0 \leq t < \infty}$ be a right-continuous, nonnegative
		supermatingale; then $X_\infty(\omega) = \lim_{t \to \infty} X_t(\omega)$ exists for
		$P$-a.e. $\omega \in \Omega$, and $\Set{X_t,\mathscr{F}_t}{0 \leq t \leq \infty}$ is a supermartingale.
	\end{itembox}
	
	\begin{prf}
		$\Set{-X_t,\mathscr{F}_t}{0 \leq t < \infty}$は右連続な$(\mathscr{F}_t)$-劣マルチンゲールとなり
		\begin{align}
			\sup{t \geq 0}{E(-X_t)^+} = 0
		\end{align}
		が満たされるから,劣マルチンゲール収束定理により或る$P$-零集合$A$が存在して
		\begin{align}
			Z_\infty \coloneqq \lim_{t \to \infty} (-X_t)\defunc_{\Omega \backslash A}
		\end{align}
		により$\mathscr{F}_\infty/\borel{\R}$-可測な可積分関数$Z_\infty$が定まる.
		すなわち
		\begin{align}
			X_\infty \coloneqq \lim_{t \to \infty} X_t\defunc_{\Omega \backslash A}
		\end{align}
		により$\mathscr{F}_\infty/\borel{\R}$-可測関数が定まり,
		かつ$X_\infty = -Z_\infty$より$X_\infty$は可積分である.
		またFatouの補題により任意の$t \geq 0$及び$A \in \mathscr{F}_t$に対し
		\begin{align}
			\int_A X_\infty\ dP \leq \liminf_{\substack{n \to \infty \\ n > t}} \int_A X_n\ dP \leq \int_A X_t\ dP
		\end{align}
		が成立するから$\Set{X_t,\mathscr{F}_t}{0 \leq t \leq \infty}$は優マルチンゲールである.
		\QED
	\end{prf}
	
	\begin{itembox}[l]{Exercise 3.18}
		Suppose that the filtration $\{\mathscr{F}_t\}$ satisfies the usual conditions.
		Then every right-continuous, uniformly integrable supermartingale $\Set{X_t,\mathscr{F}_t}{0 \leq t < \infty}$
		admits the Riesz decomposition $X_t = M_t + Z_t,\ \mbox{a.s. $P$}$, as the sum
		of a right-continuous, uniformly integrable martingale $\Set{M_t,\mathscr{F}_t}{0 \leq t < \infty}$
		and a potential $\Set{Z_t,\mathscr{F}_t}{0 \leq t < \infty}$.
 	\end{itembox}
 	条件を満たす二つの分解$X_t = M_t + Z_t = M'_t + Z'_t\ \mbox{a.s. $P$}, (\forall t \geq 0)$が存在する場合,
 	次の意味で分解は一意である:
 	\begin{align}
 		P \left( M_t = M'_t,\ Z_t = Z'_t,\ \forall t \geq 0 \right) = 1.
 		\label{eq:chapter_1_Exercise_3_18_4}
 	\end{align}
 	
 	\begin{prf}\mbox{}
		\begin{description}
			\item[第一段] $M$を構成する.いま,$t \geq 0$を固定する.
				$n > t$を満たす$n \in \N$と任意の$A \in \mathscr{F}_t$に対し
 				\begin{align}
 					\int_A \cexp{X_{n+1}}{\mathscr{F}_t}\ dP
 					&= \int_A X_{n+1}\ dP
 					= \int_A \cexp{X_{n+1}}{\mathscr{F}_n}\ dP \\
		 			&\leq \int_A X_n\ dP
 					= \int_A \cexp{X_n}{\mathscr{F}_t}\ dP
 				\end{align}
 				が成り立つから
 				\begin{align}
 					E \coloneqq \bigcup_{n > t}\Set{\omega \in \Omega}{\cexp{X_n}{\mathscr{F}_t}(\omega) < \cexp{X_{n+1}}{\mathscr{F}_t}(\omega)}
 				\end{align}
 				として$P$-零集合が定まる.また,同様に優マルチンゲール性より
 				\begin{align}
 					F \coloneqq \bigcup_{n > t}\Set{\omega \in \Omega}{\cexp{X_n}{\mathscr{F}_t}(\omega) > X_t(\omega)}
 				\end{align}
 				も$P$-零集合である.このとき,単調減少性より
 				\begin{align}
 					X^*_t \coloneqq \lim_{n \to \infty} \cexp{X_n}{\mathscr{F}_t} \defunc_{\Omega \backslash (E \cup F)}
 				\end{align}
 				が$-\infty$まで込めて確定し,$X^*_t$は$\mathscr{F}_t/\borel{[-\infty,\infty]}$-可測であり
 				\begin{align}
 					X_t(\omega) \geq X^*_t(\omega), \quad (\forall \omega \in \Omega \backslash (E \cup F))
 				\end{align}
 				を満たす.単調収束定理と$\sup{n \geq 1}{E|X_n|} < \infty$ (一様可積分性)より
 				\begin{align}
 					E\left( X_t - X^*_t \right)
 					= \int_{\Omega \backslash (E \cup F)} \lim_{n \to \infty} \left( X_t - \cexp{X_n}{\mathscr{F}_t} \right)\ dP
 					= \lim_{n \to \infty} \int_{\Omega \backslash (E \cup F)} X_t - \cexp{X_n}{\mathscr{F}_t}\ dP
 					= E X_t - \lim_{n \to \infty} EX_n < \infty
 				\end{align}
 				が成立するから$X^*_t$は可積分性であり$P$-a.s.に$|X^*_t| <\infty$となる.ここで
 				\begin{align}
 					X^{**}_t \coloneqq X^*_t \defunc_{|X^*_t| < \infty}
 				\end{align}
 				により$\mathscr{F}_t/\borel{\R}$-可測な可積分関数を定めれば,
 				単調収束定理より
 				\begin{align}
 					E X^{**}_t
 					= \lim_{n \to \infty} \int_\Omega \cexp{X_n}{\mathscr{F}_t}\ dP
 					= \lim_{n \to \infty} E X_n
 					\label{eq:chapter_1_Exercise_3_18_1}
 				\end{align}
 				となる.任意の$t \geq 0$に対し$X^{**}_t$を定めれば,任意の$0 \leq s < t$及び$A \in \mathscr{F}_s$に対して
 				\begin{align}
 					\int_A X^{**}_t\ dP
 					= \lim_{n \to \infty} \int_A \cexp{X_n}{\mathscr{F}_t}\ dP 
 					= \lim_{n \to \infty} \int_A \cexp{X_n}{\mathscr{F}_s}\ dP
 					= \int_A X^{**}_s\ dP
 					\label{eq:chapter_1_Exercise_3_18_3}
 				\end{align}
 				が成り立つから$\Set{X^{**}_t,\mathscr{F}_t}{0 \leq t < \infty}$はマルチンゲールである.
 				マルチンゲール性より$[0,\infty) \ni t \longmapsto EX^{**}_t$は定数であるから
 				Theorem 3.13により右連続な修正$\Set{M_t,\mathscr{F}_t}{0 \leq t < \infty}$が存在する.
 		
 			\item[第二段]
 				まず$\lim_{t \to \infty} EX_t$が存在することを示す.
 				任意の単調増大列$(t_k)_{k=1}^\infty,\ t_k \uparrow \infty$に対し優マルチンゲール性より
	 			\begin{align}
	 				\lim_{k \to \infty} EX_{t_k} = \inf{k \geq 1}{EX_{t_k}}
	 			\end{align}
	 			が確定し,任意の$n \in \N$に対し$n < t_k$を満たす$k$が存在するから
	 			\begin{align}
	 				\inf{n \geq 1}{EX_n} \geq \inf{k \geq 1}{EX_{t_k}}
	 			\end{align}
	 			が従う.逆に任意の$t_k$に対し$t_k < n$を満たす$n$が存在するから
	 			\begin{align}
	 				\lim_{n \to \infty} EX_n = \inf{n \geq 1}{EX_n} 
	 				= \inf{k \geq 1}{EX_{t_k}} = \lim_{k \to \infty} EX_{t_k}
	 			\end{align}
	 			が成立し,$(t_k)_{k=1}^\infty$の任意性から$\lim_{t \to \infty} EX_t$が存在して
	 			\begin{align}
	 				\lim_{t \to \infty} EX_t = \lim_{n \to \infty} EX_n
	 				\label{eq:chapter_1_Exercise_3_18_2}
	 			\end{align}
	 			となる.右連続な優マルチンゲール$\Set{Z_t,\mathscr{F}_t}{0 \leq t < \infty}$を
	 			\begin{align}
 					Z_t \coloneqq X_t - M_t,
 					\quad (\forall t \geq 0)
 				\end{align}
 				により定めれば,
 				(\refeq{eq:chapter_1_Exercise_3_18_1})より任意の$t \geq 0$に対し
 				\begin{align}
 					E(X_t - M_t)
 					= E X_t - E M_t
 					= EX_t - \lim_{n \to \infty} E X_n
 				\end{align}
 				が成り立ち,(\refeq{eq:chapter_1_Exercise_3_18_2})より
 				\begin{align}
 					\lim_{t \to \infty} E(X_t - M_t)
 					= \lim_{t \to \infty} E X_t - \lim_{n \to \infty} E X_n
 					= 0
 				\end{align}
 				が満たされるから$\Set{Z_t,\mathscr{F}_t}{0 \leq t < \infty}$はポテンシャルである.
			
			\item[第三段]
				分解の一意性を示す.任意の$t \geq 0$及び$A \in \mathscr{F}_t$に対し,
				(\refeq{eq:chapter_1_Exercise_3_18_3})と$M'$のマルチンゲール性より
				\begin{align}
					\int_A M_t\ dP
					= \lim_{\substack{n \to \infty \\ n > t}} \int_A X_n\ dP
					= \lim_{\substack{n \to \infty \\ n > t}} \left\{ \int_A M'_n - Z'_n\ dP \right\}
					= \lim_{\substack{n \to \infty \\ n > t}} \left\{ \int_A M'_t\ dP - \int_A Z'_n\ dP \right\}
				\end{align}
				が成立する.またポテンシャルは非負であるから
				\begin{align}
					0 \leq \int_A Z'_n\ dP \leq \int_\Omega Z'_n\ dP \longrightarrow 0
					\quad (n \longrightarrow \infty)
				\end{align}
				が成り立ち,$M_t = M'_t\ \mbox{$P$-a.s.}$及び$Z_t = Z'_t\ \mbox{$P$-a.s.}$が従う.パスの右連続性より
				(\refeq{eq:chapter_1_Exercise_3_18_4})が出る.
 				\QED
 		\end{description}
 	\end{prf}
	
	\begin{itembox}[l]{Problem 3.19}
		Assume that $\mathscr{F}_0$ contains all the $P$-negligible events in $\mathscr{F}$ \footnotemark.
		Then the following three conditions are equivalent for a nonnegative, right-continuous 
		submartingale $\Set{X_t,\mathscr{F}_t}{0 \leq t < \infty}$:
		\begin{description}
			\item[(a)] it is a uniformly integrable family of random variables;
			\item[(b)] is converges in $L^1$, as $t \to \infty$;
			\item[(c)] it converges $P$ a.s. (as $t \to \infty$) to an integrable random variable $X_\infty$,
			such that $\Set{X_t,\mathscr{F}_t}{0 \leq t \leq \infty}$ is a submartingale.
		\end{description}
		Observe that the implications (a) $\Rightarrow$ (b) $\Rightarrow$ (c) hold without the assumption of nonnegativity. 
	\end{itembox}
	\footnotetext{
		証明の第二段で出てくる$E$が$\mathscr{F}_\infty$に属していなければならない.
	}
	\begin{prf}\mbox{}
		\begin{description}
			\item[第一段]
				(a) $\Rightarrow$ (b)を示す.実際,一様可積分性の同値条件の補題より
				\begin{align}
					\sup{t \geq 0}{EX_t^+} \leq \sup{t \geq 0}{E|X_t|} < \infty
				\end{align}
				となるから,劣マルチンゲール収束定理より或る$\mathscr{F}_\infty/\borel{\R}$-可測な
				\footnote{
					Theorem 3.15における$X_\infty$は$\pm \infty$も取るが,可積分性より
					$P$-a.s.に$\R$値であるから$X_\infty \defunc_{|X_\infty|<\infty}$を$X_\infty$に置き換えればよい.
				}
				可積分関数$X_\infty$が存在して
				\begin{align}
					\lim_{t \to \infty} X_t = X_\infty
					\quad \mbox{$P$-a.s.}
				\end{align}
				が満たされる.一様可積分性と平均収束の補題より,$t_n \uparrow \infty$となる任意の単調増大列$(t_n)_{n=1}^\infty$に対して
				\begin{align}
					E|X_{t_n} - X_\infty| \longrightarrow 0
					\quad (n \longrightarrow \infty)
				\end{align}
				が成立するから
				\begin{align}
					E|X_t - X_\infty| \longrightarrow 0
					\quad (t \longrightarrow \infty)
				\end{align}
				が従う.
			
			\item[第二段]
				(b) $\Rightarrow$ (c)を示す.(b)の下で,或る可積分関数$X_*$が存在して
				\begin{align}
					E|X_n - X_*| \longrightarrow 0
					\quad (n \longrightarrow \infty)
				\end{align}
				が満たされるから,或る部分列$\left( X_{n_k} \right)_{k=1}^\infty$と$P$-零集合$E$が存在して
				\begin{align}
					\lim_{k \to \infty} X_{n_k}(\omega) = X_*(\omega),
					\quad (\forall \omega \in \Omega \backslash E)
				\end{align}
				となる.$X_{n_k}\defunc_{\Omega \backslash E}$は全て$\mathscr{F}_\infty/\borel{\R}$-可測であるから,
				\begin{align}
					X_\infty \coloneqq \lim_{k \to \infty} X_{n_k} \defunc_{\Omega \backslash E}
				\end{align}
				とおけば$X_\infty$は$\mathscr{F}_\infty/\borel{\R}$-可測,
				かつ$X_\infty = X^*\ \mbox{$P$-a.s.}$より可積分であり
				\begin{align}
					E|X_n - X_\infty| = E|X_n - X_*| \longrightarrow 0
					\quad (n \longrightarrow \infty)
					\label{eq:chapter_1_Problem_3_19_3}
				\end{align}
				を満たす.任意の$t \geq 0$及び$A \in \mathscr{F}_t$に対し
				\begin{align}
					\int_A X_t\ dP \leq \int_A X_n\ dP,
					\quad (\forall n > t)
					\label{eq:chapter_1_Problem_3_19_1}
				\end{align}
				が成り立つから,(\refeq{eq:chapter_1_Problem_3_19_3})より
				\begin{align}
					\int_A X_t\ dP \leq \int_A X_\infty\ dP
					\label{eq:chapter_1_Problem_3_19_2}
				\end{align}
				が出る.
				
			\item[第三段]
				$X_t \geq 0\ (\forall t \geq 0)$を仮定して(c) $\Rightarrow$ (a)を示す.実際,
				劣マルチンゲール性より
				\begin{align}
					\int_{|X_t| > \lambda} |x_t|\ dP
					= \int_{X_t > \lambda} X_t\ dP
					\leq \int_{X_t > \lambda} X_\infty\ dP
				\end{align}
				かつ
				\begin{align}
					P\left( X_t > \lambda \right)
					\leq \frac{1}{\lambda} EX_t
					\leq \frac{1}{\lambda} EX_\infty
				\end{align}
				が成り立ち,$X_\infty$の可積分性より
				\begin{align}
					\sup{t \geq 0}{\int_{|X_t| > \lambda} |x_t|\ dP} 
					\longrightarrow 0
					\quad (\lambda \longrightarrow \infty)
				\end{align}
				となる.
				\QED
		\end{description}
	\end{prf}
	
	\begin{itembox}[l]{Problem 3.20}
		Assume that $\mathscr{F}_0$ contains all the $P$-negligible events in $\mathscr{F}$.
		Then the following four conditions are equivalent for a right-continuous martingale
		$\Set{X_t,\mathscr{F}_t}{0 \leq t < \infty}$:
		\begin{description}
			\item[(a),(b)] as in Problem 3.19;
			\item[(c)] it converges $P$ a.s. (as $t \to \infty$) to an integrable random variable $X_\infty$,
				such that $\Set{X_t,\mathscr{F}_t}{0 \leq t \leq \infty}$ is a martingale;
			\item[(d)] there exists an integrable random variable $Y$, such that $X_t = \cexp{Y}{\mathscr{F}_t}$ a.s. $P$,
				for every $t \geq 0$.
		\end{description}
		Besides, if (d) holds and $X_\infty$ is the random variable in (c), then
		\begin{align}
			\cexp{Y}{\mathscr{F}_\infty} = X_\infty
			\quad \mbox{a.s. $P$}.
			\label{eq:chapter_1_Problem_3_20_1}
		\end{align}
	\end{itembox}
	
	\begin{prf}\mbox{}
		\begin{description}
			\item[第一段] マルチンゲールは劣マルチンゲールであるから,Problem 3.19より(a) $\Rightarrow$ (b)が従う.
				また今の仮定の下では
				(\refeq{eq:chapter_1_Problem_3_19_1})と(\refeq{eq:chapter_1_Problem_3_19_2})
				の不等号が等号に代わり(b) $\Rightarrow$ (c)となる.$Y \coloneqq X_\infty$として(c) $\Rightarrow$ (d)が得られ,
				一様可積分性と条件付き期待値に関する補題(P. \pageref{lem:uniformly_integrability_and_conditional_expectations})
				より(d) $\Rightarrow$ (a)が出る.
				
			\item[第二段]
				(\refeq{eq:chapter_1_Problem_3_20_1})を示す.
				いま,任意の$t \geq 0$及び$A \in \mathscr{F}_t$に対し
				\begin{align}
					\int_A Y\ dP = \int_A X_t\ dP = \int_A X_\infty\ dP
				\end{align}
				が成立するから
				\begin{align}
					\int_A Y\ dP = \int_A X_\infty\ dP,
					\quad (\forall A \in \bigcup_{t \geq 0} \mathscr{F}_t)
				\end{align}
				が従う.$Y$と$X_\infty$の可積分性より
				\begin{align}
					\mathscr{D} \coloneqq
					\Set{A \in \mathscr{F}_\infty}{\int_A Y\ dP = \int_A X_\infty\ dP}
				\end{align}
				はDynkin族をなし乗法族$\bigcup_{t \geq 0} \mathscr{F}_t$を含むから,
				Dynkin族定理より
				\begin{align}
					\int_A Y\ dP = \int_A X_\infty\ dP,
					\quad (\forall A \in \mathscr{F}_\infty)
				\end{align}
				が成立する.
				\QED
		\end{description}
	\end{prf}
\subsection{The Optional Sampling Theorem}
	\begin{itembox}[l]{Lemma: 離散時間の任意抽出定理}
		$0 = t_0 < t_1 < \cdots < t_n < \infty$とし,
		$\Set{X_{t_n},\mathscr{F}_{t_n}}{n=0,\cdots,n}$を劣マルチンゲール,
		$S,T:\Omega \longrightarrow \{t_0,t_1,\cdots,t_n,\infty\}$を$(\mathscr{F}_{t_n})$-停止時刻とする.
		また或る$\mathscr{F}/\borel{\R}$-可測関数$Y$が存在して
		\begin{align}
			X_T(\omega) \coloneqq Y(\omega)\ (\forall \omega \in \{T=\infty\}),
			\quad X_S(\omega) \coloneqq Y(\omega)\ (\forall \omega \in \{S=\infty\})
		\end{align}
		を満たしているとする.このとき,
		\begin{description}
			\item[(a)] $S,T < \infty$.
			\item[(b)] $Y$が可積分で$\cexp{Y}{\mathscr{F}_{t_n}} \geq X_{t_n}\ \mbox{a.s. $P$},\ (n=0,\cdots,n)$を満たす.
		\end{description}
		のいずれかの場合次が成り立つ:
		\begin{align}
			\cexp{X_T}{\mathscr{F}_S} \geq X_{S \wedge T}
			\quad \mbox{a.s. $P$}.
			\label{eq:lem_optional_sampling_theorem_1}
		\end{align}
	\end{itembox}
	
	\begin{prf}\mbox{}
		\begin{description}
			\item[第一段]	
				$S \leq T$と仮定して(\refeq{eq:lem_optional_sampling_theorem_1})を示す.先ず
				\begin{align}
					\int_\Omega |X_S|\ dP
					= \sum_{i=0}^n \int_{\{S=t_i\}} |X_{t_i}|\ dP
						+ \int_{\{S=\infty\}} |Y|\ dP
				\end{align}
				より(a),(b)いずれの場合も$X_S,X_T$は可積分である.
				また,劣マルチンゲール性より任意の$A \in \mathscr{F}_S$に対して
				\begin{align}
					\int_{A \cap \{S=t_i\}} X_{t_i}\ dP
					&= \int_{A \cap \{S=t_i\} \cap \{T=t_i\}} X_{t_i}\ dP
						+ \int_{A \cap \{S=t_i\} \cap \{T>t_i\}} X_{t_i}\ dP \\
					&\leq \int_{A \cap \{S=t_i\} \cap \{T=t_i\}} X_T\ dP
						+ \int_{A \cap \{S=t_i\} \cap \{T>t_i\}} X_{t_{i+1}}\ dP \\
					&= \int_{A \cap \{S=t_i\} \cap \{T=t_i\}} X_T\ dP
						+ \int_{A \cap \{S=t_i\} \cap \{T=t_{i+1}\}} X_T\ dP
						+ \int_{A \cap \{S=t_i\} \cap \{T>t_{i+1}\}} X_{t_{i+1}}\ dP \\
					&\cdots \\
					&\leq \sum_{j=i}^n \int_{A \cap \{S=t_i\} \cap \{T=t_j\}} X_T\ dP
						+ \int_{A \cap \{S=t_i\} \cap \{T>t_n\}} X_{t_n}\ dP
				\end{align}
				及び
				\begin{align}
					\int_{A \cap \{S=\infty\}} X_S\ dP
					= \int_{A \cap \{S=\infty\}} Y\ dP
					= \int_{A \cap \{S=\infty\}} X_T\ dP
				\end{align}
				が成り立つから,(a)の場合は
				\begin{align}
					\int_{A \cap \{S=t_i\}} X_{t_i}\ dP \leq
					\sum_{j=i}^n \int_{A \cap \{S=t_i\} \cap \{T=t_j\}} X_T\ dP
					= \int_{A \cap \{S=t_i\}} X_T\ dP,
				\end{align}
				(b)の場合は
				\begin{align}
					\int_{A \cap \{S=t_i\}} X_{t_i}\ dP
					&\leq \sum_{j=i}^n \int_{A \cap \{S=t_i\} \cap \{T=t_j\}} X_T\ dP
						+ \int_{A \cap \{S=t_i\} \cap \{T>t_n\}} X_{t_n}\ dP \\
					&\leq \sum_{j=i}^n \int_{A \cap \{S=t_i\} \cap \{T=t_j\}} X_T\ dP
						+ \int_{A \cap \{S=t_i\} \cap \{T>t_n\}} Y\ dP \\
					&= \int_{A \cap \{S=t_i\}} X_T\ dP
				\end{align}
				となり,いずれの場合も
				\begin{align}
					\int_A X_S\ dP
					= \sum_{i=0}^n \int_{A \cap \{S=t_i\}} X_{t_i}\ dP
						+ \int_{A \cap \{S=\infty\}} X_S\ dP
					\leq \sum_{i=0}^n \int_{A \cap \{S=t_i\}} X_T\ dP + \int_{A \cap \{S=\infty\}} X_T\ dP
					= \int_A X_T\ dP
				\end{align}
				が成立する.
			
			\item[第二段]
				一般の$S,T$に対して(\refeq{eq:lem_optional_sampling_theorem_1})を示す.
				任意の$A \in \mathscr{F}_S$に対し,Problem 2.17 (P. \pageref{chapter_1_Problem_2_17})
				と前段の結果より
				\begin{align}
					\int_A \cexp{X_T}{\mathscr{F}_S}\ dP
					&= \int_{A \cap \{S \leq T\}} \cexp{X_T}{\mathscr{F}_S}\ dP
						+ \int_{A \cap \{S > T\}} \cexp{X_T}{\mathscr{F}_S}\ dP \\
					&= \int_{A \cap \{S \leq T\}} \cexp{X_T}{\mathscr{F}_{S \wedge T}}\ dP
						+ \int_{A \cap \{S > T\}} X_T\ dP \\
					&\geq \int_{A \cap \{S \leq T\}} X_{S \wedge T}\ dP
					 	+ \int_{A \cap \{S > T\}} X_{S \wedge T}\ dP \\
					&= \int_A X_{S \wedge T}\ dP
				\end{align}
				となる.
		\end{description}
	\end{prf}
\section{The Doob-Meyer Decomposition}
	\begin{itembox}[l]{martingale transform}
		If $A = \Set{A_n,\mathscr{F}_n}{n=0,1,\cdots}$ is predictable with $E|A_n|<\infty$ for every $n$,
		and if $\Set{M_n,\mathscr{F}_n}{n=0,1,\cdots}$ is bounded martingale, then the martingale transform of $A$
		by $M$ defined by
		\begin{align}
			Y_0 = 0 \quad \mbox{and} \quad
			Y_n = \sum_{k=1}^n A_k (M_k - M_{k-1});
			\quad n \geq 1, 
		\end{align}
		is itself a martingale.
	\end{itembox}
	
	\begin{prf}
		$A_k(M_k - M_{k-1})\ (k \leq n)$は$\mathscr{F}_n/\borel{\R}$-可測であるから
		$(Y_n)_{n=1}^\infty$は$(\mathscr{F}_n)$-適合である.また
		\begin{align}
			E|Y_n| = E\left| \sum_{k=1}^n A_k (M_k - M_{k-1}) \right|
			\leq \sum_{k=1}^n \left\{\esssup{\omega \in \Omega}{\left(|M_k(\omega)|+|M_{k-1}(\omega)|\right)}\right\} E|A_k| < \infty
		\end{align}
		が成り立つ.更に任意の$n \geq 0$に対し
		\begin{align}
			\cexp{Y_{n+1} - Y_n}{\mathscr{F}_n}
			= \cexp{A_{n+1}(M_{n+1} - M_n)}{\mathscr{F}_n}
			= A_{n+1}\cexp{M_{n+1} - M_n}{\mathscr{F}_n}
			= 0,
			\quad \mbox{a.s. $P$}
		\end{align}
		が満たされる.
		\QED
	\end{prf}
	
	\begin{itembox}[l]{Doob's decomposition}\label{lem:Doob_decomposition}
		Any submartingale $\Set{X_n,\mathscr{F}_n}{n=0,1,\cdots}$ admits the unique decomposition
		$X_n = M_n + A_n$ as the summation of a martingale $\{M_n,\mathscr{F}_n\}$ and an 
		predictable and increasing sequence $\{A_n,\mathscr{F}_n\}$, where
		\begin{align}
			A_n = \sum_{k=0}^{n-1}\cexp{X_{k+1}-X_k}{\mathscr{F}_k},
			\quad \mbox{a.s. $P$},\ n \geq 1.
		\end{align}
	\end{itembox}
	
	\begin{prf}\mbox{}
		\begin{description}
			\item[第一段]
				Doob分解が存在するとして,分解の一意性を示す.
				実際,分解が存在すれば
				\begin{align}
					A_{n+1} - A_n = \cexp{A_{n+1}-A_n}{\mathscr{F}_{n}}
					= \cexp{X_{n+1}-X_n}{\mathscr{F}_{n}} - \cexp{M_{n+1}-M_n}{\mathscr{F}_{n}}
					= \cexp{X_{n+1}-X_n}{\mathscr{F}_{n}},
					\quad \mbox{a.s. $P$}
				\end{align}
				が成立し,$A_n\ (n \geq 1)$は
				\begin{align}
					A_n = \sum_{k=0}^{n-1} \cexp{X_{k+1}-X_k}{\mathscr{F}_{k}},
					\quad \mbox{a.s. $P$}
				\end{align}
				を満たすことになり分解の一意性が出る.
				
			\item[第二段]
				分解可能性を示す.
				\begin{align}
					A_0 \coloneqq 0,
					\quad A_n \coloneqq \sum_{k=0}^{n-1} \cexp{X_{k+1}-X_k}{\mathscr{F}_{k}},
					\quad (n=1,2,\cdots)
				\end{align}
				と定めれば$(A_n)$は可予測かつ可積分であり,
				\begin{align}
					A_{n+1} - A_n = \cexp{X_{k+1}-X_k}{\mathscr{F}_{k}} \geq 0,
					\quad \mbox{a.s. $P$}
					\ (\forall n \geq 1)
				\end{align}
				より増大過程である.また$M_n \coloneqq X_n - A_n$により$(\mathscr{F}_n)$-適合かつ可積分な過程を定めれば,
				\begin{align}
					\cexp{M_{n+1} - M_n}{\mathscr{F}_n}
					&= \cexp{(X_{n+1} - X_n)-(A_{n+1}-A_n)}{\mathscr{F}_n} \\
					&= \cexp{X_{n+1} - X_n}{\mathscr{F}_n} - \cexp{\cexp{X_{n+1} - X_n}{\mathscr{F}_n}}{\mathscr{F}_n}
					= 0,
					\quad \mbox{a.s. $P$}
				\end{align}
				が成り立つから$\{M_n,\mathscr{F}_n\}$はマルチンゲールである.
				\QED
		\end{description}
	\end{prf}
	
	\begin{itembox}[l]{Proposition 4.3 修正}
		An increasing random sequence $A$ \textcolor{red}{has a predictable modification}
		if and only if it is natural.
	\end{itembox}
	
	\begin{prf}
		$A$が可予測な修正$\tilde{A}$を持つとき,任意の有界マルチンゲール$M$に対して
		\begin{align}
			\tilde{Y}_0 \coloneqq 0,
			\quad \tilde{Y}_n \coloneqq \sum_{k=1}^n \tilde{A}_k(M_k - M_{k-1}); \quad n \geq 1
		\end{align}
		は$(\mathscr{F}_n)$-マルチンゲールとなる.
		このとき$M_n \tilde{A}_n$と$\sum_{k=1}^n M_{k-1}(\tilde{A}_k - \tilde{A}_{k-1})$は可積分であり
		\begin{align}
			0 = E \tilde{Y}_n = E\left[ M_n \tilde{A}_n - \sum_{k=1}^n M_{k-1}(\tilde{A}_k - \tilde{A}_{k-1}) \right]
			= E(M_n A_n) - E\sum_{k=1}^n M_{k-1}(A_k - A_{k-1}),
			\quad (\forall n \geq 1)
		\end{align}
		が成り立つから$A$はナチュラルである.逆に$A$がナチュラルであるとき,
		有界マルチンゲール$M$に対して
		\begin{align}
			0 &= E\left[ M_n A_n - \sum_{k=1}^n M_{k-1}(A_k - A_{k-1}) \right] \\
			&= E\left[ A_n(M_n-M_{n-1}) \right] - E\left[ M_{n-1} A_{n-1} - \sum_{k=1}^{n-1} M_{k-1}(A_k - A_{k-1}) \right] \\
			&= E\left[ A_n(M_n-M_{n-1}) \right],
			\quad (\forall n \geq 1)
		\end{align}
		が成り立つ.一方で
		\begin{align}
			E\left[ M_{n-1}(A_n-\cexp{A_n}{\mathscr{F}_{n-1}}) \right]
			&= E\left[ \cexp{M_{n-1} (A_n-\cexp{A_n}{\mathscr{F}_{n-1}})}{\mathscr{F}_{n-1}} \right] \\
			&= E\left[ M_{n-1} \cexp{A_n-\cexp{A_n}{\mathscr{F}_{n-1}}}{\mathscr{F}_{n-1}} \right]
			= 0,
			\quad (\forall n \geq 1)
		\end{align}
		及び
		\begin{align}
			E\left[ \cexp{A_n}{\mathscr{F}_{n-1}}(M_n-M_{n-1}) \right]
			&= E\left[ \cexp{ \cexp{A_n}{\mathscr{F}_{n-1}}(M_n-M_{n-1})}{\mathscr{F}_{n-1}} \right] \\
			&= E\left[ \cexp{A_n}{\mathscr{F}_{n-1}}\cexp{M_n-M_{n-1}}{\mathscr{F}_{n-1}} \right]
			= 0,
			\quad (\forall n \geq 1)
		\end{align}
		となるから
		\begin{align}
			E\left[ M_n(A_n - \cexp{A_n}{\mathscr{F}_{n-1}}) \right]
			&= E\left[ A_n(M_n-M_{n-1}) \right] \\
			&\quad	+ E\left[ M_{n-1}(A_n-\cexp{A_n}{\mathscr{F}_{n-1}}) \right] \\
			&\quad	- E\left[ \cexp{A_n}{\mathscr{F}_{n-1}}(M_n-M_{n-1}) \right] \\
			&= 0,
			\quad (\forall n \geq 1)
		\end{align}
		が従う.ここで各$n \geq 1$に対し,
		$\borel{\R}/\borel{\R}$-可測関数$\operatorname{sgn} = \defunc_{(0,\infty)} - \defunc_{(-\infty,0)}$を用いて
		\begin{align}
			M^{(n)}_k \coloneqq 
			\begin{cases}
				\sgn{A_n - \cexp{A_n}{\mathscr{F}_{n-1}}}, & (k \geq n), \\
				\cexp{\sgn{A_n - \cexp{A_n}{\mathscr{F}_{n-1}}}}{\mathscr{F}_k}, & (0 \leq k < n)
			\end{cases}
		\end{align}
		により有界マルチンゲール$M^{(n)} = \Set{M^{(n)}_k,\mathscr{F}_k}{k=0,1,\cdots}$を定めれば,
		\begin{align}
			0 = E\left[ M^{(n)}_n(A_n - \cexp{A_n}{\mathscr{F}_{n-1}}) \right] 
			= E\left| A_n - \cexp{A_n}{\mathscr{F}_{n-1}} \right|,
			\quad (\forall n \geq 1)
		\end{align}
		が得られ
		\begin{align}
			\tilde{A}_0 \coloneqq 0,
			\quad \tilde{A}_n \coloneqq \cexp{A_n}{\mathscr{F}_{n-1}}; \quad n \geq 1
		\end{align}
		は$A$の可予測な修正となる.
		\QED
	\end{prf}
	
	\begin{itembox}[l]{区別不能性によるパスの同値類}
		区間\footnotemark $I \subset [0,\infty)$
		の上で右連続な確率過程の全体を$RCSP(I)$と書く.また$RCSP([0,\infty))$は$RCSP$と書く.
		任意の$M = \Set{M_t}{t \in I},N = \Set{N_t}{t \in I} \in RCSP(I)$に対し,
		\begin{align}
			\{M_t = N_t,\ \forall t \in I\} = 
			\begin{cases}
				\displaystyle \bigcap_{r \in (I \cap \Q) \cup \{\sup{}{I}\}}\{M_r = N_r\}, & (\sup{}{I} \in I), \\
				\displaystyle \bigcap_{r \in I \cap \Q}\{M_r = N_r\}, & (\sup{}{I} \notin I)
			\end{cases}
		\end{align}
		となるから$\{M_t = N_t,\ \forall t \in I\}$は可測であり,
		このとき,
		\begin{align}
			M \sim N \quad \overset{\mathrm{def}}{\Longleftrightarrow} \quad 
			P(M_t = N_t,\ \forall t \in I) = 1
			\label{eq:equivalence_with_respect_to_path}
		\end{align}
		により同値関係$\sim$が定まる.
	\end{itembox}
	\footnotetext{
		この場合区間は$[a,b],(a,b),[a,b),(a,b],[a,\infty),(a,\infty),\ (0 \leq a < b < \infty)$のいずれかと考える.
	}
	
	\begin{itembox}[l]{Definition 4.4 修正}
		\textcolor{red}{Let $I \subset [0,\infty)$ be an interval.}
		An adapted process \textcolor{red}{$A = \Set{A_t,\mathscr{F}_t}{t \in I}$} 
		is called increasing if \textcolor{red}{for all $\omega \in \Omega$} we have
		\begin{description}
			\item[(a)] $A_0(\omega) = 0$
			\item[(b)] $t \longmapsto A_t(\omega)$ is nondecreasing, right-continuous function,
		\end{description}
		and $E(A_t) < \infty$ holds for every \textcolor{red}{$t \in I$}.
		An increasing process is called integrable if \textcolor{red}{$E\left(A_{\infty}\right) < \infty$,
		where $A_{\infty} = \lim_{t \to \sup{}{I}} A_t$.
		Since $A$ is nondecreasing, $A_{\infty} = A_{(\sup{}{I})-}$ if $\sup{}{I} \in I$.}
	\end{itembox}
	
	\begin{itembox}[l]{Definition 4.5 修正}
		\textcolor{red}{Let $I \subset [0,\infty)$ be an interval and $\alpha \coloneqq \inf{}{I}$.}
		An increasing processs \textcolor{red}{$A = \Set{A_t,\mathscr{F}_t}{t \in I}$} 
		is called natural if for every bounded, 
		\textcolor{red}{$RCLL$ martingale $\Set{M_t,\mathscr{F}_t}{t \in I}$} we have
		\begin{align}
			E \int_{(\alpha,t]} M_s\ dA_s = E \int_{(\alpha,t]} M_{s-}\ dA_s,
			\quad \mbox{for every $t \in (\alpha,\infty) \cap I$}.
		\end{align}
		\textcolor{red}{Let us denote the subset of $RCSP(I)$ as
		\begin{align}
			NAT(I) \coloneqq
			\Set{A \in RCSP(I)}{\mbox{natural}},
			\quad NAT \coloneqq NAT([0,\infty))
		\end{align}
		and the equivalent class of $A \in NAT$
		in the meaning of (\refeq{eq:equivalence_with_respect_to_path}) as $[A]_{NAT}\ ( \subset NAT)$.}
	\end{itembox}
	
	プロセスが$RCLL$とは全てのパスが$RCLL$であるということである.Theorem 3.8によれば
	右連続な劣マルチンゲールはa.e.のパスが$RCLL$であるから,
	(\refeq{eq:equivalence_with_respect_to_path})の意味で同値である.
	$A$も全てのパスが右連続かつ単調非減少であるから,
	全ての$\omega \in \Omega$に対し$\int_{(0,t]} M_s(\omega)\ dA_s(\omega)$と
	$\int_{(0,t]} M_{s-}(\omega)\ dA_s(\omega)$が定義される.
	たぶん余計な煩雑さを回避できる.
		
	\begin{itembox}[l]{$RCLL$なパスの不連続点は高々可算個}
		$(S,d)$を距離空間とする.写像$f:[0,\infty) \longrightarrow S$について
		各点$t \in [0,\infty)$で右連続かつ各点$t \in (0,\infty)$で左極限が存在するとき,
		$f$の不連続点は存在しても高々可算個である.
	\end{itembox}
	
	\begin{prf}
		各点$t > 0$における$f$の左極限を$f(t-)$と書けば
		\begin{align}
			\mbox{$f$が$t \in (0,\infty)$で不連続}
			\quad \Leftrightarrow \quad
			\mbox{$d(f(t),f(t-)) > 0$}
		\end{align}
		が成立するから,任意に$T > 0$を選び固定して
		\begin{align}
			D(n) \coloneqq \Set{t \in (0,T]}{\frac{1}{n+1} \leq d(f(t),f(t-)) < \frac{1}{n}},
			\quad E(n) \coloneqq \Set{t \in (0,T]}{n \leq d(f(t),f(t-)) < n+1}
		\end{align}
		とおけば
		\begin{align}
			D_T \coloneqq \Set{t \in (0,T]}{\mbox{$f$が$t \in (0,\infty)$で不連続}}
			= \bigcup_{n=1}^\infty D(n) \cup E(n)
		\end{align}
		となる.このとき$D(n),E(n)$は全て有限集合である.実際,或る$n$に対し$D(n)$が無限集合なら
		\begin{align}
			\left\{ t_k \right\}_{k=1}^\infty \subset D(n),
			\quad t_k \neq t_j\ (k \neq j)
		\end{align}
		を満たす可算集合が存在し,$[0,T]$のコンパクト性より
		或る部分列$\left( t_{k_m} \right)_{m=1}^\infty$は
		或る$y \in [0,T]$に収束する.
		$y=0$の場合,右連続の仮定より$1/2(n+1) > \epsilon > 0$に対し或る$\delta > 0$が存在して
		\begin{align}
			d(f(0),f(t)) < \epsilon, \quad (\forall 0 < t < \delta)
		\end{align}
		が成り立つが,一方で$0 < t_{k_m} < \delta$を満たす$t_{k_m}$が存在して
		\begin{align}
			\frac{1}{n+1} - \epsilon < d(f(t_{k_m}),f(t_{k_m}-)) - d(f(0),f(t_{k_m}-))
			\leq d(f(0),f(t_{k_m})) < \epsilon 
		\end{align}
		となり矛盾が生じる.
		$y > 0$の場合も,$1/2(n+1) > \epsilon > 0$に対し或る$\delta > 0$が存在して
		\begin{align}
			d(f(y-),f(t)) < \epsilon, \quad (\forall t \in (y-\delta,y))
		\end{align}
		となるが,$f$が$y$で右連続であるから(或は$y=T$のとき) $y-\delta < t_{k_m} \leq y$を満たす$t_{k_m}$が存在して
		\begin{align}
			\frac{1}{n+1} - \epsilon < 
			d(f(t_{k_m}-),f(t_{k_m})) - d(f(t_{k_m}-),f(y-)) \leq d(f(y-),f(t_{k_m})) < \epsilon
		\end{align}
		が従い矛盾が生じる.よって任意の$n \geq 1$に対して$D(n)$は有限集合であり,同様に
		$E(n)$も有限集合であるから$D_T$は高々可算集合である.
		$f$の不連続点の全体は$\bigcup_{T=1}^\infty D_T$に一致するから高々可算個である.
		\QED
	\end{prf}
	
	\begin{itembox}[l]{Remarks 4.6 (i) 修正}
		If $A$ is an increasing and $X$ a measurable process, then with $\omega \in \Omega$ fixed,
		the sample path $\Set{X_t(\omega)}{0 \leq t < \infty}$ is a measurable function from $[0,\infty)$
		into $\R$. It follows that the Lebesgue-Stieltjes integrals
		\begin{align}
			I^{\pm}_t(\omega) \coloneqq
			\int_{(0,t]} X^\pm_s(\omega)\ dA_s(\omega)
		\end{align}
		are well defined. \textcolor{red}{If $X$ is bounded, right-continuous and adapted
		to the filtration $(\mathscr{F}_t)$, then $I$ is finite, right-continuous and 
		$(\mathscr{F}_t)$-progressively measurable.}
	\end{itembox}
	
	\begin{prf}
		$X$が$\borel{[0,\infty)} \otimes \mathscr{F}/\borel{\R}$-可測なら,
		補題\ref{lem:Fubini_lemma_1} (P. \pageref{lem:Fubini_lemma_1})より
		$[0,\infty) \ni t \longmapsto X_t(\omega)$は
		$\borel{[0,\infty)}/\borel{\R}$-可測である.
		また全ての$\omega \in \Omega$に対し$t \longmapsto A_t(\omega)$は右連続非減少であるから
		\begin{align}
			\mu_\omega((a,b]) = A_b(\omega) - A_a(\omega),
			\quad (\forall (a,b] \subset [0,\infty)),
			\quad \mu_\omega(\{0\}) = 0
		\end{align}
		を満たす$\left([0,\infty),\borel{[0,\infty)}\right)$上の$\sigma$-有限測度が唯一つ存在して
		\begin{align}
			I^\pm_t(\omega) = \int_{(0,t]} X^\pm_s(\omega)\ dA_s(\omega)
			\coloneqq \int_{(0,t]} X^\pm_s(\omega)\ \mu_\omega(ds),
			\quad (0 < t < \infty)
		\end{align}
		及び$I_t \coloneqq I^+_t - I^-_t$が定義される.
		特に$\sup{s \in (0,t]}{|X^\pm_s|} \leq B < \infty$なら
		\begin{align}
			\left|I^\pm_t\right| \leq B A_t
		\end{align}
		となるから$I^\pm_t$は有限確定する.$X$が有界かつ右連続$(\mathscr{F}_t)$-適合であるとき,
		$t>0$を固定し$t^{(n)}_j \coloneqq tj/2^n$として
		\begin{align}
			X^{(n)\pm}_s \coloneqq X_0 \defunc_{\{0\}}(s) + 
				\sum_{j=0}^{2^n-1} X_{t^{(n)}_{j+1}} 
				\defunc_{\left(t^{(n)}_j,t^{(n)}_{j+1}\right]}(s)
		\end{align}
		とおけば右連続性より$X^{(n)\pm}_s \longrightarrow X^\pm_s,\ (\forall s \in [0,t])$が成立し,かつ
		\begin{align}
			I^{(n)\pm}_t \coloneqq \int_{(0,t]} X^{(n)\pm}_s\ dA_s
			= \sum_{j=0}^{2^n-1} X_{t^{(n)}_{j+1}} \left(A_{t^{(n)}_j} - A_{t^{(n)}_{j+1}}\right)
		\end{align}
		となり$I^{(n)\pm}_t$の$\mathscr{F}_t/\borel{\R}$-可測性が得られる.
		$X$が有界であるからLebesgueの収束定理より
		\begin{align}
			I^{\pm}_t = \lim_{n \to \infty} \int_{(0,t]} X^{(n)\pm}_s\ dA_s
			= \lim_{n \to \infty} I^{(n)\pm}_t
		\end{align}
		が成り立ち,定理\ref{lem:measurability_metric_space}より
		$I^{\pm}_t$の$\mathscr{F}_t/\borel{\R}$-可測性が従う.
		また$t<T$及び$\{t_n\}_{n=1}^\infty \subset (t,T],\ t_n \downarrow t$に対して,Lebesgueの収束定理より
		\begin{align}
			\lim_{n \to \infty} I^\pm_{t_n}
			= \lim_{n \to \infty} \int_{(0,T]} \defunc_{(0,t_n]}(s)X^\pm_s\ dA_s
			= \int_{(0,T]} \defunc_{(0,t]}(s)X^\pm_s\ dA_s
			= I^\pm_t
		\end{align}
		が成立し$t \longmapsto I_t(\omega)$の右連続性が出る.$I$は右連続$(\mathscr{F}_t)$-適合過程であるから
		$(\mathscr{F}_t)$-発展的可測である.
		\QED
	\end{prf}
	
	\begin{itembox}[l]{Remark 4.6 (ii) 修正}
		Every continuous, increasing process is natural. Indeed then, for \textcolor{red}{every} $\omega \in \Omega$
		we have
		\begin{align}
			\int_{(0,t]} (M_s(\omega)-M_{s-}(\omega))\ dA_s(\omega) = 0
			\quad \mbox{for every $0 < t < \infty$},
		\end{align}
		because every path $\Set{M_s(\omega)}{0 \leq s < \infty}$ has only countably many discontinuities
		(Theorem 3.8(v)).
	\end{itembox}
	
	\begin{prf}
		$RCLL$なパスの不連続点は高々可算個であり,
		連続な$A$で作る測度に対し一点集合は零集合となる.
		\QED
	\end{prf}
	
	\begin{itembox}[l]{Lemma 4.7 修正}
		If $A$ is an increasing process and $\Set{M_t,\mathscr{F}_t}{0 \leq t < \infty}$ is a bounded,
		\textcolor{red}{$RCLL$} martingale, then
		\begin{align}
			E(M_t A_t) = E \int_{(0,t]} M_s\ dA_s, \quad (\forall t > 0).
			\label{eq:chapter_1_lemma_4_7}
		\end{align}
	\end{itembox}
	
	\begin{prf}
		$t_j^{(n)} \coloneqq jt/2^n,\ (j=0,1,\cdots,2^n)$として
		\begin{align}
			M^{(n)}_s \coloneqq \sum_{j=1}^{2^n} \defunc_{\left(t^{(n)}_{j-1},t^{(n)}_j\right]}(s) M_{t^{(n)}_j},
			\quad (\forall s \in (0,t])
		\end{align}
		とおけば,$M$のパスの右連続性より任意の$s \in (0,t]$で$\lim_{n \to \infty} M^{(n)}_s = M_s$となる.また
		\begin{align}
			E \left[ A_{t^{(n)}_{j-1}} \left( M_{t^{(n)}_j} - M_{t^{(n)}_{j-1}} \right) \right]
			= E \left[ A_{t^{(n)}_{j-1}} \cexp{M_{t^{(n)}_j} - M_{t^{(n)}_{j-1}}}{\mathscr{F}_{t_{j-1}}} \right]
			= 0,
			\quad (j=1,\cdots,2^n)
		\end{align}
		が満たされるから任意の$n \geq 1$で
		\begin{align}
			E\int_{(0,t]} M^{(n)}_s\ dA_s
			&= E \sum_{j=1}^{2^n} M_{t^{(n)}_j} \left( A_{t^{(n)}_j} - A_{t^{(n)}_{j-1}} \right) \\
			&= E(M_t A_t) - \sum_{j=1}^{2^n} E \left[ A_{t^{(n)}_{j-1}} \left( M_{t^{(n)}_j} - M_{t^{(n)}_{j-1}} \right) \right] \\
			&= E(M_t A_t)
		\end{align}
		が成立する.仮定より$\sup{s \geq 0}{|M_s|} \leq b < \infty$を満たす$b$が存在して
		\begin{align}
			\left| \int_{(0,t]} M^{(n)}_s\ dA_s \right| \leq b (A_t - A_0) = b A_t,
			\quad (\forall n \geq 1)
		\end{align}
		となり,$A_t$の可積分性とLebesgueの収束定理より
		\begin{align}
			\lim_{n \to \infty} E \int_{(0,t]} M^{(n)}_s\ dA_s = E \lim_{n \to \infty} \int_{(0,t]} M^{(n)}_s\ dA_s
			= E \int_{(0,t]} M_s\ dA_s 
		\end{align}
		が従い(\refeq{eq:chapter_1_lemma_4_7})を得る.
		\QED
	\end{prf}
	
	\begin{itembox}[l]{Definition 4.8 修正}
		Let us consider the class $\mathscr{S}(\mathscr{S}_a)$ such as
		\begin{align}
			\mathscr{S} \coloneqq \Set{T:\mbox{stopping time of $(\mathscr{F}_t)$}}{\textcolor{red}{T < \infty}},
			\quad \mathscr{S}_a \coloneqq \Set{T:\mbox{stopping time of $(\mathscr{F}_t)$}}{\textcolor{red}{T \leq a}},\ (a > 0).
		\end{align}
		The right-continuous process $\Set{X_t,\mathscr{F}_t}{0 \leq t < \infty}$ is said to be 
		of class $D$, if the family $\{X_T\}_{T \in \mathscr{S}}$ is uniformly integrable;
		of class $DL$, if the family $\{X_T\}_{T \in \mathscr{S}_a}$ is uniformly integrable,
		for every $0 < a < \infty$.
	\end{itembox}
	$T \in \mathscr{S}(\mbox{resp. } \mathscr{S}_a)$, then 
	$T(\omega) < \infty\ (\mbox{resp. } \leq a)$
	for all $\omega \in \Omega$, not $P$-a.s. $\omega$.
	
	\begin{itembox}[l]{Problem 4.9 修正}
		$X = \Set{X_t,\mathscr{F}_t}{0 \leq t < \infty}$ is a right-continuous submartingale.
		Show that under any one of the following conditions, $X$ is of class $DL$.
		\begin{description}
			\item[(a)] $X_t \geq 0$ a.s. for every $t \geq 0$.
			\item[(b)] $X$ has the special form
				\begin{align}
					X_t = M_t + A_t, \quad 0 \leq t < \infty
				\end{align}
				suggested by the Doob-Meyer decomposition, where $\Set{M_t,\mathscr{F}_t}{0 \leq t < \infty}$
				is a martingale and $\Set{A_t,\mathscr{F}_t}{0 \leq t < \infty}$ is an increasing process.
		\end{description}
		Show also that if \textcolor{red}{$\mathscr{F}_0$ contains all the $P$-negligible events in $\mathscr{F}$} and
		$X$ is a uniformly integrable martingale, then it is of class $D$.
	\end{itembox}
	
	\begin{prf}\mbox{}
		\begin{description}
			\item[(a)]
				任意の$T \in \mathscr{S}_a$に対して
				$X_T$は$\mathscr{F}_T/\borel{\R}$-可測であるから
				(Proposition 2.18 修正),任意抽出定理より
				\begin{align}
					\int_{\{X_T > \lambda\}} X_T\ dP
					\leq \int_{\{X_T > \lambda\}} X_a\ dP,
					\quad (\forall \lambda > 0)
				\end{align}
				及び
				\begin{align}
					P\left( X_T > \lambda \right)
					\leq \frac{EX_T}{\lambda}
					\leq \frac{EX_a}{\lambda},
					\quad (\forall \lambda > 0)
				\end{align}
				が成立する.$X_a$が可積分であるから
				\begin{align}
					\sup{T \in \mathscr{S}_a}{\int_{\{X_T > \lambda\}} X_T\ dP}
					\longrightarrow 0
					\quad (\lambda \longrightarrow \infty)
				\end{align}
				となり,$(X_T)_{T \in \mathscr{S}_a}$の一様可積分性が得られる.
				
			\item[(b)]
				$a > 0$とすれば,任意抽出定理より
				\begin{align}
					M_T = \cexp{M_a}{\mathscr{F}_T},\ \mbox{a.s. $P$,}
					\quad (\forall T \in \mathscr{S}_a)
				\end{align}
				が成り立つから,定理\ref{lem:uniformly_integrability_and_conditional_expectations}
				(P. \pageref{lem:uniformly_integrability_and_conditional_expectations})より
				$(M_T)_{T \in \mathscr{S}_a}$は一様可積分である.このとき
				\begin{align}
					\int_{\{|X_T| > \lambda\}} |X_T|\ dP
					&\leq 2\int_{\{|M_T| > \lambda/2\}} |M_T|\ dP + 2\int_{\{|A_T| > \lambda/2\}} |A_T|\ dP \\
					&\leq 2\sup{T \in \mathscr{S}_a}{\int_{\{|M_T| > \lambda/2\}} |M_T|\ dP} + 2\int_{\{A_a > \lambda/2\}} A_a\ dP \\
					&\longrightarrow 0 \quad (\lambda \longrightarrow \infty)
				\end{align}
				が従い$(X_T)_{T \in \mathscr{S}_a}$の一様可積分性が出る.
		\end{description}
		$X$が一様可積分なマルチンゲールであるとき,Problem 3.20より
		\begin{align}
			X_t = \cexp{X_\infty}{\mathscr{F}_t},\ \mbox{a.s. $P$},
			\quad (\forall t \geq 0)
		\end{align}
		を満たす$\mathscr{F}_\infty/\borel{\R}$-可測可積分関数$X_\infty$が存在し,任意抽出定理より
		\begin{align}
			X_T = \cexp{X_\infty}{\mathscr{F}_T},\ \mbox{a.s. $P$},
			\quad (\forall T \in \mathscr{S})
		\end{align}
		が成り立つから$X$はクラス$DL$に属する.
		\QED
	\end{prf}
	
	\begin{itembox}[l]{Problem 4.11 修正}
		Let $(X,\mathscr{F},\mu)$ be a measure space and  
		$\left\{f_n\right\}_{n=1}^\infty$ be a sequence of integrable complex functions on $(X,\mathscr{F},\mu)$
		which converges weakly in $L^1$ to an integrable complex function $f$.
		Then for each $\sigma$-field $\mathscr{G} \subset \mathscr{F}$
		where $(X,\mathscr{G},\left.\mu\right|_{\mathscr{G}})$ is $\sigma$-finite,
		the sequence $\cexp{f_n}{\mathscr{G}}$ converges to $\cexp{f}{\mathscr{G}}$ weakly in $L^1$.
	\end{itembox}
	
	\begin{prf}
		$\nu \coloneqq \left.\mu\right|_{\mathscr{G}}$とおく.
		定理\ref{thm:properties_of_conditional_expectations}より
		任意の$g \in L^\infty(\mu)$と$F \in L^1(\mu)$に対して
		\begin{align}
			\int_X g\cexp{F}{\mathscr{G}}\ d\mu
			&= \int_X \cexp{g\cexp{F}{\mathscr{G}}}{\mathscr{G}}\ d\nu \\
			&= \int_X \cexp{g}{\mathscr{G}}\cexp{F}{\mathscr{G}}\ d\nu \\
			&= \int_X \cexp{\cexp{g}{\mathscr{G}}F}{\mathscr{G}}\ d\nu \\
			&= \int_X \cexp{g}{\mathscr{G}}F\ d\mu
		\end{align}
		と$\Norm{\cexp{g}{\mathscr{G}}}{L^\infty(\nu)} \leq \Norm{g}{L^\infty(\mu)}$が成り立ち
		\begin{align}
			\lim_{n \to \infty} \int_X g\cexp{f_n}{\mathscr{G}}\ d\mu
			= \lim_{n \to \infty} \int_X \cexp{g}{\mathscr{G}}f_n\ d\mu
			= \int_X \cexp{g}{\mathscr{G}}f\ d\mu
			= \int_X g\cexp{f}{\mathscr{G}}\ d\mu
		\end{align}
		となるから$\cexp{f_n}{\mathscr{G}}$は$\cexp{f}{\mathscr{G}}$に$L^1(\mu)$で弱収束する.
		\QED
	\end{prf}
	
	\begin{itembox}[l]{Lemma for theorem 4.10}\label{lem:uniqueness_of_Doob_Meyer_decomposition}
		Let $I \subset [0,\infty)$ be an interval and 
		$\Set{M_t,\mathscr{F}_t}{t \in I}$ be a right-continuous martingale,
		where the filtration $(\mathscr{F}_t)_{t \in I}$ is usual.
		If $M$ is a difference of two natural processes 
		$\Set{A_t,\mathscr{F}_t}{t \in I}$
		and $\Set{B_t,\mathscr{F}_t}{t \in I}$, namely
		\begin{align}
			M_t = A_t - B_t; \quad \forall t \in I,
		\end{align}
		then $P\Set{M_t = 0}{\forall t \in I} = 1$.
	\end{itembox}
	
	\begin{prf}
		$a_0 \coloneqq \inf{}{I}$として任意に$a \in I \cap (a_0,\infty)$を取り,
		\begin{align}
			t^{(n)}_j \coloneqq a_0 + \frac{j}{2^n}(a-a_0), \quad (j=0,1,\cdots,2^n)
		\end{align}
		とおく.任意の有界かつ$RCLL$なマルチンゲール$\xi = \Set{\xi_t,\mathscr{F}_t}{t \in I}$に対し
		\begin{align}
			\xi^{(n)}_t \coloneqq \sum_{j=1}^{2^n} \defunc_{\left(t_{j-1}^{(n)},t_j^{(n)}\right]}(t)\ \xi_{t^{(n)}_{j-1}},
			\quad (\forall t \in (a_0,a])
		\end{align}
		とおけば,任意の$\omega \in \Omega$と$t \in (a_0,a]$で
		\begin{align}
			\lim_{n \to \infty} \xi^{(n)}_t(\omega) = \xi_{t-}(\omega)
		\end{align}
		が満たされるからLebesgueの収束定理より
		\begin{align}
			&\lim_{n \to \infty} \int_{(a_0,a]} \xi^{(n)}_t(\omega)\ dA_t(\omega) 
				= \int_{(a_0,a]} \xi_{t-}(\omega)\ dA_t(\omega), \\
			&\lim_{n \to \infty} \int_{(a_0,a]} \xi^{(n)}_t(\omega)\ dB_t(\omega) 
				= \int_{(a_0,a]} \xi_{t-}(\omega)\ dB_t(\omega)
		\end{align}
		が成立する.また$A_a,B_a$の可積性と$\xi$の有界性により,再びLebesgueの収束定理を適用すれば
		\begin{align}
			E\left[ \xi_a\left( A_a - B_a \right) \right]
			&= E\left[ \xi_a A_a \right] -  E\left[ \xi_a B_a \right]
			= E \int_{(a_0,a]} \xi_{t-}\ dA_t - E\int_{(a_0,a]} \xi_{t-}\ dB_t \\
			&= E \left[ \lim_{n \to \infty} \int_{(a_0,a]} \xi^{(n)}_t\ dA_t \right]
				- E \left[ \lim_{n \to \infty} \int_{(a_0,a]} \xi^{(n)}_t\ dB_t \right] \\
			&= \lim_{n \to \infty} E\left[ \sum_{j=1}^{2^n}\xi_{t^{(n)}_{j-1}}\left( A_{t^{(n)}_j} - A_{t^{(n)}_{j-1}} \right) \right]
				-  \lim_{n \to \infty} E \left[ \sum_{j=1}^{2^n}\xi_{t^{(n)}_{j-1}}\left( B_{t^{(n)}_j} - B_{t^{(n)}_{j-1}} \right) \right] \\
			&= \lim_{n \to \infty} E \left[ \sum_{j=1}^{2^n}\xi_{t^{(n)}_{j-1}}\left( M_{t^{(n)}_j} - M_{t^{(n)}_{j-1}} \right) \right]
		\end{align}
		が従い,このとき右辺は$M$のマルチンゲール性より
		\begin{align}
			E\xi_{t^{(n)}_{j-1}}\left( M_{t^{(n)}_j} - M_{t^{(n)}_{j-1}} \right)
			= E \left[\cexp{\xi_{t^{(n)}_{j-1}}\left( M_{t^{(n)}_j} - M_{t^{(n)}_{j-1}} \right)}{\mathscr{F}_{t^{(n)}_{j-1}}} \right]
			= E \left[ \xi_{t^{(n)}_{j-1}}\cexp{M_{t^{(n)}_j} - M_{t^{(n)}_{j-1}}}{\mathscr{F}_{t^{(n)}_{j-1}}} \right]
			= 0 
		\end{align}
		となるから
		\begin{align}
			E\left[ \xi_a\left( A_a - B_a \right) \right] = 0
		\end{align}
		が得られる.$\xi$を有界マルチンゲール
		$\Set{\cexp{\sgn{A_a - B_a}}{\mathscr{F}_t},\mathscr{F}_t}{t \in I}$
		の$RCLL$な修正とすれば(usual条件よりTheorem 3.13を適用)
		\begin{align}
			0 = E\left[ \xi_a\left( A_a - B_a \right) \right]
			= E\left[ \sgn{A_a - B_a}\left( A_a - B_a \right) \right]
			= E\left| A_a - B_a \right|
		\end{align}
		が成り立ち,$a > 0$の任意性及び$A,B$のパスの右連続性より
		\begin{align}
			P\left[ \Set{A_t = B_t}{t \in I}\right] =
			\begin{cases}
				\displaystyle P\Biggl( \bigcap_{r \in (I \cap \Q) \cup \{\sup{}{I}\}}\{A_r = B_r\} \Biggr) = 1, 
					& (\sup{}{I} \in I), \\
				\displaystyle P\Biggl( \bigcap_{r \in I \cap \Q}\{A_r = B_r\} \Biggr) = 1, & (\sup{}{I} \notin I)
			\end{cases}
		\end{align}
		が出る.
		\QED
	\end{prf}
	
	\begin{itembox}[l]{Theorem 4.10 (Doob-Meyer Decomposition) 修正}
		Let $\{\mathscr{F}_t\}$ satisfy the usual conditions. If the right-continuous
		submartingale $X = \Set{X_t,\mathscr{F}_t}{0 \leq t < \infty}$ is of class $DL$, then
		\textcolor{red}{there exists a unique $[A]_{NAT}$ where $X - A'$ is right-continuous martingale
		for every $A' \in [A]_{NAT}$.}
		Further, if $X$ is of class $D$, then $M$ is a uniformly integrable martingale 
		and $A$ is integrable.	
	\end{itembox}
	
	\begin{prf}\mbox{}
		\begin{description}
			\item[第一段]
				$[A]_{NAT}$の一意性を示す.二つの右連続マルチンゲール$M,M'$とナチュラルな$A,A'$により
				\begin{align}
					X_t = M_t + A_t = M'_t + A'_t,
					\quad \forall t \geq 0
				\end{align}
				と書けるとき,
				\begin{align}
					B=\Set{B_t \coloneqq A_t - A'_t = M'_t - M_t,\mathscr{F}_t}{0 \leq t < \infty}
				\end{align}
				はLemmaの仮定を満たすマルチンゲールとなるから$[A]_{NAT} = [A']_{NAT}$が従う.
				
			\item[第二段]
				任意の区間$[0,a]$上で分解の存在を示せば
				$[0,\infty)$での分解が得られる.
				実際任意の$n \geq 1$に対し
				\begin{align}
					X_t = M^n_t + A^n_t, \quad (t \in [0,n])
				\end{align}
				と分解されるなら,$m > n$に対して
				\begin{align}
					M^n_t + A^n_t = X_t = M^m_t + A^m_t, \quad (t \in [0,n])
				\end{align}
				となり,Lemmaより或る$P$-零集合$E_{n,m}$が存在して,任意の$\omega \in \Omega \backslash E_{n,m}$で
				\begin{align}
					A^n_t(\omega) = A^m_t(\omega), \quad (\forall t \in [0,n])
				\end{align}
				が成立し,かつ$[0,n) \ni t \longmapsto A^n_t(\omega)$が右連続非減少となる.ここで
				\begin{align}
					E \coloneqq \bigcup_{\substack{n,m \in \N \\ n<m}} E_{n,m}
				\end{align}
				により$P$-零集合を定めれば,任意の$\omega \in \Omega \backslash E$及び$t \geq 0$に対して
				\begin{align}
					A^n_t(\omega) = A^m_t(\omega), \quad (\forall m > n > t)
				\end{align}
				となり$\lim_{n \to \infty} A^n_t(\omega)$が確定する.
				usual条件より$E \in \mathscr{F}_0$だから$A^n_t \defunc_{\Omega \backslash E}\ (n > t)$は
				$\mathscr{F}_t/\borel{\R}$-可測であり,
				\begin{align}
					A_t \coloneqq  \lim_{n \to \infty} A^n_t \defunc_{\Omega \backslash E},
					\quad (\forall t \geq 0)
				\end{align}
				で$A_t$を定めれば$A_t$は$\mathscr{F}_t/\borel{\R}$-可測となる.また
				任意の$n \geq 1$で
				\begin{align}
					A_t = A^n_t \defunc_{\Omega \backslash E}, \quad (\forall t \in [0,n))
				\end{align}
				が成り立つから$A_t$は可積分であり,$[0,\infty) \ni t \longmapsto A_t(\omega)$は右連続かつ非減少である.
				$\Set{\xi_t,\mathscr{F}_t}{0 \leq t < \infty}$を有界$RCLL$マルチンゲールとすれば
				任意の$t > 0$で
				\begin{align}
					E \int_{(0,t]} \xi_s\ dA_s = E \int_{(0,t]} \xi_s\ dA^n_s 
					= E \int_{(0,t]} \xi_{s-}\ dA^n_s = E \int_{(0,t]} \xi_{s-}\ dA_s,
					\quad (t < n)
				\end{align}
				が成立する.
				\begin{align}
					M \coloneqq X - A
				\end{align}
				とおけば$(M_t)_{t \geq 0}$は$(\mathscr{F}_t)$-適合かつ可積分であり,
				任意の$0 \leq s < t$及び$t < n$に対して
				\begin{align}
					M_t = X_t - A^n_t \defunc_{\Omega \backslash E} = M^n_t,
					\quad M_s = X_s - A^n_s \defunc_{\Omega \backslash E} = M^n_s,
					\quad \mbox{a.s. $P$}
				\end{align}
				となるから$\cexp{M_t}{\mathscr{F}_s} = M_s\ \mbox{a.s. $P$}$が満たされる.
				次段以降で$[0,a]$上で分解の存在を示す.
			
			\item[第三段]
				%\footnote{
				%	$X_\infty$が定義され
				%	$\Set{X_t,\mathscr{F}_t}{0 \leq t \leq \infty}$が劣マルチンゲールの場合に$a=\infty$とする.
				%}
				$\Set{Z_t,\mathscr{F}_t}{0 \leq t < \infty}$
				を$\Set{\cexp{X_a}{\mathscr{F}_t},\mathscr{F}_t}{0 \leq t < \infty}$の
				右連続な修正として(Theorem 3.13),
				\begin{align}
					Y_t \coloneqq X_t - Z_t,
					\quad (t \in [0,a])
				\end{align}
				により非正値の劣マルチンゲール$\Set{Y_t,\mathscr{F}_t}{0 \leq t \leq a}$を定め
				\begin{align}
					\Set{Y_{t^{(n)}_j},\mathscr{F}_{t^{(n)}_j}}{t^{(n)}_j = \frac{j}{2^n}a,\ j=0,1,\cdots,2^n},
					\quad n=1,2,\cdots,
					%\quad (\mbox{$a = \infty$の場合は$t^{(n)}_j = j/2^n$},\ j \in \N_0)
				\end{align}
				で離散化すれば,離散時のDoob分解 (P. \pageref{lem:Doob_decomposition})より
				\begin{align}
					&A^{(n)}_0 \coloneqq 0,
					\quad A^{(n)}_{t^{(n)}_j} \coloneqq \sum_{k=0}^{j-1} \cexp{Y_{t^{(n)}_{k+1}} - Y_{t^{(n)}_k}}{\mathscr{F}_{t^{(n)}_k}}; \\
					%\ A^{(n)}_\infty \coloneqq \sum_{k=0}^\infty \cexp{Y_{t^{(n)}_{k+1}} - Y_{t^{(n)}_k}}{\mathscr{F}_{t^{(n)}_k}}; \\
					&M^{(n)}_{t^{(n)}_j} \coloneqq Y_{t^{(n)}_j} - A^{(n)}_{t^{(n)}_j}
				\end{align}
				により可予測な増大過程$A^{(n)}$とマルチンゲール$M^{(n)}$に分解され,
				$Y_a = 0\ \mbox{a.s. $P$}$であるから
				\begin{align}
					Y_{t^{(n)}_j} = A^{(n)}_{t^{(n)}_j} +  M^{(n)}_{t^{(n)}_j}
					= A^{(n)}_{t^{(n)}_j} + \cexp{M^{(n)}_a}{\mathscr{F}_{t^{(n)}_j}}
					= A^{(n)}_{t^{(n)}_j} - \cexp{A^{(n)}_a}{\mathscr{F}_{t^{(n)}_j}},
					\quad \mbox{a.s. $P$},
					\quad j=0,1,\cdots,2^n
				\end{align}
				となる.

			\item[第四段]
				$(Y_T)_{T \in \mathscr{S}_a}$が一様可積分であることを示す.
				先ず任意の$T \in \mathscr{S}_a$に対し
				\begin{align}
					Z_T = \cexp{X_a}{\mathscr{F}_T},\quad \mbox{a.s. $P$}
					\label{eq:chapter_1_theorem_4_10_2}
				\end{align}
				が成立する.実際,任意抽出定理より
				\begin{align}
					\int_A Z_T\ dP = \int_A Z_a\ dP
					= \int_A X_a\ dP
					= \int_A \cexp{X_a}{\mathscr{F}_T}\ dP,
					\quad (\forall A \in \mathscr{F}_T)
				\end{align}
				が従い(\refeq{eq:chapter_1_theorem_4_10_2})が得られる.
				$\left(\cexp{X_a}{\mathscr{F}_T}\right)_{T \in \mathscr{S}_a}$は
				定理\ref{lem:uniformly_integrability_and_conditional_expectations}より一様可積分であるから
				$\left(Z_T\right)_{T \in \mathscr{S}_a}$も一様可積分であり,
				また$X$がクラス$DL$に属しているので$(Y_T)_{T \in \mathscr{S}_a}$の一様可積分性が従う.
				
			\item[第五段]
				$\left( A^{(n)}_a \right)_{n=1}^\infty$が一様可積分であることを示す.任意に$\lambda > 0$を取り
				\begin{align}
					T_\lambda^{(n)} \coloneqq
					a \wedge \min{}{\Set{t^{(n)}_{j-1}}{A^{(n)}_{t^{(n)}_j} > \lambda \mbox{ for some } j,\ 1 \leq j \leq 2^n}}
				\end{align}
				とおけば,$A^{(n)}$の可予測性より任意の$t \geq 0$で
				\begin{align}
					\left\{ T_\lambda^{(n)} \leq t \right\}
					= \bigcup_{j\, :\, t^{(n)}_{j-1} \leq t} \left\{ T_\lambda^{(n)} = t^{(n)}_{j-1} \right\}
					= \bigcup_{j\, :\, t^{(n)}_{j-1} \leq t}
						\left[ \bigcap_{k=1}^{j-1} \left\{ A^{(n)}_{t^{(n)}_k} \leq \lambda \right\} \right] 
						\cap \left\{ A^{(n)}_{t^{(n)}_j} > \lambda \right\}
					\in \mathscr{F}_t
				\end{align}
				が成り立つから$T_\lambda^{(n)} \in \mathscr{S}_a$が満たされ,また
				\begin{align}
					\mu < \lambda
					\quad \Longrightarrow \quad
					\left\{T^{(n)}_\lambda < a\right\} \subset \left\{T^{(n)}_\mu < a\right\}
					\label{eq:chapter_1_theorem_4_10_6}
				\end{align}
				及び
				\begin{align}
					T^{(n)}_\lambda(\omega) < a
					\quad \Longrightarrow \quad
					A^{(n)}_{T^{(n)}_\lambda}(\omega) \leq \lambda
					\label{eq:chapter_1_theorem_4_10_3}
				\end{align}
				も満たされる.
				\begin{align}
					N \coloneqq \bigcup_{k=1}^{2^n} \left\{ \cexp{Y_{t^{(n)}_k} - Y_{t^{(n)}_{k-1}}}{\mathscr{F}_{t^{(n)}_{k-1}}} < 0 \right\}
				\end{align}
				により$P$-零集合を定めれば,$\Omega \backslash N$の上で
				$A^{(n)}_0 \leq A^{(n)}_{t^{(n)}_1} \leq \cdots \leq A^{(n)}_a$となるから
				\begin{align}
					\left\{T^{(n)}_\lambda < a\right\} \cap (\Omega \backslash N)
					= \left\{A^{(n)}_a > \lambda\right\} \cap (\Omega \backslash N)
					\label{eq:chapter_1_theorem_4_10_1}
				\end{align}
				が従う.任意に$\Lambda \in \mathscr{F}_{T^{(n)}_\lambda}$を取れば,
				$\Lambda \cap \left\{T^{(n)}_\lambda=t^{(n)}_{j-1}\right\} \in \mathscr{F}_{t^{(n)}_{j-1}},
				\ (j=1,\cdots,2^n)$より
				\begin{align}
					\int_\Lambda Y_{T^{(n)}_\lambda}\ dP = 
					\sum_{j=1}^{2^n} \int_{\Lambda \cap \left\{T^{(n)}_\lambda=t^{(n)}_{j-1}\right\}} Y_{t^{(n)}_{j-1}}\ dP
					&= \sum_{j=1}^{2^n} \int_{\Lambda \cap \left\{T^{(n)}_\lambda=t^{(n)}_{j-1}\right\}} 
						A^{(n)}_{t^{(n)}_{j-1}} - \cexp{A^{(n)}_a}{\mathscr{F}_{t^{(n)}_{j-1}}}\ dP \\
					&= \sum_{j=1}^{2^n} \int_{\Lambda \cap \left\{T^{(n)}_\lambda=t^{(n)}_{j-1}\right\}} 
						A^{(n)}_{T^{(n)}_\lambda} - A^{(n)}_a\ dP \\
					&= \int_\Lambda A^{(n)}_{T^{(n)}_\lambda} - A^{(n)}_a\ dP
					\label{eq:chapter_1_theorem_4_10_5}
				\end{align}
				が成立するから,(\refeq{eq:chapter_1_theorem_4_10_3})と(\refeq{eq:chapter_1_theorem_4_10_1})と併せて
				\begin{align}
					\int_{\left\{A^{(n)}_a > \lambda\right\}} A^{(n)}_a\ dP
					= \int_{\left\{T^{(n)}_\lambda < a\right\}} A^{(n)}_{T^{(n)}_\lambda}\ dP
						- \int_{\left\{T^{(n)}_\lambda < a\right\}} Y_{T^{(n)}_\lambda}\ dP
					\leq \lambda P\left(T^{(n)}_\lambda < a\right) 
						- \int_{\left\{T^{(n)}_\lambda < a\right\}} Y_{T^{(n)}_\lambda}\ dP
				\end{align}
				となる.一方で(\refeq{eq:chapter_1_theorem_4_10_6}),(\refeq{eq:chapter_1_theorem_4_10_3}),
				(\refeq{eq:chapter_1_theorem_4_10_1}),(\refeq{eq:chapter_1_theorem_4_10_5})より
				\begin{align}
					\int_{\left\{T^{(n)}_{\lambda/2} < a\right\}} Y_{T^{(n)}_{\lambda/2}}\ dP
					&= \int_{\left\{T^{(n)}_{\lambda/2} < a\right\}} A^{(n)}_{T^{(n)}_{\lambda/2}} - A^{(n)}_a\ dP \\
					&\leq \int_{\left\{T^{(n)}_{\lambda} < a\right\}} A^{(n)}_{T^{(n)}_{\lambda/2}} - A^{(n)}_a\ dP \\
					&\leq -\frac{\lambda}{2} P\left(T^{(n)}_{\lambda} < a\right)
				\end{align}
				が成立するから
				\begin{align}
					\int_{\left\{A^{(n)}_a > \lambda\right\}} A^{(n)}_a\ dP
					\leq -2 \int_{\left\{T^{(n)}_{\lambda/2} < a\right\}} Y_{T^{(n)}_{\lambda/2}}\ dP
						- \int_{\left\{T^{(n)}_\lambda < a\right\}} Y_{T^{(n)}_\lambda}\ dP
				\end{align}
				となる.ここで
				\begin{align}
					P\left(T^{(n)}_{\lambda} < a\right)
					= P\left(A^{(n)}_a > \lambda\right)
					\leq \frac{E A^{(n)}_a}{\lambda}
					= \frac{- E M^{(n)}_a}{\lambda}
					= \frac{- E M^{(n)}_0}{\lambda}
					= \frac{- E Y_0}{\lambda}
				\end{align}
				より$P\left(T^{(n)}_{\lambda} < a\right)$は$\lambda$のみに依存して
				0に収束し,定理\ref{thm:appendix_uniform_integrability_equivalence}と$(Y_T)_{T \in \mathscr{S}_a}$の
				一様可積分性により
				\begin{align}
					\sup{n \in \N}{\int_{\left\{A^{(n)}_a > \lambda\right\}} A^{(n)}_a\ dP}
					\leq 2 \sup{n \in \N}{\int_{\left\{T^{(n)}_{\lambda/2} < a\right\}} \left|Y_{T^{(n)}_{\lambda/2}}\right|\ dP}
					+ \sup{n \in \N}{\int_{\left\{T^{(n)}_{\lambda} < a\right\}} \left|Y_{T^{(n)}_{\lambda}}\right|\ dP}
					\longrightarrow 0
					\quad (\lambda \longrightarrow \infty)
				\end{align}
				が従い$\left( A^{(n)}_a \right)_{n=1}^\infty$が一様可積分性が出る.
				
			\item[第六段]
				Dunford-Pettisの定理より$\left( A^{(n)}_a \right)_{n=1}^\infty$の或る部分列
				$\left( A^{(n_k)}_a \right)_{k=1}^\infty$は$L^1(P)$で弱収束する.つまり
				或る$A_a \in L^1(P)$が存在して
				任意の$\xi \in L^\infty(P)$に対し
				\begin{align}
					E \left( \xi A^{(n_k)}_a \right) \longrightarrow E (\xi A_a)
					\quad (k \longrightarrow \infty)
				\end{align}
				が成立する.
				\begin{align}
					\Pi_n \coloneqq \Set{t^{(n)}_j}{t^{(n)}_j = \frac{j}{2^n}a,\ j=0,1,\cdots,2^n},
					\quad \Pi \coloneqq \bigcup_{n=1}^\infty \Pi_n
				\end{align}
				とすれば,任意の$t \in \Pi$に対し或る$K \geq 1$が存在して
				$t \in \Pi_{n_k}\ (\forall k > K)$となり,Problem 4.11より
				\begin{align}
					E \left( \xi A^{(n_k)}_t \right)
					= E \xi\left\{ Y_t + \cexp{A^{(n_k)}_a}{\mathscr{F}_t} \right\}
					\longrightarrow E \xi\left\{ Y_t + \cexp{A_a}{\mathscr{F}_t} \right\}
					\quad (k > K,\ k \longrightarrow \infty)
					\label{eq:chapter_1_theorem_4_10_7}
				\end{align}
				が成り立つから$A^{(n_k)}_t$は$Y_t + \cexp{A_a}{\mathscr{F}_t}$に弱収束する.
				ここで
				\begin{align}
					\tilde{A}_t \coloneqq Y_t + \cexp{A_a}{\mathscr{F}_t},
					\quad (t \in [0,a])
				\end{align}
				と定めれば$\Set{\tilde{A}_t,\mathscr{F}_t}{0 \leq t \leq a}$は
				劣マルチンゲールとなり,$\Set{X_t,\mathscr{F}_t}{0 \leq t <\infty}$の右連続性より
				\begin{align}
					[0,a] \ni t \longmapsto E\left[ Y_t + \cexp{A_a}{\mathscr{F}_t} \right]
					= E X_t - E X_a + E A_a
				\end{align}
				は右連続であるから(Theorem 3.13),$\tilde{A}$の右連続な修正$\Set{A_t,\mathscr{F}_t}{0 \leq t \leq a}$
				が得られる.
			
			\item[第七段]
				$t \longmapsto A_t(\omega)$がa.s.に0出発かつ非減少であることを示す.
				実際,$\xi = \sgn{A_0}$として,(\refeq{eq:chapter_1_theorem_4_10_7})より
				\begin{align}
					E |A_0| = E \xi A_0 = E \xi \tilde{A}_0 = \lim_{k \to \infty} E \xi A^{(n_k)}_0 = 0
				\end{align}
				が成り立つから$A_0 = 0\ \mbox{a.s. $P$}$が従う.また任意に$s,t \in \Pi,\ (s<t)$を取れば
				或る$K \geq 1$が存在して$s,t \in \Pi_{n_k}\ (\forall k > K)$が満たされ,
				$A^{(n_k)}$は増大過程であるから$\xi = \defunc_{\{A_s > A_t\}}$として
				\begin{align}
					E \xi (A_t - A_s) = E \xi \left( \tilde{A}_t - \tilde{A}_s \right)
					= \lim_{k \to \infty} E \xi \left( A^{(n_k)}_t - A^{(n_k)}_s \right) \geq 0 
				\end{align}
				となり$P(A_s > A_t) = 0$が成り立つ.$t \longmapsto A_t$が右連続性であるから,$P$-零集合を
				\begin{align}
					N \coloneqq \Biggl(\bigcup_{\substack{s,t \in \Pi \\ s < t}} \{A_s > A_t\}\Biggr) \cup \{A_0 \neq 0\}
				\end{align}
				で定めれば$\Omega \backslash N$上で$t \longmapsto A_t$は0出発非減少となり,
				$N$上で$A \equiv 0$と修正すれば$A$は増大過程となる.
				
			\item[第八段]
				$A$がナチュラルであることを示す.$\xi = \Set{\xi_t,\mathscr{F}_t}{0 \leq t \leq a}$を有界な$RCLL$マルチンゲールとすれば
				\begin{align}
					E \xi_a A^{(n_k)}_a 
					&= E\left[ \sum_{j=1}^{2^{n_k}}\xi_{t^{(n_k)}_{j-1}} \left( A^{(n_k)}_{t^{(n_k)}_j} - A^{(n_k)}_{t^{(n_k)}_{j-1}} \right) \right] \\
					&= E\left[ \sum_{j=1}^{2^{n_k}}\xi_{t^{(n_k)}_{j-1}} \left( Y_{t^{(n_k)}_j} - Y_{t^{(n_k)}_{j-1}} \right) \right]
						+ E\left[ \sum_{j=1}^{2^{n_k}}\xi_{t^{(n_k)}_{j-1}} \left( \cexp{A^{(n_k)}_a}{\mathscr{F}_{t^{(n_k)}_j}} - \cexp{A^{(n_k)}_a}{\mathscr{F}_{t^{(n_k)}_{j-1}}} \right) \right] \\
					&= E\left[ \sum_{j=1}^{2^{n_k}}\xi_{t^{(n_k)}_{j-1}} \left( A_{t^{(n_k)}_j} - A_{t^{(n_k)}_{j-1}} \right) \right]
				\end{align}
				が任意の$k \geq 1$で成り立ち(Proposition 4.3),$k \longrightarrow \infty$として
				\begin{align}
					E \xi_a A_a = E \int_{(0,a]} \xi_{s-}\ dA_s
				\end{align}
				が得られる.任意の$t \in (0,a]$に対し
				$\xi^{(t)} = \Set{\xi^{(t)}_s \coloneqq \xi_{t \wedge s},\mathscr{F}_s}{0 \leq s \leq a}$
				も$RCLL$マルチンゲールであり
				\begin{align}
					\xi^{(t)}_{s-} &= \xi_{s-},\quad (\forall s \in (0,t]), \\
					\xi^{(t)}_{s-} &= \xi_t, \quad (\forall s \in (t,a])
				\end{align}
				より
				\begin{align}
					E \xi_t A_t + E \xi_t(A_a - A_t) = E \xi^{(t)}_a A_a 
					= E \int_{(0,a]} \xi^{(t)}_{s-}\ dA_s
					= E \int_{(0,t]} \xi_{s-}\ dA_s + E \xi_t (A_a - A_t)
				\end{align}
				となり
				\begin{align}
					E \xi_t A_t = E \int_{(0,t]} \xi_{s-}\ dA_s,
					\quad (\forall t \in (0,a])
				\end{align}
				が成立する.よって$A$はナチュラルである.
				
			\item[第九段]
				$\Set{M_t \coloneqq X_t - A_t, \mathscr{F}_t}{0 \leq t \leq a}$がマルチンゲールであることを示す.
				$M$の適合性と可積分性は$X,A$のそれより従い,また任意に
				$0 \leq s \leq t \leq a$を取れば,任意の$A \in \mathscr{F}_s$で
				\begin{align}
					\int_A M_s\ dP = \int_A X_s - A_s\ dP
					&= \int_A X_s - \left(Y_s - \cexp{A_a}{\mathscr{F}_s}\right)\ dP \\
					&= \int_A X_s - \left(X_s - Z_s - \cexp{A_a}{\mathscr{F}_s}\right)\ dP \\
					&= \int_A Z_t + \cexp{A_a}{\mathscr{F}_t}\ dP \\
					&= \int_A X_t - \left(X_t - Z_t - \cexp{A_a}{\mathscr{F}_t}\right)\ dP \\
					&= \int_A M_t\ dP
				\end{align}
				が成立する.
				\QED
		\end{description}
	\end{prf}
	
	\begin{itembox}[l]{Problem 4.13}
		Verify that a continuous, nonnegative submartingale is regular. 
	\end{itembox}
	
	\begin{prf}
		Problem 4.9 より$(X_{T_n})_{n=1}^\infty$は一様可積分であり,またパスの連続性より
		$X_{T_n} \longrightarrow X_T\ (n \longrightarrow \infty)$
		となるから,定理\ref{lem:uniformly_integrable_and_convergence_in_mean}より
		$\lim_{n \to \infty} EX_{T_n} = EX_T$が成立する.
		\QED
	\end{prf}
	
	\begin{itembox}[l]{Theorem 4.14 修正}
		Suppose that $X = \Set{X_t}{0 \leq t < \infty}$ is a right-continuous submartingale
		of class $DL$ with respect to the filtration $\{\mathscr{F}_t\}$, which
		satisfies the usual conitions, and 
		\textcolor{red}{let $[A]_{NAT}$ be of the Doob-Meyer decomposition of $X$.
		There exists a continuous version of $A$ in $[A]_{NAT}$ if and only if $X$ is regular.}
	\end{itembox}
	
	\begin{prf}\mbox{}
		\begin{description}
			\item[第一段] $A$が連続であるとき,
				増大列$\{T_n\}_{n=1}^\infty \subset \mathscr{S}_a$と
				$T \coloneqq \lim_{n \to \infty} \in T_n \mathscr{S}_a$に対し
				単調収束定理より
				\begin{align}
					\lim_{n \to \infty} EA_{T_n}
					= E \lim_{n \to \infty} A_{T_n}
					= EA_T
				\end{align}
				が成立する.また任意抽出定理より
				\begin{align}
					E(X_{T_n} - A_{T_n}) = E(X_{T} - A_{T}),
					\quad (\forall n \geq 1)
				\end{align}
				となるから
				\begin{align}
					\lim_{n \to \infty} EX_{T_n} 
					= \lim_{n \to \infty} E(X_{T_n} - A_{T_n}) + \lim_{n \to \infty} EA_{T_n} 
					= E(X_{T} - A_{T}) + EA_T
					= EX_T
				\end{align}
				が従う.
				
			\item[第二段]
				以降$X$がレギュラーであるとする.このとき任意の有界な停止時刻の増大列
				$(T_n)$と$T \coloneqq \lim T_n$に対し,$X-A$のマルチンゲール性と任意抽出定理,
				及び$X$のレギュラリティより
				\begin{align}
					EA_{T_n} &= EX_{T_n} - E(X_{T_n} - A_{T_n})
					= EX_{T_n} - E(X_T - A_T) \\
					&\qquad \longrightarrow EX_T - E(X_T - A_T)
					= EA_T
					\quad (n \longrightarrow \infty)
					\label{eq:chapter_1_theorem_4_14_1}
				\end{align}
				が得られる.いま,任意に$a \in \N$を取り
				\begin{align}
					\Pi_n \coloneqq 
					\Set{t^{(n)}_j}{t^{(n)}_j = \frac{j}{2^n}a,\ j=0,1,\cdots,2^n},
					\quad \Pi \coloneqq \bigcup_{n=1}^\infty \Pi_n
				\end{align}
				とおく.また任意に$\lambda \in \N$を取り,各$j = 0,1,\cdots,2^n$に対し
				\begin{align}
					Y^{(n),j}_t \coloneqq
					\cexp{\lambda \wedge A_{t^{(n)}_{j+1}}}{\mathscr{F}_t},
					\quad (\forall t \geq 0)
				\end{align}
				によりマルチンゲール$\Set{Y^{(n),j}_t,\mathscr{F}_t}{0 \leq t < \infty}$を定めれば,
				\begin{align}
					[0,\infty) \ni t \longmapsto EY^{(n),j}_t 
					= E\left(\lambda \wedge A_{t^{(n)}_{j+1}}\right)
				\end{align}
				と Theorem 3.13 より$RCLL$な修正$\tilde{Y}^{(n),j}$が存在する.このとき
				各$t \geq 0$で
				\begin{align}
					\int_A \tilde{Y}^{(n),j}_t\ dP 
					= \int_A \lambda \wedge A_{t^{(n)}_{j+1}}\ dP
					\leq \lambda P(A),
					\quad (\forall A \in \mathscr{F}_t)
				\end{align}
				となり,一方で各$t \in \left[t^{(n)}_j, t^{(n)}_{j+1} \right)$で
				\begin{align}
					\int_A \tilde{Y}^{(n),j}_t\ dP
					= \int_A \lambda \wedge A_{t^{(n)}_{j+1}}\ dP
					\geq \int_A \lambda \wedge A_t\ dP,
					\quad (\forall A \in \mathscr{F}_t) 
				\end{align}
				となるから,各$j$で
				\begin{align}
					E_j &\coloneqq \Set{\tilde{Y}^{(n),j}_t > \lambda}{\exists t \geq 0} 
						\cup \Set{\tilde{Y}^{(n),j}_t < \lambda \wedge A_t}{\exists t \in \left[t^{(n)}_j, t^{(n)}_{j+1} \right)} \\
					&= \left[ \bigcup_{r \in [0,\infty)\cap\Q}\left\{\tilde{Y}^{(n),j}_r > \lambda\right\} \right]
					\bigcup \left[ \bigcup_{r \in \left[t^{(n)}_j, t^{(n)}_{j+1} \right)\cap\Q}\left\{\tilde{Y}^{(n),j}_r < \lambda \wedge A_r\right\} \right]
				\end{align}
				とおけば$P$-零集合$E \coloneqq \bigcup_{j=0}^{2^n} E_j$が定まる.usual条件より$E \in \mathscr{F}_0$であるから
				\begin{align}
					\Set{Z^{(n),j}_t \coloneqq \tilde{Y}^{(n),j}_t \defunc_{\Omega \backslash E},
					\mathscr{F}_t}{0 \leq t < \infty}
				\end{align}
				で定める$Y^{(n),j}$のバージョン$Z^{(n),j}$は
				\begin{align}
					\omega \in \Omega \backslash E
					\quad \Longrightarrow \quad
					\begin{cases}
						Z^{(n),j}_t(\omega) \leq \lambda, & \forall t \geq 0, \\
						Z^{(n),j}_t(\omega) \geq \lambda \wedge A_t(\omega), & \forall t \in \left[t^{(n)}_j, t^{(n)}_{j+1} \right)
					\end{cases}
				\end{align}
				を満たす$RCLL$かつ有界なマルチンゲールとなり,
				\begin{align}
					\eta^{(n)}_t \coloneqq
					\sum_{j=0}^{2^n-1} Z^{(n),j}_t \defunc_{\left[t^{(n)}_j,t^{(n)}_{j+1}\right)}(t)
						+ (\lambda \wedge A_a) \defunc_{[a,\infty)}(t),
					\quad (t \geq 0)
				\end{align}
				とおけば
				\begin{align}
					\omega \in \Omega \backslash E
					\quad \Longrightarrow \quad
					\begin{cases}
						\eta^{(n)}_t(\omega) \leq \lambda, & (\forall t \geq 0), \\
						\eta^{(n)}_t(\omega) \geq \lambda \wedge A_t(\omega), & (\forall t \in [0,a])
					\end{cases}
					\label{eq:chapter_1_theorem_4_14_4}
				\end{align}
				が成り立つ.また$\eta^{(n)}$の右連続性,Corollary2.4,Problem2.5 及びusual条件より
				\begin{align}
					T^{(n)}_\epsilon \coloneqq
					a \wedge \inf{}{\Set{t \geq 0}{\eta^{(n)}_t - (\lambda \wedge A_t)  > \epsilon}}
				\end{align}
				は$\mathscr{S}_a$に属する停止時刻となり,このとき
				\begin{align}
					\varphi_n(t) \coloneqq 
					\begin{cases}
						t^{(n)}_{j+1}, & t^{(n)}_j \leq t < t^{(n)}_{j+1},\ j=0,1,\cdots,2^n-1 \\
						a, & t = a
					\end{cases}
				\end{align}
				を用いれば,任意抽出定理より
				\begin{align}
					E\left( \eta_{T^{(n)}_\epsilon} \right)
					&= \sum_{j=0}^{2^n-1} \int_{\left\{t^{(n)}_j \leq T^{(n)}_\epsilon < t^{(n)}_{j+1}\right\}} Z^{(n),j}_{T^{(n)}_\epsilon}\ dP
						+ \int_{\left\{T^{(n)}_\epsilon = a\right\}} \lambda \wedge A_a\ dP \\
					&= \sum_{j=0}^{2^n-1} \int_{\left\{t^{(n)}_j \leq T^{(n)}_\epsilon < t^{(n)}_{j+1}\right\}} \cexp{Z^{(n),j}_{t^{(n)}_{j+1}}}{\mathscr{F}_{T^{(n)}_\epsilon}}\ dP
						+ \int_{\left\{T^{(n)}_\epsilon = a\right\}} \lambda \wedge A_a\ dP \\
					&= \sum_{j=0}^{2^n-1} \int_{\left\{t^{(n)}_j \leq T^{(n)}_\epsilon < t^{(n)}_{j+1}\right\}} Z^{(n),j}_{t^{(n)}_{j+1}}\ dP
						+ \int_{\left\{T^{(n)}_\epsilon = a\right\}} \lambda \wedge A_a\ dP \\
					&= \sum_{j=0}^{2^n-1} \int_{\left\{t^{(n)}_j \leq T^{(n)}_\epsilon < t^{(n)}_{j+1}\right\}} \lambda \wedge A_{t^{(n)}_{j+1}}\ dP
						+ \int_{\left\{T^{(n)}_\epsilon = a\right\}} \lambda \wedge A_a\ dP \\
					&= \sum_{j=0}^{2^n-1} \int_{\left\{t^{(n)}_j \leq T^{(n)}_\epsilon < t^{(n)}_{j+1}\right\}} \lambda \wedge A_{\varphi_n\left(T^{(n)}_\epsilon\right)}\ dP
						+ \int_{\left\{T^{(n)}_\epsilon = a\right\}} \lambda \wedge A_{\varphi_n\left(T^{(n)}_\epsilon\right)}\ dP \\
					&= E\left(\lambda \wedge A_{\varphi_n\left(T^{(n)}_\epsilon\right)}\right)
				\end{align}
				が従う.また$t \longmapsto \eta^{(n)}_t - (\lambda \wedge A_t)$の右連続性より
				\begin{align}
					T^{(n)}_\epsilon(\omega) < a
					\quad \Longrightarrow
					\quad \eta^{(n)}_{T^{(n)}_\epsilon}(\omega) - \left(\lambda \wedge A_{T^{(n)}_\epsilon}(\omega)\right)
						\geq \epsilon
				\end{align}
				となるから
				\begin{align}
					E\left(\lambda \wedge A_{\varphi_n\left(T^{(n)}_\epsilon\right)}
						- \lambda \wedge A_{T^{(n)}_\epsilon} \right)
					&= E\left(\eta^{(n)}_{T^{(n)}_\epsilon}
						- \lambda \wedge A_{T^{(n)}_\epsilon} \right) \\
					&= E\defunc_{\left\{T^{(n)}_\epsilon < a\right\}}\left(\eta^{(n)}_{T^{(n)}_\epsilon}
						- \lambda \wedge A_{T^{(n)}_\epsilon} \right)
					\geq \epsilon P\left(T^{(n)}_\epsilon < a\right)
					\label{eq:chapter_1_theorem_4_14_5}
				\end{align}
				が成立する.
				
			\item[第三段]
				$\left( \eta^{(n)} \right)_{n=1}^\infty$は$n$に関して
				$P$-a.s. に減少していく.実際,任意の$t \in [0,a)$に対し
				\begin{align}
					t \in \left[t^{(n)}_j, t^{(n)}_{j+1}\right)
				\end{align}
				を満たす$0 \leq j \leq 2^n-1$を取れば
				$t \in \left[t^{(n+1)}_{2j}, t^{(n+1)}_{2j+1}\right)$或は
				$t \in \left[t^{(n+1)}_{2j+1}, t^{(n+1)}_{2j+2}\right)$となるから,
				任意の$A \in \mathscr{F}_t$で
				\begin{align}
					\int_A \eta^{(n)}_t\ dP
					= \int_A \lambda \wedge A_{t^{(n)}_{j+1}}\ dP
					\begin{cases}
						\displaystyle= \int_A \lambda \wedge A_{t^{(n+1)}_{2j+2}}\ dP \\
						\displaystyle\geq \int_A \lambda \wedge A_{t^{(n+1)}_{2j+1}}\ dP
					\end{cases}
					= \int_A \eta^{(n+1)}_t\ dP
				\end{align}
				が成り立ち$\eta^{(n)}_t \geq \eta^{(n+1)}_t,\ \mbox{a.s. $P$}$が従う.
				$\eta^{(n)},\eta^{(n+1)}$のパスは右連続であるから
				\begin{align}
					F_n \coloneqq \Set{\eta^{(n)}_t < \eta^{(n+1)}_t}{\exists t \in [0,a)}
					= \bigcup_{r \in [0,a) \cap \Q} \left\{\eta^{(n)}_r < \eta^{(n+1)}_r\right\}
				\end{align}
				で$P$-零集合が定まり,$F \coloneqq \bigcup_{n=1}^\infty F_n$とおけば
				任意の$\omega \in \Omega \backslash F$と$t \in [0,a]$で
				$\left( \eta^{(n)}_t(\omega) \right)_{n=1}^\infty$は減少し
				\begin{align}
					T^{(1)}_\epsilon \defunc_{\Omega \backslash F} 
					\leq T^{(2)}_\epsilon \defunc_{\Omega \backslash F} \leq \cdots \leq a
					\label{eq:chapter_1_theorem_4_14_2}
				\end{align}
				となる.usual条件より$F \in \mathscr{F}_0$であるから
				\begin{align}
					\left\{ T^{(n)}_\epsilon \defunc_{\Omega \backslash F} \leq t \right\}
					= \left\{ T^{(n)}_\epsilon \leq t \right\} \cap (\Omega \backslash F) + F
					\in \mathscr{F}_t,\quad (\forall t \geq 0)
				\end{align}
				が成り立つので$T^{(n)}_\epsilon \defunc_{\Omega \backslash F} \in \mathscr{S}_a$となり,
				単調増大性より
				\begin{align}
					T_\epsilon \coloneqq \lim_{n \to \infty} T^{(n)}_\epsilon \defunc_{\Omega \backslash F}
				\end{align}
				と定めれば$T_\epsilon \in \mathscr{S}_a$も満たされる.
				一方$\varphi_n\left(T^{(n)}_\epsilon\right)$についても
				\begin{align}
					\left\{\varphi_n\left(T^{(n)}_\epsilon\right) \leq t\right\}
					= \bigcup_{j\, :\, t^{(n)}_{j+1} \leq t} \left\{t^{(n)}_j \leq T^{(n)}_\epsilon < t^{(n)}_{j+1}\right\}
					\in \mathscr{F}_t,
					\quad (\forall t \geq 0)
				\end{align}
				より$\varphi_n\left(T^{(n)}_\epsilon\right) \in \mathscr{S}_a$が従い,
				また$\varphi_n(t) \geq t$と$t \longmapsto \varphi_n(t)$の増大性より
				\begin{align}
					T^{(n)}_\epsilon(\omega)
					\leq \varphi_n\left(T^{(n)}_\epsilon(\omega)\right)
					\leq \varphi_n\left(T_\epsilon(\omega)\right),
					\quad (\forall \omega \in \Omega \backslash F)
				\end{align}
				が成立し,$A$のパスの増大性と併せて
				\begin{align}
					E \left( \lambda \wedge A_{T^{(n)}_\epsilon} \right)
					\leq E \left( \lambda \wedge A_{\varphi_n\left(T^{(n)}_\epsilon\right)} \right)
					\leq E \left( \lambda \wedge A_{\varphi_n\left(T_\epsilon\right)} \right)
				\end{align}
				が満たされる.このとき(\refeq{eq:chapter_1_theorem_4_14_1})より
				\begin{align}
					\lim_{n \to \infty} E \left( \lambda \wedge A_{T^{(n)}_\epsilon} \right)
					= E \left( \lambda \wedge A_{T_\epsilon} \right)
				\end{align}
				が成り立ち,右辺も$\varphi_n(t) \downarrow t$と$A$のパスの右連続性
				及びLebesgueの収束定理より$E \left( \lambda \wedge A_{T_\epsilon} \right)$に収束するから
				\begin{align}
					\lim_{n \to \infty} E \left( \lambda \wedge A_{\varphi_n\left(T^{(n)}_\epsilon\right)} \right) 
					= E \left( \lambda \wedge A_{T_\epsilon} \right)
				\end{align}
				が得られる.
			
			\item[第五段]
				任意の$\omega \in \Omega$と$n \geq 1$に対し
				\begin{align}
					T^{(n)}_\epsilon(\omega) < a
					\quad \Longleftrightarrow \quad
					\sup{0 \leq t \leq a}{\left\{(\lambda \wedge A_t(\omega)) - \eta^{(n)}_t(\omega)\right\}} > \epsilon
				\end{align}
				が満たされ,また(\refeq{eq:chapter_1_theorem_4_14_4})より
				$\Omega \backslash E$の上で$\eta^{(n)}_t - (\lambda \wedge A_t) \geq 0,\ (\forall t \in [0,a])$だから,
				(\refeq{eq:chapter_1_theorem_4_14_5})と前段の結果と併せて
				\begin{align}
					&P\left(\sup{0 \leq t \leq a}{\left|\eta^{(n)}_t - (\lambda \wedge A_t)\right|} > \epsilon\right)
					= P\left(T^{(n)}_\epsilon < a\right) \\
					&\qquad \leq \frac{1}{\epsilon} E\left(\lambda \wedge A_{\varphi_n\left(T^{(n)}_\epsilon\right)} 
						- \lambda \wedge A_{T^{(n)}_\epsilon} \right)
					\longrightarrow \frac{1}{\epsilon} E\left(\lambda \wedge A_{T_\epsilon}
						- \lambda \wedge A_{T_\epsilon} \right) = 0 \quad (n \longrightarrow \infty)
				\end{align}
				が得られる.従って定理\ref{thm:convergence_in_measure_then_convergence_almost_everywhere}より
				或る部分列$(n_k)_{k=1}^\infty$と$P$-零集合$G$が存在して
				\begin{align}
					\sup{0 \leq t \leq a}{\left|\eta^{(n_k)}_t(\omega) - (\lambda \wedge A_t(\omega))\right|}
					\longrightarrow 0
					\quad (k \longrightarrow \infty),
					\quad (\forall \omega \in \Omega \backslash G)
					\label{eq:chapter_1_theorem_4_14_6}
				\end{align}
				が成立する.
				
			\item[第六段]
				$A$はナチュラルであり,$Z^{(n),j}$は有界かつ$RCLL$なマルチンゲールであるから
				\begin{align}
					E\int_{\left(t^{(n)}_j,t^{(n)}_{j+1}\right]} Z^{(n),j}_s\ dA_s
					&= E\int_{\left(0,t^{(n)}_{j+1}\right]} Z^{(n),j}_s\ dA_s
						- E\int_{\left(0,t^{(n)}_j\right]} Z^{(n),j}_s\ dA_s \\
					&= E\int_{\left(0,t^{(n)}_{j+1}\right]} Z^{(n),j}_{s-}\ dA_s
						- E\int_{\left(0,t^{(n)}_j\right]} Z^{(n),j}_{s-}\ dA_s \\
					&= E\int_{\left(t^{(n)}_j,t^{(n)}_{j+1}\right]} Z^{(n),j}_{s-}\ dA_s
				\end{align}
				が成立する.従って
				\begin{align}
					\xi^{(n)}_t \coloneqq
					\sum_{j=0}^{2^n-1} Z^{(n),j}_t \defunc_{\left(t^{(n)}_j,t^{(n)}_{j+1}\right]}(t),
					\quad (t \geq 0)
				\end{align}
				とおけば任意の$t \in (0,a]$で$\xi^{(n)}_{t-}$が存在し
				\begin{align}
					E\int_{(0,a]} \xi^{(n)}_s\ dA_s
					= \sum_{j=0}^{2^n-1} E\int_{\left(t^{(n)}_j,t^{(n)}_{j+1}\right]} Z^{(n),j}_s\ dA_s
					= \sum_{j=0}^{2^n-1} E\int_{\left(t^{(n)}_j,t^{(n)}_{j+1}\right]} Z^{(n),j}_{s-}\ dA_s
					= E\int_{(0,a]} \xi^{(n)}_{s-}\ dA_s
				\end{align}
				が成立する.一方で$t \notin \Pi$で$\xi^{(n)}_t = \eta^{(n)}_t,\ (\forall n \geq 1)$
				であるから(\refeq{eq:chapter_1_theorem_4_14_6})より
				\begin{align}
					\sup{t \in (0,a]\backslash\Pi}{\left|\xi^{(n_k)}_t(\omega) - \lambda \wedge A_t(\omega)\right|}
					\longrightarrow 0 \quad (k \longrightarrow \infty),
					\quad (\forall \omega \in \Omega \backslash G)
				\end{align}
				が従い,これにより
				\begin{align}
					\sup{t \in (0,a]}{\left|\xi^{(n_k)}_{t-}(\omega) - \lambda \wedge A_{t-}(\omega)\right|}
					\longrightarrow 0 \quad (k \longrightarrow \infty),
					\quad (\forall \omega \in \Omega \backslash G)
				\end{align}
				も出る.実際,$\omega \in \Omega \backslash G$を固定すれば,
				任意の$\epsilon > 0$に対し或る$K = K(\omega,\epsilon) \geq 1$が存在して
				\begin{align}
					\sup{t \in (0,a]\backslash\Pi}{\left|\xi^{(n_k)}_t(\omega) - \lambda \wedge A_t(\omega)\right|} 
					< \epsilon,\quad (\forall k \geq K)
				\end{align}
				となり,このとき任意の$t \in (0,a]$と$k \geq K$で
				\begin{align}
					&\left|\xi^{(n_k)}_{t-}(\omega) - \lambda \wedge A_{t-}(\omega)\right| \\
					&\quad \leq \left|\xi^{(n_k)}_{t-}(\omega) - \xi^{(n_k)}_{s}(\omega)\right|
						+ \left|\xi^{(n_k)}_{s}(\omega) - \lambda \wedge A_{s}(\omega)\right|
						+ \left|\lambda \wedge A_{s}(\omega) - \lambda \wedge A_{t-}(\omega)\right| \\
					&\quad < \epsilon
				\end{align}
				を満たす$s = s(t,k) \in (0,a]\backslash\Pi,\ (s < t)$が取れるから
				\begin{align}
					\sup{t \in (0,a]}{\left|\xi^{(n_k)}_{t-}(\omega) - \lambda \wedge A_{t-}(\omega)\right|} 
					\leq \epsilon,\quad (\forall k \geq K)
				\end{align}
				が成立する.$t \in \Pi$なら或る$N = N(t)$で$t \in \Pi_N$となるから
				$\xi^{(n)}_t = \lambda \wedge A_t,\ \mbox{$P$-a.s.},\ (\forall n \geq N)$となり
				\begin{align}
					H_t \coloneqq \bigcup_{n \geq N} \left\{\xi^{(n)}_t \neq \lambda \wedge A_t\right\},
					\quad H \coloneqq \bigcup_{t \in \Pi} H_t
				\end{align}
				により$P$-零集合$H$を定めれば任意の$t \in [0,a]$で
				\begin{align}
					\lim_{k \to \infty} \xi^{(n_k)}_t(\omega) = \lambda \wedge A_t(\omega),
					\quad (\forall \omega \in \Omega \backslash (G \cup H))
				\end{align}
				となる.Lebesgueの収束定理より
				\begin{align}
					E\int_{(0,a]} \lambda \wedge A_t\ dA_t
					= E\int_{(0,a]} \lambda \wedge A_{t-}\ dA_t
				\end{align}
				が得られ,$A$の単調非減少性より$A_{t-} \leq A_t$であるから
				或る$P$-零集合$U_a$が存在し,任意の$\omega \in \Omega \backslash U_a$で
				\begin{align}
					\int_{(0,a]} (\lambda \wedge A_t(\omega)) 
					- (\lambda \wedge A_{t-}(\omega))\ dA_t(\omega) = 0
				\end{align}
				が成立し$(0,a] \ni t \longmapsto \lambda \wedge A_t(\omega)$
				の連続性が出る.$a$の任意性より
				$V_\lambda \coloneqq \bigcup_{a=1}^\infty U_a$とおけば
				\begin{align}
					(0,\infty) \ni t \longmapsto \lambda \wedge A_t(\omega),
					 \quad (\forall \omega \in \Omega \backslash V_\lambda)
				\end{align}
				は連続となり,$\lambda$も任意であるから
				$V \coloneqq \bigcup_{\lambda=1}^\infty V_\lambda$として
				\begin{align}
					(0,\infty) \ni t \longmapsto A_t(\omega),
					\quad (\forall \omega \in \Omega \backslash V)
				\end{align}
				は連続となる.$\tilde{A} \coloneqq A \defunc_{\Omega \backslash V} \in [A]_{NAT}$
				が求める$A$のバージョンである.
				\QED
		\end{description}
	\end{prf}
	
	\begin{itembox}[l]{Problem 4.15}
		Let $X = \Set{X_t,\mathscr{F}_t}{0 \leq t < \infty}$ be a continuous, nonnegative process
		with $X_0 = 0$ a.s., and $A = \Set{A_t,\mathscr{F}_t}{0 \leq t < \infty}$ any continuous,
		increasing process for which
		\begin{align}
			E(X_T) \leq E(A_T)
		\end{align}
		holds for every bounded stopping time $T$ of $\{\mathscr{F}_t\}$. Introduce the process
		$V_t \coloneqq \max{0 \leq s \leq t}{X_s}$, consider a continuous, increasing function $F$
		on $[0,\infty)$ with $F(0) = 0$, and define \textcolor{red}{$G(x) \coloneqq 2F(x) + x\int_{(x,\infty)} u^{-1}\ dF(u);
		\ 0 < x < \infty$.} Establish the inequalities
		\begin{description}
			\item[(4.14)] $\displaystyle P[V_T \geq \epsilon] \leq \frac{E(A_T)}{\epsilon};\quad \forall \epsilon > 0$
			\textcolor{red}{\item[(4.15) (Lenglart inequality)] $\displaystyle P[V_T \geq \epsilon] 
				\leq \frac{E(\delta \wedge A_T)}{\epsilon} + P[A_T \geq \delta];\quad \forall \epsilon > 0,\ \delta > 0$}
			\item[(4.16)] $EF(V_T) \leq EG(A_T)$
		\end{description}
		for any stopping time $T$ of $\{\mathscr{F}_t\}$.
	\end{itembox}
	
	\begin{prf}\mbox{}
		\begin{description}
			\item[(1)] $X$のパスの連続性とProblem 2.7より
				\begin{align}
					H_\epsilon \coloneqq \inf{}{\Set{t \geq 0}{X_t \geq \epsilon}}
				\end{align}
				で$(\mathscr{F}_t)$-停止時刻が定まる.このとき
				\begin{align}
					V_T(\omega) \geq \epsilon 
					&\quad \Longrightarrow \quad
					X_t(\omega) \geq \epsilon, \quad \exists t \in [0,T(\omega)] \\
					&\quad \Longrightarrow \quad
					H_\epsilon(\omega) \leq t \leq T(\omega)
				\end{align}
				が成立するから,$\{X_0 = 0\} \cap \{V_T \geq \epsilon\}$上で
				$\epsilon = X_{H_\epsilon} = X_{T \wedge H_\epsilon}$となり
				\begin{align}
					\epsilon P(V_T \geq \epsilon)
					= \int_{\{V_T \geq \epsilon\}} X_{T \wedge H_\epsilon}\ dP
					\leq EX_{T \wedge H_\epsilon}
					\leq EA_{T \wedge H_\epsilon}
					\leq EA_T
				\end{align}
				が得られる.
				
			\item[(2)] $S_\delta \coloneqq \inf{}{\Set{t \geq 0}{A_t \geq \delta}}$により$(\mathscr{F}_t)$-停止時刻を定めれば,
				$A_{S_\delta} = \delta$と$t \longmapsto A_t(\omega)$の増大性より
				\begin{align}
					A_T(\omega) < \delta \quad \Longleftrightarrow \quad
					T(\omega) < S_\delta(\omega)
				\end{align}
				となるから
				\begin{align}
					P\left( V_T \geq \epsilon,\ A_T < \delta \right)
					&= P\left( V_{T \wedge S_\delta} \geq \epsilon,\ A_T < \delta \right)
					\leq P\left( V_{T \wedge S_\delta} \geq \epsilon \right) \\
					&\leq \frac{E(A_{S_\delta \wedge T})}{\epsilon}
					= \frac{E(A_{S_\delta} \wedge A_T)}{\epsilon}
					= \frac{E(\delta \wedge A_T)}{\epsilon}
				\end{align}
				が成立し,両辺に$P\left( V_T \geq \epsilon,\ A_T \geq \delta \right)$を加えて
				Lenglartの不等式を得る.
				
			\item[(3)] $F$は連続かつ非減少であるからLebesgue-Stieltjes積分が構成され,任意の$x \in [0,\infty)$に対し
				\begin{align}
					F(x) = \int_{[0,\infty)} \defunc_{(0,x]}(u)\ dF(u)
				\end{align}
				が満たされる.
				\begin{align}
					(\omega,u) \longmapsto \defunc_{(0,V_T(\omega)]}(u)
				\end{align}
				は,$u$の関数として左連続であり,また$\omega$の関数としては$\mathscr{F}/\borel{\R}$-可測であるから(Problem 1.16),
				二変数関数として$\mathscr{F} \otimes \borel{[0,\infty)}/\borel{\R}$-可測であり,このときFubiniの定理より
				\begin{align}
					E F(V_T) &= \int_{[0,\infty)} E\left( \defunc_{[u,\infty)}(V_T) \right)\ dF(u) \\
					&= \int_{[0,\infty)} P(V_T \geq u)\ dF(u) \\
					&\leq \int_{[0,\infty)} \frac{E(u \wedge A_T)}{u} + P(A_T \geq u)\ dF(u) \\
					&= \int_{[0,\infty)} \frac{E(u \wedge A_T \defunc_{\{A_T \geq u\}})}{u} + 
						\frac{E(u \wedge A_T \defunc_{\{A_T < u\}})}{u} + P(A_T \geq u)\ dF(u) \\
					&= \int_{[0,\infty)} 2 P(A_T \geq u) + \frac{E(A_T \defunc_{\{A_T < u\}})}{u}\ dF(u) \\
					&=  E\left(2F(A_T)\right) 
						+ E\left[A_T \int_{[0,\infty)} \frac{1}{u} \defunc_{(A_T,\infty)}(u)\ dF(u)\right] \\
					&= EG(A_T)
				\end{align}
				が得られる.
				\QED
		\end{description}
	\end{prf}
\section{Continuous, Square-Integrable Martingales}
	\begin{itembox}[l]{Processes of difference of two natural processes}
		Let denote the space of processes represented by difference of two natural processes as
		\begin{align}
			\mathscr{A} \coloneqq \Set{A^{(1)} - A^{(2)}}{A^{(j)} \ \mbox{: natural},\ j=1,2},
		\end{align}
		and the equivalent class of $A \in \mathscr{A}$ in the meaning of
		(\refeq{eq:equivalence_with_respect_to_path}) in $\mathscr{A}$ as
		$[A]_{\mathscr{A}}$. Similarly define
		\begin{align}
			\mathscr{A}_c \coloneqq \Set{A^{(1)} - A^{(2)}}{A^{(j)} \ \mbox{: natural, continuous},\ j=1,2}
		\end{align}
		and the equivalent class of $A \in \mathscr{A}_c$ in the meaning of
		(\refeq{eq:equivalence_with_respect_to_path}) in $\mathscr{A}_c$ as
		$[A]_{\mathscr{A}_c}$.
	\end{itembox}
	
	\begin{itembox}[l]{Definition 5.3 修正}
		For $X \in \mathscr{M}_2$, we define the quadratic variation of $X$ to be the process $\inprod<X>_t \coloneqq A_t$,
		where $A$ is the natural increasing process in the Doob-Meyer decomposition of $x^2$.
		\textcolor{red}{For $X \in \mathscr{M}_2^c$, the quadratic variation $\inprod<X>$ of $X$ 
		to be natural increasing and continuous process.}
	\end{itembox}
	
	\begin{itembox}[l]{Problem 5.7 修正}
		Show that $\inprod<\cdot,\cdot>$ is a bilinear form on $\mathscr{M}_2$, i.e.,
		for any members $X,Y,Z$ of $\mathscr{M}_2$ and real numbers $\alpha,\beta$, we have
		\begin{description}
			\item[(i)] $[\inprod<\alpha X + \beta Y,Z>]_{\mathscr{A}} 
				= [\alpha \inprod<X,Z> + \beta \inprod<Y,Z>]_{\mathscr{A}}$.
			\item[(ii)] $[\inprod<X,Y>]_{\mathscr{A}} = [\inprod<Y,X>]_{\mathscr{A}}$.
			\item[(iii)] $|\inprod<X,Y>|^2 \leq \inprod<X> \inprod<Y>$.
			\item[(iv)] For $P$-a.e. $\omega \in \Omega$,
				\begin{align}
					\check{\xi}_t(\omega) - \check{\xi}_s(\omega)
					\leq \frac{1}{2}[\inprod<X>_t(\omega) - \inprod<X>_s(\omega)
						+ \inprod<Y>_t(\omega) - \inprod<Y>_s(\omega)];
						\quad 0 \leq s < t < \infty,
				\end{align}
				where $\check{\xi}_t$ denotes the total variation of 
				$\check{\xi} \coloneqq \inprod<X,Y>$ on $[0,t]$.
				
			\item[(v)] For any stopping time $T$ of $(\mathscr{F}_t)_{t \geq 0}$, we have 
				\begin{align}
					P \left( \inprod<X>_{t \wedge T} = \inprod<X^T>_t,\ \forall 0 \leq t < \infty \right) = 1,
				\end{align}
				where $X^T_t \coloneqq X_{t \wedge T},\ (\forall t \geq 0)$.
		\end{description}
	\end{itembox}
	
	\begin{prf}\mbox{}
		\begin{description}
			\item[(i)] ナチュラルなプロセス
				$A^{(j)},B^{(j)},C^{(j)},\ (j=1,2)$により
				\begin{align}
					\inprod<\alpha X + \beta Y, Z> = A^{(1)} - A^{(2)},
					\quad \alpha \inprod<X,Z> = B^{(1)} - B^{(2)},
					\quad \beta \inprod<Y,Z> = C^{(1)} - C^{(2)}
				\end{align}
				と表せるから
				\begin{align}
					\inprod<\alpha X + \beta Y, Z> 
					- \left(\alpha \inprod<X,Z> + \beta \inprod<Y,Z>\right)
					= \left(A^{(1)} + B^{(2)} + C^{(2)}\right)
					- \left(A^{(2)} + B^{(1)} + C^{(1)}\right)
				\end{align}
				となり,P. \pageref{lem:uniqueness_of_Doob_Meyer_decomposition}の補題より
				\begin{align}
					\inprod<\alpha X + \beta Y, Z>_t 
					= \alpha \inprod<X,Z>_t + \beta \inprod<Y,Z>_t,
					\quad 0 \leq t < \infty,
					\quad \mbox{a.s. $P$}
				\end{align}
				が従う.
			
			\item[(ii)] 
				$XY - \inprod<X,Y>$も$YX - \inprod<Y,X>$も右連続マルチンゲールであるから
				\begin{align}
					\inprod<X,Y> - \inprod<Y,X>
				\end{align}
				も右連続マルチンゲールであり,P. \pageref{lem:uniqueness_of_Doob_Meyer_decomposition}の補題より
				\begin{align}
					\inprod<X,Y>_t = \inprod<Y,X>_t,
					\quad 0 \leq t < \infty,
					\quad \mbox{a.s. $P$}
				\end{align}
				が従う.

			\item[(iii)] Shwartzの不等式
		\end{description}
	\end{prf}
	
	\begin{itembox}[l]{Lemma 5.9}
		Let $X \in \mathscr{M}_2$ satisfy $|X_s| \leq K < \infty$ for all $s \in [0,t]$,
		a.s. $P$. Let $\Pi = \{t_0,t_1,\cdots,t_m\}$, with $0 = t_0 \leq t_1 \leq \cdots \leq
		t_m = t$, be a partition of $[0,t]$. Then $E\left( V_t^{(2)}(\Pi) \right)^2 \leq 6K^4$.
	\end{itembox}
	
	\begin{prf}
		$X$のマルチンゲール性により,任意の$0 \leq s_0 \leq s_1 \leq \cdots \leq s_n < \infty$に対して
		\begin{align}
			E \sum_{k=1}^n \left|X_{s_k} - X_{s_{k-1}}\right|^2
			&= \sum_{k=1}^n E \left\{\cexp{X_{s_k}^2 -2X_{s_k}X_{s_{k-1}} + X_{s_{k-1}}^2}{\mathscr{F}_{s_k}}\right\} \\
			&= \sum_{k=1}^n E \left\{X_{s_k}^2 -2\cexp{X_{s_k}}{\mathscr{F}_{s_k}}X_{s_{k-1}} + X_{s_{k-1}}^2\right\} \\
			&= \sum_{k=1}^n E \left( X_{s_k}^2 - X_{s_{k-1}}^2 \right) \\
			&= E X_{s_n}^2 - E X_{s_0}^2
			\label{eq:chapter_1_lemma_5_9_1}
		\end{align}
		が成立する.いま,
		\begin{align}
			E\left( V_t^{(2)}(\Pi) \right)^2
			= E \left\{ \sum_{k=1}^m \left|X_{t_k} - X_{t_{k-1}}\right|^2 \right\}^2
			= E \sum_{k=1}^m \left|X_{t_k} - X_{t_{k-1}}\right|^4
				+ 2 E \sum_{i=1}^{m-1} \sum_{j=i+1}^m 
				\left|X_{t_i} - X_{t_{i-1}}\right|^2\left|X_{t_j} - X_{t_{j-1}}\right|^2
		\end{align}
		と分解すれば,$\left|X_{t_k} - X_{t_{k-1}}\right|^2 \leq 2\left( X_{t_k}^2 + X_{t_{k-1}}^2 \right)^2 \leq 2K^2$
		と(\refeq{eq:chapter_1_lemma_5_9_1})より右辺第一項は
		\begin{align}
			E \sum_{k=1}^m \left|X_{t_k} - X_{t_{k-1}}\right|^4
			\leq 2K^2 E \sum_{k=1}^m \left|X_{t_k} - X_{t_{k-1}}\right|^2
			= 2K^2 E X_{t_m}^2
			\leq 2K^4
		\end{align}
		となる.また右辺第二項も(\refeq{eq:chapter_1_lemma_5_9_1})より
		\begin{align}
			\sum_{i=1}^{m-1} \sum_{j=i+1}^m 
				E \left|X_{t_i} - X_{t_{i-1}}\right|^2\left|X_{t_j} - X_{t_{j-1}}\right|^2
			&= \sum_{i=1}^{m-1} \sum_{j=i+1}^m 
				E \left[ \cexp{\left|X_{t_i} - X_{t_{i-1}}\right|^2 
				\left|X_{t_j} - X_{t_{j-1}}\right|^2}{\mathscr{F}_{t_j}} \right] \\
			&= \sum_{i=1}^{m-1} \sum_{j=i+1}^m 
				E \left[\left|X_{t_i} - X_{t_{i-1}}\right|^2
				\cexp{\left|X_{t_j} - X_{t_{j-1}}\right|^2}{\mathscr{F}_{t_j}} \right] \\
			&= \sum_{i=1}^{m-1} \sum_{j=i+1}^m 
				E \left|X_{t_i} - X_{t_{i-1}}\right|^2 \left( X_{t_j}^2 - X_{t_{j-1}}^2 \right) \\
			&= \sum_{i=1}^{m-1} E \left|X_{t_i} - X_{t_{i-1}}\right|^2 \left( X_t^2 - X_{t_i}^2 \right) \\
			&\leq 2K^2 E \sum_{i=1}^{m-1} \left|X_{t_i} - X_{t_{i-1}}\right|^2 \\
			&\leq 2K^4
		\end{align}
		となるから$E\left( V_t^{(2)}(\Pi) \right)^2 \leq 6K^4$が出る.
		\QED
	\end{prf}
	
	\begin{itembox}[l]{Lemma 5.10}
		Let $X \in \mathscr{M}_2^c$ satisfy $|X_s| \leq K < \infty$ for all $s \in [0,t]$, a.s. $P$.
		For partitions $\Pi$ of $[0,t]$, we have
		\begin{align}
			\lim_{\Norm{\Pi}{} \to 0} E V_t^{(4)}(\Pi) = 0.
		\end{align}
	\end{itembox}
	
	\begin{prf}\mbox{}
		\begin{description}
			\item[第一段] 
				任意の$\omega \in \Omega$と$\delta > 0$に対し
				\begin{align}
					&\sup{}{\Set{|X_r(\omega) - X_s(\omega|}{s,r \in [0,t],\ |s-r| < \delta}} \\
					&\qquad = \sup{}{\Set{|X_p(\omega) - X_q(\omega)|}{p,q \in Q \cap [0,t],\ |q-p| < \delta}}
					\label{eq:chapter_1_lemma_5_10_1}
				\end{align}
				が成立する.実際,上限を取る範囲の大小関係より$\mbox{(左辺)} \geq \mbox{(右辺)}$が成り立ち,
				一方で任意の$\mbox{(左辺)} > \alpha > 0$に対し
				$|X_r(\omega) - X_s(\omega)| > \alpha$を満たす$s,r \in [0,t],\ (|s-r| < \delta)$を取れば,
				$X$のパスの連続性より
				\begin{align}
					|X_r(\omega) - X_p(\omega)|,\ |X_s(\omega) - X_q(\omega)| < \frac{\beta-\alpha}{2}
				\end{align}
				を満たす$p,q \in \Q \cap [0,t],\ (|p-q| < \delta)$が存在して
				\begin{align}
					|X_p(\omega) - X_q(\omega)| \geq |X_r(\omega) - X_s(\omega) | - |X_r(\omega) - X_p(\omega)| - |X_q(\omega) - X_s(\omega)| > \alpha
				\end{align}
				となり(\refeq{eq:chapter_1_lemma_5_10_1})が出る.
				$\mbox{(左辺)} \leq 2K$より$m_t(X;\delta)$は$\mathscr{F}/\borel{\R}$-可測である.
				また定理\ref{thm:exponentiation_of_supremum_supremum_of_exponentiation}より任意の$a > 0$で
				\begin{align}
					m_t^a(X;\delta) 
					\coloneqq \sup{}{\Set{|X_p(\omega) - X_q(\omega)|^a}{p,q \in Q \cap [0,t],\ |q-p| < \delta}}
				\end{align}
				が満たされる.
				
			\item[第二段]
				H\Ddot{o}lderの不等式より,任意の$\Pi$に対し
				\begin{align}
					E V_t^{(4)}(\Pi) \leq E\left[ V_t^{(2)}(\Pi) \cdot m_t^2(X;\Norm{\Pi}{}) \right]
					\leq \left\{ E \left(V_t^{(2)}(\Pi)\right)^2 \right\}^{1/2}
						\left\{ E m_t^4(X;\Norm{\Pi}{}) \right\}^{1/2}
					\leq \sqrt{6} K^2 \left\{ E m_t^4(X;\Norm{\Pi}{}) \right\}^{1/2}
				\end{align}
				となる.任意に$\Norm{\Pi_n}{} \longrightarrow 0,\ (n \longrightarrow \infty)$
				を満たす分割列$(\Pi_n)_{n=1}^\infty$を取れば
				\begin{align}
					\lim_{n \to \infty} m_t(X;\Norm{\Pi_n}{}) = 0,
					\quad m_t(X;\Norm{\Pi_n}{}) \leq 2K,\ (\forall n \geq 1)
				\end{align}
				が成り立つから,Lebesgueの収束定理より
				\begin{align}
					E m_t^4(X;\Norm{\Pi_n}{}) \longrightarrow 0\quad (n \longrightarrow \infty)
				\end{align}
				が従い$E V_t^{(4)}(\Pi_n) \longrightarrow 0\ (n \longrightarrow \infty)$となる.
				$(\Pi_n)_{n=1}^\infty$の任意性より補題の主張が得られる.
				\QED
		\end{description}
	\end{prf}
	
	\begin{itembox}[l]{Theorem 5.8}
		Let $X$ be in $\mathscr{M}_2^c$. For partitions $\Pi$ of $[0,t]$, we have
		$\lim_{\Norm{\Pi}{} \to 0} V_t^{(2)} = \inprod<X>_t$ (in probability); i.e.,
		for every $\epsilon > 0,\ \eta > 0$ there exists $\delta > 0$ such that $\Norm{\Pi}{} < \delta$ implies
		\begin{align}
			P\left[ \left| V_t^{(2)}(\Pi) - \inprod<X>_t \right| > \epsilon \right] < \eta.
		\end{align}
	\end{itembox}
	
	\begin{prf}\mbox{}
		\begin{description}
			\item[第一段]
				$X^2 - \inprod<X>$のマルチンゲール性より任意の$0 \leq s < t < \infty$に対して
				\begin{align}
					\cexp{(X_t - X_s)^2 - (\inprod<X>_t - \inprod<X>_s)}{\mathscr{F}_s}
					&= \cexp{(X_t - X_s)^2}{\mathscr{F}_s} - \cexp{\inprod<X>_t - \inprod<X>_s}{\mathscr{F}_s} \\
					&= \cexp{X_t^2 - X_s^2}{\mathscr{F}_s} - \cexp{\inprod<X>_t - \inprod<X>_s}{\mathscr{F}_s} \\
					&= 0,
					\quad \mbox{a.s. $P$}
				\end{align}
				となる.従って,任意の$0 \leq u < v \leq s < t < \infty$に対し
				\begin{align}
					E\left|(X_v - X_u)^2 - (\inprod<X>_v - \inprod<X>_u)\right|
					\left|(X_t - X_s)^2 - (\inprod<X>_t - \inprod<X>_s)\right| < \infty
				\end{align}
				であれば
				\begin{align}
					&E \left[\left\{(X_v - X_u)^2 - (\inprod<X>_v - \inprod<X>_u)\right\}
						\left\{(X_t - X_s)^2 - (\inprod<X>_t - \inprod<X>_s)\right\}\right] \\
					&\quad= E \left[\cexp{\left\{(X_v - X_u)^2 - (\inprod<X>_v - \inprod<X>_u)\right\}
						\left\{(X_t - X_s)^2 - (\inprod<X>_t - \inprod<X>_s)\right\}}{\mathscr{F}_s} \right] \\
					&\quad= E \left[\left\{(X_v - X_u)^2 - (\inprod<X>_v - \inprod<X>_u)\right\}
						\cexp{\left\{(X_t - X_s)^2 - (\inprod<X>_t - \inprod<X>_s)\right\}}{\mathscr{F}_s} \right] \\
					&\quad= 0
				\end{align}
				が成立する.
				
			\item[第二段]
				$|X|$及び$\inprod<X>$のパスは全て連続であるから,Problem 2.7より
				\begin{align}
					T_n \coloneqq \inf{}{\Set{t \geq 0}{|X_t| \vee \inprod<X>_t \geq n}}
				\end{align}
				で$(\mathscr{F}_t)$-停止時刻の列$(T_n)_{n=1}^\infty$が定まる.
				このとき任意の$\omega \in \Omega$で
				\begin{align}
					\Set{t \geq 0}{|X_t(\omega)| \vee \inprod<X>_t(\omega) \geq n+1}
					\subset \Set{t \geq 0}{|X_t(\omega)| \vee \inprod<X>_t(\omega) \geq n}
				\end{align}
				となるから
				\begin{align}
					T_n \leq T_{n+1}, \quad (\forall n \geq 1)
				\end{align}
				が成立し,また任意の$K > 0$に対し$\sup{t \in [0,K]}{|X_t(\omega)| \vee \inprod<X>_t(\omega)} < N$を満たす
				$N \in \N$を取れば$T_N(\omega) > K$となり
				\begin{align}
					\lim_{n \to \infty} T_n(\omega) = \infty, \quad (\forall \omega \in \Omega)
					\label{eq:chapter_1_theorem_5_8_1}
				\end{align}
				が従う.
				
			\item[第三段]
				$X^{(n)}$を$X^{(n)}_t \coloneqq X_{t \wedge T_n},\ (\forall t \geq 0)$で定めて,
				$[0,t]$の分割$\Pi = \{t_0,t_1,\cdots,t_m\}$に対し
				\begin{align}
					V^{(2,n)}_t(\Pi) \coloneqq \sum_{k=1}^m \left|X^{(n)}_{t_k} - X^{(n)}_{t_{k-1}}\right|^2
				\end{align}
				とおけば,$\{t \leq T_n\}$の上で$X_t = X^{(n)}_t$となるから
				\begin{align}
					V^{(2,n)}_t(\Pi)(\omega) = V^{(2)}_t(\Pi)(\omega),
					\quad \left(\forall \omega \in \{t \leq T_n\}\right)
					\label{eq:chapter_1_theorem_5_8_2}
				\end{align}
				が成り立つ.$\left|X_{t \wedge T_n}\right| \vee \inprod<X>_{t \wedge T_n} \leq n$であるから,
				Lemma 5.10 と第一段の結果及び$\inprod<X^{(n)}>$の連続性により
				\begin{align}
					E \left| V_t^{(2,n)}(\Pi) - \inprod<X^{(n)}>_t \right|^2
					&= E \left[ \sum_{k=1}^m \left\{\left|X^{(n)}_{t_k} - X^{(n)}_{t_{k-1}}\right|^2 
						- \left(\inprod<X^{(n)}>_{t_k} - \inprod<X^{(n)}>_{t_{k-1}}\right)\right\} \right]^2 \\
					&= E \sum_{k=1}^m \left\{ \left|X^{(n)}_{t_k} - X^{(n)}_{t_{k-1}}\right|^2 
						- \left(\inprod<X^{(n)}>_{t_k} - \inprod<X^{(n)}>_{t_{k-1}}\right) \right\}^2 \\
					&\leq 2 E \sum_{k=1}^m \left|X^{(n)}_{t_k} - X^{(n)}_{t_{k-1}}\right|^4
						+ 2 E \left(\inprod<X^{(n)}>_{t_k} - \inprod<X^{(n)}>_{t_{k-1}}\right)^2 \\
					&\leq 2 E V_t^{(4,n)}(\Pi) 
						+ 2 n E \left[ m_t\left(\inprod<X^{(n)}>;\Norm{\Pi}{}\right) \right] \\
					&\longrightarrow 0,\quad (\Norm{\Pi}{} \longrightarrow 0)
				\end{align}
				が得られる.
			
			\item[第四段]
				任意に$\epsilon > 0$と$\eta > 0$を取る.(\refeq{eq:chapter_1_theorem_5_8_2})より任意の$n \geq 1$で
				\begin{align}
					P\left(\left|V_t^{(2)}(\Pi) - \inprod<X>_t\right| > \epsilon\right)
					&= P\left(\left\{\left|V_t^{(2)}(\Pi) - \inprod<X>_t\right| > \epsilon\right\} \cap \{t > T_n\}\right)
						+ P\left(\left\{\left|V_t^{(2,n)}(\Pi) - \inprod<X^{(n)}>_t\right| > \epsilon\right\} \cap \{t \leq T_n\}\right) \\
					&\leq P(t > T_n) + P\left(\left|V_t^{(2,n)}(\Pi) - \inprod<X^{(n)}>_t\right| > \epsilon\right)
				\end{align}
				が成立し,このとき(\refeq{eq:chapter_1_theorem_5_8_1})より或る$N \geq 1$が存在して
				\begin{align}
					P(t > T_n) < \frac{\eta}{2},\quad (\forall n \geq N)
				\end{align}
				となり,前段の結果より或る$\delta > 0$が存在して$\Norm{\Pi}{} < \delta$なら
				\begin{align}
					P\left(\left|V_t^{(2,N)}(\Pi) - \inprod<X^{(N)}>_t\right| > \epsilon\right)
					\leq \frac{1}{\epsilon} E\left|V_t^{(2,N)}(\Pi) - \inprod<X^{(N)}>_t\right|
					\leq \frac{1}{\epsilon} \left\{E\left|V_t^{(2,N)}(\Pi) - \inprod<X^{(N)}>_t\right|^2 \right\}^{1/2}
					< \frac{\eta}{2}
				\end{align}
				が満たされるから
				\begin{align}
					\Norm{\Pi}{} < \delta
					\quad \Longrightarrow \quad 
					P\left(\left|V_t^{(2)}(\Pi) - \inprod<X>_t\right| > \epsilon\right) < \eta
				\end{align}
				が従う.
				\QED
		\end{description}
	\end{prf}
	
	\begin{itembox}[l]{Theorem 5.13 修正}
		Let $X = \Set{X_t,\mathscr{F}_t}{0 \leq t < \infty}$ and $Y = \Set{Y_t,\mathscr{F}_t}{0 \leq t < \infty}$
		be members of $\mathscr{M}_2^c$. \textcolor{red}{There is a unique $[A]_{\mathscr{A}_c}$ such that
		$\Set{X_t Y_t - \tilde{A}_t,\mathscr{F}_t}{0 \leq t < \infty}$ is a continuous martingale
		for every $\tilde{A} \in [A]_{\mathscr{A}_c}$.}
	\end{itembox}
	
	\begin{prf}
		定義より$\inprod<X,Y> \in \mathscr{A}_c$に対して$XY - \inprod<X,Y>$
		は連続マルチンゲールである.また$\inprod<X,Y>$と区別不能な$A \in \mathscr{A}_c$を取れば,任意の$t \geq 0$で
		\begin{align}
			P(X_t Y_t - \inprod<X,Y>_t = X_t Y_t - A_t) = 1
		\end{align}
		となるから$XY-A$もまた連続マルチンゲールとなる.
		$A,B \in \mathscr{A}_c$に対し$XY - A,\ XY-B$が共にマルチンゲールとなるとき,
		$A - B$もマルチンゲールとなり,Theorem 4.14の補題(P. \pageref{lem:uniqueness_of_Doob_Meyer_decomposition})より
		$[A]_{\mathscr{A}_c} = [B]_{\mathscr{A}_c}$が従う.
		\QED
	\end{prf}
	
	\begin{itembox}[l]{Problem 5.17}
		Let $X$, $Y$ be in $\mathscr{M}^{c,loc}$. Then there is a unique (up to indistinguishablility) adapted,
		continuous process of bounded variation $\inprod<X,Y>$ satisfying $\inprod<X,Y>_0 = 0$,
		such that $XY - \inprod<X,Y> \in \mathscr{M}^{c,loc}$. If $X = Y$, we write $\inprod<X> = \inprod<X,X>$,
		and this process is nondecreasing.
	\end{itembox}
	
	\begin{prf}
		
	\end{prf}
	
	\begin{itembox}[l]{Problem 5.19}
		\begin{description}
			\item[(i)] A local martingale of class $DL$ is a martingale.
			\item[(ii)]
			\item[(iii)]
		\end{description}
	\end{itembox}
	
	\begin{prf}\mbox{}
		\begin{description}
			\item[(i)] $X$を局所マルチンゲールとすれば,
				或る$(\mathscr{F}_t)$-停止時刻の列$(T_n)_{n=1}^\infty$と$P$-零集合$E$が存在して
				\begin{align}
					T_1(\omega) \leq T_2(\omega) \leq \cdots \longrightarrow \infty,
					\quad (\forall \omega \in \Omega \backslash E)
				\end{align}
				かつ全ての$n \geq 1$で$\Set{X_{t \wedge T_n},\mathscr{F}_t}{0 \leq t < \infty}$はマルチンゲールとなる.
				任意に$t \geq 0$を取れば$\left\{t \wedge T_n\right\}_{n=1}^\infty \subset \mathscr{S}_t$となり,
				$X$はクラス$DL$に属しているから$\left(X_{t \wedge T_n}\right)_{n=1}^\infty$は一様可積分である.
				ここで$E \in \mathscr{F}_0$かつ
				\begin{align}
					X_t(\omega) = \lim_{n \to \infty} X_{t \wedge T_n}(\omega),
					\quad (\forall \omega \in \Omega \backslash E)
				\end{align}
				が成り立つから$X_t$は$\mathscr{F}_t/\borel{\R}$-可測であり,
				また定理\ref{lem:uniformly_integrable_and_convergence_in_mean}より
				$X_t$の可積分性及び
				\begin{align}
					E\left| X_t - X_{t \wedge T_n}(\omega) \right| \longrightarrow 0
					\quad (n \longrightarrow \infty)
				\end{align}
				が従う.よって任意に$0 \leq s < t < \infty$を取れば
				\begin{align}
					\int_A X_t\ dP = \lim_{n \to \infty} \int_A X_{t \wedge T_n}\ dP
					= \lim_{n \to \infty} \int_A X_{s \wedge T_n}\ dP
					= \int_A X_s\ dP,
					\quad (\forall A \in \mathscr{F}_s)
				\end{align}
				が満たされ,$\Set{X_t,\mathscr{F}_t}{0 \leq t < \infty}$のマルチンゲール性が得られる.
		\end{description}
	\end{prf}
	
	\begin{itembox}[l]{Definition 5.22 修正}
		\textcolor{red}{$\mathscr{M}_2$ and $\mathscr{M}_2^c$ are vector spaces, 
		where the additions and scalar multiplications are defined by
		\begin{align}
			(X+Y)_t(\omega) \coloneqq X_t(\omega) + Y_t(\omega),
			\quad (\alpha X)_t(\omega) \coloneqq \alpha X_t(\omega),
			\quad (\forall X,Y \in \mathscr{M}_2\ \mbox{(resp. $\mathscr{M}_2^c$)},\ \forall \alpha \in \R).
		\end{align}
		Let denote the quotient space of $\mathscr{M}_2$ and $\mathscr{M}_2^c$ with respect to
		the equivalent relation as in (\refeq{eq:equivalence_with_respect_to_path})
		(P. \pageref{eq:equivalence_with_respect_to_path}) by $\mathfrak{M}_2$ and $\mathfrak{M}_2^c$,
		and denote the elements of each space by $[X]_{\mathfrak{M}_2}$ and  $[X]_{\mathfrak{M}_2^c}$.
		For any $[X]_{\mathfrak{M}_2},[Y]_{\mathfrak{M}_2} \in \mathfrak{M}_2,\ 
		\mbox{(resp. $[X]_{\mathfrak{M}_2^c},[Y]_{\mathfrak{M}_2^c} \in \mathfrak{M}_2^c$)}$
		and $0 \leq t < \infty$, we define a distance by
		\begin{align}
			d\left([X]_{\mathfrak{M}_2},[Y]_{\mathfrak{M}_2}\right)
			&\coloneqq \sum_{n=1}^\infty 2^{-n}\left(\Norm{[X_n] - [Y_n]}{L^2(P)} \wedge 1\right), \\
			d_c\left([X]_{\mathfrak{M}_2^c},[Y]_{\mathfrak{M}_2^c}\right)
			&\coloneqq \sum_{n=1}^\infty 2^{-n}\left(\Norm{[X_n] - [Y_n]}{L^2(P)} \wedge 1\right),
		\end{align}
		where $\Norm{\cdot}{L^2(P)}$ denotes the $L^2$ norm on $L^2(P) = L^2(\Omega,\mathscr{F},P)$.}
	\end{itembox}
	
	\begin{itembox}[l]{Proposition 5.23 修正}
		\begin{description}
			\item[(1)]
				Suppose that the filtration $\{\mathscr{F}_t\}$ satisfies the usual conditions.
				Then $\mathfrak{M}_2$ is a complete metric space under the preceding metric $d$.
			\item[(2)]
				Suppose that for every $t \in [0,\infty)$,
				$\mathscr{F}_t$ contains all the $P$-negligible events in $\mathscr{F}$.
				Then $\mathfrak{M}_2^c$ is a complete metric space under the preceding metric $d_c$.
		\end{description}
	\end{itembox}
	
	\begin{prf} 任意の$0 \leq t < \infty$に対し,$L^2(\Omega,\mathscr{F}_t,P)$における関数類を$[\cdot]_t$と書く.
		\begin{description}
			\item[(1)] $\left([X^{(k)}]_{\mathfrak{M}_2}\right)_{k=1}^\infty$をCauchy列とすれば,
				$|X^{(k)} - X^{(j)}|^2$の劣マルチンゲール性より任意の$0 \leq t \leq n$で
				\begin{align}
					\Norm{[x^{(k)}_t]_t - [X^{(j)}_t]_t}{L^2(\Omega,\mathscr{F}_t,P)} \wedge 1
					&\leq \Norm{[x^{(k)}_n] - [X^{(j)}_n]}{L^2(P)} \wedge 1 \\
					&\leq 2^n d\left( [X^{(k)}]_{\mathfrak{M}_2}, [X^{(j)}]_{\mathfrak{M}_2}\right)
					\longrightarrow 0,
					\quad (k,j \longrightarrow \infty)
				\end{align}
				となるから,定理\ref{thm:Lp_banach}より或る$[X_t]_t \in L^2(\Omega,\mathscr{F}_t,P)$が存在して
				\begin{align}
					E \left|X^{(k)}_t - X_t\right|^2 \longrightarrow 0,
					\quad (k \longrightarrow \infty)
				\end{align}
				を満たす.特に$t = 0$なら$X_t = 0,\ \mbox{a.s. $P$}$が従う.
				H\Ddot{o}lderの不等式より任意の$A \in \mathscr{F}_t$で
				\begin{align}
					\int_A \left|X^{(k)}_t - X_t\right|\ dP
					\leq \left( E \left|X^{(k)}_t - X_t\right|^2 \right)^{1/2}
					\longrightarrow 0,
					\quad (k \longrightarrow \infty)
				\end{align}
				が成り立つから,任意に$0 \leq s < t$を取れば
				\begin{align}
					\int_A X_s\ dP
					= \lim_{k \to \infty} \int_A X^{(k)}_s\ dP
					= \lim_{k \to \infty} \int_A X^{(k)}_t\ dP
					= \int_A X_t\ dP,
					\quad (\forall A \in \mathscr{F}_s)
				\end{align}
				となり$X = \Set{X_t,\mathscr{F}_t}{0 \leq t < \infty}$のマルチンゲール性が出る.
				Theorem 3.13より$X$の$RCLL$な修正$\tilde{X} \in \mathscr{M}_2$が得られ,ここで任意に$\epsilon > 0$及び
				$1/2^N < \epsilon/2$を満たす$N$を取れば,或る$K \geq 1$が存在して
				\begin{align}
					\Norm{[X^{(k)}_n] - [\tilde{X}_n]}{L^2(P)} < \frac{\epsilon}{2},
					\quad (\forall k \geq K)
				\end{align}
				がすべての$n \leq N$で満たされるから
				\begin{align}
					d\left([X^{(k)}]_{\mathfrak{M}_2}, [\tilde{X}]_{\mathfrak{M}_2}\right) < \epsilon,
					\quad (\forall k \geq K)
				\end{align}
				が従う.
		\end{description}
	\end{prf}

\chapter{Brownian Motion}
\begin{itembox}[l]{Dynkin system theorem}
		Let $\mathscr{C}$ be a collection of subsets of $\Omega$ 
		which is closed under pairwise intersection. If $\mathscr{D}$ is 
		a Dynkin system containing $\mathscr{C}$, then $\mathscr{D}$ also 
		contains the $\sigma$-field $\sigma(\mathscr{C})$ generated by $\mathscr{C}$.
\end{itembox}

\begin{prf}
	定理\ref{thm:Dynkin_system_theorem}より
	$\sigma(\mathscr{C}) = \delta(\mathscr{C}) \subset \mathscr{D}$となる.
	\QED
\end{prf}

\begin{itembox}[l]{Problem 1.4}
\label{thm:application_dynkin_system_theorem_to_independence}
		Let $X = \Set{X_t}{0 \leq t < \infty}$ be a stochastic process 
		for which $X_0,X_{t_1} - X_{t_0}, \cdots, X_{t_n} - X_{t_{n-1}}$ are 
		independent random variables, for every integer $n \geq 1$ and indices 
		$0 = t_0 < t_1 < \cdots < t_n < \infty$. Then for any fixed $0 \leq s < t < \infty$, 
		the increment $X_t - X_s$ is independent of $\mathscr{F}^X_s$.
\end{itembox}
この主張の逆も成立する:
\begin{prf}
	先ず任意の$s \leq t \leq r$に対し$\sigma(X_t - X_s) \subset \mathscr{F}^X_r$が成り立つ.実際,
	\begin{align}
		\Phi:\R^d \times \R^d \ni (x,y) \longmapsto x - y
	\end{align}
	の連続性と$\borel{\R^d \times \R^d} = \borel{\R^d} \otimes \borel{\R^d}$より,
	任意の$E \in \borel{\R^d}$に対して
	\begin{align}
		(X_t - X_s)^{-1}(E) 
		= \left\{ \left( X_t,X_s \right) \in \Phi^{-1}(E) \right\}
		\in \sigma(X_s,X_t) \subset \mathscr{F}^X_r
		\label{eq:thm_application_dynkin_system_theorem_to_independence_1}
	\end{align}
	が満たされる.よって任意に$A_0 \in \sigma(X_0),\ A_i \in \sigma(X_{t_i} - X_{t_{i-1}})$を取れば,
	$X_{t_n} - X_{t_{n-1}}$が$\mathscr{F}^X_{t_{n-1}}$と独立であるから
	\begin{align}
		P(A_0 \cap A_1 \cap \cdots \cap A_n)
		= P(A_0 \cap A_1 \cap \cdots \cap A_{n-1}) P(A_n)
	\end{align}
	が成立する.帰納的に
	\begin{align}
		P(A_0 \cap A_1 \cap \cdots \cap A_n)
		= P(A_0) P(A_1) \cdots P(A_n)
	\end{align}
	が従い$X_0,X_{t_1} - X_{t_0}, \cdots, X_{t_n} - X_{t_{n-1}}$の独立性を得る.
	\QED
\end{prf}

\begin{prf}[Problem 1.4]\mbox{}
	\begin{description}
		\item[第一段]
			Dynkin族を次で定める:
			\begin{align}
				\mathscr{D} \coloneqq
				\Set{A \in \mathscr{F}}{P(A \cap B) = P(A)P(B),\ \forall B \in \sigma(X_t - X_s)}.
			\end{align}
			いま,任意に$0 = s_0 < \cdots < s_n = s$を取り固定し
			\begin{align}
				\mathscr{A}_{s_0, \cdots, s_n} \coloneqq
				\Set{\bigcap_{i=0}^n A_i}{A_0 \in \sigma(X_0),\ A_i \in \sigma(X_{s_i} - X_{s_j}),\ i=1,\cdots,n}
			\end{align}
			により乗法族を定めれば,仮定より$\sigma(X_{s_i} - X_{s_{i-1}})$と$\sigma(X_t - X_s)$が独立であるから
			\begin{align}
				\mathscr{A}_{s_0, \cdots, s_n}
				\subset \mathscr{D}
			\end{align}
			が成立し,Dynkin族定理により
			\begin{align}
				\sigma(X_{s_0},X_{s_1}-X_{s_0},\cdots,X_{s_n} - X_{s_{n-1}})
				= \sgmalg{\mathscr{A}_{s_0, \cdots, s_n}}
				\subset \mathscr{D}
				\label{eq:thm_application_dynkin_system_theorem_to_independence_2}
			\end{align}
			が従う.
		
		\item[第二段]
			$\sigma(X_{s_0},X_{s_1}-X_{s_0},\cdots,X_{s_n} - X_{s_{n-1}})$の全体が
			$\mathscr{F}^X_s$を生成することを示す.先ず,
			(\refeq{eq:thm_application_dynkin_system_theorem_to_independence_1})より
			\begin{align}
				\bigcup_{\substack{n \geq 1 \\ s_0 < \cdots < s_n}} 
				\sigma(X_{s_0},X_{s_1}-X_{s_0},\cdots,X_{s_n} - X_{s_{n-1}})
				\subset \mathscr{F}^X_s
				\label{eq:thm_application_dynkin_system_theorem_to_independence_3}
			\end{align}
			が成立する.一方で,任意の
			$X_r^{-1}(E)\ (\forall E \in \borel{\R^d},\ 0 < r \leq s)$について,
			\begin{align}
				\Psi:\R^d \times \R^d \ni (x,y) \longmapsto x + y
			\end{align}
			で定める連続写像を用いれば
			\begin{align}
				X_r^{-1}(E)
				= \left( X_r - X_0 + X_0 \right)^{-1}(E)
				= \left\{\left( X_r - X_0, X_0\right) \in \Psi^{-1}(E) \right\}
			\end{align}
			となり,$X_r^{-1}(E) \in \sigma(X_0, X_r - X_0)$が満たされ
			\begin{align}
				\sigma(X_r) \subset \sigma(X_0, X_r - X_0)
				\subset \sigma(X_0, X_r - X_0,X_s - X_r)
				\label{eq:thm_application_dynkin_system_theorem_to_independence_4}
			\end{align}
			が出る.
			$\sigma(X_0) \subset \sigma(X_0,X_s - X_0)$
			も成り立ち
			\begin{align}
				\bigcup_{0 \leq r \leq s} \sigma(X_r) \subset 
				\bigcup_{\substack{n \geq 1 \\ s_0 < \cdots < s_n}} \sigma(X_{s_0},X_{s_1}-X_{s_0},\cdots,X_{s_n} - X_{s_{n-1}})
			\end{align}
			が従うから,(\refeq{eq:thm_application_dynkin_system_theorem_to_independence_3})
			と併せて
			\begin{align}
				\mathscr{F}^X_s
				= \sgmalg{\bigcup_{\substack{n \geq 1 \\ s_0 < \cdots < s_n}} \sigma(X_{s_0},X_{s_1}-X_{s_0},\cdots,X_{s_n} - X_{s_{n-1}})}
				\label{eq:thm_application_dynkin_system_theorem_to_independence_5}
			\end{align}
			が得られる.
		
		\item[第三段]
			任意の$0 = s_0 < s_1 < \cdots < s_n = s$に対し,
			(\refeq{eq:thm_application_dynkin_system_theorem_to_independence_1})と
			(\refeq{eq:thm_application_dynkin_system_theorem_to_independence_4})より
			\begin{align}
				\sigma(X_{s_0},X_{s_1}-X_{s_0},\cdots,X_{s_n} - X_{s_{n-1}})
				= \sigma(X_{s_0},X_{s_1},\cdots,X_{s_n})
				\label{eq:thm_application_dynkin_system_theorem_to_independence_6}
			\end{align}
			が成り立つ.
		
		\item[第四段]
			二つの節点$0 = s_0 < \cdots < s_n = s$と$0 = r_0 < \cdots < r_m = s$
			の合併を$0 = u_0 < \cdots < u_k = s$と書けば
			\begin{align}
				\sigma(X_{s_0},\cdots,X_{s_n})
				\cup \sigma(X_{r_0},\cdots,X_{r_m})
				\subset \sigma(X_{u_0},\cdots,X_{u_k})
			\end{align}
			が成り立つから
			\begin{align}
				\bigcup_{\substack{n \geq 1 \\ s_0 < \cdots < s_n}} \sigma(X_{s_0},X_{s_1},\cdots,X_{s_n})
			\end{align}
			は交演算で閉じている.従って
			(\refeq{eq:thm_application_dynkin_system_theorem_to_independence_2}),
			(\refeq{eq:thm_application_dynkin_system_theorem_to_independence_5}),
			(\refeq{eq:thm_application_dynkin_system_theorem_to_independence_6})及び
			Dynkin族定理により
			\begin{align}
				\mathscr{F}^X_s 
				= \sgmalg{\bigcup_{\substack{n \geq 1 \\ s_0 < \cdots < s_n}} \sigma(X_{s_0},X_{s_1}-X_{s_0},\cdots,X_{s_n} - X_{s_{n-1}})}
				= \sgmalg{\bigcup_{\substack{n \geq 1 \\ s_0 < \cdots < s_n}} \sigma(X_{s_0},X_{s_1},\cdots,X_{s_n})}
				\subset \mathscr{D}
			\end{align}
			が従い定理の主張を得る.
			\QED
	\end{description}
\end{prf}
\section{The Consistency Theorem}
	Karatzas-Shreve より Bogachev の Measure Theory に載っている
	Kolmogorovの拡張定理の方が洗練された簡潔な証明になっているので
	頭に入りやすい.
	
	
\section{The Kolmogorov-\v{C}entsov Theorem}
	\begin{itembox}[l]{Exercise 2.7}
		The only $\borel{(\R^d)^{[0,\infty)}}$-measurable set contained 
		in $C[0,\infty)^d$ is the empty set.
	\end{itembox}
	
	\begin{prf}\mbox{}
		\begin{description}
			\item[第一段]
				$\borel{(\R^d)^{[0,\infty)}} = \sigma(B_t;\ 0 \leq t < \infty)$
				が成り立つことを示す.先ず,任意の$C \in \mathscr{C}$は
				\begin{align}
					C &= \Set{\omega \in (\R^d)^{[0,\infty)}}{(\omega(t_1),\cdots,\omega(t_n)) \in A} \\
					&=  \Set{\omega \in (\R^d)^{[0,\infty)}}{(B_{t_1}(\omega),\cdots,B_{t_n}(\omega)) \in A},
					\quad (A \in \borel{(\R^d)^n})
				\end{align}
				の形で表されるから$\mathscr{C} \subset \sigma(B_t;\ 0 \leq t < \infty)$
				が従い$\borel{(\R^d)^{[0,\infty)}} \subset \sigma(B_t;\ 0 \leq t < \infty)$
				を得る.逆に
				\begin{align}
					\sigma(B_t) \subset \mathscr{C},
					\quad (\forall t \geq 0)
				\end{align}
				より$\borel{(\R^d)^{[0,\infty)}} \supset \sigma(B_t;\ 0 \leq t < \infty)$
				も成立し$\borel{(\R^d)^{[0,\infty)}} = \sigma(B_t;\ 0 \leq t < \infty)$
				が出る.
				
			\item[第二段]
				高々可算集合$S = \{t_1,t_2,\cdots\} \subset [0,\infty)$に対して
				\begin{align}
					\mathcal{E}_S \coloneqq \Set{\Set{\omega \in (\R^d)^{[0,\infty)}}{(\omega(t_1),\omega(t_2),\cdots) \in A}}{A \in \borel{(\R^d)^{\# S}}}
				\end{align}
				とおけば
				\footnote{
					$S$が可算無限なら$(\R^d)^{\# S} = \R^\infty$.
				},座標過程$B$は
				$(\omega(t_1),\omega(t_2),\cdots) = (B_{t_1}(\omega),B_{t_2}(\omega),\cdots)$
				を満たすから
				\begin{align}
					\mathcal{E}_S = \Set{\left\{(B_{t_1},B_{t_2},\cdots) \in A\right\}}{A \in \borel{(\R^d)^{\# S}}} \eqqcolon \mathcal{F}^B_S
				\end{align}
				が成立する.従って第一章のLemma3 for Exercise 1.8と前段の結果より
				\begin{align}
					\borel{(\R^d)^{[0,\infty)}}
					&= \sigma(B_t;\ 0 \leq t < \infty)
					= \mathcal{F}^B_{[0,\infty)}
					= \bigcup_{S \subset [0,\infty):at\ most\ countable} \mathcal{F}^B_S\\
					&= \bigcup_{S \subset [0,\infty):at\ most\ countable} \mathcal{E}_S
				\end{align}
				を得る.すなわち,$\borel{(\R^d)^{[0,\infty)}}$の任意の元は
				$\Set{\omega \in (\R^d)^{[0,\infty)}}{(\omega(t_1),\omega(t_2),\cdots) \in A}$
				の形で表現され,$A \neq \emptyset$ならば
				$\Set{\omega \in (\R^d)^{[0,\infty)}}{(\omega(t_1),\omega(t_2),\cdots) \in A} \not\subset C[0,\infty)^d$となり主張が従う.
				\QED
		\end{description}
	\end{prf}
	
	\begin{itembox}[l]{Theorem 2.8 and Problem 2.9}
		Suppose that a process $X = \Set{X_t}{t \in  [0,T]^d}$ ($d \geq 1$)
		on a probability space $(\Omega,\mathscr{F},P)$ satisfies the condition
		\begin{align}
			E|X_t - X_s|^\alpha \leq C\Norm{t-s}{}^{d + \beta},
			\quad \mbox{where} \Norm{t-s}{} = \operatorname*{max}_{1 \leq i \leq d}|t_i - s_i|,
		\end{align}
		for some positive constants $\alpha,\beta$, and $C$. Then there exists a 
		continuous modification $\tilde{X} = \Set{\tilde{X}_t}{t \in [0,T]^d}$ of $X$, 
		which is locally H\Ddot{o}lder-continuous with exponent $\gamma$ for every 
		$\gamma \in (0,\beta/\alpha)$. \textcolor{red}{More precisely, for every $\gamma \in (0,\beta/\alpha)$,
		\begin{align}
			\forall \omega \in \Omega^*, \quad \sup{\substack{0 < \Norm{t-s}{} < h(\omega) \\ s,t \in [0,T]^d}}{\frac{\left| \tilde{X}_t(\omega) - \tilde{X}_s(\omega) \right|}{\Norm{t-s}{}^\gamma}} \leq \frac{2}{1-2^{-\gamma}}
		\end{align}
		for some $\Omega^* \in \mathscr{F}$ with $P(\Omega^*)=1$ and 
		positive random variable $h$, where $\Omega^*$ and $h$ depend on $\gamma$.}
	\end{itembox}
	
	\begin{prf}\mbox{}
		\begin{description}
			\item[第一段]
				$[0,T]^d$における順序$\prec$を
				\begin{align}
					s \prec t \Longleftrightarrow \forall i \in \{1,2,\cdots,d\}[ s_i \leq t_i] \wedge \exists i \in \{1,2,\cdots,d\}[ s_i < t_i]
				\end{align}
				で定める.$\N$の任意の要素$n$に対して
				\begin{align}
					L_n = \Set{\frac{kT}{2^n}}{k=0,1,2,\cdots,2^n-1}
				\end{align}
				として$L = \bigcup_{n \in \N} L_n$とおく.$L$は$[0,T]^d$において稠密である.
				$L_n$の要素$s$に対して
				\begin{align}
					R_n(s) = \Set{t \in L_n}{s \prec t \wedge \Norm{t-s}{} = T2^{-n}}
				\end{align}
				とおく.つまり,$R_n(s)$とは$s$の各成分を最大$T2^{-n}$だけ動かした順序対の集合である.
				いま,$L_n$の要素数は$2^{nd}$,$L_n$の各要素$s$に対して
				$R_n(s)$の要素数は$2^d$である.
				Chebyshevの不等式より,任意の正数$\epsilon$に対して
				\begin{align}
					P\left(|X_t-X_s|\geq\epsilon\right)
					\leq \epsilon^{-\alpha}E|X_t-X_s|^\alpha
					\leq C\epsilon^{-\alpha}\Norm{t-s}{}^{d+\beta}
				\end{align}
				となり,特に$\epsilon = 2^{-\gamma n}$かつ
				$\Norm{t-s}{} = T2^{-n}$の場合は
				\begin{align}
					P\left(|X_t-X_s|\geq2^{-\gamma n}\right)
					\leq C 2^{-n(d+\beta - \alpha \gamma)}
				\end{align}
				が成り立つから,
				\begin{align}
					P\left(\operatorname*{max}_{s \in L_n \wedge t \in R_n(s)}
					|X_t-X_s|\geq2^{-\gamma n}\right)
					= P\left(\bigcup_{s \in L_n} \bigcup_{t \in R_n(s)}
					\{|X_t-X_s|\geq2^{-\gamma n}\}\right)
					\leq 2^d C T^{d+\beta} 2^{-n(\beta - \alpha \gamma)}
				\end{align}
				が成り立つ.$A_n = \left\{\operatorname*{max}_{s \in L_n \wedge t \in R_n(s)}|X_t-X_s|\geq2^{-\gamma n}\right\}$とおけば,Borel-Cantelliの補題より
				\begin{align}
					N = \bigcap_{n \in \N} \bigcup_{k \geq n} A_k
				\end{align}
				は$P$-零集合となり,
				\begin{align}
					\forall \omega \in \Omega \backslash N,\
					\exists N \in \N,\
					\forall n \in \N,\quad
					N \leq n \Longrightarrow \operatorname*{max}_{s \in L_n \wedge t \in R_n(s)}
					|X_t(\omega) - X_s(\omega)| < 2^{-\gamma n}
					\label{eq:chapter_2_theorem_2_8_1}
				\end{align}
				が満たされる.
				
			\item[第二段]
				$\Omega \backslash N$の要素$\omega$に対して,
				(\refeq{eq:chapter_2_theorem_2_8_1})を満たす自然数$N$のうち
				最小なものを$n^*(\omega)$と定める(自然数の整列性).つまり$n^*$は
		\begin{align}
			n^* = \Biggl\{\ (\omega,n)\ \, : \quad &\omega \in \Omega \wedge n \in \N \wedge \Biggr.\\
			&\forall m \in \N\left[n \leq m \Longrightarrow \operatorname*{max}_{s \in L_m \wedge t \in R_m(s)}
					|X_t(\omega) - X_s(\omega)| < 2^{-\gamma m}\right] \wedge \\
			&\forall N \in \N
			\Biggl. \left[\forall m \in \N\left[N \leq m \Longrightarrow \operatorname*{max}_{s \in L_m \wedge t \in R_m(s)}
					|X_t(\omega) - X_s(\omega)| < 2^{-\gamma m}\right] \Longrightarrow n \leq N\right]\ \Biggr\}
		\end{align}
		で与えられる写像である.写像$n^*$は$\mathscr{F}/\borel{\R}$-可測性を持つ.実際,
		任意の自然数$\ell$に対して
		\begin{align}
			{n^*}^{-1}(\ell) = \left\{ \bigcap_{n = \ell}^\infty A_n^c \right\} \cap \left\{ \bigcap_{1 \leq j \leq \ell-1} \bigcap_{n = j}^\infty A_n \right\}
		\end{align}
		を満たす.
		\end{description}
	\end{prf}
	
	
	
	確率変数$h$について,厳密には
	\begin{align}
		h(\omega) \coloneqq 
		\begin{cases}
			2^{-n^*(\omega)}, & (\omega \in \Omega^*), \\
			0, & (\omega \in \Omega \backslash \Omega^*)
		\end{cases}
	\end{align}
	とおけばよい.
	
	\begin{itembox}[l]{Corollary to Theorem 2.8}
		There is a probability measure $P$ on $(\R^{[0,\infty)},\mathscr{B}(\R^{[0,\infty)}))$,
		and a stochastic process $W = \Set{W_t,\mathscr{F}_t^W}{t \geq 0}$ on the same space,
		such that under $P$, $W$ is a Brownian motion.
	\end{itembox}
	
	\begin{prf}\mbox{}
		\begin{description}
			\item[第一段]
				Corollary to Theorem 2.2より,$(\R^{[0,\infty)},\mathscr{B}(\R^{[0,\infty)}))$にただ一つの確率測度$P$が存在して,
				\begin{align}
					B = \Set{(x,y)}{\exists t \in [0,\infty) \exists \omega \in \R^{[0,\infty)}
					\left( x=(t,\omega) \wedge y=\omega(t)-\omega(0) \right)}
				\end{align}
				で定める写像$B$が$P$の下で
				\begin{itemize}
					\item $\R^{[0,\infty)}$の任意の要素$\omega$に対して
						$B_0(\omega) = 0$,
					\item 任意の実数$s,t$に対し,$0 \leq s < t$ならば
						$B_t - B_s$は$\mathscr{F}_s$と独立,
					\item 任意の実数$s,t$に対し,$0 \leq s < t$ならば
						$P(B_t - B_s)^{-1}$は平均0で分散が$t-s$の正規分布
				\end{itemize}
				となる.Theorem2.8 と Problem2.10 により,1以上の任意の自然数$N$に対し,
				$[0,N]$上で$B$の修正$W^N$が存在する.
				\begin{align}
					\Omega_N &= \Set{\omega \in \R^{[0,\infty)}}{\forall t \in [0,N] \cap \Q,
					\quad W_t^N(\omega) = B_t(\omega)} \\
					&= \bigcap_{t \in [0,N] \cap \Q} \Set{\omega \in \R^{[0,\infty)}}{W_t^N(\omega) = B_t(\omega)}
				\end{align}
				とおけば,$W^N$は$B$の修正であるから$P(\Omega_N)=1$.ここで
				$\tilde{\Omega} = \bigcap_{N \in \N}\Omega_N$とおく.
				0以上の実数$t$と$\tilde{\Omega}$の要素$\omega$が任意に与えられたとき,
				$t < N$を満たす自然数$N$を取れば,$N$以上の任意の自然数$n$で
				\begin{align}
					\forall s \in [0,N] \cap \Q, \quad
					B_s(\omega) = W^N_s(\omega) \wedge B_s(\omega) = W^n_s(\omega)
				\end{align}
				となり,$W^N(\omega)$と$W^n(\omega)$の連続性と定理
				\ref{thm:equivalence_set_of_two_mappings_into_Hausdorff_space_is_closed}より
				$W^N(\omega)$と$W^n(\omega)$は$[0,N]$上で一致する.すなわち
				\begin{align}
					\forall n \in \N,\quad N \leq n \Longrightarrow W^n_t(\omega)
					= W^N_t(\omega)
				\end{align}
				が成り立つから,このとき$\lim_{n \to \infty} W^n_t(\omega)$が確定する.
				\begin{align}
					W_t(\omega) = 
					\begin{cases}
						\lim_{n \to \infty} W^n_t(\omega), & (\omega \in \tilde{\Omega}), \\
						0, & (\omega \in \R^{[0,\infty)} \backslash \tilde{\Omega})
					\end{cases}
				\end{align}
				で$W$を定めれば,$W$は$B$の修正となる.実際,0以上の任意の実数$t$に対し,
				$t < N$を満たす自然数$N$を取れば
				\begin{align}
					\forall \omega \in \tilde{\Omega},\quad 
					W_t(\omega) = W^N_t(\omega)
				\end{align}
				となり,$W^N$が$B$の修正であるから
				\begin{align}
					P(W_t \neq B_t) \leq P(W_t \neq W^N_t) + P(W^N_t \neq B_t) = 0
				\end{align}
				が成立する.またこの$t$において,$W^N(\omega)$の連続性から$W(\omega)$の$t$での連続性が従う.
				
			\item[第二段]
				前段で定めた$W$が$(\R^{[0,\infty)},\mathscr{B}(\R^{[0,\infty)}),P)$の上の
				Brown運動であることを示す.まず$P$-a.s.に$W_0 = B_0$である.また
				$0 \leq s < t$を満たす任意の実数$s,t$に対し,
				\begin{align}
					\Omega' = \Set{\omega \in \R^{[0,\infty)}}{W_s(\omega) \neq B_s(\omega) \wedge W_t(\omega) \neq B_t(\omega)}
				\end{align}
				とおく.$\borel{\R}$の要素$E,F$が任意に与えられたとして,
				\begin{align}
					W_s^{-1}(F) \cap \Omega' = B_s^{-1}(F) \cap \Omega',
					\quad
					(W_t-W_s)^{-1}(E) \cap \Omega' = (B_t-B_s)^{-1}(E) \cap \Omega'
				\end{align}
				が成り立ち,かつ$P(\Omega') = 1$であるから
				\begin{align}
					P\left( W_s^{-1}(F) \right)
					= P\left( W_s^{-1}(F) \cap \Omega' \right)
					&= P\left( B_s^{-1}(F) \cap \Omega' \right)
					= P\left( B_s^{-1}(F) \right), \\
					P\left( (W_t-W_s)^{-1}(E) \right) &= P\left( (B_t-B_s)^{-1}(E) \right), \label{eq:chapter_2_Corollary_to_Theorem_2_8} \\
					P\left( W_s^{-1}(F) \cap (W_t-W_s)^{-1}(E) \right)
					&= P\left( B_s^{-1}(F) \cap (B_t-B_s)^{-1}(E) \right)
				\end{align}
				が従い,$B$の独立増分性と併せて
				\begin{align}
					P\left( W_s^{-1}(F) \cap (W_t-W_s)^{-1}(E) \right)
					&= P\left( B_s^{-1}(F) \cap (B_t-B_s)^{-1}(E) \right) \\
					&= P\left( B_s^{-1}(F) \right) P\left( (B_t-B_s)^{-1}(E) \right) \\
					&= P\left( W_s^{-1}(F) \right) P\left( (W_t-W_s)^{-1}(E) \right) \\
				\end{align}
				となる.以上で$W$の独立増分性が示された.また
				(\refeq{eq:chapter_2_Corollary_to_Theorem_2_8})から
				$W_t-W_s$の分布は$B_t-B_s$の分布に一致する.
				\QED
		\end{description}
	\end{prf}
\section{The Space $C[0,\infty)$, Weak Convergence, and the Wiener Measure}
	\begin{itembox}[l]{Problem 4.1}
		Show that $\rho$ defined by (4.1) is a metric on $C[0,\infty)^d$ and, under $\rho$, 
		$C[0,\infty)^d$ is a complete, separable metric space.
	\end{itembox}
	以下,$C[0,\infty)^d$には$\rho$により広義一様収束位相を導入する.
	
	\begin{prf}
		付録の定理\ref{thm:appendix_complete_separability_of_spaces_of_continuous_functions}により従う.
		\QED
	\end{prf}

\begin{itembox}[l]{Problem 4.2}\label{chapter_2_problem_4_2}
	Let $\mathscr{C}(\mathscr{C}_t)$ be the collection of finite-dimensional cylinder sets of the form (2.1); i.e.,
	\begin{align}
		C = \Set{\omega \in C[0,\infty)^d}{(\omega(t_1),\cdots,\omega(t_n)) \in A};
		\quad n \geq 1,\ A \in \borel{(\R^d)^n},
	\end{align}
	where, for all $i=1,\cdots,n,\ t_i \in [0,\infty)$ (respectively, $t_i \in [0,t]$).
	Denote by $\mathscr{G}(\mathscr{G}_t)$ the smallest $\sigma$-field containing $\mathscr{C}(\mathscr{C}_t)$.
	Show that $\mathscr{G} = \borel{C[0,\infty)^d}$, the Borel $\sigma$-field generated by the open sets in
	$C[0,\infty)^d$, and that $\mathscr{G}_t = \varphi_t^{-1}\left( \borel{C[0,\infty)^d} \right) \eqqcolon
	\mathscr{B}_t\left( C[0,\infty)^d \right)$, where $\varphi_t:C[0,\infty)^d \longrightarrow C[0,\infty)^d$ is the
	mapping $(\varphi_t\omega)(s) = \omega(t \wedge s);\ 0 \leq s < \infty$.
\end{itembox}

\begin{prf}\mbox{}
	\begin{description}
		\item[第一段]
			$w_0 \in C[0,\infty)^d$とする.任意に$w \in C[0,\infty)^d$を取れば,$w$の連続性により$d(w_0,w)$の各項について
			\begin{align}
				\sup{t \leq n}{|w_0(t) - w(t)|} = \sup{r \in [0,n]\cap\Q}{|w_0(r) - w(r)|} \quad (n = 1,2,\cdots)
			\end{align}
			とできる.いま,任意に実数$\alpha \in \R$を取れば
			\begin{align}
				\Set{w \in C[0,\infty)^d}{\sup{r \in [0,n]\cap\Q}{|w_0(r) - w(r)|} \leq \alpha}
				= \bigcap_{r \in [0,n]\cap\Q} \Set{w \in C[0,\infty)^d}{|w_0(r) - w(r)| \leq \alpha}
			\end{align}
			が成立し,右辺の各集合は
			$\mathscr{C}$に属するから$\mbox{左辺} \in \sgmalg{\mathscr{C}}$となる.従って
			\begin{align}
				\psi_n : C[0,\infty)^d \ni w \longmapsto \sup{r \in [0,n]\cap\Q}{|w_0(r) - w(r)|} \in \R, \quad (n = 1,2,\cdots)
			\end{align}
			で定める$\psi_n$は可測$\sgmalg{\mathscr{C}}/\borel{\R}$である.
			$x \longmapsto x \wedge 1$の連続性より$\psi_n \wedge 1$
			も$\sgmalg{\mathscr{C}}/\borel{\R}$-可測性を持ち,
			\begin{align}
				d(w_0,w) = \sum_{n=1}^{\infty}\frac{1}{2^n} \left( \psi_n(w) \wedge 1 \right)
			\end{align}
			により$C[0,\infty)^d \ni w \longmapsto d(w_0,w) \in \R$の
			$\sgmalg{\mathscr{C}}/\borel{\R}$-可測性が出るから,任意の$\epsilon > 0$に対する球について
			\begin{align}
				\Set{w \in C[0,\infty)^d}{d(w_0,w) < \epsilon} \in \sgmalg{\mathscr{C}}
			\end{align}
			が成り立つ.$C[0,\infty)^d$は第二可算公理を満たし,可算基底は上式の形の球で構成されるから,
			$\open{C[0,\infty)^d} \subset \sgmalg{\mathscr{C}}$が従い$\borel{C[0,\infty)^d} \subset \sgmalg{\mathscr{C}}$を得る.
			次に逆の包含関係を示す.いま任意に$n \in \Z_+$と$t_1 < \cdots < t_n$を選んで
			\begin{align}
				\phi : C[0,\infty)^d \ni w \longmapsto (w(t_1),\cdots, w(t_n)) \in (\R^d)^n
			\end{align}
			で定める写像は連続である.
			実際,任意の一点$w_0$での連続性を考えると,任意の$\epsilon > 0$に対して$t_n \leq N$を満たす$N \in \N$を取れば,
			$d(w_0,w) < \epsilon/(n2^N)$ならば
			$\sum_{i=1}^{n}|w_0(t_i) - w(t_i)| < \epsilon$が成り立つ.よって
			$\phi$は$w_0$で連続であり
			\begin{align}
				\borel{(\R^d)^n} \subset \Set{A \in \borel{(\R^d)^n}}{\phi^{-1}(A) \in \borel{C[0,\infty)^d}}
			\end{align}
			が出る.任意の$C \in \mathscr{C}$は,$n \in \N$と時点$t_1 < \cdots < t_n$によって決まる写像$\phi$によって
			$C = \phi^{-1}(B)\ (\exists B \in \borel{(\R^d)^n})$と表現できるから,
			$\mathscr{C} \subset \borel{C[0,\infty)^d}$が成り立ち
			$\sgmalg{\mathscr{C}} \subset \borel{C[0,\infty)^d}$が得られる.
			
		\item[第二段]
			$t \geq 0$とする.$C[0,\infty)^d$の位相を$\open{C[0,\infty)^d}$と書けば
			\begin{align}
				\varphi_t^{-1}\left( \borel{C[0,\infty)^d} \right)
				= \sgmalg{\Set{\varphi_t^{-1}(O)}{O \in \open{C[0,\infty)^d}}}
			\end{align}
			が成り立つ.任意の$\alpha \in \R$と$r \geq 0$に対して
			\begin{align}
				&\Set{w \in C[0,\infty)^d}{|w_0(r) - (\varphi_t w)(r)| \leq \alpha} \\
				&\qquad = \begin{cases}
					\Set{w \in C[0,\infty)^d}{|w_0(r) - (\varphi_t w)(r)| \leq \alpha}, & (r \leq t), \\
					\Set{w \in C[0,\infty)^d}{|w_0(r) - (\varphi_t w)(t)| \leq \alpha}, & (r > t),
				\end{cases}
				\quad \in \mathscr{C}_t
			\end{align}
			となるから
			\begin{align}
				\psi^t_n : C[0,\infty)^d \ni w \longmapsto \sup{r \in [0,n]\cap\Q}{|w_0(r) - (\varphi_t w)(r)|} \in \R, \quad (n = 1,2,\cdots)
			\end{align}
			で定める$\psi^t_n$は可測$\sgmalg{\mathscr{C}_t}/\borel{\R}$である.
			$x \longmapsto x \wedge 1$の連続性より$\psi^t_n \wedge 1$
			も$\sgmalg{\mathscr{C}_t}/\borel{\R}$-可測性を持ち,
			\begin{align}
				d(w_0,\varphi_t w) = \sum_{n=1}^{\infty}\frac{1}{2^n} \left( \psi^t_n(w) \wedge 1 \right)
			\end{align}
			により$C[0,\infty)^d \ni w \longmapsto d(w_0,\varphi_t w) \in \R$の
			$\sgmalg{\mathscr{C}_t}/\borel{\R}$-可測性が出るから,任意の$\epsilon > 0$に対する球について
			\begin{align}
				\Set{w \in C[0,\infty)^d}{d(w_0,\varphi_t w) < \epsilon} \in \sgmalg{\mathscr{C}_t}
			\end{align}
			が成り立つ.特に
			\begin{align}
				\varphi_t^{-1}\left( \Set{w \in C[0,\infty)^d}{d(w_0,w) < \epsilon} \right)
				= \Set{w \in C[0,\infty)^d}{d(w_0,\varphi_t w) < \epsilon}
			\end{align}
			が満たされ,$C[0,\infty)^d$の第二可算性より
			\begin{align}
				\varphi_t^{-1}(O) \in \sgmalg{\mathscr{C}_t},
				\quad (\forall O \in \open{C[0,\infty)^d})
			\end{align}
			が従う.ゆえに$\varphi_t^{-1}\left( \borel{C[0,\infty)^d} \right) \subset \sgmalg{\mathscr{C}_t}$となる.
			\QED
	\end{description}
\end{prf}

\begin{comment}
次の事柄は後の定理の証明で使うからここで証明しておく.
\begin{screen}
	\begin{thm}[$\mathscr{C}$は乗法族である]
		$\mathscr{C}$は交演算について閉じている.
	\end{thm}
\end{screen}
\begin{prf}
	任意に$A_1, A_2 \in \mathscr{C}$を取れば,$A_1,\ A_2$それぞれに対し
	$n_1,n_2 \in \N,\ C_1 \in \borel{(\R^d)^{n_1}},\ C_2 \in \borel{(\R^d)^{n_2}},\ t_1<\cdots<t_{n_1}$それから
	$s_1<\cdots<s_{n_2}$が決まっていて,
	\begin{align}
		A_1 &= \left\{\ w \in C[0,\infty)^d\quad |\quad (w(t_1), \cdots, w(t_{n_1})) \in C_1\ \right\} \\
		A_2 &= \left\{\ w \in C[0,\infty)^d\quad |\quad (w(s_1), \cdots, w(s_{n_2})) \in C_2\ \right\}
	\end{align}
	と表されている.$A_1,A_2$の時点に重複があるかないかで場合分けして示す.
	\begin{description}
	\item[時点に重複がない場合]
		集合を次のように同値な表記に直す:
		\begin{align}
			A_1 &= \left\{\ w \in C[0,\infty)^d\quad |\quad (w(t_1), \cdots, w(t_{n_1}),w(s_1), \cdots, w(s_{n_2})) \in C_1 \times (\R^d)^{n_2}\ \right\} \\
			A_2 &= \left\{\ w \in C[0,\infty)^d\quad |\quad (w(t_1), \cdots, w(t_{n_1}),w(s_1), \cdots, w(s_{n_2})) \in (\R^d)^{n_1} \times C_2\ \right\}
		\end{align}
		表現を変えれば乗法を考えやすくなり,上の場合は
		\begin{align}
			A_1 \cap A_2 = \left\{\ w \in C[0,\infty)^d\quad |\quad (w(t_1), \cdots, w(t_{n_1}),w(s_1), \cdots, w(s_{n_2})) \in C_1 \times C_2\ \right\}
		\end{align}
		と表現できる.$t_1,\cdots,s_{n_2}$の並びが気になるなら,この時点の並びを昇順に変換する$(dn_1 + dn_2) \times (dn_1 + dn_2)$行列$J_1$
		を用いて($J_1$は連続,線型,全単射),
		\begin{align}
			A_1 \cap A_2 &= \left\{\ w \in C[0,\infty)^d\quad |\quad J_1\Vector{w} \in J_1(C_1 \times C_2)\ \right\} \\
			\left(\Vector{w} \right.&= {}^{T}(w(t_1), \cdots, \left.w(t_{n_1}),w(s_1), \cdots, w(s_{n_2}))\right)
		\end{align}
		とすれば,$J(C_1 \times C_2) \in \borel{(\R^d)^{n_1 + n_2}}$であるから,$A_1 \cap A_2 \in \mathscr{C}$であることが明確になる.
	
	\item[時点に重複がある場合]
		$(r_{k_1},\cdots,r_{k_l}) \subset (t_1,\cdots,t_{n_1})$が重複時点であるとき,$A_1,A_2$の同値な表記は次のようにすればよい:
		\begin{align}
			A_1 &= \left\{\ w \in C[0,\infty)^d\quad |\quad (w(t_1),.,w(r_{k_1}),.,w(r_{k_l}),.,w(t_{n_1}),\mbox{\scriptsize($s_1, \cdots, s_{n_2}$から$r_{k_1},\cdots,r_{k_l}$を抜いたものを並べる)}) \in C_1 \times (\R^d)^{n_2 - l}\ \right\} \\
			A_2 &= \left\{\ w \in C[0,\infty)^d\quad |\quad (w(s_1),.,w(r_{k_1}),.,w(r_{k_l}),.,w(s_{n_2}),\mbox{\scriptsize($t_1, \cdots, t_{n_1}$から$r_{k_1},\cdots,r_{k_l}$を抜いたものを並べる)}) \in C_2 \times (\R^d)^{n_1 - l}\ \right\}
		\end{align}
		$A_2$について,条件中の時点の並びを変換し$A_1$の条件の順番に合わせる行列$J_2$(連続,線型,全単射)を用いて
		\begin{align}
			A_2 = \left\{\ w \in C[0,\infty)^d\quad |\quad (w(t_1),.,w(r_{k_1}),.,w(r_{k_l}),.,w(t_{n_1}),\mbox{\scriptsize($s_1, \cdots, s_{n_2}$から$r_{k_1},\cdots,r_{k_l}$を抜いたものを並べる)}) \in J_2(C_2 \times (\R^d)^{n_1 - l})\ \right\}
		\end{align}
		と書き直せば,$A_1 \cap A_2$は前段の様に表現可能であり,前段と同様に最後に時点を昇順に変換する行列を用いることで$A_1 \cap A_2 \in \mathscr{C}$となることが明確に判る.
	\end{description}
	\QED
\end{prf}
\end{comment}
\section{Weak Convergence}
	いま,$X$を局所コンパクトHausdorff空間として
	\begin{align}
		C_0(X) \coloneqq \Set{f:X \longrightarrow \C}{\mbox{連続かつ,任意の$\epsilon > 0$に対し
		$\closure{\Set{x \in X}{|f(x)| \geq \epsilon}}$がコンパクト}} 
	\end{align}
	とおく.この$C_0(X)$はノルム$\Norm{f}{C_0(X)} \coloneqq \sup{x \in X}{|f(x)|}$
	により複素Banach空間となる.また
	$(X,\borel{X})$上の複素測度$\mu$について,その総変動$|\mu|$が正則測度であるとき
	$\mu$は正則であるという.$X$上の正則複素測度の全体を$RM(X)$と書き,
	総変動ノルム$\Norm{\mu}{RM(X)} \coloneqq |\mu|(X)$によりノルム位相を導入する.
	任意の複素測度$\mu$に対し
	\begin{align}
		\Phi_\mu(f) \coloneqq \int_X f(x)\ \mu(dx)
	\end{align}
	により$C_0(X)$上の有界線型汎関数$\Phi_\mu$が定まる.
	
	\begin{screen}
		\begin{thm}[Rieszの表現定理]
			$X$を局所コンパクトHausdorff空間とする.
			$C_0(X)$に$\Norm{\cdot}{C_0(X)}$で位相を入れるとき,
			共役空間$C_0(X)^*$と書く.
			このとき$C_0(X)^*$と$RM(X)$は
			\begin{align}
				\Phi:RM(X) \ni \mu \longrightarrow \Phi_\mu \in C_0(X)
			\end{align}
			で定める対応関係$\Phi$によりBanach空間として等長同型となる.
		\end{thm}
	\end{screen}
	
	$C_0(X)^*$に汎弱位相を入れるとき,
	汎関数列$\left( \Phi_{\mu_n} \right)_{n=1}^\infty$が
	$\Phi_\mu$に汎弱収束することと
	\begin{align}
		\Phi_{\mu_n}(f) \longrightarrow \Phi_\mu(f)\ (n \longrightarrow \infty),
		\quad (\forall f \in C_0(X))
	\end{align}
	は同値になる.$C_0(X)^*$の汎弱位相の$\Phi$による逆像位相を
	$RM(X)$の弱位相と定めれば,$\Phi$は弱位相に関して位相同型となる.
	このとき,$(\mu_n)_{n=1}^\infty$が$\mu$に弱収束することは
	$\left( \Phi_{\mu_n} \right)_{n=1}^\infty$が$\Phi_\mu$に汎弱収束することと同値になり,
	すなわち
	\begin{align}
		\int_X f(x)\ \mu_n(dx) \longrightarrow \int_X f(x)\ \mu(dx)\ (n \longrightarrow \infty),
		\quad (\forall f \in C_0(X))
	\end{align}
	と同値になる.$X$上の正則な確率測度の全体を$\mathscr{P}(X)$と書けば
	$\mathscr{P}(X) \subset RM(X)$となり,正則確率測度の列$(P_n)_{n=1}^\infty$が
	$P \in \mathscr{P}(X)$に弱収束することは
	\begin{align}
		\int_X f(x)\ P_n(dx) \longrightarrow \int_X f(x)\ P(dx)\ (n \longrightarrow \infty),
		\quad (\forall f \in C_0(X))
	\end{align}
	と同値になる.
	
	\begin{itembox}[l]{Definition 4.3}
		It follows, in particular, that the weak limit $P$ is a probability measure, 
		and that it is unique.
	\end{itembox}
	
	\begin{prf}
		$f \equiv 1$として
		\begin{align}
			P(S) = \lim_{n \to \infty} P_n(S) = 1
		\end{align}
		が従うから$P$は確率測度である.また任意の有界連続関数$f:S \longrightarrow \R$に対し
		\begin{align}
			\int_S f\ dP = \int_S f\ dQ
		\end{align}
		が成り立つとき,任意の閉集合$A \subset S$に対して
		\begin{align}
			f_k(s) \coloneqq \frac{1}{1 + k d(s,A)},
			\quad (k=1,2,\cdots)
		\end{align}
		と定めれば$\lim_{k \to \infty} f_k = \defunc_A$(各点収束)が満たされるから,Lebesgueの収束定理より
		\begin{align}
			P(A) = \lim_{k \to \infty} \int_S f_k\ dP
			= \lim_{k \to \infty} \int_S f_k\ dQ
			= Q(A)
		\end{align}
		となり,測度の一致の定理より$P = Q$が得られる.すなわち弱極限は一意である.
		\QED
	\end{prf}
	
	\begin{itembox}[l]{lemma: change of variables for expectation}
		$(\Omega,\mathscr{F},P)$を確率空間,
		$(S,\mathscr{S})$を可測空間とする.
		このとき任意の
		有界$\mathscr{S}/\borel{\R}$-可測関数$f$
		と$\mathscr{F}/\mathscr{S}$-可測写像$X$に対して
		\begin{align}
			\int_\Omega f(X)\ dP = \int_S f\ dPX^{-1}
		\end{align}
		が成立する.
	\end{itembox}
	
	\begin{prf}
		任意の$A \in \mathscr{S}$に対して
		\begin{align}
			\int_S \defunc_A dPX^{-1}
			= P(X^{-1}(A))
			= \int_\Omega \defunc_{X^{-1}(A)}\ dP
			= \int_\Omega \defunc_{A}(X)\ dP
		\end{align}
		が成り立つから,任意の$\mathscr{S}/\borel{\R}$-可測単関数$g$に対し
		\begin{align}
			\int_\Omega g(X)\ dP = \int_S g\ dPX^{-1}
		\end{align}
		となる.$f$が有界なら一様有界な単関数で近似できるので,Lebesgueの収束定理より
		\begin{align}
			\int_\Omega f(X)\ dP = \int_S f\ dPX^{-1}
		\end{align}
		が出る.
		\QED
	\end{prf}
	
	\begin{itembox}[l]{Definition 4.4}
		Equivalently, $X_n \overset{\mathscr{D}}{\longrightarrow} X$ if and only if
		\begin{align}
			\lim_{n \to \infty} E_n f(X_n) = E f(X)
		\end{align} 
		for every bounded, continuous real-valued function $f$ on $S$, 
		where $E_n$ and $E$ denote expectations with respect to $P_n$ and $P$, respectively.
	\end{itembox}
	
	\begin{prf}
		任意の有界実連続関数$f:S \longrightarrow \R$に対し
		\begin{align}
			\int_\Omega f(X_n)\ dP_n = \int_S f\ dP_nX_n^{-1},
			\quad \int_\Omega f(X)\ dP = \int_S f\ dPX^{-1},
		\end{align}
		が成り立つから,$P_nX_n^{-1}$が$PX^{-1}$に弱収束することと
		$\lim_{n \to \infty} E_n f(X_n) = E f(X)$は同値である.
		\QED
	\end{prf}
	
	\begin{itembox}[l]{Problem 4.5}
		Suppose $\{X_n\}_{n=1}^\infty$ is a sequence of random variables taking values 
		in a metric space $(S_1,\rho_1)$ and converging in distribution to $X$. Suppose 
		$(S_2,\rho_2)$ is another metric space, and $\varphi:S_1 \longrightarrow S_2$ 
		is continuous. Show that $Y_n \coloneqq \varphi(X_n)$ converges in distribution 
		to $Y \coloneqq \varphi(X)$.
	\end{itembox}
	
	\begin{prf}
		任意の有界実連続関数$f:S_2 \longrightarrow \R$に対し
		$f \circ \varphi$は$S_1$上の有界実連続関数であるから
		\begin{align}
			\int_{S_2} f\ dPY_n^{-1}
			&= \int_{\Omega} f(Y_n)\ dP
			= \int_{\Omega} f(\varphi(X_n))\ dP
			= \int_{S_1} f \circ \varphi\ dPX_n^{-1} \\
			& \longrightarrow 
			\int_{S_1} f \circ \varphi\ dPX^{-1}
			= \int_{S_2} f\ dPY^{-1}
			\quad (n \longrightarrow \infty)
		\end{align}
		が成立する.
		\QED
	\end{prf}
	
\section{Tightness}
	テキスト本文において$m^T(\omega,\delta)$は
	\begin{align}
		m^T(\omega,\delta) \coloneqq \operatorname*{max}_{\substack{|s-t| \leq \delta \\ 0 \leq s,t \leq T}}|\omega(s) - \omega(t)|
	\end{align}
	で定められるが,$\operatorname{max}$と書いて妥当であることを確認しておく.
	まず
	\begin{align}
		D \coloneqq \Set{(s,t) \in \R \times \R}{|s-t| \leq \delta \wedge 0 \leq s,t \leq T}
	\end{align}
	で定められる集合は$\R \times \R$のコンパクト集合である.そして$\omega$は連続写像であるから
	\begin{align}
		\R \times \R \ni (s,t) \longmapsto \omega(s), 
		\quad \R \times \R \ni (s,t) \longmapsto \omega(t)
	\end{align}
	は共に実連続写像である.引き算は連続,絶対値も連続であるから
	\begin{align}
		\R \times \R \ni (s,t) \longmapsto |\omega(s) - \omega(t)|
	\end{align}
	は$\R \times \R$から$\R$への連続写像であり,$D$のコンパクト性から$D$上で最大値を取る.
	
	\begin{itembox}[l]{Problem 4.8}
		Show that $m^T(\omega,\delta)$ is continuous in $\omega \in C[0,\infty)$ under the metric
		$\rho$ of (4.1), is nondecreasing in $\delta$, and 
		$\lim_{\delta \downarrow 0}m^T(\omega,\delta) = 0$ for each $\omega \in C[0,\infty)$.
	\end{itembox}
	
	\begin{sketch}\mbox{}
		\begin{description}
			\item[第一段]
				$m^T(\omega,\delta)$が$\omega$に関して連続であることを示す.まず大雑把に,
				\begin{align}
					\left|\, \operatorname*{max}_{x} |f(x)| - \operatorname*{max}_{x} |g(x)|\, \right|
					\leq \operatorname*{max}_{x} |f(x) - g(x)|
				\end{align}
				が成立する.実際,
				\begin{align}
					\operatorname*{max}_{x} |f(x)| - \operatorname*{max}_{x} |g(x)|
					\leq \operatorname*{max}_{x} |f(x) - g(x)|
				\end{align}
				が成り立つことを確認するには
				\begin{align}
					|f(x_1)| = \operatorname*{max}_{x} |f(x)|
				\end{align}
				なる$x_1$を取り,
				\begin{align}
					\operatorname*{max}_{x} |f(x)| - \operatorname*{max}_{x} |g(x)|
					&= |f(x_1)| - \operatorname*{max}_{x} |g(x)| \\
					&\leq |f(x_1)| - |g(x_1)| \\
					&\leq |f(x_1) - g(x_1)| \\
					&\leq \operatorname*{max}_{x} |f(x) - g(x)|
				\end{align}
				となることを見ればよい.$f,g$を入れ替えれば
				\begin{align}
					\operatorname*{max}_{x} |g(x)| - \operatorname*{max}_{x} |f(x)|
					\leq \operatorname*{max}_{x} |f(x) - g(x)|
				\end{align}
				も成り立つから当初の主張を得る.よって$\omega_1,\omega_2$を$C[0,\infty)$の要素とすれば
				\begin{align}
					\left| m^T(\omega_1,\delta) - m^T(\omega_2,\delta) \right|
					\leq \operatorname*{max}_{\substack{|s-t| \leq \delta \\ 0 \leq s,t \leq T}}
					|(\omega_1(s) - \omega_1(t)) - (\omega_2(s) - \omega_2(t))|
				\end{align}
				が成立する.ところで,いま$\epsilon$を任意に与えられた正数とし,
				\begin{align}
					T \leq n
				\end{align}
				を満たす自然数$n$を取り
				\begin{align}
					\rho(\omega_1,\omega_2) < 2^{-n} \epsilon
				\end{align}
				が満たされていると仮定すれば,
				\begin{align}
					\operatorname*{sup}_{0 \leq t \leq n}|\omega_1(t) - \omega_2(t)|
					< \epsilon
				\end{align}
				となるから
				\begin{align}
					0 \leq t \leq T \Longrightarrow |\omega_1(t) - \omega_2(t)| < \epsilon
				\end{align}
				が満たされる.このとき
				\begin{align}
					0 \leq s,t \leq T \Longrightarrow 
					&|(\omega_1(s) - \omega_1(t)) - (\omega_2(s) - \omega_2(t))| \\
					&\leq |\omega_1(s) - \omega_2(s)| + |\omega_1(t) - \omega_2(t)| \\
					&< 2\epsilon
				\end{align}
				が成り立つので
				\begin{align}
					\left| m^T(\omega_1,\delta) - m^T(\omega_2,\delta) \right| < 2\epsilon
				\end{align}
				が従い,$m^T(\omega,\delta)$の$\omega$に関する連続性が得られた.
			
			\item[第二段]
				$\delta$に関して非減少であることを示す.いま$0 < \delta \leq \delta'$とする.
				\begin{align}
					(s,t) \longmapsto |\omega(s) - \omega(t)|
				\end{align}
				は
				\begin{align}
					\Set{(s,t)}{|s-t| \leq \delta \wedge 0 \leq s,t \leq T}
				\end{align}
				の上で最大値を取るのであるから,
				\begin{align}
					|\tilde{s} - \tilde{t}| \leq \delta \wedge
					0 \leq \tilde{s}, \tilde{t} \leq T
				\end{align}
				かつ
				\begin{align}
					|\omega(\tilde{s}) - \omega(\tilde{t})| = m^T(\omega,\delta)
				\end{align}
				を満たす$\tilde{s},\tilde{t}$を取ることが出来るが,
				\begin{align}
					|\tilde{s} - \tilde{t}| \leq \delta'
				\end{align}
				も満たされるので
				\begin{align}
					|\omega(\tilde{s}) - \omega(\tilde{t})|
					\in \Set{|\omega(s) - \omega(t)|}{|s - t| \leq \delta \wedge 0 \leq s, t \leq T}
				\end{align}
				となり
				\begin{align}
					|\omega(\tilde{s}) - \omega(\tilde{t})| \leq m^T(\omega,\delta')
				\end{align}
				が従う.よって
				\begin{align}
					\delta \leq \delta' \Longrightarrow m^T(\omega,\delta) \leq m^T(\omega,\delta')
				\end{align}
				が示された.
			
			\item[第三段]
				$\lim_{\delta \downarrow 0}m^T(\omega,\delta) = 0$が成り立つことを示す.
				$\epsilon$を任意に与えられた正数とする.$\omega$は$[0,T]$上で一様連続となるので
				\begin{align}
					|s-t| \leq \delta \Longrightarrow |\omega(s) - \omega(t)| < \epsilon
				\end{align}
				を満たす正数$\delta$が取れるが,このとき
				\begin{align}
					\delta' \leq \delta
				\end{align}
				を満たす任意の正数$\delta'$に対しても
				\begin{align}
					|s-t| \leq \delta' \Longrightarrow |\omega(s) - \omega(t)| < \epsilon
				\end{align}
				となるから
				\begin{align}
					\lim_{\delta \downarrow 0}m^T(\omega,\delta) = 0
				\end{align}
				が得られる.
				\QED
		\end{description}
	\end{sketch}
	
	\begin{itembox}[l]{Theorem 4.10}
	\end{itembox}
	
	\begin{sketch}\mbox{}
		\begin{description}
			\item[第一段]
				$\eta$を任意に与えられた正数とする.
				$\{P_n\}_{n=1}^\infty$は緊密なので,$C[0,\infty)$の或るコンパクト部分集合$K$が存在して
				\begin{align}
					\forall n \in \N\, (\, 1 - \eta \leq P_n(K)\, )
				\end{align}
				が満たされる.他方で十分大きな正数$\lambda$を取れば
				\begin{align}
					\forall \omega \in K\, (\, |\omega(0)| \leq \lambda\, )
				\end{align}
				となる.これはすなわち
				\begin{align}
					K \subset \Set{\omega}{|\omega(0)| \leq \lambda}
				\end{align}
				を表し,
				\begin{align}
					\forall n \in \N\, \left(\, P_n\Set{\omega}{\lambda < |\omega(0)|}
					\leq P_n(C[0,\infty) \backslash K) \leq \eta\, \right)
				\end{align}
				が従う.また$T,\epsilon$を任意に与えられた正数とすれば,或る正数$\delta_0$が存在して
				\begin{align}
					0 < \delta \leq \delta_0
					\Longrightarrow \forall \omega \in K\, \left(\, m^T(\omega,\delta) \leq \epsilon\, \right)
				\end{align}
				が成立する.つまり
				\begin{align}
					0 < \delta \leq \delta_0
					\Longrightarrow K \subset \Set{\omega}{m^T(\omega,\delta) \leq \epsilon}
				\end{align}
				が成り立つので,
				\begin{align}
					0 < \delta \leq \delta_0
					\Longrightarrow \forall n \in \N\, \left(\, P_n\Set{\omega}{\epsilon < m^T(\omega,\delta)}
					\leq P_n(C[0,\infty) \backslash K) \leq \eta\, \right)
				\end{align}
				が満たされる.
				
			\item[第二段]
		\end{description}
	\end{sketch}
	
	\begin{itembox}[l]{Problem 4.12}
		Suppose $\{P_n\}_{n=1}^\infty$ is a sequence of probability measures on
		$\left( C[0,\infty),\borel{C[0,\infty)} \right)$ which converges weakly to a probability
		measure $P$. Suppose, in addition, that $\{f_n\}_{n=1}^\infty$ is a uniformly bounded sequence
		of real-valued, continuous functions on $C[0,\infty)$ converging to a continuous function $f$,
		the convergence being uniform on compact subsets of $C[0,\infty)$. Then
		\begin{align}
			\lim_{n \to \infty} \int_{C[0,\infty)} f_n(\omega)\ dP_n(\omega)
			= \int_{C[0,\infty)} f(\omega)\ dP(\omega).
		\end{align}
	\end{itembox}
	
	\begin{sketch}\mbox{}
		\begin{description}
			\item[第一段]
				$\{f_n\}_{n=1}^\infty$は一様有界なので
				\begin{align}
					\forall b \in \N\, \forall \omega \in C[0,\infty)\,
					\left(\, |f_n(\omega)| < b\, \right)
				\end{align}
				を満たす正数$b$が存在する.$C[0,\infty)$の各点$\omega$で
				\begin{align}
					f_n(\omega) \longrightarrow f(\omega)\quad (n \longrightarrow \infty)
				\end{align}
				となるから
				\begin{align}
					\forall \omega \in C[0,\infty)\, (\, |f(\omega)| < b\, )
				\end{align}
				が満たされる.すなわち$f$は有界連続であり,$(P_n)_{n=1}^\infty$が$P$に弱収束するので
				\begin{align}
					\lim_{n \to \infty} \int_{C[0,\infty)} f\ dP_n
					= \int_{C[0,\infty)} f\ dP
				\end{align}
				が成立する.
				
			\item[第二段]
				前段の結果より
				\begin{align}
					\left| \int_{C[0,\infty)} f\ dP_n
					- \int_{C[0,\infty)} f\ dP\right|
					\longrightarrow 0 \quad (n \longrightarrow \infty)
				\end{align}
				が成り立つから,
				\begin{align}
					\left|\int_{C[0,\infty)} f_n\ dP_n
					- \int_{C[0,\infty)} f\ dP_n\right|
					\longrightarrow 0 \quad (n \longrightarrow \infty)
					\label{eq:chapter_2_Problem_4_12}
				\end{align}
				が成り立つことを示せば定理の主張が得られる.
				$\{P_n\}_{n=1}^\infty$は相対コンパクトであるからProhorovの定理より緊密である.
				いま$\epsilon$を任意に与えられた正数とすると,$C[0,\infty)$の或るコンパクト部分集合$K$が存在して
				\begin{align}
					\forall n \in \N\, \left(\, P_n(C[0,\infty) \backslash K) < \epsilon\, \right)
				\end{align}
				となる.他方で$K$上で$(f_n)_{n=1}^\infty$は$f$に一様収束するので,
				或る自然数$N$を取れば
				\begin{align}
					\forall n \in \N\, \left(\, N \leq n
					\Longrightarrow \forall \omega \in K\, (\, |f_n(\omega) - f(\omega)| < \epsilon\, )\, \right)
				\end{align}
				が満たされる.このとき
				\begin{align}
					N \leq n \Longrightarrow
					&\left| \int_{C[0,\infty)} f_n\ dP_n
					- \int_{C[0,\infty)} f\ dP_n\right| \\
					&\leq \int_{C[0,\infty)} |f_n - f|\ dP_n \\
					&\leq \int_K |f_n - f|\ dP_n + \int_{C[0,\infty) \backslash K} |f_n - f|\ dP_n \\ \\
					&< \epsilon P_n(K) + 2b P_n(C[0,\infty) \backslash K) \\
					&< (1+2b) \epsilon
				\end{align}
				が成り立つので(\refeq{eq:chapter_2_Problem_4_12})が示された.
				\QED
		\end{description}
	\end{sketch}

\appendix
\chapter{}
	%\section{共鳴箱定理}
	始めに,次の傲岸不遜な主張をアンサイクロペディアより引用する.
	
	\begin{screen}
		\begin{prp}[Andr$\acute{\mathrm{e}}$ Weilの共鳴箱定理]
			理論をつくるのが一流、あとの学者は(ry.
		\end{prp}
	\end{screen}
	この命題と個人的な事情から次の系を得る:
	`私は共鳴箱にすらなれない腑抜けである'.
	\begin{prf}
		私はヘタレて院を中退するので金輪際学者になりえない.
		\QED
	\end{prf}
\section{集合論理}
		\monologue{
		院生「現代的な数学では,数や関数など数学に関するあらゆるものは集合で構成されます.
			そして集合そのものは述語論理を基礎にして公理的に規定されます.
			この意味で集合論の勉強には論理学の知識が必要であると聞きますけれども,
			真に受けて論理学の本を眺めてみれば,はじめから集合そのものが出てきたり,
			変数に数で添え字をつけたり,述語関数などといったものを取り扱っていたりしているものばかりで残念です.
			論理学を基に集合論を展開しようというのですから,集合論の諸概念を予定して論理学を説明するのは本末転倒です.
			とはいえ集合論と論理学とは切っても切り離せないのですから,いっそ同時並行でそつなく理解してやりましょう.
			(いわゆるメタ数学についてはいまのところ手を出すつもりはありません.)」
	}
\subsection{言語}
	\begin{quote}
		初めに言(ことば)があった。言は神と共にあった。言は神であった。\\
		この言は、初めに神と共にあった。\\
		万物は言によって成った。成ったもので、言によらずに成ったものは何一つなかった。
	\end{quote}
	ヨハネによる福音書の冒頭である.本稿の世界もまた数学のことば,言い換えれば論理のみによって創られる(予定).

	\monologue{
		院生「私の指導教官に``新約聖書がはじめにギリシア語で書かれたとき,`ことば'にはlogosが充てられた.
			logosは`言語'の意味を持つと同時に`論理'の意味も持つ''と教わりました.
			つまり,ギリシア語版の福音書では``初めに論理があった''とも解釈できるのですね.
			一方で日本語訳では言葉ではなく言と書かれています.なぜ``言葉''ではなく``言''と書くのでしょうか.
			一説によれば言葉の葉の字の由来は万葉古今集仮名序にあり,
			現代的に説明すれば,見聞きしたり感動したりしたところを種にして生じる語彙のことを木の葉に喩えているらしいです.
			言葉は人が発するものであり,たいていの場合食い違いなく通用するのですから,すなわち
			葉が付かない``言''とは,人為の介入する前から世界を認識し,人が自覚する前から人の心に通底している
			コードとでも解釈されるでしょうか.聖書の引用文の通り%は森羅万象はことばによって成り,ことばによって尽くされるという意味であるから,
			キリスト教においてことばとは神であり森羅万象を超越しているのですから,言の字に神性を伴わせても良いですよね.
			本稿の世界もまた数学のことばによって創られますが,``はじめにことばありき''の名句が国籍や文化を問わず
			現代まで受け入れられてきたという事実を鑑みれば,ことばから始めようというのは人が生来持っている直観に対して自然な起こりなのでしょう.」
			%しかしながら,神なることばが世界の悉くを尽くせる一方で,人が創造する数学の世界は論理のみによっては完結し得ないという事実もあります.
	}
	
	\monologue{
		院生「集合論の言語の設定は思いの外厄介ですね.いや,私にとって厄介というだけですが.
			一旦言語を設定してはみるものの,行き詰れば設定をやり直すことになりますから,
			以下記述する内容はあくまで仮の形です.それから,私自身集合論も論理学も
			ド素人ですから,言語に対する認識が専門家とズレていることも十分あり得ます.
			勉強を進める中で自分の誤解に気付けばその時点で全てやり直しです.
			見る人が見れば滑稽千万な破綻が見つかるかもしれませんが,
			しかし私としてはHilbertの形式主義,つまり文字と特殊記号を一定の法則で並べただけの
			無意味な記号列に対して推論規則や公理により形式上の意味を付けるという姿勢を
			貫いているつもりです.予防的な言い訳はこの程度にして,本論に入りましょう.」
	}
	
	言語における使用文字と特殊記号は以下に指定するものである:
	\begin{description}
		\item[使用文字] 自然言語から借用する文字は表にあるものに限る.
		\item[述語記号] $=,\ \in$
		\item[論理記号] $\bot,\ \Longrightarrow,\ \wedge,\ \vee,\ \rightharpoondown$
		\item[量化記号] $\forall,\ \exists$
		\item[補助記号] $[\ ,\ ]\ ,\ (\ ,\ )\ ,\ \{\ ,\ \}\ ,\ <\ ,\ >\ ,\ |$
	\end{description}
	
	日常言語において,``あmt後右所sごぐふぉsdあじお''のように無作為に文字を並べただけでは意味不明な
	文字列が出来上がる.文字列は,何らかの規則に従って並ぶことで単語や文章として成立するのである.
	数学も同じで,一定の規則に従って並ぶ記号列のみを数学における文章として扱う.
	述語記号とは,今のところは文字同士を繋ぎ最小単位の文章を成すものとする.例えば,文字$s,t$に対し
	\begin{align}
		s \in t
	\end{align}
	は数学の文章となり,日本語には``$s$は$t$の要素である''と翻訳される.
	数学における文章を{\bf 式}\index{しき@式}
	{\bf (formula)}或は{\bf 論理式}\index{ろんりしき@論理式}と呼ぶ.
	論理記号とは式同士を繋ぐ役割を持つ.
	
	式の構成法を形式的に書き直すと次のようになる.
	\begin{description}
		\item[式] 
			\begin{itemize}
				\item $\bot$は$\mathcal{L}$の式である.
				
				\item 文字$s,t$に対して,$s=t,\ s \in t$は式である.
					
				\item $A,B$を式とするとき,
					$A$では量化されていないが$B$で量化されているといった文字が無いときに限り,
					$(A) \wedge (B),\ (A) \vee (B),\ (A)\Longrightarrow (B)$は式である.
				
				\item $A$を式とするとき,$\rightharpoondown (A)$は式である.
				
				\item $A$を式とするとき,文字$x$が$A$に現れ,かつ$x$が$A$で量化されていないときに限り
					$\forall x (A),\ \exists x (A)$は式である.
				
				\item 以上の操作を繰り返して得られる記号列のみが式である.
					ただし,繰り返しの操作は無制限に行われるものではない.
					得られる記号列は左端から辿っていけば必ず右端が見つかるものとする.
			\end{itemize}
	\end{description}
	
	\monologue{
		院生「`$A$では量化されていないが$B$で量化されているといった文字が無いときに限り'という
			制限は何のためにあるのでしょうか.例えばこの制限を外すと
			\begin{align}
				\forall x ((x \in x) \vee (\forall y (\exists x ( y = x ))))
			\end{align}
			は式となりますが,同じ式で文字$x$は二回量化されています.
			同じ文字が複数回量化されてしまうと式を解釈するときに厄介なので,
			そのような状況を排除するために制限を課しているのですね.
			では,`以上の操作を繰り返して得られる記号列のみが式である'はどういう意味でしょうか.
			例えば,最後の制限を外してしまうと
			\begin{align}
				\exists (\rightharpoondown (\exists x(\forall y (x = y))))
			\end{align}
			という記号列が式であるか式でないかは判別できませんが,
			最後の規制によりこれは式ではないと判断できます.
			また`得られる記号列は左端から辿っていけば必ず右端が見つかるものとする'というのは,
			式の長さは有限であるということを伝えているのですね.しかし未だ有限とは何かを
			規定していないのでこう書くほか術が見当たらないのです.」
	}
	
	\monologue{
		院生「式の定義では,始めに最も簡単な形の式($\bot$や$s=t$)を提示して,
			以降の段階で新しい式を作り出す手段(論理記号による式の接合)を指定しています.
			このような定義を{\bf 帰納的な定義}\index{きのうてきなていぎ@帰納的な定義}{\bf (inductive definition)}と呼びます.
			プログラミングで言うところのfor文の操作と同じですね.
			また既に量化されている文字が再び量化されるということは起こり得ません.」
	}
	
	$A$を式とする.
	$A$に$a$という文字が現れるとき,$A$に現れる全ての$a$を$x$に置き換えた式を
	\begin{align}
		(x \mid a)\, A
	\end{align}
	で表す.特に$A$に現れる文字で量化されていないものが$a$のみであるとき,
	$(x \mid a)\, A$を
	\begin{align}
		A(x)
	\end{align}
	で表す.このとき式$A$自体は$(a \mid a)\, A$とも$A(a)$とも書ける.
	
	いま文字から成る式を作ったが,例えば$x$のみが量化されていない式$A$に対して
	\begin{align}
		\Set{x}{A(x)}
	\end{align}
	という記法を導入し,これも文字同様に
	\begin{align}
		s \in \Set{x}{A(x)},\quad t = \Set{x}{A(x)}
	\end{align}
	などと式に組み込んで扱いたい.そこで{\bf 対象}\index{たいしょう@対象}{\bf (individual)}
	と{\bf 項}\index{こう@項}{\bf (term)}という概念を使う.
	
	\begin{description}
		\item[対象]
			\begin{itemize}
				\item 式$A$において文字$x$が現れ,かつ$x$のみが$A$で量化されていないとき,
					\begin{align}
						\Set{x}{A(x)}
					\end{align}
					は対象である.
					
				\item 式$A$において文字$x$が現れ,かつ$x$のみが$A$で量化されていないとき,
					\begin{align}
						\varepsilon x A(x)
					\end{align}
					は対象である.
			\end{itemize}
			
		\item[項] 対象は項である.文字も項である.またこれらのみが項である.
	\end{description}
	
	\monologue{
		院生「唐突に出てきた$\varepsilon x A(x)$という記号列は何なのでしょう.
			後述することですが,これは$\forall$と$\exists$の意味を
			公理化するための方便として導入するものなのです.ちなみに,以下で式を拡張することにより
			\begin{align}
				A(\varepsilon x A(x))
			\end{align}
			という形の記号列も式として扱えることになります.」
	}
	
	項を用いて,先ほど定義した式を拡張する.
	\begin{description}
		\item[式] 
			\begin{itemize}
				\item $\bot$は$\mathcal{L}$の式である.
				
				\item $s,t$を項とするとき,$s=t,\ s \in t$は式である.
					
				\item $A,B$を式とするとき,
					$A$では量化されていないが$B$で量化されているといった文字が無いときに限り,
					$(A) \wedge (B),\ (A) \vee (B),\ (A)\Longrightarrow (B)$は式である.
				
				\item $A$を式とするとき,$\rightharpoondown (A)$は式である.
				
				\item $A$を式とするとき,文字$x$が$A$に現れ,かつ$x$が$A$で量化されていないときに限り
					$\forall x (A),\ \exists x (A)$は式である.
				
				\item 以上の操作を繰り返して得られる記号列のみが式である.
					ただし,繰り返しの操作は無制限に行われるものではない.
					得られる記号列は左端から辿っていけば必ず右端が見つかるものとする.
			\end{itemize}
	\end{description}
	
	\monologue{
		院生「式の概念を拡張したことで
			\begin{align}
				s \in \Set{x}{A(x)},\quad t = \Set{x}{A(x)}
			\end{align}
			は式として扱えるようになりましたが,
			\begin{align}
				\forall \Set{x}{A(x)}(s \in \Set{x}{A(x)})
			\end{align}
			は式として認められないのですね.量化記号が付くのは文字に限られます.」
	}
	
	日常使用している言語と同じく,名詞にあたる対象と文法にあたる式の形成手順を
	合わせて{\bf 言語}\index{げんご@言語}{\bf (language)}と呼ぶ.
	以降は言語そのものを意識することは殆ど無いが,形式上の出発点として宣言しておく.
	
	\begin{screen}
		\begin{dfn}[閉式・命題]
			自由変項を含まない式を{\bf 閉式}\index{へいしき@閉式}{\bf (closed formula)}や{\bf 命題}\index{めいだい@命題}{\bf (proposition)}と呼ぶ.
		\end{dfn}
	\end{screen}
	
	\monologue{
		院生「命題とは真偽が定まったものであるという釈然としない説明をよく目にしますが,
			どうやら命題や真偽の哲学的議論には決着が付いていないようで,
			本によっては``異論が続出するから深い言及を避ける''と書いてあるものもあります.
			しかし本稿では命題も真偽もその概念を明確に定義して,
			つまり概念を本稿で必要な分に制限するということになりますが,
			扱うことにいたします.」
	}
	
	\begin{screen}
		\begin{dfn}[宇宙]
			文字$V$を{\bf 宇宙}\index{うちゅう@宇宙}{\bf (Universe)}と呼ぶ.
		\end{dfn}
	\end{screen}
	
	\monologue{
		院生「文字$V$を特別扱いするということですね(笑).宇宙という壮大な言葉が出てきてしまいましたが,
			集合論の世界は$V$の範囲内で語り尽くせてしまうのですから,
			現代数学にとって$V$は宇宙そのものなのですね.
			ところで,現実世界において人間が把握し得る最大の世界は宇宙空間でしょうが,
			数学の世界では宇宙の外側を見ることが出来るのです.宇宙の外側に在るものは真類と呼ばれます.
			実は宇宙そのものも真類の一つなのですが(宇宙が宇宙の外側に在るとは奇妙です),
			その話は後述にまかせましょう.」
	}
	
	数学の式を日本語に翻訳するとき,慣習上よく使われる訳し方があるので列挙する.
	\begin{itemize}
		\item 式$a = b$を``$a$は$b$に等しい''や``$a$と$b$は等しい''と翻訳する.
		\item 式$a \in b$を``$a$は$b$の要素である''や``$a$は$b$に属する''と翻訳する.
		\item 式$(A) \Longrightarrow (B)$を``$A$が成り立つならば$B$が成り立つ''と翻訳する.
		\item 式$\rightharpoondown (A)$を%``$A$でない''と翻訳する.
	\end{itemize}
	
	\begin{screen}
		\begin{dfn}[類・集合]
			対象のことを{\bf 類}\index{るい@類}{\bf (class)}と呼び直し,
			特に$V$の要素である類を{\bf 集合}\index{しゅうごう@集合}{\bf (set)}と呼ぶ.
			また$V$の要素でない類のことを{\bf 真類}\index{しんるい@真類}{\bf (proper class)}と呼ぶ.
		\end{dfn}
	\end{screen}

	\monologue{
		院生「類は$V$の要素であれば集合と呼ばれ,$V$の要素でなければ真類と呼ばれます.
			では集合であり真類でもある類や,集合でも真類でもない類はあるのでしょうか?
			答えは``現段階では確定したことは何も言えない''です.
			質問を変えましょう.集合であり真類でもある類や集合でも真類でもない類の存在を禁止するにはどうしたら良いでしょうか?
			我々は,数学において中庸が無いということや矛盾が起きるということをどう表現しようかという問題に直面しているのです.
			この問題の解決への方便として{\bf 推論規則}\index{すいろんきそく@推論規則}
			{\bf (rule of inference)}と呼ばれるものを導入します.」
	}
	
	\begin{screen}
		\begin{axm}[排中律]
			任意の論理式$A$に対し,$A \vee \rightharpoondown A$が成り立つ.
		\end{axm}
	\end{screen}
	
	\begin{screen}
		\begin{thm}[集合でも真類でもない類は存在しない]
			\begin{align}
				\forall a\ \left(\ \rightharpoondown (\ a \in V \wedge a \notin V\ )\ \right)
			\end{align}
		\end{thm}
	\end{screen}
	
	\begin{screen}
		\begin{axm}[空集合の存在公理]
			いかなる集合も要素に持たない集合が存在する:
			\begin{align}
				\exists x \in V\ \forall y \in V\ (\ y \notin x\ ).
			\end{align}
		\end{axm}
	\end{screen}
	
	\begin{screen}
		\begin{thm}[空集合はただ一つ]
			空集合の存在公理を満たす集合はただ一つである:
			\begin{align}
				\forall x \in V\ \forall y \in V
				\ (\ (\ \forall z \in V\ (\ z \notin x\ ) \wedge \forall z \in V
				\ (\ z \notin y\ )\ )
				\Longrightarrow x=y\ ).
			\end{align}
			この何も持たない空の集合を{\bf 空集合}\index{くうしゅうごう@空集合}{\bf (empty set)}と呼び$\emptyset$という記号で表す.
		\end{thm}
	\end{screen}
	
	\monologue{
		院生「ようやく存在が約束された本物の集合が出てきましたね.
			あらゆる集合は空集合を元に作られていくのですから,空集合は集合の親とでもいえるのでしょうか.」
	}
	
	\begin{screen}
		\begin{axm}[類の公理]\mbox{}
			\begin{description}
				\item[(i)] 類の要素は集合である:
					\begin{align}
						\forall a\ \forall x\ (\ x \in a \Longrightarrow x \in V\ ).
					\end{align}
				
				\item[(ii)] $\Set{x}{A(x)}$とは$A(x)$を成り立たせる集合$x$の全体である:
					\begin{align}
						\forall t\ (\ t \in \Set{x}{A(x)} \Longleftrightarrow t \in V \wedge A(t)\ ).
					\end{align}
			\end{description}
		\end{axm}
	\end{screen}
	
	\begin{screen}
		\begin{axm}[外延性の公理]
			全く同じ要素からなる類は等しい:
			\begin{align}
				\forall a\ \forall b\ \left(\ \forall t\ (\ t \in a  \Longleftrightarrow t \in b\ )
				\Longrightarrow a=b\ \right).
			\end{align}
		\end{axm}
	\end{screen}
	
	\begin{screen}
		\begin{thm}\mbox{}
			\begin{description}
				\item[(1)] $\forall a\ (\ a=a\ )$
				\item[(2)] $\forall a \in V\ (\ a = \Set{x}{x \in a}\ )$
				\item[(3)] $V=\Set{x}{x=x}$
				\item[(4)] $\forall a\ (\ a \subset V\ )$
				\item[(5)] $\Set{x}{A} = \Set{y}{(y\, |\, x)A}$
				\item[(6)] $\Set{x}{A(x)} \cup \Set{x}{\rightharpoondown A(x)} = V$.
			\end{description}
		\end{thm}
	\end{screen}
	
	\begin{screen}
		\begin{axm}[相等性の公理]\mbox{}
			\begin{description}
				\item[(1)] $\forall a\ \forall b\ \forall c\ \left(\ a=b \Longrightarrow (\ c \in a \Longleftrightarrow c \in b\ )\ \right).$
				\item[(2)] $\forall a\ \forall b\ \forall c\ \left(\ a=b \Longrightarrow (\ c = a \Longleftrightarrow c = b\ )\ \right).$
				\item[(3)] $\forall a\ \forall b\ \forall c\ \left(\ a=b \Longrightarrow (\ a \in c \Longleftrightarrow b \in c\ )\ \right).$
			\end{description}
		\end{axm}
	\end{screen}
	
	\begin{screen}
		\begin{thm}
			
		\end{thm}
	\end{screen}
	
	\begin{prf}\mbox{}
		\begin{description}
			\item[(1)] $a^{-1}$の任意の要素$t$に対し或る$V$の要素$x,y$が存在して
				\begin{align}
					(x,y) \in a \wedge t = (y,x)
				\end{align}
				を満たす.$((x,y),(y,x)) \in f$より$((x,y),t) \in f$が成り立つから
				$t \in f \ast a$となる.逆に$f \ast a$の任意の要素$t$に対して
				$a$の或る要素$x$が存在して
				\begin{align}
					x \in a \wedge (x,t) \in f
				\end{align}
				となる.$x$に対し$V$の或る要素$a,b$が存在して$x=(a,b)$となるので
				\begin{align}
					((a,b),t) \in f
				\end{align}
				となり,$V$の或る要素$c,d$が存在して
				\begin{align}
					((a,b),t) = ((c,d),(d,c))
				\end{align}
				となる.$(a,b) = (c,d)$より$a=c$かつ$b=d$となり,
				$t = (d,c)$かつ$(d,c)=(b,a)$より$t=(b,a)$,従って
				$t \in a^{-1}$が成り立つ.
		\end{description}
	\end{prf}
	\subsection{関係}
	\begin{screen}
		\begin{dfn}[対集合]
			$a,b$を集合とするとき,$\mathcal{L}$の或る対象$x,y$が存在して$a = x$かつ$b = y$を満たすが,このとき
			\begin{align}
				\{a,b\} \coloneqq \Set{t}{t \in x \wedge t \in y}
			\end{align}
			で$\{a,b\}$を定義し,これを$a$と$b$の{\bf 対集合}と呼ぶ.
		\end{dfn}
	\end{screen}
	
	\monologue{
		院生「一般に集合$a,b$に対して
			\begin{align}
				\{a,b\} \coloneqq \Set{t}{t \in a \wedge t \in b}
			\end{align}
			と定めることは出来ません.$a,b$が集合であっても$\mathcal{L}$の対象ではない場合,そもそも
			\begin{align}
				\Set{t}{t \in a \wedge t \in b}
			\end{align}
			が$\mathcal{L}'$の対象でないのです.」
	}
	
	\begin{screen}
		\begin{axm}[対集合の公理]
			$a,b$を集合とするとき次が成り立つ:
			\begin{align}
				\{a,b\} \in \Univ.
			\end{align}
		\end{axm}
	\end{screen}
	
	\begin{screen}
		\begin{dfn}[順序対]
			集合$x,y$に対し,
			\begin{align}
				(x,y) = \{\{x\},\{x,y\}\}
			\end{align}
			で定義される類$(x,y)$を$x$と$y$の{\bf 順序対}\index{じゅんじょつい@順序対}
			{\bf (ordered pair)}と呼ぶ.
		\end{dfn}
	\end{screen}
	
	\begin{screen}
		\begin{thm}[順序対は集合]\mbox{}
			\begin{description}
				\item[(1)] $\forall x,y\ \left(\ (x,y) \in \Univ\ \right)$.
				\item[(2)] $\forall x,y,s,t\ 
					\left(\ (x,y)=(s,t) \Longleftrightarrow x=s \wedge y=t\ \right)$.
			\end{description}
		\end{thm}
	\end{screen}
	
	\begin{screen}
		\begin{dfn}[Cartesian積]
			類$a,b$に対し,$a \times b$を
			\begin{align}
				a \times b = \Set{x}{\exists s \in a\ \exists t \in b\ (\ x=(s,t)\ )}
			\end{align}
			で定め,これを$a$と$b$の{\bf Cartesian 積}\index{Cartesian せき@Cartesian 積}
			{\bf (Cartesian product)}と呼ぶ.
		\end{dfn}
	\end{screen}
	
	\monologue{
		院生「類$a$と類$b$のCartesian 積は
			\begin{align}
				a \times b = \Set{(s,t)}{s \in a \wedge t \in b} 
			\end{align}
			と簡略して書かれることも多いです.
	}
	
	\begin{comment}
	\monologue{
		院生「類$a$と類$b$のCartesian 積は
			\begin{align}
				a \times b = \Set{(s,t)}{s \in a \wedge t \in b} 
			\end{align}
			と簡略して書かれることも多いです.ところで他の本やネットなどを見ていると
			Cartesian 積を直積とも呼んでいるそうです.本稿でも後で直積というものを定義いたしますが,
			本稿ではCartesian 積と直積を明確に区別いたします.
			これは巷にあふれる直積の定義の不自然さを解消するためです.
			どういう点が不自然であるか簡単に説明いたしましょう.
			まだ有限とか数だとか定義していませんが,説明の便宜のために使用いたします.
			よく見る直積の定義だと,有限か有限でないかで直積の定め方が変わります.
			\begin{align}
				I_1 \times I_2 \times \cdots \times I_n 
				= \Set{(x_1,x_2,\cdots,x_n)}{x_1 \in I_1 \wedge x_2 \in I_2 \wedge
				\cdots \wedge x_n \in I_n}
			\end{align}
			そして
			\begin{align}
				I_1 \times I_2 \times \cdots \times I_n 
				= \prod_{i=1}^n I_i
			\end{align}
			と書いている.ここで
			$\prod_{i=1}^n I_i$は$\prod_{i\in\{1,2,\cdots,n\}} I_i$の別の記法です.
			他方$I$を$\{1,2,\cdots,n\}$から$V$への写像と見ることもできますから
			\begin{align}
				\prod_{i=1}^n I_i = \Set{f}{f:\{1,2,\cdots,n\} \longrightarrow V \wedge \forall i \in \{1,2,\cdots,n\}\ (\ f(i) \in I_i\ )}
			\end{align}
			となるはずです.食い違います.
			」
	}
	\end{comment}
	
	\begin{screen}
		\begin{dfn}[関係]
			$V \times V$の部分類を{\bf 関係}\index{かんけい@関係}{\bf (relation)}と呼ぶ.
		\end{dfn}
	\end{screen}
	
	いま,関係$E$を
	\begin{align}
		E = \Set{x}{\exists s,t\ (\ x=(s,t) \wedge s = t\ )}
	\end{align}
	と定めてみる.このとき$E$は次の性質を満たす:
	\begin{description}
		\item[(a)] $\forall x\ (\ (x,x) \in E\ )$.
		\item[(b)] $\forall x,y\ (\ (x,y) \in E \Longrightarrow (y,x) \in E\ )$.
		\item[(c)] $\forall x,y,z\ (\ (x,y) \in E \wedge (y,z) \in E \Longrightarrow (x,z) \in E\ )$.
	\end{description}
	性質(a)を反射律と呼ぶ.性質(b)を対称律と呼ぶ.性質(c)を推移律と呼ぶ.
	
	\begin{screen}
		\begin{dfn}[同値関係]
			$a$を集合とするとき,
			\begin{description}
				\item[反射律] $\forall x \in a\ (\ (x,x) \in R\ )$.
				\item[対称律] $\forall x,y \in a\ (\ (x,y) \in R \Longrightarrow (y,x) \in R\ )$.
				\item[推移律] $\forall x,y,z \in a\ (\ (x,y) \in R \wedge (y,z) \in R \Longrightarrow (x,z) \in R\ )$.
			\end{description}
			を満たす関係$R$を$a$上の{\bf 同値関係}\index{どうちかんけい@同値関係}
			{\bf (equivalence relation)}と呼ぶ.
		\end{dfn}
	\end{screen}
	
	\monologue{
		院生「集合$a$に対して$R = E \cap (a \times a)$とおけば$R$は$a$上の同値関係となりますね.」
	}
	
	$E$とは別の関係$O$を
	\begin{align}
		O = \Set{x}{\exists s,t\ (\ x=(s,t) \wedge s \subset t\ )}
	\end{align}
	により定めてみる.このとき$O$は次の性質を満たす:
	\begin{description}
		\item[(a)] $\forall x\ (\ (x,x) \in O\ )$.
		\item[(b')] $\forall x,y\ (\ (x,y) \in O \wedge (y,x) \in O \Longrightarrow x=y\ )$.
		\item[(c)] $\forall x,y,z\ (\ (x,y) \in O \wedge (y,z) \in O \Longrightarrow (x,z) \in O\ )$.
	\end{description}
	性質(b')を反対称律と呼ぶ.
	
	\begin{screen}
		\begin{dfn}[順序関係]
			$a$を類とする.$a$に対し或る関係$R$が存在して
			\begin{description}
				\item[反射律] $\forall x \in a\ (\ (x,x) \in R\ )$.
				\item[反対称律] $\forall x,y \in a\ (\ (x,y) \in R \wedge (y,x) \in R \Longrightarrow x=y\ )$.
				\item[推移律] $\forall x,y,z \in a\ (\ (x,y) \in R \wedge (y,z) \in R \Longrightarrow (x,z) \in R\ )$.
			\end{description}
			が満たされているとき,$R$を$a$上の{\bf 順序}\index{じゅんじょ@順序}{\bf (order)}と呼ぶ.
			$a,R$が共に集合であるときは対$(a,R)$を{\bf 順序集合}\index{じゅんじょしゅうごう@順序集合}
			{\bf (ordered set)}と呼ぶ.特に
			\begin{align}
				\forall x,y \in a\ (\ (x,y) \in R \vee (y,x) \in R\ )
			\end{align}
			が成り立つとき,$R$を$a$上の{\bf 全順序}\index{ぜんじゅんじょ@全順序}
			{\bf (total order)}と呼ぶ.			
		\end{dfn}
	\end{screen}
	
	\monologue{
		院生「反射律と推移律のみを満たす関係を{\bf 前順序}\index{ぜんじゅんじょ@前順序}
			{\bf (preorder)}と呼びます.また全順序は{\bf 線型順序}
			\index{せんけいじゅんじょ@線型順序}{\bf (linear order)}とも呼ばれます.
			また表記上の問題ですが,集合$R$を集合$a$上の順序関係として
			\begin{align}
				x \leq y \Longleftrightarrow (x,y) \in R
			\end{align}
			で記号$\leq$を定めるとき,$(a,\leq)$と順序対の形で表して
			これを順序集合と呼ぶこともあります.」
	}
	
	\begin{screen}
		\begin{dfn}[整列集合]
			$x$が{\bf 整列集合}\index{せいれつしゅうごう@整列集合}{\bf (wellordered set)}
			であるとは,$x$が集合$a$と$a$上の順序$R$の対$(a,R)$に等しく,
			かつ$a$の空でない任意の部分集合が$R$に関する最小元を持つことをいう.
			またこのときの$R$を{\bf 整列順序}\index{せいれつじゅんじょ@整列順序}
			{\bf (wellorder)}と呼ぶ.
		\end{dfn}
	\end{screen}
	
	\begin{screen}
		\begin{thm}[整列順序は全順序]
		\end{thm}
	\end{screen}
	\section{順序数}
	$0,1,2,\cdots$で表される数字は,集合論において
	\begin{align}
		0 &\defeq \emptyset, \\
		1 &\defeq \{0\} = \{\emptyset\}, \\
		2 &\defeq \{0,1\} = \{\emptyset,\{\emptyset\}\}, \\
		3 &\defeq \{0,1,2\} = \{\emptyset,\{\emptyset\},\{\emptyset,\{\emptyset\}\}\}
	\end{align}
	といった反復操作で定められる.上の操作を受け継いで``頑張れば手で書き出せる''類を自然数と呼ぶ.
	$0$は集合であり,対集合の公理から$1$もまた集合である.
	そして和集合の公理から$2$が集合であること,更には$3,4,\cdots$と続く自然数が全て集合であることがわかる.
	自然数の冪も自然数同士の集合演算もその結果は全て集合になるが,
	ここで
	\begin{align}
		\mbox{集合は$0$に集合演算を施しただけの素姓が明らかなものに限られるか}
	\end{align}
	という疑問というか期待が自然に生まれてくる.実際それは正則性公理によって肯定されるわけだが,
	そこでキーになるのは順序数と呼ばれる概念である.
	
	\begin{screen}
		\begin{logicalthm}[論理和・論理積の結合律]\label{logicalthm:associative_law}
			$A,B,C$を$\mathcal{L}'$の閉式とするとき次が成り立つ:
			\begin{description}
				\item[(イ)] $(A \vee B) \vee C \Longleftrightarrow A \vee (B \vee C)$.
				\item[(ロ)] $(A \wedge B) \wedge C \Longleftrightarrow A \wedge (B \wedge C)$.
			\end{description}
		\end{logicalthm}
	\end{screen}
	
	\begin{screen}
		\begin{logicalthm}[論理和・論理積の分配律]\label{logicalthm:distributive_law}
			$A,B,C$を$\mathcal{L}'$の閉式とするとき次が成り立つ:
			\begin{description}
				\item[(イ)] $(A \vee B) \wedge C \Longleftrightarrow (A \wedge C) \vee (B \wedge C)$.
				\item[(ロ)] $(A \wedge B) \vee C \Longleftrightarrow (A \vee C) \wedge (B \vee C)$.
			\end{description}
		\end{logicalthm}
	\end{screen}
	
	\begin{prf}\mbox{}
		\begin{description}
			\item[(イ)] いま$(A \vee B) \wedge C$が成立していると仮定する.
				このとき論理積の除去により$A \vee B$と$C$が同時に成り立つ.ここで$A$が成り立っているとすれば,
				論理積の導入により
				\begin{align}
					A \wedge C
				\end{align}
				が成り立つので演繹法則より
				\begin{align}
					A \Longrightarrow (A \wedge C)
				\end{align}
				が成立する.他方で論理和の導入より
				\begin{align}
					(A \wedge C) \Longrightarrow (A \wedge C) \vee (B \wedge C)
				\end{align}
				も成り立つので,含意の推移律から
				\begin{align}
					A \Longrightarrow (A \wedge C) \vee (B \wedge C)
				\end{align}
				が従う.$A$と$B$を入れ替えれば
				\begin{align}
					B \Longrightarrow (B \wedge C) \vee (A \wedge C)
				\end{align}
				が成り立つが,論理和の可換律より
				\begin{align}
					(B \wedge C) \vee (A \wedge C) \Longrightarrow (A \wedge C) \vee (B \wedge C)
				\end{align}
				が成り立つので
				\begin{align}
					B \Longrightarrow (A \wedge C) \vee (B \wedge C)
				\end{align}
				が従う.よって場合分け法則から
				\begin{align}
					(A \vee B) \Longrightarrow (A \wedge C) \vee (B \wedge C)
				\end{align}
				が成立するが,いま$A \vee B$は満たされているので三段論法より
				\begin{align}
					(A \wedge C) \vee (B \wedge C)
				\end{align}
				が成立する.ここに演繹法則を適用すれば
				\begin{align}
					(A \vee B) \wedge C \Longrightarrow (A \wedge C) \vee (B \wedge C)
				\end{align}
				が得られる.次に$A \wedge C$が成り立っていると仮定する.このとき
				$A$が成り立つので$A \vee B$も成立し,同時に$C$も成り立つので
				$(A \vee B) \wedge C$が成立する.すなわち
				\begin{align}
					A \wedge C \Longrightarrow (A \vee B) \wedge C
				\end{align}
				が成立する.$A$と$B$を入れ替えれば
				\begin{align}
					B \wedge C \Longrightarrow (A \vee B) \wedge C
				\end{align}
				も成立するので
				\begin{align}
					(A \wedge C) \vee (B \wedge C) \Longrightarrow (A \vee B) \wedge C
				\end{align}
				が得られる.
				
			\item[(ロ)]
				(イ)の結果を$\rightharpoondown A,\rightharpoondown B,\rightharpoondown C$に適用すれば
				\begin{align}
					(\rightharpoondown A \vee \rightharpoondown B) \wedge \rightharpoondown C
					\Longleftrightarrow (\rightharpoondown A \wedge \rightharpoondown C) 
						\vee (\rightharpoondown B \wedge \rightharpoondown C)
				\end{align}
				が得られる.ここでDe Morganの法則と同値記号の遺伝性質から
				\begin{align}
					(\rightharpoondown A \vee \rightharpoondown B) \wedge \rightharpoondown C
					&\Longleftrightarrow\ \rightharpoondown (A \wedge B) \wedge \rightharpoondown C \\
					&\Longleftrightarrow\ \rightharpoondown ((A \wedge B) \vee C)
				\end{align}
				が成立し,一方で
				\begin{align}
					(\rightharpoondown A \wedge \rightharpoondown C) 
						\vee (\rightharpoondown B \wedge \rightharpoondown C)
					&\Longleftrightarrow\ \rightharpoondown (A \vee C) \vee \rightharpoondown (B \vee C) \\
					&\Longleftrightarrow\ \rightharpoondown ((A \vee C) \wedge (B \vee C))
				\end{align}
				も成立するから,含意の推移律より
				\begin{align}
					\rightharpoondown ((A \wedge B) \vee C)
					\Longleftrightarrow\ \rightharpoondown ((A \vee C) \wedge (B \vee C))
				\end{align}
				が従う.最後に対偶を取れば
				\begin{align}
					(A \wedge B) \vee C \Longleftrightarrow (A \vee C) \wedge (B \vee C)
				\end{align}
				が得られる.
				\QED
		\end{description}
	\end{prf}
	
	\begin{screen}
		\begin{logicalthm}[選言三段論法]\label{logicalthm:disjunctive_syllogism}
			$A,B,C$を$\mathcal{L}'$の閉式とするとき次が成り立つ:
			\begin{align}
				(A \vee B) \wedge \rightharpoondown B \Longrightarrow A.
			\end{align}
		\end{logicalthm}
	\end{screen}
	
	\begin{prf}
		分配律(推論法則\ref{logicalthm:distributive_law})より
		\begin{align}
			(A \vee B) \wedge \rightharpoondown B
			\Longrightarrow (A \wedge \rightharpoondown B) \vee (B \wedge \rightharpoondown B)
		\end{align}
		が成立する.ここで矛盾に関する規則から
		\begin{align}
			B \wedge \rightharpoondown B \Longrightarrow \bot
		\end{align}
		が満たされるので
		\begin{align}
			(A \wedge \rightharpoondown B) \vee (B \wedge \rightharpoondown B)
			\Longrightarrow (A \wedge \rightharpoondown B) \vee \bot
		\end{align}
		が従う.また,論理積の除去より
		\begin{align}
			(A \wedge \rightharpoondown B) \Longrightarrow A
		\end{align}
		が成り立ち,他方で矛盾に関する規則より
		\begin{align}
			\bot \Longrightarrow A
		\end{align}
		も成り立つから,場合分け法則より
		\begin{align}
			(A \wedge \rightharpoondown B) \vee \bot \Longrightarrow A
		\end{align}
		が従う.以上の式と含意の推移律から
		\begin{align}
			(A \vee B) \wedge \rightharpoondown B \Longrightarrow A
		\end{align}
		が得られる.
		\QED
	\end{prf}
	
	\begin{screen}
		\begin{axm}[正則性公理]
			$a$を類とするとき,$a$は空でなければ自分自身と交わらない要素を持つ:
			\begin{align}
				a \neq \emptyset \Longrightarrow 
				\exists x \in a\, (\, x \cap a = \emptyset\, ).
			\end{align}
		\end{axm}
	\end{screen}
	
	\begin{screen}
		\begin{thm}[いかなる類も自分自身を要素に持たない]
		\label{thm:no_set_is_an_element_of_itself}
			$a,b,c$を類とするとき次が成り立つ:
			\begin{description}
				\item[(イ)] $a \notin a$.
				
				\item[(ロ)] $a \notin b \vee b \notin a$.
				
				\item[(ハ)] $a \notin b \vee b \notin c \vee c \notin a$.
			\end{description}
		\end{thm}
	\end{screen}
	
	\begin{prf}\mbox{}
		\begin{description}
			\item[(イ)] $a$を類とする.まず要素の公理の対偶より
				\begin{align}
					\rightharpoondown \set{a} \Longrightarrow a \notin a
				\end{align}
				が満たされる.次に$a$が集合であるとする.
				このとき定理\ref{thm:pair_of_proper_classes_is_emptyset}より
				\begin{align}
					a \in \{a\}
				\end{align}
				が成り立つから,正則性公理より
				\begin{align}
					\exists x\, \left(\, x \in \{z\} \wedge x \cap \{a\} = \emptyset\, \right)
				\end{align}
				が従う.ここで$\chi \coloneqq \varepsilon x\, \left(\, x \in \{a\} \wedge x \cap \{a\} = \emptyset\, \right)$とおけば
				\begin{align}
					\chi = a
				\end{align}
				となるので,相等性の公理より
				\begin{align}
					a \cap \{a\} = \emptyset
				\end{align}
				が成り立つ.$a \in \{a\}$であるから
				定理\ref{thm:if_pair_is_empty_then_its_members_do_not_intersect}より
				$a \notin a$が従い,演繹法則から
				\begin{align}
					\set{a} \Longrightarrow a \notin a
				\end{align}
				が得られる.そして場合分け法則から
				\begin{align}
					\set{a} \vee \rightharpoondown \set{a} \Longrightarrow a \notin a
				\end{align}
				が成立し,排中律と三段論法から
				\begin{align}
					a \notin a
				\end{align}
				が出る.
			
			\item[(ロ)]
				要素の公理より
				\begin{align}
					a \in b \Longrightarrow \set{a}
				\end{align}
				となり,定理\ref{thm:pair_of_proper_classes_is_emptyset}より
				\begin{align}
					\set{a} \Longrightarrow a \in \{a,b\}
				\end{align}
				となるので,
				\begin{align}
					a \in b \Longrightarrow a \in \{a,b\}
				\end{align}
				が成立する.また定理\ref{thm:if_pair_is_empty_then_its_members_do_not_intersect}より
				\begin{align}
					a \in b \wedge a \in \{a,b\} 
					&\Longrightarrow \exists x\, \left(\, x \in b \wedge x \in \{a,b\}\, \right) \\
					&\Longrightarrow b \cap \{a,b\} \neq \emptyset
				\end{align}
				が成立する.他方で正則性公理より
				\begin{align}
					a \in \{a,b\} &\Longrightarrow \exists x\, \left(\, x \in \{a,b\}\, \right) \\
					&\Longrightarrow \{a,b\} \neq \emptyset \\
					&\Longrightarrow \exists x\, \left(\, x \in \{a,b\} \wedge x \cap \{a,b\} = \emptyset\, \right)
				\end{align}
				も成立する.以上を踏まえて$a \in b$が成り立っていると仮定する.このとき
				\begin{align}
					a \in \{a,b\}
				\end{align}
				が成立するので
				\begin{align}
					b \cap \{a,b\} \neq \emptyset
				\end{align}
				も成り立ち,さらに
				\begin{align}
					\exists x\, \left(\, x \in \{a,b\} \wedge x \cap \{a,b\} = \emptyset\, \right)
				\end{align}
				も満たされる.ここで
				\begin{align}
					\chi \coloneqq \varepsilon x\, \left(\, x \in \{a,b\} \wedge x \cap \{a,b\} = \emptyset\, \right)
				\end{align}
				とおけば$\chi \in \{a,b\}$から
				\begin{align}
					\chi = a \vee \chi = b
				\end{align}
				が従うが,相等性の公理より
				\begin{align}
					\chi = b \Longrightarrow b \cap \{a,b\} = \emptyset
				\end{align}
				となるので,$b \cap \{a,b\} \neq \emptyset$と併せて
				\begin{align}
					\chi \neq b
				\end{align}
				が成立する.選言三段論法(推論法則\ref{logicalthm:disjunctive_syllogism})より
				\begin{align}
					(\chi = a \vee \chi = b) \wedge \chi \neq b \Longrightarrow \chi = a
				\end{align}
				となるから
				\begin{align}
					\chi = a
				\end{align}
				が従い,相等性の公理より
				\begin{align}
					a \cap \{a,b\} = \emptyset
				\end{align}
				が成立する.いま要素の公理より
				\begin{align}
					\rightharpoondown \set{b} \Longrightarrow b \notin a
				\end{align}
				が満たされ,他方で定理\ref{thm:pair_of_proper_classes_is_emptyset}より
				\begin{align}
					\set{b} \Longrightarrow b \in \{a,b\},
				\end{align}
				$a \cap \{a,b\}$の仮定と定理\ref{thm:if_pair_is_empty_then_its_members_do_not_intersect}より
				\begin{align}
					b \in \{a,b\} = \emptyset \Longrightarrow b \notin a
				\end{align}
				が満たされるので
				\begin{align}
					\set{b} \Longrightarrow b \notin a
				\end{align}
				が成立する.従って
				\begin{align}
					b \notin a
				\end{align}
				が従い,演繹法則より
				\begin{align}
					a \in b \Longrightarrow b \notin a
				\end{align}
				が得られる.これは$a \notin b \vee b \notin a$と同値である.
				
			\item[(ハ)]
				$a \in b \wedge b \in c$が満たされていると仮定すれば,$a,b$は集合であるから
				\begin{align}
					a,b \in \{a,b,c\}
				\end{align}
				が成立する.ゆえに$b \cap \{a,b,c\} \neq \emptyset$と$c \cap \{a,b,c\} \neq \emptyset$が従う.
				他方,正則性公理より
				\begin{align}
					\tau \in \{a,b,c\} \wedge \tau \cap \{a,b,c\} = \emptyset
				\end{align}
				を満たす$\mathcal{L}$の対象$\tau$が取れる.ここで$\tau \in \{a,b,c\}$より
				\begin{align}
					\tau = a \vee \tau = b \vee \tau = a
				\end{align}
				が成り立つが,$b \cap \{a,b,c\} \neq \emptyset$と$c \cap \{a,b,c\} \neq \emptyset$
				より$\tau \neq b$かつ$\tau \neq c$となる.よって$\tau = a$となり
				\begin{align}
					a \cap \{a,b,c\} = \emptyset
				\end{align}
				が従う.$c$が真類ならば要素の公理より$c \notin a$となり,$c$が集合ならば$c \in \{a,b,c\}$となるので,
				いずれにせよ
				\begin{align}
					c \notin a
				\end{align}
				が成立する.以上で
				\begin{align}
					a \in b \wedge b \in c \Longrightarrow c \notin a
				\end{align}
				が得られる.
				\QED
		\end{description}
	\end{prf}
	
	\begin{screen}
		\begin{dfn}[順序数]
			類$a$に対して,$a$が{\bf 推移的類}\index{すいいてきるい@推移的類}{\bf (transitive class)}であるということを
			\begin{align}
				\tran{a} \defarrow
				\forall s\, (\, s \in a \Longrightarrow s \subset a\, )
			\end{align}
			で定める.また$a$が(集合であるならば){\bf 順序数}\index{じゅんじょすう@順序数}{\bf (ordinal number)}であるということを
			\begin{align}
				\ord{a} \defarrow
				\tran{a} \wedge \forall t,u \in a\, (\, t \in u \vee t = u \vee u \in t\, )
			\end{align}
			で定める.順序数の全体を
			\begin{align}
				\ON \defeq \Set{x}{\ord{x}}
			\end{align}
			とおく.
		\end{dfn}
	\end{screen}
	
	空虚な真の一例であるが,例えば$0$は順序数の性質を満たす.
	ここに一つの順序数が得られたが,いま仮に$\alpha$を順序数とすれば
	\begin{align}
		\alpha \cup \{\alpha\}
	\end{align}
	もまた順序数となることが直ちに判明する.数字の定め方から
	\begin{align}
		1 &= 0 \cup \{0\}, \\
		2 &= 1 \cup \{1\}, \\
		3 &= 2 \cup \{2\}, \\
		&\vdots
	\end{align}
	が成り立つから,数字は全て順序数である.
	
	いま$\ON$上の関係を
	\begin{align}
		\leq\ \defeq \Set{x}{\exists \alpha,\beta \in \ON\, 
		(\, x=(\alpha,\beta) \wedge \alpha \subset \beta\, )}
	\end{align}
	と定める.
		
	\begin{itembox}[l]{中置記法について}
		$x$と$y$を項とするとき,
		\begin{align}
			(x,y) \in\ \leq
		\end{align}
		なることを往々にして
		\begin{align}
			x \leq y
		\end{align}
		とも書くが,このような書き方を{\bf 中置記法}\index{ちゅうちきほう@中置記法}{\bf (infix notation)}と呼ぶ.
		同様にして,
		\begin{align}
			(x,y) \in\ \leq \wedge x \neq y
		\end{align}
		なることを
		\begin{align}
			x < y
		\end{align}
		とも書く.
	\end{itembox}
	
	以下順序数の性質を列挙するが,長いので主張だけ先に述べておく.
	\begin{itemize}
		\item $\ON$は推移的類である.
		\item $\leq$は$\ON$において整列順序となる.
		\item $a$を$a \subset \ON$なる集合とすると,$\bigcup a$は$a$の$\leq$に関する上限となる.
		\item $\ON$は集合ではない.
	\end{itemize}
	
	\begin{screen}
		\begin{thm}[推移的で$\in$が全順序となる類は$\ON$に含まれる]
		\label{thm:transitive_totally_ordered_class_is_contained_in_ON}
			$S$を類とするとき
			\begin{align}
				\ord{S} \Longrightarrow S \subset \ON.
			\end{align}
		\end{thm}
	\end{screen}
	
	\begin{sketch}
		$x$を$S$の要素とする.まず
		\begin{align}
			\forall s,t \in x\, (\, s \in t \vee s = t \vee t \in s\, )
			\label{fom:thm_transitive_totally_ordered_class_is_contained_in_ON_1}
		\end{align}
		が成り立つことを示す.実際$S$の推移性より
		\begin{align}
			x \subset S
		\end{align}
		が成り立つので,$x$の要素は全て$S$の要素となり
		(\refeq{fom:thm_transitive_totally_ordered_class_is_contained_in_ON_1})が満たされる.次に
		\begin{align}
			\tran{x}
		\end{align}
		が成り立つことを示す.$y$を$x$の要素とする.また$z$を$y$の要素とする.このとき
		\begin{align}
			x \subset S
		\end{align}
		から
		\begin{align}
			y \in S
		\end{align}
		が成り立つので
		\begin{align}
			y \subset S
		\end{align}
		が成り立ち,ゆえに
		\begin{align}
			z \in S
		\end{align}
		となる.従って
		\begin{align}
			z \in x \vee z = x \vee x \in z
			\label{fom:thm_transitive_totally_ordered_class_is_contained_in_ON_2}
		\end{align}
		が成立する.ところで定理\ref{thm:no_set_is_an_element_of_itself}より
		\begin{align}
			z \in y \Longrightarrow y \notin z
		\end{align}
		が成り立つから
		\begin{align}
			y \notin z
			\label{fom:thm_transitive_totally_ordered_class_is_contained_in_ON_3}
		\end{align}
		が成立する.また相当性の公理から
		\begin{align}
			z = x \vee y \in x \Longrightarrow y \in z
		\end{align}
		が成り立つので,その対偶と(\refeq{fom:thm_transitive_totally_ordered_class_is_contained_in_ON_2})から
		\begin{align}
			z \neq x \vee y \notin x
		\end{align}
		も満たされる.いま
		\begin{align}
			y \in x
		\end{align}
		が成り立っていて,さらに選言三段論法より
		\begin{align}
			(\, z \neq x \vee y \notin x\, ) \wedge y \in x \Longrightarrow z \neq x
		\end{align}
		も成り立つから,
		\begin{align}
			z \neq x
		\end{align}
		が成立する.他方で定理\ref{thm:no_set_is_an_element_of_itself}より
		\begin{align}
			z \in y \wedge y \in x \Longrightarrow x \notin z
		\end{align}
		が成立するから,ゆえにいま
		\begin{align}
			z \neq x \wedge x \notin z
		\end{align}
		が,つまり
		\begin{align}
			\rightharpoondown (\, z = x \vee x \in z\, )
			\label{fom:thm_transitive_totally_ordered_class_is_contained_in_ON_4}
		\end{align}
		が成立している.ここで選言三段論法より
		\begin{align}
			(\, z \in x \vee z = x \vee x \in z\, ) \wedge 
			\rightharpoondown (\, z = x \vee x \in z\, )
			\Longrightarrow z \in x
		\end{align}
		も成り立つので,(\refeq{fom:thm_transitive_totally_ordered_class_is_contained_in_ON_3})と
		(\refeq{fom:thm_transitive_totally_ordered_class_is_contained_in_ON_4})と併せて
		\begin{align}
			z \in x
		\end{align}
		が従う.以上より,$y$を$x$の要素とすれば
		\begin{align}
			\forall z \in y\, (\, z \in y \Longrightarrow z \in x\, )
		\end{align}
		が成り立ち,ゆえに
		\begin{align}
			y \subset x
		\end{align}
		が成り立つ.ゆえに$x$は推移的である.ゆえに
		\begin{align}
			\ord{x}
		\end{align}
		が成立し
		\begin{align}
			x \in \ON
		\end{align}
		となる.$x$の任意性より
		\begin{align}
			S \subset \ON
		\end{align}
		が得られる.
		\QED
	\end{sketch}
	
	\begin{screen}
		\begin{thm}[$\ON$は推移的]\label{thm:On_is_transitive}
			$\tran{\ON}$が成立する.
		\end{thm}
	\end{screen}
	
	\begin{prf} 
		$x$を順序数とすると
		\begin{align}
			\ord{x}
		\end{align}
		が成り立つので,定理\ref{thm:transitive_totally_ordered_class_is_contained_in_ON}から
		\begin{align}
			x \subset \ON
		\end{align}
		が成立する.ゆえに$\ON$は推移的である.
		\QED
	\end{prf}
	
	\begin{screen}
		\begin{thm}[$\ON$において$\in$と$<$は同義]
		\label{thm:element_and_proper_subset_correspond_between_ordinal_numbers}
			\begin{align}
				\forall \alpha,\beta \in \ON\,
				(\, \alpha \in \beta \Longleftrightarrow \alpha < \beta\, ).
			\end{align}
		\end{thm}
	\end{screen}
	
	\begin{prf}
		$\alpha,\beta$を任意に与えられた順序数とする.
		\begin{align}
			\alpha \in \beta
		\end{align}
		が成り立っているとすると,順序数の推移性より
		\begin{align}
			\alpha \subset \beta
		\end{align}
		が成り立つ.定理\ref{thm:no_set_is_an_element_of_itself}より
		\begin{align}
			\alpha \neq \beta
		\end{align}
		も成り立つから
		\begin{align}
			\alpha < \beta
		\end{align}
		が成り立つ.ゆえに
		\begin{align}
			\alpha \in \beta \Longrightarrow \alpha < \beta
		\end{align}
		が成立する.逆に
		\begin{align}
			\alpha < \beta
		\end{align}
		が成り立っているとすると
		\begin{align}
			\beta \backslash \alpha \neq \emptyset
		\end{align}
		が成り立つので,正則性公理より
		\begin{align}
			\gamma \in \beta \backslash \alpha \wedge \gamma \cap (\beta \backslash \alpha) = \emptyset
		\end{align}
		を満たす$\gamma$が取れる.このとき
		\begin{align}
			\alpha = \gamma
		\end{align}
		が成り立つことを示す.$x$を$\alpha$の任意の要素とすれば,
		$x,\gamma$は共に$\beta$に属するから
		\begin{align}
			x \in \gamma \vee x = \gamma \vee \gamma \in x
			\label{eq:thm_element_and_proper_subset_correspond_between_ordinal_numbers_1}
		\end{align}
		が成り立つ.ところで相等性の公理から
		\begin{align}
			x = \gamma \wedge x \in \alpha \Longrightarrow \gamma \in \alpha
		\end{align}
		が成り立ち,$\alpha$の推移性から
		\begin{align}
			\gamma \in x \wedge x \in \alpha \Longrightarrow \gamma \in \alpha
		\end{align}
		が成り立つから,それぞれ対偶を取れば
		\begin{align}
			\gamma \notin \alpha \Longrightarrow x \neq \gamma \vee x \notin \alpha
		\end{align}
		と
		\begin{align}
			\gamma \notin \alpha \Longrightarrow \gamma \notin x \vee x \notin \alpha
		\end{align}
		が成立する.いま
		\begin{align}
			\gamma \notin \alpha
		\end{align}
		が成り立っているので
		\begin{align}
			x \neq \gamma \vee x \notin \alpha
		\end{align}
		と
		\begin{align}
			\gamma \notin x \vee x \notin \alpha
		\end{align}
		が共に成り立ち,また
		\begin{align}
			x \in \alpha
		\end{align}
		でもあるから選言三段論法より
		\begin{align}
			x \neq \gamma
		\end{align}
		と
		\begin{align}
			\gamma \notin x
		\end{align}
		が共に成立する.そして(\refeq{eq:thm_element_and_proper_subset_correspond_between_ordinal_numbers_1})と
		選言三段論法より
		\begin{align}
			x \in \gamma
		\end{align}
		が従うので
		\begin{align}
			\alpha \subset \gamma
		\end{align}
		が得られる.逆に$x$を$\gamma$に任意の要素とすると
		\begin{align}
			x \in \beta \wedge x \notin \beta \backslash \alpha
		\end{align}
		が成り立つから,すなわち
		\begin{align}
			x \in \beta \wedge (\, x \notin \beta \vee x \in \alpha\, )
		\end{align}
		が成立する.ゆえに選言三段論法より
		\begin{align}
			x \in \alpha
		\end{align}
		が成り立ち,$x$の任意性より
		\begin{align}
			\gamma \subset \alpha
		\end{align}
		となる.従って
		\begin{align}
			\gamma = \alpha
		\end{align}
		が成立し,
		\begin{align}
			\gamma \in \beta
		\end{align}
		なので
		\begin{align}
			\alpha \in \beta
		\end{align}
		が成り立つ.以上で
		\begin{align}
			\alpha < \beta \Longrightarrow \alpha \in \beta
		\end{align}
		も得られた.
		\QED
	\end{prf}
	
	\begin{screen}
		\begin{thm}[$\ON$の整列性]\label{thm:On_is_wellordered}
			$\leq$は$\ON$上の整列順序である.また次が成り立つ.
			\begin{align}
				\forall \alpha,\beta \in \ON\,
				\left(\, \alpha \in \beta \vee \alpha = \beta \vee \beta \in \alpha\, \right).
			\end{align}
		\end{thm}
	\end{screen}
	
	\begin{prf}\mbox{}
		\begin{description}
			\item[第一段]
				$\alpha,\beta,\gamma$を順序数とすれば
				\begin{align}
					\alpha \subset \alpha
				\end{align}
				かつ
				\begin{align}
					\alpha \subset \beta \wedge \beta \subset \alpha \Longrightarrow \alpha = \beta
				\end{align}
				かつ
				\begin{align}
					\alpha \subset \beta \wedge \beta \subset \gamma \Longrightarrow \alpha \subset \gamma
				\end{align}
				が成り立つ.ゆえに$\leq$は$\ON$上の順序である.
				
			\item[第二段]
				$\leq$が全順序であることを示す.$\alpha$と$\beta$を順序数とする.このとき
				\begin{align}
					\ord{\alpha \cap \beta}
				\end{align}
				が成り立ち,他方で定理\ref{thm:no_set_is_an_element_of_itself}より
				\begin{align}
					\alpha \cap \beta \notin \alpha \cap \beta
				\end{align}
				が満たされるので
				\begin{align}
					\alpha \cap \beta \notin \alpha \vee \alpha \cap \beta \notin \beta
					\label{eq:thm_On_is_wellordered_5}
				\end{align}
				が成立する.ところで
				\begin{align}
					\alpha \cap \beta \subset \alpha
				\end{align}
				は正しいので定理\ref{thm:element_and_proper_subset_correspond_between_ordinal_numbers}から
				\begin{align}
					\alpha \cap \beta \in \alpha \vee \alpha \cap \beta = \alpha
				\end{align}
				が成立する.従って
				\begin{align}
					\alpha \cap \beta \notin \alpha \Longrightarrow 
					(\alpha \cap \beta \in \alpha \vee \alpha \cap \beta = \alpha) \wedge \alpha \cap \beta \notin \alpha
					\label{eq:thm_On_is_wellordered_2}
				\end{align}
				が成り立ち,他方で選言三段論法より
				\begin{align}
					(\alpha \cap \beta \in \alpha \vee \alpha \cap \beta = \alpha) \wedge \alpha \cap \beta \notin \alpha
					\Longrightarrow \alpha \cap \beta = \alpha
					\label{eq:thm_On_is_wellordered_3}
				\end{align}
				も成り立ち,かつ
				\begin{align}
					\alpha \cap \beta = \alpha \Longrightarrow \alpha \subset \beta
					\label{eq:thm_On_is_wellordered_4}
				\end{align}
				も成り立つので,(\refeq{eq:thm_On_is_wellordered_2})と(\refeq{eq:thm_On_is_wellordered_3})と
				(\refeq{eq:thm_On_is_wellordered_4})から
				\begin{align}
					\alpha \cap \beta \notin \alpha \Longrightarrow \alpha \subset \beta
				\end{align}
				が得られる.同様にして
				\begin{align}
					\alpha \cap \beta \notin \beta \Longrightarrow \beta \subset \alpha
				\end{align}
				も得られる.さらに論理和の規則から
				\begin{align}
					\alpha \cap \beta \notin \alpha \Longrightarrow \alpha \subset \beta \vee \beta \subset \alpha
				\end{align}
				と
				\begin{align}
					\alpha \cap \beta \notin \beta \Longrightarrow \alpha \subset \beta \vee \beta \subset \alpha
				\end{align}
				が従うので,(\refeq{eq:thm_On_is_wellordered_5})と場合分け法則より
				\begin{align}
					\alpha \subset \beta \vee \beta \subset \alpha
				\end{align}
				が成立して
				\begin{align}
					(\alpha,\beta) \in\ \leq \vee (\beta,\alpha) \in\ \leq
				\end{align}
				が成立する.ゆえに$\leq$は全順序である.
			
			\item[第三段]
				$\leq$が整列順序であることを示す.$a$を$\ON$の空でない部分集合とする.このとき正則性公理より
				\begin{align}
					x \in a \wedge x \cap a = \emptyset
				\end{align}
				を満たす集合$x$が取れるが,この$x$が$a$の最小限である.実際,任意に$a$から要素$y$を取ると
				\begin{align}
					x \cap a = \emptyset
				\end{align}
				から
				\begin{align}
					y \notin x
				\end{align}
				が従い,また前段の結果より
				\begin{align}
					x \in y \vee x = y \vee y \in x
				\end{align}
				も成り立つので,選言三段論法より
				\begin{align}
					x \in y \vee x = y
					\label{eq:thm_On_is_wellordered_6}
				\end{align}
				が成り立つ.$y$は推移的であるから
				\begin{align}
					x \in y \Longrightarrow x \subset y
				\end{align}
				が成立して,また
				\begin{align}
					x = y \Longrightarrow x \subset y
				\end{align}
				も成り立つから,(\refeq{eq:thm_On_is_wellordered_6})と場合分け法則から
				\begin{align}
					(x,y) \in\ \leq
				\end{align}
				が従う.$y$の任意性より
				\begin{align}
					\forall y \in a\, \left[\, (x,y) \in\ \leq\, \right]
				\end{align}
				が成立するので$x$は$a$の最小限である.
				\QED
		\end{description}
	\end{prf}
	
	\begin{screen}
		\begin{thm}[$\ON$の部分集合の合併は順序数となる]\label{thm:union_of_set_of_ordinal_numbers_is_ordinal}
			\begin{align}
				\forall a\,
				\left(\, a \subset \ON \Longrightarrow \bigcup a \in \ON\, \right).
			\end{align}
		\end{thm}
	\end{screen}
	
	\begin{prf}
		和集合の公理より$\bigcup a \in \Univ$となる.また順序数の推移性より
		$\bigcup a$の任意の要素は順序数であるから,定理\ref{thm:On_is_wellordered}より
		\begin{align}
			\forall x,y \in \bigcup a\ (\ x \in y \vee x = y \vee y \in x\ )
		\end{align}
		も成り立つ.最後に$\operatorname{Tran}(\bigcup a)$が成り立つことを示す.
		$b$を$\bigcup a$の任意の要素とすれば,$a$の或る要素$x$に対して
		\begin{align}
			b \in x
		\end{align}
		となるが,$x$の推移性より$b \subset x$となり,$x \subset \bigcup a$と併せて
		\begin{align}
			b \subset \bigcup a
		\end{align}
		が従う.
		\QED
	\end{prf}
	
	\begin{screen}
		\begin{thm}[Burali-Forti]\label{thm:Burali_Forti}
			順序数の全体は集合ではない.
			\begin{align}
				\rightharpoondown \set{\ON}.
			\end{align}
		\end{thm}
	\end{screen}
	
	\begin{prf}
		$a$を類とするとき,定理\ref{thm:satisfactory_set_is_an_element}より
		\begin{align}
			\ord{a} \Longrightarrow \left(\, \set{a} \Longrightarrow a \in \ON\, \right)
		\end{align}
		が成り立つ.定理\ref{thm:On_is_transitive}と定理\ref{thm:On_is_wellordered}より
		\begin{align}
			\ord{\ON}
		\end{align}
		が成り立つから
		\begin{align}
			\set{\ON} \Longrightarrow \ON \in \ON
			\label{eq:Burali_Forti_1}
		\end{align}
		が従い,また定理\ref{thm:no_set_is_an_element_of_itself}より
		\begin{align}
			\ON \notin \ON
		\end{align}
		も成り立つので,(\refeq{eq:Burali_Forti_1})の対偶から
		\begin{align}
			\rightharpoondown \set{\ON}
		\end{align}
		が成立する.
		\QED
	\end{prf}
	
	\begin{screen}
		\begin{thm}[順序数は自分自身との合併が後者となる]\label{thm:latter_element_is_ordinal}
			$\alpha$が順序数であるということと $\alpha \cup \{\alpha\}$が順序数であるということは同値である.
			\begin{align}
				\forall \alpha\ (\ \alpha \in \ON \Longleftrightarrow \alpha \cup \{\alpha\} \in \ON\ ).
			\end{align}
		\end{thm}
	\end{screen}
	
	\begin{screen}
		\begin{thm}[順序数は自分自身との合併が後者となる]
			$\alpha$を順序数とすれば,$\ON$において$\alpha \cup \{\alpha\}$は$\alpha$の後者である:
			\begin{align}
				\forall \alpha \in \ON\ 
				\left(\ \forall \beta \in \ON\ (\ \alpha < \beta 
				\Longrightarrow \alpha \cup \{\alpha\} \leq \beta\ )
				\ \right).
			\end{align}
		\end{thm}
	\end{screen}
	\section{再帰的定義}
\label{sec:recursive_definition}
	例えば
	\begin{align}
		a_1,\quad a_2,\quad a_3,\quad a_4,\quad \cdots\quad a_n,\quad \cdots
	\end{align}
	なる列が与えられたときに,その$n$重の順序対を
	\begin{align}
		(a_1,a_2,\cdots,a_n)
	\end{align}
	などと書くことがある.まあ
	\begin{align}
		(a_0,a_1)
	\end{align}
	ならば単なる順序対であり,
	\begin{align}
		(a_0,a_1,a_2)
	\end{align}
	も
	\begin{align}
		((a_0,a_1),a_2)
	\end{align}
	で定められ,
	\begin{align}
		(a_0,a_1,a_2,a_3)
	\end{align}
	も
	\begin{align}
		(((a_0,a_1),a_2),a_3)
	\end{align}
	で定められる.このように具体的に全ての要素を書き出せるうちは何も問題は無い.
	ただし,同じ操作を$n$回反復するということを表現するために
	\begin{align}
		\cdots
	\end{align}
	なる不明瞭な記号を無断で用いることは$\mathcal{L}'$において許されない.
	そもそもまだ``$n$回の反復''をどんな式で表現したら良いかもわからないのである.
	次の定理は,このような再帰的な操作が$\mathcal{L}'$で可能であることを保証する.
	
	\begin{screen}
		\begin{thm}[超限帰納法による写像の構成]
			類$G$を$\Univ$上の写像とするとき,
			\begin{align}
				K \defeq \Set{f}{\exists \alpha \in \ON\ \left(\ f:\alpha \longrightarrow V \wedge \forall \beta \in \alpha\ (\ f(\beta) = G(f|_\beta)\ )\ \right)}
			\end{align}
			とおいて
			\begin{align}
				F \defeq \bigcup K
			\end{align}
			と定めると,$F$は$\ON$上の写像であって
			\begin{align}
				\forall \alpha \in \ON\ (\ F(\alpha) = G(F|_\alpha)\ )
			\end{align}
			を満たす.また$\ON$上の写像で上式を満たすのは$F$のみである.
		\end{thm}
	\end{screen}
	
	\begin{prf}\mbox{}
		\begin{description}
			\item[第二段] $F$が写像であることを示す.
				まず$K$の任意の要素は$V \times V$の部分集合であるから
				\begin{align}
					F \subset V \times V
				\end{align}
				となる.$x,y,z$を任意の集合とする.
				$(x,y) \in F$かつ$(x,z) \in F$のとき,
				$K$の或る要素$f$と$g$が存在して
				\begin{align}
					(x,y) \in f \wedge (x,z) \in g
				\end{align}
				を満たすが,ここで$f(x) = g(x)$となることを言うために,
				$\alpha = \operatorname{dom}(f),\ 
				\beta = \operatorname{dom}(g)$とおき,
				\begin{align}
					\forall \gamma \in \ON\ (\ \gamma \in \alpha \wedge \gamma \in \beta \Longrightarrow f(\gamma) = g(\gamma)\ )
					\label{eq:thm_transfinite_recursion_theorem_1}
				\end{align}
				が成り立つことを示す.いま$\gamma$を任意の順序数とする.$\gamma = \emptyset$の場合は
				$f|_\gamma = \emptyset$かつ$g|_\gamma = \emptyset$となるから
				\begin{align}
					f(\gamma) = G(\emptyset) = g(\gamma)
				\end{align}
				が成立する.$\gamma \neq \emptyset$の場合は
				\begin{align}
					\forall \xi \in \gamma\ (\ \xi \in \alpha \wedge \xi \in \beta \Longrightarrow f(\xi) = g(\xi)\ )
				\end{align}
				が成り立っていると仮定する.このとき$\gamma \in \alpha \wedge \gamma \in \beta$ならば
				順序数の推移性より$\gamma$の任意の要素$\xi$は$\xi \in \alpha \wedge \xi \in \beta$を満たすから
				\begin{align}
					\forall \xi \in \gamma\ (\ f(\xi) = g(\xi)\ )
				\end{align}
				が成立する.従って
				\begin{align}
					f|_\gamma = g|_\gamma
				\end{align}
				が成立するので$f(\gamma) = g(\gamma)$が得られる.超限帰納法より
				(\refeq{eq:thm_transfinite_recursion_theorem_1})が得られる.
				以上より
				\begin{align}
					y = f(x) = g(x) = z
				\end{align}
				となるので$F$はsingle-valuedである.
			
			\item[第三段] $\operatorname{dom}(F) \subset \ON$が成り立つことを示す.
				実際
				\begin{align}
					\operatorname{dom}(F) = \bigcup_{f \in K} \operatorname{dom}(f)
				\end{align}
				かつ$\forall f \in K\ (\ \operatorname{dom}(f) \subset \ON\ )$だから
				$\operatorname{dom}(F) \subset \ON$となる.
				
			\item[第四段] $\operatorname{Tran}(\operatorname{dom}(F))$であることを示す.
				実際任意の集合$x,y$について
				\begin{align}
					y \in x \wedge x \in \operatorname{dom}(F)
				\end{align}
				が成り立っているとき,或る$f \in K$で$x \in \operatorname{dom}(f)$
				となり,$\operatorname{dom}(f)$は順序数なので,順序数の推移律から
				\begin{align}
					y \in \operatorname{dom}(f)
				\end{align}
				が従う.ゆえに$y \in \operatorname{dom}(F)$となる.
				
			\item[第五段] $\forall \alpha \in \operatorname{dom}(F)\ (\ F(\alpha) = G(F|_\alpha)\ )$が成り立つことを示す.
				実際,$\alpha \in \operatorname*{dom}(F)$なら
				$K$の或る要素$f$に対して$\alpha \in \operatorname*{dom}(f)$となるが,
				$f \subset F$であるから
				\begin{align}
					f(\alpha) = F(\alpha)
				\end{align}
				が成り立つ.これにより$f|_\alpha = f \cap (\alpha \times V)
				= F \cap (\alpha \times V) = F|_\alpha$より
				\begin{align}
					G(f|_\alpha) = G(F|_\alpha)
				\end{align}
				も成り立つ.$f(\alpha) = G(f|_\alpha)$と併せて
				$F(\alpha) = G(F|_\alpha)$を得る.
			
			\item[第六段] 
				$\alpha$を任意の順序数として
				$\forall \beta \in \alpha\ (\ \beta \in \operatorname{dom}(F)\ )
				\Longrightarrow \alpha \in \operatorname{dom}(F)$が成り立つことを示す.
				$\alpha = \emptyset$の場合は
				\begin{align}
					\forall f \in K\ (\ \operatorname{dom}(f) \neq \emptyset
					\Longrightarrow \emptyset \in \operatorname{dom}(f)\ )
				\end{align}
				が満たされるので$\alpha \in \operatorname{dom}(F)$となる
				(定理\ref{thm:properties_of_ordinal_numbers}).
				$\alpha \neq \emptyset$の場合,
				\begin{align}
					\forall \beta \in \alpha\ (\ \beta \in \operatorname{dom}(F)\ )
				\end{align}
				が成り立っているとして$f = F|_\alpha$とおけば,$f$は$\alpha$上の写像であり,
				$\alpha$の任意の要素$\beta$に対して
				\begin{align}
					f(\beta)
					= F|_\alpha(\beta)
					= F(\beta)
					= G(F|_\beta)
					= G(f|_\beta)
				\end{align}
				を満たすから$f \in K$である.このとき$f' = f \cup \{(\alpha,G(f))\}$も
				$K$に属するので
				\begin{align}	
					\alpha \in \operatorname{dom}(f') \subset
					\operatorname{dom}(F)
				\end{align}
				が成立する.超限帰納法より
				\begin{align}
					\forall \alpha \in \ON\ (\ \alpha \in \operatorname{dom}(F)\ )
				\end{align}
				が成立し,前段の結果と併せて
				\begin{align}
					\ON = \operatorname{dom}(F)
				\end{align}
				を得る.
				
			\item[第七段]
				$F$の一意性を示す.類$H$が
				\begin{align}
					H:\ON \longrightarrow V 
					\wedge \forall \alpha \in \ON\ (\ H(\alpha) = G(H|_\alpha)\ )
				\end{align}
				を満たすとき,$F = H$が成り立つことを示す.
				いま,$\alpha$を任意に与えられた順序数とする.$\alpha = \emptyset$の場合は
				\begin{align}
					F|_\emptyset = \emptyset = H|_\emptyset
				\end{align}
				より$F(\emptyset) = H(\emptyset)$となる.$\alpha \neq \emptyset$の場合,
				\begin{align}
					\forall \beta \in \alpha\ (\ F(\beta) = H(\beta)\ )
				\end{align}
				が成り立っていると仮定すれば
				\begin{align}
					F|_\alpha = H|_\alpha
				\end{align}
				が成り立つから$F(\alpha) = H(\alpha)$となる.以上で
				\begin{align}
					\forall \alpha \in \ON\ \left(\ \forall \beta \in \alpha\ 
					(\ F(\beta) = H(\beta)\ ) \Longrightarrow F(\alpha) = H(\alpha)\ \right)
				\end{align}
				が得られた.超限帰納法より
				\begin{align}
					\forall \alpha \in \ON\ (\ F(\alpha) = H(\alpha)\ )
				\end{align}
				が従い$F = H$が出る.
				\QED
		\end{description}
	\end{prf}
	
	\begin{itembox}[l]{再帰的定義の応用 : 多数の要素からなる順序対}
		$a$を$\Natural$から集合$A$への写像とすると,
		\begin{align}
			a_n \defeq a(n)
		\end{align}
		と書けば
		\begin{align}
			a_0, a_1, a_2, \cdots
		\end{align}
		なる列が作られる.ここでは
		\begin{align}
			(a_0,a_1,\cdots, a_n)
		\end{align}
		のような記法の集合論的意味付けを考察する.
	\end{itembox}
	
		$\Univ$上の写像$G$を
		\begin{align}
			G(x) = 
			\begin{cases}
				a_0 & \mbox{if } \dom{x} = \emptyset \\
				(x(k),a(\dom{x})) & \mbox{if } \dom{x} = k \cup \{k\} \wedge k \in \Natural \\
				\emptyset & \mbox{o.w.}
			\end{cases}
		\end{align}
		によって定めてみると,つまり$G$とは
		\begin{align}
			\{\, (x,y) \mid \quad &\left(\, \dom{x} = \emptyset \Longrightarrow y = a_0\, \right) \\
		&\wedge \forall k \in \Natural\, \left(\, \dom{x} = k \cup \{k\} \Longrightarrow y = (x(k),a(\dom{x}))\, \right) \\
		&\wedge \left[\, \dom{x} \neq \emptyset \wedge \forall k \in \Natural\, \left(\, \dom{x} \neq k \cup \{k\}\, \right)
		\Longrightarrow y = \emptyset\, \right]\, \}
		\end{align}
		のことであるが,$\ON$上の写像$p$で
		\begin{align}
			p(n) =
			\begin{cases}
				a_0 & \mbox{if } (n = 0) \\
				(a_0,a_1) & \mbox{if } (n=1) \\
				((a_0,a_1),a_2) & \mbox{if } (n=2) \\
				(((a_0,a_1),a_2),a_3) & \mbox{if } (n=3)
			\end{cases}
		\end{align}
		を満たすものが取れる.先の
		\begin{align}
			(a_0,a_1,\cdots, a_n)
		\end{align}
		という一見不正確であった記法は,この
		\begin{align}
			p(n)
		\end{align}
		によって定めると決めてしまえば無事解決である.
	
	%\input{thms/Peano_system}
	\section{半群}
	\begin{screen}
		\begin{dfn}[算法]
			$a$を類とするとき,$a \times a$から$a$への写像を$a$上の
			{\bf 算法}\index{さんぽう@算法}{\bf (operation)}と呼ぶ.
		\end{dfn}
	\end{screen}
	
	いま,$a$を類とし,$o$を$a$上の算法とする.
	\begin{description}
		\item[可換律\index{かかんりつ@可換律} (commutative law)] $\forall x,y \in a\, \left(\, o(x,y) = o(y,x)\, \right)$.
		\item[結合律\index{けつごうりつ@結合律} (associative law)] $\forall x,y,z \in a\, \left(\, o(o(x,y),z) = o(x,o(y,z))\, \right)$.
		\item[簡約律\index{かんやくりつ@簡約律} (cancellation law)] $\forall x,y,z \in a\, \left(\, o(x,z) = o(y,z) \Longrightarrow x = y\, \right)$.
	\end{description}
	
	$\ON$上の加法と乗法は$\ON$上の算法である.
	それらは結合律を満たす一方で可換律と簡約律は満たさないが,
	定義域を$\Natural \times \Natural$上に制限すれば全てを満たすようになる.ここで
	\begin{align}
		+_\Natural \defeq +|_{\Natural \times \Natural}
	\end{align}
	及び
	\begin{align}
		\cdot_\Natural \defeq \cdot|_{\Natural \times \Natural}
	\end{align}
	として$\Natural$上の加法と乗法を定義する.
	
	\begin{screen}
		\begin{dfn}[半群]
			$a$を集合とし,$o$を$a$上の算法とする.$o$が結合律を満たしているとき対$(a,o)$を
			{\bf 半群}\index{はんぐん@半群}{\bf (semi-group)}と呼ぶ.
			また$o$が結合律と可換律を満たすとき$(a,o)$を
			{\bf 可換半群}\index{かかんはんぐん@可換半群}{\bf (commutative semi-group)}と呼び,
			$o$が結合律と簡約律を満たすとき$(a,o)$を
			{\bf 簡約的半群}\index{かんやくてきはんぐん@簡約的半群}{\bf (cancellable semi-group)}と呼ぶ.
		\end{dfn}
	\end{screen}
	
	\begin{screen}
		\begin{thm}[$\omg$は加法に関して半群となる]
			$(\Natural,+_\Natural)$は簡約的可換半群である.
		\end{thm}
	\end{screen}
	
	\begin{sketch}
		
	\end{sketch}
	
	\begin{screen}
		\begin{dfn}[一般結合法則]
			空でない集合$S$に次を満たす二項演算$\ast:S \times S \longrightarrow S$
			が定義されているとき,$a_1,a_2,a_3,a_4$を$S$の元として,
			$a_1,a_2,a_3,a_4$の並びを替えずに$\ast$で評価していくと
			\begin{align}
				(a_1 \ast (a_2 \ast a_3)) \ast a_4,
				\quad ((a_1 \ast &a_2) \ast a_3) \ast a_4,
				\quad (a_1 \ast a_2) \ast (a_3 \ast a_4), \\
				\quad a_1 \ast (a_2 \ast (a_3 \ast a_4)),
				&\quad a_1 \ast ((a_2 \ast a_3) \ast a_4)
			\end{align}
			の5通りの評価法が考えうるが(括弧の中を優先して評価する),これは
			\begin{align}
				a_1 \ast a_2 \ast a_3 \ast a_4
			\end{align}
			の3つの$\ast$に演算の順番を付けることに対応している.
			特に,この場合は$\ast$が結合律を満たしていれば5通りの評価は全て同値になる.
			一般に$n$個の$a_1,a_2,\cdots,a_n \in S$を取りこれらに対して$n-1$回の評価を行うとき,
			$a_1,a_2,\cdots,a_n$の並びを替えない限り演算の順番をどう設定しても
			得られる結果に影響しない(最終的な評価がただ一つに確定する)ならば,
			$\ast$は{\bf 一般結合法則}\index{いっぱんけつごうほうそく@一般結合法則}
			{\bf (generalized associative law)}を満たすという.またその結果を
			\begin{align}
				a_1 \ast a_2 \ast \cdots \ast a_n
			\end{align}
			と書く.
		\end{dfn}
	\end{screen}
	
	\begin{screen}
		\begin{thm}[結合法則から一般結合法則が従う]
		\label{thm:generalized_associative_law_on_semigroup}
			$(S,\ast)$を半群とするとき$\ast$は一般結合法則を満たす.
		\end{thm}
	\end{screen}
	
	\begin{prf}
		$n > 3$を選ぶとき,
		任意の$k$個$(3 \leq k < n)$の元に対する演算の結果が評価順に依存しないと仮定すると
		$n$個の元に対する演算の結果も評価順に依存せず確定することを示す.
		$a_1,a_2,\cdots,a_n \in S$に対し,並びを替えずに$n-1$回評価するとき,
		$n-1$回目の演算は
		\begin{align}
			(\mbox{$a_1,a_2,\cdots,a_k$に対する評価}) \ast
			(\mbox{$a_{k+1},a_{k+2},\cdots,a_n$に対する評価})
			\label{eq:thm_generalized_associative_law_on_semigroup}
		\end{align}
		となる.ただし$k$は$1 \leq k \leq n-1$を満たす.仮定より第一項と第二項について
		\begin{align}
			\mbox{(第一項)} &= (\cdots((a_1 \ast a_2) \ast a_3)\cdots) \ast a_k, \\
			\mbox{(第二項)} &= a_{k+1} \ast (\cdots(a_{n-2} \ast (a_{n-1} \ast a_n))\cdots)
		\end{align}
		が成り立つから,ここで$\ast$の結合律を繰り返し用いることにより
		\begin{align}
			(\refeq{eq:thm_generalized_associative_law_on_semigroup}) 
			&= ((\cdots((a_1 \ast a_2) \ast a_3)\cdots) \ast a_k) \ast (a_{k+1} \ast (\cdots(a_{n-2} \ast (a_{n-1} \ast a_n))\cdots)) \\
			&= (((\cdots((a_1 \ast a_2) \ast a_3)\cdots) \ast a_k) \ast a_{k+1}) \ast (a_{k+2} \ast (\cdots(a_{n-2} \ast (a_{n-1} \ast a_n))\cdots)) \\
			&\vdots \\
			&= ((\cdots((a_1 \ast a_2) \ast a_3)\cdots) \ast a_{n-2}) \ast (a_{n-1} \ast a_n) \\
			&= (((\cdots((a_1 \ast a_2) \ast a_3)\cdots) \ast a_{n-2}) \ast a_{n-1}) \ast a_n
		\end{align}
		が得られる.$3$個の元の演算は評価順に依らないから,数学的帰納法より$\ast$は一般結合法則を満たす.
		\QED
	\end{prf}

\section{整数}
	\begin{screen}
		\begin{dfn}[商集合]
			$a$を集合とし,$R$を$a$上の同値関係とする.$x$を$a$の要素とするとき
			\begin{align}
				\Set{y}{(y,x) \in R}
			\end{align}
			を$x$の$R$に関する{\bf 同値類}\index{どうちるい@同値類}{\bf (equivalence class)}と呼び,
			$[x]$などで表す.また
			\begin{align}
				a / R \coloneqq \Set{x}{\exists y \in a\ \forall z\ (\ (y,z) \in R \Longleftrightarrow z \in x\ )}
			\end{align}
			で定められる類$a/R$を,$a$を$R$で割った
			{\bf 商集合}\index{しょうしゅうごう@商集合}{\bf (quotient set)}と呼ぶ.
		\end{dfn}
	\end{screen}
	
	$a$が空であれば$R$も$a/R$も空となる.
	
	\monologue{
		院生「商``集合''と名前を付けましたが,集合であることは後で示します.また上の定義の設定の下では
			\begin{align}
				a/R = \Set{x}{\exists y \in a\ (\ x = [y]\ )}
			\end{align}
			が成り立ちます.これも後で証明しますが,商集合とは同値類の集まりであるということが判るでしょう.」
	}
	
	\begin{screen}
		\begin{thm}[同値類の性質]
			$a$を集合とし,$R$を$a$上の同値関係として,$y$を$a$の要素とするとき$y$の$R$に関する同値類を
			$[y]$で表す.このとき次が成り立つ:
			\begin{description}
				\item[(1)] $\forall y \in a\ \left(\ [y] \subset a\ \right)$
				\item[(2)] $\forall y \in a\ \left(\ y \in [y]\ \right)$
				\item[(3)] $\forall y,z \in a\ \left(\ (y,z) \in R \Longleftrightarrow [y] = [z]\ \right)$
				\item[(4)] $\forall y,z \in a\ \left(\ (y,z) \notin R \Longleftrightarrow [y] \cap [z] = \emptyset\ \right)$
			\end{description}
		\end{thm}
	\end{screen}
	
	\begin{prf} $a$が空であれば空虚な真より(1)(2)(3)(4)は全て成立する.以下では$a \neq \emptyset$として証明する.
		\begin{description}
			\item[(1)] $s,t$を$\mathcal{L}$の任意の対象とするとき,$s \in a$であれば
				\begin{align}
					t \in [s] \Longrightarrow (s,t) \in R
				\end{align}
				が成り立つ.$R \subset a \times a$より
				\begin{align}
					(s,t) \in R \Longrightarrow t \in a
				\end{align}
				が従い
				\begin{align}
					t \in [s] \Longrightarrow t \in a
				\end{align}
				が得られる.$t$の任意性より
				\begin{align}
					[s] \subset a
				\end{align}
				となり,$s$の任意性より(1)が出る.
				
			\item[(2)] $t$を$\mathcal{L}$の任意の対象とするとき,$t \in a$であれば
				\begin{align}
					(t,t) \in R
				\end{align}
				となるから$t \in [t]$が成立する.$t$の任意性より
				\begin{align}
					\forall y \in a\ \left(\ y \in [y]\ \right)
				\end{align}
				が得られる.
				
			\item[(3)] $s,t$を$\mathcal{L}$の任意の対象として,$s,t \in a$であると仮定する.
				\begin{align}
					(s,t) \in R
				\end{align}
				が成り立っているとき,$\tau$を$\mathcal{L}$の任意の対象とすれば
				\begin{align}
					\tau \in [s] \Longleftrightarrow (\tau,s) \in R
				\end{align}
				となり,$R$の推移律より$(\tau,s) \in R$ならば$(\tau,t) \in R$となるから
				\begin{align}
					\tau \in [s] \Longrightarrow \tau \in [t]
				\end{align}
				が従う.同様に$\tau \in [t] \Longrightarrow \tau \in [s]$も成り立つので
				$[s] = [t]$となり
				\begin{align}
					(s,t) \in R \Longrightarrow [s] = [t]
				\end{align}
				が得られる.逆に$[s] = [t]$が成り立っているとき,$s \in [s]$より$s \in [t]$が従い
				\begin{align}
					[s] = [t] \Longrightarrow (s,t) \in R
				\end{align}
				も得られる.$s,t$の任意性より(2)が出る.
			
			\item[(4)] $s,t$を$\mathcal{L}$の任意の対象として,$s,t \in a$であると仮定する.
				\begin{align}
					[s] \cap [t] \neq \emptyset
				\end{align}
				が成り立っているとき,$[s] \cap [t]$の要素を$u$とすれば
				\begin{align}
					(s,u) \in R \wedge (u,t) \in R
				\end{align}
				となるので$(s,t) \in R$が従う.ゆえに
				\begin{align}
					(s,t) \notin R \Longrightarrow [s] \cap [t] = \emptyset
				\end{align}
				が得られる.逆に$(s,t) \in R$が成り立っているとき,(2)より$[s] = [t]$となるから
				\begin{align}
					[s] \cap [t] \neq \emptyset \Longrightarrow (s,t) \in R
				\end{align}
				も得られる.$s,t$の任意性より(3)が出る.
				\QED
		\end{description}
	\end{prf}
	
	\monologue{
		院生「(1)の主張は{\bf 同値類は空でない}ということですね.
			(2)の主張は{\bf 同値な要素の同値類は一致する}ということで,
			(2)と(3)を併せれば{\bf 同値類同士は一致していなければ交わらない}と言えます.」
	}
	
	\begin{screen}
		\begin{dfn}[商写像]
			$a$を集合とし,$R$を$a$上の同値関係として,$y$を$a$の要素とするとき$y$の$R$に関する同値類を
			$[y]$で表す.このとき
			\begin{align}
				f \coloneqq \Set{x}{\exists t \in a\ \left(\ x=(t,[t])\ \right)}
			\end{align}
			で定められる$f$を{\bf 商写像}\index{しょうしゃぞう@商写像}{\bf (quotient mapping)}と呼ぶ.
		\end{dfn}
	\end{screen}
	
	\monologue{
		院生「$f$が写像であることを述べる前に商写像と名前を付けましたが,以下に示す通り
			$f$は$a$から$a/R$への全射となっています.また商写像は
			{\bf 自然な全射}\index{しぜんなぜんしゃ@自然な全射}{\bf (natural surjection)}や
			{\bf 標準的全射}\index{ひょうじゅんてきぜんしゃ@標準的全射}{\bf (canonical surjection)}
			とも呼ばれます.」
	}
	
	\begin{screen}
		\begin{thm}[商写像は全射である]\label{thm:quotient_mapping_is_a_surjection}
			$a$を集合とし,$R$を$a$上の同値関係として,$y$を$a$の要素とするとき$y$の$R$に関する同値類を
			$[y]$で表す.このとき次が成り立つ:
			\begin{align}
				\forall x\ \left(\ x \in a/R \Longleftrightarrow \exists y \in a\ (\ x=[y]\ )\ \right).
				\label{eq:thm_quotient_mapping_is_a_surjection}
			\end{align}
			特に$a$から$a/R$への商写像は写像であり,さらに言えば全射である.
		\end{thm}
	\end{screen}
	
	\begin{prf}
		$a$が空である場合は$a/R$が空となるので,空虚な真より(\refeq{eq:thm_quotient_mapping_is_a_surjection})
		が成り立つ.また商写像も空となり,空写像は空集合から空集合への全単射であるから
		主張は全て従う.以下では$a$が空でない場合で証明する.
	\end{prf}
	
	\begin{screen}
		\begin{thm}[商集合の性質]
			$a$を集合とし,$R$を$a$上の同値関係として,$y$を$a$の要素とするとき$y$の$R$に関する同値類を
			$[y]$で表す.このとき次が成り立つ:
			\begin{description}
				\item[(1)] $a/R \in \Univ$
				\item[(2)] $a = \bigcup (a/R)$
			\end{description}
		\end{thm}
	\end{screen}
	
	\begin{prf} $a$が空であれば$a/R$は空となり(1)が成立する.また$\emptyset = \bigcup \emptyset$より
		(2)も成立する.以下では$a$が空でない場合で証明する.
		\begin{description}
			\item[(1)] $a/R$は$a$から$a/R$への商写像の値域であるから,置換公理より$a/R \in \Univ$が従う.
			\item[(2)] $\tau$を$\mathcal{L}$の任意の対象とすれば,$\tau \in a$ならば
				\begin{align}
					\tau \in [\tau]
				\end{align}
				となるから$\tau \in \bigcup (a/R)$が成立する.ゆえに
				\begin{align}
					\tau \in a \Longrightarrow \tau \in \bigcup (a/R)
				\end{align}
				が得られる.逆に$\tau \in \bigcup (a/R)$が成り立っているとすれば
				$\tau$に対して$a$の或る要素$y$が取れて$\tau \in [y]$となるが,
				$[y] \subset a$より$\tau \in a$が従うので
				\begin{align}
					\tau \in \bigcup (a/R) \Longrightarrow \tau \in a
				\end{align}
				も得られる.$\tau$の任意性より$a = \bigcup (a/R)$が出る.
				\QED
		\end{description}
	\end{prf}
	
	\begin{screen}
		\begin{thm}[半群の群への拡張]\label{thm:extension_of_semigroup}
			$(a,o)$を簡約的可換半群とするとき,次を満たす群$(G,\Gamma)$が存在する:
			\begin{itemize}
				\item $a \subset G$.
				\item $\forall x,y \in a\ (\ o(x,y) = \Gamma(x,y)\ )$.
			\end{itemize}
		\end{thm}
	\end{screen}
	
	\begin{screen}
		\begin{dfn}[整数]
			$(\omg,\sigma)$が生成する群を$(\Z,+)$で表す.そして$\Z$の要素を
			{\bf 整数}\index{せいすう@整数}{\bf (integer)}と呼ぶ.
		\end{dfn}
	\end{screen}
	
\section{有理数}
	\begin{screen}
		\begin{thm}[分数体]\label{thm:field_of_fractions}
			環$R$に対し,$R$が整域であるということと$R$が或る体の部分環であるということは同値である.
			$R$を整域とするとき,$R$を部分環として含む最小の体は$R$の{\bf 分数体}
			\index{ぶんすうたい@分数体}{\bf (field of fractions)}と呼ばれる.
		\end{thm}
	\end{screen}
	
	$\Z$は整域であるから,定理\ref{thm:field_of_fractions}より$\Z$を部分環として含む
	体$F$が存在する.$\Z$の任意の要素$n$に対し,$n$が$0$でなければ$F$の中に$n^{-1}$が存在するが,
	この乗法に関する逆元を用いれば$\Z$を部分環として含む最小の体は
	\begin{align}
		\Set{x}{\exists n,m \in \Z\ (\ x = n \cdot m^{-1} \wedge m \neq 0\ )}
	\end{align}
	と書ける.この集合を$\Q$で表し,{\bf 有理数体}\index{ゆうりすうたい@有理数体}{\bf (field of rationals)}と呼ぶ.

\section{実数}
	\begin{screen}
		\begin{dfn}[Dedekind切断]
			$\Q$の任意の部分集合$A$に対して,順序対$(\Q \backslash A,A)$が
			{\bf Dedekind切断}\index{Dedekindせつだん@Dedekind切断}{\bf (Dedekind cut)}であるということを
			\begin{align}
				\mbox{順序対$(\Q \backslash A,A)$がDedekind切断である} \Longleftrightarrow\ 
				&A \neq \emptyset \wedge A \neq \Q\ \wedge \\
				&\forall x \in \Q \backslash A\ \forall y \in A\ (\ x < y\ )\ \wedge \\
				&\forall x \in A\ \exists y \in A\ (\ y < x\ )
			\end{align}
			で定義する.
		\end{dfn}
	\end{screen}
	
	\monologue{
		院生「Dedekind切断とは数直線を左右に分割する操作をイメージしますね.例えば
			\begin{align}
				A = \Set{q \in \Q}{0 < q}
			\end{align}
			に対して$(\Q \backslash A,A)$はDedekind切断となります.
			実数の構成においてこの集合$A$は重要ですから,これを$\Q_+$と表して後で使いましょう.
			上の定義では$(\Q \backslash A,A)$がDedekind切断であるというとき
			$A$が最小元をもたないことを条件に入れましたが,ここは
			`$\Q \backslash A$が最大元を持たない'という条件に取り替えても構いません.」
	}
	
	いま$R = \Set{x}{\mbox{$(\Q \backslash x,x)$はDedekind切断である}}$として$R$を定め,
	\begin{align}
		T = \Set{x}{\exists a,b \in R\ (\ x = (a,b) \wedge b \subset a\ )}
	\end{align}
	と定める.この$T$は$R$上の全順序となる.
	任意の$a,b \in R$に対して,$a \not\subset b$ならば
	或る有理数$x$が$x \in a$かつ$x \notin b$を満たす.
	このとき$b$の任意の要素$y$に対して$x < y$となり,
	$x \in a$かつ$x < y$より$y \in a$となるので$b \subset a$が成り立つ.ゆえに
	\begin{align}
		\rightharpoondown (a \subset b) \Longrightarrow b \subset a
	\end{align}
	が得られた.これは$a \subset b \vee b \subset a$と同値であるから$T$は全順序である.
	
	\monologue{
		さて,高校まで扱ってきた数は`切れ目'がありませんでした.
		つまり,まるで時間の流れのように数直線は`連続'していたのです.
		集合論のことばで`数の連続性'を規定するとどうなるでしょう.
		それには同値な条件がいくつかありますが,今回述べるものは
		`上に有界な部分集合は上限を有する'という性質です.
	}
	
	$X$を$R$の部分集合で,$X \neq \emptyset$かつ$X$は$R$において上に有界であるとする.
	このとき$\bigcup X$は$X$の上限となる.
	
	\begin{screen}
		\begin{thm}
			$\Q$の部分集合$A$に対して$(\Q \backslash A,A)$をDedekind切断とするとき,
			次が成り立つ:
			\begin{description}
				\item[(1)] $\forall q \in \Q\ (\ \exists a \in A\ (\ a < q\ )\Longleftrightarrow q \in A\ )$.
				\item[(2)] $\forall q \in \Q\ (\ \exists a \in \Q \backslash A\ (\ q < a\ )\Longrightarrow q \in \Q \backslash A\ )$.
			\end{description}
		\end{thm}
	\end{screen}
	
	\begin{prf}
		$q$を任意の有理数とすれば,$A$は最小元を持たないので
		\begin{align}
			q \in A \Longrightarrow \exists a \in A\ (\ a < q\ )
		\end{align}
		となる.逆に$q \notin A$ならば$A$の任意の要素$a$に対して$q < a$となるから,対偶を取って
		\begin{align}
			\exists a \in A\ (\ a < q\ ) \Longrightarrow q \in A
		\end{align}
		を得る.$q \notin \Q \backslash A$ならば$A$の任意の要素$a$に対して$a < q$となるから,
		対偶を取って(2)を得る.
		\QED
	\end{prf}
	\section{イデアル}
	\monologue{
		まず和の記号$\sum$を定めましょう.例えば,いま実数の列
		\begin{align}
			a_0,\ a_1,\ a_2,\ a_3,\ \cdots
		\end{align}
		が与えられたとすれば,その$n$個の和
		\begin{align}
			a_0 + a_1 + \cdots + a_{n-1}
		\end{align}
		を$\sum$を用いて
		\begin{align}
			\sum_{i=0}^{n-1} a_i
		\end{align}
		と書くように定めれば便利です.$n$個の和とは直感的には添え字を順に辿って$n$個の要素を
		合計すれば良いだけですが,その操作を$\mathcal{L}'$の言葉で表現しなくては数学ではありません.
		我々が使える道具の中で,順番に足すという再帰的な操作を表現するには写像の概念が最適でしょう.
	}
	
	\begin{screen}
		\begin{dfn}[イデアル]
			$(R,\sigma,\mu)$を環とするとき,$R$の部分集合$J$が
			\begin{itemize}
				\item $\forall a,b \in J\ (\ \sigma(a,b) \in J\ )$
				\item $\forall a \in J\ \forall r \in R\ (\ \mu(r,a) \in J\ )$
			\end{itemize}
			を満たすとき,$J$を$R$の{\bf 左イデアル}\index{ひたりいである@左イデアル}{\bf (left ideal)}と呼ぶ.
			また二つ目の条件を
			\begin{itemize}
				\item $\forall a \in J\ \forall r \in R\ (\ \mu(a,r) \in J\ )$
			\end{itemize}
			に取り替えた場合,$J$を$R$の{\bf 右イデアル}\index{みぎいである@右イデアル}{\bf (right ideal)}と呼ぶ.
			左イデアルであり右イデアルでもある部分集合を{\bf イデアル}\index{いである@イデアル}{\bf (ideal)}と呼ぶ.
		\end{dfn}
	\end{screen}
	
	考察対象は主に左イデアルである.左右を反転させれば左イデアルに関する結果は右イデアルにも当てはまる.
	
	\begin{screen}
		\begin{thm}[左イデアルは加法に関して群をなす]
			$(R,\sigma,\mu)$を環とし,$J$をこの環の左イデアルとするとき,
			\begin{align}
				\sigma_J \coloneqq \sigma|_{J \times J}
			\end{align}
			とおけば$(J,\sigma_J)$は可換群となる.
		\end{thm}
	\end{screen}
	
	\monologue{
		つまり,左イデアルとは左側からの掛け算で閉じている加法部分群であると言えます.
	}
	
	
	\subsection{多項式環}
	$(R,\sigma,\mu)$を可換環として,その零元と単位元をそれぞれ$\zeta$と$\epsilon$で表す.
	また$\zeta \neq \epsilon$と仮定する.すなわち$(R,\sigma,\mu)$は零環ではない.いま
	\begin{align}
		\tilde{P} \coloneqq \Set{f}{f:\omg \longrightarrow R \wedge 
		\exists n \in \omg\ \forall m \in \omg\ (\ n < m \Longrightarrow f(m) = \zeta\ )}
	\end{align}
	により集合$\tilde{P}$を定める.$\tilde{P}$とは$\omg$から$R$への写像のうち
	或る自然数以降は$\zeta$に張り付いてしまう写像の全体である.$a$を$R$の要素として
	\begin{align}
		\varphi_a \coloneqq \Set{x}{\exists n \in \omg\ (\ n = 0 \Longrightarrow x = (0,a)
		\wedge n \neq 0 \Longrightarrow x = (n,\zeta)\ )}
	\end{align}
	として$\varphi_a$を定めれば,$\varphi_a$は$\omg$から$R$への写像であり
	\begin{align}
		\varphi_a(n) = 
		\begin{cases}
			a, & (n=0), \\
			\zeta, & (n \neq 0)
		\end{cases}
	\end{align}
	を満たすから$\tilde{P}$の要素でもある.ここで
	\begin{align}
		\varphi \coloneqq \Set{x}{\exists a \in R\ \left(\ x=(a,\varphi_a)\ \right)}
	\end{align}
	として$\varphi$を定めれば
	\underline{$\varphi$は$R$から$\tilde{P}$への埋め込み(単射環準同型)となる}.
	
	\begin{prf}
	\end{prf}
	
	\monologue{
		院生[唐突に出てきた$\tilde{P}$ですが,なぜわざわざ波線記号を載せているのか,
		本節の主題である多項式と$\tilde{P}$がどう関係しているかということを説明していきます.
		以降も回りくどい説明が続きますから,多項式環が得られる過程を簡略して述べましょう.いま
		\begin{align}
			X \coloneqq \Set{x}{\exists n \in \omg\ (\ n = 1 \Longrightarrow x=(1,\epsilon) \wedge n \neq 1 \Longrightarrow x=(n,\zeta)\ )}
		\end{align}
		とおくと,$X$は
		\begin{align}
			X(n) =
			\begin{cases}
				\epsilon, & (n = 1), \\
				\zeta, & (n \neq 1)
			\end{cases}
		\end{align}
		を満たす$\omg$から$R$への写像ですから$\tilde{P}$の要素です.ちなみに
		$X$を点列の様式で(不正確な書き方ですが直感的に解釈するには都合が良いでしょう)
		\begin{align}
			(\zeta,\ \epsilon,\ \zeta,\ \zeta,\ \zeta,\ \cdots)
		\end{align}
		と書いてみましょう.すると$X^2$や$X^3$は
		\begin{align}
			&(\zeta,\ \zeta,\ \epsilon,\ \zeta,\ \zeta,\ \cdots), \\
			&(\zeta,\ \zeta,\ \zeta,\ \epsilon,\ \zeta,\ \cdots)
		\end{align}
		と表すことが出来ますし,特に$R$の要素$a$に対して$\varphi(a) \cdot X^n$は
		\begin{align}
			(\zeta, \zeta,\ \cdots,\ \zeta,\ a,\ \zeta,\ \zeta,\ \cdots)
		\end{align}
		と書くことが出来ますね.ここで$f$を$\tilde{P}$の要素とすれば,$f$は$R$の有限個の要素
		$a_0,a_1,\cdots.a_m$を用いて
		\begin{align}
			(a_0,\ a_1,\ \cdots,\ a_m,\ \zeta,\ \zeta,\ \zeta,\ \cdots)
		\end{align}
		と表すことが出来ますから
		\begin{align}
			f = \varphi(a_0) + \varphi(a_1) \cdot X + \varphi(a_2) \cdot X^2 + \cdots + \varphi(a_m) \cdot X^m
		\end{align}
		が成り立つのです.こうして$X$の冪を有限個連ねた式が出来ましたが,これは
		まだ多項式の卵の段階です.$\varphi$が余計ですから少し手を加えて整形しますと,
		多項式環というものが得られるという寸法です.」
	}
	
	上で作った$\tilde{P}$に対して
	\begin{align}
		P \coloneqq \left( \tilde{P} \backslash (\varphi \ast R) \right) \cup R
	\end{align}
	と定める.$P$とは$\tilde{P}$の$R$が埋め込まれた部分を$R$そのものに置き換えた集合である.
	また$\tilde{P}$から$P$への写像を
	\begin{align}
		h \coloneqq \{\, x \mid \quad \exists f \in \tilde{P}\ 
		&(\\
		&\quad \exists a \in R\ (\ f = \varphi(a)\ ) \Longrightarrow x = (f,a) \\
		&\quad \wedge f \notin \varphi \ast R \Longrightarrow x = (f,f)\\
		&)\, \}
	\end{align}
	で定めれば$h$は全単射となる.$h$は$\varphi \ast R$の要素には$\varphi$で対応する$R$の要素を
	当て,$\varphi \ast R$の外側では恒等写像となっている.また
	\begin{align}
		\sigma_P &\coloneqq \Set{x}{\exists f,g \in P\ \left(\ 
			x=((f,g),h(h^{-1}(f)+h^{-1}(g)))\ \right)}, \\
		\mu_P &\coloneqq \Set{x}{\exists f,g \in P\ \left(\ 
			x=((f,g),h(h^{-1}(f) \cdot h^{-1}(g)))\ \right)}
	\end{align}
	と定めれば,
	\underline{$\sigma_P$と$\mu_P$をそれぞれ加法と乗法として$(P,\sigma_P,\mu_P)$は可換環となる}.
	
	\monologue{
		院生「下線部の証明の前に注意しておきます.$\sigma_P$も$\mu_P$も
		定義式に括弧が多くて見づらいですが,見やすいように書けば$P$の要素$f,g$に対して
		\begin{align}
			\sigma_P(f,g) &= h(h^{-1}(f)+h^{-1}(g)), \\
			\mu_P(f,g) &= h(h^{-1}(f) \cdot h^{-1}(g))
		\end{align}
		としているのです.つまり,$P$上の算法は$h$で$\tilde{P}$に引き戻して計算したものを
		再び$h$で移すことにより定めているのですね.ゆえに,$h$が環同型となることは殆ど明らかでしょう.」
	}
	
	\begin{prf}
	\end{prf}
	
	またこのとき\underline{$(\tilde{P},\tilde{\sigma},\tilde{\mu})$と$(P,\sigma_P,\mu_P)$は
	環として$h$によって同型に対応する}.
	
	\begin{prf}	
	\end{prf}
	
	$f$を$P$から任意に選ばれた要素とするとき,
	\subsection{素元分解}
	\begin{screen}
		\begin{thm}[単項イデアル整域において素元が生成するイデアルは極大イデアルである]
		\end{thm}
	\end{screen}
	
	\section{複素数}
	$K$を体とし,$p$を多項式環$K[X]$の素元とするとき,$p$の解を含んでいる
	$K$の拡大体を取ることが出来る.そのことを保証するのが次の定理である.
	
	\begin{screen}
		\begin{thm}[単拡大]
			$(K,\sigma,\mu)$を体とし,$p$を$K$上の多項式環の素元とするとき,
			$(K,\sigma,\mu)$の拡大体で$p$
		\end{thm}
	\end{screen}
	\subsection{代数閉包}	
	
	\begin{screen}
		\begin{axm}[選択公理]
			$a$を類とするとき,$a$上の写像$f$で,
			$a$の空でない要素$x$から$f(x)$を選択するもの
			(これを{\bf 選択関数}\index{せんたくかんすう@選択関数}{\bf (choice function)}と呼ぶ)
			が存在する:
			\begin{align}
				\exists f\ \left(\ 
				f:a \longrightarrow V \wedge \forall x \in a\ 
				(\ x \neq \emptyset \Longrightarrow f(x) \in x\ )\ \right). 
			\end{align}
		\end{axm}
	\end{screen}
	
	\begin{screen}
		\begin{thm}[整列可能定理]
			任意の集合は,或る順序数と全単射で結ばれる:
			\begin{align}
				\forall a\ \exists \alpha \in \ON\ 
				\exists f\ \left( f:\alpha \bij a \right).
			\end{align}
		\end{thm}
	\end{screen}
	
	\begin{screen}
		\begin{dfn}[有限・可算・無限]
			
		\end{dfn}
	\end{screen}
	
	\begin{screen}
		\begin{thm}[任意の無限集合は可算集合を含む]
			\begin{align}
				\forall a\ \left(\ \exists \alpha \in \ON \backslash {\bf \omega}\ (\  \alpha \simeq a\ )
				\Longrightarrow \exists b\ (\ b \subset a \wedge {\bf \omega} \simeq b\ )\ \right).
			\end{align}
		\end{thm}
	\end{screen}
	\subsection{Dynkin族定理}
	\begin{screen}
		\begin{dfn}[乗法族・Dynkin族]\label{def:Dynkin_system_theorem}
			集合$X$の部分集合の族$\mathscr{A}$が
			任意の$A,B \in \mathscr{A}$に対し$A \cap B \in \mathscr{A}$を満たすとき
			$\mathscr{A}$を$X$上の乗法族($\pi$-system)という.
			$X$の部分集合の族$\mathscr{D}$が
			\begin{description}
				\item[(D1)] $X \in \mathscr{D}$,
				\item[(D2)] $A,B \in \mathscr{D},
					\ A \subset B \quad \Longrightarrow \quad B \backslash A \in \mathscr{D}$,
				\item[(D3)] $\{A_n\}_{n=1}^\infty \subset \mathscr{D},
					\ A_n \cap A_m = \emptyset\ (n \neq m)
					\quad \Longrightarrow \quad \bigcup_{n=1}^\infty A_n \in \mathscr{D}$,
			\end{description}
			を満たすとき,$\mathscr{D}$を$X$上のDynkin族(Dynkin system)という.
		\end{dfn}
	\end{screen}
	
	\begin{screen}
		\begin{dfn}[Dynkin族定理]\label{thm:Dynkin_system_theorem}
			集合$X$上の乗法族$\mathscr{A}$に対し,
			$\mathscr{A}$を含む最小のDynkin族を$\delta(\mathscr{A})$と書くとき,
			\begin{align}
				\delta(\mathscr{A}) = \sigma(\mathscr{A}).
			\end{align}
		\end{dfn}
	\end{screen}
	
	\begin{prf}\mbox{}
		\begin{description}
			\item[第一段]
				$\delta(\mathscr{C})$が交演算で閉じていれば
				$\delta(\mathscr{C})$は$\sigma$-加法族となる.実際任意の$A \in \delta(\mathscr{A})$に対し
				\begin{align}
					A^c = X \backslash A \in \delta(\mathscr{A})
				\end{align}
				となるから,$\delta(\mathscr{C})$が交演算で閉じていれば任意の
				$A_n \in \delta(\mathscr{C})\ (n=1,2,\cdots)$に対し
				\begin{align}
					\bigcup_{n=1}^{\infty} A_n
					= \bigcup_{n=1}^{\infty} A_1^c \cap A_2^c \cap \cdots \cap A_{n-1}^c \cap A_n
					\in \delta(\mathscr{C})
				\end{align}
				が従う.$\sigma$-加法族はDynkin族であるから
				$\sigma(\mathscr{C}) \subset \delta(\mathscr{C})$も成り立ち
				$\sigma(\mathscr{C}) = \delta(\mathscr{C})$が得られる.
			
			\item[第二段]
				$\delta(\mathscr{C})$が交演算について閉じていることを示す.いま,
				\begin{align}
					\mathscr{D}_1 \coloneqq
					\Set{B \in \delta(\mathscr{C})}{ A \cap B \in \delta(\mathscr{C}),\ 
					\forall A \in \mathscr{C}}
				\end{align}
				により定める$\mathscr{D}_1$はDynkin族であり$\mathscr{C}$を含むから
				\begin{align}
					\delta(\mathscr{C}) \subset \mathscr{D}_1
				\end{align}
				が成立する.従って
				\begin{align}
					\mathscr{D}_2 \coloneqq
					\Set{B \in \delta(\mathscr{C})}{ A \cap B \in \delta(\mathscr{C}),\ 
					\forall A \in \delta(\mathscr{C})}
				\end{align}
				によりDynkin族$\mathscr{D}_2$を定めれば,$\mathscr{C} \subset \mathscr{D}_2$が満たされ
				\begin{align}
					\delta(\mathscr{C}) \subset \mathscr{D}_2
				\end{align}
				が得られる.よって$\delta(\mathscr{C})$は交演算について閉じている.
				\QED
		\end{description}
	\end{prf}
	
	\begin{screen}
		\begin{thm}
			集合$X$の部分集合族$\mathscr{D}$が
			の定義\ref{def:Dynkin_system_theorem}の(D1),(D2)を満たしているとき,
			$\mathscr{D}$が(D3)を満たすことと
			$\mathscr{D}$が増大列の可算和で閉じることは同値である.
		\end{thm}
	\end{screen}
	
	\begin{prf}
		$\mathscr{D}$が可算直和について閉じているとする.このとき
		単調増大列$A_1 \subset A_2 \subset \cdots$を取り
		\begin{align}
			B_1 \coloneqq A_1,
			\quad B_n \coloneqq A_n \backslash A_{n-1},
			\quad (n \geq 2)
		\end{align}
		とおけば(D2)より$B_n \in \mathscr{D},\ (\forall n \geq 1)$が満たされ,
		$n \neq m$なら$B_n \cap B_m = \emptyset$となるから
		\begin{align}
			\bigcup_{n=1}^{\infty} A_n = \bigcup_{n=1}^{\infty} B_n \in \mathscr{D} 
		\end{align}
		が成立する.逆に$\mathscr{D}$が増大列の可算和で閉じているとする.
		(D1)(D2)より互いに素な$A,B \in \mathscr{D}$に対し
		$A^c \in \mathscr{D}$及び$A^c \cap B^c = A^c \backslash B\in \mathscr{D}$が成り立つから,
		$\mathscr{D}$の互いに素な集合列$(B_n)_{n=1}^{\infty}$を取れば
		\begin{align}
			B_1^c \cap B_2^c \cap \cdots \cap B_n^c
			= \left( \cdots \left( \left( B_1^c \cap B_2^c \right) \cap B_3^c \right) \cap \cdots \cap B_{n-1}^c \right) \cap B_n^c
			\in \mathscr{D},
			\quad (n=1,2,\cdots)
		\end{align}
		が得られる.よって
		\begin{align}
			D_n \coloneqq \bigcup_{i=1}^n B_i = X \backslash \Biggl( \bigcap_{i=1}^n B_i^c \Biggr),
			\quad (n=1,2,\cdots)
		\end{align}
		により$\mathscr{D}$の単調増大列$(D_n)_{n=1}^{\infty}$を定めれば
		\begin{align}
			\bigcup_{n=1}^{\infty} B_n = \bigcup_{n=1}^{\infty} D_n \in \mathscr{D}
		\end{align}
		が成立する.
		\QED
	\end{prf}

\subsection{上限下限}
	\begin{screen}
		\begin{thm}[上限の冪と冪の上限]\label{thm:exponentiation_of_supremum_supremum_of_exponentiation}
			任意の空でない$S \subset [0,\infty)$と$t > 0$に対し次が成立する:
			\begin{align}
				(\sup{}{S})^t = \sup{}{\Set{s^t}{s \in S}}.
			\end{align}
		\end{thm}
	\end{screen}
	
	\begin{prf}
		$S=\{0\}$なら両辺0で一致するので,$S$は$\{0\}$より真に大きいとする.このとき
		任意の$s \in S$に対し$s^t \leq (\sup{}{S})^t$となるから$\sup{}{\Set{s^t}{s \in S}} \leq (\sup{}{S})^t$が従う.
		また任意の$(\sup{}{S})^t > \alpha > 0$に対し$s > \alpha^{1/t}$を満たす$s \in S$が存在し
		$(\sup{}{S})^t \geq s^t > \alpha$となるから$\sup{}{\Set{s^t}{s \in S}} = (\sup{}{S})^t$が得られる.
		\QED
	\end{prf}

	\subsection{写像}
	\begin{screen}
		\begin{dfn}[写像]
		\end{dfn}
	\end{screen}
	
	\begin{screen}
		\begin{dfn}[族・系]\label{dfn:family_collection}
			$x$を集合$A$から集合$B$への写像とするとき,
			$x$を$B$の元の集まりと見做したものを
			``$A$を添数集合\index{てんすうしゅうごう@添数集合}(index set)とする
			$B$の族\index{ぞく@族}(family) (或は系\index{けい@系}(collection))''と呼び,
			$x(a)$の代わりに$x_a$として$(x_a)$や$(x_a)_{a \in A}$,又は
			$A$の元が具体的に書き並べられるときは
			$(x_{a_1},x_{a_2},\cdots)$などとも表記する.
			$B$の元の指す対象によっては
			族を点族\index{てんぞく@点族},
			集合族(系)\index{しゅうごうぞく(けい)@集合族(系)},
			或は関数族(系)\index{かんすうぞく(けい)@関数族(系)}などと呼ぶ.
		\end{dfn}
	\end{screen}
	族$(x_a)_{a \in A}$は写像$x$そのものと同一であるが,
	丸括弧を中括弧に替えた$\{x_a\}_{a \in A}$は$B$の部分集合
	$\Set{x_a}{a \in A}$の別の記法であり,$(x_a)_{a \in A}$とは区別する.
	実際,族と集合の大きな違いは,$(x_a)_{a \in A}$の表記では重複する元も
	別個の存在と認めるのに対し,$\{x_a\}_{a \in A}$の表記では重複する元は区別しないことである.
	例えば$A = \N,\ B = \R$に対して
	\begin{align}
		x_n \coloneqq
		\begin{cases}
			1 & (n:\mbox{奇数}) \\
			-1 & (n:\mbox{偶数})
		\end{cases}
	\end{align}
	と定めるとき,$(x_n) = (1,-1,1,-1,\cdots)$と書ける一方で$\{x_n\} = \{-1,1\}$となる.
	
	\begin{screen}
		\begin{thm}[全射・単射・像・原像]\label{projective_injective_image_preimage}
			$f$を集合$A$から集合$B$への写像とするとき,
			\begin{description}
				\item[(1)] 任意の$U \subset A$に対し$f^{-1}\left(f(U)\right) \supset U$が成立し,
					特に$f$が単射なら$f^{-1}\left(f(U)\right) = U$となる.
				\item[(2)] 任意の$V \subset B$に対し$f\left(f^{-1}(V)\right) \subset V$が成立し,
					特に$f$が全射なら$f\left(f^{-1}(V)\right) = V$となる.
			\end{description}
		\end{thm}
	\end{screen}
	
	\begin{prf}\mbox{}
		\begin{description}
			\item[(1)] 任意の$x \in U$で$f(x) \in f(U)$となるから
				$x \in f^{-1}\left(f(U)\right)$が成立する.
				$f$が単射であれば,任意の$x \in f^{-1}\left(f(U)\right)$に対し
				$f(x) \in f(U)$となるから或る$x_1 \in U$で$f(x) = f(x_1)$となり,
				単射性より$x = x_1 \in U$が成り立つ.
				
			\item[(2)] 任意に$y \in f\left(f^{-1}(V)\right)$を取れば,
				或る$x \in f^{-1}(V)$で$y = f(x) \in V$となる.$f$が全射であるとき,
				任意の$y \in V$に対し或る$x \in A$が$y = f(x)$を満たすから,
				$x \in f^{-1}(V)$となり$y \in f\left(f^{-1}(V)\right)$が従う.
				\QED
		\end{description}
	\end{prf}
	
\section{位相メモ}
	\subsection{位相}
	\begin{screen}
		\begin{dfn}[位相]
			$S$を集合とし,$\mathscr{O}$を$\power{S}$の部分集合とする.
			$\mathscr{O}$が以下の三カ条を満たすとき,$\mathscr{O}$を$S$上の{\bf 位相}
			\index{いそう@位相}{\bf (topology)}と呼ぶ:
			\begin{description}
				\item[(O1)] $\mathscr{O}$は$S$と空集合を要素に持つ:
					\begin{align}
						\emptyset \in \mathscr{O} \wedge S \in \mathscr{O}.
					\end{align}
				\item[(O2)] $\mathscr{O}$は交叉で閉じる:
					\begin{align}
						\forall u,v\, \left(\, u,v \in \mathscr{O} \Longrightarrow u \cap v \in \mathscr{O}\, \right).
					\end{align}
				\item[(O3)] $\mathscr{O}$は部分集合の合併で閉じる:
					\begin{align}
						\forall U\, \left(\, U \subset \mathscr{O} \Longrightarrow \bigcup U \in \mathscr{O} \right).
					\end{align}
			\end{description}
			そして対
			\begin{align}
				(S,\mathscr{O})
			\end{align}
			を{\bf 位相空間}\index{いそうくうかん@位相空間}{\bf (topological space)}と呼ぶ.
		\end{dfn}
	\end{screen}
	
	\begin{screen}
		\begin{dfn}[開集合・閉集合]
			$(S,\mathscr{O})$を位相空間とするとき,$\mathscr{O}$の要素を$S$の{\bf 開集合}
			\index{かいしゅうごう@開集合}{\bf (open set)}と呼び,
			補集合が開である$S$の部分集合,つまり
			\begin{align}
				A \subset S \wedge S \backslash A \in \mathscr{O}
			\end{align}
			なる$A$を$S$の{\bf 閉集合}\index{へいしゅうごう@閉集合}{\bf (closed set)}と呼ぶ.
		\end{dfn}
	\end{screen}
	
	$\C$の部分集合族を
	\begin{align}
		\mathscr{O}_\C \defeq \Set{O}{O \subset \C \wedge \forall x \in O\, \exists r \in \R_+\, 
		\left(\, \forall y \in \C\, (\, |y-x| < r \Longrightarrow y \in O\, )\, \right)}
	\end{align}
	で定めると,これは$\C$上の位相となる.以降は$\mathscr{O}_\C$を$\C$の通常の位相として考える.
	つまり,$\C$の部分集合$O$は,$O$の要素$x$が与えられたときに
	\begin{align}
		\Set{y \in \C}{|x-y| < r} \subset O
	\end{align}
	なる正の実数$r$が取れるなら$\C$の通常の開集合と見做される.
	
	\begin{screen}
		\begin{dfn}[内部・閉包]
			位相空間の部分集合$A$に対し,
			$A$に含まれる最大の開集合を$A$の{\bf 内部}\index{ないぶ@内部}{\bf (interior)}と呼び
			$A^{\mathrm{o}}$や$A^i$で表す.また
			$A$を含む最大の閉集合を$A$の{\bf 閉包}\index{へいほう@閉包}{\bf (closure)}と呼び
			$\overline{A}$や$A^a$で表す.特に,
			\begin{align}
				\mbox{$A$が開}\ \Longleftrightarrow\ 
				A = A^\mathrm{o},
				\quad \mbox{$A$が閉}\ \Longleftrightarrow\ 
				A = \overline{A}.
				\label{eq:dfn_interior_closure}
			\end{align}
		\end{dfn}
	\end{screen}
	
	\begin{screen}
		\begin{thm}[内部の補集合は補集合の閉包]
		\label{thm:topology_note_closure_interior}
			$A$を位相空間の部分集合とするとき次が成り立つ.
			\begin{align}
				A^{ic} = A^{ca},
				\quad A^{cic} = A^a,
				\quad A^{ci} = A^{ac}.
			\end{align}
		\end{thm}
	\end{screen}
	
	\begin{prf}
		$A^i \subset A$より$A^{ic} \supset A^c$が従い,
		$A^{ic}$が閉であるから$A^{ic} \supset A^{ca}$となる.
		一方で$A^c \subset A^{ca}$より$A \supset A^{cac}$が従い,
		$A^{cac}$は開であるから$A^i \supset A^{cac}$すなわち
		$A^{ic} \subset A^{ca}$となる.
		$A$を$A^c$に替えれば残りの関係も得られる.
		\QED
	\end{prf}
	
	\begin{screen}
		\begin{dfn}[近傍・基本近傍系]
			空でない位相空間$S$において,$x \in S$と$U \subset S$に対し
			\begin{align}
				x \in U^{\mathrm{o}}
			\end{align}
			が満たされるとき$U$は$x$の{\bf 近傍}\index{きんぼう@近傍}
			{\bf (neighborhood)}であるという.
			同様に$A \subset S$と$V \subset S$に対し
			\begin{align}
				A \subset V^{\mathrm{o}}
			\end{align}
			が満たされるとき,$V$は$A$の近傍であるという.
			点$x$の近傍全体({\bf 近傍系}\index{きんぼうけい@近傍系}
			{\bf (neighborhood system)}と呼ぶ)を$\mathscr{V}(x)$と書くとき,
			$S$は$x$の最大の近傍であるから$\mathscr{V}(x)$は空ではない.
			また$\mathscr{V}(x)$の空でない部分集合$\mathscr{U}(x)$が
			\begin{align}
				\forall V \in \mathscr{V}(x),
				\quad \exists U \in \mathscr{U}(x),
				\quad U \subset V
			\end{align}
			を満たすとき,$\mathscr{U}(x)$を$x$の{\bf 基本近傍系}
			\index{きほんきんぼうけい@基本近傍系}{\bf (local base)}と呼ぶ.
		\end{dfn}
	\end{screen}
	
	\begin{screen}
		\begin{thm}[基本近傍系は開集合を決定する]\label{thm:local_base_defines_open_sets}
			$S$を空でない位相空間,
			$\mathscr{U}(x)$を点$x$の基本近傍系とすれば
			\begin{align}
				\mbox{$O$が$S$の開集合} \quad \Longleftrightarrow \quad 
				\mbox{$O = \emptyset$,或は任意の$x \in O$に対し
				$U \subset O$を満たす$U \in \mathscr{U}(x)$が存在する}
			\end{align}
			が成立する.すなわち,$\{\mathscr{U}(x)\}_{x \in S}$を基本近傍系とする$S$の位相は唯一つである.
		\end{thm}
	\end{screen}
	
	\begin{prf}
		空でない部分集合$O$が開集合なら任意の$x \in O$に対し$O$は$x$の近傍となるから,
		或る$U \in \mathscr{U}(x)$が存在して$U \subset O$を満たす.
		逆に任意の$x \in O$に対し$U \subset O$を満たす$U \in \mathscr{U}(x)$が存在するとき,
		\begin{align}
			x \in U^{\mathrm{o}} \subset O^{\mathrm{o}}
		\end{align}
		となり$O = O^{\mathrm{o}}$が成立するから$O$は開集合である.
		\QED
	\end{prf}
	
	\begin{screen}
		\begin{thm}[基本近傍系は位相を復元する]
		\label{thm:a_local_base_restores_the_topology}\mbox{}
			\begin{description}
				\item[(1)] 
					$(S,\mathscr{O})$を空でない位相空間とし,各点
					$x \in S$に対し$\mathscr{U}(x)$を基本近傍系とすれば以下が成り立つ:
					\begin{description}
						\item[(LB1)] $\mathscr{U}(x)$は空ではなく,また任意の$U \in \mathscr{U}(x)$は$x \in U$を満たす.
						\item[(LB2)] 任意の$U,V \in \mathscr{U}(x)$に対し或る$W \in \mathscr{U}(x)$
							が存在して$W \subset U \cap V$を満たす.
						\item[(LB3)] 任意の$U \in \mathscr{U}(x)$に対し或る$V \in \mathscr{U}(x)$が存在し,
							$V \subset U$かつ任意の$y \in V$に対し$W_y \subset U$を満たす$W_y \in \mathscr{U}(y)$が取れる.
					\end{description}
				\item[(2)]
					空でない集合$S$の各点$x$に対し(LB1)(LB2)(LB3)を満たす部分集合族$\mathscr{U}(x)$が与えられれば,
					\begin{align}
						\mathscr{O} \coloneqq
						\Set{O \subset S}{\mbox{$O = \emptyset$,或は任意の$x \in O$に対し
						$U \subset O$を満たす$U \in \mathscr{U}(x)$が存在する}}
					\end{align}
					により$S$に位相が定まり,$\{\mathscr{U}(x)\}_{x \in S}$は
					$(S,\mathscr{O})$において基本近傍系となる.
				\item[(3)] 空でない位相空間$(S,\mathscr{O})$から基本近傍系
					$\{\mathscr{U}(x)\}_{x \in S}$を得れば,
					$\{\mathscr{U}(x)\}_{x \in S}$を基本近傍系とする位相
					を(2)の手続きで構成することにより$\mathscr{O}$を復元できる.
			\end{description}
		\end{thm}
	\end{screen}
	
	\begin{prf}\mbox{}
		\begin{description}
			\item[(1)] 任意の$U \in \mathscr{U}(x)$は$x$の近傍であるから
				$(LB1)$が満たされる.また$U,V \in \mathscr{U}(x)$に対し
				\begin{align}
					x \in U^{\mathrm{o}} \cap V^{\mathrm{o}} = (U \cap V)^{\mathrm{o}}
				\end{align}
				となるから$U \cap V$は$x$の近傍であり(LB2)も従う.
				任意に$U \in \mathscr{U}(x)$を取れば,
				$U^{\mathrm{o}}$は$x$の開近傍であるから
				或る$V \in \mathscr{U}(x)$で$V \subset U^{\mathrm{o}}$
				を満たすものが存在する.このとき任意の$y \in V$に対し
				$U^{\mathrm{o}}$は$y$の開近傍となるから
				\begin{align}
					W_y \subset U^{\mathrm{o}} \subset U
				\end{align}
				を満たす$W_y \in \mathscr{U}(y)$が取れる.従って(LB3)も得られる.
			
			\item[(2)] 
				$\mathscr{U}(x)$は空ではないから$S \in \mathscr{O}$となる.
				また$O_1,O_2 \in \mathscr{O}$を取れば,
				任意の$x \in O_1 \cap O_2$に対し
				\begin{align}
					x \in U_1 \subset O_1,
					\quad x \in U_2 \subset O_2
				\end{align}
				を満たす$U_1,U_2 \in \mathscr{U}(x)$が存在し,
				(LB2)より或る$U_3 \in \mathscr{U}(x)$に対して
				\begin{align}
					U_3 \subset U_1 \cap U_2 \subset O_1 \cap O_2
				\end{align}
				が成り立つから$O_1 \cap O_2 \in \mathscr{O}$となる.
				任意に$\mathscr{G} \subset \mathscr{O}$を取れば
				任意の$x \in \bigcup \mathscr{G}$は或る$G \in \mathscr{G}$の点であるから,
				\begin{align}
					U \subset G \subset \bigcup \mathscr{G}
				\end{align}
				を満たす$U \in \mathscr{U}(x)$が存在し$\bigcup \mathscr{G} \in \mathscr{O}$が従う.
				よって$\mathscr{O}$は位相である.
				ところで,任意の$U \in \mathscr{U}(x)$に対し
				\begin{align}
					U^{\mathrm{o}} = 
					\Set{y \in U}{\mbox{或る$W_y \in \mathscr{U}(y)$が存在して
					$W_y \subset U$となる}} \eqqcolon \tilde{U}
					\label{eq:thm_a_local_base_restores_the_topology_0}
				\end{align}
				が成立する.実際$\mathscr{O}$の定義より
				\begin{align}
					y \in U^{\mathrm{o}} \quad \Longrightarrow \quad
					\mbox{或る$W_y \in \mathscr{U}(y)$で
					$W_y \subset U^{\mathrm{o}}$}
				\end{align}
				となるから$U^{\mathrm{o}}\subset\tilde{U}$が従い,
				逆に$y \in \tilde{U}$については,
				(\refeq{eq:thm_a_local_base_restores_the_topology_0})の$W_y$に対して
				(LB3)より或る$X_y \in \mathscr{U}(y)$が
				\begin{align}
					X_y \subset W_y,\quad 
					z \in X_y \ \Longrightarrow \
					\mbox{或る$Y_z \in \mathscr{U}(z)$で$Y_z \subset X_y \subset U$}
				\end{align}
				を満たすから$X_y \subset \tilde{U}$が従う.
				すなわち$\tilde{U}$は開集合であり,$U^{\mathrm{o}}\subset\tilde{U}$
				と併せて(\refeq{eq:thm_a_local_base_restores_the_topology_0})
				を得る.(LB3)より
				\begin{align}
					V \subset U, \quad y \in V \ \Longrightarrow \
					\mbox{或る$W_y \in \mathscr{U}(y)$で$W_y \subset U$}
				\end{align}
				を満たす$V \in \mathscr{U}(x)$が存在し,(LB1)と併せて
				\begin{align}
					x \in V \subset \tilde{U} = U^{\mathrm{o}}
				\end{align}
				が成り立つから任意の$U \in \mathscr{U}(x)$は$x$の近傍である.
				そして$W$を$x$の任意の近傍とすれば,
				$\mathscr{O}$の定め方より或る$U \in \mathscr{U}(x)$が
				$U \subset W^{\mathrm{o}}$を満たすから
				$\mathscr{U}(x)$は$x$の基本近傍系である.
			
			\item[(3)] 
				定理\ref{thm:local_base_defines_open_sets}より
				$\{\mathscr{U}(x)\}_{x \in S}$を基本近傍系とする位相は唯一つであるから
				主張が従う.
				\QED
		\end{description}
	\end{prf}
	
	\begin{screen}
		\begin{dfn}[集積点・密集点]
			位相空間$S$の点$x$と部分集合$A$について,
			$x$の任意の近傍$U$に対し
			\begin{align}
				(U \backslash \{x\}) \cap A \neq \emptyset
			\end{align}
			となるとき,$x$は$A$の{\bf 集積点}\index{しゅうせきてん@集積点}
			{\bf (accumulation point)}であるという.
			同様に$x$の任意の近傍$U$に対し
			\begin{align}
				U \cap A \neq \emptyset
			\end{align}
			となるとき,$x$は$A$の{\bf 密集点}\index{みっしゅうてん@密集点}
			{\bf (cluster point)}であるという.
		\end{dfn}
	\end{screen}
	
	集積点と密集点の明確な違いは$T_1$空間(後述)において現れる.
	\begin{screen}
		\begin{thm}[閉である一点集合は集積点を持たない]
		\label{thm:closed_singleton_has_no_accumulation_point}
			位相空間において,閉じている一点集合は集積点を持たない.特に
			$\{x\}$が閉であるとき,$x$は$\{x\}$の密集点ではあるが集積点ではない.
		\end{thm}
	\end{screen}
	
	\begin{prf}
		一点集合$\{x\}$が閉であるとする.このとき$y \neq x$なら
		$U \coloneqq \{x\}^c$は$y$の開近傍となり
		\begin{align}
			(U \backslash \{y\}) \cap \{x\} = \emptyset
		\end{align}
		を満たすから$y$は$\{x\}$の集積点ではない.
		$x$は$\{x\}$の集積点となりえないから$\{x\}$は集積点を持たない.
		\QED
	\end{prf}
	
	\begin{screen}
		\begin{thm}[閉集合は密集点集合]
		\label{thm:belongs_to_closure_iff_clusters}
			位相空間$S$の点$x$と部分集合$A$について次が成り立つ:
			\begin{align}
				x \in \overline{A} \quad \Longleftrightarrow \quad
				\mbox{$x$は$A$の密集点である}.
				\label{eq:thm_belongs_to_closure_iff_clusters}
			\end{align}
			特に,$A$が閉であることと$A$の密集点全体が$A$に一致することは同値になる.
		\end{thm}
	\end{screen}
	
	\begin{prf}
		$x$の或る近傍$U$が$U \cap A = \emptyset$を満たすとき,
		定理\ref{thm:topology_note_closure_interior}より
		\begin{align}
			x \in U^i \subset A^{ci} = A^{ac}
		\end{align}
		となり$x \notin \overline{A}$が従う.逆に
		$x \notin \overline{A}$なら
		$\overline{A}^c$は$A$と交わらない$x$の開近傍となるから
		(\refeq{eq:thm_belongs_to_closure_iff_clusters})が出る.
		また(\refeq{eq:dfn_interior_closure})より
		\begin{align}
			\mbox{$A$が閉} \quad \Longleftrightarrow \quad A = \overline{A}
			\quad \Longleftrightarrow \quad
			\mbox{$A$の密集点全体が$A$に一致}
		\end{align}
		が成立する.
		\QED
	\end{prf}
	
	\begin{screen}
		\begin{thm}[$x \in \overline{A \backslash \{x\}}$$\Longleftrightarrow$$x$が$A$の集積点]
			位相空間$S$の点$x$と部分集合$A$について次が成り立つ:
			\begin{align}
				x \in \overline{A \backslash \{x\}} \quad \Longleftrightarrow \quad
				\mbox{$x$は$A$の集積点である}.
			\end{align}
		\end{thm}
	\end{screen}
	
	\begin{prf}
		$x$の任意の近傍$U$に対し
		$U \cap (A \backslash \{x\}) = (U \backslash \{x\}) \cap A$となるから,
		定理\ref{thm:belongs_to_closure_iff_clusters}と併せて
		\begin{align}
			x \in \overline{A \backslash \{x\}} 
			&\quad \Longleftrightarrow \quad
			\mbox{$x$の任意の近傍$U$に対し$U \cap (A \backslash \{x\}) \neq \emptyset$} \\
			&\quad \Longleftrightarrow \quad
			\mbox{$x$の任意の近傍$U$に対し$(U \backslash \{x\}) \cap A \neq \emptyset$}
			\quad \Longleftrightarrow \quad
			\mbox{$x$は$A$の集積点}
		\end{align}
		が成立する.
		\QED
	\end{prf}
	
	\begin{screen}
		\begin{dfn}[相対位相]
			$(S,\mathscr{O})$を位相空間,$M \subset S$を部分集合,
			$i:M \longrightarrow S$を恒等写像とするとき,
			\begin{align}
				\mathscr{O}_M \coloneqq 
				\Set{i^{-1}(O) = O \cap M}{O \in \mathscr{O}}
			\end{align}
			で定める$\mathscr{O}_M$を$M$の{\bf 相対位相}
			\index{そうたいいそう@相対位相}{\bf (relative topology)}と呼ぶ.
			また相対位相が定まった部分集合をもとの空間に対し{\bf 部分位相空間}
			\index{ぶぶんいそうくうかん@部分位相空間}{\bf (topological subspace)}と呼び,
			紛れが無ければ単に{\bf 部分空間}\index{ぶぶんくうかん@部分空間}とも呼ぶ.
		\end{dfn}
	\end{screen}
	
	\begin{screen}
		\begin{dfn}[$\R$上の位相]
			$\R$上の位相は$\C$上の位相の相対位相として定める:
			\begin{align}
				\mathscr{O}_\R \defeq \Set{O \cap \R}{O \in \mathscr{O}_\C}.
			\end{align}
		\end{dfn}
	\end{screen}
	
	\begin{screen}
		\begin{thm}[$\R$の開集合はボールから成る]
			$O$を$\R$の部分集合とするとき,
			\begin{align}
				O \in \mathscr{O}_\R \Longleftrightarrow
				\forall x \in O\, \exists r \in \R_+\, \left(\, \Set{y \in \R}{|x-y| < r} \subset O\, \right).
			\end{align}
		\end{thm}
	\end{screen}
	
	\begin{screen}
		\begin{dfn}[被覆・コンパクト・相対コンパクト・局所コンパクト・$\sigma$-コンパクト]\mbox{}
			\begin{itemize}
				\item
					集合$S$の部分集合族$\mathscr{B}$が
					$S$の{\bf 被覆}\index{ひふく@被覆}{\bf (cover)}であるとは,
					\begin{align}
						S = \bigcup \mathscr{B}
					\end{align}
					を満たすことをいう.また可算(有限)個の部分集合から成る被覆を
					{\bf 可算(有限)被覆}\index{かさんひふく@可算被覆}
					\index{ゆうげんひふく@有限被覆}と呼ぶ.
					特に,位相空間において開集合のみから成る被覆を
					{\bf 開被覆}\index{かいひふく@開被覆}{\bf (open cover)}と呼ぶ.
				
				\item 集合$S$の被覆$\mathscr{B}$に対し,その部分集合で
					$S$の被覆となるものを$\mathscr{B}$の{\bf 部分被覆}
					\index{ぶぶんひふく@部分被覆}{\bf (subcover)}と呼ぶ.
					部分被覆が有限(可算)集合であるときは有限(可算)部分被覆と呼ぶ.
				\item 
					位相空間において任意の開被覆が有限部分被覆を持つとき,
					その空間は{\bf コンパクト}\index{こんぱくと@コンパクト}である
					{\bf (compact)}という.
					位相空間の部分集合は,その相対位相でコンパクト空間となるとき
					{\bf コンパクト部分集合}と呼ばれる.
				
				\item 位相空間の部分集合で,その閉包がコンパクトであるものを
					{\bf 相対コンパクト}\index{そうたいこんぱくと@相対コンパクト}な
					{\bf (relatively compact)}部分集合という.
				
				\item 位相空間の任意の点がコンパクトな近傍を持つとき,
					その空間は{\bf 局所コンパクト}である
					\index{きょくしょこんぱくと@局所コンパクト}{\bf (locally compact)}という.
					
				\item 位相空間においてコンパクト集合から成る可算被覆が存在するとき,
					その空間は{\bf $\sigma$-コンパクト}
					\index{しぐまこんぱくと@$\sigma$-コンパクト}であるという.
			\end{itemize}
		\end{dfn}
	\end{screen}
	
	集合$S$とその部分集合$A$に対し,$S$の部分集合族$\mathscr{B}$で
	$A \subset \bigcup \mathscr{B}$を満たすものを
	$A$の`{\bf $S$における被覆}'と呼ぶ.$\mathscr{B}$の構成要素が$S$の開集合である場合は
	`{\bf $S$における開被覆}'と呼び,他に`{\bf $S$における部分被覆}'や`{\bf $S$における有限被覆}'といった言い方もする.
	
	\begin{screen}
		\begin{thm}[部分集合のコンパクト性]
		\label{thm:subset_is_compact_iff_every_original_open_cover_contains_finite_subcover}
			$A$を位相空間$S$の部分集合とするとき次が成り立つ:
			\begin{align}
				\mbox{$A$がコンパクト部分集合} \quad \Longleftrightarrow \quad
				\mbox{$A$の$S$における任意の開被覆が($S$における)有限部分被覆を含む}.
			\end{align}
		\end{thm}
	\end{screen}
	
	\begin{prf}
		$A$がコンパクト部分集合であるとき,$\mathscr{B}$を$A$の$S$における開被覆とすれば
		\begin{align}
			\Set{B \cap A}{B \in \mathscr{B}}
		\end{align}
		は部分空間$A$における開被覆となり,有限個の$B_1,B_2,\cdots,B_n \in \mathscr{B}$により
		\begin{align}
			A = \bigcup_{i=1}^n (B_i \cap A) \subset \bigcup_{i=1}^n B_i
		\end{align}
		となるから$\Longrightarrow$が従う.逆に右辺が満たされているとき,
		$\mathscr{A}$を$A$の相対開集合から成る$A$の被覆として
		\begin{align}
			\mathscr{C} \coloneqq \Set{C \subset S}{\mbox{$C$は$S$の開集合で$C \cap A \in \mathscr{A}$}}
		\end{align}
		とおけば,
		\begin{align}
			\mathscr{A} = \Set{C \cap A}{C \in \mathscr{C}}
		\end{align}
		が満たされる.このとき$\mathscr{C}$は$A$を覆うから有限個の
		$C_1,C_2,\cdots,C_m \in \mathscr{C}$で$A \subset \bigcup_{j=1}^m C_j$となり,
		\begin{align}
			A = \bigcup_{j=1}^m (A \cap C_j)
		\end{align}
		かつ$A \cap C_j \in \mathscr{A}$が成り立つから$A$はコンパクトである.
		\QED
	\end{prf}
	
	\begin{screen}
		\begin{thm}[コンパクト集合の閉部分集合はコンパクト]
		\label{thm:closed_subset_of_compact_set_is_compact_on_Hausdorff_space}
			$S$を位相空間,$K,F$をそれぞれ$S$のコンパクト部分集合,閉集合とするとき,
			$K \cap F$は$S$のコンパクト部分集合である.
		\end{thm}
	\end{screen}
	
	\begin{prf}
		$K \cap F$の任意の($S$における)開被覆に$S \backslash F$を加えれば
		$K$の($S$における)開被覆となるから,そのうち$K$の有限部分被覆を取ることができる.
		$S \backslash F$を除けば$K \cap F$の有限被覆が残り
		$K \cap F$のコンパクト性が出る.
		\QED
	\end{prf}
	
	\begin{screen}
		\begin{dfn}[有限交叉性]
			集合$S$の部分集合族$\mathscr{S}$について,その任意の
			有限部分族$\mathscr{T} \subset \mathscr{S}$が
			$\bigcap \mathscr{T} \neq \emptyset$を満たすとき
			$\mathscr{S}$は{\bf 有限交叉性}\index{ゆうげんこうさせい@有限交叉性}
			{\bf (finite intersection property)}を持つという.
		\end{dfn}
	\end{screen}
	
	\begin{screen}
		\begin{thm}[コンパクト$\Longleftrightarrow$閉集合族が有限交叉的]
		\label{thm:compact_iff_closed_sets_family_finitely_intersect}
			$S$を位相空間,$A$を$S$の部分集合とするとき,
			\begin{align}
				&\mbox{$A$がコンパクト部分集合} \quad \Longleftrightarrow \\ 
				&\quad \mbox{任意の$S$の閉集合族$\mathscr{F}$に対し,
				$\Set{F \cap A}{F \in \mathscr{F}}$が有限交叉性を持つなら
				$A \cap \bigcap \mathscr{F} \neq \emptyset.$}
			\end{align}
		\end{thm}
	\end{screen}
	
	\begin{prf}
		定理\ref{thm:subset_is_compact_iff_every_original_open_cover_contains_finite_subcover}より
		\begin{align}
			&\mbox{$A$がコンパクト部分集合} \\
			&\Longleftrightarrow \quad \mbox{$A$の$S$における任意の開被覆が($S$における)有限部分被覆を含む} \\
			&\Longleftrightarrow \quad \mbox{任意の$S$の閉集合族$\mathscr{F}$に対し,
			$A \cap \bigcap \mathscr{F} = \emptyset$なら或る有限族$\mathscr{M} \subset \mathscr{F}$で
			$A \cap \bigcap \mathscr{M} = \emptyset$} \\
			&\Longleftrightarrow \quad \mbox{任意の$S$の閉集合族$\mathscr{F}$に対し,
			$\Set{F \cap A}{F \in \mathscr{F}}$が有限交叉性を持つなら
			$A \cap \bigcap \mathscr{F} \neq \emptyset$}
		\end{align}
		が従う.
		\QED
	\end{prf}
	
	\begin{screen}
		\begin{dfn}[連続・同相・開写像]
			$f$を位相空間$S$から位相空間$T$への写像とする.
			\begin{itemize}
				\item
					$x \in S$において$f(x)$の任意の任意の近傍の
					$f$による引き戻しが$x$の近傍となるとき,
					$f$は{\bf 点$x$で連続}\index{れんぞく@連続}である
					{\bf (continuous at a point $x$)}という.
					
				\item $T$の任意の開集合の$f$による引き戻しが$S$の開集合となるとき,
					$f$を{\bf 連続写像}\index{れんぞくしゃぞう@連続写像}
					{\bf (continuous mapping)}と呼ぶ.
					
				\item $f$に逆写像$f^{-1}$が存在し,$f,f^{-1}$が共に連続であるとき,
					$f$を{\bf 同相写像}\index{どうそうしゃぞう@同相写像}{\bf (homeomorphism)}
					や{\bf 位相同型写像}\index{いそうどうけいしゃぞう@位相同型写像},
					或は単に{\bf 同相}や{\bf 位相同型}と呼ぶ.
					また$S,T$間に同相写像が存在するとき$S$と$T$は
					{\bf 同相}\index{どうそう@同相}である{\bf (homeomorphic)},
					或は{\bf 位相同型}であるという.
					
				\item $S$の任意の開集合の$f$による像が$T$の開集合となるとき,
					$f$を{\bf 開写像}\index{かいしゃぞう@開写像}{\bf (open mapping)}と呼ぶ.
			\end{itemize}
		\end{dfn}
	\end{screen}
	
	\begin{screen}
		\begin{thm}[コンパクト集合の連続写像による像はコンパクト]
		\end{thm}
	\end{screen}
	
	\begin{screen}
		\begin{thm}[各点連続$\Longleftrightarrow$連続]
		\label{thm:continuous_on_every_point_iff_continuous}
			$f$を位相空間$S$から位相空間$T$への写像とするとき次が成り立つ:
			\begin{align}
				\mbox{$f$が連続} \quad \Longleftrightarrow \quad
				\mbox{$f$が$S$の各点で連続}.
			\end{align}
		\end{thm}
	\end{screen}
	
	\begin{prf}
		$f$が連続であるとき,各点$x \in S$で$f(x)$の任意の近傍$U$に対し
		$f(x) \in U^{\mathrm{o}}$が満たされるから
		$f^{-1}(U^{\mathrm{o}})$は$x$を含む開集合となる.
		$f^{-1}(U^{\mathrm{o}})$は$f^{-1}(U)$に含まれる開集合であるから
		\begin{align}
			x \in f^{-1}(U^{\mathrm{o}}) \subset f^{-1}(U)^{\mathrm{o}}
		\end{align}
		が成り立ち,従って$f$は$x$で連続である.
		逆に$f$が各点連続であるとき,
		$T$の任意の開集合$O$に対し
		$f^{-1}(O)$は任意の$x \in f^{-1}(O)$の近傍となるから
		定理\ref{thm:local_base_defines_open_sets}より
		$f^{-1}(O)$は開集合である.よって$f$は連続である.
		\QED
	\end{prf}
	
	\begin{screen}
		\begin{thm}[部分空間と制限写像の連続性]
			$S,T$を位相空間,$f$を$S$から$T$への写像とする.
			また$g:S \longrightarrow f(S)$を
			$f$の終集合を$f(S)$へ制限した写像とする.このとき次が成り立つ:
			\begin{align}
				\mbox{$f:S \longrightarrow T$が連続である} 
				\quad \Longleftrightarrow \quad
				\mbox{$g:S \longrightarrow f(S)$が($f(S)$の相対位相に関して)連続である}.
			\end{align}
		\end{thm}
	\end{screen}
	
	\begin{prf}
		$U \coloneqq f(S)$とおけば$T$の任意の開集合$O$に対し
		\begin{align}
			g^{-1}(U \cap O) = f^{-1}(U \cap O) = f^{-1}(O)
		\end{align}
		が成り立つから,$f$と$g$の連続性は一致する.
		\QED
	\end{prf}
	
	\begin{screen}
		\begin{thm}[位相の生成]
			$S$を集合,$\mathscr{M}$を$S$の部分集合の族として
			\begin{align}
				\mathscr{A} \coloneqq
				\Set{\bigcap \mathscr{F}}{\mbox{$\mathscr{F}$は$\mathscr{M}$の有限部分集合}}
			\end{align}
			とおくとき,$\mathscr{M}$を含む最小の位相は
			\begin{align}
				\mathscr{O} \coloneqq
				\Set{\bigcup \Lambda}{\Lambda \subset \mathscr{A}}
				\cup \{S\}
			\end{align}
			で与えられる.この$\mathscr{O}$を$\mathscr{M}$が生成する$S$の位相と呼ぶ.
		\end{thm}
	\end{screen}
	
	\begin{prf}
		$\mathscr{O}$は定め方より$S$と$\emptyset$を含む.また
		任意の$O_1 = \bigcup \Lambda_1,\ O_2=\bigcup \Lambda_2 \in \mathscr{O},\ 
		(\Lambda_1,\Lambda_2 \subset \mathscr{A})$に対し
		\begin{align}
			\Lambda \coloneqq
			\Set{I \cap J}{I \in \Lambda_1,\ J \in \Lambda_2} \subset \mathscr{A}
		\end{align}
		となるから
		\begin{align}
			O_1 \cap O_2 = \bigcup_{I \in \Lambda_1,\ J \in \Lambda_2} I \cap J
			= \bigcup \Lambda \in \mathscr{O}
		\end{align}
		が成立する.任意に$\emptyset \neq \mathscr{U} \subset \mathscr{O}$を取れば,
		各$U \in \mathscr{U}$に$U = \bigcup \Lambda_U$を満たす
		$\Lambda_U \subset \mathscr{A}$が対応し,このとき
		\begin{align}
			\bigcup_{U \in \mathscr{U}} \Lambda_U \subset \mathscr{A}
		\end{align}
		となるから
		\begin{align}
			\bigcup \mathscr{U} = \bigcup \Biggl(\bigcup_{U \in \mathscr{U}} \Lambda_U\Biggr)
			\in \mathscr{O}
		\end{align}
		が従う.$\mathscr{M}$を含む任意の位相は$\mathscr{A}$を含みかつその任意和で閉じるから$\mathscr{O}$を含む.
		\QED
	\end{prf}
	
	\begin{screen}
		\begin{thm}[Alexanderの定理]
		\end{thm}
	\end{screen}
	
	\begin{screen}
		\begin{dfn}[始位相]
			$f \in \mathscr{F}$を集合$S$から位相空間$(T_f,\mathscr{O}_f)$への写像とするとき,
			全ての$f \in \mathscr{F}$を連続にする最弱の位相を$S$の$\mathscr{F}$-始位相
			(initial topology)と呼ぶ.$\mathscr{F}$-始位相は次が生成する位相である:
			\begin{align}
				\bigcup_{f \in \mathscr{F}} \Set{f^{-1}(O)}{O \in \mathscr{O}_f}.
			\end{align}
		\end{dfn}
	\end{screen}
	\subsection{分離公理}
	\begin{screen}
		\begin{dfn}[位相的に識別可能・分離]
			$S$を位相空間とする.
			\begin{itemize}
				\item $x,y \in S$に対し$x \notin \overline{\{y\}}$
					或は$y \notin \overline{\{x\}}$が満たされるとき,
					$x$と$y$は{\bf 位相的に識別可能}
					\index{いそうてきにしきべつかのう@位相的に識別可能}である
					{\bf (topologically distinguishable)}という.
				\item $A,B \subset S$に対し$\overline{A} \cap B = \emptyset$
					かつ$A \cap \overline{B} = \emptyset$が満たされるとき,
					$A$と$B$は{\bf 分離される}
					\index{ぶんりされる@(集合が)分離される}
					{\bf (separeted)}という.点と点,点と集合の分離は一点集合を考える.
				\item $A,B \subset S$が{\bf 近傍で分離される}
					\index{きんぼうでぶんりされる@近傍で分離される}
					{\bf (separated by neighborhoods)}とは,
					$A,B$が互いに交わらない近傍を持つことをいう.
				\item 閉集合$A,B \subset S$が
					{\bf 関数で分離される}
					\index{かんすうでぶんりされる@関数で分離される}
					{\bf (separated by a function)}とは,
					或る連続関数$f:S \longrightarrow [0,1]$によって$f(A) = \{0\},\ f(B) = \{1\}$
					が満たされることをいう.
				\item 閉集合$A,B \subset S$が
					{\bf 関数でちょうど分離される}
					\index{かんすうでちょうどぶんりされる@関数でちょうど分離される}
					{\bf (precisely separated by a function)}とは,
					或る連続関数$f:S \longrightarrow [0,1]$によって
					$A = f^{-1}(\{0\}),\ B = f^{-1}(\{1\})$が満たされることをいう.
			\end{itemize}
		\end{dfn}
	\end{screen}
	
	\begin{screen}
		\begin{thm}[位相的に識別可能な二点は相異なる]
			$S$を位相空間とするとき,任意の$x,y \in S$に対し
			\begin{align}
				\mbox{$x$と$y$が位相的に識別可能} \quad \Longrightarrow \quad
				x \neq y .
			\end{align}
		\end{thm}
	\end{screen}
	
	\begin{prf}
		$x = y$なら$y \in \overline{\{x\}}$かつ$x \in \overline{\{y\}}$となる.
		後述の$T_0$空間とは,この逆が満たされる位相空間である.
		\QED
	\end{prf}
	
	\begin{screen}
		\begin{thm}[分離される集合は他方を含まない近傍を持つ]
		\label{thm:the_equivalent_condition_of_separatedness}
			位相空間$S$において,$A,B \subset S$が分離されることと
			\begin{align}
				A \subset U,\quad B \subset V,\quad 
				A \cap V = \emptyset,
				\quad B \cap U = \emptyset
				\label{eq:thm_the_equivalent_condition_of_separatedness}
			\end{align}
			を満たす開集合$U,V$が存在することは同値である.
		\end{thm}
	\end{screen}
	
	\begin{prf}
		$A,B \subset S$が分離されるとき,$U \coloneqq \overline{B}^c,\ V \coloneqq \overline{A}^c$
		とおけば(\refeq{eq:thm_the_equivalent_condition_of_separatedness})が成立する.
		逆に$A,B$に対し(\refeq{eq:thm_the_equivalent_condition_of_separatedness})を満たす
		開集合$U,V$が存在するとき,$\closure{A} \subset V^c \subset B^c$及び
		$\closure{B} \subset U^c \subset A^c$となるから$A,B$は分離される.
		\QED
	\end{prf}
	
	\begin{screen}
		\begin{thm}[部分空間の互いに素な閉集合はもとの空間で分離される]
		\label{thm:disjoint_relative_closed_sets_are_separated}
			$S$を位相空間,$T$を$S$の部分集合とする.
			このとき$T$上の相対閉集合$A,B$に対し,
			$A \cap B = \emptyset$ならば$A \cap \overline{B} = \emptyset$
			かつ$\overline{A} \cap B = \emptyset$が成り立つ.ただし上線は$S$における閉包を表す.
		\end{thm}
	\end{screen}
	
	\begin{prf}
		$A,B$は一方が空なら分離される.そうでない場合は対偶を示す.
		$A \cap \overline{B} \neq \emptyset$のとき,
		$x \in A \cap \overline{B}$を取り,
		$U$を$x$の$T$における近傍とすれば,
		$x$の$S$における近傍$V$で$U = T \cap V$を満たすものが存在する.
		このとき
		\begin{align}
			U \cap B = (T \cap V) \cap B = T \cap (V \cap B) = V \cap B
		\end{align}
		となるが,一方で$x \in \overline{B}$と定理\ref{thm:belongs_to_closure_iff_clusters}より
		\begin{align}
			V \cap B \neq \emptyset
		\end{align}
		が成り立ち,$B$は$T$で閉じているから$x \in B$が従う.
		対称的に$\overline{A} \cap B \neq \emptyset$の場合も
		$A \cap B \neq \emptyset$が成立する.
		\QED
	\end{prf}
	
	\begin{screen}
		\begin{dfn}[分離公理]\mbox{}
			\begin{itemize}
				\item 任意の二点が位相的に識別可能である位相空間を{\bf $T_0$空間}
					\index{$T_0$くうかん@$T_0$空間},
					或は{\bf Kolmogorov空間}という.
				\item 任意の二点が分離される位相空間を{\bf $T_1$空間}
					\index{$T_1$くうかん@$T_1$空間}という.
				\item 任意の二点が近傍で分離される位相空間を{\bf $T_2$空間}
					\index{$T_2$くうかん@$T_2$空間},
					或は{\bf Hausdorff空間}\index{Hausdorffくうかん@Hausdorff空間}という.
				\item 任意の交わらない点と閉集合が近傍で分離される位相空間を
					{\bf 正則(regular)空間}\index{せいそくくうかん@正則空間}という.
				\item $T_0$かつ正則な位相空間を{\bf $T_3$空間}
					\index{$T_3$くうかん@$T_3$空間},
					或は{\bf 正則Hausdorff空間}
					\index{せいそくHausdorffくうかん@正則Hausdorff空間}という.
				\item 任意の交わらない点と閉集合が関数で分離される位相空間を
					{\bf 完全正則(completely regular)空間}
					\index{かんぜんせいそくくうかん@完全正則空間}という.
				\item $T_0$かつ完全正則な位相空間を{\bf $T_{3{}^1{\mskip -5mu/\mskip -3mu}_2}$空間}
					\index{$T_{3{}^1{\mskip -5mu/\mskip -3mu}_2}$くうかん@$T_{3{}^1{\mskip -5mu/\mskip -3mu}_2}$空間}や
					{\bf 完全正則Hausdorff空間}
					\index{かんぜんせいそくHausdorffくうかん@完全正則Hausdorff空間},
					或は{\bf Tychonoff空間}\index{Tychonoffくうかん@Tychonoff空間}という.
				\item 任意の交わらない二つの閉集合が近傍で分離される位相空間を
					{\bf 正規(normal)空間}\index{せいきくうかん@正規空間}という.
				\item $T_1$かつ正規な位相空間を{\bf $T_4$空間}
					\index{$T_4$くうかん@$T_4$空間},
					或は{\bf 正規Hausdorff空間}
					\index{せいきHausdorffくうかん@正規Hausdorff空間}という.
				\item 任意の部分位相空間が正規である位相空間を
					{\bf 全部分正規(completely normal)空間}
					\index{ぜんぶぶんせいきくうかん@全部分正規空間}という.
				\item $T_1$かつ全部分正規な位相空間を{\bf $T_5$空間}
					\index{$T_5$くうかん@$T_5$空間},
					或は{\bf 全部分正規Hausdorff空間}
					\index{ぜんぶぶんせいきHausdorffくうかん@全部分正規Hausdorff空間}という.
				\item 任意の交わらない二つの閉集合が関数でちょうど分離される位相空間を
					{\bf 完全正規(perfectly normal)空間}
					\index{かんぜんせいきくうかん@完全正規空間}という.
				\item $T_1$かつ完全正規な位相空間を{\bf $T_6$空間}
					\index{$T_6$くうかん@$T_6$空間},
					或は{\bf 完全正規Hausdorff空間}
					\index{かんぜんせいきHausdorffくうかん@完全正規Hausdorff空間}という.
			\end{itemize}
		\end{dfn}
	\end{screen}
	
	\begin{screen}
		\begin{thm}[$T_1$空間とは一点集合が閉である空間]
			位相空間$S$に対し,以下は全て同値になる:
			\begin{description}
				\item[(a)] $S$が$T_1$である.
				\item[(b)] $S$が$T_0$であり,位相的に識別可能な任意の二点が分離される.
				\item[(c)] $S$の任意の一点集合が閉である.
				\item[(d)] $x \in S$が$A \subset S$の集積点であることと$x$の任意の開近傍が$A$と交わることは同値になる.
			\end{description}
		\end{thm}
	\end{screen}
	
	\begin{prf}
		$x$が$A$の集積点であるとき,任意に$x$の近傍$U$を取る.
		いま,$x$の或る開近傍$U_{n-1}$と$x_{n-1} \in U_{n-1},\ (x \neq x_{n-1})$
		が取れたとして,
		\begin{align}
			U_n \coloneqq U_{n-1} \cap (S \backslash \{x_{n-1}\})
		\end{align}
		は$x$の開近傍となり或る$x_n \in (U_{n-1} \backslash \{x\}) \cap A$が取れる.
		$U_0 \coloneqq U^{\mathrm{o}},\ 
		x_0 \in (U^{\mathrm{o}} \backslash \{x\}) \cap A$を出発点とすれば
		$A$は$U$の無限集合$\{x_n\}_{n=1}^\infty$を含む.
	\end{prf}
	
	$T_1$空間でもHausdorffであるとは限らない.実際,$\N$において
	\begin{align}
		\Set{O \subset \N}{\mbox{$O = \emptyset$,又は$\N \backslash O$が有限集合}}
	\end{align}
	で位相を定めるとき,一点集合は常に閉となるが,
	任意の空でない二つの開集合は必ず交叉する(そうでないと有限集合が無限集合を包含することになる)
	のでHausdorff空間とはならない.一方でHausdorff空間は常に$T_1$である.
	
	\begin{screen}
		\begin{thm}[$T_2 \Longrightarrow T_1$]
			Hausdorff空間は$T_1$である.
		\end{thm}
	\end{screen}
	
	\begin{prf}
		$x$をHausdorff空間の点とする.$x$と異なる任意の点$y$に対して
		\begin{align}
			x \in U_y,\quad y \in V_y,\quad U_y \cap V_y = \emptyset
		\end{align}
		を満たす開集合$U_y,V_y$が存在し,このとき
		\begin{align}
			\{x\} = \bigcap_{y\, :\, x \neq y} V_y^c
		\end{align}
		となるから$\{x\}$は閉である.つまりHausdorff空間は$T_1$である.
		\QED
	\end{prf}
	
	\begin{screen}
		\begin{thm}[Hausdorff空間のコンパクト部分集合は閉]
			Hausdorff空間のコンパクト部分集合は閉である.
		\end{thm}
	\end{screen}
	
	\begin{prf}
		$S$をHausdorff空間,$K \subset S$をコンパクト部分集合とするとき,
		任意に$x \in S \backslash K,\ y \in K$を取れば
		\begin{align}
			x \in U_y,\quad y \in V_y, \quad U_y \cap V_y = \emptyset
		\end{align}
		を満たす開集合$U_y,V_y$が取れる.或る$\{y_i\}_{i=1}^n \subset K$に対し
		$K \subset \bigcup_{i=1}^n V_{y_i}$となるから,
		$U \coloneqq \bigcap_{i=1}^n U_{y_i}$とおけば
		\begin{align}
			x \in U,\quad U \subset \bigcap_{i=1}^n \left(S\backslash V_{y_i}\right)
			\subset S \backslash K
		\end{align}
		が成立する.従って$S \backslash K$は開集合であり,$K$は閉集合である.
		\QED
	\end{prf}
	
	\begin{screen}
		\begin{thm}[Hausdorff空間とは交わらない二つのコンパクト集合が近傍で分離される空間]
		\label{thm:Hausdorff_space_two_disjoint_compact_sets_are_separated_by_nbh}
			位相空間において,Hausdorffであることと,
			交わらない二つのコンパクト部分集合が近傍で分離されることは同値である.
		\end{thm}
	\end{screen}
	
	\begin{prf}
		$A,B$をHausdorff空間の交わらないコンパクト集合とするとき,
		任意の$p \in A$に対し
		\begin{align}
			p \in U_p,\quad B \subset V_p,\quad U_p \cap V_p = \emptyset
			\label{eq:thm_Hausdorff_space_two_disjoint_compact_sets_are_separated_by_nbh_1}
		\end{align}
		を満たす開集合$U_p,V_p$が存在する.実際
		任意の$q \in B$に対し
		\begin{align}
			p \in U_p(q),\quad q \in V_p(q),\quad U_p(q) \cap U_p(q) = \emptyset
		\end{align}
		を満たす開集合$U_p(q), U_p(q)$が取れ,$B$のコンパクト性より
		或る$\{q_i\}_{i=1}^n \subset B$で$B \subset \bigcup_{i=1}^n U_p(q_i)$となるから,
		\begin{align}
			U_p \coloneqq \bigcap_{i=1}^n U_p(q_i),
			\quad V_p \coloneqq \bigcup_{i=1}^n V_p(q_i)
		\end{align}
		とおけば(\refeq{eq:thm_Hausdorff_space_two_disjoint_compact_sets_are_separated_by_nbh_1})
		が成立する.$A$のコンパクト性より或る$\{p_j\}_{j=1}^m \subset A$で
		$A \subset \bigcup_{j=1}^m U_{p_j}$となるから,
		\begin{align}
			U \coloneqq \bigcup_{j=1}^m U_{p_j},
			\quad V \coloneqq \bigcap_{j=1}^m V_{p_j}
		\end{align}
		とおけば$A$と$B$は$U,V$により分離される.
		逆の主張は一点集合がコンパクトであることより従う.
		\QED
	\end{prf}
	
	\begin{screen}
		\begin{thm}[Hausdorff空間値連続写像の等価域は閉]
		\label{thm:equivalence_set_of_two_mappings_into_Hausdorff_space_is_closed}
			$S$を位相空間,$T$をHausdorff空間,$f,g$を
			$S$から$T$への連続写像とするとき,$E \coloneqq \Set{x \in S}{f(x) = g(x)}$は$S$で閉じている.
			特に$\overline{E}=X$なら$f=g$となる.
		\end{thm}
	\end{screen}
	
	\begin{prf}
		任意に$x \in \Set{x \in S}{f(x) \neq g(x)}$を取れば,Hausdorff性より
		\begin{align}
			f(x) \in A,\quad g(x) \in B,\quad A \cap B = \emptyset
		\end{align}
		を満たす$T$の開集合$A,B$が存在する.
		$f^{-1}(A) \cap g^{-1}(B)$は$x$の開近傍であり,
		\begin{align}
			f^{-1}(A) \cap g^{-1}(B) \subset \Set{x \in S}{f(x) \neq g(x)}
		\end{align}
		となるから$\Set{x \in S}{f(x) \neq g(x)}$は$S$の開集合である.
		従って$E$は閉である.
		\QED
	\end{prf}
	
	\begin{screen}
		\begin{thm}[$T_3 \Longrightarrow T_2$]
			$T_3$空間はHausdorffである.
		\end{thm}
	\end{screen}
	
	\begin{prf}
		$T_3$空間は$T_0$であるから,相異なる二点$x,y$に対して
		$x \in \overline{\{y\}}$或は$y \in \overline{\{x\}}$が成り立つ.正則性より
		\begin{align}
			f(x) = 0,\quad f(y) = 1
		\end{align}
		を満たす連続写像が存在し,$x,y$は$f^{-1}([0,1/2))$と$f^{-1}((1/2,1])$で分離される.
		\QED
	\end{prf}
	
	\begin{screen}
		\begin{thm}[正則空間とは交わらないコンパクト集合と閉集合が近傍で分離される空間]
		\label{thm:each_point_in_regular_space_has_closesd_local_base}\mbox{}
			\begin{description}
				\item[(1)] 位相空間において,正則性と,交わらないコンパクト集合と閉集合が近傍で分離されることは同値である.
					
				\item[(2)]
					$K,W,\ (K \subset W)$をそれぞれ局所コンパクトな$T_3$空間のコンパクト集合,
					開集合とするとき,相対コンパクトな開集合$U$が存在して次を満たす:
					\begin{align}
						K \subset U \subset \overline{U} \subset W.
						\label{eq:thm_each_point_in_regular_space_has_closesd_local_base}
					\end{align}
			\end{description}
		\end{thm}
	\end{screen}
	
	\begin{prf}\mbox{}
		\begin{description}
			\item[(1)]
				$K,F$を正則空間のコンパクト集合,閉集合とするとき,
				$K \cap F = \emptyset$なら任意の点$x \in K$に対して
				\begin{align}
					x \in U_x,\ \quad F \subset V_x,
					\quad U_x \cap V_x = \emptyset
				\end{align}
				を満たす開集合$U_x,V_x$が取れる.
				$K$はコンパクトであるから或る$\{x_i\}_{i=1}^n \subset K$で
				$K \subset \bigcup_{i=1}^n U_{x_i}$となり
				\begin{align}
					K \subset U \coloneqq \bigcup_{i=1}^n U_{x_i},
					\quad F \subset V \coloneqq \bigcap_{i=1}^n V_{x_i},
					\quad U \cap V = \emptyset
				\end{align}
				が成立する.逆の主張は一点集合がコンパクトであることにより従う.
			\item[(2)]
				任意の$x \in K$に対し,$F_x \subset W$
				を満たす閉近傍$F_x$とコンパクトな近傍$C_x$が存在する.
				或る$\{y_i\}_{i=1}^m \subset K$で
				\begin{align}
					K \subset 
					\left(C_{y_1}^{\mathrm{o}} \cap F_{y_1}^{\mathrm{o}}\right) 
					\cup \cdots \cup 
					\left(C_{y_m}^{\mathrm{o}} \cap F_{y_m}^{\mathrm{o}}\right)
				\end{align}
				となるが,ここで
				$U \coloneqq 
				\bigcup_{i=1}^m C_{y_i}^{\mathrm{o}} \cap F_{y_i}^{\mathrm{o}}$
				とおけば,Hausdorff空間において$C_{y_i}$は閉じているから
				\begin{align}
					\overline{U} \subset \bigcup_{i=1}^m C_{y_i}
				\end{align}
				が成り立つ.
				定理\ref{thm:closed_subset_of_compact_set_is_compact_on_Hausdorff_space}
				より$\overline{U}$のコンパクト性が得られ,かつこのとき
				\begin{align}
					K \subset U \subset \overline{U} \subset \bigcup_{i=1}^m F_{y_i}
					\subset W
				\end{align}
				も満たされる.
				\QED
		\end{description}
	\end{prf}
	
	\begin{screen}
		\begin{thm}[完全正則なら正則]
		\end{thm}
	\end{screen}
	
	\begin{screen}
		\begin{thm}[完全正則空間とは交わらないコンパクト集合と閉集合が関数で分離される空間]
			位相空間において,完全正則であることと,交わらないコンパクト集合と閉集合が
			関数で分離されることは同値である.
		\end{thm}
	\end{screen}
	
	\begin{prf}
		$K,C$をそれぞれ完全正則空間$S$のコンパクト部分集合と閉集合とする.
		任意の$x \in K$に対し
		\begin{align}
			f_x(y) = 
			\begin{cases}
				0, & (y=x), \\
				1, & (y \in C)
			\end{cases} 
		\end{align}
		を満たす連続写像$f_x:S \longrightarrow [0,1]$が存在し,
		$K$のコンパクト性より或る$x_1,x_2,\cdots,x_n \in K$で
		\begin{align}
			K \subset \bigcup_{i=1}^n \Set{x \in K}{f_{x_i}(x) < \frac{1}{2}}
		\end{align}
		が成り立つ.$x \in K$なら$\prod_{i=1}^n f_{x_i}(x) < 1/2$,
		$x \in C$なら$\prod_{i=1}^n f_{x_i}(x) = 1$となるから,
		$f \coloneqq \prod_{i=1}^n f_{x_i}$として
		\begin{align}
			g(x) \coloneqq 2 \operatorname{max}\left\{f(x),\frac{1}{2}\right\} - 1
		\end{align}
		により連続写像$g:S \longrightarrow [0,1]$を定めれば
		\begin{align}
			g(x) = 
			\begin{cases}
				0, & (x \in K), \\
				1, & (x \in C)
			\end{cases}
		\end{align}
		が従う.すなわち$K,C$は$g$で分離される.
		一点はコンパクトであるから逆の主張も得られる.
		\QED
	\end{prf}
	
	\begin{screen}
		\begin{thm}[実数値関数の族が生成する始位相は完全正則]
		\label{thm:initial_topology_of_continuous_functions_is_completely_regular}
			$S$を集合とし,$\mathscr{C}$を$S$から$\R$への実数値関数の集合とする.このとき
			$S$は$\mathscr{C}$-始位相により完全正則空間となる.
		\end{thm}
	\end{screen}
	
	\begin{prf}
		$S$に$\mathscr{C}$-始位相を入れるとき,任意の$x \in S$と$x$を含まない(空でない)始位相の閉集合$F$に対して
		\begin{align}
			x \in \bigcap_{i=1}^n f_i^{-1}(O_i) \subset S \backslash F
		\end{align}
		を満たす$f_i \in \mathscr{C}$と$\R$の開集合$O_i,\ (i=1,\cdots,n)$が取れる.
		$\R$は完全正則であるから,各$i$で
		\begin{align}
			g_i:\R \longrightarrow [0,1]
		\end{align}
		かつ
		\begin{align}
			g_i(f_i(x)) = 1
		\end{align}
		かつ
		\begin{align}
			r \in \R \backslash O_i \Longrightarrow g_i(r) = 0
		\end{align}
		を満たす連続写像$g_i$が存在して,
		\begin{align}
			y \in S \backslash f_i^{-1}(O_i)
			\Longrightarrow g_i(f_i(y)) = 0
		\end{align}
		が成立する.
		\begin{align}
			x \in S \Longrightarrow h(x) = \operatorname{min}\left\{g_1(f_1(x)),\, g_2(f_2(x)),\cdots,g_n(f_n(x))\right\}
		\end{align}
		なる写像$h$を定めれば,$h$は
		\begin{align}
			h:S \longrightarrow [0,1]
		\end{align}
		を満たし,$\mathscr{C}$-始位相に関して連続であり,
		$h(x)=1$かつ$F$上で$0$となる.
		\QED
	\end{prf}
	
	\begin{screen}
		\begin{thm}[完全正則空間の位相は実連続写像全体の始位相に一致する]
			$(S,\mathscr{O})$を位相空間とし,$C(S)$を実連続写像の全体とし,
			\begin{align}
				\mathscr{Z} \coloneqq \Set{\bigcap_{f \in \mathscr{F}} f^{-1}\ast\{0\}}{\mathscr{F} \subset C(S) \wedge \mathscr{F} \neq \emptyset}
			\end{align}
			とおくとき,以下は同値となる:
			\begin{description}
				\item[(a)] $S$が完全正則である.
				\item[(b)] $S$の$C(S)$-始位相が$\mathscr{O}$に一致する.
				\item[(c)] $S$の閉集合全体と$\mathscr{Z}$が一致する.
			\end{description}
		\end{thm}
	\end{screen}
		
	\begin{prf}\mbox{}
		\begin{description}
			\item[$(a) \Longrightarrow (c)$]
				$S$が完全正則であるとき,$C=\emptyset$なら
				\begin{align}
					f:S \longrightarrow \{1\}
				\end{align}
				なる$f$により,$C = S$なら
				\begin{align}
					f:S \longrightarrow \{0\}
				\end{align}
				なる$f$により
				\begin{align}
					C = f^{-1} \ast \{0\}
				\end{align}
				となる.$C$が$\emptyset$でも$S$でもない閉集合であるとき,
				任意の$x \in S \backslash C$に対し或る$f_x \in C(S)$で
				\begin{align}
					f_x(y) = \begin{cases}
						1, & (y=x),\\
						0, & (y \in C)
					\end{cases}
				\end{align}
				を満たすものが存在する.このとき
				\begin{align}
					C \subset \bigcap_{x \in S \backslash C} f_x^{-1} \ast \{0\}
				\end{align}
				となるが,一方で
				\begin{align}
					x \notin C \Longrightarrow x \notin f_x^{-1} \ast \{0\}
				\end{align}
				も成り立つので
				\begin{align}
					C = \bigcap_{x \in S \backslash C} f_x^{-1} \ast \{0\}
				\end{align}
				が成り立つ.従って
				\begin{align}
					C \in \mathscr{Z}
				\end{align}
				となる.一方で$f \in C(S)$に対し$f^{-1} \ast \{0\}$は閉であるから
				$\mathscr{Z}$は$S$の閉集合の族であり$(c)$が満たされる.
				
			\item[$(c) \Longrightarrow (b)$]
				$C(S)$の要素は$\mathscr{O}$に関して連続であり,
				$C(S)$-始位相は$C(S)$のすべての要素を連続にする最弱の位相であるから,
				\begin{align}
					\mbox{$C(S)$-始位相} \subset \mathscr{O}
				\end{align}
				が成り立つ.一方で$(c)$が満たされているとき,
				$O$を$\mathscr{O}$の要素とすると
				\begin{align}
					S \backslash O = \bigcap_{f \in \mathscr{F}} f^{-1}\ast\{0\}
				\end{align}
				を満たす$\subset \Set{f^{-1}(\{0\})}{f \in C(S)}$の部分集合$\mathscr{F}$が存在して,
				\begin{align}
					O = \bigcup_{f \in \mathscr{F}} f^{-1} \ast (\R \backslash \{0\})
				\end{align}
				となる.各$f$で$f^{-1} \ast (\R \backslash \{0\})$は
				$C(S)$-始位相の要素であるから,その合併である$O$も
				$C(S)$-始位相の要素である.ゆえに$(b)$が従う.
			
			\item[$(b) \Longrightarrow (a)$] 
				定理\ref{thm:initial_topology_of_continuous_functions_is_completely_regular}
				より従う.
				\QED
		\end{description}
	\end{prf}
	
	\begin{screen}
		\begin{thm}[正規空間とは交わらない二つの閉集合が関数で分離される空間(Urysohnの補題)]
		\label{thm:Urysohn_lemma}
			位相空間において,正規性と,任意の交わらない二つの閉集合が関数で分離されることは同値である.
		\end{thm}
	\end{screen}
	
	\begin{screen}
		\begin{thm}[正則かつ正規なら完全正則]
		\label{thm:if_regular_and_normal_then_completely_normal}
			正則かつ正規な(空でない)位相空間は完全正則である.
		\end{thm}
	\end{screen}
	
	\begin{prf}
		点$x$と空でない閉集合$F,\ (x \notin F)$に対し,
		正則なら$x$の閉近傍$E$で$E \cap F = \emptyset$を満たすものが取れる.
		加えて正規なら,Urysohnの補題より$E$と$F$は関数で分離されるから$x$と$F$も関数で分離される.
		\QED
	\end{prf}
	
	\begin{screen}
		\begin{thm}[$T_4 \Longrightarrow T_3$]
		\end{thm}
	\end{screen}
	
	\begin{screen}
		\begin{thm}[$T_6 \Longrightarrow T_5 \Longrightarrow T_4$]
			完全正規空間は全部分正規である.
			特に,全部分正規なら正規であるから
			$T_6 \Longrightarrow T_5 \Longrightarrow T_4$となる.
		\end{thm}
	\end{screen}
	
	\begin{prf}
		$S$を$T_6$空間,$T$を$S$の部分位相空間,$A,B$を$T$の空でない閉集合とするとき,
		定理\ref{thm:disjoint_relative_closed_sets_are_separated}より
		\begin{align}
			A \cap \overline{B} = \emptyset,\quad \overline{A} \cap B = \emptyset
		\end{align}
		となる.ただし上線は$S$における閉包を表す.完全正規性より
		\begin{align}
			\overline{A} = f^{-1}(\{0\}),
			\quad \overline{B} = g^{-1}(\{0\}),
			\quad \left( f^{-1}(\{1\}) = \emptyset = g^{-1}(\{1\}) \right)
		\end{align}
		を満たす連続写像$f,g:S \longrightarrow [0,1]$が取れるから,ここで
		$h:S \longrightarrow \R$を$h \coloneqq f - g$で定めれば
		\begin{align}
			\begin{cases}
				h(x) < 0, & (x \in A), \\
				h(x) > 0, & (x \in B)
			\end{cases}
		\end{align}
		が成り立ち,$A \subset T \cap h^{-1}((-\infty,0))$かつ
		$B \subset T \cap h^{-1}((0,\infty))$より$A,B$は$T$における開近傍で分離される.
		\QED
	\end{prf}
	
	\begin{screen}
		\begin{dfn}[$G_\delta$集合・$F_\sigma$集合]
			位相空間の部分集合で,開集合の可算交叉で表されるものを$G_\delta$集合,
			閉集合の可算和で表されるものを$F_\sigma$集合と呼ぶ.
			特に,任意の閉集合が$G_\delta$である空間では任意の開集合が$F_\sigma$となる.
		\end{dfn}
	\end{screen}
	
	\begin{screen}
		\begin{thm}[完全正規空間とは正規かつ閉集合が全て$G_\delta$である空間]
		\label{thm:perfectly_normal_Hausdorff_is_normal_and_closed_is_G_delta}\mbox{}
			\begin{description}
				\item[(1)]
					$F$を完全正規空間の閉集合とすれば,次を満たす閉集合系$(F_n)_{n=1}^\infty$が存在する:
					\begin{align}
						F = \bigcap_{n=1}^\infty F_n,
						\quad F_n^{\mathrm{o}} \supset F_{n+1}. 
					\end{align}
					
				\item[(2)]
					位相空間において,完全正規であることと,正規かつ任意の閉集合が$G_\delta$であることは同値である.
			\end{description}
		\end{thm}
	\end{screen}
	
	\begin{prf}
		$S$を完全正規空間,$A,B$を互いに交わらない$S$の閉集合とすれば,
		$A=f^{-1}(\{0\}),\ B = f^{-1}(\{1\})$を満たす連続関数
		$f:S \longrightarrow \R$が存在する.このとき
		$U \coloneqq f^{-1}([0,1/2)),\ V \coloneqq f^{-1}((1/2,1])$
		で開集合$U,V$を定めれば
		\begin{align}
			A \subset U,\quad B \subset V,\quad U \cap V = \emptyset
		\end{align}
		となるから$S$は正規である.また$F$を閉集合とすれば
		或る連続関数$g:S \longrightarrow \R,\ (\emptyset = g^{-1}(\{1\}))$により
		\begin{align}
			F = g^{-1}(\{0\}) 
			= g^{-1}\Biggl(\bigcap_{n=1}^\infty\left[0,n^{-1}\right)\Biggr)
			= \bigcap_{n=1}^\infty g^{-1}\left(\left[0,n^{-1}\right)\right)
		\end{align}
		が成立するから$F$は$G_\delta$である.特に,このとき
		$F_n \coloneqq g^{-1}\left(\left[0,n^{-1}\right]\right)$とおけば
		\begin{align}
			F = \bigcap_{n=1}^\infty g^{-1}\left(\left[0,n^{-1}\right]\right)
			= \bigcap_{n=1}^\infty F_n,
			\quad F_n^{\mathrm{o}} \supset g^{-1}\left(\left[0,n^{-1}\right)\right)
			\supset g^{-1}\left(\left[0,(n+1)^{-1}\right]\right)
			= F_{n+1}
		\end{align}
		となり(1)の主張が得られる.逆に$S$が正規かつ
		閉集合が全て$G_\delta$であるとき,任意の交わらない閉集合$A,B$に対し
		$A = \bigcap_{n=1}^\infty U_n,\ B = \bigcap_{n=1}^\infty V_n$
		を満たす開集合系$(U_n)_{n=1}^\infty,\ (V_n)_{n=1}^\infty$が取れて,
		定理\ref{thm:Urysohn_lemma}より各$n \geq 1$で
		\begin{align}
			f_n(A) = \{0\},\quad f_n(S \backslash U_n) = \{1\},
			\quad g_n(B) = \{0\},\quad g_n(S \backslash V_n) = \{1\}
		\end{align}
		を満たす連続写像$f_n,g_n:S \longrightarrow [0,1]$が存在する.
		ここで連続写像を$f \coloneqq \sum_{n=1}^\infty 2^{-n} f_n,\ 
		g \coloneqq \sum_{n=1}^\infty 2^{-n} g_n$で定めれば
		\begin{align}
			\begin{cases}
				f(x) = 0, & (x \in A), \\
				f(x) > 0, & (x \notin A),
			\end{cases}
			\quad \begin{cases}
				g(x) = 0, & (x \in B), \\
				g(x) > 0, & (x \notin B),
			\end{cases}
		\end{align}
		となり,$h \coloneqq f/(f+g)$とおけば$A = h^{-1}(\{0\}),\ B = h^{-1}(\{1\})$が成立する.
		従って$S$は完全正規である.
		\QED
	\end{prf}
	
	\begin{screen}
		\begin{thm}[連続な単射の引き戻しによる分離性の遺伝]
			$S,T$を位相空間とする.$S$から$T$への連続単射が存在するとき,
			$T$が$T_k$-空間$(k=0,1,\cdots,6)$なら
			$S$もまた$T_k$-空間となる.
		\end{thm}
	\end{screen}
	
	\begin{prf}
		任意に異なる二点$s_1,s_2 \in S$を取れば単射性より$f(s_1) \neq f(s_2)$となる.
		$T$の分離性より
	\end{prf}
	\section{可算公理}
	\begin{screen}
		\begin{thm}[可算コンパクト性の同値条件]
		\end{thm}
	\end{screen}
	
	\begin{screen}
		\begin{dfn}[開基]
			位相空間$(S,\mathscr{O})$において,
			$\mathscr{O}$の部分集合$\mathscr{B}$で
			\begin{align}
				\mathscr{O}
				= \Set{\bigcup \mathscr{U}}{\mathscr{U} \subset \mathscr{B}}
			\end{align}
			を満たすもの,ただし$\bigcup \emptyset = \emptyset$,
			を$\mathscr{O}$の{\bf 開基}\index{かいき@開基}や
			{\bf 基底}\index{きてい@基底},{\bf 基}\index{き@基}{\bf (base)}と呼ぶ.
			基底は一意に定まるものではない.
		\end{dfn}
	\end{screen}
	
	\begin{screen}
		\begin{dfn}[可算公理]
			位相空間$S$において,任意の点が高々可算な基本近傍系を持つとき
			$S$は{\bf 第一可算公理}
			\index{だいいちかさんこうり@第一可算公理}
			{\bf (the first axiom of countability)}を満たす,或は
			$S$は第一可算であるといい,
			$S$が高々可算な基底を持つとき
			$S$は{\bf 第二可算公理}
			\index{だいにかさんこうり@第二可算公理}
			{\bf (the second axiom of countability)}を満たす,或は
			$S$は第二可算であるという.
		\end{dfn}
	\end{screen}
	空集合(要素数0)を含む任意の有限位相空間は,その冪集合が有限集合であるから
	第二可算公理を満たす.
	
	\begin{screen}
		\begin{thm}[第二可算なら第一可算]
			空でない第二可算空間は第一可算である.
		\end{thm}
	\end{screen}
	
	\begin{prf}
		$\mathscr{B}$を空でない第二可算空間$S$の可算基とするとき,任意の$x \in S$に対して
		\begin{align}
			\mathscr{U}(x) \coloneqq
			\Set{B \in \mathscr{B}}{x \in B}
		\end{align}
		で可算な基本近傍系が定まる.実際
		$x$の任意の近傍$U$に対し或る$B \in \mathscr{B}$で
		\begin{align}
			x \in B \subset U^{\mathrm{o}}
		\end{align}
		が成立し,定義より$B \in \mathscr{U}(x)$が満たされる.
		\QED
	\end{prf}
	
	\begin{screen}
		\begin{dfn}[稠密・可分]
			位相空間$S$において,$\overline{M} = S$を満たすような部分集合$M$を
			$S$で{\bf 稠密}\index{ちゅうみつ@稠密}な{\bf (dense)}部分集合と呼ぶ.
			また高々可算かつ稠密な部分集合$M$が存在するとき$S$は{\bf 可分}
			\index{かぶん@可分}である{\bf (separable)}という.
		\end{dfn}
	\end{screen}
	
	\begin{screen}
		\begin{thm}[第二可算なら可分]\label{thm:second_countable_then_separable}
			第二可算位相空間は可分である.
		\end{thm}
	\end{screen}
	
	\begin{prf}
		$\mathscr{B}$を第二可算空間$S$の可算基とするとき,
		$S = \emptyset$なら$\emptyset$は$S$の唯一の部分集合であり,
		要素数$0$かつ$\overline{\emptyset} = \emptyset = S$を満たすから
		$S$は可分である.$S \neq \emptyset$のとき,
		選択関数$\Phi \in \prod \mathscr{B} = \prod_{B \in \mathscr{B}} B$を取り
		\begin{align}
			M \coloneqq \Set{\Phi(B)}{B \in \mathscr{B}}
		\end{align}
		で可算集合を定めれば,任意の$x \in S$及び$x$の任意の近傍$U$に対し
		$x \in B \subset U^{\mathrm{o}}$を満たす
		$B \in \mathscr{B}$が存在して
		\begin{align}
			\Phi(B) \in B \cap M \subset U \cap M
		\end{align}
		となるから,定理\ref{thm:belongs_to_closure_iff_clusters}より
		$S = \overline{M}$が成立する.
		\QED
	\end{prf}
	
	\begin{screen}
		\begin{dfn}[局所有限]
			$\mathscr{F}$を位相空間$S$の部分集合族とする.
			任意の$x \in S$が$\mathscr{F}$の高々有限個の元としか交叉しない近傍を持つとき,
			$\mathscr{F}$は{\bf 局所有限}\index{きょくしょゆうげん@局所有限}
			{\bf (locally finite)}であるという.
			つまり,$\mathscr{F}$が局所有限であることの論理式で表現すると
			\begin{align}
				\forall x \in S \exists V \in \mathscr{V}(x) \exists \mathscr{G} \in
				\mathcal{P}(\mathscr{F})
				\left(\exists i \in \omega(\mathscr{G} \simeq i) \wedge
				\forall G \in \mathscr{G}(V \cap G \neq \emptyset) \wedge
				\forall F \in \mathscr{F} \backslash \mathscr{G}(V \cap F = \emptyset)\right).
			\end{align}
			また
			$\mathscr{F}$が局所有限な部分集合族の高々可算個の合併で表されるとき,
			$\mathscr{F}$は{\bf $\sigma$-局所有限}
			\index{しぐまきょくしょゆうげん@$\sigma$-局所有限}であるという.
		\end{dfn}
	\end{screen}
	
	後述の一様位相空間(距離空間や位相線型空間に共通する構造が定義された空間)の或るクラスは
	$\sigma$-局所有限な基底を持つ(定理\ref{thm:if_uniformity_has_countable_base_then_has_topology_has_sigma_locally_finite_base}).従って以下のいくつかの定理はそのまま
	距離空間や第一可算位相線型空間に適用される.
	
	\begin{screen}
		\begin{thm}[$\sigma$-局所有限な基底が存在すれば第一可算]
			$\sigma$-局所有限な基底が存在する空でない位相空間は第一可算である.
		\end{thm}
	\end{screen}
	
	\begin{prf}
		$S$を空でない位相空間,$\mathscr{B} = \bigcup_{n=1}^\infty \mathscr{B}_n$を
		$\sigma$-局所有限な基底とする(各$\mathscr{B}_n$は局所有限).
		任意の$x \in S$で
		\begin{align}
			\mathscr{U}_n(x) \coloneqq \Set{B \in \mathscr{B}_n}{x \in B},
			\quad \mathscr{U}(x) \coloneqq \bigcup_{n=1}^\infty \mathscr{U}_n(x)
		\end{align}
		と定めれば,局所有限性より$\mathscr{U}_n(x)$は有限であるから
		$\mathscr{U}(x)$は高々可算である.また$x$の任意の近傍$U$に対し
		\begin{align}
			x \in B \subset U^{\mathrm{o}}
		\end{align}
		を満たす$B \in \mathscr{B}$が存在し,定義より$B \in \mathscr{U}(x)$
		が成り立つから$\mathscr{U}(x)$は$x$の高々可算な基本近傍系をなす.
		\QED
	\end{prf}
	
	\begin{screen}
		\begin{thm}[可分空間の局所有限な開集合族は高々可算集合]
		\label{thm:locally_finite_family_of_open_sets_is_countable_in_separable_space}
			$S$を空でない可分位相空間,
			$M$を$S$で稠密な高々可算集合,$\mathscr{B}$を
			$S$の空でない開集合から成る族とするとき,
			\begin{align}
				\mathscr{B} = \bigcup_{m \in M} \Set{B \in \mathscr{B}}{m \in B}
				\label{eq:thm_locally_finite_family_of_open_sets_is_countable_in_separable_space}
			\end{align}
			が成立する.特に$\mathscr{B}$が局所有限なら$\mathscr{B}$は高々可算集合である.
		\end{thm}
	\end{screen}
	
	\begin{prf}
		稠密性より任意の$E \in \mathscr{B}$は
		$E \cap M \neq \emptyset$を満たすから,$m \in E \cap M$で$
		E \in \Set{B \in \mathscr{B}}{m \in B}$となり
		(\refeq{eq:thm_locally_finite_family_of_open_sets_is_countable_in_separable_space})が出る.
		$\mathscr{B}$が局所有限なら$\Set{B \in \mathscr{B}}{m \in B}$は全て有限集合となり
		$\mathscr{B}$は高々可算集合となる.
		\QED	
	\end{prf}
	
	\begin{screen}
		\begin{thm}[$\sigma$-局所有限な基底が存在すれば,可分$\Longleftrightarrow$第二可算]
			$\sigma$-局所有限な基底が存在する空でない位相空間において,
			可分であることと第二可算であることは同値になる.
		\end{thm}
	\end{screen}
	
	\begin{prf}
		空でない可分位相空間において$\sigma$-局所有限な基底が存在するとき,
		定理\ref{thm:locally_finite_family_of_open_sets_is_countable_in_separable_space}
		よりその基底は高々可算集合であるから第二可算性が満たされる.
		逆に第二可算なら可分であるから定理の主張を得る.
		\QED
	\end{prf}
	
	\begin{screen}
		\begin{thm}[正則かつ$\sigma$-局所有限な基底を持つ$\Longrightarrow$完全正規]
		\end{thm}
	\end{screen}
	
	\begin{screen}
		\begin{dfn}[細分・パラコンパクト]\mbox{}
			\begin{itemize}
				\item $\mathscr{A}$と$\mathscr{B}$を或る集合の被覆とする.
					任意の$B \in \mathscr{B}$に対し$B \subset A$を満たす
					$A \in \mathscr{A}$が存在するとき,
					$\mathscr{B}$を$\mathscr{A}$の{\bf 細分}
					\index{さいぶん@細分}{\bf (refinement)}と呼ぶ.
					位相空間において,被覆の細分で元が全て開(閉)集合であるものを
					{\bf 開(閉)細分}\index{かいさいぶん@開細分}
					{\bf (open(closed) refinement)}と呼ぶ.
					
				\item 任意の開被覆が局所有限な開細分を持つ
					位相空間は{\bf パラコンパクト}\index{ぱらこんぱくと@パラコンパクト}
					{\bf (paracompact)}であるという.
			\end{itemize}
		\end{dfn}
	\end{screen}
	
	\begin{screen}
		\begin{thm}[正則空間の開被覆に対し,$\sigma$-局所有限な開細分が存在する
		$\Longleftrightarrow$局所有限な開細分が存在する]
			$S$を正則空間,$\mathscr{S}$を$S$の開被覆とするとき,以下は全て同値になる:
			\begin{description}
				\item[(a)] $\mathscr{S}$が$\sigma$-局所有限な開細分を持つ.
				\item[(b)] $\mathscr{S}$が局所有限な細分を持つ.
				\item[(c)] $\mathscr{S}$が局所有限な閉細分を持つ.
				\item[(d)] $\mathscr{S}$が局所有限な開細分を持つ.
			\end{description}
		\end{thm}	
	\end{screen}
	
	\begin{screen}
		\begin{thm}[第二可算空間の任意の基底は可算基を内包する]\label{thm:countable_base_of_second_countable_space}
			$\mathscr{B}$を第二可算空間$S$の任意の基底とするとき,或る可算部分集合
			$\mathscr{B}_0 \subset \mathscr{B}$もまた$S$の基底となる.
		\end{thm}
	\end{screen}
	
	\begin{prf}
		$\mathscr{D}$を$S$の可算基とする.
		任意の開集合$U$に対し或る$\mathscr{B}_U \subset \mathscr{B}$が存在して
		$U = \bigcup_{V \in \mathscr{B}_U}V$を満たすから,
		\begin{align}
			\mathscr{D}_U \coloneqq
			\Set{W \in \mathscr{D}}{W \subset V,\ V \in \mathscr{B}_U}
			\label{eq:thm_countable_base_of_second_countable_space_1}
		\end{align}
		とおけば$U = \bigcup_{V \in \mathscr{B}_U} V
			= \bigcup_{V \in \mathscr{B}_U} \bigcup_{\substack{W \in \mathscr{D}_U \\ W \subset V}} W
			\subset \bigcup_{W \in \mathscr{D}_U} W
			\subset U$より
		\begin{align}
			U = \bigcup_{W \in \mathscr{D}_U} W
			\label{eq:thm_countable_base_of_second_countable_space_2}
		\end{align}
		が成り立つ.ここで(\refeq{eq:thm_countable_base_of_second_countable_space_1})より
		任意の$W \in \mathscr{D}_U$に対して
		$\Set{V \in \mathscr{B}}{W \subset V} \neq \emptyset$であるから
		\begin{align}
			\Phi_U \in \prod_{W \in \mathscr{D}_U} \Set{V \in \mathscr{B}}{W \subset V}
		\end{align}
		が取れる.$\mathscr{B}_U' \coloneqq \Set{\Phi_U(W)}{W \in \mathscr{D}_U}$とすれば
		$U = \bigcup_{W \in \mathscr{D}_U} W \subset \bigcup_{W \in \mathscr{D}_U} \Phi(W)
		\subset \bigcup_{V \in \mathscr{B}_U'} V \subset U$より
		\begin{align}
			U = \bigcup_{V \in \mathscr{B}_U'} V
			\label{eq:thm_countable_base_of_second_countable_space_3}
		\end{align}
		が満たされ,
		\begin{align}
			\mathscr{B}_0 \coloneqq \bigcup_{W \in \mathscr{D}} \mathscr{B}_W'
		\end{align}
		と定めれば$\mathscr{B}_0$は求める$S$の可算基となる.実際,任意の開集合$U$に対し
		(\refeq{eq:thm_countable_base_of_second_countable_space_2})と
		(\refeq{eq:thm_countable_base_of_second_countable_space_3})より
		\begin{align}
			U = \bigcup_{W \in \mathscr{D}_U} W
			= \bigcup_{W \in \mathscr{D}_U} \bigcup_{V \in \mathscr{B}_W'} V
		\end{align}
		となる.
		\QED
	\end{prf}
	
	\begin{screen}
		\begin{thm}[局所コンパクトHausdorff空間が第二可算なら$\sigma$-コンパクト]\label{thm:second_countable_Hausdorff_sigma_compact}
			$S$が第二可算性をもつ局所コンパクトHausdorff空間なら,
			次を満たすコンパクト部分集合の列$(K_n)_{n=1}^\infty$が存在する:
			\begin{align}
				K_n \subset K_{n+1}^{\mathrm{o}},
				\quad S = \bigcup_{n=1}^\infty K_n.
			\end{align}
		\end{thm}
	\end{screen}
	
	\begin{prf}
		任意の$x \in S$に対して閉包がコンパクトな開近傍$U_x$を取っておく.
		$\mathscr{O}$を$S$の開集合系として
		\begin{align}
			\mathscr{B} \coloneqq
			\Set{U \in \mathscr{O}}{\mbox{$\overline{U}$がコンパクト}}
		\end{align}
		とおけば,$\mathscr{B}$は$\mathscr{O}$の基底となる.実際,
		任意の$O \in \mathscr{O}$に対し$O \cap U_x \in \mathscr{B}$かつ
		\begin{align}
			O = \bigcup_{x \in O} O \cap U_x
		\end{align}
		となる.従って定理\ref{thm:countable_base_of_second_countable_space}より
		或る可算部分集合$\{U_n\}_{n=1}^\infty \subset \mathscr{B}$が
		$\mathscr{O}$の基底となる.いま,$K_1 \coloneqq \overline{U_1}$として,
		またコンパクト集合$K_n$が選ばれたとして,
		$K_n$の有限被覆$\mathscr{U}_n \subset \mathscr{B}_0$を取り
		\begin{align}
			K_{n+1} \coloneqq \overline{U_{n+1}} \cup \bigcup_{V \in \mathscr{U}_n} \overline{V}
		\end{align}
		とすれば,$K_{n+1}$はコンパクトであり$K_n \subset K_{n+1}^{\mathrm{o}}$を満たす.
		この操作で$(K_n)_{n=1}^\infty$を構成すれば
		\begin{align}
			S = \bigcup_{n=1}^\infty U_n \subset \bigcup_{n=1}^\infty K_n \subset S
		\end{align}
		が成立する.
		\QED
	\end{prf}
	\subsection{商位相}
	\begin{screen}
		\begin{thm}[商位相]
			位相空間$(S,\mathscr{O})$に同値関係$\sim$が定まっているとき,
			$x \in S$からその同値類$\pi(x)$への対応
			\begin{align}
				\pi: S \ni x \longmapsto \pi(x) \in S/\sim
			\end{align}
			を商写像\index{しょうしゃぞう@商写像}(quotient mapping)という.
			すなわち商写像は
			\begin{align}
				x \sim y \quad \Longleftrightarrow \quad
				\pi(x) = \pi(y)
			\end{align}
			を満たす.また,商写像を連続にする$S/\sim$の最強の位相,つまり
			\begin{align}
				\mathscr{O}(S/\sim) \coloneqq
				\Set{V \subset S/\sim}{\pi^{-1}(V) \in \mathscr{O}}
			\end{align}
			で定まる位相を$S/\sim$の商位相
			\index{しょういそう@商位相}(quotient topology)という.
		\end{thm}
	\end{screen}
	
	\begin{screen}
		\begin{thm}[商空間が$T_1 \Longleftrightarrow$同値類が元の空間で閉じている]
		\label{thm:quotient_space_T_1_iff_each_equivalence_class_closed}
			$S$を位相空間,$\sim$を$S$上の同値関係,$\pi:S \longrightarrow S/\sim$を商写像
			とする.このとき次が成り立つ:
			\begin{align}
				\mbox{$S/\sim$が$T_1$空間である}
				\quad \Longleftrightarrow \quad
				\mbox{任意の$x \in S$に対し$\pi(x)$が$S$の閉集合である}.
			\end{align}
		\end{thm}
	\end{screen}
	
	\begin{prf}
		任意の$F \subset S/\sim$に対し
		\begin{align}
			\mbox{$F$が閉} \quad \Longleftrightarrow \quad
			\mbox{$\pi^{-1}(F^c) = \pi^{-1}(F)^c$が開} \quad \Longleftrightarrow \quad
			\mbox{$\pi^{-1}(F)$が閉}
		\end{align}
		となる.いま任意の$x \in S$に対し
		$\pi(x) = \pi^{-1}(\pi(x))$が満たされているから定理の主張を得る.
		\QED
	\end{prf}
	
	\begin{screen}
		\begin{thm}[商写像が開なら,商空間がHausdorff
		$\Longleftrightarrow$対角線集合が閉]
		\label{thm:quotient_space_Hausdorff_iff_diagonal_set_closed}
			$S$を位相空間,$\sim$を$S$上の同値関係,$\pi:S \longrightarrow S/\sim$を商写像
			とする.このとき,$\pi$が開写像であれば次が成立する:
			\begin{align}
				\mbox{$S/\sim$がHausdorff} \quad \Longleftrightarrow \quad
				\mbox{$\Set{(x,y) \in S \times S}{x \sim y}$が閉}.
			\end{align}
		\end{thm}
	\end{screen}
	
	\begin{prf}
		$S/\sim$がHausdorffであるとき,$x \not\sim y$を満たす$(x,y) \in S \times S$に対し
		$\pi(x) \neq \pi(y)$となるから
		\begin{align}
			\pi(x) \in U,\quad \pi(y) \in V,\quad U \cap V = \emptyset
		\end{align}
		を満たす$S/\sim$の開集合$U,V$が取れる.このとき
		$\pi^{-1}(U) \times \pi^{-1}(V)$は$S \times S$の開集合であり
		\begin{align}
			(x,y) \in \pi^{-1}(U) \times \pi^{-1}(V)
			\subset \Set{(s,t) \in S \times S}{s \not\sim t}
		\end{align}
		が成り立つから$\Longrightarrow$が得られる.
		逆に$\Set{(s,t) \in S \times S}{s \not\sim t}$が開集合であるとき,
		$\pi(x) \neq \pi(y)$なら
		\begin{align}
			(x,y) \in U \times V \subset \Set{(s,t) \in S \times S}{s \not\sim t}
		\end{align}
		を満たす$S$の開集合$U,V$が存在し,このとき
		\begin{align}
			\pi(x) \in \pi(U),\quad \pi(y) \in \pi(V),
			\quad \pi(U) \cap \pi(V) = \emptyset
		\end{align}
		となりかつ$\pi$が開写像であるから$\Longleftarrow$が従う.
		\QED
	\end{prf}
	
	\begin{screen}
		\begin{cor}[Hausdorff
		$\Longleftrightarrow$対角線集合が閉]
		\label{cor:quotient_space_Hausdorff_iff_diagonal_set_closed}
			$S$を位相空間とするとき,
			\begin{align}
				\mbox{$S$がHausdorffである}
				\quad \Longleftrightarrow \quad
				\mbox{$\Set{(x,x)}{x \in S}$が$S \times S$で閉じている}.
			\end{align}
		\end{cor}
	\end{screen}
	
	\begin{prf}
		等号$=$を同値関係と見れば$S$と$S/=$は商写像により同相となるから,
		定理\ref{thm:quotient_space_Hausdorff_iff_diagonal_set_closed}より
		\begin{align}
			\mbox{$S$がHausdorff} \quad \Longleftrightarrow \quad
			\mbox{$S/=$がHausdorff} \quad \Longleftrightarrow \quad
			\mbox{$\Set{(x,x)}{x \in S}$が閉}
		\end{align}
		が成立する.
		\QED
	\end{prf}
	
	\begin{screen}
		\begin{dfn}[(位相的)埋め込み写像]
			$S,T$を位相空間とするとき,$S$から$T$への{\bf (位相的)埋め込み}\index{うめこみ@埋め込み}
			{\bf (embedding)}とは,連続な単射$i:S \longrightarrow T$で,$S$と(相対位相を入れた)
			$i(S)$を$i$
			(の終集合を$i(S)$に制限した全単射)により同相とするものである.
		\end{dfn}
	\end{screen}
	
	\begin{screen}
		\begin{dfn}[コンパクト化]
			$S$をコンパクトではない位相空間,$K$をコンパクト位相空間として,
			$S$が$K$に稠密に埋め込まれるとき,言い換えれば,$S$から$K$への
			位相的埋め込み$i$が存在して$i(S)$が$K$で稠密となるとき,$K$を(埋め込み$i$による)
			$S$の{\bf コンパクト化}\index{こんぱくとか@コンパクト化}
			{\bf (compactification)}と呼ぶ.
		\end{dfn}
	\end{screen}
	
	\begin{screen}
		\begin{thm}[一点を追加すればコンパクト空間となる(Alexandroff拡大)]
		\end{thm}
	\end{screen}
	
	\begin{prf}\mbox{}
		\begin{description}
			\item[第一段]
				$\mathscr{O}$が$K$上の位相であることを示す.
				先ず$\emptyset\ (= K \backslash K)$は$S$で閉かつコンパクトであるから
				$K,\emptyset \in \mathscr{O}$となる.
				また$U,V \in \mathscr{O}$を取れば,
				\begin{itemize}
					\item $x_\infty \notin U$かつ$x_\infty \notin V$なら
						$U,V$は$S$の開集合であるから$U \cap V \in \mathscr{O}$.
					
					\item $x_\infty \in U$かつ$x_\infty \notin V$のとき,
						$V' \coloneqq V \backslash \{x_\infty\}$とおけば
						\begin{align}
							K \backslash V = S \backslash V'
						\end{align}
						となり,$K \backslash V$は$S$で閉じているから$V'$は$S$の開集合であり
						$U \cap V = U \cap V' \in \mathscr{O}$が従う.
						
					\item $x_\infty \in U$かつ$x_\infty \in V$のとき,
						$K \backslash (U \cap V)= (K \backslash U) \cup (K \backslash V)$
						より$K \backslash (U \cap V)$は$S$で閉かつコンパクトなので
						$U \cap V \in \mathscr{O}$となる.
				\end{itemize}
				従って$\mathscr{O}$は有限交叉で閉じる.
				任意の$\mathscr{U} \subset \mathscr{O}$に対し
				$\mathscr{U}_1 \coloneqq \Set{U \in \mathscr{U}}{x_\infty \in U},
				\mathscr{U}_2 \coloneqq \Set{U \in \mathscr{U}}{x_\infty \notin U}$
				とおけば,$\mathscr{U}_2$の元は$S$の開集合なので
				$x_\infty \notin \bigcup \mathscr{U}$なら
				$\bigcup \mathscr{U} = \bigcup \mathscr{U}_2 \in \mathscr{O}$となる.
				$x_\infty \in \bigcup \mathscr{U}$のとき,
				\begin{align}
					K \left\backslash \bigcup \mathscr{U} \right.
					= \left(K \left\backslash \bigcup \mathscr{U}_1 \right.\right)
					\bigcap \left(S \left\backslash \bigcup \mathscr{U}_2 \right.\right)
					= \Biggl( \bigcap_{U \in \mathscr{U}_1} K \backslash U \Biggr)
					\bigcap \Biggl( \bigcap_{U \in \mathscr{U}_2} S \backslash U \Biggr)
				\end{align}
				より$K\left\backslash \bigcup \mathscr{U}\right.$は$S$で閉じ,
				また定理\ref{thm:closed_subset_of_compact_set_is_compact_on_Hausdorff_space}
				より$S$でコンパクトでもあるから$\bigcup \mathscr{U} \in \mathscr{O}$が従う.
				
			\item[第二段] $S$から$K$への恒等写像$i$が埋め込みであることを示す.
				実際$i$は単射であり,また$\mathscr{O}$が$S$の位相を含むから
				$i$は開写像でもある.$i$の連続性は
		\end{description}
	\end{prf}
	\subsection{有向点族}
	第一可算性が仮定された空間では
	可算個の点族(点列)の収束を用いることでいくつかの位相的概念を記述できるが,
	一般に位相空間では近傍が`多すぎる'ため位相概念を記述するのに点列では間に合わない.
	有向点族の理論では,非可算個の集合に或る種の`向き'を与えることで
	それを添数集合とする点族に収束の概念が定式化され,
	一般の位相空間における閉包や連続性,コンパクト性の概念を点族の収束により記述することが可能となる.
	
	\begin{screen}
		\begin{dfn}[有向集合]
			空でない集合$\Lambda$において
			任意の有限部分集合が上界を持つような前順序が定まっているとき,
			つまり次を満たす二項関係$\leq$が定まっているとき,
			対$(\Lambda,\leq)$を有向集合\index{ゆうこうしゅうごう@有向集合}(directed set)と呼ぶ:
			\begin{description}
				\item[(反射律)] $\lambda \leq \lambda,\quad (\forall \lambda \in \Lambda)$,
				\item[(推移律)] $\lambda \leq \mu,\ \mu \leq \nu 
					\quad \Longrightarrow \quad \lambda \leq \nu,\quad 
					(\forall \lambda,\mu,\nu \in \Lambda)$,
				\item[(有向律)] 
					$M \subset \Lambda$が有限なら
					$\mu \leq \lambda,\ (\forall \mu \in M)$を満たす
					$\lambda \in \Lambda$が存在する.
			\end{description}
			また$\lambda < \mu \overset{\mathrm{def}}{\Longleftrightarrow} 
			\mbox{$\lambda \leq \mu$かつ$\lambda \neq \mu$}$と定める.
		\end{dfn}
	\end{screen}
	正の自然数全体$\N$や実数全体$\R$は,通常の順序により
	有向集合となっている.また位相空間の一点の近傍全体も
	\begin{align}
		U \leq V \quad \overset{\mathrm{def}}{\Longleftrightarrow} \quad
		U \supset V
	\end{align}
	により有向集合となる.
	
	\begin{screen}
		\begin{dfn}[有向点族]
			有向集合を添数集合とする点族
			(P. \pageref{dfn:family_collection})
			を有向点族\index{ゆうこうてんぞく@有向点族}(net)と呼ぶ.
			$(\Lambda,\leq),\ (\Gamma,\preceq)$を有向集合,
			$(x_\lambda)_{\lambda \in \Lambda}$を有向点族とするとき,
			共終かつ序列を保つ写像$f:\Gamma \longrightarrow \Lambda$:
			つまり
			\begin{description}
				\item[(単調性)] $\gamma \preceq \xi \quad \Longrightarrow \quad
					f(\gamma) \leq f(\xi),\quad (\forall \gamma,\xi \in \Gamma)$,
				\item[(共終性)] $f(\Gamma)$が非有界:
					任意の$\lambda \in \Lambda$に対し
					$\lambda \leq f(\gamma)$を満たす$\gamma \in \Gamma$が存在する
			\end{description}
			を満たす写像$f$に対して,$\left(x_{f(\gamma)}\right)_{\gamma \in \Gamma}$を
			$(x_\lambda)$の部分有向点族\index{ぶぶんゆうこうてんぞく@部分有向点族}
			(subnet)と呼ぶ:
		\end{dfn}
	\end{screen}
	特に$\N$を有向集合とする有向点族を点列\index{てんれつ@点列}(sequence)と呼ぶ.
	また点列$(x_n)_{n \in \N}$に対し
	\begin{align}
		f:\N \ni k \longmapsto n_k \in \N,
		\quad (n_1 < n_2 < n_3 < \cdots)
	\end{align}
	で定まる部分有向点族$\left(x_{n_k}\right)_{k \in \N}$
	を部分列\index{ぶぶんれつ@部分列}(subsequence)と呼ぶ.
	一般の部分有向点族ではそれを定める写像$f$に単射性を仮定していないが
	(cf. Tychonoff plank),部分列は$k < j$なら$n_k < n_j$が満たされるものと約束する.
	従って点列の部分有向点族といってもそれが部分列となっているとは限らない.
	
	\begin{screen}
		\begin{dfn}[有向点族の収束\index{ゆうこうてんぞくのしゅうそく@有向点族の収束}]
			$x = (x_\lambda)_{\lambda \in \Lambda}$を位相空間$S$と
			有向集合$(\Lambda,\leq)$で定まる有向点族とする.
			点$a \in S$において,$a$の任意の近傍$U$に対し或る
			$\lambda_0 \in \Lambda$が存在して
			\begin{align}
				\lambda_0 \leq \lambda \quad \Longrightarrow \quad
				x_\lambda \in U
			\end{align}
			となるとき,$(x_\lambda)$は$a$に収束する(converge)といい
			$\lim x_\lambda = a$や$\lim_{\lambda} x_\lambda = a$と書く.
			また$(x_\lambda)_{\lambda \in \Lambda}$が
			部分集合$A$上の有向点族である場合,$(x_\lambda)_{\lambda \in \Lambda}$が
			$A$の点に収束するとき$(x_\lambda)_{\lambda \in \Lambda}$は`$A$で収束する'という.
		\end{dfn}
	\end{screen}
	
	\begin{screen}
		\begin{thm}[有向点族が収束する$\Longleftrightarrow$任意の部分点族が収束する]
		\label{thm:a_net_converges_iff_every_subnet_converges}
			$(x_\lambda)_{\lambda \in \Lambda}$を位相空間$S$
			と有向集合$(\Lambda,\leq)$で定まる有向点族とし,
			また$a$を$S$の任意の点とするとき
			\begin{align}
				\mbox{$(x_\lambda)_{\lambda \in \Lambda}$が$a$に収束する}
				\quad \Longleftrightarrow \quad
				\mbox{$(x_\lambda)_{\lambda \in \Lambda}$
				の任意の部分有向点族が$a$に収束する}
				\label{eq:thm_a_net_converges_iff_every_subnet_converges_2}
			\end{align}
			が成立する.特に$(x_\lambda)_{\lambda \in \Lambda}$が点列であるとき,
			右辺で部分有向点族を部分列に替えても同値関係は成立する.
		\end{thm}
	\end{screen}
	
	\begin{prf}
				$(x_\lambda)_{\lambda \in \Lambda}$が$a$に収束するとき,
				$a$の任意の近傍$U$に対し或る$\lambda_0 \in \Lambda$が存在して
				\begin{align}
					\lambda_0 \leq \lambda
					\quad \Longrightarrow \quad
					x_\lambda \in U
				\end{align}
				を満たす.$(y_\gamma)_{\gamma \in \Gamma}$
				を$(x_\lambda)_{\lambda \in \Lambda}$
				の部分有向点族とするとき,つまりこのとき或る有向集合$(\Gamma,\preceq)$と
				$f:\Gamma \longrightarrow \Lambda$により
				$y_\gamma = x_{f(\gamma)}$と表せるが,$f$の共終性から
				$\lambda_0 \leq f(\gamma_0)$を満たす$\gamma_0 \in \Gamma$が存在し,
				$f$の単調性と$\leq$の推移律より
				\begin{align}
					\gamma_0 \preceq \gamma
					\quad \Longrightarrow \quad
					f(\gamma_0) \leq f(\gamma)
					\quad \Longrightarrow \quad
					\lambda_0 \leq f(\gamma)
					\quad \Longrightarrow \quad
					y_\gamma = x_{f(\gamma)} \in U
				\end{align}
				が従うから$(y_\lambda)_{\gamma \in \Gamma}$は$a$に収束する.
				逆に$(x_\lambda)_{\lambda \in \Lambda}$が$a$に収束しないとき,
				$a$の或る近傍$V$では任意の$\lambda \in \Lambda$に対し
				\begin{align}
					\lambda \leq \mu,
					\quad x_\mu \notin V
					\label{eq:thm_a_net_converges_iff_every_subnet_converges_1}
				\end{align}
				を満たす$\mu \in \Lambda$が取れる.
				ここで
				\begin{align}
					\Gamma \coloneqq \Set{\lambda \in \Lambda}{x_\lambda \notin U}
				\end{align}
				とおけば,任意の有限集合$M \subset \Gamma$に対し
				$\Lambda$における上界$\lambda$が存在するが,
				(\refeq{eq:thm_a_net_converges_iff_every_subnet_converges_1})
				より$\lambda \leq \mu$を満たす
				$\mu \in \Gamma$が取れるから$(\Gamma,\leq)$は有向集合となる.
				恒等写像$\Gamma \longrightarrow \Lambda$は単調性と共終性を満たし,
				この場合の部分有向点族$(x_\gamma)_{\gamma \in \Gamma}$は$a$に収束しないから
				(\refeq{eq:thm_a_net_converges_iff_every_subnet_converges_2})が出る.
				$(x_\lambda)_{\lambda \in \Lambda}$が$a$に収束しない点列であるとき,
				任意の$n \in \N$に対して
				\begin{align}
					\inprod<n> \coloneqq
					\Set{m \in \N}{n < m,\ x_m \notin U}
				\end{align}
				は空ではない.
				$\N$の空でない部分集合の全体を$\mathscr{N}$として
				選択関数$\Phi \in \prod \mathscr{N}$を取り
				\begin{align}
					n_1 &\coloneqq \Phi(\inprod<1>), \\
					n_2 &\coloneqq \Phi(\inprod<n_1>), \\
					n_3 &\coloneqq \Phi(\inprod<n_2>), \\
					&\vdots
				\end{align}
				で$\{n_k\}_{k \in \N}$を定めれば,
				$(x_{n_k})_{k \in \N}$は$a$に収束しない部分列となる.
				\QED
	\end{prf}
	
	\begin{screen}
		\begin{thm}[有向点族の密集点に対する収束部分点族の存在]
			$(x_\lambda)_{\lambda \in \Lambda}$を位相空間$S$
			と有向集合$(\Lambda,\leq)$で定まる有向点族,$a$を$S$の点とするとき
			\begin{align}
				\mbox{$a$が$\{x_\lambda\}_{\lambda \in \Lambda}$の密集点である}
				\quad \Longleftrightarrow \quad
				\mbox{$a$に収束する$(x_\lambda)_{\lambda \in \Lambda}$の部分有向列が存在する}.
			\end{align}
			特に$\Lambda = \N$かつ
			$a$が可算な基本近傍系を持つ場合,右辺の部分有向点族を部分列に替えて
			同値関係が成立する.
		\end{thm}
	\end{screen}
	
	\begin{prf}
		$a$が$\{x_\lambda\}_{\lambda \in \Lambda}$の密集点であるとき,
		$\mathscr{U}$を$a$の基本近傍系とすれば
		任意の$U \in \mathscr{U}$に対し
		或る$\lambda \in \Lambda$が存在して$x_\lambda \in U$となるから,
		選択関数$\Phi \in \prod_{U \in \mathscr{U}} 
		\Set{\lambda \in \Lambda}{x_\lambda \in U}$を取り
		\begin{align}
			\Gamma \coloneqq \Set{(\Phi(U),U)}{U \in \mathscr{U}}
		\end{align}
		と定める.$\Lambda = \N$かつ$\mathscr{U} = \{U_k\}_{k \in \N}$と書けるときは
		\begin{align}
			\Gamma \coloneqq \Set{(n_k,U_k)}{k \in \N,\ x_{n_k} \in U_k,\ n_1 < n_2 < \cdots}
		\end{align}
		で定める.$\Gamma$において二項関係$\preceq$を
		\begin{align}
			(\lambda,U) \preceq (\mu,V) 
			\quad \overset{\mathrm{def}}{\Longleftrightarrow} \quad
			\mbox{$\lambda \leq \mu$かつ$U \supset V$}
		\end{align}
		で定めれば$(\Gamma,\preceq)$は有向集合となる.実際
		$\lambda \leq \lambda$かつ$U \supset U$より
		$(\lambda,U) \preceq (\lambda,U),\ (\forall (\lambda,U) \in \Gamma)$となり,
		\begin{align}
			(\lambda_1,U_1) \preceq (\lambda_2,U_2),\
			(\lambda_2,U_2) \preceq (\lambda_3,U_3) 
			&\quad \Longrightarrow \quad
			\lambda_1 \leq \lambda_2,\ \lambda_2 \leq \lambda_3,
			\ U_1 \supset U_2,\ U_2 \supset U_3 \\
			&\quad \Longrightarrow \quad
			\lambda_1 \leq \lambda_3,\ U_1 \supset U_3 \\
			&\quad \Longrightarrow \quad
			(\lambda_1,U_1) \preceq (\lambda_3,U_3)
		\end{align}
		より推移律も出る.また任意の有限個の$(\lambda_i,U_i) \in \Gamma,\ (i=1,2,\cdots,n)$
		に対し或る$\lambda \in \Lambda$と$U \in \mathscr{U}$が存在して
		\begin{align}
			\lambda_i \leq \lambda,\ (1 \leq i \leq n);
			\quad \bigcap_{i=1}^n U_i \supset U
		\end{align}
		を満たすが,$x_\mu \in U$を満たす$\mu \in \Lambda$を取れば
		或る$\nu \in \Lambda$で$\lambda,\mu \leq \nu$となり
		
		,$\Gamma$の任意の有限部分集合は上界を持つ.そして次の写像
		\begin{align}
			f:\Gamma \ni (\lambda,U) \longmapsto \lambda \in \Lambda
		\end{align}
		は単調かつ共終であるから$(x_{f(\gamma)})_{\gamma \in \Gamma}$は部分有向点族となる.
		そして任意に$U_0 \in \mathscr{U}$を取り
		$\lambda_0 \coloneqq \Phi(U_0)$とおけば
		\begin{align}
			(\lambda_0,U_0) \preceq (\lambda,U)
			\quad \Longrightarrow \quad
			x_{f(\lambda,U)} = x_\lambda \in U \subset U_0
		\end{align}
		となるから$(x_{f(\gamma)})_{\gamma \in \Gamma}$は$a$に収束する.
		逆に$a$に収束する$(x_\lambda)$の部分有向族(又は部分列)$(y_\gamma)$が存在するとき,
		$\{y_\gamma\}$は$\{x_\lambda\}$の部分集合でありかつ
		$a$の任意の近傍と交わるから,$\{x_\lambda\}$も$a$の任意の近傍と交叉する.
		\QED
	\end{prf}
	
	\begin{screen}
		\begin{thm}[コンパクト$\Longleftrightarrow$任意の有向点族が収束部分有向点族を持つ]
			位相空間$S$の部分集合$A$に対し,
			\begin{align}
				\mbox{$A$がコンパクト部分集合}
				\quad \Longleftrightarrow \quad
				\mbox{$A$上の任意の有向点族が$A$で収束する部分有向点族を持つ}.
			\end{align}
		\end{thm}
	\end{screen}
	\subsection{一様空間}
	一様空間は後述する距離空間や位相線型空間の上位概念である.
	距離空間では距離により,位相線型空間では$0$ベクトル周りの近傍を
	任意の点に移すことにより,空間全体で共通する点同士の`近さ'を規定することが出来る.
	一般の位相空間では点同士の`近さ'を相対的に比較することはできない
	(つまり点$x,y$の`近さ'と点$a,b$の`近さ'を比較する術がない)が,
	一様構造が導入された空間では各点に共通する近傍概念が定義されるため`近さ'の相対比較が可能になり,
	一様連続,一様収束,完備,全有界といった性質が定式化される.
	
	始めに次の集合演算を定義する.
	$S$を集合とするとき,任意の$V \subset S \times S$に対して,
	その反転$V^{-1}$を
	\begin{align}
		V^{-1} \coloneqq \Set{(y,x)}{(x,y) \in V}
	\end{align}
	により定め,また$S \times S$における二項演算$\circ$を
	\begin{align}
		U \circ V \coloneqq
		\Set{(x,z)}{\mbox{或る$y \in S$で$(x,y) \in U$かつ$(y,z) \in V$となる}},
		\quad (U,V \subset S \times S)
	\end{align}
	で定める.このとき演算$\circ$について次が成り立つ:
	$V,W \subset S \times S$を空でない部分集合とすれば
	\begin{align}
		W \circ W \subset V
		\quad \Longleftrightarrow \quad
		\mbox{任意の$x,y,z \in S$に対し$(x,y),(y,z) \in W$なら$(x,z) \in V$}.
	\end{align}
	
	\begin{screen}
		\begin{dfn}[近縁系]\label{dfn:uniform_structure}
			$S$を空でない集合とするとき,次の(US1)$\sim$(US5)を満たす
			$S \times S$の部分集合族$\mathscr{V}$を
			$S$の近縁系\index{きんえんけい@近縁系}(system of entourages)
			や一様構造\index{いちようこうぞう@一様構造}(uniform structure)と呼び,
			対$(S,\mathscr{V})$を一様空間\index{いちようくうかん@一様空間}
			(uniform space)と呼ぶ:
			\begin{description}
				\item[(US1)] $\mathscr{V} \neq \emptyset$かつ任意の$V \in \mathscr{V}$は
					$\Set{(x,x)}{x \in S} \subset V$を満たす.
					
				\item[(US2)] 任意の$V \subset S \times S$に対し
					$V \in \mathscr{V} \Longleftrightarrow V^{-1} \in \mathscr{V}$.
				\item[(US3)] 任意の$U,V \in \mathscr{V}$に対し$U \cap V \in \mathscr{V}$.
				\item[(US4)] 任意の$V \in \mathscr{V}$に対し或る$W \in \mathscr{V}$が存在して$W \circ W \subset V$.またこれは次と同値である:
					\begin{align}
						\forall V \in \mathscr{V};\ 
						\exists W \in \mathscr{V};\ 
						\forall x,y,z \in S;\quad
						(x,y),(y,z) \in W \Longrightarrow (x,z) \in V.
					\end{align}
					
				\item[(US5)] 任意の$V \subset \mathscr{V}$に対し
					$V \subset R$なら$R \in \mathscr{V}$.
			\end{description}
			$\mathscr{V}$の元を近縁\index{きんえん@近縁}(entourage)と呼び,
			近縁$V$が$V = V^{-1}$を満たすとき$V$は対称\index{たいしょう@対称}
			である(symmetric)という.また基本近傍系と同様に
			$\mathscr{V}$の部分集合$\mathscr{U}$で
			$\mathscr{V}$の任意の近縁に対しそれに含まれる$\mathscr{U}$の元が取れるとき,
			$\mathscr{U}$を$\mathscr{V}$の基本近縁系
			\index{きほんきんえんけい@基本近縁系}(fundamental system of entourages)と呼ぶ.
		\end{dfn}
	\end{screen}
	(US3)について,$V$に対し$W$を対称なものとして取ることができる.実際
	$U \in \mathscr{V}$が$U \circ U \subset V$を満たすとき,
	\begin{align}
		W \coloneqq U \cap U^{-1}
	\end{align}
	で$W \in \mathscr{V}$を定めれば,$W$は対称であり
	$W \circ W \subset U \circ U \subset V$となる.
	
	\begin{screen}
		\begin{thm}[近縁は対角線集合を覆う`ベルト'を持つ]\label{thm:uniform_structure}
			$(S,\mathscr{V})$を一様空間とするとき,
			任意の$V \in \mathscr{V}$に対し
			\begin{align}
				W_x \times W_x \subset V,\quad (\forall x \in S)
			\end{align}
			を満たす対称な$W \in \mathscr{V}$が存在する.
			ただし$W_x = \Set{y \in S}{(x,y) \in W}$である.
		\end{thm}
	\end{screen}
	
	\begin{prf}
		近縁系の定義より$U \circ U \subset V$を満たす
		$U \in \mathscr{V}$が存在する.ここで
		\begin{align}
			W \coloneqq U \cap U^{-1}
		\end{align}
		で対称な$W \in \mathscr{V}$を定めれば,任意の$x \in S$に対し
		\begin{align}
			y,z \in W_x \quad \Longrightarrow \quad
			(x,y),(x,z) \in W \quad \Longrightarrow \quad
			(y,x),(x,z) \in W \quad \Longrightarrow \quad
			(y,z) \in V
		\end{align}
		が成立し$W_x \times W_x \subset V$が得られる.
		\QED
	\end{prf}
	
	\begin{screen}
		\begin{thm}[近縁系で導入する位相]\label{thm:topology_induced_by_the_uniformity}
			$\mathscr{V}$を集合$S$の近縁系,$\mathscr{U}$を
			$\mathscr{V}$の基本近縁系とする.$V_x$を
			\begin{align}
				V_x \coloneqq \Set{y \in S}{(x,y) \in V},
				\quad (V \in \mathscr{V},\ x \in S)
			\end{align}
			で定義するとき,各$x \in S$で
			\begin{align}
				\mathscr{U}(x) \coloneqq \Set{U_x}{U \in \mathscr{U}}
			\end{align}
			とおけば$\{\mathscr{U}(x)\}_{x \in S}$は定理
			\ref{thm:a_local_base_restores_the_topology}
			の(LB1)(LB2)(LB3)を満たす.このとき$\{\mathscr{U}(x)\}_{x \in S}$が基本近傍系となる
			$S$の位相が唯一つ定まるが,別の基本近縁系を用いても同じ位相が定まる.
		\end{thm}
	\end{screen}
	
	\begin{prf}
		$\mathscr{U}$は空でないから$\mathscr{U}(x)$も空ではない.
		そして任意の$U \in \mathscr{U}$は$\Set{(x,x)}{x \in S}$を含むから
		$x \in U_x$となり(LB1)が満たされる.また任意の$U_x,V_x \in \mathscr{U}(x)$に対し
		或る$W \in \mathscr{U}$で$W \subset U \cap V$となるから,
		$W_x \subset U_x \cap V_x$が従い(LB2)も出る.
		任意の$U_x \in \mathscr{U}(x)$に対し
		定理\ref{thm:uniform_structure}より
		\begin{align}
			W_y \times W_y \subset U,\quad (\forall y \in S)
		\end{align}
		を満たす対称な$W \in \mathscr{V}$が存在する.
		$R \subset W$を満たす$R \in \mathscr{U}$を取れば
		\begin{align}
			y \in R_x \quad \Longrightarrow \quad
			y \in W_x \quad \Longrightarrow \quad
			(x,y) \in W_x \times W_x \subset U \quad \Longrightarrow \quad
			y \in U_x
		\end{align}
		となるから$R_x \subset U_x$が成り立ち,このとき任意の$y \in R_x$に対し
		\begin{align}
			z \in R_y \quad \Longrightarrow \quad
			(y,z) \in W = W^{-1} \quad \Longrightarrow \quad
			(x,z) \in W_y \times W_y \subset U \quad \Longrightarrow \quad
			z \in U_x
		\end{align}
		より$R_y \subset U_x$が満たされるから(LB3)も得られる.
		従って定理\ref{thm:local_base_defines_open_sets}と
		定理\ref{thm:a_local_base_restores_the_topology}より
		$\{\mathscr{U}(x)\}_{x \in S}$が基本近傍系となる$S$の位相
		$\tau_{\mathscr{U}}$が唯一つ定まる.
		いま,$\tilde{\mathscr{U}}$を$\mathscr{V}$の別の基本近縁系として
		\begin{align}
			\tilde{\mathscr{U}}(x) \coloneqq \Set{\tilde{U}_x}{\tilde{U} \in \tilde{\mathscr{U}}},
			\quad (\forall x \in S)
		\end{align}
		とおけば,$\left\{\tilde{\mathscr{U}}(x)\right\}_{x \in S}$は
		$(S,\tau_{\mathscr{U}})$における基本近傍系となる.
		実際,任意の$\tilde{U}_x \in \tilde{\mathscr{U}}(x)$に対し或る$U \in \mathscr{U}$で
		$U_x \subset \tilde{U}_x$となるから$\tilde{U}_x$は$x$の近傍であり,
		一方で任意の$V_x \in \mathscr{U}(x)$に対し
		或る$\tilde{V} \in \tilde{\mathscr{U}}$で
		$\tilde{V}_x \subset V_x$となるから
		$\tilde{\mathscr{U}}(x)$は$x$の基本近傍系をなしている.
		$\left\{\tilde{\mathscr{U}}(x)\right\}_{x \in S}$が基本近傍系となる位相
		は唯一つであるから$\tau_{\tilde{\mathscr{U}}} = \tau_{\mathscr{U}}$が成り立つ.
		\QED
	\end{prf}
	
	\begin{screen}
		\begin{dfn}[一様位相]
			$\mathscr{V}$を集合$S$の近縁系,$\mathscr{U}$を
			$\mathscr{V}$の基本近縁系とする.$U \in \mathscr{U}$と$x \in S$に対し$U_x$を
			\begin{align}
				U_x \coloneqq \Set{y \in S}{(x,y) \in U}
			\end{align}
			で定義するとき,定理\ref{thm:topology_induced_by_the_uniformity}より
			\begin{align}
				\mathscr{U}(x) \coloneqq \Set{U_x}{U \in \mathscr{U}}
			\end{align}
			を各点$x$の基本近傍系とする位相が定まるが,
			別の基本近縁系を取っても同じ位相が定まるので
			これを近縁系$\mathscr{V}$で導入する
			$S$の一様位相\index{いちよういそう@一様位相}(uniform topology)と呼ぶ.
			$S$が位相空間であるとき,$\mathscr{V}$で導入する位相と元の位相が一致することを
			$\mathscr{V}$と元の位相が両立\index{りょうりつ@両立}する(compatible)という.
		\end{dfn}
	\end{screen}
	
	\begin{screen}
		\begin{thm}[部分一様空間]
			$(S,\mathscr{V})$を一様空間とするとき,任意の空でない部分集合$A \subset S$に対し
			\begin{align}
				\mathscr{V}_A \coloneqq 
				\Set{(A \times A) \cap V}{V \in \mathscr{V}}
			\end{align}
			は$A$上の近縁系となる.また$S$に$\mathscr{V}$で位相を導入するとき,
			$A$上の相対位相と$\mathscr{V}_A$は両立する.
		\end{thm}
	\end{screen}
	
	\begin{prf}\mbox{}
		\begin{description}
			\item[第一段] $\mathscr{V}_A$が定義
				\ref{dfn:uniform_structure}の(US1)$\sim$(US5)を満たすことを示す.先ず
				$\mathscr{V} \neq \emptyset$より$\mathscr{V}_A \neq \emptyset$であり,
				\begin{align}
					V \in \mathscr{V} \quad \Longrightarrow \quad
					(a,a) \in V,\ (\forall a \in A) \quad \Longrightarrow \quad
					(a,a) \in (A \times A) \cap V,\ (\forall a \in A)
				\end{align}
				となるから(US1)が満たされる.また任意に$E \in \mathscr{V}_A$を取れば
				或る$V \in \mathscr{V}$で$E = (A \times A) \cap V$と表され,
				\begin{align}
					(x,y) \in E^{-1}
					\quad \Longleftrightarrow \quad
					(y,x) \in (A \times A) \cap V 
					\quad \Longleftrightarrow \quad
					(x,y) \in (A \times A) \cap V^{-1}
				\end{align}
				が成り立つから$E^{-1} \in \mathscr{V}_A$が従い(US2)も満たされる.
				任意の$U,V \in \mathscr{V}$に対し
				\begin{align}
					((A \times A) \cap U) \cap ((A \times A) \cap V)
					= (A \times A) \cap (U \cap V) \in \mathscr{V}_A
				\end{align}
				より(US3)が得られ,また$V \in \mathscr{V}$に対し
				$W \circ W \subset V$となる$W \in \mathscr{V}$を取れば
				\begin{align}
					(x,y),(y,z) \in (A \times A) \cap W
					\quad \Longrightarrow \quad
					x,z \in A,\ (x,z) \in V
					\quad \Longrightarrow \quad
					(x,z) \in (A \times A) \cap V
				\end{align}
				となるから(US4)が出る.
				$(A \times A) \cap V \subset R,\ (V \in \mathscr{V})$を満たす任意の
				$R \subset A \times A$に対し,
				$V \cup R \in \mathscr{V}$より
				\begin{align}
					R = (A \times A) \cap (V \cup R) \in \mathscr{V}_A
				\end{align}
				が成立し(US5)も従う.
			
			\item[第二段] $\mathscr{V}_A$で導入する$A$の位相を
				$\tau_A$と書く.任意の$a \in A$と$V \in \mathscr{V}$に対して
				\begin{align}
					[(A \times A) \cap V]_a \coloneqq
					&\Set{x \in A}{(a,x) \in (A \times A) \cap V} \\
					=& \Set{x \in S}{(a,x) \in V} \cap A
					\eqqcolon V_a \cap A
				\end{align}
				となる.$\Set{[(A \times A) \cap V]_a}{V \in \mathscr{V}}$
				は$\tau_A$における$a$の基本近傍系をなし,
				$\Set{V_a \cap A}{V \in \mathscr{V}}$
				は$A$の相対位相における$a$の基本近傍系をなすが,
				両者が一致するので位相も一致する.
				\QED
		\end{description}
	\end{prf}
	
	\begin{screen}
		\begin{thm}[一様位相空間において$T_0 \Longleftrightarrow T_2$]
		\label{thm:T_0_iff_T_2_on_uniform_topological_space}
			$(S,\mathscr{V})$を一様空間とし,$S$に一様位相を導入する.このとき
			\begin{align}
				\mbox{$S$が$T_0$} \quad \Longleftrightarrow \quad
				\bigcap_{V \in \mathscr{V}}V = \Set{(x,x)}{x \in S}
				\quad \Longleftrightarrow \quad
				\mbox{$S$が$T_2$}
				\label{eq:thm_T_0_iff_T_2_on_uniform_topological_space}
			\end{align}
		\end{thm}
	\end{screen}
	
	\begin{prf} 位相空間が$T_2$なら$T_0$であるから,二つの$\Longrightarrow$を示せば
		(\refeq{eq:thm_T_0_iff_T_2_on_uniform_topological_space})が従う.
		\begin{description}
			\item[一つ目の$\Longrightarrow$]
				$\bigcap_{V \in \mathscr{V}}V \neq \Set{(x,x)}{x \in S}$
				が満たされるとき,或る相異なる二点$x,y \in S$に対し
				\begin{align}
					(x,y),(y,x) \in V, \quad (\forall V \in \mathscr{V})
				\end{align}
				となる.$\Set{V_x \coloneqq \Set{s \in S}{(x,s) \in V}}{V \in \mathscr{V}}$は$x$の基本近傍系をなすから
				\begin{align}
					y \in V_x, \quad (\forall V \in \mathscr{V})
				\end{align}
				が成立し,定理\ref{thm:belongs_to_closure_iff_clusters}より
				$x \in \overline{\{y\}}$が従う.
				対称的に$y \in \overline{\{x\}}$も出るから
				$x$と$y$は位相的に区別不能である.
				
			\item[二つ目の$\Longrightarrow$]
				$\bigcap_{V \in \mathscr{V}}V = \Set{(x,x)}{x \in S}$
				が満たされるとき,任意の相異なる二点$x,y \in S$に対し
				\begin{align}
					(x,y) \in V
				\end{align}
				を満たす$V \in \mathscr{V}$が存在する.
				定理\ref{thm:uniform_structure}より或る対称な$W \in \mathscr{V}$で
				\begin{align}
					W \circ W \subset V,
					\quad W_x \times W_x \subset V,
					\quad W_y \times W_y \subset V
				\end{align}
				となるが,このとき$W_x \cap W_y = \emptyset$が成り立つ.実際,
				$W_x \cap W_y$が空でないとき,$z \in W_x \cap W_y$を取れば
				\begin{align}
					(x,z),(y,z) \in W \quad \Longrightarrow \quad
					(x,z),(z,y) \in W \quad \Longrightarrow \quad
					(x,y) \in V
				\end{align}
				が従い矛盾が生じる.$W_x,W_y$はそれぞれ$x,y$の近傍であるから二つ目の$\Longrightarrow$を得る.
				\QED
		\end{description}
	\end{prf}
	
	\begin{screen}
		\begin{dfn}[一様連続性]
			$(S,\mathscr{U})$と$(T,\mathscr{V})$を一様空間として
			$\mathscr{U},\mathscr{V}$により$S,T$に一様位相を導入し,
			$f:S \longrightarrow T$を連続写像とする.
			任意の$V \in \mathscr{V}$に対し或る$U \in \mathscr{U}$が存在して
			\begin{align}
				(x,y) \in U \quad \Longrightarrow \quad (f(x),f(y)) \in V
			\end{align}
			となるとき,$f$は一様連続\index{いちようれんぞく@一様連続}である(uniformly continuous)という.
		\end{dfn}
	\end{screen}
	
	\begin{screen}
		\begin{thm}[コンパクト集合上で連続写像は一様連続]
			$(S,\mathscr{U})$と$(T,\mathscr{V})$を一様空間として
			$\mathscr{U},\mathscr{V}$により$S,T$に一様位相を導入し,
			$f:S \longrightarrow T$を連続写像とする.
			$A \subset S$をコンパクト部分集合とするとき,
			$f$は$A$上で一様連続となる.つまり,
			任意の$V \in \mathscr{V}$に対し或る$U \in \mathscr{U}$が存在して
			\begin{align}
				(x,y) \in U \cap A \quad \Longrightarrow \quad (f(x),f(y)) \in V.
			\end{align}
		\end{thm}
	\end{screen}
	
	\begin{prf}
		任意の$M \in \mathscr{U}, s \in S$に対し
		\begin{align}
			M_s \coloneqq \Set{x \in S}{(s,x) \in M}
		\end{align}
		と定め,$W \in \mathscr{V},\ t \in T$に対しても同様に$W_t$を定める.
		任意に$V \in \mathscr{V}$を取れば,定理\ref{thm:uniform_structure}より
		或る$W \in \mathscr{V}$で
		\begin{align}
			W_t \times W_t \subset V,
			\quad (\forall t \in T)
		\end{align}
		となる.$f$は連続であるから任意の$s \in S$に対し或る$N(s) \in \mathscr{U}$が存在して
		\begin{align}
			(s,x) \in N(s) \quad \Longrightarrow \quad
			f(x) \in W_{f(s)}
		\end{align}
		が成り立ち,$M(s) \circ M(s) \subset N(s)$を満たす対称な$M(s) \in \mathscr{U}$を取れば,
		定理\ref{thm:subset_is_compact_iff_every_original_open_cover_contains_finite_subcover}より
		或る$a_1,\cdots,a_n \in A$で
		\begin{align}
			A \subset \bigcup_{i=1}^n M(a_i)_{a_i}
		\end{align}
		となる.近縁系は有限交叉で閉じるから
		\begin{align}
			U \coloneqq \bigcap_{i=1}^n M(a_i)
		\end{align}
		は$\mathscr{U}$の元であり,このとき任意に$(x,y) \in U \cap A$を取れば,
		或る$i$で$y \in M(a_i)_{a_i}$となり,
		\begin{align}
			(a_i,a_i),(a_i,y) \in M(a_i) \quad \Longrightarrow \quad
			(a_i,y) \in N(a_i)
		\end{align}
		及び$M(a_i)$の対称性から
		\begin{align}
			(a_i,y),(y,x) \in M(a_i) \quad \Longrightarrow \quad
			(a_i,x) \in N(a_i)
		\end{align}
		が満たされ,$f(x),f(y) \in W_{f(a_i)}$が従うから
		$(f(x),f(y)) \in V$が成立し$f$の$A$の上での一様連続性が出る.
		\QED
	\end{prf}
	
	\begin{screen}
		\begin{thm}[擬距離空間の一様構造]
		\label{thm:uniform_structure_on_pseudometric_spaces}
			$(S,d)$を擬距離空間とするとき,
			\begin{align}
				\mathscr{V} \coloneqq
				\Set{V(r)}{r > 0},
				\quad (V(r) \coloneqq \Set{(x,y) \in S \times S}{d(x,y) < r})
			\end{align}
			とおけば$\mathscr{V}$は$S$上の一様構造となり,
			$\mathscr{V}$で導入する一様位相は$d$-位相に一致する.
		\end{thm}
	\end{screen}
	
	\begin{screen}
		\begin{thm}[擬距離空間のCauchy列]
		\label{thm:Cauchy_sequences_on_pseudometric_spaces}
			$(S,d)$を擬距離空間とし,一様構造$\mathscr{V}$を
			定理\ref{thm:uniform_structure_on_pseudometric_spaces}の要領で定めるとき,
			$S$の任意の点列$(x_n)_{n \in \N}$に対し,$(x_n)_{n \in \N}$がCauchy列であることと
			\begin{align}
				\forall \epsilon > 0;\ 
				\exists N \in \N;\quad
				n,m \geq N \Longrightarrow d(x_n,x_m) < \epsilon
				\label{eq:thm_Cauchy_sequences_on_pseudometric_spaces}
			\end{align}
			が成り立つことは同値になる.
		\end{thm}
	\end{screen}
	
	\begin{prf}
		任意の$\epsilon$と$n,m \in \N$で
		\begin{align}
			(x_n,x_m) \in V(\epsilon) \quad \Longleftrightarrow \quad
			d(x_n,x_m) < \epsilon
		\end{align}
		となるから,$(x_n)_{n \in \N}$がCauchy列であるとき,任意の$\epsilon > 0$に対し
		或る$N \in \N$が存在して
		\begin{align}
			n,m \geq N \quad \Longrightarrow \quad
			(x_n,x_m) \in V(\epsilon) \quad \Longrightarrow \quad
			d(x_n,x_m) < \epsilon
		\end{align}
		が成り立つ.逆に$(x_n)_{n \in \N}$に対して
		(\refeq{eq:thm_Cauchy_sequences_on_pseudometric_spaces})が
		満たされているとき,任意の$V(\epsilon) \in \mathscr{V}$に対し
		或る$M \in \N$が存在して
		\begin{align}
			n,m \geq N \quad \Longrightarrow \quad
			d(x_n,x_m) < \epsilon \quad \Longrightarrow \quad
			(x_n,x_m) \in V(\epsilon)
		\end{align}
		となるから$(x_n)_{n \in \N}$はCauchy列である.
		\QED
	\end{prf}
	
	\begin{screen}
		\begin{thm}[点列の擬距離に関する収束]
			点列$(x_n)_{n \in \N}$が$a$に収束する
			ことと$d(x_n,a) \longrightarrow 0$は同値.
		\end{thm}
	\end{screen}
	
	\begin{screen}
		\begin{thm}[可算な基本近縁系が存在するとき,完備$\Longleftrightarrow$任意のCauchy列が収束する]
		\label{thm:complete_iff_every_Cauchy_seq_converges_if_entourage_contains_some_countable_subset}
			$(S,\mathscr{V})$を一様空間とする.
			$\mathscr{V}$に対して可算な基本近縁系$\{V_n\}_{n \in \N}$が存在するとき次が成立する:
			\begin{align}
				\mbox{$S$が完備である} \quad \Longleftrightarrow \quad
				\mbox{$S$の任意のCauchy列が収束する}.
			\end{align}
		\end{thm}
	\end{screen}
	
	\begin{prf}
		$\Longleftarrow$を示す.近縁系は有限交叉で閉じるから
		\begin{align}
			U_n \coloneqq V_1 \cap V_2 \cap \cdots \cap V_n,
			\quad (n = 1,2,\cdots)
		\end{align}
		により単調減少な$\mathscr{V}$の基本近縁系$\{U_n\}_{n \in \N}$が定まる.
		$(x_\lambda)_{\lambda \in \Lambda}$を$S$のCauchy有向点族として
		\begin{align}
			A_\lambda \coloneqq \Set{x_\mu}{\lambda \leq \mu},
			\quad (\forall \lambda \in \Lambda)
		\end{align}
		とおけば,任意の$n \in \N$で或る$\lambda_n \in \Lambda$が存在して
		\begin{align}
			A_{\lambda_n} \times A_{\lambda_n} \subset U_n
		\end{align}
		となる.任意の$V \in \mathscr{V}$に対し$W \circ W \subset V$
		を満たす$W \in \mathscr{V}$を取れば,或る$N \in \N$で$U_N \subset W$となるから
		\begin{align}
			U_N \circ U_N \subset V
		\end{align}
		が成り立つ.また任意の$n,m \geq N$に対し,有向集合の定義より
		$\lambda_n,\lambda_m \leq \mu$を満たす$\mu \in \Lambda$が存在して
		\begin{align}
			(x_{\lambda_n},x_\mu) \in U_n \subset U_N,
			\quad (x_\mu, x_{\lambda_m}) \in U_m \subset U_N
		\end{align}
		となり$(x_{\lambda_n},x_{\lambda_m}) \in V$が従うから,
		$(x_{\lambda_n})_{n \in \N}$はCauchy列であり或る$a \in S$に収束する.このとき
		\begin{align}
			\lim x_\lambda = a
			\label{eq:thm_complete_iff_every_Cauchy_seq_converges_if_entourage_contains_some_countable_subset}
		\end{align}
		が成立する.実際,任意に$a$の近傍$B$を取れば或る$\tilde{V} \in \mathscr{V}$で
		\begin{align}
			\tilde{V}_a \coloneqq \Set{x \in S}{(a,x) \in \tilde{V}} \subset B
		\end{align}
		となり,$\tilde{W} \circ \tilde{W} \subset V$を満たす$\tilde{W} \in \mathscr{V}$に対し
		或る$N_1 \in \N$が存在して
		\begin{align}
			n \geq N_1 \quad \Longrightarrow \quad
			x_{\lambda_n} \in \tilde{W}_a \quad \Longrightarrow \quad
			(a,x_{\lambda_n}) \in \tilde{W}
		\end{align}
		を満たす.また或る$N_2 \geq N_1$で$U_{N_2} \subset \tilde{W}$となるから
		\begin{align}
			A_{\lambda_{N_2}} \times A_{\lambda_{N_2}} \subset U_{N_2} \subset \tilde{W}
		\end{align}
		が従い,このとき$(a,x_{\lambda_{N_2}}) \in \tilde{W}$かつ
		$(x_{\lambda_{N_2}},x) \in \tilde{W},\ (\forall x \in A_{\lambda_{N_2}})$より
		$(a,x) \in \tilde{V},\ (\forall x \in A_{\lambda_{N_2}})$となるから
		\begin{align}
			A_{\lambda_{N_2}} \subset \tilde{V}_a 
		\end{align}
		が得られ(
		\refeq{eq:thm_complete_iff_every_Cauchy_seq_converges_if_entourage_contains_some_countable_subset})
		が出る.任意のCauchy有向点族が収束するから$S$は完備である.
		\QED
	\end{prf}
	
	\begin{screen}
		\begin{thm}[完備かつ全有界$\Longleftrightarrow$コンパクト]
			$(S,\mathscr{V})$を一様空間として$\mathscr{V}$で一様位相を導入するとき,
			\begin{align}
				\mbox{$S$が完備かつ全有界} \quad \Longleftrightarrow \quad
				\mbox{$S$がコンパクト}.
			\end{align}
		\end{thm}
	\end{screen}
	
	\begin{prf}\mbox{}
		\begin{description}
			\item[第一段]
				任意の有向点族が収束する部分有向点族を持てばコンパクトである.
				$(x_\lambda)_{\lambda \in \Lambda}$を$S$の有向点族とする.
				任意の$V \in \mathscr{V}$に対し或る$\{A_i\}_{i=1}^n$が存在して
				\begin{align}
					A_i \times A_i \subset U;\ (\forall i=1,\cdots,n),\quad
					\bigcup_{i=1}^n A_i = S
				\end{align}
				を満たす.この$\{A_i\}_{i=1}^n$が生成する$S$の位相を$\tau_V$とし,
				$\tau_V$を導入した$S$を$S_V$と書けば,$S_V$はコンパクトであるからTyconovの定理より
				\begin{align}
					T \coloneqq \prod_{V \in \mathscr{V}} S_V
				\end{align}
				はコンパクト空間であり,
			\item[第二段]
		\end{description}
	\end{prf}
	\subsection{距離空間}
	\begin{screen}
		\begin{dfn}[(擬)距離関数・距離位相]
			空でない集合$S$において,
			\begin{description}
				\item[(PM1)] $d(x,x) = 0,\quad (\forall x \in S)$
				\item[(PM2)] $d(x,y) = d(y,x),\quad (\forall x,y \in S)$
				\item[(PM3)] $d(x,y) \leq d(x,z) + d(z,y),\quad (\forall x,y,z \in S)$
			\end{description}
			を満たす関数$d:S \times S \longrightarrow [0,\infty)$を
			擬距離(pseudometric)と呼ぶ.これらに加えて
			\begin{description}
				\item[(PM4)] $d(x,y) = 0 \quad \Longleftrightarrow \quad x=y,
				\quad (\forall x,y \in S)$
			\end{description}
			が満たされるとき$d$を距離(metric)と呼び,
			$S$と(擬)距離$d$との対$(S,d)$を(擬)距離空間と呼ぶ.また
			\begin{align}
				&\mbox{$O \subset S$が開集合である}
				\quad \overset{\mathrm{def}}{\Longleftrightarrow} \quad \\
				&\quad\mbox{$O \neq \emptyset$,或は任意の$x \in O$に対し或る$r_x > 0$が存在して
					$\Set{y \in S}{d(x,y) < r_x} \subset O$となる}
			\end{align}
			で定める開集合系を(擬)距離位相と呼ぶ.
		\end{dfn}
	\end{screen}
	
	\begin{screen}
		\begin{thm}[擬距離位相は第一可算]
			$(S,d)$を擬距離空間として擬距離位相を導入すれば,
			任意の$x \in S$に対して
			\begin{align}
				\left\{\Set{y \in S}{d(x,y) < \frac{1}{n}}\right\}_{n=1}^\infty
			\end{align}
			は$x$の基本近傍系となる.すなわち擬距離位相は第一可算空間を定める.
		\end{thm}
	\end{screen}
	
	\begin{prf}
		$U$を$x$を近傍とすれば或る$r > 0$で
		$\Set{y \in S}{d(x,y) < r} \subset U$となる.
		このとき$1/n < r$なら
		\begin{align}
			\Set{y \in S}{d(x,y) < \frac{1}{n}}
			\subset \Set{y \in S}{d(x,y) < r} \subset U
		\end{align}
		が成り立つ.
		\QED
	\end{prf}
	
	\begin{screen}
		\begin{thm}
		\end{thm}
	\end{screen}
	
	\begin{screen}
		\begin{thm}[距離でないと分離性が成り立たない]
		\label{thm:pseudometric_is_metric_iff_T_0}
			$(S,d)$を擬距離空間として擬距離位相を入れるとき,
			\begin{align}
				\mbox{$d$が距離である} \quad \Longleftrightarrow \quad
				\mbox{$S$が$T_0$である} \quad \Longleftrightarrow \quad
				\mbox{$S$が$T_2$である}.
				\label{eq:thm_pseudometric_is_metric_iff_T_0}
			\end{align}
		\end{thm}
	\end{screen}
	
	\begin{prf}\mbox{}
		\begin{description}
			\item[第一段] 	$d$が距離なら$S$はHausdorffである.
				実際,相異なる二点$x,y \in S$に対し
				\begin{align}
					B_\epsilon(x) \coloneqq \Set{s \in S}{d(s,x) < \frac{\epsilon}{2}},
					\quad B_\epsilon(y) \coloneqq \Set{s \in S}{d(s,y) < \frac{\epsilon}{2}},
					\quad (\epsilon \coloneqq d(x,y))
				\end{align}
				で交わらない開球を定めれば,$x$と$y$は
				$B_\epsilon(x)$と$B_\epsilon(y)$で分離される.
			
			\item[第二段]
				$S$が$T_0$であるとき,相異なる二点$x,y$に対し
				$x \notin \overline{\{y\}}$或は$y \notin \overline{\{x\}}$となる.
				$x \notin \overline{\{y\}}$とすれば
				$\overline{\{y\}} \subset S \backslash B_r(x)$
				を満たす$r > 0$が存在し,$d(x,y) \geq r > 0$が成り立つから
				$d$は距離となる.
				$T_2 \Longrightarrow T_0$より
				(\refeq{eq:thm_pseudometric_is_metric_iff_T_0})を得る.
				\QED
		\end{description}
	\end{prf}
	
	\begin{screen}
		\begin{thm}[擬距離関数の連続性]\label{thm:continuity_of_pseudometrics}
			$(S,d)$を擬距離空間として擬距離位相を導入するとき,以下が成り立つ:
			\begin{description}
				\item[(1)] $S \times S \ni (x,y) \longmapsto d(x,y)$は直積位相に関し連続である.
				
				\item[(2)] 任意の空でない部分集合$A$に対し
					$S \ni x \longmapsto d(x,A)$は連続である.特に$A$が閉なら
					\begin{align}
						x \in A \quad \Longleftrightarrow \quad
						d(x,A) = 0. 
					\end{align} 
			\end{description}
		\end{thm}
	\end{screen}
	
	\begin{screen}
		\begin{thm}[擬距離空間は完全正規]
			任意の擬距離位相空間は完全正規である.特に
			\begin{align}
				\mbox{擬距離が距離である} \quad \Longleftrightarrow \quad
				\mbox{擬距離位相が$T_6$である}.
			\end{align}
		\end{thm}
	\end{screen}
	
	\begin{prf}
		$(S,d)$を擬距離空間として擬距離位相を入れるとき,$A,B$を交わらない$S$の閉集合として
		\begin{align}
			f(x) \coloneqq \frac{d(x,A)}{d(x,A) + d(x,B)},
			\quad (\forall x \in S)
		\end{align}
		により$f:S \longrightarrow \R$を定めれば,
		定理\ref{thm:continuity_of_pseudometrics}より$f$は連続であり
		\begin{align}
			A = f^{-1}(\{0\}),\quad B = f^{-1}(\{1\})
		\end{align}
		が満たされるから$S$は完全正規である.また$d$が距離であるとき,
		定理\ref{thm:pseudometric_is_metric_iff_T_0}より
		$S$はHausdorffかつ完全正規となるから$T_6$となる.
		逆に$S$が$T_6$なら$T_0$であるから$d$は距離となる.
		\QED
	\end{prf}
	
	\begin{screen}
		\begin{thm}[距離空間の部分空間の距離]
			$(S,d)$を距離空間,$M$を$S$の空でない部分集合とし,
			$S$に距離位相を入れる.このとき$M$の相対位相$\mathscr{O}_M$は
			\begin{align}
				d_M(x,y) \coloneqq d(x,y),
				\quad (\forall x,y \in M)
			\end{align}
			で定める相対距離により導入する距離位相$\mathscr{O}_{d_M}$と一致する.
		\end{thm}
	\end{screen}
	
	\begin{prf} 任意の$x \in M$と$r > 0$に対し
		\begin{align}
			\Set{y \in M}{d_M(x,y) < r}
			= M \cap \Set{y \in S}{d(x,y) < r}
		\end{align}
		が成り立つから,相対開集合は$d_M$-開球の合併で表され,
		逆に$d_M$-開集合は$M$と$d$-開集合の交叉で表せる.
		\QED
	\end{prf}
	
	\begin{screen}
		\begin{thm}[距離空間の高々可算直積の距離]
			$((S_n,d_n))_{n=1}^N$を距離空間の族として距離位相を導入し,
			$S$をその直積位相空間とする.また$x \in S$に対し
			$x(n)$を$x_n$と書く.このとき$N < \infty$なら
			\begin{align}
				d(x,y)
				\coloneqq \left\{\sum_{n=1}^N d_n(x_n,y_n)^2\right\}^{1/2},
				\quad (\forall x,y \in S)
			\end{align}
			により,$N = \infty$なら
			\begin{align}
				d(x,y) \coloneqq
				\sum_{n=1}^\infty 2^{-n}\left(d_n(x_n,y_n) \wedge 1\right),
				\quad (\forall x,y \in S)
			\end{align}
			により,$S$は距離化可能である.特に$(S_n,d_n)$が全て完備(resp. 可分)なら
			$(S,d)$も完備(resp. 可分)となる.
		\end{thm}
	\end{screen}
	
	\begin{screen}
		\begin{thm}[距離空間において可分$\Longrightarrow$第二可算]
		\end{thm}
	\end{screen}
	
	\begin{screen}
		\begin{thm}[擬距離の距離化]
			$(S,d)$を擬距離空間とするとき,
			$x \sim y \overset{\mathrm{def}}{\Longleftrightarrow} d(x,y) = 0$
			で$S$に同値関係が定まる.また商写像を$\pi:S \longrightarrow S/\sim$と書けば
			\begin{align}
				\rho(\pi(x),\pi(y)) \coloneqq d(x,y),
				\quad (\forall \pi(x),\pi(y) \in S/\sim)
			\end{align}
			により$S/\sim$に距離$\rho$が定まり,$\rho$-位相は
			$S$の$d$-位相の商位相に一致する.
		\end{thm}
	\end{screen}
	
\subsection{範疇定理}
	\begin{screen}
		\begin{thm}[Cantorの共通部分定理]\label{thm:Cantor_intersection_theorem}
			$S$をHausdorff空間とし,
			$(K_n)_{n=1}^\infty$をコンパクト部分集合の列とする.
			このとき,任意の$n \geq 1$に対して$\bigcap_{i=1}^n K_i \neq \emptyset$なら
			$\bigcap_{i=1}^\infty K_i \neq \emptyset$が成り立つ.
		\end{thm}
	\end{screen}
	
	\begin{prf}
		$\bigcap_{i=1}^\infty K_i = \emptyset$と仮定すれば,
		$K_1 \subset \bigcup_{n=1}^\infty K_n^c = S$と$K_1$のコンパクト性より
		\begin{align}
			K_1 \subset \bigcup_{n=1}^N K_n^c = \Biggl( \bigcap_{n=1}^N K_n \Biggr)^c
		\end{align}
		を満たす$N \geq 1$が存在し,$\bigcap_{n=1}^N K_n \subset K_1$より$\bigcap_{n=1}^N K_n = \emptyset$が従う.
		\QED
	\end{prf}
	
	\begin{screen}
		\begin{dfn}[疎集合・第一類集合・第二類集合]
			位相空間$S$の部分集合$A$が疎である(nowhere dense)とは
			$A$の閉包の内核が$\overline{A}^{\mathrm{o}} = \emptyset$を満たすことをいう.
			$S$が可算個の疎集合の合併で表せるとき$S$を第一類集合(the set of the first category)と呼び,
			そうでない場合はこれを第二類集合と呼ぶ.
		\end{dfn}
	\end{screen}
	
	\begin{screen}
		\begin{thm}[Baireの範疇定理]\label{thm:Baire_category_theorem}
			空でない完備距離空間と局所コンパクトHausdorff空間は第二類集合である.
		\end{thm}
	\end{screen}
	
	\begin{prf} $S \neq \emptyset$を完備距離空間,或は局所コンパクトHausdorff空間とする.\mbox{}
		\begin{description}
			\item[第一段]
				$(V_n)_{n=1}^\infty$を$S$で稠密な開集合系とするとき
				\begin{align}
					\overline{\bigcap_{n=1}^\infty V_n} = S,
					\label{eq:thm_Baire_category_theorem_1}
				\end{align}
				となることを示す.実際(\refeq{eq:thm_Baire_category_theorem_1})が満たされていれば,
				任意の疎集合系$(E_n)_{n=1}^\infty$に対して
				\begin{align}
					V_n \coloneqq \overline{E_n}^c,
					\quad n=1,2,\cdots
				\end{align}
				で開集合系$(V_n)$を定めると定理\ref{thm:topology_note_closure_interior}より
				\begin{align}
					\overline{V_n} = \overline{E_n}^{ca} = \overline{E_n}^{ic} = \emptyset^c = S
				\end{align}
				となるから,$\bigcap_{n=1}^\infty V_n \neq \emptyset$が従い
				$S \neq \bigcup_{n=1}^\infty \overline{E_n} \supset \bigcup_{n=1}^\infty E_n$
				が成り立つ.従って$S$は第二類である.
				
			\item[第二段]
				任意の空でない開集合$B_0$に対し$B_0 \cap \left( \bigcap_{n=1}^\infty V_n \right) \neq \emptyset$
				となることを示せば(\refeq{eq:thm_Baire_category_theorem_1})が従う.
				$V_1$は稠密であるから$B_0 \cap V_1 \neq \emptyset$となり,
				点$x_1 \in B_0 \cap V_1$を取れば,
				$S$が距離空間なら或る半径$<1$の開球$B_1$が存在して
				\begin{align}
					x_1 \in B_1 \subset \overline{B_1} \subset B_0 \cap V_1
					\label{eq:thm_Baire_category_theorem_2}
				\end{align}
				を満たす.$S$が局所コンパクトHausdorffの場合も,
				定理\ref{thm:each_point_in_regular_space_has_closesd_local_base}と
				定理\ref{thm:T_2_equals_to_T_3_in_locally_compact_spaces}より
				(\refeq{eq:thm_Baire_category_theorem_2})を満たす
				相対コンパクトな開集合$B_1$が取れる.
				同様に半径$<1/n$の開球,或は相対コンパクトな開集合$B_n$と$x_n \in S$で
				\begin{align}
					x_n \in B_n \subset \overline{B_n} \subset B_{n-1} \cap V_n
				\end{align}
				を満たすものが存在する.このとき$S$が完備距離空間なら$(x_n)_{n=1}^\infty$は
				Cauchy列をなし,その極限点$x_\infty$は
				\begin{align}
					x_\infty \in \bigcap_{n=1}^\infty \overline{B_n}
				\end{align}
				を満たす.$S$が局所コンパクトHausdorff空間なら定理\ref{thm:Cantor_intersection_theorem}より
				\begin{align}
					\bigcap_{n=1}^\infty \overline{B_n} \neq \emptyset
				\end{align}
				となるから,いずれの場合も
				\begin{align}
					\emptyset \neq \bigcap_{n=1}^\infty \overline{B_n} 
					\subset B_0 \cap \Biggl( \bigcap_{n=1}^\infty V_n \Biggr)
				\end{align}
				が従い定理の主張が得られる.
				\QED
		\end{description}
	\end{prf}
	
	\begin{screen}
		\begin{lem}[同相写像に関して閉包(内部)の像は像の閉包(内部)に一致する]
		\label{lem:image_of_closure_is_closure_of_image}
			$A$を位相空間$S$の部分集合,$h:S \longrightarrow S$を同相写像とするとき
			次が成り立つ:
			\begin{description}
				\item[(1)] $h(A^a) = h(A)^a$.
				\item[(2)] $h(A^i) = h(A)^i$.
			\end{description}
		\end{lem}
	\end{screen}
	
	\begin{prf}\mbox{}
		\begin{description}
			\item[(1)]
				$h(A) \subset h(A^a)$かつ$h(A^a)$は閉であるから$h(A)^a \subset h(A^a)$が従う.一方で
				任意の$x \in h(A^a)$に対し$x = h(y)$を満たす
				$y \in A^a$と$x$の任意の近傍$V$を取れば,
				$h^{-1}(V) \cap A \neq \emptyset$より
				$V \cap h(A) \neq \emptyset$が成り立ち
				$x \in h(A)^a$となる.
				
			\item[(2)]
				$h(A^i) \subset h(A)$かつ$h(A^i)$は開であるから
				$h(A^i) \subset h(A)^i$が従う.一方で
				任意の開集合$O \subset h(A)$に対し
				$h^{-1}(O) \subset A$より
				$h^{-1}(O) \subset A^i$となり,
				$O \subset h(A^i)$が成り立つから
				$h(A)^i \subset h(A^i)$が得られる.
				\QED
		\end{description}
	\end{prf}
	
	\begin{screen}
		\begin{thm}[第一類集合の性質]
			$S$を位相空間とする.
			\begin{description}
				\item[(a)] $A \subset B \subset S$に対し$B$が第一類なら$A$も第一類である.
				\item[(b)] 第一類集合の可算和も第一類である.
				\item[(c)] 内核が空である閉集合は第一類である.
				\item[(d)] $S$から$S$への位相同型$h$と$E \subset S$に対し次が成り立つ:
					\begin{align}
						\mbox{$E$が第一類} \quad \Longleftrightarrow \quad
						\mbox{$h(E)$が第一類}.
					\end{align}
			\end{description}
		\end{thm}
	\end{screen}
	
	\begin{prf}\mbox{}
		\begin{description}
			\item[(a)] $B = \bigcup_{n=1}^\infty E_n$
				を満たす疎集合系$(E_n)_{n=1}^\infty$に対し
				$A \cap E_n$は疎であり$A = \bigcup_{n=1}^\infty (A \cap E_n)$となる.
			\item[(b)] $A_n \subset S,\ (n=1,2,\cdots)$が第一類集合とし
				$(E_{n,i})_{i=1}^\infty$を$A_n = \bigcup_{i=1}^\infty E_{n,i}$
				を満たす疎集合系とすれば
				\begin{align}
					\bigcup_{n=1}^\infty A_n
					= \bigcup_{n,i=1}^\infty E_{n,i}
				\end{align}
				が成り立つ.
				
			\item[(c)] 内核が空である閉集合はそれ自身が疎であり,自身の可算和に一致する.
			\item[(d)] $E$が第一類のとき,$E = \bigcup_{i=1}^\infty E_i$を満たす
				疎集合系$(E_i)_{i=1}^\infty$に対し
				定理\ref{thm:topology_note_closure_interior}と
				補題\ref{lem:image_of_closure_is_closure_of_image}より
				\begin{align}
					\emptyset = h(E_i^{ai})
					= h(E_i^a)^i
					= h(E_i)^{ai}
				\end{align}
				が成り立つから$h(E_i)$は疎であり,
				\begin{align}
					h(E) = \bigcup_{i=1}^\infty h(E_i)
				\end{align}
				となるから$h(E)$も第一類である.$h(E)$が第一類なら$E = h^{-1}(h(E))$も第一類である.
				\QED
		\end{description}
	\end{prf}
	\section{連結性}
	
	\begin{screen}
		\begin{thm}
			$\R$の任意の区間は連結である.
		\end{thm}
	\end{screen}
	
	\begin{screen}
		\begin{thm}
			連結集合の連続写像による像は連結である.
		\end{thm}
	\end{screen}
	
	\begin{screen}
		\begin{thm}[弧状連結なら連結]\label{thm:connected_path_connected}
			弧状連結位相空間は連結空間である.
		\end{thm}
	\end{screen}
	
	\begin{prf}
		$S$を連結でない位相空間とする.このとき
		或る空でない開集合$U_1,U_2$が存在して
		\begin{align}
			U_1 \cup U_2 = S,
			\quad U_1 \cap U_2 = \emptyset
		\end{align}
		を満たす.$x \in U_1,\ y \in U_2$に対し
		$f(0) = x,\ f(1) = y$を満たす連続写像
		$f:[0,1] \longrightarrow S$が存在する場合,
		\begin{align}
			f([0,1]) = \left( U_1 \cap f([0,1]) \right) \cup \left( U_2 \cap f([0,1]) \right),
			\quad \left( U_1 \cap f([0,1]) \right) \cap \left( U_2 \cap f([0,1]) \right) = \emptyset
		\end{align}
		となり$f([0,1])$の連結性に矛盾する.
		従って$x,y$を結ぶ道は存在しないから$S$は弧状連結ではない.
		\QED
	\end{prf}
\subsection{位相線型空間 (Rudin note)}
	\begin{screen}
		\begin{thm}[多変数連続写像は一変数写像として連続]
		\label{thm:multivariable_continuous_mapping_is_one_variable_continuous}
			$\Lambda$を任意濃度の空でない集合とし,
			$\left( (S_\lambda,\tau_\lambda) \right)_{\lambda \in \Lambda}$を位相空間の族とする.
			
		\end{thm}
	\end{screen}
	
	以降扱う線型空間はすべて体$\Phi (=\C,\R)$をスカラーとして考え,位相はEuclid距離による距離位相を導入する.
	
	\begin{screen}
		\begin{dfn}[位相線型空間]\label{def:topological_vector_space}
			$\Phi$上の線型空間$X$で定められる位相$\tau$が
			\begin{description}
				\item[(tvs1)] $X \times X \ni (x,y) \longmapsto x+y \in X$
					及び$\Phi \times X \ni (\alpha,x) \longmapsto \alpha x \in X$
					が$\tau$及びその直積位相に関し連続である.
				\item[(tvs2)]
					$(X,\tau)$は$T_1$位相空間である.
			\end{description}
			を満たすとき線型位相(vector topology)と呼び,
			$(X,\tau)$を位相線型空間(topological vector space)と呼ぶ.
		\end{dfn}
	\end{screen}
	
	\begin{screen}
		\begin{thm}[位相線型空間は$T_3$]
		\end{thm}
	\end{screen}
	
	\begin{screen}
		\begin{dfn}[平行移動不変距離]
			線型空間$X$上に定まる距離$d$が
			\begin{align}
				d(x+z, y+z) = d(x,y),\quad (\forall x,y,z \in X)
			\end{align}
			を満たすとき,$d$を平行移動不変距離(invariant metric)と呼ぶ.
			平行移動不変距離$d$がさらに
			\begin{align}
				d(\alpha x, \alpha y) = |\alpha| d(x,y),
				\quad (\forall \alpha \in \Phi,\ x,y \in X)
			\end{align}
			を満たすとき$d$は斉次的である(homogeneous)という.
			例えば$X$にノルム$\Norm{\cdot}{}$が定まっている場合,
			$d(x,y) \coloneqq \Norm{x-y}{}$により定まる距離$d$は斉次的かつ平行移動不変である.
		\end{dfn}
	\end{screen}
	
	\begin{screen}
		\begin{thm}[斉次的な平行移動不変距離による距離位相は線型位相]
			$X$を線型空間とする.$X$において斉次的な平行移動不変距離$d$が存在するとき,
			$d$で導入する距離位相は線型位相となる.
		\end{thm}
	\end{screen}
	
	\begin{prf}
		距離位相は$T_4$位相空間を定めるから$X$は定義\ref{def:topological_vector_space}の(tvs2)を満たす.また
		\begin{align}
			d(x+y,x'+y') \leq d(x+y,x'+y) + d(x'+y,x'+y') = d(x,x') + d(y,y')
		\end{align}
		より加法の連続性が得られ,
		\begin{align}
			d(\alpha x, \alpha'x') &\leq d(\alpha x, \alpha'x) + d(\alpha'x,\alpha'x') \\
			&= d((\alpha - \alpha') x, 0) + |\alpha'|d(x,x')
			= |\alpha-\alpha'|d(x,0) + |\alpha'|d(x,x')
		\end{align}
		よりスカラ倍の連続性も出る.
		\QED
	\end{prf}
	
	\begin{screen}
		\begin{thm}[平行移動・スカラ倍の連続性]\label{thm:continuity_of_translations_multiples}
			$(X,\tau)$を位相線型空間とするとき,任意の$a \in X$に対し
			\begin{align}
				X \ni x \longmapsto a + x \in X,
				\quad \Phi \ni \alpha \longmapsto \alpha a \in X
			\end{align}
			はいずれも連続である.同様に任意の$\beta \in \Phi$に対し
			$X \ni x \longmapsto \beta x$もまた連続である.
		\end{thm}
	\end{screen}
	
	\begin{prf}
		定理\ref{thm:multivariable_continuous_mapping_is_one_variable_continuous}より従う.
		\QED
	\end{prf}
	
	\begin{screen}
		\begin{thm}[位相線型空間の連結性]\label{thm:topological_vector_spaces_connected}
			位相線型空間は連結である.
		\end{thm}
	\end{screen}
	
	\begin{prf}
		零元のみの空間は密着空間であるから連結である.
		$X \neq \{0\}$を位相線型空間とするとき,任意に$a,b \in X$を取り
		\begin{align}
			f:[0,1] \ni t \longmapsto a + t(b - a) \in X
		\end{align}
		と定めれば$f$は$[0,1]$から$X$への連続写像である.実際,
		定理\ref{thm:continuity_of_translations_multiples}より
		$\Phi \ni t \longmapsto t(b-a)$が連続であるから
		\begin{align}
			g:[0,1] \ni t \longmapsto t(b-a)
		\end{align}
		は$[0,1]$の相対位相に関して連続であり,かつ$h:X \ni x \longmapsto a + x$もまた連続であるから
		$f = h \circ g$の連続性が従う.
		よって$X$は弧状連結であるから定理\ref{thm:connected_path_connected}より連結である.
		\QED
	\end{prf}
	
	\begin{screen}
		\begin{dfn}[位相線形空間の有界集合]
			$X$を位相線型空間,$E$を$X$の部分集合とする.0の任意の近傍$V$に対し
			或る$s = s(V) > 0$が存在して
			\begin{align}
				E \subset t V, \quad (\forall t > s)
			\end{align}
			となるとき,$E$は有界であるという.
		\end{dfn}
	\end{screen}
	
	\begin{screen}
		\begin{thm}
		\end{thm}
	\end{screen}
	
	位相線形空間$(X,\tau)$に対し,その部分集合$Y$上の相対位相を$\tau_Y$と書き,
	また$X$が或る距離$d$で距離付け可能なとき,
	$d$により導入する位相を$\tau_d$と書く.位相$\tau$に関する開集合,閉集合,近傍,
	Cauchy列は$\tau$-開集合(resp. 閉集合,近傍,Cauchy列)と書く.
	
	\begin{screen}
		\begin{dfn}[局所基・局所凸・局所コンパクト・局所有界]
			$(X,\tau)$を位相線型空間とする.
			\begin{description}
				\item[(1)] $0 \in X$の基本近傍系を$X$の局所基(local base)と呼ぶ.
				\item[(2)] すべての元が凸集合であるような局所基が取れるとき,$X$は局所凸(locally convex)であるという.
				\item[(3)] $0 \in X$がコンパクトな近傍を持つとき,$X$は局所コンパクト(locally compact)であるという.
				\item[(4)] $0 \in X$が有界な近傍を持つとき,$X$は局所有界(locally bounded)であるという.
			\end{description}
		\end{dfn}
	\end{screen}
	
	\begin{screen}
		\begin{thm}[局所基は平行移動により任意の点の近傍系となる]
			$X$を位相線型空間とするとき以下が成り立つ:
			\begin{description}
				\item[(1)] 任意の$x \in X$に対し$\Set{x + V}{V \in \mathscr{B}}$は
					$x$の近傍系となる.
				\item[(2)] $x$の近傍系$\mathbb{V}(x)$に対し$\mathscr{B} = \Set{-x + V}{V \in \mathbb{V}(x)}$
					となる.
				\item[(3)] $X$が局所コンパクト空間であるとき,任意の点はコンパクトな近傍を持つ. 
 			\end{description}
		\end{thm}
	\end{screen}
	
	\begin{screen}
		\begin{dfn}[$F$-空間・Frechet空間・ノルム空間]
			$(X,\tau)$を位相線型空間とする.
			$d$により$X$が距離化可能でかつ完備距離空間となるとき,
			$X$を$F$-空間と呼ぶ.局所凸な$F$-空間をFrechet空間と呼び
		\end{dfn}
	\end{screen}
	
	\begin{screen}
		\begin{thm}[部分空間が$F$-空間なら閉]
			$(X,\tau)$を位相線形空間,$Y \subset X$を部分空間とする.
			このとき$Y$が$F$-空間なら$Y$は$\tau$-閉である.
		\end{thm}
	\end{screen}
	
	\begin{prf}
		$Y$に対し或る平行移動不変な距離$d$が存在して$\tau_Y = \tau_d$を満たす.
		このとき
		\begin{align}
			B_{1/n} \coloneqq \Set{y \in Y}{d(y,0) < \frac{1}{n}},
			\quad n=1,2,\cdots
		\end{align}
		で$\tau_Y$-開集合を定めれば,$B_{1/n}$は$0$を含むから
		或る0の$\tau$-近傍$U_n$が存在して
		\begin{align}
			B_{1/n} = Y \cap U_n, \quad n=1,2,\cdots
		\end{align}
		を満たす.
	\end{prf}
	
	\begin{screen}
		\begin{dfn}[集合の線型演算]
			$X$を体$\Phi$上の位相線型空間,$A,B$を$X$の部分集合,$\alpha,\beta \in \Phi$とする.
			このとき
			\begin{align}
				\alpha A + \beta B \coloneqq \Set{\alpha a + \beta b}{a \in A,\ b \in B}
			\end{align}
			と書く.
		\end{dfn}
	\end{screen}
	
	\begin{screen}
		\begin{thm}
			$X$を位相線型空間,$A,B$を部分集合とする.
			\begin{description}
				\item[(1)] $\alpha \overline{A} = \overline{\alpha A}$
				\item[(2)] $\alpha (A^{\mathrm{o}}) = (\alpha A)^{\mathrm{o}}$
			\end{description}
		\end{thm}
	\end{screen}
	
	\begin{prf}\mbox{}
		\begin{description}
			\item[(1)] $\alpha = 0$或は$A = \emptyset$の場合は両辺が
				$\{0\}$或は$\emptyset$となって等号が成立する.
				$\alpha \neq 0$かつ$A \neq \emptyset$の場合,
				\begin{align}
					x \in \alpha \overline{A}
					\quad &\Longleftrightarrow \quad
					\alpha^{-1}x \in \overline{A} \\
					\quad &\Longleftrightarrow \quad
					\left(\alpha^{-1}x + V\right) \cap A \neq \emptyset, \quad 
						(\mbox{$\forall V$: neighborhood of 0}) \\
					\quad &\Longleftrightarrow \quad
					\left(x + V\right) \cap \alpha A \neq \emptyset, \quad 
						(\mbox{$\forall V$: neighborhood of 0}) \\
					\quad &\Longleftrightarrow \quad
					x \in \overline{\alpha A}
				\end{align}
				が成り立つ.
				
			\item[(2)] $\alpha = 0$或は$A = \emptyset$の場合は両辺が
				$\{0\}$或は$\emptyset$となって等号が成立する.
				$\alpha \neq 0$かつ$A \neq \emptyset$の場合,
				\begin{align}
					x \in \alpha (A^{\mathrm{o}})
					\quad &\Longleftrightarrow \quad
					\alpha^{-1}x \in A^{\mathrm{o}} \\
					\quad &\Longleftrightarrow \quad
					\mbox{$\exists V$: neighborhood of 0},\quad \alpha^{-1}x + V \subset A \\
					\quad &\Longleftrightarrow \quad
					\mbox{$\exists V$: neighborhood of 0},\quad x + V \subset \alpha A \\
					\quad &\Longleftrightarrow \quad
					x \in (\alpha A)^{\mathrm{o}}
				\end{align}
				が成り立つ.
				
		\end{description}
	\end{prf}
	
	\begin{screen}
		\begin{dfn}[位相線型空間における同程度連続性]
			$X,Y$を位相線形空間,$\mathscr{F}$を$X$から$Y$への連続線型写像の族とする.
			このとき,$\mathscr{F}$が同程度連続であるとは,$0 \in Y$の任意の近傍$V$に対し
			\begin{align}
				f(U) \subset V,\quad (\forall f \in \mathscr{F})
			\end{align}
			を満たす$0 \in X$の近傍$U$が存在することである.
		\end{dfn}
	\end{screen}
	
	\begin{screen}
		\begin{thm}[同程度連続な写像族の有界性]
			$X,Y$を位相線形空間,$\mathscr{F}$を$X$から$Y$への連続線型写像の族とする.
			$\mathscr{F}$が同程度連続であるとき,
		\end{thm}
	\end{screen}
	
	\begin{screen}
		\begin{thm}[Banach-Steinhaus]
			
		\end{thm}
	\end{screen}
	
	\begin{screen}
		\begin{thm}[開写像原理]
			$X$
		\end{thm}
	\end{screen}
\section{測度}
	\subsection{Lebesgue拡大}
		\begin{screen}
			\begin{dfn}[Lebesgue拡大]
				$(X,\mathcal{B},\mu)$を測度空間とするとき,
				\begin{align}
					\overline{\mathcal{B}} &\coloneqq
					\Set{B \subset X}{\exists A_1,A_2 \in \mathcal{B},\ \mbox{s.t.}\quad A_1 \subset B \subset A_2,\ \mu(A_2 - A_1)=0 }, \\
					\overline{\mu}(B) &\coloneqq \mu(A_1) \quad (\forall B \in \overline{\mathcal{B}},\ \mbox{$A_1$ as in above})
				\end{align}
				により得られる完備測度空間$(X,\overline{\mathcal{B}},\overline{\mu})$を
				$(X,\mathcal{B},\mu)$のLebesgue拡大と呼ぶ.
			\end{dfn}
		\end{screen}
		$\overline{\mu}$はwell-definedである.実際,$B \subset X$に対し
		$A_1,A_2,B_1,B_2 \in \mathcal{B}$が
		\begin{align}
			&A_1 \subset B \subset A_2, \quad \mu(A_2 - A_1) = 0, \\
			&B_1 \subset B \subset B_2, \quad \mu(B_2 - B_1) = 0,
		\end{align}
		を満たすとき,$A_1 \cup B_1 \subset B \subset A_2 \cap B_2$となるが,
		\begin{align}
			(A_2 \cap B_2) \cap (A_1 \cup B_1)^c
			\subset A_2 \backslash A_1
		\end{align}
		より$\mu(A_1 \cup B_1) = \mu(A_2 \cap B_2)$が従い
		\begin{align}
			\mu(A_2) &= \mu(A_1) \leq \mu(A_1 \cup B_1) = \mu(A_2 \cap B_2) \leq \mu(B_2), \\
			\mu(B_2) &= \mu(B_1) \leq \mu(A_1 \cup B_1) = \mu(A_2 \cap B_2) \leq \mu(A_2)
		\end{align}
		が成り立つから$\mu(A_2) = \mu(B_2)$が出る.
		また,任意の$B \subset X$について
		\begin{align}
			\overline{\mathcal{B}}
			= \Set{B \subset X}{\exists A,N \in \mathcal{B},\ \mbox{s.t.}\quad \mu(N)=0,
			\ B \cap A^c, A \cap B^c \subset N}
			\label{eq:appendix_Lebesgue_expansion_note_1}
		\end{align}
		が成立する.実際,$B \in \overline{\mathcal{B}}$なら
		$A_1 \subset B \subset A_2$かつ$\mu(A_2 - A_1) = 0$を満たす$A_1,A_2 \in \mathcal{B}$が存在するから
		\begin{align}
			A = A_2, \quad N = A_2 - A_1
		\end{align}
		として$(\subset)$を得る.逆に右辺を満たす$A,N$が存在するとき,
		\begin{align}
			A \cap N^c &\subset A \cap B \subset B 
			\subset A \cup (A^c \cap B)
			\subset A \cup N
		\end{align}
		より$A_1 = A\cap N^c,\ A_2 = A \cup N$として$(\supset)$を得る.
	
		\begin{screen}
			\begin{thm}[完備化前後の可測関数の関係]
				$(X,\mathcal{B},\mu)$を測度空間,そのLebesgue拡大を
				$(X,\overline{\mathcal{B}},\overline{\mu})$と書き,
				$f:X \longrightarrow [-\infty,\infty]$とする.
				このとき次は同値である:
				\begin{description}
					\item[(a)] 或る$\mathcal{B}/\borel{[-\infty,\infty]}$-可測関数$g$が存在して
						$f = g\quad \mbox{$\overline{\mu}$-a.e.}$を満たす.
					\item[(b)] 或る$\mathcal{B}/\borel{[-\infty,\infty]}$-可測関数$g_1,g_2$が存在して
						$g_1(x) \leq f(x) \leq g_2(x)\ (\forall x \in X)$かつ$g_1 = g_2\quad \mbox{$\overline{\mu}$-a.e.}$を満たす.
					\item[(c)] $f$は$\overline{\mathcal{B}}/\borel{[-\infty,\infty]}$-可測である.
				\end{description}
			\end{thm}
		\end{screen}
		
		\begin{prf}\mbox{}
			\begin{description}
				\item[第一段]
					$B \subset X$に対して$f = \defunc_B$と表せるとき,
					\begin{align}
						\mbox{$f$が$\overline{\mathcal{B}}/\borel{[-\infty,\infty]}$-可測}
						&\Leftrightarrow B = f^{-1}(\{1\}) \in \overline{\mathcal{B}} \\
						&\Leftrightarrow \exists A_1,A_2 \in \mathcal{B},\ \mbox{s.t.}\quad A_1 \subset B \subset A_2,\ \mu(A_2 - A_1)=0 \\
						&\Rightarrow \defunc_{A_1} \leq f \leq \defunc_{A_2},\quad \defunc_{A_1}=\defunc_{A_2}\ \mbox{$\overline{\mu}$-a.e.} \\
						&\Rightarrow (b) \\
						&\Rightarrow (c)
					\end{align}
					となる.また$(c)$が満たされているとき,
					\begin{align}
						f(x) = g(x) \quad (\forall x \in X \backslash N),
						\label{eq:appendix_Lebesgue_expansion_note_2}
					\end{align}
					を満たす$\mu$-零集合$N \in \mathcal{B}$が存在して
					\begin{align}
						f^{-1}(E) \cap \left( g^{-1}(E) \right)^c \subset N,
						\quad g^{-1}(E) \cap \left( f^{-1}(E) \right)^c \subset N,
						\quad (\forall E \in \borel{[-\infty,\infty]})
					\end{align}
					が成り立つから,(\refeq{eq:appendix_Lebesgue_expansion_note_1})より
					$f^{-1}(E) \in \overline{\mathcal{B}}$が従い$(c) \Rightarrow (a)$が出る.
				
				\item[第二段]
					$0 \leq f \leq \infty$かつ単関数$f = \sum_{n=0}^N \alpha_n \defunc_{B_n}\ 
					(\alpha_0 = 0,\ i \neq j \Rightarrow \alpha_i \neq \alpha_j)$
					として表されるとき,
					\begin{align}
						\mbox{$f$が$\overline{\mathcal{B}}/\borel{[-\infty,\infty]}$-可測}
						&\Leftrightarrow B_n = f^{-1}(\{\alpha_n\}) \in \overline{\mathcal{B}},\ (n=0,1,\cdots,N) \\
						&\Leftrightarrow \exists g_{1,n},g_{2,n}:\ \mbox{$\mathcal{B}/\borel{[-\infty,\infty]}$-measurable }, \\
							&\qquad \mbox{s.t.}\quad g_{1,n} \leq \defunc_{B_n} \leq g_{2,n},
							\ \mu(g_{1,n} \neq g_{1,n})=0,\ (n=0,1,\cdots,N) \\
						&\Rightarrow g_1 \coloneqq \sum_{n=0}^N \alpha_n g_{1,n},
							\quad g_2 \coloneqq \sum_{n=0}^N \alpha_n g_{2,n}, \\
							&\qquad g_1 \leq f \leq g_2,\quad \mu(g_1 \neq g_2) \leq \mu\Biggl( \bigcup_{n=0}^N \left\{g_{1,n} \neq g_{2,n}\right\} \Biggr)=0 \\
						&\Rightarrow (b) \\
						&\Rightarrow (c)
					\end{align}
					が成り立つ.また前段と同じ理由で$(c) \Rightarrow (a)$が出る.
					
				\item[第三段]
					$0 \leq f \leq \infty$のとき,
					$f$が$\overline{\mathcal{B}}/\borel{[-\infty,\infty]}$-可測なら
					$f_n(x) \uparrow f(x)\ (\forall x \in X)$を満たす
					非負$\overline{\mathcal{B}}$-可測単関数列$(f_n)_{n=1}^\infty$が存在し,
					第二段の結果より各$f_n$に対して
					\begin{align}
						g_{1,n} \leq f_n \leq g_{2,n},
						\quad \mu\left( g_{1,n} \neq g_{2,n} \right)=0
					\end{align}
					を満たす$\mathcal{B}/\borel{[-\infty,\infty]}$-可測写像$g_{1,n},g_{2,n}$が存在する.
					\begin{align}
						g_1 \coloneqq \liminf_{n \to \infty} g_{1,n},
						\quad g_2 \coloneqq \limsup_{n \to \infty} g_{2,n}
					\end{align}
					とおけば
					\begin{align}
						g_{1,n}(x) = g_{2,n}(x)\ (\forall n \geq 1)
						\quad \Rightarrow \quad g_1(x) = \lim_{n \to \infty} f_n(x) = g_2(x)
					\end{align}
					が成り立ち
					\begin{align}
						\mu(g_1 \neq g_2)
						\leq \mu\Biggl( \bigcup_{n=1}^\infty \left\{g_{1,n} \neq g_{2,n}\right\} \Biggr)=0 \\
					\end{align}
					が従うから$(a) \Rightarrow (b)$及び$(b) \Rightarrow (c)$が得られる.
					第一段と同じ理由で$(c) \Rightarrow (a)$も成立する.
					
				\item[第四段]
					一般の$f:X \longrightarrow [-\infty,\infty]$に対し
					$f^+ \coloneqq f \defunc_{\{f \geq 0\}},\ f^- \coloneqq -f \defunc_{\{f < 0\}}$とおけば,
					$f$が$\overline{\mathcal{B}}/\borel{[-\infty,\infty]}$-可測なら
					$f^+,f^-$も$\overline{\mathcal{B}}/\borel{[-\infty,\infty]}$-可測である.従って
					\begin{align}
						g_1^\pm \leq f^\pm \leq g_2^\pm, \quad \mu\left( g_1^\pm \neq g_2^\pm \right) = 0,
						\quad \mbox{(複合同順)}
					\end{align}
					を満たす$\mathcal{B}/\borel{[-\infty,\infty]}$-可測写像$g_1^{\pm},g_2^{\pm}$が存在する.
					ここで
					\begin{align}
						g_1 \coloneqq g_1^+ - g_2^-,
						\quad g_2 \coloneqq g_2^+ - g_1^+
					\end{align}
					とおけば$(a) \Rightarrow (b)$成り立ち,前段と同様に$(b) \Rightarrow (c) \Rightarrow (a)$も得られる.
					\QED
			\end{description}
		\end{prf}
		
		\begin{screen}
			\begin{cor}
				$(X,\mathcal{B},\mu)$を測度空間,そのLebesgue拡大を
				$(X,\overline{\mathcal{B}},\overline{\mu})$と書き,
				$f:X \longrightarrow \C$とする.このとき次は同値である:
				\begin{description}
					\item[(a)] 或る$\mathcal{B}/\borel{\C}$-可測関数$g$が存在して
						$f = g\quad \mbox{$\overline{\mu}$-a.e.}$を満たす.
					\item[(b)] $f$は$\overline{\mathcal{B}}/\borel{\C}$-可測である.
				\end{description}
			\end{cor}
		\end{screen}
	
	\subsection{測度の構成1: 外測度による方法}
	\subsection{測度の構成2: Riesz-Markov-角谷の定理による方法}
	\subsection{測度の構成1と2の関係}
	\subsection{有限加法的測度の拡張}
		\begin{screen}
			\begin{thm}[Kolmogorov-Hopf]
				$(X,\mathcal{B},\mu_0)$を有限加法的測度空間($\mathcal{B}$は有限加法族,$\mu_0$は有限加法的)とし,
				\begin{align}
					\mu^*(A) \coloneqq \inf{}{}\Set{\sum_{n=1}^\infty \mu_0(B_n)}{B_n \in \mathcal{B},\ A \subset \bigcup_{n=1}^\infty B_n},
					\quad (\forall A \subset X)
				\end{align}
				により$X$上に外測度を定め,$\mu^*$-可測集合を$\mathcal{B}^*$と書く.このとき,
				\begin{description}
					\item[(1)] $\sigma[\mathcal{B}] \subset \mathcal{B}^*$が成り立つ.
						ここで$\mu' \coloneqq\left.\mu^*\right|_{\mathcal{B}^*},
						\ \mu \coloneqq \left.\mu^*\right|_{\sigma[\mathcal{B}]}$とおく.
					\item[(2)] $\mu_0$が$\mathcal{B}$上で$\sigma$-加法的なら
						\begin{align}
							\mu_0(B) = \mu(B),\quad (\forall B \in \mathcal{B})
							\label{eq:appendix_finite_additive_measure_expansion_1}
						\end{align}
						となる.つまり$\mu$は$\mu_0$の拡張である.
						
					\item[(3)] $\mu_0$が$\mathcal{B}$上で$\sigma$-有限的であるとき,
						$\left( X,\sigma[\mathcal{B}] \right)$上の測度$\mu_1,\mu_2$が
						(\refeq{eq:appendix_finite_additive_measure_expansion_1})を満たせば
						$\mu_1 = \mu_2$となる.
					
					\item[(4)] $\mu_0$が$\mathcal{B}$上で$\sigma$-加法的かつ$\sigma$-有限的ならば,
						$\mu$は$\mu_0$の$\left( X,\sigma[\mathcal{B}] \right)$への唯一つの拡張測度であり,
						$\left( X,\mathcal{B}^*,\mu' \right)$は$(X,\sigma[\mathcal{B}],\mu)$の
						Lebesgue拡大に一致する:
						\begin{align}
							\left( X,\mathcal{B}^*,\mu' \right) 
							= \left( X,\overline{\sigma[\mathcal{B}]},\overline{\mu} \right).
						\end{align}
				\end{description}
			\end{thm}
		\end{screen}
		
		\begin{prf}\mbox{}
			\begin{description}
				\item[(1)の証明]
					任意の$B \in \mathcal{B}$が$\mu^*$-可測であること,つまり任意の$A \subset X$に対し
					\begin{align}
						\mu^*(A) \geq \mu^*(A \cap B) + \mu^*(A \cap B^c)
						\label{eq:appendix_finite_additive_measure_expansion_2}
					\end{align}
					となることを示せば,$\mathcal{B} \subset \mathcal{B}^*$すなわち
					$\sigma[\mathcal{B}] \subset \mathcal{B}^*$が従う.
					任意の$A \subset X,\ \epsilon > 0$に対し
					\begin{align}
						A \subset \bigcup_{n=1}^\infty B_n,
						\quad \sum_{n=1}^\infty \mu_0(B_n) < \mu^*(A) + \epsilon
					\end{align}
					を満たす$\{B_n\}_{n=1}^\infty \subset \mathcal{B}$が存在する.
					このとき$A \cap B \subset \bigcup_{n=1}^\infty (B_n \cap B)
					,\ A \cap B^c \subset \bigcup_{n=1}^\infty (B_n \cap B^c)$より
					\begin{align}
						\mu^*(A \cap B) \leq \sum_{n=1}^\infty \mu_0(B_n \cap B),
						\quad \mu^*(A \cap B^c) \leq \sum_{n=1}^\infty \mu_0(B_n \cap B^c)
					\end{align}
					となるから
					\begin{align}
						\mu^*(A) + \epsilon
						&\geq \sum_{n=1}^\infty \mu_0(B_n)
						= \sum_{n=1}^\infty \left\{ \mu_0(B_n \cap B) + \mu_0(B_n \cap B^c) \right\} \\
						&= \sum_{n=1}^\infty \mu_0(B_n \cap B) + \sum_{n=1}^\infty \mu_0(B_n \cap B^c) \\
						&\geq \mu^*(A \cap B) + \mu^*(A \cap B^c)
					\end{align}
					が成り立つ.$\epsilon$の任意性より
					(\refeq{eq:appendix_finite_additive_measure_expansion_2})が出る.
				
				\item[(2)の証明]
					任意に$B \in \mathcal{B}$を取る.まず,
					$B \subset B \cup \emptyset \cup \emptyset \cup \cdots$より
					\begin{align}
						\mu^*(B) \leq \mu_0(B)
					\end{align}
					が成り立つ.一方で
					$B \subset \bigcup_{n=1}^\infty B_n$を満たす$\{B_n\}_{n=1}^\infty \subset \mathcal{B}$に対し
					\begin{align}
						B = \sum_{n=1}^\infty \Biggl( B \cap \Biggl( B_n \backslash \bigcup_{k=1}^{n-1}B_k \Biggr) \Biggr)
					\end{align}
					かつ$B \cap \left( B_n \backslash \bigcup_{k=1}^{n-1}B_k \right) \in \mathcal{B}$が満たされるから,
					$\mu_0$の$\sigma$-加法性より
					\begin{align}
						\mu_0(B) = \sum_{n=1}^\infty \mu_0\Biggl( B \cap \Biggl( B_n \backslash \bigcup_{k=1}^{n-1}B_k \Biggr) \Biggr)
						\leq \sum_{n=1}^\infty \mu_0(B_n)
					\end{align}
					が成り立ち$\mu_0(B) \leq \mu^*(B)$が従う.よって$\mu_0(B) = \mu^*(B) = \mu(B)$が得られる.
				
				\item[(3)の証明]
					$\sigma$-有限の仮定より,或る増大列$X_1 \subset X_2 \subset \cdots
					,\ \{X_n\}_{n=1}^\infty \subset \mathcal{B}$が存在して
					\begin{align}
						\mu_0 (X_n) < \infty \quad \bigcup_{n=1}^\infty X_n = X
						\label{eq:appendix_finite_additive_measure_expansion_3}
					\end{align}
					を満たす.このとき
					\begin{align}
						\mathscr{D}_n \coloneqq \Set{B \in \sigma[\mathcal{B}]}{\mu_1(B \cap X_n) = \mu_2(B \cap X_n)},
						\quad (n=1,2,\cdots)
					\end{align}
					とおけば,(\refeq{eq:appendix_finite_additive_measure_expansion_1})より
					$\mathscr{D}_n$は$\mathcal{B}$を含むDynkin族である.従ってDynkin族定理より
					\begin{align}
						\mathscr{D}_n = \sigma[\mathcal{B}],
						\quad (\forall n \geq 1)
					\end{align}
					が成り立ち
					\begin{align}
						\mu_1(B) = \lim_{n \to \infty} \mu_1(B \cap X_n)
						= \lim_{n \to \infty} \mu_2(B \cap X_n) = \mu_2(B),
						\quad (\forall B \in \sigma[\mathcal{B}])
					\end{align}
					が出る.
					
				\item[(4)の証明]
					(2)と(3)の結果より$\mu$は$\mu_0$の唯一つの拡張測度である.次に
					\begin{align}
						\mathcal{B}^* = \overline{\sigma[\mathcal{B}]}
					\end{align}
					を示す.$E \in \overline{\sigma[\mathcal{B}]}$なら
					或る$B_1,B_2 \in \sigma[\mathcal{B}]$が存在して
					\begin{align}
						B_1 \subset E \subset B_2, \quad \mu(B_2 - B_1) = 0
					\end{align}
					を満たす.このとき(1)より
					$\mu'(B_2 - B_1) = 0$であり,$\left( X,\mathcal{B}^*,\mu' \right)$の完備性より
					$E \backslash B_1 \in \mathcal{B}^*$が満たされ
					\begin{align}
						E = B_1 + E \backslash B_1 \in \mathcal{B}^*
					\end{align}
					が従う.いま,(\refeq{eq:appendix_finite_additive_measure_expansion_3})を満たす
					$\{X_n\}_{n=1}^\infty \subset \mathcal{B}$を取り,
					$E \in \mathcal{B}^*$に対して$E_n \coloneqq E \cap X_n$とおく.このとき
					\begin{align}
						\mu'(E_n) \leq \mu'(X_n) = \mu_0(X_n) < \infty
					\end{align}
					となり,任意の$k = 1,2,\cdots$に対して
					\begin{align}
						E_n \subset \bigcup_{j=1}^\infty B^n_{k,j},
						\quad
						\sum_{j=1}^\infty \mu_0\left( B^n_{k,j} \right)
						< \mu'(E_n) + \frac{1}{k}
					\end{align}
					を満たす$\left\{B^n_{k,j}\right\}_{j=1}^\infty \subset \mathcal{B}$が存在する.
					\begin{align}
						B_{2,n} \coloneqq \bigcap_{k=1}^\infty \bigcup_{j=1}^\infty B^n_{k,j}
					\end{align}
					とおけば$E_n \subset B_{2,n} \in \sigma[\mathcal{B}]$であり,
					任意の$k = 1,2,\cdots$に対して
					\begin{align}
						&\mu'(B_{2,n} - E_n) = \mu'(B_{2,n}) - \mu'(E_n)
						\leq \mu'\Biggl( \bigcup_{j=1}^\infty B^n_{k,j} \Biggr) - \mu'(E_n) \\
						&\qquad \leq \sum_{j=1}^\infty \mu'\left( B^n_{k,j} \right) - \mu'(E_n)
						< \mu'(E_n) + \frac{1}{k} - \mu'(E_n)
						= \frac{1}{k}
					\end{align}
					が成り立つから$\mu'(B_{2,n} - E_n) = 0$となる.
			\end{description}
		\end{prf}
	\section{拡張定理}
	確率空間を実数空間の上に作ってしまうと
	Kolmogorovの拡張定理の証明は論理的にも見た目の上でも煩雑になるようだが,
	Bogachevが説明する通り,コンパクトクラスの概念を使って
	一般化されたKolmogorovの拡張定理は主張のみならず証明も洗練されたものとなる.
	一見長い証明となるが,内容は抽象的で捉えやすい.
	
	いま,$T$を空でない集合とし,$T$の任意の要素$t$に対して可測空間$(X_t,\mathscr{B}_t)$が
	定まっていて,また
	\begin{align}
		\forall t \in T\, (\, X_t \neq \emptyset\, )
	\end{align}
	が満たされているとする.$\mathscr{F}$を$T$の空でない任意の有限部分集合の全体として,$\mathscr{F}$の任意の要素$\Lambda$に対して
	\begin{align}
		X_\Lambda \defeq \prod_{t \in \Lambda} X_t,
		\quad \mathscr{B}_\Lambda \defeq \bigotimes_{t \in \Lambda} \mathscr{B}_t
	\end{align}
	により可測空間$(X_\Lambda,\mathscr{B}_\Lambda)$を定める.また
	\begin{align}
		X \defeq \prod_{t \in T} X_t,
		\quad \mathscr{B} \defeq \bigotimes_{t \in T} \mathscr{B}_t
	\end{align}
	とおく.$\mathscr{F}$の任意の要素$\Lambda,\Lambda'$に対し,
	$\Lambda \subset \Lambda'$であるとき$X_{\Lambda'}$から
	$X_{\Lambda}$への射影を$\pi_{\Lambda',\Lambda}$と書き,
	また$X$から$X_\Lambda$への射影を$\pi_{\Lambda}$と書く.以上が準備となる.
	
	定理の首脳部に入る前に次の補題を証明する.
	\begin{screen}
		\begin{lem}[射影の可測性]\label{lem:Kolmogorov_extension_theorem}
			$T$の任意の空でない部分集合$\Lambda,\Lambda'$に対し(有限性は要らない),$\Lambda \subset \Lambda'$であるとき
			射影$\pi_{\Lambda',\Lambda}$は$\mathscr{B}_{\Lambda'}/\mathscr{B}_\Lambda$-可測である.
			また射影$\pi_\Lambda$は$\mathscr{B}/\mathscr{B}_\Lambda$-可測である.
		\end{lem}
	\end{screen}
	
	\begin{prf}
		$\Lambda,\Lambda'$を$T$の空でない部分集合とする.このとき,まず$\Lambda$の任意の要素$t$に対し$\pi_{\Lambda,\{t\}}$は
		$X_\Lambda$から$X_t$への射影であるから,直積$\sigma$-加法族の定義より$\pi_{\Lambda,\{t\}}$は
		$\mathscr{B}_\Lambda/\mathscr{B}_t$-可測性を持つ.特に$\Lambda=T$の場合
		\begin{align}
			\pi_{\Lambda,\{t\}} = \pi_\Lambda,\quad \mathscr{B}_\Lambda = \mathscr{B}
		\end{align}
		であるから$\pi_\Lambda$の$\mathscr{B}/\mathscr{B}_\Lambda$-可測性が得られる.
		また$\Lambda \subset \Lambda'$であるとき,$\Lambda$の任意の要素$t$と$\mathscr{B}_t$の任意の要素$B$に対し
		\begin{align}
			\pi_{\Lambda',\Lambda}^{-1}\left(\pi_{\Lambda,\{t\}}^{-1}(B)\right)
			= \pi_{\Lambda',\{t\}}^{-1}(B) \in \mathscr{B}_{\Lambda'}
		\end{align}
		が成立する.従って
		\begin{align}
			\bigcup_{t\in\Lambda} \Set{\pi_{\Lambda,\{t\}}^{-1}(B)}{B \in \mathscr{B}_t}
			\subset \Set{B \in \mathscr{B}_\Lambda}{\pi_{\Lambda',\Lambda}^{-1}(B) \in \mathscr{B}_{\Lambda'}}
		\end{align}
		となり,左辺は$\mathscr{B}_\Lambda$を生成し右辺は$\sigma$-加法族であるから
		$\pi_{\Lambda',\Lambda}$の$\mathscr{B}_{\Lambda'}/\mathscr{B}_\Lambda$-可測性が従う.
		\QED
	\end{prf}
	
	本節の主題は次である.いま,$\mathscr{F}$の任意の要素$\Lambda$に対し,
	$(X_\Lambda,\mathscr{B}_\Lambda)$上に確率測度$\mu_\Lambda$が定まっていて
	\begin{align}
		\forall \Lambda,\Lambda' \in \mathscr{F},\quad
		\Lambda \subset \Lambda' \Longrightarrow
		\mu_{\Lambda'} \pi_{\Lambda',\Lambda}^{-1}
		= \mu_\Lambda
	\end{align}
	が成り立っているとする.この式を{\bf 両立条件}\index{りょうりつじょうけん@両立条件}
	{\bf (consistency condition)}と呼ぶ.
	加えて,$T$の任意の要素$t$に対して$\mathscr{B}_t$に含まれるコンパクトクラス$\mathcal{K}_t$が取れて,
	任意の正数$\epsilon$と$\mathscr{B}_t$の任意の要素$B$に対して
	\begin{align}
		K \subset B \wedge \mu_{\{t\}}(B \backslash K) < \epsilon
	\end{align}
	なる$\mathcal{K}_t$の要素$K$が取れるとする.このとき,
	\begin{align}
		\forall \Lambda \in \mathscr{F},\quad 
		\mu \pi_{\Lambda}^{-1} = \mu_\Lambda.
	\end{align}
	を満たす$(X,\mathscr{B})$上の確率測度$\mu$が唯一つだけ取れる.
	
	\begin{prf}\mbox{}
		\begin{description}
			\item[第一段]
				$\mathscr{R} = \bigcup_{\Lambda\in\mathscr{F}} \Set{\pi_\Lambda^{-1}(B)}{B \in \mathscr{B}_\Lambda}$
				で$\mathscr{R}$を定めれば,$\mathscr{R}$は$X$上の有限加法族となり$\mathscr{B}$を生成する.
				実際
				\begin{align}
					X = \pi_\Lambda^{-1}(X_\Lambda) \in \mathscr{R}
				\end{align}
				が成立し,また任意に$\mathscr{R}$の要素$A$を取れば,
				$A$は或る$\Lambda \in \mathscr{F}$と$B \in \mathscr{B}_\Lambda$を用いて
				\begin{align}
					A = \pi_\Lambda^{-1}(B)
				\end{align}
				と表され,$X_\Lambda \backslash B \in \mathscr{B}_\Lambda$であるから
				\begin{align}
					X \backslash A = \pi_\Lambda^{-1}(X_\Lambda \backslash B) \in \mathscr{R}
				\end{align}
				が成り立つ.また任意に$\mathscr{R}$の要素$A,A'$を取れば,
				或る$\mathscr{F}$の要素$\Lambda$と$\mathscr{B}_\Lambda$の要素$B$及び
				或る$\mathscr{F}$の要素$\Lambda'$と$\mathscr{B}_{\Lambda'}$の要素$B'$を用いて
				\begin{align}
					A = \pi_\Lambda^{-1}(B),\quad A' = \pi_\Lambda^{-1}(B')
				\end{align}
				と表され,このとき$\Lambda'' = \Lambda \cup \Lambda'$とおけば
				\begin{align}
					A = \pi_{\Lambda''}^{-1}\left(\pi_{\Lambda'',\Lambda}^{-1}(B)\right),
					\quad A' = \pi_{\Lambda''}^{-1}\left(\pi_{\Lambda'',\Lambda'}^{-1}(B')\right)
				\end{align}
				が成り立つ.補題\ref{lem:Kolmogorov_extension_theorem}より
				$\pi_{\Lambda'',\Lambda}^{-1}(B)$と$\pi_{\Lambda'',\Lambda'}^{-1}(B')$は共に$\mathscr{B}_{\Lambda''}$に属するので
				\begin{align}
					A \cup A' =  \pi_{\Lambda''}^{-1}\left(\pi_{\Lambda'',\Lambda}^{-1}(B) \cup 
					\pi_{\Lambda'',\Lambda'}^{-1}(B')\right)
				\end{align}
				となり$A \cup A' \in \mathscr{R}$が成立する.以上より$\mathscr{R}$は$X$上の有限加法族をなしている.
				また$T$の任意の要素$t$に対して$\{t\}$は$\mathscr{F}$に属するから
				\begin{align}
					\bigcup_{t \in T}\Set{\pi_{\{t\}}^{-1}(B)}{B \in \mathscr{B}_t} \subset \mathscr{R}
				\end{align}
				が成り立ち,左辺は$\mathscr{B}$を生成するので$\mathscr{B} \subset \sigma(\mathscr{R})$を得る.
				一方で$\mathscr{F}$の任意の要素$\Lambda$に対し$\pi_\Lambda$は
				$\mathscr{B}/\mathscr{B}_\Lambda$-可測である(補題\ref{lem:Kolmogorov_extension_theorem})
				から$\mathscr{R} \subset \mathscr{B}$も得られ,以上で
				\begin{align}
					\sigma(\mathscr{R}) = \mathscr{B}
				\end{align}
				が出る.
				
			\item[第二段]
				$\mu$を
				\begin{align}
					\mu = \Set{(x,y)}{\exists \Lambda \in \mathscr{F}
					\exists B \in \mathscr{B}_\Lambda
					\left(x=\pi_\Lambda^{-1}(B) \wedge y = P_\Lambda(B)\right)}
				\end{align}
				により定める.このとき$\mu$はsigle-valuedであり,
				$\mathscr{R}$上の有限加法的測度となる.まず$\mu$がsingle-valuedであることを示す.
				\begin{align}
					\pi_\Lambda^{-1}(B) = \pi_{\Lambda'}^{-1}(B')
				\end{align}
				であるとき,$\Lambda'' \coloneqq \Lambda \cup \Lambda'$とおけば
				\begin{align}
					\pi_{\Lambda''}^{-1}\left( \pi_{\Lambda'',\Lambda}^{-1}(B) \right)
					= \pi_\Lambda^{-1}(B)
					= \pi_{\Lambda'}^{-1}(B')
					= \pi_{\Lambda''}^{-1}\left( \pi_{\Lambda'',\Lambda'}^{-1}(B') \right)
				\end{align}
				が成り立つから$\pi_{\Lambda'',\Lambda}^{-1}(B) = \pi_{\Lambda'',\Lambda'}^{-1}(B')$
				が従い(定理\ref{projective_injective_image_preimage}),整合性条件より
				\begin{align}
					\mu_\Lambda(B) 
					= \mu_{\Lambda''} \circ \pi_{\Lambda'',\Lambda}^{-1}(B)
					= \mu_{\Lambda''} \circ \pi_{\Lambda'',\Lambda'}^{-1}(B')
					= \mu_{\Lambda'}(B')
				\end{align}
				が満たされ$\mu$の一意性を得る.次に$\mu$の加法性を示す.
				\begin{align}
					\pi_{\Lambda_1}^{-1}(B_1) \cap \pi_{\Lambda_2}^{-1}(B_2) = \emptyset
				\end{align}
				であるとき,$\Lambda_3 \coloneqq \Lambda_1 \cup \Lambda_2$とおけば
				\begin{align}
					\emptyset 
					= \pi_{\Lambda_3}^{-1}\left( \pi_{\Lambda_3,\Lambda_1}^{-1}(B_1) \right)
					\cap \pi_{\Lambda_3}\left( \pi_{\Lambda_3,\Lambda_2}^{-1}(B_2) \right)
					= \pi_{\Lambda_3}^{-1}\left( \pi_{\Lambda_3,\Lambda_1}^{-1}(B_1) \cap \pi_{\Lambda_3,\Lambda_2}^{-1}(B_2) \right)
				\end{align}
				となるから$\pi_{\Lambda_3,\Lambda_1}^{-1}(B_1) \cap \pi_{\Lambda_3,\Lambda_2}^{-1}(B_2)
				= \emptyset$が従い(全射の性質),
				\begin{align}
					\mu\left( \pi_{\Lambda_1}^{-1}(B_1) \cup \pi_{\Lambda_2}^{-1}(B_2) \right)
					&= \mu\left[\pi_{\Lambda_3}^{-1}\left( \pi_{\Lambda_3,\Lambda_1}^{-1}(B_1) \right)
					\cup \pi_{\Lambda_3}\left( \pi_{\Lambda_3,\Lambda_2}^{-1}(B_2) \right) \right] \\
					&= \mu\left[ \pi_{\Lambda_3}^{-1}\left( \pi_{\Lambda_3,\Lambda_1}^{-1}(B_1) \cup \pi_{\Lambda_3,\Lambda_2}^{-1}(B_2) \right) \right] \\
					&= \mu_{\Lambda_3} \left( \pi_{\Lambda_3,\Lambda_1}^{-1}(B_1) \cup \pi_{\Lambda_3,\Lambda_2}^{-1}(B_2) \right) \\
					&= \mu_{\Lambda_3} \left( \pi_{\Lambda_3,\Lambda_1}^{-1}(B_1) \right)
						+ \mu_{\Lambda_3} \left( \pi_{\Lambda_3,\Lambda_2}^{-1}(B_2) \right) \\
					&= \mu\left( \pi_{\Lambda_1}^{-1}(B_1) \right)
						+ \mu\left( \pi_{\Lambda_2}^{-1}(B_2) \right)
				\end{align}
				が成立する.
			
			\item[第三段]
				$\mu$が$\mathscr{R}$の上で完全加法的であることを定理\ref{thm:compact_class_intersection}と併せて示す.
		\end{description}
	\end{prf}
	\section{Kolmogorov-\v{C}entsovの定理}
	\begin{itembox}[l]{Exercise 2.7}
		The only $\borel{(\R^d)^{[0,\infty)}}$-measurable set contained 
		in $C[0,\infty)^d$ is the empty set.
	\end{itembox}
	
	\begin{prf}\mbox{}
		\begin{description}
			\item[第一段]
				$\borel{(\R^d)^{[0,\infty)}} = \sigma(B_t;\ 0 \leq t < \infty)$
				が成り立つことを示す.先ず,任意の$C \in \mathscr{C}$は
				\begin{align}
					C &= \Set{\omega \in (\R^d)^{[0,\infty)}}{(\omega(t_1),\cdots,\omega(t_n)) \in A} \\
					&=  \Set{\omega \in (\R^d)^{[0,\infty)}}{(B_{t_1}(\omega),\cdots,B_{t_n}(\omega)) \in A},
					\quad (A \in \borel{(\R^d)^n})
				\end{align}
				の形で表されるから$\mathscr{C} \subset \sigma(B_t;\ 0 \leq t < \infty)$
				が従い$\borel{(\R^d)^{[0,\infty)}} \subset \sigma(B_t;\ 0 \leq t < \infty)$
				を得る.逆に
				\begin{align}
					\sigma(B_t) \subset \mathscr{C},
					\quad (\forall t \geq 0)
				\end{align}
				より$\borel{(\R^d)^{[0,\infty)}} \supset \sigma(B_t;\ 0 \leq t < \infty)$
				も成立し$\borel{(\R^d)^{[0,\infty)}} = \sigma(B_t;\ 0 \leq t < \infty)$
				が出る.
				
			\item[第二段]
				高々可算集合$S = \{t_1,t_2,\cdots\} \subset [0,\infty)$に対して
				\begin{align}
					\mathcal{E}_S \coloneqq \Set{\Set{\omega \in (\R^d)^{[0,\infty)}}{(\omega(t_1),\omega(t_2),\cdots) \in A}}{A \in \borel{(\R^d)^{\# S}}}
				\end{align}
				とおけば
				\footnote{
					$S$が可算無限なら$(\R^d)^{\# S} = \R^\infty$.
				},座標過程$B$は
				$(\omega(t_1),\omega(t_2),\cdots) = (B_{t_1}(\omega),B_{t_2}(\omega),\cdots)$
				を満たすから
				\begin{align}
					\mathcal{E}_S = \Set{\left\{(B_{t_1},B_{t_2},\cdots) \in A\right\}}{A \in \borel{(\R^d)^{\# S}}} \eqqcolon \mathcal{F}^B_S
				\end{align}
				が成立する.従って第一章のLemma3 for Exercise 1.8と前段の結果より
				\begin{align}
					\borel{(\R^d)^{[0,\infty)}}
					&= \sigma(B_t;\ 0 \leq t < \infty)
					= \mathcal{F}^B_{[0,\infty)}
					= \bigcup_{S \subset [0,\infty):at\ most\ countable} \mathcal{F}^B_S\\
					&= \bigcup_{S \subset [0,\infty):at\ most\ countable} \mathcal{E}_S
				\end{align}
				を得る.すなわち,$\borel{(\R^d)^{[0,\infty)}}$の任意の元は
				$\Set{\omega \in (\R^d)^{[0,\infty)}}{(\omega(t_1),\omega(t_2),\cdots) \in A}$
				の形で表現され,$A \neq \emptyset$ならば
				$\Set{\omega \in (\R^d)^{[0,\infty)}}{(\omega(t_1),\omega(t_2),\cdots) \in A} \not\subset C[0,\infty)^d$となり主張が従う.
				\QED
		\end{description}
	\end{prf}
	
	\begin{itembox}[l]{Theorem 2.8 の主張は次のように変更するべきである:}
		Suppose that a process $X = \Set{X_t}{0 \leq t \leq T}$ on a probability space 
		$(\Omega,\mathscr{F},P)$ satisfies the condition
		\begin{align}
			E|X_t - X_s|^\alpha \leq C|t-s|^{1 + \beta},
			\quad 0 \leq s,t \leq T,
		\end{align}
		for some positive constants $\alpha,\beta$, and $C$. Then there exists a 
		continuous modification $\tilde{X} = \Set{\tilde{X}_t}{0 \leq t \leq T}$ of $X$, 
		which is locally H\Ddot{o}lder-continuous with exponent $\gamma$ for every 
		$\gamma \in (0,\beta/\alpha)$. More precisely, for every $\gamma \in (0,\beta/\alpha)$,
		\begin{align}
			\sup{\substack{0 < |t-s| < h(\omega) \\ s,t \in [0,T]}}{\frac{\left| \tilde{X}_t(\omega) - \tilde{X}_s(\omega) \right|}{|t-s|^\gamma}} \leq \frac{2}{1-2^{-\gamma}},
			\quad \forall \omega \in \Omega^*,
		\end{align}
		for some $\Omega^* \in \mathscr{F}$ with $P(\Omega^*)=1$ and 
		positive random variable $h$, where $\Omega^*$ and $h$ depend on $\gamma$.
	\end{itembox}
	
	なぜならば,式(2.8)において$P$の中身が$\Omega^*$に一致するかどうかわからないためである.
	可測集合でなければ$P$で測ることはできない.ただし
	今の場合は$(\Omega,\mathscr{F},P)$が完備確率空間ならば式(2.8)の表記で問題ない.
	
	
	\begin{itembox}[l]{Theorem 2.8 memo}
		証明中の式(2.10)直後の
		``where $n^*(\omega)$ is a positive, integer-valued random variable''
		について.
	\end{itembox}
	
	\begin{prf}
		$\N \coloneqq \{1,2,\cdots\}$とおき,
		$\N$の冪集合を$2^\N$で表せば,$(\N,2^\N)$は可測空間となる.
		示せばよいのは$n^*$の$\mathscr{F}/2^\N$-可測性である.
		ただし,$n^*$は証明文中においてwell-definedでないため,明確な意味を持たせる必要がある.
		\begin{align}
			A_0 \coloneqq \Omega,
			\quad A_n \coloneqq \Set{\omega \in \Omega}{\max{1 \leq k \leq 2^n}{\left| X_{k/2^n}(\omega) - X_{(k-1)/2^n}(\omega) \right|} \geq 2^{-\gamma n}},
			\quad (n=1,2,\cdots)
		\end{align}
		とおくとき,$\Omega^*$は
		\begin{align}
			\Omega^* \coloneqq \bigcup_{\ell =1}^{\infty} \bigcap_{n = \ell}^\infty A_n^c
		\end{align}
		により定まる集合である.
		任意の$\omega \in \Omega^*$に対して或る$\ell \geq 1$が存在し,
		\begin{align}
			\max{1 \leq k \leq 2^n}{\left| X_{k/2^n}(\omega) - X_{(k-1)/2^n}(\omega) \right|} < 2^{-\gamma n},
			\quad (\forall n \geq \ell)
		\end{align}
		を満たす.このような$\ell$のうち最小なものを$n^*(\omega)$と定めれば
		\begin{align}
			{n^*}^{-1}(\ell) = \left\{ \bigcap_{n = \ell}^\infty A_n^c \right\} \cap \left\{ \bigcap_{n = \ell-1}^\infty A_n^c \right\}^c,
			\quad (\ell =1,2,\cdots)
		\end{align}
		が成立し$n^*$の$\mathscr{F}/2^\N$-可測性が従う.
		\QED
	\end{prf}
	
	確率変数$h$について,厳密には
	\begin{align}
		h(\omega) \coloneqq 
		\begin{cases}
			2^{-n^*(\omega)}, & (\omega \in \Omega^*), \\
			0, & (\omega \in \Omega \backslash \Omega^*)
		\end{cases}
	\end{align}
	とおけばよい.
	
	\begin{itembox}[l]{Theorem 2.8 memo}
	\end{itembox}
\section{積分}
\subsection{積分}
	\begin{screen}
		\begin{thm}[複素数値可測$\Longleftrightarrow$実部虚部が可測]\label{thm:measurability_of_complex_measurable_functions}
			$(X,\mathscr{F})$を可測空間,$f:X \longrightarrow \C$とするとき,
			$f$が$\mathscr{F}/\borel{\C}$-可測であることと
			$f$の実部$u$と虚部$v$がそれぞれ$\mathscr{F}/\borel{\R}$-可測であることは同値である.
		\end{thm}
	\end{screen}
	
	\begin{prf}
		$z \in \C$に対し$x,y \in \C$の組が唯一つ対応し$z = x + i y$を満たす.この対応関係により定める写像
		\begin{align}
			\varphi:\C \ni z \longmapsto (x,y) \in \R^2
		\end{align}
		は位相同型である.射影を$p_1:\R^2 \ni (x,y) \longmapsto x,
		\ p_2:\R^2 \ni (x,y) \longmapsto y$とすれば
		\begin{align}
			u = p_1 \circ \varphi \circ f,
			\quad v = p_2 \circ \varphi \circ f
		\end{align}
		となるから,$f$が$\mathscr{F}/\borel{\C}$-可測であるなら
		$p_1,p_2,\varphi$の連続性より
		\begin{align}
			u^{-1}(A) = f^{-1} \circ \varphi^{-1} \circ p_1^{-1}(A) \in \mathscr{F},
			\quad v^{-1}(A) = f^{-1} \circ \varphi^{-1} \circ p_2^{-1}(A) \in \mathscr{F},
			\quad (\forall A \in \borel{\R})
		\end{align}
		が成り立ち$u,v$の$\mathscr{F}/\borel{\R}$-可測性が従う.逆に$u,v$が$\mathscr{F}/\borel{\R}$-可測であるとき,
		\begin{align}
			f^{-1}(B) = \Set{x \in X}{(u(x),v(x)) \in \varphi(B)} \in \mathscr{F},
			\quad (\forall B \in \borel{\C})
		\end{align}
		が成り立ち$f$の$\mathscr{F}/\borel{\C}$-可測性が出る.
		\QED
	\end{prf}
	
	\begin{screen}
		\begin{thm}[和・積・商の可測性]
			
		\end{thm}
	\end{screen}
	
	\begin{screen}
		\begin{thm}[相対位相のBorel集合族]\label{thm:Borel_algebra_of_relative_topology}
			$(S,\mathscr{O})$を位相空間とする.部分集合$A \subset S$に対して
			\begin{align}
				\borel{A} \coloneqq \sigma\left[ \Set{A \cap O}{O \in \mathscr{O}} \right]
			\end{align}
			とおくとき次が成り立つ:
			\begin{align}
				\borel{A} = \Set{A \cap E}{E \in \borel{S}}.
			\end{align}
			また$A \in \borel{S}$なら$\borel{A} \subset \borel{S}$となる.
		\end{thm}
	\end{screen}
	
	$\R$-値可測関数は$\C$-値可測関数でもある.
	
	\begin{screen}
		\begin{thm}[単関数近似列の存在]
			$(X,\mathscr{F})$を可測空間とする.
			\begin{description} 
				\item[(1)] 任意の$\mathscr{F}/\borel{[0,\infty]}$-可測写像$f$に対し
					\begin{align}
						0 \leq f_1 \leq f_2 \leq \cdots \leq f;
						\quad \lim_{n \to \infty} f_n(x) = f(x),\ (\forall x \in X)
					\end{align}
					を満たす$\mathscr{F}/\borel{[0,\infty)}$-可測単関数列$(f_n)_{n=1}^\infty$が存在する.
					
				\item[(2)]
					 任意の$\mathscr{F}/\borel{\C}$-可測写像$f$に対し
					\begin{align}
						0 \leq |f_1| \leq |f_2| \leq \cdots \leq |f|;
						\quad \lim_{n \to \infty} f_n(x) = f(x),\ (\forall x \in X)
					\end{align}
					を満たす$\mathscr{F}/\borel{\C}$-可測単関数列$(f_n)_{n=1}^\infty$が存在する.
				
				\item[(3)] (1)または(2)において,$f$が$E \in \mathscr{F}$上で有界なら
					$f_n \defunc_E$は一様に$f \defunc_E$を近似する:
					\begin{align}
						\sup{x \in E}{\left| f_n(x) - f(x) \right|} \longrightarrow 0
						\quad (n \longrightarrow \infty).
					\end{align}
			\end{description}
		\end{thm}
	\end{screen}
	
	\begin{screen}
		\begin{dfn}[複素数値可測関数の正値測度に関する積分]
			$(X,\mathscr{F},\mu)$を正値測度空間,
			$f$を$\mathscr{F}/\borel{\C}$-可測関数とする.
			$u \coloneqq \Re{f},\ v \coloneqq \Im{f}$とおけば
			$|u|,|v| \leq |f| \leq |u| + |v|$より
			\begin{align}
				\mbox{$|f|$が可積分} \quad \Longleftrightarrow \quad
				\mbox{$u,v$が共に可積分}
			\end{align}
			が成り立つ.$|f|$が可積分のとき,$f$は可積分であるといい$f$の$\mu$に関する積分を次で定める:
			\begin{align}
				\int_X f\ d\mu
				\coloneqq \int_X u\ d\mu + i \int_X v\ d\mu.
			\end{align}
		\end{dfn}
	\end{screen}
	
	\begin{screen}
		\begin{thm}[Lebesgueの収束定理]
			$(X,\mathscr{F},\mu)$を正値測度空間,
			$f,\ f_n\ (n=1,2,\cdots)$を$\mathscr{F}/\borel{\C}$-可測な可積分関数とする.
			このとき,$f = \lim_{n \to \infty} f_n\ \mbox{$\mu$-a.e.}$かつ
			\begin{align}
				|f_n| \leq g, \quad \mbox{$\mu$-a.e.}
			\end{align}
			を満たす可積分関数$g$が存在するとき
			\begin{align}
				\int_X |f - f_n|\ d\mu \longrightarrow 0
				\quad (n \longrightarrow \infty).
			\end{align}
		\end{thm}
	\end{screen}
	
	\begin{screen}
		\begin{thm}[積分の線形性・積分作用素の有界性]
			$(X,\mathscr{F})$を可測空間とし,$\mu$を$\mathscr{F}$上の正値測度とする.
			\begin{description}
				\item[(1)] 任意の$\mathscr{F}/\borel{\C}$-可測可積分関数$f,g$と
					$\alpha,\beta \in \C$に対して次が成り立つ:
					\begin{align}
						\int_X \alpha f + \beta g\ d\mu
						= \alpha \int_X f\ d\mu + \beta \int_X g\ d\mu.
					\end{align}
					
				\item[(2)] 任意の$\mathscr{F}/\borel{\C}$-可測可積分関数$f$に対して次が成り立つ:
					\begin{align}
						\left| \int_X f\ d\mu \right| \leq \int_X |f|\ d\mu.
					\end{align}
			\end{description}	
		\end{thm}
	\end{screen}
	
	\begin{prf}\mbox{}
		\begin{description}
			\item[(1)] 
			
			\item[(2)]
				$\alpha \coloneqq \int_X f\ d\mu$とおけば,$\alpha \neq 0$の場合
				\begin{align}
					|\alpha|
					= \frac{\overline{\alpha}}{|\alpha|} \int_X f\ d\mu
					= \int_X \frac{\overline{\alpha}}{|\alpha|} f\ d\mu
				\end{align}
				が成り立ち
				\begin{align}
					|\alpha| = \Re{|\alpha|}
					= \Re{\int_X \frac{\overline{\alpha}}{|\alpha|} f\ d\mu}
					= \int_X \Re{\frac{\overline{\alpha}}{|\alpha|} f}\ d\mu
					\leq \int_X |f|\ d\mu
				\end{align}
				が従う.$\alpha = 0$の場合も不等式は成立する.
				\QED
		\end{description}
	\end{prf}
	
	\begin{screen}
		\begin{lem}
			$S$を実数の集合とする.$-S \coloneqq \Set{-s}{s \in S}$とおくとき次が成り立つ:
			\begin{align}
				\inf{}{S} = -\sup{}{(-S)},
				\quad \sup{}{S} = -\inf{}{(-S)}.
			\end{align}
		\end{lem}
	\end{screen}
	
	\begin{prf}
		任意の$s \in S$に対して$-s \leq \sup{}{(-S)}$より
		$\inf{}{S} \geq -\sup{}{(-S)}$となる.一方で任意の$s \in S$に対し
		$\inf{}{S} \leq s$より$-s \leq -\inf{}{S}$となり
		$\sup{}{(-S)} \leq -\inf{}{S}$が従うから
		$-\sup{}{(-S)} \geq \inf{}{S}$も成り立ち
		$\inf{}{S} = -\sup{}{(-S)}$が出る.
		\QED
	\end{prf}
	
	\begin{screen}
		\begin{thm}[写像の値域は積分の平均値の範囲を出ない]\label{thm:mean_value_of_integral_and_closed_set}
			$(X,\mathscr{F},\mu)$を$\sigma$-有限測度空間,
			$f:X \longrightarrow \C$を$\mathscr{F}/\borel{\C}$-可測かつ$\mu$-可積分な関数,
			$C \subset \C$を閉集合とする.このとき
			\begin{align}
				\frac{1}{\mu(E)}\int_E f\ d\mu \in C,
				\quad (\forall E \in \mathscr{F},\ 0 < \mu(E) < \infty)
				\label{eq:thm_mean_value_of_integral_and_closed_set}
			\end{align}
			なら次が成り立つ:
			\begin{align}
				f(x) \in C \quad \mbox{$\mu$-a.e.}x \in X.
			\end{align}
		\end{thm}
	\end{screen}
	$C=\R$なら$f$は殆ど至る所$\R$値であり,
	$C=\{0\}$なら殆ど至る所$f=0$である.
	\begin{prf}
		$\sigma$-有限の仮定より次を満たす$\{X_n\}_{n=1}^\infty \subset \mathscr{F}$が存在する:
		\begin{align}
			\mu(X_n) < \infty,\ (\forall n \geq 1);
			\quad X = \bigcup_{n=1}^\infty X_n.
		\end{align}
		$C = \C$なら$f(x) \in C\ (\forall x \in X)$である.
		$C \neq \C$の場合,任意の$\alpha \in \C \backslash C$に対し
		或る$r > 0$が存在して
		\begin{align}
			B_r(\alpha) \coloneqq \Set{z \in \C}{|z - \alpha| \leq r} \subset \C \backslash C
		\end{align}
		を満たす.ここで
		\begin{align}
			E \coloneqq f^{-1}\left( B_r(\alpha) \right),
			\quad E_n \coloneqq E \cap X_n
		\end{align}
		とおけば,任意の$n \geq 1$について$\mu(E_n) > 0$なら
		\begin{align}
			\left| \frac{1}{\mu(E_n)}\int_{E_n} f\ d\mu - \alpha \right|
			= \left| \frac{1}{\mu(E_n)}\int_{E_n} f - \alpha\ d\mu \right|
			\leq \frac{1}{\mu(E_n)}\int_{E_n} |f - \alpha|\ d\mu
			\leq r
		\end{align}
		となり(\refeq{eq:thm_mean_value_of_integral_and_closed_set})に反するから,
		$\mu(E_n) = 0\ (\forall n \geq 1)$及び
		\begin{align}
			\mu(E) = \mu\Biggl( \bigcup_{n=1}^\infty E_n \Biggr) 
			\leq \sum_{n=1}^\infty \mu(E_n) = 0
		\end{align}
		が従う.$\C \backslash C$は開集合であり$B_r(\alpha)$の形の集合の可算和で表せるから
		\begin{align}
			\mu\left( f^{-1}\left( \C \backslash C \right) \right) = 0
		\end{align}
		が成り立ち主張が得られる.
		\QED
	\end{prf}
	
	\begin{screen}
		\begin{thm}[可積分なら積分値を一様に小さくできる]\label{thm:integrable_intvalue_uniformly_shrinking}
			$(X,\mathscr{F},\mu)$を正値測度空間,$f:X \longrightarrow \C$
			を$\mathscr{F}/\borel{\C}$-可測関数とするとき,
			$f$が可積分なら,任意の$\epsilon > 0$に対して或る$\delta > 0$が存在し次を満たす:
			\begin{align}
				\mu(E) < \delta \quad \Longrightarrow \quad \int_E |f|\ d\mu < \epsilon.
			\end{align}
		\end{thm}
	\end{screen}
	
	\begin{prf}
		$X_n \coloneqq \{|f| \leq n\}$により増大列$(X_n)_{n=1}^\infty$を定めれば
		単調収束定理より
		\begin{align}
			\int_X |f|\ d\mu = \lim_{n \to \infty} \int_{X_n} |f|\ d\mu
		\end{align}
		となるから,任意の$\epsilon > 0$に対し或る$n_0 \geq 1$が存在して
		\begin{align}
			\int_{X \backslash X_{n_0}} |f|\ d\mu < \epsilon
		\end{align}
		が成り立つ.このとき$\mu(E) < \delta \coloneqq \epsilon/n_0$なら
		\begin{align}
			\int_E |f|\ d\mu
			= \int_{E \cap X_{n_0}} |f|\ d\mu + \int_{E \cap (X \backslash X_{n_0})} |f|\ d\mu
			\leq n_0 \mu(E) + \int_{X \backslash X_{n_0}} |f|\ d\mu
			< 2\epsilon
		\end{align}
		が従う.
		\QED
	\end{prf}
	
	\subsection{関数列の収束}
		\begin{screen}
			\begin{dfn}[概収束すれば測度収束する]
				$(X,\mathscr{F},\mu)$を正値有限測度空間とする.
				$(f_n)_{n=1}^\infty,f$を全て$\mathscr{F}/\borel{\C}$-可測関数とするとき,
				$\lim_{n \to \infty} f_n = f,\ \mbox{$\mu$-a.e.}$なら
				$(f_n)_{n=1}^\infty$は$f$に測度収束する.
			\end{dfn}
		\end{screen}
		
		\begin{prf}
			任意の$\epsilon > 0$に対し
			\begin{align}
				A^n_\epsilon \coloneqq \left\{ |f_n - f| > \epsilon \right\}
			\end{align}
			とおけば,Lebesgueの収束定理より任意の$k \geq 1$で
			\begin{align}
				\epsilon \mu\left(A^n_\epsilon\right)
				\leq \int_{A^n_\epsilon} |f_n - f| \wedge \epsilon\ d\mu
				\leq \int_{X} |f_n - f| \wedge \epsilon\ d\mu
				\longrightarrow 0
				\quad (n \longrightarrow \infty)
			\end{align}
			が成立する.
			\QED
		\end{prf}
		
		上の定理で有限性を外すときの反例を示す.
		$X = \R$,$\mu$を一次元Lebesgue測度とするとき,
		\begin{align}
			f_n \coloneqq \defunc_{\R \backslash (-n,n)}
		\end{align}
		で定める関数列$(f_n)_{n=1}^\infty$は零写像に各点収束するが,$0 < \epsilon < 1$に対し
		\begin{align}
			\mu\left( f_n > \epsilon \right) = \mu((-\infty,-n] \cup [n,\infty)) = \infty,
			\quad (\forall n \geq 1)
		\end{align}
		を満たすから測度収束しない.
		
		\begin{screen}
			\begin{thm}[測度収束列の概収束部分列]\label{thm:convergence_in_measure_then_convergence_almost_everywhere}
				$(X,\mathscr{F},\mu)$を正値測度空間,
				$(f_n)_{n=1}^\infty,f$を全て$\mathscr{F}/\borel{\C}$-可測関数とするとき,
				$(f_n)_{n=1}^\infty$が$f$に測度収束するなら
				或る部分列$(f_{n_k})_{k=1}^\infty$は$f$に概収束する.
			\end{thm}
		\end{screen}
		
		\begin{prf}
			$(f_n)_{n=1}^\infty$が$f$に測度収束するとき,任意の$k \geq 1$に対し
			\begin{align}
				\mu\left( |f_{n_k} - f| > \frac{1}{2^k}\right) < \frac{1}{2^k}
			\end{align}
			を満たす添数列$n_1 < n_2 < n_3 < \cdots$が取れる.
			\begin{align}
				A_k \coloneqq \left\{|f_{n_k} - f| > \frac{1}{2^k}\right\},
				\quad A \coloneqq \bigcup_{k\geq1} \bigcap_{j>k} A_j^c
			\end{align}
			とおけば,$\mu(A^c) \leq \mu\left(\bigcup_{j>k} A_j\right),\ (\forall k \geq 1)$かつ
			\begin{align}
				\mu\Biggl(\bigcup_{j>k} A_j\Biggr) \leq \sum_{j>k} \frac{1}{2^j}
				= \frac{1}{2^k}
			\end{align}
			より$\mu(A^c) = 0$が従い,$x \in A$なら或る$k = k(x)$が存在して
			\begin{align}
				|f_{n_j}(x) - f(x)| \leq \frac{1}{2^j}, \quad (\forall j > k)
			\end{align}
			となるから$\lim_{k \to \infty} f_{n_k}(x) = f(x)$が満たされる.
			\QED
		\end{prf}
		
		\begin{screen}
			\begin{thm}[平均収束すれば測度収束する]
				$p \in (0,\infty)$,$(X,\mathscr{F},\mu)$を正値測度空間,
				$(f_n)_{n=1}^\infty,f$を全て$\mathscr{F}/\borel{\C}$-可測関数とするとき,
				\begin{align}
					\int_X |f_n - f|^p\ d\mu \longrightarrow 0
					\quad (n \longrightarrow \infty)
				\end{align}
				なら$(f_n)_{n=1}^\infty$は$f$に測度収束する.
			\end{thm}
		\end{screen}
		
		\begin{prf}
			任意の$\epsilon > 0$に対し
			\begin{align}
				\epsilon^p \mu\left(|f_n - f| > \epsilon\right)
				\leq \int_{\left\{|f_n - f| > \epsilon\right\}} |f_n-f|^p\ d\mu
				\leq \int_X |f_n - f|^p\ d\mu \longrightarrow 0
				\quad (n \longrightarrow \infty)
			\end{align}
			が成立する.
			\QED
		\end{prf}
		
		\begin{screen}
			\begin{thm}[Egorov]
			\end{thm}
		\end{screen}
		
	\subsection{Radon測度}
		\begin{screen}
			\begin{thm}[Riesz-Markov-Kakutaniの表現定理]
			\end{thm}
		\end{screen}
		
		\begin{screen}
			\begin{thm}[正値Borel測度の正則性定理]
				
			\end{thm}
		\end{screen}
	
\section{Stieltjes積分}
	$\R$の左半開区間とは
	$(a,b],\ (-\infty \leq a \leq b \leq \infty)$を指す.ただし
	\begin{align}
		(a,b] =
		\begin{cases}
			\emptyset, & a=b, \\
			(-\infty,b], & a=-\infty,\ b < \infty, \\
			(a,\infty), & -\infty < a,\ b = \infty, \\
			(-\infty,\infty), & a=-\infty,\ b = \infty, \\
		\end{cases}
	\end{align}
	と考える.ここで$d \geq 1$に対し$\left(a_1,b_1\right] \times \left(a_2,b_2\right] \times
	\cdots \times \left(a_d,b_d\right]$の形の集合を$\R^d$の左半開区間として
	\begin{align}
		\mathfrak{F} \coloneqq \Set{\sum_{i=1}^n I_i}{I_i \subset \R^d:\mbox{左半開区間},\ n=1,2,\cdots}
	\end{align}
	とおけば,定理\ref{thm:forming_finitely_additive_class}より$\mathfrak{F}$は$\R^d$の上の加法族となる.
	$f_\lambda:\R \longrightarrow \R,\ (\lambda = 1,\cdots,d)$を単調非減少関数として,
	任意の左半開区間$I = I^1 \times \cdots \times I^d \subset \R^d$($I^\lambda$は$\R$の左半開区間)に対し
	\begin{align}
		m_0(I) \coloneqq \prod_{\lambda=1}^d 
		\sup{}{\Set{f_\lambda(\beta_\lambda) - f_\lambda(\alpha_\lambda)}{\left(\alpha_\lambda,\beta_\lambda\right] \subset I^\lambda}}
	\end{align}
	とおき,また$I = \emptyset$なら$m_0(I) \coloneqq 0$として
	\begin{align}
		\mu_0(F) \coloneqq \sum_{i=1}^n m_0(I_i),
		\quad (\forall F = I_1 + I_2 + \cdots + I_n \in \mathfrak{F})
	\end{align}
	により$\mu_0$を定める.この$\mu_0$はwell-definedであり,有限劣加法的かつ有限加法的な$m_0$の拡張である.実際,
	\begin{align}
		I_1 + I_2 + \cdots + I_n = J_1 + J_2 + \cdots + J_m
	\end{align}
	に対し,$\sum_{i=1}^n I_i = \sum_{i=1}^n \sum_{j=1}^m I_i \cap J_j = \sum_{j=1}^m J_i$かつ$I_i \cap J_j$は区間であるから
	\begin{align}
		\mu_0\Biggl(\sum_{i=1}^n I_i\Biggr)
		= \sum_{i=1}^n \sum_{j=1}^m m_0(I_i \cap J_j)
		= \mu_0\Biggl(\sum_{j=1}^m J_j\Biggr)
	\end{align}
	が成り立ち,また有限加法性は$\mu_0$の定義より従う.
	
	\begin{screen}
		\begin{thm}[右連続性と完全加法性]
			単調非減少関数$f_\lambda:\R \longrightarrow \R,\ (\lambda=1,\cdots,d)$を用いて定める$\mu_0$について,
			全ての$f_\lambda$が右連続であることと$\mu_0$が$\mathfrak{F}$の上で完全加法的であることは同値である.
		\end{thm}
	\end{screen}
	
	$\mathfrak{F}$は$\borel{\R^d}$を生成する.
	任意の$n \geq 1$に対して
	\begin{align}
		\mu_0((-n,n] \times \cdots \times (-n,n]) 
		= \prod_{\lambda=1}^d \left\{f_\lambda(n) - f_\lambda(-n)\right\} < \infty
	\end{align}
	であるから
	$\mu_0$は$\mathfrak{F}$上で$\sigma$-有限的であり,$f_\lambda$が右連続であれば
	定理\ref{thm:appendix_Kolmogorov_Hopf}より$\mu_0$の拡張測度$\mu$が唯一つ存在する.
	
	$\mathfrak{F}$が加法族であるから,空でない任意の区間$I \neq \emptyset$に対して
	\begin{align}
		\mathfrak{F}_I \coloneqq \Set{I \cap F}{F \in \mathfrak{F}}
	\end{align}
	は加法族をなす.$I$上右連続単調非減少な,
	ただし$I$が有界なら$I$上で有界な関数$f_I$に対し
	\begin{align}
		a_0 \coloneqq \inf{}{\Set{f(x)}{\inf{}{I} < x < \sup{}{I}}},
		\quad b_0 \coloneqq \sup{}{\Set{f(x)}{\inf{}{I} < x < \sup{}{I}}}
	\end{align}
	とすれば,$\inf{}{I} \in I$なら$a_0 = f(\inf{}{I})$,
	$\sup{}{I} \in I$なら$b_0 = f(\sup{}{I})$であり,
	\begin{align}
		f(x) \coloneqq 
		\begin{cases}
			a_0 & -\infty < x \leq \inf{}{I} \\
			f_I(x) & \inf{}{I} < x < \sup{}{I} \\
			b_0 & \sup{}{I} \leq x < \infty
		\end{cases}
	\end{align}
	により$f_I$を$f$に拡張して$\mu_0$を定めるとき,
	\begin{align}
		\mu_{0,I}(I \cap F) \coloneqq \mu_0(I \cap F)
	\end{align}
	は$\mathfrak{F}_I$上で完全加法的となる.
	$\mu_{0,I}$の拡張測度を$\mu_I$と書き,これを$f_I$のStieltjes測度と呼ぶ.
	$\mu_I$のLebesgue拡大をLebesgue-Stieltjes測度と呼び,
	特に$I = \R,\ f(x) = x$のときLebesgue測度と呼ぶ.
	
	\begin{screen}
		\begin{thm}[左半開区間のStiletjes測度]
			$(\alpha,\beta] \subset I,\ (-\infty < \alpha < \beta < \infty)$に対して
			\begin{align}
				\mu((\alpha,\beta]) = f(\beta) - f(\alpha).
			\end{align}
		\end{thm}
	\end{screen}
	
	\begin{screen}
		\begin{thm}[Riemann-Stieltjes積分との関係]
			$F:I \longrightarrow \C$が右連続或は左連続なら
		\end{thm}
	\end{screen}
	
	\begin{screen}
		\begin{thm}[時間変更]
			
		\end{thm}
	\end{screen}
\section{Fubiniの定理}
	$(X,\mathcal{M}),(Y,\mathcal{N})$を可測空間とするとき,
	任意の$x \in X$に対し
	\begin{align}
		p_x:Y \ni y \longmapsto (x,y) \in X \times Y
	\end{align}
	で定める$p_x$は$\mathcal{N}/\mathcal{M} \otimes \mathcal{N}$-可測である.
	実際,$A \in \mathcal{M},\ B \in \mathcal{N}$に対しては
	\begin{align}
		p_x^{-1}(A \times B) = 
		\begin{cases}
			\emptyset, & (x \notin A), \\
			B, & (x \in A),
		\end{cases}
		\in \mathcal{N}
	\end{align}
	となるから,
	\begin{align}
		\Set{A \times B}{A \in \mathcal{M},\ B \in \mathcal{N}}
		\subset \Set{E \in \mathcal{M} \otimes \mathcal{N}}{p_x^{-1}(E) \in \mathcal{N}}
	\end{align}
	が従い$p_x$の$\mathcal{N}/\mathcal{M} \otimes \mathcal{N}$-可測性が出る.
	同様に任意の$y \in Y$に対し
	\begin{align}
		q_y:X \ni x \longmapsto (x,y) \in X \times Y
	\end{align}
	で定める$q_y$は$\mathcal{M}/\mathcal{M} \otimes \mathcal{N}$-可測である.
	
	\begin{screen}
		\begin{lem}[二変数可測写像は片変数で可測]\label{lem:Fubini_lemma_1}
			$(X,\mathcal{M}),(Y,\mathcal{N}),(Z,\mathcal{L})$を可測空間とするとき,
			写像$f: X \times Y \longmapsto Z$が
			$\mathcal{M}\otimes \mathcal{N}/ \mathcal{L}$-可測であれば,
			任意の$x_0 \in X,\ y_0 \in Y$に対し
			\begin{align}
				X \ni x \longmapsto f(x,y_0),
				\quad Y \ni y \longmapsto f(x_0,y)
			\end{align}
			はそれぞれ$\mathcal{M}/\mathcal{L}$-可測,
			$\mathcal{N}/\mathcal{L}$-可測である.
		\end{lem}
	\end{screen}
	
	\begin{prf}
		$X \ni x \longmapsto f(x,y_0)$は$f$と$q_{y_0}$の合成$f \circ q_{y_0}$であり,
		$Y \ni y \longmapsto f(x_0,y)$は$f \circ p_{x_0}$である.
		\QED
	\end{prf}
	
	\begin{screen}
		\begin{lem}\label{lem:Fubini_theorem}
			$(X,\mathcal{M},\mu),(Y,\mathcal{N},\nu)$を$\sigma$-有限な測度空間とするとき,
			任意の$Q \in \mathcal{M} \otimes \mathcal{N}$に対し
			\begin{align}
				\varphi_Q: X \ni x \longmapsto \int_Y \defunc_{Q}\circ p_x\ d\nu,
				\quad \psi_Q: Y \ni y \longmapsto \int_X \defunc_{Q} \circ q_y\ d\mu,
			\end{align}
			はそれぞれ$\mathcal{M}/\borel{[0,\infty]}$-可測,
			$\mathcal{N}/\borel{[0,\infty]}$-可測であり
			\begin{align}
				\int_X \varphi_Q\ d\mu
				= (\mu \otimes \nu)(Q)
				= \int_Y \psi_Q\ d\nu
				\label{eq:lem_Fubini_theorem_1}
			\end{align}
			が成立する.
		\end{lem}
	\end{screen}
	
	\begin{prf}\mbox{}
		\begin{description}
			\item[第一段]
				$\sigma$-有限の仮定より,
				\begin{align}
					\bigcup_{n=1}^\infty X_n = X,
					\quad \bigcup_{n=1}^\infty Y_n = Y,
					\quad \mu(X_n),\ \nu(Y_n) < \infty;
					\ n = 1,2,\cdots
				\end{align}
				を満たす増大列$\{X_n\}_{n=1}^\infty \subset \mathcal{M}$と
				$\{Y_n\}_{n=1}^\infty \subset \mathcal{N}$が存在する.ここで
				\begin{align}
					\mathcal{M}_n \coloneqq \Set{A \cap X_n}{A \in \mathcal{M}},
					\quad \mathcal{N}_n \coloneqq \Set{B \cap Y_n}{B \in \mathcal{N}}
				\end{align}
				により$X_n,Y_n$上の$\sigma$-加法族を定めて
				\begin{align}
					\mathcal{D}_n \coloneqq
					\Set{Q_n \in \mathcal{M}_n \otimes \mathcal{N}_n}{
					\substack{
					\displaystyle \varphi_{Q_n}: X \ni x \longmapsto \int_Y \defunc_{Q_n} \circ p_x\ d\nu \mbox{ が$\mathcal{M}/\borel{[0,\infty]}$-可測},\\
					\displaystyle \psi_{Q_n}: Y \ni y \longmapsto \int_X \defunc_{Q_n} \circ q_y\ d\mu \mbox{ が$\mathcal{N}/\borel{[0,\infty]}$-可測},\\
					\displaystyle \int_X \varphi_{Q_n}\ d\mu
					= (\mu \otimes \nu)(Q_n)
					= \int_Y \psi_{Q_n}\ d\nu}} 
				\end{align}
				とおけば,$\mathcal{D}_n$は$X_n \times Y_n$上のDynkin族であり
				\begin{align}
					\Set{A \times B}{A \in \mathcal{M}_n,\ B \in \mathcal{N}_n}
					\subset \mathcal{D}_n
				\end{align}
				を満たすから$\mathcal{M}_n \otimes \mathcal{N}_n = \mathcal{D}_n$が従う.
			
			\item[第二段]
				$\mathcal{M}_n \otimes \mathcal{N}_n = \Set{Q \cap (X_n \times Y_n)}{Q \in \mathcal{M} \otimes \mathcal{N}}$より,任意の$Q \in \mathcal{M} \otimes \mathcal{N}$に対して
				\begin{align}
					Q_n \coloneqq Q \cap (X_n \times Y_n) \in \mathcal{D}_n,
					\ (\forall n \geq 1),
					\quad Q_1 \subset Q_2 \subset \cdots \longrightarrow Q
				\end{align}
				が従い,単調収束定理より
				\begin{align}
					\varphi_Q(x) = \int_Y \defunc_Q \circ p_x\ d\nu
					= \lim_{n \to \infty} \int_Y \defunc_{Q_n} \circ p_x\ d\nu
					= \lim_{n \to \infty} \varphi_{Q_n}(x),
					\quad (\forall x \in X)
				\end{align}
				となるから$\varphi_Q$の$\mathcal{M}/\borel{[0,\infty]}$-可測性が出る.
				また,
				\begin{align}
					\varphi_{Q_n}(x) = \int_Y \defunc_{Q_n} \circ p_x\ d\nu
					\leq \int_Y \defunc_{Q_{n+1}} \circ p_x\ d\nu
					= \varphi_{Q_{n+1}}(x),
					\quad (n=1,2,\cdots)
				\end{align}
				が満たされているから,再び単調収束定理により
				\begin{align}
					\int_X \varphi_Q\ d\mu
					= \lim_{n \to \infty} \int_X \varphi_{Q_n}\ d\mu
					= \lim_{n \to \infty} (\mu \otimes \nu)(Q_n)
					= (\mu \otimes \nu)(Q)
				\end{align}
				が得られる.同様に$\psi_Q$は$\mathcal{N}/\borel{[0,\infty]}$-可測であり
				(\refeq{eq:lem_Fubini_theorem_1})を満たす.
				\QED
		\end{description}
	\end{prf}
	
	\begin{screen}
		\begin{thm}[Fubini]
			$(X,\mathcal{M},\mu),(Y,\mathcal{N},\nu)$を$\sigma$-有限な測度空間とする.
			\begin{description}
				\item[(1)]
					$f:X \times Y \longrightarrow [0,\infty]$を
					$\mathcal{M} \otimes \mathcal{N}/\borel{[0,\infty]}$-可測写像とするとき,
					\begin{align}
						\varphi: X \ni x \longmapsto \int_Y f \circ p_x\ d\nu,
						\quad \psi: Y \ni y \longmapsto \int_X f \circ q_y\ d\mu
					\end{align}
					により定める$\varphi,\psi$はそれぞれ$\mathcal{M}/\borel{[0,\infty]}$-可測,
					$\mathcal{N}/\borel{[0,\infty]}$-可測であり,
					\begin{align}
						\int_X \varphi\ d\mu
						= \int_{X \times Y} f\ d(\mu \otimes \nu)
						= \int_Y \psi\ d\nu
					\end{align}
					が成立する.
					
				\item[(2)]
					$f:X \times Y \longrightarrow \C$を
					$\mathcal{M} \otimes \mathcal{N}/\borel{\C}$-可測な
					可積分関数とするとき,
			\end{description}
		\end{thm}
	\end{screen}
	
	\begin{screen}
		\begin{thm}[$n$変数関数のFubiniの定理]
			$\left((X_i,\mathcal{M}_i,\mu_i)\right)_{i=1}^n,\ (n \geq 3)$を
			$\sigma$-有限な測度空間の族とし,
			\begin{align}
				\{i_1,\cdots,i_k\} \cup \{j_1,\cdots,j_h\} = \{1,2,\cdots,n\},
				\quad \{i_1,\cdots,i_k\} \cap \{j_1,\cdots,j_h\} = \emptyset
			\end{align}
			を満たす添数列$i_1, \cdots, i_k$と$j_1, \cdots, j_h,\ (1 \leq k,h \leq n-1)$を任意に取り
			\begin{align}
				&Y \coloneqq \prod_{i=1}^n X_i,
				\quad Y_1 \coloneqq \prod_{\ell=1}^k X_{i_\ell},
				\quad Y_2 \coloneqq \prod_{\ell=1}^h X_{j_\ell}, \\
				&\mathcal{N} \coloneqq \bigotimes_{i=1}^n \mathcal{M}_i,
				\quad \mathcal{N}_1 \coloneqq \bigotimes_{\ell=1}^k \mathcal{M}_{i_\ell},
				\quad \mathcal{N}_2 \coloneqq \bigotimes_{\ell=1}^h \mathcal{M}_{j_\ell}, \\
				&\mu \coloneqq \bigotimes_{i=1}^n \mu_i,
				\quad \nu_1 \coloneqq \bigotimes_{\ell=1}^k \mu_{i_\ell},
				\quad \nu_2 \coloneqq \bigotimes_{\ell=1}^h \mu_{j_\ell}
			\end{align}
			とおく.また
			\begin{align}
				p_{y_1}:Y_2 \ni y_2 \longmapsto (y_1,y_2),\ (\forall y_1 \in Y_1),
				\quad q_{y_2}:Y_1 \ni y_1 \longmapsto (y_1,y_2),\ (\forall y_2 \in Y_2)
			\end{align}
			とする.このとき,射影$\pi_1:Y \longrightarrow Y_1,\ \pi_2:Y \longrightarrow Y_2$に対し
			\begin{align}
				\varphi: Y_1 \times Y_2 \ni (y_1,y_2) \longmapsto \pi_1^{-1}(y_1) \cap \pi_2^{-1}(y_2)
			\end{align}
			により$\varphi:Y_1 \times Y_2 \longrightarrow Y$を定めれば
			$\varphi$は$\mathcal{N}_1 \otimes \mathcal{N}_2/\mathcal{N}$-可測であり,
			更に以下が成立する:
			\begin{description}
				\item[(1)] $f:Y \longrightarrow [0,\infty]$が$\mathcal{N}/\borel{[0,\infty]}$-可測なら次が成り立つ:
					\begin{align}
						\int_Y f\ d\mu
						= \int_{Y_1} \int_{Y_2} f \left(\varphi\left(p_{y_1}(y_2)\right)\right)\ \nu_2(dy_2)\ \nu_1(dy_1)
						= \int_{Y_2} \int_{Y_1} f \left(\varphi\left(q_{y_2}(y_1)\right)\right)\ \nu_1(dy_1)\ \nu_2(dy_2).
					\end{align}
			\end{description}
		\end{thm}
	\end{screen}
	
	\begin{prf}\mbox{}
		\begin{description}
			\item[第一段]
				$\varphi$の$\mathcal{N}_1 \otimes \mathcal{N}_2/\mathcal{N}$-可測性を示す.
				実際,$\varphi:Y_1 \times Y_2 \longrightarrow Y$が全単射であることより
				\begin{align}
					\varphi^{-1}(E_1 \times \cdots \times E_n)
					= \prod_{\ell=1}^k E_{i_\ell} \times \prod_{\ell=1}^h E_{j_\ell}
					\in \mathcal{N}_1 \otimes \mathcal{N}_2,
					\quad (\forall E_i \in \mathcal{M}_i,\ i=1,\cdots,n)
					\label{eq:Fubini_theorem_n_variables_1}
				\end{align}
				が成り立つから
				\begin{align}
					\Set{E_1 \times \cdots \times E_n}{E_i \in \mathcal{M}_i,\ i=1,\cdots,n}
					\subset \Set{E \in \mathcal{N}}{\varphi^{-1}(E) \in \mathcal{N}_1 \otimes \mathcal{N}_2}
				\end{align}
				となり,左辺は$\mathcal{N}$を生成するから$\varphi$は$\mathcal{N}_1 \otimes \mathcal{N}_2/\mathcal{N}$-可測である.
				
			\item[第二段]
				$f = \defunc_E\ (E \in \mathcal{N})$に対し
				\begin{align}
					\int_Y f\ d\mu 
					= \int_{Y_1 \times Y_2} f \circ \varphi\ d(\nu_1 \otimes \nu_2)
				\end{align}
				となることを示す.実際,(\refeq{eq:Fubini_theorem_n_variables_1})より
				\begin{align}
					\Set{E_1 \times \cdots \times E_n}{E_i \in \mathcal{M}_i,\ i=1,\cdots,n}
					\subset \Set{E \in \mathcal{N}}{\mu(E) = \nu_1 \otimes \nu_2\left( \varphi^{-1}(E) \right)}
				\end{align}
				となるから,Dinkin族定理より任意の$E \in \mathcal{N}$に対して
				$\mu(E) = \nu_1 \otimes \nu_2\left( \varphi^{-1}(E) \right)$が成立し
				\begin{align}
					\int_Y f\ d\mu
					= \mu(E)
					= \nu_1 \otimes \nu_2\left( \varphi^{-1}(E) \right)
					= \int_{Y_1 \times Y_2} f \circ \varphi\ d(\nu_1 \otimes \nu_2)
				\end{align}
				が従う.
		\end{description}
	\end{prf}
\section{$L^p$空間}

測度空間を$(X,\mathscr{F},\mu)$とする.$\mathscr{F}/\borel{\C}$-可測関数$f$に対して
\begin{align}
	\Norm{f}{\mathscr{L}^p} \coloneqq
	\begin{cases}
		\inf{}{\Set{r \in \C}{|f(x)| \leq r\quad \mbox{$\mu$-a.e.}x \in X}} & (p = \infty) \\
		\displaystyle\left( \int_{X} |f(x)|^p\ \mu(dx) \right)^{1/p} & (0 < p < \infty)
	\end{cases}
\end{align}
により$\Norm{\cdot}{\mathscr{L}^p}$を定め,
\begin{align}
	\mathscr{L}^p(X,\mathscr{F},\mu) \coloneqq \Set{f:X \rightarrow \C}{f:\mbox{可測}\mathscr{F}/\borel{\C},\ \Norm{f}{\mathscr{L}^p} < \infty} \quad (1 \leq p \leq \infty)
\end{align}
で空間$\mathscr{L}^p(X,\mathscr{F},\mu)$を定義する.$\mathscr{L}^p(\mu)$とも略記する.

\begin{screen}
	\begin{lem}\label{lem:holder_inequality}
		任意の$f \in \mathscr{L}^\infty(X,\mathscr{F},\mu)$に対して次が成り立つ:
		\begin{align}
			|f| \leq \Norm{f}{\mathscr{L}^\infty} \quad \mbox{$\mu$-a.e.}
		\end{align}
	\end{lem}
\end{screen}

\begin{prf}
	$\mathscr{L}^\infty(X,\mathscr{F},\mu)$の定義より任意の実数$\alpha > \Norm{f}{\mathscr{L}^\infty}$に対して
	\begin{align}
		\mu\left( \Set{x \in X}{|f(x)| > \alpha} \right) = 0
	\end{align}
	が成り立つから,
	\begin{align}
		\Set{x \in X}{|f(x)| > \Norm{f}{\mathscr{L}^\infty}} = \bigcup_{n =1}^{\infty} \Set{x \in X}{|f(x)| > \Norm{f}{\mathscr{L}^\infty} + \frac{1}{n}}
	\end{align}
	の右辺は$\mu$-零集合であり主張が従う.
	\QED
\end{prf}

\begin{screen}
	\begin{thm}[H\Ddot{o}lderの不等式]\label{thm:holder_inequality}
		$1 \leq p, q \leq \infty$,$p + q = pq\ (p = \infty$なら$q = 1)$とする.このとき
		任意の$\mathscr{F}/\borel{\C}$-可測関数$f,g$に対して次が成り立つ:
		\begin{align}
			\int_{X} |fg|\ d\mu \leq \Norm{f}{\mathscr{L}^p} \Norm{g}{\mathscr{L}^q}. \label{ineq:holder}
		\end{align}
	\end{thm}
\end{screen}

\begin{prf}
	$\Norm{f}{\mathscr{L}^p} = \infty$又は$\Norm{g}{\mathscr{L}^q} = \infty$なら(\refeq{ineq:holder})
		は成り立つから,$\Norm{f}{\mathscr{L}^p} < \infty$かつ$\Norm{g}{\mathscr{L}^q} < \infty$とする.
	\begin{description}
		\item[$p = \infty,\ q = 1$の場合]
			補題\ref{lem:holder_inequality}により或る零集合$A$が存在して
			\begin{align}
				|f(x)g(x)| \leq \Norm{f}{\mathscr{L}^\infty}|g(x)| \quad (\forall x \in X \backslash A).
			\end{align}
			が成り立つから,
			\begin{align}
				\int_{X} |fg|\ d\mu = \int_{X \backslash A} |fg|\ d\mu
				\leq \Norm{f}{\mathscr{L}^\infty} \int_{X \backslash A} |g|\ d\mu 
				= \Norm{f}{\mathscr{L}^\infty} \Norm{g}{\mathscr{L}^1}
			\end{align}
			が従い不等式(\refeq{ineq:holder})を得る.
		
		\item[$1 < p,q < \infty$の場合]
			$\Norm{f}{\mathscr{L}^p} = 0$のとき
			\begin{align}
				B \coloneqq \Set{x \in X}{|f(x)| > 0}
			\end{align}
			は零集合であるから,
			\begin{align}
				\int_{X} |fg|\ d\mu = \int_{X \backslash B} |fg|\ d\mu = 0
			\end{align}
			となり(\refeq{ineq:holder})を得る.$\Norm{g}{\mathscr{L}^q} = 0$の場合も同じである.
			次に$0 < \Norm{f}{\mathscr{L}^p},\Norm{g}{\mathscr{L}^q} < \infty$の場合を示す.
			実数値対数関数$(0,\infty) \ni t \longmapsto -\Log{t}$は凸であるから,$1/p + 1/q = 1$に対して
			\begin{align}
				-\Log{\left( \frac{s}{p} + \frac{t}{q} \right)} \leq \frac{1}{p}(-\Log{s}) + \frac{1}{q}(-\Log{t}) \quad (\forall s,t > 0)
			\end{align}
			を満たし
			\begin{align}
				s^{1/p}t^{1/q} \leq \frac{s}{p} + \frac{t}{q} \quad (\forall s,t > 0)
			\end{align}
			が従う.ここで
			\begin{align}
				F \coloneqq \frac{|f|^p}{\Norm{f}{\mathscr{L}^p}^p},
				\quad G \coloneqq \frac{|g|^q}{\Norm{g}{\mathscr{L}^q}^q}
			\end{align}
			により可積分関数$F,G$を定めれば,
			\begin{align}
				F(x)^{1/p}G(x)^{1/q} \leq \frac{1}{p}F(x) + \frac{1}{q}G(x) \quad (\forall x \in X)
			\end{align}
			が成り立つから
			\begin{align}
				\frac{1}{\Norm{f}{\mathscr{L}^p}\Norm{g}{\mathscr{L}^q}}\int_{X} |fg|\ d\mu
				= \int_{X} F^{1/p}G^{1/q}\ d\mu
				\leq \frac{1}{p} \int_{X} F\ d\mu + \frac{1}{q} \int_{X} G\ d\mu
				= \frac{1}{p} + \frac{1}{q} = 1
			\end{align}
			が従い,$\Norm{f}{\mathscr{L}^p}\Norm{g}{\mathscr{L}^q}$を移項して不等式(\refeq{ineq:holder})を得る.
			\QED
	\end{description}
\end{prf}

\begin{screen}
	\begin{thm}[Minkowskiの不等式]\label{thm:minkowski_inequality}
		$1 \leq p \leq \infty$のとき,
		任意の$\mathscr{F}/\borel{\C}$-可測関数$f,g$に対して次が成り立つ:
		\begin{align}
			\Norm{f+g}{\mathscr{L}^p} \leq \Norm{f}{\mathscr{L}^p} + \Norm{g}{\mathscr{L}^p}. \label{ineq:minkowski}
		\end{align}
	\end{thm}
\end{screen}

\begin{prf}
	$\Norm{f+g}{\mathscr{L}^p} = 0,\ \Norm{f}{\mathscr{L}^p} = \infty,\ \Norm{g}{\mathscr{L}^p} = \infty$
	のいずれかが満たされていれば(\refeq{ineq:minkowski})は成り立つから,
	$\Norm{f+g}{\mathscr{L}^p} > 0$かつ$\Norm{f}{\mathscr{L}^p} < \infty$かつ$\Norm{g}{\mathscr{L}^p} < \infty$
	の場合を考える.
	\begin{description}
		\item[$p = \infty$の場合]
			補題\ref{lem:holder_inequality}により
			\begin{align}
				C \coloneqq \Set{x \in X}{|f(x)| > \Norm{f}{\mathscr{L}^\infty}} \cup \Set{x \in X}{|g(x)| > \Norm{g}{\mathscr{L}^\infty}}
			\end{align}
			は零集合であり,
			\begin{align}
				|f(x) + g(x)| \leq |f(x)| + |g(x)| \leq \Norm{f}{\mathscr{L}^\infty} + \Norm{g}{\mathscr{L}^\infty} \quad (\forall x \in X \backslash C)
			\end{align}
			が成り立ち(\refeq{ineq:minkowski})が従う.
		
		\item[$p = 1$の場合]
			\begin{align}
				\int_X |f + g|\ d\mu \leq \int_X |f| + |g|\ d\mu = \Norm{f}{\mathscr{L}^1} + \Norm{g}{\mathscr{L}^1}
			\end{align}
			より(\refeq{ineq:minkowski})が従う.
		
		\item[$1 < p < \infty$の場合]
			$q$を$p$の共役指数とする.
			\begin{align}
				|f+g|^p = |f+g||f+g|^{p-1} \leq |f||f+g|^{p-1} + |g||f+g|^{p-1}
			\end{align}
			が成り立つから,H\Ddot{o}lderの不等式より
			\begin{align}
				\Norm{f+g}{\mathscr{L}^p}^p &= \int_{X} |f+ g|^p\ d\mu \\
				&\leq \int_{X} |f||f+g|^{p-1}\ d\mu + \int_{X} |g||f+g|^{p-1}\ d\mu \\
				&\leq \Norm{f}{\mathscr{L}^p}\Norm{f+g}{\mathscr{L}^p}^{p-1} + \Norm{g}{\mathscr{L}^p}\Norm{f+g}{\mathscr{L}^p}^{p-1}
				\label{Minkowski_1}
			\end{align}
			が得られる.また$|f|^p,|g|^p$の可積性と
			\begin{align}
				|f + g|^p \leq 2^p \left( |f|^p + |g|^p \right)
			\end{align}
			により$\Norm{f+g}{\mathscr{L}^p} < \infty$が従うから,
			(\refeq{Minkowski_1})の両辺を$\Norm{f+g}{\mathscr{L}^p}^{p-1}$で割って(\refeq{ineq:minkowski})を得る.
			\QED
	\end{description}
\end{prf}

以上の結果より$\mathscr{L}^p(X,\mathscr{F},\mu)$は線形空間となる.実際線型演算は
\begin{align}
	(f+g)(x) \coloneqq f(x) + g(x), \quad (\alpha f)(x) \coloneqq \alpha f(x),
	\quad (\forall x \in X,\ f,g \in \mathscr{L}^p(\mu),\ \alpha \in \C)
\end{align}
により定義され,Minkowskiの不等式により加法について閉じている.

\begin{screen}
	\begin{lem}
		$1 \leq p \leq \infty$に対し,$\Norm{\cdot}{\mathscr{L}^p}$は線形空間$\mathscr{L}^p(X,\mathscr{F},\mu)$のセミノルムである.
	\end{lem}
\end{screen}

\begin{prf}\mbox{}
	\begin{description}
	\item[半正値性] $\Norm{\cdot}{\mathscr{L}^p}$が正値であることは定義による.
		一方で,$E \neq \emptyset$を満たす$\mu$-零集合$E$が存在するとき,
		\begin{align}
			f(x) \coloneqq
			\begin{cases}
				1 & (x \in E) \\
				0 & (x \in \Omega \backslash E)
			\end{cases}
		\end{align}
		で定める$f$は零写像ではないが$\Norm{f}{\mathscr{L}^p} = 0$となる.
		
	\item[同次性] 
		任意に$\alpha \in \C,\ f \in \mathscr{L}^p(\mu)$を取る.
		$1 \leq p < \infty$の場合は
		\begin{align}
			\left( \int_{X} |\alpha f|^p\ d\mu \right)^{1/p} = \left( |\alpha|^p \int_{X} |f|^p\ d\mu \right)^{1/p} 
			= |\alpha| \left( \int_{X} |f|^p\ d\mu \right)^{1/p}
		\end{align}
		により,$p = \infty$の場合は
		\begin{align}
			\inf{}{\Set{r \in \R}{|\alpha f(x)| \leq r \quad \mbox{$\mu$-a.e.}x \in X}} = |\alpha|\inf{}{\Set{r \in \R}{|f(x)|  \leq r \quad \mbox{$\mu$-a.e.}x \in X}}
		\end{align}
		により$\Norm{\alpha f}{\mathscr{L}^p} = |\alpha|\Norm{f}{\mathscr{L}^p}$が成り立つ.
		
	\item[三角不等式] Minkowskiの不等式より従う.
	\QED
	\end{description}
\end{prf}

$\mathscr{L}^p$はノルム空間ではないが,同値類でまとめることによりノルム空間となる.
\begin{description}
	\item[可測関数全体の商集合]
		$\mathscr{F}/\borel{\C}$-可測関数全体の集合を
		\begin{align}
			\mathscr{L}^0(X,\mathscr{F},\mu) \coloneqq \Set{f:X \rightarrow \C}{f:\mbox{可測}\mathscr{F}/\borel{\C}}
		\end{align}
		とおく.$f,g \in \mathscr{L}^0(X,\mathscr{F},\mu)$に対し
		\begin{align}
			 f \sim g \quad \overset{\mathrm{def}}{\Longleftrightarrow} \quad f = g \quad \mbox{$\mu$-a.e.}
		\end{align}
		により定める$\sim$は同値関係であり,$\sim$による$\mathscr{L}^0(X,\mathscr{F},\mu)$の商集合を
		$L^0(X,\mathscr{F},\mu)$と表す.
	
	\item[商集合における算法]
		$L^0(\mu)$の元である関数類(同値類)を$[f]\ $($f$は関数類の代表)と表せば,$L^0(\mu)$は
		\begin{align}
			[f] + [g] \coloneqq [f+g],
			\quad \alpha [f] \coloneqq [\alpha f], \quad ([f],[g] \in L^0(\mu),\ \alpha \in \C).
		\end{align}
		を線型演算として$\C$上の線形空間となる.また
		\begin{align}
			[f][g] \coloneqq [fg] \quad \left([f],[g] \in L^{0}(\mu) \right).
		\end{align}
		を乗法として$L^0(\mu)$は環となる.$L^0(\mu)$の零元は零写像の関数類でありこれを[0]と書く.また
		単位元は恒等的に$1$を取る関数の関数類でありこれを[1]と書く.
		減法は
		\begin{align}
			[f] - [g] \coloneqq [f] + (-[g]) = [f] + [-g] = [f - g]
		\end{align}
		により定める.
	
	\item[関数類の順序]
		$[f],[g] \in L^0(\mu)$に対して次の関係$<(>)$を定める:
		\begin{align}
			[f] < [g]\ \left( [g] > [f] \right) \quad
			\overset{\mathrm{def}}{\Longleftrightarrow}
			\quad f < g \quad \mbox{$\mu$-a.s.} \label{dfn:equiv_class_order}
		\end{align}
		この定義はwell-definedである.実際任意の$f' \in [f],g' \in [g]$に対して
		\begin{align}
			\left\{ f' \geq g' \right\} \subset \left\{ f \neq f' \right\} \cup \left\{ f \geq g \right\} \cup \left\{ g \neq g' \right\}
		\end{align}
		の右辺は零集合であるから
		\begin{align}
			[f] < [g] \Leftrightarrow [f'] < [g']
		\end{align}
		が従う.$<(>)$または$=$であることを$\leq(\geq)$と書くとき,任意の$[f],[g],[h] \in L^0(\mu)$に対し,
		\begin{itemize}
			\item $[f] \leq [f]$が成り立つ.
			\item $[f] \leq [g]$かつ$[g] \leq [f]$ならば$[f] = [g]$が成り立つ.
			\item $[f] \leq [g],\ [g] \leq [h]$ならば$[f] \leq [h]$が成り立つ.
		\end{itemize}
		が満たされるから$\leq$は$L^0(\mu)$における順序となる.
\end{description}

\begin{screen}
	\begin{dfn}[商空間におけるノルムの定義]
		\begin{align}
			\Norm{[f]}{L^p} \coloneqq \Norm{f}{\mathscr{L}^p} 
			\quad (f \in \mathscr{L}^p(X,\mathscr{F},\mu),\ 1 \leq p \leq \infty)
		\end{align}
		により定める$\Norm{\cdot}{L^p}:L^0(X,\mathscr{F},\mu) \rightarrow \R$は関数類の代表に依らずに値が確定する.
		そして
		\begin{align}
			L^p(X,\mathscr{F},\mu) \coloneqq \Set{[f] \in L^0(X,\mathscr{F},\mu)}{\Norm{[f]}{L^p} < \infty} \quad (1 \leq p \leq \infty)
		\end{align}
		として定める空間は$\Norm{\cdot}{L^p}$をノルムとしてノルム空間となる.
	\end{dfn}
\end{screen}

\begin{screen}
	\begin{thm}[$L^p$はBanach空間]\label{thm:Lp_banach}
		ノルム空間$L^p(X,\mathscr{F},\mu)\ (1 \leq p \leq \infty)$の任意のCauchy列$\left( [f_n] \right)_{n=1}^\infty$
		に対してノルム収束極限$[f] \in L^p(\mu)$が存在する.
		また,このとき或る部分列$\left( \left[f_{n_k}\right] \right)_{k=1}^\infty$の代表は
		$f$に概収束する:
		\begin{align}
			\lim_{k \to \infty} f_{n_k} = f, \quad \mbox{$\mu$-a.e.}
		\end{align}
	\end{thm}
\end{screen}

\begin{prf}
	任意にCauchy列$[f_n] \in L^p(\mu)\ (n=1,2,3,\cdots)$を取れば,
	或る$N_1 \in \N$が存在して
	\begin{align}
		\Norm{[f_n]-[f_m]}{L^p} < \frac{1}{2}
		\quad (\forall n > m \geq N_1)
	\end{align}
	を満たす.ここで$m > N_1$を一つ選び$n_1$とおく.
	同様に$N_2 > N_1$を満たす$N_2 \in \N$が存在して
	\begin{align}
		\Norm{[f_n]-[f_m]}{L^p} < \frac{1}{2^2}
		\quad (\forall n > m \geq N_2)
	\end{align}
	を満たすから,$m > N_2$を一つ選び$n_2$とおけば
	\begin{align}
		\Norm{\left[f_{n_1}\right] - \left[f_{n_2}\right]}{L^p} < \frac{1}{2}
	\end{align}
	が成り立つ.同様の操作を繰り返して
	\begin{align}
		\Norm{\left[f_{n_k}\right] - \left[f_{n_{k+1}}\right]}{L^p} < \frac{1}{2^k} 
		\quad (n_k < n_{k+1},\ k=1,2,3,\cdots) \label{ineq:Lp_banach_2}
	\end{align}
	を満たす部分添数列$(n_k)_{k=1}^{\infty}$を構成する.
	\begin{description}
		\item[$p = \infty$の場合]
			$\left[f_{n_k}\right]$の代表$f_{n_k}\ (k=1,2,\cdots)$に対して
			\begin{align}
				A_k &\coloneqq \Set{x \in X}{\left| f_{n_k}(x) \right| > \Norm{f_{n_k}}{\mathscr{L}^\infty}}, \\
				A^k &\coloneqq \Set{x \in X}{\left| f_{n_k}(x) - f_{n_{k+1}}(x) \right| > \Norm{f_{n_k} - f_{n_{k+1}}}{\mathscr{L}^\infty}}
			\end{align}
			とおけば,補題\ref{lem:holder_inequality}より$\mu(A_k) = \mu(A^k) = 0\ (k=1,2,\cdots)$が成り立つ.
			\begin{align}
				A_\circ \coloneqq \bigcup_{k=1}^{\infty} A_k,
				\quad A^\circ \coloneqq \bigcup_{k=1}^{\infty}A^k,
				\quad A \coloneqq A_\circ \cup A^\circ
			\end{align}
			として$\mu$-零集合$A$を定めて
			\begin{align}
				\hat{f}_{n_k} \coloneqq f_{n_k} \defunc_{X \backslash A}
				\quad (\forall k=1,2,\cdots)
			\end{align}
			とおけば
			各$\hat{f}_{n_k}$は$\left[\hat{f}_{n_k}\right] = \left[f_{n_k}\right]$を満たす有界可測関数であり,
			(\refeq{ineq:Lp_banach_2})より
			\begin{align}
				\sup{x \in X}{\left|\hat{f}_{n_k}(x) - \hat{f}_{n_{k+1}}(x)\right|}
				\leq \Norm{\hat{f}_{n_k} - \hat{f}_{n_{k+1}}}{\mathscr{L}^\infty} < \frac{1}{2^k} \quad (k=1,2,3,\cdots) 
				\label{ineq:Lp_banach_1}
			\end{align}
			が成り立つ.
			このとき任意の$\epsilon > 0$に対し$1/2^N < \epsilon$を満たす$N \in \N$を取れば,$\ell > k > N$なら
			\begin{align}
				\left|\hat{f}_{n_k}(x) - \hat{f}_{n_{\ell}}(x)\right| 
				\leq \sum_{j=k}^{\ell-1}\left|\hat{f}_{n_j}(x) - \hat{f}_{n_{j+1}}(x)\right| 
				< \sum_{k > N} \frac{1}{2^k} = \frac{1}{2^N} < \epsilon
				\quad (\forall x \in X)
			\end{align}
			となるから,各点$x \in X$で$\left( \hat{f}_{n_k}(x) \right)_{k=1}^{\infty}$は$\C$のCauchy列となり収束する.
			\begin{align}
				\hat{f}(x) \coloneqq \lim_{k \to \infty} \hat{f}_{n_k}(x)
				\quad (\forall x \in X)
			\end{align}
			として$\hat{f}$を定めれば,$\hat{f}$は可測$\mathscr{F}/\borel{\C}$であり,且つ任意に$k \in \N$を取れば
			\begin{align}
				\sup{x \in X}{|\hat{f}_{n_k}(x) - \hat{f}(x)|} \leq \frac{1}{2^{k-1}} \label{ineq:Lp_banach_3}
			\end{align}
			を満たす.実際或る$y \in X$で$\alpha \coloneqq |\hat{f}_{n_k}(y) - \hat{f}(y)| > 1/2^{k-1}$が成り立つと仮定すれば,
			\begin{align}
				\left| \hat{f}_{n_k}(y) - \hat{f}_{n_\ell}(y) \right|
				\leq \sum_{j=k}^{\ell-1} \sup{x \in X}{\left|\hat{f}_{n_j}(x) - \hat{f}_{n_{j+1}}(x)\right|}
				< \sum_{j=k}^{\infty} \frac{1}{2^j}
				= \frac{1}{2^{k-1}}
				\quad (\forall \ell > k)
			\end{align}
			より
			\begin{align}
				0 < \alpha - \frac{1}{2^{k-1}} < \left| \hat{f}_{n_k}(y) - \hat{f}(y) \right| - \left| \hat{f}_{n_k}(y) - \hat{f}_{n_\ell}(y) \right|
				\leq \left| \hat{f}(y) - \hat{f}_{n_\ell}(y) \right|
				\quad (\forall \ell > k)
			\end{align}
			が従い各点収束に反する.不等式(\refeq{ineq:Lp_banach_3})により
			\begin{align}
				\sup{x \in X}{\left| \hat{f}(x) \right|} 
				< \sup{x \in X}{\left| \hat{f}(x) - \hat{f}_{n_k}(x) \right|} + \sup{x \in X}{\left| \hat{f}_{n_k}(x) \right|} 
				\leq \frac{1}{2^{k-1}} + \Norm{\hat{f}_{n_k}}{\mathscr{L}^\infty}
			\end{align}
			が成り立つから$\left[\hat{f}\right] \in L^\infty(\mu)$が従い,
			\begin{align}
				\Norm{\left[f_{n_k}\right] - \left[\hat{f}\right]}{L^\infty}
				= \Norm{\left[\hat{f}_{n_k}\right] - \left[\hat{f}\right]}{L^\infty}
				\leq \sup{x \in X}{|\hat{f}_{n_k}(x) - \hat{f}(x)|}
				\longrightarrow 0 \quad (k \longrightarrow \infty)
			\end{align}
			により部分列$\left( \left[f_{n_k}\right] \right)_{k=1}^{\infty}$が$\left[\hat{f}\right]$に収束するから
			元のCauchy列も$\left[\hat{f}\right]$に収束する.
			
		\item[$1 \leq p < \infty$の場合]
			$\left[f_{n_k}\right]$の代表$f_{n_k}\ (k=1,2,\cdots)$は
			\begin{align}	
				f_{n_k}(x) = f_{n_1}(x) + \sum_{j=1}^{k}\left( f_{n_j}(x) - f_{n_{j-1}}(x) \right) \quad (\forall x \in X)
				\label{eq:Lp_banach_3}
			\end{align}
			を満たし,これに対して
			\begin{align}
				g_k(x) &\coloneqq \left| f_{n_1}(x) \right| + \sum_{j=1}^{k} \left| f_{n_j}(x) - f_{n_{j-1}}(x) \right|
				\quad (\forall x \in X,\ k=1,2,\cdots)
			\end{align}
			により単調非減少な可測関数列$(g_k)_{k=1}^{\infty}$を定めれば,Minkowskiの不等式と(\refeq{ineq:Lp_banach_2})により
			\begin{align}
				\Norm{g_k}{\mathscr{L}^p} \leq \Norm{f_{n_1}}{\mathscr{L}^p} + \sum_{j=1}^{k}\Norm{f_{n_j} - f_{n_{j-1}}}{\mathscr{L}^p}
				< \Norm{f_{n_1}}{\mathscr{L}^p} + 1 < \infty
				\quad (k = 1,2,\cdots)
				\label{eq:thm_Lp_banach_1}
			\end{align}
			が成り立つ.ここで
			\begin{align}
				B_N \coloneqq \bigcap_{k=1}^{\infty} \Set{x \in X}{g_k(x) \leq N},
				\quad B \coloneqq \bigcup_{N=1}^{\infty} B_N
			\end{align}
			とおけば$(g_k)_{k=1}^{\infty}$は$B$上で各点収束し$X \backslash B$上では発散するが,
			$X \backslash B$は零集合である.実際
			\begin{align}
				\int_X g_k^p\ d\mu
				= \int_B g_k^p\ d\mu + \int_{X \backslash B} g_k^p\ d\mu
				\leq \left( \Norm{f_{n_1}}{\mathscr{L}^p} + 1 \right)^p,
				\quad (k=1,2,\cdots)
			\end{align}
			が満たされているから,単調収束定理より
			\begin{align}
				\int_B \lim_{k \to \infty} g_k^p\ d\mu + \int_{X \backslash B} \lim_{k \to \infty} g_k^p\ d\mu
				\leq \left( \Norm{f_{n_1}}{\mathscr{L}^p} + 1 \right)^p
			\end{align}
			が成り立ち$\mu(X \backslash B) = 0$が従う.$\mathscr{F}/\borel{\C}$-可測関数$g,f$を
			\begin{align}
				g \coloneqq \lim_{k \to \infty} g_k \defunc_B,
				\quad f \coloneqq \lim_{k \to \infty} f_{n_k} \defunc_B
			\end{align}
			で定義すれば,$|f| \leq g$と
			$g^p$の可積分性により$\left[f\right] \in L^p(\mu)$が成り立つ.
			また$\left|f_{n_k} - f\right|^p \leq 2^p g^p\ (\forall k=1,2,\cdots)$が満たされているから,
			Lebesgueの収束定理により
			\begin{align}
				\lim_{k \to \infty}\Norm{\left[f_{n_k}\right] - \left[f\right]}{L^p}^p
				= \lim_{k \to \infty} \int_X \left| f_{n_k} - f \right|^p\ d\mu = 0
			\end{align}
			が従い,部分列の収束により元のCauchy列も$\left[f\right]$に収束する.
			\QED
	\end{description}
\end{prf}

\section{複素測度}
	\begin{screen}
		\begin{dfn}[複素測度]
			$(X,\mathscr{F})$を可測空間とする.
			$\lambda: \mathscr{F} \longrightarrow \C$が
			任意の互いに素な列$(E_i)_{i=1}^{\infty} \subset \mathscr{F}$に対し
			\begin{align}
				\lambda\biggl( \sum_{i=1}^{\infty} E_i \biggr) = \sum_{i=1}^{\infty} \lambda(E_i)
				\label{eq:dfn_complex_measure}
			\end{align}
			を満たすとき,$\lambda$を複素測度(complex measure)という.
		\end{dfn}
	\end{screen}
	
	任意の全単射$\sigma:\N \rightarrow \N$に対し
	\begin{align}
		(E \coloneqq)\ \sum_{i=1}^{\infty}E_i = \sum_{i=1}^{\infty}E_{\sigma(i)}
	\end{align}
	が成り立つから
	\begin{align}
		\sum_{i=1}^{\infty} \lambda(E_i) = \lambda(E) = \sum_{i=1}^{\infty} \lambda(E_{\sigma(i)})
	\end{align}
	が従い,Riemannの級数定理より
	$\sum_{i=1}^{\infty} \lambda(E_i)$は絶対収束する.
	ここで,
	\begin{align}
		|\lambda(E)| \leq \mu(E) \quad (\forall E \in \mathscr{F})
		\label{radon_nikodym_1}
	\end{align}
	を満たすような或る$(X,\mathscr{F})$上の測度$\mu$が存在すると考える.
	このとき$\mu$は
	\begin{align}
		\sum_{i=1}^{\infty} |\lambda(E_i)| \leq \sum_{i=1}^{\infty} \mu(E_i) 
		= \mu\Biggl(\sum_{i=1}^{\infty} E_i\Biggr)
	\end{align}
	を満たすから
	\begin{align}
		\sup{}{\Set{\sum_{i=1}^{\infty} |\lambda(A_i)|}{E = \sum_{i=1}^\infty A_i,\ \{A_i\}_{i=1}^\infty \subset \mathscr{F}}} 
		\leq \mu(E),
		\quad (\forall E \in \mathscr{F})
		\label{radon_nikodym_2}
	\end{align}
	が成立する.実は,
	\begin{align}
		|\lambda|(E) \coloneqq 
		\sup{}{\Set{\sum_{i=1}^{\infty} |\lambda(A_i)|}{E = \sum_{i=1}^\infty A_i,\ \{A_i\}_{i=1}^\infty \subset \mathscr{F}}},
		\quad (\forall E \in \mathscr{F})
		\label{radon_nikodym_3}
	\end{align}
	で定める$|\lambda|$は(\refeq{radon_nikodym_1})を満たす最小の有限測度となる
	(定理\ref{thm:total_variation_measure},定理\ref{thm:total_variation_measure_bounded}).
	
	\begin{screen}
		\begin{dfn}[総変動・総変動測度]
			可測空間$(X,\mathscr{F})$上の複素測度$\lambda$に対し,(\refeq{radon_nikodym_3})で定める
			$|\lambda|$を$\lambda$の総変動測度(total variation measure)といい,$|\lambda|(X)$を
			$\lambda$の総変動(total variation)という.
		\end{dfn}
	\end{screen}
	特に$\lambda$が正値有限測度である場合は$\lambda = |\lambda|$が成り立つ.実際,任意の$E \in \mathscr{F}$に対して
	\begin{align}
		|\lambda|(E) = \sup{}{\Set{\sum_{i=1}^{\infty} |\lambda(A_i)|}{E = \sum_{i=1}^\infty A_i,\ \{A_i\}_{i=1}^\infty \subset \mathscr{F}}}
		= \lambda(E)
	\end{align}
	が成立する.

	\begin{screen}
		\begin{thm}[$|\lambda|$は測度]
			可測空間$(X,\mathscr{F})$上の複素測度$\lambda$に対して,
			(\refeq{radon_nikodym_3})で定める$|\lambda|$は正値測度である.
			\label{thm:total_variation_measure}
		\end{thm}
	\end{screen}
	
	\begin{prf}
		$|\lambda|$の正値性は(\refeq{radon_nikodym_3})より従うから,
		$|\lambda|$の完全加法性を示す.
		いま,互いに素な集合列$E_i \in \mathscr{F}\ (i=1,2,\cdots)$を取り
		$E \coloneqq \sum_{i=1}^{\infty} E_i$とおく.
		このとき,任意の$\epsilon > 0$に対して
		$E_i$の或る分割$(A_{ij})_{j=1}^{\infty} \subset \mathscr{F}$が存在して
		\begin{align}
			|\lambda|(E_i) \geq \sum_{j=1}^{\infty} |\lambda(A_{ij})| 
			> |\lambda|(E_i) - \frac{\epsilon}{2^i}
		\end{align}
		を満たすから,$E = \sum_{i,j=1}^{\infty} A_{ij}$と併せて
		\begin{align}
			|\lambda|(E) \geq \sum_{i,j=1}^{\infty} |\lambda(A_{ij})| \geq \sum_{i=1}^{\infty}\sum_{j=1}^{\infty} |\lambda(A_{ij})| > \sum_{i=1}^{\infty} |\lambda|(E_i) - \epsilon
		\end{align}
		となり,$\epsilon > 0$の任意性より
		\begin{align}
			|\lambda|(E) \geq \sum_{j=1}^{\infty} |\lambda|(E_j)
		\end{align}
		が従う.一方で$E$の任意の分割$(A_j)_{j=1}^{\infty} \subset \mathscr{F}$に対し
		\begin{align}
			\sum_{j=1}^{\infty} |\lambda(A_j)| 
			= \sum_{j=1}^{\infty} \left| \sum_{i=1}^{\infty} \lambda(A_j \cap E_i) \right|
			\leq \sum_{j=1}^{\infty} \sum_{i=1}^{\infty} |\lambda|(A_j \cap E_i)
			\leq \sum_{i=1}^{\infty} |\lambda|(E_i)
		\end{align}
		が成り立つから,$E$の分割について上限を取って
		\begin{align}
			|\lambda|(E) \leq \sum_{i=1} |\lambda|(E_i)
		\end{align}
		を得る.
		\QED
	\end{prf}
	
	\begin{screen}
		\begin{lem}\label{lem:total_variation_measure_bounded}
			$z_1,\cdots,z_N$を複素数とする.このとき,次を満たす或る部分集合$S \subset \{1,\cdots,N\}$が存在する:
			\begin{align}
				\left| \sum_{k \in S} z_k \right| \geq \frac{1}{2\pi} \sum_{k=1}^{N} |z_k|.
			\end{align}
		\end{lem}
	\end{screen}
	
	\begin{prf}
		$i = \sqrt{-1}$として,
		$z_k = |z_k|\exp{i \alpha_k}\ (-\pi \leq \alpha_k < \pi,\ k=1,\cdots,N)$を満たす$\alpha_1,\cdots,\alpha_N$を取り
		\begin{align}
			S(\theta) \coloneqq \Set{k \in \{1,\cdots,N\}}{\cos{(\alpha_k - \theta)}{} > 0},
			\quad (-\pi \leq \theta \leq \pi)
		\end{align}
		とおく.このとき,$\cos{x}{+} \coloneqq 0 \vee \cos{x}{}\ (x \in \R)$とすれば
		\begin{align}
			\left| \sum_{k \in S(\theta)} z_k \right| &= |\exp{-i\theta}|\left| \sum_{k \in S(\theta)} z_k \right| = \left| \sum_{k \in S(\theta)} |z_k|\exp{i(\alpha_k - \theta)} \right| \\
			&\geq \Re{\sum_{k \in S(\theta)} |z_k|\exp{i(\alpha_k - \theta)}} = \sum_{k \in S(\theta)} |z_k| \cos{(\alpha_k - \theta)}{} = \sum_{k=1}^{N} |z_k| \cos{(\alpha_k - \theta)}{+}
		\end{align}
		が成り立ち,最右辺は$\theta$に関して連続であるから最大値を達成する$\theta_0 \in [-\pi,\pi]$が存在する.
		$S \coloneqq S(\theta_0)$として
		\begin{align}
			\left| \sum_{k \in S} z_k \right| \geq \sum_{k=1}^{N} |z_k| \cos{(\alpha_k - \theta_0)}{+} \geq \sum_{k=1}^{N} |z_k| \cos{(\alpha_k - \theta)}{+}
			\quad (\forall \theta \in [-\pi, \pi])
		\end{align}
		となり,積分して
		\begin{align}
			\left| \sum_{k \in S} z_k \right| 
			&\geq \sum_{k=1}^{N} |z_k| \frac{1}{2\pi} \int_{[-\pi,\pi]} \cos{(\alpha_k - \theta)}{+}\ d\theta \\
			&= \sum_{k=1}^{N} |z_k| \frac{1}{2\pi} \int_{[-\pi,\pi]} \cos{\theta}{+}\ d\theta
			= \frac{1}{2\pi} \sum_{k=1}^{N} |z_k|
		\end{align}
		が得られる.
		\QED
	\end{prf}
	
	\begin{screen}
		\begin{thm}[総変動は有限]\label{thm:total_variation_measure_bounded}
			可測空間$(X,\mathscr{F})$上の複素測度$\lambda$の総変動測度$|\lambda|$について次が成り立つ:
			\begin{align}
				|\lambda|(X) < \infty.
			\end{align}
			特に,複素測度は有界である.
		\end{thm}
	\end{screen}

	\begin{prf} $|\lambda|(X) = \infty$と仮定して背理法により定理を導く.
		\begin{description}
		\item[第一段]
			或る$E \in \mathscr{F}$に対し$|\lambda|(E) = \infty$が成り立っているなら,
			\begin{align}
				|\lambda(A)| > 1, \quad |\lambda(B)| > 1, \quad E = A + B
			\end{align}
			を満たす$A,B \in \mathscr{F}$が存在することを示す.いま,$t \coloneqq 2\pi(1 + |\lambda(E)|)$とおけば
			\begin{align}
				\sum_{i=1}^{\infty} |\lambda(E_i)| > t
			\end{align}
			を満たす$E$の分割$(E_i)_{i=1}^{\infty}$が存在する.従って或る$N \in \N$に対し
			\begin{align}
				\sum_{i=1}^{N} |\lambda(E_i)| > t
			\end{align}
			が成り立ち,補題\ref{lem:total_variation_measure_bounded}より
			\begin{align}
				\left| \sum_{k \in S} \lambda(E_k) \right| \geq \frac{1}{2\pi} \sum_{k=1}^{N} |\lambda(E_k)| > \frac{t}{2\pi} > 1
			\end{align}
			を満たす$S \subset \{1,\cdots,N\}$が取れる.ここで$A \coloneqq \sum_{k \in S} E_k,\ B \coloneqq E - A$とおけば,
			$|\lambda(A)| > 1$かつ
			\begin{align}
				|\lambda(B)| = |\lambda(E)-\lambda(A)| \geq |\lambda(A)| - |\lambda(E)| > \frac{t}{2\pi} - |\lambda(E)| = 1
			\end{align}
			が成り立つ.また,
			\begin{align}
				|\lambda|(E) = |\lambda|(A) + |\lambda|(B)
			\end{align}
			より$|\lambda|(A),\ |\lambda|(B)$の少なくとも一方は$\infty$となる.
		
		\item[第二段]
			いま,$|\lambda|(X) = \infty$と仮定すると,前段の結果より
			\begin{align}
				|\lambda|(B_1) = \infty, \quad |\lambda(A_1)| > 1, \quad |\lambda(B_1)| > 1,
				\quad X = A_1 + B_1
			\end{align}
			を満たす$A_1,B_1 \in \mathscr{F}$が存在する.同様に$B_1$に対しても
			\begin{align}
				|\lambda|(B_2) = \infty, \quad |\lambda(A_2)| > 1, \quad |\lambda(B_2)| > 1,
				\quad B_1 = A_2 + B_2
			\end{align}
			を満たす$A_2,B_2 \in \mathscr{F}$が存在する.
			繰り返せば$|\lambda(A_j)| > 1\ (j=1,2,\cdots)$
			を満たす互いに素な集合列$(A_j)_{j=1}^{\infty}$が構成され,
			このとき$\sum_{j=1}^{\infty} |\lambda(A_j)| = \infty$となる.
			一方でRiemannの級数定理より$\sum_{j=1}^{\infty} |\lambda(A_j)| < \infty$
			が成り立つから矛盾が生じ,$|\lambda|(X) < \infty$が出る.
			\QED
		\end{description}
	\end{prf}
	
	\begin{screen}
		\begin{thm}[複素測度全体は線型空間・総変動ノルム]
			可測空間$(X,\mathscr{F})$上の複素測度の全体を$CM(X,\mathscr{F})$と書く.
			\begin{align}
				&(\lambda + \mu)(E) \coloneqq \lambda(E) + \mu(E), \\
				&(c\lambda)(E) \coloneqq c\lambda(E)
				\label{complex_measure_linear}
			\end{align}
			を線型演算として$CM(X,\mathscr{F})$は線形空間となり,また
			\begin{align}
				\Norm{\lambda}{TV} \coloneqq |\lambda|(X) \quad (\lambda \in CM(X,\mathscr{F}))
			\end{align}
			により$CM(X,\mathscr{F})$に総変動ノルム$\Norm{\cdot}{TV}$が定まる.
		\end{thm}
	\end{screen}
	
	\begin{prf}\mbox{}
		$\Norm{\cdot}{TV}$がノルムであることを示す.
		\begin{description}
			\item[第一段]
				$\lambda = 0$なら$\Norm{\lambda}{TV} = |\lambda|(X) = 0$となる.また
				$|\lambda(E)| \leq |\lambda|(E) \leq \Norm{\lambda}{TV}$より
				$\Norm{\lambda}{TV} = 0$なら$\lambda=0$が従う.
			
			\item[第二段]
				任意の$\lambda \in CM(X,\mathscr{F})$と$c \in \C$に対し
				\begin{align}
					\Norm{c\lambda}{TV} = \sup{}{\sum_{i}|(c\lambda)(E_i)|} = \sup{}{\sum_{i}|c\lambda(E_i)|} = |c|\sup{}{\sum_{i}|\lambda(E_i)|} = |c|\Norm{\lambda}{TV}
				\end{align}
				が成り立ち同次性が得られる.
			
			\item[第三段]
				$\lambda,\mu \in CM(X,\mathscr{F})$を任意に取る.このとき,
				$X$の任意の分割$X = \sum_{i=1}^\infty E_i\ (E_i \in \mathscr{F})$に対して
				\begin{align}
					\sum_{i=1}^\infty |(\lambda + \mu)(E_i)| 
					= \sum_{i=1}^\infty |\lambda(E_i) + \mu(E_i)| 
					\leq \sum_{i=1}^\infty |\lambda(E_i)| + \sum_{i=1}^\infty |\mu(E_i)| \leq \Norm{\lambda}{TV} + \Norm{\mu}{TV}
				\end{align}
				が成り立つから$\Norm{\lambda + \mu}{TV} \leq \Norm{\lambda}{TV} + \Norm{\mu}{TV}$が従う.
				\QED
		\end{description}
	\end{prf}
	
	可測空間$(X,\mathscr{F})$において,$\R$にしか値を取らない複素測度を符号付き測度(signed measure)という.
	\begin{screen}
		\begin{dfn}[正変動と負変動・Jordanの分解]
			$(X,\mathscr{F})$を可測空間とする.$(X,\mathscr{F})$上の符号付き測度$\mu$に対し
			\begin{align}
				\mu^+ \coloneqq \frac{1}{2}(|\mu| + \mu) , \quad \mu^- \coloneqq \frac{1}{2}(|\mu| - \mu)
			\end{align}
			として正値有限測度$\mu^+,\mu^-$を定める.
			$\mu^+\ (\mu^-)$を$\mu$の正(負)変動(positive (negative) variation)と呼び,
			\begin{align}
				\mu = \mu^+ - \mu^-
			\end{align}
			を符号付き測度$\mu$のJordan分解(Jordan decomposition)という.同時に$|\mu| = \mu^+ + \mu^-$も成り立つ.
		\end{dfn}
	\end{screen}
	
	\begin{screen}
		\begin{dfn}[絶対連続・特異]
			$(X,\mathscr{F})$を可測空間,
			$\mu$を$\mathscr{F}$上の正値測度
			,$\lambda,\lambda_1,\lambda_2$を$\mathscr{F}$上の任意の測度とする.
			\begin{itemize}
				\item $\mu(E)=0$ならば$\lambda(E)=0$となるとき,
					$\lambda$は$\mu$に関して絶対連続である(absolutely continuous)といい
					\begin{align}
						\lambda \ll \mu
					\end{align}
					と書く.
				
				\item 或る$A \in \mathscr{F}$が存在して
					\begin{align}
						\lambda(E) = \lambda(A \cap E),\quad (\forall E \in \mathscr{F})
					\end{align}
					が成り立つとき,$\lambda$は$A$に集中している(concentrated on A)という.
					$\lambda_1$が$A_1$に,$\lambda_2$が$A_2$に集中し,かつ
					$A_1 \cap A_2 = \emptyset$であるとき,
					$\lambda_1,\lambda_2$は互いに特異である(mutually singular)といい
					\begin{align}
						\lambda_1 \perp \lambda_2
					\end{align}
					と書く.
			\end{itemize}
		\end{dfn}
	\end{screen}
	
	\begin{screen}
		\begin{thm}[絶対連続性の同値条件]\label{thm:equivalent_condition_of_absolute_continuity}
			$\lambda,\mu$をそれぞれ可測空間$(X,\mathscr{F})$上の複素測度,正値測度とするとき,
			次は同値である:
			\begin{description}
				\item[(1)] $\lambda \ll \mu$,
				\item[(2)] $|\lambda| \ll \mu$
				\item[(3)] 任意の$\epsilon > 0$に対し或る$\delta > 0$が存在して
					$\mu(E) < \delta$なら$|\lambda|(E) < \epsilon$となる.
			\end{description}
		\end{thm}
	\end{screen}
	
	\begin{prf}\mbox{}
		\begin{description}
			\item[第一段]
				$(1) \Leftrightarrow (2)$を示す.
				任意の$E \in \mathscr{F}$に対し$|\lambda(E)| \leq |\lambda|(E)$より
				$(2) \Rightarrow (1)$が従う.また$\lambda \ll \mu$のとき,
				\begin{align}
					|\lambda|(E) =
					\sup{}{\Set{\sum_{i=1}^{\infty} |\lambda(A_i)|}{E = \sum_{i=1}^\infty A_i,\ \{A_i\}_{i=1}^\infty \subset \mathscr{F}}},
					\quad (\forall E \in \mathscr{F})
				\end{align}
				より$\mu(E) = 0$なら$\mu(A_i) = 0$となり
				$\lambda(A_i) = 0\ (\forall i \geq 1)$が満たされ
				$(1) \Rightarrow (2)$が従う.
				
			\item[第二段]
				$(2) \Leftrightarrow (3)$を示す.
				実際(3)が満たされているとき,$\mu(E) = 0$なら任意の$\delta > 0$に対し
				$\mu(E) < \delta$となるから$|\lambda|(E) < \epsilon\ (\forall \epsilon > 0)$
				となり$|\mu|(E) = 0$が出る.逆に(3)が満たされていないとき,或る$\epsilon > 0$に対して
				\begin{align}
					\mu(E_n) < \frac{1}{2^{n+1}}, \quad |\lambda|(E_n) \geq \epsilon,
					\quad (n=1,2,\cdots)
				\end{align}
				を満たす$\{E_n\}_{n=1}^\infty \subset \mathscr{F}$が存在する.このとき
				\begin{align}
					A_n \coloneqq \bigcup_{i=n}^\infty E_i,
					\quad A \coloneqq \bigcap_{n=1}^\infty A_n
				\end{align}
				とおけば
				\begin{align}
					\mu(A) = \lim_{n \to \infty} \mu(A_n) 
					\leq \lim_{n \to \infty} \frac{1}{2^n} = 0
				\end{align}
				かつ
				\begin{align}
					|\lambda|(A) = \lim_{n \to \infty} |\lambda|(A_n) 
					\geq \lim_{n \to \infty} |\lambda|(E_n) \geq \epsilon 
				\end{align}
				が成り立ち,対偶を取れば$(2) \Rightarrow (3)$が従う.
				\QED
		\end{description}
	\end{prf}
	
	\begin{screen}
		\begin{lem}\label{lem:Lebesgue_Radon_Nikodym}
			$(X,\mathscr{F},\mu)$を$\sigma$-有限測度空間とするとき,
			$0 < w < 1$を満たす可積分関数$w$が存在する.
		\end{lem}
	\end{screen}
	
	\begin{prf}
		$\mu(X) = 0$なら$w \equiv 1/2$とすればよい.$\mu(X) > 0$の場合,$\sigma$-有限の仮定より
		\begin{align}
			0 < \mu(X_n) < \infty,\ (\forall n \geq 1),
			\quad X = \bigcup_{n=1}^\infty X_n
		\end{align}
		を満たす$\{X_n\}_{n=1}^\infty \subset \mathscr{F}$が存在する.ここで
		\begin{align}
			w_n(x) \coloneqq
			\begin{cases}
				\displaystyle\frac{1}{2^n\left(1+\mu(X_n)\right)}, & x \in X_n, \\
				0, & x \in X \backslash X_n,
			\end{cases}
			\quad n=1,2,\cdots
		\end{align}
		に対して
		\begin{align}
			w \coloneqq \sum_{n=1}^\infty w_n
		\end{align}
		と定めれば,任意の$x \in X$は或る$X_n$に属するから
		\begin{align}
			0 < w_n(x) \leq w(x)
		\end{align}
		が成り立ち,かつ
		\begin{align}
			w(x) = w_1(x) + \sum_{n=2}^\infty w_n(x)
			\leq \frac{1}{2\left(1+\mu(X_1)\right)} + \frac{1}{2}
			< 1,
			\quad (\forall x \in X)
		\end{align}
		が満たされる.また単調収束定理より
		\begin{align}
			\int_X w\ d\mu \leq \sum_{n=1}^\infty \int_X w_n\ d\mu
			\leq \sum_{n=1}^\infty \frac{\mu(X_n)}{2^n\left(1+\mu(X_n)\right)}
			\leq 1
		\end{align}
		となり$w$の可積分性が出る.
		\QED
	\end{prf}
	
	\begin{screen}
		\begin{thm}[Lebesgue-Radon-Nikodym]
			$(X,\mathscr{F})$を可測空間,$\lambda$を$(X,\mathscr{F})$上の複素測度,
			$\mu$を$(X,\mathscr{F})$上の$\sigma$-有限正値測度$(\mu(X)>0)$とするとき,以下が成立する:
			\begin{description}
				\item[Lebesgue分解]
					$\lambda$は$\mu$に関して絶対連続な$\lambda_a$及び$\mu$と互いに特異な
					$\lambda_s$に一意に分解される:
					\begin{align}
						\lambda = \lambda_a + \lambda_s,
						\quad \lambda_a \ll \mu,
						\quad \lambda_s \perp \mu.
					\end{align}
				
				\item[密度関数の存在]
					$\lambda_a$に対し或る$g \in L^1(\mu) = L^1(X,\mathscr{F},\mu)$が唯一つ存在して次を満たす:
					\begin{align}
						\lambda_a(E) = \int_E g\ d\mu,
						\quad (\forall E \in \mathscr{F}).
					\end{align}
			\end{description}
		\end{thm}
	\end{screen}
	
	\begin{prf}\mbox{}
		\begin{description}
			\item[第一段] Lebesgueの分解の一意性を示す.
				$\lambda'_a \ll \mu$と$\lambda'_s \perp \mu$により
				\begin{align}
					\lambda_a + \lambda_s = \lambda'_a + \lambda'_s
				\end{align}
				が成り立つとき,
				\begin{align}
					\Lambda \coloneqq \lambda_a - \lambda'_a = \lambda'_s - \lambda_s,
					\quad \Lambda \ll \mu,
					\quad \Lambda \perp \mu
				\end{align}
				となり$\Lambda = 0$が従い分解の一意性が出る.
			
			\item[第二段] 密度関数の一意性を示す.実際,可積分関数$f$に対して
				\begin{align}
					\int_E f\ d\mu = 0, \quad (\forall E \in \mathscr{F})
				\end{align}
				が成り立つとき,定理\ref{thm:mean_value_of_integral_and_closed_set}より
				$f = 0,\ \mbox{$\mu$-a.e.}$が成り立つ.
				
			\item[第三段] Lebesgueの分解と密度関数の存在を示す.
				
		\end{description}
	\end{prf}
	
	\begin{screen}
		\begin{thm}[Vitali-Hahn-Saks]\label{thm;Vitali_Hahn_Saks}
			$(X,\mathscr{F})$を可測空間,$(\lambda_n)_{n=1}^\infty$をこの上の複素測度の列とするとき,
			\begin{align}
				\lambda(E) \coloneqq \lim_{n \to \infty} \lambda_n(E),
				\quad (\forall E \in \mathscr{F})
				\label{eq:thm_Vitali_Hahn_Saks}
			\end{align}
			が存在すれば$\lambda$もまた$(X,\mathscr{F})$上の複素測度となる.
			つまり$(CM(X,\mathscr{F}),\Norm{\cdot}{TV})$はBanach空間である.
		\end{thm}
	\end{screen}
	
	\begin{prf}$\lambda_n \equiv 0\ (\forall n \geq 1)$なら$\lambda \equiv 0$で複素測度となるから,
		或る$n$と$E \in \mathscr{F}$に対し$\lambda_n(E) \neq 0$と仮定する.
		\begin{description}
			\item[第一段] $(X,\mathscr{F})$上の有限測度を
				\begin{align}
					\mu \coloneqq \sum_{n=1}^\infty \frac{1}{2^n(1 + \Norm{\lambda_n}{TV})} |\lambda_n|
				\end{align}
				により定めるとき,%定理\ref{thm:equivalent_condition_of_absolute_continuity}より
				任意の$\epsilon > 0$に対して或る$\delta > 0$が存在し
				\begin{align}
					\mu(E) < \delta \quad \Rightarrow \quad |\lambda_n|(E) < \epsilon\ (\forall n \geq 1)
					\label{eq:thm_Vitali_Hahn_Saks_2}
				\end{align}
				となることを示す.
				任意の$n \geq 1$に対して$\lambda_n \ll \mu$であるから
				Lebesgue-Radon-Nikodymの定理より
				\begin{align}
					\lambda_n(E) = \int_E g_n\ d\mu,
					\quad (\forall E \in \mathscr{F})
				\end{align}
				を満たす$g_n \in L^1(\mu)$が存在し,このとき
				\begin{align}
					\left| \int_E g_n\ d\mu \right|
					\leq |\lambda_n|(E)
					\leq 2^n(1+\Norm{\lambda_n}{TV})\mu(E),
					\quad (\forall E \in \mathscr{F})
				\end{align}
				が成立するから定理\ref{thm:mean_value_of_integral_and_closed_set}より
				\begin{align}
					\Norm{g_n}{L^\infty(\mu)} \leq 2^n(1+\Norm{\lambda_n}{TV})
				\end{align}
				が従う.いま,任意の$E \in \mathscr{F}$に対し$f_E \coloneqq [\defunc_E]$として
				\begin{align}
					L \coloneqq \Set{f_E}{E \in \mathscr{F}}
				\end{align}
				とおけば,$\mu(X) < \infty$より$L \subset L^1(\mu)$となり,また
				\begin{align}
					d(f_E,f_{E'}) \coloneqq \Norm{f_E - f_{E'}}{L^1(\mu)}
				\end{align}
				で定める距離$d$により$L$は完備距離空間となる.実際,定理\ref{thm:Lp_banach}より
				$L$の任意のCauchy列$\left(f_{E_n}\right)_{n=1}^\infty$に対し極限$f \in L^1(\mu)$が存在し,
				或る部分列$\left(\defunc_{E_{n_k}}\right)_{k=1}^\infty$は或る$\mu$-零集合$A$を除いて各点収束するから
				\begin{align}
					\varphi \coloneqq \lim_{k \to \infty} \defunc_{E_{n_k}} \defunc_{X \backslash A}
				\end{align}
				に対し$E \coloneqq \{\varphi = 1\}$とおけば$f = [\defunc_E] \in L$が満たされる.ここで
				\begin{align}
					\Phi_n: L \ni f_E \longmapsto \int_X |g_n| f_E\ d\mu
				\end{align}
				とおけば,任意の$E \in \mathscr{F}$に対し$|\lambda_n|(E) \leq \Phi_n(f_E)$が満たされ,
				またH\Ddot{o}lderの不等式より
				\begin{align}
					\left| \Phi_n(f_E) - \Phi_n(f_{E'}) \right|
					\leq \int_X |g_n| |f_E - f_{E'}|\ d\mu
					\leq \Norm{g_n}{L^\infty(\mu)} d(f_E,f_{E'}),
					\quad (\forall f_E,f_{E'} \in L)
				\end{align}
				がとなるから$\Phi_n$は$L$上の連続写像である.いま$\epsilon > 0$を任意に取れば,
				$\eta \coloneqq \epsilon/4$に対して
				\begin{align}
					F_n(\eta) 
					\coloneqq \Set{f_E \in L}{\sup{k \geq 1}{\left| \Phi_n(f_E)-\Phi_{n+k}(f_E) \right|} \leq \eta}
					= \bigcap_{k \geq 1} \Set{f_E \in L}{\left| \Phi_n(f_E)-\Phi_{n+k}(f_E) \right| \leq \eta}
				\end{align}
				により定める$F_n(\delta)$は閉集合であり,任意の$f_E \in L$は
				\begin{align}
					\sup{k \geq 1}{\left| \Phi_n(f_E)-\Phi_{n+k}(f_E) \right|}
					&\leq \left| \Phi_n(f_E)-\lambda(E) \right|
						+ \sup{k \geq 1}{\left| \lambda(E)-\Phi_{n+k}(f_E) \right|} \\
					&= \left| \lambda_n(E)-\lambda(E) \right|
						+ \sup{k \geq 1}{\left| \lambda(E)-\lambda_{n+k}(E) \right|} \\
					&\longrightarrow 0 \quad (n \longrightarrow \infty)
				\end{align}
				を満たすから
				\begin{align}
					L = \bigcup_{n=1}^\infty F_n(\eta)
				\end{align}
				が成り立ち,Baireの範疇定理より或る$F_{n_0}(\eta)$は内点$f_{E_0}$を持つ.
				つまり或る$\delta_0 > 0$が存在して
				\begin{align}
					d(f_{E_0},f_E) < \delta_0
					\quad \Rightarrow \quad
					\sup{k \geq 1}{\left| \Phi_n(f_E)-\Phi_{n+k}(f_E) \right|} \leq \eta
				\end{align}
				となる.$\mu(E) < \delta_0$ならば,
				\begin{align}
					E_1 \coloneqq E \cup E_0,
					\quad E_2 \coloneqq E_0 \backslash (E \cap E_0)
				\end{align}
				とすれば$f_E = [\defunc_E] = [\defunc_{E_1} - \defunc_{E_2}]
				= [\defunc_{E_1}] - [\defunc_{E_2}] = f_{E_1} - f_{E_2}$かつ
				\begin{align}
					d(f_{E_0},f_{E_1}) = \mu(E \backslash E_0) < \delta_0,
					\quad d(f_{E_0},f_{E_2}) = \mu(E \cap E_0) < \delta_0
				\end{align}
				が満たされるから,$n > n_0$なら
				\begin{align}
					\left|\Phi_n(f_E)\right| 
					&\leq \left|\Phi_{n_0}(f_E)\right| + \left|\Phi_n(f_E) - \Phi_{n_0}(f_E)\right| \\
					&\leq \left|\Phi_{n_0}(f_E)\right| + \left|\Phi_n(f_{E_1}) - \Phi_{n_0}(f_{E_1})\right|
						+ \left|\Phi_n(f_{E_2}) - \Phi_{n_0}(f_{E_2})\right| \\
					&\leq \left|\Phi_{n_0}(f_E)\right| + 2\eta
				\end{align}
				が従い,一方で$n=1,2,\cdots,n_0$に対しては,
				定理\ref{thm:integrable_intvalue_uniformly_shrinking}より
				或る$\delta_n > 0$が存在して
				\begin{align}
					\mu(E) < \delta_n \Longrightarrow \Phi_n(f_E) = \int_E |g_n|\ d\mu < \frac{\epsilon}{2}
				\end{align}
				が成立し,$\delta \coloneqq \min{}{\{\delta_0,\delta_1,\cdots,\delta_{n_0}\}}$として
				\begin{align}
					\mu(E) < \delta_n \Longrightarrow |\lambda_n|(E) \leq \Phi_n(f_E) < \epsilon,\ (\forall n \geq 1)
				\end{align}
				が得られる.
				
			\item[第二段] $\lambda$の可算加法性を示す.任意の互いに素な$A,B \in \mathscr{F}$を取れば
				\begin{align}
					\lambda(A + B) = \lim_{n \to \infty} \lambda_n(A + B)
					= \lim_{n \to \infty} \lambda_n(A) + \lim_{n \to \infty} \lambda_n(B)
					= \lambda(A) + \lambda(B)
				\end{align}
				となるから$\lambda$は有限加法的であり,このとき任意の互いに素な列$\{E_i\}_{i=1}^\infty \subset \mathscr{F}$に対し
				\begin{align}
					\lambda\Biggl( \sum_{i=1}^\infty E_i \Biggr)
					= \lambda\Biggl( \sum_{i=1}^N E_i \Biggr) + \lambda\Biggl( \sum_{i=N+1}^\infty E_i \Biggr)
					= \sum_{i=1}^N \lambda(E_i) + \lambda\Biggl( \sum_{i=N+1}^\infty E_i \Biggr)
				\end{align}
				が任意の$N \geq 1$について満たされるが,
				\begin{align}
					\mu\Biggl( \sum_{i=N+1}^\infty E_i \Biggr) \longrightarrow 0 \quad (N \longrightarrow \infty)
				\end{align}
				と(\refeq{eq:thm_Vitali_Hahn_Saks_2})より
				\begin{align}
					\lambda\Biggl( \sum_{i=N+1}^\infty E_i \Biggr) \longrightarrow 0 \quad (N \longrightarrow \infty)
				\end{align}
				が従い
				\begin{align}
					\lambda\Biggl( \sum_{i=1}^\infty E_i \Biggr) = \sum_{i=1}^\infty \lambda(E_i)
				\end{align}
				が得られる.よって$\lambda$は複素測度である.
				\QED
		\end{description}
	\end{prf}
	
	\begin{screen}
		\begin{thm}[$L^p$の共役空間]\label{thm:dual_space_of_L_p}
			$1 \leq p < \infty$,$q$を$p$の共役指数とし,また$(X,\mathscr{F},\mu)$を$\sigma$-有限な測度空間とするとき,
			$g \in L^q(\mu)$に対して
			\begin{align}
				\Phi_g: L^p(\mu) \ni f \longmapsto \int_X fg\ d\mu
				\label{eq:thm_dual_space_of_L_p_1}
			\end{align}
			は有界線形作用素となる.また
			\begin{align}
				\Phi: L^q(\mu) \ni g \longmapsto \Phi_g \in \left( L^p(\mu) \right)^*
			\end{align}
			で定める$\Phi$は$\left( L^p(\mu) \right)^*$から$L^q(\mu)$への線型同型であり,
			次の意味で等長である:
			\begin{align}
				\Norm{g}{L^q(\mu)} = \Norm{\Phi_g}{\left( L^p(\mu) \right)^*}.
				\label{eq:thm_dual_space_of_L_p_asseretion_2}
			\end{align}
			$p=\infty$の場合,$\mu(X) < \infty$かつ$\varphi \in \left( L^\infty(\mu) \right)^*$に対し
			$\mathscr{F} \ni A \longmapsto \varphi(\defunc_A)$が可算加法的ならば,
			$\varphi$に対し或る$g \in L^1(\mu)$が唯一つ存在して
			$\varphi = \Phi_g$と(\refeq{eq:thm_dual_space_of_L_p_asseretion_2})を満たす.
		\end{thm}
	\end{screen}
	
	\begin{prf}\mbox{}
		\begin{description}
			\item[第一段]
				$\Phi_g$が(\refeq{eq:thm_dual_space_of_L_p_1})で与えられていれば,H\Ddot{o}lderの不等式より
				\begin{align}
					\left|\Phi_g(f)\right| \leq \Norm{g}{L^q(\mu)}\Norm{f}{L^p(\mu)}
				\end{align}
				が成り立つから
				\begin{align}
					\Norm{\Phi_g}{\left( L^p(\mu) \right)^*} \leq \Norm{g}{L^q(\mu)}
					\label{eq:thm_dual_space_of_L_p_3}
				\end{align}
				が従う.よって$\Phi_g \in \left( L^p(\mu) \right)^*$となる.
			
			\item[第二段]
				$\varphi \in \left( L^p(\mu) \right)^*$に対して
				$\Phi(g) = \varphi$を満たす$g \in L^q(\mu)$が存在するとき,
				$g$が$\varphi$に対して一意に決まることを示す.$\sigma$-有限の仮定より
				\begin{align}
					\mu(X_n) < \infty,\ (\forall n \geq 1);
					\quad X = \bigcup_{n=1}^\infty X_n
					\label{eq:thm_dual_space_of_L_p_6}
				\end{align}
				を満たす$\{X_n\}_{n=1}^\infty \subset \mathscr{F}$が存在する.
				いま,$g,g' \in L^q(\mu)$に対して
				\begin{align}
					\int_X fg\ d\mu = \int_X fg'\ d\mu,
					\quad (\forall f \in L^p(\mu))
				\end{align}
				が成り立っているとすれば,任意の$E \in \mathscr{F}$に対して
				$\defunc_{E \cap X_n} \in L^p(\mu)$であるから
				\begin{align}
					\int_{E \cap X_n} g-g'\ d\mu = 0,
					\quad (\forall n \geq 1)
				\end{align}
				となり,Lebesgueの収束定理より
				\begin{align}
					\int_E g-g'\ d\mu = 0
				\end{align}
				が従い$L^q(\mu)$で$g = g'$が成立する.
				
			\item[第三段]
				$1 \leq p < \infty$の場合,$\mu(X) < \infty$なら
				任意の$\varphi \in \left( L^p(\mu) \right)^*$に対して
				$\Phi(g) = \varphi$を満たす$g \in L^q(\mu)$が存在することを示す.
				\begin{align}
					\lambda(E) \coloneqq \varphi(\defunc_E)
					\label{eq:thm_dual_space_of_L_p_7}
				\end{align}
				により$\lambda$を定めれば
				\begin{align}
					\lambda(A + B) = \varphi(\defunc_{A+B}) = \varphi(\defunc_A + \defunc_B)
					= \varphi(\defunc_A) + \varphi(\defunc_B)
					= \lambda(A) + \lambda(B)
				\end{align}
				となり$\lambda$の加法性が出る.また
				任意の互いに素な$\{E_n\}_{n=1}^\infty \in \mathscr{F}$に対して
				\begin{align}
					A_k \coloneqq \sum_{n=1}^k E_n,
					\quad A \coloneqq \sum_{n=1}^\infty E_n
				\end{align}
				とおけば
				\begin{align}
					\left| \lambda(A) - \sum_{n=1}^k \lambda(E_n) \right|
					&= \left| \lambda(A) - \lambda(A_k) \right|
					= \left| \varphi(\defunc_A - \defunc_{A_k}) \right| \\
					&\leq \Norm{\varphi}{\left( L^p(\mu) \right)^*} \Norm{\defunc_A - \defunc_{A_k}}{L^p(\mu)}
					= \Norm{\varphi}{\left( L^p(\mu) \right)^*} \mu(A - A_k)^{1/p}
					\longrightarrow 0
					\quad (k \longrightarrow \infty)
				\end{align}
				が成り立つから$\lambda$は複素測度である.また
				\begin{align}
					|\lambda(E)| \leq \Norm{\varphi}{\left( L^p(\mu) \right)^*} \mu(E)^{1/p}
				\end{align}
				より$\lambda \ll \mu$となるから,Lebesgue-Radon-Nikodymの定理より
				\begin{align}
					\varphi(\defunc_E) = \lambda(E) = \int_X \defunc_E g\ d\mu,
					\quad (\forall E \in \mathscr{F})
					\label{eq:thm_dual_space_of_L_p_8}
				\end{align}
				を満たす$g \in L^1(\mu)$が存在する.$\varphi$の線型性より
				任意の単関数の同値類$f$に対して
				\begin{align}
					\varphi(f) = \int_X fg\ d\mu
					\label{eq:thm_dual_space_of_L_p_2}
				\end{align}
				が成立し,特に$f \in L^\infty(\mu)$に対しては
				\begin{align}
					B \coloneqq \Set{x \in X}{|f(x)| > \Norm{f}{L^\infty(\mu)}}
				\end{align}
				とおけば$\mu(B) = 0$となり,有界可測関数$f \defunc_{X \backslash B}$を
				一様に近似する単関数列$(f_n)_{n=1}^\infty$が存在して
				\begin{align}
					\left| \varphi(f) - \int_X fg\ d\mu \right|
					&\leq \left| \varphi(f) - \varphi(f_n) \right| + \left| \int_X f_ng\ d\mu - \int_X fg\ d\mu \right| \\
					&\leq \Norm{\varphi}{\left( L^p(\mu) \right)^*} \Norm{f - f_n}{L^p(\mu)}
						+ \int_X |f_n - f||g|\ d\mu \\
					&\longrightarrow 0 \quad (n \longrightarrow \infty)
				\end{align}
				となるから(\refeq{eq:thm_dual_space_of_L_p_2})が成立する.
			
			\item[第四段]
				$p = \infty,\ \mu(X) < \infty$の場合,
				$\varphi \in \left( L^p(\mu) \right)^*$に対して
				$\mathscr{F} \ni A \longmapsto \varphi(\defunc_A)$が可算加法的ならば
				(\refeq{eq:thm_dual_space_of_L_p_7})で定める$\lambda$は複素測度となり,
				前段と同じ理由で(\refeq{eq:thm_dual_space_of_L_p_8})を満たす$g \in L^1(\mu)$が存在し
				\begin{align}
					\varphi(f) = \int_X fg\ d\mu,
					\quad (\forall f \in L^\infty(\mu))
				\end{align}
				が成立する.すなわち$\varphi = \Phi_g$であり,
				このとき$f \coloneqq \defunc_{\{g \neq 0\}}\overline{g}/g \in L^\infty(\mu)$
				に対して
				\begin{align}
					\Norm{g}{L^1(\mu)} = \int_X fg\ d\mu = \varphi(f) 
					\leq \Norm{\varphi}{\left( L^\infty(\mu) \right)^*} 
				\end{align}
				となるから,(\refeq{eq:thm_dual_space_of_L_p_3})と併せて
				(\refeq{eq:thm_dual_space_of_L_p_asseretion_2})が満たされる.
				以降は$p < \infty$とする.
				
			\item[第五段]
				$g \in L^q(\mu)$であることを示す.$p = 1$の場合,
				任意の$E \in \mathscr{F}$に対して$f = \defunc_E$とすれば,
				(\refeq{eq:thm_dual_space_of_L_p_2})より
				\begin{align}
					\left| \int_E g\ d\mu \right| = \left| \varphi(\defunc_E) \right|
					\leq \Norm{\varphi}{\left( L^p(\mu) \right)^*} \mu(E)
				\end{align}
				が成立し
				\begin{align}
					\Norm{g}{L^q(\mu)} \leq \Norm{\varphi}{\left( L^p(\mu) \right)^*}
					\label{eq:thm_dual_space_of_L_p_4}
				\end{align}
				が従う.$1 < p < \infty$の場合は
				$\alpha \coloneqq \defunc_{\{g \neq 0\}}\overline{g}/g$と
				\begin{align}
					E_n \coloneqq \Set{x \in X}{|g(x)| \leq n},
					\quad (n=1,2,\cdots)
				\end{align}
				に対して$f \coloneqq \defunc_{E_n} |g|^{q-1} \alpha$とおけば,
				\begin{align}
					fg = \defunc_{E_n} |g|^q = |f|^p
				\end{align}
				が成り立ち$|f|^p \in L^\infty(\mu)$となるから(\refeq{eq:thm_dual_space_of_L_p_2})より
				\begin{align}
					\int_X \defunc_{E_n} |g|^q\ d\mu
					= \int_X fg\ d\mu
					= \varphi(f)
					\leq \Norm{\varphi}{\left( L^p(\mu) \right)^*} \Norm{f}{L^p(\mu)}
					= \Norm{\varphi}{\left( L^p(\mu) \right)^*} \left\{ \int_X \defunc_{E_n} |g|^q\ d\mu \right\}^{1/p}
				\end{align}
				が従い
				\begin{align}
					\left\{ \int_X \defunc_{E_n} |g|^q\ d\mu \right\}^{1/q} \leq \Norm{\varphi}{\left( L^p(\mu) \right)^*}
				\end{align}
				が得られ,単調収束定理より
				\begin{align}
					\Norm{g}{L^q(\mu)} \leq \Norm{\varphi}{\left( L^p(\mu) \right)^*}
					\label{eq:thm_dual_space_of_L_p_5}
				\end{align}
				が出る.
				
			\item[第六段]
				任意の$f \in L^p(\mu)$に対して,単関数近似列$(f_n)_{n=1}^\infty$は(\refeq{eq:thm_dual_space_of_L_p_2})を満たすから,
				H\Ddot{o}lderの不等式とLebesgueの収束定理より
				\begin{align}
					\left| \varphi(f) - \int_X fg\ d\mu \right|
					&\leq \left| \varphi(f) - \varphi(f_n) \right| + \left| \int_X f_ng\ d\mu - \int_X fg\ d\mu \right| \\
					&\leq \Norm{\varphi}{\left( L^p(\mu) \right)^*} \Norm{f - f_n}{L^p(\mu)}
						+ \Norm{f - f_n}{L^p(\mu)}\Norm{g}{L^q(\mu)} \\
					&\longrightarrow 0 \quad (n \longrightarrow \infty)
				\end{align}
				となり
				\begin{align}
					\varphi = \Phi(g)
				\end{align}
				が成り立つ.また,このとき(\refeq{eq:thm_dual_space_of_L_p_3})と(\refeq{eq:thm_dual_space_of_L_p_4})或は
				(\refeq{eq:thm_dual_space_of_L_p_5})より
				\begin{align}
					\Norm{g}{L^q(\mu)} = \Norm{\varphi}{\left( L^p(\mu) \right)^*}
				\end{align}
				が満たされる.
				
			\item[第七段]
				$\mu(X) = \infty$の場合,補題\ref{lem:Lebesgue_Radon_Nikodym}の関数$w$を用いて
				\begin{align}
					\tilde{\mu}(E) \coloneqq \int_E w\ d\mu,
					\quad (\forall E \in \mathscr{F})
				\end{align}
				により有限測度$\tilde{\mu}$を定める.このとき
				任意の$f \in L^p(\mu)$に対して
				\begin{align}
					F \coloneqq w^{-1/p} f
				\end{align}
				とおけば
				\begin{align}
					\int_X |F|^p\ d\tilde{\mu} = \int_X |F|^p w\ d\mu = \int_X |f|^p\ d\mu
					\label{eq:thm_dual_space_of_L_p_6}
				\end{align}
				が成立し,
				\begin{align}
					L^p \ni f \longmapsto w^{-1/p} f \in L^p(\tilde{\mu})
				\end{align}
				は等長な線型同型となる.ここで任意の$\varphi \in \left( L^p(\mu) \right)^*$に対して
				\begin{align}
					\Psi(F) \coloneqq \varphi\left( w^{1/p} F \right),
					\quad (\forall F \in L^p(\tilde{\mu}))
				\end{align}
				で線形作用素$\Psi$を定めれば
				\begin{align}
					\left| \Psi(F) \right| = \left| \varphi\left( w^{1/p} F \right) \right|
					\leq \Norm{\varphi}{\left( L^p(\mu) \right)^*}\Norm{w^{1/p} F}{L^p(\mu)}
					= \Norm{\varphi}{\left( L^p(\mu) \right)^*}\Norm{F}{L^p(\tilde{\mu})}
				\end{align}
				より$\Psi \in \left( L^p(\tilde{\mu}) \right)^*$が満たされ,かつ
				任意の$f \in L^p(\mu)$に対して
				\begin{align}
					\left| \varphi(f) \right| = \left| \Psi\left( w^{-1/p} f \right) \right|
					\leq \Norm{\Psi}{\left( L^p(\mu) \right)^*}\Norm{w^{-1/p} f}{L^p(\tilde{\mu})}
					= \Norm{\Psi}{\left( L^p(\tilde{\mu}) \right)^*}\Norm{f}{L^p(\mu)}
				\end{align}
				も成り立ち
				\begin{align}
					\Norm{\varphi}{\left( L^p(\mu) \right)^*} = \Norm{\Psi}{\left( L^p(\tilde{\mu}) \right)^*}
				\end{align}
				が得られる.前段までの結果より$\Psi$に対し或る$G \in L^q(\tilde{\mu})$が存在して
				\begin{align}
					\Psi(F) = \int_X FG\ d\tilde{\mu}
				\end{align}
				が成立するから,任意の$f \in L^p(\mu)$に対して
				\begin{align}
					\varphi(f) = \Psi\left( w^{-1/p} f \right)
					= \int_X w^{-1/p} f G w\ d\mu
					= \begin{cases}
						\displaystyle\int_X f G\ d\mu, & (p = 1), \\
						\displaystyle\int_X f w^{1/q} G\ d\mu, & (1 < p < \infty)
					\end{cases}
				\end{align}
				が従い,
				\begin{align}
					g \coloneqq
					\begin{cases}
						G, & (p = 1), \\
						w^{1/q} G, & (1 < p < \infty)
					\end{cases}
				\end{align}
				とおけば(\refeq{eq:thm_dual_space_of_L_p_6})より$g \in L^q(\mu)$となり,
				$\varphi = \Phi(g)$かつ
				\begin{align}
					\Norm{\varphi}{\left( L^p(\mu) \right)^*} = \Norm{\Psi}{\left( L^p(\tilde{\mu}) \right)^*}
					= \Norm{G}{L^q(\tilde{\mu})}
					= \Norm{g}{L^q(\mu)}
				\end{align}
				が満たされる.
				\QED
		\end{description}
	\end{prf}
\section{複素測度に関する積分}
	\begin{screen}
		\begin{thm}[複素測度の極分解]\label{thm:polar_decomposition_of_complex_measures}
			可測空間$(X,\mathscr{F})$上の任意の複素測度$\mu$に対し,次の意味での極分解
			\begin{align}
				\quad \mu(E) = \int_E e^{i\theta}\ d|\mu|,
				\quad (\forall E \in \mathscr{F})
			\end{align}
			を満たす$\mathscr{F}/\borel{\C}$-可測関数$\theta$が存在する.
			$\lambda \not\equiv 0$なら$e^{i \theta}$は
			$L^1(|\mu|)$の元として唯一つに決まる.
		\end{thm}
	\end{screen}
	
	\begin{prf} $\mu \equiv 0$なら$|\mu| \equiv 0$より$\theta \equiv \pi$でよい.
		$\mu \not\equiv 0$の場合,
		Lebesgue-Radon-Nikodymの定理より
		\begin{align}
			\mu(E) = \int_E h\ d|\mu|,
			\quad (\forall E \in \mathscr{F})
		\end{align}
		を満たす$[h] \in L^1(|\mu|)$が唯一つ存在する.このとき$|\mu|(E) > 0$なら
		\begin{align}
			\frac{1}{|\mu|(E)} \left| \int_E h\ d|\mu| \right|
			= \frac{|\mu(E)|}{|\mu|(E)} \leq 1
		\end{align}
		となるから,定理\refeq{thm:mean_value_of_integral_and_closed_set}より
		$|\mu|$-a.e.に$|h| \leq 1$となる.また
		\begin{align}
			E_r \coloneqq \{|h| \leq r\}
		\end{align}
		とおき$\{A_n\}_{n=1}^\infty \subset \mathscr{F}$を$E_r$の任意の分割とすれば,
		\begin{align}
			\sum_{n=1}^\infty |\mu(A_n)|
			= \sum_{n=1}^\infty \left|\int_{A_n} h\ d|\mu|\right|
			\leq \sum_{n=1}^\infty \int_{A_n} |h|\ d|\mu|
			\leq r \sum_{n=1}^\infty |\mu|(A_n)
			= r |\mu|(E_r)
		\end{align}
		が成り立つから$r < 1$なら$|\mu|(E_r) = 0$となり
		\begin{align}
			|\mu|\left(|h|< 1 \right)
			= |\mu| \Biggl(\bigcap_{n=1}^\infty E_{1-1/n} \Biggr)
			= 0
		\end{align}
		が従う.よって$|\mu|$-a.e.に$|h|=1$となる.ここで
		\begin{align}
			\theta(x) \coloneqq
			\begin{cases}
				0, & h(x) = 1, \\
				\pi, & h(x) \neq 1
			\end{cases}
		\end{align}
		と定めれば$[h] = [e^{i \theta}]$が成立する.
		\QED
	\end{prf}
	
	\begin{screen}
		\begin{dfn}[複素測度に関する積分]
			$(X,\mathscr{F})$を可測空間,$\mu$を$(X,\mathscr{F})$上の複素測度,
			$f$を$\mathscr{F}/\borel{\C}$-可測関数とする.
			$f$が$|\mu|$-可積分であるとき,極分解$d\mu = e^{i\theta}\ d|\mu|$を用いて
			\begin{align}
				\int_X f\ d\mu \coloneqq \int_X f e^{i \theta}\ d|\mu|
			\end{align}
			により$f$の$\mu$に関する積分を定める.
		\end{dfn}
	\end{screen}
	
	$\mu \not\equiv 0$なら極分解は定理\ref{thm:polar_decomposition_of_complex_measures}
	の意味で一意であるから$\mu$に関する積分はwell-definedである.
	$\mu \equiv 0$なら$|\mu| \equiv 0$であるから任意の可測写像は$|\mu|$について可積分となり,
	$\mu$に関する積分値は0で確定する(well-defined).
	
	\begin{screen}
		\begin{thm}[総変動測度の積分表現]
			$(X,\mathscr{F},\mu)$を正値測度空間,
			$f$を$\mathscr{F}/\borel{\C}$-可測な$\mu$-可積分関数とするとき,
			\begin{align}
				\lambda(E) \coloneqq \int_E f\ d\mu, \quad (\forall E \in \mathscr{F})
			\end{align}
			で複素測度$\lambda$を定めれば次が成り立つ:
			\begin{align}
				|\lambda|(E) = \int_E |f|\ d\mu, \quad (\forall E \in \mathscr{F}).
			\end{align}
		\end{thm}
	\end{screen}
	
	\begin{screen}
		\begin{thm}[積分の測度に関する線型性]\label{thm:linearity_of_integral_respect_to_complex_measure}
			$(X,\mathscr{F})$を可測空間,$\mu,\nu$をこの上の複素測度とする.$f:X \rightarrow \C$が$|\mu|$と$|\nu|$について可積分であるなら,
			$\alpha,\beta \in \C$に対し$|\alpha \mu + \beta \nu|$についても可積分であり,更に次が成り立つ:
			\begin{align}
				\int_X f\ d(\alpha\mu + \beta\nu) = \alpha \int_X f\ d\mu + \beta \int_X f\ d\nu.
			\end{align}
		\end{thm}
	\end{screen}
	
	\begin{prf}
		\begin{description}
			\item[第一段]
				$f$が可測単関数の場合について証明する.
				$a_i \in \C,\ A_i \in \mathcal{M}\ (i=1,\cdots,n,\ \sum_{i=1}^{n} A_i = X)$を用いて
				\begin{align}
					f = \sum_{i=1}^{n} a_i \defunc_{A_i}
				\end{align}
				と表されている場合,
				\begin{align}
					&\int_X f(x)\ (\alpha\mu + \beta\nu)(dx)
					= \sum_{i=1}^{n} a_i (\alpha\mu + \beta\nu)(A_i) \\
					&\qquad = \alpha \sum_{i=1}^{n} a_i \mu(A_i) + \beta \sum_{i=1}^{n} a_i \nu(A_i)
					= \alpha \int_X f(x)\ \mu(dx) + \beta \int_X f(x)\ \nu(dx)
				\end{align}
				が成り立つ.
				
			\item[第二段]
			$f$が一般の可測関数の場合について証明する.任意の$A \in \mathcal{M}$に対して
			\begin{align}
				\left| (\alpha \mu + \beta \nu)(A) \right| \leq |\alpha||\mu(A)| + |\beta||\nu(A)| \leq |\alpha||\mu|(A) + |\beta||\nu|(A)
 			\end{align}
 			が成り立つから,左辺で$A$を任意に分割しても右辺との大小関係は変わらず
 			\begin{align}
 				|\alpha \mu + \beta \nu|(A) \leq |\alpha||\mu|(A) + |\beta||\nu|(A)
 			\end{align}
 			となる.従って$f$が$|\mu|$と$|\nu|$について可積分であるなら
 			\begin{align}
 				\int_X |f(x)|\ |\alpha \mu + \beta \nu|(dx) \leq |\alpha| \int_X |f(x)|\ |\mu|(dx) + |\beta| \int_X |f(x)|\ |\nu|(dx) < \infty
 			\end{align}
 			が成り立ち前半の主張を得る.$f$の単関数近似列$(f_n)_{n=1}^{\infty}$を取れば,前段の結果と積分の定義より
 			\begin{align}
 				&\left| \int_X f(x)\ (\alpha\mu + \beta\nu)(dx) - \alpha \int_X f(x)\ \mu(dx) - \beta \int_X f(x)\ \nu(dx) \right| \\
 					&\qquad \leq \left| \int_X f(x)\ (\alpha\mu + \beta\nu)(dx) - \int_X f_n(x)\ (\alpha\mu + \beta\nu)(dx) \right| \\
 					&\qquad \quad + |\alpha| \left| \int_X f(x)\ \mu(dx) - \int_X f_n(x)\ \mu(dx) \right|
 					+ |\beta| \left| \int_X f(x)\ \nu(dx) - \int_X f_n(x)\ \nu(dx) \right| \\
 				&\qquad \longrightarrow 0 \quad (n \longrightarrow \infty)
 			\end{align}
 			が成り立ち後半の主張が従う.
 			\QED
		\end{description}
	\end{prf}
	
	\begin{screen}
		\begin{thm}[積分の複素共役]
			$(X,\mathscr{F})$を可測空間,$\mu$を複素測度,
			$f:X \rightarrow \C$を$|\mu|$について可積分な$\mathscr{F}/\borel{\C}$-可測関数とするとき
			次が成り立つ:
			\begin{align}
				\int_X f\ d\overline{\mu}
				= \overline{\int_X \overline{f}\ d\mu}.
			\end{align}
		\end{thm}
	\end{screen}
	
	\begin{prf}
		$u = \Re{f},\ v = \Im{f},\ \gamma = \Re{\mu},\ \theta = \Im{\mu}$とすれば,
		定理\refeq{thm:linearity_of_integral_respect_to_complex_measure}より
		\begin{align}
			\int_X f\ d\overline{\mu} &= \int_X f\ d\gamma - i \int_X f\ d\theta \\
			&= \int_X u\ d\gamma + i \int_X v\ d\gamma - i \int_X u\ d\theta + \int_X v\ d\theta \\
			&= \overline{\int_X u\ d\gamma - i \int_X v\ d\gamma + i \int_X u\ d\theta + \int_X v\ d\theta} \\
			&= \overline{\int_X \overline{f}\ d\gamma + i \int_X \overline{f}\ d\theta} \\
			&= \overline{\int_X \overline{f}\ d\mu}
		\end{align}
		が成立する.
		\QED
	\end{prf}
	
	\begin{screen}
		\begin{thm}[Rieszの表現定理(複素測度)]
		\end{thm}
	\end{screen}
\subsection{正則関数}
	$\alpha$を複素数とするとき,$f$が$\alpha$で微分可能であるということを
	\begin{align}
		f \diffble \alpha \defarrow
		\exists a \in \C\, \forall \epsilon \in \R_+\, \exists \delta \in \R_+\,
		\forall z \in \dom{f}\, 
		\left(\, 0 < |z - \alpha| < \delta \Longrightarrow 
		\left| \frac{f(z) - f(\alpha)}{z-\alpha} - a\right| < \epsilon\, \right)
	\end{align}
	で定め,$f$が$\Omega$上の{\bf 正則関数}\index{せいそくかんすう@正則関数}{\bf (holomorphic function)}であるということを
	\begin{align}
		\hol_\Omega(f) \defarrow f:\Omega \longrightarrow \C 
		\wedge \forall \alpha \in \Omega\, \left(\, f \diffble \alpha\, \right)
	\end{align}
	で定める.また$\Omega$上の正則関数の全体を
	\begin{align}
		\Holomorphic{\Omega} \defeq \Set{f}{\hol_\Omega(f)}
	\end{align}
	と表す.ここで
	\begin{align}
		\operatorname{deriv}_{f,\alpha}(a)
		\defarrow
		\forall \epsilon \in \R_+\, \exists \delta \in \R_+\,
		\forall z \in \dom{f}\, 
		\left(\, 0 < |z - \alpha| < \delta \Longrightarrow 
		\left| \frac{f(z) - f(\alpha)}{z-\alpha} - a\right| < \epsilon\, \right)
	\end{align}
	と略記しておく.
	
	\begin{screen}
		\begin{thm}[微係数の一意性]
			$f$を$\Holomorphic{\Omega}$の要素とし,$\alpha$を$\Omega$の要素とする.このとき
			\begin{align}
				\forall a,b \in \C\, 
				\left(\, \operatorname{deriv}_{f,\alpha}(a) \wedge \operatorname{deriv}_{f,\alpha}(b)
				\Longrightarrow a = b\, \right)
			\end{align}
			が成り立つ.
		\end{thm}
	\end{screen}
	
	\begin{screen}
		\begin{dfn}[導関数]
			$f$を$\Holomorphic{\Omega}$の要素とするとき,
			\begin{align}
				f' \defeq \Set{x}{\exists \alpha \in \Omega\, 
				\exists a \in \C\, \left(\, x=(\alpha,a) \wedge \operatorname{deriv}_{f,\alpha}(a)\, \right)}
			\end{align}
			で定める写像を$f$の{\bf 導関数}\index{どうかんすう@導関数}{\bf (derivative function)}と呼ぶ.
		\end{dfn}
	\end{screen}
	
	\begin{screen}
		\begin{thm}[連鎖律]
			$f$を$\Holomorphic{\Omega}$の要素とし,$\Omega'$を
			\begin{align}
				f \ast \Omega \subset \Omega'
			\end{align}
			を満たす開集合とするとき,$g$を$H(\Omega')$の要素とすれば
			\begin{align}
				g \circ f \in \Holomorphic{\Omega}
			\end{align}
			が成立する.特に
			\begin{align}
				h \coloneqq g \circ f
			\end{align}
			とおけば
			\begin{align}
				\alpha \in \Omega \Longrightarrow h'(\alpha) = g'(f(\alpha)) \cdot f'(\alpha).
			\end{align}
		\end{thm}
	\end{screen}
	
	\begin{sketch}
		$\alpha$を$\Omega$の要素とし,$\epsilon$を正数とする.ここで
		\begin{align}
			\eta^2 + \left(|f'(\alpha)| + |g'(f(\alpha))| \right) \eta = \epsilon
		\end{align}
		を満たす正数$\eta$を取る.$\eta$に対し,
		\begin{align}
			|z-\alpha| < \delta_1 \Longrightarrow
			\left| (f(z) - f(\alpha)) - f'(\alpha)(z-\alpha) \right| < \eta |z-\alpha|
		\end{align}
		を満たす正数$\delta_1$と,
		\begin{align}
			|w-f(\alpha)| < \delta_2 \Longrightarrow
			\left| (g(w) - g(f(\alpha))) - g'(f(\alpha))(w-f(\alpha)) \right| < \eta |w-f(\alpha)|
		\end{align}
		を満たす正数$\delta_2$を取る.また$f$は$\alpha$で連続であるから
		\begin{align}
			|z-\alpha| < \delta_3 \Longrightarrow \left| f(z) - f(\alpha) \right| < \delta_2
		\end{align}
		を満たす正数$\delta_3$が取れる.このとき
		\begin{align}
			\delta \defeq \operatorname{min}\{\delta_1, \delta_3\}
		\end{align}
		とおけば,
		\begin{align}
			|z-\alpha| < \delta
		\end{align}
		なる$\Omega$の任意の要素$z$に対して
		\begin{align}
			\left| \left(g(f(z)) - g(f(\alpha))\right) - g'(f(\alpha))(f(z)-f(\alpha)) \right| 
			&< \eta |f(z)-f(\alpha)| \\
			&< \eta \left( \eta|z-\alpha| + |f'(\alpha)||z-\alpha| \right),
		\end{align}
		及び
		\begin{align}
			&\left| \left(g(f(z)) - g(f(\alpha))\right) - g'(f(\alpha))(f(z)-f(\alpha)) \right| \\
			&= \left| \left(g(f(z)) - g(f(\alpha))\right) - g'(f(\alpha))f'(\alpha)(z-\alpha)
			- g'(f(\alpha)) \left( (f(z) - f(\alpha)) - f'(\alpha)(z-\alpha) \right) \right|
		\end{align}
		から
		\begin{align}
			&\left| \left(g(f(z)) - g(f(\alpha))\right) - g'(f(\alpha))f'(\alpha)(z-\alpha) \right| \\
			&\leq \left| \left(g(f(z)) - g(f(\alpha))\right) - g'(f(\alpha))(f(z)-f(\alpha)) \right|
			+ \left| g'(f(\alpha)) \right| \left| (f(z) - f(\alpha)) - f'(\alpha)(z-\alpha) \right|
		\end{align}
		が成り立つので,
		\begin{align}
			\left| \left(g(f(z)) - g(f(\alpha))\right) - g'(f(\alpha))f'(\alpha)(z-\alpha) \right|
			< \left[ \eta^2 + \left(|f'(\alpha)| + |g'(f(\alpha))| \right) \eta \right] |z-\alpha|
		\end{align}
		が従う.ゆえに
		\begin{align}
			0 < |z-\alpha| < \delta
			\Longrightarrow \left| \frac{h(z) - h(\alpha)}{z-\alpha} - g'(f(\alpha)) \cdot f'(\alpha) \right| < \epsilon
		\end{align}
		が成り立つ.$\alpha$の任意性から
		\begin{align}
			h \in \Holomorphic{\Omega}
		\end{align}
		が従い,また
		\begin{align}
			\alpha \in \Omega \Longrightarrow h'(\alpha) = g'(f(\alpha)) \cdot f'(\alpha)
		\end{align}
		も示された.
		\QED
	\end{sketch}
	
\subsection{解析関数}
	\begin{screen}
		\begin{dfn}[計数測度]
			可測空間$(\Natural,\dirpro{\Natural})$において,
			\begin{align}
				\mu_c \defeq \Set{x}{\exists a\, \left(\, a \subset \Natural \wedge x = (a,\card{a})\, \right)}
			\end{align}
			により定める測度$\mu_c$を{\bf 計数測度}\index{けいすうそくど@計数測度}{\bf (counting measure)}と呼ぶ.
		\end{dfn}
	\end{screen}
	
	\begin{screen}
		\begin{dfn}[級数]
			$f$を$\Natural$上の$\C$値関数とする.
			\begin{align}
				\exists \alpha \in \C\, \left(\, 
				\sum_{k=0}^n f(k) \longrightarrow \alpha\, \right)
			\end{align}
			が成り立つとき,
			\begin{align}
				\sum_{n=0}^\infty f(n) \defeq \lim_{n \to \infty} \sum_{k=0}^n f(k)
			\end{align}
			と書いてこれを$f$の{\bf 級数}\index{きゅうすう@級数}{\bf (series)}と呼ぶ.また
			\begin{align}
				\exists \alpha \in \C\, \left(\, 
				\sum_{k=0}^n |f(k)| \longrightarrow \alpha\, \right)
			\end{align}
			が成り立つとき$f$の級数は{\bf 絶対収束する}\index{ぜったいしゅうそく@絶対収束}{\bf (absolutely converge)}という.
		\end{dfn}
	\end{screen}
	
	\begin{screen}
		\begin{thm}[級数は積分]
			$\Natural$上の$\C$値関数$f$に対して
			\begin{align}
				\sum_{n=0}^\infty |f(n)| < \infty
				\Longleftrightarrow \int_\Natural |f|\ d\mu_c < \infty
			\end{align}
			が成り立ち,かつこのとき
			\begin{align}
				\sum_{n=0}^\infty f(n) = \int_\Natural f\ d\mu_c
			\end{align}
			が成立する.
		\end{thm}
	\end{screen}
	
	$\alpha$を複素数とし,$r$を正数とするとき,中心$\alpha$半径$r$の円板を
	\begin{align}
		D(\alpha;r) \defeq \Set{z}{z \in \C \wedge |z - \alpha| < r}
	\end{align}
	で定める.また中心を抜いた円板を
	\begin{align}
		D'(\alpha;r) \defeq \Set{z}{z \in \C \wedge 0 < |z - \alpha| < r}
	\end{align}
	と定め,$D(\alpha;r)$の閉包は
	\begin{align}
		\overline{D}(\alpha;r)
	\end{align}
	と書く.
	
	\begin{screen}
		\begin{dfn}[解析関数]
			$f$を$\Omega$上の$\C$値関数とするとき,
			\begin{align}
				D(\alpha;r) \subset \Omega
			\end{align}
			なる円板の上で$f$が{\bf 収束級数展開可能である}{\bf (representable by a convergent series)}ということを
			\begin{align}
				&f \representable D(\alpha;r) \defarrow \\
				&\exists c\, \left(\, c:\Natural \longrightarrow \C \wedge
				\limsup_{n \to \infty} \sqrt[n]{c(n)} < \frac{1}{r} \wedge
				\forall z \in D(\alpha;r)\, \left(\, f(z) = \sum_{n=0}^\infty C(n)(z-\alpha)^n\, \right)\, \right)
			\end{align}
			で定める.また$f$が$\Omega$で{\bf 解析的である}\index{かいせきてき@解析的}{\bf (analytic)}ということを
			\begin{align}
				f \analytic \Omega \defarrow
				\forall \alpha \in \C\, \forall r \in \R_+\,
				\left(\, D(\alpha;r) \subset \Omega \Longrightarrow
				f \representable D(\alpha;r)\, \right)
			\end{align}
			で定義する.つまり,解析的とは局所的に収束級数展開可能であるということである.
			$\Omega$上の解析関数の全体を
			\begin{align}
				\Analytic{\Omega} \defeq
				\Set{f}{f:\Omega \longrightarrow \C \wedge f \analytic \Omega}
			\end{align}
			で定める.
		\end{dfn}
	\end{screen}
	
	後述することであるが
	\begin{align}
		\Holomorphic{\Omega} = \Analytic{\Omega}
	\end{align}
	が成立する.解析関数は,後述できないかもしれないWeierstrass解析関数とは別物である.
\input{thms/Radon_Nikodym_theorem}
\section{条件付き期待値}
	\begin{screen}
		\begin{lem}
			$(X,\mathscr{F},\mu)$を$\sigma$-有限測度空間 $(\mu(X) > 0)$とするとき,
			$0 < w < 1$を満たす可積分関数$w$が存在する.
		\end{lem}
	\end{screen}
	
	\begin{prf}
		$\sigma$-有限の仮定より
		\begin{align}
			0 < \mu(X_n) < \infty,\ (\forall n \geq 1),
			\quad X = \bigcup_{n=1}^\infty X_n
		\end{align}
		を満たす$\{X_n\}_{n=1}^\infty \subset \mathscr{F}$が存在する.ここで
		\begin{align}
			w_n(x) \coloneqq
			\begin{cases}
				\displaystyle\frac{1}{2^n\left(1+\mu(X_n)\right)}, & x \in X_n, \\
				0, & x \in X \backslash X_n,
			\end{cases}
			\quad n=1,2,\cdots
		\end{align}
		に対して
		\begin{align}
			w \coloneqq \sum_{n=1}^\infty w_n
		\end{align}
		と定めれば,任意の$x \in X$は或る$X_n$に属するから
		\begin{align}
			0 < w_n(x) \leq w(x)
		\end{align}
		が成り立ち,かつ
		\begin{align}
			w(x) = w_1(x) + \sum_{n=2}^\infty w_n(x)
			\leq \frac{1}{2\left(1+\mu(X_1)\right)} + \frac{1}{2}
			< 1,
			\quad (\forall x \in X)
		\end{align}
		が満たされる.また単調収束定理より
		\begin{align}
			\int_X w\ d\mu \leq \sum_{n=1}^\infty \int_X w_n\ d\mu
			\leq \sum_{n=1}^\infty \frac{\mu(X_n)}{2^n\left(1+\mu(X_n)\right)}
			\leq 1
		\end{align}
		となり$w$の可積分性が出る.
		\QED
	\end{prf}
	
	\begin{screen}
		\begin{thm}[Lebesgue-Radon-Nikodym]
			$(X,\mathscr{F})$を可測空間,$\lambda$を$(X,\mathscr{F})$上の複素測度,
			$\mu$を$(X,\mathscr{F})$上の$\sigma$-有限正値測度とするとき,以下が成立する:
			\begin{description}
				\item[Lebesgue分解]
					$\lambda$は$\mu$に関して絶対連続な$\lambda_a$及び$\mu$と互いに特異な
					$\lambda_s$に一意に分解される:
					\begin{align}
						\lambda = \lambda_a + \lambda_s,
						\quad \lambda_a \ll \mu,
						\quad \lambda_s \perp \mu.
					\end{align}
				
				\item[密度関数の存在]
					$\lambda_a$に対し或る$g \in L^1(\mu) = L^1(X,\mathscr{F},\mu)$が唯一つ存在して次を満たす:
					\begin{align}
						\lambda_a(E) = \int_E g\ d\mu,
						\quad (\forall E \in \mathscr{F}).
					\end{align}
			\end{description}
		\end{thm}
	\end{screen}
	
	\begin{prf}\mbox{}
		\begin{description}
			\item[第一段] Lebesgueの分解の一意性を示す.
				$\lambda'_a \ll \mu$と$\lambda'_s \perp \mu$により
				\begin{align}
					\lambda_a + \lambda_s = \lambda'_a + \lambda'_s
				\end{align}
				が成り立つとき,
				\begin{align}
					\Lambda \coloneqq \lambda_a - \lambda'_a = \lambda'_s - \lambda_s,
					\quad \Lambda \ll \mu,
					\quad \Lambda \perp \mu
				\end{align}
				となり$\Lambda = 0$が従い分解の一意性が出る.
			
			\item[第二段] 密度関数の一意性は
			\item[第三段] Lebesgueの分解と密度関数の存在を示す.
		\end{description}
	\end{prf}
	
	\begin{screen}
		\begin{dfn}[条件付き期待値]
			$(X,\mathscr{F},\mu)$を測度空間,$f \in L^1(\mu)$とする.
			部分$\sigma$-加法族$\mathscr{G} \subset \mathscr{F}$に対し
			$\nu \coloneqq \left. \mu \right|_{\mathscr{G}}$が$\sigma$-有限であるとき,
			\begin{align}
				\lambda(A) \coloneqq \int_A f\ d\mu,
				\quad (\forall A \in \mathscr{G})
			\end{align}
			により$(X,\mathscr{G})$上に複素測度$\lambda$が定まり,$\lambda \ll \nu$であるから
			Lebesgue-Radon-Nikodymの定理より
			\begin{align}
				\lambda(A) = \int_A g\ d\nu,
				\quad (\forall A \in \mathscr{G})
			\end{align}
			を満たす$g \in L^1(\nu) = L^1\left(X,\mathscr{G},\nu\right)$
			が存在する.この$g$を$\mathscr{G}$で条件付けた$f$の条件付き期待値と呼び
			\begin{align}
				g = \cexp{f}{\mathscr{G}}
			\end{align}
			と書く.
		\end{dfn}
	\end{screen}
	
	\begin{screen}
		\begin{thm}
			\begin{description}
				\item[(1)]
					$X_n \leq X_{n+1}$
					$X_n \longrightarrow X\ a.s.P$
					$\cexp{X_n}{\mathscr{G}} \longrightarrow \cexp{X}{\mathscr{G}}\ a.s.P$
				\item[(2)]
					$X_n \geq 0$
					$\cexp{\liminf X_n}{\mathscr{G}} \leq \liminf \cexp{X_n}{\mathscr{G}}$
				\item[(3)]
					$|X_n| \leq Y$ $X_n \longrightarrow X\ a.s.P$
					$\cexp{X_n}{\mathscr{G}} \longrightarrow \cexp{X}{\mathscr{G}}\ a.s.P$
			\end{description}
		\end{thm}
	\end{screen}
	
	\begin{screen}
	\begin{lem}[凸関数の片側微係数の存在]
		任意の凸関数$\varphi:\R \longrightarrow \R$には
		各点で左右の微係数が存在する.特に,凸関数は連続であり,すなわちBorel可測である.
	\end{lem}
	\end{screen}
	
	\begin{prf}
		凸性より任意の$x < y < z$に対して
		\begin{align}
			\frac{\varphi(y) - \varphi(x)}{y - x} 
			\leq \frac{\varphi(z) - \varphi(x)}{z - x}
			\leq \frac{\varphi(z) - \varphi(y)}{z - y}
			\label{ineq:lem:convex_function_measurability_1}
		\end{align}
		が満たされる.従って,$x$を固定すれば,$x$に単調減少に近づく任意の点列$(x_n)_{n=1}^{\infty}$に対し
		 \begin{align}
		 	\left(\frac{f(x_n)-f(x)}{x_n-x}\right)_{n=1}^{\infty} 
		 	\label{seq:lem:convex_function_measurability_2}
		 \end{align}
		 は下に有界な単調減少列となり下限が存在する.$x$に単調減少に近づく別の点列$(y_k)_{k=1}^{\infty}$を取れば
		 \begin{align}
		 	\inf{k \in \N}{\frac{f(y_k)-f(x)}{y_k-x}} \leq \frac{f(x_n)-f(x)}{x_n-x} \quad (n=1,2,\cdots)
		 \end{align}
		 より
		 \begin{align}
		 	\inf{k \in \N}{\frac{f(y_k)-f(x)}{y_k-x}} \leq \inf{n \in \N}{\frac{f(x_n)-f(x)}{x_n-x}}
		 \end{align}
		 が成立し,$(x_n),(y_k)$の立場を変えれば逆向きの不等号も得られる.
		 すなわち極限は点列に依らず確定し,$\varphi$は$x$で右側微係数を持つ.
		 同様に左側微係数も存在し,特に$\varphi$の連続性及びBorel可測性が従う.
		 \QED
	\end{prf}
	
	\begin{screen}
	\begin{thm}[Jensenの不等式]
		$(X,\mathscr{F},\mu)$を測度空間,
		$\mathscr{G} \subset \mathscr{F}$を部分$\sigma$-加法族とし,
		$\left. \mu \right|_{\mathscr{G}}$が$\sigma$-有限であるとする.
		このとき,任意の可積分関数
		$f:X \longrightarrow \R$と
		凸関数$\varphi:\R \longrightarrow \R$に対し,
		$\varphi(f)$が可積分なら次が成立する:
		\begin{align}
			\varphi\left(\cexp{f}{\mathscr{G}} \right)
			\leq \cexp{\varphi(f)}{\mathscr{G}},
			\quad \mbox{$\mu$-a.e.}
		\end{align}
	\end{thm}
	\end{screen}
	
	\begin{prf}
			$\varphi$は各点$x \in \R$で右側接線を持つから,
			それを$\R \ni t \longmapsto a_x t + b_x$と表せば,
			\begin{align}
				\varphi(t) = \sup{r \in \Q}{\left\{ a_r t + b_r \right\}} \quad (\forall t \in \R)
				\label{eq:prp_properties_of_expanded_conditional_expectation_1}
			\end{align}
			が成立する.
			よって任意の$r \in \Q$に対して
			\begin{align}
				\varphi(f(x)) \geq a_r f(x) + b_r
			\end{align}
			が満たされるから
			\begin{align}
				\cexp{\varphi(f)}{\mathscr{G}}
				\geq a_r \cexp{f}{\mathscr{G}} + b_r 
				\quad \mbox{$\mu$-a.e.},
				\quad \forall r \in \Q 
			\end{align}
			が従い,各$r \in \Q$に対し
			\begin{align}
				N_r \coloneqq \Set{x \in X}{\cexp{\varphi(f)}{\mathscr{G}}(x)
				< a_r \cexp{f}{\mathscr{G}}(x) + b_r}
			\end{align}
			とおけば$\mu(N_r) = 0$かつ
			\begin{align}
				\cexp{\varphi(f)}{\mathscr{G}}(x)
				\geq a_r \cexp{f}{\mathscr{G}}(x) + b_r, 
				\quad \forall r \in \Q,\ x \notin \bigcup_{r \in \Q} N_r
			\end{align}
			となる.$r$の任意性と(\refeq{eq:prp_properties_of_expanded_conditional_expectation_1})より
			\begin{align}
				\cexp{\varphi(f)}{\mathscr{G}} \geq \varphi\left( \cexp{f}{\mathscr{G}} \right),
				\quad \mbox{$\mu$-a.e.}
			\end{align}
			が得られる.
			\QED
	\end{prf}
\section{正則条件付測度}
	\begin{screen}
		\begin{dfn}[正則条件付複素測度]
			$(X,\mathscr{F})$を可測空間,$\mathscr{G} \subset \mathscr{F}$を部分$\sigma$-加法族,
			$\mu$を$\mathscr{F}$上の複素測度とするとき,次の(1)(2)(3)を満たす写像
			\begin{align}
				\mu_{\mathscr{G}}(\cdot\, |\, \cdot):\mathscr{F} \times X \longrightarrow \C
			\end{align}
			を$\mathscr{G}$の下での$\mu$の正則条件付複素測度
			(regular conditional complex measure of $\mu$ with respect to $\mathscr{G}$)と呼ぶ:
			\begin{description}
				\item[(1)] 任意の$x \in X$で$\mathscr{F} \ni A \longmapsto \mu_{\mathscr{G}}(A\, |\, x)$は複素測度である.
				\item[(2)] 任意の$A \in \mathscr{F}$で$X \ni x \longmapsto \mu_{\mathscr{G}}(A\, |\, x)$は
					$\mathscr{G}/\borel{\C}$-可測かつ$|\mu|$-可積分である.
				\item[(3)] 任意の$A \in \mathscr{F}$と$B \in \mathscr{G}$に対し次を満たす:
					\begin{align}
						\mu(A \cap B) = \int_B \mu_{\mathscr{G}}(A\, |\, x)\ |\mu|(dx).
					\end{align}
			\end{description}
		\end{dfn}
	\end{screen}
	
	\begin{screen}
		\begin{thm}[正則条件付複素測度の一意性]
			$(X,\mathscr{F})$を可測空間,$\mathscr{G} \subset \mathscr{F}$を部分$\sigma$-加法族,
			$\mu$を$\mathscr{F}$上の複素測度とする.
			$\mathscr{F}$が可算乗法族で生成されるなら,
			$\mu$に対し$\mathscr{G}$の下での正則条件付複素測度
			$\mu_{\mathscr{G}},\nu_{\mathscr{G}}$が存在するとき,
			或る$|\mu|$-零集合$N \in \mathscr{G}$が存在して次が成立する:
			\begin{align}
				\mu_{\mathscr{G}}(A\, |\, x) = \nu_{\mathscr{G}}(A\, |\, x),
				\quad (\forall A \in \mathscr{F},\ \forall x \in X \backslash N).
			\end{align}
		\end{thm}
	\end{screen}
	
	\begin{prf}
		$\mathscr{F}$を生成する可算乗法族を$\{A_n\}_{n=1}^\infty$と書けば,任意の$A_n$に対し
		\begin{align}
			\int_{B} \mu_{\mathscr{G}}(A_n\, |\, x)\ |\mu|(dx)
			= \int_{B} \nu_{\mathscr{G}}(A_n\, |\, x)\ |\mu|(dx),
			\quad (\forall B \in \mathscr{G})
		\end{align}
		が満たされるから$N_n \coloneqq \Set{x \in X}{\mu_{\mathscr{G}}(A_n\, |\, x) \neq \nu_{\mathscr{G}}(A_n\, |\, x)}$
		は$\mathscr{G}$の$|\mu|$-零集合となる.$N \coloneqq \bigcup_{n=1}^\infty N_n$とおけば
		\begin{align}
			\mathscr{D} \coloneqq \Set{A \in \mathscr{F}}{\mu_{\mathscr{G}}(A\, |\, x) = \nu_{\mathscr{G}}(A\, |\, x),\ 
			\forall x \in X \backslash N}
		\end{align}
		によりDynkin族が定まり,$\mathscr{D}$は$\{A_n\}_{n=1}^\infty$を含むから
		Dynkin族定理より定理の主張が従う.
		\QED
	\end{prf}
	
	\begin{screen}
		\begin{thm}[正則条件付測度の存在]
			$(X,\mathscr{F})$を可測空間,$\mathscr{G} \subset \mathscr{F}$を部分$\sigma$-加法族,
			$\mu$を$\mathscr{F}$上の複素測度とする.
			また$\mathscr{F}$が可算族で生成され,かつ或る
			コンパクトクラス$\mathcal{K}$が存在して,
			任意の$\epsilon > 0$と$A \in \mathscr{F}$に対し
			\begin{align}
				A_\epsilon \subset K_\epsilon \subset A,\quad |\mu|(A \backslash A_\epsilon) < \epsilon
			\end{align}
			を満たす$K_\epsilon \in \mathcal{K},\ A_\epsilon \in \mathscr{F}$が取れると仮定する.
			このとき$\mathscr{G}$の下での$\mu$の正則条件付複素測度が存在する.
		\end{thm}
	\end{screen}
\section{一様可積分性}
	\begin{screen}
		\begin{dfn}[一様可積分]
			$(X,\mathscr{F},\mu)$を測度空間とし,$\mathscr{U}$を$\mathscr{L}^1(X,\mathscr{F},\mu)$の部分集合とする.
			\begin{itemize}
				\item $\mathscr{U}$が$\mathscr{L}^1(X,\mathscr{F},\mu)$で有界である:
					\begin{align}
						\sup{f \in \mathscr{U}}\int_X|f|\ d\mu < \infty.
					\end{align}
				
				\item $\epsilon$を任意に与えられた正数とすると,次を満たす正数$\delta$が取れる:
					\begin{align}
						\forall f \in \mathscr{U}\, \forall B \in \mathscr{F}\, \left(\, \mu(B) < \delta
						\Longrightarrow \int_B |f|\ d\mu < \epsilon\, \right).
					\end{align}
			\end{itemize}
			が満たされているとき,$\mathscr{U}$は{\bf 一様可積分}\index{いちようかせきぶん@一様可積分}
			{\bf (uniformly integrable)}であるという.
		\end{dfn}
	\end{screen}
	
	一様可積分な集合の部分集合もまた一様可積分である.
	
	\begin{screen}
	\begin{thm}[一様可積分性の同値条件]\label{thm:appendix_uniform_integrability_equivalence}
		$(X,\mathscr{F},\mu)$を測度空間とし,$\mathscr{U}$を$\mathscr{L}^1(X,\mathscr{F},\mu)$の部分集合とする.
		このとき次の(1)と(2)が成り立つ:
		\begin{description}
			\item[(1)] $\mathscr{U}$が一様可積分であるとき,$\epsilon$を任意に与えられた正数とすると,次を満たす正数$a$が取れる:
				\begin{align}
					\forall f \in \mathscr{U}\, \forall \lambda \in \R_+\,
					\left(\, a < \lambda \Longrightarrow \int_{\{|f| > \lambda\}} |f|\ d\mu < \epsilon\, \right).
				\end{align}
			
			\item[(2)] $\mu(X) < \infty$の場合(1)の逆が成立する.つまり,
				\begin{align}
					\forall \epsilon \in \R_+\, \exists a \in \R_+\, 
					\forall f \in \mathscr{U}\, \forall \lambda \in \R_+\,
					\left(\, a < \lambda \Longrightarrow \int_{\{|f| > \lambda\}} |f|\ d\mu < \epsilon\, \right)
				\end{align}
				が成り立つとき$\mathscr{U}$は一様可積分である.
		\end{description}
	\end{thm}
	\end{screen}
	
	\begin{sketch}\mbox{}
		\begin{description}
			\item[(1)]
				$\mathscr{U}$が一様可積分であるとする.
				いま$\epsilon$を任意に与えられた正数とする.このとき
				\begin{align}
					\forall f \in \mathscr{U}\, \forall B \in \mathscr{F}\, \left(\, \mu(B) < \delta
					\Longrightarrow \int_B |f|\ d\mu < \epsilon\, \right)
				\end{align}
				を満たす$\delta$が取れる.ここで
				\begin{align}
					\frac{1}{a}\sup{f \in \mathscr{U}}{\int_X|f|\ d\mu} < \delta
				\end{align}
				を満たす正の実数$a$を取れば,$a < \lambda$なる正数$\lambda$と$\mathscr{U}$の任意の要素$f$に対して
				\begin{align}
					\mu(|f| > \lambda) \leq \frac{1}{\lambda} \int_X |f|\ d\mu < \delta
				\end{align}
				となるので
				\begin{align}
					\forall f \in \mathscr{U}\, \forall \lambda \in \R_+\,
					\left(\, a < \lambda \Longrightarrow \int_{\{|f| > \lambda\}} |f|\ d\mu < \epsilon\, \right)
				\end{align}
				が成立する.
			
			\item[(2)]
				いま$\epsilon$を任意に与えられた正数とする.このとき
				\begin{align}
					\forall f \in \mathscr{U}\, 
					\left(\, \int_{\{|f| > a\}} |f|\ d\mu < \frac{\epsilon}{2}\, \right)
				\end{align}
				を満たす正数$a$が取れる.$f$を$\mathscr{U}$の任意の要素とし,$B$を$\mathscr{F}$の任意の要素とすれば
				\begin{align}
					\int_B |f|\ d\mu
					= \int_{\{|f|>a\} \cap B} |f|\ d\mu
						+ \int_{\{|f| \leq a\} \cap B} |f|\ d\mu
					\leq \frac{\epsilon}{2} + a\mu(B)
				\end{align}				
				が成り立つから,
				\begin{align}
					\sup{f \in \mathscr{U}}\int_X|f|\ d\mu < \infty
				\end{align}
				及び
				\begin{align}
					\forall B \in \mathscr{F}\, \left(\, \mu(B) < \frac{\epsilon}{2a}
					\Longrightarrow \int_B |f|\ d\mu < \epsilon\, \right)
				\end{align}
				が成立する.すなわち$\mathscr{U}$は一様可積分である.
				\QED
		\end{description}
	\end{sketch}
	
	\begin{screen}
	\begin{thm}[一様可積分性と平均収束]\label{lem:uniformly_integrable_and_convergence_in_mean}
		$(X,\mathscr{F},\mu)$を正値測度空間とし,
		\begin{align}
			\mu(X) < \infty
		\end{align}
		とする.また$\{f_n\}_{n=1}^\infty$を$\mathscr{L}^1(X,\mathscr{F},\mu)$の部分集合とし,
		$A$を$\mu$-零集合とし,$X \backslash A$の各点$x$で$(f_n(x))_{n=1}^\infty$が
		$\C$で収束するとする.このとき次の(1)と(2)は同値である:
		\begin{description}
			\item[(1)] $\{f_n\}_{n=1}^\infty$が一様可積分.
			\item[(2)] $f \defeq \lim_{n \to \infty} f_n \defunc_A$と$f$を定めると,$f$は可積分で
				\begin{align}
					\int_X |f - f_n|\ d\mu 
					\longrightarrow 0
					\quad (n \longrightarrow \infty).
				\end{align}
		\end{description}
	\end{thm}
	\end{screen}
	
	\begin{sketch}
		
	\end{sketch}
	
	\begin{screen}
	\begin{thm}[一様可積分性と条件付き期待値]\label{lem:uniformly_integrability_and_conditional_expectations}
		$(X,\mathscr{F},\mu)$を測度空間とし,$\mu(X) < \infty$とし,
		$f$を$\mathscr{L}^1(X,\mathscr{F},\mu)$の要素とする.
		また$\mathscr{S}$を$X$上の$\sigma$-加法族であり$\mathscr{F}$の部分集合であるものの全体とする.
		このとき$\left\{ \cexp{f}{\mathscr{G}} \right\}_{\mathscr{G} \in \mathscr{S}}$は一様可積分である.
	\end{thm}
	\end{screen}
	
	\begin{prf}
		定理\ref{thm:properties_of_conditional_expectations}より
				\begin{align}
					\int_{\left| \cexp{f}{\mathscr{G}} \right| > \lambda} \left| \cexp{f}{\mathscr{G}} \right|\ d\mu
					\leq \int_{\cexp{|f|}{\mathscr{G}} > \lambda} \cexp{|f|}{\mathscr{G}}\ d\mu
					= \int_{\cexp{|f|}{\mathscr{G}} > \lambda} |f|\ d\mu
				\end{align}
				が成り立つ.また$X$の可積分性より,任意の$\epsilon > 0$に対して
				或る$\delta > 0$が存在し
				\begin{align}
					\mu(B) < \delta \Rightarrow \int_B |f|\ d\mu < \epsilon
				\end{align}
				が満たされる.いま,Chebyshevの不等式より
				\begin{align}
					\mu\left( \cexp{|f|}{\mathscr{G}} > \lambda \right)
					\leq \frac{1}{\lambda} \int_X \cexp{|f|}{\mathscr{G}}\ d\mu
					= \frac{1}{\lambda} \int_X |f|\ d\mu
				\end{align}
				となるから,$\epsilon > 0$に対し或る$\lambda_0 > 0$が存在して
				\begin{align}
					\sup{\mathscr{G} \in \mathscr{S}}{\mu\left( \cexp{|f|}{\mathscr{G}} > \lambda \right)}
					< \delta,
					\quad (\forall \lambda > \lambda_0)
				\end{align}
				が満たされ
				\begin{align}
					\sup{\mathscr{G} \in \mathscr{S}}{\int_{\cexp{|f|}{\mathscr{G}} > \lambda}|f|\ d\mu}
					< \epsilon,
					\quad (\forall \lambda > \lambda_0)
				\end{align}
				が従う.
		\QED
	\end{prf}
\section{連続写像の空間の位相}
	$(X,d_X),(Y,d_Y)$を距離空間とし,
	\begin{align}
		C(X,Y) \coloneqq \Set{f:X \longrightarrow Y}{\mbox{$f$は連続写像}}
	\end{align}
	とおく.$X$が$\sigma$-コンパクトであるとき,つまり
	\begin{align}
		K_1 \subset K_2 \subset K_3 \subset \cdots,
		\quad \bigcup_{n=1}^\infty K_n = X 
	\end{align}
	を満たすコンパクト部分集合の列$(K_n)_{n=1}^\infty$が存在するとき,
	\begin{align}
		\rho(f,g) \coloneqq \sum_{n=1}^\infty 2^{-n} \left( 1 \wedge \sup{x \in K_n}{d_Y(f(x),g(x))} \right),
		\quad (f,g \in C(X,Y))
	\end{align}
	により定める$\rho$は$C(X,Y)$上の距離関数となる.
	実際,$f \in C(X,Y)$に対し$f(K_n)$はコンパクトであるから
	$\operatorname{diam}(f(K_n)) < \infty$
	\bddddegin{align}
		d_Y(f(x),g(x)) \leq d_Y(f(x),f(x_0)) + d_Y(f(x_0),g(x_0)) + d_Y(g(x_0),g(x))
		\leq \operatorname{diam}(f(K_n)) + d_Y(f(x_0),g(x_0)) + \operatorname{diam}(g(K_n))
	\end{align}
	
	\begin{screen}
		\begin{thm}
			$X$を$\sigma$-コンパクトな距離空間,$Y$を距離空間とするとき$C(X,Y)$は可分距離空間である.
		\end{thm}
	\end{screen}

\chapter{}
\section{メタ数学}
	\section{推論}
	本節では,「集合でも真類でもない類は存在しない」と「集合であり真類でもある類は存在しない」の二つの言明の正否の決定を主軸にして
	{\bf 推論規則}\index{すいろんきそく@推論規則}{\bf (rule of inference)}を導入し,基本的な推論法則を導出する.
	
	\begin{screen}
		\begin{logicalaxm}[排中律]
			$A$を任意の文とするとき次は定理である:
			\begin{align}
				A \vee \rightharpoondown A.
			\end{align}
		\end{logicalaxm}
	\end{screen}
	
	排中律の言明は``どんな文でも持ってくれば,その式に対して排中律が適用される''という意味である.
	このように無数に存在し得る定理を一括して表す式は{\bf 公理図式}\index{こうりずしき@公理図式}{\bf (schema)}と呼ばれる.
	
	いま$a,b$を類とするとき,
	\begin{align}
		a \notin b \defarrow\ \rightharpoondown a \in b
	\end{align}
	で$a \notin b$を定める.同様に
	\begin{align}
		a \neq b \defarrow\ \rightharpoondown a = b
	\end{align}
	で$a \neq b$を定める.
	
	\monologue{
		定義記号$\defeq$と同様に,`$A \defarrow B$'とは
		式$B$を記号列$A$で置き換えて良いという意味で使われます.また,式中に記号列$A$が出てくるときは,
		暗黙裡にその$A$を$B$に戻して式を解釈します.
		$\defeq$も$\defarrow$も略記することと同じですね.
	}
	
	\begin{screen}
		\begin{thm}[類は集合であるか真類であるかのいずれかに定まる]
			$a$を類とするとき次は定理である:
			\begin{align}
				\set{a} \vee \rightharpoondown \set{a}.
			\end{align}
		\end{thm}
	\end{screen}
	
	\begin{prf}
		排中律を適用することにより従う.
		\QED
	\end{prf}
	
	排中律をそのまま適用することにより上の定理は導かれたが,``集合であり真類でもある類は存在しない''という主張はまだ得られない.
	以下はこの言明を証明することを目標にしてしばらく推論規則の話が続くが,提示される規則はどれも基本的で直感に反しないため
	通常は無断で使用されてしまうものである.
	
	ここで論理記号の名称を書いておく.
	\begin{itemize}
		\item $\bot$を{\bf 矛盾}\index{むじゅん@矛盾}{\bf (contradiction)}と呼ぶ.
		\item $\vee$を{\bf 論理和}\index{ろんりわ@論理和}{\bf (logical disjunction)}と呼ぶ.
		\item $\wedge$を{\bf 論理積}\index{ろんりせき@論理積}{\bf (logical conjunction)}と呼ぶ.
		\item $\Longrightarrow$を{\bf 含意}\index{がんい@含意}{\bf (implication)}と呼ぶ.
		\item $\rightharpoondown$を{\bf 否定}\index{ひてい@否定}{\bf (negation)}と呼ぶ.
	\end{itemize}
	
	\begin{screen}
		\begin{logicalaxm}[基本的な推論規則]\label{logicalaxm:fundamental_rules_of_inference}
			$A,B,C$を$\mathcal{L}'$の閉式とするとき,次の規則を認める:
			\begin{description}
				\item[三段論法] $A$ならびに$A \Longrightarrow B$が定理なら$B$は定理である.
				\item[演繹法則] $A$を公理に追加した下で$B$が定理であるなら,
					$A$を外した公理系で$A \Longrightarrow B$は定理である.
				\item[論理和の導入イ] $A \Longrightarrow (A \vee B)$は定理である.
				\item[論理和の導入ロ] $A \Longrightarrow (B \vee A)$は定理である.
				\item[論理積の導入] $A,B$が共に定理なら$A \wedge B$は定理である.
				\item[論理積の除去イ] $(A \wedge B) \Longrightarrow A$は定理である.
				\item[論理積の除去ロ] $(A \wedge B) \Longrightarrow B$は定理である.
				\item[場合分け法則] $A \Longrightarrow C$と$B \Longrightarrow C$が共に定理であるとき
					$(A \vee B) \Longrightarrow C$は定理である.
			\end{description}	
		\end{logicalaxm}
	\end{screen}
	
	\monologue{
		演繹法則について,``$A$を公理に追加する''ことを``$A$が成り立っていると仮定する''
		などの言明により示唆することが多いです.
	}
	
	\begin{itembox}[l]{演繹法則の意味}
		我々は公理か,或いは公理図式として,複数の式を選び出し$\mathcal{L}'$の世界において正しいと決める.
		それらは以降小出しに登場させるが,その全体は現段階ですでに決めているのでそれを
		\begin{align}
			\mathscr{S}
		\end{align}
		と呼ぶことにする.本稿で出てくる``正しい式''とは$\mathscr{S}$のみを公理系とした体系において証明される式を指す.
		演繹法則は,``$A$が成り立つとする''などの
		言明により$\mathscr{S}$に式$A$を加えたとき,その新しい公理系$\mathscr{S}'$の下で式$B$が成り立つなら,
		$\mathscr{S}$のみを公理とした体系において
		\begin{align}
			A \Longrightarrow B
		\end{align}
		が成立する,と主張している.複数の式を$\mathscr{S}$に追加する場合もある.
		たとえば$\mathscr{S}'$に式$C$を追加し,その新しい公理体系$\mathscr{S}''$の下で
		式$D$が成り立つ場合,演繹法則に則れば$\mathscr{S}'$の下で
		\begin{align}
			C \Longrightarrow D
		\end{align}
		が成立する.(ちなみに$\mathscr{S}''$から式$A$のみを抜いた公理系の下では
		$A \Longrightarrow D$が正しくなる.)
		このとき$\mathscr{S}$を公理系とした下では,再び演繹法則を適用することにより
		\begin{align}
			A \Longrightarrow (C \Longrightarrow D)
		\end{align}
		が成立するとわかる,が,
		\begin{align}
			C \Longrightarrow D
		\end{align}
		が成り立つ保証は無い.非常に屡々いくつも仮定を重ねたところに演繹法則を運用することがあるが,
		その都度どの段階の公理系を扱っているかを明確に把握しておかないと推論が破綻してしまう恐れがある.
	\end{itembox}
	
	\begin{screen}
		\begin{logicalthm}[含意の反射律]\label{logicalthm:reflective_law_of_implication}
			$A$を文とするとき
			\begin{align}
				\vdash A \Longrightarrow A.
			\end{align}
		\end{logicalthm}
	\end{screen}
	
	\begin{prf}
		$A \vdash A$であるから,演繹法則より$\vdash A \Longrightarrow A$となる.
		\QED
	\end{prf}
	
	\begin{screen}
		\begin{logicalthm}[論理和・論理積の可換律]
		\label{logicalthm:commutative_law_of_disjunction_and_conjunction}
			$A,B$を文とするとき
			\begin{itemize}
				\item $\vdash (A \vee B) \Longrightarrow (B \vee A)$.
				\item $\vdash (A \wedge B) \Longrightarrow (B \wedge A)$.
			\end{itemize}
		\end{logicalthm}
	\end{screen}
	
	\begin{prf}
		$\vee$の導入により
		\begin{align}
			\vdash A \Longrightarrow (B \vee A)
		\end{align}
		と
		\begin{align}
			\vdash B \Longrightarrow (A \vee B)
		\end{align}
		が成り立つので,場合分け法則より
		\begin{align}
			\vdash (A \vee B) \Longrightarrow (B \vee A)
		\end{align}
		が成り立つ.また,$\wedge$の除去より
		\begin{align}
			A \wedge B \vdash A
		\end{align}
		と
		\begin{align}
			A \wedge B \vdash B
		\end{align}
		となるので,$\wedge$の導入により
		\begin{align}
			A \wedge B \vdash B \wedge A
		\end{align}
		が成り立つ.よって演繹法則より
		\begin{align}
			\vdash (A \wedge B) \Longrightarrow (B \wedge A)
		\end{align}
		が成り立つ.
		\QED
	\end{prf}
	
	\begin{screen}
		\begin{logicalthm}[含意の推移律]\label{logicalthm:transitive_law_of_implication}
			$A,B,C$を文とするとき
			\begin{align}
				\vdash ((A \Longrightarrow B) \wedge (B \Longrightarrow C)) 
				\Longrightarrow (A \Longrightarrow C).
			\end{align}
		\end{logicalthm}
	\end{screen}
	
	\begin{prf}
		\begin{align}
			(A \Longrightarrow B) \wedge (B \Longrightarrow C),A \vdash 
			(A \Longrightarrow B) \wedge (B \Longrightarrow C)
		\end{align}
		であるから,$\wedge$の除去より
		\begin{align}
			(A \Longrightarrow B) \wedge (B \Longrightarrow C),A \vdash A \Longrightarrow B
		\end{align}
		となる.また
		\begin{align}
			(A \Longrightarrow B) \wedge (B \Longrightarrow C),A \vdash A
		\end{align}
		でもあるから,三段論法より
		\begin{align}
			(A \Longrightarrow B) \wedge (B \Longrightarrow C),A \vdash B
		\end{align}
		となる.$\wedge$の除去より
		\begin{align}
			(A \Longrightarrow B) \wedge (B \Longrightarrow C),A \vdash B \Longrightarrow C
		\end{align}
		も成り立つから,再び三段論法より
		\begin{align}
			(A \Longrightarrow B) \wedge (B \Longrightarrow C),A \vdash C
		\end{align}
		となる.よって演繹法則より
		\begin{align}
			(A \Longrightarrow B) \wedge (B \Longrightarrow C) \vdash A \Longrightarrow C
		\end{align}
		となり,
		\begin{align}
			\vdash ((A \Longrightarrow B) \wedge (B \Longrightarrow C)) 
			\Longrightarrow (A \Longrightarrow C)
		\end{align}
		を得る.
		\QED
	\end{prf}
	
	\begin{screen}
		\begin{logicalthm}[二式が同時に導かれるならその論理積が導かれる]
		\label{logicalthm:conjunction_of_consequences}
			$A,B,C$を文とするとき
			\begin{align}
				\vdash ((A \Longrightarrow B) \wedge (A \Longrightarrow C))
				\Longrightarrow (A \Longrightarrow (B \wedge C))
			\end{align}
		\end{logicalthm}
	\end{screen}
	
	\begin{prf}
		\begin{align}
			(A \Longrightarrow B) \wedge (A \Longrightarrow C),A \vdash
			(A \Longrightarrow B) \wedge (A \Longrightarrow C)
		\end{align}
		であるから,$\wedge$の除去より
		\begin{align}
			(A \Longrightarrow B) \wedge (A \Longrightarrow C),A \vdash
			A \Longrightarrow B
		\end{align}
		が成り立つ.
		\begin{align}
			(A \Longrightarrow B) \wedge (A \Longrightarrow C),A \vdash A
		\end{align}
		でもあるから
		\begin{align}
			(A \Longrightarrow B) \wedge (A \Longrightarrow C),A \vdash B
		\end{align}
		となる.同様にして
		\begin{align}
			(A \Longrightarrow B) \wedge (A \Longrightarrow C),A \vdash C
		\end{align}
		となるので,$\wedge$の導入により
		\begin{align}
			(A \Longrightarrow B) \wedge (A \Longrightarrow C),A \vdash B \wedge C
		\end{align}
		となり,演繹法則より
		\begin{align}
			(A \Longrightarrow B) \wedge (A \Longrightarrow C) \vdash
			A \Longrightarrow (B \wedge C)
		\end{align}
		が成り立つ.ゆえに
		\begin{align}
			\vdash ((A \Longrightarrow B) \wedge (A \Longrightarrow C))
			\Longrightarrow (A \Longrightarrow (B \wedge C))
		\end{align}
		が得られる.
		\QED
	\end{prf}
	
	\begin{screen}
		\begin{logicalthm}[含意は遺伝する]\label{logicalthm:rule_of_inference_1}
			$A,B,C$を$\mathcal{L}'$の閉式とするとき以下が成り立つ:
			\begin{description}
				\item[(a)] $(A \Longrightarrow B) \Longrightarrow ( (A \vee C) \Longrightarrow (B \vee C) )$.
				
				\item[(b)] $(A \Longrightarrow B) \Longrightarrow ( (A \wedge C) \Longrightarrow (B \wedge C) )$.
				
				\item[(c)] $(A \Longrightarrow B) \Longrightarrow ( (B \Longrightarrow C) \Longrightarrow (A \Longrightarrow C) )$.
				
				\item[(c)] $(A \Longrightarrow B) \Longrightarrow ( (C \Longrightarrow A) \Longrightarrow (C \Longrightarrow B) )$.
			\end{description}
		\end{logicalthm}
	\end{screen}
	
	\begin{prf}\mbox{}
		\begin{description}
			\item[(a)]
				いま$A \Longrightarrow B$が成り立っていると仮定する.
				論理和の導入により
				\begin{align}
					C \Longrightarrow (B \vee C)
				\end{align}
				は定理であるから,含意の推移律より
				\begin{align}
					A \Longrightarrow (B \vee C)
				\end{align}
				が従い,場合分け法則より
				\begin{align}
					(A \vee C) \Longrightarrow (B \vee C)
				\end{align}
				が成立する.ここに演繹法則を適用して
				\begin{align}
					(A \Longrightarrow B) \Longrightarrow 
					( (A \vee C) \Longrightarrow (B \vee C) )
				\end{align}
				が得られる.
				
			\item[(b)]
				いま$A \Longrightarrow B$が成り立っていると仮定する.論理積の除去より
				\begin{align}
					(A \wedge C) \Longrightarrow A
				\end{align}
				は定理であるから,含意の推移律より
				\begin{align}
					(A \wedge C) \Longrightarrow B
				\end{align}
				が従い,他方で論理積の除去より
				\begin{align}
					(A \wedge C) \Longrightarrow C
				\end{align}
				も満たされる.そして推論法則\ref{logicalthm:conjunction_of_consequences}から
				\begin{align}
					(A \wedge C) \Longrightarrow (B \wedge C)
				\end{align}
				が成り立ち,演繹法則より
				\begin{align}
					(A \Longrightarrow B) \Longrightarrow ((A \wedge C) \Longrightarrow (B \wedge C))
				\end{align}
				が得られる.
				
			\item[(c)]
				いま$A \Longrightarrow B$,$B \Longrightarrow C$および
				$A$が成り立っていると仮定する.このとき三段論法より$B$が成り立つので再び三段論法より
				$C$が成立する.ゆえに演繹法則より$A \Longrightarrow B$と$B \Longrightarrow C$が
				成り立っている下で
				\begin{align}
					A \Longrightarrow C
				\end{align}
				が成立し,演繹法則を更に順次適用すれば
				\begin{align}
					(A \Longrightarrow B) \Longrightarrow ( (B \Longrightarrow C) \Longrightarrow (A \Longrightarrow C) )
				\end{align}
				が得られる.
				
			\item[(d)]
				いま$A \Longrightarrow B$,$C \Longrightarrow A$および
				$C$が成り立っていると仮定する.このとき三段論法より$A$が成り立つので再び三段論法より$B$が成立し,
				ここに演繹法則を適用すれば,$A \Longrightarrow B$と$C \Longrightarrow A$が成立している下で
				\begin{align}
					C \Longrightarrow B
				\end{align}
				が成立する.演繹法則を更に順次適用すれば
				\begin{align}
					(A \Longrightarrow B) \Longrightarrow ( (C \Longrightarrow A) \Longrightarrow (C \Longrightarrow B) )
				\end{align}
				が得られる.
				\QED
		\end{description}
	\end{prf}
	
	\begin{screen}
		\begin{logicalthm}[正しい式は仮定を選ばない]\label{logicalthm:rule_of_inference_2}
			$A,B$を$\mathcal{L}'$の閉式とするとき,
			$B \Longrightarrow (A \Longrightarrow B)$は定理である.
		\end{logicalthm}
	\end{screen}
	
	\begin{prf}
		$B$を公理に追加した場合,$A$を公理に追加しても$B$は真であるから,このとき
		\begin{align}
			A \Longrightarrow B
		\end{align}
		は定理となる.従って演繹法則より$B \Longrightarrow (A \Longrightarrow B)$は定理である.
		\QED
	\end{prf}
	
	$A$と$B$を$\mathcal{L}'$の式とするとき,
	\begin{align}
		(A \Longleftrightarrow B) \defarrow
		(A \Longrightarrow B \wedge B \Longrightarrow A)
	\end{align}
	により$\Longleftrightarrow$を定め,式`$A \Longleftrightarrow B$'を
	``$A$と$B$は{\bf 同値である}\index{どうち@同値}{\bf (equivalent)}''と翻訳する.
	
	\begin{screen}
		\begin{logicalthm}[同値記号の可換律]\label{logicalthm:commutative_law_of_equivalence}
			$A$と$B$を$\mathcal{L}'$の閉式とするとき
			\begin{align}
				(A \Longleftrightarrow B) \Longrightarrow (B \Longleftrightarrow A).
			\end{align}
		\end{logicalthm}
	\end{screen}
	
	\begin{sketch}
		$A \Longleftrightarrow B$が成り立っているならば,推論法則\ref{logicalthm:commutative_law_of_disjunction_and_conjunction}より
		\begin{align}
			B \Longrightarrow A \wedge A \Longrightarrow B
		\end{align}
		が成立する.すなわち
		\begin{align}
			B \Longleftrightarrow A
		\end{align}
		が成立する.そして演繹法則から
		\begin{align}
			(A \Longleftrightarrow B) \Longrightarrow (B \Longleftrightarrow A)
		\end{align}
		が成立する.
		\QED
	\end{sketch}
	
	\begin{screen}
		\begin{logicalthm}[同値記号の遺伝性質]\label{logicalthm:hereditary_of_equivalence}
			$A,B,C$を$\mathcal{L}'$の閉式とするとき以下の式が成り立つ:
			\begin{description}
				\item[(a)] $(A \Longleftrightarrow B) \Longrightarrow ((A \vee C) \Longleftrightarrow (B \vee C))$.
				\item[(b)] $(A \Longleftrightarrow B) \Longrightarrow ((A \wedge C) \Longleftrightarrow (B \wedge C))$.
				\item[(c)] $(A \Longleftrightarrow B) \Longrightarrow ((B \Longrightarrow C) \Longleftrightarrow (A \Longrightarrow C))$.
				
				\item[(d)] $(A \Longleftrightarrow B) \Longrightarrow ((C \Longrightarrow A) \Longleftrightarrow (C \Longrightarrow B))$.
			\end{description}
		\end{logicalthm}
	\end{screen}
	
	\begin{prf}
		まず(a)を示す.いま$A \Longleftrightarrow B$が成り立っていると仮定する.このとき$A \Longrightarrow B$と
		$B \Longrightarrow A$が共に成立し,他方で含意の遺伝性質より
		\begin{align}
			&(A \Longrightarrow B) \Longrightarrow ((A \vee C) \Longrightarrow (B \vee C)), \\
			&(B \Longrightarrow A) \Longrightarrow ((B \vee C) \Longrightarrow (A \vee C))
		\end{align}
		が成立するから三段論法より$(A \vee C) \Longrightarrow (B \vee C)$と
		$(B \vee C) \Longrightarrow (A \vee C)$が共に成立する.ここに$\wedge$の導入を適用すれば
		\begin{align}
			(A \vee C) \Longleftrightarrow (B \vee C)
		\end{align}
		が成立し,演繹法則を適用すれば
		\begin{align}
			(A \Longleftrightarrow B) \Longrightarrow ((A \vee C) \Longleftrightarrow (B \vee C))
		\end{align}
		が得られる.(b)(c)(d)も含意の遺伝性を適用すれば得られる.
		\QED
	\end{prf}
	
	\begin{screen}
		\begin{logicalaxm}[矛盾と否定に関する規則]\label{logicalaxm:rules_of_contradiction}
			$A$を$\mathcal{L}'$の閉式とするとき以下の式が成り立つ:
			\begin{description}
				\item[矛盾の発生] 否定が共に成り立つとき矛盾が導かれる:
					\begin{align}
						(A \wedge \rightharpoondown A) \Longrightarrow \bot.
					\end{align}
				\item[否定の導出] 矛盾が導かれるとき否定が成り立つ:
					\begin{align}
						(A \Longrightarrow \bot) \Longrightarrow\ \rightharpoondown A.
					\end{align}
				\item[二重否定の法則] 二重に否定された式は元の式を導く:
					\begin{align}
						\rightharpoondown \rightharpoondown A \Longrightarrow A.
					\end{align}
			\end{description}
		\end{logicalaxm}
	\end{screen}
	
	\monologue{
		$A$を$\mathcal{L}'$の閉式とするとき,式$A \Longrightarrow \bot$を
		``$A$は{\bf 偽である}\index{ぎ@偽}{\bf (false)}''と翻訳します.
	}
	
	否定の導出の逆は定理として得られる.
	\begin{screen}
		\begin{logicalthm}[否定が正しい式は偽である]\label{logicalthm:false_and_negation_are_equivalent}
			$A$を$\mathcal{L}'$の閉式とするとき次が成り立つ:
			\begin{align}
				\rightharpoondown A \Longrightarrow (A \Longrightarrow \bot).
			\end{align}
		\end{logicalthm}
	\end{screen}
	
	\begin{prf}
		$\rightharpoondown A$が成り立っていると仮定する.このとき$A$が成り立っていれば
		推論規則\ref{logicalaxm:rules_of_contradiction}より$\bot$が成立するから,演繹法則より
		\begin{align}
			\rightharpoondown A \Longrightarrow (A \Longrightarrow \bot)
		\end{align}
		が成り立つ.
		\QED
	\end{prf}
	
	\begin{screen}
		\begin{logicalthm}[矛盾からはあらゆる式が導かれる]\label{logicalthm:contradiction_derives_any_formula}
			$A$を$\mathcal{L}'$の閉式とするとき
			\begin{align}
				\bot \Longrightarrow A.
			\end{align}
		\end{logicalthm}
	\end{screen}
	
	\begin{prf}
		推論法則\ref{logicalthm:rule_of_inference_2}より
		\begin{align}
			\bot \Longrightarrow (\rightharpoondown A \Longrightarrow \bot)
		\end{align}
		が成り立つ.また否定の導出より
		\begin{align}
			(\rightharpoondown A \Longrightarrow \bot) \Longrightarrow\ \rightharpoondown \rightharpoondown A
		\end{align}
		も成り立ち,さらに二重否定の法則から
		\begin{align}
			\rightharpoondown \rightharpoondown A \Longrightarrow A
		\end{align}
		も成り立つ.上の式に含意の推移律を適用すれば
		\begin{align}
			\bot \Longrightarrow A
		\end{align}
		が得られる.
		\QED
	\end{prf}
	
	\begin{screen}
		\begin{logicalthm}[背理法の原理]
			$A$を$\mathcal{L}'$の閉式とするとき
			\begin{align}
				(\rightharpoondown A \Longrightarrow \bot) \Longrightarrow A.
			\end{align}
		\end{logicalthm}
	\end{screen}
	
	\begin{prf}
		$\rightharpoondown A \Longrightarrow \bot$が成り立つとき,否定の導出より
		$\rightharpoondown \rightharpoondown A$が成り立つが,二重否定の法則より
		$A$も成立する.
		\QED
	\end{prf}
	
	\begin{screen}
		\begin{logicalthm}[矛盾を導く式はあらゆる式を導く]\label{logicalthm:formula_leading_to_contradiction_derives_any_formula}
			$A,B$を$\mathcal{L}'$の閉式とするとき,次が成り立つ:
			\begin{align}
				(A \Longrightarrow \bot) \Longrightarrow (A \Longrightarrow B).
			\end{align}
		\end{logicalthm}
	\end{screen}
	
	\begin{prf}
		$A \Longrightarrow \bot$が成り立っているとする.推論法則\ref{logicalthm:contradiction_derives_any_formula}より
		\begin{align}
			\bot \Longrightarrow B
		\end{align}
		が満たされるので,含意の推移律より
		\begin{align}
			A \Longrightarrow B
		\end{align}
		が成り立つ.従って演繹法則を適用すれば
		\begin{align}
			(A \Longrightarrow \bot) \Longrightarrow (A \Longrightarrow B)
		\end{align}
		が得られる.
		\QED
	\end{prf}
	
	\begin{screen}
		\begin{logicalthm}[含意は否定と論理和で表せる]\label{logicalthm:rule_of_inference_3}
			$A,B$を$\mathcal{L}'$の閉式とするとき,次が成り立つ:
			\begin{align}
				(A \Longrightarrow B) \Longleftrightarrow (\rightharpoondown A \vee B).
			\end{align}
		\end{logicalthm}
	\end{screen}
	
	\begin{prf}
		$A \Longrightarrow B$が成り立っていると仮定する.含意の遺伝性質より
		\begin{align}
			(A \Longrightarrow B) \Longrightarrow 
			((A \vee \rightharpoondown A) \Longrightarrow (B \vee \rightharpoondown A))
		\end{align}
		が満たされているから三段論法より
		\begin{align}
			(A \vee \rightharpoondown A) \Longrightarrow (B \vee \rightharpoondown A)
		\end{align}
		は定理となり,ここに排中律と三段論法を適用すれば
		\begin{align}
			B \vee \rightharpoondown A
		\end{align}
		が定理となる.
		ここで論理和の可換律より$\rightharpoondown A \vee B$が成り立つので,演繹法則を適用して
		\begin{align}
			(A \Longrightarrow B) \Longrightarrow (\rightharpoondown A \vee B)
		\end{align}
		が得られる.また矛盾に関する推論規則より
		\begin{align}
			\rightharpoondown A \Longrightarrow (A \Longrightarrow \bot)
		\end{align}
		が成り立ち,同時に推論法則\ref{logicalthm:formula_leading_to_contradiction_derives_any_formula}より
		\begin{align}
			(A \Longrightarrow \bot) \Longrightarrow (A \Longrightarrow B)
		\end{align}
		も成り立つので,含意の推移律より
		\begin{align}
			\rightharpoondown A \Longrightarrow (A \Longrightarrow B)
		\end{align}
		が成立する.他方で推論法則\ref{logicalthm:rule_of_inference_2}より
		\begin{align}
			B \Longrightarrow (A \Longrightarrow B)
		\end{align}
		も成り立つから,場合分けの法則より
		\begin{align}
			(\rightharpoondown A \vee B) \Longrightarrow (A \Longrightarrow B)
		\end{align}
		が成り立つ.以上で$(A \Longrightarrow B) \Longleftrightarrow (\rightharpoondown A \vee B)$が得られた.
		\QED
	\end{prf}
	
	\monologue{
		$A,B$を$\mathcal{L}'$の閉式とするとき,$A$が偽であれば$\rightharpoondown A$が成立する
		(推論規則\ref{logicalaxm:rules_of_contradiction})ので
		$\rightharpoondown A \vee B$が成立します(推論規則\ref{logicalaxm:fundamental_rules_of_inference}).
		すなわちこのとき$A \Longrightarrow B$が成り立つのですが,式の解釈としては
		``偽な式からはあらゆる式が導かれる''となりますね.この現象を
		{\bf 空虚な真}\index{くうきょなしん@空虚な真}{\bf (vacuous truth)}と呼びます.
	}
	
	\begin{screen}
		\begin{logicalthm}[二重否定の法則の逆が成り立つ]
			$A$を$\mathcal{L}'$の閉式とするとき,次が成り立つ:
			\begin{align}
				A \Longrightarrow\ \rightharpoondown \rightharpoondown A.
			\end{align}
		\end{logicalthm}
	\end{screen}
	
	\begin{prf}
		排中律より
		\begin{align}
			\rightharpoondown A \vee \rightharpoondown \rightharpoondown A
		\end{align}
		が成立し,また推論法則\ref{logicalthm:rule_of_inference_3}より
		\begin{align}
			(\rightharpoondown A \vee \rightharpoondown \rightharpoondown A)
			\Longrightarrow (A \Longrightarrow\ \rightharpoondown \rightharpoondown A)
		\end{align}
		も成り立つので,三段論法より
		\begin{align}
			A \Longrightarrow\ \rightharpoondown \rightharpoondown A
		\end{align}
		が成立する.
		\QED
	\end{prf}
	
	\begin{screen}
		\begin{logicalthm}[対偶命題は同値]\label{thm:contraposition_is_true}
			$A,B$を$\mathcal{L}'$の閉式とするとき,次が成り立つ:
			\begin{align}
				(A \Longrightarrow B) \Longleftrightarrow (\rightharpoondown B \Longrightarrow\ \rightharpoondown A).
			\end{align}
		\end{logicalthm}
	\end{screen}
	
	\begin{prf}
		推論法則\ref{logicalthm:rule_of_inference_3},論理和の可換律,二重否定の法則(とその逆)を順に用いれば
		\begin{align}
			(A \Longrightarrow B) &\Longleftrightarrow (\rightharpoondown A \vee B) \\
			&\Longleftrightarrow (B \vee \rightharpoondown A) \\
			&\Longleftrightarrow (\rightharpoondown \rightharpoondown B \vee \rightharpoondown A) \\
			&\Longleftrightarrow (\rightharpoondown B \Longrightarrow\ \rightharpoondown A)
		\end{align}
		が成り立つ.
		\QED
	\end{prf}
	
	\monologue{
		対偶命題を述べるときには``対偶を取る''と表現することが多いです.
	}
	
	\begin{screen}
		\begin{logicalthm}[De Morganの法則]
			$A,B$を$\mathcal{L}'$の閉式とするとき,次が成り立つ:
			\begin{itemize}
				\item $\rightharpoondown (A \vee B) \Longleftrightarrow\ \rightharpoondown A \wedge \rightharpoondown B$.
			
				\item $\rightharpoondown (A \wedge B) \Longleftrightarrow\ \rightharpoondown A \vee \rightharpoondown B$.
			\end{itemize}
		\end{logicalthm}
	\end{screen}
	
	\begin{prf}
		$A \Longrightarrow (A \vee B)$は定理であるから,その対偶命題
		\begin{align}
			\rightharpoondown (A \vee B) \Longrightarrow\ \rightharpoondown A
		\end{align}
		も定理となる.同様に$\rightharpoondown (A \vee B) \Longrightarrow\ \rightharpoondown B$は定理となるので,
		$\rightharpoondown (A \vee B)$が成り立っていると仮定すれば$\rightharpoondown A \wedge \rightharpoondown B$が成り立つ.
		ゆえに
		\begin{align}
			\rightharpoondown (A \vee B) \Longrightarrow\ \rightharpoondown A \wedge \rightharpoondown B
		\end{align}
		が得られる.また$A$が成り立っていると仮定すれば,この下で$\rightharpoondown A \wedge \rightharpoondown B$が成り立っているなら
		$A$と$\rightharpoondown A$が同時に成り立つことになるので$\bot$が成立する.つまり
		$A$が成り立っているとき
		\begin{align}
			\rightharpoondown A \wedge \rightharpoondown B \Longrightarrow \bot
		\end{align}
		が成り立つが,このとき$\rightharpoondown(\rightharpoondown A \wedge \rightharpoondown B)$が成り立つので
		\begin{align}
			A \Longrightarrow\ \rightharpoondown(\rightharpoondown A \wedge \rightharpoondown B)
		\end{align}
		が得られる.同様にして
		\begin{align}
			B \Longrightarrow\ \rightharpoondown(\rightharpoondown A \wedge \rightharpoondown B)
		\end{align}
		も得られるから,場合分け法則より
		\begin{align}
			(A \vee B) \Longrightarrow\ \rightharpoondown(\rightharpoondown A \wedge \rightharpoondown B)
		\end{align}
		が成立する.この対偶を取れば
		\begin{align}
			\rightharpoondown A \wedge \rightharpoondown B
			\Longrightarrow\ \rightharpoondown (A \vee B)
		\end{align}
		が出る.以上で一つ目の式が示された.一つ目の式で$A$を$\rightharpoondown A$に,
		$B$を$\rightharpoondown B$に置き換えると
		\begin{align}
			\rightharpoondown \rightharpoondown A \wedge \rightharpoondown \rightharpoondown B
			\Longleftrightarrow\ \rightharpoondown (\rightharpoondown A \vee \rightharpoondown B)
		\end{align}
		が得られるが,このとき二重否定の法則より
		\begin{align}
			A \wedge B
			\Longleftrightarrow\ \rightharpoondown (\rightharpoondown A \vee \rightharpoondown B)
		\end{align}
		が成立し,対偶命題の同値性から
		\begin{align}
			\rightharpoondown (A \wedge B)
			\Longleftrightarrow\ (\rightharpoondown A \vee \rightharpoondown B)
		\end{align}
		は定理となる.
		\QED
	\end{prf}
	
	\monologue{
		以上で``集合であり真類でもある類は存在しない''という言明を証明する準備が整いました.
	}
	
	\begin{screen}
		\begin{thm}[集合であり真類でもある類は存在しない]
			$a$を類とするとき次が成り立つ:
			\begin{align}
				\rightharpoondown (\ \set{a} \wedge \rightharpoondown \set{a}\ ).
			\end{align}
		\end{thm}
	\end{screen}
	
	\begin{prf}
		$a$を類とするとき,排中律より$\set{a} \vee \rightharpoondown \set{a}$
		が成り立ち,論理和の可換律より
		\begin{align}
			\rightharpoondown \set{a} \vee \set{a}
		\end{align}
		も成立する.そしてDe Morganの法則より
		\begin{align}
			\rightharpoondown (\ \rightharpoondown \rightharpoondown \set{a} \wedge \rightharpoondown \set{a}\ )
		\end{align}
		が成り立つが,二重否定の法則より$\rightharpoondown \rightharpoondown \set{a}$と
		$\set{a}$は同値となるので
		\begin{align}
			\rightharpoondown (\ \set{a} \wedge \rightharpoondown \set{a}\ )
		\end{align}
		が成り立つ.
		\QED
	\end{prf}
	
	\monologue{
		次は量化記号が推論操作の上でどのような働きを持つのかを規定しましょう.
	}
	
	\begin{screen}
		\begin{logicalaxm}[量化記号に関する規則]\label{logicalaxm:rules_of_quantifiers}
			$A$を$\mathcal{L}'$の式とし,$x$を$A$に現れる文字とするとき,$x$のみが$A$で量化されていないならば以下を認める:
			\begin{description}
				\item[$\varepsilon$記号の導入] $\varepsilon x A(x)$は$\mathcal{L}$の或る対象に代用される.
				\item[存在記号の規則] $A (\varepsilon x A(x)) \Longleftrightarrow \exists x A(x)$が成り立つ.
				\item[全称記号の規則] $A (\varepsilon x \rightharpoondown A(x)) \Longleftrightarrow \forall x A(x)$が成り立つ.
				\item[存在記号の基本性質] $\tau$を$\mathcal{L}$の対象とするとき
					$A(\tau) \Longrightarrow \exists x A(x)$が成り立つ.
			\end{description}
		\end{logicalaxm}
	\end{screen}
	
	$\varepsilon$記号はHilbertのイプシロン関数と呼ばれるもので,
	量化記号の働きを形式的に表現するには簡便かつ有能である.
	また$\varepsilon$記号が指定する対象を$\mathcal{L}$のものと約束することで,
	$\exists$と$\forall$の作用範囲を$\mathcal{L}$の対象全体に制限している.
	
	\begin{screen}
		\begin{logicalthm}[全称記号と任意性]\label{logicalthm:fundamental_law_of_universal_quantifier}
			$A$を$\mathcal{L}'$の式とし,$x$を$A$に現れる文字とし,$x$のみが$A$で量化されていないとする.このとき
			$\forall x A(x)$が成り立つならば$\mathcal{L}$のいかなる対象$\tau$に対しても$ A(\tau)$が成り立つ.
			逆に,$\mathcal{L}$のいかなる対象$\tau$に対しても$A(\tau)$が成り立てば$\forall x A(x)$が成り立つ.
		\end{logicalthm}
	\end{screen}
	
	\begin{prf}
		$\tau$を$\mathcal{L}$の任意の対象とすれば,存在記号に関する推論規則より
		\begin{align}
			\rightharpoondown A(\tau) \Longrightarrow\ \exists x \rightharpoondown A(x)
		\end{align}
		と
		\begin{align}
			\exists x \rightharpoondown A(x) \Longrightarrow\ \rightharpoondown A
			\left( \varepsilon x \rightharpoondown A(x) \right)
		\end{align}
		が成り立つから,推論法則\ref{logicalthm:transitive_law_of_implication}より
		\begin{align}
			\rightharpoondown A(\tau) \Longrightarrow\ \rightharpoondown A
			\left( \varepsilon x \rightharpoondown A(x) \right)
		\end{align}
		が成り立ち,対偶を取って
		\begin{align}
			A \left( \varepsilon x \rightharpoondown A(x) \right)
			\Longrightarrow A(\tau)
		\end{align}
		が成り立つ.全称記号に関する推論規則より
		\begin{align}
			\forall x A(x) \Longrightarrow A \left( \varepsilon x \rightharpoondown A(x) \right)
		\end{align}
		が満たされているので
		\begin{align}
			\forall x A(x) \Longrightarrow A(\tau)
		\end{align}
		が従う.逆にいかなる対象$\tau$に対しても$A(\tau)$が成り立つとき,特に
		\begin{align}
			A \left( \varepsilon x \rightharpoondown A(x) \right)
		\end{align}
		が成り立つので$\forall x A(x)$も成り立つ.
		\QED
	\end{prf}
	
	\monologue{
		推論法則\ref{logicalthm:fundamental_law_of_universal_quantifier}を根拠にして,
		当面は$\forall x A(x)$という式を``$\mathcal{L}$の任意の対象$x$に対して
		$A(x)$が成立する''と翻訳することにします.また後述する相等性の公理によれば,
		これは``任意の集合$x$に対して$A(x)$が成立する''と翻訳しても同義です.
	}
	
	\begin{screen}
		\begin{logicalthm}[量化記号の性質(イ)]\label{logicalthm:properties_of_quantifiers}
			$A,B$を$\mathcal{L}'$の式とし,$x$を$A,B$に現れる文字とし,$x$のみが$A,B$で量化されていないとする.
			$\mathcal{L}$の任意の対象$\tau$に対して
			\begin{align}
				A(\tau) \Longleftrightarrow B(\tau)
			\end{align}
			が成り立っているとき,
			\begin{align}
				\exists x A(x) \Longleftrightarrow \exists x B(x)
			\end{align}
			および
			\begin{align}
				\forall x A(x) \Longleftrightarrow \forall x B(x)
			\end{align}
			が成り立つ.
		\end{logicalthm}
	\end{screen}
	
	\begin{prf}
		いま,$\mathcal{L}$の任意の対象$\tau$に対して
		\begin{align}
			A(\tau) \Longleftrightarrow B(\tau)
			\label{logicalthm:properties_of_quantifiers_1}
		\end{align}
		が成り立っているとする.
		ここで
		\begin{align}
			\exists x A(x)
		\end{align}
		が成り立っていると仮定すると,
		\begin{align}
			\tau \defeq \varepsilon x A(x)
		\end{align}
		とおけば存在記号に関する規則より
		\begin{align}
			A(\tau)
		\end{align}
		が成立し,(\refeq{logicalthm:properties_of_quantifiers_1})と併せて
		\begin{align}
			B(\tau)
		\end{align}
		が成立する.再び存在記号に関する規則より
		\begin{align}
			\exists x B(x)
		\end{align}
		が成り立つので,演繹法則から
		\begin{align}
			\exists x A(x) \Longrightarrow \exists x B(x)
		\end{align}
		が得られる.$A$と$B$の立場を入れ替えれば
		\begin{align}
			\exists x B(x) \Longrightarrow \exists x A(x)
		\end{align}
		も得られる.今度は
		\begin{align}
			\forall x A(x)
		\end{align}
		が成り立っていると仮定すると,
		推論法則\ref{logicalthm:fundamental_law_of_universal_quantifier}より
		$\mathcal{L}$の任意の対象$\tau$に対して
		\begin{align}
			A(\tau)
		\end{align}
		が成立し,(\refeq{logicalthm:properties_of_quantifiers_1})と併せて
		\begin{align}
			B(\tau)
		\end{align}
		が成立する.$\tau$の任意性と推論法則\ref{logicalthm:fundamental_law_of_universal_quantifier}より
		\begin{align}
			\forall x B(x)
		\end{align}
		が成り立つので,演繹法則から
		\begin{align}
			\forall x A(x) \Longrightarrow \forall x B(x)
		\end{align}
		が得られる.$A$と$B$の立場を入れ替えれば
		\begin{align}
			\forall x B(x) \Longrightarrow \forall x A(x)
		\end{align}
		も得られる.
		\QED
	\end{prf}
	
	\begin{screen}
		\begin{logicalthm}[量化記号に対する De Morgan の法則]\label{logicalthm:De_Morgan_law_for_quantifiers}
			$A$を$\mathcal{L}'$の式とし,$x$を$A$に現れる文字とし,$x$のみが$A$で量化されていないとする.このとき
			\begin{align}
				\exists x \rightharpoondown A(x) \Longleftrightarrow\ \rightharpoondown \forall x A(x)
			\end{align}
			および
			\begin{align}
				\forall x \rightharpoondown A(x) \Longleftrightarrow\ \rightharpoondown \exists x A(x)
			\end{align}
			が成り立つ.
		\end{logicalthm}
	\end{screen}
	
	\begin{sketch}
		推論規則\ref{logicalaxm:rules_of_quantifiers}より
		\begin{align}
			\exists x \rightharpoondown A(x) \Longleftrightarrow\ 
			\rightharpoondown A(\varepsilon x \rightharpoondown A(x))
		\end{align}
		は定理である.他方で推論規則\ref{logicalaxm:rules_of_quantifiers}より
		\begin{align}
			A(\varepsilon x \rightharpoondown A(x)) \Longleftrightarrow \forall x A(x) 
		\end{align}
		もまた定理であり,この対偶を取れば
		\begin{align}
			\rightharpoondown A(\varepsilon x \rightharpoondown A(x)) \Longleftrightarrow\ 
			\rightharpoondown \forall x A(x)
		\end{align}
		が成り立つ.ゆえに
		\begin{align}
			\exists x \rightharpoondown A(x) \Longleftrightarrow\ \rightharpoondown \forall x A(x)
		\end{align}
		が従う.$A$を$\rightharpoondown A$に置き換えれば
		\begin{align}
			\forall x \rightharpoondown A(x) \Longleftrightarrow\ 
			\rightharpoondown \exists x \rightharpoondown \rightharpoondown A(x)
		\end{align}
		が成り立ち,また$\mathcal{L}$の任意の対象$\tau$に対して
		\begin{align}
			A(\tau) \Longleftrightarrow\ \rightharpoondown \rightharpoondown A(\tau)
		\end{align}
		が成り立つので,推論法則\ref{logicalthm:properties_of_quantifiers}より
		\begin{align}
			\exists x \rightharpoondown \rightharpoondown A(x)
			\Longleftrightarrow \exists x A(x)
		\end{align}
		も成り立つ.ゆえに
		\begin{align}
			\forall x \rightharpoondown A(x) \Longleftrightarrow\ 
			\rightharpoondown \exists x A(x)
		\end{align}
		が従う.
		\QED
	\end{sketch}
\begin{thebibliography}{数字}
	\bibitem{key1} Moser, G. and Zach, R., ``The Epsilon Calculus and Herbrand Complexity'',
		Studia Logica 82, 133-155 (2006)
	
	\bibitem{key2} 高橋優太, ``1階述語論理に対する$\varepsilon$計算'', \\
		http://www2.kobe-u.ac.jp/~mkikuchi/ss2018files/takahashi1.pdf 
		
	\bibitem{key3} キューネン数学基礎論講義
	
	\bibitem{key5} ブルバキ, 数学原論 集合論 1, 
	
	\bibitem{key4} 竹内外史, 現代集合論入門, 増強版第5刷, 日本評論社, 2016, pp. 138-183, ISBN 978-4-535-60116-1
	
	\bibitem{key6} 島内剛一, 数学の基礎, 第1版第10刷, 日本評論社, 2016, ISBN 978-4-535-60106-2
	
	\bibitem{key7} 戸次大介, 数理論理学, 第2刷, 東京大学出版会, 2016, pp. 148-166, ISBN 978-4-13-062915-7
	
	\bibitem{key8} K. G$\ddot{\mbox{o}}$del, $The\ Consistency\ of\ the\ Continuum\ Hypothesis$, 8th printing, Princeton University Press 1970, p. 3, ISBN 0-691-07927-7.
	
	\bibitem{key9} 菊地誠, 不完全性定理, 初版3刷, 共立出版株式会社, 2017, pp. 86-91, ISBN 978-4-320-11096-0
	
	\bibitem{key10} 前原昭二, 記号論理入門, 新装版第8刷, 日本評論社, 2018, pp. 106-115, ISBN 4-535-60144-5
	
	\bibitem{key11} Kenji Miyamoto and Georg Moser, The Epsilon Calculus with Equality and Herbrand Complexity
\end{thebibliography}
\newpage
\printindex
%
%
\end{document}
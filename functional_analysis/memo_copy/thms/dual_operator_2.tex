\section{Hilbert空間の共役作用素}
	以降は係数体を$\C$とし,Hilbert空間$X$における内積を$\inprod<\cdot,\cdot>_X$,ノルムを$\Norm{\cdot}{X}$と表しノルム位相を導入する.
	
	\begin{screen}
		\begin{dfn}[共役作用素]
			$X,Y$をHilbert空間,$T:X \oparrow Y$を線型作用素とし,
			$\closure{\Dom{T} } = X$を満たすとする.
			\begin{align}
				\mathcal{D} \coloneqq \Set{y \in Y}{\mbox{或る$u \in X$が存在して$\inprod<x,u>_X = \inprod<T x,y>_Y \quad (\forall x \in \Dom{T} ) $を満たす.}}
				\label{eq:dfn_dual_operator_hilbert}
			\end{align}
			と定めれば,$y \in \mathcal{D}$に対して
			(\refeq{eq:dfn_dual_operator_hilbert})の関係を満たす
			$u$の存在は一意的であり\footnotemark
			この対応を
			\begin{align}
				T^*: \mathcal{D} \ni y \longmapsto u
			\end{align}
			で表し$\Dom{T^*} \coloneqq \mathcal{D}$とおく.この$T^*:Y \oparrow X$を$T$の共役作用素という.
		\end{dfn}
	\end{screen}
	
	\footnotetext{
		$y$に対し$u$とは別に(\refeq{eq:dfn_dual_operator_hilbert})を満たす$u' \in X$が存在すれば
		\begin{align}
			\inprod<x,u>_X = \inprod<x,u'>_X \quad (\forall x \in \Dom{T} )
		\end{align}
		が成り立つ.$\Dom{T} $は$X$で稠密であるから
		内積の連続性より$u = u'$が従う.
	}
	
	\begin{screen}
		\begin{thm}
			$H$を複素Hilbert空間,$T:H \oparrow H$を線型作用素,
			$I$を$H$上の恒等写像とする.$\closure{\Dom{T} } = H$が満たされていれば
			$(\lambda I - T)^* = \conj{\lambda} I - T^*\ (\forall \lambda \in \C)$が成り立つ.
		\end{thm}
	\end{screen}
	
	\begin{prf}
		任意に$\lambda \in \C$を取り固定する.
		$\Dom{\lambda I - T} = \Dom{T} $が成り立つから$(\lambda I - T)^*$が定義され
		\begin{align}
			\inprod<(\lambda I - T)u, v>_H = \inprod<u, (\lambda I - T)^*v>_H
			\quad \left( \forall u \in \Dom{T} , \ v \in \Dom{(\lambda I - T)^*} \right)
		\end{align}
		が満たされる.一方で任意の$u \in \Dom{T} , \ v \in \Dom{(\lambda I - T)^*} $に対し
		\begin{align}
			\inprod<(\lambda I - T)u, v>_H
			= \inprod<u, \conj{\lambda} v>_H - \inprod<T u, v>_H
		\end{align}
		が成り立つから,
		\begin{align}
			\inprod<T u, v>_H = \inprod<u, \conj{\lambda} v>_H - \inprod<u, (\lambda I - T)^*v>_H
			\quad \left( \forall u \in \Dom{T} \right)
		\end{align}
		となり$v \in \Dom{T^*} = \Dom{\conj{\lambda} I - T^*} $かつ
		\begin{align}
			(\lambda I - T)^*v = \left( \conj{\lambda} I - T^* \right)v \quad \left( v \in \Dom{(\lambda I - T)^*} \right)
		\end{align}
		が従う.後は$\Dom{\conj{\lambda} I - T^*} \subset \Dom{(\lambda I - T)^*} $を示せば主張が得られる.
		実際任意に$v \in \Dom{\conj{\lambda} I - T^*} $を取れば,
		\begin{align}
			\inprod<T u, v>_H = \inprod<u, T^*v>_H
			\quad \left( \forall u \in \Dom{T} \right)
		\end{align}
		が成り立つから
		\begin{align}
			\inprod<(\lambda I - T) u, v>_H
			= \inprod<\lambda u, v>_H - \inprod<T u, v>_H
			= \inprod<u, \left( \conj{\lambda} I - T^* \right) v>_H
			\quad \left( \forall u \in \Dom{T} \right)
		\end{align}
		が従い$v \in \Dom{(\lambda I - T)^*} $を得る.
		\QED
	\end{prf}
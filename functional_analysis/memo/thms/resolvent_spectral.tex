本章を通じて$X$を複素Banach空間とし,ノルムを$\Norm{\cdot}{}$で表す.

\section{レゾルベントとスペクトル}
	\begin{screen}
		\begin{e.g.}[微分作用素のスペクトル]
			$I \coloneqq [0,a]\ (a > 0),\ X \coloneqq \c{I}$とし,
			$X$上の作用素$T_0,\cdots,T_4$を次で定める:
			\begin{align}
				\Dom{T_0} &\coloneqq \Set{u \in \cn{I}{1}}{u(0) = u(a) = 0}, \\
				\Dom{T_1} &\coloneqq \Set{u \in \cn{I}{1}}{u(0) = 0}, \\
				\Dom{T_2} &\coloneqq \Set{u \in \cn{I}{1}}{u(0) = u(a)}, \\
				\Dom{T_3} &\coloneqq \cn{I}{1}, \\
				\Dom{T_4} &\coloneqq \Set{u \in \cn{I}{1}}{u(0) + u(a) = 0}
			\end{align}
			$u \in \Dom{T_j} $に対し$T_j u = u'\ (j=0,\cdots,4)$.
			この下で$\pSpctr{T_j} $及び$\Spctr{T_j} \ (j=0,\cdots,4)$を求める.
			\label{ex:spectral_of_differential_operator}
		\end{e.g.}
	\end{screen}
	
	\begin{screen}
		\begin{lem}[微分方程式の解]
			$I \coloneqq [0,a]\ (a > 0),\ X \coloneqq \c{I}$とおく.
			任意の$f \in \cn{I}{1},\ \lambda,u_0 \in \C$に対し
			\begin{align}
				\begin{cases}
					u'(x) - \lambda u(x) = f(x) \\
					u(0) = u_0
				\end{cases}
				\quad (x \in I)
				\label{eq:lem_solution_of_differential_equation_2}
			\end{align}
			は$\cn{I}{1}$において唯一つの解
			\begin{align}
				u(x) = \exp{\lambda x}u_0 + \int_0^x \exp{\lambda (x-s)} f(s)\ ds
				\quad (x \in I)
				\label{eq:lem_solution_of_differential_equation_1}
			\end{align}
			を持つ.
			\label{lem:solution_of_differential_equation}
		\end{lem}
	\end{screen}
	
	\begin{prf}[補題\ref{lem:solution_of_differential_equation}]
		(\refeq{eq:lem_solution_of_differential_equation_1})で与えられる$u \in \cn{I}{1}$は
		\begin{align}
			u'(x) = \lambda u_0 \exp{\lambda x} + \lambda \exp{\lambda x} \int_0^x \exp{-\lambda s} f(s)\ ds + f(x) 
			= \lambda u(x) + f(x)
			\quad (\forall x \in I)
		\end{align}
		を満たすから微分方程式(\refeq{eq:lem_solution_of_differential_equation_2})の解であるから,あとは
		解の一意性を示せばよい.$v \in \cn{I}{1}$が(\refeq{eq:lem_solution_of_differential_equation_2})の解であるとき,
		\begin{align}
			u'(x) - \lambda u(x) = v'(x) - \lambda v(x) \quad (\forall x \in I),
			\quad u(0) - v(0) = 0
		\end{align}
		が成り立つから,$w \coloneqq u - v \in \cn{I}{1}$は次の確率微分方程式を満たす:
		\begin{align}
			\begin{cases}
				w'(x) = \lambda w(x) \\
				w(0) = 0
			\end{cases}
			\quad (\forall x \in I).
		\end{align}
		これを満たす$w$は$w = 0$のみであるから$u = v$が得られる.
		\QED
	\end{prf}
	
	\begin{prf}[例\ref{ex:spectral_of_differential_operator}]
		先ず各$T_j$が閉作用素であることを示す.
		$u_n \in \Dom{T_j} \ (n=1,2,\cdots)$に対し或る$u,v \in \c{I}$が存在して,$\Norm{u_n - u}{} \longrightarrow 0$
		かつ$\Norm{T_j u_n - v}{} \longrightarrow 0$が成り立つとき,任意の$x \in I$に対して
		\begin{align}
			\left| u(x) - \int_0^x v(t)\ dt \right|
			&\leq \left| u(x) - u_n(x) \right| + \left| \int_0^x T_j u_n(t)\ dt - \int_0^x v(t)\ dt \right| \\
			&\leq \Norm{u - u_n}{} + a \Norm{T_j u_n - v}{}
			\longrightarrow 0 \quad (n \longrightarrow \infty)
		\end{align}
		が成り立つから
		\begin{align}
			u(x) = \int_0^x v(t)\ dt \quad (\forall x \in I)
		\end{align}
		となり,$u \in \cn{I}{1}$かつ$u' = v$が得られる.後は$u \in \Dom{T_j} $を満たせばよい.
		$j = 0,1$の場合は
		\begin{align}
			\left| u(x) \right| = \left| u(x) - u_n(x) \right| \leq \Norm{u - u_n}{} \longrightarrow 0 \quad (n \longrightarrow \infty,\ x=0,a)
		\end{align}
		により,$j=2$の場合は
		\begin{align}
			\left| u(0) - u(a) \right| \leq \left| u(0) - u_n(0) \right| 
				+ \left| u_n(a) - u(a) \right|
			\leq 2 \Norm{u - u_n}{}
			\longrightarrow 0 \quad (n \longrightarrow \infty)
		\end{align}
		により,$j=4$の場合は
		\begin{align}
			\left| u(0) + u(a) \right| \leq \left| u(0) - u_n(0) \right| 
				+ \left| u_n(a) - u(a) \right|
			\leq 2 \Norm{u - u_n}{}
			\longrightarrow 0 \quad (n \longrightarrow \infty)
		\end{align}
		により,いずれの場合も$u \in \Dom{T_j} $が成り立つ.
		次に$\pSpctr{T_j} $と$\Spctr{T_j} $を考察する.
		\begin{description}
			\item[$j=0$の場合]
			\item[$j=1$の場合]
			\item[$j=4$の場合]
				%レポート問題4
	\begin{prf}\mbox{}
		\begin{description}
			\item[点スペクトルについて]
				$(\lambda I - T) u = 0$を満たす$\lambda$に対し,微分方程式を解けば
				\begin{align}
					u(x) = C\exp{\lambda x}
					\quad (x \in I,\ \exists C \ni \C)
				\end{align}
				と表せる.$u(0) + u(a) = 0$が満たされているから,
				\begin{align}
					C + C\exp{\lambda a} = 0
				\end{align}
				が成り立つ.従って$C = 0$或は,複素対数を用いて
				$\lambda = \frac{1}{a} \log{(-1)}$となる.
				$C = 0$の場合は$u = 0$となり固有ベクトルになりえないから
				\begin{align}
					\pSpctr{T} = \Set{\sqrt{-1} \frac{(2 n + 1)\pi}{a}}{n \in \Z}
				\end{align}
				
			\item[スペクトルについて]
		\end{description}
	\end{prf}
		\end{description}
	\end{prf}
	
	\begin{screen}
		\begin{lem}[レゾルベントは閉作用素]
			$T$を$X$上の線型作用素,$\lambda$を複素数とするとき,次は同値である:
			\begin{description}
				\item[(1)] $T$は閉作用素である.
				\item[(2)] $\lambda I - T$は閉作用素である.
				\item[(3)] $(\lambda I - T)^{-1}$が存在すれば,これは閉作用素である.
			\end{description}
			\label{lem:resolvent_is_closed}
		\end{lem}
	\end{screen}
	
	\begin{prf}\mbox{}
		\begin{description}
			\item[(1)$\Rightarrow$(2)]
				$[u_n,(\lambda I - T) u_n] \in \Graph{(\lambda I - T)} \ (n=1,2,\cdots)$が
				$u_n \longrightarrow u \in X,\ (\lambda I - T) u_n \longrightarrow v \in X$を満たすとき,
				\begin{align}
					\Norm{(\lambda u - v) - T u_n}{}
					\leq \Norm{(\lambda I - T) u_n - v}{} + \Norm{\lambda u_n - \lambda u}{}
					\longrightarrow 0 \quad (n \longrightarrow \infty)
				\end{align}
				より$T u_n \longrightarrow \lambda u - v$が成り立つ.$T$が閉作用素なら
				$T u = \lambda u - v$が従い$v = (\lambda I - T) u$を得る.
				
			\item[(2)$\Rightarrow$(1)]
				$[u_n,T u_n] \in \Graph{T} \ (n=1,2,\cdots)$が
				$u_n \longrightarrow u \in X,\ T u_n \longrightarrow v \in X$を満たすとき,
				\begin{align}
					\Norm{(\lambda I - T)u_n - (\lambda u - v)}{}
					\leq \Norm{\lambda u_n - \lambda u}{} + \Norm{v - T u_n}{}
					\longrightarrow 0 \quad (n \longrightarrow \infty)
				\end{align}
				より$(\lambda I - T)u_n \longrightarrow \lambda u - v$が成り立つ.
				$(\lambda I - T)$が閉作用素なら$(\lambda I - T)u = \lambda u - v$が従い
				$T u = v$を得る.
				
			\item[(2)$\Leftrightarrow$(3)]
				定理\ref{thm:closed_linear_op_inverse}による.
				\QED
		\end{description}
	\end{prf}
	
	\begin{screen}
		\begin{thm}[レゾルベントの共役]
			$X$を複素banach空間,$T:X \oparrow X$を閉線型作用素,
			$I$を$X$上の恒等写像とする.$\closure{\Dom{T} } = X$が満たされていれば,$\Res{T^*} = \Res{T} $かつ
			$\left( \lambda I - T^* \right)^{-1} = \left((\lambda I - T)^{-1}\right)^*\ (\forall \lambda \in \Res{T^*})$
			が成り立つ.
		\end{thm}
	\end{screen}
	
	\begin{prf}\mbox{}
		\begin{description}
			\item[(1)]
				
			\item[(2)]
		\end{description}
	\end{prf}
	
	\begin{screen}
		\begin{thm}[レゾルベントの共役]
			$H$を複素Hilbert空間,$T:H \oparrow H$を閉線型作用素,
			$I$を$H$上の恒等写像とする.$\closure{\Dom{T} } = H$が満たされていれば,$\Res{T^*} = \Res{T} $かつ
			$\left( \lambda I - T^* \right)^{-1} = \left((\lambda I - T)^{-1}\right)^*\ (\forall \lambda \in \Res{T^*})$
			が成り立つ.
		\end{thm}
	\end{screen}
	
	\begin{screen}
		\begin{e.g.}
			$(X,\mathcal{M},\mu)$を$\sigma$-有限な測度空間,$H = \mathrm{L}^2(X,\mathcal{M},\mu) = \mathrm{L}^2(\mu)$とする.
			$\mathcal{M}$-可測関数$a:X \rightarrow \C$に対して,$H$から$H$へのかけ算作用素$M_a$を次で定める:
			\begin{align}
				\Dom{M_a} = \Set{u \in H}{au \in H},
				\quad (M_a u)(x) = a(x) u(x) \quad (x \in X).
			\end{align}
			\begin{description}
				\item[(1)] $M_a$は線型作用素で,$\Dom{M_a} $は$H$で稠密なことを示せ.
				\item[(2)] $M_a^* = M_{\conj{a}}$が成り立つことを示せ.
				\item[(3)] $\Spctr{M_a} = \Set{\lambda \in \C}{\mbox{$\forall \epsilon > 0$に対し$\mu\left( a^{-1}(U_\epsilon(\lambda)) \right) > 0$}}$を示せ.
					(ただし$U_\epsilon(\lambda)$は$\lambda$の$\epsilon$-近傍.)
				\item[(4)] $\pSpctr{M_a} = \Set{\lambda \in \C}{\mu\left( a^{-1}(\{\lambda\}) \right) > 0}$を示せ.
			\end{description}
		\end{e.g.}
	\end{screen}
	
	%レポート問題8
	\begin{prf}
		$\sigma$-有限の仮定により,或る集合の系$(X_n)_{n=1}^{\infty} \subset \mathcal{M}$
		が存在して$X_1 \subset X_2 \subset \cdots,\ \mu(X_n) < \infty\ (\forall n \in \N),\ \cup_{n \in \N} X_n = X$を満たす.また$H$におけるノルムと内積をそれぞれ$\Norm{\cdot}{},\inprod<\cdot,\cdot>$と表し,
		$H$上の恒等写像を$I$とする.
		
		\begin{description}
			\item[(1)] 
				\begin{description}
					\item[$M_a$が線型作用素であること]
						先ず$\Dom{M_a} $が$H$の線型部分空間であることを示す.
						任意に$u,v \in \Dom{M_a} ,\ \alpha,\beta \in \C$を取れば,
						$H$が線形空間であることにより$\alpha u + \beta v \in H$が満たされ,
						且つ$a u, a v \in H$により
						\begin{align}
							a( \alpha u + \beta v)
							= \alpha a u + \beta a v 
							\in H
						\end{align}
						も成り立つから$\alpha u + \beta v \in \Dom{M_a} $が従う.
						また任意の$u,v \in \Dom{M_a} ,\ \alpha,\beta \in \C$に対して
						\begin{align}
							&M_a(\alpha u + \beta v)(x)
							= a(x)( \alpha u(x) + \beta v(x)) \\
							&\qquad = \alpha a(x) u(x) + \beta a(x) v(x)
							= \alpha (M_a u) (x) + \beta (M_a v) (x)
							\quad (\mbox{$\mu$-a.e.} x \in X)
						\end{align}
						が満たされるから$M_a$は線型作用素である.
						
					\item[$\Dom{M_a} $が$H$で稠密なこと]
						任意に$v \in H$を取り$v_n \coloneqq v\defunc_{\{|a| \leq n\}}\ (n=1,2,3,\cdots)$として関数列$(v_n)_{n=1}^{\infty}$を作る.
						全ての$x \in X$で$|v_n(x)| \leq |v(x)|$が満たされているから
						$(v_n)_{n \in \N} \subset H$である.また全ての$n \in \N$について
						\begin{align}
							\int_X |a(x)v_n(x)|^2 \mu(dx) = \int_{\{|a| \leq n\}} |a(x)v(x)|^2 \mu(dx) \leq n^2  \int_X |v(x)|^2 \mu(dx) < \infty
						\end{align}
						が成り立つから$(v_n)_{n \in \N} \subset \Dom{M_a} $が従う.
						\begin{align}
							\Norm{v - v_n}{}^2 = \int_X |v(x) - v_n(x)|^2\, \mu(dx) = \int_X \defunc_{\{|a| > n\}}(x)|v(x)|^2\, \mu(dx)
							\label{eq:func_report_Q9_1}
						\end{align}
						の右辺の被積分関数は各点で$0$に収束し,かつ可積分関数$|v|^2$で抑えられるから,
						Lebesgueの収束定理より
						\begin{align}
							\lim_{n \to \infty} \Norm{v - v_n}{}^2 
							= \lim_{n \to \infty} \int_X \defunc_{\{|a| > n\}}(x)|v(x)|^2\, \mu(dx)
							= \int_X \lim_{n \to \infty} \defunc_{\{|a| > n\}}(x)|v(x)|^2\, \mu(dx)
							= 0
						\end{align}
						が得られる.$v$は任意に選んでいたから$\Dom{M_a} $の稠密性が従う.
				\end{description}
				
			\item[(2)]
				$\Dom{M_a} $が$H$で稠密であるから$M_a$の共役作用素を定義できる.
				任意の$u,v \in \Dom{M_a} = \Dom{M_{\conj{a}}} $に対して
				\begin{align}
					\inprod<M_a u,v> 
					= \int_X a(x) u(x) \conj{v(x)}\ \mu(dx)
					= \int_X u(x) \conj{\conj{a(x)} v(x)}\ \mu(dx)
					= \inprod<u,M_{\conj{a}}v>
				\end{align}
				が成り立つから$v \in \Dom{M_a^*} $且つ$M_a^* v = M_{\conj{a}} v\ \left(\forall v \in \Dom{M_{\conj{a}}} \right)$が従う.
				逆に任意の$u \in \Dom{M_a} , v \in \Dom{M_a^*} $に対し
				\begin{align}
					\inprod<u,M_a^* v> = \inprod<M_a u,v> = \inprod<u,M_{\conj{a}}v>
				\end{align}
				が成り立つから,$\Dom{M_a} $の稠密性により$M_a^* v = M_{\conj{a}}v\ \left(\forall v \in \Dom{M_a^*} \right)$が従う.
				以上より$M_a^* = M_{\conj{a}}$を得る.
				
			\item[(3)]
				$\lambda \in \C$を任意に取り固定し,
				$V_\epsilon \coloneqq a^{-1}(U_\epsilon(\lambda))\ (\forall \epsilon > 0)$とおく.
			 	或る$\epsilon > 0$が存在して$\mu(V_\epsilon) = 0$が成り立つ場合,
			 	\begin{align}
			 		b(x) \coloneqq 
			 		\begin{cases}
			 			1/\left( \lambda - a(x) \right) & (x \in X \backslash V_\epsilon) \\
			 			0 & (x \in V_\epsilon)
			 		\end{cases}
			 	\end{align}
				として$b$を定めれば,任意の$u \in H$に対して
				\begin{align}
					\int_X |b(x)u(x)|^2\ \mu(dx)
					= \int_{X \backslash V_\epsilon} \frac{1}{|\lambda - a(x)|^2} |u(x)|^2\ \mu(dx)
					\leq \frac{1}{\epsilon^2} \int_X |u(x)|^2\ \mu(dx) < \infty 
				\end{align}
				が成り立つから,$M_b$は$\Dom{M_b} = H$を満たす有界線型作用素である.更に$b(x) \left( \lambda - a(x) \right) = 1\ (\forall x \in V_\epsilon)$により
				\begin{align}
					b(x) \left( \lambda - a(x) \right) u(x) &= u(x) \quad (\mbox{$\mu$-a.e.}x \in X,\ \forall u \in \Dom{M_a} ), \\
					\left( \lambda - a(x) \right) b(x) u(x) &= u(x) \quad (\mbox{$\mu$-a.e.}x \in X,\ \forall u \in H )
				\end{align}
				が成り立つから,$M_b = (\lambda I - M_a)^{-1}$となり
				$\lambda \in \Res{M_a} $が従う.
				以上より
				\begin{align}
					\Spctr{M_a} \subset \Set{\lambda \in \C}{\mbox{$\forall \epsilon > 0$に対し$\mu\left( a^{-1}(U_\epsilon(\lambda)) \right) > 0$}}
					\label{eq:report_8_1}
				\end{align}
				が成立する.
				次に逆の包含関係を示す.$\mu(V_\epsilon) > 0\ (\forall \epsilon > 0)$が満たされている時,
				任意に$\epsilon > 0$を取り固定する.
				\begin{align}
					\mu(V_\epsilon) = \lim_{n \to \infty} \mu(V_\epsilon \cap X_n)
				\end{align}
				が成り立つから,或る$N \in \N$が存在して$\mu(V_\epsilon \cap X_N) > 0$を満たす.
				\begin{align}
					u_\epsilon(x) \coloneqq
					\begin{cases}
						1 & (x \in V_\epsilon \cap X_N) \\
						0 & (x \notin V_\epsilon \cap X_N)
					\end{cases}
				\end{align}
				として$u_\epsilon$を定めれば,$u_\epsilon$は二乗可積分であり
				\begin{align}
					\int_X |a(x)u_\epsilon(x)|^2\ \mu(dx)
					= \int_{V_\epsilon \cap X_N} |a(x)u_\epsilon(x)|^2\ \mu(dx)
					\leq \left( \epsilon + |\lambda| \right)^2 \mu\left( V_\epsilon \cap X_N \right)
					< \infty
				\end{align}
				を満たすから$u_\epsilon \in \Dom{M_a} $が従う.また
				\begin{align}
					&\Norm{(\lambda I- M_a)u_\epsilon}{}^2
					= \int_X |\lambda - a(x)|^2|u_\epsilon(x)|^2\ \mu(dx) \\
					&\qquad = \int_{V_\epsilon \cap X_N} |\lambda - a(x)|^2|u_\epsilon(x)|^2\ \mu(dx)
					\leq \epsilon^2 \int_X |u_\epsilon(x)|^2\ \mu(dx)
					= \epsilon^2 \Norm{u_\epsilon}{}^2
				\end{align}
				を満たす.$\epsilon > 0$は任意に選んでいたから,任意の$\epsilon > 0$に対し或る$u_\epsilon \in \Dom{M_a} $が存在して
				\begin{align}
					\Norm{(\lambda I- M_a)u_\epsilon}{} \leq \epsilon \Norm{u_\epsilon}{}
				\end{align}
				が成り立つ.この場合$(\lambda I- M_a)^{-1}$が存在しても,
				$u_\epsilon = (\lambda I- M_a)^{-1}v_\epsilon$を満たす
				$v_\epsilon \in \Dom{(\lambda I- M_a)^{-1}} $に対して
				\begin{align}
					\frac{1}{\epsilon} \leq \frac{\Norm{(\lambda I- M_a)^{-1}v_\epsilon}{}}{\Norm{v_\epsilon}{}}
				\end{align}
				が従い,$\epsilon$の任意性より$(\lambda I- M_a)^{-1}$の作用素ノルムは非有界である.
				ゆえに$\lambda \in \Spctr{M_a} $が成立し,(\refeq{eq:report_8_1})と併せて
				\begin{align}
					\Spctr{M_a} = \Set{\lambda \in \C}{\mbox{$\forall \epsilon > 0$に対し$\mu\left( a^{-1}(U_\epsilon(\lambda)) \right) > 0$}}
				\end{align}
				が得られる.
				
			\item[(4)] 
				先ず$\pSpctr{M_a} \subset \Set{z \in \C}{\mu\left( a^{-1}(\{z\}) \right) > 0}$が成り立つことを示す.
				任意の$\lambda \in \pSpctr{M_a}$に対しては固有ベクトル$u \in H$が存在し,
				固有ベクトルは$u \neq 0$を満たすから
				\begin{align}
					N \coloneqq \Set{x \in X}{u(x) \neq 0}
				\end{align}
				とおけば$\mu(N) > 0$が成り立つ.一方で点スペクトルの定義より$u$は$(\lambda I - M_a)u = 0$を満たすから,
				\begin{align}
					0 = \Norm{(\lambda I - M_a)u}{}^2 = \int_X |\lambda - a(x)|^2 |u(x)|^2\ \mu(dx)
					= \int_{N} |\lambda - a(x)|^2 |u(x)|^2\ \mu(dx)
				\end{align}
				が成り立ち
				\begin{align}
					\mu\left( \Set{x \in N}{|\lambda - a(x)| > 0} \right) = 0
				\end{align}
				が従う.よって
				\begin{align}
					\mu\left( a^{-1}(\{\lambda\}) \right)
					\geq \mu\left( \Set{x \in N}{|\lambda - a(x)| = 0} \right)
					= \mu(N)
					> 0
				\end{align}
				となり$\lambda \in \Set{z \in \C}{\mu\left( a^{-1}(\{z\}) \right) > 0}$が成り立つ.
				次に$\pSpctr{M_a} \supset \Set{z \in \C}{\mu\left( a^{-1}(\{z\}) \right) > 0}$が成り立つことを示す.
				任意に$\lambda \in \Set{z \in \C}{\mu\left( a^{-1}(\{z\}) \right) > 0}$を取り
				\begin{align}
					\Lambda \coloneqq a^{-1}(\{\lambda\})
				\end{align}
				とおく.
				\begin{align}
					0 < \mu(\Lambda) = \lim_{n \to \infty} \mu(\Lambda \cap X_n)
				\end{align}
				が成り立つから,或る$n \in \N$が存在して$\mu(\Lambda \cap X_n) > 0$を満たす.
				\begin{align}
					u(x) \coloneqq 
					\begin{cases}
						1 & (x \in \Lambda \cap X_n), \\
						0 & (x \notin \Lambda \cap X_n)
					\end{cases}
				\end{align}
				として$u$を定めれば$u$は二乗可積分であり,$\mu(\Lambda \cap X_n) > 0$であるから$u \neq 0$を満たす.また
				\begin{align}
					\Norm{(\lambda I - M_a)u}{}^2
					= \int_X |\lambda - a(x)|^2 |u(x)|^2\ \mu(dx)
					= \int_{\Lambda \cap X_n} |\lambda - a(x)|^2 |u(x)|^2\ \mu(dx)
					= 0
				\end{align}
				により$(\lambda I - M_a)u = 0$が従うから$u$は$\lambda$の固有ベクトルであり,$\lambda \in \pSpctr{M_a}$が成立する.
				\QED
		\end{description}
	\end{prf}
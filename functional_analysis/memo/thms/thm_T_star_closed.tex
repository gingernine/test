\section{共役作用素は閉作用素}
	係数体を$\K$,$X,Y$をノルム空間,$T$を$X \rightarrow Y$の線型作用素とする.
	以下では$X,Y$及びその共役空間$X^*,Y^*$におけるノルムを
	$\Norm{\cdot}{X},\ \Norm{\cdot}{Y},\ \Norm{\cdot}{X^*},\ \Norm{\cdot}{Y^*}$と表記し,
	位相は全てこれらのノルムにより導入されるものと考える.
	$T$の定義域$\mathscr{D}(T)$が$X$で稠密であるときは$T$の共役作用素$T^*$
	が定義される.
	\begin{itembox}[l]{}
		\begin{thm}[共役作用素は閉線型]
			$T^*$は閉線型作用素である.\label{thm:T_star_closed}
		\end{thm}
	\end{itembox}

	この定理を証明するために以下にいくつか準備をする.
	
	$x \in X$と$f \in X^*$に対して$f(x)$を次の形式で表現する:
	\begin{align}
		f(x) = \inprod<x,f>_{X,X^*}.
	\end{align}
	これは双線型形式,つまり
	$\inprod<\alpha x_1 + \beta x_2,f>_{X,X^*} = \alpha \inprod<x_1,f>_{X,X^*} + \beta\inprod<x_2,f>_{X,X^*}$と$\inprod<x,\alpha f_1 + \beta f_2>_{X,X^*} = \alpha \inprod<x,f_1>_{X,X^*} + \beta\inprod<x,f_2>_{X,X^*}$を満たす.これは$f$の線型性と$X^*$における線型演算の定義による.
	双線型形式で表現することで内積空間を扱っているように捉えることができ,
	例えば,$A \subset X$に対し全ての$x \in A$で$\inprod<x,f>_{X,X^*} = 0$となるような
	$f \in X^*$の全体は$A$の直交空間である様に見做すことができる.
	
	\begin{itembox}[l]{}
		\begin{lem}
			$A \subset X$に対し
			\begin{align}
				A^{\perp} \coloneqq \Set{f \in X^*}{\forall x \in A,\ \inprod<x,f>_{X,X^*} = 0}
			\end{align}
			とおけば,$A^{\perp}$は$X^*$において閉部分空間となる.
			\label{lem:T_star_closed_1}
		\end{lem}
	\end{itembox}
	
	\begin{prf}
		$A^{\perp}$が$X^*$において完備部分空間であることを示せばよい.
		\begin{description}
			\item[線型性]
				任意の$f_1,f_2 \in A^{\perp}$と$\alpha \in \K$に対し
				\begin{align}
					(f_1 + f_2)(x) = f_1(x) + f_2(x) = 0, \quad (\alpha f_1)(x) = \alpha f_1(x) = 0
					,\quad (\forall x \in A)
				\end{align}
				が成り立つ.
				
			\item[完備性]
				$f_n \in A^{\perp}$が収束列であるとすれば$X^*$の完備性から$(f_n)_{n=1}^{\infty}$は或る$f \in X^*$
				に(作用素ノルムで)収束する.任意の$x \in A$に対して
				\begin{align}
					|f(x)| = |f(x) - f_n(x)| \leq \Norm{f-f_n}{X^*}\Norm{x}{X} \longrightarrow 0 \quad (n \longrightarrow \infty)
				\end{align}
				が成り立ち$f \in A^{\perp}$となる.
		\end{description}
		
		\QED
	\end{prf}
	
	\begin{description}
		\item[補助定理について補足]
			実際はさらに
			\begin{align}
				(A^{\perp})^{\perp} = \overline{\LH{A}}
			\end{align}
			となることが証明される.ここで$(A^{\perp})^{\perp} = \Set{x \in X}{\forall f \in A^{\perp},\ \inprod<x,f>_{X,X^*} = 0}$
			である.$A \subset (A^{\perp})^{\perp}$かつ$(A^{\perp})^{\perp}$は$X$の閉部分空間であるから
			$\overline{\LH{A}} \subset (A^{\perp})^{\perp}$が先ず判る.
			逆向きの包含関係について,$X = \overline{\LH{A}}$の場合は成り立つが,そうでない場合は次のように考える.
			Hahn-Banachの定理の系によれば任意の$x_0 \in X \backslash \overline{\LH{A}}$を一つ取って
			\begin{align}
				f_0(x) = 
				\begin{cases}
					0 & (x \in \overline{\LH{A}}) \\
					f_0(x_0) \neq 0 & (x = x_0)
				\end{cases}
			\end{align}
			を満たす$f_0 \in X^*$が存在する.$f_0 \in A^{\perp}$であるが$x_0 \notin (A^{\perp})^{\perp}$となり
			$(A^{\perp})^{\perp} \subset \overline{\LH{A}}$が従う.
			\QED
	\end{description}
	
	二つのノルム空間$X,Y$の直積空間$X \times Y$における直積ノルムを
	\begin{align}
		\Norm{[x,y]}{X \times Y} = \Norm{x}{X} + \Norm{y}{Y} \quad (\forall [x,y] \in X \times Y)
	\end{align}
	と表すことにする.$Y \times X$の共役空間$(Y \times X)^*$の任意の元$F$に対し
	\begin{align}
		F_Y(y) &\coloneqq F[y,0] \quad (y \in Y) \\
		F_X(x) &\coloneqq F[0,x] \quad (x \in X) \label{eq:thm_T_star_closed_1}
	\end{align}
	として$F_Y, F_X$を定義すれば,$F$の線型性,有界性から$F_Y \in Y^*,\ F_X \in X^*$となり,
	特に$F[y,x] = F_Y(y) + F_X(x)$が成り立つ.逆に$g \in Y^*$と$f \in X^*$に対し
	\begin{align}
		F[y,x] = g(y) + f(x) \quad (\forall [y,x] \in Y \times X)
	\end{align}
	と定義すれば$F \in (Y \times X)^*$となり,従って対応$(Y \times X)^* \ni F \longmapsto [F_Y,F_X] \in Y^* \times X^*$
	は全単射である.
	\begin{itembox}[l]{}
		\begin{lem}
			次の写像
			\begin{align}
				\varphi : (Y \times X)^* \ni F \longmapsto [F_Y,F_X] \in Y^* \times X^*
			\end{align}
			は線形,同相である.\label{lem:T_star_closed_2}
		\end{lem}
	\end{itembox}
	
	\begin{prf}\mbox{}
		\begin{description}
			\item[線型性]
				対応のさせ方(\refeq{eq:thm_T_star_closed_1})に基づけば,任意の$[y,x] \in Y \times X$と
				$F_1,F_2 \in (Y \times X)^*$,$\alpha \in \K$に対して
				\begin{align}
					\varphi(F_1 + F_2)[y,x] &= (F_1 + F_2)[y,0] + (F_1 + F_2)[0,x] = \varphi(F_1)[y,x] + \varphi(F_2)[y,x] \\
					\varphi(\alpha F_1)[y,x] &= (\alpha F_1)[y,0] + (\alpha F_1)[0,x] = \alpha \varphi(F_1)[y,x]
				\end{align}
				が成り立つ.
			
			\item[同相] $\varphi$はBanach空間からBanach空間への線型全単射であるから,
				$\varphi^{-1}$が有界であるなら値域定理より$\varphi$も線型有界となり,従って$\varphi$は同相写像となる.
				実際
				\begin{align}
					\Norm{[F_Y,F_X]}{Y^* \times X^*} = \Norm{F_Y}{Y^*} + \Norm{F_X}{X^*}
				\end{align}
				であることと
				\begin{align}
					\Norm{\varphi^{-1}[F_Y,F_X]}{(Y \times X)^*}
					= \sup{\substack{[y,x] \in Y \times X \\ [y,x] \neq [0,0]}}{\frac{|F_Y(y) + F_X(x)|}{\Norm{[y,x]}{Y \times X}}}
					\leq \Norm{F_Y}{Y^*} + \Norm{F_X}{X^*}
				\end{align}
				により
				\begin{align}
					\sup{\substack{[F_Y,F_X] \in Y^* \times X^* \\ [F_Y,F_X] \neq [0,0]}}{\frac{\Norm{\varphi^{-1}[F_Y,F_X]}{(Y \times X)^*}}{\Norm{[F_Y,F_X]}{Y^* \times X^*}}}
					\leq 1
				\end{align}
				が成り立つ.
		\end{description}
		\QED
	\end{prf}
	
	\begin{prf}[定理\ref{thm:T_star_closed}]
		\begin{align}
			U : X \times Y \ni [x,y] \longmapsto [y,-x] \in Y \times X
		\end{align}
		として写像$U$(等長,全単射)を定義する.$T^*$のグラフ$\mathscr{G}(T^*)$は
		\begin{align}
			\mathscr{G}(T^*) = \Set{[g,T^*g] \in Y^* \times X^*}{\forall [x,Tx] \in \mathscr{G}(T),\quad \inprod<Tx,g>_{Y,Y^*} = \inprod<x,T^*g>_{X,X^*}}
		\end{align}
		で表される.補助定理\ref{lem:T_star_closed_2}により$[g,T^*g]$に対応する$F_g \in (Y \times X)^*$がただ一つ存在して
		$\inprod<Tx,g>_{Y,Y^*} - \inprod<x,T^*g>_{X,X^*} = F_g[Tx,-x] = F_gU[x,Tx]\ ([x,Tx] \in \mathscr{G}(T))$と書き直せるから,
		補助定理\ref{lem:T_star_closed_2}の同相写像$\varphi$により
		\begin{align}
			\left[U\mathscr{G}(T) \right]^{\perp} = \Set{F \in (Y \times X)^*}{\forall [x,Tx] \in \mathscr{G}(T),\quad FU[x,Tx] = 0}
			= \varphi^{-1}\mathscr{G}(T^*)
		\end{align}
		が成り立つ.補助定理\ref{lem:T_star_closed_1}より$\left[U\mathscr{G}(T) \right]^{\perp}$が$Y^* \times X^*$の閉部分空間であるから,
		$\mathscr{G}(T^*) = \varphi \left[U\mathscr{G}(T) \right]^{\perp}$は$(Y \times X)^*$において閉部分空間となり,従って$T^*$が閉線型作用素であると示された.
		\QED
	\end{prf}
	
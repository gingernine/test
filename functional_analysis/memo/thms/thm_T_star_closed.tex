\section{共役作用素は閉作用素}
	係数体を$\K$,$X,Y$をノルム空間,$T$を$X \rightarrow Y$の線型作用素とする.
	以下では$X,Y$及びその共役空間$X^*,Y^*$におけるノルムを
	$\Norm{\cdot}{X},\ \Norm{\cdot}{Y},\ \Norm{\cdot}{X^*},\ \Norm{\cdot}{Y^*}$と表記し,
	位相は全てこれらのノルムにより導入されるものと考える.
	$T$の定義域$\mathscr{D}(T)$が$X$で稠密であるときは$T$の共役作用素$T^*$
	が定義される.この共役作用素について,本節の目標は次の定理の証明である.
	\footnote{関数解析の教科書では証明が略されていたので簡単に証明できると思い試みたが,$T$が有界とは限らない場合にどうもうまくいかなかった.
	土居先生の講義ノートを基に電子化しておく.}
	\begin{itembox}[l]{}
		\begin{thm}
			$T^*$は閉作用素である.\label{thm:T_star_closed}
		\end{thm}
	\end{itembox}

	この定理を証明するために以下にいくつか準備をする.
	
	$x \in X$と$f \in X^*$に対して$f(x)$を次の形式で表現する:
	\begin{align}
		f(x) = \inprod<x,f>_{X,X^*}.
	\end{align}
	これは双線型形式,つまり
	$\inprod<\alpha x_1 + \beta x_2,f>_{X,X^*} = \alpha \inprod<x_1,f>_{X,X^*} + \beta\inprod<x_2,f>_{X,X^*}$と$\inprod<x,\alpha f_1 + \beta f_2>_{X,X^*} = \alpha \inprod<x,f_1>_{X,X^*} + \beta\inprod<x,f_2>_{X,X^*}$を満たす.これは$f$の線型性と$X^*$における線型演算の定義による.
	双線型形式で表現することで内積空間を扱っているように捉えることができ,
	例えば,$A \subset X$に対し全ての$x \in A$で$\inprod<x,f>_{X,X^*} = 0$となるような
	$f \in X^*$の全体は$A$の直交空間である様に見做すことができる.
	
	\begin{itembox}[l]{}
		\begin{lem}
			$A \subset X$に対し
			\begin{align}
				A^{\perp} \coloneqq \left\{\ f \in X^*\quad |\quad \forall x \in A,\ \inprod<x,f>_{X,X^*} = 0\ \right\}
			\end{align}
			とおけば,$A^{\perp}$は$X^*$において閉となる.
			\label{lem:T_star_closed_1}
		\end{lem}
	\end{itembox}
	
	\begin{prf}
		$A^{\perp}$が$X^*$において完備部分集合であることを示せばよい.
		$f_n \in A^{\perp}$が収束列であるとすれば$X^*$の完備性から$(f_n)_{n=1}^{\infty}$は或る$f \in X^*$
		に(作用素ノルムで)収束する.任意の$x \in A$に対して
		\begin{align}
			|f(x)| = |f(x) - f_n(x)| \leq \Norm{f-f_n}{X^*}\Norm{x}{X} \longrightarrow 0 \quad (n \longrightarrow \infty)
		\end{align}
		が成り立ち$f \in A^{\perp}$となるから$A^{\perp}$は完備であると示された.
		\QED
	\end{prf}
	
	\begin{description}
		\item[補助定理について補足]
			実際はさらに
			\begin{align}
				(A^{\perp})^{\perp} = \overline{\LH{A}}
			\end{align}
			となることが証明される.ここで$(A^{\perp})^{\perp} = \left\{\ x \in X\quad |\quad \forall f \in A^{\perp},\ \inprod<x,f>_{X,X^*} = 0\ \right\}$
			と表現している.$A \subset (A^{\perp})^{\perp}$により$(A^{\perp})^{\perp} \supset \overline{\LH{A}}$が先ず判る.
			逆向きの包含関係について,$X = \overline{\LH{A}}$の場合は成り立つからそうでない場合を考える.
			Hahn-Banachの定理の系によれば任意の$x_0 \in X \backslash \overline{\LH{A}}$を一つ取って
			\begin{align}
				f_0(x) = 
				\begin{cases}
					0 & (x \in \overline{\LH{A}}) \\
					f_0(x_0) \neq 0 & (x = x_0)
				\end{cases}
			\end{align}
			を満たす$f_0 \in X^*$が存在する.$f_0 \in A^{\perp}$ではあるが$x_0 \notin (A^{\perp})^{\perp}$となり,
			$(A^{\perp})^{\perp} \subset \overline{\LH{A}}$が示されたことになる.
			\QED
	\end{description}
	
	二つのノルム空間$X,Y$の直積空間$X \times Y$における直積ノルムを
	\begin{align}
		\Norm{[x,y]}{X \times Y} = \Norm{x}{X} + \Norm{y}{Y} \quad (\forall [x,y] \in X \times Y)
	\end{align}
	と表すことにする.$Y \times X$の共役空間$(Y \times X)^*$の任意の元$F$に対し
	\begin{align}
		F_Y(y) &\coloneqq F[y,0] \quad (y \in Y) \\
		F_X(x) &\coloneqq F[0,x] \quad (x \in X) \label{eq:thm_T_star_closed_1}
	\end{align}
	として$F_Y, F_X$を定義すれば,$F$の線型性,有界性から$F_Y \in Y^*,\ F_X \in X^*$となり,
	特に$F[y,x] = F_Y(y) + F_X(x)$が成り立つ.逆に$g \in Y^*$と$f \in X^*$に対し
	\begin{align}
		F[y,x] = g(y) + f(x) \quad (\forall [y,x] \in Y \times X)
	\end{align}
	と定義すれば$F \in (Y \times X)^*$となり,対応$(Y \times X)^* \ni F \longmapsto [F_Y,F_X] \in Y^* \times X^*$
	が全単射であると判る.これについて次の事実を定理として載せておく.
	\begin{itembox}[l]{}
		\begin{lem}
			次の写像
			\begin{align}
				\phi : (Y \times X)^* \ni F \longmapsto [F_Y,F_X] \in Y^* \times X^*
			\end{align}
			は線形,同相である.\label{lem:T_star_closed_2}
		\end{lem}
	\end{itembox}
	
	\begin{prf}\mbox{}
		\begin{description}
			\item[線型性]
				対応のさせ方(\refeq{eq:thm_T_star_closed_1})に基づけば,任意の$[y,x] \in Y \times X$と
				$F_1,F_2 \in (Y \times X)^*$,$\alpha \in \K$に対して
				\begin{align}
					\phi(F_1 + F_2)[y,x] &= (F_1 + F_2)[y,0] + (F_1 + F_2)[0,x] = \phi(F_1)[y,x] + \phi(F_2)[y,x] \\
					\phi(\alpha F_1)[y,x] &= (\alpha F_1)[y,0] + (\alpha F_1)[0,x] = \alpha \phi(F_1)[y,x]
				\end{align}
				が成り立つから$\phi$が線型であることが判る.
			
			\item[同相] $\phi$がBanach空間からBanach空間への線型全単射であることが示されたから,
				$\phi$が有界であるなら値域定理より逆写像$\phi^{-1}$も線型有界となり,従って$\phi$が同相写像であると判る.
				\begin{align}
					\Norm{\phi(F)}{Y^* \times X^*} = \Norm{[F_Y,F_X]}{Y^* \times X^*} = \Norm{F_Y}{Y^*} + \Norm{F_X}{X^*}
				\end{align}
				であることと
				\begin{align}
					\Norm{F}{(Y \times X)^*}
					= \sup{\substack{[y,x] \in Y \times X \\ [y,x] \neq [0,0]}}{\frac{|F_Y(y) + F_X(x)|}{\Norm{[y,x]}{Y \times X}}}
					\leq \Norm{F_Y}{Y^*} + \Norm{F_X}{X^*}
				\end{align}
				により
				\begin{align}
					\sup{\substack{F \in (Y \times X)^* \\ F \neq 0}}{\frac{\Norm{\phi(F)}{Y^* \times X^*}}{\Norm{F}{(Y \times X)^*}}}
					= \sup{\substack{F \in (Y \times X)^* \\ F \neq 0}}{\frac{\Norm{[F_Y,F_X]}{Y^* \times X^*}}{\Norm{F}{(Y \times X)^*}}} \leq 1
				\end{align}
				が成り立つから$\phi$は有界である.
		\end{description}
		\QED
	\end{prf}
	
	以上の準備の下定理の証明に入る.
	\begin{prf}[定理\ref{thm:T_star_closed}]
		\begin{align}
			U : X \times Y \ni [x,y] \longmapsto [y,-x] \in Y \times X
		\end{align}
		として写像$U$(等長,全単射)を定義する.$T^*$のグラフ$\mathscr{G}(T^*)$は
		\begin{align}
			\mathscr{G}(T^*) = \left\{\ [g,T^*g] \in Y^* \times X^*\quad |\quad \forall [x,Tx] \in \mathscr{G}(T),\quad g(Tx) + T^*g(-x) = 0\ \right\}
		\end{align}
		で表される.補助定理\ref{lem:T_star_closed_2}により$[g,T^*g]$に対応する$F_g \in (Y \times X)^*$がただ一つ存在して
		$g(Tx) + T^*g(-x) = F_g[Tx,-x] = F_gU[x,Tx]\ ([x,Tx] \in \mathscr{G}(T))$と書き直せるから,
		補助定理\ref{lem:T_star_closed_2}の同相写像$\phi$により$\mathscr{G}(T^*)$に対して
		\begin{align}
			\left[U\mathscr{G}(T) \right]^{\perp} = \left\{\ F \in (Y \times X)^*\quad |\quad F[Tx,-x] = 0\ \right\}
		\end{align}
		という表現が対応する.補助定理\ref{lem:T_star_closed_1}より$\left[U\mathscr{G}(T) \right]^{\perp}$が$Y^* \times X^*$で閉となるから
		$\mathscr{G}(T^*) = \phi \left[U\mathscr{G}(T) \right]^{\perp}$は$(Y \times X)^*$において閉となり,以上で$T^*$が閉作用素であると示された.
		\QED
	\end{prf}
	
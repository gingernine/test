\section{Fredholm性}

	\begin{screen}
		\begin{lem}[商空間のコンパクト作用素]
			$X$を複素ノルム空間,$Y$を$X$の閉部分空間とする.
			$A \in \selfCop{X} $が$AY \subset Y$を満たすとき次が成り立つ:
			\begin{description}
				\item[(1)] $A_1:Y \ni y \longmapsto A y \in Y$として$A_1$を定めれば$A_1 \in \selfCop{Y} $が成り立つ.
				\item[(2)] $A_2:X/Y \ni [x] \longmapsto [Ax] \in X/Y$として$A_2$を定めれば$A_2 \in \selfCop{X/Y} $が成り立つ.
			\end{description}
			\label{thm:compact_operator_on_quotient_normed_space}
		\end{lem}
	\end{screen}
	
	\begin{prf}\mbox{}
		\begin{description}
			\item[(1)] 任意に$Y$から有界点列$(x_n)_{n=1}^{\infty}$を取る.
				補助定理\ref{lem:compact_operator_equiv_cond}
				より$(A x_n)_{n=1}^{\infty}$の部分列$\left( A x_{n_k} \right)_{k=1}^{\infty}$は
				或る$y \in X$に収束し,$Y$が閉であるから$y \in Y$を満たす.
				$A_1 x_{n_k} = A x_{n_k}\ (k=1,2,\cdots)$より$A_1 x_{n_k} \longrightarrow y\ (k \longrightarrow \infty)$
				が従い,補助定理\ref{lem:compact_operator_equiv_cond}より$A_1 \in \selfCop{Y} $が成り立つ.
				
			\item[(2)]
				\begin{description}
					\item[well-defined] $A_2$の定義はwell-definedである.つまり同値類の表示の仕方に依らない.実際$[x] = [x']$なら
						\begin{align}
							A x - A x' = A(x - x') \in Y
						\end{align}
						が成り立つから$A_2[x] = [Ax] = [Ax'] = A_2[x']$が従う.
						また$[x],[y] \in X/Y$と$\alpha,\beta \in \K$に対し
						\begin{align}
							A_2(\alpha[x] + \beta[y]) = A_2[\alpha x + \beta y] 
							= [A(\alpha x + \beta y)] = [\alpha A x + \beta A y] = \alpha [Ax] + \beta [Ay] = \alpha A_2[x] + \beta A_2[y]
						\end{align}
						が成り立つから$A_2$は線型作用素である.
						
					\item[コンパクト性]
						$B$を$X/Y$の単位開球とする.$B$から任意に取った点列$\left( [x_n] \right)_{n=1}^{\infty}$に対して
						$\left( A_2[x_n] \right)_{n=1}^{\infty}$が$X/Y$で収束する部分列を含むなら,
						定理\ref{lem:compact_operator_equiv_cond}の証明中の(※)の主張により$A_2 B$は相対コンパクトとなり,
						定理\ref{lem:unit_ball_and_compact_operator}により$A$のコンパクト性が従う.
						各$n \in \N$について$\Norm{[x_n]}{X/Y} < 1$であるから$\Norm{u_n}{X} \leq 2$を満たす$u_n \in [x_n]$が存在する.
						定理\ref{lem:compact_operator_equiv_cond}より$(A u_n)_{n=1}^{\infty}$の或る部分列
						$\left( A u_{n_k} \right)_{k=1}^{\infty}$は或る$y \in Y$に収束するから
						\begin{align}
							\Norm{A_2 \left[x_{n_k}\right] - [y]}{X/Y} = \Norm{\left[ A x_{n_k} - y \right]}{X/Y} 
							\leq \Norm{A x_{n_k} - y}{X} \longrightarrow 0 \quad (k \longrightarrow \infty)
						\end{align}
						が成り立つ.
						\QED
				\end{description}
		\end{description}
	\end{prf}
	
	\begin{screen}
		\begin{thm}[複素Banach空間上のコンパクト作用素の値域の余次元,核の次元]\mbox{}\\
			$X$を複素Banach空間,$I$を$X$上の恒等写像とし,$0 \neq \lambda \in \C$と$A \in \selfCop{X} $に対して
			$T \coloneqq \lambda I - A$とおく.このとき
			$\Ran{T}$は$X$の閉部分空間であり,$\Dim{\Ker{T}} < \infty$かつ$\Codim{\Ran{T}} < \infty$\footnotemark
			が成り立つ.
			\label{thm:Banach_space_compact_operator_kernel_dimension}
		\end{thm}
	\end{screen}
	
	\footnotetext{
		$\Codim{\Ran{T}} = \Dim{X/\Ran{T}}$である.
	}
	
	\begin{prf}\mbox{}
		\begin{description}
			\item[$\Ran{T}$が閉となること]
				\begin{align}
					\hat{T}:X/\Ker{T} \ni [x] \longmapsto Tx \in \Ran{T}
				\end{align}
				と定めれば$\hat{T}$は線型同型かつ連続となる:
				\begin{description}
					\item[全単射]
						$\hat{T}$が単射であることは,$T[x] = T[x']$ならば$x - x' \in \Ker{T}$より
						$[x] = [x']$が従い,また任意の$y \in \Ran{T}$に対して,$y = Tx$を満たす$x \in X$の
						同値類$[x] \in X/\Ker{T}$が$\hat{T}[x] = y$を満たすから$\hat{T}$は全射である.
					
					\item[線型性]
						任意に$[x],[y] \in X/\Ker{T}$と$\alpha,\beta \in \C$を取れば
						\begin{align}
							\hat{T}\left( \alpha[x] + \beta[y] \right)
							= \hat{T}\left( [\alpha x] + [\beta y] \right)
							= T(\alpha x + \beta y)
							= \alpha T x + \beta T y
							= \alpha \hat{T} [x] + \beta \hat{T} [y]
						\end{align}
						が成立する.
						
					\item[連続性]
						定理\ref{prp:compact_operator_bounded_composition_of_compact_operators}より$A$は有界であるから
						\begin{align}
							\Norm{T}{\selfBop{X}} = \Norm{\lambda I - A}{\selfBop{X}} \leq |\lambda| + \Norm{A}{\selfBop{X}} < \infty
						\end{align}
						が成り立ち,任意の$[x] \in X/\Ker{T}$に対して
						\begin{align}
							\Norm{\hat{T}[x]}{X} = \Norm{Tx}{X} \leq \Norm{T}{\selfBop{X}} \Norm{x}{X}
						\end{align}
						が従うから$\hat{T}$は連続である.
				\end{description}		
				$\Ran{T} = \Ran{\hat{T}}$であるから$\Ran{\hat{T}}$が$X$の閉部分空間となることを示せばよい.
				まず或る$C > 0$が存在して
				\begin{align}
					C \Norm{\hat{T}[x]}{X} \geq \Norm{[x]}{X/\Ker{T}} \quad (\forall x \in X)
					\label{eq:thm_Banach_space_compact_operator_kernel_dimension_3}
				\end{align}
				を満たすことを示す.
				\begin{description}
					\item[(\refeq{eq:thm_Banach_space_compact_operator_kernel_dimension_3})の証明]
						このような$C$が存在しないなら
						\begin{align}
							\Norm{\hat{T}[x_n]}{X} < \frac{1}{n} \Norm{[x_n]}{X/\Ker{T}}
							\quad (n=1,2,\cdots) 
							\label{eq:thm_Banach_space_compact_operator_kernel_dimension}
						\end{align}
						を満たす$X/\Ker{T}$の点列$\left( [x_n] \right)_{n=1}^{\infty}$が存在する.
						\begin{align}
							[y_n] \coloneqq \frac{1}{\Norm{[x_n]}{X/\Ker{T}}} [x_n] \quad (n=1,2,\cdots)
						\end{align}
						とおけば$\left( [y_n] \right)_{n=1}^{\infty}$も(\refeq{eq:thm_Banach_space_compact_operator_kernel_dimension})
						を満たし,かつ$\hat{T}[y_n] = \hat{T}[u_n] = T u_n$であるから
						\begin{align}
							\Norm{Tu_n}{X} = \Norm{\hat{T}[y_n]}{X} < \frac{1}{n} \longrightarrow 0
							\quad (n \longrightarrow \infty)
							\label{eq:thm_Banach_space_compact_operator_kernel_dimension_2}
						\end{align}
						が成立する.$\Norm{[y_n]}{X/\Ker{T}} = 1$であるから
						ノルムの定義(\refeq{eq:thm_quotient_space_norm})より$\Norm{u_n}{X} \leq 2$となる$u_n \in [y_n]$が存在し,
						定理\ref{lem:compact_operator_equiv_cond}より
						$\left( Au_n \right)_{n=1}^{\infty}$の或る部分列$\left( Au_{n_k} \right)_{k=1}^{\infty}$は
						或る$y \in X$に収束するから
						\begin{align}
							\Norm{y - \lambda u_{n_k}}{X} = \Norm{y - Au_{n_k} - Tu_{n_k}}{X} \leq \Norm{y - Au_{n_k}}{X} + \Norm{Tu_{n_k}}{X}
							\longrightarrow 0 \quad (k \longrightarrow \infty)
						\end{align}
						が成り立ち,更に$T$の有界性と(\refeq{eq:thm_Banach_space_compact_operator_kernel_dimension_2})より
						\begin{align}
							\Norm{Ty}{X} \leq \Norm{Ty - \lambda Tu_{n_k}}{X} + |\lambda| \Norm{Tu_{n_k}}{X} \longrightarrow 0
							\quad (k \longrightarrow \infty)
						\end{align}
						となり$y \in \Ker{T}$が従う.一方で
						\begin{align}
							\left|\, \Norm{[y]}{X/\Ker{T}} - \Norm{\lambda \left[y_{n_k}\right]}{X/\Ker{T}}\, \right| 
							\leq \Norm{[y] - \lambda \left[y_{n_k}\right]}{X/\Ker{T}} %= \Norm{\left[y - \lambda u_{n_k}\right]}{X/\Ker{T}}
							\leq \Norm{y - \lambda u_{n_k}}{X} \longrightarrow 0 \quad (k \longrightarrow \infty)
						\end{align}
						が成り立つから$\Norm{[y]}{X/\Ker{T}} = |\lambda| > 0$が従い$y \in \Ker{T}$に矛盾する.
				\end{description}
				$\Ran{\hat{T}}$の点列$\left( \hat{T}[v_n] \right)_{n=1}^{\infty}$が
				$\hat{T}[v_n] \rightarrow x \in X$を満たすなら,(\refeq{eq:thm_Banach_space_compact_operator_kernel_dimension_3})より
				\begin{align}
					\Norm{[v_n] - [v_m]}{X/\Ker{T}} \leq C \Norm{\hat{T}[v_n] - \hat{T}[v_m]}{X} \longrightarrow 0
					\quad (n,m \longrightarrow \infty)
				\end{align}
				が成り立ち,定理\ref{thm:quotient_normed_space_Banach}より$\left( [v_n] \right)_{n=1}^{\infty}$は或る$[v] \in X/\Ker{T}$に収束する.
				よって$\hat{T}$の連続性から
				\begin{align}
					\Norm{x - \hat{T}[v]}{X} 
					%\leq \Norm{x - \hat{T}[v_n]}{X} + \Norm{\hat{T}[v_n] - \hat{T}[v]}{X}
					\leq \Norm{x - \hat{T}[v_n]}{X} + \Norm{\hat{T}}{\selfBop{\hat{T}}} \Norm{[v_n] - [v]}{X}
					\longrightarrow 0 \quad (n \longrightarrow \infty)
				\end{align}
				が成り立ち$x = \hat{T}[v] \in \Ran{\hat{T}}$が従う.
			
			\item[$\Dim{\Ker{T}} < \infty$となること]	
				$T = \lambda I - A$より
				\begin{align}
					\lambda x = A x \quad (\forall x \in \Ker{T})
					\label{eq:thm_Banach_space_compact_operator_kernel_dimension_4}
				\end{align}
				が成り立つから
				\begin{align}
					T A x = T \lambda x = \lambda T x = 0 \quad (\forall x \in \Ker{T})
				\end{align}
				となり$A \Ker{T} \subset \Ker{T}$が従う.
				よって$\Ker{T}$から任意に有界点列$(x_n)_{n=1}^{\infty}$を取れば
				$\left(A x_n \right)_{n=1}^{\infty}$は閉部分空間$\Ker{T}$に含まれ,
				定理\ref{lem:compact_operator_equiv_cond}より或る部分列
				$\left(A x_{n_k} \right)_{k=1}^{\infty}$は或る$x \in \Ker{T}$に収束する.
				そして(\refeq{eq:thm_Banach_space_compact_operator_kernel_dimension_4})より
				\begin{align}
					\Norm{\frac{1}{\lambda}x - x_{n_k}}{X}
					= \frac{1}{|\lambda|} \Norm{x - \lambda x_{n_k}}{X}
					= \frac{1}{|\lambda|} \Norm{x - A x_{n_k}}{X}
					\longrightarrow 0 \quad (k \longrightarrow \infty)
				\end{align}
				が成り立つから,定理\ref{thm:compact_identity_operator_and_dimension}より
				$\Dim{X} < \infty$が従う.
				
			\item[$\Codim{\Ran{T}} < \infty$となること]
				$\Ran{T}$は$X$の閉部分空間であるから商ノルム空間$X/\Ran{T}$を定義できる.
				\begin{align}
					U:X/\Ran{T} \ni [x] \longmapsto [Ax] \in X/\Ran{T}
				\end{align}
				と定めれば定理\ref{thm:compact_operator_on_quotient_normed_space}より
				$U$はコンパクト作用素である.
				\begin{align}
					[0] = [Tx] = [\lambda x - Ax] = \lambda [x] - [Ax] = \lambda [x] - U[x] \quad (\forall x \in X)
				\end{align}
				が成り立つから,定理\ref{thm:compact_identity_operator_and_dimension}より
				$\Dim{X/\Ran{T}} < \infty$が従う.
				\QED
		\end{description}
	\end{prf}
	
	\begin{screen}
		\begin{thm}[Fredholmの交代定理]
			
		\end{thm}
	\end{screen}
	
	\begin{screen}
		\begin{lem}
			$E$を複素ノルム空間,$E_1,E_2$を$E$の線型部分空間とし
			$E = E_1 + E_2$が成り立っているとする\footnotemark.
			また$E,E_1 \times E_2$におけるノルムをそれぞれ$\Norm{\cdot}{E},\Norm{\cdot}{E_1 \times E_2}$としてノルム位相を導入し
			\begin{align}
				\Phi:E \ni x \longmapsto [x_1,x_2] \in E_1 \times E_2
				\quad (x = x_1 + x_2)
			\end{align}
			を定める.このとき次が成り立つ:
			\begin{description}
				\item[(1)] $\Phi$は全単射かつ閉線型である.
				\item[(2)] $\Phi^{-1}$は連続である.
				\item[(3)] $\Phi$が連続ならば$E_1,E_2$は閉部分空間である.
				\item[(4)] $E$がBanach空間で$E_1,E_2$が閉部分空間ならば$\Phi$は線型同型かつ同相である.
				\item[(5)] $\Dim{E_1} < \infty$かつ$E_2$が閉ならば$\Phi$は線型同型かつ同相である.
			\end{description}
		\end{lem}
	\end{screen}
	
	\footnotetext{
		つまり$E_1 \cap E_2 = \{0\}$であり,かつ$E$の任意の元$x$は
		或る$x_1 \in E_1,x_2 \in E_2$によって$x = x_1 + x_2$と一意に表される.
		一意性について,$x = y_1 + y_2\ (y_1 \in E_1,y_2 \in E_2)$が同時に成り立っているとすれば
		\begin{align}
			E_1 \ni x_1 - y_1 = y_2 - x_2 \in E_2
		\end{align}
		となるから$ x_1 - y_1 = y_2 - x_2 = 0$が従う.
	}
	
	\begin{prf}\mbox{}
		\begin{description}
			\item[(1)] 
				\begin{description}
					\item[全単射であること]
						任意に$[x_1,x_2] \in E_1 \times E_2$を取れば
						$x_1 + x_2 \in E$を満たすから$\Phi$は全射である.
						また$E_1 \times E_2$の二元が$[x_1,x_2] = [y_1,y_2]$を満たせば
						$x_1 = y_1$かつ$x_2 = y_2$となるから$\Phi$は単射である.
						
					\item[閉線型であること]
						$x,y \in E,\alpha \in \C$を任意に取り$\Phi x = [x_1,x_2], \Phi y = [y_1,y_2]$とすれば,
						\begin{align}
							\Phi(x + y) &= [x_1 + y_1, x_2 + y_2] = [x_1,x_2] + [y_1,y_2] = \Phi x + \Phi y, \\
							\Phi(\alpha x) &= [\alpha x_1, \alpha x_2] = \alpha [x_1,x_2] = \alpha \Phi x
						\end{align}
						より$\Phi$の線型性が従う.
						また$(x_n)_{n=1}^{\infty} \subset E$が$x_n \rightarrow u \in X$かつ
						$\Phi x_n \rightarrow [u_1,u_2] \in E_1 \times E_2$を満たす場合,
						\begin{align}
							\Norm{u - (u_1 + u_2)}{E} \leq \Norm{u - x_n}{E} + \Norm{\Phi x_n - [u_1,u_2]}{E_1 \times E_2}
							\longrightarrow 0 \quad (n \longrightarrow \infty)
						\end{align}
						が成り立ち$\Phi u = [u_1,u_2]$が従うから$\Phi$は閉作用素である.
				\end{description}
				
			\item[(2)] (1)より逆写像$\Phi^{-1}:E_1 \times E_2 \rightarrow E\ $(線形全単射)が存在し,任意の$[0,0] \neq [x_1,x_2] \in E_1 \times E_2$に対して
				\begin{align}
					\frac{\Norm{\Phi^{-1}[x_1, x_2]}{E}}{\Norm{[x_1, x_2]}{E_1 \times E_2}} 
					= \frac{\Norm{x_1 + x_2}{E}}{\Norm{x_1}{E} + \Norm{x_2}{E}} \leq 1
				\end{align}
				を満たす.
				
			\item[(3)] ノルム空間において一点集合$\{0\}$は閉であるから,直積位相において$E_1 \times \{0\}$及び$\{0\} \times E_2$は閉集合である.
				従って$\Phi$の連続性と$E_1 = \Phi^{-1}(E_1 \times \{0\})$及び$E_2 = \Phi^{-1}(\{0\} \times E_2)$が成り立つことから
				$E_1,E_2$は閉集合となる.
			
			\item[(4)] $E,E_1 \times E_2$はBanach空間でありかつ
				$\Dom{\Phi} = E$が満たされているから,閉グラフ定理より$\Phi$は有界となる.
				(1)(2)と併せれば$\Phi,\Phi^{-1}$は共に連続且つ線型全単射であるから主張が従う.
			
			\item[(5)] $E \rightarrow E$の恒等写像を$I$と表す.また
				\begin{align}
					p_1:E \ni x \longmapsto [x] \in E/E_2,
					\quad p_2:E/E_2 \ni [x] \longmapsto x_1 \in E_1 \quad (x = x_1 + x_2,\ x_1 \in E_1,\ x_2 \in E_2)
				\end{align}
				と定めれば$p_1$は線型連続であり$p_2$は線型同型かつ連続である:
				\begin{description}
					\item[$p_1$について] 任意に$x,y \in E$と$\alpha, \beta \in \C$を取れば
						\begin{align}
							p_1(\alpha x + \beta y) = [\alpha x + \beta y] = [\alpha x] + [\beta y] 
							= \alpha [x] + \beta [y] = \alpha p_1 x + \beta p_1 y
						\end{align}
						が成り立ち$p_1$の線型性が従う.また$x \in E,\ x \neq 0$に対して
						\begin{align}
							\frac{\Norm{p_1 x}{E/E_2}}{\Norm{x}{E}} = \frac{\Norm{[x]}{E/E_2}}{\Norm{x}{E}} \leq \frac{\Norm{x}{E}}{\Norm{x}{E}} = 1
						\end{align}
						となるから$p_1$は連続である.
						
					\item[$p_2$について] $E$から$E_1$への線型準同型を
						\begin{align}
							p:E \ni x \longmapsto x_1 \in E_1 \quad (x = x_1 + x_2,\ x_1 \in E_1,\ x_2 \in E_2)
						\end{align}
						で定める.$\Ran{p} = E_1$かつ$\Ker{p} = E_2$であるから,準同型定理より$p_2$は線型同型となる.また
						$\Dim{E_1} < \infty$であるから$\Dim{E/E_2} = \Dim{E_1} < \infty$となり
						\footnote{
							一般の線形空間$X,Y$に対し,$\Dim{X} = k < \infty$且つ線型同型$f:X \rightarrow Y$が存在するなら
							$\Dim{Y} = k$が成り立つ.実際
							$X$の基底を$x_1,\cdots,x_k$とすれば
							$f(x_1),\cdots,f(x_k)$は$Y$の基底となる.
							$\alpha_1,\cdots,\alpha_k \in \C$に対し
							\begin{align}
								\alpha_1 f(x_1) + \cdots + \alpha_k f(x_k) = 0
							\end{align}
							が成り立っている場合,$f$が線型かつ単射であるから
							\begin{align}
								\alpha_1 x_1 + \cdots + \alpha_k x_k = 0
							\end{align}
							となり$f(x_1),\cdots,f(x_k)$の線型独立性が従う.また任意に$y \in Y$を取れば或る$x \in X$が対応し$f(x) = y$を満たすから,
							\begin{align}
								y = f(x) = f(\alpha_1 x_1 + \cdots + \alpha_k x_k) = \alpha f(x_1) + \cdots + \alpha f(x_k)
							\end{align}
							が成り立ち$Y = \LH{\left\{\, f(x_1),\cdots,f(x_k)\, \right\}}$が従う.
						}
						$p_2$の連続性が従う.
				\end{description}
				$\Phi$は$p_1,p_2$を用いて
				\begin{align}
					\Phi x = [p_2 p_1 x, (I - p_2 p_1) x] \quad (\forall x \in E)
				\end{align}
				と表現できるから
				\begin{align}
					\Norm{\Phi x}{E_1 \times E_2} = \Norm{p_2 p_1 x}{E} + \Norm{(I - p_2 p_1) x}{E} 
				\end{align}
				により$\Phi$の連続性が従い,(1)(2)と併せて主張を得る.
				\QED
		\end{description}
	\end{prf}
	
	\begin{screen}
		\begin{lem}[$T$が単射なら全射]	$X$を複素Banach空間,$I$を$X$上の恒等写像とし,$0 \neq \lambda \in \C$と$A \in \selfCop{X} $に対して
			$T \coloneqq \lambda I - A$とおく.このとき$T$が単射のならば$T$は全射である.
			\label{lem:T_injective_then_surjective}
		\end{lem}
	\end{screen}
	
	\begin{prf}
		背理法で示す.今$T$が単射であり全射ではないとする.このとき
		\begin{align}
			\Ran{T^k} \supsetneq \Ran{T^{k+1}} \quad (k = 1,2,\cdots)
		\end{align}
		が成り立つ.実際或る$k \in \N$で$\Ran{T^k} = \Ran{T^{k+1}}$が成り立つなら,
		任意の$y \in X$に対し或る$x \in X$が存在して
		\begin{align}
			T^{k} y = T^{k+1} x = T^k T x
		\end{align}
		を満たすが,$T^k$が単射であるから$y = T x$が従い$T$が全射でないという仮定に反する.
		\begin{align}
			X_k \coloneqq \Ran{T^k} \quad (k = 1,2,\cdots)
		\end{align}
		と簡単に表せば,定理\ref{thm:Banach_space_compact_operator_kernel_dimension}より
		$X_k$は$X$の閉部分空間であり,定理\ref{thm:closed_subspace_distance}より
		\begin{align}
			\Norm{x_k}{X} = 1,
			\quad \inf{x \in X_k}{\Norm{x_k - x}{X}} > \frac{1}{2}
			\label{eq:lem_T_injective_then_surjective}
		\end{align}
		を満たす$x_k \in X_k \backslash X_{k+1}\ (k=1,2,\cdots)$が存在する.
		$n < m$となる$n,m \in \N$を取れば
		\begin{align}
			T x_n + A x_m = T x_n + \lambda x_m - T x_m \in X_{n+1}
		\end{align}
		が成り立つから,(\refeq{eq:lem_T_injective_then_surjective})より
		\begin{align}
			\Norm{A x_n - A x_m}{X} = \Norm{\lambda x_n - T x_n - A x_m}{X} > \frac{|\lambda|}{2}
		\end{align}
		が従い$\left( A x_k \right)_{k=1}^{\infty}$
		は収束部分列を含み得ないが,これは定理\ref{lem:compact_operator_equiv_cond}に矛盾する.
		\QED
	\end{prf}
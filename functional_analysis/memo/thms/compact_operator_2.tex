\section{Fredholm性}
	
	以降ノルム空間の商空間は付録\ref{chap:quotient_normed_space}章に従って定義する.
	
	\begin{screen}
		\begin{lem}[商空間のコンパクト作用素]
			$X$を複素ノルム空間,$Y$を$X$の閉部分空間とする.
			$A \in \selfCop{X} $が$AY \subset Y$を満たすとき次が成り立つ:
			\begin{description}
				\item[(1)] $A_1:Y \ni y \longmapsto A y \in Y$として$A_1$を定めれば$A_1 \in \selfCop{Y} $が成り立つ.
				\item[(2)] $A_2:X/Y \ni [x] \longmapsto [Ax] \in X/Y$として$A_2$を定めれば$A_2 \in \selfCop{X/Y} $が成り立つ.
			\end{description}
		\end{lem}
	\end{screen}
	
	\begin{prf}\mbox{}
		\begin{description}
			\item[(1)] 任意に$Y$から有界点列$(x_n)_{n=1}^{\infty}$を取る.
				補助定理\ref{lem:compact_operator_equiv_cond}
				より$(A x_n)_{n=1}^{\infty}$の部分列$\left( A x_{n_k} \right)_{k=1}^{\infty}$は
				或る$y \in X$に収束し,$Y$が閉であるから$y \in Y$を満たす.
				$A_1 x_{n_k} = A x_{n_k}\ (k=1,2,\cdots)$より$A_1 x_{n_k} \longrightarrow y\ (k \longrightarrow \infty)$
				が従い,補助定理\ref{lem:compact_operator_equiv_cond}より$A_1 \in \selfCop{Y} $が成り立つ.
				
			\item[(2)]
				\begin{description}
					\item[well-defined] $A_2$の定義はwell-definedである.つまり同値類の表示の仕方に依らない.実際$[x] = [x']$なら
						\begin{align}
							A x - A x' = A(x - x') \in Y
						\end{align}
						が成り立つから$A_2[x] = [Ax] = [Ax'] = A_2[x']$が従う.
						また$[x],[y] \in X/Y$と$\alpha,\beta \in \K$に対し
						\begin{align}
							A_2(\alpha[x] + \beta[y]) = A_2[\alpha x + \beta y] 
							= [A(\alpha x + \beta y)] = [\alpha A x + \beta A y] = \alpha [Ax] + \beta [Ay] = \alpha A_2[x] + \beta A_2[y]
						\end{align}
						が成り立つから$A_2$は線型作用素である.
						
					\item[コンパクト性]
						$B$を$X/Y$の単位開球とする.$B$から任意に取った点列$\left( [x_n] \right)_{n=1}^{\infty}$に対して
						$\left( A_2[x_n] \right)_{n=1}^{\infty}$が$X/Y$で収束する部分列を含むなら,
						定理\ref{lem:compact_operator_equiv_cond}の証明中の(※)の主張により$A_2 B$は相対コンパクトとなり,
						定理\ref{lem:unit_ball_and_compact_operator}により$A$のコンパクト性が従う.
						各$n \in \N$について$\Norm{[x_n]}{X/Y} < 1$であるから$\Norm{u_n}{X} \leq 2$を満たす$u_n \in [x_n]$が存在する.
						定理\ref{lem:compact_operator_equiv_cond}より$(A u_n)_{n=1}^{\infty}$の或る部分列
						$\left( A u_{n_k} \right)_{k=1}^{\infty}$は或る$y \in Y$に収束するから
						\begin{align}
							\Norm{A_2 \left[x_{n_k}\right] - [y]}{X/Y} = \Norm{\left[ A x_{n_k} - y \right]}{X/Y} 
							\leq \Norm{A x_{n_k} - y}{X} \longrightarrow 0 \quad (k \longrightarrow \infty)
						\end{align}
						が成り立つ.
						\QED
				\end{description}
		\end{description}
	\end{prf}
	
	\begin{screen}
		\begin{thm}[複素Banach空間上のコンパクト作用素の値域,核の次元]
			
		\end{thm}
	\end{screen}
	
	\begin{screen}
		\begin{thm}[Fredholmの交代定理]
			
		\end{thm}
	\end{screen}
	
	\begin{screen}
		\begin{lem}
			$E$を複素ノルム空間,$E_1,E_2$を$E$の線型部分空間とし
			$E = E_1 + E_2$が成り立っているとする\footnotemark.
			また$E,E_1 \times E_2$におけるノルムをそれぞれ$\Norm{\cdot}{E},\Norm{\cdot}{E_1 \times E_2}$としてノルム位相を導入し
			\begin{align}
				\Phi:E \ni x \longmapsto [x_1,x_2] \in E_1 \times E_2
				\quad (x = x_1 + x_2)
			\end{align}
			を定める.このとき次が成り立つ:
			\begin{description}
				\item[(1)] $\Phi$は全単射かつ閉線型である.
				\item[(2)] $\Phi^{-1}$は連続である.
				\item[(3)] $\Phi$が連続ならば$E_1,E_2$は閉部分空間である.
				\item[(4)] $E$がBanach空間で$E_1,E_2$が閉部分空間ならば$\Phi$は線型同型かつ同相である.
				\item[(5)] $\Dim{E_1} < \infty$かつ$E_2$が閉ならば$\Phi$は線型同型かつ同相である.
			\end{description}
		\end{lem}
	\end{screen}
	
	\footnotetext{
		つまり$E_1 \cap E_2 = \{0\}$であり,かつ$E$の任意の元$x$は
		或る$x_1 \in E_1,x_2 \in E_2$によって$x = x_1 + x_2$と一意に表される.
		一意性について,$x = y_1 + y_2\ (y_1 \in E_1,y_2 \in E_2)$が同時に成り立っているとすれば
		\begin{align}
			E_1 \ni x_1 - y_1 = y_2 - x_2 \in E_2
		\end{align}
		となるから$ x_1 - y_1 = y_2 - x_2 = 0$が従う.
	}
	
	\begin{prf}\mbox{}
		\begin{description}
			\item[(1)] 
				\begin{description}
					\item[全単射であること]
						任意に$[x_1,x_2] \in E_1 \times E_2$を取れば
						$x_1 + x_2 \in E$を満たすから$\Phi$は全射である.
						また$E_1 \times E_2$の二元が$[x_1,x_2] = [y_1,y_2]$を満たせば
						$x_1 = y_1$かつ$x_2 = y_2$となるから$\Phi$は単射である.
						
					\item[閉線型であること]
						$x,y \in E,\alpha \in \C$を任意に取り$\Phi x = [x_1,x_2], \Phi y = [y_1,y_2]$とすれば,
						\begin{align}
							\Phi(x + y) &= [x_1 + y_1, x_2 + y_2] = [x_1,x_2] + [y_1,y_2] = \Phi x + \Phi y, \\
							\Phi(\alpha x) &= [\alpha x_1, \alpha x_2] = \alpha [x_1,x_2] = \alpha \Phi x
						\end{align}
						より$\Phi$の線型性が従う.
						また$(x_n)_{n=1}^{\infty} \subset E$が$x_n \rightarrow u \in X$かつ
						$\Phi x_n \rightarrow [u_1,u_2] \in E_1 \times E_2$を満たす場合,
						\begin{align}
							\Norm{u - (u_1 + u_2)}{E} \leq \Norm{u - x_n}{E} + \Norm{\Phi x_n - [u_1,u_2]}{E_1 \times E_2}
							\longrightarrow 0 \quad (n \longrightarrow \infty)
						\end{align}
						が成り立ち$\Phi u = [u_1,u_2]$が従うから$\Phi$は閉作用素である.
				\end{description}
				
			\item[(2)] (1)より逆写像$\Phi^{-1}:E_1 \times E_2 \rightarrow E\ $(線形全単射)が存在し,任意の$[0,0] \neq [x_1,x_2] \in E_1 \times E_2$に対して
				\begin{align}
					\frac{\Norm{\Phi^{-1}[x_1, x_2]}{E}}{\Norm{[x_1, x_2]}{E_1 \times E_2}} 
					= \frac{\Norm{x_1 + x_2}{E}}{\Norm{x_1}{E} + \Norm{x_2}{E}} \leq 1
				\end{align}
				を満たす.
				
			\item[(3)] ノルム空間において一点集合$\{0\}$は閉であるから,直積位相において$E_1 \times \{0\}$及び$\{0\} \times E_2$は閉集合である.
				従って$\Phi$の連続性と$E_1 = \Phi^{-1}(E_1 \times \{0\})$及び$E_2 = \Phi^{-1}(\{0\} \times E_2)$が成り立つことから
				$E_1,E_2$は閉集合となる.
			
			\item[(4)] $E,E_1 \times E_2$はBanach空間でありかつ
				$\Dom{\Phi} = E$が満たされているから,閉グラフ定理より$\Phi$は有界となる.
				(1)(2)と併せれば$\Phi,\Phi^{-1}$は共に連続且つ線型全単射であるから主張が従う.
			
			\item[(5)] 
				$E_2$が閉部分空間であるから,
				(\refeq{eq:thm_quotient_space_norm})をノルムとする商ノルム空間$E/E_2$を定義することができる.
				\begin{align}
					p_1:E \ni x \longmapsto [x] \in E/E_2,
					\quad p_2:E/E_2 \ni [x] \longmapsto x \in E_1
				\end{align}
				と定めれば$p_1$は線型連続であり$p_2$は線型同型である.
				\begin{description}
					\item[$p_1$について]
						$p_1$の線型性は
					\item[$p_2$について]
				\end{description}
				
		\end{description}
	\end{prf}
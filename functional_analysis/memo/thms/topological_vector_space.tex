\section{位相線形空間}
	位相空間$X$が分離公理$T_1$を満たすとは,
	或いは$T_1$-空間であるとは,
	任意の$x,y \in X$に対して$y$の近傍$U_y$
	で$x \notin U_y$となるものが存在することをいう.
	
	\begin{screen}
		\begin{thm}[$T_1$-空間の特徴づけ]
			位相空間$X$が$T_1$-空間であるための必要十分条件は,
			$X$の任意の一点集合が閉であることである.
		\end{thm}
	\end{screen}
	
	\begin{screen}
		\begin{dfn}[位相線形空間]
			$X$を線形空間とする.
			$X$に対し次を満たす位相$\tau$が付随しているとき,
			$\tau$を線型位相(vector topology),
			$X$を位相線形空間(topological vector space)と呼ぶ.
			\begin{description}
				\item[(1)] $(X,\tau)$は$T_1$-空間である.
				\item[(2)] $X$上の線型演算(加法・スカラ倍)が連続である.
			\end{description}
		\end{dfn}
	\end{screen}
\section{緩増加$C^m$-関数}
	\begin{screen}
		\begin{dfn}[緩増加$C^m$-関数]
			$f:\R^n \longrightarrow \C$が$C^m$-級で,各$\alpha \in \N^n$に対して
			或る定数$C_\alpha$と$\ell_\alpha \in \N$が存在し
			\begin{align}
				\left| \partial^\alpha f(x) \right| \leq C_\alpha(1+|x|^2)^{\ell_\alpha},
				\quad (\forall x \in \R^n,\ |\alpha| \leq m)
				\label{eq:tempered_c^m_function}
			\end{align}
			が成り立つとき,$f$を緩増加$C^m$-関数(tempered $C^m$-function)という.
			$m=0$の場合$f$を緩増加連続関数と呼び,
			$m=\infty$の場合は緩増加関数と呼ぶ.
			また緩増加連続関数,緩増加関数の全体のなす線形空間を$\mathscr{O}_C,\ \mathscr{O}_M$で表す.
		\end{dfn}
	\end{screen}
	
	緩増加$C^m$-関数$f$に対し(\refeq{eq:tempered_c^m_function})を満たす
	$C_\alpha$と$\ell_\alpha$について,
	$C \coloneqq \max{|\alpha| \leq m}{C_\alpha}
	,\ \ell \coloneqq \max{|\alpha| \leq m}{\ell_\alpha}$とおけば
	\begin{align}
		\left| \partial^\alpha f(x) \right| \leq C(1+|x|^2)^\ell,
		\quad (\forall x \in \R^n,\ |\alpha| \leq m)
		\label{eq:tempered_c^m_function_2}
	\end{align}
	が成立する.特に緩増加関数に対しては任意の$m \in \N$ごとに$C,\ \ell$が定まり
	(\refeq{eq:tempered_c^m_function_2})が満たされる.
	
	\begin{screen}
		\begin{thm}[緩増加連続関数により定まる緩増加超関数]
		\label{thm:tempered_continuous_functions_and_tempered_distributions}
			緩増加連続関数$f:\R^n \longrightarrow \C$に対して
			\begin{align}
				u_f: \rapid{\R^n} \ni \varphi \longmapsto
				\int_{\R^n} f(x) \varphi(x)\ dx
			\end{align}
			により定める$u_f$は緩増加超関数である.またこの対応
			$\mathscr{O}_C \longrightarrow \tempdist{\R^n}\ (f \longmapsto u_f)$は
			線型単射である.
		\end{thm}
	\end{screen}
	
	\begin{prf}
		$f$に対し或る定数$C$と$\ell \in \N$が存在して(\refeq{eq:tempered_c^m_function})が満たされ
		\begin{align}
			\int_{\R^n} |f(x)| |\varphi(x)|\ dx
			\leq C \int_{\R^n} (1+|x|^2)^\ell |\varphi(x)|\ dx
			\leq \left[ C \int_{\R^n} \frac{1}{(1+|x|^2)^n}\ dx \right] p_{m+\ell}(\varphi) 
		\end{align}
		が成立する.可積分性より$u_f$は線型性をもち,また半ノルム$p_{m+\ell}$で抑えられているから$u_f$の連続性も出る.
		上の可積分性より$f \longmapsto u_f$の線型性も従い,
		単射であることは変分法の基本補題より得られる.
		\QED
	\end{prf}
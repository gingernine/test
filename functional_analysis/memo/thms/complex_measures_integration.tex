\section{複素測度に関する積分}
	\begin{screen}
		\begin{thm}[複素数値可測関数]
			$(X,\mathcal{M})$を可測空間とする.$f:X \rightarrow \C$について次が成り立つ:
			\begin{description}
				\item[(1)] $f$が可測$\mathcal{M}/\borel{\C}$であることと
					$f$の実部虚部がそれぞれ可測$\mathcal{M}/\borel{\R}$であることは同値である.
					
				\item[(2)] $\mathcal{M}/\borel{\C}$-可測関数列$(f_n)_{n=1}^{\infty}$が$f$に各点収束するなら
					$f$もまた可測$\mathcal{M}/\borel{\C}$となる.
			\end{description}
		\end{thm}
	\end{screen}
	
	\begin{screen}
		\begin{dfn}[積分の定義]
			$(X,\mathcal{M})$を可測空間とし,$\mu$を$(X,\mathcal{M})$上の複素測度とする.
			$\mu$の総変動測度$|\mu|$に関して可積分となる関数$f:X \rightarrow \C$について
			,$f$の$\mu$に関する積分を次で定める:
			\begin{description}
				\item[$f$が可測単関数の場合]
					有限個の複素数$\alpha_1,\cdots,\alpha_k$と集合$A_1,\cdots,A_k \in \mathcal{M}$によって
					\begin{align}
						f = \sum_{i=1}^{k} \alpha_i \defunc_{A_i}
						\label{eq:expression_simple_function}
					\end{align}
					と表されるとき\footnotemark
					,$f$の$\mu$に関する積分を
					\begin{align}
						\int_X f(x)\ \mu(dx) \coloneqq \sum_{i=1}^{k} \alpha_i \mu(A_i)
						\label{eq:complex_measures_integration_simple_function}
					\end{align}
					で定める.
					
				\item[$f$が一般の可測関数の場合]
					\begin{align}
						\int_X f_n(x)\ |\mu|(dx) \longrightarrow \int_X f(x)\ |\mu|(dx)
						\label{eq:integration_simple_function_approximation}
					\end{align}
					を満たす$f$の可測単関数近似列$(f_n)_{n=1}^{\infty}$を取り,
					$f$の$\mu$に関する積分を
					\begin{align}
						\int_X f(x)\ \mu(dx) \coloneqq \lim_{n \to \infty} \int_X f_n(x)\ \mu(dx)
						\label{eq:complex_measures_integration_measurable_function}
					\end{align}
					で定める.
					
			\end{description}
			\label{dfn:complex_measures_integration}
		\end{dfn}
	\end{screen}
	
	\footnotetext{
		$A_1,\cdots,A_k$は互いに素であり$X = \sum_{i=1}^{k} A_i$を満たす.
	}
	
	\begin{screen}
		\begin{thm}[積分の定義はwell-defined]
			定義\ref{dfn:complex_measures_integration}において,
			(\refeq{eq:complex_measures_integration_simple_function})
			は(\refeq{eq:expression_simple_function})の表示の仕方に依らずに定まり,
			(\refeq{eq:complex_measures_integration_measurable_function})
			も(\refeq{eq:integration_simple_function_approximation})を満たす単関数近似列の選び方に
			依らずに定まる.更に任意の$f \in MF$に対して次が成り立つ:
			\begin{align}
				\left| \int_X f(x)\ \mu(dx) \right| \leq \int_X |f(x)|\ |\mu|(dx).
				\label{eq:thm_complex_measure_integration_well_defined_2}
			\end{align}
			\label{thm:complex_measure_integration_well_defined}
		\end{thm}
	\end{screen}
	
	\begin{prf}\mbox{}
		\begin{description}
			\item[$f$が可測単関数の場合]
				$f$が(\refeq{eq:expression_simple_function})の表示とは別に
				\begin{align}
					f = \sum_{j=1}^{m} \beta_j \defunc_{B_j}
					\quad (\beta_j \in \C,\ B_j \in \mathcal{M},\ \mbox{$X=\sum_{j=1}^{m} B_j$})
				\end{align}
				と表現できるとしても
				\begin{align}
					\sum_{i=1}^{k} \alpha_i \mu(A_i)
					= \sum_{i=1}^{k} \sum_{j=1}^{m} \alpha_i \mu(A_i \cap B_j)
					= \sum_{j=1}^{m} \sum_{i=1}^{k} \beta_j \mu(A_i \cap B_j)
					= \sum_{j=1}^{m} \beta_j \mu(B_j)
				\end{align}
				が成り立つ.また(\refeq{radon_nikodym_3})より
				\begin{align}
					\left| \int_X f(x)\ \mu(dx) \right|
					= \left| \sum_{i=1}^{k} \alpha_i \mu(A_i) \right|
					\leq \sum_{i=1}^{k} \left| \alpha_i \right| |\mu|(A_i)
					= \int_X |f(x)|\ |\mu|(dx)
					\label{eq:thm_complex_measure_integration_well_defined}
				\end{align}
				も成り立つ.
				
			\item[$f$が一般の可測関数の場合]
				(\refeq{eq:complex_measures_integration_measurable_function})
				は有限確定している.実際
				(\refeq{eq:integration_simple_function_approximation})を
				満たす単関数近似列$(f_n)_{n=1}^{\infty}$に対して
				(\refeq{eq:thm_complex_measure_integration_well_defined})より
				\begin{align}
					\left| \int_X f_n(x)\ \mu(dx) - \int_X f_m(x)\ \mu(dx) \right|
					\leq \int_X \left| f_n(x) - f_m(x) \right|\ |\mu|(dx)
					\quad (\forall n,m \in \N)
				\end{align}
				が成り立つから,$\left( \int_X f_n(x)\ \mu(dx) \right)_{n=1}^{\infty}$
				は$\C$においてCauchy列をなし極限が存在する.
				$(f_n)_{n=1}^{\infty}$とは別に(\refeq{eq:integration_simple_function_approximation})
				を満たす$f$の単関数近似列$(g_n)_{n=1}^{\infty}$が存在しても
				\begin{align}
					&\left| \int_X f_n(x)\ \mu(dx) - \int_X g_m(x)\ \mu(dx) \right|
					\leq \int_X \left| f_n(x) - g_m(x) \right|\ |\mu|(dx) \\
					&\qquad \leq \int_X \left| f_n(x) - f(x) \right|\ |\mu|(dx)
						+ \int_X \left| f(x) - g_m(x) \right|\ |\mu|(dx)
					\longrightarrow 0 \quad (n,m \longrightarrow \infty)
				\end{align}
				が成り立つから,
				\begin{align}
					\alpha \coloneqq \lim_{n \to \infty} \int_X f_n(x)\ \mu(dx),
					\quad \beta \coloneqq \lim_{n \to \infty} \int_X g_n(x)\ \mu(dx)
				\end{align}
				とおけば
				\begin{align}
					|\alpha - \beta|
					&\leq \left| \alpha - \int_X f_n(x)\ \mu(dx) \right|
						+ \left| \int_X f_n(x)\ \mu(dx) - \int_X g_m(x)\ \mu(dx) \right|
						+ \left| \int_X g_m(x)\ \mu(dx) - \beta \right| \\
					&\longrightarrow 0 \quad (n,m \longrightarrow \infty)
				\end{align}
				が従い$\alpha = \beta$を得る.また(\refeq{eq:thm_complex_measure_integration_well_defined})より
				\begin{align}
					\left| \int_X f_n(x)\ \mu(dx) \right| \leq \int_X |f_n(x)|\ |\mu|(dx) \quad (n = 1,2,\cdots)
				\end{align}
				が満たされているから,両辺で$n \longrightarrow \infty$として(\refeq{eq:thm_complex_measure_integration_well_defined_2})を得る.
				\QED
		\end{description}
	\end{prf}
	
	定義\ref{dfn:complex_measures_integration}において,
	(\refeq{eq:complex_measures_integration_measurable_function})は
	(\refeq{eq:complex_measures_integration_simple_function})の拡張となっている.
	実際$f$が可測単関数の場合,(\refeq{eq:integration_simple_function_approximation})を満たす単関数近似列
	として$f$自身を選べばよい.定理\ref{thm:complex_measure_integration_well_defined}より
	(\refeq{eq:complex_measures_integration_measurable_function})による$f$の積分は一意に確定し
	(\refeq{eq:complex_measures_integration_simple_function})の左辺に一致する.
	
	\begin{screen}
		\begin{thm}[積分の線型性]
			定義\ref{dfn:complex_measures_integration}で定めた積分について,
			任意の$f,g \in \mathscr{L}^1(X,\mathcal{M},|\mu|)$と
			$\alpha,\beta \in \C$に対し
			\begin{align}
				\int_X \alpha f(x) + \beta g(x)\ \mu(dx)
				= \alpha \int_X f(x)\ \mu(dx) + \beta \int_X g(x)\ \mu(dx)
			\end{align}
			が成り立つ.
			\label{thm:complex_measure_integral_linearity}
		\end{thm}
	\end{screen}
	
	\begin{prf}\mbox{}
		\begin{description}
			\item[第一段]
				$f,g$が可測単関数の場合,(\refeq{eq:complex_measures_integration_simple_function})で定める積分が線型性を持つことを示す.
				$u_1,\cdots,u_k,v_1,\cdots,v_r \in \C$と$A_1,\cdots,A_k,B_1,\cdots,B_r \in \mathcal{M}\ (X = \sum_{i=1}^{k} A_i = \sum_{j=1}^{r} B_j)$によって
				\begin{align}
					f = \sum_{i=1}^{k} u_i \defunc_{A_i},
					\quad g = \sum_{j=1}^{r} v_j \defunc_{B_j}
				\end{align}
				と表示されているとき,
				\begin{align}
					\alpha f + \beta g = \sum_{i=1}^{k} \sum_{j=1}^{r} (\alpha u_i + \beta v_j) \defunc_{A_i \cap B_j}
				\end{align}
				と表現できるから
				\begin{align}
					&\int_X \alpha f(x) + \beta g(x)\ \mu(dx) 
					= \sum_{i=1}^{k} \sum_{j=1}^{r} (\alpha u_i + \beta v_j) \mu(A_i \cap B_j) \\
					&\qquad = \alpha \sum_{i=1}^{k} u_i \mu(A_i) + \beta \sum_{j=1}^{r} v_j \mu(B_j)
					= \alpha \int_X f(x)\ \mu(dx) + \beta \int_X g(x)\ \mu(dx)
				\end{align}
				が成り立つ.
			\item[第二段]
				$f,g$を一般の可測関数とし,$f,g$それぞれについて(\refeq{eq:integration_simple_function_approximation})を満たす
				単関数近似列$(f_n)_{n=1}^{\infty},(g_n)_{n=1}^{\infty}$を一つ選ぶ.
				\begin{align}
					&\int_X \left|(\alpha f_n(x) + \beta g_n(x)) - (\alpha f(x) + \beta g(x)) \right|\ |\mu|(dx) \\
					&\qquad \leq |\alpha| \int_X \left| f_n(x) - f(x) \right|\ |\mu|(dx)
						+ |\beta| \int_X \left| g_n(x) - g(x) \right|\ |\mu|(dx)
					\longrightarrow 0 \quad (n \longrightarrow \infty)
				\end{align}
				が成り立つから$\alpha f + \beta g$の$\mu$に関する積分は
				\begin{align}
					\int_X \alpha f(x) + \beta g(x)\ \mu(dx) \coloneqq \lim_{n \to \infty} \int_X \alpha f_n(x) + \beta g_n(x)\ \mu(dx)
				\end{align}
				で定義され,前段の結果より
				\begin{align}
					&\left| \int_X \alpha f(x) + \beta g(x)\ \mu(dx) - \alpha \int_X f(x)\ \mu(dx) - \beta \int_X g(x)\ \mu(dx) \right| \\
					&\quad\leq \left| \int_X \alpha f(x) + \beta g(x)\ \mu(dx) - \int_X \alpha f_n(x) + \beta g_n(x)\ \mu(dx) \right| \\
						&\qquad+ \left| \alpha \int_X f_n(x)\ \mu(dx) + \beta \int_X g_n(x)\ \mu(dx) - \alpha \int_X f(x)\ \mu(dx) - \beta \int_X g(x)\ \mu(dx) \right| \\
					&\longrightarrow 0 \quad (n \longrightarrow \infty)
				\end{align}
				が従う.
				\QED
		\end{description}
	\end{prf}
	
	\begin{screen}
		\begin{thm}[積分の測度に関する線型性]
			$(X,\mathcal{M})$を可測空間,$\mu,\nu$をこの上の複素測度とする.$f:X \rightarrow \C$が$|\mu|$と$|\nu|$について可積分であるなら,
			$\alpha,\beta \in \C$に対し$|\alpha \mu|, |\beta \nu|, |\alpha \mu + \beta \nu|$についても可積分であり,更に次が成り立つ:
			\begin{align}
				\int_X f(x)\ (\alpha\mu + \beta\nu)(dx) = \alpha \int_X f(x)\ \mu(dx) + \beta \int_X f(x)\ \nu(dx).
			\end{align}
			\label{thm:linearity_of_integral_respect_to_measure}
		\end{thm}
	\end{screen}
	
	\begin{prf}
		\begin{description}
			\item[第一段]
				$f$が可測単関数の場合について証明する.
				$a_i \in \C,\ A_i \in \mathcal{M}\ (i=1,\cdots,n,\ \sum_{i=1}^{n} A_i = X)$を用いて
				\begin{align}
					f = \sum_{i=1}^{n} a_i \defunc_{A_i}
				\end{align}
				と表されている場合,
				\begin{align}
					&\int_X f(x)\ (\alpha\mu + \beta\nu)(dx)
					= \sum_{i=1}^{n} a_i (\alpha\mu + \beta\nu)(A_i) \\
					&\qquad = \alpha \sum_{i=1}^{n} a_i \mu(A_i) + \beta \sum_{i=1}^{n} a_i \nu(A_i)
					= \alpha \int_X f(x)\ \mu(dx) + \beta \int_X f(x)\ \nu(dx)
				\end{align}
				が成り立つ.
				
			\item[第二段]
			$f$が一般の可測関数の場合について証明する.任意の$A \in \mathcal{M}$に対して
			\begin{align}
				\left| (\alpha \mu + \beta \nu)(A) \right| \leq |\alpha||\mu(A)| + |\beta||\nu(A)| \leq |\alpha||\mu|(A) + |\beta||\nu|(A)
 			\end{align}
 			が成り立つから,左辺で$A$を任意に分割しても右辺との大小関係は変わらず
 			\begin{align}
 				|\alpha \mu + \beta \nu|(A) \leq |\alpha||\mu|(A) + |\beta||\nu|(A)
 			\end{align}
 			となる.従って$f$が$|\mu|$と$|\nu|$について可積分であるなら
 			\begin{align}
 				\int_X |f(x)|\ |\alpha \mu + \beta \nu|(dx) \leq |\alpha| \int_X |f(x)|\ |\mu|(dx) + |\beta| \int_X |f(x)|\ |\nu|(dx) < \infty
 			\end{align}
 			が成り立ち前半の主張を得る.$f$の単関数近似列$(f_n)_{n=1}^{\infty}$を取れば,前段の結果と積分の定義より
 			\begin{align}
 				&\left| \int_X f(x)\ (\alpha\mu + \beta\nu)(dx) - \alpha \int_X f(x)\ \mu(dx) - \beta \int_X f(x)\ \nu(dx) \right| \\
 					&\qquad \leq \left| \int_X f(x)\ (\alpha\mu + \beta\nu)(dx) - \int_X f_n(x)\ (\alpha\mu + \beta\nu)(dx) \right| \\
 					&\qquad \quad + |\alpha| \left| \int_X f(x)\ \mu(dx) - \int_X f_n(x)\ \mu(dx) \right|
 					+ |\beta| \left| \int_X f(x)\ \nu(dx) - \int_X f_n(x)\ \nu(dx) \right| \\
 				&\qquad \longrightarrow 0 \quad (n \longrightarrow \infty)
 			\end{align}
 			が成り立ち後半の主張が従う.
 			\QED
		\end{description}
	\end{prf}
	
	\begin{screen}
		\begin{thm}[収束定理]
			$(X,\mathcal{M})$を可測空間,$\mu$をこの上の複素測度とする.$\mathcal{M}/\borel{C}$-可測関数列$(f_n)_{n=1}^{\infty}$
			が各点で収束し,かつ或る$g \in \mathscr{L}^1(X,\mathcal{M},|\mu|)$が存在して$|f_n| \leq |g|\ (n=1,2,\cdots)$
			を満たすとき,次が成り立つ:
			\begin{align}
				\int_X\lim_{n \to \infty} f_n(x)\ \mu(dx) = \lim_{n \to \infty} \int_X f_n(x)\ \mu(dx).
			\end{align}
			\label{eq:lebesgue_convergence_theorem_complex_measure}
		\end{thm}
	\end{screen}
	
	\begin{prf}
		\begin{align}
			f(x) \coloneqq \lim_{n \to \infty} f_n(x) \quad (\forall x \in X)
		\end{align}
		とおく.Lebesgueの収束定理より$f \in \mathscr{L}^1(X,\mathcal{M},|\mu|)$かつ
		\begin{align}
			\int_X f(x)\ |\mu|(dx) = \lim_{n \to \infty} \int_X f_n(x)\ |\mu|(dx)
		\end{align}
		が成り立つから,定理\ref{thm:complex_measure_integral_linearity}及び定理\ref{thm:complex_measure_integration_well_defined}より
		\begin{align}
			\left| \int_X f(x)\ \mu(dx) - \int_X f_n(x)\ \mu(dx) \right| \leq \int_X \left| f(x) - f_n(x) \right|\ |\mu|(dx) \longrightarrow 0
			\quad (n \longrightarrow \infty)
		\end{align}
		が従う.
		\QED
	\end{prf}
	
	\begin{screen}
		\begin{thm}[順序交換定理]
		\end{thm}
	\end{screen}
	
	\begin{prf}
		$|\mu| \times |\nu| \leq |\mu \times \nu|$より$|\mu|,|\nu|,|\mu \times \nu|$にFubiniの定理を適用.
		\begin{align}
			\int_X \int_Y f_n(x,y)\ \nu(dy)\ \mu(dx) = \int_{X \times Y} f_n(x,y)\ (\mu \times \nu)(dx \times dy)
			= \int_Y \int_X f_n(x,y)\ \mu(dy)\ \nu(dx)
		\end{align}
		$f$が$|\mu \times \nu|$に関して可積分なら
		\begin{align}
			\int_{X \times Y} f(x,y)\ (\mu \times \nu)(dx \times dy)
		\end{align}
		が定義され,更に
		\begin{align}
			\int_Y |f(x,y)|\ |\nu|(dy) < \infty
		\end{align}
		だから
		\begin{align}
			\int_Y f(x,y)\ \nu(dy)
		\end{align}
		も定義される.
		\begin{align}
			\left| \int_Y f(x,y)\ \nu(dy) \right| \leq \int_Y |f(x,y)|\ |\nu|(dy)
		\end{align}
		が$|\mu|$について可積分であるから
		\begin{align}
			\int_X \int_Y f(x,y)\ \nu(dy)\ \mu(dx)
		\end{align}
		も定義される.
	\end{prf}
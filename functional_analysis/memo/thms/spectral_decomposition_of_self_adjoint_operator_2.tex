	\begin{screen}
		\begin{lem}[$\Dom{T_f} $は線型・稠密]
			(\refeq{eq:dfn_operator_introduced_by_measurable_functions_3})で定めた$\Dom{T_f} $は$H$の線型部分空間で$\closure{\Dom{T_f} }=H$を満たす.
			\label{lem:domain_T_f_linear_dense}
		\end{lem}
	\end{screen}
	
	\begin{prf}\mbox{}
		\begin{description}
			\item[線型性]
				$u,v \in \Dom{T_f} $に対して
				\begin{align}
					\int_X |f(x)|^2\ \mu_u(dx) < \infty,\quad \int_X |f(x)|^2\ \mu_v(dx) < \infty
				\end{align}
				が満たされている.(\refeq{eq:lem_complex_measure_introduced_by_spectral_measure_2})より任意の$\Lambda \in \mathcal{M}$に対して
				\begin{align}
					\mu_{u+v}(\Lambda) = \Norm{E(\Lambda)(u+v)}{}^2 \leq 2 \Norm{E(\Lambda)u}{}^2 + 2 \Norm{E(\Lambda)v}{}^2 = 2 \mu_u(\Lambda) + 2 \mu_v(\Lambda)
				\end{align}
				が成り立つから
				\begin{align}
					\int_X |f(x)|^2\ \mu_{u+v}(dx) \leq 2 \int_X |f(x)|^2\ \mu_u(dx) + 2 \int_X |f(x)|^2\ \mu_v(dx) < \infty 
				\end{align}
				が従い$u+v \in \Dom{T_f} $を得る.また任意に$\lambda \in \C$を取れば
				\begin{align}
					\mu_{\lambda u}(\Lambda) = \Norm{\lambda E(\Lambda)u}{}^2 = |\lambda|^2 \mu_{u}(\Lambda)
				\end{align}
				が成り立ち$\lambda u \in \Dom{T_f} $も従う.
				
			\item[稠密性]
				任意に$u \in H$を取る.
				\begin{align}
					A_k \coloneqq \Set{x \in X}{|f(x)| \leq k} \quad (k=1,2,\cdots)
				\end{align}
				に対して$u_k \coloneqq E(A_k)u$とおけば,$(A_k)_{k=1}^{\infty}$は単調に増加し$X$に収束するから
				\begin{align}
					\Norm{u - u_k}{} = \Norm{E(X)u - E(A_k)u}{} \longrightarrow 0 \quad (k \longrightarrow \infty)
					\label{eq:lem_domain_T_f_linear_dense}
				\end{align}
				が成り立つ.一方で任意の$\Lambda \in \mathcal{M}$に対して,
				命題\ref{lem:product_of_spectral_measure}と(\refeq{eq:lem_complex_measure_introduced_by_spectral_measure_2})より
				\begin{align}
					\mu_{u_k}(\Lambda) = \inprod<E(\Lambda)E(A_k)u, E(A_k)u> = \inprod<E(\Lambda \cap A_k)u,u> = \mu_u(\Lambda \cap A_k)
				\end{align}
				と表せるから$\mu_{u_k}$は$A_k$に集中している.よって
				\begin{align}
					\int_X |f(x)|^2\ \mu_{u_k}(dx) = \int_{A_k} |f(x)|^2\ \mu_{u_k}(dx) \leq k^2 \mu_u(A_k) < \infty
				\end{align}
				が成り立ち$u_k \in \Dom{T_f} $が従い,(\refeq{eq:lem_domain_T_f_linear_dense})より主張を得る.
				\QED
		\end{description}
	\end{prf}
	
	\begin{screen}
		\begin{thm}[$T_f$の定義域は0ではない]
				任意の$f \in MF$に対し$\Dom{T_f} \neq \{0\}$が成り立つ.
		\end{thm}
	\end{screen}
	
	\begin{prf}
		Hausdorff位相空間において一点集合は閉だから,$\Dom{T_f} = \{0\}$なら$H = \{0\}$が従い本章の仮定に反する.
		\QED
	\end{prf}
	
	\begin{screen}
		\begin{thm}[$T$の性質]
			$f,g \in MF$とする.
			\begin{description}
				\item[(1)] $T_f$は$H$から$H$への線型作用素である.
				\item[(2)] $u \in \Dom{T_f} ,\ v \in \Dom{T_g} $ならば次が成り立つ:
					\begin{align}
						\int_X \left| f(x) \conj{g(x)} \right|\ |\mu_{u,v}|(dx) \leq \Norm{f}{\Lp{2}{\mu_u}} \Norm{g}{\Lp{2}{\mu_v}}, \quad 
						\int_X f(x) \conj{g(x)}\ \mu_{u,v}(dx) = \inprod<T_f u, T_g v>.
					\end{align}
				\item[(3)] $T_f + T_g \subset T_{f+g}$が成り立ち,特に$g$が有界なら等号が成立する.
				\item[(4)] $T_f T_g \subset T_{fg}$が成り立ち,特に$g$が有界なら等号が成立する.
				\item[(5)] $T_f^* = T_{\conj{f}}$が成り立つ.特に$T_f$は閉作用素であり,また$f$が$\R$値なら$T_f$は自己共役である.
				\item[(6)] $\lambda \in \C$が$\lambda \neq 0$なら$T_{\lambda f} = \lambda T_f$が成り立つ.
			\end{description}
			\label{thm:properties_of_T_f}
		\end{thm}
	\end{screen}
	
	\begin{prf}\mbox{}
		\begin{description}
			\item[(1)]	補題\ref{lem:domain_T_f_linear_dense}より$T_f$の定義域は線形空間であるから,後は$T_f$が線型演算を満たすことを示せばよい.
					$f$の$MSF$-近似列$(f_n)_{n=1}^{\infty}$を取れば,定義式(\refeq{eq:dfn_operator_introduced_by_measurable_functions})より$T_{f_n}$は線型作用素であるから
					\begin{align}
						\Norm{T_f (\alpha u + \beta v) - \alpha T_f u - \beta T_f v}{}
						&\leq \Norm{T_f (\alpha u + \beta v) - T_{f_n} (\alpha u + \beta v)}{}
							+ |\alpha| \Norm{T_f u - T_{f_n} u}{} + |\beta| \Norm{T_f v - T_{f_n} v}{} \\
						&\longrightarrow 0 \quad (n \longrightarrow \infty)
					\end{align}
					が成り立つ.
					
			\item[(2)] $f,g \in MSF$のとき,任意の$u,v \in H$に対して
				\begin{align}
					\int_X \left| f(x) \conj{g(x)} \right|\ |\mu_{u,v}|(dx) \leq \Norm{f}{\Lp{2}{\mu_u}} \Norm{g}{\Lp{2}{\mu_v}},
					\quad \inprod<T_f u, T_g v> = \int_X f(x) \conj{g(x)}\ \mu_{u,v}(dx)
				\end{align}
				が成り立つ.第二式は補題(\ref{lem:MSF_properties_of_T_f})による.第一式について,
				\begin{align}
					f = \sum_{i=1}^{n} \alpha_i \defunc_{A_i},\quad 
					g = \sum_{i=1}^{n} \beta_i \defunc_{A_i}
				\end{align}
				と表示されているとして
				\begin{align}
					\int_X \left| f(x) \conj{g(x)} \right|\ |\mu_{u,v}|(dx)
					= \sum_{i=1}^{n} |\alpha_i||\beta_i| |\mu_{u,v}|(A_i)
					\leq \sum_{i=1}^{n} |\alpha_i||\beta_i| \mu_u(A_i)^{\frac{1}{2}} \mu_v(A_i)^{\frac{1}{2}}
					\leq \left( \int_X \left| f(x) \right|^2\ \mu_u(dx) \right)^{\frac{1}{2}} \left( \int_X \left| g(x) \right|^2\ \mu_v(dx) \right)^{\frac{1}{2}}
				\end{align}
				が成り立つ.
				一般の$f,g \in MF$については,$MSF$-近似列とFatouの補題より従う.
				
			\item[(3)]
				$\Dom{T_f + T_g} = \Dom{T_f} \cap \Dom{T_g} $であるから,任意の$u \in \Dom{T_f + T_g} $に対して
				\begin{align}
					\int_X |f(x)|^2\ \mu_u(dx) < \infty,
					\quad \int_X |g(x)|^2\ \mu_u(dx) < \infty
				\end{align}
				が満たされ
				\begin{align}
					\int_X |f(x) + g(x)|^2\ \mu_u(dx) \leq 
					2 \int_X |f(x)|^2\ \mu_u(dx) + 2 \int_X |g(x)|^2\ \mu_u(dx) < \infty
				\end{align}
				が従い$u \in \Dom{T_{f+g}} $が成り立つ.また任意の$u \in \Dom{T_f + T_g} $に対して,内積を展開し(2)の結果を適用すれば
				\begin{align}
					\Norm{T_{f+g}u - T_f u - T_g u}{}^2
					&= \int_X |f+g|^2\ d\mu_u + \int_X |f|^2\ d\mu_u + \int_X |g|^2\ \mu_u \\
						&\qquad - 2 \int_X \Re{(f+g)f}\ d\mu_u - 2 \int_X \Re{(f+g)g}\ d\mu_u + 2 \int_X \Re{fg}\ d\mu_u
					= 0 
				\end{align}
				が成り立ち$T_f + T_g \subset T_{f+g}$が従う.$g$が有界な場合,補題\ref{lem:complex_measure_introduced_by_spectral_measure}より
				全ての$u \in H$に対して$\mu_u$が有限測度であるから,$\Dom{T_g} $は$H$に一致し$\Dom{T_f + T_g} = \Dom{T_f} $が成り立つ.
				また任意の$u \in \Dom{T_{f+g}} $に対して
				\begin{align}
					\int_X |f(x)|^2\ \mu_u(dx) \leq 2 \int_X |f(x)+g(x)|^2\ \mu_u(dx) + 2 \int_X |g(x)|^2\ \mu_u(dx) < \infty
				\end{align}
				となり$u \in \Dom{T_f + T_g} $が従うから,前半の結果と併せて$T_f + T_g = T_{f+g}$が得られる.
			
			\item[(4)]
			
			\item[(5)]
				補題\ref{lem:domain_T_f_linear_dense}より$\Dom{T_f} $が$H$で稠密であるから$T_f^*$が定義される.
				(2)の結果より
				\begin{align}
					\inprod<T_f u,v> = \int_X f(x)\ \mu_{u,v}(dx) = \inprod<u, T_{\conj{f}} v>
					\quad \left( \forall u,v \in \Dom{T_f} = \Dom{T_{\conj{f}}} \right)
					\label{eq:thm_properties_of_T_f_1}
				\end{align}
				が成り立ち,先ず$T_{\conj{f}} \subset T_f^*$が従う.
				後は$\Dom{T_f^*} = \Dom{T_{\conj{f}}} $が成り立つことを示せばよい.
				\begin{align}
					A_k \coloneqq \Set{x \in X}{|f(x)| \leq k}
					\quad (k=1,2,\cdots)
				\end{align}
				とおいて,任意に$v \in \Dom{T_f^*} $を取り
				\begin{align}
					v_k \coloneqq T_{\conj{f}\defunc_{A_k}} v
					\quad (k=1,2,\cdots)
				\end{align}
				とすれば,各$k \in \N$について
				\begin{align}
					\Norm{T_f v_k}{} = \Norm{T_f T_{\conj{f}\defunc_{A_k}} v}{}^2 = \int_{A_k} |f(x)|^4\ \mu_v(dx) < k^4 \mu_v(A_k) < \infty
				\end{align}
				が成り立つから$v_k \in \Dom{T_f} $である.(\refeq{eq:thm_properties_of_T_f_1})と同様にすれば
				\begin{align}
					\Norm{v_k}{}^2 = \inprod<T_{\conj{f}\defunc_{A_k}} v, T_{\conj{f}\defunc_{A_k}} v>
						= \inprod<T_{f\defunc_{A_k}} T_{\conj{f}\defunc_{A_k}} v, v>
						= \inprod<T_f v_k, v>
						= \inprod<v_k, T_f^* v>
				\end{align}
				となり,Schwartzの不等式より
				\begin{align}
					\Norm{v_k}{} \leq \Norm{T_f^* v}{}
					\label{eq:thm_properties_of_T_f_2}
				\end{align}
				が得られる.一方で
				\begin{align}
					\Norm{v_k}{}^2 = \Norm{T_{\conj{f}\defunc_{A_k}} v}{}^2 = \int_{A_k} |f(x)|^2\ \mu_v(dx)
				\end{align}
				が成り立つから,(\refeq{eq:thm_properties_of_T_f_2})と併せて
				\begin{align}
					\int_{A_k} |f(x)|^2\ \mu_v(dx) \leq \Norm{T_f^* v}{}^2
				\end{align}
				が従う.$(A_k)_{k=1}^{\infty}$は単調増大列で$\cup_{k=1}^{\infty} A_k = X$を満たすから,単調収束定理より
				\begin{align}
					\int_X |f(x)|^2\ \mu_v(dx) \leq \Norm{T_f^* v}{}^2
				\end{align}
				となり$v \in \Dom{T_f} $が得られる.
				特に$T_f = T_{\conj{f}}^*$が従い,共役作用素が閉線型であるから$T_f$も閉作用素である.
				
			\item[(6)]
				$\lambda = 0$の場合は,$\Dom{T_{\lambda f}} = \Dom{T_0} = H$であるが$\Dom{T_f} = H$とは限らないから主張が従わない.
				$\lambda \neq 0$の場合
				\begin{align}
					\int_X |\lambda f(x)|^2\ \mu_u(dx) < \infty \quad \Leftrightarrow \quad
					\int_X |f(x)|^2\ \mu_u(dx) < \infty
				\end{align}
				が成り立つから$\Dom{T_{\lambda f}} = \Dom{T_f} = \Dom{\lambda T_f} $である.また$f$の$MSF$-近似列$(f_n)_{n=1}^{\infty}$
				については補題\ref{lem:MSF_properties_of_T_f}より
				\begin{align}
					T_{\lambda f_n} u = \lambda T_{f_n} u \quad \left( u \in \Dom{T_{\lambda f}} \right)
				\end{align}
				が満たされているから,任意の$u \in \Dom{T_{\lambda f}} $に対して
				\begin{align}
					\Norm{T_{\lambda f}u - \lambda T_f u}{}
					\leq \Norm{T_{\lambda f}u - T_{\lambda f_n} u}{} + |\lambda| \Norm{T_f u - T_{f_n} u}{}
					\longrightarrow 0 \quad (n \longrightarrow \infty)
				\end{align}
				が従う.
				\QED
		\end{description}
	\end{prf}
	
	\begin{screen}
		\begin{cor}
			$f,g \in MF$とする.
			\begin{description}
				\item[(1)] $T_f = T_g$であることと$E\left( \Set{x \in X}{f(x) \neq g(x)} \right) = 0\ $(零写像)であることは同値である.
				\item[(2)] $f$が有界ならば$T_f \in \selfBop{H} $であり$\Norm{T_f}{\selfBop{H}} \leq \sup{x \in X}{|f(x)|}$が成り立つ.
				\item[(3)] 或る$L > 0$に対し$E\left( \Set{x \in X}{|f(x)| > L} \right) = 0$が成り立つとき,$T_f \in \selfBop{H} $であり次が成り立つ:
					\begin{align}
						\Norm{T_f}{\selfBop{H}} = \inf{}{\Set{L > 0}{E\left( \Set{x \in X}{|f(x)| > L} \right) = 0}}.
					\end{align}
				\item[(4)] $\lambda \in \C,\epsilon > 0$に対し$U_\epsilon(\lambda) \coloneqq \Set{z \in \C}{|z-\lambda| < \epsilon}$とおく.
					$T_f$のレゾルベント集合\footnotemark は
					\begin{align}
						\Res{T_f} = \Set{\lambda \in \C}{\mbox{或る$\epsilon > 0$が存在して$E\left(f^{-1}(U_\epsilon(\lambda))\right) = 0$を満たす.}}
					\end{align}
					で与えられ,さらに$\lambda \in \Res{T_f} $に対して$\epsilon > 0$が$E\left(f^{-1}(U_\epsilon(\lambda))\right) = 0$を満たすとすれば
					\begin{align}
						\left( \lambda I - T_f \right)^{-1} = T_{\frac{1}{\lambda - f} \defunc_{X \backslash f^{-1}(U_\epsilon(\lambda))}}
					\end{align}
					が成り立つ.
			\end{description}
			\label{cor:properties_of_T_f}
		\end{cor}
	\end{screen}
	
	\footnotetext{
		定理\ref{thm:properties_of_T_f}より$T_f$は閉作用素であるからレゾルベントを考察できる.
	}
	
	\begin{prf}\mbox{}
		\begin{description}
			\item[(1)] 今$N \coloneqq \Set{x \in X}{f(x) \neq g(x)}$とおく.
				$T_f = T_g$が成り立っているとすると,$u \in \Dom{T_f} $に対し
				\begin{align}
					0 = \Norm{T_f u - T_g u}{}^2 = \Norm{T_{f-g} u}{}^2 = \int_X |f(x) - g(x)|^2\ \mu_u(dx)
					\label{eq:cor_properties_of_T_f_1}
				\end{align}
				が従い$\mu_u(N) = \Norm{E(N)u}{}^2 = 0$となる.$\Dom{T_f} $の稠密性と直交射影$E(N)$の連続性より$E(N) = 0$を得る.
				逆に$E(N) = 0$の場合,任意の$u \in \Dom{T_f} $に対して$\mu_u(N) = \Norm{E(N)u}{}^2 = 0$が成り立つから
				\begin{align}
					\int_X |g(x)|^2\ \mu_u(dx) \leq 2 \int_X |f(x) - g(x)|^2\ \mu_u(dx) + 2 \int_X |f(x)|^2\ \mu_u(dx) = 2 \int_X |f(x)|^2\ \mu_u(dx) < \infty
				\end{align}
				となり,(\refeq{eq:cor_properties_of_T_f_1})と併せて$T_f \subset T_g$が従う.同様に$T_g \subset T_f$も成り立つから$T_f = T_g$を得る.	
				
			\item[(3)]
				$E\left( \Set{x \in X}{|f(x)| > L} \right) = 0$を満たす$L > 0$に対し
				\begin{align}
					A_L \coloneqq \Set{x \in X}{|f(x)| \leq L}
				\end{align}
				とおけば,任意の$u \in H$に対し
				\begin{align}
					\mu_u(\Lambda) = \inprod<E(\Lambda)u,u> = \inprod<E(\Lambda \cap A_L)u,u> = \mu_u(\Lambda \cap A_L)
				\end{align}
				が成り立つから$\mu_u$は$A_L$に集中している.従って定理\ref{thm:properties_of_T_f}(2)と
				$\mu_u$の定義(\refeq{eq:lem_complex_measure_introduced_by_spectral_measure_1})より
				\begin{align}
					\Norm{T_f u}{}^2 = \int_X |f(x)|^2\ \mu_u(dx) = \int_{A_L} |f(x)|^2\ \mu_u(dx) \leq L^2 \mu_u(X) = L^2 \Norm{u}{}^2 < \infty
				\end{align}
				となるから,$\Dom{T_f} = H$且つ$\Norm{T_f}{\selfBop{H}} \leq L$を得る.これにより$T_f \in \selfBop{H} $と
				\begin{align}
					\Norm{T_f}{\selfBop{H}} \leq \inf{}{\Set{L > 0}{E\left( \Set{x \in X}{|f(x)| > L} \right) = 0}}
				\end{align}
				が成り立つ.ここで$\Norm{T_f}{\selfBop{H}} < \inf{}{\Set{L > 0}{E\left( \Set{x \in X}{|f(x)| > L} \right) = 0}}$が成り立つとすると
				
			\item[(4)] $\lambda \in \C$を固定する.任意の$\epsilon > 0$に対し$V_\epsilon \coloneqq f^{-1}(U_\epsilon(\lambda))$とおけば
				$f$の可測性から$V_\epsilon \in \mathcal{M}$であり,また
				\begin{align}
					x \in V_\epsilon \quad \Leftrightarrow \quad |\lambda - f(x)| < \epsilon
				\end{align}
				が成り立つから,$X \backslash V_\epsilon$上で$1/(\lambda - f) \leq 1/\epsilon$が満たされる.
				\begin{description}
					\item[第一段] $E(V_\epsilon) = 0$を満たす$\epsilon$が存在しない場合,
				\end{description}
		\end{description}
	\end{prf}
	
	\begin{screen}
		\begin{cor}
			$(X,\mathcal{M}) = \left( \R^d,\borel{\R^d} \right)$の場合,
			\begin{align}
				\supp{E} \coloneqq \Set{x \in \R^d}{\mbox{$x$の任意の開近傍$V$に対して$E(V)=0$が成り立つ.}}
			\end{align}
			として$E$の台を定める.このとき任意の連続写像$f:\R^d \rightarrow \C$について
			\begin{align}
				\Spctr{T_f} = \closure{f(\supp{E})}
			\end{align}
			が成り立つ.特に$\supp{E}$がコンパクトなら$\Spctr{T_f} = f(\supp{E})$となる.
		\end{cor}
	\end{screen}
	
	\begin{prf}
		任意に$x \in \supp{E}$を取る.$f(x)$の任意の$\epsilon$近傍$U_\epsilon = U_\epsilon(f(x))$に対し,
		$f$の連続性から$f^{-1}(U_\epsilon)$は$x$の開近傍となるから,
		$E(f^{-1}(U_\epsilon)) = 0$が成り立ち$f(x) \in \Spctr{T_f}$が従う.$\Spctr{T_f}$は閉集合であるから
		$\closure{f(\supp{E})} \subset \Spctr{T_f}$を得る.
		逆に任意に$\lambda \in \closure{f(\supp{E})}$を取れば,或る$\epsilon > 0$が存在して
		$U_\epsilon(\lambda) \cap \closure{f(\supp{E})} = \emptyset$を満たすから
		$f^{-1}(U_\epsilon(\lambda)) \cap \supp{E} = \emptyset$が成り立つ.
		$f^{-1}(U_\epsilon(\lambda))$に属する$\R^d$の有理点全体を$\Q_f$と表せば,
		各$r \in \Q_f$に対し或る開近傍$V_r$が存在して$E(V_r) = 0$を満たすから
		$E\left( V_r \cap f^{-1}(U_\epsilon(\lambda)) \right) = 0\ (\forall r \in \Q_f)$が従う.
		$\Q_f$は可付番だから添数を変えれば
		\begin{align}
			f^{-1}(U_\epsilon(\lambda)) = \bigcup_{n \in \N} V_n \cap f^{-1}(U_\epsilon(\lambda))
		\end{align}
		と表され
		\footnote{
			もし或る$x \in f^{-1}(U_\epsilon(\lambda))$が$x \notin \bigcup_{n \in \N} V_n \cap f^{-1}(U_\epsilon(\lambda))$
			を満たすとすれば,$x$に近づく有理点列$(x_n)_{n=1}^{\infty} \subset \Q_f$に対し
			$(\epsilon_n)_{n=1}^{\infty}$が存在して
			\begin{align}
				x \notin \bigcup_{n=1}^{\infty} U_{\epsilon_n}(x_n)
			\end{align}
			を満たすことになり$x_n \longrightarrow x$に反する.
		}
		,
	\end{prf}
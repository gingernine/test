\section{緩増加超関数の台}
	$f:\R^n \longrightarrow \C$を緩増加連続関数として,定理\ref{thm:tempered_continuous_functions_and_tempered_distributions}
	で定める緩増加超関数$u_f$の台を考察する.
	いま,$\Omega \coloneqq (\supp{f})^c$とおけば,
	$\supp{\varphi} \subset \Omega$を満たす全ての$\varphi \in \rapid{\R^n}$に対して
	(定理\ref{construction_of_test_function}により$\varphi$は存在する.)
	\begin{align}
		\inprod<u_f,\varphi> = \int_{\R^n} f(x)\varphi(x)\ dx = 0
	\end{align}
	が成立する.これをもとに次を定義する.
	\begin{screen}
		\begin{dfn}[緩増加関数の台]
			$\Omega \subset \R^n$を開集合とする.緩増加超関数$u \in \tempdist{\R^n}$が
			$\Omega$で0であるとは,$\supp{\varphi} \subset \Omega$を満たす任意の
			$\varphi \in \rapid{\R^n}$に対し$\inprod<u,\varphi> = 0$が成り立つことであると定める.
			そして$u$の台$\supp{u}$を次で定める:
			\begin{align}
				\supp{u} \coloneqq \left\{ \mbox{$u$が0になる最大の開集合} \right\}^c = \left[ \bigcup_{\substack{\Omega:\mathrm{open} \\ u:0\ \mathrm{on}\ \Omega}} \Omega \right]^c.
			\end{align}
		\end{dfn}	
	\end{screen}
	
	\begin{screen}
		\begin{thm}
			緩増加連続関数$f$に対して$\supp{u_f} = \supp{f}$が成り立つ.
		\end{thm}
	\end{screen}
	
	\begin{prf}
		上で述べたことにより
		$u_f$は$(\supp{f})^c$で0になるから,
		$(\supp{f})^c \subset (\supp{u})^c$すなわち
		$\supp{u_f} \subset \supp{f}$が成り立つ.
		一方で任意の$x \in (\supp{u_f})^c$に対し
		\footnote{
			$\supp{u_f} = \R^n$の場合は自動的に$\supp{u_f} \supset \supp{f}$が成立する.
		}
		或る開近傍$U_x$が存在し,
		$\supp{\varphi} \subset U_x$を満たす任意の$\varphi \in \rapid{\R^n}$に対して
		\begin{align}
			\int_{U_x}  f(y) \varphi(y)\ dy = 0
		\end{align}
		が成立する.よって変分法の基本補題と$f$の連続性より
		$f$は$U_x$上で0に張り付くから,
		\begin{align}
			(\supp{u_f})^c \subset \Set{x \in \R^n}{f(x) = 0}^\circ = (\supp{f})^c
		\end{align}
		が従い$\supp{u_f} \supset \supp{f}$を得る.
		\QED
	\end{prf}

	\begin{screen}
		\begin{lem}[閉集合とコンパクト集合の和は閉]\label{lem:combination_closed_compact}
			$A,K \subset \R^n$をそれぞれ閉集合,コンパクト集合とする.このとき
			\begin{align}
				A + K = \Set{z = x + y \in \R^n}{x \in A,\ y \in K}
			\end{align}
			は$\R^n$の閉集合である.
		\end{lem}
	\end{screen}
	
	\begin{prf}
		$A+K$の完備性を示す.$(z_n)_{n=1}^{\infty}$を
		$A+K$のCauchy列とし,$z_n = x_n + y_n\ (x_n \in A,\ y_n \in K,\ n=1,2,\cdots)$と考える.
		\begin{description}
			\item[第一段]
				$(x_n)_{n=1}^{\infty}$がCauchy列である場合,
				$|y_n - y_m| \leq |z_n - z_m| + |x_n - x_m| \longrightarrow 0\ (n,m \longrightarrow \infty)$より
				$(y_n)_{n=1}^{\infty}$もCauchy列である.$A,K$が閉集合であるから
				$x_n \longrightarrow {}^\exists x \in A,\ y_n \longrightarrow {}^\exists y \in K$が満たされ,
				$z \coloneqq x + y \in A + K$とおけば
				\begin{align}
					|z_n - z| \leq |x_n - x| + |y_n - y| \longrightarrow 0
					\quad (n \longrightarrow \infty)
				\end{align}
				が従う.$(x_n)_{n=1}^{\infty}$と$(y_n)_{n=1}^{\infty}$の立場を交換しても同じことが言える.
			
			\item[第二段]
				$(x_n)_{n=1}^{\infty}$と$(y_n)_{n=1}^{\infty}$のどちらもCauchy列でない場合,
				$(y_n)_{n=1}^{\infty}$は有界であるから或る部分列
				$(y_{n_k})_{k=1}^{\infty}$が$K$で収束する.
				前段の結果より$(z_{n_k})_{k=1}^{\infty}$は$A + K$で収束し,
				$(z_n)_{n=1}^{\infty}$も部分列と同じ極限に収束する.
				\QED
		\end{description}
	\end{prf}
	
	\begin{screen}
		\begin{thm}\label{thm:supp_of_tempered_distributions_1}
			$u \in \tempdist{\R^n}$とする.
			\begin{description}
				\item[(1)]	任意のテスト関数$\varphi \in \Test{\R^n}$に対し次が成り立つ:
					\begin{align}
						\supp{u \ast \varphi} \subset \supp{u} + \supp{\varphi}.
						\label{eq:thm_supp_of_tempered_distributions_1}
					\end{align}
					
				\item[(2)] 
					$\tempdist{\R^n}$の位相に関して$u \ast \rho_\epsilon \longrightarrow u\ (\epsilon \longrightarrow 0)$が成立する.
					つまり次が成り立つ.
					\begin{align}
						\inprod<u \ast \rho_\epsilon,\varphi> \longrightarrow \inprod<u,\varphi>,
						\quad (\forall \varphi \in \rapid{\R^n}).
					\end{align}
			\end{description}
		\end{thm}
	\end{screen}
	
	\begin{prf}\mbox{}
		\begin{description}
			\item[(1)]
				$\supp{u} \neq \R^n$の場合を考える.
				補題\ref{lem:combination_closed_compact}より$\supp{u} + \supp{\varphi}$は閉集合であるから,
				$x \notin \supp{u}+\supp{\varphi}$を取れば
				$U \cap \supp{u}+\supp{\varphi} = \emptyset$を満たす開近傍$U$が存在する.
				このとき$U \subset (\supp{u \ast \varphi})^c$が成立し,
				\begin{align}
					(\supp{u} + \supp{\varphi})^c \subset (\supp{u \ast \varphi})^c
				\end{align}
				が従い(\refeq{eq:thm_supp_of_tempered_distributions_1})が得られる.
				実際,$\supp{\psi} \subset U$を満たす任意の$\psi \in \rapid{\R^n}$について
				\begin{align}
					\check{\varphi} \ast \psi (y)
					= \int_{U} \varphi(z - y)\psi(z)\ dz
					= 0,
					\quad (\forall y \in \supp{u})
				\end{align}
				が成り立ち
				\footnote{
					任意の$y \in \supp{u}$に対して$z - y \notin \supp{\varphi}\ (\forall z \in U)$が満たされる.
				}
				\begin{align}
					\inprod<u \ast \varphi, \psi>
					= \inprod<u, \check{\varphi} \ast \psi>
					= 0
				\end{align}
				が出る.すなわち$u \ast \varphi$は$U$で0であり$U \subset (\supp{u \ast \varphi})^c$が満たされる.
				
			\item[(2)]
				任意の$\varphi \in \rapid{\R^n}$に対して
				$\inprod<u \ast \rho_\epsilon, \varphi> = \inprod<u, \check{\rho}_\epsilon \ast \varphi> = \inprod<u, \rho_\epsilon \ast \varphi>$
				が成り立つから,$|\inprod<u, \rho_\epsilon \ast \varphi - \varphi>| \longrightarrow 0\ (\epsilon \longrightarrow 0)$となることを示せばよい.
				補題\ref{lem:tempered_distributions_continuity}より$u$に対し或る$c > 0$と$m \in \N$が存在して
				\begin{align}
					|\inprod<u, \rho_\epsilon \ast \varphi - \varphi>| \leq c p_m(\rho_\epsilon \ast \varphi - \varphi),
					\quad (\forall \varphi \in \rapid{\R^n})
				\end{align}
				を満たす.
				\begin{align}
					p_m(\rho_\epsilon \ast \varphi - \varphi)
					&= \sum_{|\alpha| + k \leq m} \sup{x \in \R^n}{(1+|x|^2)^k \left| \rho_\epsilon \ast \partial^\alpha \varphi(x) - \partial^\alpha \varphi(x) \right|} \\
					&= \sum_{|\alpha| + k \leq m} 
						\sup{x \in \R^n}{(1+|x|^2)^k \left| \int_{|y| \leq \epsilon} \left( \partial^\alpha \varphi(x-y) - \partial^\alpha \varphi(x) \right) \rho_\epsilon(y)\ dy \right|} \\
					&\leq \sum_{|\alpha| + k \leq m} 
						\sup{x \in \R^n}{\int_{|y| \leq \epsilon} (1+|x|^2)^k \left| \partial^\alpha \varphi(x-y) - \partial^\alpha \varphi(x) \right| \rho_\epsilon(y)\ dy}
					\label{eq:thm_supp_of_tempered_distributions_1_2}
				\end{align}
				まで半ノルムを展開し
				\begin{align}
					\partial^\alpha \varphi(x-y) - \partial^\alpha \varphi(x)
					= \int_0^1 \frac{d}{dt} \partial^\alpha \varphi(x-ty)\ dt
					= -\sum_{j=1}^{n} y_j \int_0^1 \partial_j \partial^\alpha \varphi(x-ty)\ dt
				\end{align}
				と式変形すれば,$\epsilon < 1$なら$(1+|x|^2) \leq 2 (1+|ty|^2)(1+|x-ty|^2) \leq 4(1+|x-ty|^2)\ (|y|\leq \epsilon,\ |t| \leq 1)$より
				\begin{align}
					(\refeq{eq:thm_supp_of_tempered_distributions_1_2})
					&= \sum_{|\alpha| + k \leq m} 
						\sup{x \in \R^n}{\int_{|y| \leq \epsilon} (1+|x|^2)^k  \left| \sum_{j=1}^{n} y_j \int_0^1 \partial_j \partial^\alpha \varphi(x-ty)\ dt \right| \rho_\epsilon(y)\ dy} \\
					&\leq n \epsilon \sum_{|\alpha| + k \leq m} 
						\sup{x \in \R^n}{\int_{|y| \leq \epsilon} (1+|x|^2)^k \int_0^1 \left| \partial_j \partial^\alpha \varphi(x-ty) \right|\ dt\ \rho_\epsilon(y)\ dy} \\
					&\leq 4^m n \epsilon \sum_{|\alpha| + k \leq m} 
						\sup{x \in \R^n}{\int_{|y| \leq \epsilon} \int_0^1 (1+|x-ty|^2)^k \left| \partial_j \partial^\alpha \varphi(x-ty) \right|\ dt\ \rho_\epsilon(y)\ dy} \\
					&\leq \left( 4^m n p_{m+1}(\varphi) \right) \epsilon 
				\end{align}
				が成り立ち,$\epsilon$を潰して$|\inprod<u \ast \rho_\epsilon, \varphi> - \inprod<u, \varphi>| 
				\leq c p_m(\rho_\epsilon \ast \varphi - \varphi) \longrightarrow 0\ (\epsilon \longrightarrow 0)$を得る.
				\QED
		\end{description}
	\end{prf}
	
	\begin{screen}
		\begin{thm}[コンパクト台を持つ緩増加超関数のFourier変換]
			$u \in \comtempdist{\R^n}$に対し,$K \coloneqq \supp{u}$とおく.このとき
			$K$上で$1$を満たす任意の$\varphi \in \rapid{\R^n}$に対して
			\begin{align}
				\hat{u}(\xi) \coloneqq \inprod<u,\exp{-ix \cdot \xi}\varphi>\ \footnotemark,
				\quad (\xi \in \R^n)
			\end{align}
			と定めれば,左辺は$K$上で$1$を満たす急減少関数の選び方に依らずに確定する.また
			$\xi \longmapsto \hat{u}(\xi)$は緩増加である.
		\end{thm}
	\end{screen}
	\footnotetext{
		正確には$a_\xi:\R^n \ni x \longmapsto \exp{-ix \cdot \xi}\varphi(x)$で定まる急減少関数$a_\xi$を用いて
		$\hat{u}(\xi) \coloneqq \inprod<u,a_\xi>$と表す.
	}
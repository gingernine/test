\section{緩増加超関数の台}
	\begin{screen}
		\begin{lem}
			$A,K \subset \R^n$をそれぞれ閉集合,コンパクト集合とする.このとき
			\begin{align}
				A + K \coloneqq \Set{z = x + y \in \R^n}{x \in A,\ y \in K}
			\end{align}
			で定める$A + K$は閉集合である.
		\end{lem}
	\end{screen}
	
	\begin{prf}
		$A+K$の完備性を示す.$(z_n)_{n=1}^{\infty}$を
		$A+K$のCauchy列とし,$z_n = x_n + y_n\ (x_n \in A,\ y_n \in K,\ n=1,2,\cdots)$と考える.
		\begin{description}
			\item[第一段]
				$(x_n)_{n=1}^{\infty}$がCauchy列である場合,
				$|y_n - y_m| \leq |z_n - z_m| + |x_n - x_m| \longrightarrow 0\ (n,m \longrightarrow \infty)$より
				$(y_n)_{n=1}^{\infty}$もCauchy列である.$A,K$が閉集合であるから
				$x_n \longrightarrow {}^\exists x \in A,\ y_n \longrightarrow {}^\exists y \in K$が満たされ,
				$z \coloneqq x + y \in A + K$とおけば
				\begin{align}
					|z_n - z| \leq |x_n - x| + |y_n - y| \longrightarrow 0
					\quad (n \longrightarrow \infty)
				\end{align}
				が従う.
			
			\item[第二段]
				$(x_n)_{n=1}^{\infty}$と$(y_n)_{n=1}^{\infty}$のどちらもCauchy列でない場合,
				$(y_n)_{n=1}^{\infty}$は有界であるから或る部分列
				$(y_{n_k})_{k=1}^{\infty}$が$K$で収束する.
				前段の結果より$(z_{n_k})_{k=1}^{\infty}$は$A + K$で収束し,
				$(z_n)_{n=1}^{\infty}$も部分列と同じ極限に収束する.
				\QED
		\end{description}
	\end{prf}
	
	\begin{prf}\mbox{}
		\begin{description}
			\item[(1)]
				$\supp{u} + \supp{\varphi}$は閉集合であるから,
				任意の$x \notin \supp{u}+\supp{\varphi}$に対して
				$U \cap \supp{u}+\supp{\varphi} = \emptyset$を満たす近傍$U$が存在する.
				このとき$U \subset (\supp{u \ast \varphi})^c$が成立し
				\begin{align}
					(\supp{u \ast \varphi})^c
					\supset (\supp{u} + \supp{\varphi})^c
				\end{align}
				が得られる.実際,冒頭に書いたような$U$を取れば,
				$\supp{\psi} \subset U$を満たす$\psi \in \rapid{\R^n}$について
				\begin{align}
					\check{\varphi} \ast \psi (y)
					= \int_{U} \varphi(z - y)\psi(z)\ dz
					= 0,
					\quad (\forall y \in \supp{u})
				\end{align}
				が成り立ち
				\footnote{
					任意の$y \in \supp{u}$に対して$z - y \notin \supp{\varphi}\ (\forall z \in U)$が満たされる.
				}
				\begin{align}
					\inprod<u \ast \varphi, \psi>
					= \inprod<u, \check{\varphi} \ast \psi>
					= 0
				\end{align}
				が従う.すなわち$u \ast \varphi$は$U$で0であり$U \subset (\supp{u \ast \varphi})^c$が満たされる.
				\QED
		\end{description}
	\end{prf}
	\begin{screen}
		\begin{dfn}[直交射影]
			$H$を複素Hilbert空間とする.
			線型写像$p:H \rightarrow H$が直交射影であるとは,
			或る$H$の閉部分空間$H_0$が存在し,
			$x \in H$とその直交分解$x = x_1 + x_2\ (x_1 \in H_0, x_2 \in H_0^{\perp})$
			に対し次を満たすことをいう\footnotemark:
			\begin{align}
				p:H \ni x \longmapsto x_1 \in H_0.
			\end{align}
			また$H$上の直交射影全体を$\Oproj{H}$と書く.
		\end{dfn}
	\end{screen}
	
	\footnotetext{
		射影定理より$x \in H$の直交分解は一意に定まるから,
		$p$は写像としてwell-definedである.
	}
	
	\begin{screen}
		\begin{prp}[直交射影の存在]
			$H$を複素Hilbert空間とする.$H$の任意の閉部分空間$L$に対し
			或る$p \in \Oproj{H}$が存在して$p:H \rightarrow L$を満たす.
			特に$\Ran{p} = L$が成り立つ.
		\end{prp}
	\end{screen}
	
	\begin{prf}
		Hilbert空間の射影定理により,任意の$x \in H$は
		$x = x_1 + x_2\ (x_1 \in L,x_2 \in L^\perp)$の形に一意に分解されるから
		\begin{align}
			p:H \ni x \longmapsto x_1 \in L
		\end{align}
		として線型写像を定めれば
		$p \in \Oproj{H}$が従う.特に任意の$u \in L$に対しては$p u = u$が満たされる.
		\QED
	\end{prf}
	
	\begin{screen}
		\begin{prp}[直交射影は冪等・自己共役]
			$H$を複素Hilbert空間とする.任意の$p:H \rightarrow H$に対し次は同値である:
			\begin{description}
				\item[(1)] $p \in \Oproj{H}$.
				\item[(2)] $p$は有界で$\Norm{p}{\selfBop{H} } \leq 1,\ p^2 = p,\ p^* = p$を満たす.
			\end{description}
			\label{prp:orthogonal_projection_idempotent_self_adjoint}
		\end{prp}
	\end{screen}
	
	\begin{prf}\mbox{}
		\begin{description}
			\item[(1)] $p \in \Oproj{H}$ならば或る$H$の閉部分空間$L$が存在して
				\begin{align}
					p: H \ni x \longmapsto x_1 \quad (x = x_1 + x_2,\ x_1 \in L,\ x_2 \in L^\perp)
				\end{align}
				を満たすから$\Norm{p}{\selfBop{H} } \leq 1$が従う.また$p u = u\ (\forall u \in L)$より
				$p^2 = p$が成り立ち,更に任意に$x,y \in H$を取れば
				\begin{align}
					x = x_1 + x_2, \quad y = y_1 + y_2 \quad (x_1,y_1 \in L,\ x_2,y_2 \in L^\perp)
				\end{align}
				と分解されるから,直交性より
				\begin{align}
					\inprod<p x, y>
					= \inprod<x_1, y_1 + y_2>
					= \inprod<x_1, y_1>
					= \inprod<x_1 + x_2, y_1>
					= \inprod<x, p y>
				\end{align}
				が成り立ち$y \in \Dom{p^*} $及び$p^* = p$を得る.
			
			\item[(2)] 仮定を満たす$p$について,$\Ran{p} \perp \Ran{I - p} $が成り立つことを示す.
				任意に$x \in \Ran{p} , y \in \Ran{I - p} $を取れば,
				$x = p x',\ y = (I - p) y'$を満たす$x',y' \in H$が存在する.
				$p^* = p$より$\Dom{p^*} = \Dom{p} $が満たされているから,
				\begin{align}
					\inprod<x,y> = \inprod<p x', (I - p) y'>
					= \inprod<x', p^* (I - p) y'>
					= \inprod<x', p (I - p) y'>
					= \inprod<x', (p - p^2) y'>
					= 0
				\end{align}
				が成り立ち$\Ran{p} \perp \Ran{I - p} $が従う.
				\begin{align}
					\Ran{p} = \Ran{I - p} {}^\perp = \left( \Ran{p} {}^\perp \right)^\perp
				\end{align}
				となるから$\Ran{p} $は$H$の閉部分空間であり$p \in \Oproj{H}$が得られる.
				\QED
		\end{description}
	\end{prf}
	
	\begin{screen}
		\begin{lem}[一様有界な作用素の極限は有界]
			$X$をノルム空間,$Y$をBanach空間とし,ノルムをそれぞれ$\Norm{\cdot}{X},\Norm{\cdot}{Y}$で表す.
			$A_n \in \Bop{X}{Y} \ (n=1,2,\cdots)$が
			\begin{align}
				\sup{n \in \N}{\Norm{A_n}{\Bop{X}{Y} }} < \infty
			\end{align}
			を満たし,かつ或る$X$で稠密な部分集合$S$が存在して全ての$x \in S$に対し
			$\left( A_n x \right)_{n=1}^{\infty}$が$Y$で収束するとき,
			\begin{align}
				\lim_{n \to \infty} A_n x = A x \quad (\forall x \in X)
			\end{align}
			を満たす$A \in \Bop{X}{Y} $が一意に存在する.
			\label{lem:limit_operator_of_uniformly_bounded_operators}
		\end{lem}
	\end{screen}
	
	\begin{prf}
		先ず任意の$x \in X$に対し$\left( A_n x \right)_{n=1}^{\infty}$が$Y$で収束することを示す.
		任意に$\epsilon > 0$を取る.
		\begin{align}
			a \coloneqq \sup{n \in \N}{\Norm{A_n}{\Bop{X}{Y} }}
		\end{align}
		とおき$\Norm{x - z}{X} < \epsilon/a$を満たす$z \in S$を一つ選べば,仮定より或る$N \in \N$が存在して
		\begin{align}
			\Norm{A_n z - A_m z}{Y} < \epsilon \quad (\forall n > m \geq N)
		\end{align}
		が成り立つから,
		\begin{align}
			\Norm{A_n x - A_m x}{Y} \leq a \Norm{x - z}{X} + \Norm{A_n z - A_m z}{Y} + a \Norm{x - z}{X} < 3\epsilon
		\end{align}
		が従う.よって$\left( A_n x \right)_{n=1}^{\infty}$は$Y$のCauchy列であり,$Y$の完備性より収束する.
		\begin{align}
			A x \coloneqq \lim_{n \to \infty} A_n x \quad (\forall x \in X)
		\end{align}
		として$A$を定めれば,任意の$x,y \in X$と$\alpha, \beta \in \C$に対し
		\begin{align}
			&\Norm{A(\alpha x + \beta y) - \alpha A x - \beta A y}{Y} \\
			&\qquad \leq \Norm{A(\alpha x + \beta y) - A_n(\alpha x + \beta y)}{Y}
				+ |\alpha| \Norm{A x - A_n x}{Y} + |\beta| \Norm{A x - A_n x}{Y}
			\longrightarrow 0 \quad (n \longrightarrow \infty)
		\end{align}
		が満たされるから$A$は線形作用素であり,かつ任意の$x \in X$に対して
		\begin{align}
			\Norm{A x}{Y} \leq \Norm{A x - A_n x}{Y} + \Norm{A_n x}{Y}
			\leq \Norm{A x - A_n x}{Y} + \Norm{A_n}{\Bop{X}{Y} } \Norm{x}{X}
		\end{align}
		が成り立ち,右辺で下極限を取れば
		\begin{align}
			\Norm{A x}{Y} \leq \liminf_{n \to \infty} \Norm{A_n}{\Bop{X}{Y} } \Norm{x}{X}
		\end{align}
		が従う.
		\begin{align}
			\Norm{A}{\Bop{X}{Y} } \leq \liminf_{n \to \infty} \Norm{A_n}{\Bop{X}{Y} } = \sup{n \in \N}{\inf{\nu \geq n}{\Norm{A_\nu}{\Bop{X}{Y} }}} \leq \sup{n \in \N}{\Norm{A_n}{\Bop{X}{Y} }}
		\end{align}
		より$A \in \Bop{X}{Y} $を得る.
		\QED
	\end{prf}
	
	\begin{screen}
		\begin{prp}[直交射影の積・和の性質]
			$H$を複素Hilbert空間とする.
			\begin{description}
				\item[(1)] $P,Q \in \Oproj{H}$に対し次が成り立つ:
					\begin{align}
						\Ran{P} \perp \Ran{Q}
						\quad \Leftrightarrow \quad  PQ = 0
						\quad \Leftrightarrow \quad  QP = 0.
					\end{align}
				
				\item[(2)] 
					$P_1,\cdots,P_n \in \Oproj{H}$に対し
					$P \coloneqq \sum_{i=1}^{n} P_i$とおけば次が成り立つ:
					\begin{align}
						P \in \Oproj{H}
						\quad \Leftrightarrow \quad P_i P_j = \delta_{ij} P_j \quad (i,j = 1,\cdots,n).
					\end{align}
					ただし$\delta_{ij}$はKroneckerのデルタである.
				
				\item[(3)] 
					$P_1,P_2,\cdots \in \Oproj{H}$が
					$P_i P_j = \delta_{ij} P_j \ (\forall i,j \in \N)$を満たしているとする.
					\begin{align}
						H_0 \coloneqq \closure{\LH{\cup_{i=1}^{\infty}\Ran{P_i}}}
					\end{align}
					に対して,$P \in \Oproj{H}$が$\Ran{P} = H_0$を満たすとき次が成り立つ:
					\begin{align}
						Px = \sum_{i=1}^{\infty} P_i x \quad (\forall x \in H).
					\end{align}
			\end{description}
			\label{prp:orthogonal_projection_product_sum}
		\end{prp}
	\end{screen}
	
	\begin{prf}\mbox{}
		\begin{description}
			\item[(1)] $\Ran{P} \perp \Ran{Q}$なら,命題\ref{prp:orthogonal_projection_idempotent_self_adjoint}より
				任意の$x,y \in H$に対し
				\begin{align}
					0 = \inprod<P x,Q y> = \inprod<x, P^*Q y> = \inprod<x, PQ y>
				\end{align}
				が成り立つ.特に$x = PQy$とすれば
				\begin{align}
					\Norm{PQ y}{} = 0 \quad (\forall y \in H)
				\end{align}
				が従い$PQ = 0$を得る.逆に$PQ = 0$ならば
				\begin{align}
					\inprod<P x,Q y> = 0 \quad (\forall x,y \in H)
				\end{align}
				が成り立つから$\Ran{P} \perp \Ran{Q}$が得られる.
				同様にして$\Ran{P} \perp \Ran{Q} \Leftrightarrow QP = 0$も成り立つ.
				
			\item[(2)] 
				\begin{description}
					\item[$\Leftarrow$]
						命題\ref{prp:orthogonal_projection_idempotent_self_adjoint}を使う.
						$\selfBop{H} $は線形空間であるから先ず$P \in \selfBop{H} $が成り立つ.
						また$P_i P_j = \delta_{ij} P_j$より
						\begin{align}
							P^2 = \sum_{i=1}^n P_i^2 = \sum_{i=1}^n P_i = P
						\end{align}
						が従い,更に任意の$x,y \in H$に対して
						\begin{align}
							\inprod<Px,y> = \sum_{i=1}^n \inprod<P_i x,y> 
							= \sum_{i=1}^n \inprod<x,P_i y> 
							= \inprod<x,P y>
						\end{align}
						が成り立つから$P^* = P$が得られる.
						
					\item[$\Rightarrow$] 
						$(P_i)_{i=1}^{n} \in \Oproj{H}$であるから,$i = j$なら
						命題\ref{prp:orthogonal_projection_idempotent_self_adjoint}より$P_i P_j = P_i$が成り立つ.また
						任意の$x \in H$に対し
						\begin{align}
							&\sum_{i,j=1}^{n} \Norm{P_i P_j x}{}^2
							= \sum_{i=1}^{n} \sum_{j=1}^{n} \inprod<P_i P_j x, P_i P_j x>
							= \sum_{i=1}^{n} \sum_{j=1}^{n} \inprod<P_i P_j x, P_j x> \\
							&\qquad = \sum_{j=1}^{n} \inprod<\sum_{i=1}^{n} P_i P_j x, P_j x>
							= \sum_{j=1}^{n} \inprod<P P_j x, P_j x>
							= \sum_{j=1}^{n} \Norm{P P_j x}{}^2
							\leq \sum_{j=1}^{n} \Norm{P_j x}{}^2
						\end{align}
						が成り立ち,一方で
						\begin{align}
							\sum_{i,j=1}^{n} \Norm{P_i P_j x}{}^2
							= \sum_{\substack{i,j=1 \\ i \neq j}}^{n} \Norm{P_i P_j x}{}^2
								+ \sum_{j=1}^{n} \Norm{P_j^2 x}{}^2
							= \sum_{\substack{i,j=1 \\ i \neq j}}^{n} \Norm{P_i P_j x}{}^2
								+ \sum_{j=1}^{n} \Norm{P_j x}{}^2
						\end{align}
						も成り立つから,
						\begin{align}
							\sum_{\substack{i,j=1 \\ i \neq j}}^{n} \Norm{P_i P_j x}{}^2 = 0
							\quad (\forall x \in H)
						\end{align}
						が従い$P_i P_j = 0\ (i \neq j)$を得る.
						\QED
				\end{description}
			\item[(3)] 
				$P_i P_j = \delta_{ij} P_j \ (\forall i,j \in \N)$が満たされているから,任意の$x \in H$と$N \in \N$に対し
				\begin{align}
					\Norm{\sum_{n=1}^{N} P_n x}{}^2 
					= \inprod<\sum_{n=1}^{N} P_n x, \sum_{n=1}^{N} P_n x>
					= \sum_{n=1}^{N} \inprod<P_n x, P_n x> + \sum_{\substack{n,m=1 \\ n \neq m}}^{N} \inprod<P_m P_n x, x>
					= \sum_{n=1}^{N} \Norm{P_n x}{}^2
				\end{align}
				が成り立つ.また
				\begin{align}
					x = \sum_{n=1}^{N} P_n x + \left( x - \sum_{n=1}^{N} P_n x \right)
				\end{align}
				とすれば,(2)より$\sum_{n=1}^{N} P_n \in \Oproj{H}$であるから
				\begin{align}
					\inprod<\sum_{n=1}^{N} P_n x, x - \sum_{n=1}^{N} P_n x> = 0
				\end{align}
				が満たされ,
				\begin{align}
					\Norm{x}{}^2 = \Norm{\sum_{n=1}^{N} P_n x}{}^2 + \Norm{x - \sum_{n=1}^{N} P_n x}{}^2
					\leq \sum_{n=1}^{N} \Norm{P_n x}{}^2
				\end{align}
				を得る.ゆえに
				\begin{align}
					\sum_{n=1}^{\infty} \Norm{P_n x}{}^2 \leq \Norm{x}{}^2 \quad (\forall x \in H)
				\end{align}
				かつ
				\begin{align}
					\Norm{\sum_{n=1}^{N} P_n}{\selfBop{H} } \leq 1 \quad (\forall N \in \N)
				\end{align}
				が成り立つ.
				\begin{align}
					 \Norm{\sum_{n=p}^{q} P_n x}{}^2 = \sum_{n=p}^{q} \Norm{P_n x}{}^2
				\end{align}
				が成り立つから$\sum_{n=1}^{\infty} P_n x$は$H$で収束する.
				補題\ref{lem:limit_operator_of_uniformly_bounded_operators}より
				\begin{align}
					Q x \coloneqq \sum_{n=1}^{\infty} P_n x \quad (\forall x \in H)
				\end{align}
				として$Q$を定めれば$Q \in \selfBop{H} $となる.$Q = P$となることを示す.
				任意の$x \in \Ran{P_n} $に対しては,$P_i P_j = \delta_{ij} P_j \ (\forall i,j \in \N)$より
				\begin{align}
					x = P x = Q x
				\end{align}
				が成り立つから$P,Q$は$\LH{\cup_{i=1}^{\infty}\Ran{P_i}}$で一致し,
				連続性から$H_0$上で一致する.
				また$x \in H_0^\perp$に対しても,$P_n x = 0\ (n=1,2,\cdots)$より
				\begin{align}
					0 = P x = Q x
				\end{align}
				が成り立つ.任意の$x \in X$は$x = x_1 + x_2\ (x_1 \in H_0,\ x_2 \in H_0^\perp)$
				と分解されるから$H$上で$P = Q$が成り立つ.
				\QED
		\end{description}
	\end{prf}
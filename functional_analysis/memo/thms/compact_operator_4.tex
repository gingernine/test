	\begin{screen}
		\begin{dfn}[直交射影]
			$H$を複素Hilbert空間とする.
			線型写像$p:H \rightarrow H$が直交射影であるとは,
			或る$H$の閉部分空間$H_0$が存在し,
			$x \in H$とその直交分解$x = x_1 + x_2\ (x_1 \in H_0, x_2 \in H_0^{\perp})$
			に対し次を満たすことをいう\footnotemark:
			\begin{align}
				p:H \ni x \longmapsto x_1 \in H_0.
			\end{align}
			また$H$上の直交射影全体を$\Oproj{H}$と書く.
		\end{dfn}
	\end{screen}
	
	\footnotetext{
		射影定理より$x \in H$の直交分解は一意に定まるから,
		$p$は写像としてwell-definedである.
	}
	
	\begin{screen}
		\begin{prp}[直交射影の存在]
			$H$を複素Hilbert空間とする.$H$の任意の閉部分空間$L$に対し
			或る$p \in \Oproj{H}$が存在して$p:H \rightarrow L$を満たす.
			特に$\Ran{p} = L$が成り立つ.
		\end{prp}
	\end{screen}
	
	\begin{prf}
		Hilbert空間の射影定理により,任意の$x \in H$は
		$x = x_1 + x_2\ (x_1 \in L,x_2 \in L^\perp)$の形に一意に分解されるから
		\begin{align}
			p:H \ni x \longmapsto x_1 \in L
		\end{align}
		として線型写像を定めれば
		$p \in \Oproj{H}$が従う.特に任意の$u \in L$に対しては$p u = u$が満たされる.
		\QED
	\end{prf}
	
	\begin{screen}
		\begin{prp}[直交射影は冪等・自己共役]
			$H$を複素Hilbert空間とする.任意の$p:H \rightarrow H$に対し次は同値である:
			\begin{description}
				\item[(1)] $p \in \Oproj{H}$.
				\item[(2)] $p$は有界で$p^2 = p$と$p^* = p$を満たす.
			\end{description}
			\label{prp:orthogonal_projection_idempotent_self_adjoint}
		\end{prp}
	\end{screen}
	
	\begin{screen}
		\begin{lem}[一様有界な作用素の極限は有界]
			$X$をノルム空間,$Y$をBanach空間とし,ノルムをそれぞれ$\Norm{\cdot}{X},\Norm{\cdot}{Y}$で表す.
			$A_n \in \Bop{X}{Y} \ (n=1,2,\cdots)$が
			\begin{align}
				\sup{n \in \N}{\Norm{A_n}{\Bop{X}{Y} }} < \infty
			\end{align}
			を満たし,かつ或る$X$で稠密な部分集合$S$が存在して全ての$x \in S$に対し
			$\left( A_n x \right)_{n=1}^{\infty}$が$Y$で収束するとき,
			\begin{align}
				\lim_{n \to \infty} A_n x = A x \quad (\forall x \in X)
			\end{align}
			を満たす$A \in \Bop{X}{Y} $が一意に存在する.
		\end{lem}
	\end{screen}
	
	\begin{prf}
		先ず任意の$x \in X$に対し$\left( A_n x \right)_{n=1}^{\infty}$が$Y$で収束することを示す.
		任意に$\epsilon > 0$を取る.
		\begin{align}
			a \coloneqq \sup{n \in \N}{\Norm{A_n}{\Bop{X}{Y} }}
		\end{align}
		とおき$\Norm{x - z}{X} < \epsilon/a$を満たす$z \in S$を一つ選べば,仮定より或る$N \in \N$が存在して
		\begin{align}
			\Norm{A_n z - A_m z}{Y} < \epsilon \quad (\forall n > m \geq N)
		\end{align}
		が成り立つから,
		\begin{align}
			\Norm{A_n x - A_m x}{Y} \leq a \Norm{x - z}{X} + \Norm{A_n z - A_m z}{Y} + a \Norm{x - z}{X} < 3\epsilon
		\end{align}
		が従う.よって$\left( A_n x \right)_{n=1}^{\infty}$は$Y$のCauchy列であり,$Y$の完備性より収束する.
		\begin{align}
			A x \coloneqq \lim_{n \to \infty} A_n x \quad (\forall x \in X)
		\end{align}
		として$A$を定めれば,任意の$x,y \in X$と$\alpha, \beta \in \C$に対し
		\begin{align}
			&\Norm{A(\alpha x + \beta y) - \alpha A x - \beta A y}{Y} \\
			&\qquad \leq \Norm{A(\alpha x + \beta y) - A_n(\alpha x + \beta y)}{Y}
				+ |\alpha| \Norm{A x - A_n x}{Y} + |\beta| \Norm{A x - A_n x}{Y}
			\longrightarrow 0 \quad (n \longrightarrow \infty)
		\end{align}
		が満たされるから$A$は線形作用素であり,かつ任意の$x \in X$に対して
		\begin{align}
			\Norm{A x}{Y} \leq \Norm{A x - A_n x}{Y} + \Norm{A_n x}{Y}
			\leq \Norm{A x - A_n x}{Y} + \Norm{A_n}{\Bop{X}{Y} } \Norm{x}{X}
		\end{align}
		が成り立ち,右辺で下極限を取れば
		\begin{align}
			\Norm{A x}{Y} \leq \liminf_{n \to \infty} \Norm{A_n}{\Bop{X}{Y} } \Norm{x}{X}
		\end{align}
		が従う.
		\begin{align}
			\Norm{A}{\Bop{X}{Y} } \leq \liminf_{n \to \infty} \Norm{A_n}{\Bop{X}{Y} } = \sup{n \in \N}{\inf{\nu \geq n}{\Norm{A_\nu}{\Bop{X}{Y} }}} \leq \sup{n \in \N}{\Norm{A_n}{\Bop{X}{Y} }}
		\end{align}
		より$A \in \Bop{X}{Y} $を得る.
		\QED
	\end{prf}
	
	\begin{screen}
		\begin{prp}[直交射影の積・和の性質]
			$H$を複素Hilbert空間とする.
			\begin{description}
				\item[(1)] $p,q \in \Oproj{H}$に対し次が成り立つ:
					\begin{align}
						\Ran{p} \perp \Ran{q}
						\quad \Leftrightarrow \quad  pq = 0
						\quad \Leftrightarrow \quad  qp = 0.
					\end{align}
				
				\item[(2)] 
					$p_1,\cdots,p_n \in \Oproj{H}$が
					$p_i \neq p_j\ (i \neq j)$を満たすなら,
					$p \coloneqq \sum_{i=1}^{n} p_i$とおいて次が成り立つ:
					\begin{align}
						p \in \Oproj{H}
						\quad \Leftrightarrow \quad p_i p_j = \delta_{ij} p_j \quad (i,j = 1,\cdots,n).
					\end{align}
					ただし$\delta_{ij}$はKroneckerのデルタである.
				
				\item[(3)] 
					$p_1,p_2,\cdots \in \Oproj{H}$が
					$p_i p_j = \delta_{ij} p_j \ (\forall i,j \in \N)$を満たすとして
					\begin{align}
						H_0 \coloneqq \closure{\LH{\bigcup_{i=1}^{\infty}\Ran{p_i}}}
					\end{align}
					とおく.$p \in \Oproj{H}$が$\Ran{p} = H_0$であるとき次が成り立つ:
					\begin{align}
						px = \sum_{i=1}^{\infty} p_i x \quad (\forall x \in H).
					\end{align}
			\end{description}
			\label{prp:orthogonal_projection_product_sum}
		\end{prp}
	\end{screen}
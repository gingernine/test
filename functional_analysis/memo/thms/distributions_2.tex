\section{あんまり意味ないメモ}
	$\Omega \neq \emptyset$を$\R^n$の開集合とする.
	$\Omega$に含まれるコンパクト集合を台とする$C^\infty$-級関数$(\Omega \longrightarrow \C)$を($\Omega$上の){\bf テスト関数(test function)}と呼び,
	その全体を$\Test{\Omega}$で表す.以後考察対象となるデルタ近似関数は$\R^n$上のテスト関数である.
	\begin{align}
		f(t) \coloneqq 
		\begin{cases}
			\exp{-\frac{1}{t}} & (t > 0), \\
			0 & (t \leq 0)
		\end{cases}
	\end{align}
	により$\R$上無限回微分可能な$f$を定めれば,
	\begin{align}
		g(t) \coloneqq f(1-t)f(1+t) = 
		\begin{cases}
			\exp{-\frac{2}{1-t^2}} & (|t| < 1), \\
			0 & (|t| \geq 1)
		\end{cases}
	\end{align}
	で定める$g$もまたLibnizの公式より$\R$上で無限回微分可能である.指数関数は0を取りえないから
	\begin{align}
		\Set{t \in \R}{g(t) \neq 0} = \Set{t \in \R}{|t| < 1}
	\end{align}
	が成り立つ.$\R^n \ni x \longmapsto |x|^2$もまた無限回微分可能であるから
	\begin{align}
		h(x) \coloneqq g(|x|^2) = g(x_1^2+ \cdots + x_n^2),
		\quad (\forall x=(x_1,\cdots,x_n) \in \R^n)
	\end{align}
	とおけば$h \in \Test{\R^n}$が満たされ,
	\begin{align}
		c \coloneqq \int_{\R^n} h(x)\ dx > 0
	\end{align}
	に対し$\rho \coloneqq (1/c)h$で定める$\rho \in \Test{\R^n}$は
	\begin{align}
		\supp{\rho} = \Set{x \in \R^n}{|x| \leq 1}
	\end{align}
	かつ
	\begin{align}
		\int_{\R^n} \rho(x)\ dx = 1
	\end{align}
	を満たす.
	
	\begin{screen}
		\begin{dfn}[デルタ近似関数]
			Diracのデルタ関数を近似する関数をデルタ近似関数と呼び,任意の$\epsilon > 0$に対し
			\begin{align}
				\rho_\epsilon(x) \coloneqq \epsilon^n \rho(x/\epsilon),
				\quad (\forall x \in \R^n)
			\end{align}
			により定めるテスト関数$\rho_\epsilon$はデルタ近似関数である.
		\end{dfn}
	\end{screen}
	
	\begin{screen}
		\begin{thm}[デルタ近似関数$\rho_\epsilon$の性質]\mbox{}
			\begin{description}
				\item[(1)] $\rho_\epsilon \geq 0$.
				\item[(2)] $\check{\rho}_\epsilon = \rho_\epsilon$.
				\item[(3)] $\supp{\rho_\epsilon} = \Set{x \in \R^n}{|x| \leq \epsilon}$.
				\item[(4)] $\int_{\R^n} \rho_\epsilon(x)\ dx = 1$.
			\end{description}
		\end{thm}
	\end{screen}
	
	\begin{screen}
		\begin{thm}[テスト関数を任意に構成する]
		\label{construction_of_test_function}
			任意に$\R^n$のコンパクト集合$K$と開集合$U\ (K \subset U)$を取れば,
			\begin{align}
				0 \leq \eta \leq 1,
				\quad 
				\eta(x) =
				\begin{cases}
					1 & (x \in K), \\
					0 & (x \in \R^n \backslash U) 
				\end{cases}
			\end{align}
			を満たす$\eta \in \Test{\R^n}$が存在する.
		\end{thm}
	\end{screen}
	
	\begin{prf}
		任意の$x \in K$に対し
		$B(x;\epsilon_x) \coloneqq \Set{y \in \R^n}{|x-y| < \epsilon_x} \subset U$を満たす$\epsilon_x > 0$が存在し,
		コンパクト性から有限個の$x_1,\cdots,x_m \in K$により$K \subset B(x_1,\epsilon_1/3) \cup \cdots \cup B(x_m,\epsilon_m/3)$が成り立つ.
		ここで$\epsilon \coloneqq \min{}{\left\{ \epsilon_{x_1}/3,\cdots,\epsilon_{x_m}/3 \right\}}$とおけば
		\begin{align}
			\eta(x) \coloneqq \rho_\epsilon \ast \defunc_K(x) = \int_{K} \rho_\epsilon(x - y)\ dy,
			\quad (\forall x \in \R^n)
		\end{align}
		が求めるテスト関数である.先ず任意の$\alpha \in \N^n$に対して
		\begin{align}
			\partial^\alpha \eta(x) = \int_K \partial^\alpha \rho_\epsilon(x-y)\ dy,
			\quad (\forall x \in \R^n)
		\end{align}
		が成り立つから$\eta$は$\R^n$上で無限回微分可能であり,
		$x \in \R^n \backslash U$なら
		$|x - y| > \epsilon\ (\forall y \in K)$より$\eta(x) = 0$が従う.
		\begin{align}
			\eta(x) = \int_K \rho_\epsilon(x-y)\ dy \leq \int_{\R^n} \rho_\epsilon(x-y)\ dy = 1
		\end{align}
		より$0 \leq \eta \leq 1$が成り立ち,また
	\end{prf}
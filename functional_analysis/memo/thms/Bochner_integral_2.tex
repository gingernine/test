\section{Pettisの強可測性定理}
	\begin{screen}
		\begin{lem}[距離空間値の可測関数列の極限は可測]
			$(S,d)$を距離空間,$(X,\mathcal{M})$を可測空間とする.
			$\mathcal{M}/\borel{S}$-可測関数列$(f_n)_{n=1}^{\infty}$が
			各点$x \in X$で極限を持てば,
			$f \coloneqq \lim_{n \to \infty} f_n$で定める関数$f$もまた可測$\mathcal{M}/\borel{S}$となる.
			\label{lem:measurability_metric_space}
		\end{lem}
	\end{screen}
	
	\begin{prf}
		任意に$S$の閉集合$C$を取り,閉集合の系$(C_m)_{m=1}^{\infty}$を次で定める:
		\begin{align}
			C_m \coloneqq \Set{y \in S}{d(y,C) \leq \frac{1}{m}}, \quad (m=1,2,\cdots).\ \footnotemark
		\end{align}
		\footnotetext{
			$S \ni y \longmapsto d(y,C) \in [0,\infty)$は連続であるから,
			閉集合$[0,1/m]$は$S$の閉集合に引き戻される.
		}
		$f(x) \in C$を取れば$f_n(x) \longrightarrow f(x)$が満たされているから,任意の$m \in \N$に対し或る$N = N(x,m) \in \N$が対応して
		\begin{align}
			d\left( f_n(x),C \right) \leq d\left( f_n(x),f(x) \right) < \frac{1}{m}
			\quad (\forall n \geq N)
		\end{align}
		が成り立ち
		\begin{align}
			f^{-1}(C) \subset \bigcap_{m \geq 1} \bigcup_{N \in \N} \bigcap_{n \geq N} f_n^{-1}(C_m)
			\label{eq:lem_measurability_metric_space}
		\end{align}
		が従う.一方$f(x) \notin C$については,$0 < \epsilon < d(f(x),C)$を満たす$\epsilon$に対し
		或る$N = N(x,\epsilon) \in \N$が存在して
		\begin{align}
			d\left( f_n(x), f(x) \right) < \epsilon
			\quad (\forall n \geq N)
		\end{align}
		が成り立つから,$1/m < d(f(x),C) - \epsilon$を満たす$m \in \N$を取れば
		\begin{align}
			\frac{1}{m} < d(f(x),C) - d(f(x),f_n(x)) \leq d(f_n(x),C)
			\quad (\forall n \geq N)
		\end{align}
		が従い
		\begin{align}
			f^{-1}(C^c) \subset \bigcup_{m \geq 1} \bigcup_{N \in \N} \bigcap_{n \geq N} f_n^{-1}(C_m^c)
			\subset \bigcup_{m \geq 1} \bigcap_{N \in \N} \bigcup_{n \geq N} f_n^{-1}(C_m^c)
		\end{align}
		が得られる.(\refeq{eq:lem_measurability_metric_space})と併せれば
		\begin{align}
			f^{-1}(C) = \bigcap_{m \geq 1} \bigcup_{N \in \N} \bigcap_{n \geq N} f_n^{-1}(C_m)
		\end{align}
		が成り立つ.$S$の閉集合は$f$により$\mathcal{M}$の元に引き戻されるから$f$は可測$\mathcal{M}/\borel{S}$である.
		\QED
	\end{prf}
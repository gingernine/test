	本章を通じて係数体を$\C$とし,ノルム空間$E$におけるノルムを$\Norm{\cdot}{E}$と書きノルム位相を導入する.
	また$(X,\mathcal{M},\mu)$を$\sigma$-有限な測度空間($\mu$:正値測度),$B$を複素Banach空間とする.

\section{ノーミング}
	\begin{screen}
		\begin{thm}[Hahn-Banachの拡張定理]
			$E$を線形空間,$F$を$E$の線型部分空間とし,
			或る$p:E \rightarrow \C$が存在して
			\begin{align}
				p(x + y) \leq p(x) + p(y),
				\quad p(\lambda x) = |\lambda|p(x)
				\quad (\forall x,y \in E,\ \lambda \in \C)
				\label{eq:thm_hahn_banach_extension}
			\end{align}
			が成り立つとする.このとき$F$上の線型汎関数$f$が
			\begin{align}
				|f(x)| \leq p(x) \quad (\forall x \in F)
			\end{align}
			を満たすなら,次の関係を持つ$f$の拡張線型汎関数$\tilde{f}:E \rightarrow \C$が存在する:
			\begin{align}
				\left| \tilde{f}(x) \right| \leq p(x) \quad (\forall x \in E).
			\end{align}
			\label{thm:hahn_banach_extension}
		\end{thm}
	\end{screen}
	
	\begin{prf}
		see Kreyszig.
	\end{prf}
	
	\begin{screen}
		\begin{cor}[ノルム空間における拡張定理]
			$E$をノルム空間,$F$を$E$の部分ノルム空間とする.
			\begin{description}
				\item[(1)]
					任意の$f^* \in F^*$に対し,$f^*$の拡張である$g^* \in E^*$が存在して
					$\Norm{g^*}{E^*} = \Norm{f^*}{F^*}$を満たす.
				
				\item[(2)]
					任意の$x \in E$に対し$\Norm{x}{E} = \sup{}{\Set{\left| g^*(x) \right|}{g^* \in E^*,\ \Norm{g^*}{E^*} = 1}}$が成り立つ.
			\end{description}
			\label{cor:hahn_banach_extension}
		\end{cor}
	\end{screen}
	
	\begin{prf}\mbox{}
		\begin{description}
			\item[(1)]
				$f^*$は有界であるから
				\begin{align}
					\left| f^*(x) \right| \leq \Norm{f^*}{F^*} \Norm{x}{E} \quad (\forall x \in F)
				\end{align}
				が成り立つ.$E \ni x \longmapsto \Norm{f^*}{F^*} \Norm{x}{E}$は(\refeq{eq:thm_hahn_banach_extension})
				を満たすから,定理\refeq{thm:hahn_banach_extension}より或る$f^*$の拡張$g^* \in E^*$が存在して
				\begin{align}
					\left| g^*(x) \right| \leq \Norm{f^*}{F^*} \Norm{x}{E} \quad (\forall x \in E)
				\end{align}
				となり$\Norm{g^*}{E^*} \leq \Norm{f^*}{F^*}$が従う.また$\left. g^* \right|_F = f^*$であるから
				\begin{align}
					\Norm{f^*}{F^*} = \sup{\substack{x \in F \\ \Norm{x}{E} \leq 1}}{\left| f^*(x) \right|}
					= \sup{\substack{x \in F \\ \Norm{x}{E} \leq 1}}{\left| g^*(x) \right|}
					\leq \sup{\substack{x \in E \\ \Norm{x}{E} \leq 1}}{\left| g^*(x) \right|}
					= \Norm{g^*}{E^*}
				\end{align}
				も成り立ち$\Norm{g^*}{E^*} = \Norm{f^*}{F^*}$を得る.
			
			\item[(2)]
				$x = 0$の場合は全ての$g^* \in E^*$に対して$g^*(x) = 0$となるから主張が得られる.
				$x \neq 0$の場合,まずは
				\begin{align}
					h^*(x) = \Norm{x}{E},
					\quad \Norm{h^*}{E^*} = 1
					\label{eq:cor_hahn_banach_extension}
				\end{align}
				を満たす$h^* \in E^*$が存在することを示す.実際
				\begin{align}
					F \coloneqq \Set{\lambda x}{\lambda \in \C}
				\end{align}
				として$E$の部分ノルム空間を構成し
				\begin{align}
					f^*:F \ni \lambda x \longrightarrow \lambda \Norm{x}{E}
				\end{align}
				として等長作用素$f^* \in F^*$を定めれば,
				\begin{align}
					f^*(x) = \Norm{x}{E},
					\quad \Norm{f^*}{F^*} = 1
				\end{align}
				が成り立ち,(1)より(\refeq{eq:cor_hahn_banach_extension})を満たす$f^*$の拡張$h^* \in E^*$が存在する.
				今,任意の$g^* \in E^*$に対して
				\begin{align}
					\frac{\left| g^*(x) \right|}{\Norm{g^*}{E^*}} \leq \Norm{x}{E}
				\end{align}
				が成り立っているが,$g^* = h^*$とすれば等号が成立するから
				\begin{align}
					\sup{\substack{g^* \in E^* \\ \Norm{g^*}{E^*} = 1}}{\left| g^*(x) \right|} = \Norm{x}{E}
				\end{align}
				を得る.
				\QED
		\end{description}
	\end{prf}
	
	\begin{screen}
		\begin{dfn}[ノーミング]
			$E$をノルム空間,$E_0$を$E$の部分集合とする.或る$E^*$の部分集合$\tilde{E^*}$が存在して
			\begin{align}
				\Norm{x}{E} = \sup{\substack{g^* \in \tilde{E^*} \\ \Norm{g^*}{E^*} = 1}}{\left| g^*(x) \right|}
				\quad (\forall x \in E_0)
				\label{eq:dfn_norming}
			\end{align}
			を満たすとき,$\tilde{E^*}$を$E_0$のノーミング(norming)と呼ぶ.
			系\ref{cor:hahn_banach_extension}より$E^*$はノーミングの一つである.
		\end{dfn}
	\end{screen}
	
	\begin{screen}
		\begin{lem}[単位球面上にノーミングが存在する] 
			$E$をノルム空間とし,$E$の部分集合$E_0$が可分であるとする.
			$E^*$の部分集合$\tilde{E^*}$が$E_0$のノーミングであるなら,
			$E_0$のノーミングとなる単位点列$\left( g^*_n \right)_{n=1}^{\infty} \subset \tilde{E^*}$が存在する.
		\end{lem}
	\end{screen}
	
	\begin{prf}
		$(x_n)_{n=1}^{\infty}$が$E_0$において稠密であるとし,$\delta_n \coloneqq 1/2^n$とおく.
		(\refeq{eq:dfn_norming})より各$n \in \N$に対し或る$g^*_n \in \tilde{E^*}$が存在して
		\begin{align}
			\Norm{g^*_n}{E^*} = 1,
			\quad (1 - \delta_n) \Norm{x_n}{E} \leq \left| g^*_n(x_n) \right|
		\end{align}
		を満たす.任意に$x \in E_0,\ \epsilon > 0$を取れば,$(x_n)_{n=1}^{\infty}$
		の稠密性と$\delta_n \longrightarrow 0$より或る$n_0 \in \N$が存在して
		\begin{align}
			\Norm{x - x_{n_0}}{E} < \epsilon,
			\quad \delta_{n_0} < \epsilon
		\end{align}
		を同時に満たす.$\Norm{g^*_n}{E^*} = 1$より
		\begin{align}
			\left| g^*_{n_0}(x_{n_0}) \right| \leq \left| g^*_{n_0}(x_{n_0}) - g^*_{n_0}(x) \right| + \left| g^*_{n_0}(x) \right|
			< \epsilon + \left| g^*_{n_0}(x) \right|
		\end{align}
		が成り立つから
		\begin{align}
			(1 - \epsilon) \Norm{x}{E}
			\leq (1 - \delta_{n_0}) \Norm{x}{E}
			\leq \left| g^*_n(x_{n_0}) \right|
			< \epsilon + \left| g^*_{n_0}(x) \right|
		\end{align}
		が従い,$\epsilon > 0$の任意性から
		\begin{align}
			\Norm{x}{E} \leq \sup{n \in \N}{\left| g^*_n(x) \right|}
		\end{align}
		を得る.系\ref{cor:hahn_banach_extension}と併せれば
		\begin{align}
			\Norm{x}{E} \leq \sup{n \in \N}{\left| g^*_n(x) \right|}
			\leq \sup{\substack{g^* \in E^* \\ \Norm{g^*}{E^*} = 1}}{\left| g^*(x) \right|}
			= \Norm{x}{E}
		\end{align}
		が成り立つから,$\left( g^*_n \right)_{n=1}^{\infty}$は$E_0$のノーミングである.
		\QED
	\end{prf}
	
	\begin{screen}
		\begin{dfn}[分離]
			$E$をノルム空間,$E_0$を$E$の部分集合とする.
			或る$E^*$の部分集合$\tilde{E^*}$が存在して,
			任意に二点$x,y \in E_0,\ x \neq y$を選んでも
			$g^*(x) \neq g^*(y)$を満たす$g^* \in \tilde{E^*}$が取れるとき,
			$\tilde{E^*}$は$E_0$を分離するという.
		\end{dfn}
	\end{screen}
	
	\begin{screen}
		\begin{thm}[ノーミングは分離する]
			$E$をノルム空間,$E_0$を$E$の部分集合とする.$E^*$の部分集合$\tilde{E^*}$が
			$E_0$のノーミングであるなら,$\tilde{E^*}$は$E_0$を分離する.
		\end{thm}
	\end{screen}
	
	\begin{prf}
		背理法で証明する.$\tilde{E^*}$が$E_0$のノーミングであるとき,$x \neq y$を満たす或る組$x,y \in E_0$に対して
		\begin{align}
			g^*(x) = g^*(y) \quad \left( \forall g^* \in \tilde{E^*} \right)
		\end{align}
		が成り立つとすると,(\refeq{eq:dfn_norming})より
		\begin{align}
			\Norm{x - y}{E} = \sup{\substack{g^* \in \tilde{E^*} \\ \Norm{g^*}{E^*} = 1}}{\left| g^*(x - y) \right|} = 0
		\end{align}
		が従い$x \neq y$に矛盾する.
		\QED
	\end{prf}
	
	\begin{screen}
		\begin{lem}[可分な集合は可算列により分離される]
			$E$をノルム空間,$E_0$を$E$の可分な部分集合とする.
			$E^*$の部分集合$\tilde{E^*}$が$E_0$を分離するとき,
			$E_0$を分離する可算列$\left( g^*_n \right)_{n=1}^{\infty} \subset \tilde{E^*}$が存在する.
		\end{lem}
	\end{screen}
	
	\begin{prf}
		$\tilde{E^*}$が$E_0$を分離するなら,
		任意に$x \in E_0 \backslash \{0\}$を取れば或る$g^*_x \in \tilde{E^*}$が存在して
		$g^*_x(x) \neq g^*_x(0) = 0$を満たす.
		\begin{align}
			V_x \coloneqq \Set{y \in E_0}{g^*_x(y) \neq 0}
		\end{align}
		と定めれば,$g^*_x$の連続性より$V_x$は$E_0 \backslash \{0\}$の開集合である
		\footnote{
			$E_0 \backslash \{0\}$は$E$の部分位相空間であり$V_x$は$E$の開集合であるから,
			相対位相の意味で$V_x$は$E_0 \backslash \{0\}$の開集合となる.
		}
		.$x \in V_x\ (\forall x \in E_0 \backslash \{0\})$が満たされているから
		$\left( V_x \right)_{x \in E_0 \backslash \{0\}}$は$E_0 \backslash \{0\}$の開被覆である.
		更に可分性より$E_0 \backslash \{0\}$は第二可算公理を満たし
		\footnote{
			$E$のノルム位相は$d:(x,y) \longmapsto \Norm{x-y}{E}$で定まる距離で導入する位相に一致し,また
			相対位相としての$E_0 \backslash \{0\}$の位相は相対距離により導入される位相に一致する.
			従って$E_0 \backslash \{0\}$において可分であることと第二可算公理が満たされることは同値になる.
		}
		Lindel\Ddot{o}f性が従うから,$E_0 \backslash \{0\}$を覆う$\left( V_x \right)_{x \in E_0 \backslash \{0\}}$の可算部分列
		$\left( V_{x_n} \right)_{n=1}^{\infty}$が取れる.ここで$g^*_n \coloneqq g^*_{x_n}\ (n=1,2,\cdots)$とおけば
		$\left( g^*_n \right)_{n=1}^{\infty}$は$E_0$を分離する.実際
		$x \neq y$を満たす任意の組$x,y \in E_0$に対し,或る$n \in \N$が存在して$x - y \in V_{x_n}$となるから
		\begin{align}
			g^*_n(x - y) \neq 0 \quad \Rightarrow \quad g^*_n(x) \neq g^*_n(y)
		\end{align}
		が成り立つ.
		\QED
	\end{prf}
	
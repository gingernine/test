	本章を通じて係数体を$\C$とし,ノルム空間$E$におけるノルムを$\Norm{\cdot}{E}$と書きノルム位相を導入する.
	また$(X,\mathcal{M},\mu)$を$\sigma$-有限な測度空間($\mu$:正値測度),$B$を複素Banach空間とする.

\section{ノーミング}
	\begin{screen}
		\begin{thm}[Hahn-Banachの拡張定理]
			$E$を線形空間,$F$を$E$の線型部分空間とし,
			或る$p:E \rightarrow \C$が存在して
			\begin{align}
				p(x + y) \leq p(x) + p(y),
				\quad p(\lambda x) = |\lambda|p(x)
				\quad (\forall x,y \in E,\ \lambda \in \C)
				\label{eq:thm_hahn_banach_extension}
			\end{align}
			が成り立つとする.このとき$F$上の線型汎関数$f$が
			\begin{align}
				|f(x)| \leq p(x) \quad (\forall x \in F)
			\end{align}
			を満たすなら,次の関係を持つ$f$の拡張線型汎関数$\tilde{f}:E \rightarrow \C$が存在する:
			\begin{align}
				\left| \tilde{f}(x) \right| \leq p(x) \quad (\forall x \in E).
			\end{align}
			\label{thm:hahn_banach_extension}
		\end{thm}
	\end{screen}
	
	\begin{prf}
		see Kreyszig.
	\end{prf}
	
	\begin{screen}
		\begin{cor}[ノルム空間における拡張定理]
			$E$をノルム空間,$F$を$E$の部分ノルム空間とする.
			\begin{description}
				\item[(1)]
					任意の$f^* \in F^*$に対し,$f^*$の拡張である$g^* \in E^*$が存在して
					$\Norm{g^*}{E^*} = \Norm{f^*}{F^*}$を満たす.
				
				\item[(2)]
					任意の$x \in E$に対し$\Norm{x}{E} = \sup{}{\Set{\left| g^*(x) \right|}{g^* \in E^*,\ \Norm{g^*}{E^*} = 1}}$が成り立つ.
			\end{description}
			\label{cor:hahn_banach_extension}
		\end{cor}
	\end{screen}
	
	\begin{prf}\mbox{}
		\begin{description}
			\item[(1)]
				$f^*$は有界であるから
				\begin{align}
					\left| f^*(x) \right| \leq \Norm{f^*}{F^*} \Norm{x}{E} \quad (\forall x \in F)
				\end{align}
				が成り立つ.$E \ni x \longmapsto \Norm{f^*}{F^*} \Norm{x}{E}$は(\refeq{eq:thm_hahn_banach_extension})
				を満たすから,定理\refeq{thm:hahn_banach_extension}より或る$f^*$の拡張$g^* \in E^*$が存在して
				\begin{align}
					\left| g^*(x) \right| \leq \Norm{f^*}{F^*} \Norm{x}{E} \quad (\forall x \in E)
				\end{align}
				となり$\Norm{g^*}{E^*} \leq \Norm{f^*}{F^*}$が従う.また$\left. g^* \right|_F = f^*$であるから
				\begin{align}
					\Norm{f^*}{F^*} = \sup{\substack{x \in F \\ \Norm{x}{E} \leq 1}}{\left| f^*(x) \right|}
					= \sup{\substack{x \in F \\ \Norm{x}{E} \leq 1}}{\left| g^*(x) \right|}
					\leq \sup{\substack{x \in E \\ \Norm{x}{E} \leq 1}}{\left| g^*(x) \right|}
					= \Norm{g^*}{E^*}
				\end{align}
				も成り立ち$\Norm{g^*}{E^*} = \Norm{f^*}{F^*}$を得る.
			
			\item[(2)]
				$x = 0$の場合は全ての$g^* \in E^*$に対して$g^*(x) = 0$となるから主張が得られる.
				$x \neq 0$の場合,まずは
				\begin{align}
					h^*(x) = \Norm{x}{E},
					\quad \Norm{h^*}{E^*} = 1
					\label{eq:cor_hahn_banach_extension}
				\end{align}
				を満たす$h^* \in E^*$が存在することを示す.実際
				\begin{align}
					F \coloneqq \Set{\lambda x}{\lambda \in \C}
				\end{align}
				として$E$の部分ノルム空間を構成し
				\begin{align}
					f^*:F \ni \lambda x \longrightarrow \lambda \Norm{x}{E}
				\end{align}
				として等長作用素$f^* \in F^*$を定めれば,
				\begin{align}
					f^*(x) = \Norm{x}{E},
					\quad \Norm{f^*}{F^*} = 1
				\end{align}
				が成り立ち,(1)より(\refeq{eq:cor_hahn_banach_extension})を満たす$f^*$の拡張$h^* \in E^*$が存在する.
				今,任意の$g^* \in E^*$に対して
				\begin{align}
					\frac{\left| g^*(x) \right|}{\Norm{g^*}{E^*}} \leq \Norm{x}{E}
				\end{align}
				が成り立っているが,$g^* = h^*$とすれば等号が成立するから
				\begin{align}
					\sup{\substack{g^* \in E^* \\ \Norm{g^*}{E^*} = 1}}{\left| g^*(x) \right|} = \Norm{x}{E}
				\end{align}
				を得る.
				\QED
		\end{description}
	\end{prf}
	
	\begin{screen}
		\begin{dfn}[ノーミング]
			$E$をノルム空間,$E_0$を$E$の部分集合とする.或る$E^*$の線型部分空間$\tilde{E^*}$が存在して
			\begin{align}
				\Norm{x}{E} = \sup{\substack{g^* \in \tilde{E^*} \\ \Norm{g^*}{E^*} = 1}}{\left| g^*(x) \right|}
				\quad (\forall x \in E_0)
			\end{align}
			を満たすとき,$\tilde{E^*}$を$E_0$のノーミング(norming)と呼ぶ.
			系\ref{cor:hahn_banach_extension}より$E^*$はノーミングの一つである.
		\end{dfn}
	\end{screen}
	
	\begin{screen}
		\begin{lem}[単位球面上にノーミングが存在する] 
			
		\end{lem}
	\end{screen}
	
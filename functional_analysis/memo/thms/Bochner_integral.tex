	係数体$\K$を$\R$或は$\C$と考え,ノルム空間$E$におけるノルムを$\Norm{\cdot}{E}$と書きノルム位相を導入する.
	また$(X,\mathcal{M},\mu)$を$\sigma$-有限な測度空間($\mu$:正値測度),$B$を複素Banach空間とする.

\section{ノーミング}
	\begin{screen}
		\begin{thm}[Hahn-Banachの拡張定理の系]
			$E$をノルム空間,$F$を$E$の部分空間とする.
			\begin{description}
				\item[(1)]
					任意の$y^* \in F^*$に対し,$y^*$の拡張である$x^* \in E^*$が存在して
					$\Norm{x^*}{E^*} = \Norm{y^*}{F^*}$を満たす.
				
				\item[(2)]
					任意の$x \in E$に対し$\Norm{x}{E} = \sup{\substack{x^* \in E^* \\ \Norm{x^*}{E^*} = 1}}{\left| \inprod<x,x^*> \right|}$が成り立つ.
			\end{description}
			\label{cor:hahn_banach_extension}
		\end{thm}
	\end{screen}
	
	\begin{prf}\mbox{}
		\begin{description}
			\item[(1)]
				$y^*$は有界であるから
				\begin{align}
					\left| \inprod<x,y^*> \right| \leq \Norm{y^*}{F^*} \Norm{x}{E} \quad (\forall x \in F)
				\end{align}
				が成り立つ.$E \ni x \longmapsto \Norm{y^*}{F^*} \Norm{x}{E}$は
				セミノルムであるから,Hahn-Banachの拡張定理より$y^*$の拡張$x^* \in E^*$が存在して
				\begin{align}
					\left| \inprod<x,x^*> \right| \leq \Norm{y^*}{F^*} \Norm{x}{E} \quad (\forall x \in E)
				\end{align}
				を満たし,$\Norm{x^*}{E^*} \leq \Norm{y^*}{F^*}$が従う.一方で
				\begin{align}
					\Norm{y^*}{F^*} = \sup{\substack{x \in F \\ \Norm{x}{E} \leq 1}}{\left| \inprod<x,y^*> \right|}
					= \sup{\substack{x \in F \\ \Norm{x}{E} \leq 1}}{\left| \inprod<x,x^*> \right|}
					\leq \sup{\substack{x \in E \\ \Norm{x}{E} \leq 1}}{\left| \inprod<x,x^*> \right|}
					= \Norm{x^*}{E^*}
				\end{align}
				も成り立ち$\Norm{x^*}{E^*} = \Norm{y^*}{F^*}$を得る.
			
			\item[(2)]
				$x = 0$の場合は全ての$x^* \in E^*$に対して$\inprod<x,x^*> = 0$となるから主張が得られる.
				$x \neq 0$の場合,
				\begin{align}
					F \coloneqq \Set{\lambda x}{\lambda \in \C}
				\end{align}
				により$E$の部分空間を作り
				\begin{align}
					y^*:F \ni \lambda x \longrightarrow \lambda \Norm{x}{E}
				\end{align}
				として等長作用素$y^* \in F^*$を定めれば,
				\begin{align}
					\inprod<x,y^*> = \Norm{x}{E},
					\quad \Norm{y^*}{F^*} = 1
				\end{align}
				が成立し,(1)より或る$y^*$の拡張$z^* \in E^*$が存在して
				\begin{align}
					\inprod<x,z^*> = \Norm{x}{E},
					\quad \Norm{z^*}{E^*} = 1
				\end{align}
				を満たす.今,任意の$0 \neq x^* \in E^*$に対して
				\begin{align}
					\frac{\left| \inprod<x,x^*> \right|}{\Norm{x^*}{E^*}} \leq \Norm{x}{E}
				\end{align}
				が成り立っているが,$x^* = z^*$のとき等号で成立するから
				\begin{align}
					\sup{\substack{x^* \in E^* \\ \Norm{x^*}{E^*} = 1}}{\left| \inprod<x,x^*> \right|} = \Norm{x}{E}
				\end{align}
				を得る.
				\QED
		\end{description}
	\end{prf}
	
	\begin{screen}
		\begin{dfn}[ノーミング]
			$E$をノルム空間,$S$を$E$の部分集合とする.$E^*$の或る部分空間$F$が
			\begin{align}
				\Norm{x}{E} = \sup{\substack{x^* \in F \\ \Norm{x^*}{E^*} = 1}}{\left| \inprod<x,x^*> \right|}
				\quad (\forall x \in S)
				\label{eq:dfn_norming}
			\end{align}
			を満たすとき,$F$を$S$のノーミング(norming)と呼ぶ.
			定理\ref{cor:hahn_banach_extension}より$E^*$は$E$のノーミングである.
		\end{dfn}
	\end{screen}
	
	\begin{screen}
		\begin{lem}[単位球面上のノーミング] 
			$E$をノルム空間とし,$S$を$E$の可分な部分集合とする.
			$E^*$の部分空間$F$が$S$のノーミングであるなら,
			$F$の単位球面上の或る点列$\left( x^*_n \right)_{n=1}^{\infty}$も
			$S$のノーミングとなる.
		\end{lem}
	\end{screen}
	
	\begin{prf}
		$(x_n)_{n=1}^{\infty}$が$S$において稠密であるとし,$\delta_n \coloneqq 1/2^n$とおけば,
		(\refeq{eq:dfn_norming})より各$n \in \N$に対し
		\begin{align}
			\Norm{x^*_n}{E^*} = 1,
			\quad (1 - \delta_n) \Norm{x_n}{E} \leq \left| \inprod<x_n,x^*_n> \right|
		\end{align}
		を満たす$x^*_n \in F$が取れる.ここで任意に$x \in S,\ \epsilon > 0$を取れば,$(x_n)_{n=1}^{\infty}$
		の稠密性と$\delta_n \longrightarrow 0$より
		\begin{align}
			\Norm{x - x_{n_0}}{E} < \epsilon,
			\quad \delta_{n_0} < \epsilon
		\end{align}
		を同時に満たす$n_0 \in \N$が存在する.$\Norm{x^*_{n_0}}{E^*} = 1$より
		\begin{align}
			(1 - \epsilon) \Norm{x}{E}
			&\leq (1 - \delta_{n_0}) \Norm{x}{E}
			\leq (1 - \delta_{n_0}) \Norm{x_{n_0}}{E} + (1 - \delta_{n_0}) \epsilon
			\leq \left| \inprod<x_{n_0},x^*_{n_0}> \right| + (1 - \delta_{n_0}) \epsilon \\
			&\leq \left\{ \left| \inprod<x_{n_0}-x,x^*_{n_0}> \right| 
				+ \left| \inprod<x,x^*_{n_0}> \right| \right\} + (1 - \delta_{n_0}) \epsilon \\
			&< \left\{ \epsilon + \left| \inprod<x,x^*_{n_0}> \right| \right\} + (1 - \delta_{n_0}) \epsilon \\
			&< \sup{n \in \N}{\left| \inprod<x,x^*_n> \right|} + 2 \epsilon
		\end{align}
		が従い,$\epsilon > 0$の任意性から
		\begin{align}
			\Norm{x}{E} \leq \sup{n \in \N}{\left| \inprod<x,x^*_n> \right|}
		\end{align}
		が出る.定理\ref{cor:hahn_banach_extension}より
		\begin{align}
			\Norm{x}{E} \leq \sup{n \in \N}{\left| \inprod<x,x^*_n> \right|}
			\leq \sup{\substack{x^* \in E^* \\ \Norm{x^*}{E^*} = 1}}{\left| \inprod<x,x^*> \right|}
			= \Norm{x}{E}
		\end{align}
		も得られるから,$\left( x^*_n \right)_{n=1}^{\infty}$は$S$のノーミングとなる.
		\QED
	\end{prf}
	
	\begin{screen}
		\begin{dfn}[分離]
			$E$をノルム空間,$S$を$E$の部分集合,
			$F$を$E^*$の部分空間とする.
			任意の相異なる二点$x,y \in S$に対し
			$\inprod<x,x^*> \neq \inprod<y,x^*>$を満たす$x^* \in F$が取れるとき,
			$F$は$S$を分離する(separates)という.
		\end{dfn}
	\end{screen}
	
	\begin{screen}
		\begin{thm}[ノーミングは分離する]
			$E$をノルム空間,$S$を$E$の部分集合とする.$E^*$の部分空間$F$が
			$S$のノーミングであるなら,$F$は$S$を分離する.
		\end{thm}
	\end{screen}
	
	\begin{prf}
		$x,y \in S$に対して
		\begin{align}
			\inprod<x,x^*> = \inprod<y,x^*> \quad \left( \forall x^* \in F \right)
		\end{align}
		が満たされるとき,(\refeq{eq:dfn_norming})より
		\begin{align}
			\Norm{x - y}{E} = \sup{\substack{x^* \in F \\ \Norm{x^*}{E^*} = 1}}{\left| \inprod<x-y,x^*> \right|} = 0
		\end{align}
		となり$x = y$が従う.
		\QED
	\end{prf}
	
	\begin{screen}
		\begin{lem}[可分な集合は可算列により分離される]
			$E$をノルム空間,$E_0$を$E$の可分な部分空間とする.
			$E^*$の部分空間$F$が$E_0$を分離するとき,
			$F$の或る可算点列$\left( x^*_n \right)_{n=1}^{\infty}$
			も$E_0$を分離する.
		\end{lem}
	\end{screen}
	
	\begin{prf}
		任意に$x \in E_0 \backslash \{0\}$を取れば,或る$f^*_x \in F$が存在して
		$\inprod<x,f^*_x> \neq \inprod<0,f^*_x> = 0$を満たす.
		\begin{align}
			V_x \coloneqq \Set{y \in E_0 \backslash \{0\}}{\inprod<y,f^*_x> \neq 0}
		\end{align}
		とおけば,$\left. f^*_x \right|_{E_0 \backslash \{0\}}$
		の連続性より$V_x$は$E_0 \backslash \{0\}$の相対位相に関して開であり,
		また$x \in V_x\ (\forall x \in E_0 \backslash \{0\})$より
		\begin{align}
			E_0 \backslash \{0\} = \bigcup_{x \in E_0 \backslash \{0\}} V_x
		\end{align}
		が成り立つが,いま$E_0 \backslash \{0\}$は可分な($E$の部分)距離空間であるから,
		Lindel\Ddot{o}f性より
		\begin{align}
			E_0 \backslash \{0\} = \bigcup_{n=1}^\infty V_{x_n}
		\end{align}
		を満たす$\left\{ V_{x_n} \right\}_{n=1}^{\infty} 
		\subset \left\{ V_x \right\}_{x \in E_0 \backslash \{0\}}$が取れる.
		このとき$x^*_n \coloneqq f^*_{x_n}\ (n=1,2,\cdots)$で定める可算族
		$\left( x^*_n \right)_{n=1}^{\infty}$は$E_0$を分離する.実際
		$x,y \in E_0$が相異なるとき,$x - y \in E_0 \backslash \{0\}$より
		或る$n \in \N$に対して$x - y \in V_{x_n}$となり
		\begin{align}
			\inprod<x,x^*_n> - \inprod<y,x^*_n> = \inprod<x-y,x^*_n> \neq 0
		\end{align}
		が成り立つ.
		\QED
	\end{prf}
	
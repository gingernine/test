\section{自己共役作用素のスペクトル分解}
	$H$を複素Hilbert空間とし,内積とノルムをそれぞれ$\inprod<\cdot,\cdot>,\Norm{\cdot}{}$と表す.
	また$I$を$H$上の恒等写像とし,
	$\CM = \CM(\R^d,\borel{\R^d})$と$\cvan{\R^d}$のノルムをそれぞれ$\Norm{\cdot}{\CM},\Norm{\cdot}{\infty}$で表す.
	
	\begin{screen}
		\begin{thm}[スペクトル分解定理]
			写像$T:\cvan{\R^d} \rightarrow \selfBop{H} $が有界線型で次を満たすとする:
			\begin{description}
				\item[(1)] $T_f T_g = T_{fg} \quad (\forall f,g \in \cvan{\R^d}).$
				\item[(2)] $T_f^* = T_{\conj{f}} \quad (\forall f \in \cvan{\R^d}).$
				\item[(3)] 或る$\cvan{\R^d}$の列$(\phi_n)_{n=1}^{\infty}$が存在して次を満たす:
					\begin{align}
						&\sup{n \in \N}{\Norm{\phi_n}{\infty}} < \infty,
						\quad \lim_{n \to \infty} \phi_n(x) = 1 \quad (\forall x \in \R^d), \\
						&\quad \Norm{T_{\phi_n}u - u}{} \longrightarrow 0
							\quad (n \longrightarrow \infty,\ \forall u \in H).
					\end{align}
			\end{description}
			このとき次の表現を持つスペクトル測度$E:\borel{\R^d} \rightarrow \Oproj{H}$が唯一つ存在する:
			\begin{align}
				T_f = \int_{\R^d} f(x)\ E(dx) \quad (\forall f \in \cvan{\R^d}).
				\label{eq:thm_pectral_decomposition_of_bounded_linear_operators_0}
			\end{align}
			\label{thm:spectral_decomposition_of_bounded_linear_operators}
		\end{thm}
	\end{screen}
	
	定理\ref{thm:spectral_decomposition_of_bounded_linear_operators}において
	$T$の条件(3)が要求されていないとする.
	この場合$T$を零写像としても(1)(2)は満たされるが,
	\begin{align}
		E(\R^d)u = \lim_{n \to \infty} \int_{\R^d} \phi_n(x)\ E(dx) u = 0 \quad (\forall u \in H)
	\end{align}
	が従い$E(\R^d) = I$に反する.
	
	\begin{prf}\mbox{}
		\begin{description}
			\item[第一段]
				$E$の存在を示す.任意の$u,v \in H$に対し
				\begin{align}
					S_{u,v}:\cvan{\R^d} \ni f \longmapsto \inprod<T_f u,v> \in \C
				\end{align}
				と定めれば$S_{u,v} \in \cvan{\R^d}^*$となる.実際$T,T_f$及び内積の線型性より,任意の$f,g \in \cvan{\R^d},\alpha,\beta \in \C$に対して
				\begin{align}
					S_{u,v} (\alpha f + \beta g) = \inprod<T_{\alpha f + \beta g} u,v>
					= \inprod<(\alpha T_f + \beta T_g) u,v>
					= \alpha S_{u,v} f + \beta S_{u,v} g
				\end{align}
				が成り立つから$S_{u,v}$の線型性が従い,またSchwartzの不等式と$T,T_f$の有界性より
				\begin{align}
					\left| S_{u,v}f \right| \leq \Norm{T_f u}{} \Norm{v}{} \leq \Norm{T}{\Bop{\cvan{\R^d}}{\selfBop{H} } } \Norm{u}{} \Norm{v}{} \Norm{f}{\infty}
					\quad (\forall f \in \cvan{\R^d})
					\label{eq:thm_spectral_decomposition_of_bounded_linear_operators_5}
				\end{align}
				も成り立つから$S_{u,v}$の有界性が従う.よって定理\ref{thm:complex_measure_riesz_representation_theorem}より
				或る$\mu_{u,v}$が唯一つ対応し
				\begin{align}
					\inprod<T_f u,v> = \int_{\R^d} f(x)\ \mu_{u,v}(dx) \quad (\forall f \in \cvan{\R^d})
					\label{eq:thm_spectral_decomposition_of_bounded_linear_operators_3}
				\end{align}
				と表現できる.このとき任意に$\Lambda \in \borel{\R^d}$に取り固定すれば,対応
				\begin{align}
					H \times H \ni [u,v] \longmapsto \mu_{u,v}(\Lambda)
					\label{eq:thm_spectral_decomposition_of_bounded_linear_operators_1}
				\end{align}
				は準双線型であり
				\begin{align}
					\left| \mu_{u,v}(\Lambda) \right| \leq \Norm{T}{\Bop{\cvan{\R^d}}{\selfBop{H} } } \Norm{u}{} \Norm{v}{}
					\quad (\forall u,v \in H)
					\label{eq:thm_spectral_decomposition_of_bounded_linear_operators_2}
				\end{align}
				を満たすから,或る$E(\Lambda) \in \selfBop{H} $が唯一つ存在して
				\begin{align}
					\inprod<E(\Lambda) u, v> = \mu_{u,v}(\Lambda) \quad (\forall u,v \in H)
					\label{eq:thm_spectral_decomposition_of_bounded_linear_operators_6}
				\end{align}
				が成り立つ.以下,上に列記した事柄を証明する.
				\begin{description}
					\item[(\refeq{eq:thm_spectral_decomposition_of_bounded_linear_operators_1})の準双線型性]
						先ず任意の$v \in H$に対し$u \longmapsto \mu_{u,v}(\Lambda)$が線型であることを示す.
						任意に$u,w \in H,\alpha,\beta \in \C$と$f \in \cvan{\R^d}$を取れば,$T_f$の線型性と
						(\refeq{eq:thm_spectral_decomposition_of_bounded_linear_operators_3})の表現
						及び定理\ref{thm:linearity_of_integral_respect_to_measure}より
						\begin{align}
							&\int_{\R^d} f(x)\ \mu_{\alpha u + \beta w,v}(dx)
							= \inprod<T_f (\alpha u + \beta w),v> \\
							&\qquad = \alpha \inprod<T_f u,v> + \beta \inprod<T_f w,v>
							= \int_{\R^d} f(x)\ \left( \alpha \mu_{u,v} + \beta \mu_{w,v} \right)(dx)
							\label{eq:thm_spectral_decomposition_of_bounded_linear_operators_4}
						\end{align}
						が成り立つ.$\R^d$の任意の開集合$A$に対し$|f_n| \leq 1$且つ$f_n \longrightarrow \defunc_A\ $(各点)を満たす
						$\cvan{\R^d}$の列$(f_n)_{n=1}^{\infty}$が存在するから,
						定理\ref{eq:lebesgue_convergence_theorem_complex_measure}より
						\begin{align}
							\mu_{\alpha u + \beta w,v}(A)
							= \lim_{n \to \infty} \int_{\R^d} f_n(x)\ \mu_{\alpha u + \beta w,v}(dx)
							= \lim_{n \to \infty} \int_{\R^d} f_n(x)\ \left( \alpha \mu_{u,v} + \beta \mu_{w,v} \right)(dx)
							= \left( \alpha \mu_{u,v} + \beta \mu_{w,v} \right)(A)
						\end{align}
						が従い,定理\ref{thm:identity_theorem_of_complex_measures}より
						$\mu_{\alpha u + \beta w,v} = \alpha \mu_{u,v} + \beta \mu_{w,v}$が得られる.
						$v \longmapsto \mu_{u,v}(\Lambda)$についても同様であるが,内積の準双線型性より
						(\refeq{eq:thm_spectral_decomposition_of_bounded_linear_operators_4})の
						スカラーが共役に替わり
						$\mu_{u,\alpha v + \beta w} = \conj{\alpha} \mu_{u,v} + \conj{\beta} \mu_{u,w}\ (\forall u,v,w \in H,\alpha,\beta \in \C)$が従う.
						
					\item[(\refeq{eq:thm_spectral_decomposition_of_bounded_linear_operators_2})の証明]
						定理\ref{thm:complex_measure_riesz_representation_theorem}の等長性と
						(\refeq{eq:thm_spectral_decomposition_of_bounded_linear_operators_5})
						(\refeq{eq:thm_spectral_decomposition_of_bounded_linear_operators_3})より,
						任意の$u,v \in H$に対して次が成り立つ:
						\begin{align}
							\Norm{\mu_{u,v}}{\CM} 
							= \sup{\substack{f \in \cvan{\R^d} \\ \Norm{f}{\infty} \leq 1}}{\left| \int_{\R^d} f(x)\ \mu_{u,v}(dx) \right|}
							= \sup{\substack{f \in \cvan{\R^d} \\ \Norm{f}{\infty} \leq 1}}{\left| \inprod<T_f u,v> \right|}
							\leq \Norm{T}{\Bop{\cvan{\R^d}}{\selfBop{H} } } \Norm{u}{} \Norm{v}{}.
						\end{align}
						
					\item[$E(\Lambda)$の存在と一意性]
						上の結果より任意の$u \in H$に対して$Q_u:H \ni v \longmapsto \mu_{u,v}(\Lambda)$は$\conj{Q_u} \in H^*$を満たすから,
						Hilbert空間におけるRieszの表現定理より或る$a(\Lambda)_u \in H$が唯一つ存在し
						\begin{align}
							\conj{Q_u(v)} = \inprod<v,a(\Lambda)_u> = \conj{\inprod<a(\Lambda)_u,v>} \quad (\forall v \in H)
						\end{align}
						を満たす.写像$H \ni u \longmapsto a(\Lambda)_u \in H$を$E(\Lambda)$と表せば
						\begin{align}
							\inprod<E(\Lambda)u,v> = Q_u(v) = \mu_{u,v}(\Lambda) \quad (\forall u,v \in H)
						\end{align}
						が成り立つ.次に$E(\Lambda)$の一意性を示す.或る$F(\Lambda):H \rightarrow H$が存在し
						\begin{align}
							\inprod<F(\Lambda)u,v> = \mu_{u,v}(\Lambda) \quad (\forall u,v \in H)
						\end{align}
						を満たすなら
						\begin{align}
							\inprod<\left( E(\Lambda) - F(\Lambda) \right)u,v> = 0
							\quad (\forall u,v \in H)
						\end{align}
						となるから,特に$v = \left( E(\Lambda) - F(\Lambda) \right)u$として
						\begin{align}
							\left( E(\Lambda) - F(\Lambda) \right)u = 0 \quad (\forall u \in H)
						\end{align}
						が従い$E(\Lambda) = F(\Lambda)$が得られる.
						
					\item[$E(\Lambda)$の線型有界性]
						(\refeq{eq:thm_spectral_decomposition_of_bounded_linear_operators_6})と
						(\refeq{eq:thm_spectral_decomposition_of_bounded_linear_operators_1})の準双線型性より,
						任意の$u,v,w \in H$と$\alpha,\beta \in \C$に対して
						\begin{align}
							\inprod<E(\Lambda)(\alpha u + \beta v),w>
							= \mu_{\alpha u + \beta v, w}(\Lambda)
							= \alpha \mu_{u,w}(\Lambda) + \beta \mu_{v,w}(\Lambda)
							= \alpha \inprod<E(\Lambda)u,w> + \beta \inprod<E(\Lambda)v,w>
						\end{align}
						が成り立つ.特に$w = E(\Lambda)(\alpha u + \beta v) - \alpha E(\Lambda) u - \beta E(\Lambda) v$とすれば
						\begin{align}
							E(\Lambda)(\alpha u + \beta v) - \alpha E(\Lambda) u - \beta E(\Lambda) v = 0
							\quad (\forall u,v \in H,\ \alpha,\beta \in \C)
						\end{align}
						が従い$E(\Lambda)$の線型性を得る.また(\refeq{eq:thm_spectral_decomposition_of_bounded_linear_operators_2})より
						任意の$u,v \in H$に対して
						\begin{align}
							\left| \inprod<E(\Lambda)u,v> \right| \leq \Norm{T}{\Bop{\cvan{\R^d}}{\selfBop{H} } } \Norm{u}{} \Norm{v}{}
						\end{align}
						が成り立つから,特に$v = E(\Lambda) u$とすれば
						\begin{align}
							\Norm{E(\Lambda)u}{} \leq \Norm{T}{\Bop{\cvan{\R^d}}{\selfBop{H} } } \Norm{u}{}
							\quad (\forall u \in H)
						\end{align}
						が得られ$E(\Lambda)$の有界性が従う.
				\end{description}
			
			\item[第二段] 任意の$\Lambda \in \borel{\R^d}$に対して$E(\Lambda)$が直交射影であることを示す.
				前段で$E(\Lambda) \in \selfBop{H} $が示されたから,
				命題\label{prp:orthogonal_projection_idempotent_self_adjoint}より後は
				$E(\Lambda)^2 = E(\Lambda)$と$E(\Lambda)^* = E(\Lambda)$を示せばよい.
				$T_f^* = T_{\conj{f}}\ (\forall f \in \cvan{\R^d})$の仮定より
				\begin{align}
					\inprod<T_f u,v> = \inprod<u,T_f^*v> = \inprod<u,T_{\conj{f}}v> \quad (\forall u,v \in H)
				\end{align}
				が成り立つから,(\refeq{eq:thm_spectral_decomposition_of_bounded_linear_operators_3})より
				\begin{align}
					\int_{\R^d} f(x)\ \mu_{u,v}(dx) = \inprod<T_f u,v> = \conj{\inprod<T_{\conj{f}}v,u>}
					= \int_{\R^d} f(x)\ \conj{\mu_{v,u}}(dx)
					\quad (\forall f \in \cvan{\R^d})
				\end{align}
				が従う.前段で(\refeq{eq:thm_spectral_decomposition_of_bounded_linear_operators_1})の準双線型性
				を示した時と同様にして$\mu_{u,v} = \conj{\mu_{v,u}}$が成り立つから
				\begin{align}
					\inprod<E(\Lambda)u,v> = \mu_{u,v}(\Lambda)
					= \conj{\mu_{v,u}(\Lambda)}
					= \conj{\inprod<E(\Lambda)v,u>}
					= \inprod<u,E(\Lambda)v>
					\quad (\forall u,v \in H)
				\end{align}
				となり$E(\Lambda)^* = E(\Lambda)$が得られる.
				次に$\Lambda$が開集合であるとする.
				$0 \leq f_1 \leq f_2 \leq \cdots \leq 1$かつ$\defunc_\Lambda$に各点収束する
				関数列$(f_n)_{n=1}^{\infty} \subset \cvan{\R^d}$を取れば,
				(\refeq{eq:thm_spectral_decomposition_of_bounded_linear_operators_3})と
				(\refeq{eq:thm_spectral_decomposition_of_bounded_linear_operators_6})より
				\begin{align}
					\inprod<T_{f_n}u,u> = \int_{\R^d} f_n(x)\ \mu_u(dx)
					\leq \int_{\R^d} \defunc_\Lambda(x)\ \mu_u(dx) = \inprod<E(\Lambda)u,u>
					\quad (\forall u \in H,\ n=1,2,\cdots)
				\end{align}
				が成り立ち,更に$\mu_u(\Lambda) < \infty\ (\forall u \in H)$であるからLebesgueの収束定理より
				\begin{align}
					\inprod<\left( E(\Lambda) - T_{f_n} \right)u,u>
					= \int_{\R^d} \defunc_{\Lambda}(x) - f_n(x)\ \mu_u(dx) 
					\longrightarrow 0 \quad (n \longrightarrow \infty,\ \forall u \in H)
				\end{align}
				となる.$\Norm{T_{f_n} u}{} \leq \Norm{T}{\Bop{\cvan{\R^d}}{\selfBop{H} } } \Norm{u}{}$より
				\begin{align}
					\Norm{E(\Lambda) - T_{f_n}}{\selfBop{H} }
					\leq \Norm{E(\Lambda)}{\selfBop{H} } + \Norm{T}{\Bop{\cvan{\R^d}}{\selfBop{H} } }
					\quad (\forall n=1,2,\cdots)
				\end{align}
				が成り立つから,$C \coloneqq \Norm{E(\Lambda)}{\selfBop{H} } + \Norm{T}{\Bop{\cvan{\R^d}}{\selfBop{H} } }$とおけば
				\begin{align}
					\Norm{\left( E(\Lambda) - T_{f_n} \right)u}{}
					\leq \sqrt{C\inprod<\left( E(\Lambda) - T_{f_n} \right)u,u>}
					\longrightarrow 0 \quad (n \longrightarrow \infty,\ \forall u \in H)
				\end{align}
				が従う.またこれは$g_n \coloneqq f_n^2$に対しても成り立つ.ゆえに
				\begin{align}
					\left| \inprod<E(\Lambda)u,E(\Lambda)u> - \inprod<E(\Lambda)u,u> \right|
					\leq \left| \inprod<E(\Lambda)u,E(\Lambda)u> - \inprod<T_{f_n} u,T_{f_n} u> \right|
						+ \left| \inprod<T_{f_n^2} u,u> - \inprod<E(\Lambda)u,u> \right|
					\longrightarrow 0 \quad (n \longrightarrow \infty)
				\end{align}
				となり$E(\Lambda)^* = E(\Lambda)$と併せて$E(\Lambda)^2 = E(\Lambda)$が得られる.
				
			\item[第三段] $E(\R^d) = I$を示す.任意の$n \in \N$に対して
				\begin{align}
					\inprod<\phi_n u,v> = \int_{\R^d} \phi_n(x)\ \mu_{u,v}(dx)
				\end{align}
				が成り立つ.また仮定より任意の$u \in H$に対して
				\begin{align}
					\left| \inprod<\phi_n u,v> - \inprod<u,v> \right|
					\leq \Norm{\phi_n u - u}{} \Norm{v}{}
					\longrightarrow 0 \quad (n \longrightarrow \infty)
				\end{align}
				かつ
				\begin{align}
					\int_{\R^d} \phi_n(x)\ \mu_{u,v}(dx) \longrightarrow \mu_{u,v}(\R^d) = \inprod<E(\R^d)u,v>
				\end{align}
				が成り立つから,
				\begin{align}
					\left| \inprod<u,v> - \inprod<E(\R^d)u,v> \right|
					\leq \left| \inprod<u,v> - \inprod<\phi_n u,v> \right|
						+ \left| \int_{\R^d} \phi_n(x)\ \mu_{u,v}(dx) - \inprod<E(\R^d)u,v> \right|
					\longrightarrow 0 \quad (n \longrightarrow \infty)
				\end{align}
				となり$\inprod<u,v> = \inprod<E(\R^d)u,v>\ (\forall u,v \in H)$が成り立つ.特に$v = u - E(\R^d)u$とすれば
				$u = E(\R^d)u\ (\forall u \in H)$が従い$E(\R^d) = I$を得る.
				
			\item[第四段] $E$の完全加法性を示す.
				任意の$A,B \in \borel{\R^d},\ u,v \in H$に対し
				\begin{align}
					\inprod<\left(E(A+B) - E(A) - E(B) \right)u,v>
					&= \inprod<E(A+B)u,v> - \inprod<E(Au,v> - \inprod<E(B)u,v> \\
					&= \mu_{u,v}(A+B) - \mu_{u,v}(A) - \mu_{u,v}(B)
					= 0
				\end{align}
				が成り立つから
				\begin{align}
					E(A+B)u = E(A)u + E(B)u 
					\quad (\forall u \in H)
				\end{align}
				が得られる.次に任意に互いに素な列$\Lambda_1,\Lambda_2,\cdots \in \borel{\R^d}$を取り
				$\Lambda = \sum_{n=1}^{\infty} \Lambda_n,\ \Lambda_N \coloneqq \sum_{n=1}^{N} \Lambda_n$とおけば,
				\begin{align}
					\Norm{\left( E(\Lambda) - E(\Lambda_N) \right)u}{}
					\leq \sqrt{\inprod<\left( E(\Lambda) - E(\Lambda_N) \right)u, u>}
					= \leq \sqrt{\mu_u(\Lambda) - \mu_u(\Lambda_N)}
					\longrightarrow 0 \quad (N \longrightarrow \infty,\ \forall u \in H)
				\end{align}
				が成り立ち,一方で
				\begin{align}
					E(\Lambda_N) u = \sum_{n=1}^{N} E(\Lambda_n) u \quad (\forall N \in \N,\ u \in H)
				\end{align}
				となるから,命題\ref{prp:orthogonal_projection_product_sum}より
				任意の$i,j \in \N$に対して$E(\Lambda_i)E(\Lambda_j) = \delta_{ij}E(\Lambda_i)$が成り立ち
				$\sum_{n=1}^{\infty} E(\Lambda_n) u$は絶対収束する.
				\begin{align}
					E(\Lambda) u = \sum_{n=1}^{\infty} E(\Lambda_n) u \quad (\forall u \in H)
				\end{align}
				が得られる.
			
			\item[第五段]
				(\refeq{eq:thm_pectral_decomposition_of_bounded_linear_operators_0})を示す.
				$f \in \cvan{\R^d}$が単関数の場合は
				\begin{align}
					\int_{\R^d} f_n(x)\ \mu_{u,v}(dx) = \inprod<\int_{\R^d} f_n(x)\ E(dx) u, v>
					\quad (\forall u,v \in H)
				\end{align}
				が成り立つ.一般の$f \in \cvan{\R^d}$に対しては,$MSF$-単調近似列$(f_n)_{n=1}^{\infty}$を取れば
				\begin{align}
					&\left| \inprod<T_f u,v> - \inprod<\int_{\R^d} f(x)\ E(dx)\ u, v> \right|
					= \left| \int_{\R^d} f(x)\ \mu_{u,v}(dx) - \inprod<\int_{\R^d} f(x)\ E(dx)\ u, v> \right| \\
					&\qquad \leq \left| \int_{\R^d} f(x)\ \mu_{u,v}(dx) - \int_{\R^d} f_n(x)\ \mu_{u,v}(dx) \right| 
						+ \Norm{\int_{\R^d} f_n(x)\ E(dx) u - \int_{\R^d} f(x)\ E(dx) u}{} \Norm{v}{}
					\longrightarrow 0 \quad (n \longrightarrow \infty)
				\end{align}
				が成り立つ.
				
			\item[第六段]
				$E$の一意性を示す.開集合の上で$E$は一意.一致の定理より$E$は一意.
			\QED
		\end{description}
	\end{prf}
	
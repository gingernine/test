\section{複素測度のRieszの表現定理}
	\begin{screen}
		\begin{dfn}[空間$C_\infty$]
			局所コンパクトなHausdorff空間$X$に対し$\c{X} \coloneqq \Set{f:X \rightarrow \C}{連続}$とおく.
			\begin{align}
				\cvan{X} \coloneqq
				\Set{f \in \c{X}}{\mbox{任意の$\epsilon > 0$に対して$\Set{x \in X}{|f(x)| \geq \epsilon}$がコンパクト.}}
			\end{align}
			として$\cvan{X}$を定め,またコンパクトな台を持つ$f \in \c{X}$の全体を$\ckon{X}$と表す.
		\end{dfn}
	\end{screen}
	
	$f \in \cvan{X}$は遠方で0になる関数である.特に$X = \R^d$の場合は
	\begin{align}
		\cvan{\R^d} = \Set{f:\R^d \rightarrow \C}{\lim_{|x| \to \infty}|f(x)| = 0}
	\end{align}
	が成り立つ.
	
	\begin{screen}
		\begin{thm}[$C_c$は$C_\infty$で稠密]
			
		\end{thm}
	\end{screen}
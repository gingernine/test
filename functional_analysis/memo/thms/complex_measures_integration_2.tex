\section{Rieszの表現定理}
	\begin{screen}
		\begin{dfn}[記号の定義]
			位相空間$X$に対し$\c{X} \coloneqq \Set{f:X \rightarrow \C}{連続}$とおく.先ず
			\begin{align}
				\ckon{X} &\coloneqq
				\Set{f \in \c{X}}{\mbox{$\supp{f}$がコンパクト.}}, \\
				\cvan{X} &\coloneqq
				\Set{f \in \c{X}}{\mbox{任意の$\epsilon > 0$に対して$\Set{x \in X}{|f(x)| \geq \epsilon}$がコンパクト.}}
			\end{align}
			により二つの空間を定め,そして開集合$V$,コンパクト集合$K$,$f \in \ckon{X}$に対し
			\begin{align}
				f \prec V &\quad \DEF \quad 0 \leq f \leq 1, \quad \supp{f} \subset V, \\
				K \prec f &\quad \DEF \quad 0 \leq f \leq 1, \quad f(x) = 1 \ (\forall x \in K)
			\end{align}
			により二項関係$\prec$を定める.
		\end{dfn}
	\end{screen}
	
	特に$X = \R^d$の場合,$f \in \cvan{X}$は原点を中心にして遠方で0になる関数である.実際
	\begin{align}
		\cvan{\R^d} = \Set{f:\R^d \rightarrow \C}{\lim_{|x| \to \infty}|f(x)| = 0}
	\end{align}
	が成り立つ.
	
	
	\begin{screen}
		\begin{thm}[$C_c$は$C_0$で稠密]
		\end{thm}
	\end{screen}
	
	\begin{screen}
		\begin{lem}
			$X$を局所コンパクトなHausdorff空間,$K$を$X$のコンパクト集合,
			$V$を$X$の開集合とすると,或る閉包がコンパクトな開集合$U$が次を満たす:
			\begin{align}
				K \subset U \subset \closure{U} \subset V.
			\end{align}
		\end{lem}
	\end{screen}
	
	\begin{prf}
		
	\end{prf}
	
	\begin{screen}
		\begin{lem}[コンパクト集合上の一の分割]
			$X$を局所コンパクトなHausdorff空間,$K$を$X$のコンパクト集合,
			$V_1,\cdots, V_n$を$X$の開集合とし
			$K \subset V_1 \cup \cdots \cup V_n$を満たすものとする.
			このとき或る$h_1,\cdots,h_n \in \ckon{X}$が存在して
			\begin{align}
				h_i \prec V_i,  \quad (i=1,\cdots,n),
				\quad h_1(x) + \cdots + h_n(x) = 1, \quad (\forall x \in K)
			\end{align}
			を満たす.
		\end{lem}
	\end{screen}
	
	\begin{prf}
		
	\end{prf}
		
	\begin{screen}
		\begin{thm}[Rieszの表現定理]
			$X$を局所コンパクトなHausdorff空間とする.各$\mu \in \CM = \CM(X,\mathcal{M})$に対し
			\begin{align}
				\Psi_\mu: \cvan{X} \ni f \longmapsto \int_X f(x)\ \mu(dx)
				\label{eq:thm_complex_measure_riesz_representation_theorem_1}
			\end{align}
			で定める$\Psi_\mu$は$\cvan{X}^*$の元であり,次で定める写像
			\begin{align}
				\Psi: \CM \ni \mu \longmapsto \Psi_\mu \in \cvan{X}^*
			\end{align}
			は$\CM$から$\cvan{X}^*$へのBanach空間としての等長同型写像である.
			\label{thm:complex_measure_riesz_representation_theorem}
		\end{thm}
	\end{screen}
	
	\begin{prf}
		任意に$\Phi \in \cvan{X}^*$を取る.このとき$\Phi = \Psi_\mu$
		を満たす複素測度$\mu$がただ一つだけ存在することを示す.
		\begin{description}
			\item[第一段] $\mu$の一意性を示す.
			\item[第二段] $\mu$の存在を示す.
				次を満たす$\cvan{X}$上の正値有界線形汎関数$\Lambda$を構成する:
				\begin{align}
					|\Phi f| \leq \Lambda |f| \leq \Norm{\Phi}{} \Norm{f}{\cvan{X}}
					\quad (\forall f \in \cvan{X}).
				\end{align}
				実際これを満たす$\Lambda$が存在すれば,
				$\Lambda$により導入する$\mathcal{M}$上の正値測度$\lambda$は
				\begin{align}
					\lambda(X)
					= \sup{}{\Set{\Lambda f}{f \prec X}}
				\end{align}
				を満たすから
				\begin{align}
					\lambda(X) \leq \Norm{\Phi}{}
				\end{align}
				が従う.また
				\begin{align}
					\Lambda f = \int_X f(x)\ \lambda(dx)
					\quad (\forall f \in \ckon{X})
				\end{align}
				の関係から
				\begin{align}
					|\Phi f| \leq \Lambda |f| = \Norm{f}{\mathrm{L}^1(\lambda)}
					\quad (\forall f \in \ckon{X})
				\end{align}
				が得られ,$\ckon{X}$は$\cvan{X}$においてsup-normで稠密だから
				\begin{align}
					|\Phi f| \leq \Norm{f}{\mathrm{L}^1(\lambda)}
					\quad (\forall f \in \cvan{X})
				\end{align}
				が成立する.実際任意の$f \in \cvan{X}$に対して
				\begin{align}
					&\bigl| |\Phi f| - |\Phi f_n| \bigr|
					\leq \Norm{\Phi}{} \Norm{f - f_n}{\cvan{X}}
					\longrightarrow 0 \quad (n \longrightarrow \infty), \\
					&\left| \Norm{f}{\mathrm{L}^1(\lambda)} - \Norm{f_n}{\mathrm{L}^1(\lambda)} \right| \leq \lambda(X) \Norm{f - f_n}{\cvan{X}}
					\longrightarrow 0 \quad (n \longrightarrow \infty)
				\end{align}
				を満たすように$f_n \in \ckon{X}$を選ぶことができる.
				また$\cvan{X}$は$\mathrm{L}^1(\lambda)$において
				稠密であるから,$\Phi$は$\mathrm{L}^1(\lambda)$上の
				有界線形作用素$\tilde{\Phi}$にノルム保存拡張される.
				従って或る$g \in \mathrm{L}^\infty(\lambda)$が存在して
				\begin{align}
					\tilde{\Phi}f = \int_X f(x) g(x)\ \lambda(dx)
					\quad (\forall f \in \mathrm{L}^1(\lambda))
				\end{align}
				を満たす.このとき任意の$E \in \mathcal{M}$に対して,
				$\lambda$の有限性より$\defunc_E \in \mathrm{L}^1(\lambda)$であるから
				\begin{align}
					\left| \int_E g(x)\ \lambda(dx) \right|
					= \left| \tilde{\Phi} \defunc_E \right|
					\leq \Norm{\defunc_E}{\mathrm{L}^1(\lambda)}
					= \lambda(E)
				\end{align}
				が成り立ち$|g| \leq 1$が得られる.従って$d\mu \coloneqq g d\lambda$により
				$\mu \in \CM$を定めれば
				\begin{align}
					|\mu|(X) = \int_X |g(x)|\ \lambda(dx) \leq \lambda(X) \leq \Norm{\Phi}{}
				\end{align}
				が出る.一方で
				\begin{align}
					|\mu|(X) = \int_X |g(x)|\ \lambda(dx)
					\geq \sup{}{\Set{|\tilde{\Phi} f|}{ 0 \leq f \leq 1,\ f \in \mathrm{L}^1(\lambda)}} = \Norm{\tilde{\Phi}}{} = \Norm{\Phi}{}
				\end{align}
				も成り立つ.
				
			\item[第三段] $\Psi$の線形等長性を示す.
		\end{description}
	\end{prf}
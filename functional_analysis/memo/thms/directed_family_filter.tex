\section{有向族とフィルター}
	\begin{screen}
		\begin{thm}[有向点族による連続性の特徴づけ]
			$X,Y$を位相空間とする.
			写像$f:X \longrightarrow Y$が点$x \in X$で連続
			であるための必要十分条件について,次がいえる:
			\begin{description}
				\item[(1)] $x$が可算な基本近傍系をもつとき,
					必要十分条件は,$x$に収束する任意の点列$(x_n)_{n=1}^{\infty}$に対し
					$\left( f(x_n) \right)_{n=1}^\infty$
					が$f(x)$に収束することである.
					
				\item[(2)] $X$が一般の位相空間の場合,
					必要十分条件は,$x$に収束する任意の有向点族$(x_\lambda)_{\lambda \in \Lambda}$に対し
					$\left( f(x_\lambda) \right)_{\lambda \in \Lambda}$
					が$f(x)$に収束することである.
			\end{description}
		\end{thm}
	\end{screen}
	
	\begin{prf}\mbox{}
		\begin{description}
			\item[必要性]
				$x$に収束する任意の点列$(x_n)_{n=1}^{\infty}$は有向点族であるから,
				必要性の証明は(2)に対して示せばよい.
				$(x_\lambda)_{\lambda \in \Lambda}$を$x$に収束する有向点族とする.
				$f$が$x$で連続であるとき,$f(x)$の任意の近傍$V$に対して或る
				$x$の近傍$U$が存在し
				\begin{align}
					f(U) \subset V
				\end{align}
				が満たされる.この$U$に対し或る$\lambda_0 \in \Lambda$が存在して
				$x_\lambda \in U\ (\forall \lambda \geq \lambda_0)$が成り立ち
				$f(x_\lambda) \in V\ (\forall \lambda \geq \lambda_0)$が従う.
				
			\item[十分性] (1)と(2)それぞれの場合について,対偶を証明する.
				いま,$f$が$x$で連続ではないと仮定する.
				\begin{description}
					\item[(1)]
						$x$に対し可算基本近傍系$(U_n)_{n=1}^{\infty}$が存在する.
						近傍系は有限回の交演算で閉じるから
						\begin{align}
							W_n \coloneqq U_1 \cap U_2 \cap \cdots \cap U_n,
							\quad (n=1,2,\cdots)
						\end{align}
						により$x$の単調減少な可算近傍系$(W_n)_{n=1}^{\infty}$が定まり,
						仮定より$f(x)$の或る近傍$V$が存在して
						\begin{align}
							f(W_n) \not\subset V,
							\quad (n=1,2,\cdots)
						\end{align}
						が成り立つから,各$n$に対し$f(x_n) \notin V$を満たす
						$x_n \in W_n$が取れる.ゆえに$f(x_n) \not\rightarrow f(x)$であるが,
						一方で$x$の任意の近傍$U$に対し或る
						$U_{n_0}$が$U$に含まれ,
						$x_n \in W_n \subset U\ (\forall n \geq n_0)$が従い
						$(x_n)_{n=1}^{\infty}$は$x$に収束する.
					\item[(2)]
				\end{description}
		\end{description}
	\end{prf}
	
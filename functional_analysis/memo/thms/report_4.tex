%レポート問題4
	\begin{prf}\mbox{}
		\begin{description}
			\item[点スペクトルについて]
				$(\lambda I - T) u = 0$を満たす$\lambda$に対し,微分方程式を解けば
				\begin{align}
					u(x) = C\exp{\lambda x}
					\quad (x \in I,\ C \in \C)
				\end{align}
				と表せる.今$u(0) + u(a) = 0$が仮定されているから,
				\begin{align}
					C + C\exp{\lambda a} = 0
				\end{align}
				が成り立つ.これは$C = 0$或は
				$\lambda = \frac{1}{a} \log{(-1)}$の場合に実現する.
				$\lambda = \frac{1}{a} \log{(-1)}$でなければ$C = 0$でなくてはならないが,
				このとき$u = 0$となり固有ベクトルにならないから$\lambda = \frac{1}{a} \log{(-1)}$
				でなくてはならない.従って
				\begin{align}
					\pSpctr{T} \subset \Set{\sqrt{-1} \frac{(2 n + 1)\pi}{a}}{n \in \Z}
				\end{align}
				が得られる.逆に或る$n \in \Z$に対して$\lambda = (2 n + 1)\pi/a$と表されているとき,
				任意の$C \in \C$に対して$u(x) = C\exp{\lambda x}\ (x \in I)$とおけば
				\begin{align}
					\lambda u(x) - T u(x) = 0 \quad (\forall x \in I),
					\quad u(0) + u(a) = 0
				\end{align}
				が満たされるから
				\begin{align}
					\pSpctr{T} \supset \Set{\sqrt{-1} \frac{(2 n + 1)\pi}{a}}{n \in \Z}
				\end{align}
				が成り立ち,$\pSpctr{T} = \Set{\sqrt{-1} \frac{(2 n + 1)\pi}{a}}{n \in \Z}$が得られる.
				
			\item[スペクトルについて]
				$\Res{T} = \C \backslash \pSpctr{T} $が成り立つことを示す.これにより
				$\Spctr{T} = \pSpctr{T} $が従う.
				$\lambda \in \C \backslash \pSpctr{T} $を任意に取る.$f \in \c{I}$に対し
				\begin{align}
					\begin{cases}
						u'(x) - \lambda u(x) = f(x) \\
						u(0) + u(a) = 0
					\end{cases}
					\quad (x \in I)
				\end{align}
				を満たす$u$を考えれば,
				\begin{align}
					&\begin{cases}
						u'(x) - \lambda u(x) = f(x) \\
						u(0) + u(a) = 0
					\end{cases}
					\quad (x \in I) \\
					&\quad \Leftrightarrow 
					\begin{cases}
						u(x) = \exp{\lambda x}u_0 + \int_0^x \exp{\lambda (x-s)} f(s)\ ds \\
						u(0) + u(a) = 0
					\end{cases}
					\quad (x \in I) \\
					&\quad \Leftrightarrow 
					\begin{cases}
						u(x) = \exp{\lambda x}u_0 + \int_0^x \exp{\lambda (x-s)} f(s)\ ds \\
						u(0) + \exp{\lambda a}u_0 + \int_0^a \exp{\lambda (a-s)} f(s)\ ds = 0
					\end{cases}
					\quad (x \in I) \\
					&\quad \Leftrightarrow 
					u(x) = -\frac{\exp{\lambda x}}{1 + \exp{\lambda a}} \int_0^a \exp{\lambda (a-s)} f(s)\ ds + \int_0^x \exp{\lambda (x-s)} f(s)\ ds
					\quad (x \in I)
				\end{align}
				より$f$に対して$u$は唯一つ定まる.この対応を$R_\lambda:\c{I} \oparrow \c{I}$と表せば,$\Dom{R_\lambda} = \c{I}$且つ
				積分の線型性より$R_\lambda$線型写像である.また
				\begin{align}
					\Norm{R_\lambda f}{} \leq \Norm{f}{}
				\end{align}
				を満たすから$R_\lambda$は有界で,さらに
				\begin{align}
					R_\lambda (\lambda - T) u &= u \quad (\forall u \in \Dom{T} ), \\
					(\lambda - T) R_\lambda f &= f \quad (\forall f \in \c{I})
				\end{align}
				が成り立つから$R_\lambda = (\lambda - T)^{-1}$が従い$\lambda \in \Res{T} $を得る.
		\end{description}
	\end{prf}
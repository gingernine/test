%レポート問題4
	\begin{prf} sup-normを$\Norm{\cdot}{}$,$\c{I}$上の恒等写像をI$(\neq I)$と表す.
		\begin{description}
			\item[$T$が閉作用素であること]
				先ず$T$が閉作用素であることを示す.
				$u_n \in \Dom{T} \ (n=1,2,\cdots)$に対し或る$u,v \in \c{I}$が存在して,$\Norm{u_n - u}{} \longrightarrow 0$
				かつ$\Norm{T u_n - v}{} \longrightarrow 0$が成り立つとき,任意の$x \in I$に対して
				\begin{align}
					\left| u(x) - \int_0^x v(t)\ dt \right|
					&\leq \left| u(x) - u_n(x) \right| + \left| \int_0^x T u_n(t)\ dt - \int_0^x v(t)\ dt \right| \\
					&\leq \Norm{u - u_n}{} + a \Norm{T u_n - v}{}
					\longrightarrow 0 \quad (n \longrightarrow \infty)
				\end{align}
				が成り立つから
				\begin{align}
					u(x) = \int_0^x v(t)\ dt \quad (\forall x \in I)
				\end{align}
				となり,$u \in \cn{I}{1}$かつ$Tu = v$が従う.そして
				\begin{align}
					\left| u(0) + u(a) \right| \leq \left| u(0) - u_n(0) \right| 
						+ \left| u_n(a) - u(a) \right|
					\leq 2 \Norm{u - u_n}{}
					\longrightarrow 0 \quad (n \longrightarrow \infty)
				\end{align}
				により,$u \in \Dom{T} $が成り立つ.これにより$T$は閉作用素である.
				
			\item[点スペクトルについて]
				$u \in \Dom{T} $とする.$\lambda u - T u = 0$を満たす$\lambda \in \C$に対し,微分方程式を解けば
				\begin{align}
					u(x) = C\exp{\lambda x}
					\quad (x \in I,\ C \in \C)
				\end{align}
				と表せる.今$u(0) + u(a) = 0$が満たされているから,
				\begin{align}
					C + C\exp{\lambda a} = 0
				\end{align}
				が成り立つ.これは$C = 0$或は
				$\lambda \in \Set{\sqrt{-1} (2 n + 1)\pi/a}{n \in \Z}$の場合に実現する.
				$\lambda \notin \Set{\sqrt{-1} (2 n + 1)\pi/a}{n \in \Z}$ならば$C = 0$となり,
				この場合$\lambda u - T u = 0$を満たす$u \neq 0$が存在しないから
				\begin{align}
					\pSpctr{T} \subset \Set{\sqrt{-1} \frac{(2 n + 1)\pi}{a}}{n \in \Z}
				\end{align}
				が従う.逆に$n \in \Z$を取り$\lambda = (2 n + 1)\pi/a$とおけば,
				任意の$0 \neq C \in \C$に対して$u(x) = C\exp{\lambda x}\ (x \in I)$は
				\begin{align}
					\lambda u(x) - T u(x) = 0 \quad (\forall x \in I),
					\quad u(0) + u(a) = 0
				\end{align}
				を満たすから
				\begin{align}
					\pSpctr{T} \supset \Set{\sqrt{-1} \frac{(2 n + 1)\pi}{a}}{n \in \Z}
				\end{align}
				が成り立ち,$\pSpctr{T} = \Set{\sqrt{-1} (2 n + 1)\pi/a}{n \in \Z}$が得られる.
				
			\item[スペクトルについて]
				レゾルベント集合が$\Res{T} = \C \backslash \pSpctr{T} $を満たすことを示す.これにより
				$\Spctr{T} = \pSpctr{T} $が従う.
				$\lambda \in \C \backslash \pSpctr{T} ,f \in \c{I}$を任意に取り
				\begin{align}
					\begin{cases}
						u'(x) - \lambda u(x) = f(x) \\
						u(0) + u(a) = 0
					\end{cases}
					\quad (x \in I)
				\end{align}
				を満たす$u$を考えれば,
				\begin{align}
					&\begin{cases}
						u'(x) - \lambda u(x) = f(x) \\
						u(0) + u(a) = 0
					\end{cases}
					\quad (x \in I) \\
					&\quad \Leftrightarrow 
					\begin{cases}
						u(x) = \exp{\lambda x}u(0) + \int_0^x \exp{\lambda (x-s)} f(s)\ ds \\
						u(0) + u(a) = 0
					\end{cases}
					\quad (x \in I) \\
					&\quad \Leftrightarrow 
					\begin{cases}
						u(x) = \exp{\lambda x}u(0) + \int_0^x \exp{\lambda (x-s)} f(s)\ ds \\
						u(0) + \exp{\lambda a}u(0) + \int_0^a \exp{\lambda (a-s)} f(s)\ ds = 0
					\end{cases}
					\quad (x \in I) \\
					&\quad \Leftrightarrow 
					u(x) = -\frac{\exp{\lambda x}}{1 + \exp{\lambda a}} \int_0^a \exp{\lambda (a-s)} f(s)\ ds + \int_0^x \exp{\lambda (x-s)} f(s)\ ds
					\quad (x \in I) \footnotemark
					\label{eq:report_4_1}
				\end{align}
				より$f$に対して$u \in \Dom{T} $は唯一つ定まる
				\footnotetext{
					$\lambda \in \C \backslash \pSpctr{T} $より$1 + \exp{\lambda a} \neq 0$である.
				}
				.この$f$から$u$への単射対応を$R_\lambda:\c{I} \oparrow \Dom{T} $と表せば,
				$f$の任意性より$\Dom{R_\lambda} = \c{I}$が成り立ち,且つ
				積分の線型性により$R_\lambda$も線型性を持つ.また(\refeq{eq:report_4_1})の最終式より
				\begin{align}
					\Norm{R_\lambda f}{} 
					\leq \left(\frac{\sup{x \in I}{\left| \exp{\lambda x} \right|}}{\left| 1 + \exp{\lambda a} \right|} \int_0^a \left| \exp{\lambda (a-s)} \right|\ ds + \sup{x \in I}{\left| \exp{\lambda x} \right|} \int_0^a \left| \exp{- \lambda s} \right|\ ds \right) \Norm{f}{}
					\quad (\forall f \in \c{I})
				\end{align}
				が成り立つから$R_\lambda$は有界であり,さらに$R_\lambda$の定め方と(\refeq{eq:report_4_1})より
				\begin{align}
					-R_\lambda (\lambda \mathrm{I} - T) u &= u \quad (\forall u \in \Dom{T} ), \\
					-(\lambda \mathrm{I} - T) R_\lambda f &= f \quad (\forall f \in \c{I})
				\end{align}
				が満たされるから$-R_\lambda = (\lambda \mathrm{I} - T)^{-1}$が成り立ち$\lambda \in \Res{T} $が従う.
				以上より$\Res{T} = \C \backslash \pSpctr{T} $である.
				\QED
		\end{description}
	\end{prf}
%レポート問題9
	\begin{prf} $H$のノルムと内積をそれぞれ$\Norm{\cdot}{},\inprod<\cdot,\cdot>$と表す.
		\begin{description}
			\item[(1)] 
				任意の$A \in \borel{\C}$に対し
				$\defunc_{a^{-1}(A)}$は有界であるから,
				春学期のレポート問題より
				$M_{\defunc_{a^{-1}(A)}} \in \selfBop{H} $が成り立つ.
				ゆえに$E$は$\borel{\C}$全体で定義される.次に任意の$A \in \borel{\C}$に対し
				$E(A)$が$H$上の直交射影であることを示す.
				実際
				\begin{align}
					E(A)^2 u = M_{\defunc_{a^{-1}(A)}}M_{\defunc_{a^{-1}(A)}}u
					= \defunc_{a^{-1}(A)}\defunc_{a^{-1}(A)}u
					= \defunc_{a^{-1}(A)}u
					= E(A) u
					\quad (\forall u \in H)
				\end{align}
				により$E(A)^2 = E(A)$が成り立ち,また前問[8]の(2)により
				\begin{align}
					E(A)^* = M_{\defunc_{a^{-1}(A)}}^* = M_{\defunc_{a^{-1}(A)}} = E(A)
				\end{align}
				が成り立つから$E(A)$は自己共役である.従って$E(A)$は
				$H$上の直交射影である.
				最後に$E$がスペクトル測度であることを示す.
				先ず$a^{-1}(\C) = X$より
				\begin{align}
					\left( E(\C)u \right)(x) = \left( M_{\defunc_X}u \right)(x) 
					= \defunc_X(x) u(x) = u(x)
					\quad (\mbox{$\mu$-a.e.}x \in X,\ \forall u \in H)
				\end{align}
				が成り立ち$E(\C) = I$を得る.後は任意の互いに素な集合列
				$A_1,A_2,\cdots \in \borel{\C}$に対して,$A \coloneqq \sum_{n=1}^{\infty}A_n$として
				\begin{align}
					E(A) u = \sum_{n=1}^{\infty} E(A_n) u \quad (\forall u \in H)
					\label{eq:report_9_1}
				\end{align}
				が成立することを示せばよい:
				\begin{description}
					\item[第一段]
						先ず(\refeq{eq:report_9_1})の右辺の級数が$H$で収束することを示す.
						任意に$u \in H$を取り
						\begin{align}
							v_n \coloneqq \sum_{i=1}^{n} E(A_i) u
							\quad (n=1,2,\cdots)
						\end{align}
						として$(v_n)_{n=1}^{\infty} \subset H$を定める.任意の$n \in \N$に対し
						\begin{align}
							&\sum_{i=1}^{n} \Norm{E(A_i) u}{}^2
							= \sum_{i=1}^{n} \int_X \defunc_{a^{-1}(A_i)}(x) \left| u(x) \right|^2\ \mu(dx) \\
							&\qquad = \int_X \defunc_{a^{-1}\left( \sum_{i=1}^{n}A_i \right)}(x) \left| u(x) \right|^2\ \mu(dx)
							\leq \int_X |u(x)|^2\ \mu(dx)
							= \Norm{u}{}^2
						\end{align}
						が満たされるから
						\begin{align}
							\sum_{i=1}^{\infty} \Norm{E(A_i) u}{}^2 \leq \Norm{u}{}^2 < \infty
							\label{eq:report_9_2}
						\end{align}
						が従い,一方で任意の$p,q \in \N,\ p < q$に対し
						\begin{align}
							\Norm{v_q - v_p}{}^2
							= \int_X \left| \sum_{i=p+1}^{q} \defunc_{a^{-1}(A_i)}(x) u(x) \right|^2\ \mu(dx)
							= \int_X \sum_{i=p+1}^{q} \defunc_{a^{-1}(A_i)}(x) |u(x)|^2\ \mu(dx)
							= \sum_{i=p+1}^{q} \Norm{E(A_i) u}{}^2
						\end{align}
						が成り立つから,(\refeq{eq:report_9_2})より$(v_n)_{n=1}^{\infty}$は
						$H$でCauchy列をなし,$H$の完備性により或る$v \in H$に強収束する.
					\item[第二段]
						$E(A)u = v$が成り立つことを示す.
						任意の$n \in \N$に対し
						\begin{align}
							\left| v_n(x) - \left( E(A)u \right)(x) \right| \leq 2|u(x)|
							\quad (\forall x \in X)
						\end{align}
						が満たされ,かつ
						\begin{align}
							v_n(x) \longrightarrow \left( E(A)u \right)(x)
							\quad (n \longrightarrow \infty,\ \forall x \in X)
						\end{align}
						が成り立つからLebesgueの収束定理より
						\begin{align}
							\Norm{E(A)u - v_n}{}^2
							= \int_X \left| \left( E(A)u \right)(x) - v_n(x) \right|^2\ \mu(dx)
							\longrightarrow 0 \quad (n \longrightarrow \infty)
						\end{align}
						が従う.前段の結果と併せれば
						\begin{align}
							\Norm{E(A)u - v}{}
							\leq \Norm{E(A)u - v_n}{} + \Norm{v_n - v}{}
							\longrightarrow 0 \quad (n \longrightarrow \infty)
						\end{align}
						により$E(A)u = v$が成立する.$u$の任意性により(\refeq{eq:report_9_1})が得られる.
				\end{description}
				
			\item[(2)]
				先ず$\Dom{T_f} = \Dom{M_{f \circ a}} $が成り立つことを示す.
				$f$が可測単関数の場合,
				\begin{align}
					f = \sum_{i=1}^{n} \alpha_i \defunc_{A_i}
					\quad \left( \alpha_i \in \C,\ A_i \in \borel{\C}\ (i=1,\cdots,n),\ 
					\C = \sum_{i=1}^{n} A_i \right)
				\end{align}
				と表せば,任意の$u \in H$に対して
				\begin{align}
					\int_\C |f(z)|^2\ \mu_u(dz)
					= \sum_{i=1}^{n} |\alpha_i|^2 \inprod<E(A_i)u,u>
					= \sum_{i=1}^{n} |\alpha_i|^2 \int_X \defunc_{a^{-1}(A_i)}|u(x)|^2\ \mu(dx)
					= \int_X |f(a(x))|^2 |u(x)|^2\ \mu(dx)
				\end{align}
				が成り立つ
				\footnote{
					$u \in H$に対する$\mu_u$は
					\begin{align}
						\mu_u(\Lambda) \coloneqq \inprod<E(\Lambda)u,u>
						\quad (\Lambda \in \borel{\C})
					\end{align}
					により定められる有限正値測度を表す.
				}
				.$f$が一般の可測関数の場合は$f$の$MSF$-単調近似列$(f_n)_{n=1}^{\infty}$を取れば,
				任意の$n \in \N$に対して
				\begin{align}
					\int_\C |f_n(x)|^2\ \mu_u(dx) = \int_X |f_n(a(x))|^2 |u(x)|^2\ \mu(dx)
					\quad (\forall n \in \N)
				\end{align}
				が満たされる.そして単調収束定理より
				\begin{align}
					\int_\C |f(x)|^2\ \mu_u(dx) = \int_X |f(a(x))|^2 |u(x)|^2\ \mu(dx)
				\end{align}
				が成り立つから
				\begin{align}
					u \in \Dom{T_f} \quad \Leftrightarrow \quad u \in \Dom{M_{f \circ a}} 
				\end{align}
				が従う.次に$T_f u = M_{f \circ a} u\ (\forall u \in \Dom{T_f} )$を示す.
				$f$が可測単関数の場合,
				\begin{align}
					T_f u = \sum_{i=1}^n \alpha_i E(A_i) u
					= \sum_{i=1}^n \alpha_i \defunc_{a^{-1}(A_i)} u
					= f_n \circ a u
					= M_{f_n \circ a} u
					\quad (\forall u \in H)
				\end{align}
				が成り立つ.
				一般の$f$に対しては,$MSF$-単調近似列$(f_n)_{n=1}^{\infty}$を取れば任意の$n \in \N$に対して
				\begin{align}
					\Norm{T_f u - M_{f \circ a} u}{}
					\leq \Norm{T_f u - T_{f_n} u}{}
						+ \Norm{M_{f_n \circ a} u - M_{f \circ a} u}{}
				\end{align}
				が成立する.スペクトル積分$T_f$の定義より
				\begin{align}
					\Norm{T_f u - T_{f_n} u}{}
					\longrightarrow 0 \quad (n \longrightarrow \infty)
				\end{align}
				が満たされ,またLebesgueの収束定理より
				\begin{align}
					\Norm{M_{f_n \circ a} u - M_{f \circ a} u}{}^2
					= \int_X \left| f_n(a(x)) u(x) - f(a(x)) u(x) \right|^2\ \mu(dx)
					\longrightarrow 0 
					\quad (n \longrightarrow \infty)
				\end{align}
				も成り立つから
				\begin{align}
					\Norm{T_f u - M_{f \circ a} u}{} = 0
					\quad (\forall u \in \Dom{T_f} )
				\end{align}
				が従い$T_f = M_{f \circ a}$が得られる.
				\QED
		\end{description}
	\end{prf}
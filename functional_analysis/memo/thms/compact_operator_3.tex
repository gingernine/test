\section{コンパクト自己共役作用素のスペクトル分解}
	$H$を複素Hilbert空間とし内積を$\inprod<\cdot,\cdot>$,ノルムを$\Norm{\cdot}{}$と表す.
	
	\begin{screen}
		\begin{dfn}[自己共役作用素]
			$H$上の線型作用素$A$が
			$\closure{\Dom{A} } = H$かつ$A = A^*$を満たすとき,
			$A$を自己共役作用素(self adjoint operator)という.
			自己共役作用素は閉作用素である.
		\end{dfn}
	\end{screen}
	
	\begin{screen}
		\begin{thm}
			$H$上の線型作用素$A$が自己共役なら,
			任意の$u \in \Dom{A} $に対し$\inprod<Au,u>$は実数値である.
		\end{thm}
	\end{screen}
	
	\begin{prf}
		$A = A^*$であるから$u \in \Dom{A} \Leftrightarrow u \in \Dom{A^*} $となり
		\begin{align}
			\inprod<Au,u> = \inprod<u,A^*u> = \inprod<u,Au> = \conj{\inprod<Au,u>}
			\quad \left(\forall u \in \Dom{A} \right)
			\label{eq:self_adjoint_1}
		\end{align}
		が成り立つ.
		\QED
	\end{prf}
	
	\begin{screen}
		\begin{prp}[自己共役作用素のスペクトルは実数]
			$A$を$H$上の自己共役作用素とする.
			\begin{description}
				\item[(1)] $\Spctr{A} \subset \R$であり,かつ任意の$\lambda \in \C \backslash \R$に対し次が満たされる:
					\begin{align}
						\Norm{(\lambda I - A)^{-1}}{\selfBop{H} } \leq \frac{1}{\Im{\lambda}}.
					\end{align}
					ただし$I$は$H$上の恒等写像である.
					
				\item[(2)] $u,v \in H$を$A$の異なる固有値$\lambda,\mu$に対する固有ベクトルとすれば
					$\inprod<u,v> = 0$が成り立つ.
			\end{description}
		\end{prp}
	\end{screen}
	
	\begin{prf}\mbox{}
		\begin{description}
			\item[(1)]
				
			\item[(2)]
				今$Au = \lambda u,\ Av = \mu v$が満たされているから,(\refeq{eq:self_adjoint_1})と同様にして
				\begin{align}
					\lambda \inprod<u,v> = \inprod<Au,v> = \inprod<u,Av> = \conj{\mu} \inprod<u,v>
				\end{align}
				が成り立つ.(1)より$\mu \in \R$であるから$(\lambda - \mu) \inprod<u,v> = 0$
				が得られ,$\lambda \neq \mu$の仮定より$\inprod<u,v> = 0$が従う.
				\QED
		\end{description}
	\end{prf}
	
	
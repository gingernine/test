\section{直交射影}
	\begin{screen}
		\begin{dfn}[直交射影]
			$H$を複素Hilbert空間とする.
			$p:H \rightarrow H$が直交射影であるとは,
			$p$に対して或る$H$の閉部分空間$H_0$が存在し,
			$x \in H$とその直交分解$x = x_1 + x_2\ (x_1 \in H_0, x_2 \in H_0^{\perp})$
			に対し次を満たすことをいう\footnotemark:
			\begin{align}
				p:H \ni x \longmapsto x_1.
			\end{align}
		\end{dfn}
		また$H$上の直交射影全体を$\Oproj{H}$と書く.
	\end{screen}
	
	\footnotetext{
		射影定理より$x \in H$の直交分解は一意に定まるから,
		$p$は写像としてwell-definedである.
	}
	
	\begin{screen}
		\begin{prp}[直交射影の存在]
			$H$を複素Hilbert空間とする.$H$の任意の閉部分空間$L$に対し
			或る$p \in \Oproj{H}$が存在して$p:H \rightarrow L$を満たす.
			特に$\Ran{p} = L$が成り立つ.
		\end{prp}
	\end{screen}
	
	\begin{screen}
		\begin{prp}[直交射影は冪等・自己共役]
			$H$を複素Hilbert空間とする.任意の$p:H \rightarrow H$に対し次は同値である:
			\begin{description}
				\item[(1)] $p \in \Oproj{H}$.
				\item[(2)] $p$は有界で$p^2 = p$と$p^* = p$を満たす.
			\end{description}
			\label{prp:orthogonal_projection_idempotent_self_adjoint}
		\end{prp}
	\end{screen}
	
	\begin{screen}
		\begin{prp}[直交射影の積・和の性質]
			$H$を複素Hilbert空間とする.
			\begin{description}
				\item[(1)] $p,q \in \Oproj{H}$に対し次が成り立つ:
					\begin{align}
						\Ran{p} \perp \Ran{q}
						\quad \Leftrightarrow \quad  pq = 0
						\quad \Leftrightarrow \quad  qp = 0.
					\end{align}
				
				\item[(2)] 
					$p_1,\cdots,p_n \in \Oproj{H}$が
					$p_i \neq p_j\ (i \neq j)$を満たすなら,
					$p \coloneqq \sum_{i=1}^{n} p_i$とおいて次が成り立つ:
					\begin{align}
						p \in \Oproj{H}
						\quad \Leftrightarrow \quad p_i p_j = \delta_{ij} p_j \quad (i,j = 1,\cdots,n).
					\end{align}
					ただし$\delta_{ij}$はKroneckerのデルタである.
				
				\item[(3)] 
					$p_1,p_2,\cdots \in \Oproj{H}$が
					$p_i p_j = \delta_{ij} p_j \ (\forall i,j \in \N)$を満たすとして
					\begin{align}
						H_0 \coloneqq \closure{\LH{\bigcup_{i=1}^{\infty}\Ran{p_i}}}
					\end{align}
					とおく.$p \in \Oproj{H}$が$\Ran{p} = H_0$であるとき次が成り立つ:
					\begin{align}
						px = \sum_{i=1}^{\infty} p_i x \quad (\forall x \in H).
					\end{align}
			\end{description}
			\label{prp:orthogonal_projection_product_sum}
		\end{prp}
	\end{screen}
本章を通じて$X,Y$を複素ノルム空間とし,ノルムをそれぞれ$\Norm{\cdot}{X},\Norm{\cdot}{Y}$と表す.

\section{直積ノルム空間の位相}
	\begin{screen}
		\begin{prp}[直積ノルム空間]
			$x \in X, y \in Y$の組を$[x,y]$と表し,$X$と$Y$の直積(product)を
			\begin{align}
				X \times Y \coloneqq \Set{[x,y]}{x \in X, y \in Y}
			\end{align}
			で定める.そして写像$\Norm{\cdot}{X \times Y}:X \times Y \rightarrow \R$を次で定める:
			\begin{align}
				\Norm{[x,y]}{X \times Y} \coloneqq \Norm{x}{X} + \Norm{y}{Y}
				\quad (\forall [x,y] \in X \times Y).
			\end{align}
			このとき次が成り立つ:
			\begin{description}
				\item[(1)] $X \times Y$は
					\begin{align}
						[x,y] + [s, t] \coloneqq [x + s, y + t],
						\quad \alpha [x,y] \coloneqq [\alpha x, \alpha y]
						\quad (\forall [x,y],[s,t] \in X \times Y,\ \alpha \in \C)
					\end{align}
					を線型演算として線形空間となる.
					
				\item[(2)] 線形空間$X \times Y$は$\Norm{\cdot}{X \times Y}$をノルムとしてノルム空間となる.
			\end{description}
		\end{prp}
	\end{screen}
	
	\begin{screen}
		\begin{thm}[ノルム位相と直積位相は一致する]
			$\Norm{\cdot}{X \times Y}$によるノルム位相と$X,Y$の直積位相は一致する.
		\end{thm}
	\end{screen}
	
	\begin{screen}
		\begin{thm}[$X,Y$が完備なら$X \times Y$も完備]
			$X,Y$がBanach空間であるとき,$X \times Y$もBanach空間である.
		\end{thm}
	\end{screen}

\section{閉作用素と閉グラフ定理}
	\begin{screen}
		\begin{thm}[Banach空間値の有界な閉作用素の定義域は閉]
			$Y$がBanach空間であるとき,閉作用素$T:X \oparrow Y$が有界なら
			$\Dom{T} $は$X$の閉部分空間である.
			\label{thm:domain_of_banach_valued_closed_op_is_closed}
		\end{thm}
	\end{screen}
	
	\begin{prf}
		$\Dom{T} $の点列$(u_n)_{n=1}^{\infty}$が$u_n \longrightarrow u \in X$を満たせば
		\begin{align}
			\Norm{T u_n - T u_m}{Y} \leq \Norm{T}{\Bop{X}{Y} }\Norm{u_n - u_m}{X}
			\longrightarrow 0 \quad (n,m \longrightarrow \infty)
		\end{align}
		が成り立ち,$Y$の完備性より$(T u_n)_{n=1}^{\infty}$は或る$v \in Y$に強収束する.
		$T$は閉作用素であるから$v = T u$が従う.
		\QED
	\end{prf}
	
	\begin{screen}
		\begin{thm}[閉作用素の逆も閉]
			$T:X \oparrow Y$を閉作用素とするとき,
			$T^{-1}$が存在すればこれも閉作用素となる.
			\label{thm:closed_linear_op_inverse}
		\end{thm}
	\end{screen}
	
	\begin{prf}
		$U:X \times Y \ni [x,y] \longmapsto [y,x] \in Y \times X$
		として同相写像$U$を定める.仮定より$\Graph{T} $は$X \times Y$の閉集合であり
		\begin{align}
			\Graph{T^{-1}} = U \Graph{T}  
		\end{align}
		が成り立つから,$\Graph{T^{-1}} $は$Y \times X$の閉集合である.
		\QED
	\end{prf}
	
	\begin{screen}
		\begin{thm}[閉グラフ定理]
		\end{thm}
	\end{screen}
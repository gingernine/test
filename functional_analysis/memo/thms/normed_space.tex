\chapter{ノルム空間}
	$\K$を$\R$又は$\C$とする.$\K$上のノルム空間$X$におけるノルムを$\Norm{\cdot}{X}$と表記し,$X$にノルム位相を導入する.
	\begin{screen}
		\begin{thm}[有限次元空間における有界点列の収束(局所コンパクト性)]\mbox{}\\
			$\K$を$\R$又は$\C$とし,$X$を$\K$上のノルム空間とする.$\Dim{X} < \infty$ならば$X$の任意の有界点列は
			収束部分列を含む.
			\label{thm:normed_space_locally_compact}
		\end{thm}
	\end{screen}

	\begin{prf}\mbox{}
		$X$の次元数$n$による帰納法で証明する.
		\begin{description}
			\item[第一段]
				$n=1$のとき$X$の基底を$u_1$とすれば,$X$の任意の有界点列は
				$( \alpha_m u_1)_{m=1}^{\infty} \quad (\alpha_m \in \K,\ m=1,2,\cdots)$と表せる.
				$\left( \alpha_m \right)_{m=1}^{\infty}$は有界列であるから,
				Bolzano-Weierstrassの定理より部分列$\left( \alpha_{m_k} \right)_{k=1}^{\infty}$と$\alpha \in \K$が存在して
				\begin{align}
					\left| \alpha_{m_k} - \alpha \right| \longrightarrow 0 \quad (k \longrightarrow \infty)
				\end{align}
				を満たし
				\begin{align}
					\Norm{\alpha_{m_k} u_1 - \alpha u_1}{X} \longrightarrow 0 \quad (k \longrightarrow \infty)
				\end{align}
				が従う.
			
			\item[第二段]
				$n=k$のとき定理の主張が成り立つと仮定し,$n = k+1$として$X$の基底を$u_1,\cdots,u_{k+1}$と表す.
				$X$から任意に有界列$(x_j)_{j=1}^{\infty}$を取れば,各$x_j$は
				\begin{align}
					x_j = y_j + \beta_j u_{k+1} \quad (y_j \in \LH{\left\{\, u_1,\cdots,u_k\, \right\}},\ \beta_j \in \K)
				\end{align}
				として一意に表示される.$(\beta_j)_{j=1}^{\infty}$が有界でないと仮定すると
				$\beta_{j_s} \geq s\ (j_s < j_{s+1},\ s=1,2,\cdots)$を満たす部分列が存在し,$(x_j)_{j=1}^{\infty}$の有界性と併せて
				\begin{align}
					\Norm{u_{k+1} + \tfrac{1}{\beta_{j_s}}y_{j_s}}{X}
					\leq \Norm{u_{k+1} + \tfrac{1}{\beta_{j_s}}y_{j_s} - \tfrac{1}{\beta_{j_s}}x_{j_s}}{X}
						+ \Norm{\tfrac{1}{\beta_{j_s}}x_{j_s}}{X}
					= \Norm{\tfrac{1}{\beta_{j_s}}x_{j_s}}{X} \longrightarrow 0 \quad (s \longrightarrow \infty)
				\end{align}
				が成り立つが,有限次元空間は閉であるから
				$u_{k+1} \in \LH{\left\{\, u_1,\cdots,u_k\, \right\}}$が従い矛盾が生じる.よって$(\beta_j)_{j=1}^{\infty}$は
				$\K$の有界列でなくてはならず,Bolzano-Weierstrassの定理より部分列$\left( \beta_{j(1,i)} \right)_{i=1}^{\infty}$と$\beta \in \K$が存在して
				\begin{align}
					\left| \beta_{j(1,i)} - \beta \right| \longrightarrow 0 \quad (i \longrightarrow \infty)
				\end{align}
				を満たす.また$\left(y_{j(1,i)}\right)_{i=1}^{\infty}$も有界列となるから,或る
				$y \in \LH{\left\{\, u_1,\cdots,u_k\, \right\}}$と部分列$\left(y_{j(2,i)}\right)_{i=1}^{\infty}$が存在して
				\begin{align}
					\Norm{y_{j(2,i)} - y}{X} \longrightarrow 0 \quad (i \longrightarrow \infty)
				\end{align}
				を満たす.従って
				\begin{align}
					\Norm{x_{j(2,i)} - \left(y + \beta u_{k+1}\right)}{X} \leq
					\Norm{y_{j(2,i)} - y}{X} + \left| \beta_{j(1,i)} - \beta \right| \Norm{u_{k+1}}{X}
					\longrightarrow 0 \quad (i \longrightarrow \infty)
				\end{align}
				が成り立つ.
				\QED
		\end{description}
	\end{prf}
	
	\begin{screen}
		\begin{thm}[閉部分空間との点の距離]
			$X$をノルム空間,$L \subsetneq X$を閉部分空間とする.このとき任意の$\epsilon > 0$に対して
			或る$e \in X$が存在し,$\Norm{e}{X} = 1$かつ次を満たす:
			\begin{align}
				\inf{x \in L}{\Norm{e - x}{X}} > 1 - \epsilon.
			\end{align}
			\label{thm:closed_subspace_distance}
		\end{thm}
	\end{screen}
	
	\begin{prf}
		
	\end{prf}
	
	\begin{screen}
		\begin{thm}[単位球面がコンパクトなら有限次元]
			$X$をノルム空間,$S$を$X$の単位球面とする.
			$S$がコンパクトならば$\Dim{X} < \infty$である.
			\label{thm:normed_space_unit_sphere_compact_finite_dimension}
		\end{thm}
	\end{screen}
	
	\begin{prf}
		対偶を証明する.距離空間のコンパクト性についての一般論より,$S$がコンパクトであることと$S$の任意の点列が
		$S$で収束する部分列を含むことは同値である.$\Dim{X} = \infty$と仮定する.
		任意に一つ$e_1 \in S$を取り$L_1 \coloneqq \LH{\{\, e_1\, \}}$とおけば,$L_1$は$X$の閉部分空間であるから
		定理\ref{thm:closed_subspace_distance}より或る$e_2 \in S$が存在して
		\begin{align}
			\inf{x \in L_1}{\Norm{e_2 - x}{X}} > \frac{1}{2}
		\end{align}
		を満たす.$L_2 \coloneqq \LH{\{\, e_1,e_2\, \}}$も$X$の閉部分空間であるから
		或る$e_3 \in S$が存在して
		\begin{align}
			\inf{x \in L_2}{\Norm{e_3 - x}{X}} > \frac{1}{2}
		\end{align}
		を満たす.この操作を繰り返して$S$の点列$e_1,e_2,\cdots$を構成すれば,
		\begin{align}
			\Norm{e_n - e_m}{X} > \frac{1}{2} \quad (\forall n,m \in \N,\ n \neq m)
		\end{align}
		が成り立ち$\left( e_n \right)_{n=1}^{\infty}$は収束部分列を含みえない.
		\QED
	\end{prf}
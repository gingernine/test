\section{緩増加超関数の構造定理}
	\begin{screen}
		\begin{lem}\label{lem:isomorphism_on_product_of_dual_spaces}
			$\Lambda$を有限集合,$\Bigl( (X_\lambda,\Norm{\cdot}{\lambda}) \Bigr)_{\lambda \in \Lambda}$
			をノルム空間の系とする.このとき,$f_\lambda \in X^*_\lambda\ (\lambda \in \Lambda)$
			を取り
			\begin{align}
				F(x) \coloneqq \sum_{\lambda \in \Lambda} f_\lambda(x_\lambda),
				\quad \biggl(\forall x = (x_\lambda)_{\lambda \in \Lambda} \in \prod_{\lambda \in \Lambda} X_\lambda \biggr)
				\label{eq:lem_isomorphism_on_product_of_dual_spaces}
			\end{align}
			により線型汎関数$F$を定めれば$F \in \left( \prod_{\lambda \in \Lambda} X_\lambda \right)^*$
			が満たされる.そしてこの対応により定まる次の写像
			\begin{align}
				W:\prod_{\lambda \in \Lambda} X_\lambda^* \ni f = (f_\lambda)_{\lambda \in \Lambda}
				\longmapsto F \in \Biggl( \prod_{\lambda \in \Lambda} X_\lambda \Biggr)^*
			\end{align}
			は線型・位相同型である.
		\end{lem}
	\end{screen}
	
	\begin{prf}
		$X_\lambda^*,\prod_{\lambda \in \Lambda} X_\lambda^*,\left( \prod_{\lambda \in \Lambda} X_\lambda \right)^*$におけるノルムをそれぞれ
		$\Norm{\cdot}{X_\lambda^*},\Norm{\cdot}{\prod_{\lambda \in \Lambda} X_\lambda^*},\Norm{\cdot}{\left( \prod_{\lambda \in \Lambda} X_\lambda \right)^*}$と表す.
		先ず(\refeq{eq:lem_isomorphism_on_product_of_dual_spaces})において,
		\begin{align}
			|F(x)| \leq \sum_{\lambda \in \Lambda} |f_\lambda(x_\lambda)|
			\leq \max{\lambda \in \Lambda}{\Norm{f_\lambda}{X_\lambda^*}} \sum_{\lambda \in \Lambda} \Norm{x_\lambda}{\lambda}
			\label{eq:lem_isomorphism_on_product_of_dual_spaces_2}
		\end{align}
		となるから$F \in \left( \prod_{\lambda \in \Lambda} X_\lambda \right)^*$を得る.
		次に$W$が全単射であることを示す.実際,任意の$G \in \left( \prod_{\lambda \in \Lambda} X_\lambda \right)^*$に対して
		\begin{align}
			x^{(\lambda)}_\nu \coloneqq 
			\begin{cases}
				x_\lambda & (\nu = \lambda) \\
				0 & (\nu \neq \lambda)
			\end{cases},
			\quad g_\lambda(x_\lambda) \coloneqq G(x^{(\lambda)})
			\quad (\forall x_\lambda \in X_\lambda)
		\end{align}
		と定めれば$g_\lambda \in X_\lambda^*$が成り立つから$W$は全射であり,
		また$f,g \in \prod_{\lambda \in \Lambda} X_\lambda^*$に対し$Wf = Wg$が満たされているとき,
		\begin{align}
			f_\lambda(x_\lambda) = (Wf)(x^{(\lambda)})
			= (Wg)(x^{(\lambda)}) = g(x_\lambda),
			\quad (\forall x_\lambda \in X_\lambda,\ \forall \lambda \in \Lambda)
		\end{align}
		が従い$W$の単射性が出る.$W$の線型性は
		\begin{align}
			&W(\alpha f + \beta g)(x)
			= \sum_{\lambda \in \Lambda} (\alpha f_\lambda + \beta g_\lambda) (x_\lambda) \\
			&\qquad = \alpha \sum_{\lambda \in \Lambda} f_\lambda(x_\lambda)
				+ \beta \sum_{\lambda \in \Lambda} g_\lambda(x_\lambda)
			= (\alpha W f + \beta W g)(x),
			\quad (\forall x=(x_\lambda),\ f=(f_\lambda),g=(g_\lambda),\ \alpha,\beta \in \C)
		\end{align}
		により得られ,かつ(\refeq{eq:lem_isomorphism_on_product_of_dual_spaces_2})より
		\begin{align}
			\Norm{Wf}{\left( \prod_{\lambda \in \Lambda} X_\lambda \right)^*}
			\leq \max{\lambda \in \Lambda}{\Norm{f_\lambda}{X_\lambda^*}}
			\leq \Norm{f}{\prod_{\lambda \in \Lambda} X_\lambda^*},
			\quad \biggl( \forall f = (f_\lambda)_{\lambda \in \Lambda} \in \prod_{\lambda \in \Lambda} X_\lambda^* \biggr)
		\end{align}
		が成り立つから$W$は連続であり,開写像定理より$W^{-1}$もまた連続である.
		\QED
	\end{prf}
	
	\begin{screen}
		\begin{lem}\label{lem:tempered_distributions_continuity}
			任意の$u \in \tempdist{\R^n}$に対して或る$c = c(u) > 0$と$m = m(u) \in \N$が存在し次を満たす:
			\begin{align}
				|\inprod<u,\varphi>| \leq c p_m(\varphi),
				\quad (\forall \varphi \in \rapid{\R^n}).
			\end{align}
		\end{lem}
	\end{screen}
	
	\begin{prf}
		背理法で証明する.主張が満たされない場合,
		任意の$k \in \Z_+$に対して或る$\varphi_k \in \rapid{\R^k}$が存在し
		\begin{align}
			|\inprod<u,\varphi_k>| > k p_k(\varphi_k)
		\end{align}
		が成立するから,$\psi_k \coloneqq \varphi_k / [k p_k(\varphi_k)]\ (k \in \Z_+)$とおけば
		\footnote{
			$|\inprod<u,\varphi_k>| > 0$より$\varphi_k$は零写像ではない.従って$\rapid{\R^n}$の半ノルム
			$p_k$の定義より$p_k(\varphi_k) > 0$が満たされている.
		}
		\begin{align}
			|\inprod<u,\psi_k>| = \frac{|\inprod<u,\varphi_k>|}{k p_k(\varphi_k)} > 1
			\quad (\forall k \in \Z_+)
			\label{eq:lem_tempered_distributions_continuity}
		\end{align}
		が従う.一方で半ノルム系$(p_m)_{m \in \N}$は$p_0 \leq p_1 \leq p_2 \leq \cdots$を満たすから,
		任意の$m \in \N$に対して
		\begin{align}
			p_m(\psi_k) = \frac{p_m(\varphi_k)}{k p_k(\varphi_k)} \leq \frac{1}{k}
			\longrightarrow 0 \quad (k \longrightarrow \infty)
		\end{align}
		が成り立ち,$u$の連続性から$\inprod<u,\psi_k> \longrightarrow \inprod<u,0> = 0$となるはずであるが,
		これは(\refeq{eq:lem_tempered_distributions_continuity})と矛盾する.
		\QED
	\end{prf}
	
	\begin{screen}
		\begin{thm}[緩増加超関数の構造定理]\label{thm:structure_theorem_of_tempered_distributions}
			$u \in \tempdist{\R^n}$とし,補題\ref{lem:tempered_distributions_continuity}
			により対応する$m \in \N$を取る.また以下では$\alpha$は$n$次元多重指数を表すものとする.
			このとき或る系$(g_\alpha)_{|\alpha| \leq m} \subset \mathrm{L}^\infty(\R^n)$が存在して次を満たす:
			\begin{align}
				\inprod<u,\varphi> =
				\sum_{|\alpha| \leq m} \int_{\R^n} (1+|x|^2)^m g_\alpha(x) \left[ \partial_1 \cdots \partial_n \partial^\alpha \varphi(x) \right]\ dx,
				\quad (\forall \varphi \in \rapid{\R^n}).
			\end{align}
		\end{thm}
	\end{screen}
	
	\begin{prf}以下$u \in \tempdist{\R^n}$と$c>0,\ m \in \N$は固定する.
		\begin{description}
			\item[第一段]
				任意の$\varphi \in \rapid{\R^n}$に対して
				\begin{align}
					p_m(\varphi)
					\leq (m+1) \sum_{|\alpha| \leq m} \int_{\R^n} (1+|y|^2)^m \left| \partial_1 \cdots \partial_n\partial^\alpha \varphi(y) \right|\ dy
					\label{eq:structure_theorem_of_tempered_distributions_1}
				\end{align}
				が成り立つことを示す.
				任意の$\alpha \in \N^n$と$x = (x_1,\cdots,x_n) \in \R^n$に対して,
				$x_j \geq 0$を満たす$1 \leq j \leq n$の個数を$\#$で表し,
				$x_j < 0$なら積分範囲$I_j$を$(-\infty,x_j]$,$x_j \geq 0$なら$I_j = [x_j,\infty)$とすれば
				\begin{align}
					\partial^\alpha \varphi(x)
					= (-1)^{\#} \int_{I_1}\cdots\int_{I_n} \partial_1 \cdots \partial_n\partial^\alpha \varphi(y)\ dy_n \cdots dy_1
				\end{align}
				が成り立つ.そして$y \in I_1 \times \cdots \times I_n$なら$|x| \leq |y|$が満たされるから,
				$0 \leq k \leq m$に対し
				\begin{align}
					(1+|x|^2)^k \left| \partial^\alpha \varphi(x) \right|
					&\leq \int_{I_1}\cdots\int_{I_n} (1+|x|^2)^k \left| \partial_1 \cdots \partial_n\partial^\alpha \varphi(y) \right|\ dy_n \cdots dy_1 \\
					&\leq \int_{I_1}\cdots\int_{I_n} (1+|y|^2)^k \left| \partial_1 \cdots \partial_n\partial^\alpha \varphi(y) \right|\ dy_n \cdots dy_1 \\
					&\leq \int_{\R^n} (1+|y|^2)^k \left| \partial_1 \cdots \partial_n\partial^\alpha \varphi(y) \right|\ dy \\
					&\leq \int_{\R^n} (1+|y|^2)^m \left| \partial_1 \cdots \partial_n\partial^\alpha \varphi(y) \right|\ dy
				\end{align}
				が従い
				\begin{align}
					p_m(\varphi) &= \sum_{|\alpha|+k \leq m} \sup{x \in \R^n}{(1+|x|^2)^k \left| \partial^\alpha \varphi(x) \right|} \\
					&\leq \sum_{|\alpha|+k \leq m} \int_{\R^n} (1+|y|^2)^m \left| \partial_1 \cdots \partial_n\partial^\alpha \varphi(y) \right|\ dy \\
					&\leq (m+1) \sum_{|\alpha| \leq m} \int_{\R^n} (1+|y|^2)^m \left| \partial_1 \cdots \partial_n\partial^\alpha \varphi(y) \right|\ dy
				\end{align}
				を得る.
				
			\item[第二段]
				$\varphi \in \rapid{\R^n}$に対し,
				$\psi^\varphi_\alpha(y) \coloneqq (1+|y|^2)^m \partial_1 \cdots \partial_n\partial^\alpha \varphi(y)$により$\psi^\varphi = (\psi^\varphi_\alpha)_{|\alpha| \leq m}$を定める.このとき
				\begin{align}
					\Delta \coloneqq \Set{\psi^\varphi = (\psi^\varphi_\alpha)_{|\alpha| \leq m}}{\varphi \in \rapid{\R^n}}
				\end{align}
				で定める$\Delta$は対応$\varphi \longmapsto \psi^\varphi$により$\rapid{\R^n}$と線型同型となる.実際,この写像の線型性は微分の性質から従い,
				$\Delta$の作り方より全射である.また
				$\varphi,\eta \in \rapid{\R^n}$に対して,
				$(\psi^\varphi_\alpha)_{|\alpha| \leq m} = (\psi^\eta_\alpha)_{|\alpha| \leq m}$
				ならば
				\begin{align}
					(1+|y|^2)^m \partial_1 \cdots \partial_n \varphi(y)
					= (1+|y|^2)^m \partial_1 \cdots \partial_n \eta(y),
					\quad (\forall y \in \R^n)
				\end{align}
				が従い
				\begin{align}
					&\varphi(x)
					= \int_{-\infty}^{x_1}\cdots\int_{-\infty}^{x_n} \partial_1 \cdots \partial_n \varphi(y)\ dy_n\cdots dy_1 \\
					&\qquad = \int_{-\infty}^{x_1}\cdots\int_{-\infty}^{x_n} \partial_1 \cdots \partial_n \eta(y)\ dy_n\cdots dy_1
					= \eta(x),
					\quad (\forall x \in \R^n)
				\end{align}
				が得られるから$\varphi \longmapsto \psi^\varphi$は単射である.
				いま,
				\begin{align}
					\Norm{\psi^\varphi}{\Delta}
					\coloneqq \sum_{|\alpha| \leq m} \int_{\R^n} \left| \psi^\varphi_\alpha(y) \right|\ dy
					= \sum_{|\alpha| \leq m} \int_{\R^n} (1+|y|^2)^m \left| \partial_1 \cdots \partial_n\partial^\alpha \varphi(y) \right|\ dy
					\label{eq:structure_theorem_of_tempered_distributions_2}
				\end{align}
				として$\Norm{\cdot}{\Delta}$を定めれば,
				$\psi^\varphi_\alpha$の連続性よりこれは$\Delta$上のノルムとなる.
				
			\item[第三段]
				ノルム空間$\Delta$上の線型汎関数$L$を
				\begin{align}
					L\psi^\varphi
					\coloneqq \inprod<u,\varphi>,
					\quad (\forall \varphi \in \rapid{\R^n})
				\end{align}
				により定めれば,$L$は連続である.
				実際(\refeq{eq:structure_theorem_of_tempered_distributions_1})と
				(\refeq{eq:structure_theorem_of_tempered_distributions_2})より
				\begin{align}
					\left| L\psi^\varphi \right|
					\leq c (m+1) \Norm{\psi^\varphi}{\Delta},
					\quad (\forall \varphi \in \rapid{\R^n})
				\end{align}
				が成立する.
				
			\item[第四段]
				$\psi^\varphi = (\psi^\varphi_\alpha)_{|\alpha| \leq m}$に対し,
				各成分$\psi^\varphi_\alpha$にこれを代表とする$\mathrm{L}^1(\R^n)$の元を対応させる等長な線型単射を
				$U$と表す:
				\begin{align}
					U:\Delta \ni \psi^\varphi \longmapsto 
					U \psi^\varphi \in \prod_{|\alpha| \leq m} \mathrm{L}^1(\R^n).
				\end{align}
				$U$の値域をその像に制限すれば逆写像$U^{-1}$が存在し,
				合成写像$LU^{-1}$は部分空間$U\Delta$上で線型連続であるからHahn-Banachの拡張定理より
				$LU^{-1}$のノルム保存拡張$\tilde{L} \in \left( \prod_{|\alpha| \leq m} \mathrm{L}^1(\R^n) \right)^*$が存在する.従って,
				\begin{align}
					W:\prod_{|\alpha| \leq m} \mathrm{L}^1(\R^n)^* \longrightarrow 
					\biggl( \prod_{|\alpha| \leq m} \mathrm{L}^1(\R^n) \biggr)^*
				\end{align}
				を補題\ref{lem:isomorphism_on_product_of_dual_spaces}における同型写像とすれば,
				或る$(\Phi_\alpha)_{|\alpha| \leq m} \in \prod_{|\alpha| \leq m} \mathrm{L}^1(\R^n)^*$がただ一つ存在して$\tilde{L} = W \left((\Phi_\alpha)_{|\alpha| \leq m}\right)$と表される.一方で
				それぞれの$\Phi_\alpha$には或る$g_\alpha \in \mathrm{L}^\infty(\R^n)$
				がただ一つ対応して
				\begin{align}
					\Phi_\alpha (f) = \int_{\R^n} g_\alpha(x)f(x)\ dx,
					\quad \left( \forall f \in \mathrm{L}^1(\R^n) \right)
				\end{align}
				を満たすから,(\refeq{eq:lem_isomorphism_on_product_of_dual_spaces})で定める演算により
				\begin{align}
					\inprod<u,\varphi>
					&= L\psi^\varphi
					= \tilde{L}U\psi^\varphi
					= \left[ W \left((\Phi_\alpha)_{|\alpha| \leq m}\right) \right] U\psi^\varphi \\
					&= \sum_{|\alpha| \leq m} \int_{\R^n} g_\alpha(x) \psi^\varphi_\alpha(x)\ dx
					= \sum_{|\alpha| \leq m} \int_{\R^n} g_\alpha(x) (1+|x|^2)^m \partial_1 \cdots \partial_n\partial^\alpha \varphi(x)\ dx
				\end{align}
				が成り立ち定理の主張を得る.
				\QED
		\end{description}
	\end{prf}
	
	\begin{screen}
		\begin{thm}[緩増加連続関数と緩増加超関数の一対一対応]
			緩増加連続関数$f:\R^n \longrightarrow \C$に対し
			$u_f \in \tempdist{\R^n}$を
			\begin{align}
				u_f: \rapid{\R^n} \ni \varphi \longmapsto
				\int_{\R^n} f(x) \varphi(x)\ dx
			\end{align}
			で定めるとき,
		\end{thm}
	\end{screen}
	
	\begin{prf}
		定理\ref{thm:tempered_continuous_functions_and_tempered_distributions}より
		$u_f$は緩増加超関数であり,可積分性より$f \longmapsto u_f$の線型性が出る.
		また緩増加連続関数は局所可積分であるから,変分法の基本補題より
		$f \longmapsto u_f$は単射である.後は$f \longmapsto u_f$が全射であることを示せばよい.
	\end{prf}
	
	\begin{screen}
		\begin{thm}[緩増加超関数の合成積の性質]\mbox{}
			\begin{description}
				\item[(1)] $u \in \tempdist{\R^n}$と$\varphi \in \rapid{\R^n}$の合成積
					$u \ast \varphi$に対し或る緩増加関数$f$が存在して
					$u \ast \varphi = u_f$を満たす.
			\end{description}
		\end{thm}
	\end{screen}
	
	\begin{prf}\mbox{}
		\begin{description}
			\item[(1)] 構造定理より$u$に対して或る$(g_\alpha)_{|\alpha| \leq m} \subset \mathrm{L}^\infty(\R^n)$が存在し
			\begin{align}
				\inprod<u,\varphi> =
				\sum_{|\alpha| \leq m} \int_{\R^n} (1+|x|^2)^m g_\alpha(x) \left[ \partial_1 \cdots \partial_n \partial^\alpha \varphi(x) \right]\ dx,
				\quad (\forall \varphi \in \rapid{\R^n}).
			\end{align}
			と表現できるから,Fubiniの定理より任意の$\psi \in \rapid{\R^n}$に対して
			\begin{align}
				\inprod<u \ast \varphi,\psi> &= \inprod<u, \check{\varphi} \ast \psi> \\
				&= \sum_{|\alpha| \leq m} \int_{\R^n} (1+|x|^2)^m g_\alpha(x) \left[ \partial_1 \cdots \partial_n \partial^\alpha \left(\check{\varphi} \ast \psi\right)(x) \right]\ dx \\
				&= \sum_{|\alpha| \leq m} \int_{\R^n} (1+|x|^2)^m g_\alpha(x) \left[ \partial_1 \cdots \partial_n \partial^\alpha \int_{\R^n} \varphi(y-x) \psi (y)\ dy \right]\ dx \\
				&= \sum_{|\alpha| \leq m} (-1)^{|\alpha|} \int_{\R^n} (1+|x|^2)^m g_\alpha(x) \left[ \int_{\R^n} \partial_1 \cdots \partial_n \partial^\alpha\varphi(y-x) \psi (y)\ dy \right]\ dx \\
				&= \sum_{|\alpha| \leq m} (-1)^{|\alpha|} \int_{\R^n} \left[ \int_{\R^n} (1+|x|^2)^m g_\alpha(x)\ \partial_1 \cdots \partial_n \partial^\alpha\varphi(y-x)\ dx\right] \psi(y)\ dy
			\end{align}
			が成立する.このとき,各$\alpha$に対して
			\begin{align}
				f_\alpha:\R^n \ni y \longmapsto \int_{\R^n} (1+|x|^2)^m g_\alpha(x)\ \partial_1 \cdots \partial_n \partial^\alpha\varphi(y-x)\ dx
			\end{align}
			は緩増加関数である.表記上簡単にするため$\partial_1 \cdots \partial_n \partial^\alpha\varphi$を$\eta$と書き,
			先ずは$f_\alpha$の微分可能性を示す.実際
			\begin{align}
				(1+|x|^2) \leq (1 + 2|y-x|^2 + 2|y|^2) \leq 2(1+|y-x|^2)(1+|y|^2)
			\end{align}
			であることを用いれば,任意の$1 \leq k \leq n$に対して
			\begin{align}
				(1+|x|^2)^m |g_\alpha(x)|\ |\partial_k \eta(y-x)|
				\leq 2 \Norm{g_\alpha}{\mathrm{L}^\infty(\R^n)} (1+|y|^2)^m p_{m+n+1}(\eta) \frac{1}{(1+|y-x|^2)^n},
				\quad (\forall x \in \R^n)
			\end{align}
			が成り立ち,右辺は$x$の関数として$\R^n$上可積分である.よってLebesgueの収束定理より
			\begin{align}
				\partial_k \int_{\R^n} (1+|x|^2)^m g_\alpha(x)\ \eta(y-x)\ dx
				= \int_{\R^n} (1+|x|^2)^m g_\alpha(x)\ \partial_k \eta(y-x)\ dx
			\end{align}
			が従い,帰納法により任意の$\beta \in \N^n$に対して
			\begin{align}
				\partial^\beta \int_{\R^n} (1+|x|^2)^m g_\alpha(x)\ \eta(y-x)\ dx
				= \int_{\R^n} (1+|x|^2)^m g_\alpha(x)\ \partial^\beta \eta(y-x)\ dx
			\end{align}
			が出る.そして
			\begin{align}
				\left| \partial^\beta f_\alpha(y) \right|
				&\leq \int_{\R^n} (1+|x|^2)^m |g_\alpha(x)|\ \left| \partial^\beta \eta(y-x) \right|\ dx \\
				&\leq \left[2 \Norm{g_\alpha}{\mathrm{L}^\infty(\R^n)} p_{m+n+|\beta|}(\eta) \int_{\R^n} \frac{1}{(1+|y-x|^2)^n}\ dx \right] (1+|y|^2)^m
			\end{align}
			が満たされるから$f_\alpha$は緩増加関数であり,
			\begin{align}
				f \coloneqq \sum_{|\alpha| \leq m} (-1)^{|\alpha|} f_\alpha
			\end{align}
			により緩増加関数$f$を定めれば$u \ast \varphi = u_f$が得られる.
			\QED
		\end{description}
	\end{prf}
%レポート問題8
	\begin{prf}
		$\sigma$-有限であるから或る系$(X_n)_{n=1}^{\infty} \subset \mathcal{M}$
		が存在して$X_1 \subset X_2 \subset \cdots,\ \mu(X_n) < \infty\ (\forall n \in \N),\ \cup_{n \in \infty} X_n = X$を満たす.
		\begin{description}
			\item[(1)] 
				任意に$v \in H$を取り$v_n \coloneqq v\defunc_{\{|a| \leq n\}}\ (n=1,2,3,\cdots)$として関数列$(v_n)_{n=1}^{\infty}$を作る.
				全ての$x \in S$で$|v_n(x)| \leq |v(x)|$が満たされているから
				$(v_n)_{n \in \N} \subset H$である.また全ての$n \in \N$について
				\begin{align}
					\int_{S} |a(x)v_n(x)|^2 \mu(dx) = \int_{\{|a| \leq n\}} |a(x)v(x)|^2 \mu(dx) \leq n^2  \int_{S} |v(x)|^2 \mu(dx)
				\end{align}
				が成り立つから$(v_n)_{n \in \N} \subset D(M_a)$も満たされる.
				\begin{align}
					\Norm{v - v_n}{}^2 = \int_{S} |v(x) - v_n(x)|^2\, \mu(dx) = \int_{S} \defunc_{\{|a| > n\}}(x)|v(x)|^2\, \mu(dx)
					\label{eq:func_report_Q9_1}
				\end{align}
				となり,右辺の被積分関数は各点で$0$に収束し,かつ$n$に関係なく可積分関数$|v|^2$で抑えられるから,
				Lebesgueの収束定理より
				\begin{align}
					\lim_{n \to \infty} \Norm{v - v_n}{\Lp{2}{\mu}}^2 
					= \lim_{n \to \infty} \int_{S} \defunc_{\{|a| > n\}}(x)|v(x)|^2\, \mu(dx)
					= \int_{S} \lim_{n \to \infty} \defunc_{\{|a| > n\}}(x)|v(x)|^2\, \mu(dx)
					= 0
				\end{align}
				が得られる.$v$は任意に選んでいたから$D(M_a)$は$X$において稠密である.
				
			\item[(2)]
				任意の$u,v \in \Dom{M_a} = \Dom{M_{\conj{a}}} $に対して
				\begin{align}
					\inprod<M_a u,v> 
					= \int_X a(x) u(x) \conj{v(x)}\ \mu(dx)
					= \int_X u(x) \conj{\conj{a(x)} v(x)}\ \mu(dx)
					= \inprod<u,M_{\conj{a}}v>
				\end{align}
				が成り立つから,$v \in \Dom{M_a^*} $且つ$M_a^* v = M_{\conj{a}} v\ \left(\forall v \in \Dom{M_{\conj{a}}} \right)$が従う.
				逆に任意に$u \in \Dom{M_a} , v \in \Dom{M_a^*} $を取れば,
				\begin{align}
					\inprod<u,M_a^* v> = \inprod<M_a u,v> = \inprod<u,M_{\conj{a}}v>
				\end{align}
				となり$M_a^* v = M_{\conj{a}}v\ \left(\forall v \in \Dom{M_a^*} \right)$が従う.
			
			\item[(3)]
				$\lambda \in \C$を任意に取り固定し,任意の$\epsilon > 0$に対して
				$V_\epsilon \coloneqq a^{-1}(U_\epsilon(\lambda))$とおく.
			 	或る$\epsilon > 0$が存在して$\mu(V_\epsilon) = 0$が成り立つ場合,
			 	\begin{align}
			 		b(x) \coloneqq 
			 		\begin{cases}
			 			\frac{1}{\lambda - a(x)} & (x \in X \backslash V_\epsilon) \\
			 			0 & (x \in V_\epsilon)
			 		\end{cases}
			 	\end{align}
				と定めれば,任意の$u \in H$に対して
				\begin{align}
					\int_X |b(x)u(x)|^2\ \mu(dx)
					= \int_{X \backslash V_\epsilon} \frac{1}{|\lambda - a(x)|^2} |u(x)|^2\ \mu(dx)
					\leq \frac{1}{\epsilon^2} \int_X |u(x)|^2\ \mu(dx) < \infty 
				\end{align}
				が成り立つから$\Dom{M_b} = H$である.更に
				\begin{align}
					b (\lambda - a) u &= u \quad (\mbox{$\mu$-a.e.},\ \forall u \in \Dom{M_a} ), \\
					(\lambda - a) b u &= u \quad (\mbox{$\mu$-a.e.},\ \forall u \in H )
				\end{align}
				が成り立つから$M_b = (\lambda I - M_a)^{-1}$であり,$M_b \in \selfBop{H} $であるから
				$\lambda \in \Res{M_a} $が成り立つ.
				一方任意の$\epsilon > 0$に対して$\mu(V_\epsilon)  = 0$となる場合,
				任意に$\epsilon > 0$を取り固定する.
				\begin{align}
					\mu(V_\epsilon) = \lim_{n \to \infty} \mu(V_\epsilon \cap X_n)
				\end{align}
				が成り立つから,或る$N \in \N$が存在して$\mu(V_\epsilon \cap X_N) > 0$を満たす.
				\begin{align}
					u_\epsilon(x) \coloneqq
					\begin{cases}
						1 & (x \in V_\epsilon \cap X_N) \\
						0 & (x \notin V_\epsilon \cap X_N)
					\end{cases}
				\end{align}
				と定めれば,$u_\epsilon$は二乗可積分であり
				\begin{align}
					\int_X |a(x)u_\epsilon(x)|^2\ \mu(dx)
					= \int_{V_\epsilon \cap X_N} |a(x)u_\epsilon(x)|^2\ \mu(dx)
					< \infty
				\end{align}
				を満たすから$u_\epsilon \in \Dom{M_a} $である.また
				\begin{align}
					\Norm{(\lambda I- M_a)u_\epsilon}{}^2
					= \int_X |\lambda - a(x)|^2|u_\epsilon(x)|^2\ \mu(dx)
					= \int_{V_\epsilon \cap X_N} |\lambda - a(x)|^2|u_\epsilon(x)|^2\ \mu(dx)
					\leq \epsilon^2 \int_X |u_\epsilon(x)|^2\ \mu(dx)
					= \epsilon^2 \Norm{u_\epsilon}{}^2
				\end{align}
				を満たす.つまり任意の$\epsilon$に対し或る$u_\epsilon \in \Dom{M_a} $が存在し
				\begin{align}
					\frac{1}{\epsilon} \leq \frac{\Norm{u_\epsilon}{}}{\Norm{(\lambda I- M_a)u_\epsilon}{}}
				\end{align}
				を満たす.ここで$(\lambda I- M_a)$に対し逆作用素$(\lambda I- M_a)^{-1}$が存在するとしても,
				$u_\epsilon \in \Dom{M_a} $に対して或る$v_\epsilon \in \Dom{(\lambda I- M_a)^{-1}} $が存在して
				$u_\epsilon = (\lambda I- M_a)^{-1}v_\epsilon$を満たすが,
				\begin{align}
					\frac{1}{\epsilon} \leq \frac{\Norm{(\lambda I- M_a)^{-1}v_\epsilon}{}}{\Norm{v_\epsilon}{}}
				\end{align}
				が従い,$\epsilon$の任意性より$(\lambda I- M_a)^{-1}$の作用素ノルムは非有界である.
				従ってこの場合$\lambda \in \Spctr{M_a} $が成り立つ.
				
			\item[(4)] 
				先ず$\pSpctr{M_a} \subset \Set{\lambda \in \C}{\mu\left( a^{-1}(\{\lambda\}) \right) > 0}$が成り立つことを示す.
				任意の$\lambda \in \pSpctr{M_a}$に対して固有ベクトル$u \in H$が存在する.$u \neq 0\ $(関数類の意味で)より
				\begin{align}
					N \coloneqq \Set{x \in X}{u(x) \neq 0}
				\end{align}
				とおけば$\mu(N) > 0$が満たされる.一方で点スペクトルの定義より$(\lambda I - M_a)u = 0$が成り立つから
				\begin{align}
					0 = \Norm{(\lambda I - M_a)u}{}^2 = \int_X |\lambda - a(x)|^2 |u(x)|^2\ \mu(dx)
					= \int_{N} |\lambda - a(x)|^2 |u(x)|^2\ \mu(dx)
				\end{align}
				となり
				\begin{align}
					\mu\left( \Set{x \in N}{|\lambda - a(x)| > 0} \right) = 0
				\end{align}
				が従う.$\mu(N) > 0$であるから
				\begin{align}
					\mu\left( a^{-1}(\{\lambda\}) \right)
					\geq \mu\left( \Set{x \in N}{|\lambda - a(x)| = 0} \right)
					> 0
				\end{align}
				が成り立ち$\lambda \in \Set{\lambda \in \C}{\mu\left( a^{-1}(\{\lambda\}) \right) > 0}$を得る.
				次に$\pSpctr{M_a} \supset \Set{\lambda \in \C}{\mu\left( a^{-1}(\{\lambda\}) \right) > 0}$が成り立つことを示す.
				任意の$\lambda \in \Set{\lambda \in \C}{\mu\left( a^{-1}(\{\lambda\}) \right) > 0}$に対して
				\begin{align}
					\Lambda \coloneqq a^{-1}(\{\lambda\})
				\end{align}
				とおけば$\mu(\Lambda) > 0$が満たされている.
				\begin{align}
					\mu(\Lambda) = \lim_{n \to \infty} \mu(\Lambda \cap X_n)
				\end{align}
				が成り立つから,或る$n \in \N$が存在して$\mu(\Lambda \cap X_n) > 0$を満たす.
				\begin{align}
					u(x) \coloneqq 
					\begin{cases}
						1 & (x \in \Lambda \cap X_n), \\
						0 & (x \notin \Lambda \cap X_n)
					\end{cases}
				\end{align}
				として$u$を定めれば$u$は二乗可積分であり,$\mu(\Lambda \cap X_n) > 0$であるから関数類として$u \neq 0$を満たす.また
				\begin{align}
					\Norm{(\lambda I - M_a)u}{}^2
					= \int_X |\lambda - a(x)|^2 |u(x)|^2\ \mu(dx)
					= \int_{\Lambda \cap X_n} |\lambda - a(x)|^2 |u(x)|^2\ \mu(dx)
					= 0
				\end{align}
				が成り立ち$(\lambda I - M_a)u = 0$が従うから$u$は$\lambda$の固有ベクトルであり,$\lambda \in \pSpctr{M_a}$を得る.
		\end{description}
	\end{prf}
\chapter{コンパクト作用素}
	
	係数体を$\C$,$X,Y$をノルム空間とし,$K$を$X$から$Y$への線型作用素とする.
	また$X,Y$及び共役空間$X^*,Y^*$におけるノルムを
	$\Norm{\cdot}{X},\ \Norm{\cdot}{Y},\ \Norm{\cdot}{X^*},\ \Norm{\cdot}{Y^*}$と表記し,
	位相はこれらのノルムにより導入する.
	
	\begin{screen}
		\begin{dfn}[コンパクト作用素]
			$K$がコンパクト作用素(compact operator)であるということを次で定義する:
			\begin{itemize}
				\item $\Dom{K} = X$を満たし,かつ$X$の任意の有界部分集合$B$に対して$KB$が相対コンパクト
					($KB$の閉包$\closure{KB}$がコンパクト)となる.
			\end{itemize}
		\end{dfn}
	\end{screen}
	
	\begin{screen}
		\begin{lem}[コンパクト作用素となるための十分条件の一つ]
			$\Dom{K} = X$とする.$B_1 \coloneqq \Set{x \in X}{\Norm{x}{X} < 1}$に対して$\closure{KB_1}$が
			コンパクトであるなら$K$はコンパクト作用素となる.
		\end{lem}
	\end{screen}
	
	\begin{prf}
		$B \subset X$が有界集合なら或る$\lambda > 0$が存在して$B \subset \lambda B_1\ (= \Set{\lambda x}{x \in B_1})$
		が成り立つ.$\closure{K(\lambda B_1)}$がコンパクトとなるならその閉部分集合である$\closure{KB}$もコンパクトとなるから,
		$\closure{K(\lambda B_1)}$がコンパクトとなることを示せばよい.先ず
		\begin{align}
			\closure{K(\lambda B_1)} = \lambda \closure{KB_1}
		\end{align}
		が成り立つことを示す.$x \in \closure{K(\lambda B_1)}$に対しては点列$(x_n)_{n=1}^{\infty} \subset K(\lambda B_1)$が取れて
		$\Norm{x_n - x}{X} \longrightarrow 0\ (n \longrightarrow \infty)$を満たす.
		$y_n \coloneqq x_n/\lambda$とおけば$K$の線型性により$y_n \in KB_1$となり,
		$\Norm{y_n - x/\lambda}{X}= \Norm{x_n - x}{X}/\lambda \longrightarrow 0\ (n \longrightarrow \infty)$
		となるから$x/\lambda \in \closure{KB_1}$,すなわち$x \in \lambda\closure{KB_1}$である.
		逆に$x \in \lambda \closure{KB_1}$に対しては$x/\lambda \in \closure{KB_1}$となるから,
		或る点列$(t_n)_{n=1}^{\infty} \subset KB_1$が存在して$\Norm{t_n - x/\lambda}{X} \longrightarrow 0\ (n \longrightarrow \infty)$
		を満たす.$s_n = \lambda t_n$とおけば$K$の線型性により$s_n \in K(\lambda B_1)$となり,
		$\Norm{s_n - x}{X}= \lambda \Norm{t_n - x/\lambda}{X} \longrightarrow 0\ (n \longrightarrow \infty)$
		が成り立つから$x \in \closure{K(\lambda B_1)}$である.以上で$\closure{K(\lambda B_1)} = \lambda \closure{KB_1}$が示された.
		$\closure{K(\lambda B_1)}$を覆う任意の開被覆$\cup_{\mu \in M}O_\mu\ $($M$は任意濃度)に対し
		\begin{align}
			\closure{KB_1} \subset \bigcup_{\mu \in M} \tfrac{1}{\lambda}O_\mu
		\end{align}
		が成り立ち\footnote{開集合$O_\mu$は$1/\lambda$でスケールを変えてもまた開集合となる.},仮定より$\closure{KB_1}$はコンパクトであるから,$M$から有限個の添数$\mu_i\ (i=1,\cdots,n)$を取り出して
		\begin{align}
			\closure{KB_1} \subset \bigcup_{i=1}^{n} \tfrac{1}{\lambda}O_{\mu_i}
		\end{align}
		となる.
		\begin{align}
			\closure{K(\lambda B_1)} = \lambda \closure{KB_1} \subset \bigcup_{i=1}^{n} O_{\mu_i}
		\end{align}
		が従うから$\closure{K(\lambda B_1)}$はコンパクトである.
		\QED
	\end{prf}
	
	\begin{screen}
		\begin{lem}[コンパクト作用素であることの同値条件]
			$\Dom{K} = X$とする.(1)$K$がコンパクトであることと,(2)$X$の任意の有界点列$(x_n)_{n=1}^{\infty}$に対し点列$(Tx_n)_{n=1}^{\infty}$が
			$\closure{(Tx_n)_{n=1}^{\infty}}$で収束する部分列を含むことは同値である.
			\label{lem:compact_operator_equiv_cond}
		\end{lem}
	\end{screen}
	
	\begin{prf}\mbox{}
		\begin{description}
			\item[(1)$\Rightarrow$(2)]
				$(x_n)_{n=1}^{\infty}$は$X$において有界集合であるから$(Kx_n)_{n=1}^{\infty}$は相対コンパクトである.
				距離空間におけるコンパクト性の一般論により$\closure{(Kx_n)_{n=1}^{\infty}}$は点列コンパクトとなり(2)が従う.
			\item[(2)$\Rightarrow$(1)]
				距離空間の一般論より,任意の有界集合$B \subset X$に対して$\closure{TB}$がコンパクトとなることと
				$\closure{TB}$が点列コンパクトとなることは同値である.従って次の主張
				\begin{itembox}[l]{主張(※)}
					$TB$の任意の点列が$\closure{TB}$で収束する部分列を含むなら$\closure{TB}$は点列コンパクトである.
				\end{itembox}
				を示せばよい.実際(※)が示されたとする.
				$TB$から任意に点列$(y_n)_{n=1}^{\infty}$を取れば,これに対し或る$(x_n)_{n=1}^{\infty} \subset B$が対応して
				$y_n = Tx_n\ (n=1,2,\cdots)$と表現され,(2)の仮定より$(y_n)_{n=1}^{\infty}$は
				$\closure{(y_n)_{n=1}^{\infty}}$で収束する部分列を持つ.
				よって(※)と上の一般論により$\closure{TB}$はコンパクトとなる.
				(※)を示す.$\closure{TB}$の任意の点列$(y_n)_{n=1}^{\infty}$に対して
				$\Norm{y_n - z_n}{Y} < 1/n\ (n=1,2,\cdots)$を満たす$(z_n)_{n=1}^{\infty} \subset TB$が存在する.
				部分列$(z_{n_k})_{k=1}^{\infty}$が$y \in \closure{TB}$に収束するなら,任意の$\epsilon > 0$に対し
				或る$K_1 \in \N$が取れて$k \geq K_1$ならば$\Norm{y - z_{n_k}}{Y} < \epsilon/2$を満たす.
				更に或る$K_2 \in \N$が取れて$k \geq K_2$なら$1/n_k < \epsilon/2$も満たされるから,全ての$k \geq \max{}{\{K_1,K_2\}}$
				に対して
				\begin{align}
					\Norm{y - y_{n_k}}{Y} \leq \Norm{y - z_{n_k}}{Y} + \Norm{z_{n_k} - y_{n_k}}{Y} < \epsilon
				\end{align}
				が成り立つ.
		\end{description}
		\QED
	\end{prf}
	
	\begin{screen}
		\begin{dfn}[コンパクト作用素の空間]
			ここで新しく次の表記を導入する:
			\begin{align}
				\Cop{X}{Y} \coloneqq \Set{K:X \rightarrow Y}{\mbox{$K$はコンパクト作用素}}.
			\end{align}
			有界作用素の空間に似た表記をしているが,定義右辺では作用素の有界性を要件に入れていない.しかし実際コンパクト作用素は有界である(命題\ref{prp:compact_operator_bounded_composition_of_compact_operators}).
		\end{dfn}
	\end{screen}
	
	\begin{screen}
		\begin{prp}[コンパクト作用素の有界性・コンパクト作用素の合成のコンパクト性]\mbox{}
			\begin{description}
				\item[(1)] $\Cop{X}{Y} $は$\Bop{X}{Y} $の線型部分空間となる.
				\item[(2)] $Z$をノルム空間とする.$A \in \Bop{X}{Y} $と$B \in \Bop{Y}{Z} $に対して$A$又は$B$がコンパクト作用素なら$BA$もまたコンパクト作用素となる.
			\end{description}
			\label{prp:compact_operator_bounded_composition_of_compact_operators}
		\end{prp}
	\end{screen}
	
	\begin{prf}\mbox{}
		\begin{description}
			\item[(1)] 
				任意に$K \in \Cop{X}{Y} $を取れば,コンパクト作用素の定義より$\Dom{K} = X$が満たされている.
				また$B_1 \coloneqq \Set{x \in X}{\Norm{x}{X} \leq 1}$とおけば,
				$\closure{KB_1}$のコンパクト性により$KB_1$は有界であるから
				\begin{align}
					\sup{0 < \Norm{x}{X} \leq 1}{\Norm{Kx}{Y}} = \sup{x \in B_1 \backslash \{0\}}{\Norm{Kx}{Y}} < \infty
				\end{align}
				となり$K \in \Bop{X}{Y} $が従う.次に$\Cop{X}{Y} $が線形空間であることを示す.
				$K_1, K_2 \in \Cop{X}{Y} $と$\alpha \in \C$を任意に取る.
				補助定理\ref{lem:compact_operator_equiv_cond}より,
				$X$の任意の有界点列$(x_n)_{n=1}^{\infty}$に対して$\left( (K_1 + K_2)\left(x_n\right) \right)_{n=1}^{\infty}$
				と$\left( (\alpha K_1)\left(x_n\right) \right)_{n=1}^{\infty}$が収束部分列を含むことを示せばよい.
				補助定理\ref{lem:compact_operator_equiv_cond}により,$(K_1x_n)_{n=1}^{\infty}$
				は$\closure{(K_1x_n)_{n=1}^{\infty}}$で収束する部分列$\left(K_1x_{n(1,k)}\right)_{k=1}^{\infty}$を持つ.
				また$\left(K_2x_{n(1,k)}\right)_{k=1}^{\infty}$も$\closure{\left(K_2x_{n(1,k)}\right)_{k=1}^{\infty}}$で
				収束する部分列$\left(K_2x_{n(2,k)}\right)_{k=1}^{\infty}$を持ち,
				更に$\left(K_1x_{n(2,k)}\right)_{k=1}^{\infty}$は収束列$\left(K_1x_{n(1,k)}\right)_{k=1}^{\infty}$の部分列となるから,
				$\left( (K_1 + K_2)\left(x_{n(2,k)}\right) \right)_{k=1}^{\infty}$が収束列となり
				$K_1 + K_2 \in \Cop{X}{Y} $が従う.$(\alpha K_1x_{n(1,k)})_{k=1}^{\infty}$
				もまた収束列であるから$\alpha K_1 \in \Cop{X}{Y} $も従う.以上より$\Cop{X}{Y} $は線形空間である.
			
			\item[(2)]
				\begin{description}
					\item[$A$がコンパクト作用素である場合]
						補助定理\ref{lem:compact_operator_equiv_cond}により,$X$の任意の点列$(x_n)_{n=1}^{\infty}$に対し$(Ax_n)_{n=1}^{\infty}$は収束部分列
						$\left(Ax_{n_k}\right)_{k=1}^{\infty}$を持つ.$B$の連続性により$\left(BAx_{n_k}\right)_{k=1}^{\infty}$も
						収束列となるから,補助定理\ref{lem:compact_operator_equiv_cond}より$BA$はコンパクト作用素である.
					
					\item[$B$がコンパクト作用素である場合]
						任意の有界集合$S \subset X$に対して,$A$の有界性と併せて$AS$は有界となる.従って$\closure{BAS}$がコンパクトとなるから
						$BA$はコンパクト作用素である.
				\end{description}
		\end{description}
		\QED
	\end{prf}
	
	
	\begin{screen}
		\begin{prp}[$\Cop{X}{Y} $の位相]
			$Y$がBanach空間ならば$\Cop{X}{Y} $は$\Bop{X}{Y} $の閉部分空間である.
		\end{prp}
	\end{screen}
	
	\begin{prf}
		$Y$がBanach空間ならば$\Bop{X}{Y} $は作用素ノルム$\Norm{\cdot}{\Bop{X}{Y} }$についてBanach空間となる.従って
		$\Cop{X}{Y} $の任意のCauchy列は少なくとも$\Bop{X}{Y} $で収束する.
		補助定理\ref{lem:compact_operator_equiv_cond}により,次のことを示せばよい.
		\begin{itemize}
			\item $A_n \in \Cop{X}{Y} \ (n=1,2,\cdots)$がCauchy列をなし$A \in \Bop{X}{Y} $に収束するとき,
			$X$の任意の有界点列$(x_n)_{n=1}^{\infty}$に対して$(Ax_n)_{n=1}^{\infty}$が$Y$で収束する部分列を持つ.
		\end{itemize}
		証明には対角線論法を使う.先ず$A_1$について,補助定理\ref{lem:compact_operator_equiv_cond}により
		$\left(A_1x_n\right)_{n=1}^{\infty}$の或る部分列$\left(A_1x_{k(1,j)}\right)_{j=1}^{\infty}$は収束する.$A_2$についても
		$\left(A_2x_{k(1,j)}\right)_{j=1}^{\infty}$の或る部分列$\left(A_2x_{k(2,j)}\right)_{j=1}^{\infty}$は収束する.
		以下収束部分列を抜き取る操作を繰り返し,一般の$A_n$に対して$\left(A_nx_{k(n,j)}\right)_{j=1}^{\infty}$
		が収束列となるようにできる.ここで$x_{k_j} \coloneqq x_{k(j,j)}\ (j=1,2,\cdots)$として点列$(x_{k_j})_{j=1}^{\infty}$
		を定めれば,これは$(x_n)_{n=1}^{\infty}$の部分列であり,また全ての$n=1,2,\cdots$に対して
		$\left(A_nx_{k_j}\right)_{j=n}^{\infty}$は収束列$\left(A_nx_{k(n,j)}\right)_{j=1}^{\infty}$の部分列となるから
		$\left(A_nx_{k_j}\right)_{j=1}^{\infty}$は収束列である.
		この$(x_{k_j})_{j=1}^{\infty}$に対して$\left(Ax_{k_j}\right)_{j=1}^{\infty}$が
		Cauchy列をなすならば$A$のコンパクト性が従う
		\footnote{
			$Y$がBanach空間であるからCauchy列であることと収束列であることは同値である.
		}
		.$A_n \rightarrow A$を書き直せば,任意の$\epsilon > 0$に対して或る$N = N(\epsilon) \in \N$が存在し,$n > N$なら$\Norm{A_n - A}{\Bop{X}{Y} } < \epsilon$となる.
		また$n > N$を満たす$n$を一つ取れば,$\left(A_nx_{k_j}\right)_{j=1}^{\infty}$は収束列であるから或る$J=J(n,\epsilon) \in \N$が存在し
		全ての$j_1,j_2 > J$に対して$\Norm{A_nx_{k_{j_1}} - A_nx_{k_{j_2}}}{Y} < \epsilon$が成り立つ.
		$M \coloneqq \sup{n\in\N}{\Norm{x_n}{X}} < \infty$とおけば,全ての$j_1,j_2 > J$に対して
		\begin{align}
			\Norm{Ax_{k_{j_1}} - Ax_{k_{j_2}}}{Y} \leq M\Norm{A- A_n}{\Bop{X}{Y} } + \Norm{A_nx_{k_{j_1}} - A_nx_{k_{j_2}}}{Y} + M\Norm{A- A_n}{\Bop{X}{Y} } < (2M+1)\epsilon
		\end{align}
		が従うから,$\left(Ax_{k_j}\right)_{j=1}^{\infty}$はCauchy列すなわち収束列である.
		\QED
	\end{prf}
	
	\begin{screen}
		\begin{thm}[コンパクト作用素の共役作用素のコンパクト性]
			$X,Y$がBanach空間ならば次が成り立つ:
			\begin{align}
				A \in \Cop{X}{Y}  \quad \Leftrightarrow \quad A^* \in \Cop{Y^*}{X^*} .
			\end{align}
			\label{thm:dual_operator_of_compact_operators}
		\end{thm}
	\end{screen}
	
	\begin{prf}\mbox{}
		\begin{description}
			\item[$\Rightarrow$について]
				定理\ref{thm:dual_operator_bounded}より$A \in \Bop{X}{Y} $なら$A^* \in \Bop{Y^*}{X^*} $が成り立つ.
				\begin{align}
					S_1 \coloneqq \Set{x \in X}{0 < \Norm{x}{X} \leq 1}
				\end{align}
				とおけば仮定より$L \coloneqq \closure{AS}$は$Y$のコンパクト部分集合であり,
				任意に有界点列$(y_n^*)_{n=1}^{\infty} \subset Y^*$を取り
				\begin{align}
					f_n:L \ni y \longmapsto y_n^*(y) \in \C
					\quad (n=1,2,\cdots)
				\end{align}
				と定める.関数族$(f_n)_{n=1}^{\infty}$は正規族となる
				\footnote{
					関数族$(f_n)_{n=1}^{\infty}$の同等連続性と各点での有界性を示す.
					\begin{description}
						\item[同等連続性]
							$(y_n^*)_{n=1}^{\infty}$は有界であるから,$M \coloneqq \sup{n\in\N}{\Norm{y_n^*}{Y^*}}$とおけば
							\begin{align}
								|f_n(y_1) - f_n(y_2)| = |y_n^*(y_1) - y_n^*(y_2)| \leq M\Norm{y_1 - y_2}{Y} \quad (\forall y_1,y_2 \in L,\ n=1,2,\cdots)
							\end{align}
							が成り立ち同等連続性が従う.
						
						\item[各点で有界]
							上で定めた$M$に対し
							\begin{align}
								|f_n(y)| \leq M \Norm{y}{Y} \quad (\forall y \in L,\ n=1,2,\cdots)
							\end{align}
							が成り立つ.
					\end{description}
				}
				から,
				Ascoli-Arzelaの定理により$L$上の連続関数の全体$C(L)$において収束する部分列$\left(f_{n_k}\right)_{k=1}^{\infty}$を含む.
				\begin{align}
					\Norm{A^*y_{n_k}^* - A^*y_{n_j}^*}{X^*} &= \sup{x \in S_1}{\left| \inprod<x,A^*y_{n_k}^* - A^*y_{n_j}^*>_{X,X^*} \right|} && (\scriptsize\because \mbox{作用素ノルムの定義より.}) \\
					&= \sup{x \in S_1}{\left| \inprod<Ax, y_{n_k}^* - y_{n_j}^*>_{Y,Y^*} \right|} && (\scriptsize\because \mbox{$\Dom{A^*} =Y^*$より.}) \\
					&= \sup{y \in AS_1}{\left| \inprod<y, y_{n_k}^* - y_{n_j}^*>_{Y,Y^*} \right|} \\
					&= \sup{y \in L}{\left| \inprod<y, y_{n_k}^* - y_{n_j}^*>_{Y,Y^*} \right|} && (\scriptsize\because \mbox{$y_{n_k}^* - y_{n_j}^*$の連続性より.}) \\
					&= \Norm{f_{n_k} - f_{n_j}}{C(L)} && (\scriptsize\because \mbox{$C(L)$におけるsup-normを表す.})
				\end{align}
				が成り立つ.$\left(f_{n_k}\right)_{k=1}^{\infty}$がsup-normについてCauchy列をなすから
				$\left(A^*y_{n_k}^*\right)_{k=1}^{\infty}$もCauchy列となり,$X^*$の完備性と
				補助定理\ref{lem:compact_operator_equiv_cond}より$A^* \in \Cop{Y^*}{X^*} $が従う.
				
			\item[$\Leftarrow$について]\mbox{}
				\begin{description}
					\item[証明1]
						$J_X:X \longrightarrow X^{**},\ J_Y:Y \longrightarrow Y^{**}$を自然な等長埋め込みとする.任意に$x \in X$を取れば
						\begin{align}
							\inprod<A^*y^*, J_Xx>_{X^*,X^{**}} = \inprod<x,A^*y^*>_{X,X^*} = \inprod<Ax,y^*>_{Y,Y^*} = \inprod<y^*,J_YAx>_{Y^*,Y^{**}} \quad (\forall y^* \in Y^*=\Dom{A^*} )
						\end{align}
						が成り立ち,$\mathscr{D}(A^*)=Y^*$であるから$A^{**}$が定義され
						\begin{align}
							A^{**} J_X x = J_Y A x \quad (\forall x \in X)
							\label{eq:thm_dual_operator_of_compact_operators}
						\end{align}
						が従う.また前段の結果と$A^*$のコンパクト性から$A^{**}$もコンパクト作用素となる.
						$X$から任意に有界点列$(x_n)_{n=1}^{\infty}$を取れば,
						$J_X$の等長性より$\left(J_Xx_n\right)_{n=1}^{\infty}$も$X^{**}$において有界となり,
						補助定理\ref{lem:compact_operator_equiv_cond}により
						$\left(A^{**}J_Xx_n\right)_{n=1}^{\infty}$の或る部分列$\left(A^{**}J_Xx_{n_k}\right)_{k=1}^{\infty}$
						はCauchy列となる.(\refeq{eq:thm_dual_operator_of_compact_operators})より
						$\left(J_YAx_{n_k}\right)_{k=1}^{\infty}$もCauchy列となるから,
						$J_Y$の等長性より$\left(Ax_{n_k}\right)_{k=1}^{\infty}$はBanach空間$Y$で収束し$A \in \Cop{X}{Y} $が従う.
					\item[証明2]
						$X$の任意の有界点列$(x_n)_{n=1}^{\infty}$に対して
						\begin{align}
							\Norm{Ax_n}{Y} = \sup{\Norm{y^*}{Y^*} \leq 1}{|y^*(Ax_n)|}
							= \sup{\Norm{y^*}{Y^*} \leq 1}{|\inprod<y^*,Ax_n>_{Y^*,Y}|}
							= \sup{\Norm{y^*}{Y^*} \leq 1}{|\inprod<A^*y^*,x_n>_{X^*,X}|}
							= \sup{x^* \in V}{|\inprod<x^*,x_n>_{X^*,X}|}
						\end{align}
						が成り立つ.ただし$V \coloneqq \closure{\left\{\ A^*y^*\quad |\quad \Norm{y^*}{Y^*} \leq 1\ \right\}}$としていて,
						また第1の等号は
						\begin{align}
							\Norm{y}{Y} = \sup{\substack{0 \neq g \in Y^* \\ \Norm{g}{Y^*}\leq1}}{\frac{|g(y)|}{\Norm{g}{Y^*}}} 
							= \sup{\Norm{g}{Y^*}=1}{|g(y)|} = \sup{\Norm{g}{Y^*} \leq 1}{|g(y)|}
						\end{align}
						の関係を使った\footnote{Hahn-Banachの定理の系を参照.始めのsupは$\Norm{g}{Y^*}\leq1$の範囲で制限しているが,等号成立する$g$のノルムが1であるから問題ない.}.
						$A^*$がコンパクトだから$V$が$X^*$のコンパクト集合となるから$M \coloneqq \sup{x^* \in V}{\Norm{x^*}{X^*}}$とおけば$M < \infty$
						である.また$\left(\Norm{x_n}{X}\right)_{n=1}^{\infty}$は$\R$において有界列となるから
						収束する部分列$\left(\Norm{x_{n_k}}{X}\right)_{k=1}^{\infty}$を取ることができる.
						この部分列と全ての$x^* \in V$に対して
						\begin{align}
							|x^*(x_{n_k}) - x^*(x_{n_j})| \leq M\Norm{x_{n_k} - x_{n_j}}{X} \longrightarrow 0 \quad (k,j \longrightarrow \infty)
						\end{align}
						が成り立つから,
						\begin{align}
							\Norm{Ax_{n_k} - Ax_{n_j}}{Y} = \sup{x^* \in V}{\left|\inprod<x^*,x_{n_k} - x_{n_j}>_{X^*,X}\right|} \longrightarrow 0 \quad (k,j \longrightarrow \infty)
						\end{align}
						が従い$A \in B_c(X,Y)$が判明する.
				\end{description}
				\QED
		\end{description}
	\end{prf}
	
	\begin{screen}
		\begin{thm}[反射的Banach空間の弱点列コンパクト性]\mbox{}\\
			$X$が反射的Banach空間なら,$X$の任意の有界点列は弱収束する部分列を含む.
			\label{thm:weak_seq_compact}
		\end{thm}
	\end{screen}
	
	\begin{screen}
		\begin{thm}[有限次元空間における有界点列の収束]
			$X,Y$をノルム空間とし$A \in B(X,Y)$とする.このとき
			$\Rank{A} = \Dim{\Ran{A}} < \infty$ならば$A \in B_c(X,Y)$となる.\footnote{$X,Y$がHilbert空間であるなら逆が成立する.}
		\end{thm}
	\end{screen}
	
	\begin{prf}
		$A$の線型性と有界性から$\Ran{A} = AX$は有界な有限次元空間となるから,後の結論までは次の定理による.
		\QED
	\end{prf}
	%有限次元空間における有界点列の収束
\begin{screen}
	\begin{thm}[有限次元空間における有界点列の収束(局所コンパクト性)]\mbox{}\\
		$\K$を$\R$又は$\C$とし,$X$を$\K$上のノルム空間とする.$\Dim{X} < \infty$ならば$X$の任意の有界点列は
		収束部分列を含む.
	\end{thm}
\end{screen}

\begin{prf}\mbox{}
	$X$の次元数$n$による帰納法で証明する.
	\begin{description}
		\item[第一段]
			$n=1$のとき$X$の基底を$u_1$とすれば,$X$の任意の有界点列は
			$( \alpha_m u_1)_{m=1}^{\infty} \quad (\alpha_m \in \K,\ m=1,2,\cdots)$と表せる.
			$\left( \alpha_m \right)_{m=1}^{\infty}$は有界列であるから,
			Bolzano-Weierstrassの定理より部分列$\left( \alpha_{m_k} \right)_{k=1}^{\infty}$と$\alpha \in \K$が存在して
			\begin{align}
				\left| \alpha_{m_k} - \alpha \right| \longrightarrow 0 \quad (k \longrightarrow \infty)
			\end{align}
			を満たし
			\begin{align}
				\Norm{\alpha_{m_k} u_1 - \alpha u_1}{X} \longrightarrow 0 \quad (k \longrightarrow \infty)
			\end{align}
			が従う.
		
		\item[第二段]
			$n=k$のとき定理の主張が成り立つと仮定し,$n = k+1$として$X$の基底を$u_1,\cdots,u_{k+1}$と表す.
			$X$から任意に有界列$(x_j)_{j=1}^{\infty}$を取れば,各$x_j$は
			\begin{align}
				x_j = y_j + \beta_j u_{k+1} \quad (y_j \in \LH{\left\{\, u_1,\cdots,u_k\, \right\}},\ \beta_j \in \K)
			\end{align}
			として一意に表示される.$(\beta_j)_{j=1}^{\infty}$が有界でないと仮定すると
			$\beta_{j_s} \geq s\ (j_s < j_{s+1},\ s=1,2,\cdots)$を満たす部分列が存在し,$(x_j)_{j=1}^{\infty}$の有界性と併せて
			\begin{align}
				\Norm{u_{k+1} + \tfrac{1}{\beta_{j_s}}y_{j_s}}{X}
				\leq \Norm{u_{k+1} + \tfrac{1}{\beta_{j_s}}y_{j_s} - \tfrac{1}{\beta_{j_s}}x_{j_s}}{X}
					+ \Norm{\tfrac{1}{\beta_{j_s}}x_{j_s}}{X}
				= \Norm{\tfrac{1}{\beta_{j_s}}x_{j_s}}{X} \longrightarrow 0 \quad (s \longrightarrow \infty)
			\end{align}
			が成り立つが,有限次元空間は閉であるから
			$u_{k+1} \in \LH{\left\{\, u_1,\cdots,u_k\, \right\}}$が従い矛盾が生じる.よって$(\beta_j)_{j=1}^{\infty}$は
			$\K$の有界列でなくてはならず,Bolzano-Weierstrassの定理より部分列$\left( \beta_{j(1,i)} \right)_{i=1}^{\infty}$と$\beta \in \K$が存在して
			\begin{align}
				\left| \beta_{j(1,i)} - \beta \right| \longrightarrow 0 \quad (i \longrightarrow \infty)
			\end{align}
			を満たす.また$\left(y_{j(1,i)}\right)_{i=1}^{\infty}$も有界列となるから,或る
			$y \in \LH{\left\{\, u_1,\cdots,u_k\, \right\}}$と部分列$\left(y_{j(2,i)}\right)_{i=1}^{\infty}$が存在して
			\begin{align}
				\Norm{y_{j(2,i)} - y}{X} \longrightarrow 0 \quad (i \longrightarrow \infty)
			\end{align}
			を満たす.従って
			\begin{align}
				\Norm{x_{j(2,i)} - \left(y + \beta u_{k+1}\right)}{X} \leq
				\Norm{y_{j(2,i)} - y}{X} + \left| \beta_{j(1,i)} - \beta \right| \Norm{u_{k+1}}{X}
				\longrightarrow 0 \quad (i \longrightarrow \infty)
			\end{align}
			が成り立つ.
			\QED
	\end{description}
\end{prf}
	
	\begin{screen}
		\begin{thm}[反射的Banach空間上のコンパクト作用素は弱収束列を強収束列に写す]\mbox{}\\
			$X$を反射的Banach空間,$Y$をBanach空間,そして$A \in B(X,Y)$とする.このとき次が成り立つ.
			\begin{description}
				\item[(1)] $A \in B_c(X,Y)$なら$A$は$X$の任意の弱収束列を強収束列に写す.
				\item[(2)] 逆に$A$が$X$の任意の弱収束列を強収束列に写すなら$A \in B_c(X,Y)$である.
			\end{description}
		\end{thm}
	\end{screen}
	
	\begin{prf}
		\begin{description}
			\item[(1)] $(x_n)_{n=1}^{\infty}$を$X$の任意の弱収束列とする.$\wlim x_n = x$として,示すべきことは(i)$(Ax_n)_{n=1}^{\infty}$
				の任意の部分列が収束部分列を含み,(ii)収束先は全て$Ax$である,の二つである.この二つが示されれば,距離空間における
				点列の収束の一般論より(1)の主張が従う.(i)について,一般にノルム空間の弱収束列は有界であるから
			\item[(2)]
		\end{description}
	\end{prf}
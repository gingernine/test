\section{スペクトル測度}
	$H$を複素Hilbert空間,その上の直交射影全体を$\Oproj{H}$とし,$H$における内積とノルムをそれぞれ$\inprod<\cdot,\cdot>,\Norm{\cdot}{}$で表す.
	また$(X,\mathcal{M})$を可測空間とする.
	
	\begin{screen}
		\begin{dfn}[スペクトル測度]
			$I$を$H$上の恒等写像とする.$E:\mathcal{M} \rightarrow \Oproj{H}$がスペクトル測度(spectral measure)であるとは,
			$E(X) = I$かつ,互いに素な列$A_n \in \mathcal{M}\ (n=1,2,\cdots)$に対して次を満たすことをいう:
			\begin{align}
				E(A_n) \neq E(A_m) \quad (n \neq m,\ A_n,A_m \neq \emptyset), \quad
				\sum_{n=1}^{\infty} E(A_n)u = E(\sum_{n=1}^{\infty} A_n)u \quad (\forall u \in H).
				\label{eq:sigma_additivity_spectral_measures}
			\end{align}
			\label{dfn:spectral_measure}
		\end{dfn}
	\end{screen}
	
	\begin{screen}
		\begin{lem}[スペクトル測度の積]
			$\mathcal{M}$から$\Oproj{H}$へのスペクトル測度$H$は次を満たす:
			\begin{description}
				\item[(1)] $E(\emptyset) = 0.$
				\item[(2)] $A,B \in \mathcal{M}$に対し$E(A) E(B) = E(A \cap B)$.
			\end{description}
			\label{lem:product_of_spectral_measure}
		\end{lem}
	\end{screen}
	
	\begin{prf}\mbox{}
		\begin{description}
			\item[(1)] (\refeq{eq:sigma_additivity_spectral_measures})において$A_n = \emptyset\ (n=1,2,\cdots)$とすれば
				\begin{align}
					E(\emptyset) = \sum_{n=1}^{\infty} E(\emptyset) = E(\emptyset) + \sum_{n=2}^{\infty} E(\emptyset)
				\end{align}
				が成り立ち
				\footnote{
					\begin{align}
						a \coloneqq \sum_{n=1}^{\infty} E(\emptyset),\quad a' \coloneqq \sum_{n=2}^{\infty} E(\emptyset),
						\quad a_N \coloneqq \sum_{n=1}^{N} E(\emptyset),\quad a'_N \coloneqq \sum_{n=2}^{N} E(\emptyset)
					\end{align}
					とおけば
					\begin{align}
						\Norm{a - \left( E(\emptyset) - a' \right)}{}
						\leq \Norm{a - a_N}{} + \Norm{a_N - \left( E(\emptyset) - a'_N \right)}{} + \Norm{a'_N - a'}{}
						\longrightarrow 0 \quad (N \longrightarrow \infty)
					\end{align}
					が成り立つ.
				}
				\begin{align}
					\sum_{n=2}^{\infty} E(\emptyset) = 0
				\end{align}
				が従う.
				\begin{align}
					S \coloneqq \sum_{n=2}^{\infty} E(\emptyset),
					\quad S_N \coloneqq \sum_{n=2}^{N} E(\emptyset) \quad (N=1,2,\cdots)
				\end{align}
				とおけば,$(S_N)_{N=2}^{\infty}$は$S$にノルム収束するからCauchy列であり,
				\begin{align}
					\Norm{S_{N+1} - S_{N}}{} \longrightarrow 0 \quad (N \longrightarrow \infty)
				\end{align}
				より$E(\emptyset) = 0$が得られる.
			
			\item[(2)]
				$F \cap G = \emptyset$となる$F,G \in \mathcal{M}$に対し,$F$又は$G$が$\emptyset$なら
				(1)より$E(F)E(G) = 0$,そうでない場合は(\refeq{eq:sigma_additivity_spectral_measures})より
				\begin{align}
					E(F) \neq E(G), \quad E(F) + E(G) = E(F + G) \in \Oproj{H}
				\end{align}
				が成り立つから,命題\ref{prp:orthogonal_projection_product_sum}より$E(F)E(G) = 0$が従う.
				これと命題\ref{prp:orthogonal_projection_idempotent_self_adjoint}より,$A,B \in \mathcal{M}$に対し
				\begin{align}
					E(A)E(B) = \left( E(A \cap B) + E(A \cap B^c) \right)\left( E(A \cap B) + E(B \cap A^c) \right) = E(A \cap B)
				\end{align}
				が得られる.
				\QED
		\end{description}
	\end{prf}
	
	\begin{screen}
		\begin{lem}[スペクトル測度で導入する複素測度]
			$E:\mathcal{M} \rightarrow \Oproj{H}$をスペクトル測度とする.各$u,v \in H$に対し
			$\mu_{u,v}:\mathcal{M} \rightarrow \C$と$\mu_u:\mathcal{M} \rightarrow [0,\infty)$を次で定める:
			\begin{align}
				\mu_{u,v}(\Lambda) \coloneqq \inprod<E(\Lambda)u,v> \quad (\forall \Lambda \in \mathcal{M}),
				\quad \mu_u \coloneqq \mu_{u,u}
				\label{eq:lem_complex_measure_introduced_by_spectral_measure_1}
			\end{align}
			\begin{description}
				\item[(1)] $\mu_{u,v}$は$(X,\mathcal{M})$上の複素測度であり,$\mu_u$は$(X,\mathcal{M})$上の実数値有限測度である.
				\item[(2)] 任意の$\Lambda \in \mathcal{M}$に対し次が成り立つ:
					\begin{align}
						|\mu_{u.v}(\Lambda)| \leq \mu_u(\Lambda)^{\frac{1}{2}} \mu_v(\Lambda)^{\frac{1}{2}}.
						\label{eq:lem_complex_measure_introduced_by_spectral_measure_3}
					\end{align}
				\item[(3)] $\mathcal{M}/\borel{[0,\infty)}$-可測関数$f,g$に対して次が成り立つ:
					\begin{align}
						\int_X f(x)g(x)\ |\mu_{u,v}|(dx) \leq \left( \int_X |f(x)|^2\ \mu_u(dx) \right)^{\frac{1}{2}} \left( \int_X |g(x)|^2\ \mu_v(dx) \right)^{\frac{1}{2}}.
					\end{align}
			\end{description}
			\label{lem:complex_measure_introduced_by_spectral_measure}
		\end{lem}
	\end{screen}
	
	\begin{prf}\mbox{}
		\begin{description}
			\item[(1)] (\refeq{eq:lem_complex_measure_introduced_by_spectral_measure_1})より$\mu_{u,v}$は複素数値である.
				また命題\ref{prp:orthogonal_projection_idempotent_self_adjoint}より
				\begin{align}
					\inprod<E(\Lambda)u,v> = \inprod<E(\Lambda)^2u,v> = \inprod<E(\Lambda)u,E(\Lambda)^*v> = \inprod<E(\Lambda)u,E(\Lambda)v>
					\label{eq:lem_complex_measure_introduced_by_spectral_measure_2}
				\end{align}
				が成り立つから$\mu_u(\Lambda) = \Norm{E(\Lambda)u}{}^2$を得る.
				互いに素な列$A_1,A_2,\cdots \in \mathcal{M}$を取れば,(\refeq{eq:sigma_additivity_spectral_measures})より
				\begin{align}
					\left| \inprod<\sum_{n=1}^{\infty}E(A_n)u,v> - \inprod<\sum_{n=1}^{N}E(A_n)u,v> \right|
					\leq \Norm{\sum_{n=1}^{\infty}E(A_n)u - \sum_{n=1}^{N}E(A_n)u}{} \Norm{v}{}
					\longrightarrow 0 \quad (N \longrightarrow \infty)
				\end{align}
				が成り立つから,
				\begin{align}
					\inprod<\sum_{n=1}^{N}E(A_n)u,v>= \sum_{n=1}^{N} \inprod<E(A_n)u,v> 
				\end{align}
				の右辺も収束し
				\begin{align}
					\inprod<\sum_{n=1}^{\infty}E(A_n)u,v> = \sum_{n=1}^{\infty}\inprod<E(A_n)u,v>
				\end{align}
				が得られ$\mu_{u,v}$の完全加法性が従う.
				
			\item[(2)] 
				(\refeq{eq:lem_complex_measure_introduced_by_spectral_measure_2})より
				\begin{align}
					\left| \mu_{u,v}(A) \right| = \left| \inprod<E(A)u,E(A)v> \right|
					\leq \Norm{E(A)u}{}\Norm{E(A)v}{}
					= \mu_u(A)^{\frac{1}{2}} \mu_v(A)^{\frac{1}{2}} \quad (\forall A \in \mathcal{M})
				\end{align}
				が成り立つから,
				任意の$\Lambda \in \mathcal{M}$とその有限分割$\Lambda = \sum_{i=1}^{n} A_i\ (A_i \in \mathcal{M})$に対し
				\begin{align}
					\sum_{i=1}^{n} \left| \mu_{u,v}(A_i) \right|
					\leq \sum_{i=1}^{n} \mu_u(A_i)^{\frac{1}{2}} \mu_v(A_i)^{\frac{1}{2}}
					\leq \left( \sum_{i=1}^{n} \mu_u(A_i) \right)^{\frac{1}{2}} \left( \sum_{i=1}^{n} \mu_v(A_i) \right)^{\frac{1}{2}}
					= \mu_u(\Lambda)^{\frac{1}{2}} \mu_v(\Lambda)^{\frac{1}{2}}
				\end{align}
				が得られ,左辺で分割の取り方の上限を取り(\refeq{eq:lem_complex_measure_introduced_by_spectral_measure_3})が従う.
			
			\item[(3)] $f,g$が可測単関数の場合,
				\begin{align}
					f = \sum_{i=1}^{n} \alpha_i \defunc_{A_i},\quad g = \sum_{i=1}^{n} \beta_i \defunc_{A_i}
					\quad (\alpha_i,\beta_i \in [0,\infty),\ \sum_{i=1}^{n} A_i = X)
				\end{align}
				と表されているとして
				\begin{align}
					&\int_X f(x)g(x)\ |\mu_{u,v}|(dx) 
					= \sum_{i=1}^{n} \alpha_i \beta_i |\mu_{u,v}|(A_i) \\
					&\qquad \leq \sum_{i=1}^{n} \alpha_i \mu_u(A_i)^{\frac{1}{2}} \beta_i \mu_v(A_i)^{\frac{1}{2}}
					= \left( \int_X |f(x)|^2\ \mu_u(dx) \right)^{\frac{1}{2}} \left( \int_X |g(x)|^2\ \mu_v(dx) \right)^{\frac{1}{2}}
				\end{align}
				が成り立つ.一般の可測関数については,単関数近似と単調収束定理より主張が従う.
				\QED
		\end{description}
	\end{prf}
	
	以後は$(X,\mathcal{M})$を可測空間,$E:\mathcal{M} \rightarrow \Oproj{H}$をスペクトル測度とし,次の記号を定める:
	\begin{align}
		MF = MF(X,\mathcal{M}) &\coloneqq \Set{f:X \rightarrow \C}{\mbox{$f$は$\mathcal{M}/\borel{\C}$-可測関数.}}, \\
		MSF = MSF(X,\mathcal{M}) &\coloneqq \Set{f:X \rightarrow \C}{\mbox{$f$は$\mathcal{M}/\borel{\C}$-可測単関数.}}
	\end{align}
	\begin{screen}
		\begin{dfn}[$MSF$-近似列]
			$f \in MF$に対し$\lim_{n \to \infty} f_n(x) = f(x)\ (\forall x \in X)$かつ
			$|f_n| \leq |f|\ (n=1,2,\cdots)$を満たす$f_n \in MSF\ (n=1,2,\cdots)$を
			$f$の$MSF$-近似列と呼び,特に$\left( |f_n| \right)_{n=1}^{\infty}$が単調増加なら$MSF$-単調近似列と呼ぶ.
		\end{dfn}
	\end{screen}
	
	\begin{screen}
		\begin{prp}[$MSF$-単調近似列の存在]
			任意の$f \in MF$に対して$MSF$-単調近似列が存在する.
		\end{prp}
	\end{screen}
	
	\begin{prf}
		任意に$f \in MF$を取り,$f$の実部と虚部をそれぞれ$g,h$と表す.
		$g^+ \coloneqq g \defunc_{\{g \geq 0\}},\ g^- \coloneqq g^+ - g$と定め,同様に
		$h^+,h^-$を定めれば,$g^+,g^-,h^+,h^-$はそれぞれ非負で可測$\mathcal{M}/\borel{\R}$であるから
		$MSF$-単調近似列$\left( g^+_n \right)_{n=1}^{\infty}$が存在する.
		\begin{align}
			\left| g_n \right|^2 = \left| g^+_n \right|^2 + \left| g^-_n \right|^2
			\leq \left| g^+ \right|^2 + \left| g^- \right|^2
			= |g|^2
		\end{align}
		が成り立つから,
		\begin{align}
			f_n \coloneqq g^+_n - g^-_n + i\left( h^+_n - h^-_n \right)
			\quad (n=1,2,\cdots)
		\end{align}
		とおけば
		\begin{align}
			|f_n|^2 = \left| g^+_n - g^-_n \right|^2 + \left| h^+_n - h^-_n \right|^2
			\leq |g|^2 + |h|^2
			= |f|^2
		\end{align}
		が得られる.
		\QED
	\end{prf}
	
	\begin{screen}
		\begin{dfn}[可測関数で導入する作用素]
			$f \in MF$に対し,$H$上の作用素$T_f$を次で定義する:
			\begin{description}
				\item[$f$が可測単関数の場合]
					$\alpha_i \in \C$と$A_i \in \mathcal{M},\ \sum_{i=1}^{n} A_i = X$によって
					\begin{align}
						f = \sum_{i=1}^{n} \alpha_i \defunc_{A_i}
					\end{align}
					と表示されているとき,
					\begin{align}
						\Dom{T_f} \coloneqq H,\quad T_f \coloneqq \sum_{i=1}^{n} \alpha_i E(A_i)
						\label{eq:dfn_operator_introduced_by_measurable_functions}
					\end{align}
					と定める.
				
				\item[$f$が一般の可測関数の場合]
					$f$の$MSF$-近似列$(f_n)_{n=1}^{\infty}$を一つ取り
					\begin{align}
						\Dom{T_f} &\coloneqq \Set{u \in H}{\int_X |f(x)|^2\ \mu_u(dx) < \infty}, \label{eq:dfn_operator_introduced_by_measurable_functions_3} \\
						T_f u &\coloneqq \lim_{n \to \infty} T_{f_n} u \quad (\forall u \in \Dom{T_f} )
						\label{eq:dfn_operator_introduced_by_measurable_functions_2}
					\end{align}
					と定める.
			\end{description}
			\label{dfn:operator_introduced_by_measurable_functions}
		\end{dfn}
	\end{screen}
	
	\begin{screen}
		\begin{lem}
			(\refeq{eq:dfn_operator_introduced_by_measurable_functions})による$T_f$の定義は$f$の表示に依らない.
		\end{lem}	
	\end{screen}
	
	\begin{prf}
			$\alpha_i ,\beta_j \in \C$と$A_i,B_j \in \mathcal{M},\ \sum_{i=1}^{n} A_i = \sum_{j=1}^{m} B_j = X$によって
			\begin{align}
				f = \sum_{i=1}^{n} \alpha_i \defunc_{A_i} = \sum_{j=1}^{m} \beta_i \defunc_{B_i}
			\end{align}
			と表示されているとき,
			\begin{align}
				\sum_{i=1}^{n} \alpha_i E(A_i)
				= \sum_{i=1}^{n} \sum_{j=1}^{m} \alpha_i E(A_i \cap B_j)
				= \sum_{j=1}^{m} \sum_{i=1}^{n} \beta_j E(A_i \cap B_j)
				= \sum_{j=1}^{m} \beta_j E(B_j)
			\end{align}
			が成り立つ.
		\QED
	\end{prf}
	
	\begin{screen}
		\begin{lem}
			$f,g \in MSF,\ \alpha,\beta \in \C,\ u,v \in H$に対して次が成り立つ:
			\begin{description}
				\item[(1)] 
					$T_{\alpha f + \beta g} = \alpha T_f + \beta T_f,
						\quad T_f T_g = T_{fg},
						\quad T_f^* = T_{\conj{f}},
						\quad T_{\defunc_A} = E(A) \quad (\forall A \in \mathcal{M}).$
				
				\item[(2)] 
					$\inprod<T_f u, T_g v> = \int_X f(x) \conj{g(x)}\ \mu_{u,v}(dx), 
						\quad \Norm{T_f u}{}^2 = \int_X |f(x)|^2\ \mu_u(dx).$
			\end{description}
			\label{lem:MSF_properties_of_T_f}
		\end{lem}
	\end{screen}
	
	\begin{prf}\mbox{}
		\begin{description}
			\item[(1)] $a_i,b_i \in \C,\ A_i \in \mathcal{M}\ \sum_{i=1}^{n} A_i = X$によって
				\begin{align}
					f = \sum_{i=1}^{n} a_i \defunc_{A_i},
					\quad g = \sum_{i=1}^{n} b_i \defunc_{A_i}
				\end{align}
				と表示されているとする.先ず
				\begin{align}
					T_{\alpha f + \beta g} = \sum_{i=1}^{n} (\alpha a_i + \beta b_i) E(A_i) = \alpha \sum_{i=1}^{n} a_i E(A_i) + \beta \sum_{i=1}^{n} b_i E(A_i) = \alpha T_f + \beta T_f
				\end{align}
				が成り立つ.また補題\ref{lem:product_of_spectral_measure}より
				\begin{align}
					T_f T_g = \sum_{i=1} a_i b_i E(A_i) = T_{fg}
				\end{align}
				が従い,また命題\ref{prp:orthogonal_projection_idempotent_self_adjoint}より
				\begin{align}
					T_f^* = \left( \sum_{i=1}^n a_i E(A_i) \right)^* = \sum_{i=1}^n \conj{a_i} E(A_i)^* = T_{\conj{f}}
				\end{align}
				も得られる.$T_{\defunc_A} = E(A)$は(\refeq{eq:dfn_operator_introduced_by_measurable_functions})による.
			
			\item[(2)] 命題\ref{prp:orthogonal_projection_idempotent_self_adjoint}と命題\ref{lem:product_of_spectral_measure}より
				$\Ran{E(A_j)} \perp \Ran{E(A_k)}\ (j \neq k)$が成り立つから,
				(\refeq{eq:lem_complex_measure_introduced_by_spectral_measure_2})より
				\begin{align}
					\inprod<T_f u, T_g v> 
					= \sum_{i=1}^{n} a_i \conj{b_i} \inprod<E(A_i) u, E(A_i) v> 
					= \sum_{i=1}^{n} a_i \conj{b_i} \inprod<E(A_i) u, v>
					= \int_X f(x) \conj{g(x)}\ \mu_{u,v}(dx)
				\end{align}
				を得る.
				\QED
		\end{description}
	\end{prf}
	
	\begin{screen}
		\begin{thm}
			(\refeq{eq:dfn_operator_introduced_by_measurable_functions_2})で定める$T_f$はwell-definedであり,
			特に(\refeq{eq:dfn_operator_introduced_by_measurable_functions})による定義の拡張となっている.
		\end{thm}
	\end{screen}
	
	\begin{prf}\mbox{}
		\begin{description}
			\item[第一段]
				(\refeq{eq:dfn_operator_introduced_by_measurable_functions_2})の極限が存在することを示す.
				$f \in MF$に対し$MSF$-近似列$(f_n)_{n=1}^{\infty}$を取れば
				\begin{align}
					\Norm{T_{f_n}u - T_{f_m}u}{}
					= \Norm{T_{f_n - f_m}u}{}
					= \int_X |f_n(x) - f_m(x)|^2\ \mu_u(dx)
					\quad \left( \forall u \in \Dom{T_f} \right)
				\end{align}
				が成り立つ.$|f_n - f| \leq 2|f|$かつ各点で$|f_n(x) - f(x)| \longrightarrow 0$となるから,
				Lebesgueの収束定理より
				\begin{align}
					\int_X |f_n(x) - f_m(x)|^2\ \mu_u(dx)
					\leq 2 \int_X |f_n(x) - f(x)|^2\ \mu_u(dx)
						+ 2 \int_X |f_m(x) - f(x)|^2\ \mu_u(dx)
					\longrightarrow 0 \quad (n \longrightarrow \infty)
				\end{align}
				が得られる.従って$\left( T_{f_n}u \right)_{n=1}^{\infty}$はHilbert空間$H$においてCauchy列であり極限が存在する.
				
			\item[第二段]
				(\refeq{eq:dfn_operator_introduced_by_measurable_functions_2})の極限が近似列に依存しないことを示す.
				前段の$f$に対し別の$MSF$-近似列$(g_m)_{m=1}^{\infty}$を取り
				\begin{align}
					T_1 u \coloneqq \lim_{n \to \infty} T_{f_n} u,
					\quad T_2 u \coloneqq \lim_{m \to \infty} T_{g_m} u
					\quad \left( \forall u \in \Dom{T_f} \right)
				\end{align}
				とおく.各$u \in \Dom{T_f} $に対し 
				\begin{align}
					&\Norm{T_{f_n} u - T_{g_m} u}{}^2
					= \int_X \left| f_n(x) - g_m(x) \right|^2\ \mu_u(dx) \\
					&\qquad \leq 2 \int_X \left| f_n(x) - f(x) \right|^2\ \mu_u(dx)
						+ 2 \int_X \left| f(x) - g_m(x) \right|^2\ \mu_u(dx)
					\longrightarrow 0 \quad (n,m \longrightarrow \infty)
				\end{align}
				が成り立つから
				\begin{align}
					\Norm{T_1 u - T_2 u}{} 
					\leq \Norm{T_1 u - T_{f_n} u}{} + \Norm{T_{f_n} u - T_{g_m} u}{} + \Norm{T_{g_m} u - T_2 u}{}
					\longrightarrow 0 \quad (n,m \longrightarrow \infty)
				\end{align}
				が従い$T_1 u = T_2 u$を得る.
				\QED
		\end{description}
	\end{prf}
	
	\begin{screen}
		\begin{lem}[$\Dom{T_f} $の線型性・稠密性]
			(\refeq{eq:dfn_operator_introduced_by_measurable_functions_3})で定めた$\Dom{T_f} $は$H$の線型部分空間で$\closure{\Dom{T_f} }=H$を満たす.
			\label{lem:domain_T_f_linear_dense}
		\end{lem}
	\end{screen}
	
	\begin{prf}\mbox{}
		\begin{description}
			\item[線型性]
				$u,v \in \Dom{T_f} $に対して
				\begin{align}
					\int_X |f(x)|^2\ \mu_u(dx) < \infty,\quad \int_X |f(x)|^2\ \mu_v(dx) < \infty
				\end{align}
				が満たされている.(\refeq{eq:lem_complex_measure_introduced_by_spectral_measure_2})より任意の$\Lambda \in \mathcal{M}$に対して
				\begin{align}
					\mu_{u+v}(\Lambda) = \Norm{E(\Lambda)(u+v)}{}^2 \leq 2 \Norm{E(\Lambda)u}{}^2 + 2 \Norm{E(\Lambda)v}{}^2 = 2 \mu_u(\Lambda) + 2 \mu_v(\Lambda)
				\end{align}
				が成り立つから
				\begin{align}
					\int_X |f(x)|^2\ \mu_{u+v}(dx) \leq 2 \int_X |f(x)|^2\ \mu_u(dx) + 2 \int_X |f(x)|^2\ \mu_v(dx) < \infty 
				\end{align}
				が従い$u+v \in \Dom{T_f} $を得る.また任意に$\lambda \in \C$を取れば
				\begin{align}
					\mu_{\lambda u}(\Lambda) = \Norm{\lambda E(\Lambda)u}{}^2 = |\lambda|^2 \mu_{u}(\Lambda)
				\end{align}
				が成り立ち$\lambda u \in \Dom{T_f} $も従う.
				
			\item[稠密性]
				任意に$u \in H$を取る.
				\begin{align}
					A_k \coloneqq \Set{x \in X}{|f(x)| \leq k} \quad (k=1,2,\cdots)
				\end{align}
				に対して$u_k \coloneqq E(A_k)u$とおけば
				\begin{align}
					\Norm{u_k - u}{} \longrightarrow 0 \quad (k \longrightarrow \infty)
					\label{eq:lem_domain_T_f_linear_dense}
				\end{align}
				が成り立つ.一方で任意の$\Lambda \in \mathcal{M}$に対して,
				命題\ref{lem:product_of_spectral_measure}と(\refeq{eq:lem_complex_measure_introduced_by_spectral_measure_2})より
				\begin{align}
					\mu_{u_k}(\Lambda) = \inprod<E(\Lambda)E(A_k)u, E(A_k)u> = \inprod<E(\Lambda \cap A_k)u,u> = \mu_u(\Lambda \cap A_k)
				\end{align}
				が成り立つから$\mu_{u_k}$は$A_k$に集中している.よって
				\begin{align}
					\int_X |f(x)|^2\ \mu_{u_k}(dx) = \int_{A_k} |f(x)|^2\ \mu_{u_k}(dx) \leq k^2 \mu_u(A_k) < \infty
				\end{align}
				となり$u_k \in \Dom{T_f} $が従い,(\refeq{eq:lem_domain_T_f_linear_dense})より主張を得る.
				\QED
		\end{description}
	\end{prf}
	
	\begin{screen}
		\begin{thm}
			$f,g \in MF$とする.
			\begin{description}
				\item[(1)] $T_f$は線型作用素である.
				\item[(2)] $u \in \Dom{T_f} ,\ v \in \Dom{T_g} $ならば次が成り立つ:
					\begin{align}
						\int_X \left| f(x) \conj{g(x)} \right|\ |\mu_{u,v}|(dx) \leq \Norm{f}{\Lp{2}{\mu_u}} \Norm{g}{\Lp{2}{\mu_v}}, \quad 
						\int_X f(x) \conj{g(x)}\ \mu_{u,v}(dx) = \inprod<T_f u, T_g v>.
					\end{align}
				\item[(3)] $T_f T_g \subset T_{fg}$が成り立ち,特に$g$が有界なら等号が成立する.
				\item[(4)] $T_f + T_g \subset T_{f+g}$が成り立ち,特に$g$が有界なら等号が成立する.
				\item[(5)] $T_f^* = T_{\conj{f}}$が成り立ち,特に$T_f$は閉作用素である.
				\item[(6)] $\lambda \in \C$が$\lambda = 0$なら$T_{\lambda f} = \lambda T_f$が成り立つ.
			\end{description}
		\end{thm}
	\end{screen}
	
	\begin{prf}\mbox{}
		\begin{description}
			\item[(1)]	$f$の$MSF$-近似列$(f_n)_{n=1}^{\infty}$に対し$T_{f_n}$は線型であるから.
			\item[(2)] $f,g \in MSF$のとき,任意の$u,v \in H$に対して
				\begin{align}
					\int_X \left| f(x) \conj{g(x)} \right|\ |\mu_{u,v}|(dx) \leq \Norm{f}{\Lp{2}{\mu_u}} \Norm{g}{\Lp{2}{\mu_v}},
					\quad \inprod<T_f u, T_g v> = \int_X f(x) \conj{g(x)}\ \mu_{u,v}(dx)
				\end{align}
				が成り立つ.第二式は補題(\ref{lem:MSF_properties_of_T_f})による.第一式について,
				\begin{align}
					f = \sum_{i=1}^{n} \alpha_i \defunc_{A_i},\quad 
					g = \sum_{i=1}^{n} \beta_i \defunc_{A_i}
				\end{align}
				と表示されているとして
				\begin{align}
					\int_X \left| f(x) \conj{g(x)} \right|\ |\mu_{u,v}|(dx)
					= \sum_{i=1}^{n} |\alpha_i||\beta_i| |\mu_{u,v}|(A_i)
					\leq \sum_{i=1}^{n} |\alpha_i||\beta_i| \mu_u(A_i)^{\frac{1}{2}} \mu_v(A_i)^{\frac{1}{2}}
					\leq \left( \int_X \left| f(x) \right|^2\ \mu_u(dx) \right)^{\frac{1}{2}} \left( \int_X \left| g(x) \right|^2\ \mu_v(dx) \right)^{\frac{1}{2}}
				\end{align}
				が成り立つ.
				一般の$f,g \in MF$については,$MSF$-近似列とFatouの補題より従う.
		\end{description}
	\end{prf}
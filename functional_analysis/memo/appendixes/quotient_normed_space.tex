\chapter{商ノルム空間}
	$\K$を$\R$又は$\C$として$X$を$\K$上のノルム空間とする.
	$X$のノルムを$\Norm{\cdot}{X}$と表記し$X$にノルム位相を導入する.
	また$X$の閉部分空間$Y$に対し
	\begin{align}
		x \sim y \DEF x - y \in Y \quad (\forall x,y \in X)
	\end{align}
	として$X$における同値関係$\sim$を定める
	\footnote{
		$x,y,z \in X$を取る.$Y$は線形空間であるから,
		反射率は$x - x = 0 \in Y$により従い,対称律は$x - y \in Y$なら
		$y - x = -(x - y) \in Y$が成り立つことにより従う.推移律についても,
		$x \sim y$かつ$y \sim z$が満たされているなら
		$x - z = (x - y) + (y - z) \in Y$が成り立ち$x \sim z$が従う.
	}
	.以降,関係$\sim$による$x \in X$の同値類を$[x]$と表し,商集合を$X/Y$と表す.
	
	\begin{screen}
		\begin{thm}[商集合における線型演算]
			$X/Y$において
			\begin{align}
				[x] + [y] \coloneqq [x + y], \quad
				\alpha [x] \coloneqq [\alpha x] \quad (\forall [x],[y] \in X/Y,\ \alpha \in \K)
				\label{eq:quotient_set_linear_calculation_1}
			\end{align}
			として演算を定義すれば,$X/Y$はこれを線型演算として線形空間となる.
		\end{thm}
	\end{screen}
	
	\begin{prf}\mbox{}
		\begin{description}
			\item[well-defined]
				先ず(\ref{eq:quotient_set_linear_calculation_1})の定義がwell-definedであることを示す.
				任意に$u \in [x],v \in [y],\alpha \in \K$を取り
				\begin{align}
					[u + v] = [x + y], \quad [\alpha u] = [\alpha x]
					\label{eq:quotient_set_linear_calculation_2}
				\end{align}
				が成り立つことをいえばよい.実際$x \sim u$かつ$y \sim v$であるから
				\begin{align}
					(x + y) - (u + v) = (x - u) + (y - v) \in Y,
					\quad \alpha x - \alpha v = \alpha(x - u) \in Y
					\end{align}
				が成り立ち(\refeq{eq:quotient_set_linear_calculation_1})が従う.
		\end{description}
		$X$が線形空間であるから$X/Y$は(\ref{eq:quotient_set_linear_calculation_1})の演算で閉じている.
		よってあとは以下の事項を確認すればよい.
		\begin{description}
			\item[加法]
				$X/Y$が加法について可換群をなすことを示す.任意に$[x],[y],[z] \in X/Y$を取れば
				\begin{align}
					([x]+[y]) + [z] = [x+y] + [z] = [(x+y) + z] = [x + (y+z)] = [x] + [y+z] = [x] + ([y]+[z])
				\end{align}
				が成り立ち結合律が従う.可換性は
				\begin{align}
					[x] + [y] = [x + y] = [y + x] = [y] + [x]
				\end{align}
				により従い,また$[x]$の逆元は$(-1)[x]$
				\footnote{
					$[x] + (-1)[y]$は$[x] - [y]$と表す.
				}
				,$X/Y$の零元は$Y = [0]$である.
				
			\item[スカラ倍]
				任意に$[x],[y] \in X/Y$と$\alpha,\beta \in \K$を取れば以下が成り立つ:
				\begin{description}
					\item[$(1)$] $(\alpha\beta)[x] = [(\alpha\beta)x] = [\alpha(\beta x)] = \alpha[\beta x] = \alpha(\beta [x]),$
					\item[$(2)$] $(\alpha + \beta)[x] = [(\alpha + \beta)x] = [\alpha x + \beta x] = [\alpha x] + [\beta x] = \alpha [x] + \beta [x],$
					\item[$(3)$] $\alpha ([x] + [y]) = \alpha [x+y] = [\alpha(x+y)] = [\alpha x + \alpha y] = [\alpha x] + [\alpha y] = \alpha [x] + \alpha [y],$
					\item[$(4)$] $1 [x] = [x].$
				\end{description}
		\end{description}
		\QED
	\end{prf}
	
	\begin{screen}
		\begin{lem}[同値類は閉]
			任意の$[x] \in X/Y$は$X$において閉集合となる.
			\label{lem:equivalence_class_closed}
		\end{lem}
	\end{screen}
	
	\begin{prf}
		任意に$[x] \in X/Y$を取る.距離空間の一般論より$u_n \in [x]\ (n=1,2,\cdots)$が或る$u \in X$に収束するとき
		$u \in [x]$が成り立つことを示せばよい.各$n \in \N$について$u_n - x \in Y$であり,かつ
		\begin{align}
			\Norm{(u_n - x) - (u - x)}{X} = \Norm{u_n - u}{X} \longrightarrow 0
			\quad (n \longrightarrow \infty)
		\end{align}
		が成り立つから,$Y$が閉であることにより$u - x \in Y$が従う.
		\QED
	\end{prf}
	
	\begin{screen}
		\begin{thm}[商空間におけるノルムの定義]
			$X/Y$において
			\begin{align}
				\Norm{[x]}{X/Y} \coloneqq \inf{u \in [x]}{\Norm{u}{X}} \quad (\forall [x] \in X/Y)
				\label{eq:thm_quotient_space_norm}
			\end{align}
			として$\Norm{\cdot}{X/Y}:X/Y \rightarrow \R$を定めれば,これはノルムとなる.
		\end{thm}
	\end{screen}
	
	\begin{prf}\mbox{}
		\begin{description}
			\item[正値性]
				$\Norm{\cdot}{X/Y}$が非負値であることは定義式(\refeq{eq:thm_quotient_space_norm})右辺の非負性による.
				また$[x] = [0]$である場合,
				\begin{align}
					\inf{u \in [x]}{\Norm{u}{X}} = \Norm{0}{X} = 0
				\end{align}
				が成り立ち$\Norm{[x]}{X/Y} = 0$が従う.逆に$\Norm{[x]}{X/Y} = 0$である場合,
				\begin{align}
					\Norm{u_n}{X} \leq \frac{1}{n} \quad (n=1,2,\cdots)
				\end{align}
				を満たす点列$u_n \in [x]\ (n=1,2,\cdots)$が存在する.
				すなわち$u_n \longrightarrow 0 \quad (n \longrightarrow \infty)$
				であるから,補助定理\ref{lem:equivalence_class_closed}により$0 \in [x]$が成り立ち$[x] = [0]$が従う.
				
			\item[同次性]
				任意に$[x] \in X/Y$と$\alpha \in \K$を取る.$\alpha = 0$の場合は
				\begin{align}
					\Norm{0 [x]}{X/Y} = \Norm{[0]}{X/Y} = 0 = 0 \Norm{[x]}{X/Y}
				\end{align}
				が成り立つ.$\alpha \neq 0$の場合は
				\begin{align}
					u \in [\alpha x] \quad \Leftrightarrow \quad \frac{1}{\alpha} u \in [x]
				\end{align}
				が成り立つから
				\begin{align}
					\Norm{\alpha [x]}{X/Y} = \Norm{[\alpha x]}{X/Y} = \inf{u \in [\alpha x]}{\Norm{u}{X}}
					= |\alpha| \inf{u \in [\alpha x]}{\Norm{(1/\alpha)u}{X}}
					= |\alpha| \inf{v \in [x]}{\Norm{v}{X}}
					= |\alpha |\Norm{[x]}{X/Y}
				\end{align}
				が従う.
			
			\item[劣加法性]
				任意に$[x],[y] \in X/Y$を取り
				\begin{align}
					L \coloneqq \Set{u + v}{u \in [x],\ v \in [y]}
				\end{align}
				とおけば,任意の$u+v \in L$に対し$(u+v) - (x+y) \in Y$となるから$L \subset [x+y]$が成り立つ.また
				\begin{align}
					\Norm{u + v}{X} \leq \Norm{u}{X} + \Norm{v}{X}
				\end{align}
				により
				\begin{align}
					\inf{u'+v' \in L}{\Norm{u' + v'}{X}} \leq \Norm{u}{X} + \Norm{v}{X} \quad (\forall u \in [x],\ v \in [y])
				\end{align}
				が成り立つから,
				\begin{align}
					\inf{u'+v' \in L}{\Norm{u' + v'}{X}} 
					\leq \inf{u \in [x]}{\Norm{u}{X}} + \inf{v \in [y]}{\Norm{v}{X}}
					= \Norm{[x]}{X/Y} + \Norm{[y]}{X/Y}
				\end{align}
				が従い
				\begin{align}
					\Norm{[x] + [y]}{X/Y} = \Norm{[x+y]}{X/Y} 
					= \inf{w \in [x+y]}{\Norm{w}{X}} 
					\leq \inf{u+v \in L}{\Norm{u + v}{X}}
					\leq \Norm{[x]}{X/Y} + \Norm{[y]}{X/Y}
				\end{align}
				を得る.
		\end{description}
		\QED
	\end{prf}
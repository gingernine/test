\chapter{弱収束}
\section{ノルム空間における弱収束}

	$\K$を$\R$又は$\C$とする.ノルム空間$X$に対しノルムを$\Norm{\cdot}{X}$で表記し,
	また$J_X: X \rightarrow X^{**}$を自然な等長単射とする.
	\begin{screen}
		\begin{dfn}[弱収束]
			$X$を$\K$上のノルム空間とする.$X$の点列$(x_n)_{n=1}^{\infty}$が$x \in X$に弱収束するとは
			\begin{align}
				\lim_{n \to \infty} f(x_n) = f(x) \quad (\forall f \in X^*)
			\end{align}
			が成り立つことを言い,$\wlim_{n \to \infty} x_n = x$と表記する.
		\end{dfn}
	\end{screen}
	
	\begin{screen}
		\begin{dfn}[汎弱収束]
			$X$を$\K$上のノルム空間とする.$X^*$の列$(f_n)_{n=1}^{\infty}$が$f \in X^*$に汎弱収束するとは
			\begin{align}
				\lim_{n \to \infty} f_n(x) = f(x) \quad (\forall x \in X)
			\end{align}
			が成り立つことを言い,$\wstarlim_{n \to \infty} f_n = f$と表記する.
		\end{dfn}
	\end{screen}
	
	\begin{screen}
		\begin{thm}[弱収束及び汎弱収束極限の一意性]
			$X$を$\K$上のノルム空間とする.$X$の点列$(x_n)_{n=1}^{\infty}$が$u,v \in X$に弱収束
			するなら$u = v$が従い,$X^*$の列$(f_n)_{n=1}^{\infty}$が$f,g \in X^*$に汎弱収束するなら
			$f = g$が従う.
		\end{thm}
	\end{screen}
	
	\begin{prf}
		$(x_n)_{n=1}^{\infty}$が$u,v \in X$に弱収束するとき,任意の$f \in X^*$に対して
		\begin{align}
			\left| f(u) - f(v) \right| \leq \left| f(u) - f(x_n) \right| + \left| f(x_n) - f(v) \right| \longrightarrow 0 
			\quad (n \longrightarrow \infty)
		\end{align}
		が成り立ち,Hahn-Banachの定理の系より$u = v$が従う.また
		$(f_n)_{n=1}^{\infty}$が$f,g \in X^*$に汎弱収束するとき,任意の$x \in X$に対して
		\begin{align}
			\left| f(x) - g(x) \right| \leq \left| f(x) - f_n(x) \right| + \left| f_n(x) - g(x) \right| \longrightarrow 0 
			\quad (n \longrightarrow \infty)
		\end{align}
		が成り立ち$f = g$が従う.
		\QED
	\end{prf}
	
	\begin{screen}
		\begin{thm}[弱収束と自然な等長単射の関係]
			$X$を$\K$上のノルム空間とする.$x_n \in X\ (n=1,2,\cdots)$が$x \in X$に弱収束することと
			$J_Xx_n \in X^{**}\ (n=1,2,\cdots)$が$J_Xx \in X^{**}$に汎弱収束することは同値である.
			\label{thm:weak_convergence_and_canonical_injection}
		\end{thm}
	\end{screen}
	
	\begin{prf}
		自然な等長単射の定義より任意の$f \in X^*$について$f(x_n) = J_X x_n(f)$であるから,
		\begin{align}
			\lim_{n \to \infty} f(x_n) = f(x) \quad (\forall f \in X^*)
		\end{align}
		が成り立つことと
		\begin{align}
			\lim_{n \to \infty} J_X x_n(f) = J_X x(f) \quad (\forall f \in X^*)
		\end{align}
		が成り立つことは同じである.
		\QED
	\end{prf}
	
	\begin{screen}
		\begin{thm}[汎弱収束列の有界性]
			$X$を$\K$上のノルム空間とし$X \neq \{0\}$を仮定する.$X^*$の列$(f_n)_{n=1}^{\infty}$が各点$x \in X$でCauchy列をなすとき,
			$(f_n)_{n=1}^{\infty}$は有界となりさらに汎弱収束極限$f \in X^*$が存在して次が成り立つ \footnotemark:
			\begin{align}
				\Norm{f}{X^*} \leq \liminf_{n \to \infty} \Norm{f_n}{X^*}.
			\end{align}
			\label{thm:weak_star_convergence_bonded}
		\end{thm}
	\end{screen}
	
	\footnotetext{
		右辺は有限確定する.
		実際$(f_n)_{n=1}^{\infty}$が有界であるとして$M \coloneqq \sup{n \in \N}{\Norm{f_n}{X^*}}$とおけば,
		任意の$n \in \N$に対し
		\begin{align}
			\inf{\nu \geq n}{\Norm{f_n}{X^*}} \leq \sup{n \in \N}{\Norm{f_n}{X^*}} = M
		\end{align}
		が成り立つから
		\begin{align}
			\liminf_{n \to \infty} \Norm{f_n}{X^*} \leq M
		\end{align}
		が従う.
	}
	
	\begin{prf}
		任意の$x \in X$に対して$\left( f_n(x) \right)_{n=1}^{\infty}$は有界であるから,
		一様有界性の原理より$\left( \Norm{f_n}{X^*} \right)_{n=1}^{\infty}$が有界となる.また
		\begin{align}
			f(x) \coloneqq \lim_{n \to \infty} f_n(x) \quad (\forall x \in X)
			\label{eq:thm_weak_star_convergence_bonded}
		\end{align}
		として$f:X \rightarrow \K$を定めれば,$f$は$X^*$に属する:
		\begin{description}
			\item[線型性]
				任意に$x,x_1,x_2 \in X$と$\alpha \in \K$を取れば
				\begin{align}
					\left| f(x_1 + x_2) - f(x_1) - f(x_2) \right| &\leq \left| f(x_1 + x_2) - f_n (x_1 + x_2)\right| + \left| f(x_1) - f_n(x_1) \right| 
						+ \left| f(x_2) - f_n(x_2) \right| \longrightarrow 0 \quad (n \longrightarrow \infty) \\
					\left| f(\alpha x) - \alpha f(x) \right| &\leq \left| f(\alpha x) - f_n(\alpha x) \right| + |\alpha| \left| f(x) - f_n(x) \right|
						\longrightarrow 0 \quad (n \longrightarrow \infty)
				\end{align}
				が成り立つ.
			
			\item[有界性]
				絶対値の連続性より
				\begin{align}
					\left| f(x) \right| = \lim_{n \to \infty} \left| f_n(x) \right| \leq \liminf_{n \to \infty} \Norm{f_n}{X^*} \Norm{x}{X}
				\end{align}
				が成り立ち,特に$\Norm{x}{X} = 1$として
				\begin{align}
					\sup{\Norm{x}{X} = 1}{\left| f(x) \right|} \leq \liminf_{n \to \infty} \Norm{f_n}{X^*} < \infty
				\end{align}
				が従う.
		\end{description}
		$f$が$f_n$の汎弱収束極限であることは(\refeq{eq:thm_weak_star_convergence_bonded})より従う.
		\QED
	\end{prf}
	
	\begin{screen}
		\begin{thm}[弱収束列の有界性]
			$X$を$\K$上のノルム空間とし$X \neq \{0\}$を仮定する.$X$の列$(x_n)_{n=1}^{\infty}$が$x \in X$に弱収束するとき,
			$(x_n)_{n=1}^{\infty}$は有界列であり次が成り立つ:
			\begin{align}
				\Norm{x}{X} \leq \liminf_{n \to \infty} \Norm{x_n}{X}.
			\end{align}
			\label{thm:weak_convergence_bounded}
		\end{thm}
	\end{screen}
	
	\begin{prf}
		定理\ref{thm:weak_convergence_and_canonical_injection}より
		$(J_X x_n)_{n=1}^{\infty}$が$J_X x \in X^{**}$に汎弱収束するから,
		定理\ref{thm:weak_star_convergence_bonded}より$(J_X x_n)_{n=1}^{\infty}$は有界列で
		\begin{align}
			\Norm{J_X x}{X^{**}} \leq \liminf_{n \to \infty} \Norm{J_X x_n}{X^{**}}
		\end{align}
		が成り立つ.$J_X$は等長であるから定理の主張が従う.
		\QED
	\end{prf}
	
	\begin{screen}
		\begin{thm}[反射的Banach空間の点列が弱収束するための十分条件]
			$X$を$\K$上の反射的Banach空間として点列$(x_n)_{n=1}^{\infty}$を取る.任意の$f \in X^*$に対して
			$\left( f(x_n) \right)_{n=1}^{\infty}$がCauchy列となるなら,$(x_n)_{n=1}^{\infty}$は或る$x \in X$に弱収束する.
		\end{thm}
	\end{screen}
	
	\begin{prf}
		$f(x_n) = J_X x_n(f)$であることと定理の仮定より,任意の$f \in X^*$で$\left( J_X x_n(f) \right)_{n=1}^{\infty}$は$\K$のCauchy列をなすから,
		\begin{align}
			J(f) \coloneqq \lim_{n \to \infty} J_X x_n(f) \quad (\forall f \in X^*)
		\end{align}
		として$J:X^* \rightarrow \K$を定めれば定理\ref{thm:weak_star_convergence_bonded}より$J \in X^{**}$が成り立つ.
		$X$の反射性から$J$に対し或る$x \in X$が存在して$J = J_X x$を満たし,
		定理\ref{thm:weak_convergence_and_canonical_injection}より定理の主張を得る.
		\QED
	\end{prf}
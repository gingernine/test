%有限次元空間における有界点列の収束
\begin{itembox}[l]{}
	\begin{thm}[有限次元空間における有界点列の収束(局所コンパクト性)]\mbox{}\\
		$X$を$\K (=\R\ \mathrm{or, }\ \C)$上のノルム空間とする.$\Dim{X} < \infty$ならば$X$の任意の有界点列は
		収束部分列を持つ.
	\end{thm}
\end{itembox}

\begin{prf}\mbox{}
	次元数$n$による帰納法で証明する.
	\begin{description}
		\item[$n=1$のとき]
			$X$の基底を$u_1$として取る.$X$の任意の有界点列は
			$\alpha_m \in \K\ (m=1,2,\cdots)$によって$( \alpha_m u_1)_{m=1}^{\infty}$と表現できるが,
			$\left( \alpha_mu_1 \right)_{m=1}^{\infty}$が有界だから$\left( \alpha_m \right)_{m=1}^{\infty}$も有界となる.
			Bolzano-Weierstrassの定理より$\K$において収束する\footnote{以降出てくる点列も収束先は$X$に含まれる.これは有限次元空間が完備であることによる.}
			部分列$\left( \alpha_{m_k} \right)_{k=1}^{\infty}$が存在し,
			同じ添数で$\left( \alpha_{m_k}u_1 \right)_{k=1}^{\infty}$も収束列となる.
		
		\item[一般の$n$について]
			$n=k$のとき定理の主張が正しいと仮定する.$n=k+1$のとき$X$の基底を$u_1,\cdots,u_{k+1}$
			として取ると,$X$から任意に有界列$(x_j)_{j=1}^{\infty}$を取れば各$x_j$は
			\begin{align}
				x_j = y_j + \alpha_j u_{k+1} \quad (y_j \in \LH{u_1,\cdots,u_k},\ \alpha_j \in \K)
			\end{align}
			として一意に表示される.もし$(\alpha_j)_{j=1}^{\infty}$が有界でないとすれば,
			$\alpha_{j_s} \geq s\ (s=1,2,\cdots)$となるように部分列を抜き出すことができて,$(x_j)_{j=1}^{\infty}$の有界性から
			\begin{align}
				\Norm{u_{k+1} - \tfrac{1}{\alpha_{j_s}}y_{j_s}}{X}
				\leq \Norm{u_{k+1} - \left(\tfrac{1}{\alpha_{j_s}}x_{j_s} - \tfrac{1}{\alpha_{j_s}}y_{j_s}\right)}{X}
					+ \Norm{\tfrac{1}{\alpha_{j_s}}x_{j_s}}{X}
				= \Norm{\tfrac{1}{\alpha_{j_s}}x_{j_s}}{X} \longrightarrow 0 \quad (s \longrightarrow \infty)
			\end{align}
			が成り立つ.有限次元空間は閉であることに注意すれば,$u_{k+1}$が$\LH{u_1,\cdots,u_k}$の列
			$\left(\tfrac{1}{\alpha_{j_s}}y_{j_s}\right)_{s=1}^{\infty}$の極限となっているから
			$u_{k+1} \in \LH{u_1,\cdots,u_k}$となり矛盾ができた.背理法により$(\alpha_j)_{j=1}^{\infty}$は
			$\K$の有界列と判り,Bolzano-Weierstrassの定理より部分列$(\alpha_{j(1,t)})_{t=1}^{\infty}$が収束列となる.
			帰納法の仮定から$(y_{j(1,t)})_{t=1}^{\infty}$から収束する部分列$(y_{j(2,t)})_{t=1}^{\infty}$を取ることができて,
			これと同じ添数について$x_{j(2,t)} = y_{j(2,t)} + \alpha_{j(2,t)}u_{k+1}\ (t=1,2,\cdots)$は収束列となる.
	\end{description}
	\QED
\end{prf}
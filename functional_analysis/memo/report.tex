	
	\begin{screen}
		[4]$a > 0,I = [0,a]$とおく.$\c{I}$から$\c{I}$への線型作用素$T$を次で定める:
			\begin{align}
				\Dom{T} \coloneqq \Set{u \in \cn{I}{1}}{u(0) + u(a) = 0},
				\quad Tu(x) = u'(x) \quad (x \in I).
			\end{align}
			このとき,$\pSpctr{T} $及び$\Spctr{T} $を求めよ.
	\end{screen}
	
	%レポート問題4
	\begin{prf}\mbox{}
		\begin{description}
			\item[点スペクトルについて]
				$(\lambda I - T) u = 0$を満たす$\lambda$に対し,微分方程式を解けば
				\begin{align}
					u(x) = C\exp{\lambda x}
					\quad (x \in I,\ \exists C \ni \C)
				\end{align}
				と表せる.$u(0) + u(a) = 0$が満たされているから,
				\begin{align}
					C + C\exp{\lambda a} = 0
				\end{align}
				が成り立つ.従って$C = 0$或は,複素対数を用いて
				$\lambda = \frac{1}{a} \log{(-1)}$となる.
				$C = 0$の場合は$u = 0$となり固有ベクトルになりえないから
				\begin{align}
					\pSpctr{T} = \Set{\sqrt{-1} \frac{(2 n + 1)\pi}{a}}{n \in \Z}
				\end{align}
				
			\item[スペクトルについて]
		\end{description}
	\end{prf}
	
	\begin{screen}
		[6] $a \in \cbound{\R^d}, \lambda > d$とする.$f \in \mathrm{L}^2(\R^d)$に対し,
		\begin{align}
			T_a f(x) = \int_{|x - y| > 1} \frac{a(x)f(y)}{|x-y|^\lambda}\ dy
			\quad (\mbox{a.e.}x \in \R^d)
		\end{align}
		により$T_a:\mathrm{L}^2(\R^d) \rightarrow \mathrm{L}^2(\R^d)$を定める.
		\begin{description}
			\item[(1)] $T_a$は連続であることを示せ.
		\end{description}
	\end{screen}
	
	\begin{prf}\mbox{}
		\begin{description}
			\item[(1)] 
				任意の$f \in \mathrm{L}^2(\R^d)$に対し$T_a f$が二乗可積分であることと$T_a$の連続性を同時に示す.
				以下,$\mathrm{L}^2(\R^d)$のノルムを$\Norm{\cdot}{}$と書き,$M \coloneqq \sup{x \in \R^d}{|a(x)|} < \infty$とおく.
				$f \in \mathrm{L}^2(\R^d)$に対し,H\Ddot{o}lderの不等式より
				\begin{align}
					\int_{\R^d} \defunc_{|x-y| > 1} \frac{|a(x)f(y)|}{|x-y|^\lambda}\ dy
					&= \int_{\R^d} \defunc_{|x-y| > 1} \frac{|a(x)|}{|x-y|^{\frac{\lambda}{2}}} \defunc_{|x-y| > 1} \frac{|f(y)|}{|x-y|^{\frac{\lambda}{2}}}\ dy \\
					&\leq \left( \int_{\R^d} \defunc_{|x-y| > 1} \frac{|a(x)|^2}{|x-y|^{\lambda}}\ dy \right)^{\frac{1}{2}}
						\left( \int_{\R^d} \defunc_{|x-y| > 1} \frac{|f(y)|^2}{|x-y|^{\lambda}}\ dy \right)^{\frac{1}{2}}
				\end{align}
				が任意の$x \in \R^d$で成立する.右辺第一項について,$\lambda > d \geq 1$であるから,変数変換を用いて
				\begin{align}
					\int_{\R^d} \defunc_{|x-y| > 1} \frac{|a(x)|^2}{|x-y|^{\lambda}}\ dy
					\leq M^2 \int_{\R^d} \defunc_{|x-y| > 1} \frac{1}{|x-y|^{\lambda}}\ dy
					= M^2 \int_{\R^d} \defunc_{|u| > 1} \frac{1}{|u|^\lambda}\ du < \infty
				\end{align}
				が満たされる.従って$U \coloneqq \int_{\R^d} \defunc_{|x - y| > 1} \frac{1}{|x-y|^{\lambda}}\ dy$とおけば$U$は$x$に依らない定数である.
				今,$\R^d \ni x \longmapsto (a(x)f(y))/|x-y|^\lambda$は各$y \in \R^d \backslash \{x\}$で連続,
				$\R^d \ni y \longmapsto \defunc_{\left\{|x-y|>1\right\}}(a(x)f(y))/|x-y|^\lambda$は各$x \in \R^d$で$\borel{\R^d}/\borel{\C}$-可測より,
				$\R^d \times \R^d \ni (x,y) \longmapsto \defunc_{\left\{|x-y|>1\right\}} (a(x)f(y))/|x-y|^\lambda$
				は$\borel{\R^d} \times \borel{\R^d}/\borel{\C}$-可測であるから,
				Fubiniの定理より
				\begin{align}
					\Norm{T_a f}{}^2
					&= \int_{\R^d} \left| \int_{|x-y| > 1} \frac{a(x)f(y)}{|x-y|^\lambda}\ dy \right|^2\ dx \\
					&\leq \int_{\R^d} \left( \int_{\R^d} \defunc_{|x-y| > 1} \frac{|a(x)|^2}{|x-y|^{\lambda}}\ dy \right)
						\left( \int_{\R^d} \defunc_{|x-y| > 1} \frac{|f(y)|^2}{|x-y|^{\lambda}}\ dy \right)\ dx \\
					&\leq M^2 U \int_{\R^d} \int_{\R^d} \defunc_{|x-y| > 1} \frac{|f(y)|^2}{|x-y|^{\lambda}}\ dy\ dx \\
					&= M^2 U \int_{\R^d} |f(y)|^2\ dy \int_{\R^d} \defunc_{|x-y| > 1} \frac{1}{|x-y|^{\lambda}}\ dx \\
					&= M^2 U^2 \Norm{f}{}^2
 				\end{align}
 				が得られる.$T_a$が線型性を持てば有界性と連続性は一致するから,
 				あとは$T_a$が線型性を持つことを示せばよい.
 				
			\item[(2)] $\mathrm{L}^2(\R^d)$がBanach空間であるから,
				$\selfCop{\mathrm{L}^2} $は
				$\selfBop{\mathrm{L}^2} $の閉部分空間である.従って
				$T_a$に作用素ノルムで収束する$\selfCop{\mathrm{L}^2} $の列が存在すれば
				$T_a \in \selfCop{\mathrm{L}^2} $が従う.
				\begin{align}
					a_n(x) \coloneqq a(x)\defunc_{|x| \leq n}
					\quad (\forall x \in \R^d,\ n=1,2,\cdots)
				\end{align}
				により$(a_n)_{n=1}^{\infty}$を定めれば,各$n$に対し
				\begin{align}
					\defunc_{|x - y| > 1} \frac{|a(x) - a_n(x)||f(y)|}{|x-y|^\lambda}
					\leq \defunc_{|x - y| > 1} \frac{2|a(x)||f(y)|}{|x-y|^\lambda}
					\quad (\forall (x,y) \in \R^d \times \R^d)
				\end{align}
				が成り立ち,右辺の関数は(1)の結果より$\R^d \times \R^d$で二乗まで可積分である.従ってFubiniの定理とLebesgueの収束定理より
				\begin{align}
					\Norm{T_a f - T_{a_n}f}{}^2
					\leq \int_{\R^d} \int_{\R^d} \defunc_{|x - y| > 1} \frac{|a(x) - a_n(x)|^2|f(y)|^2}{|x-y|^\lambda}\ dy\ dx
				\end{align}
		\end{description}
	\end{prf}
	
	\begin{screen}
		[7] $a = (a_n)_{n=1}^{\infty} \in \ell^{\infty}$に対して$T:\ell^2 \rightarrow \ell^2$を
		$Tx = (a_n x_n)_{n=1}^{\infty}\ (x = (x_n)_{n=1}^{\infty} \in \ell^2)$で定める.
		\begin{description}
			\item[(1)] $T$がコンパクト作用素であるための必要十分条件を求めよ.
			\item[(2)] $T$がHilbert-Schmidt型作用素であるための必要十分条件を求めよ.
		\end{description}
	\end{screen}
	
	\begin{prf}\mbox{}
		\begin{description}
			\item[(1)] 求める必要十分条件が,或る$n_0 \in \N$が存在して$a_n = 0\ (\forall n > n_0)$が満たされていること,であることを示す.
				\begin{description}
					\item[十分性] 任意に$\ell^2$の有界列
						$\left( x^{\nu} \right)_{\nu=1}^{\infty}\ \left(x^{\nu} = \left( x^{\nu}_n \right)_{n=1}^{\infty} \right)$
						を取る.対角線論法により,或る部分添数列$(\nu(k))_{k=1}^{\infty}$が存在して,全ての$n \in \N$について
						$\left(x_n^{\nu(k)} \right)_{k=1}^{\infty}$が$\C$のCauchy列となるようにできる.実際
						$\left( x^{\nu}_1 \right)_{\nu=1}^{\infty}$は$\C$において有界列であるから,
						Bolzano-Weierstrassの定理より或る部分列$\left( x^{\nu(k,1)}_1 \right)_{k=1}^{\infty}$は$\C$のCauchy列となる.
						$\left( x^{\nu}_2 \right)_{\nu=1}^{\infty}$も$\C$において有界列であるから,
						$(\nu(k,1))_{k=1}^{\infty}$の部分添数列$(\nu(k,2))_{k=1}^{\infty}$が存在し
						$\left( x^{\nu(k,2)}_2 \right)_{k=1}^{\infty}$は$\C$のCauchy列となる.
						同様に部分列を取る操作を繰り返し,任意の$n \in \N$に対し
						$\left( x^{\nu(k,n)}_n \right)_{k=1}^{\infty}$が$\C$のCauchy列となる
						ようにできる.$\nu(k) \coloneqq \nu(k,k)\ (k=1,2,\cdots)$として$(\nu(k))_{k=1}^{\infty}$を定めればよい.
						
						
					\item[必要性]
				\end{description}
			\item[(2)]
		\end{description}
	\end{prf}
	
	\begin{screen}
		[8]$(X,\mathcal{M},\mu)$を$\sigma$-有限な測度空間,$H = \mathrm{L}^2(X,\mathcal{M},\mu) = \mathrm{L}^2(\mu)$とする.
			$\mathcal{M}$-可測関数$a:X \rightarrow \C$に対して,$H$から$H$へのかけ算作用素$M_a$を次で定める:
			\begin{align}
				\Dom{M_a} = \Set{u \in H}{au \in H},
				\quad (M_a u)(x) = a(x) u(x) \quad (x \in X).
			\end{align}
			\begin{description}
				\item[(1)] $M_a$は線型作用素で,$\Dom{M_a} $は$H$で稠密なことを示せ.
				\item[(2)] $M_a^* = M_{\conj{a}}$が成り立つことを示せ.
				\item[(3)] $\Spctr{M_a} = \Set{\lambda \in \C}{\mbox{$\forall \epsilon > 0$に対し$\mu\left( a^{-1}(U_\epsilon(\lambda)) \right) > 0$}}$を示せ.
					(ただし$U_\epsilon(\lambda)$は$\lambda$の$\epsilon$-近傍.)
				\item[(4)] $\pSpctr{M_a} = \Set{\lambda \in \C}{\mu\left( a^{-1}(\{\lambda\}) \right) > 0}$を示せ.
			\end{description}
	\end{screen}
	
	%レポート問題8
	\begin{prf}
		$\sigma$-有限の仮定により,或る集合の系$(X_n)_{n=1}^{\infty} \subset \mathcal{M}$
		が存在して$X_1 \subset X_2 \subset \cdots,\ \mu(X_n) < \infty\ (\forall n \in \N),\ \cup_{n \in \N} X_n = X$を満たす.また$H$におけるノルムと内積をそれぞれ$\Norm{\cdot}{},\inprod<\cdot,\cdot>$と表し,
		$H$上の恒等写像を$I$とする.
		
		\begin{description}
			\item[(1)] 
				\begin{description}
					\item[$M_a$が線型作用素であること]
						先ず$\Dom{M_a} $が$H$の線型部分空間であることを示す.
						任意に$u,v \in \Dom{M_a} ,\ \alpha,\beta \in \C$を取れば,
						$H$が線形空間であることにより$\alpha u + \beta v \in H$が満たされ,
						且つ$a u, a v \in H$により
						\begin{align}
							a( \alpha u + \beta v)
							= \alpha a u + \beta a v 
							\in H
						\end{align}
						も成り立つから$\alpha u + \beta v \in \Dom{M_a} $が従う.
						また任意の$u,v \in \Dom{M_a} ,\ \alpha,\beta \in \C$に対して
						\begin{align}
							&M_a(\alpha u + \beta v)(x)
							= a(x)( \alpha u(x) + \beta v(x)) \\
							&\qquad = \alpha a(x) u(x) + \beta a(x) v(x)
							= \alpha (M_a u) (x) + \beta (M_a v) (x)
							\quad (\mbox{$\mu$-a.e.} x \in X)
						\end{align}
						が満たされるから$M_a$は線型作用素である.
						
					\item[$\Dom{M_a} $が$H$で稠密なこと]
						任意に$v \in H$を取り$v_n \coloneqq v\defunc_{\{|a| \leq n\}}\ (n=1,2,3,\cdots)$として関数列$(v_n)_{n=1}^{\infty}$を作る.
						全ての$x \in X$で$|v_n(x)| \leq |v(x)|$が満たされているから
						$(v_n)_{n \in \N} \subset H$である.また全ての$n \in \N$について
						\begin{align}
							\int_X |a(x)v_n(x)|^2 \mu(dx) = \int_{\{|a| \leq n\}} |a(x)v(x)|^2 \mu(dx) \leq n^2  \int_X |v(x)|^2 \mu(dx) < \infty
						\end{align}
						が成り立つから$(v_n)_{n \in \N} \subset \Dom{M_a} $が従う.
						\begin{align}
							\Norm{v - v_n}{}^2 = \int_X |v(x) - v_n(x)|^2\, \mu(dx) = \int_X \defunc_{\{|a| > n\}}(x)|v(x)|^2\, \mu(dx)
							\label{eq:func_report_Q9_1}
						\end{align}
						の右辺の被積分関数は各点で$0$に収束し,かつ可積分関数$|v|^2$で抑えられるから,
						Lebesgueの収束定理より
						\begin{align}
							\lim_{n \to \infty} \Norm{v - v_n}{}^2 
							= \lim_{n \to \infty} \int_X \defunc_{\{|a| > n\}}(x)|v(x)|^2\, \mu(dx)
							= \int_X \lim_{n \to \infty} \defunc_{\{|a| > n\}}(x)|v(x)|^2\, \mu(dx)
							= 0
						\end{align}
						が得られる.$v$は任意に選んでいたから$\Dom{M_a} $の稠密性が従う.
				\end{description}
				
			\item[(2)]
				$\Dom{M_a} $が$H$で稠密であるから$M_a$の共役作用素を定義できる.
				任意の$u,v \in \Dom{M_a} = \Dom{M_{\conj{a}}} $に対して
				\begin{align}
					\inprod<M_a u,v> 
					= \int_X a(x) u(x) \conj{v(x)}\ \mu(dx)
					= \int_X u(x) \conj{\conj{a(x)} v(x)}\ \mu(dx)
					= \inprod<u,M_{\conj{a}}v>
				\end{align}
				が成り立つから$v \in \Dom{M_a^*} $且つ$M_a^* v = M_{\conj{a}} v\ \left(\forall v \in \Dom{M_{\conj{a}}} \right)$が従う.
				逆に任意の$u \in \Dom{M_a} , v \in \Dom{M_a^*} $に対し
				\begin{align}
					\inprod<u,M_a^* v> = \inprod<M_a u,v> = \inprod<u,M_{\conj{a}}v>
				\end{align}
				が成り立つから,$\Dom{M_a} $の稠密性により$M_a^* v = M_{\conj{a}}v\ \left(\forall v \in \Dom{M_a^*} \right)$が従う.
				以上より$M_a^* = M_{\conj{a}}$を得る.
				
			\item[(3)]
				$\lambda \in \C$を任意に取り固定し,
				$V_\epsilon \coloneqq a^{-1}(U_\epsilon(\lambda))\ (\forall \epsilon > 0)$とおく.
			 	或る$\epsilon > 0$が存在して$\mu(V_\epsilon) = 0$が成り立つ場合,
			 	\begin{align}
			 		b(x) \coloneqq 
			 		\begin{cases}
			 			1/\left( \lambda - a(x) \right) & (x \in X \backslash V_\epsilon) \\
			 			0 & (x \in V_\epsilon)
			 		\end{cases}
			 	\end{align}
				として$b$を定めれば,任意の$u \in H$に対して
				\begin{align}
					\int_X |b(x)u(x)|^2\ \mu(dx)
					= \int_{X \backslash V_\epsilon} \frac{1}{|\lambda - a(x)|^2} |u(x)|^2\ \mu(dx)
					\leq \frac{1}{\epsilon^2} \int_X |u(x)|^2\ \mu(dx) < \infty 
				\end{align}
				が成り立つから,$M_b$は$\Dom{M_b} = H$を満たす有界線型作用素である.更に$b(x) \left( \lambda - a(x) \right) = 1\ (\forall x \in V_\epsilon)$により
				\begin{align}
					b(x) \left( \lambda - a(x) \right) u(x) &= u(x) \quad (\mbox{$\mu$-a.e.}x \in X,\ \forall u \in \Dom{M_a} ), \\
					\left( \lambda - a(x) \right) b(x) u(x) &= u(x) \quad (\mbox{$\mu$-a.e.}x \in X,\ \forall u \in H )
				\end{align}
				が成り立つから,$M_b = (\lambda I - M_a)^{-1}$となり
				$\lambda \in \Res{M_a} $が従う.
				以上より
				\begin{align}
					\Spctr{M_a} \subset \Set{\lambda \in \C}{\mbox{$\forall \epsilon > 0$に対し$\mu\left( a^{-1}(U_\epsilon(\lambda)) \right) > 0$}}
					\label{eq:report_8_1}
				\end{align}
				が成立する.
				次に逆の包含関係を示す.$\mu(V_\epsilon) > 0\ (\forall \epsilon > 0)$が満たされている時,
				任意に$\epsilon > 0$を取り固定する.
				\begin{align}
					\mu(V_\epsilon) = \lim_{n \to \infty} \mu(V_\epsilon \cap X_n)
				\end{align}
				が成り立つから,或る$N \in \N$が存在して$\mu(V_\epsilon \cap X_N) > 0$を満たす.
				\begin{align}
					u_\epsilon(x) \coloneqq
					\begin{cases}
						1 & (x \in V_\epsilon \cap X_N) \\
						0 & (x \notin V_\epsilon \cap X_N)
					\end{cases}
				\end{align}
				として$u_\epsilon$を定めれば,$u_\epsilon$は二乗可積分であり
				\begin{align}
					\int_X |a(x)u_\epsilon(x)|^2\ \mu(dx)
					= \int_{V_\epsilon \cap X_N} |a(x)u_\epsilon(x)|^2\ \mu(dx)
					\leq \left( \epsilon + |\lambda| \right)^2 \mu\left( V_\epsilon \cap X_N \right)
					< \infty
				\end{align}
				を満たすから$u_\epsilon \in \Dom{M_a} $が従う.また
				\begin{align}
					&\Norm{(\lambda I- M_a)u_\epsilon}{}^2
					= \int_X |\lambda - a(x)|^2|u_\epsilon(x)|^2\ \mu(dx) \\
					&\qquad = \int_{V_\epsilon \cap X_N} |\lambda - a(x)|^2|u_\epsilon(x)|^2\ \mu(dx)
					\leq \epsilon^2 \int_X |u_\epsilon(x)|^2\ \mu(dx)
					= \epsilon^2 \Norm{u_\epsilon}{}^2
				\end{align}
				を満たす.$\epsilon > 0$は任意に選んでいたから,任意の$\epsilon > 0$に対し或る$u_\epsilon \in \Dom{M_a} $が存在して
				\begin{align}
					\Norm{(\lambda I- M_a)u_\epsilon}{} \leq \epsilon \Norm{u_\epsilon}{}
				\end{align}
				が成り立つ.この場合$(\lambda I- M_a)^{-1}$が存在しても,
				$u_\epsilon = (\lambda I- M_a)^{-1}v_\epsilon$を満たす
				$v_\epsilon \in \Dom{(\lambda I- M_a)^{-1}} $に対して
				\begin{align}
					\frac{1}{\epsilon} \leq \frac{\Norm{(\lambda I- M_a)^{-1}v_\epsilon}{}}{\Norm{v_\epsilon}{}}
				\end{align}
				が従い,$\epsilon$の任意性より$(\lambda I- M_a)^{-1}$の作用素ノルムは非有界である.
				ゆえに$\lambda \in \Spctr{M_a} $が成立し,(\refeq{eq:report_8_1})と併せて
				\begin{align}
					\Spctr{M_a} = \Set{\lambda \in \C}{\mbox{$\forall \epsilon > 0$に対し$\mu\left( a^{-1}(U_\epsilon(\lambda)) \right) > 0$}}
				\end{align}
				が得られる.
				
			\item[(4)] 
				先ず$\pSpctr{M_a} \subset \Set{z \in \C}{\mu\left( a^{-1}(\{z\}) \right) > 0}$が成り立つことを示す.
				任意の$\lambda \in \pSpctr{M_a}$に対しては固有ベクトル$u \in H$が存在し,
				固有ベクトルは$u \neq 0$を満たすから
				\begin{align}
					N \coloneqq \Set{x \in X}{u(x) \neq 0}
				\end{align}
				とおけば$\mu(N) > 0$が成り立つ.一方で点スペクトルの定義より$u$は$(\lambda I - M_a)u = 0$を満たすから,
				\begin{align}
					0 = \Norm{(\lambda I - M_a)u}{}^2 = \int_X |\lambda - a(x)|^2 |u(x)|^2\ \mu(dx)
					= \int_{N} |\lambda - a(x)|^2 |u(x)|^2\ \mu(dx)
				\end{align}
				が成り立ち
				\begin{align}
					\mu\left( \Set{x \in N}{|\lambda - a(x)| > 0} \right) = 0
				\end{align}
				が従う.よって
				\begin{align}
					\mu\left( a^{-1}(\{\lambda\}) \right)
					\geq \mu\left( \Set{x \in N}{|\lambda - a(x)| = 0} \right)
					= \mu(N)
					> 0
				\end{align}
				となり$\lambda \in \Set{z \in \C}{\mu\left( a^{-1}(\{z\}) \right) > 0}$が成り立つ.
				次に$\pSpctr{M_a} \supset \Set{z \in \C}{\mu\left( a^{-1}(\{z\}) \right) > 0}$が成り立つことを示す.
				任意に$\lambda \in \Set{z \in \C}{\mu\left( a^{-1}(\{z\}) \right) > 0}$を取り
				\begin{align}
					\Lambda \coloneqq a^{-1}(\{\lambda\})
				\end{align}
				とおく.
				\begin{align}
					0 < \mu(\Lambda) = \lim_{n \to \infty} \mu(\Lambda \cap X_n)
				\end{align}
				が成り立つから,或る$n \in \N$が存在して$\mu(\Lambda \cap X_n) > 0$を満たす.
				\begin{align}
					u(x) \coloneqq 
					\begin{cases}
						1 & (x \in \Lambda \cap X_n), \\
						0 & (x \notin \Lambda \cap X_n)
					\end{cases}
				\end{align}
				として$u$を定めれば$u$は二乗可積分であり,$\mu(\Lambda \cap X_n) > 0$であるから$u \neq 0$を満たす.また
				\begin{align}
					\Norm{(\lambda I - M_a)u}{}^2
					= \int_X |\lambda - a(x)|^2 |u(x)|^2\ \mu(dx)
					= \int_{\Lambda \cap X_n} |\lambda - a(x)|^2 |u(x)|^2\ \mu(dx)
					= 0
				\end{align}
				により$(\lambda I - M_a)u = 0$が従うから$u$は$\lambda$の固有ベクトルであり,$\lambda \in \pSpctr{M_a}$が成立する.
				\QED
		\end{description}
	\end{prf}
	
	\begin{screen}
		[9]$(\C,\borel{\C},\mu)$を$\sigma$-有限な測度空間,$H = \mathrm{L}^2(\C,\borel{\C},\mu) = \mathrm{L}^2(\mu)$とする.
			Borel可測関数$a:\C \rightarrow \C$に対して,$H$から$H$へのかけ算作用素$M_a$を次で定める:
			\begin{align}
				\Dom{M_a} = \Set{u \in H}{au \in H},
				\quad (M_a u)(z) = a(z) u(z) \quad (z \in \C).
			\end{align}
			また$E(A) = M_{\defunc_{a^{-1}(A)}}\ (A \in \borel{\C})$と定める.
			\begin{description}
				\item[(1)] $E$は,$\borel{\C}$で定義され,$H$上の直交射影を値とするスペクトル測度であることを示せ.
				\item[(2)] Borel可測関数$f:\C \rightarrow \C$に対し,
					\begin{align}
						T_f \coloneqq \int_\C f(z)\ E(dxdy) \quad (z = x + iy,\ (x,y) \in \R^2)
					\end{align}
					と定める.このとき,$T_f = M_{f \circ a}$を示せ.
			\end{description}
	\end{screen}
	
	%レポート問題9
	\begin{prf} $H$のノルムと内積をそれぞれ$\Norm{\cdot}{},\inprod<\cdot,\cdot>$と表す.
		\begin{description}
			\item[(1)] 
				任意の$A \in \borel{\C}$に対し
				$\defunc_{a^{-1}(A)}$は有界であるから,
				春学期のレポート問題より
				$M_{\defunc_{a^{-1}(A)}} \in \selfBop{H} $が成り立つ.
				ゆえに$E$は$\borel{\C}$全体で定義される.次に任意の$A \in \borel{\C}$に対し
				$E(A)$が$H$上の直交射影であることを示す.
				実際
				\begin{align}
					E(A)^2 u = M_{\defunc_{a^{-1}(A)}}M_{\defunc_{a^{-1}(A)}}u
					= \defunc_{a^{-1}(A)}\defunc_{a^{-1}(A)}u
					= \defunc_{a^{-1}(A)}u
					= E(A) u
					\quad (\forall u \in H)
				\end{align}
				により$E(A)^2 = E(A)$が成り立ち,また前問[8]の(2)により
				\begin{align}
					E(A)^* = M_{\defunc_{a^{-1}(A)}}^* = M_{\defunc_{a^{-1}(A)}} = E(A)
				\end{align}
				が成り立つから$E(A)$は自己共役である.従って$E(A)$は
				$H$上の直交射影である.
				最後に$E$がスペクトル測度であることを示す.
				先ず$a^{-1}(\C) = X$より
				\begin{align}
					\left( E(\C)u \right)(x) = \left( M_{\defunc_X}u \right)(x) 
					= \defunc_X(x) u(x) = u(x)
					\quad (\mbox{$\mu$-a.e.}x \in X,\ \forall u \in H)
				\end{align}
				が成り立ち$E(\C) = I$を得る.後は任意の互いに素な集合列
				$A_1,A_2,\cdots \in \borel{\C}$に対して,$A \coloneqq \sum_{n=1}^{\infty}A_n$として
				\begin{align}
					E(A) u = \sum_{n=1}^{\infty} E(A_n) u \quad (\forall u \in H)
					\label{eq:report_9_1}
				\end{align}
				が成立することを示せばよい:
				\begin{description}
					\item[第一段]
						先ず(\refeq{eq:report_9_1})の右辺の級数が$H$で収束することを示す.
						任意に$u \in H$を取り
						\begin{align}
							v_n \coloneqq \sum_{i=1}^{n} E(A_i) u
							\quad (n=1,2,\cdots)
						\end{align}
						として$(v_n)_{n=1}^{\infty} \subset H$を定める.任意の$n \in \N$に対し
						\begin{align}
							&\sum_{i=1}^{n} \Norm{E(A_i) u}{}^2
							= \sum_{i=1}^{n} \int_X \defunc_{a^{-1}(A_i)}(x) \left| u(x) \right|^2\ \mu(dx) \\
							&\qquad = \int_X \defunc_{a^{-1}\left( \sum_{i=1}^{n}A_i \right)}(x) \left| u(x) \right|^2\ \mu(dx)
							\leq \int_X |u(x)|^2\ \mu(dx)
							= \Norm{u}{}^2
						\end{align}
						が満たされるから
						\begin{align}
							\sum_{i=1}^{\infty} \Norm{E(A_i) u}{}^2 \leq \Norm{u}{}^2 < \infty
							\label{eq:report_9_2}
						\end{align}
						が従い,一方で任意の$p,q \in \N,\ p < q$に対し
						\begin{align}
							\Norm{v_q - v_p}{}^2
							= \int_X \left| \sum_{i=p+1}^{q} \defunc_{a^{-1}(A_i)}(x) u(x) \right|^2\ \mu(dx)
							= \int_X \sum_{i=p+1}^{q} \defunc_{a^{-1}(A_i)}(x) |u(x)|^2\ \mu(dx)
							= \sum_{i=p+1}^{q} \Norm{E(A_i) u}{}^2
						\end{align}
						が成り立つから,(\refeq{eq:report_9_2})より$(v_n)_{n=1}^{\infty}$は
						$H$でCauchy列をなし,$H$の完備性により或る$v \in H$に強収束する.
					\item[第二段]
						$E(A)u = v$が成り立つことを示す.
						任意の$n \in \N$に対し
						\begin{align}
							\left| v_n(x) - \left( E(A)u \right)(x) \right| \leq 2|u(x)|
							\quad (\forall x \in X)
						\end{align}
						が満たされ,かつ
						\begin{align}
							v_n(x) \longrightarrow \left( E(A)u \right)(x)
							\quad (n \longrightarrow \infty,\ \forall x \in X)
						\end{align}
						が成り立つからLebesgueの収束定理より
						\begin{align}
							\Norm{E(A)u - v_n}{}^2
							= \int_X \left| \left( E(A)u \right)(x) - v_n(x) \right|^2\ \mu(dx)
							\longrightarrow 0 \quad (n \longrightarrow \infty)
						\end{align}
						が従う.前段の結果と併せれば
						\begin{align}
							\Norm{E(A)u - v}{}
							\leq \Norm{E(A)u - v_n}{} + \Norm{v_n - v}{}
							\longrightarrow 0 \quad (n \longrightarrow \infty)
						\end{align}
						により$E(A)u = v$が成立する.$u$の任意性により(\refeq{eq:report_9_1})が得られる.
				\end{description}
				
			\item[(2)]
				先ず$\Dom{T_f} = \Dom{M_{f \circ a}} $が成り立つことを示す.
				$f$が可測単関数の場合,
				\begin{align}
					f = \sum_{i=1}^{n} \alpha_i \defunc_{A_i}
					\quad \left( \alpha_i \in \C,\ A_i \in \borel{\C}\ (i=1,\cdots,n),\ 
					\C = \sum_{i=1}^{n} A_i \right)
				\end{align}
				と表せば,任意の$u \in H$に対して
				\begin{align}
					\int_\C |f(z)|^2\ \mu_u(dz)
					= \sum_{i=1}^{n} |\alpha_i|^2 \inprod<E(A_i)u,u>
					= \sum_{i=1}^{n} |\alpha_i|^2 \int_X \defunc_{a^{-1}(A_i)}|u(x)|^2\ \mu(dx)
					= \int_X |f(a(x))|^2 |u(x)|^2\ \mu(dx)
				\end{align}
				が成り立つ
				\footnote{
					$u \in H$に対する$\mu_u$は
					\begin{align}
						\mu_u(\Lambda) \coloneqq \inprod<E(\Lambda)u,u>
						\quad (\Lambda \in \borel{\C})
					\end{align}
					により定められる有限正値測度を表す.
				}
				.$f$が一般の可測関数の場合は$f$の$MSF$-単調近似列$(f_n)_{n=1}^{\infty}$を取れば,
				任意の$n \in \N$に対して
				\begin{align}
					\int_\C |f_n(x)|^2\ \mu_u(dx) = \int_X |f_n(a(x))|^2 |u(x)|^2\ \mu(dx)
					\quad (\forall n \in \N)
				\end{align}
				が満たされる.そして単調収束定理より
				\begin{align}
					\int_\C |f(x)|^2\ \mu_u(dx) = \int_X |f(a(x))|^2 |u(x)|^2\ \mu(dx)
				\end{align}
				が成り立つから
				\begin{align}
					u \in \Dom{T_f} \quad \Leftrightarrow \quad u \in \Dom{M_{f \circ a}} 
				\end{align}
				が従う.次に$T_f u = M_{f \circ a} u\ (\forall u \in \Dom{T_f} )$を示す.
				$f$が可測単関数の場合,
				\begin{align}
					T_f u = \sum_{i=1}^n \alpha_i E(A_i) u
					= \sum_{i=1}^n \alpha_i \defunc_{a^{-1}(A_i)} u
					= f_n \circ a u
					= M_{f_n \circ a} u
					\quad (\forall u \in H)
				\end{align}
				が成り立つ.
				一般の$f$に対しては,$MSF$-単調近似列$(f_n)_{n=1}^{\infty}$を取れば任意の$n \in \N$に対して
				\begin{align}
					\Norm{T_f u - M_{f \circ a} u}{}
					\leq \Norm{T_f u - T_{f_n} u}{}
						+ \Norm{M_{f_n \circ a} u - M_{f \circ a} u}{}
				\end{align}
				が成立する.スペクトル積分$T_f$の定義より
				\begin{align}
					\Norm{T_f u - T_{f_n} u}{}
					\longrightarrow 0 \quad (n \longrightarrow \infty)
				\end{align}
				が満たされ,またLebesgueの収束定理より
				\begin{align}
					\Norm{M_{f_n \circ a} u - M_{f \circ a} u}{}^2
					= \int_X \left| f_n(a(x)) u(x) - f(a(x)) u(x) \right|^2\ \mu(dx)
					\longrightarrow 0 
					\quad (n \longrightarrow \infty)
				\end{align}
				も成り立つから
				\begin{align}
					\Norm{T_f u - M_{f \circ a} u}{} = 0
					\quad (\forall u \in \Dom{T_f} )
				\end{align}
				が従い$T_f = M_{f \circ a}$が得られる.
				\QED
		\end{description}
	\end{prf}
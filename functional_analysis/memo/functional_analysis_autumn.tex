\documentclass[a4j,10.5pt,oneside,openany]{jsbook}
%
\usepackage{amsmath,amssymb}
\usepackage{amsthm}
\usepackage{makeidx}
\usepackage{txfonts}
\usepackage{mathrsfs} %花文字
\usepackage{mathtools} %参照式のみ式番号表示
\usepackage{latexsym} %qed
\usepackage{ascmac}
\usepackage{color}
\usepackage{relsize}
\newtheoremstyle{mystyle}% % Name
	{20pt}%                      % Space above
	{20pt}%                      % Space below
	{\rm}%           % Body font
	{}%                      % Indent amount
	{\gt}%             % Theorem head font
	{.}%                      % Punctuation after theorem head
	{10pt}%                     % Space after theorem head, ' ', or \newline
	{}%                      % Theorem head spec (can be left empty, meaning `normal')
\theoremstyle{mystyle}

\allowdisplaybreaks[1]
\newcommand{\bhline}[1]{\noalign {\hrule height #1}} %表の罫線を太くする.
\newcommand{\bvline}[1]{\vrule width #1} %表の罫線を太くする.
\newtheorem{Prop}{$Proposition.$}
\newtheorem{Proof}{$Proof.$}
\newcommand{\QED}{% %証明終了
	\relax\ifmmode
		\eqno{%
		\setlength{\fboxsep}{2pt}\setlength{\fboxrule}{0.3pt}
		\fcolorbox{black}{black}{\rule[2pt]{0pt}{1ex}}}
	\else
		\begingroup
		\setlength{\fboxsep}{2pt}\setlength{\fboxrule}{0.3pt}
		\hfill\fcolorbox{black}{black}{\rule[2pt]{0pt}{1ex}}
		\endgroup
	\fi}
\newtheorem{thm}{定理}[section]
\newtheorem{dfn}[thm]{定義}
\newtheorem{prp}[thm]{命題}
\newtheorem{cor}[thm]{系}
\newtheorem{lem}[thm]{補題}
\newtheorem*{prf}{証明}
\newtheorem{rem}[thm]{注意}
\newtheorem{e.g.}[thm]{例}
\newcommand{\defunc}{\mbox{1}\hspace{-0.25em}\mbox{l}} %定義関数
\def\supp#1{\operatorname{supp} #1 } %support
\def\Box#1{$(\mbox{#1})$} %丸括弧つきコメント
\def\Ddot#1{$\ddot{\mathrm{#1}}$} %文中ddot
\def\DEF{\overset{\mathrm{def}}{\Leftrightarrow}} %定義記号
\def\Set#1#2{\left\{\ #1\ \, ; \quad #2\ \right\}} %集合の書き方
\def\eqqcolon{=\mathrel{\mathop:}} %定義=:
\def\oparrow{\overset{\mathrm{op}}{\rightarrow}} %作用素の矢印
\newcommand{\wlim}{\mbox{w-}\lim} %弱収束
\newcommand{\wstarlim}{\ast \mbox{w-}\lim} %弱収束
\def\max#1#2{\operatorname*{max}_{#1} #2 } %最大
\def\min#1#2{\operatorname*{min}_{#1} #2 } %最小
\def\sin#1#2{\operatorname{sin}^{#2} #1} %sin
\def\cos#1#2{\operatorname{cos}^{#2} #1} %cos
\def\tan#1#2{\operatorname{tan}^{#2} #1} %tan
\def\inprod<#1>{\left\langle #1 \right\rangle} %内積
\def\sup#1#2{\operatorname*{sup}_{#1} #2 } %上限
\def\inf#1#2{\operatorname*{inf}_{#1} #2 } %下限
\def\esssup#1#2{\mathrm{ess.sup}_{#1} #2 } %本質的上限
\def\Vector#1{\mbox{\boldmath $#1$}} %ベクトルを太字表示
\def\Norm#1#2{\left\|\, #1\, \right\|_{#2} } %ノルム
\def\Log#1{\operatorname{log} #1} %log
\def\Rank#1{\operatorname{rank} #1} %階数
\def\Det#1{\operatorname{det} (#1)} %行列式
\def\Diag#1{\operatorname{diag} \left(#1\right)} %行列の対角成分
\def\Tmat#1{#1^\mathrm{T}} %転置行列
\def\Dim#1{\operatorname{dim} #1} %次元
\def\Codim#1{\operatorname{codim} #1} %余次元
\def\Dom#1 {\mathcal{D}\left(#1\right)} %作用素の定義域
\def\Graph#1 {\mathcal{G}\left(#1\right)} %作用素のグラフ
\def\Ran#1{\mathcal{R}\left(#1\right)} %作用素の値域
\def\Ker#1{\mathcal{N}\left(#1\right)} %作用素の核
\def\Bop#1#2{\mathrm{B}(#1,#2)} %有界作用素の空間
\def\selfBop#1{\mathrm{B}(#1)} %有界作用素の空間[始集合=終集合]
\def\Cop#1#2{\mathrm{B}_c(#1,#2)} %コンパクト作用素の空間
\def\selfCop#1{\mathrm{B}_c(#1)} %コンパクト作用素の空間[始集合=終集合]
\def\Res#1{\mathfrak{\rho} (#1)} %レゾルベント集合
\def\Spctr#1{\mathfrak{\sigma} (#1)} %スペクトル集合
\def\pSpctr#1{\mathfrak{\sigma}_p (#1)} %点スペクトル集合
\def\Oproj#1{\mathrm{Proj}( #1 )} %直交射影(orthogonal projection)
\def\Exp#1{\operatorname{E} \left[ #1 \right]} %期待値
\def\Var#1{\operatorname{V} \left[ #1 \right]} %分散
\def\Cov#1#2{\operatorname{Cov} \left[ #1,\ #2 \right]} %共分散
\def\LH#1{\mathrm{L.h.}\left[\, #1\, \right] } %線型包(linear hull)
\def\exp#1{e^{#1}} %指数関数
\def\N{\mathbb{N}} %自然数全体
\def\Q{\mathbb{Q}} %有理数全体
\def\R{\mathbb{R}} %実数全体
\def\Z{\mathbb{Z}} %整数全体
\def\C{\mathbb{C}} %複素数全体
\def\K{\mathbb{K}} %係数体K
\def\im{\mathrm{i}} %虚数単位
\def\Re#1{\mathrm{Re}\left[ #1 \right]} %実部
\def\Im#1{\mathrm{Im}\left[ #1 \right]} %虚部
\def\conj#1{\overline{#1}} %共役複素数
\def\CM{\mathrm{Meas}_{\C}} %複素測度全体
\def\borel#1{\mathfrak{B}(#1)} %Borel集合族
\def\open#1{\mathfrak{O}(#1)} %位相空間 #1 の位相
\def\close#1{\mathfrak{A}(#1)} %%位相空間 #1 の閉集合系
\def\nbh#1{\mathscr{V}_{#1}} %位相空間の点 #1 の近傍全体
\def\fnbh#1{\mathscr{B}_{#1}} %基本近傍系 fundamental neighborhood
\def\rapid#1{\mathcal{S}(#1)} %急減少空間
\def\tempdist#1{\mathcal{S}'(#1)} %緩増加超関数空間 tempered distribution
\def\comtempdist#1{\mathcal{E}'(#1)} %コンパクト台を持つ緩増加超関数空間
\def\ft#1{\mathscr{F}[#1]} %Fourier変換
\def\cft#1{\overline{\mathscr{F}}[#1]} %逆Fourier変換
\def\c#1{C(#1)} %連続関数
\def\cbound#1{C_b (#1)} %有界連続関数
\def\supp#1{\operatorname{supp} #1} %関数の台
\def\ckon#1{C_c (#1)} %コンパクトな台を持つ連続関数
\def\cvan#1{C_0 (#1)} %遠方で0になる連続関数
\def\Lp#1#2{\operatorname{L}^{#1} \left(#2\right)} %L^p
\def\cn#1#2{C^{#2} (#1)} %n回連続微分可能関数
\def\cinf#1{C^{\infty} (#1)} %無限回連続微分可能関数
\def\Test#1{\mathscr{D}(#1)} %テスト関数の空間
\def\sgmalg#1{\sigma \left[#1\right]} %#1が生成するσ加法族
\def\ball#1#2{\operatorname{B} \left(#1\, ;\, #2 \right)} %開球
\def\closure#1{\overline{#1}} %閉包
\def\prob#1{\operatorname{P} \left(#1\right)} %確率
\def\cprob#1#2{\operatorname{P} \left(\left\{ #1 \ \middle|\ #2 \right\}\right)} %条件付確率
\def\cexp#1#2{\operatorname{E} \left[ #1 \ \middle|\ #2 \right]} %条件付期待値
%\renewcommand{\contentsname}{\bm Index}
%
\makeindex
%
\setlength{\textwidth}{\fullwidth}
\setlength{\textheight}{40\baselineskip}
\addtolength{\textheight}{\topskip}
%\setlength{\voffset}{-0.55in}
%

\title{関数解析(土居先生)後期メモ}
\author{百合川}
\date{\today}

\begin{document}
%
%

\mathtoolsset{showonlyrefs = true}
\maketitle
%
%
\tableofcontents
\frontmatter
%
\mainmatter
%
%本文
\chapter{ノルム空間}
	$\K$を$\R$又は$\C$とする.$\K$上のノルム空間$X$におけるノルムを$\Norm{\cdot}{X}$と表記し,$X$にノルム位相を導入する.
	\begin{screen}
		\begin{thm}[有限次元空間における有界点列の収束(局所コンパクト性)]\mbox{}\\
			$\K$を$\R$又は$\C$とし,$X$を$\K$上のノルム空間とする.$\Dim{X} < \infty$ならば$X$の任意の有界点列は
			収束部分列を含む.
			\label{thm:normed_space_locally_compact}
		\end{thm}
	\end{screen}

	\begin{prf}\mbox{}
		$X$の次元数$n$による帰納法で証明する.
		\begin{description}
			\item[第一段]
				$n=1$のとき$X$の基底を$u_1$とすれば,$X$の任意の有界点列は
				$( \alpha_m u_1)_{m=1}^{\infty} \quad (\alpha_m \in \K,\ m=1,2,\cdots)$と表せる.
				$\left( \alpha_m \right)_{m=1}^{\infty}$は有界列であるから,
				Bolzano-Weierstrassの定理より部分列$\left( \alpha_{m_k} \right)_{k=1}^{\infty}$と$\alpha \in \K$が存在して
				\begin{align}
					\left| \alpha_{m_k} - \alpha \right| \longrightarrow 0 \quad (k \longrightarrow \infty)
				\end{align}
				を満たし
				\begin{align}
					\Norm{\alpha_{m_k} u_1 - \alpha u_1}{X} \longrightarrow 0 \quad (k \longrightarrow \infty)
				\end{align}
				が従う.
			
			\item[第二段]
				$n=k$のとき定理の主張が成り立つと仮定し,$n = k+1$として$X$の基底を$u_1,\cdots,u_{k+1}$と表す.
				$X$から任意に有界列$(x_j)_{j=1}^{\infty}$を取れば,各$x_j$は
				\begin{align}
					x_j = y_j + \beta_j u_{k+1} \quad (y_j \in \LH{\left\{\, u_1,\cdots,u_k\, \right\}},\ \beta_j \in \K)
				\end{align}
				として一意に表示される.$(\beta_j)_{j=1}^{\infty}$が有界でないと仮定すると
				$\beta_{j_s} \geq s\ (j_s < j_{s+1},\ s=1,2,\cdots)$を満たす部分列が存在し,$(x_j)_{j=1}^{\infty}$の有界性と併せて
				\begin{align}
					\Norm{u_{k+1} + \tfrac{1}{\beta_{j_s}}y_{j_s}}{X}
					\leq \Norm{u_{k+1} + \tfrac{1}{\beta_{j_s}}y_{j_s} - \tfrac{1}{\beta_{j_s}}x_{j_s}}{X}
						+ \Norm{\tfrac{1}{\beta_{j_s}}x_{j_s}}{X}
					= \Norm{\tfrac{1}{\beta_{j_s}}x_{j_s}}{X} \longrightarrow 0 \quad (s \longrightarrow \infty)
				\end{align}
				が成り立つが,有限次元空間は閉であるから
				$u_{k+1} \in \LH{\left\{\, u_1,\cdots,u_k\, \right\}}$が従い矛盾が生じる.よって$(\beta_j)_{j=1}^{\infty}$は
				$\K$の有界列でなくてはならず,Bolzano-Weierstrassの定理より部分列$\left( \beta_{j(1,i)} \right)_{i=1}^{\infty}$と$\beta \in \K$が存在して
				\begin{align}
					\left| \beta_{j(1,i)} - \beta \right| \longrightarrow 0 \quad (i \longrightarrow \infty)
				\end{align}
				を満たす.また$\left(y_{j(1,i)}\right)_{i=1}^{\infty}$も有界列となるから,或る
				$y \in \LH{\left\{\, u_1,\cdots,u_k\, \right\}}$と部分列$\left(y_{j(2,i)}\right)_{i=1}^{\infty}$が存在して
				\begin{align}
					\Norm{y_{j(2,i)} - y}{X} \longrightarrow 0 \quad (i \longrightarrow \infty)
				\end{align}
				を満たす.従って
				\begin{align}
					\Norm{x_{j(2,i)} - \left(y + \beta u_{k+1}\right)}{X} \leq
					\Norm{y_{j(2,i)} - y}{X} + \left| \beta_{j(1,i)} - \beta \right| \Norm{u_{k+1}}{X}
					\longrightarrow 0 \quad (i \longrightarrow \infty)
				\end{align}
				が成り立つ.
				\QED
		\end{description}
	\end{prf}
	
	\begin{screen}
		\begin{thm}[閉部分空間との点の距離]
			$X$をノルム空間,$L \subsetneq X$を閉部分空間とする.このとき任意の$\epsilon > 0$に対して
			或る$e \in X$が存在し,$\Norm{e}{X} = 1$かつ次を満たす:
			\begin{align}
				\inf{x \in L}{\Norm{e - x}{X}} > 1 - \epsilon.
			\end{align}
			\label{thm:closed_subspace_distance}
		\end{thm}
	\end{screen}
	
	\begin{prf}
		
	\end{prf}
	
	\begin{screen}
		\begin{thm}[単位球面がコンパクトなら有限次元]
			$X$をノルム空間,$S$を$X$の単位球面とする.
			$S$がコンパクトならば$\Dim{X} < \infty$である.
			\label{thm:normed_space_unit_sphere_compact_finite_dimension}
		\end{thm}
	\end{screen}
	
	\begin{prf}
		対偶を証明する.距離空間のコンパクト性についての一般論より,$S$がコンパクトであることと$S$の任意の点列が
		$S$で収束する部分列を含むことは同値である.$\Dim{X} = \infty$と仮定する.
		任意に一つ$e_1 \in S$を取り$L_1 \coloneqq \LH{\{\, e_1\, \}}$とおけば,$L_1$は$X$の閉部分空間であるから
		定理\ref{thm:closed_subspace_distance}より或る$e_2 \in S$が存在して
		\begin{align}
			\inf{x \in L_1}{\Norm{e_2 - x}{X}} > \frac{1}{2}
		\end{align}
		を満たす.$L_2 \coloneqq \LH{\{\, e_1,e_2\, \}}$も$X$の閉部分空間であるから
		或る$e_3 \in S$が存在して
		\begin{align}
			\inf{x \in L_2}{\Norm{e_3 - x}{X}} > \frac{1}{2}
		\end{align}
		を満たす.この操作を繰り返して$S$の点列$e_1,e_2,\cdots$を構成すれば,
		\begin{align}
			\Norm{e_n - e_m}{X} > \frac{1}{2} \quad (\forall n,m \in \N,\ n \neq m)
		\end{align}
		が成り立ち$\left( e_n \right)_{n=1}^{\infty}$は収束部分列を含みえない.
		\QED
	\end{prf}
\chapter{商ノルム空間}
	$\K$を$\R$又は$\C$として$X$を$\K$上のノルム空間とする.
	$X$のノルムを$\Norm{\cdot}{X}$と表記し$X$にノルム位相を導入する.
	また$X$の閉部分空間$Y$に対し
	\begin{align}
		x \sim y \DEF x - y \in Y \quad (\forall x,y \in X)
	\end{align}
	として$X$における同値関係$\sim$を定める
	\footnote{
		$x,y,z \in X$を取る.$Y$は線形空間であるから,
		反射率は$x - x = 0 \in Y$により従い,対称律は$x - y \in Y$なら
		$y - x = -(x - y) \in Y$が成り立つことにより従う.推移律についても,
		$x \sim y$かつ$y \sim z$が満たされているなら
		$x - z = (x - y) + (y - z) \in Y$が成り立ち$x \sim z$が従う.
	}
	.以降,関係$\sim$による$x \in X$の同値類を$[x]$と表し,商集合を$X/Y$と表す.
	
	\begin{screen}
		\begin{thm}[商集合における線型演算]
			$X/Y$において
			\begin{align}
				[x] + [y] \coloneqq [x + y], \quad
				\alpha [x] \coloneqq [\alpha x] \quad (\forall [x],[y] \in X/Y,\ \alpha \in \K)
				\label{eq:quotient_set_linear_calculation_1}
			\end{align}
			として演算を定義すれば,$X/Y$はこれを線型演算として線形空間となる.
		\end{thm}
	\end{screen}
	
	\begin{prf}\mbox{}
		\begin{description}
			\item[well-defined]
				先ず(\ref{eq:quotient_set_linear_calculation_1})の定義がwell-definedであることを示す.
				任意に$u \in [x],v \in [y],\alpha \in \K$を取り
				\begin{align}
					[u + v] = [x + y], \quad [\alpha u] = [\alpha x]
					\label{eq:quotient_set_linear_calculation_2}
				\end{align}
				が成り立つことをいえばよい.実際$x \sim u$かつ$y \sim v$であるから
				\begin{align}
					(x + y) - (u + v) = (x - u) + (y - v) \in Y,
					\quad \alpha x - \alpha v = \alpha(x - u) \in Y
					\end{align}
				が成り立ち(\refeq{eq:quotient_set_linear_calculation_1})が従う.
		\end{description}
		$X$が線形空間であるから$X/Y$は(\ref{eq:quotient_set_linear_calculation_1})の演算で閉じている.
		よってあとは以下の事項を確認すればよい.
		\begin{description}
			\item[加法]
				$X/Y$が加法について可換群をなすことを示す.任意に$[x],[y],[z] \in X/Y$を取れば
				\begin{align}
					([x]+[y]) + [z] = [x+y] + [z] = [(x+y) + z] = [x + (y+z)] = [x] + [y+z] = [x] + ([y]+[z])
				\end{align}
				が成り立ち結合律が従う.可換性は
				\begin{align}
					[x] + [y] = [x + y] = [y + x] = [y] + [x]
				\end{align}
				により従い,また$[x]$の逆元は$(-1)[x]$
				\footnote{
					$[x] + (-1)[y]$は$[x] - [y]$と表す.
				}
				,$X/Y$の零元は$Y = [0]$である.
				
			\item[スカラ倍]
				任意に$[x],[y] \in X/Y$と$\alpha,\beta \in \K$を取れば以下が成り立つ:
				\begin{description}
					\item[$(1)$] $(\alpha\beta)[x] = [(\alpha\beta)x] = [\alpha(\beta x)] = \alpha[\beta x] = \alpha(\beta [x]),$
					\item[$(2)$] $(\alpha + \beta)[x] = [(\alpha + \beta)x] = [\alpha x + \beta x] = [\alpha x] + [\beta x] = \alpha [x] + \beta [x],$
					\item[$(3)$] $\alpha ([x] + [y]) = \alpha [x+y] = [\alpha(x+y)] = [\alpha x + \alpha y] = [\alpha x] + [\alpha y] = \alpha [x] + \alpha [y],$
					\item[$(4)$] $1 [x] = [x].$
				\end{description}
		\end{description}
		\QED
	\end{prf}
	
	\begin{screen}
		\begin{lem}[同値類は閉]
			任意の$[x] \in X/Y$は$X$において閉集合となる.
			\label{lem:equivalence_class_closed}
		\end{lem}
	\end{screen}
	
	\begin{prf}
		任意に$[x] \in X/Y$を取る.距離空間の一般論より$u_n \in [x]\ (n=1,2,\cdots)$が或る$u \in X$に収束するとき
		$u \in [x]$が成り立つことを示せばよい.各$n \in \N$について$u_n - x \in Y$であり,かつ
		\begin{align}
			\Norm{(u_n - x) - (u - x)}{X} = \Norm{u_n - u}{X} \longrightarrow 0
			\quad (n \longrightarrow \infty)
		\end{align}
		が成り立つから,$Y$が閉であることにより$u - x \in Y$が従う.
		\QED
	\end{prf}
	
	\begin{screen}
		\begin{thm}[商空間におけるノルムの定義]
			$X/Y$において
			\begin{align}
				\Norm{[x]}{X/Y} \coloneqq \inf{u \in [x]}{\Norm{u}{X}} \quad (\forall [x] \in X/Y)
				\label{eq:thm_quotient_space_norm}
			\end{align}
			として$\Norm{\cdot}{X/Y}:X/Y \rightarrow \R$を定めれば,これはノルムとなる.
		\end{thm}
	\end{screen}
	
	\begin{prf}\mbox{}
		\begin{description}
			\item[正値性]
				$\Norm{\cdot}{X/Y}$が非負値であることは定義式(\refeq{eq:thm_quotient_space_norm})右辺の非負性による.
				また$[x] = [0]$である場合,
				\begin{align}
					\inf{u \in [x]}{\Norm{u}{X}} = \Norm{0}{X} = 0
				\end{align}
				が成り立ち$\Norm{[x]}{X/Y} = 0$が従う.逆に$\Norm{[x]}{X/Y} = 0$である場合,
				\begin{align}
					\Norm{u_n}{X} \leq \frac{1}{n} \quad (n=1,2,\cdots)
				\end{align}
				を満たす点列$u_n \in [x]\ (n=1,2,\cdots)$が存在する.
				すなわち$u_n \longrightarrow 0 \quad (n \longrightarrow \infty)$
				であるから,補助定理\ref{lem:equivalence_class_closed}により$0 \in [x]$が成り立ち$[x] = [0]$が従う.
				
			\item[同次性]
				任意に$[x] \in X/Y$と$\alpha \in \K$を取る.$\alpha = 0$の場合は
				\begin{align}
					\Norm{0 [x]}{X/Y} = \Norm{[0]}{X/Y} = 0 = 0 \Norm{[x]}{X/Y}
				\end{align}
				が成り立つ.$\alpha \neq 0$の場合は
				\begin{align}
					u \in [\alpha x] \quad \Leftrightarrow \quad \frac{1}{\alpha} u \in [x]
				\end{align}
				が成り立つから
				\begin{align}
					\Norm{\alpha [x]}{X/Y} = \Norm{[\alpha x]}{X/Y} = \inf{u \in [\alpha x]}{\Norm{u}{X}}
					= |\alpha| \inf{u \in [\alpha x]}{\Norm{(1/\alpha)u}{X}}
					= |\alpha| \inf{v \in [x]}{\Norm{v}{X}}
					= |\alpha |\Norm{[x]}{X/Y}
				\end{align}
				が従う.
			
			\item[劣加法性]
				任意に$[x],[y] \in X/Y$を取り
				\begin{align}
					L \coloneqq \Set{u + v}{u \in [x],\ v \in [y]}
				\end{align}
				とおけば,任意の$u+v \in L$に対し$(u+v) - (x+y) \in Y$となるから$L \subset [x+y]$が成り立つ.また
				\begin{align}
					\Norm{u + v}{X} \leq \Norm{u}{X} + \Norm{v}{X}
				\end{align}
				により
				\begin{align}
					\inf{u'+v' \in L}{\Norm{u' + v'}{X}} \leq \Norm{u}{X} + \Norm{v}{X} \quad (\forall u \in [x],\ v \in [y])
				\end{align}
				が成り立つから,
				\begin{align}
					\inf{u'+v' \in L}{\Norm{u' + v'}{X}} 
					\leq \inf{u \in [x]}{\Norm{u}{X}} + \inf{v \in [y]}{\Norm{v}{X}}
					= \Norm{[x]}{X/Y} + \Norm{[y]}{X/Y}
				\end{align}
				が従い
				\begin{align}
					\Norm{[x] + [y]}{X/Y} = \Norm{[x+y]}{X/Y} 
					= \inf{w \in [x+y]}{\Norm{w}{X}} 
					\leq \inf{u+v \in L}{\Norm{u + v}{X}}
					\leq \Norm{[x]}{X/Y} + \Norm{[y]}{X/Y}
				\end{align}
				を得る.
		\end{description}
		\QED
	\end{prf}
\chapter{閉作用素}
	本章を通じて$X,Y$を複素ノルム空間とし,ノルムをそれぞれ$\Norm{\cdot}{X},\Norm{\cdot}{Y}$と表す.

\section{直積ノルム空間の位相}
	\begin{screen}
		\begin{prp}[直積ノルム空間]
			$x \in X, y \in Y$の組を$[x,y]$と表し,$X$と$Y$の直積(product)を
			\begin{align}
				X \times Y \coloneqq \Set{[x,y]}{x \in X, y \in Y}
			\end{align}
			で定める.そして写像$\Norm{\cdot}{X \times Y}:X \times Y \rightarrow \R$を次で定める:
			\begin{align}
				\Norm{[x,y]}{X \times Y} \coloneqq \Norm{x}{X} + \Norm{y}{Y}
				\quad (\forall [x,y] \in X \times Y).
			\end{align}
			このとき次が成り立つ:
			\begin{description}
				\item[(1)] $X \times Y$は
					\begin{align}
						[x,y] + [s, t] \coloneqq [x + s, y + t],
						\quad \alpha [x,y] \coloneqq [\alpha x, \alpha y]
						\quad (\forall [x,y],[s,t] \in X \times Y,\ \alpha \in \C)
					\end{align}
					を線型演算として線形空間となる.
					
				\item[(2)] 線形空間$X \times Y$は$\Norm{\cdot}{X \times Y}$をノルムとしてノルム空間となる.
			\end{description}
		\end{prp}
	\end{screen}
	
	\begin{screen}
		\begin{thm}[ノルム位相と直積位相は一致する]
			$\Norm{\cdot}{X \times Y}$によるノルム位相と$X,Y$の直積位相は一致する.
		\end{thm}
	\end{screen}
	
	\begin{screen}
		\begin{thm}[$X,Y$が完備なら$X \times Y$も完備]
			$X,Y$がBanach空間であるとき,$X \times Y$もBanach空間である.
		\end{thm}
	\end{screen}

\section{閉作用素と閉グラフ定理}
	\begin{screen}
		\begin{thm}[Banach空間値の有界な閉作用素の定義域は閉]
			$Y$がBanach空間であるとき,閉作用素$T:X \oparrow Y$が有界なら
			$\Dom{T} $は$X$の閉部分空間である.
			\label{thm:domain_of_banach_valued_closed_op_is_closed}
		\end{thm}
	\end{screen}
	
	\begin{prf}
		$\Dom{T} $の点列$(u_n)_{n=1}^{\infty}$が$u_n \longrightarrow u \in X$を満たせば
		\begin{align}
			\Norm{T u_n - T u_m}{Y} \leq \Norm{T}{\Bop{X}{Y} }\Norm{u_n - u_m}{X}
			\longrightarrow 0 \quad (n,m \longrightarrow \infty)
		\end{align}
		が成り立ち,$Y$の完備性より$(T u_n)_{n=1}^{\infty}$は或る$v \in Y$に強収束する.
		$T$は閉作用素であるから$v = T u$が従う.
		\QED
	\end{prf}
	
	\begin{screen}
		\begin{thm}[閉作用素の逆も閉]
			$T:X \oparrow Y$を閉作用素とするとき,
			$T^{-1}$が存在すればこれも閉作用素となる.
			\label{thm:closed_linear_op_inverse}
		\end{thm}
	\end{screen}
	
	\begin{prf}
		$U:X \times Y \ni [x,y] \longmapsto [y,x] \in Y \times X$
		として同相写像$U$を定める.仮定より$\Graph{T} $は$X \times Y$の閉集合であり
		\begin{align}
			\Graph{T^{-1}} = U \Graph{T}  
		\end{align}
		が成り立つから,$\Graph{T^{-1}} $は$Y \times X$の閉集合である.
		\QED
	\end{prf}
	
	\begin{screen}
		\begin{thm}[閉グラフ定理]
		\end{thm}
	\end{screen}
\chapter{Bochner積分}
	\section{Bochner積分の定義}
	本章では$(X,\mathcal{M},\mu)$を測度空間,$(B,\Norm{\cdot}{})$を複素Banach空間とする.
	
	\begin{screen}
		\begin{lem}[距離空間値の可測関数列の極限は可測]
			
		\end{lem}
	\end{screen}
	
	\begin{screen}
		\begin{dfn}[Bochner積分]
			e
		\end{dfn}
	\end{screen}
	\section{Pettisの強可測性定理}
	\begin{screen}
		\begin{lem}[距離空間値の可測関数列の各点極限は可測]
			$(S,d)$を距離空間,$(X,\mathcal{M})$を可測空間とする.
			$\mathcal{M}/\borel{S}$-可測関数列$(f_n)_{n=1}^{\infty}$が
			各点収束すれば,
			$f \coloneqq \lim_{n \to \infty} f_n$で定める関数$f$もまた可測$\mathcal{M}/\borel{S}$となる.
			\label{lem:measurability_metric_space}
		\end{lem}
	\end{screen}
	
	\begin{prf}
		任意に$S$の閉集合$A$を取り,閉集合の系$(A_m)_{m=1}^{\infty}$を次で定める:
		\begin{align}
			A_m \coloneqq \Set{y \in S}{d(y,A) \leq \frac{1}{m}}, \quad (m=1,2,\cdots).\ \footnotemark
		\end{align}
		\footnotetext{
			$S \ni y \longmapsto d(y,A) \in [0,\infty)$は連続であるから,
			閉集合$[0,1/m]$は$S$の閉集合に引き戻される.
		}
		$f(x) \in A$を取れば$f_n(x) \longrightarrow f(x)$が満たされているから,任意の$m \in \N$に対し或る$N = N(x,m) \in \N$が対応して
		\begin{align}
			d\left( f_n(x),A \right) \leq d\left( f_n(x),f(x) \right) < \frac{1}{m}
			\quad (\forall n \geq N)
		\end{align}
		が成り立ち
		\begin{align}
			f^{-1}(A) \subset \bigcap_{m \geq 1} \bigcup_{N \in \N} \bigcap_{n \geq N} f_n^{-1}(A_m)
			\label{eq:lem_measurability_metric_space}
		\end{align}
		が従う.一方$f(x) \notin A$については,$0 < \epsilon < d(f(x),A)$を満たす$\epsilon$に対し
		或る$N = N(x,\epsilon) \in \N$が存在して
		\begin{align}
			d\left( f_n(x), f(x) \right) < \epsilon
			\quad (\forall n \geq N)
		\end{align}
		が成り立つから,$1/m < d(f(x),A) - \epsilon$を満たす$m \in \N$を取れば
		\begin{align}
			\frac{1}{m} < d(f(x),A) - d(f(x),f_n(x)) \leq d(f_n(x),A)
			\quad (\forall n \geq N)
		\end{align}
		が従い
		\begin{align}
			f^{-1}(A^c) \subset \bigcup_{m \geq 1} \bigcup_{N \in \N} \bigcap_{n \geq N} f_n^{-1}(A_m^c)
			\subset \bigcup_{m \geq 1} \bigcap_{N \in \N} \bigcup_{n \geq N} f_n^{-1}(A_m^c)
		\end{align}
		が得られる.(\refeq{eq:lem_measurability_metric_space})と併せれば
		\begin{align}
			f^{-1}(A) = \bigcap_{m \geq 1} \bigcup_{N \in \N} \bigcap_{n \geq N} f_n^{-1}(A_m)
		\end{align}
		が成り立つ.$S$の閉集合は$f$により$\mathcal{M}$の元に引き戻されるから$f$は可測$\mathcal{M}/\borel{S}$である.
		\QED
	\end{prf}
	\section{Bochner積分}
	\begin{screen}
		\begin{dfn}[Bochner積分]
			e
		\end{dfn}
	\end{screen}
\chapter{共役作用素}
	\input{thms/dual_operator}
	\section{Hilbert空間の共役作用素}
\chapter{レゾルベントとスペクトル}
	本章を通じて$X$を複素Banach空間とし,ノルムを$\Norm{\cdot}{}$で表す.また$I$を$X$上の恒等写像とする.

\section{レゾルベントとスペクトル}
	\begin{screen}
		\begin{lem}[レゾルベントは閉作用素]
			$T$を$X$上の線型作用素,$\lambda$を複素数とするとき,次は同値である:
			\begin{description}
				\item[(1)] $T$は閉作用素である.
				\item[(2)] $\lambda I - T$は閉作用素である.
				\item[(3)] $(\lambda I - T)^{-1}$が存在すれば,これは閉作用素である.
			\end{description}
		\end{lem}
	\end{screen}
	
	\begin{prf}\mbox{}
		\begin{description}
			\item[(1)$\Rightarrow$(2)]
				$[u_n,(\lambda I - T) u_n] \in \Graph{(\lambda I - T)} \ (n=1,2,\cdots)$が
				$u_n \longrightarrow u \in X,\ (\lambda I - T) u_n \longrightarrow v \in X$を満たすとき,
				\begin{align}
					\Norm{(\lambda u - v) - T u_n}{}
					\leq \Norm{(\lambda I - T) u_n - v}{} + \Norm{\lambda u_n - \lambda u}{}
					\longrightarrow 0 \quad (n \longrightarrow \infty)
				\end{align}
				より$T u_n \longrightarrow \lambda u - v$が成り立つ.$T$が閉作用素なら
				$T u = \lambda u - v$が従い$v = (\lambda I - T) u$を得る.
				
			\item[(2)$\Rightarrow$(1)]
				$[u_n,T u_n] \in \Graph{T} \ (n=1,2,\cdots)$が
				$u_n \longrightarrow u \in X,\ T u_n \longrightarrow v \in X$を満たすとき,
				\begin{align}
					\Norm{(\lambda I - T)u_n - (\lambda u - v)}{}
					\leq \Norm{\lambda u_n - \lambda u}{} + \Norm{v - T u_n}{}
					\longrightarrow 0 \quad (n \longrightarrow \infty)
				\end{align}
				より$(\lambda I - T)u_n \longrightarrow \lambda u - v$が成り立つ.
				$(\lambda I - T)$が閉作用素なら$(\lambda I - T)u = \lambda u - v$が従い
				$T u = v$を得る.
				
			\item[(2)$\Leftrightarrow$(3)]
				定理\ref{thm:closed_linear_op_inverse}による.
				\QED
		\end{description}
	\end{prf}
	
\section{コンパクト作用素}
	係数体を$\C$,$X,Y$をノルム空間,$K$を$X \rightarrow Y$の線型写像($\mathscr{D}(K) = X$)とする.
	以下では$X,Y$におけるノルムを$\Norm{\cdot}{X},\ \Norm{\cdot}{Y}$と表記し,
	位相はこれらのノルムにより導入されるものと考える.
	
	\begin{itembox}[l]{}
		\begin{dfn}[コンパクト作用素]
			任意の有界部分集合$B \subset X$に対して$KB$が相対コンパクトとなるとき,
			つまり$KB$の閉包$\closure{KB}$がコンパクトとなるとき,
			$K$をコンパクト作用素(compact operator)という.
		\end{dfn}
	\end{itembox}
	
	\begin{itembox}[l]{}
		\begin{lem}[コンパクト作用素となるための十分条件の一つ]
			$B_1 \coloneqq \left\{\ x \in X\quad |\quad \Norm{x}{X} < 1\ \right\}$に対して$\closure{KB_1}$が
			コンパクトであるなら$K$はコンパクト作用素となる.
		\end{lem}
	\end{itembox}
	
	\begin{prf}
		任意の有界集合$B \subset X$に対しては或る$\lambda$が取れて$B \subset \lambda B_1\ (= \left\{\ \lambda x\quad |\quad x \in B_1\ \right\})$
		となるようにできる.$K(\lambda B_1)$の閉包がコンパクトとなるなら$KB$の閉包もコンパクトとなる(コンパクト集合の閉部分集合はコンパクトとなる)から,
		$\closure{K(\lambda B_1)}$がコンパクトとなることを示せばよい.先ず
		\begin{align}
			\closure{K(\lambda B_1)} = \lambda \closure{KB_1}
		\end{align}
		が成り立つことを示す.$x \in \closure{K(\lambda B_1)}$に対しては点列$(x_n)_{n=1}^{\infty} \subset K(\lambda B_1)$が取れて
		$\Norm{x_n - x}{X} \longrightarrow 0\ (n \longrightarrow \infty)$が成り立つ.
		$y_n \coloneqq x_n/\lambda$とおけば$K$の線型性により$y_n \in KB_1$となり,
		$\Norm{y_n - x/\lambda}{X}= \Norm{x_n - x}{X}/\lambda \longrightarrow 0\ (n \longrightarrow \infty)$
		となるから$x/\lambda \in \closure{KB_1}$すなわち$x \in \lambda\closure{KB_1}$が判る.
		逆に$x \in \lambda \closure{KB_1}$に対しては$x/\lambda \in \closure{KB_1}$となるから,
		或る点列$(t_n)_{n=1}^{\infty} \subset KB_1$が存在して$\Norm{t_n - x/\lambda}{X} \longrightarrow 0\ (n \longrightarrow \infty)$
		が成り立つ.$s_n = \lambda t_n$とおけば$K$の線型性により$s_n \in K(\lambda B_1)$となり,
		$\Norm{s_n - x}{X}= \lambda \Norm{t_n - x/\lambda}{X} \longrightarrow 0\ (n \longrightarrow \infty)$
		が成り立つから$x \in \closure{K(\lambda B_1)}$が判る.以上で$\closure{K(\lambda B_1)} = \lambda \closure{KB_1}$が示された.
		$\closure{K(\lambda B_1)}$を覆う任意の開被覆$\cup_{\mu \in M}O_\mu\ $($M$は任意濃度)に対し
		\begin{align}
			\closure{KB_1} \subset \bigcup_{\mu \in M} \tfrac{1}{\lambda}O_\mu
		\end{align}
		が成り立ち\footnote{開集合$O_\mu$は$1/\lambda$でスケールを変えてもまた開集合となる.},仮定より$\closure{KB_1}$はコンパクトであるから$M$から有限個の$\mu_i\ (i=1,\cdots,n)$を取り出して
		\begin{align}
			\closure{KB_1} \subset \bigcup_{i=1}^{n} \tfrac{1}{\lambda}O_{\mu_i}
		\end{align}
		とできる.従って$\closure{K(\lambda B_1)}$は$O_{\mu_1}\cup \cdots \cup O_{\mu_n}$で覆われることになるからコンパクトであると示された.
		\QED
	\end{prf}
	
	\begin{itembox}[l]{}
		\begin{lem}[コンパクト作用素であることの同値条件]
			(1)$K$がコンパクトであることと,(2)$X$の任意の有界点列$(x_n)_{n=1}^{\infty}$に対し点列$(Tx_n)_{n=1}^{\infty}$が
				$\closure{(Tx_n)_{n=1}^{\infty}}$で収束する部分列を含むことは同値である.
		\end{lem}
	\end{itembox}
	
	\begin{prf}\mbox{}
		\begin{description}
			\item[(1)$\Rightarrow$(2)]
				$B = (x_n)_{n=1}^{\infty}$とおけば$B$は$X$において有界集合となるから$KB$は相対コンパクトである.
				点列$(x_n)_{n=1}^{\infty}$は$\closure{KB}$の点列でもあるから,コンパクト性の一般論により
				$(x_n)_{n=1}^{\infty}$は点列コンパクト,つまり収束部分列を持つ.
			\item[(2)$\Rightarrow$(1)]
				一般論より任意の有界集合$B \subset X$に対して$\closure{TB}$がコンパクトとなるための同値条件は
				$\closure{TB}$が点列コンパクトとなることである.
				このためには「$TB$が点列コンパクトなら$\closure{TB}$も点列コンパクトとなる」---(※)を示せばよい.
				(※)が示されたとして,(2)を仮定すれば$TB$の任意の点列は$(x_n)_{n=1}^{\infty} \subset B$(有界)によって
				$(Tx_n)_{n=1}^{\infty}$と表現できるから収束する部分列を持ち,(※)の主張と上の一般論により$\closure{TB}$はコンパクトとなる.
				これより(※)を示す.$\closure{TB}$の任意の点列$(y_n)_{n=1}^{\infty}$に対して
				$\Norm{y_n - z_n}{Y} < 1/n\ (n=1,2,\cdots)$を満たす$(z_n)_{n=1}^{\infty} \subset TB$が存在する.
				部分列$(z_{n_k})_{k=1}^{\infty}$が$z \in TB$に収束するなら,任意の$\epsilon > 0$に対し
				或る$K_1 \in \N$が取れて$k \geq K_1$ならば$\Norm{z - z_{n_k}}{Y} < \epsilon/2$を満たす.
				更に或る$K_2 \in \N$が取れて$k \geq K_2$なら$1/n_k < \epsilon/2$も満たされ,$\forall k \geq \max{}{\{K_1,K_2\}}$
				に対して
				\begin{align}
					\Norm{z - y_{n_k}}{Y} \leq \Norm{z - z_{n_k}}{Y} + \Norm{z_{n_k} - y_{n_k}}{Y} < \epsilon
				\end{align}
				が成り立つ.これで$(y_n)_{n=1}^{\infty}$が収束部分列$(y_{n_k})_{k=1}^{\infty}$を持つと示された.
		\end{description}
		\QED
	\end{prf}
	
	\begin{itembox}[l]{}
		\begin{prp}[コンパクト作用素の空間・作用素の合成がコンパクトとなるための十分条件]\mbox{}
			\begin{description}
				\item[(1)] $B_c(X,Y) \coloneqq \left\{\ K:X \rightarrow Y\quad |\quad \mbox{$K$:コンパクト作用素}\ \right\}$
					とおけば$B_c(X,Y)$は$B(X,Y)$の線型部分空間となる.
				\item[(2)] $Z$をノルム空間とする.$A \in B(X,Y)$と$B \in B(Y,Z)$に対して$A$又は$B$がコンパクト作用素なら$BA$もまたコンパクト作用素となる.
			\end{description}
		\end{prp}
	\end{itembox}
	
	\begin{prf}\mbox{}
		\begin{description}
			\item[(1)] $B_1 \coloneqq \left\{\ x \in X \quad |\quad \Norm{x}{X} \leq 1\ \right\}$とおけば
				任意の$K \in B_c(X,Y)$に対して$\closure{TB_1}$はコンパクトとなる.従って$TB_1$は有界で
				\begin{align}
					\infty > \sup{x \in B_1 \backslash \{0\}}{\Norm{Kx}{Y}} = \sup{0 < \Norm{x} \leq 1}{\Norm{Kx}{Y}}
				\end{align}
				が成り立ち,$K \in B(X,Y)$であると示された.次に$B_c(X,Y)$が線形空間であることを示す.
				$K_1, K_2 \in B_c(X,Y)$と$\alpha \in \C$を任意に取り,前補助定理を使う.
				補助定理によれば,任意の有界点列$(x_n)_{n=1}^{\infty} \subset X$に対して$(K_1x_n)_{n=1}^{\infty}$
				は収束部分列$(K_1x_{n_k})_{k=1}^{\infty}$を持つ.この部分列$(n_k)_{k=1}^{\infty}$
				に対して$(K_2x_{n_k})_{k=1}^{\infty}$もまた収束部分列$(K_2x_{n_{kl}})_{l=1}^{\infty}$
				を持ち,$(K_1x_{n_{kl}})_{l=1}^{\infty}$もまた収束列であることに注意すれば
				$\left( (K_1 + K_2)(x_{n_{kl}}) \right)_{l=1}^{\infty}$が収束部分列となるから
				前補助定理より$K_1 + K_2$もコンパクト作用素となる.$K_1$に対して,$(\alpha K_1x_{n_k})_{k=1}^{\infty}$
				もまた収束列であるから$\alpha K_1$もコンパクト作用素となる.以上で$B_c(X,Y)$が線形空間であると示された.
			
			\item[(2)]\mbox{}
				\begin{description}
					\item[$A$がコンパクト作用素である場合]
						補助定理により,$X$の任意の点列$(x_n)_{n=1}^{\infty}$に対し$(Ax_n)_{n=1}^{\infty}$は収束部分列
						$(Ax_{n_k})_{k=1}^{\infty}$を持つ.$B$の連続性により$(BAx_{n_k})_{k=1}^{\infty}$も
						収束列となるから,再び補助定理を適用して$BA$がコンパクト作用素であると示される.
					
					\item[$B$がコンパクト作用素である場合]
						任意の有界集合$S \subset X$に対して,$A$の有界性と併せて$AS$は有界となる.従って$\closure{BAS}$がコンパクトとなり
						$BA$はコンパクト作用素であると示された.
				\end{description}
		\end{description}
		\QED
	\end{prf}
\section{Fredholm性}

	\begin{screen}
		\begin{lem}[商空間のコンパクト作用素]
			$X$を複素ノルム空間,$Y$を$X$の閉部分空間とする.
			$A \in \selfCop{X} $が$AY \subset Y$を満たすとき次が成り立つ:
			\begin{description}
				\item[(1)] $A_1:Y \ni y \longmapsto A y \in Y$として$A_1$を定めれば$A_1 \in \selfCop{Y} $が成り立つ.
				\item[(2)] $A_2:X/Y \ni [x] \longmapsto [Ax] \in X/Y$として$A_2$を定めれば$A_2 \in \selfCop{X/Y} $が成り立つ.
			\end{description}
			\label{thm:compact_operator_on_quotient_normed_space}
		\end{lem}
	\end{screen}
	
	\begin{prf}\mbox{}
		\begin{description}
			\item[(1)] 任意に$Y$から有界点列$(x_n)_{n=1}^{\infty}$を取る.
				補助定理\ref{lem:compact_operator_equiv_cond}
				より$(A x_n)_{n=1}^{\infty}$の部分列$\left( A x_{n_k} \right)_{k=1}^{\infty}$は
				或る$y \in X$に収束し,$Y$が閉であるから$y \in Y$を満たす.
				$A_1 x_{n_k} = A x_{n_k}\ (k=1,2,\cdots)$より$A_1 x_{n_k} \longrightarrow y\ (k \longrightarrow \infty)$
				が従い,補助定理\ref{lem:compact_operator_equiv_cond}より$A_1 \in \selfCop{Y} $が成り立つ.
				
			\item[(2)]
				\begin{description}
					\item[well-defined] $A_2$の定義はwell-definedである.つまり同値類の表示の仕方に依らない.実際$[x] = [x']$なら
						\begin{align}
							A x - A x' = A(x - x') \in Y
						\end{align}
						が成り立つから$A_2[x] = [Ax] = [Ax'] = A_2[x']$が従う.
						また$[x],[y] \in X/Y$と$\alpha,\beta \in \K$に対し
						\begin{align}
							A_2(\alpha[x] + \beta[y]) = A_2[\alpha x + \beta y] 
							= [A(\alpha x + \beta y)] = [\alpha A x + \beta A y] = \alpha [Ax] + \beta [Ay] = \alpha A_2[x] + \beta A_2[y]
						\end{align}
						が成り立つから$A_2$は線型作用素である.
						
					\item[コンパクト性]
						$B$を$X/Y$の単位開球とする.$B$から任意に取った点列$\left( [x_n] \right)_{n=1}^{\infty}$に対して
						$\left( A_2[x_n] \right)_{n=1}^{\infty}$が$X/Y$で収束する部分列を含むなら,
						定理\ref{lem:compact_operator_equiv_cond}の証明中の(※)の主張により$A_2 B$は相対コンパクトとなり,
						定理\ref{lem:unit_ball_and_compact_operator}により$A$のコンパクト性が従う.
						各$n \in \N$について$\Norm{[x_n]}{X/Y} < 1$であるから$\Norm{u_n}{X} \leq 2$を満たす$u_n \in [x_n]$が存在する.
						定理\ref{lem:compact_operator_equiv_cond}より$(A u_n)_{n=1}^{\infty}$の或る部分列
						$\left( A u_{n_k} \right)_{k=1}^{\infty}$は或る$y \in Y$に収束するから
						\begin{align}
							\Norm{A_2 \left[x_{n_k}\right] - [y]}{X/Y} = \Norm{\left[ A x_{n_k} - y \right]}{X/Y} 
							\leq \Norm{A x_{n_k} - y}{X} \longrightarrow 0 \quad (k \longrightarrow \infty)
						\end{align}
						が成り立つ.
						\QED
				\end{description}
		\end{description}
	\end{prf}
	
	\begin{screen}
		\begin{thm}[複素Banach空間上のコンパクト作用素の値域の余次元,核の次元]\mbox{}\\
			$X$を複素Banach空間,$I$を$X$上の恒等写像とし,$0 \neq \lambda \in \C$と$A \in \selfCop{X} $に対して
			$T \coloneqq \lambda I - A$とおく.このとき
			$\Ran{T}$は$X$の閉部分空間であり,$\Dim{\Ker{T}} < \infty$かつ$\Codim{\Ran{T}} < \infty$\footnotemark
			が成り立つ.
			\label{thm:Banach_space_compact_operator_kernel_dimension}
		\end{thm}
	\end{screen}
	
	\footnotetext{
		$\Codim{\Ran{T}} = \Dim{X/\Ran{T}}$である.
	}
	
	\begin{prf}\mbox{}
		\begin{description}
			\item[$\Ran{T}$が閉となること]
				\begin{align}
					\hat{T}:X/\Ker{T} \ni [x] \longmapsto Tx \in \Ran{T}
				\end{align}
				と定めれば$\hat{T}$は線型同型かつ連続となる:
				\begin{description}
					\item[全単射]
						$\hat{T}$が単射であることは,$T[x] = T[x']$ならば$x - x' \in \Ker{T}$より
						$[x] = [x']$が従い,また任意の$y \in \Ran{T}$に対して,$y = Tx$を満たす$x \in X$の
						同値類$[x] \in X/\Ker{T}$が$\hat{T}[x] = y$を満たすから$\hat{T}$は全射である.
					
					\item[線型性]
						任意に$[x],[y] \in X/\Ker{T}$と$\alpha,\beta \in \C$を取れば
						\begin{align}
							\hat{T}\left( \alpha[x] + \beta[y] \right)
							= \hat{T}\left( [\alpha x] + [\beta y] \right)
							= T(\alpha x + \beta y)
							= \alpha T x + \beta T y
							= \alpha \hat{T} [x] + \beta \hat{T} [y]
						\end{align}
						が成立する.
						
					\item[連続性]
						定理\ref{prp:compact_operator_bounded_composition_of_compact_operators}より$A$は有界であるから
						\begin{align}
							\Norm{T}{\selfBop{X}} = \Norm{\lambda I - A}{\selfBop{X}} \leq |\lambda| + \Norm{A}{\selfBop{X}} < \infty
						\end{align}
						が成り立ち,任意の$[x] \in X/\Ker{T}$に対して
						\begin{align}
							\Norm{\hat{T}[x]}{X} = \Norm{Tx}{X} \leq \Norm{T}{\selfBop{X}} \Norm{x}{X}
						\end{align}
						が従うから$\hat{T}$は連続である.
				\end{description}		
				$\Ran{T} = \Ran{\hat{T}}$であるから$\Ran{\hat{T}}$が$X$の閉部分空間となることを示せばよい.
				まず或る$C > 0$が存在して
				\begin{align}
					C \Norm{\hat{T}[x]}{X} \geq \Norm{[x]}{X/\Ker{T}} \quad (\forall x \in X)
					\label{eq:thm_Banach_space_compact_operator_kernel_dimension_3}
				\end{align}
				を満たすことを示す.
				\begin{description}
					\item[(\refeq{eq:thm_Banach_space_compact_operator_kernel_dimension_3})の証明]
						このような$C$が存在しないなら
						\begin{align}
							\Norm{\hat{T}[x_n]}{X} < \frac{1}{n} \Norm{[x_n]}{X/\Ker{T}}
							\quad (n=1,2,\cdots) 
							\label{eq:thm_Banach_space_compact_operator_kernel_dimension}
						\end{align}
						を満たす$X/\Ker{T}$の点列$\left( [x_n] \right)_{n=1}^{\infty}$が存在する.
						\begin{align}
							[y_n] \coloneqq \frac{1}{\Norm{[x_n]}{X/\Ker{T}}} [x_n] \quad (n=1,2,\cdots)
						\end{align}
						とおけば$\left( [y_n] \right)_{n=1}^{\infty}$も(\refeq{eq:thm_Banach_space_compact_operator_kernel_dimension})
						を満たし,かつ$\hat{T}[y_n] = \hat{T}[u_n] = T u_n$であるから
						\begin{align}
							\Norm{Tu_n}{X} = \Norm{\hat{T}[y_n]}{X} < \frac{1}{n} \longrightarrow 0
							\quad (n \longrightarrow \infty)
							\label{eq:thm_Banach_space_compact_operator_kernel_dimension_2}
						\end{align}
						が成立する.$\Norm{[y_n]}{X/\Ker{T}} = 1$であるから
						ノルムの定義(\refeq{eq:thm_quotient_space_norm})より$\Norm{u_n}{X} \leq 2$となる$u_n \in [y_n]$が存在し,
						定理\ref{lem:compact_operator_equiv_cond}より
						$\left( Au_n \right)_{n=1}^{\infty}$の或る部分列$\left( Au_{n_k} \right)_{k=1}^{\infty}$は
						或る$y \in X$に収束するから
						\begin{align}
							\Norm{y - \lambda u_{n_k}}{X} = \Norm{y - Au_{n_k} - Tu_{n_k}}{X} \leq \Norm{y - Au_{n_k}}{X} + \Norm{Tu_{n_k}}{X}
							\longrightarrow 0 \quad (k \longrightarrow \infty)
						\end{align}
						が成り立ち,更に$T$の有界性と(\refeq{eq:thm_Banach_space_compact_operator_kernel_dimension_2})より
						\begin{align}
							\Norm{Ty}{X} \leq \Norm{Ty - \lambda Tu_{n_k}}{X} + |\lambda| \Norm{Tu_{n_k}}{X} \longrightarrow 0
							\quad (k \longrightarrow \infty)
						\end{align}
						となり$y \in \Ker{T}$が従う.一方で
						\begin{align}
							\left|\, \Norm{[y]}{X/\Ker{T}} - \Norm{\lambda \left[y_{n_k}\right]}{X/\Ker{T}}\, \right| 
							\leq \Norm{[y] - \lambda \left[y_{n_k}\right]}{X/\Ker{T}} %= \Norm{\left[y - \lambda u_{n_k}\right]}{X/\Ker{T}}
							\leq \Norm{y - \lambda u_{n_k}}{X} \longrightarrow 0 \quad (k \longrightarrow \infty)
						\end{align}
						が成り立つから$\Norm{[y]}{X/\Ker{T}} = |\lambda| > 0$が従い$y \in \Ker{T}$に矛盾する.
				\end{description}
				$\Ran{\hat{T}}$の点列$\left( \hat{T}[v_n] \right)_{n=1}^{\infty}$が
				$\hat{T}[v_n] \rightarrow x \in X$を満たすなら,(\refeq{eq:thm_Banach_space_compact_operator_kernel_dimension_3})より
				\begin{align}
					\Norm{[v_n] - [v_m]}{X/\Ker{T}} \leq C \Norm{\hat{T}[v_n] - \hat{T}[v_m]}{X} \longrightarrow 0
					\quad (n,m \longrightarrow \infty)
				\end{align}
				が成り立ち,定理\ref{thm:quotient_normed_space_Banach}より$\left( [v_n] \right)_{n=1}^{\infty}$は或る$[v] \in X/\Ker{T}$に収束する.
				よって$\hat{T}$の連続性から
				\begin{align}
					\Norm{x - \hat{T}[v]}{X} 
					%\leq \Norm{x - \hat{T}[v_n]}{X} + \Norm{\hat{T}[v_n] - \hat{T}[v]}{X}
					\leq \Norm{x - \hat{T}[v_n]}{X} + \Norm{\hat{T}}{\selfBop{\hat{T}}} \Norm{[v_n] - [v]}{X}
					\longrightarrow 0 \quad (n \longrightarrow \infty)
				\end{align}
				が成り立ち$x = \hat{T}[v] \in \Ran{\hat{T}}$が従う.
			
			\item[$\Dim{\Ker{T}} < \infty$となること]	
				$T = \lambda I - A$より
				\begin{align}
					\lambda x = A x \quad (\forall x \in \Ker{T})
					\label{eq:thm_Banach_space_compact_operator_kernel_dimension_4}
				\end{align}
				が成り立つから
				\begin{align}
					T A x = T \lambda x = \lambda T x = 0 \quad (\forall x \in \Ker{T})
				\end{align}
				となり$A \Ker{T} \subset \Ker{T}$が従う.
				よって$\Ker{T}$から任意に有界点列$(x_n)_{n=1}^{\infty}$を取れば
				$\left(A x_n \right)_{n=1}^{\infty}$は閉部分空間$\Ker{T}$に含まれ,
				定理\ref{lem:compact_operator_equiv_cond}より或る部分列
				$\left(A x_{n_k} \right)_{k=1}^{\infty}$は或る$x \in \Ker{T}$に収束する.
				そして(\refeq{eq:thm_Banach_space_compact_operator_kernel_dimension_4})より
				\begin{align}
					\Norm{\frac{1}{\lambda}x - x_{n_k}}{X}
					= \frac{1}{|\lambda|} \Norm{x - \lambda x_{n_k}}{X}
					= \frac{1}{|\lambda|} \Norm{x - A x_{n_k}}{X}
					\longrightarrow 0 \quad (k \longrightarrow \infty)
				\end{align}
				が成り立つから,定理\ref{thm:compact_identity_operator_and_dimension}より
				$\Dim{X} < \infty$が従う.
				
			\item[$\Codim{\Ran{T}} < \infty$となること]
				$\Ran{T}$は$X$の閉部分空間であるから商ノルム空間$X/\Ran{T}$を定義できる.
				\begin{align}
					U:X/\Ran{T} \ni [x] \longmapsto [Ax] \in X/\Ran{T}
				\end{align}
				と定めれば定理\ref{thm:compact_operator_on_quotient_normed_space}より
				$U$はコンパクト作用素である.
				\begin{align}
					[0] = [Tx] = [\lambda x - Ax] = \lambda [x] - [Ax] = \lambda [x] - U[x] \quad (\forall x \in X)
				\end{align}
				が成り立つから,定理\ref{thm:compact_identity_operator_and_dimension}より
				$\Dim{X/\Ran{T}} < \infty$が従う.
				\QED
		\end{description}
	\end{prf}
	
	\begin{screen}
		\begin{thm}[Fredholmの交代定理]
			
		\end{thm}
	\end{screen}
	
	\begin{screen}
		\begin{lem}
			$E$を複素ノルム空間,$E_1,E_2$を$E$の線型部分空間とし
			$E = E_1 + E_2$が成り立っているとする\footnotemark.
			また$E,E_1 \times E_2$におけるノルムをそれぞれ$\Norm{\cdot}{E},\Norm{\cdot}{E_1 \times E_2}$としてノルム位相を導入し
			\begin{align}
				\Phi:E \ni x \longmapsto [x_1,x_2] \in E_1 \times E_2
				\quad (x = x_1 + x_2)
			\end{align}
			を定める.このとき次が成り立つ:
			\begin{description}
				\item[(1)] $\Phi$は全単射かつ閉線型である.
				\item[(2)] $\Phi^{-1}$は連続である.
				\item[(3)] $\Phi$が連続ならば$E_1,E_2$は閉部分空間である.
				\item[(4)] $E$がBanach空間で$E_1,E_2$が閉部分空間ならば$\Phi$は線型同型かつ同相である.
				\item[(5)] $\Dim{E_1} < \infty$かつ$E_2$が閉ならば$\Phi$は線型同型かつ同相である.
			\end{description}
		\end{lem}
	\end{screen}
	
	\footnotetext{
		つまり$E_1 \cap E_2 = \{0\}$であり,かつ$E$の任意の元$x$は
		或る$x_1 \in E_1,x_2 \in E_2$によって$x = x_1 + x_2$と一意に表される.
		一意性について,$x = y_1 + y_2\ (y_1 \in E_1,y_2 \in E_2)$が同時に成り立っているとすれば
		\begin{align}
			E_1 \ni x_1 - y_1 = y_2 - x_2 \in E_2
		\end{align}
		となるから$ x_1 - y_1 = y_2 - x_2 = 0$が従う.
	}
	
	\begin{prf}\mbox{}
		\begin{description}
			\item[(1)] 
				\begin{description}
					\item[全単射であること]
						任意に$[x_1,x_2] \in E_1 \times E_2$を取れば
						$x_1 + x_2 \in E$を満たすから$\Phi$は全射である.
						また$E_1 \times E_2$の二元が$[x_1,x_2] = [y_1,y_2]$を満たせば
						$x_1 = y_1$かつ$x_2 = y_2$となるから$\Phi$は単射である.
						
					\item[閉線型であること]
						$x,y \in E,\alpha \in \C$を任意に取り$\Phi x = [x_1,x_2], \Phi y = [y_1,y_2]$とすれば,
						\begin{align}
							\Phi(x + y) &= [x_1 + y_1, x_2 + y_2] = [x_1,x_2] + [y_1,y_2] = \Phi x + \Phi y, \\
							\Phi(\alpha x) &= [\alpha x_1, \alpha x_2] = \alpha [x_1,x_2] = \alpha \Phi x
						\end{align}
						より$\Phi$の線型性が従う.
						また$(x_n)_{n=1}^{\infty} \subset E$が$x_n \rightarrow u \in X$かつ
						$\Phi x_n \rightarrow [u_1,u_2] \in E_1 \times E_2$を満たす場合,
						\begin{align}
							\Norm{u - (u_1 + u_2)}{E} \leq \Norm{u - x_n}{E} + \Norm{\Phi x_n - [u_1,u_2]}{E_1 \times E_2}
							\longrightarrow 0 \quad (n \longrightarrow \infty)
						\end{align}
						が成り立ち$\Phi u = [u_1,u_2]$が従うから$\Phi$は閉作用素である.
				\end{description}
				
			\item[(2)] (1)より逆写像$\Phi^{-1}:E_1 \times E_2 \rightarrow E\ $(線形全単射)が存在し,任意の$[0,0] \neq [x_1,x_2] \in E_1 \times E_2$に対して
				\begin{align}
					\frac{\Norm{\Phi^{-1}[x_1, x_2]}{E}}{\Norm{[x_1, x_2]}{E_1 \times E_2}} 
					= \frac{\Norm{x_1 + x_2}{E}}{\Norm{x_1}{E} + \Norm{x_2}{E}} \leq 1
				\end{align}
				を満たす.
				
			\item[(3)] ノルム空間において一点集合$\{0\}$は閉であるから,直積位相において$E_1 \times \{0\}$及び$\{0\} \times E_2$は閉集合である.
				従って$\Phi$の連続性と$E_1 = \Phi^{-1}(E_1 \times \{0\})$及び$E_2 = \Phi^{-1}(\{0\} \times E_2)$が成り立つことから
				$E_1,E_2$は閉集合となる.
			
			\item[(4)] $E,E_1 \times E_2$はBanach空間でありかつ
				$\Dom{\Phi} = E$が満たされているから,閉グラフ定理より$\Phi$は有界となる.
				(1)(2)と併せれば$\Phi,\Phi^{-1}$は共に連続且つ線型全単射であるから主張が従う.
			
			\item[(5)] $E \rightarrow E$の恒等写像を$I$と表す.また
				\begin{align}
					p_1:E \ni x \longmapsto [x] \in E/E_2,
					\quad p_2:E/E_2 \ni [x] \longmapsto x_1 \in E_1 \quad (x = x_1 + x_2,\ x_1 \in E_1,\ x_2 \in E_2)
				\end{align}
				と定めれば$p_1$は線型連続であり$p_2$は線型同型かつ連続である:
				\begin{description}
					\item[$p_1$について] 任意に$x,y \in E$と$\alpha, \beta \in \C$を取れば
						\begin{align}
							p_1(\alpha x + \beta y) = [\alpha x + \beta y] = [\alpha x] + [\beta y] 
							= \alpha [x] + \beta [y] = \alpha p_1 x + \beta p_1 y
						\end{align}
						が成り立ち$p_1$の線型性が従う.また$x \in E,\ x \neq 0$に対して
						\begin{align}
							\frac{\Norm{p_1 x}{E/E_2}}{\Norm{x}{E}} = \frac{\Norm{[x]}{E/E_2}}{\Norm{x}{E}} \leq \frac{\Norm{x}{E}}{\Norm{x}{E}} = 1
						\end{align}
						となるから$p_1$は連続である.
						
					\item[$p_2$について] $E$から$E_1$への線型準同型を
						\begin{align}
							p:E \ni x \longmapsto x_1 \in E_1 \quad (x = x_1 + x_2,\ x_1 \in E_1,\ x_2 \in E_2)
						\end{align}
						で定める.$\Ran{p} = E_1$かつ$\Ker{p} = E_2$であるから,準同型定理より$p_2$は線型同型となる.また
						$\Dim{E_1} < \infty$であるから$\Dim{E/E_2} = \Dim{E_1} < \infty$となり
						\footnote{
							一般の線形空間$X,Y$に対し,$\Dim{X} = k < \infty$且つ線型同型$f:X \rightarrow Y$が存在するなら
							$\Dim{Y} = k$が成り立つ.実際
							$X$の基底を$x_1,\cdots,x_k$とすれば
							$f(x_1),\cdots,f(x_k)$は$Y$の基底となる.
							$\alpha_1,\cdots,\alpha_k \in \C$に対し
							\begin{align}
								\alpha_1 f(x_1) + \cdots + \alpha_k f(x_k) = 0
							\end{align}
							が成り立っている場合,$f$が線型かつ単射であるから
							\begin{align}
								\alpha_1 x_1 + \cdots + \alpha_k x_k = 0
							\end{align}
							となり$f(x_1),\cdots,f(x_k)$の線型独立性が従う.また任意に$y \in Y$を取れば或る$x \in X$が対応し$f(x) = y$を満たすから,
							\begin{align}
								y = f(x) = f(\alpha_1 x_1 + \cdots + \alpha_k x_k) = \alpha f(x_1) + \cdots + \alpha f(x_k)
							\end{align}
							が成り立ち$Y = \LH{\left\{\, f(x_1),\cdots,f(x_k)\, \right\}}$が従う.
						}
						$p_2$の連続性が従う.
				\end{description}
				$\Phi$は$p_1,p_2$を用いて
				\begin{align}
					\Phi x = [p_2 p_1 x, (I - p_2 p_1) x] \quad (\forall x \in E)
				\end{align}
				と表現できるから
				\begin{align}
					\Norm{\Phi x}{E_1 \times E_2} = \Norm{p_2 p_1 x}{E} + \Norm{(I - p_2 p_1) x}{E} 
				\end{align}
				により$\Phi$の連続性が従い,(1)(2)と併せて主張を得る.
				\QED
		\end{description}
	\end{prf}
	
	\begin{screen}
		\begin{lem}[$T$が単射なら全射]	$X$を複素Banach空間,$I$を$X$上の恒等写像とし,$0 \neq \lambda \in \C$と$A \in \selfCop{X} $に対して
			$T \coloneqq \lambda I - A$とおく.このとき$T$が単射のならば$T$は全射である.
			\label{lem:T_injective_then_surjective}
		\end{lem}
	\end{screen}
	
	\begin{prf}
		背理法で示す.今$T$が単射であり全射ではないとする.このとき
		\begin{align}
			\Ran{T^k} \supsetneq \Ran{T^{k+1}} \quad (k = 1,2,\cdots)
		\end{align}
		が成り立つ.実際或る$k \in \N$で$\Ran{T^k} = \Ran{T^{k+1}}$が成り立つなら,
		任意の$y \in X$に対し或る$x \in X$が存在して
		\begin{align}
			T^{k} y = T^{k+1} x = T^k T x
		\end{align}
		を満たすが,$T^k$が単射であるから$y = T x$が従い$T$が全射でないという仮定に反する.
		\begin{align}
			X_k \coloneqq \Ran{T^k} \quad (k = 1,2,\cdots)
		\end{align}
		と簡単に表せば,定理\ref{thm:Banach_space_compact_operator_kernel_dimension}より
		$X_k$は$X$の閉部分空間であり,定理\ref{thm:closed_subspace_distance}より
		\begin{align}
			\Norm{x_k}{X} = 1,
			\quad \inf{x \in X_k}{\Norm{x_k - x}{X}} > \frac{1}{2}
			\label{eq:lem_T_injective_then_surjective}
		\end{align}
		を満たす$x_k \in X_k \backslash X_{k+1}\ (k=1,2,\cdots)$が存在する.
		$n < m$となる$n,m \in \N$を取れば
		\begin{align}
			T x_n + A x_m = T x_n + \lambda x_m - T x_m \in X_{n+1}
		\end{align}
		が成り立つから,(\refeq{eq:lem_T_injective_then_surjective})より
		\begin{align}
			\Norm{A x_n - A x_m}{X} = \Norm{\lambda x_n - T x_n - A x_m}{X} > \frac{|\lambda|}{2}
		\end{align}
		が従い$\left( A x_k \right)_{k=1}^{\infty}$
		は収束部分列を含み得ないが,これは定理\ref{lem:compact_operator_equiv_cond}に矛盾する.
		\QED
	\end{prf}
\section{コンパクト自己共役作用素のスペクトル分解}
	$H$を複素Hilbert空間とし内積を$\inprod<\cdot,\cdot>$,ノルムを$\Norm{\cdot}{}$と表す.
	
	\begin{screen}
		\begin{dfn}[自己共役作用素]
			$H$上の線型作用素$A$が
			$\closure{\Dom{A} } = H$かつ$A = A^*$を満たすとき,
			$A$を自己共役作用素(self adjoint operator)という.
			自己共役作用素は閉作用素である.
		\end{dfn}
	\end{screen}
	
	\begin{screen}
		\begin{thm}
			$H$上の線型作用素$A$が自己共役なら,
			任意の$u \in \Dom{A} $に対し$\inprod<Au,u>$は実数値である.
		\end{thm}
	\end{screen}
	
	\begin{prf}
		$A = A^*$であるから$u \in \Dom{A} \Leftrightarrow u \in \Dom{A^*} $となり
		\begin{align}
			\inprod<Au,u> = \inprod<u,A^*u> = \inprod<u,Au> = \conj{\inprod<Au,u>}
			\quad \left(\forall u \in \Dom{A} \right)
			\label{eq:self_adjoint_1}
		\end{align}
		が成り立つ.
		\QED
	\end{prf}
	
	\begin{screen}
		\begin{prp}[自己共役作用素のスペクトルは実数]
			$A$を$H$上の自己共役作用素とする.
			\begin{description}
				\item[(1)] $\Spctr{A} \subset \R$であり,かつ任意の$\lambda \in \C \backslash \R$に対し次が満たされる:
					\begin{align}
						\Norm{(\lambda I - A)^{-1}}{\selfBop{H} } \leq \frac{1}{\Im{\lambda}}.
					\end{align}
					ただし$I$は$H$上の恒等写像である.
					
				\item[(2)] $u,v \in H$を$A$の異なる固有値$\lambda,\mu$に対する固有ベクトルとすれば
					$\inprod<u,v> = 0$が成り立つ.
			\end{description}
		\end{prp}
	\end{screen}
	
	\begin{prf}\mbox{}
		\begin{description}
			\item[(1)]
				
			\item[(2)]
				今$Au = \lambda u,\ Av = \mu v$が満たされているから,(\refeq{eq:self_adjoint_1})と同様にして
				\begin{align}
					\lambda \inprod<u,v> = \inprod<Au,v> = \inprod<u,Av> = \conj{\mu} \inprod<u,v>
				\end{align}
				が成り立つ.(1)より$\mu \in \R$であるから$(\lambda - \mu) \inprod<u,v> = 0$
				が得られ,$\lambda \neq \mu$の仮定より$\inprod<u,v> = 0$が従う.
				\QED
		\end{description}
	\end{prf}
	
	
	\begin{screen}
		\begin{dfn}[直交射影]
			$H$を複素Hilbert空間とする.
			線型写像$p:H \rightarrow H$が直交射影であるとは,
			或る$H$の閉部分空間$H_0$が存在し,
			$x \in H$とその直交分解$x = x_1 + x_2\ (x_1 \in H_0, x_2 \in H_0^{\perp})$
			に対し次を満たすことをいう\footnotemark:
			\begin{align}
				p:H \ni x \longmapsto x_1 \in H_0.
			\end{align}
			また$H$上の直交射影全体を$\Oproj{H}$と書く.
		\end{dfn}
	\end{screen}
	
	\footnotetext{
		射影定理より$x \in H$の直交分解は一意に定まるから,
		$p$は写像としてwell-definedである.
	}
	
	\begin{screen}
		\begin{prp}[直交射影の存在]
			$H$を複素Hilbert空間とする.$H$の任意の閉部分空間$L$に対し
			或る$p \in \Oproj{H}$が存在して$p:H \rightarrow L$を満たす.
			特に$\Ran{p} = L$が成り立つ.
		\end{prp}
	\end{screen}
	
	\begin{prf}
		Hilbert空間の射影定理により,任意の$x \in H$は
		$x = x_1 + x_2\ (x_1 \in L,x_2 \in L^\perp)$の形に一意に分解されるから
		\begin{align}
			p:H \ni x \longmapsto x_1 \in L
		\end{align}
		として線型写像を定めれば
		$p \in \Oproj{H}$が従う.特に任意の$u \in L$に対しては$p u = u$が満たされる.
		\QED
	\end{prf}
	
	\begin{screen}
		\begin{prp}[直交射影は冪等・自己共役]
			$H$を複素Hilbert空間とする.任意の$p:H \rightarrow H$に対し次は同値である:
			\begin{description}
				\item[(1)] $p \in \Oproj{H}$.
				\item[(2)] $p$は有界で$p^2 = p$と$p^* = p$を満たす.
			\end{description}
			\label{prp:orthogonal_projection_idempotent_self_adjoint}
		\end{prp}
	\end{screen}
	
	\begin{screen}
		\begin{lem}[一様有界な作用素の極限は有界]
			$X$をノルム空間,$Y$をBanach空間とし,ノルムをそれぞれ$\Norm{\cdot}{X},\Norm{\cdot}{Y}$で表す.
			$A_n \in \Bop{X}{Y} \ (n=1,2,\cdots)$が
			\begin{align}
				\sup{n \in \N}{\Norm{A_n}{\Bop{X}{Y} }} < \infty
			\end{align}
			を満たし,かつ或る$X$で稠密な部分集合$S$が存在して全ての$x \in S$に対し
			$\left( A_n x \right)_{n=1}^{\infty}$が$Y$で収束するとき,
			\begin{align}
				\lim_{n \to \infty} A_n x = A x \quad (\forall x \in X)
			\end{align}
			を満たす$A \in \Bop{X}{Y} $が一意に存在する.
		\end{lem}
	\end{screen}
	
	\begin{prf}
		先ず任意の$x \in X$に対し$\left( A_n x \right)_{n=1}^{\infty}$が$Y$で収束することを示す.
		任意に$\epsilon > 0$を取る.
		\begin{align}
			a \coloneqq \sup{n \in \N}{\Norm{A_n}{\Bop{X}{Y} }}
		\end{align}
		とおき$\Norm{x - z}{X} < \epsilon/a$を満たす$z \in S$を一つ選べば,仮定より或る$N \in \N$が存在して
		\begin{align}
			\Norm{A_n z - A_m z}{Y} < \epsilon \quad (\forall n > m \geq N)
		\end{align}
		が成り立つから,
		\begin{align}
			\Norm{A_n x - A_m x}{Y} \leq a \Norm{x - z}{X} + \Norm{A_n z - A_m z}{Y} + a \Norm{x - z}{X} < 3\epsilon
		\end{align}
		が従う.よって$\left( A_n x \right)_{n=1}^{\infty}$は$Y$のCauchy列であり,$Y$の完備性より収束する.
		\begin{align}
			A x \coloneqq \lim_{n \to \infty} A_n x \quad (\forall x \in X)
		\end{align}
		として$A$を定めれば,任意の$x,y \in X$と$\alpha, \beta \in \C$に対し
		\begin{align}
			&\Norm{A(\alpha x + \beta y) - \alpha A x - \beta A y}{Y} \\
			&\qquad \leq \Norm{A(\alpha x + \beta y) - A_n(\alpha x + \beta y)}{Y}
				+ |\alpha| \Norm{A x - A_n x}{Y} + |\beta| \Norm{A x - A_n x}{Y}
			\longrightarrow 0 \quad (n \longrightarrow \infty)
		\end{align}
		が満たされるから$A$は線形作用素であり,かつ任意の$x \in X$に対して
		\begin{align}
			\Norm{A x}{Y} \leq \Norm{A x - A_n x}{Y} + \Norm{A_n x}{Y}
			\leq \Norm{A x - A_n x}{Y} + \Norm{A_n}{\Bop{X}{Y} } \Norm{x}{X}
		\end{align}
		が成り立ち,右辺で下極限を取れば
		\begin{align}
			\Norm{A x}{Y} \leq \liminf_{n \to \infty} \Norm{A_n}{\Bop{X}{Y} } \Norm{x}{X}
		\end{align}
		が従う.
		\begin{align}
			\Norm{A}{\Bop{X}{Y} } \leq \liminf_{n \to \infty} \Norm{A_n}{\Bop{X}{Y} } = \sup{n \in \N}{\inf{\nu \geq n}{\Norm{A_\nu}{\Bop{X}{Y} }}} \leq \sup{n \in \N}{\Norm{A_n}{\Bop{X}{Y} }}
		\end{align}
		より$A \in \Bop{X}{Y} $を得る.
		\QED
	\end{prf}
	
	\begin{screen}
		\begin{prp}[直交射影の積・和の性質]
			$H$を複素Hilbert空間とする.
			\begin{description}
				\item[(1)] $p,q \in \Oproj{H}$に対し次が成り立つ:
					\begin{align}
						\Ran{p} \perp \Ran{q}
						\quad \Leftrightarrow \quad  pq = 0
						\quad \Leftrightarrow \quad  qp = 0.
					\end{align}
				
				\item[(2)] 
					$p_1,\cdots,p_n \in \Oproj{H}$が
					$p_i \neq p_j\ (i \neq j)$を満たすなら,
					$p \coloneqq \sum_{i=1}^{n} p_i$とおいて次が成り立つ:
					\begin{align}
						p \in \Oproj{H}
						\quad \Leftrightarrow \quad p_i p_j = \delta_{ij} p_j \quad (i,j = 1,\cdots,n).
					\end{align}
					ただし$\delta_{ij}$はKroneckerのデルタである.
				
				\item[(3)] 
					$p_1,p_2,\cdots \in \Oproj{H}$が
					$p_i p_j = \delta_{ij} p_j \ (\forall i,j \in \N)$を満たすとして
					\begin{align}
						H_0 \coloneqq \closure{\LH{\bigcup_{i=1}^{\infty}\Ran{p_i}}}
					\end{align}
					とおく.$p \in \Oproj{H}$が$\Ran{p} = H_0$であるとき次が成り立つ:
					\begin{align}
						px = \sum_{i=1}^{\infty} p_i x \quad (\forall x \in H).
					\end{align}
			\end{description}
			\label{prp:orthogonal_projection_product_sum}
		\end{prp}
	\end{screen}
\chapter{自己共役作用素のスペクトル分解}
	\section{複素測度}

	\begin{screen}
		\begin{dfn}[複素測度]
			$(X,\mathcal{M})$を可測空間とする.
			$\lambda: \mathcal{M} \rightarrow \C$が
			任意の互いに素な集合列$E_i \in \mathcal{M}\ (i=1,2,\cdots)$と
			$E \coloneqq \sum_{i=1}^{\infty} E_i$に対して
			\begin{align}
				\lambda(E) = \sum_{i=1}^{\infty} \lambda(E_i)
				\label{eq:dfn_complex_measure}
			\end{align}
			を満たすとき,$\lambda$を複素測度(complex measure)という.
		\end{dfn}
	\end{screen}
	
	$\lambda$は複素数値であるから任意の$E \in \mathcal{M}$に対して
	$|\lambda(E)| < \infty$を満たす.
	従って(\refeq{eq:dfn_complex_measure})において右辺の級数は収束していなくてはならない.
	$\sigma:\N \rightarrow \N$を任意の並び替え
	\footnote{
		$\sigma$は$\N$から$\N$への全単射である.
	}とすれば
	\begin{align}
		E = \sum_{i=1}^{\infty}E_{\sigma(i)}
	\end{align}
	が成り立つから
	\begin{align}
		\lambda(E) = \sum_{i=1}^{\infty} \lambda(E_{\sigma(i)})
	\end{align}
	を得る.従って複素数列$\left( \lambda(E_i) \right)_{i=1}^{\infty}$
	は無条件に$\lambda(E)$に収束し,Riemannの級数定理より
	$\left( \lambda(E_i) \right)_{i=1}^{\infty}$は絶対収束する:
	\begin{align}
		\sum_{i=1}^{\infty} |\lambda(E_i)| < \infty.
	\end{align}
	今,$\lambda$を支配するような或る$\mathcal{M}$上の測度$\mu$,つまり
	\begin{align}
		|\lambda(E)| \leq \mu(E) \quad (\forall E \in \mathcal{M})
		\label{radon_nikodym_1}
	\end{align}
	を満たすものを,できるだけ小さいものとして取ろうと考える
	\footnote{
		つまり(\refeq{radon_nikodym_1})を満たす$\mu$のうちから,
		同様に(\refeq{radon_nikodym_1})を満たす任意の測度$\mu'$に対し
		\begin{align}
			\mu(E) \leq \mu'(E) \quad (\forall E \in \mathcal{M})
		\end{align}
		を満たすものを選べるかどうかを考える.
	}.
	このような$\mu$は次を満たすことになる:
	\begin{align}
		\sum_{i=1}^{\infty} |\lambda(E_i)| \leq \sum_{i=1}^{\infty} \mu(E_i) = \mu(E).
	\end{align}
	ゆえに
	\begin{align}
		\mu(E) \geq \sup{}{\sum_{i=1}^{\infty} |\lambda(E_i)|} 
		\label{radon_nikodym_2}
	\end{align}
	でなくてはならず(上限は$E$のあらゆる分割$E = \sum_{i}E_i$に対して取るものである),
	ここで$\mathcal{M}$上の関数を
	\begin{align}
		|\lambda|(E) \coloneqq \sup{}{\sum_{i=1}^{\infty} |\lambda(E_i)|} \quad (\forall E \in \mathcal{M})
		\label{radon_nikodym_3}
	\end{align}
	として定義してみれば,$E$自体が$E$の一つの分割であるから(\refeq{radon_nikodym_3})より$|\lambda|$は$\lambda$を支配し,
	更に,後述することであるが,$|\lambda|$は$\mathcal{M}$上の測度でもあり(\refeq{radon_nikodym_2})と併せて当座の問題の解となる.
	
	\begin{itembox}[l]{}
		\begin{dfn}[総変動・総変動測度]
			可測空間$(X,\mathcal{M})$上の複素測度$\lambda$に対し,上で定めた
			測度$|\lambda|:\mathcal{M} \longrightarrow [0,\infty)$を
			$\lambda$の総変動測度(total variation measure)といい,$|\lambda|(X)$を
			$\lambda$の総変動(total variation)という.
			特に$\lambda$が正値有限測度である場合は$\lambda = |\lambda|$が成り立つ.\footnotemark
		\end{dfn}
	\end{itembox}
	\footnotetext{
		複素測度の虚部が0であるものとして考えれば$0 \leq \lambda(E) \leq \lambda(X) < \infty\ (\forall E \in \mathcal{M})$が成り立つ.
		また実際任意の$E \in \mathcal{M}$とその分割$(E_i)_{i=1}^{\infty}$に対して
		\begin{align}
			|\lambda|(E) = \sup{}{\sum_{i=1}^{\infty} |\lambda(E_i)|} = \sup{}{\sum_{i=1}^{\infty} \lambda(E_i)} = \sup{}{\lambda(E)} = \lambda(E)
		\end{align}
		が成り立つ.
	}
	
	以降で$|\lambda|$の性質
	\begin{description}
		\item[(1)] $|\lambda|$は測度である.
		\item[(2)] $|\lambda|(X) < \infty$が成り立つ.
	\end{description}
	を証明する.特に(2)により任意の$E \in \mathcal{M}$に対し
	\begin{align}
		|\lambda(E)| \leq |\lambda|(E) \leq |\lambda|(X) < \infty
	\end{align}
	が従うから,複素測度は有界であると判明する.

	\begin{itembox}[l]{}
		\begin{thm}[$|\lambda|$は測度となる]
			可測空間$(X,\mathcal{M})$上の複素測度$\lambda$に対し(\refeq{radon_nikodym_3})で定義する$|\lambda|$は
			$(X,\mathcal{M})$において測度となる.
		\end{thm}
	\end{itembox}
	
	\begin{prf}
		(\refeq{radon_nikodym_3})により$|\lambda|$は正値であるから,ここで示すことは$|\lambda|$が完全加法的であるということである.
		任意に$\epsilon > 0$とどの二つも互いに素な集合列$E_i \in \mathcal{M}\ (i=1,2,\cdots)$を取る.
		示すことは$E \coloneqq \sum_{i=1}^{\infty} E_i$に対して
		\begin{align}
			|\lambda|(E) = \sum_{i=1}^{\infty} |\lambda|(E_i)
		\end{align}
		が成り立つことである.(\refeq{radon_nikodym_3})により$E_i$の分割$(A_{ij})_{j=1}^{\infty} \subset \mathcal{M}$を
		\begin{align}
			|\lambda|(E_i) \geq \sum_{j=1}^{\infty} |\lambda(A_{ij})| > |\lambda|(E_i) - \epsilon/2^i
		\end{align}
		となるように取ることができる.また$E = \sum_{i,j=1}^{\infty} A_{ij}$でもあるから
		\begin{align}
			|\lambda|(E) \geq \sum_{i,j=1}^{\infty} |\lambda(A_{ij})| \geq \sum_{i=1}^{\infty}\sum_{j=1}^{\infty} |\lambda(A_{ij})| > \sum_{i=1}^{\infty} |\lambda|(E_i) - \epsilon
		\end{align}
		が成り立つ.$\epsilon > 0$は任意であるから
		\begin{align}
			|\lambda|(E) \geq \sum_{j=1}^{\infty} |\lambda|(E_j)
		\end{align}
		が従う.逆向きの不等号について,$E$の任意の分割$(A_j)_{j=1}^{\infty} \subset \mathcal{M}$に対し
		\begin{align}
			\sum_{j=1}^{\infty} |\lambda(A_j)| 
			= \sum_{j=1}^{\infty} \left| \sum_{i=1}^{\infty} \lambda(A_j \cap E_i) \right|
			\leq \sum_{j=1}^{\infty} \sum_{i=1}^{\infty} |\lambda(A_j \cap E_i)|
			\leq \sum_{i=1}^{\infty} |\lambda|(E_i)
			\footnotemark
		\end{align}
		\footnotetext{
			正項級数は和の順序に依らないから
			\begin{align}
				\sum_{j=1}^{\infty} \sum_{i=1}^{\infty} |\lambda(A_j \cap E_i)|
				= \sum_{i=1}^{\infty} \sum_{j=1}^{\infty} |\lambda(A_j \cap E_i)|
			\end{align}
			が成り立つ.これと(\refeq{radon_nikodym_3})を併せれば最後の不等号が従う.
		}
		が成り立つから,左辺の上限を取って
		\begin{align}
			|\lambda|(E) \leq \sum_{i=1} |\lambda|(E_i)
		\end{align}
		を得る.
		\QED
	\end{prf}
	
	\begin{itembox}[l]{}	
		\begin{thm}[総変動測度は有界]
			可測空間$(X,\mathcal{M})$上の複素測度$\lambda$の総変動測度$|\lambda|$について次が成り立つ:
			\begin{align}
				|\lambda|(X) < \infty.
			\end{align}
			\label{thm:total_variation_measure_bounded}
		\end{thm}
	\end{itembox}
	
	先ずは次の補題を示す.
	
	\begin{itembox}[l]{}
		\begin{lem}
			$z_1,\cdots,z_N$を複素数とする.これらの添数集合の或る部分$S \subset \{1,\cdots,N\}$を抜き取れば次が成り立つ:
			\begin{align}
				\left| \sum_{k \in S} z_k \right| \geq \frac{1}{2\pi} \sum_{k=1}^{N} |z_k|.
			\end{align}
			\label{lem:total_variation_measure_bounded}
		\end{lem}
	\end{itembox}
	
	\begin{prf}[補題]
		$z_k = |z_k|\exp{i \alpha_k}\ (-\pi \leq \alpha_k < \pi,\ k=1,\cdots,N)$となるように$\alpha_1,\cdots,\alpha_N$を取る.
		ここで$i$は虚数単位である.また$-\pi \leq \theta \leq \pi$に対し
		\begin{align}
			S(\theta) \coloneqq \Set{k \in \{1,\cdots,N\}}{\cos{(\alpha_k - \theta)}{} > 0}
		\end{align}
		とおく.このとき
		\begin{align}
			\left| \sum_{k \in S(\theta)} z_k \right| &= |\exp{-i\theta}|\left| \sum_{k \in S(\theta)} z_k \right| = \left| \sum_{k \in S(\theta)} |z_k|\exp{i(\alpha_k - \theta)} \right| \\
			&\geq \Re{\sum_{k \in S(\theta)} |z_k|\exp{i(\alpha_k - \theta)}} = \sum_{k \in S(\theta)} |z_k| \cos{(\alpha_k - \theta)}{} = \sum_{k=1}^{N} |z_k| \cos{(\alpha_k - \theta)}{+}
			\footnotemark
		\end{align}
		\footnotetext{$\cos{x}{+} = 0 \vee \cos{x}{}\ (x \in \R)$である.}
		が成り立ち,最右辺は$\theta$に関して連続となるから$[-\pi,\pi]$上で式を最大にする$\theta_0$が存在する.$S \coloneqq S(\theta_0)$とおき,$\theta_0$と任意の$\theta \in [-\pi,\pi]$に対して
		\begin{align}
			\left| \sum_{k \in S} z_k \right| \geq \sum_{k=1}^{N} |z_k| \cos{(\alpha_k - \theta_0)}{+} \geq \sum_{k=1}^{N} |z_k| \cos{(\alpha_k - \theta)}{+}
		\end{align}
		が成り立つから,左辺右辺を積分して
		\begin{align}
			\left| \sum_{k \in S} z_k \right| \geq \sum_{k=1}^{N} |z_k| \frac{1}{2\pi} \int_{[-\pi,\pi]} \cos{(\alpha_k - \theta)}{+}\ d\theta
			= \frac{1}{2\pi} \sum_{k=1}^{N} |z_k|
		\end{align}
		が成り立つ\footnote{
			最後の積分について,実際三角関数の周期性を使えば任意の$\alpha \in \R$に対して
			\begin{align}
				\int_{[-\pi,\pi]} \cos{(\alpha - \theta)}{+}\ d\theta = \int_{[\alpha-\pi,\alpha+\pi]} \cos{\theta}{+}\ d\theta = \int_{[-\pi,\pi]} \cos{\theta}{+}\ d\theta = 1
			\end{align}
			が成り立つ.
		}.
		\QED
	\end{prf}
	
	\begin{prf}[定理\ref{thm:total_variation_measure_bounded}]\mbox{}
		\begin{description}
		\item[第一段]
			或る$E \in \mathcal{M}$に対し$|\lambda|(E) = \infty$が成り立っていると仮定する.
			$t \coloneqq 2\pi(1 + |\lambda(E)|)$とおけば({\scriptsize 複素測度であるから$|\lambda(E)| < \infty$})
			$|\lambda|(E) > t$となるから,(\refeq{radon_nikodym_3})より$E$の分割$(E_i)_{i=1}^{\infty}$を
			\begin{align}
				\sum_{i=1}^{\infty} |\lambda(E_i)| > t
			\end{align}
			となるように取ることができる.従って或る$N \in \N$を取れば
			\begin{align}
				\sum_{i=1}^{N} |\lambda(E_i)| > t
			\end{align}
			が成り立つ.$z_i \coloneqq \lambda(E_i)\ (i=1,\cdots,N)$として補題\ref{lem:total_variation_measure_bounded}を使えば,
			或る$S \subset \{1,\cdots,N\}$に対し
			\begin{align}
				\left| \sum_{k \in S} \lambda(E_k) \right| \geq \frac{1}{2\pi} \sum_{k=1}^{N} |\lambda(E_k)| > \frac{t}{2\pi} > 1
			\end{align}
			となる.$A \coloneqq \sum_{k \in S} E_k$とおいて$B \coloneqq E - A$とすれば
			\begin{align}
				|\lambda(B)| = |\lambda(E)-\lambda(A)| \geq |\lambda(A)| - |\lambda(E)| > \frac{t}{2\pi} - |\lambda(E)| = 1
			\end{align}
			が成り立つから,つまり$|\lambda|(E) = \infty$の場合,$E$の直和分割$A,B$で
			\begin{align}	
				|\lambda(A)| > 1, \quad |\lambda(B)| > 1
			\end{align}
			を満たすものが取れると示された.そして$|\lambda|$の加法性から
			\begin{align}
				|\lambda|(E) = |\lambda|(A) + |\lambda|(B)
			\end{align}
			も成り立つから,この場合右辺の少なくとも一方は$\infty$となる.
		
		\item[第二段]
			背理法により定理の主張することを証明する.今$|\lambda|(X) = \infty$と仮定すると,前段の結果より$X$の或る直和分割$A_1,B_1$で
			\begin{align}
				|\lambda|(B_1) = \infty, \quad |\lambda(A_1)| > 1, \quad |\lambda(B_1)| > 1
			\end{align}
			を満たすものが取れる.$B_1$についてもその直和分割$A_2,B_2$で
			\begin{align}
				|\lambda|(B_2) = \infty, \quad |\lambda(A_2)| > 1, \quad |\lambda(B_2)| > 1
			\end{align}
			を満たすものが取れる.この操作を繰り返せば,どの二つも互いに素な集合列$(A_j)_{j=1}^{\infty}$で$|\lambda(A_j)| > 1\ (j=1,2,\cdots)$
			を満たすものを構成できる.$A \coloneqq \sum_{j=1}^{\infty}$について,$|\lambda(A)| < \infty$でなくてはならないから,
			Riemannの級数定理より
			\begin{align}
				\lambda(A) = \sum_{j=1}^{\infty} \lambda(A_j)
			\end{align}
			の右辺は絶対収束する.従って$0 < \epsilon < 1$に対し或る$N \in \N$が存在して$n > N$なら$|\lambda(A_n)| < \epsilon$
			が成り立つはずであるが,これは$|\lambda(A_n)| > 1$であることに矛盾する.
			背理法により$|\lambda|(X) < \infty$であることが示された.
		\end{description}
		\QED
	\end{prf}
	
	\begin{itembox}[l]{}
		\begin{dfn}[複素測度の空間・ノルムの定義]
			可測空間$(X,\mathcal{M})$上の複素測度の全体を$\CM$と表す.$\lambda,\mu \in \CM,\ c \in \C,\ E \in \mathcal{M}$に対し
			\begin{align}
				&(\lambda + \mu)(E) \coloneqq \lambda(E) + \mu(E), \\
				&(c\lambda)(E) \coloneqq c\lambda(E)
				\label{complex_measure_linear}
			\end{align}
			を線型演算として$\CM$は線形空間となり,特に定理\ref{thm:total_variation_measure_bounded}により
			$\lambda \in \CM$に対して$|\lambda| \in \CM$が成り立つ.
			また$\Norm{\cdot}{}:\CM \rightarrow \R$を
			\begin{align}
				\Norm{\lambda}{} \coloneqq |\lambda|(X) \quad (\lambda \in \CM)
			\end{align}
			と定義すればこれは$\CM$においてノルムとなる.
		\end{dfn}
	\end{itembox}
	
	上で定義した$\Norm{\cdot}{}$がノルムとなることを証明する.総変動の正値性からノルムの正値性が従うから,以下示すのは
	同次性と三角不等式である.
	\begin{description}
		\item[同次性]
			総変動測度の定義(\refeq{radon_nikodym_3})とスカラ倍の定義(\refeq{complex_measure_linear})より,任意の$\lambda \in \CM$と$c \in \C$に対し
			\begin{align}
				\Norm{c\lambda}{} = \sup{}{\sum_{i}|(c\lambda)(E_i)|} = \sup{}{\sum_{i}|c\lambda(E_i)|} = |c|\sup{}{\sum_{i}|\lambda(E_i)|} = |c|\Norm{\lambda}{}
			\end{align}
			が成り立つ.
			
		\item[三角不等式]
			任意の$\lambda,\mu \in \CM$に対し
			\begin{align}
				\Norm{\lambda + \mu}{} = |\lambda + \mu|(X) = \sup{}{\sum_{i} |(\lambda + \mu)(E_i)|} = \sup{}{\sum_{i} |\lambda(E_i) + \mu(E_i)|}
			\end{align}
			となるが,ここで
			\begin{align}
				\sum_{i} |\lambda(E_i) + \mu(E_i)| \leq \sum_{i} |\lambda(E_i)| + \sum_{i} |\mu(E_i)| \leq \Norm{\lambda}{} + \Norm{\mu}{}
			\end{align}
			が成り立つから
			\begin{align}
				\Norm{\lambda + \mu}{} = \sup{}{\sum_{i} |\lambda(E_i) + \mu(E_i)|} \leq \Norm{\lambda}{} + \Norm{\mu}{}
			\end{align}
			が従う.
	\end{description}
	
	\begin{itembox}[l]{}
		\begin{dfn}[正変動と負変動・Jordanの分解]
			可測空間$(X,\mathcal{M})$上の複素測度の全体を$\CM$とし,
			実数値の$\mu \in \CM$を取る({\scriptsize このような$\mu$を符号付き測度(signed measure)という}).
			\begin{align}
				\mu^+ \coloneqq \frac{1}{2}(|\mu| + \mu) , \quad \mu^- \coloneqq \frac{1}{2}(|\mu| - \mu)
			\end{align}
			とおけば$\mu^+,\mu^-$はどちらも正値有限測度となる\footnotemark
			.$\mu^+$を$\mu$の正変動(positive variation)といい$\mu^-$を$\mu$の負変動(negative variation)という.
			また
			\begin{align}
				\mu = \mu^+ - \mu^-, \quad |\mu| = \mu^+ + \mu^-
			\end{align}
			が成り立ち,ここで示した符号付き測度の正変動と負変動による表現をJordanの分解という.
		\end{dfn}
	\end{itembox}
	
	\footnotetext{
		$\mathcal{M}$上で$|\mu|(E) \geq |\mu(E)|$であることと定理\ref{thm:total_variation_measure_bounded}による.
	}
	
	\begin{itembox}[l]{}
		\begin{dfn}[絶対連続・特異]
			$(X,\mathcal{M})$を可測空間,
			$\mu$を$\mathcal{M}$上の正値測度\footnotemark
			,$\lambda,\lambda_1,\lambda_2$を$\mathcal{M}$上の任意の測度({\scriptsize 正値測度或は複素測度})とする.
			\begin{itemize}
				\item $\lambda$が$\mu$に関して絶対連続である(absolutely continuous)ということを
					\begin{align}
						\lambda \ll \mu
					\end{align}
					と書き,その意味は,「$\mu(E)=0$となる全ての$E \in \mathcal{M}$について$\lambda(E)=0$」である.
				
				\item 或る$A \in \mathcal{M}$があって$\lambda(E) = \lambda(A \cap E)\ (\forall E \in \mathcal{M})$
					が成り立っているとき,$\lambda$は$A$に集中している(concentrated on A)という.
					$\lambda_1,\lambda_2$に対し或る$A_1,A_2 \in \mathcal{M}$があって,
					$\lambda_1$が$A_1$に集中,$\lambda_2$が$A_2$に集中しかつ
					$A_1 \cap A_2 = \emptyset$を満たしているとき,これを互いに特異である(mutually singular)といい
					\begin{align}
						\lambda_1 \perp \lambda_2
					\end{align}
					と書く.
			\end{itemize}
		\end{dfn}
	\end{itembox}
	
	\footnotetext{
		正値測度という場合は$\infty$も取りうる.従って正値測度は複素測度の範疇にはない.$\mu$として例えば$k$次元Lebesgue測度を想定している.
	}
	
	\begin{itembox}[l]{}
		\begin{prp}[絶対連続性と特異性に関する性質]
			$(X,\mathcal{M})$を可測空間,$\mu$を$\mathcal{M}$上の正値測度,
			$\lambda,\lambda_1,\lambda_2$を$\mathcal{M}$上の複素測度とする.このとき以下に羅列する事柄が成り立つ.
			\begin{description}
				\item[(1)] $\lambda$が$A \in \mathcal{M}$に集中しているなら$|\lambda|$も$A$に集中している.
				\item[(2)] $\lambda_1 \perp \lambda_2$ならば$|\lambda_1| \perp |\lambda_2|$.
				\item[(3)] $\lambda_1 \perp \mu$かつ$\lambda_2 \perp \mu$ならば$\lambda_1 + \lambda_2 \perp \mu$.
				\item[(4)] $\lambda_1 \ll \mu$かつ$\lambda_2 \ll \mu$ならば$\lambda_1 + \lambda_2 \ll \mu$.
				\item[(5)] $\lambda \ll \mu$ならば$|\lambda| \ll \mu$.
				\item[(6)] $\lambda_1 \ll \mu$かつ$\lambda_2 \perp \mu$ならば$\lambda_1 \perp \lambda_2$.
				\item[(7)] $\lambda \ll \mu$かつ$\lambda \perp \mu$ならば$\lambda = 0$.
			\end{description}
			\label{prp:absolute_continuous_singular}
		\end{prp}
	\end{itembox}
	\section{複素測度に関する積分}
	\begin{screen}
		\begin{thm}[複素数値可測関数]
			$(X,\mathcal{M})$を可測空間とする.$f:X \rightarrow \C$について次が成り立つ:
			\begin{description}
				\item[(1)] $f$が可測$\mathcal{M}/\borel{\C}$であることと
					$f$の実部虚部がそれぞれ可測$\mathcal{M}/\borel{\R}$であることは同値である.
					
				\item[(2)] $\mathcal{M}/\borel{\C}$-可測関数列$(f_n)_{n=1}^{\infty}$が$f$に各点収束するなら
					$f$もまた可測$\mathcal{M}/\borel{\C}$となる.
			\end{description}
		\end{thm}
	\end{screen}
	
	\begin{screen}
		\begin{dfn}[積分の定義]
			$(X,\mathcal{M})$を可測空間とし,$\mu$を$(X,\mathcal{M})$上の複素測度とする.
			$\mu$の総変動測度$|\mu|$に関して可積分となる関数$f:X \rightarrow \C$について
			,$f$の$\mu$に関する積分を次で定める:
			\begin{description}
				\item[$f$が可測単関数の場合]
					有限個の複素数$\alpha_1,\cdots,\alpha_k$と集合$A_1,\cdots,A_k \in \mathcal{M}$によって
					\begin{align}
						f = \sum_{i=1}^{k} \alpha_i \defunc_{A_i}
						\label{eq:expression_simple_function}
					\end{align}
					と表されるとき\footnotemark
					,$f$の$\mu$に関する積分を
					\begin{align}
						\int_X f(x)\ \mu(dx) \coloneqq \sum_{i=1}^{k} \alpha_i \mu(A_i)
						\label{eq:complex_measures_integration_simple_function}
					\end{align}
					で定める.
					
				\item[$f$が一般の可測関数の場合]
					\begin{align}
						\int_X f_n(x)\ |\mu|(dx) \longrightarrow \int_X f(x)\ |\mu|(dx)
						\label{eq:integration_simple_function_approximation}
					\end{align}
					を満たす$f$の可測単関数近似列$(f_n)_{n=1}^{\infty}$を取り,
					$f$の$\mu$に関する積分を
					\begin{align}
						\int_X f(x)\ \mu(dx) \coloneqq \lim_{n \to \infty} \int_X f_n(x)\ \mu(dx)
						\label{eq:complex_measures_integration_measurable_function}
					\end{align}
					で定める.
					
			\end{description}
			\label{dfn:complex_measures_integration}
		\end{dfn}
	\end{screen}
	
	\footnotetext{
		$A_1,\cdots,A_k$は互いに素であり$X = \sum_{i=1}^{k} A_i$を満たす.
	}
	
	\begin{screen}
		\begin{thm}[積分の定義はwell-defined]
			定義\ref{dfn:complex_measures_integration}において,
			(\refeq{eq:complex_measures_integration_simple_function})
			は(\refeq{eq:expression_simple_function})の表示の仕方に依らずに定まり,
			(\refeq{eq:complex_measures_integration_measurable_function})
			も(\refeq{eq:integration_simple_function_approximation})を満たす単関数近似列の選び方に
			依らずに定まる.更に任意の$f \in MF$に対して次が成り立つ:
			\begin{align}
				\left| \int_X f(x)\ \mu(dx) \right| \leq \int_X |f(x)|\ |\mu|(dx).
				\label{eq:thm_complex_measure_integration_well_defined_2}
			\end{align}
			\label{thm:complex_measure_integration_well_defined}
		\end{thm}
	\end{screen}
	
	\begin{prf}\mbox{}
		\begin{description}
			\item[$f$が可測単関数の場合]
				$f$が(\refeq{eq:expression_simple_function})の表示とは別に
				\begin{align}
					f = \sum_{j=1}^{m} \beta_j \defunc_{B_j}
					\quad (\beta_j \in \C,\ B_j \in \mathcal{M},\ \mbox{$X=\sum_{j=1}^{m} B_j$})
				\end{align}
				と表現できるとしても
				\begin{align}
					\sum_{i=1}^{k} \alpha_i \mu(A_i)
					= \sum_{i=1}^{k} \sum_{j=1}^{m} \alpha_i \mu(A_i \cap B_j)
					= \sum_{j=1}^{m} \sum_{i=1}^{k} \beta_j \mu(A_i \cap B_j)
					= \sum_{j=1}^{m} \beta_j \mu(B_j)
				\end{align}
				が成り立つ.また(\refeq{radon_nikodym_3})より
				\begin{align}
					\left| \int_X f(x)\ \mu(dx) \right|
					= \left| \sum_{i=1}^{k} \alpha_i \mu(A_i) \right|
					\leq \sum_{i=1}^{k} \left| \alpha_i \right| |\mu|(A_i)
					= \int_X |f(x)|\ |\mu|(dx)
					\label{eq:thm_complex_measure_integration_well_defined}
				\end{align}
				も成り立つ.
				
			\item[$f$が一般の可測関数の場合]
				(\refeq{eq:complex_measures_integration_measurable_function})
				は有限確定している.実際
				(\refeq{eq:integration_simple_function_approximation})を
				満たす単関数近似列$(f_n)_{n=1}^{\infty}$に対して
				(\refeq{eq:thm_complex_measure_integration_well_defined})より
				\begin{align}
					\left| \int_X f_n(x)\ \mu(dx) - \int_X f_m(x)\ \mu(dx) \right|
					\leq \int_X \left| f_n(x) - f_m(x) \right|\ |\mu|(dx)
					\quad (\forall n,m \in \N)
				\end{align}
				が成り立つから,$\left( \int_X f_n(x)\ \mu(dx) \right)_{n=1}^{\infty}$
				は$\C$においてCauchy列をなし極限が存在する.
				$(f_n)_{n=1}^{\infty}$とは別に(\refeq{eq:integration_simple_function_approximation})
				を満たす$f$の単関数近似列$(g_n)_{n=1}^{\infty}$が存在しても
				\begin{align}
					&\left| \int_X f_n(x)\ \mu(dx) - \int_X g_m(x)\ \mu(dx) \right|
					\leq \int_X \left| f_n(x) - g_m(x) \right|\ |\mu|(dx) \\
					&\qquad \leq \int_X \left| f_n(x) - f(x) \right|\ |\mu|(dx)
						+ \int_X \left| f(x) - g_m(x) \right|\ |\mu|(dx)
					\longrightarrow 0 \quad (n,m \longrightarrow \infty)
				\end{align}
				が成り立つから,
				\begin{align}
					\alpha \coloneqq \lim_{n \to \infty} \int_X f_n(x)\ \mu(dx),
					\quad \beta \coloneqq \lim_{n \to \infty} \int_X g_n(x)\ \mu(dx)
				\end{align}
				とおけば
				\begin{align}
					|\alpha - \beta|
					&\leq \left| \alpha - \int_X f_n(x)\ \mu(dx) \right|
						+ \left| \int_X f_n(x)\ \mu(dx) - \int_X g_m(x)\ \mu(dx) \right|
						+ \left| \int_X g_m(x)\ \mu(dx) - \beta \right| \\
					&\longrightarrow 0 \quad (n,m \longrightarrow \infty)
				\end{align}
				が従い$\alpha = \beta$を得る.また(\refeq{eq:thm_complex_measure_integration_well_defined})より
				\begin{align}
					\left| \int_X f_n(x)\ \mu(dx) \right| \leq \int_X |f_n(x)|\ |\mu|(dx) \quad (n = 1,2,\cdots)
				\end{align}
				が満たされているから,両辺で$n \longrightarrow \infty$として(\refeq{eq:thm_complex_measure_integration_well_defined_2})を得る.
				\QED
		\end{description}
	\end{prf}
	
	定義\ref{dfn:complex_measures_integration}において,
	(\refeq{eq:complex_measures_integration_measurable_function})は
	(\refeq{eq:complex_measures_integration_simple_function})の拡張となっている.
	実際$f$が可測単関数の場合,(\refeq{eq:integration_simple_function_approximation})を満たす単関数近似列
	として$f$自身を選べばよい.定理\ref{thm:complex_measure_integration_well_defined}より
	(\refeq{eq:complex_measures_integration_measurable_function})による$f$の積分は一意に確定し
	(\refeq{eq:complex_measures_integration_simple_function})の左辺に一致する.
	
	\begin{screen}
		\begin{thm}[積分の線型性]
			定義\ref{dfn:complex_measures_integration}で定めた積分について,
			任意の$f,g \in \mathscr{L}^1(X,\mathcal{M},|\mu|)$と
			$\alpha,\beta \in \C$に対し
			\begin{align}
				\int_X \alpha f(x) + \beta g(x)\ \mu(dx)
				= \alpha \int_X f(x)\ \mu(dx) + \beta \int_X g(x)\ \mu(dx)
			\end{align}
			が成り立つ.
			\label{thm:complex_measure_integral_linearity}
		\end{thm}
	\end{screen}
	
	\begin{prf}\mbox{}
		\begin{description}
			\item[第一段]
				$f,g$が可測単関数の場合,(\refeq{eq:complex_measures_integration_simple_function})で定める積分が線型性を持つことを示す.
				$u_1,\cdots,u_k,v_1,\cdots,v_r \in \C$と$A_1,\cdots,A_k,B_1,\cdots,B_r \in \mathcal{M}\ (X = \sum_{i=1}^{k} A_i = \sum_{j=1}^{r} B_j)$によって
				\begin{align}
					f = \sum_{i=1}^{k} u_i \defunc_{A_i},
					\quad g = \sum_{j=1}^{r} v_j \defunc_{B_j}
				\end{align}
				と表示されているとき,
				\begin{align}
					\alpha f + \beta g = \sum_{i=1}^{k} \sum_{j=1}^{r} (\alpha u_i + \beta v_j) \defunc_{A_i \cap B_j}
				\end{align}
				と表現できるから
				\begin{align}
					&\int_X \alpha f(x) + \beta g(x)\ \mu(dx) 
					= \sum_{i=1}^{k} \sum_{j=1}^{r} (\alpha u_i + \beta v_j) \mu(A_i \cap B_j) \\
					&\qquad = \alpha \sum_{i=1}^{k} u_i \mu(A_i) + \beta \sum_{j=1}^{r} v_j \mu(B_j)
					= \alpha \int_X f(x)\ \mu(dx) + \beta \int_X g(x)\ \mu(dx)
				\end{align}
				が成り立つ.
			\item[第二段]
				$f,g$を一般の可測関数とし,$f,g$それぞれについて(\refeq{eq:integration_simple_function_approximation})を満たす
				単関数近似列$(f_n)_{n=1}^{\infty},(g_n)_{n=1}^{\infty}$を一つ選ぶ.
				\begin{align}
					&\int_X \left|(\alpha f_n(x) + \beta g_n(x)) - (\alpha f(x) + \beta g(x)) \right|\ |\mu|(dx) \\
					&\qquad \leq |\alpha| \int_X \left| f_n(x) - f(x) \right|\ |\mu|(dx)
						+ |\beta| \int_X \left| g_n(x) - g(x) \right|\ |\mu|(dx)
					\longrightarrow 0 \quad (n \longrightarrow \infty)
				\end{align}
				が成り立つから$\alpha f + \beta g$の$\mu$に関する積分は
				\begin{align}
					\int_X \alpha f(x) + \beta g(x)\ \mu(dx) \coloneqq \lim_{n \to \infty} \int_X \alpha f_n(x) + \beta g_n(x)\ \mu(dx)
				\end{align}
				で定義され,前段の結果より
				\begin{align}
					&\left| \int_X \alpha f(x) + \beta g(x)\ \mu(dx) - \alpha \int_X f(x)\ \mu(dx) - \beta \int_X g(x)\ \mu(dx) \right| \\
					&\quad\leq \left| \int_X \alpha f(x) + \beta g(x)\ \mu(dx) - \int_X \alpha f_n(x) + \beta g_n(x)\ \mu(dx) \right| \\
						&\qquad+ \left| \alpha \int_X f_n(x)\ \mu(dx) + \beta \int_X g_n(x)\ \mu(dx) - \alpha \int_X f(x)\ \mu(dx) - \beta \int_X g(x)\ \mu(dx) \right| \\
					&\longrightarrow 0 \quad (n \longrightarrow \infty)
				\end{align}
				が従う.
				\QED
		\end{description}
	\end{prf}
	
	\begin{screen}
		\begin{thm}[積分の測度に関する線型性]
			$(X,\mathcal{M})$を可測空間,$\mu,\nu$をこの上の複素測度とする.$f:X \rightarrow \C$が$|\mu|$と$|\nu|$について可積分であるなら,
			$\alpha,\beta \in \C$に対し$|\alpha \mu|, |\beta \nu|, |\alpha \mu + \beta \nu|$についても可積分であり,更に次が成り立つ:
			\begin{align}
				\int_X f(x)\ (\alpha\mu + \beta\nu)(dx) = \alpha \int_X f(x)\ \mu(dx) + \beta \int_X f(x)\ \nu(dx).
			\end{align}
			\label{thm:linearity_of_integral_respect_to_measure}
		\end{thm}
	\end{screen}
	
	\begin{prf}
		\begin{description}
			\item[第一段]
				$f$が可測単関数の場合について証明する.
				$a_i \in \C,\ A_i \in \mathcal{M}\ (i=1,\cdots,n,\ \sum_{i=1}^{n} A_i = X)$を用いて
				\begin{align}
					f = \sum_{i=1}^{n} a_i \defunc_{A_i}
				\end{align}
				と表されている場合,
				\begin{align}
					&\int_X f(x)\ (\alpha\mu + \beta\nu)(dx)
					= \sum_{i=1}^{n} a_i (\alpha\mu + \beta\nu)(A_i) \\
					&\qquad = \alpha \sum_{i=1}^{n} a_i \mu(A_i) + \beta \sum_{i=1}^{n} a_i \nu(A_i)
					= \alpha \int_X f(x)\ \mu(dx) + \beta \int_X f(x)\ \nu(dx)
				\end{align}
				が成り立つ.
				
			\item[第二段]
			$f$が一般の可測関数の場合について証明する.任意の$A \in \mathcal{M}$に対して
			\begin{align}
				\left| (\alpha \mu + \beta \nu)(A) \right| \leq |\alpha||\mu(A)| + |\beta||\nu(A)| \leq |\alpha||\mu|(A) + |\beta||\nu|(A)
 			\end{align}
 			が成り立つから,左辺で$A$を任意に分割しても右辺との大小関係は変わらず
 			\begin{align}
 				|\alpha \mu + \beta \nu|(A) \leq |\alpha||\mu|(A) + |\beta||\nu|(A)
 			\end{align}
 			となる.従って$f$が$|\mu|$と$|\nu|$について可積分であるなら
 			\begin{align}
 				\int_X |f(x)|\ |\alpha \mu + \beta \nu|(dx) \leq |\alpha| \int_X |f(x)|\ |\mu|(dx) + |\beta| \int_X |f(x)|\ |\nu|(dx) < \infty
 			\end{align}
 			が成り立ち前半の主張を得る.$f$の単関数近似列$(f_n)_{n=1}^{\infty}$を取れば,前段の結果と積分の定義より
 			\begin{align}
 				&\left| \int_X f(x)\ (\alpha\mu + \beta\nu)(dx) - \alpha \int_X f(x)\ \mu(dx) - \beta \int_X f(x)\ \nu(dx) \right| \\
 					&\qquad \leq \left| \int_X f(x)\ (\alpha\mu + \beta\nu)(dx) - \int_X f_n(x)\ (\alpha\mu + \beta\nu)(dx) \right| \\
 					&\qquad \quad + |\alpha| \left| \int_X f(x)\ \mu(dx) - \int_X f_n(x)\ \mu(dx) \right|
 					+ |\beta| \left| \int_X f(x)\ \nu(dx) - \int_X f_n(x)\ \nu(dx) \right| \\
 				&\qquad \longrightarrow 0 \quad (n \longrightarrow \infty)
 			\end{align}
 			が成り立ち後半の主張が従う.
 			\QED
		\end{description}
	\end{prf}
	
	\begin{screen}
		\begin{thm}[収束定理]
			$(X,\mathcal{M})$を可測空間,$\mu$をこの上の複素測度とする.$\mathcal{M}/\borel{C}$-可測関数列$(f_n)_{n=1}^{\infty}$
			が各点で収束し,かつ或る$g \in \mathscr{L}^1(X,\mathcal{M},|\mu|)$が存在して$|f_n| \leq |g|\ (n=1,2,\cdots)$
			を満たすとき,次が成り立つ:
			\begin{align}
				\int_X\lim_{n \to \infty} f_n(x)\ \mu(dx) = \lim_{n \to \infty} \int_X f_n(x)\ \mu(dx).
			\end{align}
			\label{eq:lebesgue_convergence_theorem_complex_measure}
		\end{thm}
	\end{screen}
	
	\begin{prf}
		\begin{align}
			f(x) \coloneqq \lim_{n \to \infty} f_n(x) \quad (\forall x \in X)
		\end{align}
		とおく.Lebesgueの収束定理より$f \in \mathscr{L}^1(X,\mathcal{M},|\mu|)$かつ
		\begin{align}
			\int_X f(x)\ |\mu|(dx) = \lim_{n \to \infty} \int_X f_n(x)\ |\mu|(dx)
		\end{align}
		が成り立つから,定理\ref{thm:complex_measure_integral_linearity}及び定理\ref{thm:complex_measure_integration_well_defined}より
		\begin{align}
			\left| \int_X f(x)\ \mu(dx) - \int_X f_n(x)\ \mu(dx) \right| \leq \int_X \left| f(x) - f_n(x) \right|\ |\mu|(dx) \longrightarrow 0
			\quad (n \longrightarrow \infty)
		\end{align}
		が従う.
		\QED
	\end{prf}
	
	\begin{screen}
		\begin{thm}[順序交換定理]
		\end{thm}
	\end{screen}
	
	\begin{prf}
		$|\mu| \times |\nu| \leq |\mu \times \nu|$より$|\mu|,|\nu|,|\mu \times \nu|$にFubiniの定理を適用.
		\begin{align}
			\int_X \int_Y f_n(x,y)\ \nu(dy)\ \mu(dx) = \int_{X \times Y} f_n(x,y)\ (\mu \times \nu)(dx \times dy)
			= \int_Y \int_X f_n(x,y)\ \mu(dy)\ \nu(dx)
		\end{align}
		$f$が$|\mu \times \nu|$に関して可積分なら
		\begin{align}
			\int_{X \times Y} f(x,y)\ (\mu \times \nu)(dx \times dy)
		\end{align}
		が定義され,更に
		\begin{align}
			\int_Y |f(x,y)|\ |\nu|(dy) < \infty
		\end{align}
		だから
		\begin{align}
			\int_Y f(x,y)\ \nu(dy)
		\end{align}
		も定義される.
		\begin{align}
			\left| \int_Y f(x,y)\ \nu(dy) \right| \leq \int_Y |f(x,y)|\ |\nu|(dy)
		\end{align}
		が$|\mu|$について可積分であるから
		\begin{align}
			\int_X \int_Y f(x,y)\ \nu(dy)\ \mu(dx)
		\end{align}
		も定義される.
	\end{prf}
	\section{複素測度のRieszの表現定理}
	\begin{screen}
		\begin{dfn}[空間$C_\infty$]
			局所コンパクトなHausdorff空間$X$に対し$\c{X} \coloneqq \Set{f:X \rightarrow \C}{連続}$とおく.
			\begin{align}
				\cvan{X} \coloneqq
				\Set{f \in \c{X}}{\mbox{任意の$\epsilon > 0$に対して$\Set{x \in X}{|f(x)| \geq \epsilon}$がコンパクト.}}
			\end{align}
			として$\cvan{X}$を定め,またコンパクトな台を持つ$f \in \c{X}$の全体を$\ckon{X}$と表す.
		\end{dfn}
	\end{screen}
	
	$f \in \cvan{X}$は遠方で0になる関数である.特に$X = \R^d$の場合は
	\begin{align}
		\cvan{\R^d} = \Set{f:\R^d \rightarrow \C}{\lim_{|x| \to \infty}|f(x)| = 0}
	\end{align}
	が成り立つ.
	
	\begin{screen}
		\begin{thm}[$C_c$は$C_\infty$で稠密]
			
		\end{thm}
	\end{screen}
	\chapter{自己共役作用素のスペクトル分解}
\section{複素測度}
		\begin{screen}
		\begin{lem}[$\Dom{T_f} $は線型・稠密]
			(\refeq{eq:dfn_operator_introduced_by_measurable_functions_3})で定めた$\Dom{T_f} $は$H$の線型部分空間で$\closure{\Dom{T_f} }=H$を満たす.
			\label{lem:domain_T_f_linear_dense}
		\end{lem}
	\end{screen}
	
	\begin{prf}\mbox{}
		\begin{description}
			\item[線型性]
				$u,v \in \Dom{T_f} $に対して
				\begin{align}
					\int_X |f(x)|^2\ \mu_u(dx) < \infty,\quad \int_X |f(x)|^2\ \mu_v(dx) < \infty
				\end{align}
				が満たされている.(\refeq{eq:lem_complex_measure_introduced_by_spectral_measure_2})より任意の$\Lambda \in \mathcal{M}$に対して
				\begin{align}
					\mu_{u+v}(\Lambda) = \Norm{E(\Lambda)(u+v)}{}^2 \leq 2 \Norm{E(\Lambda)u}{}^2 + 2 \Norm{E(\Lambda)v}{}^2 = 2 \mu_u(\Lambda) + 2 \mu_v(\Lambda)
				\end{align}
				が成り立つから
				\begin{align}
					\int_X |f(x)|^2\ \mu_{u+v}(dx) \leq 2 \int_X |f(x)|^2\ \mu_u(dx) + 2 \int_X |f(x)|^2\ \mu_v(dx) < \infty 
				\end{align}
				が従い$u+v \in \Dom{T_f} $を得る.また任意に$\lambda \in \C$を取れば
				\begin{align}
					\mu_{\lambda u}(\Lambda) = \Norm{\lambda E(\Lambda)u}{}^2 = |\lambda|^2 \mu_{u}(\Lambda)
				\end{align}
				が成り立ち$\lambda u \in \Dom{T_f} $も従う.
				
			\item[稠密性]
				任意に$u \in H$を取る.
				\begin{align}
					A_k \coloneqq \Set{x \in X}{|f(x)| \leq k} \quad (k=1,2,\cdots)
				\end{align}
				に対して$u_k \coloneqq E(A_k)u$とおけば,$(A_k)_{k=1}^{\infty}$は単調に増加し$X$に収束するから
				\begin{align}
					\Norm{u - u_k}{} = \Norm{E(X)u - E(A_k)u}{} \longrightarrow 0 \quad (k \longrightarrow \infty)
					\label{eq:lem_domain_T_f_linear_dense}
				\end{align}
				が成り立つ.一方で任意の$\Lambda \in \mathcal{M}$に対して,
				命題\ref{lem:product_of_spectral_measure}と(\refeq{eq:lem_complex_measure_introduced_by_spectral_measure_2})より
				\begin{align}
					\mu_{u_k}(\Lambda) = \inprod<E(\Lambda)E(A_k)u, E(A_k)u> = \inprod<E(\Lambda \cap A_k)u,u> = \mu_u(\Lambda \cap A_k)
				\end{align}
				と表せるから$\mu_{u_k}$は$A_k$に集中している.よって
				\begin{align}
					\int_X |f(x)|^2\ \mu_{u_k}(dx) = \int_{A_k} |f(x)|^2\ \mu_{u_k}(dx) \leq k^2 \mu_u(A_k) < \infty
				\end{align}
				が成り立ち$u_k \in \Dom{T_f} $が従い,(\refeq{eq:lem_domain_T_f_linear_dense})より主張を得る.
				\QED
		\end{description}
	\end{prf}
	
	\begin{screen}
		\begin{thm}[$T_f$の定義域は0ではない]
				任意の$f \in MF$に対し$\Dom{T_f} \neq \{0\}$が成り立つ.
		\end{thm}
	\end{screen}
	
	\begin{prf}
		Hausdorff位相空間において一点集合は閉だから,$\Dom{T_f} = \{0\}$なら$H = \{0\}$が従い本章の仮定に反する.
		\QED
	\end{prf}
	
	\begin{screen}
		\begin{thm}[$T$の性質]
			$f,g \in MF$とする.
			\begin{description}
				\item[(1)] $T_f$は$H$から$H$への線型作用素である.
				\item[(2)] $u \in \Dom{T_f} ,\ v \in \Dom{T_g} $ならば次が成り立つ:
					\begin{align}
						\int_X \left| f(x) \conj{g(x)} \right|\ |\mu_{u,v}|(dx) \leq \Norm{f}{\Lp{2}{\mu_u}} \Norm{g}{\Lp{2}{\mu_v}}, \quad 
						\int_X f(x) \conj{g(x)}\ \mu_{u,v}(dx) = \inprod<T_f u, T_g v>.
					\end{align}
				\item[(3)] $T_f + T_g \subset T_{f+g}$が成り立ち,特に$g$が有界なら等号が成立する.
				\item[(4)] $T_f T_g \subset T_{fg}$が成り立ち,特に$g$が有界なら等号が成立する.
				\item[(5)] $T_f^* = T_{\conj{f}}$が成り立つ.特に$T_f$は閉作用素であり,また$f$が$\R$値なら$T_f$は自己共役である.
				\item[(6)] $\lambda \in \C$が$\lambda \neq 0$なら$T_{\lambda f} = \lambda T_f$が成り立つ.
			\end{description}
			\label{thm:properties_of_T_f}
		\end{thm}
	\end{screen}
	
	\begin{prf}\mbox{}
		\begin{description}
			\item[(1)]	補題\ref{lem:domain_T_f_linear_dense}より$T_f$の定義域は線形空間であるから,後は$T_f$が線型演算を満たすことを示せばよい.
					$f$の$MSF$-近似列$(f_n)_{n=1}^{\infty}$を取れば,定義式(\refeq{eq:dfn_operator_introduced_by_measurable_functions})より$T_{f_n}$は線型作用素であるから
					\begin{align}
						\Norm{T_f (\alpha u + \beta v) - \alpha T_f u - \beta T_f v}{}
						&\leq \Norm{T_f (\alpha u + \beta v) - T_{f_n} (\alpha u + \beta v)}{}
							+ |\alpha| \Norm{T_f u - T_{f_n} u}{} + |\beta| \Norm{T_f v - T_{f_n} v}{} \\
						&\longrightarrow 0 \quad (n \longrightarrow \infty)
					\end{align}
					が成り立つ.
					
			\item[(2)] $f,g \in MSF$のとき,任意の$u,v \in H$に対して
				\begin{align}
					\int_X \left| f(x) \conj{g(x)} \right|\ |\mu_{u,v}|(dx) \leq \Norm{f}{\Lp{2}{\mu_u}} \Norm{g}{\Lp{2}{\mu_v}},
					\quad \inprod<T_f u, T_g v> = \int_X f(x) \conj{g(x)}\ \mu_{u,v}(dx)
				\end{align}
				が成り立つ.第二式は補題(\ref{lem:MSF_properties_of_T_f})による.第一式について,
				\begin{align}
					f = \sum_{i=1}^{n} \alpha_i \defunc_{A_i},\quad 
					g = \sum_{i=1}^{n} \beta_i \defunc_{A_i}
				\end{align}
				と表示されているとして
				\begin{align}
					\int_X \left| f(x) \conj{g(x)} \right|\ |\mu_{u,v}|(dx)
					= \sum_{i=1}^{n} |\alpha_i||\beta_i| |\mu_{u,v}|(A_i)
					\leq \sum_{i=1}^{n} |\alpha_i||\beta_i| \mu_u(A_i)^{\frac{1}{2}} \mu_v(A_i)^{\frac{1}{2}}
					\leq \left( \int_X \left| f(x) \right|^2\ \mu_u(dx) \right)^{\frac{1}{2}} \left( \int_X \left| g(x) \right|^2\ \mu_v(dx) \right)^{\frac{1}{2}}
				\end{align}
				が成り立つ.
				一般の$f,g \in MF$については,$MSF$-近似列とFatouの補題より従う.
				
			\item[(3)]
				$\Dom{T_f + T_g} = \Dom{T_f} \cap \Dom{T_g} $であるから,任意の$u \in \Dom{T_f + T_g} $に対して
				\begin{align}
					\int_X |f(x)|^2\ \mu_u(dx) < \infty,
					\quad \int_X |g(x)|^2\ \mu_u(dx) < \infty
				\end{align}
				が満たされ
				\begin{align}
					\int_X |f(x) + g(x)|^2\ \mu_u(dx) \leq 
					2 \int_X |f(x)|^2\ \mu_u(dx) + 2 \int_X |g(x)|^2\ \mu_u(dx) < \infty
				\end{align}
				が従い$u \in \Dom{T_{f+g}} $が成り立つ.また任意の$u \in \Dom{T_f + T_g} $に対して,内積を展開し(2)の結果を適用すれば
				\begin{align}
					\Norm{T_{f+g}u - T_f u - T_g u}{}^2
					&= \int_X |f+g|^2\ d\mu_u + \int_X |f|^2\ d\mu_u + \int_X |g|^2\ \mu_u \\
						&\qquad - 2 \int_X \Re{(f+g)f}\ d\mu_u - 2 \int_X \Re{(f+g)g}\ d\mu_u + 2 \int_X \Re{fg}\ d\mu_u
					= 0 
				\end{align}
				が成り立ち$T_f + T_g \subset T_{f+g}$が従う.$g$が有界な場合,補題\ref{lem:complex_measure_introduced_by_spectral_measure}より
				全ての$u \in H$に対して$\mu_u$が有限測度であるから,$\Dom{T_g} $は$H$に一致し$\Dom{T_f + T_g} = \Dom{T_f} $が成り立つ.
				また任意の$u \in \Dom{T_{f+g}} $に対して
				\begin{align}
					\int_X |f(x)|^2\ \mu_u(dx) \leq 2 \int_X |f(x)+g(x)|^2\ \mu_u(dx) + 2 \int_X |g(x)|^2\ \mu_u(dx) < \infty
				\end{align}
				となり$u \in \Dom{T_f + T_g} $が従うから,前半の結果と併せて$T_f + T_g = T_{f+g}$が得られる.
			
			\item[(4)]
			
			\item[(5)]
				補題\ref{lem:domain_T_f_linear_dense}より$\Dom{T_f} $が$H$で稠密であるから$T_f^*$が定義される.
				(2)の結果より
				\begin{align}
					\inprod<T_f u,v> = \int_X f(x)\ \mu_{u,v}(dx) = \inprod<u, T_{\conj{f}} v>
					\quad \left( \forall u,v \in \Dom{T_f} = \Dom{T_{\conj{f}}} \right)
					\label{eq:thm_properties_of_T_f_1}
				\end{align}
				が成り立ち,先ず$T_{\conj{f}} \subset T_f^*$が従う.
				後は$\Dom{T_f^*} = \Dom{T_{\conj{f}}} $が成り立つことを示せばよい.
				\begin{align}
					A_k \coloneqq \Set{x \in X}{|f(x)| \leq k}
					\quad (k=1,2,\cdots)
				\end{align}
				とおいて,任意に$v \in \Dom{T_f^*} $を取り
				\begin{align}
					v_k \coloneqq T_{\conj{f}\defunc_{A_k}} v
					\quad (k=1,2,\cdots)
				\end{align}
				とすれば,各$k \in \N$について
				\begin{align}
					\Norm{T_f v_k}{} = \Norm{T_f T_{\conj{f}\defunc_{A_k}} v}{}^2 = \int_{A_k} |f(x)|^4\ \mu_v(dx) < k^4 \mu_v(A_k) < \infty
				\end{align}
				が成り立つから$v_k \in \Dom{T_f} $である.(\refeq{eq:thm_properties_of_T_f_1})と同様にすれば
				\begin{align}
					\Norm{v_k}{}^2 = \inprod<T_{\conj{f}\defunc_{A_k}} v, T_{\conj{f}\defunc_{A_k}} v>
						= \inprod<T_{f\defunc_{A_k}} T_{\conj{f}\defunc_{A_k}} v, v>
						= \inprod<T_f v_k, v>
						= \inprod<v_k, T_f^* v>
				\end{align}
				となり,Schwartzの不等式より
				\begin{align}
					\Norm{v_k}{} \leq \Norm{T_f^* v}{}
					\label{eq:thm_properties_of_T_f_2}
				\end{align}
				が得られる.一方で
				\begin{align}
					\Norm{v_k}{}^2 = \Norm{T_{\conj{f}\defunc_{A_k}} v}{}^2 = \int_{A_k} |f(x)|^2\ \mu_v(dx)
				\end{align}
				が成り立つから,(\refeq{eq:thm_properties_of_T_f_2})と併せて
				\begin{align}
					\int_{A_k} |f(x)|^2\ \mu_v(dx) \leq \Norm{T_f^* v}{}^2
				\end{align}
				が従う.$(A_k)_{k=1}^{\infty}$は単調増大列で$\cup_{k=1}^{\infty} A_k = X$を満たすから,単調収束定理より
				\begin{align}
					\int_X |f(x)|^2\ \mu_v(dx) \leq \Norm{T_f^* v}{}^2
				\end{align}
				となり$v \in \Dom{T_f} $が得られる.
				特に$T_f = T_{\conj{f}}^*$が従い,共役作用素が閉線型であるから$T_f$も閉作用素である.
				
			\item[(6)]
				$\lambda = 0$の場合は,$\Dom{T_{\lambda f}} = \Dom{T_0} = H$であるが$\Dom{T_f} = H$とは限らないから主張が従わない.
				$\lambda \neq 0$の場合
				\begin{align}
					\int_X |\lambda f(x)|^2\ \mu_u(dx) < \infty \quad \Leftrightarrow \quad
					\int_X |f(x)|^2\ \mu_u(dx) < \infty
				\end{align}
				が成り立つから$\Dom{T_{\lambda f}} = \Dom{T_f} = \Dom{\lambda T_f} $である.また$f$の$MSF$-近似列$(f_n)_{n=1}^{\infty}$
				については補題\ref{lem:MSF_properties_of_T_f}より
				\begin{align}
					T_{\lambda f_n} u = \lambda T_{f_n} u \quad \left( u \in \Dom{T_{\lambda f}} \right)
				\end{align}
				が満たされているから,任意の$u \in \Dom{T_{\lambda f}} $に対して
				\begin{align}
					\Norm{T_{\lambda f}u - \lambda T_f u}{}
					\leq \Norm{T_{\lambda f}u - T_{\lambda f_n} u}{} + |\lambda| \Norm{T_f u - T_{f_n} u}{}
					\longrightarrow 0 \quad (n \longrightarrow \infty)
				\end{align}
				が従う.
				\QED
		\end{description}
	\end{prf}
	
	\begin{screen}
		\begin{cor}
			$f,g \in MF$とする.
			\begin{description}
				\item[(1)] $T_f = T_g$であることと$E\left( \Set{x \in X}{f(x) \neq g(x)} \right) = 0\ $(零写像)であることは同値である.
				\item[(2)] $f$が有界ならば$T_f \in \selfBop{H} $であり$\Norm{T_f}{\selfBop{H}} \leq \sup{x \in X}{|f(x)|}$が成り立つ.
				\item[(3)] 或る$L > 0$に対し$E\left( \Set{x \in X}{|f(x)| > L} \right) = 0$が成り立つとき,$T_f \in \selfBop{H} $であり次が成り立つ:
					\begin{align}
						\Norm{T_f}{\selfBop{H}} = \inf{}{\Set{L > 0}{E\left( \Set{x \in X}{|f(x)| > L} \right) = 0}}.
					\end{align}
				\item[(4)] $\lambda \in \C,\epsilon > 0$に対し$U_\epsilon(\lambda) \coloneqq \Set{z \in \C}{|z-\lambda| < \epsilon}$とおく.
					$T_f$のレゾルベント集合\footnotemark は
					\begin{align}
						\Res{T_f} = \Set{\lambda \in \C}{\mbox{或る$\epsilon > 0$が存在して$E\left(f^{-1}(U_\epsilon(\lambda))\right) = 0$を満たす.}}
					\end{align}
					で与えられ,さらに$\lambda \in \Res{T_f} $に対して$\epsilon > 0$が$E\left(f^{-1}(U_\epsilon(\lambda))\right) = 0$を満たすとすれば
					\begin{align}
						\left( \lambda I - T_f \right)^{-1} = T_{\frac{1}{\lambda - f} \defunc_{X \backslash f^{-1}(U_\epsilon(\lambda))}}
					\end{align}
					が成り立つ.
			\end{description}
			\label{cor:properties_of_T_f}
		\end{cor}
	\end{screen}
	
	\footnotetext{
		定理\ref{thm:properties_of_T_f}より$T_f$は閉作用素であるからレゾルベントを考察できる.
	}
	
	\begin{prf}\mbox{}
		\begin{description}
			\item[(1)] 今$N \coloneqq \Set{x \in X}{f(x) \neq g(x)}$とおく.
				$T_f = T_g$が成り立っているとすると,$u \in \Dom{T_f} $に対し
				\begin{align}
					0 = \Norm{T_f u - T_g u}{}^2 = \Norm{T_{f-g} u}{}^2 = \int_X |f(x) - g(x)|^2\ \mu_u(dx)
					\label{eq:cor_properties_of_T_f_1}
				\end{align}
				が従い$\mu_u(N) = \Norm{E(N)u}{}^2 = 0$となる.$\Dom{T_f} $の稠密性と直交射影$E(N)$の連続性より$E(N) = 0$を得る.
				逆に$E(N) = 0$の場合,任意の$u \in \Dom{T_f} $に対して$\mu_u(N) = \Norm{E(N)u}{}^2 = 0$が成り立つから
				\begin{align}
					\int_X |g(x)|^2\ \mu_u(dx) \leq 2 \int_X |f(x) - g(x)|^2\ \mu_u(dx) + 2 \int_X |f(x)|^2\ \mu_u(dx) = 2 \int_X |f(x)|^2\ \mu_u(dx) < \infty
				\end{align}
				となり,(\refeq{eq:cor_properties_of_T_f_1})と併せて$T_f \subset T_g$が従う.同様に$T_g \subset T_f$も成り立つから$T_f = T_g$を得る.	
				
			\item[(3)]
				$E\left( \Set{x \in X}{|f(x)| > L} \right) = 0$を満たす$L > 0$に対し
				\begin{align}
					A_L \coloneqq \Set{x \in X}{|f(x)| \leq L}
				\end{align}
				とおけば,任意の$u \in H$に対し
				\begin{align}
					\mu_u(\Lambda) = \inprod<E(\Lambda)u,u> = \inprod<E(\Lambda \cap A_L)u,u> = \mu_u(\Lambda \cap A_L)
				\end{align}
				が成り立つから$\mu_u$は$A_L$に集中している.従って定理\ref{thm:properties_of_T_f}(2)と
				$\mu_u$の定義(\refeq{eq:lem_complex_measure_introduced_by_spectral_measure_1})より
				\begin{align}
					\Norm{T_f u}{}^2 = \int_X |f(x)|^2\ \mu_u(dx) = \int_{A_L} |f(x)|^2\ \mu_u(dx) \leq L^2 \mu_u(X) = L^2 \Norm{u}{}^2 < \infty
				\end{align}
				となるから,$\Dom{T_f} = H$且つ$\Norm{T_f}{\selfBop{H}} \leq L$を得る.これにより$T_f \in \selfBop{H} $と
				\begin{align}
					\Norm{T_f}{\selfBop{H}} \leq \inf{}{\Set{L > 0}{E\left( \Set{x \in X}{|f(x)| > L} \right) = 0}}
				\end{align}
				が成り立つ.ここで$\Norm{T_f}{\selfBop{H}} < \inf{}{\Set{L > 0}{E\left( \Set{x \in X}{|f(x)| > L} \right) = 0}}$が成り立つとすると
				
			\item[(4)] $\lambda \in \C$を固定する.任意の$\epsilon > 0$に対し$V_\epsilon \coloneqq f^{-1}(U_\epsilon(\lambda))$とおけば
				$f$の可測性から$V_\epsilon \in \mathcal{M}$であり,また
				\begin{align}
					x \in V_\epsilon \quad \Leftrightarrow \quad |\lambda - f(x)| < \epsilon
				\end{align}
				が成り立つから,$X \backslash V_\epsilon$上で$1/(\lambda - f) \leq 1/\epsilon$が満たされる.
				\begin{description}
					\item[第一段] $E(V_\epsilon) = 0$を満たす$\epsilon$が存在しない場合,
				\end{description}
		\end{description}
	\end{prf}
	
	\begin{screen}
		\begin{cor}
			$(X,\mathcal{M}) = \left( \R^d,\borel{\R^d} \right)$の場合,
			\begin{align}
				\supp{E} \coloneqq \Set{x \in \R^d}{\mbox{$x$の任意の開近傍$V$に対して$E(V)=0$が成り立つ.}}
			\end{align}
			として$E$の台を定める.このとき任意の連続写像$f:\R^d \rightarrow \C$について
			\begin{align}
				\Spctr{T_f} = \closure{f(\supp{E})}
			\end{align}
			が成り立つ.特に$\supp{E}$がコンパクトなら$\Spctr{T_f} = f(\supp{E})$となる.
		\end{cor}
	\end{screen}
	
	\begin{prf}
		任意に$x \in \supp{E}$を取る.$f(x)$の任意の$\epsilon$近傍$U_\epsilon = U_\epsilon(f(x))$に対し,
		$f$の連続性から$f^{-1}(U_\epsilon)$は$x$の開近傍となるから,
		$E(f^{-1}(U_\epsilon)) = 0$が成り立ち$f(x) \in \Spctr{T_f}$が従う.$\Spctr{T_f}$は閉集合であるから
		$\closure{f(\supp{E})} \subset \Spctr{T_f}$を得る.
		逆に任意に$\lambda \in \closure{f(\supp{E})}$を取れば,或る$\epsilon > 0$が存在して
		$U_\epsilon(\lambda) \cap \closure{f(\supp{E})} = \emptyset$を満たすから
		$f^{-1}(U_\epsilon(\lambda)) \cap \supp{E} = \emptyset$が成り立つ.
		$f^{-1}(U_\epsilon(\lambda))$に属する$\R^d$の有理点全体を$\Q_f$と表せば,
		各$r \in \Q_f$に対し或る開近傍$V_r$が存在して$E(V_r) = 0$を満たすから
		$E\left( V_r \cap f^{-1}(U_\epsilon(\lambda)) \right) = 0\ (\forall r \in \Q_f)$が従う.
		$\Q_f$は可付番だから添数を変えれば
		\begin{align}
			f^{-1}(U_\epsilon(\lambda)) = \bigcup_{n \in \N} V_n \cap f^{-1}(U_\epsilon(\lambda))
		\end{align}
		と表され
		\footnote{
			もし或る$x \in f^{-1}(U_\epsilon(\lambda))$が$x \notin \bigcup_{n \in \N} V_n \cap f^{-1}(U_\epsilon(\lambda))$
			を満たすとすれば,$x$に近づく有理点列$(x_n)_{n=1}^{\infty} \subset \Q_f$に対し
			$(\epsilon_n)_{n=1}^{\infty}$が存在して
			\begin{align}
				x \notin \bigcup_{n=1}^{\infty} U_{\epsilon_n}(x_n)
			\end{align}
			を満たすことになり$x_n \longrightarrow x$に反する.
		}
		,
	\end{prf}
	\section{自己共役作用素のスペクトル分解}
	$H$を複素Hilbert空間とし,内積とノルムをそれぞれ$\inprod<\cdot,\cdot>,\Norm{\cdot}{}$と表す.
	また$I$を$H$上の恒等写像とし,
	$\CM = \CM(\R^d,\borel{\R^d})$と$\cvan{\R^d}$のノルムをそれぞれ$\Norm{\cdot}{\CM},\Norm{\cdot}{\infty}$で表す.
	
	\begin{screen}
		\begin{thm}[スペクトル分解定理]
			写像$T:\cvan{\R^d} \rightarrow \selfBop{H} $が有界線型で次を満たすとする:
			\begin{description}
				\item[(1)] $T_f T_g = T_{fg} \quad (\forall f,g \in \cvan{\R^d}).$
				\item[(2)] $T_f^* = T_{\conj{f}} \quad (\forall f \in \cvan{\R^d}).$
				\item[(3)] 或る$\cvan{\R^d}$の列$(\phi_n)_{n=1}^{\infty}$が存在して次を満たす:
					\begin{align}
						&\sup{n \in \N}{\Norm{\phi_n}{\infty}} < \infty,
						\quad \lim_{n \to \infty} \phi_n(x) = 1 \quad (\forall x \in \R^d), \\
						&\quad \Norm{T_{\phi_n}u - u}{} \longrightarrow 0
							\quad (n \longrightarrow \infty,\ \forall u \in H).
					\end{align}
			\end{description}
			このとき次の表現を持つスペクトル測度$E:\borel{\R^d} \rightarrow \Oproj{H}$が唯一つ存在する:
			\begin{align}
				T_f = \int_{\R^d} f(x)\ E(dx) \quad (\forall f \in \cvan{\R^d}).
				\label{eq:thm_pectral_decomposition_of_bounded_linear_operators_0}
			\end{align}
			\label{thm:spectral_decomposition_of_bounded_linear_operators}
		\end{thm}
	\end{screen}
	
	定理\ref{thm:spectral_decomposition_of_bounded_linear_operators}において
	$T$の条件(3)が要求されていないとする.
	この場合$T$を零写像としても(1)(2)は満たされるが,
	\begin{align}
		E(\R^d)u = \lim_{n \to \infty} \int_{\R^d} \phi_n(x)\ E(dx) u = 0 \quad (\forall u \in H)
	\end{align}
	が従い$E(\R^d) = I$に反する.
	
	\begin{prf}\mbox{}
		\begin{description}
			\item[第一段]
				$E$の存在を示す.任意の$u,v \in H$に対し
				\begin{align}
					S_{u,v}:\cvan{\R^d} \ni f \longmapsto \inprod<T_f u,v> \in \C
				\end{align}
				と定めれば$S_{u,v} \in \cvan{\R^d}^*$となる.実際$T,T_f$及び内積の線型性より,任意の$f,g \in \cvan{\R^d},\alpha,\beta \in \C$に対して
				\begin{align}
					S_{u,v} (\alpha f + \beta g) = \inprod<T_{\alpha f + \beta g} u,v>
					= \inprod<(\alpha T_f + \beta T_g) u,v>
					= \alpha S_{u,v} f + \beta S_{u,v} g
				\end{align}
				が成り立つから$S_{u,v}$の線型性が従い,またSchwartzの不等式と$T,T_f$の有界性より
				\begin{align}
					\left| S_{u,v}f \right| \leq \Norm{T_f u}{} \Norm{v}{} \leq \Norm{T}{\Bop{\cvan{\R^d}}{\selfBop{H} } } \Norm{u}{} \Norm{v}{} \Norm{f}{\infty}
					\quad (\forall f \in \cvan{\R^d})
					\label{eq:thm_spectral_decomposition_of_bounded_linear_operators_5}
				\end{align}
				も成り立つから$S_{u,v}$の有界性が従う.よって定理\ref{thm:complex_measure_riesz_representation_theorem}より
				或る$\mu_{u,v}$が唯一つ対応し
				\begin{align}
					\inprod<T_f u,v> = \int_{\R^d} f(x)\ \mu_{u,v}(dx) \quad (\forall f \in \cvan{\R^d})
					\label{eq:thm_spectral_decomposition_of_bounded_linear_operators_3}
				\end{align}
				と表現できる.このとき任意に$\Lambda \in \borel{\R^d}$に取り固定すれば,対応
				\begin{align}
					H \times H \ni [u,v] \longmapsto \mu_{u,v}(\Lambda)
					\label{eq:thm_spectral_decomposition_of_bounded_linear_operators_1}
				\end{align}
				は準双線型であり
				\begin{align}
					\left| \mu_{u,v}(\Lambda) \right| \leq \Norm{T}{\Bop{\cvan{\R^d}}{\selfBop{H} } } \Norm{u}{} \Norm{v}{}
					\quad (\forall u,v \in H)
					\label{eq:thm_spectral_decomposition_of_bounded_linear_operators_2}
				\end{align}
				を満たすから,或る$E(\Lambda) \in \selfBop{H} $が唯一つ存在して
				\begin{align}
					\inprod<E(\Lambda) u, v> = \mu_{u,v}(\Lambda) \quad (\forall u,v \in H)
					\label{eq:thm_spectral_decomposition_of_bounded_linear_operators_6}
				\end{align}
				が成り立つ.以下,上に列記した事柄を証明する.
				\begin{description}
					\item[(\refeq{eq:thm_spectral_decomposition_of_bounded_linear_operators_1})の準双線型性]
						先ず任意の$v \in H$に対し$u \longmapsto \mu_{u,v}(\Lambda)$が線型であることを示す.
						任意に$u,w \in H,\alpha,\beta \in \C$と$f \in \cvan{\R^d}$を取れば,$T_f$の線型性と
						(\refeq{eq:thm_spectral_decomposition_of_bounded_linear_operators_3})の表現
						及び定理\ref{thm:linearity_of_integral_respect_to_measure}より
						\begin{align}
							&\int_{\R^d} f(x)\ \mu_{\alpha u + \beta w,v}(dx)
							= \inprod<T_f (\alpha u + \beta w),v> \\
							&\qquad = \alpha \inprod<T_f u,v> + \beta \inprod<T_f w,v>
							= \int_{\R^d} f(x)\ \left( \alpha \mu_{u,v} + \beta \mu_{w,v} \right)(dx)
							\label{eq:thm_spectral_decomposition_of_bounded_linear_operators_4}
						\end{align}
						が成り立つ.$\R^d$の任意の開集合$A$に対し$|f_n| \leq 1$且つ$f_n \longrightarrow \defunc_A\ $(各点)を満たす
						$\cvan{\R^d}$の列$(f_n)_{n=1}^{\infty}$が存在するから,
						定理\ref{eq:lebesgue_convergence_theorem_complex_measure}より
						\begin{align}
							\mu_{\alpha u + \beta w,v}(A)
							= \lim_{n \to \infty} \int_{\R^d} f_n(x)\ \mu_{\alpha u + \beta w,v}(dx)
							= \lim_{n \to \infty} \int_{\R^d} f_n(x)\ \left( \alpha \mu_{u,v} + \beta \mu_{w,v} \right)(dx)
							= \left( \alpha \mu_{u,v} + \beta \mu_{w,v} \right)(A)
						\end{align}
						が従い,定理\ref{thm:identity_theorem_of_complex_measures}より
						$\mu_{\alpha u + \beta w,v} = \alpha \mu_{u,v} + \beta \mu_{w,v}$が得られる.
						$v \longmapsto \mu_{u,v}(\Lambda)$についても同様であるが,内積の準双線型性より
						(\refeq{eq:thm_spectral_decomposition_of_bounded_linear_operators_4})の
						スカラーが共役に替わり
						$\mu_{u,\alpha v + \beta w} = \conj{\alpha} \mu_{u,v} + \conj{\beta} \mu_{u,w}\ (\forall u,v,w \in H,\alpha,\beta \in \C)$が従う.
						
					\item[(\refeq{eq:thm_spectral_decomposition_of_bounded_linear_operators_2})の証明]
						定理\ref{thm:complex_measure_riesz_representation_theorem}の等長性と
						(\refeq{eq:thm_spectral_decomposition_of_bounded_linear_operators_5})
						(\refeq{eq:thm_spectral_decomposition_of_bounded_linear_operators_3})より,
						任意の$u,v \in H$に対して次が成り立つ:
						\begin{align}
							\Norm{\mu_{u,v}}{\CM} 
							= \sup{\substack{f \in \cvan{\R^d} \\ \Norm{f}{\infty} \leq 1}}{\left| \int_{\R^d} f(x)\ \mu_{u,v}(dx) \right|}
							= \sup{\substack{f \in \cvan{\R^d} \\ \Norm{f}{\infty} \leq 1}}{\left| \inprod<T_f u,v> \right|}
							\leq \Norm{T}{\Bop{\cvan{\R^d}}{\selfBop{H} } } \Norm{u}{} \Norm{v}{}.
						\end{align}
						
					\item[$E(\Lambda)$の存在と一意性]
						上の結果より任意の$u \in H$に対して$Q_u:H \ni v \longmapsto \mu_{u,v}(\Lambda)$は$\conj{Q_u} \in H^*$を満たすから,
						Hilbert空間におけるRieszの表現定理より或る$a(\Lambda)_u \in H$が唯一つ存在し
						\begin{align}
							\conj{Q_u(v)} = \inprod<v,a(\Lambda)_u> = \conj{\inprod<a(\Lambda)_u,v>} \quad (\forall v \in H)
						\end{align}
						を満たす.写像$H \ni u \longmapsto a(\Lambda)_u \in H$を$E(\Lambda)$と表せば
						\begin{align}
							\inprod<E(\Lambda)u,v> = Q_u(v) = \mu_{u,v}(\Lambda) \quad (\forall u,v \in H)
						\end{align}
						が成り立つ.次に$E(\Lambda)$の一意性を示す.或る$F(\Lambda):H \rightarrow H$が存在し
						\begin{align}
							\inprod<F(\Lambda)u,v> = \mu_{u,v}(\Lambda) \quad (\forall u,v \in H)
						\end{align}
						を満たすなら
						\begin{align}
							\inprod<\left( E(\Lambda) - F(\Lambda) \right)u,v> = 0
							\quad (\forall u,v \in H)
						\end{align}
						となるから,特に$v = \left( E(\Lambda) - F(\Lambda) \right)u$として
						\begin{align}
							\left( E(\Lambda) - F(\Lambda) \right)u = 0 \quad (\forall u \in H)
						\end{align}
						が従い$E(\Lambda) = F(\Lambda)$が得られる.
						
					\item[$E(\Lambda)$の線型有界性]
						(\refeq{eq:thm_spectral_decomposition_of_bounded_linear_operators_6})と
						(\refeq{eq:thm_spectral_decomposition_of_bounded_linear_operators_1})の準双線型性より,
						任意の$u,v,w \in H$と$\alpha,\beta \in \C$に対して
						\begin{align}
							\inprod<E(\Lambda)(\alpha u + \beta v),w>
							= \mu_{\alpha u + \beta v, w}(\Lambda)
							= \alpha \mu_{u,w}(\Lambda) + \beta \mu_{v,w}(\Lambda)
							= \alpha \inprod<E(\Lambda)u,w> + \beta \inprod<E(\Lambda)v,w>
						\end{align}
						が成り立つ.特に$w = E(\Lambda)(\alpha u + \beta v) - \alpha E(\Lambda) u - \beta E(\Lambda) v$とすれば
						\begin{align}
							E(\Lambda)(\alpha u + \beta v) - \alpha E(\Lambda) u - \beta E(\Lambda) v = 0
							\quad (\forall u,v \in H,\ \alpha,\beta \in \C)
						\end{align}
						が従い$E(\Lambda)$の線型性を得る.また(\refeq{eq:thm_spectral_decomposition_of_bounded_linear_operators_2})より
						任意の$u,v \in H$に対して
						\begin{align}
							\left| \inprod<E(\Lambda)u,v> \right| \leq \Norm{T}{\Bop{\cvan{\R^d}}{\selfBop{H} } } \Norm{u}{} \Norm{v}{}
						\end{align}
						が成り立つから,特に$v = E(\Lambda) u$とすれば
						\begin{align}
							\Norm{E(\Lambda)u}{} \leq \Norm{T}{\Bop{\cvan{\R^d}}{\selfBop{H} } } \Norm{u}{}
							\quad (\forall u \in H)
						\end{align}
						が得られ$E(\Lambda)$の有界性が従う.
				\end{description}
			
			\item[第二段] 任意の$\Lambda \in \borel{\R^d}$に対して$E(\Lambda)$が直交射影であることを示す.
				前段で$E(\Lambda) \in \selfBop{H} $が示されたから,
				命題\label{prp:orthogonal_projection_idempotent_self_adjoint}より後は
				$E(\Lambda)^2 = E(\Lambda)$と$E(\Lambda)^* = E(\Lambda)$を示せばよい.
				$T_f^* = T_{\conj{f}}\ (\forall f \in \cvan{\R^d})$の仮定より
				\begin{align}
					\inprod<T_f u,v> = \inprod<u,T_f^*v> = \inprod<u,T_{\conj{f}}v> \quad (\forall u,v \in H)
				\end{align}
				が成り立つから,(\refeq{eq:thm_spectral_decomposition_of_bounded_linear_operators_3})より
				\begin{align}
					\int_{\R^d} f(x)\ \mu_{u,v}(dx) = \inprod<T_f u,v> = \conj{\inprod<T_{\conj{f}}v,u>}
					= \int_{\R^d} f(x)\ \conj{\mu_{v,u}}(dx)
					\quad (\forall f \in \cvan{\R^d})
				\end{align}
				が従う.前段で(\refeq{eq:thm_spectral_decomposition_of_bounded_linear_operators_1})の準双線型性
				を示した時と同様にして$\mu_{u,v} = \conj{\mu_{v,u}}$が成り立つから
				\begin{align}
					\inprod<E(\Lambda)u,v> = \mu_{u,v}(\Lambda)
					= \conj{\mu_{v,u}(\Lambda)}
					= \conj{\inprod<E(\Lambda)v,u>}
					= \inprod<u,E(\Lambda)v>
					\quad (\forall u,v \in H)
				\end{align}
				となり$E(\Lambda)^* = E(\Lambda)$が得られる.
				次に$\Lambda$が開集合であるとする.
				$0 \leq f_1 \leq f_2 \leq \cdots \leq 1$かつ$\defunc_\Lambda$に各点収束する
				関数列$(f_n)_{n=1}^{\infty} \subset \cvan{\R^d}$を取れば,
				(\refeq{eq:thm_spectral_decomposition_of_bounded_linear_operators_3})と
				(\refeq{eq:thm_spectral_decomposition_of_bounded_linear_operators_6})より
				\begin{align}
					\inprod<T_{f_n}u,u> = \int_{\R^d} f_n(x)\ \mu_u(dx)
					\leq \int_{\R^d} \defunc_\Lambda(x)\ \mu_u(dx) = \inprod<E(\Lambda)u,u>
					\quad (\forall u \in H,\ n=1,2,\cdots)
				\end{align}
				が成り立ち,更に$\mu_u(\Lambda) < \infty\ (\forall u \in H)$であるからLebesgueの収束定理より
				\begin{align}
					\inprod<\left( E(\Lambda) - T_{f_n} \right)u,u>
					= \int_{\R^d} \defunc_{\Lambda}(x) - f_n(x)\ \mu_u(dx) 
					\longrightarrow 0 \quad (n \longrightarrow \infty,\ \forall u \in H)
				\end{align}
				となる.$\Norm{T_{f_n} u}{} \leq \Norm{T}{\Bop{\cvan{\R^d}}{\selfBop{H} } } \Norm{u}{}$より
				\begin{align}
					\Norm{E(\Lambda) - T_{f_n}}{\selfBop{H} }
					\leq \Norm{E(\Lambda)}{\selfBop{H} } + \Norm{T}{\Bop{\cvan{\R^d}}{\selfBop{H} } }
					\quad (\forall n=1,2,\cdots)
				\end{align}
				が成り立つから,$C \coloneqq \Norm{E(\Lambda)}{\selfBop{H} } + \Norm{T}{\Bop{\cvan{\R^d}}{\selfBop{H} } }$とおけば
				\begin{align}
					\Norm{\left( E(\Lambda) - T_{f_n} \right)u}{}
					\leq \sqrt{C\inprod<\left( E(\Lambda) - T_{f_n} \right)u,u>}
					\longrightarrow 0 \quad (n \longrightarrow \infty,\ \forall u \in H)
				\end{align}
				が従う.またこれは$g_n \coloneqq f_n^2$に対しても成り立つ.ゆえに
				\begin{align}
					\left| \inprod<E(\Lambda)u,E(\Lambda)u> - \inprod<E(\Lambda)u,u> \right|
					\leq \left| \inprod<E(\Lambda)u,E(\Lambda)u> - \inprod<T_{f_n} u,T_{f_n} u> \right|
						+ \left| \inprod<T_{f_n^2} u,u> - \inprod<E(\Lambda)u,u> \right|
					\longrightarrow 0 \quad (n \longrightarrow \infty)
				\end{align}
				となり$E(\Lambda)^* = E(\Lambda)$と併せて$E(\Lambda)^2 = E(\Lambda)$が得られる.
				
			\item[第三段] $E(\R^d) = I$を示す.任意の$n \in \N$に対して
				\begin{align}
					\inprod<\phi_n u,v> = \int_{\R^d} \phi_n(x)\ \mu_{u,v}(dx)
				\end{align}
				が成り立つ.また仮定より任意の$u \in H$に対して
				\begin{align}
					\left| \inprod<\phi_n u,v> - \inprod<u,v> \right|
					\leq \Norm{\phi_n u - u}{} \Norm{v}{}
					\longrightarrow 0 \quad (n \longrightarrow \infty)
				\end{align}
				かつ
				\begin{align}
					\int_{\R^d} \phi_n(x)\ \mu_{u,v}(dx) \longrightarrow \mu_{u,v}(\R^d) = \inprod<E(\R^d)u,v>
				\end{align}
				が成り立つから,
				\begin{align}
					\left| \inprod<u,v> - \inprod<E(\R^d)u,v> \right|
					\leq \left| \inprod<u,v> - \inprod<\phi_n u,v> \right|
						+ \left| \int_{\R^d} \phi_n(x)\ \mu_{u,v}(dx) - \inprod<E(\R^d)u,v> \right|
					\longrightarrow 0 \quad (n \longrightarrow \infty)
				\end{align}
				となり$\inprod<u,v> = \inprod<E(\R^d)u,v>\ (\forall u,v \in H)$が成り立つ.特に$v = u - E(\R^d)u$とすれば
				$u = E(\R^d)u\ (\forall u \in H)$が従い$E(\R^d) = I$を得る.
				
			\item[第四段] $E$の完全加法性を示す.
				任意の$A,B \in \borel{\R^d},\ u,v \in H$に対し
				\begin{align}
					\inprod<\left(E(A+B) - E(A) - E(B) \right)u,v>
					&= \inprod<E(A+B)u,v> - \inprod<E(Au,v> - \inprod<E(B)u,v> \\
					&= \mu_{u,v}(A+B) - \mu_{u,v}(A) - \mu_{u,v}(B)
					= 0
				\end{align}
				が成り立つから
				\begin{align}
					E(A+B)u = E(A)u + E(B)u 
					\quad (\forall u \in H)
				\end{align}
				が得られる.次に任意に互いに素な列$\Lambda_1,\Lambda_2,\cdots \in \borel{\R^d}$を取り
				$\Lambda = \sum_{n=1}^{\infty} \Lambda_n,\ \Lambda_N \coloneqq \sum_{n=1}^{N} \Lambda_n$とおけば,
				\begin{align}
					\Norm{\left( E(\Lambda) - E(\Lambda_N) \right)u}{}
					\leq \sqrt{\inprod<\left( E(\Lambda) - E(\Lambda_N) \right)u, u>}
					= \leq \sqrt{\mu_u(\Lambda) - \mu_u(\Lambda_N)}
					\longrightarrow 0 \quad (N \longrightarrow \infty,\ \forall u \in H)
				\end{align}
				が成り立ち,一方で
				\begin{align}
					E(\Lambda_N) u = \sum_{n=1}^{N} E(\Lambda_n) u \quad (\forall N \in \N,\ u \in H)
				\end{align}
				となるから,命題\ref{prp:orthogonal_projection_product_sum}より
				任意の$i,j \in \N$に対して$E(\Lambda_i)E(\Lambda_j) = \delta_{ij}E(\Lambda_i)$が成り立ち
				$\sum_{n=1}^{\infty} E(\Lambda_n) u$は絶対収束する.
				\begin{align}
					E(\Lambda) u = \sum_{n=1}^{\infty} E(\Lambda_n) u \quad (\forall u \in H)
				\end{align}
				が得られる.
			
			\item[第五段]
				(\refeq{eq:thm_pectral_decomposition_of_bounded_linear_operators_0})を示す.
				$f \in \cvan{\R^d}$が単関数の場合は
				\begin{align}
					\int_{\R^d} f_n(x)\ \mu_{u,v}(dx) = \inprod<\int_{\R^d} f_n(x)\ E(dx) u, v>
					\quad (\forall u,v \in H)
				\end{align}
				が成り立つ.一般の$f \in \cvan{\R^d}$に対しては,$MSF$-単調近似列$(f_n)_{n=1}^{\infty}$を取れば
				\begin{align}
					&\left| \inprod<T_f u,v> - \inprod<\int_{\R^d} f(x)\ E(dx)\ u, v> \right|
					= \left| \int_{\R^d} f(x)\ \mu_{u,v}(dx) - \inprod<\int_{\R^d} f(x)\ E(dx)\ u, v> \right| \\
					&\qquad \leq \left| \int_{\R^d} f(x)\ \mu_{u,v}(dx) - \int_{\R^d} f_n(x)\ \mu_{u,v}(dx) \right| 
						+ \Norm{\int_{\R^d} f_n(x)\ E(dx) u - \int_{\R^d} f(x)\ E(dx) u}{} \Norm{v}{}
					\longrightarrow 0 \quad (n \longrightarrow \infty)
				\end{align}
				が成り立つ.
				
			\item[第六段]
				$E$の一意性を示す.開集合の上で$E$は一意.一致の定理より$E$は一意.
			\QED
		\end{description}
	\end{prf}
	

\chapter{位相線形空間}
	\section{有向集合とフィルター}
	\begin{screen}
		\begin{dfn}[有向集合]
			集合$\Lambda$において,次の性質を持つ二項関係$\leq\ (\geq)$\footnotemark
			が定まっているとき,$\Lambda$を有向集合(directed set),或は有向擬順序集合(directed preorder)と呼ぶ.
			\begin{description}
				\item[反射的] $\lambda \leq \lambda\ (\forall \lambda \in \Lambda)$,
				\item[推移的] $\lambda \leq \mu$かつ$\mu \leq \nu$なら$\lambda \leq \nu\ (\forall \lambda,\mu,\nu \in \Lambda)$,
				\item[有向的] 任意に$\lambda,\mu \in \Lambda$を取れば或る$\nu \in \Lambda$が存在し
					$\lambda \leq \nu$かつ$\mu \leq \nu$を満たす.
			\end{description}
			(順序集合に対して三番目の性質を追加しても有向集合となるが,本節では反対称性を使うことはない.)
		\end{dfn}
	\end{screen}
	\footnotetext{
		$\lambda,\mu \in \Lambda$について
		$\lambda \leq \mu \Leftrightarrow \mu \geq \lambda$.
	}
	\begin{screen}
		\begin{dfn}[有向族による収束の記述]
			位相空間$X$において,有向集合$\Lambda$を添数集合とする
			系$(x_\lambda)_{\lambda \in \Lambda}$を有向族(directed family)と呼ぶ.
			$(x_\lambda)_{\lambda \in \Lambda}$が$x \in X$に収束するとは,
			$x$の任意の近傍$U$に対して或る$\lambda_0 \in \Lambda$が存在し,
			$x_\lambda \in U\ (\forall \lambda \geq \lambda_0)$が成り立つことにより定める.
			これは距離空間における点列収束の一般化である.
		\end{dfn}
	\end{screen}
	
	\begin{screen}
		\begin{thm}[有向点族による連続性の特徴づけ]
			$X,Y$を位相空間とする.
			写像$f:X \longrightarrow Y$が点$x \in X$で連続
			であるための必要十分条件について,次がいえる:
			\begin{description}
				\item[(1)] $x$が可算な基本近傍系をもつとき,
					必要十分条件は,$x$に収束する任意の点列$(x_n)_{n=1}^{\infty}$に対し
					$\left( f(x_n) \right)_{n=1}^\infty$
					が$f(x)$に収束することである.
					すなわち,$X$が第一可算公理を満たすなら点列連続性と連続性は一致する.
				\item[(2)] $X$が一般の位相空間の場合,
					必要十分条件は,$x$に収束する任意の有向点族$(x_\lambda)_{\lambda \in \Lambda}$に対し
					$\left( f(x_\lambda) \right)_{\lambda \in \Lambda}$
					が$f(x)$に収束することである.
			\end{description}
		\end{thm}
	\end{screen}
	
	\begin{prf}\mbox{}
		\begin{description}
			\item[必要性]
				点列$(x_n)_{n=1}^{\infty}$は有向点族であるから,
				必要性の証明は(2)に対して示せばよい.
				$(x_\lambda)_{\lambda \in \Lambda}$を$x$に収束する有向点族とする.
				$f$が$x$で連続であるとき,$f(x)$の任意の近傍$V$に対して或る
				$x$の近傍$U$が存在し
				\begin{align}
					f(U) \subset V
				\end{align}
				が満たされる.この$U$に対し或る$\lambda_0 \in \Lambda$が存在して
				$x_\lambda \in U\ (\forall \lambda \geq \lambda_0)$が成り立ち
				$f(x_\lambda) \in V\ (\forall \lambda \geq \lambda_0)$が従う.
				
			\item[十分性] (1)と(2)それぞれの場合について,対偶を証明する.
				いま,$f$が$x$で連続ではないと仮定する.
				\begin{description}
					\item[(1)]
						$x$に対し可算基本近傍系$(U_n)_{n=1}^{\infty}$が存在する.
						近傍系は有限回の交演算で閉じるから
						\begin{align}
							W_n \coloneqq U_1 \cap U_2 \cap \cdots \cap U_n,
							\quad (n=1,2,\cdots)
						\end{align}
						により$x$の単調減少な可算近傍系$(W_n)_{n=1}^{\infty}$が定まり,
						仮定より$f(x)$の或る近傍$V$が存在して
						\begin{align}
							f(W_n) \not\subset V,
							\quad (n=1,2,\cdots)
						\end{align}
						が成り立つから,各$n$に対し$f(x_n) \notin V$を満たす
						$x_n \in W_n$が取れる.ゆえに$f(x_n) \not\rightarrow f(x)$であるが,
						一方で$x$の任意の近傍$U$に対し或る
						$U_{n_0}$が$U$に含まれ,
						$x_n \in W_n \subset U\ (\forall n \geq n_0)$が従い
						$(x_n)_{n=1}^{\infty}$は$x$に収束する.
					\item[(2)]
						$x$の近傍全体を$\Lambda$とおけば,
						$U \leq V \DEF U \supset V\ (U,V \in \Lambda)$により
						$\Lambda$は有向集合となる.
						仮定より$f(x)$の或る近傍$V$が存在して
						全ての$U \in \Lambda$に対し或る$x_U \in U$が存在して
						$f(x_U) \notin V$を満たすから,
						$$
				\end{description}
		\end{description}
	\end{prf}
	
	\subsection{位相線型空間 (Rudin note)}
	\begin{screen}
		\begin{thm}[多変数連続写像は一変数写像として連続]
		\label{thm:multivariable_continuous_mapping_is_one_variable_continuous}
			$\Lambda$を任意濃度の空でない集合とし,
			$\left( (S_\lambda,\tau_\lambda) \right)_{\lambda \in \Lambda}$を位相空間の族とする.
			
		\end{thm}
	\end{screen}
	
	以降扱う線型空間はすべて体$\Phi (=\C,\R)$をスカラーとして考え,位相はEuclid距離による距離位相を導入する.
	
	\begin{screen}
		\begin{dfn}[位相線型空間]\label{def:topological_vector_space}
			$\Phi$上の線型空間$X$で定められる位相$\tau$が
			\begin{description}
				\item[(tvs1)] $X \times X \ni (x,y) \longmapsto x+y \in X$
					及び$\Phi \times X \ni (\alpha,x) \longmapsto \alpha x \in X$
					が$\tau$及びその直積位相に関し連続である.
				\item[(tvs2)]
					$(X,\tau)$は$T_1$位相空間である.
			\end{description}
			を満たすとき線型位相(vector topology)と呼び,
			$(X,\tau)$を位相線型空間(topological vector space)と呼ぶ.
		\end{dfn}
	\end{screen}
	
	\begin{screen}
		\begin{thm}[位相線型空間は$T_3$]
		\end{thm}
	\end{screen}
	
	\begin{screen}
		\begin{dfn}[平行移動不変距離]
			線型空間$X$上に定まる距離$d$が
			\begin{align}
				d(x+z, y+z) = d(x,y),\quad (\forall x,y,z \in X)
			\end{align}
			を満たすとき,$d$を平行移動不変距離(invariant metric)と呼ぶ.
			平行移動不変距離$d$がさらに
			\begin{align}
				d(\alpha x, \alpha y) = |\alpha| d(x,y),
				\quad (\forall \alpha \in \Phi,\ x,y \in X)
			\end{align}
			を満たすとき$d$は斉次的である(homogeneous)という.
			例えば$X$にノルム$\Norm{\cdot}{}$が定まっている場合,
			$d(x,y) \coloneqq \Norm{x-y}{}$により定まる距離$d$は斉次的かつ平行移動不変である.
		\end{dfn}
	\end{screen}
	
	\begin{screen}
		\begin{thm}[斉次的な平行移動不変距離による距離位相は線型位相]
			$X$を線型空間とする.$X$において斉次的な平行移動不変距離$d$が存在するとき,
			$d$で導入する距離位相は線型位相となる.
		\end{thm}
	\end{screen}
	
	\begin{prf}
		距離位相は$T_4$位相空間を定めるから$X$は定義\ref{def:topological_vector_space}の(tvs2)を満たす.また
		\begin{align}
			d(x+y,x'+y') \leq d(x+y,x'+y) + d(x'+y,x'+y') = d(x,x') + d(y,y')
		\end{align}
		より加法の連続性が得られ,
		\begin{align}
			d(\alpha x, \alpha'x') &\leq d(\alpha x, \alpha'x) + d(\alpha'x,\alpha'x') \\
			&= d((\alpha - \alpha') x, 0) + |\alpha'|d(x,x')
			= |\alpha-\alpha'|d(x,0) + |\alpha'|d(x,x')
		\end{align}
		よりスカラ倍の連続性も出る.
		\QED
	\end{prf}
	
	\begin{screen}
		\begin{thm}[平行移動・スカラ倍の連続性]\label{thm:continuity_of_translations_multiples}
			$(X,\tau)$を位相線型空間とするとき,任意の$a \in X$に対し
			\begin{align}
				X \ni x \longmapsto a + x \in X,
				\quad \Phi \ni \alpha \longmapsto \alpha a \in X
			\end{align}
			はいずれも連続である.同様に任意の$\beta \in \Phi$に対し
			$X \ni x \longmapsto \beta x$もまた連続である.
		\end{thm}
	\end{screen}
	
	\begin{prf}
		定理\ref{thm:multivariable_continuous_mapping_is_one_variable_continuous}より従う.
		\QED
	\end{prf}
	
	\begin{screen}
		\begin{thm}[位相線型空間の連結性]\label{thm:topological_vector_spaces_connected}
			位相線型空間は連結である.
		\end{thm}
	\end{screen}
	
	\begin{prf}
		零元のみの空間は密着空間であるから連結である.
		$X \neq \{0\}$を位相線型空間とするとき,任意に$a,b \in X$を取り
		\begin{align}
			f:[0,1] \ni t \longmapsto a + t(b - a) \in X
		\end{align}
		と定めれば$f$は$[0,1]$から$X$への連続写像である.実際,
		定理\ref{thm:continuity_of_translations_multiples}より
		$\Phi \ni t \longmapsto t(b-a)$が連続であるから
		\begin{align}
			g:[0,1] \ni t \longmapsto t(b-a)
		\end{align}
		は$[0,1]$の相対位相に関して連続であり,かつ$h:X \ni x \longmapsto a + x$もまた連続であるから
		$f = h \circ g$の連続性が従う.
		よって$X$は弧状連結であるから定理\ref{thm:connected_path_connected}より連結である.
		\QED
	\end{prf}
	
	\begin{screen}
		\begin{dfn}[位相線形空間の有界集合]
			$X$を位相線型空間,$E$を$X$の部分集合とする.0の任意の近傍$V$に対し
			或る$s = s(V) > 0$が存在して
			\begin{align}
				E \subset t V, \quad (\forall t > s)
			\end{align}
			となるとき,$E$は有界であるという.
		\end{dfn}
	\end{screen}
	
	\begin{screen}
		\begin{thm}
		\end{thm}
	\end{screen}
	
	位相線形空間$(X,\tau)$に対し,その部分集合$Y$上の相対位相を$\tau_Y$と書き,
	また$X$が或る距離$d$で距離付け可能なとき,
	$d$により導入する位相を$\tau_d$と書く.位相$\tau$に関する開集合,閉集合,近傍,
	Cauchy列は$\tau$-開集合(resp. 閉集合,近傍,Cauchy列)と書く.
	
	\begin{screen}
		\begin{dfn}[局所基・局所凸・局所コンパクト・局所有界]
			$(X,\tau)$を位相線型空間とする.
			\begin{description}
				\item[(1)] $0 \in X$の基本近傍系を$X$の局所基(local base)と呼ぶ.
				\item[(2)] すべての元が凸集合であるような局所基が取れるとき,$X$は局所凸(locally convex)であるという.
				\item[(3)] $0 \in X$がコンパクトな近傍を持つとき,$X$は局所コンパクト(locally compact)であるという.
				\item[(4)] $0 \in X$が有界な近傍を持つとき,$X$は局所有界(locally bounded)であるという.
			\end{description}
		\end{dfn}
	\end{screen}
	
	\begin{screen}
		\begin{thm}[局所基は平行移動により任意の点の近傍系となる]
			$X$を位相線型空間とするとき以下が成り立つ:
			\begin{description}
				\item[(1)] 任意の$x \in X$に対し$\Set{x + V}{V \in \mathscr{B}}$は
					$x$の近傍系となる.
				\item[(2)] $x$の近傍系$\mathbb{V}(x)$に対し$\mathscr{B} = \Set{-x + V}{V \in \mathbb{V}(x)}$
					となる.
				\item[(3)] $X$が局所コンパクト空間であるとき,任意の点はコンパクトな近傍を持つ. 
 			\end{description}
		\end{thm}
	\end{screen}
	
	\begin{screen}
		\begin{dfn}[$F$-空間・Frechet空間・ノルム空間]
			$(X,\tau)$を位相線型空間とする.
			$d$により$X$が距離化可能でかつ完備距離空間となるとき,
			$X$を$F$-空間と呼ぶ.局所凸な$F$-空間をFrechet空間と呼び
		\end{dfn}
	\end{screen}
	
	\begin{screen}
		\begin{thm}[部分空間が$F$-空間なら閉]
			$(X,\tau)$を位相線形空間,$Y \subset X$を部分空間とする.
			このとき$Y$が$F$-空間なら$Y$は$\tau$-閉である.
		\end{thm}
	\end{screen}
	
	\begin{prf}
		$Y$に対し或る平行移動不変な距離$d$が存在して$\tau_Y = \tau_d$を満たす.
		このとき
		\begin{align}
			B_{1/n} \coloneqq \Set{y \in Y}{d(y,0) < \frac{1}{n}},
			\quad n=1,2,\cdots
		\end{align}
		で$\tau_Y$-開集合を定めれば,$B_{1/n}$は$0$を含むから
		或る0の$\tau$-近傍$U_n$が存在して
		\begin{align}
			B_{1/n} = Y \cap U_n, \quad n=1,2,\cdots
		\end{align}
		を満たす.
	\end{prf}
	
	\begin{screen}
		\begin{dfn}[集合の線型演算]
			$X$を体$\Phi$上の位相線型空間,$A,B$を$X$の部分集合,$\alpha,\beta \in \Phi$とする.
			このとき
			\begin{align}
				\alpha A + \beta B \coloneqq \Set{\alpha a + \beta b}{a \in A,\ b \in B}
			\end{align}
			と書く.
		\end{dfn}
	\end{screen}
	
	\begin{screen}
		\begin{thm}
			$X$を位相線型空間,$A,B$を部分集合とする.
			\begin{description}
				\item[(1)] $\alpha \overline{A} = \overline{\alpha A}$
				\item[(2)] $\alpha (A^{\mathrm{o}}) = (\alpha A)^{\mathrm{o}}$
			\end{description}
		\end{thm}
	\end{screen}
	
	\begin{prf}\mbox{}
		\begin{description}
			\item[(1)] $\alpha = 0$或は$A = \emptyset$の場合は両辺が
				$\{0\}$或は$\emptyset$となって等号が成立する.
				$\alpha \neq 0$かつ$A \neq \emptyset$の場合,
				\begin{align}
					x \in \alpha \overline{A}
					\quad &\Longleftrightarrow \quad
					\alpha^{-1}x \in \overline{A} \\
					\quad &\Longleftrightarrow \quad
					\left(\alpha^{-1}x + V\right) \cap A \neq \emptyset, \quad 
						(\mbox{$\forall V$: neighborhood of 0}) \\
					\quad &\Longleftrightarrow \quad
					\left(x + V\right) \cap \alpha A \neq \emptyset, \quad 
						(\mbox{$\forall V$: neighborhood of 0}) \\
					\quad &\Longleftrightarrow \quad
					x \in \overline{\alpha A}
				\end{align}
				が成り立つ.
				
			\item[(2)] $\alpha = 0$或は$A = \emptyset$の場合は両辺が
				$\{0\}$或は$\emptyset$となって等号が成立する.
				$\alpha \neq 0$かつ$A \neq \emptyset$の場合,
				\begin{align}
					x \in \alpha (A^{\mathrm{o}})
					\quad &\Longleftrightarrow \quad
					\alpha^{-1}x \in A^{\mathrm{o}} \\
					\quad &\Longleftrightarrow \quad
					\mbox{$\exists V$: neighborhood of 0},\quad \alpha^{-1}x + V \subset A \\
					\quad &\Longleftrightarrow \quad
					\mbox{$\exists V$: neighborhood of 0},\quad x + V \subset \alpha A \\
					\quad &\Longleftrightarrow \quad
					x \in (\alpha A)^{\mathrm{o}}
				\end{align}
				が成り立つ.
				
		\end{description}
	\end{prf}
	
	\begin{screen}
		\begin{dfn}[位相線型空間における同程度連続性]
			$X,Y$を位相線形空間,$\mathscr{F}$を$X$から$Y$への連続線型写像の族とする.
			このとき,$\mathscr{F}$が同程度連続であるとは,$0 \in Y$の任意の近傍$V$に対し
			\begin{align}
				f(U) \subset V,\quad (\forall f \in \mathscr{F})
			\end{align}
			を満たす$0 \in X$の近傍$U$が存在することである.
		\end{dfn}
	\end{screen}
	
	\begin{screen}
		\begin{thm}[同程度連続な写像族の有界性]
			$X,Y$を位相線形空間,$\mathscr{F}$を$X$から$Y$への連続線型写像の族とする.
			$\mathscr{F}$が同程度連続であるとき,
		\end{thm}
	\end{screen}
	
	\begin{screen}
		\begin{thm}[Banach-Steinhaus]
			
		\end{thm}
	\end{screen}
	
	\begin{screen}
		\begin{thm}[開写像原理]
			$X$
		\end{thm}
	\end{screen}

\chapter{垣田高夫「シュワルツ超関数入門」だんだんわからなくなってきたからメモ}
	$\N = \{0,1,2,\cdots\},\ \Z_+ = \{1,2,\cdots\}$とする.
	また複素線形空間$E_i\ (i=1,\cdots,n)$の直積$E_1 \times \cdots \times E_n$において
	\begin{align}
		(e_1,\cdots,e_n) + (f_1,\cdots,f_n) \coloneqq (e_1 + f_1,\cdots,e_n + f_n), 
		\quad \alpha (e_1,\cdots,e_n) \coloneqq (\alpha e_1,\cdots,\alpha e_n)
	\end{align}
	により線型演算を定め,ノルム空間$(X_i,\Norm{\cdot}{i})\ (i=1,\cdots,n)$の
	直積空間$X_1 \times \cdots \times X_n$におけるノルムは
	$\Norm{x_1}{1} + \cdots + \Norm{x_n}{n}\ (x_i \in X_i)$とする.
	以降,線型空間は全て複素係数である.
	\section{あんまり意味ないメモ}
	$\Omega \neq \emptyset$を$\R^n$の開集合とする.
	$\Omega$に含まれるコンパクト集合を台とする$C^\infty$-級関数$(\Omega \longrightarrow \C)$を($\Omega$上の){\bf テスト関数(test function)}と呼び,
	その全体を$\Test{\Omega}$で表す.以後考察対象となるデルタ近似関数は$\R^n$上のテスト関数である.
	\begin{align}
		f(t) \coloneqq 
		\begin{cases}
			\exp{-\frac{1}{t}} & (t > 0), \\
			0 & (t \leq 0)
		\end{cases}
	\end{align}
	により$\R$上無限回微分可能な$f$を定めれば,
	\begin{align}
		g(t) \coloneqq f(1-t)f(1+t) = 
		\begin{cases}
			\exp{-\frac{2}{1-t^2}} & (|t| < 1), \\
			0 & (|t| \geq 1)
		\end{cases}
	\end{align}
	で定める$g$もまたLibnizの公式より$\R$上で無限回微分可能である.指数関数は0を取りえないから
	\begin{align}
		\Set{t \in \R}{g(t) \neq 0} = \Set{t \in \R}{|t| < 1}
	\end{align}
	が成り立つ.$\R^n \ni x \longmapsto |x|^2$もまた無限回微分可能であるから
	\begin{align}
		h(x) \coloneqq g(|x|^2) = g(x_1^2+ \cdots + x_n^2),
		\quad (\forall x=(x_1,\cdots,x_n) \in \R^n)
	\end{align}
	とおけば$h \in \Test{\R^n}$が満たされ,
	\begin{align}
		c \coloneqq \int_{\R^n} h(x)\ dx > 0
	\end{align}
	に対し$\rho \coloneqq (1/c)h$で定める$\rho \in \Test{\R^n}$は
	\begin{align}
		\supp{\rho} = \Set{x \in \R^n}{|x| \leq 1}
	\end{align}
	かつ
	\begin{align}
		\int_{\R^n} \rho(x)\ dx = 1
	\end{align}
	を満たす.
	
	\begin{screen}
		\begin{dfn}[デルタ近似関数]
			Diracのデルタ関数を近似する関数をデルタ近似関数と呼び,任意の$\epsilon > 0$に対し
			\begin{align}
				\rho_\epsilon(x) \coloneqq \epsilon^n \rho(x/\epsilon),
				\quad (\forall x \in \R^n)
			\end{align}
			により定めるテスト関数$\rho_\epsilon$はデルタ近似関数である.
		\end{dfn}
	\end{screen}
	
	\begin{screen}
		\begin{thm}[デルタ近似関数$\rho_\epsilon$の性質]\mbox{}
			\begin{description}
				\item[(1)] $\rho_\epsilon \geq 0$.
				\item[(2)] $\check{\rho}_\epsilon = \rho_\epsilon$.
				\item[(3)] $\supp{\rho_\epsilon} = \Set{x \in \R^n}{|x| \leq \epsilon}$.
				\item[(4)] $\int_{\R^n} \rho_\epsilon(x)\ dx = 1$.
			\end{description}
		\end{thm}
	\end{screen}
	
	\begin{screen}
		\begin{thm}[テスト関数を任意に構成する]
		\label{construction_of_test_function}
			任意に$\R^n$のコンパクト集合$K$と開集合$U\ (K \subset U)$を取れば,
			\begin{align}
				0 \leq \eta \leq 1,
				\quad 
				\eta(x) =
				\begin{cases}
					1 & (x \in K), \\
					0 & (x \in \R^n \backslash U) 
				\end{cases}
			\end{align}
			を満たす$\eta \in \Test{\R^n}$が存在する.
		\end{thm}
	\end{screen}
	
	\begin{prf}
		任意の$x \in K$に対し
		$B(x;\epsilon_x) \coloneqq \Set{y \in \R^n}{|x-y| < \epsilon_x} \subset U$を満たす$\epsilon_x > 0$が存在し,
		コンパクト性から有限個の$x_1,\cdots,x_m \in K$により$K \subset B(x_1,\epsilon_1/3) \cup \cdots \cup B(x_m,\epsilon_m/3)$が成り立つ.
		ここで$\epsilon \coloneqq \min{}{\left\{ \epsilon_{x_1}/3,\cdots,\epsilon_{x_m}/3 \right\}}$とおけば
		\begin{align}
			\eta(x) \coloneqq \rho_\epsilon \ast \defunc_K(x) = \int_{K} \rho_\epsilon(x - y)\ dy,
			\quad (\forall x \in \R^n)
		\end{align}
		が求めるテスト関数である.先ず任意の$\alpha \in \N^n$に対して
		\begin{align}
			\partial^\alpha \eta(x) = \int_K \partial^\alpha \rho_\epsilon(x-y)\ dy,
			\quad (\forall x \in \R^n)
		\end{align}
		が成り立つから$\eta$は$\R^n$上で無限回微分可能であり,
		$x \in \R^n \backslash U$なら
		$|x - y| > \epsilon\ (\forall y \in K)$より$\eta(x) = 0$が従う.
		\begin{align}
			\eta(x) = \int_K \rho_\epsilon(x-y)\ dy \leq \int_{\R^n} \rho_\epsilon(x-y)\ dy = 1
		\end{align}
		より$0 \leq \eta \leq 1$が成り立ち,また
	\end{prf}
	\section{緩増加$C^m$-関数}
	\begin{screen}
		\begin{dfn}[緩増加$C^m$-関数]
			$f:\R^n \longrightarrow \C$が$C^m$-級で,各$\alpha \in \N^n$に対して
			或る定数$C_\alpha$と$\ell_\alpha \in \N$が存在し
			\begin{align}
				\left| \partial^\alpha f(x) \right| \leq C_\alpha(1+|x|^2)^{\ell_\alpha},
				\quad (\forall x \in \R^n,\ |\alpha| \leq m)
				\label{eq:tempered_c^m_function}
			\end{align}
			が成り立つとき,$f$を緩増加$C^m$-関数(tempered $C^m$-function)という.
			$m=0$の場合$f$を緩増加連続関数と呼び,
			$m=\infty$の場合は緩増加関数と呼ぶ.
			また緩増加連続関数,緩増加関数の全体のなす線形空間を$\mathscr{O}_C,\ \mathscr{O}_M$で表す.
		\end{dfn}
	\end{screen}
	
	緩増加$C^m$-関数$f$に対し(\refeq{eq:tempered_c^m_function})を満たす
	$C_\alpha$と$\ell_\alpha$について,
	$C \coloneqq \max{|\alpha| \leq m}{C_\alpha}
	,\ \ell \coloneqq \max{|\alpha| \leq m}{\ell_\alpha}$とおけば
	\begin{align}
		\left| \partial^\alpha f(x) \right| \leq C(1+|x|^2)^\ell,
		\quad (\forall x \in \R^n,\ |\alpha| \leq m)
		\label{eq:tempered_c^m_function_2}
	\end{align}
	が成立する.特に緩増加関数に対しては任意の$m \in \N$ごとに$C,\ \ell$が定まり
	(\refeq{eq:tempered_c^m_function_2})が満たされる.
	
	\begin{screen}
		\begin{thm}[緩増加連続関数により定まる緩増加超関数]
		\label{thm:tempered_continuous_functions_and_tempered_distributions}
			緩増加連続関数$f:\R^n \longrightarrow \C$に対して
			\begin{align}
				u_f: \rapid{\R^n} \ni \varphi \longmapsto
				\int_{\R^n} f(x) \varphi(x)\ dx
			\end{align}
			により定める$u_f$は緩増加超関数である.またこの対応
			$\mathscr{O}_C \longrightarrow \tempdist{\R^n}\ (f \longmapsto u_f)$は
			線型単射である.
		\end{thm}
	\end{screen}
	
	\begin{prf}
		$f$に対し或る定数$C$と$\ell \in \N$が存在して(\refeq{eq:tempered_c^m_function})が満たされ
		\begin{align}
			\int_{\R^n} |f(x)| |\varphi(x)|\ dx
			\leq C \int_{\R^n} (1+|x|^2)^\ell |\varphi(x)|\ dx
			\leq \left[ C \int_{\R^n} \frac{1}{(1+|x|^2)^n}\ dx \right] p_{m+\ell}(\varphi) 
		\end{align}
		が成立する.可積分性より$u_f$は線型性をもち,また半ノルム$p_{m+\ell}$で抑えられているから$u_f$の連続性も出る.
		上の可積分性より$f \longmapsto u_f$の線型性も従い,
		単射であることは変分法の基本補題より得られる.
		\QED
	\end{prf}
	\section{緩増加超関数の構造定理}
	\begin{screen}
		\begin{lem}\label{lem:isomorphism_on_product_of_dual_spaces}
			$\Lambda$を有限集合,$\Bigl( (X_\lambda,\Norm{\cdot}{\lambda}) \Bigr)_{\lambda \in \Lambda}$
			をノルム空間の系とする.このとき,$f_\lambda \in X^*_\lambda\ (\lambda \in \Lambda)$
			を取り
			\begin{align}
				F(x) \coloneqq \sum_{\lambda \in \Lambda} f_\lambda(x_\lambda),
				\quad \biggl(\forall x = (x_\lambda)_{\lambda \in \Lambda} \in \prod_{\lambda \in \Lambda} X_\lambda \biggr)
				\label{eq:lem_isomorphism_on_product_of_dual_spaces}
			\end{align}
			により線型汎関数$F$を定めれば$F \in \left( \prod_{\lambda \in \Lambda} X_\lambda \right)^*$
			が満たされる.そしてこの対応により定まる次の写像
			\begin{align}
				W:\prod_{\lambda \in \Lambda} X_\lambda^* \ni f = (f_\lambda)_{\lambda \in \Lambda}
				\longmapsto F \in \Biggl( \prod_{\lambda \in \Lambda} X_\lambda \Biggr)^*
			\end{align}
			は線型・位相同型である.
		\end{lem}
	\end{screen}
	
	\begin{prf}
		$X_\lambda^*,\prod_{\lambda \in \Lambda} X_\lambda^*,\left( \prod_{\lambda \in \Lambda} X_\lambda \right)^*$におけるノルムをそれぞれ
		$\Norm{\cdot}{X_\lambda^*},\Norm{\cdot}{\prod_{\lambda \in \Lambda} X_\lambda^*},\Norm{\cdot}{\left( \prod_{\lambda \in \Lambda} X_\lambda \right)^*}$と表す.
		先ず(\refeq{eq:lem_isomorphism_on_product_of_dual_spaces})において,
		\begin{align}
			|F(x)| \leq \sum_{\lambda \in \Lambda} |f_\lambda(x_\lambda)|
			\leq \max{\lambda \in \Lambda}{\Norm{f_\lambda}{X_\lambda^*}} \sum_{\lambda \in \Lambda} \Norm{x_\lambda}{\lambda}
			\label{eq:lem_isomorphism_on_product_of_dual_spaces_2}
		\end{align}
		となるから$F \in \left( \prod_{\lambda \in \Lambda} X_\lambda \right)^*$を得る.
		次に$W$が全単射であることを示す.実際,任意の$G \in \left( \prod_{\lambda \in \Lambda} X_\lambda \right)^*$に対して
		\begin{align}
			x^{(\lambda)}_\nu \coloneqq 
			\begin{cases}
				x_\lambda & (\nu = \lambda) \\
				0 & (\nu \neq \lambda)
			\end{cases},
			\quad g_\lambda(x_\lambda) \coloneqq G(x^{(\lambda)})
			\quad (\forall x_\lambda \in X_\lambda)
		\end{align}
		と定めれば$g_\lambda \in X_\lambda^*$が成り立つから$W$は全射であり,
		また$f,g \in \prod_{\lambda \in \Lambda} X_\lambda^*$に対し$Wf = Wg$が満たされているとき,
		\begin{align}
			f_\lambda(x_\lambda) = (Wf)(x^{(\lambda)})
			= (Wg)(x^{(\lambda)}) = g(x_\lambda),
			\quad (\forall x_\lambda \in X_\lambda,\ \forall \lambda \in \Lambda)
		\end{align}
		が従い$W$の単射性が出る.$W$の線型性は
		\begin{align}
			&W(\alpha f + \beta g)(x)
			= \sum_{\lambda \in \Lambda} (\alpha f_\lambda + \beta g_\lambda) (x_\lambda) \\
			&\qquad = \alpha \sum_{\lambda \in \Lambda} f_\lambda(x_\lambda)
				+ \beta \sum_{\lambda \in \Lambda} g_\lambda(x_\lambda)
			= (\alpha W f + \beta W g)(x),
			\quad (\forall x=(x_\lambda),\ f=(f_\lambda),g=(g_\lambda),\ \alpha,\beta \in \C)
		\end{align}
		により得られ,かつ(\refeq{eq:lem_isomorphism_on_product_of_dual_spaces_2})より
		\begin{align}
			\Norm{Wf}{\left( \prod_{\lambda \in \Lambda} X_\lambda \right)^*}
			\leq \max{\lambda \in \Lambda}{\Norm{f_\lambda}{X_\lambda^*}}
			\leq \Norm{f}{\prod_{\lambda \in \Lambda} X_\lambda^*},
			\quad \biggl( \forall f = (f_\lambda)_{\lambda \in \Lambda} \in \prod_{\lambda \in \Lambda} X_\lambda^* \biggr)
		\end{align}
		が成り立つから$W$は連続であり,開写像定理より$W^{-1}$もまた連続である.
		\QED
	\end{prf}
	
	\begin{screen}
		\begin{lem}\label{lem:tempered_distributions_continuity}
			任意の$u \in \tempdist{\R^n}$に対して或る$c = c(u) > 0$と$m = m(u) \in \N$が存在し次を満たす:
			\begin{align}
				|\inprod<u,\varphi>| \leq c p_m(\varphi),
				\quad (\forall \varphi \in \rapid{\R^n}).
			\end{align}
		\end{lem}
	\end{screen}
	
	\begin{prf}
		背理法で証明する.主張が満たされない場合,
		任意の$k \in \Z_+$に対して或る$\varphi_k \in \rapid{\R^k}$が存在し
		\begin{align}
			|\inprod<u,\varphi_k>| > k p_k(\varphi_k)
		\end{align}
		が成立するから,$\psi_k \coloneqq \varphi_k / [k p_k(\varphi_k)]\ (k \in \Z_+)$とおけば
		\footnote{
			$|\inprod<u,\varphi_k>| > 0$より$\varphi_k$は零写像ではない.従って$\rapid{\R^n}$の半ノルム
			$p_k$の定義より$p_k(\varphi_k) > 0$が満たされている.
		}
		\begin{align}
			|\inprod<u,\psi_k>| = \frac{|\inprod<u,\varphi_k>|}{k p_k(\varphi_k)} > 1
			\quad (\forall k \in \Z_+)
			\label{eq:lem_tempered_distributions_continuity}
		\end{align}
		が従う.一方で半ノルム系$(p_m)_{m \in \N}$は$p_0 \leq p_1 \leq p_2 \leq \cdots$を満たすから,
		任意の$m \in \N$に対して
		\begin{align}
			p_m(\psi_k) = \frac{p_m(\varphi_k)}{k p_k(\varphi_k)} \leq \frac{1}{k}
			\longrightarrow 0 \quad (k \longrightarrow \infty)
		\end{align}
		が成り立ち,$u$の連続性から$\inprod<u,\psi_k> \longrightarrow \inprod<u,0> = 0$となるはずであるが,
		これは(\refeq{eq:lem_tempered_distributions_continuity})と矛盾する.
		\QED
	\end{prf}
	
	\begin{screen}
		\begin{thm}[緩増加超関数の構造定理]\label{thm:structure_theorem_of_tempered_distributions}
			$u \in \tempdist{\R^n}$とし,補題\ref{lem:tempered_distributions_continuity}
			により対応する$m \in \N$を取る.また以下では$\alpha$は$n$次元多重指数を表すものとする.
			このとき或る系$(g_\alpha)_{|\alpha| \leq m} \subset \mathrm{L}^\infty(\R^n)$が存在して次を満たす:
			\begin{align}
				\inprod<u,\varphi> =
				\sum_{|\alpha| \leq m} \int_{\R^n} (1+|x|^2)^m g_\alpha(x) \left[ \partial_1 \cdots \partial_n \partial^\alpha \varphi(x) \right]\ dx,
				\quad (\forall \varphi \in \rapid{\R^n}).
			\end{align}
		\end{thm}
	\end{screen}
	
	\begin{prf}以下$u \in \tempdist{\R^n}$と$c>0,\ m \in \N$は固定する.
		\begin{description}
			\item[第一段]
				任意の$\varphi \in \rapid{\R^n}$に対して
				\begin{align}
					p_m(\varphi)
					\leq (m+1) \sum_{|\alpha| \leq m} \int_{\R^n} (1+|y|^2)^m \left| \partial_1 \cdots \partial_n\partial^\alpha \varphi(y) \right|\ dy
					\label{eq:structure_theorem_of_tempered_distributions_1}
				\end{align}
				が成り立つことを示す.
				任意の$\alpha \in \N^n$と$x = (x_1,\cdots,x_n) \in \R^n$に対して,
				$x_j \geq 0$を満たす$1 \leq j \leq n$の個数を$\#$で表し,
				$x_j < 0$なら積分範囲$I_j$を$(-\infty,x_j]$,$x_j \geq 0$なら$I_j = [x_j,\infty)$とすれば
				\begin{align}
					\partial^\alpha \varphi(x)
					= (-1)^{\#} \int_{I_1}\cdots\int_{I_n} \partial_1 \cdots \partial_n\partial^\alpha \varphi(y)\ dy_n \cdots dy_1
				\end{align}
				が成り立つ.そして$y \in I_1 \times \cdots \times I_n$なら$|x| \leq |y|$が満たされるから,
				$0 \leq k \leq m$に対し
				\begin{align}
					(1+|x|^2)^k \left| \partial^\alpha \varphi(x) \right|
					&\leq \int_{I_1}\cdots\int_{I_n} (1+|x|^2)^k \left| \partial_1 \cdots \partial_n\partial^\alpha \varphi(y) \right|\ dy_n \cdots dy_1 \\
					&\leq \int_{I_1}\cdots\int_{I_n} (1+|y|^2)^k \left| \partial_1 \cdots \partial_n\partial^\alpha \varphi(y) \right|\ dy_n \cdots dy_1 \\
					&\leq \int_{\R^n} (1+|y|^2)^k \left| \partial_1 \cdots \partial_n\partial^\alpha \varphi(y) \right|\ dy \\
					&\leq \int_{\R^n} (1+|y|^2)^m \left| \partial_1 \cdots \partial_n\partial^\alpha \varphi(y) \right|\ dy
				\end{align}
				が従い
				\begin{align}
					p_m(\varphi) &= \sum_{|\alpha|+k \leq m} \sup{x \in \R^n}{(1+|x|^2)^k \left| \partial^\alpha \varphi(x) \right|} \\
					&\leq \sum_{|\alpha|+k \leq m} \int_{\R^n} (1+|y|^2)^m \left| \partial_1 \cdots \partial_n\partial^\alpha \varphi(y) \right|\ dy \\
					&\leq (m+1) \sum_{|\alpha| \leq m} \int_{\R^n} (1+|y|^2)^m \left| \partial_1 \cdots \partial_n\partial^\alpha \varphi(y) \right|\ dy
				\end{align}
				を得る.
				
			\item[第二段]
				$\varphi \in \rapid{\R^n}$に対し,
				$\psi^\varphi_\alpha(y) \coloneqq (1+|y|^2)^m \partial_1 \cdots \partial_n\partial^\alpha \varphi(y)$により$\psi^\varphi = (\psi^\varphi_\alpha)_{|\alpha| \leq m}$を定める.このとき
				\begin{align}
					\Delta \coloneqq \Set{\psi^\varphi = (\psi^\varphi_\alpha)_{|\alpha| \leq m}}{\varphi \in \rapid{\R^n}}
				\end{align}
				で定める$\Delta$は対応$\varphi \longmapsto \psi^\varphi$により$\rapid{\R^n}$と線型同型となる.実際,この写像の線型性は微分の性質から従い,
				$\Delta$の作り方より全射である.また
				$\varphi,\eta \in \rapid{\R^n}$に対して,
				$(\psi^\varphi_\alpha)_{|\alpha| \leq m} = (\psi^\eta_\alpha)_{|\alpha| \leq m}$
				ならば
				\begin{align}
					(1+|y|^2)^m \partial_1 \cdots \partial_n \varphi(y)
					= (1+|y|^2)^m \partial_1 \cdots \partial_n \eta(y),
					\quad (\forall y \in \R^n)
				\end{align}
				が従い
				\begin{align}
					&\varphi(x)
					= \int_{-\infty}^{x_1}\cdots\int_{-\infty}^{x_n} \partial_1 \cdots \partial_n \varphi(y)\ dy_n\cdots dy_1 \\
					&\qquad = \int_{-\infty}^{x_1}\cdots\int_{-\infty}^{x_n} \partial_1 \cdots \partial_n \eta(y)\ dy_n\cdots dy_1
					= \eta(x),
					\quad (\forall x \in \R^n)
				\end{align}
				が得られるから$\varphi \longmapsto \psi^\varphi$は単射である.
				いま,
				\begin{align}
					\Norm{\psi^\varphi}{\Delta}
					\coloneqq \sum_{|\alpha| \leq m} \int_{\R^n} \left| \psi^\varphi_\alpha(y) \right|\ dy
					= \sum_{|\alpha| \leq m} \int_{\R^n} (1+|y|^2)^m \left| \partial_1 \cdots \partial_n\partial^\alpha \varphi(y) \right|\ dy
					\label{eq:structure_theorem_of_tempered_distributions_2}
				\end{align}
				として$\Norm{\cdot}{\Delta}$を定めれば,
				$\psi^\varphi_\alpha$の連続性よりこれは$\Delta$上のノルムとなる.
				
			\item[第三段]
				ノルム空間$\Delta$上の線型汎関数$L$を
				\begin{align}
					L\psi^\varphi
					\coloneqq \inprod<u,\varphi>,
					\quad (\forall \varphi \in \rapid{\R^n})
				\end{align}
				により定めれば,$L$は連続である.
				実際(\refeq{eq:structure_theorem_of_tempered_distributions_1})と
				(\refeq{eq:structure_theorem_of_tempered_distributions_2})より
				\begin{align}
					\left| L\psi^\varphi \right|
					\leq c (m+1) \Norm{\psi^\varphi}{\Delta},
					\quad (\forall \varphi \in \rapid{\R^n})
				\end{align}
				が成立する.
				
			\item[第四段]
				$\psi^\varphi = (\psi^\varphi_\alpha)_{|\alpha| \leq m}$に対し,
				各成分$\psi^\varphi_\alpha$にこれを代表とする$\mathrm{L}^1(\R^n)$の元を対応させる等長な線型単射を
				$U$と表す:
				\begin{align}
					U:\Delta \ni \psi^\varphi \longmapsto 
					U \psi^\varphi \in \prod_{|\alpha| \leq m} \mathrm{L}^1(\R^n).
				\end{align}
				$U$の値域をその像に制限すれば逆写像$U^{-1}$が存在し,
				合成写像$LU^{-1}$は部分空間$U\Delta$上で線型連続であるからHahn-Banachの拡張定理より
				$LU^{-1}$のノルム保存拡張$\tilde{L} \in \left( \prod_{|\alpha| \leq m} \mathrm{L}^1(\R^n) \right)^*$が存在する.従って,
				\begin{align}
					W:\prod_{|\alpha| \leq m} \mathrm{L}^1(\R^n)^* \longrightarrow 
					\biggl( \prod_{|\alpha| \leq m} \mathrm{L}^1(\R^n) \biggr)^*
				\end{align}
				を補題\ref{lem:isomorphism_on_product_of_dual_spaces}における同型写像とすれば,
				或る$(\Phi_\alpha)_{|\alpha| \leq m} \in \prod_{|\alpha| \leq m} \mathrm{L}^1(\R^n)^*$がただ一つ存在して$\tilde{L} = W \left((\Phi_\alpha)_{|\alpha| \leq m}\right)$と表される.一方で
				それぞれの$\Phi_\alpha$には或る$g_\alpha \in \mathrm{L}^\infty(\R^n)$
				がただ一つ対応して
				\begin{align}
					\Phi_\alpha (f) = \int_{\R^n} g_\alpha(x)f(x)\ dx,
					\quad \left( \forall f \in \mathrm{L}^1(\R^n) \right)
				\end{align}
				を満たすから,(\refeq{eq:lem_isomorphism_on_product_of_dual_spaces})で定める演算により
				\begin{align}
					\inprod<u,\varphi>
					&= L\psi^\varphi
					= \tilde{L}U\psi^\varphi
					= \left[ W \left((\Phi_\alpha)_{|\alpha| \leq m}\right) \right] U\psi^\varphi \\
					&= \sum_{|\alpha| \leq m} \int_{\R^n} g_\alpha(x) \psi^\varphi_\alpha(x)\ dx
					= \sum_{|\alpha| \leq m} \int_{\R^n} g_\alpha(x) (1+|x|^2)^m \partial_1 \cdots \partial_n\partial^\alpha \varphi(x)\ dx
				\end{align}
				が成り立ち定理の主張を得る.
				\QED
		\end{description}
	\end{prf}
	
	\begin{screen}
		\begin{thm}[緩増加連続関数と緩増加超関数の一対一対応]
			緩増加連続関数$f:\R^n \longrightarrow \C$に対し
			$u_f \in \tempdist{\R^n}$を
			\begin{align}
				u_f: \rapid{\R^n} \ni \varphi \longmapsto
				\int_{\R^n} f(x) \varphi(x)\ dx
			\end{align}
			で定めるとき,
		\end{thm}
	\end{screen}
	
	\begin{prf}
		定理\ref{thm:tempered_continuous_functions_and_tempered_distributions}より
		$u_f$は緩増加超関数であり,可積分性より$f \longmapsto u_f$の線型性が出る.
		また緩増加連続関数は局所可積分であるから,変分法の基本補題より
		$f \longmapsto u_f$は単射である.後は$f \longmapsto u_f$が全射であることを示せばよい.
	\end{prf}
	
	\begin{screen}
		\begin{thm}[緩増加超関数の合成積の性質]\mbox{}
			\begin{description}
				\item[(1)] $u \in \tempdist{\R^n}$と$\varphi \in \rapid{\R^n}$の合成積
					$u \ast \varphi$に対し或る緩増加関数$f$が存在して
					$u \ast \varphi = u_f$を満たす.
			\end{description}
		\end{thm}
	\end{screen}
	
	\begin{prf}\mbox{}
		\begin{description}
			\item[(1)] 構造定理より$u$に対して或る$(g_\alpha)_{|\alpha| \leq m} \subset \mathrm{L}^\infty(\R^n)$が存在し
			\begin{align}
				\inprod<u,\varphi> =
				\sum_{|\alpha| \leq m} \int_{\R^n} (1+|x|^2)^m g_\alpha(x) \left[ \partial_1 \cdots \partial_n \partial^\alpha \varphi(x) \right]\ dx,
				\quad (\forall \varphi \in \rapid{\R^n}).
			\end{align}
			と表現できるから,Fubiniの定理より任意の$\psi \in \rapid{\R^n}$に対して
			\begin{align}
				\inprod<u \ast \varphi,\psi> &= \inprod<u, \check{\varphi} \ast \psi> \\
				&= \sum_{|\alpha| \leq m} \int_{\R^n} (1+|x|^2)^m g_\alpha(x) \left[ \partial_1 \cdots \partial_n \partial^\alpha \left(\check{\varphi} \ast \psi\right)(x) \right]\ dx \\
				&= \sum_{|\alpha| \leq m} \int_{\R^n} (1+|x|^2)^m g_\alpha(x) \left[ \partial_1 \cdots \partial_n \partial^\alpha \int_{\R^n} \varphi(y-x) \psi (y)\ dy \right]\ dx \\
				&= \sum_{|\alpha| \leq m} (-1)^{|\alpha|} \int_{\R^n} (1+|x|^2)^m g_\alpha(x) \left[ \int_{\R^n} \partial_1 \cdots \partial_n \partial^\alpha\varphi(y-x) \psi (y)\ dy \right]\ dx \\
				&= \sum_{|\alpha| \leq m} (-1)^{|\alpha|} \int_{\R^n} \left[ \int_{\R^n} (1+|x|^2)^m g_\alpha(x)\ \partial_1 \cdots \partial_n \partial^\alpha\varphi(y-x)\ dx\right] \psi(y)\ dy
			\end{align}
			が成立する.このとき,各$\alpha$に対して
			\begin{align}
				f_\alpha:\R^n \ni y \longmapsto \int_{\R^n} (1+|x|^2)^m g_\alpha(x)\ \partial_1 \cdots \partial_n \partial^\alpha\varphi(y-x)\ dx
			\end{align}
			は緩増加関数である.表記上簡単にするため$\partial_1 \cdots \partial_n \partial^\alpha\varphi$を$\eta$と書き,
			先ずは$f_\alpha$の微分可能性を示す.実際
			\begin{align}
				(1+|x|^2) \leq (1 + 2|y-x|^2 + 2|y|^2) \leq 2(1+|y-x|^2)(1+|y|^2)
			\end{align}
			であることを用いれば,任意の$1 \leq k \leq n$に対して
			\begin{align}
				(1+|x|^2)^m |g_\alpha(x)|\ |\partial_k \eta(y-x)|
				\leq 2 \Norm{g_\alpha}{\mathrm{L}^\infty(\R^n)} (1+|y|^2)^m p_{m+n+1}(\eta) \frac{1}{(1+|y-x|^2)^n},
				\quad (\forall x \in \R^n)
			\end{align}
			が成り立ち,右辺は$x$の関数として$\R^n$上可積分である.よってLebesgueの収束定理より
			\begin{align}
				\partial_k \int_{\R^n} (1+|x|^2)^m g_\alpha(x)\ \eta(y-x)\ dx
				= \int_{\R^n} (1+|x|^2)^m g_\alpha(x)\ \partial_k \eta(y-x)\ dx
			\end{align}
			が従い,帰納法により任意の$\beta \in \N^n$に対して
			\begin{align}
				\partial^\beta \int_{\R^n} (1+|x|^2)^m g_\alpha(x)\ \eta(y-x)\ dx
				= \int_{\R^n} (1+|x|^2)^m g_\alpha(x)\ \partial^\beta \eta(y-x)\ dx
			\end{align}
			が出る.そして
			\begin{align}
				\left| \partial^\beta f_\alpha(y) \right|
				&\leq \int_{\R^n} (1+|x|^2)^m |g_\alpha(x)|\ \left| \partial^\beta \eta(y-x) \right|\ dx \\
				&\leq \left[2 \Norm{g_\alpha}{\mathrm{L}^\infty(\R^n)} p_{m+n+|\beta|}(\eta) \int_{\R^n} \frac{1}{(1+|y-x|^2)^n}\ dx \right] (1+|y|^2)^m
			\end{align}
			が満たされるから$f_\alpha$は緩増加関数であり,
			\begin{align}
				f \coloneqq \sum_{|\alpha| \leq m} (-1)^{|\alpha|} f_\alpha
			\end{align}
			により緩増加関数$f$を定めれば$u \ast \varphi = u_f$が得られる.
			\QED
		\end{description}
	\end{prf}
	\section{緩増加超関数の台}
	\begin{screen}
		\begin{lem}[閉集合とコンパクト集合の和は閉]\label{lem:combination_closed_compact}
			$A,K \subset \R^n$をそれぞれ閉集合,コンパクト集合とする.このとき
			\begin{align}
				A + K \coloneqq \Set{z = x + y \in \R^n}{x \in A,\ y \in K}
			\end{align}
			で定める$A + K$は閉集合である.
		\end{lem}
	\end{screen}
	
	\begin{prf}
		$A+K$の完備性を示す.$(z_n)_{n=1}^{\infty}$を
		$A+K$のCauchy列とし,$z_n = x_n + y_n\ (x_n \in A,\ y_n \in K,\ n=1,2,\cdots)$と考える.
		\begin{description}
			\item[第一段]
				$(x_n)_{n=1}^{\infty}$がCauchy列である場合,
				$|y_n - y_m| \leq |z_n - z_m| + |x_n - x_m| \longrightarrow 0\ (n,m \longrightarrow \infty)$より
				$(y_n)_{n=1}^{\infty}$もCauchy列である.$A,K$が閉集合であるから
				$x_n \longrightarrow {}^\exists x \in A,\ y_n \longrightarrow {}^\exists y \in K$が満たされ,
				$z \coloneqq x + y \in A + K$とおけば
				\begin{align}
					|z_n - z| \leq |x_n - x| + |y_n - y| \longrightarrow 0
					\quad (n \longrightarrow \infty)
				\end{align}
				が従う.$(x_n)_{n=1}^{\infty}$と$(y_n)_{n=1}^{\infty}$の立場を交換しても同じことが言える.
			
			\item[第二段]
				$(x_n)_{n=1}^{\infty}$と$(y_n)_{n=1}^{\infty}$のどちらもCauchy列でない場合,
				$(y_n)_{n=1}^{\infty}$は有界であるから或る部分列
				$(y_{n_k})_{k=1}^{\infty}$が$K$で収束する.
				前段の結果より$(z_{n_k})_{k=1}^{\infty}$は$A + K$で収束し,
				$(z_n)_{n=1}^{\infty}$も部分列と同じ極限に収束する.
				\QED
		\end{description}
	\end{prf}
	
	\begin{screen}
		\begin{thm}[緩増加超関数$u$とデルタ近似関数の合成積は$u$に収束する]\label{thm:supp_of_tempered_distributions_1}
			$u \in \tempdist{\R^n}$とする.
			\begin{description}
				\item[(1)]	任意のテスト関数$\varphi \in \Test{\R^n}$に対し
					\begin{align}
						\supp{u \ast \varphi} \subset \supp{u} + \supp{\varphi}
						\label{eq:thm_supp_of_tempered_distributions_1}
					\end{align}
					が成り立つ.ただし$\supp{u} + \supp{\varphi} = \Set{x \in \R^n}{x = y+z,\ y \in \supp{u},\ z \in \supp{\varphi}}$である.
					
				\item[(2)] 
					$\tempdist{\R^n}$の位相に関して$u \ast \rho_\epsilon \longrightarrow u\ (\epsilon \longrightarrow 0)$が成立する.
					つまり次が成り立つ.
					\begin{align}
						\inprod<u \ast \rho_\epsilon,\varphi> \longrightarrow \inprod<u,\varphi>,
						\quad (\forall \varphi \in \rapid{\R^n}).
					\end{align}
			\end{description}
		\end{thm}
	\end{screen}
	
	\begin{prf}\mbox{}
		\begin{description}
			\item[(1)]
				$\supp{u} \neq \R^n$の場合を考える.
				補題\ref{lem:combination_closed_compact}より$\supp{u} + \supp{\varphi}$は閉集合であるから,
				$x \notin \supp{u}+\supp{\varphi}$を取れば
				$U \cap \supp{u}+\supp{\varphi} = \emptyset$を満たす開近傍$U$が存在する.
				このとき$U \subset (\supp{u \ast \varphi})^c$が成立するから,
				\begin{align}
					(\supp{u} + \supp{\varphi})^c \subset (\supp{u \ast \varphi})^c
				\end{align}
				が従い(\refeq{eq:thm_supp_of_tempered_distributions_1})が得られる.
				実際,$\supp{\psi} \subset U$を満たす任意の$\psi \in \rapid{\R^n}$について
				\begin{align}
					\check{\varphi} \ast \psi (y)
					= \int_{U} \varphi(z - y)\psi(z)\ dz
					= 0,
					\quad (\forall y \in \supp{u})
				\end{align}
				が成り立ち
				\footnote{
					任意の$y \in \supp{u}$に対して$z - y \notin \supp{\varphi}\ (\forall z \in U)$が満たされる.
				}
				\begin{align}
					\inprod<u \ast \varphi, \psi>
					= \inprod<u, \check{\varphi} \ast \psi>
					= 0
				\end{align}
				が出る.すなわち$u \ast \varphi$は$U$で0であり$U \subset (\supp{u \ast \varphi})^c$が満たされる.
				
			\item[(2)]
				任意の$\varphi \in \rapid{\R^n}$に対して
				$\inprod<u \ast \rho_\epsilon, \varphi> = \inprod<u, \check{\rho}_\epsilon \ast \varphi> = \inprod<u, \rho_\epsilon \ast \varphi>$
				が成り立つから,$|\inprod<u, \rho_\epsilon \ast \varphi - \varphi>| \longrightarrow 0\ (\epsilon \longrightarrow 0)$となることを示せばよい.
				補題\ref{lem:tempered_distributions_continuity}より$u$に対し或る$c > 0$と$m \in \N$が存在して
				\begin{align}
					|\inprod<u, \rho_\epsilon \ast \varphi - \varphi>| \leq c p_m(\rho_\epsilon \ast \varphi - \varphi),
					\quad (\forall \varphi \in \rapid{\R^n})
				\end{align}
				を満たす.
				\begin{align}
					p_m(\rho_\epsilon \ast \varphi - \varphi)
					&= \sum_{|\alpha| + k \leq m} \sup{x \in \R^n}{(1+|x|^2)^k \left| \rho_\epsilon \ast \partial^\alpha \varphi(x) - \partial^\alpha \varphi(x) \right|} \\
					&= \sum_{|\alpha| + k \leq m} 
						\sup{x \in \R^n}{(1+|x|^2)^k \left| \int_{|y| \leq \epsilon} \left( \partial^\alpha \varphi(x-y) - \partial^\alpha \varphi(x) \right) \rho_\epsilon(y)\ dy \right|} \\
					&\leq \sum_{|\alpha| + k \leq m} 
						\sup{x \in \R^n}{\int_{|y| \leq \epsilon} (1+|x|^2)^k \left| \partial^\alpha \varphi(x-y) - \partial^\alpha \varphi(x) \right| \rho_\epsilon(y)\ dy}
					\label{eq:thm_supp_of_tempered_distributions_1_2}
				\end{align}
				まで半ノルムを展開し
				\begin{align}
					\partial^\alpha \varphi(x-y) - \partial^\alpha \varphi(x)
					= \int_0^1 \frac{d}{dt} \partial^\alpha \varphi(x-ty)\ dt
					= -\sum_{j=1}^{n} y_j \int_0^1 \partial_j \partial^\alpha \varphi(x-ty)\ dt
				\end{align}
				と式変形すれば,$\epsilon < 1$なら$(1+|x|^2) \leq 2 (1+|ty|^2)(1+|x-ty|^2) \leq 4(1+|x-ty|^2)\ (|y|\leq \epsilon,\ |t| \leq 1)$より
				\begin{align}
					(\refeq{eq:thm_supp_of_tempered_distributions_1_2})
					&= \sum_{|\alpha| + k \leq m} 
						\sup{x \in \R^n}{\int_{|y| \leq \epsilon} (1+|x|^2)^k  \left| \sum_{j=1}^{n} y_j \int_0^1 \partial_j \partial^\alpha \varphi(x-ty)\ dt \right| \rho_\epsilon(y)\ dy} \\
					&\leq n \epsilon \sum_{|\alpha| + k \leq m} 
						\sup{x \in \R^n}{\int_{|y| \leq \epsilon} (1+|x|^2)^k \int_0^1 \left| \partial_j \partial^\alpha \varphi(x-ty) \right|\ dt\ \rho_\epsilon(y)\ dy} \\
					&\leq 4^m n \epsilon \sum_{|\alpha| + k \leq m} 
						\sup{x \in \R^n}{\int_{|y| \leq \epsilon} \int_0^1 (1+|x-ty|^2)^k \left| \partial_j \partial^\alpha \varphi(x-ty) \right|\ dt\ \rho_\epsilon(y)\ dy} \\
					&\leq \left( 4^m n p_{m+1}(\varphi) \right) \epsilon 
				\end{align}
				が成り立ち,$\epsilon$を潰して$|\inprod<u \ast \rho_\epsilon, \varphi> - \inprod<u, \varphi>| 
				\leq c p_m(\rho_\epsilon \ast \varphi - \varphi) \longrightarrow 0\ (\epsilon \longrightarrow 0)$を得る.
				\QED
		\end{description}
	\end{prf}
\appendix
\chapter{弱収束}
\section{ノルム空間における弱収束}

	$\K$を$\R$又は$\C$とする.ノルム空間$X$に対しノルムを$\Norm{\cdot}{X}$で表記し,
	また$J_X: X \rightarrow X^{**}$を自然な等長単射とする.
	\begin{screen}
		\begin{dfn}[弱収束]
			$X$を$\K$上のノルム空間とする.$X$の点列$(x_n)_{n=1}^{\infty}$が$x \in X$に弱収束するとは
			\begin{align}
				\lim_{n \to \infty} f(x_n) = f(x) \quad (\forall f \in X^*)
			\end{align}
			が成り立つことを言い,$\wlim_{n \to \infty} x_n = x$と表記する.
		\end{dfn}
	\end{screen}
	
	\begin{screen}
		\begin{dfn}[汎弱収束]
			$X$を$\K$上のノルム空間とする.$X^*$の列$(f_n)_{n=1}^{\infty}$が$f \in X^*$に汎弱収束するとは
			\begin{align}
				\lim_{n \to \infty} f_n(x) = f(x) \quad (\forall x \in X)
			\end{align}
			が成り立つことを言い,$\wstarlim_{n \to \infty} f_n = f$と表記する.
		\end{dfn}
	\end{screen}
	
	\begin{screen}
		\begin{thm}[弱収束及び汎弱収束極限の一意性]
			$X$を$\K$上のノルム空間とする.$X$の点列$(x_n)_{n=1}^{\infty}$が$u,v \in X$に弱収束
			するなら$u = v$が従い,$X^*$の列$(f_n)_{n=1}^{\infty}$が$f,g \in X^*$に汎弱収束するなら
			$f = g$が従う.
		\end{thm}
	\end{screen}
	
	\begin{prf}
		$(x_n)_{n=1}^{\infty}$が$u,v \in X$に弱収束するとき,任意の$f \in X^*$に対して
		\begin{align}
			\left| f(u) - f(v) \right| \leq \left| f(u) - f(x_n) \right| + \left| f(x_n) - f(v) \right| \longrightarrow 0 
			\quad (n \longrightarrow \infty)
		\end{align}
		が成り立ち,Hahn-Banachの定理の系より$u = v$が従う.また
		$(f_n)_{n=1}^{\infty}$が$f,g \in X^*$に汎弱収束するとき,任意の$x \in X$に対して
		\begin{align}
			\left| f(x) - g(x) \right| \leq \left| f(x) - f_n(x) \right| + \left| f_n(x) - g(x) \right| \longrightarrow 0 
			\quad (n \longrightarrow \infty)
		\end{align}
		が成り立ち$f = g$が従う.
		\QED
	\end{prf}
	
	\begin{screen}
		\begin{thm}[弱収束と自然な等長単射の関係]
			$X$を$\K$上のノルム空間とする.$x_n \in X\ (n=1,2,\cdots)$が$x \in X$に弱収束することと
			$J_Xx_n \in X^{**}\ (n=1,2,\cdots)$が$J_Xx \in X^{**}$に汎弱収束することは同値である.
			\label{thm:weak_convergence_and_canonical_injection}
		\end{thm}
	\end{screen}
	
	\begin{prf}
		自然な等長単射の定義より任意の$f \in X^*$について$f(x_n) = J_X x_n(f)$であるから,
		\begin{align}
			\lim_{n \to \infty} f(x_n) = f(x) \quad (\forall f \in X^*)
		\end{align}
		が成り立つことと
		\begin{align}
			\lim_{n \to \infty} J_X x_n(f) = J_X x(f) \quad (\forall f \in X^*)
		\end{align}
		が成り立つことは同じである.
		\QED
	\end{prf}
	
	\begin{screen}
		\begin{thm}[汎弱収束列の有界性]
			$X$を$\K$上のノルム空間とし$X \neq \{0\}$を仮定する.$X^*$の列$(f_n)_{n=1}^{\infty}$が各点$x \in X$でCauchy列をなすとき,
			$(f_n)_{n=1}^{\infty}$は有界となりさらに汎弱収束極限$f \in X^*$が存在して次が成り立つ \footnotemark:
			\begin{align}
				\Norm{f}{X^*} \leq \liminf_{n \to \infty} \Norm{f_n}{X^*}.
			\end{align}
			\label{thm:weak_star_convergence_bonded}
		\end{thm}
	\end{screen}
	
	\footnotetext{
		右辺は有限確定する.
		実際$(f_n)_{n=1}^{\infty}$が有界であるとして$M \coloneqq \sup{n \in \N}{\Norm{f_n}{X^*}}$とおけば,
		任意の$n \in \N$に対し
		\begin{align}
			\inf{\nu \geq n}{\Norm{f_n}{X^*}} \leq \sup{n \in \N}{\Norm{f_n}{X^*}} = M
		\end{align}
		が成り立つから
		\begin{align}
			\liminf_{n \to \infty} \Norm{f_n}{X^*} \leq M
		\end{align}
		が従う.
	}
	
	\begin{prf}
		任意の$x \in X$に対して$\left( f_n(x) \right)_{n=1}^{\infty}$は有界であるから,
		一様有界性の原理より$\left( \Norm{f_n}{X^*} \right)_{n=1}^{\infty}$が有界となる.また
		\begin{align}
			f(x) \coloneqq \lim_{n \to \infty} f_n(x) \quad (\forall x \in X)
			\label{eq:thm_weak_star_convergence_bonded}
		\end{align}
		として$f:X \rightarrow \K$を定めれば,$f$は$X^*$に属する:
		\begin{description}
			\item[線型性]
				任意に$x,x_1,x_2 \in X$と$\alpha \in \K$を取れば
				\begin{align}
					\left| f(x_1 + x_2) - f(x_1) - f(x_2) \right| &\leq \left| f(x_1 + x_2) - f_n (x_1 + x_2)\right| + \left| f(x_1) - f_n(x_1) \right| 
						+ \left| f(x_2) - f_n(x_2) \right| \longrightarrow 0 \quad (n \longrightarrow \infty) \\
					\left| f(\alpha x) - \alpha f(x) \right| &\leq \left| f(\alpha x) - f_n(\alpha x) \right| + |\alpha| \left| f(x) - f_n(x) \right|
						\longrightarrow 0 \quad (n \longrightarrow \infty)
				\end{align}
				が成り立つ.
			
			\item[有界性]
				絶対値の連続性より
				\begin{align}
					\left| f(x) \right| = \lim_{n \to \infty} \left| f_n(x) \right| \leq \liminf_{n \to \infty} \Norm{f_n}{X^*} \Norm{x}{X}
				\end{align}
				が成り立ち,特に$\Norm{x}{X} = 1$として
				\begin{align}
					\sup{\Norm{x}{X} = 1}{\left| f(x) \right|} \leq \liminf_{n \to \infty} \Norm{f_n}{X^*} < \infty
				\end{align}
				が従う.
		\end{description}
		$f$が$f_n$の汎弱収束極限であることは(\refeq{eq:thm_weak_star_convergence_bonded})より従う.
		\QED
	\end{prf}
	
	\begin{screen}
		\begin{thm}[弱収束列の有界性]
			$X$を$\K$上のノルム空間とし$X \neq \{0\}$を仮定する.$X$の列$(x_n)_{n=1}^{\infty}$が$x \in X$に弱収束するとき,
			$(x_n)_{n=1}^{\infty}$は有界列であり次が成り立つ:
			\begin{align}
				\Norm{x}{X} \leq \liminf_{n \to \infty} \Norm{x_n}{X}.
			\end{align}
			\label{thm:weak_convergence_bounded}
		\end{thm}
	\end{screen}
	
	\begin{prf}
		定理\ref{thm:weak_convergence_and_canonical_injection}より
		$(J_X x_n)_{n=1}^{\infty}$が$J_X x \in X^{**}$に汎弱収束するから,
		定理\ref{thm:weak_star_convergence_bonded}より$(J_X x_n)_{n=1}^{\infty}$は有界列で
		\begin{align}
			\Norm{J_X x}{X^{**}} \leq \liminf_{n \to \infty} \Norm{J_X x_n}{X^{**}}
		\end{align}
		が成り立つ.$J_X$は等長であるから定理の主張が従う.
		\QED
	\end{prf}
	
	\begin{screen}
		\begin{thm}[反射的Banach空間の点列が弱収束するための十分条件]
			$X$を$\K$上の反射的Banach空間として点列$(x_n)_{n=1}^{\infty}$を取る.任意の$f \in X^*$に対して
			$\left( f(x_n) \right)_{n=1}^{\infty}$がCauchy列となるなら,$(x_n)_{n=1}^{\infty}$は或る$x \in X$に弱収束する.
		\end{thm}
	\end{screen}
	
	\begin{prf}
		$f(x_n) = J_X x_n(f)$であることと定理の仮定より,任意の$f \in X^*$で$\left( J_X x_n(f) \right)_{n=1}^{\infty}$は$\K$のCauchy列をなすから,
		\begin{align}
			J(f) \coloneqq \lim_{n \to \infty} J_X x_n(f) \quad (\forall f \in X^*)
		\end{align}
		として$J:X^* \rightarrow \K$を定めれば定理\ref{thm:weak_star_convergence_bonded}より$J \in X^{**}$が成り立つ.
		$X$の反射性から$J$に対し或る$x \in X$が存在して$J = J_X x$を満たし,
		定理\ref{thm:weak_convergence_and_canonical_injection}より定理の主張を得る.
		\QED
	\end{prf}
\chapter{Riemannの級数定理}
	複素数列$(a_n)_{n=1}^{\infty}$に対し,
	$\N$から$\N$への全単射$\varphi$を用いて
	\begin{align}
		a'_n \coloneqq a_{\varphi(n)} \quad (n=1,2,\cdots)
	\end{align}
	により定める数列$\left( a'_n \right)_{n=1}^{\infty}$を
	$(a_n)_{n=1}^{\infty}$の並び替え(rearrangement)という.
	
	\begin{screen}
		\begin{thm}[Riemannの級数定理]
			$(\alpha_n)_{n=1}^{\infty}$を実数列とする.
			$\sum_{n=1}^{\infty} \alpha_n$が条件収束するとき,
			任意の実数$\beta \in \R$に対して$(\alpha_n)_{n=1}^{\infty}$の
			或る並び替え$(\alpha'_n)_{n=1}^{\infty}$が存在し$\beta = \sum_{n=1}^{\infty} \alpha'_n$を満たす.
			\label{thm:Riemann_series}
		\end{thm}
	\end{screen}
	
	\begin{prf}
		\begin{align}
			p_n \coloneqq \frac{|\alpha_n| + \alpha_n}{2},
			\quad q_n \coloneqq \frac{|\alpha_n| - \alpha_n}{2}
			\quad (n=1,2,\cdots)
		\end{align}
		とおけば,$(p_n)_{n=1}^{\infty},(q_n)_{n=1}^{\infty}$は全て非負項で構成され,仮定より
		\begin{align}
			p_n \longrightarrow 0,
			\quad q_n \longrightarrow 0,
			\quad \sum_{n=1}^{\infty} p_n = \infty,
			\quad \sum_{n=1}^{\infty} q_n = \infty
		\end{align}
		が成り立つ.実際$\alpha_n \longrightarrow 0$により$p_n,q_n \longrightarrow 0$が満たされ,また
		$\sum_{n=1}^{\infty} p_n$か$\sum_{n=1}^{\infty} q_n$の一方が収束すると
		\begin{align}
			\sum_{n=1}^N \alpha_n = \sum_{n=1}^N p_n - \sum_{n=1}^N q_n
			\quad (\forall N \in \N)
		\end{align}
		よりもう一方の級数も収束するから,
		$\infty = \sum_{n=1}^{\infty} |\alpha_n| = \sum_{n=1}^{\infty} (p_n + q_n) < \infty$が従い矛盾が生じる.
		$\alpha_n\ (n=1,2,\cdots)$から添数の順に非負項を取り出し,取り出した順番に並べて$(P_n)_{n=1}^{\infty}$とおく.
		同様に負値の項を取り出しその絶対値の列を$(Q_n)_{n=1}^{\infty}$とおく.このとき$(P_n)_{n=1}^{\infty},(Q_n)_{n=1}^{\infty}$はそれぞれ
		$(p_n)_{n=1}^{\infty},(q_n)_{n=1}^{\infty}$と0の項を除いて一致するから
		\begin{align}
			P_n \longrightarrow 0,
			\quad Q_n \longrightarrow 0,
			\quad \sum_{n=1}^{\infty} P_n = \infty,
			\quad \sum_{n=1}^{\infty} Q_n = \infty
			\label{eq:thm_riemann_series_1}
		\end{align}
		が満たされる.今,任意に$\beta \in \R$を取り,
		\begin{align}
			P_1 + P_2 + \cdots + P_m > \beta
		\end{align}
		を満たす最小の$m$を$s_1$と決める.次に
		\begin{align}
			P_1 + \cdots + P_{s_1} - Q_1 - Q_2 - \cdots - Q_m < \beta
		\end{align}
		を満たす最小の$m$を$u_1$と決める.始めの要領で
		\begin{align}
			P_1 + \cdots + P_{s_1} - Q_1 - \cdots - Q_{u_1}
			+ P_{s_1 + 1} + \cdots + P_m > \beta
		\end{align}
		を満たす最小の$m$を$s_2$とし,
		\begin{align}
			P_1 + \cdots + P_{s_1} - Q_1 - \cdots - Q_{u_1}
			+ P_{s_1 + 1} + \cdots + P_{s_2} - Q_{u_1 + 1} - \cdots - Q_m < \beta
		\end{align}
		を満たす最小の$m$を$u_2$とする.(\refeq{eq:thm_riemann_series_1})より正項級数$\sum_{n=1}^{\infty} P_n,\ \sum_{n=1}^{\infty} Q_n$が発散するから
		この操作は可能であり,
		同様の操作を繰り返して$(s_n)_{n=1}^{\infty},(u_n)_{n=1}^{\infty}$を構成すれば
		$(\alpha_n)_{n=1}^{\infty}$の並び替え$P_1,\cdots,P_{s_1},Q_1,\cdots,Q_{u_1},P_{s_1+1},\cdots,P_{s_2},Q_{u_1+1},\cdots$を得る.
		この並び替えの部分和を$A_n\ (n=1,2,\cdots)$と表す.
		(\refeq{eq:thm_riemann_series_1})より任意の$\epsilon > 0$に対し或る$K \in \N$が存在して
		\begin{align}
			P_m,\ Q_m < \epsilon
			\quad ( \forall m \geq s_K + u_K )
		\end{align}
		を満たすから,$s_k \leq n < u_k\ (k > K)$なら
		\begin{align}
			A_n \geq \beta,
			\quad A_n - Q_{u_k} < \beta
		\end{align}
		より
		\begin{align}
			0 \leq A_n - \beta < Q_{u_k} < \epsilon
		\end{align}
		が成立し,或は$u_k \leq n < s_{k+1}\ (k \geq K)$なら
		\begin{align}
			A_n \leq \beta,
			\quad A_n + P_{s_{k+1}} > \beta
		\end{align}
		より
		\begin{align}
			0 \leq \beta - A_n < P_{s_{k+1}} < \epsilon
		\end{align}
		が成立するから,いずれの場合も
		\begin{align}
			|A_n - \beta| < \epsilon \quad ( \forall n \geq s_K + u_K )
		\end{align}
		が満たされ,$\epsilon > 0$の任意性より定理の主張が得られる.
		\QED
	\end{prf}
	
	\begin{screen}
		\begin{cor}[絶対収束と無条件収束は同値]
			任意の複素数列$(a_n)_{n=1}^{\infty}$に対し次は同値である:
			\begin{description}
				\item[(1)] $\sum_{n=1}^{\infty} |a_n| < \infty$.
				\item[(2)] 任意の全単射$\varphi:\N \rightarrow \N$に対し$\sum_{n=1}^{\infty} a_{\varphi(n)} = \sum_{n=1}^{\infty} a_n \in \C$.
			\end{description}
		\end{cor}
	\end{screen}
	
	\begin{prf}\mbox{}
		\begin{description}
			\item[(1) $\Rightarrow$ (2)]
				$\sum_{n=1}^{\infty} |a_n| < \infty$が満たされている場合,任意の$N \in \N$に対し
				\begin{align}
					\left| \sum_{n=p}^{q} a_n \right| \leq \sum_{n=p}^{q} |a_n| \longrightarrow 0
					\quad (p,q \longrightarrow \infty)
				\end{align}
				より$\sum_{n=1}^{\infty} a_n$は収束する.また$\varphi:\N \rightarrow \N$を全単射とすれば,
				任意の$N \in \N$に対し或る$N' \in \N$が存在して
				\begin{align}
					\sum_{n=1}^{N} \left| a_{\varphi(n)} \right|
					\leq \sum_{n=1}^{N'} \left| a_n \right|
					\leq \sum_{n=1}^{\infty} \left| a_n \right|
				\end{align}
				が成り立つから$\sum_{n=1}^{\infty} \left| a_{\varphi(n)} \right|$及び$\sum_{n=1}^{\infty} a_{\varphi(n)}$も収束する.
				任意に$\epsilon > 0$を取れば,或る$N \in \N$が存在して
				\begin{align}
					\sum_{n=p}^{q} |a_n| < \epsilon
					\quad (p,q > N)
				\end{align}
				を満たすから,$\{1,\cdots,N\} \subset \{\varphi(1),\cdots,\varphi(K)\}$を満たす$K \in \N\ (K > N)$が存在して
				\begin{align}
					\left| \sum_{n=1}^{K} a_n - \sum_{n=1}^{K} a_{\varphi(n)} \right|
					\leq \left| \sum_{n=N+1}^K a_n \right| + \left| \sum_{n \in \{\varphi(1),\cdots,\varphi(K)\} \backslash \{1,\cdots,N\}} a_n \right|
					< 2 \epsilon
				\end{align}
				が成立し,
				\begin{align}
					\left| \sum_{n=1}^{\infty} a_n - \sum_{n=1}^{\infty} a_{\varphi(n)} \right|
					\leq \left| \sum_{n=1}^{\infty} a_n - \sum_{n=1}^{K} a_n \right|
						+ \left| \sum_{n=1}^{K} a_n - \sum_{n=1}^{K} a_{\varphi(n)} \right|
						+ \left| \sum_{n=1}^{K} a_{\varphi(n)} - \sum_{n=1}^{\infty} a_{\varphi(n)} \right|
					\longrightarrow 0
					\quad (K \longrightarrow \infty)
				\end{align}
				が従う.
				
			\item[(2) $\Rightarrow$ (1)]
				任意の$a_n$に対し或る$\alpha_n,\beta_n \in \R$が存在して$a_n = \alpha_n + i \beta_n$と表せる.
				仮定より任意の$\varphi$に対し
				\begin{align}
					\sum_{n=1}^{\infty} \alpha_{\varphi(n)} = \sum_{n=1}^{\infty} \alpha_n,
					\quad \sum_{n=1}^{\infty} \beta_{\varphi(n)} = \sum_{n=1}^{\infty} \beta_n
				\end{align}
				が満たされているから,定理\ref{thm:Riemann_series}より
				\begin{align}
					\sum_{n=1}^\infty |a_n| \leq \sum_{n=1}^{\infty} |\alpha_n| + \sum_{n=1}^{\infty} |\beta_n| < \infty
				\end{align}
				が従い(1)が出る.
				\QED
				
		\end{description}
	\end{prf}
\newpage
\printindex
%
%
\end{document}
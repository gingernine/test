
(約束及び定義)
\begin{itemize}
	\item 係数体は複素数体$\C$.
	\item 位相空間$X,Y$に対し,$C(X,Y) = \{f\ |\ \mbox{$f$は$X$から$Y$への連続写像}\};\ C(X) = C(X, \C)$.
		$C_b(X) = \{f\in C(X)\ |\ \mbox{$f$は有界}\}$は$\Norm{u}{} = \sup{x \in X}{|u(x)|}$をノルム(sup-norm)としてBanach空間である.
	\item $s$を複素数列全体のなす線形空間とする.
		$l^{\infty} = \{a = (a_n)_{n=1}^{\infty} \in s\ |\ \Norm{a}{l^{\infty}} = \sup{n}{|a_n|} < \infty\},\ 
		c_0=\{a=(a_n)_{n=1}^{\infty} \in \ |\ \lim_{n \to \infty}a_n=0\}.$
		このとき$l^{\infty}$は$\Norm{a}{l^{\infty}}$をノルムとしてBanach空間である.
\end{itemize}

[4].\ $k \in \mathbb{N}_0$,$I = [a,b]$とする.$C^k(I)$は$|f|_k = \sum_{j=0}^{k}\sup{x \in I}{\left|f^{(j)}(x)\right|}$
をノルムとしてBanach空間であることを示せ.

\begin{prf}
	以下の手順で示す.
	\begin{description}
		\item[\rm{(i)}] $|\cdot|_k$が$C^k(I)$におけるノルムである.
		\item[\rm{(ii)}] $C^k(I)$の$|\cdot|_k$によるCauchy列を取ると,各$j\ (=0,1,2,\cdots,k)$
			階導関数列に対し或る$I$上の連続関数$f^j$が存在し,$j$階導関数列は$f^j$に$I$上で一様収束する.
		\item[\rm{(iii)}] 各$j=0,1,2,\cdots,k-1,$について$f^j$は$I$上連続微分可能で
			$f^{j+1}(x) = \tfrac{d}{dx}f^{j}(x)\ (\forall x \in I)$が成り立っている.
	\end{description}
	\begin{description}
	\item[\rm{(i)}] $f,g \in C^k(I)$と$\alpha \in \C$を任意に取る.正値性$|f|_k \geq 0$は右辺の各項が$\geq 0$であることから成り立つ.また$|f|_k=0$の場合,
	右辺で$\sup{x \in I}{|f^{(j)}(x)|} = 0\ (j=0,1,2,\cdots,k)$が成り立ち,特に$f$は$I$上で零写像であるとわかるから$f = 0$である.
	逆に$f$が$I$上で零写像ならば全ての導関数が零写像になるため右辺は$0$になり,従って$|f|_k=0$となる.同次性は
	\begin{align}
		|\alpha f|_k = \sum_{j=0}^{k}\sup{x \in I}{\left|\left(\alpha f^{(j)}\right)(x)\right|} 
		= \sum_{j=0}^{k}\sup{x \in I}{\left|\alpha f^{(j)}(x)\right|}
		= \sum_{j=0}^{k}\sup{x \in I}{|\alpha|\left|f^{(j)}(x)\right|} 
		= |\alpha| \sum_{j=0}^{k}\sup{x \in I}{\left|f^{(j)}(x)\right|} = |\alpha||f|_k
	\end{align}
	により示される.三角不等式は
	\begin{align}
		|f+g|_k 
		&= \sum_{j=0}^{k}\sup{x \in I}{\left|(f + g)^{(j)}(x)\right|} \\
		&= \sum_{j=0}^{k}\sup{x \in I}{\left|\left(f^{(j)} + g^{(j)}\right)(x)\right|} \\
		&\leq \sum_{j=0}^{k}\sup{x \in I}{\left|f^{(j)}(x) + g^{(j)}(x)\right|} \\
		&\leq \sum_{j=0}^{k}\left(\sup{x \in I}{\left|f^{(j)}(x)\right|} + \sup{x \in I}{\left|g^{(j)}(x)\right|}\right)
		= |f|_k + |g|_k
	\end{align}
	により示される.
	\item[\rm{(ii)}] $f_n \in C^k(I) \ (n=1,2,3,\cdots)$を$C^k(I)$の$|\cdot|_k$によるCauchy列とする.任意の$\epsilon > 0$に対し或る
	$N \in \N$が存在して全ての$n,m > N$で
	\begin{align}
		\epsilon > |f_n - f_m|_k =  \sum_{j=0}^{k}\sup{x \in I}{\left|f_n^{(j)}(x) - f_m^{(j)}(x)\right|}
	\end{align}
	が成り立っているから,各$j = 0,1,2,\cdots,k$について$\left(f_n^{(j)}\right)_{n=1}^{+\infty}$はsup-normに関してCauchy列をなしている.
	$(C(I),\ \mathrm{sup-norm})$がBanach空間であることが認められているから,各$j = 0,1,2,\cdots,k$についてそれぞれ或る$I$上の連続関数$f^j$が存在して,
	$\left(f_n^{(j)}\right)_{n=1}^{+\infty}$は$f^j$にsup-normで収束,即ち$I$上で一様収束する.\\
	\item[\rm{(iii)}] 上で取った$(f_n)_{n=1}^{+\infty} \subset C^k(I)$について,全ての$n \in \N$と$j = 0,1,2,\cdots,k-1$に対して次の関係が成り立っている.
	\begin{align}
		f_n^{(j)}(x) - f_n^{(j)}(b) = \int_{b}^{x} f_n^{(j+1)}(t)\, dt, \quad (\forall x \in I).
	\end{align}
	ここで(ii)の結果から,任意の$x \in I$と$\epsilon > 0$に対し或る$N = N(j,\epsilon) \in \N$が存在して全ての$n > N$で
	\begin{align}
		\left|f^j(x) - f_n^{(j)}(x)\right| < \frac{\epsilon}{3},\quad \left|f_n^{(j+1)}(x) - f^{j+1}(x)\right| < \frac{\epsilon}{3(b-a)}
	\end{align}
	が成り立つようにできるから,同じ$n$について
	\begin{align}
		\left|\left(f^j(x) - f^j(b)\right) - \int_{b}^{x} f^{j+1}(t)\, dt\right| 
		&= \left|\left(f^j(x) - f^j(b)\right) - \left(f_n^{(j)}(x) - f_n^{(j)}(b)\right)
			+ \int_{b}^{x} f_n^{(j+1)}(t)\, dt - \int_{b}^{x} f^{j+1}(t)\, dt\right| \\
		&\leq \left|f^j(x) - f_n^{(j)}(x)\right| + \left|f^j(b) - f_n^{(j)}(b)\right| 
			+ \int_{b}^{x} \left| f_n^{(j+1)}(t) - f^{j+1}(t) \right|\, dt \\
		&< \frac{\epsilon}{3} + \frac{\epsilon}{3} + (b-a) \frac{\epsilon}{3(b-a)} = \epsilon
 	\end{align}
 	が成り立つ.$\epsilon$は任意だから
 	\begin{align}
 		f^j(x) - f^j(b) = \int_{b}^{x} f^{j+1}(t)\, dt, \quad (\forall x \in I,\ j=0,1,2,\cdots,k-1)
 	\end{align}
 	が示されたことになる.右辺は連続関数$f^{j+1}$の積分だから左辺$f^j$は$x$に関して微分可能関数(端点は片側微分を考える)となり,導関数は$f^{j+1}$である.
 	ゆえに$f^0 \in C^k(I)$が示される.表記を改めて$f \coloneqq f^0,\ f^{(j)} \coloneqq f^j\ (j=1,2,\cdots,k)$と表せば,(ii)の結果より
 	\begin{align}
 		|f_n - f|_k = \sum_{j=0}^{k}\sup{x \in I}{\left|f_n^{(j)}(x) - f^{(j)}(x)\right|} \longrightarrow 0\quad (n \rightarrow +\infty)
 	\end{align}
 	が成り立つことで$C^k(I)$が$|\cdot|_k$をノルムとしてBanach空間をなしていると示される.
 	\QED
 	\end{description}
\end{prf}

[5].\ $X=C([0,1])$をsup-normの入ったBanach空間とする.$0 < a <1,\ Y=\{f \in X\ ;\ [0,\ a]\mbox{上で}f(t)=0\}$とおく.
\begin{description}
	\item[(1)] $Y$が$X$の閉線形部分空間であることを示せ.
	\item[(2)] $X/Y$と$C([0,\ a])$(sup-normを入れる)はBanach空間として同型であることを示せ.
\end{description}

\begin{prf}
\begin{description}\mbox{}
	\item[(1)]
		$Y$が$X$の線形部分空間であることは,任意の$f,g \in Y$と任意の複素数$\alpha \in \C$に対して
		\begin{align}
			&(f+g)(t) = f(t) + g(t) = 0, \quad (\forall t \in [0,\ a]) \\
			&(\alpha f) (t) = \alpha f(t) = 0, \quad (\forall t \in [0,\ a])
		\end{align}
		が成り立つことから示される.後はsup-normに関して$Y$が閉集合となっていることを示せばよい.
		$f_n \in Y\ (n=1,2,3,\cdots)$をsup-normに関するCauchy列とする.$(C([0,\ 1]),\ \rm{sup-norm})$の完備性から
		$(f_n)_{n=1}^{\infty}$は或る$f \in C([0,\ 1])$に$[0,\ 1]$上で一様に収束するが,もし或る$x \in [0,\ a]$について
		$|f(x)| > 0$であるならば,この$x$において$f_n(x)=0\ (n=1,2,3,\cdots)$であることから
		\begin{align}
			0 < |f(x)| = |f_n(x) - f(x)| \leq \sup{t \in [0,\ 1]}{|f_n(t) - f(t)|}, \qquad (n=1,2,3,\cdots)
		\end{align}
		となり$(f_n)_{n=1}^{\infty}$が$f$に収束することに反する.従って$f$も$[0,\ a]$上で0でなくてはならず,
		$f$は$Y$に属することになる.これは$Y$がsup-normの下で完備ノルム空間となっていることを主張し,以上より$Y$は
		$X$の閉線形部分空間であると示された.
	\item[(2)]
		$X$のsup-normを$\Norm{\cdot}{}$で表現する.
		$X$を$Y$で割った商空間$X/Y$の元を$[f]\ (\mbox{代表元}\ f \in X)$で表現して,$Y$が閉線形部分空間であるから$X/Y$は
		\begin{align}
			\Norm{[f]}{X/Y} \coloneqq \inf{g \in Y}{\Norm{f - g}{}},\qquad ([f] \in X/Y)
		\end{align}
		をノルムとしてノルム空間となり,さらに$(X,\ \Norm{\cdot}{})$がBanach空間であるから$(X/Y,\ \Norm{\cdot}{X/Y})$もBanach空間となる.
		任意の$[f] \in X/Y$について$f_1, f_2 \in [f]$は$f_1 - f_2 \in Y$を満たすから即ち
		\begin{align}
			f_1(t) = f_2(t) \quad (\forall t \in [0,\ a])
		\end{align}
		が成り立っている.
		\begin{align}
			\Norm{[f]}{X/Y} = \inf{g \in Y}{\Norm{f - g}{}} = \inf{g \in Y}\sup{t \in [0,\ 1]}{|f(t) - g(t)|} = \sup{t \in [0,\ a]}{|f(t)|}
		\end{align}
		が成り立つことに注意する.
		\begin{align}
			\sup{t \in [0,\ a]}{|f(t)|} = \sup{t \in [0,\ a]}{|f(t) - g(t)|} \leq \sup{t \in [0,\ 1]}{|f(t) - g(t)|}\quad (\forall g \in Y)
		\end{align}
		\begin{align}
			\sup{t \in [0,\ a]}{|f(t)|} \leq \inf{g \in Y}{\sup{t \in [0,\ 1]}{|f(t) - g(t)|}}
		\end{align}
		$[f]$の元の定義域を$[0,a]$に制限した関数は$C([0,\ a])$の或る元に一致する.$C([0,\ a])$の任意の元は定義域を$[0,1]$に拡張
		(例えば$[a,\ 1]$上では適当な一次関数でおく)すれば$X=C([0,\ 1])$の或る元に一致するから或る$X/Y$の或る元(同値類)に属するしていることになる.
		写像$X/Y \ni [f] \longmapsto f|_{[0,\ a]} \in C([0,\ a])$は$X/Y$から$C([0,\ a])$への全単射である.この写像を$T$と表すとする.
		$T$は次の意味で等長である.$\Norm{[f]}{X/Y} = \Norm{f|_{[0,\ a]}}{} = \Norm{T[f]}{}$.$T$の線型性は
		\begin{align}
			&T([f] + [h]) = T[f+h] = (f+h)|_{[0,\ a]} = f|_{[0,\ a]} + h|_{[0,\ a]} = T[f] + T[h], \\
			&T(\alpha[f]) = T[\alpha f] = (\alpha f)|_{[0,\ a]} = \alpha f|_{[0,\ a]} = \alpha T[f]
		\end{align}
		により示される.ゆえに$T$は$X/Y \longmapsto C([0,\ a])$の同型写像であり,$X/Y$と$C([0,\ a])$がBanach空間として同型であることが示された.
		\QED
\end{description}
\end{prf}

[6].\ $I=[0,\ 1]$とし,$X=C(I)$をsup-normの入ったBanach空間とする.$K \in C(I \times I)$とし,$A = \sup{(t,s)\in I \times I}{|K(t,s)|}$
とおく.$u \in X$に対して$Tu\, :\, I \longmapsto \C$を次で定める:
\begin{align}
	Tu(t) = \int_{0}^{t} K(t,s)u(s)\, ds,\ (t \in I).
\end{align}
\begin{description}
	\item[(1)] $u \in X$ならば$Tu \in X$を示せ.
	\item[(2)] 写像$X \ni u \longmapsto Tu \in X$を同じ記号$T$であらわすとき,$T \in B(X)$及び$\Norm{T^n}{} \leq A^n/n!\ (n \in \N_0)$を示せ.
	\item[(3)] $I-T$は逆作用素を持ち,$(I-T)^{-1} \in B(X)$であることを示せ.ただし$I$は$X$の高等写像である.
\end{description}

\begin{prf}
	$X$におけるsup-normを$\Norm{\cdot}{X}$と表す.また$T^0 = I$(恒等写像)として考える.
\begin{description}
	\item[(1)] $u \in X$に対して$Tu$が$I$上で連続であることを示せばよい.
		任意の正数$\epsilon > 0$に対して$\delta = \epsilon/{A\Norm{u}{X}}$と取れば,
		$t \in [0,\ 1)$と$t+h \in [0,\ 1] \cap (t,t+\delta)$に対して
		\begin{align}
			|Tu(t + h) - Tu(t)| 
			&= \left|\int_{0}^{t+h} K(t,s)u(s)\, ds - \int_{0}^{t} K(t,s)u(s)\, ds \right| \\
			&\leq \left|\int_{t}^{t+h} K(t,s)u(s)\, ds \right| \\
			&\leq \int_{t}^{t+h} |K(t,s)||u(s)|\, ds \\
			&\leq \int_{t}^{t+h} \sup{(t,s)\in I \times I}{|K(t,s)|}\, \sup{s \in I}{|u(s)|}\, ds \\
			&\leq A \Norm{u}{X} h < \epsilon
		\end{align}
		が成り立つことにより$Tu$が$[0,\ 1)$上右連続であることが示される.同様にして
		$Tu$が$(0,\ 1]$上で左連続であることも示されるから,$Tu \in X$が示される.
	\item[(2)] (1)の結果より$X \ni u \longmapsto Tu \in X$が判っているから,
		後は写像$T:X \longmapsto X$の線型性を示せば,$T$が$X$を定義域とする線型作用素であること,即ち$T \in B(X)$が示される.
		$T$の線型性は,任意の$u, v \in X,\ \alpha \in \C,\ t \in I$に対して
		\begin{align}
			T(u+v)(t) &= \int_{0}^{t} K(t,s)(u+v)(s)\, ds \\
			&= \int_{0}^{t} K(t,s)(u(s) + v(s))\, ds \\
			&= \int_{0}^{t} K(t,s)u(s)\, ds + \int_{0}^{t} K(t,s)v(s)\, ds = Tu(t) + Tv(t), \\
			T(\alpha u)(t) &= \int_{0}^{t} K(t,s)(\alpha u)(s)\, ds \\
			&= \int_{0}^{t} K(t,s)(\alpha u(s))\, ds \\
			&= \alpha\int_{0}^{t} K(t,s)u(s)\, ds = \alpha Tu(t)
		\end{align}
		が成り立つことにより示される.
		次に$\Norm{T^n}{} \leq A^n/n!\ (n \in \N_0)$を示すが,その準備に次のことを示す.
		\begin{align}
			|T^nu(t)| \leq \frac{A^n}{n!}\Norm{u}{X}t^n, \quad(t \in I,\ n=1,2,3,\cdots). \label{eq:func_analy_induction}
		\end{align}
		証明は数学的帰納法による.$n=1$のとき
		\begin{align}
			|Tu(t)| &= \left|\int_{0}^{t} K(t,s)u(s)\, ds\right| \\
			&\leq \int_{0}^{t} |K(t,s)||u(s)|\, ds \\
			&\leq \int_{0}^{t} \sup{(t,s)\in I \times I}{|K(t,s)|}\, \sup{s \in I}{|u(s)|}\, ds \\
			&\leq A\Norm{u}{X} \int_{0}^{t}\, ds \\
			&= A\Norm{u}{X}t
		\end{align}
		が成り立つ.$n=k$のとき(\refeq{eq:func_analy_induction})を仮定すると,
		\begin{align}
			|T^{k+1}u(t)| &= |T(T^ku)(t)| = \left|\int_{0}^{t} K(t,s)T^ku(s)\, ds\right| \\
			&\leq \int_{0}^{t} |K(t,s)||T^ku(s)|\, ds \\
			&\leq \int_{0}^{t} \sup{(t,s)\in I \times I}{|K(t,s)|}\, \frac{A^k}{k!}\Norm{u}{X}s^k\, ds \\
			&\leq \frac{A^{k+1}}{(k+1)!}\Norm{u}{X}t^{k+1}
		\end{align}
		となることにより(\refeq{eq:func_analy_induction})が任意の$n \in \N$で成立すると示された.従って$t \in I$についての上限を取れば
		\begin{align}
			\Norm{T^nu}{X} = \sup{t \in I}{|T^nu(t)|} \leq \sup{t \in I}{\frac{A^n}{n!}\Norm{u}{X}t^n} = \frac{A^n}{n!}\Norm{u}{X}, 
			\quad (\forall u \in X,\ n = 1,2,3,\cdots)
		\end{align}
		となる.$n=0$の場合は,$\Norm{Iu}{X} = \Norm{u}{X}\ (\forall u \in X)$により
		$\Norm{T^0}{} = \Norm{I}{} = 1$である.以上より$\Norm{T^n}{} \leq A^n/n!\ (n \in \N_0)$となることが示された.
	\item[(3)] $(X,\ \Norm{\cdot}{X})$がBanach空間であるから$B(X)$も作用素ノルムの下でBanach空間となっている.従って級数
		\begin{align}
			\sum_{n=0}^{+\infty} T^n \label{eq:func_analy_series}
		\end{align}
		が収束することの十分条件は
		\begin{align}
			\sum_{n=0}^{+\infty} \Norm{T^n}{} < +\infty
		\end{align}
		が成り立つことである.今,(2)の結果より
		\begin{align}
			\sum_{n=0}^{+\infty} \Norm{T^n}{} \leq \sum_{n=0}^{+\infty} \frac{A^n}{n!} = \exp{A} < +\infty
		\end{align}
		が成り立っているから(\refeq{eq:func_analy_series})は収束する.つまり
		\begin{align}
			T^* \coloneqq \sum_{n=0}^{+\infty} T^n
		\end{align}
		と表せば$T^*$は$B(X)$の元であり,部分和を$T_N \coloneqq \sum_{n=0}^{N} T^n\ (N = 0,1,2,\cdots)$と表現して
		$\Norm{T_N - T^*}{} \rightarrow 0\ (N \rightarrow +\infty)$が成り立っていることになる.任意の$u \in X$に対して
		\begin{align}
			\Norm{(TT^* - TT_N)u}{X} = \Norm{TT^*u - TT_Nu}{X} = \Norm{T(T^*u) - T(T_Nu)}{X} = \Norm{T(T^*u - T_Nu)}{X} = \Norm{T(T^* - T_N)u}{X}
		\end{align}
		と
		\begin{align}
			\Norm{(T^*T - T_NT)u}{X} = \Norm{T^*Tu - T_NTu}{X} = \Norm{T^*(Tu) - T_N(Tu)}{X} = \Norm{(T^*-T_N)Tu}{X}
		\end{align}
		が成り立つことから,
		\begin{align}
			\Norm{TT^* - TT_N}{} \leq \Norm{T}{}\Norm{T^* - T_N}{} \longrightarrow 0\ (N \longrightarrow +\infty) \label{eq:func_analy_conv1}
		\end{align}
		と
		\begin{align}
			\Norm{T^*T - T_NT}{} \leq \Norm{T^* - T_N}{}\Norm{T}{} \longrightarrow 0\ (N \longrightarrow +\infty) \label{eq:func_analy_conv2}
		\end{align}
		が成り立つ.(\refeq{eq:func_analy_conv1})により
		\begin{align}
			TT^* = \lim_{N \to \infty} TT_N = \lim_{N \to \infty} \sum_{n=1}^{N+1} T^n = \sum_{n=1}^{+\infty} T^n = T^* - I
		\end{align}
		が成り立つから$I = T^* - TT^* = (I - T)T^*$と表現でき,また(\refeq{eq:func_analy_conv2})により
		\begin{align}
			T^*T = \lim_{N \to \infty} T_NT = \lim_{N \to \infty} \sum_{n=1}^{N+1} T^n = \sum_{n=1}^{+\infty} T^n = T^* - I
		\end{align}
		も成り立つから$I = T^* - T^*T = T^*(I - T)$と表現できる.ゆえに$I = (I - T)T^* = T^*(I - T)$が成り立ち,この等式は
		$I-T$が$X \longmapsto X$の全単射であり$T^* \in B(X)$を逆写像にもつことを示している.
		\QED
\end{description}
\end{prf}


[9].\ $(S, \mathfrak{M}, \mu)$は$\sigma-$有限な測度空間,$X=\Lp{2}{S, \mathfrak{M}, \mu}=\Lp{2}{\mu}$とする.可測関数$a:S \rightarrow \C$に対して,
$X$上の掛け算作用素$M_a$を次で定める:
\begin{align}
	D(M_a) = \{u \in X\, |\, au \in X\}, \quad (M_au)(x) = a(x)u(x)\ (x \in S).
\end{align}
\begin{description}
	\item[(1)] $D(M_a)$が$X$で稠密であることを示せ.
	\item[(2)] $a \in \Lp{\infty}{S, \mathfrak{M}, \mu}$ならば$M_a \in B(X)$であり,$\Norm{M_a}{} = \Norm{a}{\Lp{\infty}{\mu}}$が成り立つことを示せ.
	\item[(3)] 逆に$M_a \in B(X)$ならば$a \in \Lp{\infty}{S, \mathfrak{M}, \mu}$であることを示せ.
\end{description}

\begin{prf}
\begin{description}\mbox{}
	\item[(1)] 任意の$v \in X$に対して
		$v_n \coloneqq vI_{(|a| \leq n)}\ (n=1,2,3,\cdots)$として関数列$(v_n)_{n=1}^{\infty}$を作る.$I_{(|a| \leq n)}$は定義関数で
		$|a(x)| \leq n$となる$x \in S$の上で1,その外では0となる.全ての$x \in S$で$|v_n(x)| \leq |v(x)|$となるから
		$(v_n)_{n \in \N} \subset X$である.さらに$(v_n)_{n \in \N} \subset D(M_a)$でもあることが示される.全ての$n \in \N$について
		\begin{align}
			\int_{S} |a(x)v_n(x)|^2 \mu(dx) = \int_{(|a| \leq n)} |a(x)v(x)|^2 \mu(dx) \leq n^2  \int_{S} |v(x)|^2 \mu(dx)
		\end{align}
		が成り立つからである.$X$のノルムを$\Norm{\cdot}{\Lp{2}{\mu}}$で表せば
		\begin{align}
			\Norm{v - v_n}{\Lp{2}{\mu}}^2 = \int_{S} |v(x) - v_n(x)|^2\, \mu(dx) = \int_{S} I_{(|a| > n)}(x)|v(x)|^2\, \mu(dx)
			\label{eq:func_report_Q9_1}
		\end{align}
		が成り立つ.$a$は$\C$値であるから,各点$x \in S$で$\lim_{n \to +\infty} I_{(|a| > n)}(x) = 0$が成り立つ.
		式(\refeq{eq:func_report_Q9_1})の右辺の被積分関数は$n$に関係なく可積分関数$|v|^2$で抑えられ各点で$n \to +\infty$で$0$に収束
		するから,Lebesgueの収束定理を適用すれば
		\begin{align}
			\lim_{n \to +\infty} \Norm{v - v_n}{\Lp{2}{\mu}}^2 
			= \lim_{n \to +\infty} \int_{S} I_{(|a| > n)}(x)|v(x)|^2\, \mu(dx)
			= \int_{S} \lim_{n \to +\infty} I_{(|a| > n)}(x)|v(x)|^2\, \mu(dx)
			= 0
		\end{align}
		が成立する.これは$X = \left(X,\ \Norm{\cdot}{\Lp{2}{\mu}}\right)$において$v$の任意の近傍に$D(M_a)$の元$v_n$が存在することを表していて,
		$v$は任意に選んでいたから$D(M_a)$は$X$で稠密であると示された.
		
	\item[(2)] $a \in \Lp{\infty}{S, \mathfrak{M}, \mu}$ならば,$\Norm{\cdot}{\Lp{\infty}{\mu}}$の定義より
		\begin{align}
			\Norm{a}{\Lp{\infty}{\mu}} = \inf{}{\left\{b \in [0, +\infty)\ |\ \mu(x \in S\ |\ |a(x)| > b) = 0 \right\}} < +\infty
		\end{align}
		である.特に
		\begin{align}
			N_m \coloneqq \left\{x \in S\ \left|\ |a(x)| \geq \Norm{a}{\Lp{\infty}{\mu}} + \frac{1}{m} \right.\right\}, \quad(m = 1,2,3,\cdots)
		\end{align}
		と置けば$\mu(N_m)=0\ (m=1,2,3,\cdots)$であって,
		$\mu$零集合$N$を
		\begin{align}
			N \coloneqq \bigcup_{m=1}^{\infty} N_m
		\end{align}
		で定めれば
		\begin{align}
			|a(x)| \leq \Norm{a}{\Lp{\infty}{\mu}}, \quad (\forall x \in S \cap N^c) \label{eq:func_analy_Q9}
		\end{align}
		が成り立つ.また$0 < c < \Norm{a}{\Lp{\infty}{\mu}}$となるような任意の$c$については
		\begin{align}
			\mu(\ \{x \in S\ |\ |a(x)| > c\}\ ) > 0 \label{eq:func_analy_Linfty}
		\end{align}
		が成り立つことにも注意しておく.全ての$u \in X$に対して,(\refeq{eq:func_analy_Q9})により
		\begin{align}
			\Norm{M_a u}{\Lp{2}{\mu}}^2 
			&= \int_{S} |a(x)u(x)|^2\, \mu(dx) \\
			&= \int_{S/N} |a(x)u(x)|^2\, \mu(dx) \\
			&\leq \int_{S/N} \Norm{a}{\Lp{\infty}{\mu}}^2|u(x)|^2\, \mu(dx) \\
			&= \int_{S} \Norm{a}{\Lp{\infty}{\mu}}^2|u(x)|^2\, \mu(dx)
			= \Norm{a}{\Lp{\infty}{\mu}}^2 \Norm{u}{\Lp{2}{\mu}}
		\end{align}
		が成り立っていることから,$\Norm{M_a}{} \leq \Norm{a}{\Lp{\infty}{\mu}}$であり$M_a \in B(X)$が示される.
		さらに,$(S, \mathfrak{M}, \mu)$が$\sigma-$有限な測度空間であるという条件の下では,$\Norm{M_a}{} = \Norm{a}{\Lp{\infty}{\mu}}$であることが
		次のように示される.$\Norm{a}{\Lp{\infty}{\mu}} = 0$ならば明らかに$M_a$は零作用素で$0 = \Norm{M_a}{} = \Norm{a}{\Lp{\infty}{\mu}}$である.
		$\Norm{a}{\Lp{\infty}{\mu}} > 0$である場合,$\Norm{a}{\Lp{\infty}{\mu}} >> \epsilon > 0$を満たすような$\epsilon$を任意に取り,
		\begin{align}
			G_\epsilon \coloneqq \left\{ x \in S \cap N^c\ \left|\ \Norm{a}{\Lp{\infty}{\mu}} - \epsilon < |a(x)| \right.\right\}
		\end{align}
		として$\mu$可測集合$G_\epsilon$を作る.(\refeq{eq:func_analy_Linfty})により,$\mu(G_\epsilon) > 0$であることが次で示される.
		\begin{align}
			\mu(G_\epsilon) &= \mu\left(\left\{x \in S\ \left|\ \Norm{a}{\Lp{\infty}{\mu}}-\epsilon < |a(x)|\right.\right\} \cap N^c\right) \\
			&= \mu\left(\left\{x \in S\ \left|\ \Norm{a}{\Lp{\infty}{\mu}}-\epsilon < |a(x)|\right.\right\}\right) & (\because \mu(N)=0) \\
			&> 0. & (\because (\refeq{eq:func_analy_Linfty})).
 		\end{align}
		$\sigma-$有限の仮定より,単調増大な$\mu$可測集合列$S_1 \subset S_2 \subset S_3 \subset \cdots$で$\mu(S_k) < +\infty\ (k=1,2,3,\cdots)$と
		$\cup_{k=1}^{\infty}S_k = S$を満たすものが存在して
		\begin{align}
			0 < \mu(G_\epsilon) = \lim_{k \to \infty} \mu(S_k \cap G_\epsilon)
		\end{align}
		となるから,必ず或る$S_{k}$に対して
		\begin{align}
			r_\epsilon \coloneqq \mu(S_{k} \cap G_\epsilon) > 0
		\end{align}
		となっている.
		\begin{align}
			u_\epsilon(x) \coloneqq 
			\begin{cases}
				1/\sqrt{r_\epsilon} & x \in S_{k} \cap G_\epsilon \\
				0 & x \notin S_{k} \cap G_\epsilon
			\end{cases}
		\end{align}
		と定義すれば,$\mu(S_{k} \cap G_\epsilon) < +\infty$であるから$u_\epsilon \in X$であって
		\begin{align}
			\Norm{u_\epsilon}{\Lp{2}{\mu}} = 1
		\end{align}
		となっている.この$u_\epsilon$に対して
		\begin{align}
			\left(\Norm{a}{\Lp{\infty}{\mu}} - \epsilon\right)^2 
			< \int_{S_k \cap G_\epsilon} |a(x)u_\epsilon(x)|^2\, \mu(dx) 
			= \int_{S} |a(x)u_\epsilon(x)|^2\, \mu(dx)
			\leq \Norm{a}{\Lp{\infty}{\mu}}^2
		\end{align}
		により
		\begin{align}
			\Norm{a}{\Lp{\infty}{\mu}} - \epsilon < \Norm{M_a u_\epsilon}{\Lp{2}{\mu}} \leq \Norm{a}{\Lp{\infty}{\mu}}
		\end{align}
		が成り立っているから
		\begin{align}
			\Norm{a}{\Lp{\infty}{\mu}} - \epsilon 
			< \sup{0 \neq u \in X}{\frac{\Norm{M_a u}{\Lp{2}{\mu}}}{\Norm{u}{\Lp{2}{\mu}}}} = \Norm{M_a}{} 
			\leq \Norm{a}{\Lp{\infty}{\mu}}
		\end{align}
		である.$\epsilon > 0$は任意に取っていたから,$\epsilon \rightarrow 0$で考えれば
		\begin{align}
			\Norm{a}{\Lp{\infty}{\mu}} = \Norm{M_a}{}
		\end{align}
		が示されたことになる.
		
	\item[(3)] $M_a \in B(X)$ならば
		\begin{align}
			\int_{S} |a(x)u(x)|^2\, \mu(dx) 
			= \Norm{M_au}{\Lp{2}{\mu}}^2 \leq \Norm{M_a}{}^2 \Norm{u}{\Lp{2}{\mu}}^2 
			= \Norm{M_a}{}^2 \int_{S} |u(x)|^2\, \mu(dx)
		\end{align}
		が成立している.$(S, \mathfrak{M}, \mu)$が$\sigma-$有限な測度空間であるという条件の下では
		\begin{align}
			\mu\left(\ \{x \in S\ |\ |a(x)| > \Norm{M_a}{}\}\ \right) = 0
		\end{align}
		が成り立つことを示す.これが示されれば$a \in \Lp{\infty}{S, \mathfrak{M}, \mu}$が従う.
		$\sigma-$有限の仮定より,単調増大な$\mu$可測集合列$S_1 \subset S_2 \subset S_3 \subset \cdots$で$\mu(S_k) < +\infty\ (k=1,2,3,\cdots)$と
		$\cup_{k=1}^{\infty}S_k = S$を満たすものが存在する.
		$\mu$可測集合
		\begin{align}
			G \coloneqq \{x \in S\ |\ |a(x)| > \Norm{M_a}{}\}
		\end{align}
		について,これが仮に$\mu(G) > 0$であるとすると
		\begin{align}
			0 < \mu(G) = \lim_{k \to \infty} \mu(S_k \cap G)
		\end{align}
		により或る$K \in \N$が存在して$\mu(S_k \cap G) > 0 \ (\forall k > K)$が成立する.$k > K$を満たす$k$を選んで
		\begin{align}
			u(x) = \begin{cases}
				1 & x \in S_k \cap G \\
				0 & x \notin S_k \cap G
			\end{cases}
		\end{align}
		と置けば,$\mu(S_k \cap G) < +\infty$であることから$u \in X$であって,
		\begin{align}
			\Norm{u}{\Lp{2}{\mu}}^2 = \int_{S} |u(x)|^2\, \mu(dx) = \mu(S_k \cap G)
		\end{align}
		が成り立っている.$G$上で$|a(x)| > \Norm{M_a}{}$であるから
		\begin{align}
			\Norm{M_a}{}^2 \Norm{u}{\Lp{2}{\mu}}^2 
			&= \Norm{M_a}{}^2 \int_{S} |u(x)|^2\, \mu(dx) \\
			&= \Norm{M_a}{}^2 \int_{S_k \cap G} |u(x)|^2\, \mu(dx) \\
			&< \int_{S_k \cap G} |a(x)u(x)|^2\, \mu(dx) \\
			&\leq \int_{S} |a(x)u(x)|^2\, \mu(dx) \\
			&\leq \Norm{M_a}{}^2 \Norm{u}{\Lp{2}{\mu}}^2 
		\end{align}
		となるが,最右辺と最左辺を$\Norm{u}{\Lp{2}{\mu}}^2$で割ると
		\begin{align}
			\Norm{M_a}{} < \Norm{M_a}{}
		\end{align}
		と矛盾が出る.従って$\mu(G)=0$でなくてはならない.
		\QED
\end{description}
\end{prf}

[10].\ $X$をノルム空間,$X_0$をその部分空間とする.$i\ :\ X_0 \rightarrow X$を$i(x) = x\ (x \in X_0)$なる包含写像とする.
$T\ :\ X^* \ni x^* \longmapsto x^* \circ i \in X_0^*$により線型写像を定める.
\begin{description}
	\item[(1)] $T$は連続かつ全射であることを示せ.
	\item[(2)] $X_0$が$X$で稠密ならば,$T$はノルム空間としての同型写像であることを示せ.
	\item[(3)] $X_0$が$X$で稠密でないならば,$T$は単射でないことを示せ.
\end{description}

\begin{prf}
\begin{description}
	\item[(1)] 任意の$x^* \in X^*$と$u \in X_0$に対して
		\begin{align}
			\left|\left( x^* \circ i \right) (u)\right| = \left| x^*(u) \right| \leq \Norm{x^*}{X^*} \Norm{u}{X}
		\end{align}
		が成り立つから$\Norm{x^* \circ i}{X_0^*} \leq \Norm{x^*}{X^*}\ (\forall x^* \in X^*)$が成立する.
		従って
		\begin{align}
			\Norm{Tx^*}{X_0^*} = \Norm{x^* \circ i}{X_0^*} \leq \Norm{x^*}{X^*} \quad \forall x^* \in X^*
		\end{align}
		となり$T$が有界作用素であることが示される.$T$が全射であることはHahn-Banachの定理による.
		任意の$f_0 \in X_0^*$に対して$X$上のセミノルムとして$X \ni x \longmapsto \Norm{f_0}{X_0^*}\Norm{x}{X} \in \C$を考えれば
		作用素ノルムの定義より
		\begin{align}
			|f_0 (x)| \leq  \Norm{f_0}{X_0^*}\Norm{x}{X}\quad (\forall x \in X_0)
		\end{align}
		が成り立っているから,Hahn-Banachの定理が適用されて$X$上の線型汎関数$f \in X^*$で
		\begin{align}
			\begin{cases}
				|f (x)| \leq  \Norm{f_0}{X_0^*}\Norm{x}{X} & (\forall x \in X), \\
				f(x) = f_0(x) & (\forall x \in X_0)
			\end{cases}
		\end{align}
		を満たすものが存在する.明らかに$f \circ i = f_0$が成り立っているから$T$が全射であることが示された.
	
	\item[(2)] $X_0$が$X$で稠密であるなら,任意の$f_0 \in X_0^*$はノルムを変えることなく$f \in X^*$に一意に拡張されるということが示される.
		まずはこのことを証明する.これが示されれば,(1)の結果と合わせて
		\begin{itemize}
			\item $T\ :\ X \longmapsto X_0$は線型全単射である.
			\item $\Norm{T x^*}{X_0^*} = \Norm{x^* \circ i}{X_0^*} = \Norm{x^*}{X^*}$
		\end{itemize}
		が成り立つことが示され,$T$がノルム空間としての同型写像であると判る.
		$X_0$が$X$で稠密であるということにより,任意の$x \in X$に対して
		$x_k \in X_0 \ (k=1,2,\cdots)$で$\Norm{x_k - x}{X} \longrightarrow 0\ (k \longrightarrow +\infty)$
		となるものを取れる.任意に$f_0 \in X_0^*$を取れば,任意の$m,n \in \N$に対して
		\begin{align}
			|f_0(x_m) - f_0(x_n)| \leq \Norm{f_0}{X_0^*} \Norm{x_m - x_n}{X}
		\end{align}
		が成り立つから,右辺が$X_0$のCauchy列をなすことにより$(f_0(x_n))_{n=1}^{+\infty}$も$\C$のCauchy列となる.
		$\C$の完備性から$(f_0(x_n))_{n=1}^{+\infty}$は或る$u \in \C$に収束する.この$u$は$x \in X$に対して一意に定まる.
		なぜならば,$x$への別の収束列$y_k \in X_0 \ (k=1,2,\cdots)$を取った場合,$(f_0(y_n))_{n=1}^{+\infty}$の収束先が
		$v \in \C$であるとして,任意の$n,m \in \N$に対して
		\begin{align}
			|u - v| &= |u - f_0(x_n) + f_0(x_n) - f_0(y_m) + f_0(y_m) - v| \\
			&\leq |u - f_0(x_n)| + |f_0(x_n) - f_0(y_m)| + |f_0(y_m) - v| \\
			&\leq |u - f_0(x_n)| + \Norm{f_0}{X_0^*}\Norm{x_n - y_m}{X} + |f_0(y_m) - v| \\
			&\leq |u - f_0(x_n)| + \Norm{f_0}{X_0^*}\left(\Norm{x_n - x}{X} + \Norm{x - y_m}{X}\right)+ |f_0(y_m) - v|
		\end{align}
		となるから$n,m \longrightarrow +\infty$で右辺は0に収束し,$u = v$が示されるためである.
		$x$に$u$を対応させる関係は$X \longmapsto \C$の写像となり,この写像を$f$と表すことにする.$f$の線型性も次のように示される.
		任意の$x,\ y \in X,\ \alpha,\ \beta \in \C$に対して,$x,y$への収束列$(x_k)_{k=1}^{+\infty},\ (y_k)_{k=1}^{+\infty} \subset X_0$
		を取れば$(\alpha x_k + \beta y_k)_{k=1}^{+\infty}$が$\alpha x+ \beta y$への収束列となるから
		\begin{align}
			|f(\alpha x + \beta y) - \alpha f(x) - \beta f(y)| 
			&= |f(\alpha x + \beta y) - f_0(\alpha x_k + \beta y_k) + f_0(\alpha x_k) + f_0(\beta y_k) - \alpha f(x) - \beta f(y)| \\
			&\leq |f(\alpha x + \beta y) - f_0(\alpha x_k + \beta y_k)| 
				+ |\alpha f_0(x_k) - \alpha f(x)| + |\beta f_0(y_k) - \beta f(y)| \\
			&\leq |f(\alpha x + \beta y) - f_0(\alpha x_k + \beta y_k)| 
				+ |\alpha| |f_0(x_k) - f(x)| + |\beta| |f_0(y_k) - f(y)| \\
			&\longrightarrow 0\quad (k \longrightarrow +\infty)
		\end{align}
		が成り立つ.ゆえに$f(\alpha x + \beta y) = \alpha f(x) + \beta f(y)\ (\forall x,\ y \in X,\ \alpha,\ \beta \in \C)$である.
		また$f$は有界な線型作用素である.なぜなら
		$x \in X_0$に対しては$x= x_k \in X_0\ (k=1,2,\cdots)$が$x$への一つの収束列になるから
		$\displaystyle f(x) = \lim_{k \to +\infty} f_0(x_k) = f_0(x)$となり$f$と$f_0$は$X_0$の上で一致する.($f$は$f_0$の拡張となっている.)
		従って
		\begin{align}
			|f(x)| = |f_0(x)| \leq \Norm{f_0}{X_0^*} \Norm{x}{X}\quad (\forall x \in X_0)
		\end{align}
		となり$f \in X^*$であることと$\Norm{f}{X^*} \leq \Norm{f_0}{X_0^*}$が判る.
		さらに作用素ノルムの性質より
		\begin{align}
			\Norm{f}{X^*} = \sup{\substack{x \in X \\ \Norm{x}{X} = 1}}{|f(x)|} 
			\geq \sup{\substack{x \in X_0 \\ \Norm{x}{X} = 1}}{|f(x)|} 
			= \sup{\substack{x \in X_0 \\ \Norm{x}{X} = 1}}{|f_0(x)|} = \Norm{f_0}{X_0^*}
		\end{align}
		も成り立つから結局$\Norm{f}{X^*} = \Norm{f_0}{X_0^*}$であると判る.以上より
		任意の$f_0 \in X_0^*$がノルムを変えないまま或る$f \in X^*$に一意に拡張されることが示された.
		最後にを示して終わる.
		任意に$x^* \circ i\ (x^* \in X^*)$を取り$x^* \circ i$を$x^* \circ i = f_0 \in X_0^*$と表現する.
		上記の結果より$f_0$は或る$f \in X^*$に拡張されるが,このとき$f = x^*$であることを示す,任意の$x \in X \backslash X_0$
		に対して$x_k \in X_0\ (k = 1,2,\cdots)$で
		$x_k \longrightarrow x\ (k \longrightarrow +\infty)$となるものを取れば
		\begin{align}
			\left| f(x) - x^*(x) \right| &= \left| f(x) - f_0(x_k) + \left(x^* \circ i \right)(x_k) - x^*(x) \right| \\
			&\leq \left| f(x) - f_0(x_k) \right| + \left|\left(x^* \circ i \right)(x_k) - x^*(x) \right| \\
			&\leq \left| f(x) - f(x_k) \right| + \left|x^*(x_k) - x^*(x) \right| \\
			&\leq \Norm{f}{X^*}\Norm{x_k - x}{X} + \Norm{x^*}{X^*}\Norm{x_k - x}{X} \\
			&\longrightarrow 0 \quad (k \longrightarrow +\infty)
		\end{align}
		が成り立つから$f(x) = x^*(x)\ (\forall x \in X \backslash X_0)$,すなわち$f(x) = x^*(x)\ (\forall x \in X)$
		が成り立つ.
	\item[(3)] 
\end{description}
\end{prf}

[11].\ 
	\begin{description}
		\item[(1)] $c_0$は$l^{\infty}$の閉線形部分空間であることを示せ.
		\item[(2)] $l^{\infty}$と$c_0$が可分であるかどうか判定せよ.
	\end{description}
	
\begin{prf}\mbox{}
\begin{description}
	\item[(1)] まず$c_0 \subset l^{\infty}$であることを示す.$\forall a=(a_n)_{n=1}^{\infty} \in c_0$は収束点列である.
		任意の$\epsilon > 0$に対して或る$N \in \N$を取れば,$N$以降の$n \in \N$については$|a_n| < \epsilon$で抑えられるから
		\begin{align}
			\sup{n \in \N}{|a_n|} \leq \max{}{\{|a_1|,\ |a_2|,\ \cdots,\ |a_N|,\ \epsilon\}} < +\infty
		\end{align}
		が成り立ち$a \in l^{\infty}$が判る.従って$c_0 \subset l^{\infty}$である.
		つぎに$c_0$が線型空間$l^{\infty}$の線形部分空間であることを示す.
		任意の$a=(a_n)_{n=1}^{\infty},\ b=(b_n)_{n=1}^{\infty} \in c_0,\ \alpha \in \C$
		に対し
		\begin{align}
			&|a_n + b_n| \leq |a_n| + |b_n| \longrightarrow 0 \quad (n \longrightarrow +\infty), \\
			&|\alpha a_n|= |\alpha||a_n| \longrightarrow 0 \quad (n \longrightarrow +\infty)
		\end{align}
		が成り立つことにより$a + b \in c_0$と$\alpha a \in c_0$が示される.従って$c_0$は$l^{\infty}$の線形部分空間である.
		最後に$c_0$が$l^{\infty}$で閉集合となっていることを示す.$l^{\infty}$は$\Norm{\cdot}{l^{\infty}}$をノルムとしてBanach空間となっているから,
		その部分空間である$c_0$が$\Norm{\cdot}{l^{\infty}}$をノルムとしてBanach空間をなしていることを示せばよい.
		$a^{(n)} = \left(a_m^{(n)}\right)_{m=1}^{\infty} \in c_0\ (m=1,2,3,\cdots)$を$\Norm{\cdot}{l^{\infty}}$に関するCauchy列とする. 
		$(l^{\infty},\ \Norm{\cdot}{l^{\infty}})$が完備であるから,$\left(a^{(n)}\right)_{n=1}^{\infty}$
		は或る$a^{*} = \left(a_m^{*}\right)_{m=1}^{\infty} \in l^{\infty}$
		に$m$に関して一様に収束している.つまり任意の$\epsilon > 0$に対して或る$N \in \N$が存在して,全ての$n > N$について
		\begin{align}
			\Norm{a^{(n)} - a^{*}}{l^{\infty}} = \sup{m \in \N}{\left|a_m^{(n)} - a_m^{*}\right|} < \epsilon \label{eq:func_analy_Q11}
		\end{align}
		が成り立っている.
		$a^{*} = (a_m^{*})_{m=1}^{\infty}$が$c_0$の元であることは帰謬法で示す.$a^* \notin c_0$であると仮定すると,或る$\delta > 0$に対しては,いかなる$N \in \N$を取っても
		必ず$n > N$なる自然数で
		\begin{align}
			|a_n^*| \geq \delta
		\end{align}
		を満たすものが存在する.$(a_m^*)_{m=1}^{\infty}$の部分列$(a_{m_k}^*)_{k=1}^{\infty}$を
		\begin{align}
			|a_{m_k}^*| \geq \delta, \qquad (m_k < m_{k+1},\ k= 1,2,3,\cdots)
		\end{align}
		を満たすものとして取ることが出来て,(\refeq{eq:func_analy_Q11})により
		或る$N_\delta \in \N$を取れば,全ての$n > N_\delta$で
		\begin{align}
			\sup{k \in \N}{\left|a_{m_k}^{(n)} - a_{m_k}^{*}\right|} < \frac{\delta}{2}
		\end{align}
		が成立することになる.
		$n > N_\delta$番目の数列$a^{(n)}=\left(a_m^{(n)}\right)_{m=1}^{\infty}$は$c_0$の元であるから,
		或る$N_\delta^n \in \N$が存在して全ての$p > N_\delta^n$に対し
		\begin{align}
			\left|a_p^{(n)}\right| < \frac{\delta}{2}
		\end{align}
		となるはずであるが,$m_1 < m_2 < m_3 < \cdots \rightarrow +\infty$であるから$m_k > N_\delta^n$となるような添数$m_k$が存在してしまい,
		\begin{align}
			\frac{\delta}{2} \leq \left|a_{m_k}^*\right| - \frac{\delta}{2} < \left|a_{m_k}^{(n)}\right| < \frac{\delta}{2}
		\end{align}
		と矛盾が出る.従って$a^* \in c_0$であるべきで,これは$c_0$が$\Norm{\cdot}{l^{\infty}}$をノルムとして完備であることを示したことになる.
		ゆえに$c_0$は$l^{\infty}$の閉線形部分空間である.
		
	\item[(2)]
		結論は,$l^{\infty}$は可分ではなく$c_0$は可分である.順番に示す.
		$l^{\infty}$の部分集合として0と1のみで成る数列全体
		\begin{align}
			M \coloneqq \left\{a \in l^{\infty}\, \left|\, a = (a_n)_{n=1}^{+\infty},\ a_n \in \{0,1\},\, n=1,2,3,\cdots \right.\right\}
		\end{align}
		を考える.また任意の$a=(a_n)_{n=1}^{\infty},\, b=(b_n)_{n=1}^{\infty} \in M$に対し
		\begin{align}
			\Norm{a-b}{l^{\infty}} = \sup{n \in \N}{|a_n - b_n|} = \begin{cases}
				1 & (a = b) \\
				0 & (a \neq b)
			\end{cases}
		\end{align}
		が成り立つから,$M$の異なる2元の$\Norm{\cdot}{l^{\infty}}$による距離は1で固定されている.もし$l^{\infty}$が可分であるとすれば,
		$\Norm{\cdot}{l^{\infty}}$に関して$l^{\infty}$で稠密な可算部分集合$C$が存在することになる.任意の$a \in M$に対して
		その$1/2$近傍(sup-norm)の内部に$C$の元が存在していることになるから,そのうちの一つを$c_a$と表し対応を付ける.
		$a \in M$に対応する$c_a \in C$は他の$M$の元の$1/2$近傍に属することはない.もし$c_a$が或る$a \neq b \in M$の$1/2$近傍に入ると
		\begin{align}
			1 = \Norm{a-b}{l^{\infty}} \leq \Norm{a - c_a}{l^{\infty}} + \Norm{b-c_a}{l^{\infty}} < \frac{1}{2} + \frac{1}{2} = 1
		\end{align}
		と矛盾ができるからである.即ち$M$から$C$への対応関係$M \ni a \longmapsto c_a \in C$は単射である.
		ここで$M$の濃度$2^{\N}$が連続体濃度であることに注意すれば,単射の存在により$C$の濃度は連続体濃度以上で
		$C$が可算集合であることに反する.従って$l^{\infty}$は可分ではない.
		一方で$c_0$は$\Norm{\cdot}{l^{\infty}}$をノルムとして可分なノルム空間をなす.$c_0$の可算部分集合を
		\begin{align}
			S \coloneqq 
			\left\{b \in c_0\, \left|\, 
			b = (\alpha_n + i \beta_n)_{n=1}^{+\infty},\ 
			\begin{cases}
				\alpha_n,\ \beta_n \in \Q, & n=1,2,\cdots,N,\\
				\alpha_n=\beta_n=0, & n \geq N+1,
			\end{cases}
			\ (N = 1,2,3,\cdots) \right.\right\}
		\end{align}
		として取る.ただし$i$は$i^2 = -1$なる虚数単位で$\Q$は有理数全体である.任意の$a=(a_n)_{n=1}^{+\infty} \in c_0$について,
		任意の正数$\epsilon > 0$に対して或る$N \in \N$を取れば全ての$n > N$で
		\begin{align}
			|a_n| < \epsilon
		\end{align}
		が成り立つから,後は$(a_n)_{n=1}^{N}$の部分で
		\begin{align}
			\sup{n=1,2,\cdots,N}{|a_n - b_n|} < \epsilon
		\end{align}
		となるように$S$の元$b=(b_n)_{n=1}^{+\infty}\ (b_n = 0,\ n>N)$を取れば
		\begin{align}
			\Norm{a - b}{l^{\infty}} < \epsilon
		\end{align}
		が成り立つ.即ち$S$が$c_0$において$\Norm{\cdot}{l^{\infty}}$に関して稠密であるとわかり,$c_0$が可分であると示された.
		\QED
\end{description}
\end{prf}

[12].\ $a = (a_n)_{n=1}^{\infty} \in l^1$に対して$T_a:c_0 \longmapsto \C$を次で定める:
	\begin{align}
		T_a(x) = \sum_{n=1}^{\infty}a_n x_n\ (x=(x_n) \in c_0).
	\end{align}
	\begin{description}
		\item[(1)] $\forall a = (a_n) \in l^1,\ T_a \in c_o^*$かつ$\Norm{T_a}{} = \Norm{a}{l^1}$であることを示せ.
		\item[(2)] $T:l^1 \ni a \longmapsto T_a \in c_0^*$はBanach空間としての同系写像であることを示せ.
	\end{description}

\begin{prf}
\begin{description}\mbox{}
	\item[(1)] 
		設問[11]の結果により,$T_a$の定義域である$c_0$は$\Norm{\cdot}{l^{\infty}}$をノルムとして
		$l^{\infty}$の閉線型部分空間であり,よってBanach空間である.このことに留意して以下進む.
		$a = (a_n)_{n=1}^{\infty} \in l^1$を任意に取って固定する.
		任意の$x=(x_n) \in c_0$に対して
		\begin{align}
			\sum_{n=1}^{\infty} |a_n| |x_n| \leq \sum_{n=1}^{\infty} |a_n| \Norm{x}{l^{\infty}} = \Norm{a}{l^1} \Norm{x}{l^{\infty}} < +\infty
			\label{eq:func_analy_Q12}
		\end{align}
		となり級数$T_a (x)\ (\forall x \in c_0)$は有限確定するから,
		任意の$x=(x_n), y=(y_n) \in c_0,\ \alpha \in \C$に対して以下に示す式変形が正当化される.
		\begin{align}
			\sum_{n=1}^{+\infty} a_n x_n + a_n y_n = \sum_{n=1}^{+\infty} a_n x_n + \sum_{n=1}^{+\infty} a_n y_n,\quad 
			\sum_{n=1}^{+\infty} \alpha a_n x_n = \alpha \sum_{n=1}^{+\infty} a_n x_n,\quad
			\left| \sum_{n=1}^{+\infty} a_n x_n \right| \leq \sum_{n=1}^{+\infty} |a_n||x_n|.
		\end{align}
		従って任意に$x=(x_n), y=(y_n) \in c_0$と$\alpha \in \C$を取れば.
		\begin{align}
			&T_a(x + y) = \sum_{n=1}^{\infty}a_n (x_n + y_n) 
			= \sum_{n=1}^{\infty} (a_n x_n + a_n y_n) = \sum_{n=1}^{\infty}a_n x_n + \sum_{n=1}^{\infty}a_n y_n = T_a x + T_a y, \\
			&T_a(\alpha x) = \sum_{n=1}^{\infty}a_n(\alpha x_n) = \sum_{n=1}^{\infty}(a_n\alpha)x_n = \sum_{n=1}^{\infty}(\alpha a_n)x_n
			= \sum_{n=1}^{\infty}\alpha (a_n x_n) = \alpha \sum_{n=1}^{\infty} a_n x_n = \alpha T_a(x)
		\end{align}
		により$T_a$の線型性が示されるから,$T_a$は$c_0 \rightarrow \C$の線型汎関数である.
		有界性は式(\refeq{eq:func_analy_Q12})により
		\begin{align}
			|T_a(x)| \leq \Norm{a}{l^1} \Norm{x}{l^{\infty}}
		\end{align}
		から$\Norm{T_a}{} \leq \Norm{a}{l^1}$となるとわかる.ゆえに$T_a \in c_0^*\ (\forall a \in l^1)$である.
		また$\Norm{T_a}{} = \Norm{a}{l^1}$について,$a=0$の場合は$T_a$が零作用素になるから明らかに成り立つ.
		$a \neq 0$の場合,任意の$\Norm{a}{l^1} >> \epsilon >0$に対して或る$N \in \N$が存在して
		\begin{align}
			\Norm{a}{l^1} - \epsilon < \sum_{n = 1}^{N} |a_n|
		\end{align}
		とできる.$x \in c_0$を
		\begin{align}
			x_n = \begin{cases}
				\overline{a_n}/|a_n| & a_n \neq 0,\rm{and}\ n \leq N \\
				0 & a_n = 0,\rm{or}\ n > N
			\end{cases}
		\end{align}
		となっているもので取れば,
		\begin{align}
			\Norm{a}{l^1} - \epsilon < \sum_{n = 1}^{N} |a_n| = \sum_{n = 1}^{\infty} a_n x_n = T_a(x) = |T_a(x)|
		\end{align}
		が成立することになる.$\Norm{x}{l^{\infty}}=1$であることに注意すれば
		\begin{align}
			\Norm{a}{l^1} - \epsilon 
			< T_a(x) = \frac{|T_a(x)|}{\Norm{x}{l^{\infty}}} 
			\leq \sup{0 \neq y \in c_0}{\frac{|T_a(y)|}{\Norm{y}{l^{\infty}}}} = \Norm{T_a}{} \leq \Norm{a}{l^1}
		\end{align}
		が成り立ち,$\epsilon$が任意であるから,$a=0$の場合と合わせて$\Norm{T_a}{} = \Norm{a}{l^1}\ (\forall a \in l^1)$が示された.
	
	\item[(2)] まず,$l^1$は$\Norm{\cdot}{l^1}$をノルムとしてBanach空間となり,$c_0^*$は$T_a$の値域$\C$がBanach空間であるから作用素ノルム
		によりBanach空間となっている.写像$T$が
		\begin{align}
			\Norm{Ta}{} = \Norm{T_a}{} = \Norm{a}{l^1}\quad (\forall a \in l^1)
		\end{align}
		の意味で等長であることは(1)で示してあるから,後は$T$が線型全単射であることを証明すればよい.
		任意の$a = (a_n), b=(b_n) \in l^1,\ \alpha \in \C$に対して
		\begin{align}
			&\sum_{n=1}^{\infty} (a_n + b_n) x_n 
			= \sum_{n=1}^{\infty} (a_n x_n + b_n x_n) 
			=  \sum_{n=1}^{\infty} a_n x_n + \sum_{n=1}^{\infty} b_n x_n, \\
			&\sum_{n=1}^{\infty} (\alpha a_n) x_n
			= \sum_{n=1}^{\infty} \alpha (a_n x_n)
			= \alpha \sum_{n=1}^{\infty} a_n x_n
		\end{align}
		が成り立つから
		\begin{align}
			&T(a+b) = T_{a+b} = T_a + T_b = Ta + Tb \\
			&T(\alpha a) = T_{\alpha a} = \alpha T_a = \alpha Ta
		\end{align}
		も成り立つことにより$T$の線型性が示される.
		また$a = (a_n), b=(b_n) \in l^1$に対して,$a \neq b$であるなら或る$N \in \N$番目で
		$a_N \neq b_N$となっているはずであるから,
		\begin{align}
			x_N = \begin{cases}
				1 & n=N \\
				0 & n \neq N
			\end{cases}
		\end{align}
		となる$x=(x_n) \in c_0$に対して
		\begin{align}
			T_a(x) = a_N \neq b_N = T_b(x)
		\end{align}
		となり,$T$が単射であることが示される.最後に$T$が全射であることを示す.
		任意に$L \in c_0^*$を取る.或る$a \in l^1$に対して$L$が$T_a$に一致することを見ればよい.
		Kroneckerのデルタを用いて
		\begin{align}
			e_n \coloneqq (\delta_{jn})_{j=1}^{\infty}
		\end{align}
		で表現される$e_n\ (n=1,2,3,\cdots)$は$c_0$の元であり,各$n = 1,2,3,\cdots$に対して
		\begin{align}
			a_n \coloneqq L(e_n) \label{eq:def_a_n_in_l1}
		\end{align}
		とおく.任意の$x=(x_n) \in c_0$に対して
		\begin{align}
			x^{(N)} = \sum_{n=1}^{N} x_n e_n, \quad (N = 1,2,3,\cdots)
		\end{align}
		として作られる数列の族$\left(x^{(N)}\right)_{N=1}^{\infty}$は$l^{\infty}$において収束し,$x^{(N)} \rightarrow x\ (N \rightarrow +\infty)$が成り立つ.
		これは$x = (x_n)$が収束数列であることによる.任意の$\epsilon > 0$に対して或る$N \in \N$を選べば
		\begin{align}
			|x_n| < \epsilon, \quad (\forall n > N)
		\end{align}
		が成り立つから,$n>N$なる任意の自然数$n$に対して
		\begin{align}
			\Norm{x - x^{(n)}}{l^{\infty}} = \sup{m > n}{|x_m|} < \epsilon
		\end{align}
		となり,$x^{(N)} \rightarrow x\ (N \rightarrow +\infty)$が示されるのである.
		$L$が有界線型汎関数であることも併せれば
		\begin{align}
			\left|L(x) - L(x^{(N)})\right| \leq \Norm{L}{}\Norm{x - x^{(N)}}{l^{\infty}} \longrightarrow 0\quad (N \longrightarrow +\infty) 
		\end{align}
		により
		\begin{align}
			L(x) = \lim_{N \to +\infty} L(x^{(N)}) 
			= \lim_{N \to +\infty} \sum_{n=1}^{N} L(x_n e_n)
			= \lim_{N \to +\infty} \sum_{n=1}^{N} a_n x_n
			= \sum_{n=1}^{+\infty} a_n x_n \quad (\forall x \in c_0)
		\end{align}
		が成り立つ.後は(\refeq{eq:def_a_n_in_l1})で定義した複素数列$(a_n)_{n=1}^{\infty}$が$l^1$に属していることを示せば,
		写像$L:c_0 \longmapsto \C$が$T_a:c_0 \longmapsto \C$に一致していると証明される.これは次のように示される.
		$a_n \neq 0\ (\exists n \leq M)$となるような$M \in \N$を取って,この$M$に対して$x = (x_n) \in c_0$として
		\begin{align}
			x_n = \begin{cases}
				\overline{a_n}/|a_n| & a_n \neq 0,\rm{and}\ n \leq M \\
				0 & a_n = 0,\rm{or}\ n > M
			\end{cases}
		\end{align}
		となるものを取れば,
		\begin{align}
			L(x) = \lim_{N \to +\infty} \sum_{n=1}^{N} a_n x_n = \sum_{n=1}^{M} |a_n|
		\end{align}
		が成り立つ.$\Norm{x}{l^{\infty}} = 1$であることに注意すれば
		\begin{align}
			\sum_{n=1}^{M} |a_n| = L(x) = |L(x)| \leq \Norm{L}{} \Norm{x}{l^{\infty}} = \Norm{L}{}
		\end{align}
		となる.$M$は任意に大きく取って問題ないから,
		\begin{align}
			\sum_{n=1}^{M'} |a_n| \leq \Norm{L}{}, \quad (\forall M' \geq M)
		\end{align}
		が成り立ち,従って
		\begin{align}
			\sum_{n=1}^{\infty} |a_n| \leq \Norm{L}{}
		\end{align}
		となることにより$a \in l^1$であることが示された.
\end{description}
\end{prf}

[13].\ $1 < p < \infty$とする.$l^p$の点列$x(1),x(2),x(3),\cdots$(ただし$x(j)=\left(x(j)_n\right)_{n=1}^{\infty}$)と
$x=(x_n)_{n=1}^{\infty} \in l^p$に対し,次の(i),\ (ii)が同値であることを示せ:
\begin{description}
	\item[(1)] $\displaystyle \wlim_{n \to \infty} x(j) = x.$
	\item[(2)] $\left\{x(j);j\in\N\right\}$は有界でかつ$\forall n \in \N\ \displaystyle \lim_{j \to \infty}x(j)_n = x_n$.
\end{description}

\begin{prf}
\begin{description}\mbox{}
	\item[$l^p$の共役空間] まず$l^p$の共役空間の任意の元$f \in \left(l^p\right)^*$が或る$v \in l^q$に対応して
		\begin{align}
			f(x) = \sum_{n=1}^{\infty} x_n v_n, \quad \left(x = (x_n)_{n=1}^{\infty} \in l^p,\ v = (v_n)_{n=1}^{\infty} \in l^q\right)
		\end{align}
		で表現されることを示す.
	\item[(1) $\Rightarrow$ (2)] 任意の$f \in \left(l^p\right)^*$に対して,$f$には或る$v = (v_n)_{n=1}^{\infty} \in l^q$が対応して
		\begin{align}
			f(x(j)) = \sum_{n=1}^{\infty} x(j)_n v_n
		\end{align}
		と表すことができる.弱収束の仮定より任意の$\epsilon > 0$に対してある$J \in \N$が存在して,全ての$j > J$で
		\begin{align}
			|f(x(j)) - f(x)| < \epsilon
		\end{align}
		が成り立っている.$x(j),x \in l^p,\ v \in l^q$であるからH\Ddot{o}lderの不等式により
		\begin{align}
			|f(x(j)) - f(x)| = \left|\sum_{n=1}^{\infty} x(j)_n v_n - \sum_{n=1}^{\infty} x_n v_n  \right|
			= \left|\sum_{n=1}^{\infty} \left(x(j)_n - x_n\right) v_n \right|
			\leq \Norm{x(j) - x}{l^p}\Norm{v}{l^q}
		\end{align}
		と表現することができる.
	\item[(2) $\Rightarrow$ (1)] $x(j)$が有界であるから,或る正数$M$が存在して
		\begin{align}
			\sup{j \in \N}{\Norm{x(j)}{l^p}} \leq M
		\end{align}
		が成り立っている.即ち
		\begin{align}
			\sup{j,n \in \N}{\left|x(j)_n\right|} < +\infty
		\end{align}
		が成り立っていることになる.また$\forall n \in \N\ \displaystyle \lim_{j \to \infty}x(j)_n = x_n$の意味は
		$n \in \N$に関して$x(j)$が$x$に各点収束しているということである.$f \in \left(l^p\right)^*$を任意に取り,
		$f(x(j)),\ f(x)$が$v=(v_n)_{n=1}^{\infty} \in l^q$を用いて
		\begin{align}
			f(x(j)) = \sum_{n=1}^{\infty} x(j)_n v_n,\quad f(x) = \sum_{n=1}^{\infty} x_n v_n
		\end{align}
		と表現できているとする.測度空間$(\N, 2^{\N}, v)\ (v(n) = v_n)$における
		積分の収束を考えているとみなせば,以上の仮定によりLebesgueの優収束定理が適用できて
		\begin{align}
			\lim_{j \to \infty} \sum_{n=1}^{\infty} x(j)_n v_n = \sum_{n=1}^{\infty} \lim_{j \to \infty} x(j)_n v_n = \sum_{n=1}^{\infty} x_n v_n
		\end{align}
		が成り立つ.これは$f(x(j)) \longrightarrow f(x)\ (j \longrightarrow +\infty)$を意味していて,線型汎関数$f \in \left(l^p\right)^*$
		は任意に取っているから$\displaystyle \wlim_{n \to \infty} x(j) = x$が示されたことになる.
\end{description}
\end{prf}

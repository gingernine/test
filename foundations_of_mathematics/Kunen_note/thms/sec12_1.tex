\section{完全性定理}
	\begin{screen}
		\begin{thm}[補題2.12.3]
			補題2.12.2から定理2.12.1が得られる.
		\end{thm}
	\end{screen}
	
	\begin{sketch}
		健全性定理より$\Sigma \vdash \varphi$なら$\Sigma \models \varphi$となる.
		$\CON{\models}{\Sigma}$であれば,$\mathfrak{A} \models \Sigma$なるモデル$\mathfrak{A}$が取れるが,
		$\Sigma \vdash \varphi$ならば$\mathfrak{A} \models \varphi$となる.従って
		$\mathfrak{A} \not\models\ \negation \varphi$となる.従って
		$\Sigma \not\vdash\ \negation \varphi$となる.以上より
		\begin{align}
			\CON{\models}{\Sigma} \Longrightarrow \CON{\vdash}{\Sigma}
		\end{align}
		となる.$\Sigma \models \varphi$ならば$\Sigma \cup \{\negation \varphi\}$を充足する
		モデルは存在しない.つまり$\negation \CON{\models}{\Sigma \cup \{\negation \varphi\}}$.
		すなわち
		\begin{align}
			\Sigma \cup \{\negation \varphi\} \vdash \bot.
		\end{align}
		すなわち$\Sigma \vdash \varphi$.
		\QED
	\end{sketch}
	
	\begin{screen}
		\begin{thm}[補題2.12.6]
			$\tau \in CT_{0}(\mathcal{L})$のとき
			$\val{\mathfrak{A}_{0}}{\tau} \equiv \tau$.
		\end{thm}
	\end{screen}
	
	\begin{sketch}
		$\tau \in \mathcal{F}_{0}$なら
		$\val{\mathfrak{A}_{0}}{\tau} \equiv \tau_{\mathfrak{A}_{0}} \equiv \tau$.
		いま$\tau_{1},\cdots,\tau_{n} \in CT_{0}(\mathcal{L})$に対して
		\begin{align}
			\val{\mathfrak{A}_{0}}{\tau_{i}} \equiv \tau_{i},
			\quad (i=1,\cdots,n)
		\end{align}
		と仮定すると,
		\begin{align}
			\val{\mathfrak{A}_{0}}{f\tau_{1}\cdots\tau_{n}} 
			&\equiv f_{\mathfrak{A}_{0}}(\val{\mathfrak{A}_{0}}{\tau_{1}},\cdots,\val{\mathfrak{A}_{0}}{\tau_{n}}) \\
			&\equiv f_{\mathfrak{A}_{0}}(\tau_{1},\cdots,\tau_{n}) \\
			&\equiv f\tau_{1}\cdots\tau_{n}
		\end{align}
		となる.
	\end{sketch}
	
	\begin{screen}
		\begin{thm}[定義2.12.9の正当性の検証]
			$\val{\mathfrak{A}}{}$は商写像であるから同地類の代表云々は関係ない.
			問題は$[\tau_{i}] \equiv [\sigma_{i}]$のとき
			$f\tau_{1}\cdots\tau_{n} \sim f\sigma_{1}\cdots\sigma_{n}$となり,
			$\Sigma \vdash p\tau_{1}\cdots\tau_{n} \Longleftrightarrow 
			\Sigma \vdash p\sigma_{1}\cdots\sigma_{n}$となるか.
		\end{thm}
	\end{screen}
\section{完全性定理}
	\begin{screen}
		\begin{thm}[補題2.12.3]
			補題2.12.2から定理2.12.1が得られる.
		\end{thm}
	\end{screen}
	
	\begin{sketch}
		健全性定理より$\Sigma \vdash \varphi$なら$\Sigma \models \varphi$となる.
		$\CON{\models}{\Sigma}$であれば,$\mathfrak{A} \models \Sigma$なるモデル$\mathfrak{A}$が取れるが,
		$\Sigma \vdash \varphi$ならば$\mathfrak{A} \models \varphi$となる.従って
		$\mathfrak{A} \not\models\ \negation \varphi$となる.従って
		$\Sigma \not\vdash\ \negation \varphi$となる.以上より
		\begin{align}
			\CON{\models}{\Sigma} \Longrightarrow \CON{\vdash}{\Sigma}
		\end{align}
		となる.$\Sigma \models \varphi$ならば$\Sigma \cup \{\negation \varphi\}$を充足する
		モデルは存在しない.つまり$\negation \CON{\models}{\Sigma \cup \{\negation \varphi\}}$.
		すなわち
		\begin{align}
			\Sigma \cup \{\negation \varphi\} \vdash \bot.
		\end{align}
		すなわち$\Sigma \vdash \varphi$.
		\QED
	\end{sketch}
	
	\begin{screen}
		\begin{thm}[補題2.12.6]
			$\tau \in CT_{0}(\mathcal{L})$のとき
			$\val{\mathfrak{A}_{0}}{\tau} \equiv \tau$.
		\end{thm}
	\end{screen}
	
	\begin{sketch}
		$\tau \in \mathcal{F}_{0}$なら
		$\val{\mathfrak{A}_{0}}{\tau} \equiv \tau_{\mathfrak{A}_{0}} \equiv \tau$.
		いま$\tau_{1},\cdots,\tau_{n} \in CT_{0}(\mathcal{L})$に対して
		\begin{align}
			\val{\mathfrak{A}_{0}}{\tau_{i}} \equiv \tau_{i},
			\quad (i=1,\cdots,n)
		\end{align}
		と仮定すると,
		\begin{align}
			\val{\mathfrak{A}_{0}}{f\tau_{1}\cdots\tau_{n}} 
			&\equiv f_{\mathfrak{A}_{0}}(\val{\mathfrak{A}_{0}}{\tau_{1}},\cdots,\val{\mathfrak{A}_{0}}{\tau_{n}}) \\
			&\equiv f_{\mathfrak{A}_{0}}(\tau_{1},\cdots,\tau_{n}) \\
			&\equiv f\tau_{1}\cdots\tau_{n}
		\end{align}
		となる.
	\end{sketch}
	
	\begin{screen}
		\begin{thm}[定義2.12.9の正当性の検証]
			$\val{\mathfrak{A}}{}$は商写像であるから同地類の代表云々は関係ない.
			問題は$[\tau_{i}] \equiv [\sigma_{i}]$のとき
			$f\tau_{1}\cdots\tau_{n} \sim f\sigma_{1}\cdots\sigma_{n}$となり,
			$\Sigma \vdash p\tau_{1}\cdots\tau_{n} \Longleftrightarrow 
			\Sigma \vdash p\sigma_{1}\cdots\sigma_{n}$となるか.
		\end{thm}
	\end{screen}
	
	つまり$f_{\mathfrak{A}}$が写像であるということを示すということ.
	定義2.12.9では商写像$\val{\mathfrak{A}}{\tau} \equiv [\tau]$から始めて,
	$f \in \mathcal{F}_{n}$の解釈を
	\begin{align}
		f_{\mathfrak{A}}:
		([\tau_{1}],\cdots,[\tau_{n}]) \longmapsto
		\val{\mathfrak{A}}{f\tau_{1}\cdots\tau_{n}}
		\equiv [f\tau_{1}\cdots\tau_{n}]
	\end{align}
	と定めている.
	
	\begin{screen}
		\begin{thm}[補題2.12.10]
			語彙$\mathcal{L}$の文の集合$\Sigma$を考える.ただし$\mathcal{F}_{0} \neq \emptyset$と仮定する.
			$\mathfrak{A} = \mathfrak{CT}(\mathcal{L},\Sigma)$としよう.このとき
			\begin{description}
				\item[(1)] $\mathcal{L}$の閉項$\tau$に対して$\val{\mathfrak{A}}{\tau} \equiv [\tau]$.
			\end{description}
		\end{thm}
	\end{screen}
	
	\begin{sketch}
		(1)について,$\val{\mathfrak{A}}{}$はもともと商写像として設定されているのでこの問いはナンセンス.
		
	\end{sketch}
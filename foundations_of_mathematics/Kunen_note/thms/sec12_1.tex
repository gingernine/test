\section{完全性定理}
	\begin{screen}
		\begin{thm}[補題2.12.3]
			補題2.12.2から定理2.12.1が得られる.
		\end{thm}
	\end{screen}
	
	\begin{sketch}
		健全性定理より$\Sigma \vdash \varphi$なら$\Sigma \models \varphi$となる.
		$\CON{\models}{\Sigma}$であれば,$\mathfrak{A} \models \Sigma$なるモデル$\mathfrak{A}$が取れるが,
		$\Sigma \vdash \varphi$ならば$\mathfrak{A} \models \varphi$となる.従って
		$\mathfrak{A} \not\models\ \negation \varphi$となる.従って
		$\Sigma \not\vdash\ \negation \varphi$となる.以上より
		\begin{align}
			\CON{\models}{\Sigma} \Longrightarrow \CON{\vdash}{\Sigma}
		\end{align}
		となる.$\Sigma \models \varphi$ならば$\Sigma \cup \{\negation \varphi\}$を充足する
		モデルは存在しない.つまり$\negation \CON{\models}{\Sigma \cup \{\negation \varphi\}}$.
		すなわち
		\begin{align}
			\Sigma \cup \{\negation \varphi\} \vdash \bot.
		\end{align}
		すなわち$\Sigma \vdash \varphi$.
		\QED
	\end{sketch}
	
	\begin{screen}
		\begin{thm}[補題2.12.6]
			$\tau \in CT_{0}(\mathcal{L})$のとき
			$\val{\mathfrak{A}_{0}}{\tau} \equiv \tau$.
		\end{thm}
	\end{screen}
	
	\begin{sketch}
		$\tau \in \mathcal{F}_{0}$なら
		$\val{\mathfrak{A}_{0}}{\tau} \equiv \tau_{\mathfrak{A}_{0}} \equiv \tau$.
		いま$\tau_{1},\cdots,\tau_{n} \in CT_{0}(\mathcal{L})$に対して
		\begin{align}
			\val{\mathfrak{A}_{0}}{\tau_{i}} \equiv \tau_{i},
			\quad (i=1,\cdots,n)
		\end{align}
		と仮定すると,
		\begin{align}
			\val{\mathfrak{A}_{0}}{f\tau_{1}\cdots\tau_{n}} 
			&\equiv f_{\mathfrak{A}_{0}}(\val{\mathfrak{A}_{0}}{\tau_{1}},\cdots,\val{\mathfrak{A}_{0}}{\tau_{n}}) \\
			&\equiv f_{\mathfrak{A}_{0}}(\tau_{1},\cdots,\tau_{n}) \\
			&\equiv f\tau_{1}\cdots\tau_{n}
		\end{align}
		となる.
	\end{sketch}
	
	\begin{screen}
		\begin{thm}[定義2.12.9の正当性の検証]
			$\val{\mathfrak{A}}{}$は商写像であるから同地類の代表云々は関係ない.
			問題は$[\tau_{i}] \equiv [\sigma_{i}]$のとき
			$f\tau_{1}\cdots\tau_{n} \sim f\sigma_{1}\cdots\sigma_{n}$となり,
			$\Sigma \vdash p\tau_{1}\cdots\tau_{n} \Longleftrightarrow 
			\Sigma \vdash p\sigma_{1}\cdots\sigma_{n}$となるか.
		\end{thm}
	\end{screen}
	
	つまり$f_{\mathfrak{A}}$が写像であるということを示すということ.
	定義2.12.9では商写像$\val{\mathfrak{A}}{\tau} \equiv [\tau]$から始めて,
	$f \in \mathcal{F}_{n}$の解釈を
	\begin{align}
		f_{\mathfrak{A}}:
		([\tau_{1}],\cdots,[\tau_{n}]) \longmapsto
		\val{\mathfrak{A}}{f\tau_{1}\cdots\tau_{n}}
		\equiv [f\tau_{1}\cdots\tau_{n}]
	\end{align}
	と定めている.また$p_{\mathfrak{A}}$は
	\begin{align}
		p_{\mathfrak{A}} \defeq \Set{([\tau_{1}],\cdots,[\tau_{n}])}{
		\Sigma \vdash p\tau_{1}\cdots\tau_{n}}
	\end{align}
	により定めている.原子式に対する$\val{\mathfrak{A}}{}$は
	\begin{align}
		\val{\mathfrak{A}}{\varphi} = 
		\begin{cases}
			T &\mbox{if } \Sigma \provable{\mathcal{L}} \varphi, \\
			F &\mbox{if } \neg\Sigma \provable{\mathcal{L}} \varphi
		\end{cases}
	\end{align}
	ということになるが,$\Sigma$が極大無矛盾ならば量化子フリーな任意の文$\varphi$に対しても上が成り立つ.
	
	\begin{screen}
		\begin{thm}[補題2.12.10]
			語彙$\mathcal{L}$の文の集合$\Sigma$を考える.ただし$\mathcal{F}_{0} \neq \emptyset$と仮定する.
			$\mathfrak{A} = \mathfrak{CT}(\mathcal{L},\Sigma)$としよう.このとき
			\begin{description}
				\item[(1)] $\mathcal{L}$の閉項$\tau$に対して$\val{\mathfrak{A}}{\tau} \equiv [\tau]$.
				\item[(3)] $\varphi$が原子論理式である文のときは,$\Sigma \vdash \varphi$と
					$\mathfrak{A} \models \varphi$が同値.
				\item[(4)] 
			\end{description}
		\end{thm}
	\end{screen}
	
	\begin{sketch}
		(1)について,$\val{\mathfrak{A}}{}$はもともと商写像として設定されているのでこの問いはナンセンス.
		(3)について,$\Sigma \vdash p\tau_{1}\cdots\tau_{n}$と
			$([\tau_{1}],\cdots,[\tau_{n}]) \in p_{\mathfrak{A}}$は同値.
			\begin{align}
				([\tau_{1}],\cdots,[\tau_{n}]) \equiv 
				(\val{\mathfrak{A}}{\tau_{1}},\cdots,\val{\mathfrak{A}}{\tau_{n}}).
			\end{align}
			$(\val{\mathfrak{A}}{\tau_{1}},\cdots,\val{\mathfrak{A}}{\tau_{n}}) \in p_{\mathfrak{A}}$
			と$\mathfrak{A} \models p\tau_{1}\cdots\tau_{n}$は同値.
			他方で,$\Sigma \vdash \tau = \sigma$と$[\tau] \equiv [\sigma]$は同値.
			つまり
			\begin{align}
				\Sigma \vdash \tau = \sigma 
				&\Longleftrightarrow 
				\val{\mathfrak{A}}{\tau} \equiv \val{\mathfrak{A}}{\sigma} \\
				&\Longleftrightarrow \val{\mathfrak{A}}{\tau = \sigma} = T
			\end{align}
			となる.
	\end{sketch}
	
	$\Sigma$が矛盾していれば,任意の文$\varphi$で$\val{\mathfrak{A}}{\varphi} = 1$となり,
	$\val{\mathfrak{A}}{\negation \varphi} = 0$となるが,
	$\val{\mathfrak{A}}{\negation\varphi} = 1$と衝突する.
	従って$\val{\mathfrak{A}}{}$は写像ではない.ゆえに$\mathfrak{A}$はモデルとはなりえない.
	
	\begin{screen}
		\begin{dfn}[定義2.12.11]
			語彙$\mathcal{L}$の文の集合$\Sigma$は次の条件を満たすとき極大$(\vdash,\mathcal{L})$
			無矛盾であるといわれる:
			\begin{description}
				\item[(1)] $\CON{\vdash,\mathcal{L}}{\Sigma}$
				\item[(2)] $\Sigma$に属さないいかなる文$\varphi$に対しても
					$\Sigma \cup \{\varphi\} \provable{\mathcal{L}} \bot$.
			\end{description}
		\end{dfn}
	\end{screen}
	
	\begin{screen}
		\begin{thm}[補題2.12.12]
			$\Delta$が$\mathcal{L}$の文の集合で$\CON{\vdash,\mathcal{L}}{\Delta}$であるものとする.
			$\mathcal{L}$の文の集合$\Sigma$で,$\Delta$を含んで極大$(\vdash,\mathcal{L})$無矛盾である
			ようなものが存在する.
		\end{thm}
	\end{screen}
	
	\begin{sketch}
		$S$を$\mathcal{L}$の文の全体とする.このとき
		\begin{align}
			\forall \Pi \in \power{S}\, 
			(\, \CON{\vdash,\mathcal{L}}{\Pi} \Longleftrightarrow 
			\forall \Omega \subset \Pi\, (\, \Fin{\Omega} \Longrightarrow \CON{\vdash,\mathcal{L}}{\Omega}\, )\, )
		\end{align}
		が成り立つ.実際,$\CON{\vdash,\mathcal{L}}{\Pi}$のとき,$\Omega$を$\Pi$の有限部分集合として
		$\Omega \provable{\mathcal{L}} \varphi$であるとすると,$\Omega$からの任意の演繹の列$\pi$は
		$\Pi$からの演繹の列でもあって,$\CON{\vdash,\mathcal{L}}{\Pi}$より$\pi$の終点は
		$\negation \varphi$ではありえない.すなわち$\CON{\vdash,\mathcal{L}}{\Omega}$である.
		今度は$\Pi$の任意の有限部分集合$\Omega$に対して$\CON{\vdash,\mathcal{L}}{\Omega}$であるとし,
		\begin{align}
			\Pi \provable{\mathcal{L}} \varphi
		\end{align}
		とする.この演繹の列を$\pi$とし,$\pi$に使われている$\Pi$の公理の全体を$Ax(\pi)$とする.
		このとき,$\Pi$からの任意の演繹の列$\rho$を取っても,
		\begin{align}
			\Omega \defeq Ax(\pi) \cup Ax(\rho)
		\end{align}
		とおけば$\CON{\vdash,\mathcal{L}}{\Omega}$かつ$\Omega \provable{\mathcal{L}} \varphi$なので,
		$\Omega$から$\negation \varphi$は導かれない.$\rho$は$\Omega$からの演繹の列でもあるから
		$\rho$の終点は$\negation \varphi$ではありえない.
		つまりどのように$\rho$を取っても$\negation \varphi$に到着することはない.ゆえに
		$\CON{\vdash,\mathcal{L}}{\Pi}$である.
	\end{sketch}
	
	\begin{screen}
		\begin{thm}[補題2.12.13]
			$\mathcal{L}$の文の極大$(\vdash,\mathcal{L})$無矛盾な集合$\Sigma$を考える.
			このとき$\mathcal{L}$の任意の文$\varphi$と$\psi$について,
			\begin{description}
				\item[(1)] $\varphi \in \Sigma \Longleftrightarrow 
					\Sigma \provable{\mathcal{L}} \varphi$;
				\item[(2)] $\varphi \notin \Sigma \Longleftrightarrow 
					\Sigma \provable{\mathcal{L}}\ \negation \varphi$;
				\item[(3)] $\varphi \vee \psi \in \Sigma \Longleftrightarrow$
					$\varphi \in \Sigma$または$\psi \in \Sigma$.
			\end{description}
		\end{thm}
	\end{screen}
	
	極大無矛盾なら,任意の文は「証明される」か「否定が証明される」の二択.
	
	\begin{sketch}
		証明可能性の定義より$\varphi \in \Sigma$ならば$\Sigma \provable{\mathcal{L}} \varphi$である.
		また極大無矛盾の定義より
		\begin{align}
			\varphi \notin \Sigma 
			&\Longrightarrow \Sigma \cup \{\varphi\} \provable{\mathcal{L}} \bot \\
			&\Longrightarrow \Sigma \provable{\mathcal{L}}\ \negation \varphi
		\end{align}
		が成り立つから,$\Sigma \provable{\mathcal{L}} \varphi$なら無矛盾性より
		$\neg \Sigma \provable{\mathcal{L}}\ \negation \varphi$となるので
		$\varphi \in \Sigma$が従う.(2)については,$\Sigma \provable{\mathcal{L}}\ \negation \varphi$
		であれば無矛盾性より$\neg \Sigma \provable{\mathcal{L}} \varphi$となるので
		(1)から$\varphi \notin \Sigma$が従う.(3)について,$\varphi \vee \psi \in \Sigma$ならば
		(1)より$\Sigma \provable{\mathcal{L}} \varphi \vee \psi$となるが,ここで選言三段論法より
		\begin{align}
			\Sigma \provable{\mathcal{L}}\ \negation \varphi \rarrow \psi
		\end{align}
		が成り立つ.このとき,$\varphi \notin \Sigma$ならば(2)より
		$\Sigma \provable{\mathcal{L}}\ \negation \varphi$となり,
		$\Sigma \provable{\mathcal{L}} \psi$が従い,(1)より$\psi \in \Sigma$となる.
		逆に$\varphi \in \Sigma$のときは(1)より$\Sigma \provable{\mathcal{L}} \varphi$となり
		論理和の導入より$\Sigma \provable{\mathcal{L}} \varphi \vee \psi$となり
		再び(1)より$\varphi \vee \psi \in \Sigma$となる.同様に$\psi \in \Sigma$のときも
		$\varphi \vee \psi \in \Sigma$となるから,(3)の$\Longleftarrow$が得られる.
		\QED
	\end{sketch}
	
	\begin{screen}
		\begin{thm}[補題2.12.14]
			語彙$\mathcal{L}$は$\mathcal{F}_{0} \neq \emptyset$であるものとし,文の集合$\Sigma$は
			極大$(\vdash,\mathcal{L})$無矛盾であるとする.$\mathfrak{A} = \mathfrak{CT}
			(\mathcal{L},\Sigma)$とする.$\mathcal{L}$の文$\varphi$が量化記号を用いないものであれば,
			$\Sigma \provable{\mathcal{L}} \varphi$と$\mathfrak{A} \models \varphi$は同値である.
		\end{thm}
	\end{screen}
\section{超限帰納法}
	$x$を任意に与えられた集合としたとき,$x$の任意の要素$y$で
	\begin{align}
		A(y)
	\end{align}
	が成り立つならば
	\begin{align}
		A(x)
	\end{align}
	が成り立つとする.すると,なんと$A(x)$は普遍的に成り立つのである.つまり
	\begin{align}
		\forall x\, \left[\, \forall y \in x\, A(y) \Longrightarrow A(x)\, \right]
		\Longrightarrow \forall x A(x)
	\end{align}
	が成り立つわけだが,この事実を本稿では{\bf 集合の帰納法}と呼ぶ.また派生形としては,
	集合を順序数に制限した場合の{\bf 超限帰納法}\index{ちょうげんきのうほう@超限帰納法}{\bf (transfinite induction)}と,
	自然数に制限した場合の{\bf 数学的帰納法}\index{すうがくてききのうほう@数学的帰納法}{\bf (mathematical induction)}がある.
	
	\begin{screen}
		\begin{thm}[集合の帰納法]\label{thm:equivalent_condition_of_axiom_of_regularity}
			$A$を$\mathcal{L}'$の式とし,$x$を$A$に現れる文字とし,$y$を$A$に現れない文字とし,
			$A$に現れる文字で$x$のみが量化されていないとする.このとき
			\begin{align}
				\forall x\, \left[\, \forall y \in x\, A(y) \Longrightarrow A(x)\, \right]
				\Longrightarrow \forall x A(x).
			\end{align}
		\end{thm}
	\end{screen}
	
	\begin{sketch}
		いま
		\begin{align}
			a \defeq \Set{x}{\rightharpoondown A(x)}
		\end{align}
		とおく.正則性公理より
		\begin{align}
			a \neq \emptyset \Longrightarrow 
			\exists x\, (\, x \in a \wedge x \cap a = \emptyset\, )
		\end{align}
		が成り立つので,対偶を取れば
		\begin{align}
			\forall x\, (\, x \notin a \vee x \cap a \neq \emptyset\, )
			\Longrightarrow a = \emptyset
			\label{fom:thm_equivalent_condition_of_axiom_of_regularity_1}
		\end{align}
		が成り立つ.ここで
		\begin{align}
			x \cap a \neq \emptyset \Longleftrightarrow \exists y \in x\, (\, y \in a\, )
		\end{align}
		が成り立つので(\refeq{fom:thm_equivalent_condition_of_axiom_of_regularity_1})から
		\begin{align}
			\forall x\, \left[\, x \notin a \vee \exists y \in x\, (\, y \in a\, )\, \right]
			\Longrightarrow a = \emptyset
			\label{fom:thm_equivalent_condition_of_axiom_of_regularity_2}
		\end{align}
		が従い,そして論理和は否定と含意で書き直せる(推論法則\ref{logicalthm:rule_of_inference_3})から
		\begin{align}
			\forall x\, \left[\, \forall y \in x\, (\, y \notin a\, ) \Longrightarrow x \notin a\, \right]
			\Longrightarrow a = \emptyset
		\end{align}
		が従う.ところで類の公理より
		\begin{align}
			x \notin a \Longleftrightarrow A(x)
		\end{align}
		が成り立つから
		\begin{align}
			\forall x\, \left[\, \forall y \in x\, A(y)
			\Longrightarrow A(x)\, \right]
			\Longrightarrow \forall x A(x)
		\end{align}
		を得る.
		\QED
	\end{sketch}
	
	本稿では正則性公理を認めているが,いまだけは認めないことにして代わりに
	集合の帰納法が正しいと仮定してみると,今度は正則性公理が定理として導かれる.実際,$a$を類とすれば
	\begin{align}
		\forall x\, \left[\, \forall y \in x\, (\, y \notin a\, )
		\Longrightarrow\ x \notin a\, \right]
		\Longrightarrow \forall x\, (\, x \notin a\, )
	\end{align}
	が成立するが,ここで対偶を取れば
	\begin{align}
		\exists x\, (\, x \in a\, ) \Longrightarrow 
		\exists x \in a\, \left[\, \forall y \in x\, (\, y \notin a\, )\, \right]
	\end{align}
	が成立し,
	\begin{align}
		a \neq \emptyset \Longleftrightarrow \exists x\, (\, x \in a\, )
	\end{align}
	と
	\begin{align}
		\forall y \in x\, (\, y \notin a\, ) \Longleftrightarrow x \cap a = \emptyset 
	\end{align}
	が成り立つことを併せれば
	\begin{align}
		a \neq \emptyset \Longrightarrow 
		\exists x \in a\, (\, x \cap a = \emptyset\, )
	\end{align}
	が出る.この意味で正則性公理は{\bf 帰納法の公理}とも呼ばれる.
	
	\begin{screen}
		\begin{thm}[超限帰納法]\label{thm:transfinite_induction}
			$A$を$\mathcal{L}'$の式,$\alpha$を$A$に現れる文字,$\beta$を$A$に現れない文字とする.
			このとき,$A$に現れる文字で$\alpha$のみが$A$で量化されていない場合,次が成り立つ:
			\begin{align}
				\forall \alpha \in \ON\, \left(\, \forall \beta \in \alpha\, A(\beta) \Longrightarrow A(\alpha)\, \right)
				\Longrightarrow \forall \alpha \in \ON\, A(\alpha).
			\end{align}
		\end{thm}
	\end{screen}
	
	\begin{prf}
		定理\ref{thm:equivalent_condition_of_axiom_of_regularity}より
		\begin{align}
			\forall \alpha\, \left[\, \forall \beta \in \alpha\, (\, \beta \in \ON \Longrightarrow A(\beta)\, )
			\Longrightarrow (\, \alpha \in \ON \Longrightarrow A(\alpha)\, )\, \right]
			\Longrightarrow \forall \alpha\, (\, \alpha \in \ON \Longrightarrow A(\alpha)\, )
			\label{fom:thm_transfinite_induction}
		\end{align}
		が成り立つ.いま
		\begin{align}
			\forall \alpha \in \ON\, \left(\, \forall \beta \in \alpha\, A(\beta) \Longrightarrow A(\alpha)\, \right)
			\label{fom:thm_transfinite_induction_1}
		\end{align}
		が成り立っているとする.その上で$\alpha$を集合とし,
		\begin{align}
			\forall \beta \in \alpha\, (\, \beta \in \ON \Longrightarrow A(\beta)\, )
			\label{fom:thm_transfinite_induction_2}
		\end{align}
		が成り立っているとする.さらにその上で
		\begin{align}
			\alpha \in \ON
		\end{align}
		が成り立っているとする.このとき$\beta$を
		\begin{align}
			\beta \in \alpha
		\end{align}
		なる集合とすると,順序数の推移性より
		\begin{align}
			\beta \in \ON
		\end{align}
		が成り立つので,(\refeq{fom:thm_transfinite_induction_2})と併せて
		\begin{align}
			A(\beta)
		\end{align}
		が成り立つ.すなわちいま
		\begin{align}
			\forall \beta \in \alpha\, A(\beta)
		\end{align}
		が成り立つ.また(\refeq{fom:thm_transfinite_induction_1})より
		\begin{align}
			\forall \beta \in \alpha\, A(\beta) \Longrightarrow A(\alpha)
		\end{align}
		が成り立つので,いま
		\begin{align}
			A(\alpha)
		\end{align}
		が成立する.つまり,(\refeq{fom:thm_transfinite_induction_2})までを仮定したときには
		\begin{align}
			\alpha \in \ON \Longrightarrow A(\alpha)
		\end{align}
		が成立する.ゆえに(\refeq{fom:thm_transfinite_induction_1})までを仮定したときには
		\begin{align}
			\forall \beta \in \alpha\, (\, \beta \in \ON \Longrightarrow A(\beta)\, )
			\Longrightarrow (\, \alpha \in \ON \Longrightarrow A(\alpha)\, )
		\end{align}
		が成立し,$\alpha$の任意性から
		\begin{align}
			\forall \alpha\, \left[\, \forall \beta \in \alpha\, (\, \beta \in \ON \Longrightarrow A(\beta)\, )
			\Longrightarrow (\, \alpha \in \ON \Longrightarrow A(\alpha)\, )\, \right]
		\end{align}
		が成立する.ゆえに,何も仮定しなくても
		\begin{align}
			(\refeq{fom:thm_transfinite_induction_1}) \Longrightarrow
			\forall \alpha\, \left[\, \forall \beta \in \alpha\, (\, \beta \in \ON \Longrightarrow A(\beta)\, )
			\Longrightarrow (\, \alpha \in \ON \Longrightarrow A(\alpha)\, )\, \right]
			\label{fom:thm_transfinite_induction_3}
		\end{align}
		が成立する.(\refeq{fom:thm_transfinite_induction})と(\refeq{fom:thm_transfinite_induction_3})と含意の推移性より
		\begin{align}
			\forall \alpha \in \ON\, \left(\, \forall \beta \in \alpha\, A(\beta) \Longrightarrow A(\alpha)\, \right)
			\Longrightarrow \forall \alpha\, (\, \alpha \in \ON \Longrightarrow A(\alpha)\, )
		\end{align}
		が従うが,
		\begin{align}
			\forall \alpha\, (\, \alpha \in \ON \Longrightarrow A(\alpha)\, )
		\end{align}
		を略記したものが
		\begin{align}
			\forall \alpha \in \ON\, A(\alpha)
		\end{align}
		であるから
		\begin{align}
			\forall \alpha \in \ON\, \left(\, \forall \beta \in \alpha\, A(\beta) \Longrightarrow A(\alpha)\, \right)
			\Longrightarrow \forall \alpha \in \ON\, A(\alpha)
		\end{align}
		が成り立つことになる.
		\QED
	\end{prf}
	
	
	以後本稿では超限帰納法を頻繁に扱うので,ここでその{\bf 利用方法}を述べておく.
	順序数に対する何らかの言明$A$が与えられたとき,それがいかなる順序数に対しても真であることを示したいとする.往々にして
	\begin{align}
		\forall \alpha \in \ON\, A(\alpha)
	\end{align}
	をいきなり示すのは難しく,一方で
	\begin{align}
		\forall \beta \in \alpha\, A(\beta)
	\end{align}
	から
	\begin{align}
		A(\alpha)
	\end{align}
	を導くことは容易い.それは順序数の``順番''的な性質の良さによるが,超限帰納法のご利益は
	\begin{align}
		\forall \alpha \in \ON\, \left(\, \forall \beta \in \alpha\, A(\beta) \Longrightarrow A(\alpha)\, \right)
	\end{align}
	が成り立つことさえ示してしまえばいかなる順序数に対しても$A$が真となってくれるところにある.
	
	$\alpha$を任意に与えられた順序数とするとき,
	\begin{align}
		\alpha = 0
	\end{align}
	であると空虚な真によって
	\begin{align}
		\forall \beta \in \alpha\, A(\beta)
	\end{align}
	は必ず真となるから,まずは
	\begin{align}
		A(0)
	\end{align}
	が成り立つことを示さなければならない.$A(0)$が偽であると
	\begin{align}
		\forall \beta \in \alpha\, A(\beta) \wedge \rightharpoondown A(0)
	\end{align}
	が真となって
	\begin{align}
		\forall \beta \in \alpha\, A(\beta) \Longrightarrow A(0)
	\end{align}
	が偽となってしまうからである.$\alpha$が$0$でないときは素直に
	\begin{align}
		\forall \beta \in \alpha\, A(\beta)
	\end{align}
	が成り立つとき
	\begin{align}
		A(\alpha)
	\end{align}
	が成り立つことを示せば良い.以上超限帰納法の利用法をまとめると,
	
	\begin{itembox}[l]{超限帰納法の利用手順}
		順序数に対する何らかの言明$A$が与えられて,それがいかなる順序数に対しても真なることを示したいならば,
		\begin{itemize}
			\item まずは$A(0)$が成り立つことを示し,
			\item 次は$\alpha$を$0$でない順序数として
				$\forall \beta \in \alpha\, A(\beta) \Longrightarrow A(\alpha)$が成り立つことを示す.
		\end{itemize}
	\end{itembox}
	
	\begin{screen}
		\begin{thm}[数学的帰納法の原理]
		\label{thm:the_principle_of_mathematical_induction}
			$\Natural$は次の意味で最小の無限集合である:
			\begin{align}
				\forall a\, \left[\, \emptyset \in a \wedge \forall x\, 
				(\, x \in a \Longrightarrow x \cup \{x\} \in a\, ) 
				\Longrightarrow \Natural \subset a\, \right].
			\end{align}
		\end{thm}
	\end{screen}
	
	\begin{prf}
		$a$を集合とし,
		\begin{align}
			\emptyset \in a \wedge \forall x\, 
			(\, x \in a \Longrightarrow x \cup \{x\} \in a\, )
			\label{fom:thm_the_principle_of_mathematical_induction_1}
		\end{align}
		が成り立っているとする.このとき
		\begin{align}
			\forall \alpha \in \ON\, (\, \alpha \in \Natural \Longrightarrow \alpha \in a\, )
		\end{align}
		が成り立つことを超限帰納法で示す.まずは
		\begin{align}
			0 \in a
		\end{align}
		から
		\begin{align}
			\emptyset \in \Natural \Longrightarrow \emptyset \in a
		\end{align}
		が成立する.次に$\alpha$を任意に与えられた$0$でない順序数とする.
		\begin{align}
			\forall \beta \in \alpha\, (\, \beta \in \Natural \Longrightarrow \beta \in a\, )
			\label{fom:thm_the_principle_of_mathematical_induction_2}
		\end{align}
		が成り立っているとすると,
		\begin{align}
			\alpha \in \Natural
		\end{align}
		なら$\alpha$は極限数でないから
		\begin{align}
			\alpha = \beta \cup \{\beta\}
		\end{align}
		を満たす自然数$\beta$が取れて,(\refeq{fom:thm_the_principle_of_mathematical_induction_2})より
		\begin{align}
			\beta \in a
		\end{align}
		が成り立ち,(\refeq{fom:thm_the_principle_of_mathematical_induction_1})より
		\begin{align}
			\alpha \in a
		\end{align}
		が従う.以上で
		\begin{align}
			\forall \alpha \in \ON\, \left[\ 
				\forall \beta \in \alpha\, (\, \beta \in \Natural \Longrightarrow \beta \in a\, )
				\Longrightarrow (\, \alpha \in \Natural \Longrightarrow \alpha \in a\, )\, \right]
		\end{align}
		が得られた.超限帰納法により
		\begin{align}
			\forall \alpha \in \ON\, (\, \alpha \in \Natural \Longrightarrow \alpha \in a\, )
		\end{align}
		が成り立つから
		\begin{align}
			\Natural \subset a
		\end{align}
		が従う.
		\QED
	\end{prf}
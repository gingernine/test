	\begin{screen}
		\begin{logicalaxm}[量化記号に関する公理]
		\label{logicalaxm:rules_of_quantifiers}
			$A$を$\mathcal{L}$の式とし,$A$を$\mathcal{L}$の式とし,$x$を変項とし,
			$A$には$x$のみが自由に現れるとする.
			また$\tau$を主要$\varepsilon$項とする.このとき以下を公理とする.
			\begin{align}
				\negation \forall x A(x) &\rarrow \exists x \negation A(x), \\
				\forall x A(x) &\rarrow A(\tau), \\
				A(\tau) &\rarrow \exists x A(x), \\
				\exists x A(x) &\rarrow A(\varepsilon x \widehat{A}(x)).
			\end{align}
			ただし$\widehat{A}$とは,$A$が$\lang{\varepsilon}$の式でないときは
			$A$を$\lang{\varepsilon}$の式に書き直したものであり,
			小節\ref{subsec:formula_rewriting}の通りの書き換えならばどんな式でも良い.
			$A$が$\lang{\varepsilon}$の式であるときは$\widehat{A}$は$A$そのものとする.
		\end{logicalaxm}
	\end{screen}
	
	$\widehat{A}$を,$A$が$\lang{\varepsilon}$の式でないときは$A$を$\lang{\varepsilon}$の式に
	書き直したものとし,$A$が$\lang{\varepsilon}$の式であるときは$A$そのものとする.
	存在記号の論理的公理より
	\begin{align}
		\vdash \exists x \negation A(x) \rarrow\ \negation A(\varepsilon x \negation \widehat{A})
	\end{align}
	が成り立つので,
	\begin{align}
		\negation \forall x A(x) \vdash \exists x \negation A(x)
	\end{align}
	との三段論法で
	\begin{align}
		\negation \forall x A(x) \vdash\ \negation A(\varepsilon x \negation \widehat{A})
	\end{align}
	が従う.そして対偶律4 (論理的定理\ref{logicalthm:proof_by_contraposition})より
	\begin{align}
		\vdash A(\varepsilon x \negation \widehat{A}) \rarrow \forall x A(x)
	\end{align}
	が得られる.これは非常に有用な結果であるから一つの定理として述べておく.
	
	\begin{screen}
		\begin{logicalthm}[$\varepsilon$項による全称の導出]
		\label{logicalthm:derivation_of_universal_by_epsilon}
			$A$を$\mathcal{L}$の式とし,$x$を変項とし,
			$A$には$x$のみが自由に現れるとする.このとき
			\begin{align}
				\vdash A(\varepsilon x \negation \widehat{A}(x)) \rarrow \forall x A(x).
			\end{align}
			ただし$\widehat{A}$とは,$A$が$\lang{\varepsilon}$の式でないときは
			$A$を$\lang{\varepsilon}$の式に書き直したものであり,
			小節\ref{subsec:formula_rewriting}の通りの書き換えならばどんな式でも良い.
			$A$が$\lang{\varepsilon}$の式であるときは$\widehat{A}$は$A$そのものとする.
		\end{logicalthm}
	\end{screen}
	
	どれでも一つ,$A(\tau)$を成り立たせるような主要$\varepsilon$項$\tau$が取れれば
	$\exists x A(x)$が成り立つのだし,逆に$\exists x A(x)$が成り立つならば
	$\varepsilon x A(x)$なる$\epsilon$項が$A(\varepsilon x A(x))$を満たすのである.
	そして主要$\varepsilon$項は集合であるから(定理\ref{thm:critical_epsilon_term_is_set}),
	「$A(x)$を満たす集合$x$が存在する」ということと
	「$A(x)$を満たす集合$x$が{\bf ``実際に取れる''}」ということが同じ意味になる.
	
	$\forall x A(x)$が成り立つならばいかなる主要$\varepsilon$項$\tau$も$A(\tau)$を満たすし,
	逆にいかなる主要$\varepsilon$項$\tau$も$A(\tau)$を満たすならば,
	特に$\varepsilon x \negation A(x)$なる$\varepsilon$項も
	$A(\varepsilon x \negation A(x))$を満たすのだから$\forall x A(x)$が成立する.
	つまり,「$\forall x A(x)$が成り立つ」ということと
	「任意の主要$\varepsilon$項$\tau$が$A(\tau)$を満たす」ということは同じ意味になる.
	
	後述することであるが,主要$\varepsilon$項はどれも集合であって
	(定理\ref{thm:critical_epsilon_term_is_set}),また集合であるクラスは
	いずれかの主要$\varepsilon$項と等しい
	(定理\ref{thm:if_a_class_is_a_set_then_equal_to_some_epsilon_term}).
	ゆえに,{\bf 量化子の亘る範囲は集合に制限される}のである.
	
	量化記号についても De Morgan の法則があり,それを
	\begin{description}
		\item[弱 De Morgan の法則] $\exists x \negation A(x) \lrarrow\ \negation \forall x A(x)$,
		\item[強 De Morgan の法則] $\forall x \negation A(x) \lrarrow\ \negation \exists x A(x)$,
	\end{description}
	と呼ぶことにする.
	
	\begin{screen}
		\begin{logicalthm}[量化記号に対する弱 De Morgan の法則(1)]
		\label{logicalthm:weak_De_Morgan_law_for_quantifiers_1}
			$A$を$\mathcal{L}$の式とし,$x$を$A$に自由に現れる変項とし,
			また$A$に自由に現れる変項は$x$のみであるとする.このとき
			\begin{align}
				\vdash \exists x \negation A(x) \rarrow\ \negation \forall x A(x).
			\end{align}
		\end{logicalthm}
	\end{screen}
	
	\begin{sketch}
		必要に応じて$A$を$\lang{\varepsilon}$の式に書き換えたものを$\widehat{A}$とする.
		存在記号の論理的公理より
		\begin{align}
			\exists x \negation A(x) \vdash\ \negation A(\varepsilon x \negation \widehat{A}(x))
			\label{fom:weak_De_Morgan_law_for_quantifiers_1_1}
		\end{align}
		となる.また全称記号の論理的公理より
		\begin{align}
			\vdash \forall x A(x) \rarrow A(\varepsilon x \negation \widehat{A}(x))
		\end{align}
		が成り立つので,対偶を取って
		\begin{align}
			\vdash\ \negation A(\varepsilon x \negation \widehat{A}(x)) \rarrow\ \negation \forall x A(x)
			\label{fom:weak_De_Morgan_law_for_quantifiers_1_2}
		\end{align}
		となる(論理的定理\ref{logicalthm:introduction_of_contraposition}).
		(\refeq{fom:weak_De_Morgan_law_for_quantifiers_1_1})と
		(\refeq{fom:weak_De_Morgan_law_for_quantifiers_1_2})の三段論法より
		\begin{align}
			\exists x \negation A(x) \vdash\ \negation \forall x A(x)
		\end{align}
		が従い,演繹定理より
		\begin{align}
			\vdash \exists x \negation A(x) \rarrow\ \negation \forall x A(x)
		\end{align}
		が得られる.
		\QED
	\end{sketch}
	
	\begin{screen}
		\begin{logicalthm}[量化記号に対する弱 De Morgan の法則(2)]
		\label{logicalthm:weak_De_Morgan_law_for_quantifiers_2}
			$A$を$\mathcal{L}$の式とし,$x$を$A$に自由に現れる変項とし,
			また$A$に自由に現れる変項は$x$のみであるとする.このとき
			\begin{align}
				\vdash\ \negation \forall x A(x) \rarrow \exists x \negation A(x).
			\end{align}
		\end{logicalthm}
	\end{screen}
	
	\begin{sketch}
		論理的公理
		\begin{align}
			\negation \forall x A(x) \rarrow \exists x \negation A(x)
		\end{align}
		により得られる.
		\QED
	\end{sketch}
	
	\begin{comment}
		必要に応じて$A$を$\lang{\varepsilon}$の式に書き換えたものを$\widehat{A}$とする.
		存在記号の推論公理より
		\begin{align}
			\vdash\ \negation A(\varepsilon x \negation \widehat{A}(x)) \rarrow \exists x \negation A(x)
		\end{align}
		が成り立つので,対偶を取って
		\begin{align}
			\vdash\ \negation \exists x \negation A(x) \rarrow\ \negation\negation A(\varepsilon x \negation \widehat{A}(x))
		\end{align}
		が従い(論理的定理\ref{logicalthm:introduction_of_contraposition}),演繹定理の逆より
		\begin{align}
			\negation \exists x \negation A(x) \vdash\ \negation\negation A(\varepsilon x \negation \widehat{A}(x))
		\end{align}
		となる.そして二重否定の除去より
		\begin{align}
			\negation \exists x \negation A(x) \vdash A(\varepsilon x \negation \widehat{A}(x))
		\end{align}
		が成立し,全称の導出(論理的定理\ref{derivation_of_universal_by_epsilon})より
		\begin{align}
			\negation \exists x \negation A(x) \vdash \forall x A(x)
		\end{align}
		が従う.演繹定理より
		\begin{align}
			\vdash\ \negation \exists x \negation A(x) \rarrow \forall x A(x)
		\end{align}
		となり,対偶を取れば
		\begin{align}
			\vdash\ \negation \forall x A(x) \rarrow\ \negation\negation \exists x \negation A(x)
		\end{align}
		となるが,先と同様に二重否定の除去によって
		\begin{align}
			\vdash\ \negation \forall x A(x) \rarrow \exists x \negation A(x)
		\end{align}
		が得られる.
		\QED
	\end{comment}
	
	\begin{screen}
		\begin{logicalthm}[量化記号に対する強 De Morgan の法則(1)]
		\label{logicalthm:strong_De_Morgan_law_for_quantifiers_1}
			$A$を$\mathcal{L}$の式とし,$x$を$A$に自由に現れる変項とし,
			また$A$に自由に現れる変項は$x$のみであるとする.このとき
			\begin{align}
				\vdash \forall x \negation A(x) \rarrow\ \negation \exists x A(x).
			\end{align}
		\end{logicalthm}
	\end{screen}
	
	\begin{sketch}
		必要に応じて$A$を$\lang{\varepsilon}$の式に書き換えたものを$\widehat{A}$とする.
		まず存在記号の論理的公理より
		\begin{align}
			\vdash \exists x A(x) \rarrow A(\varepsilon x \widehat{A}(x))
		\end{align}
		が成り立つので,対偶を取って
		\begin{align}
			\vdash\ \negation A(\varepsilon x \widehat{A}(x)) 
			\rarrow\ \negation \exists x A(x)
		\end{align}
		が成り立つ(論理的定理\ref{logicalthm:introduction_of_contraposition}).また
		全称記号の論理的公理より
		\begin{align}
			\forall x \negation A(x) \vdash\ \negation A(\varepsilon x \widehat{A}(x))
		\end{align}
		が成り立つので,三段論法より
		\begin{align}
			\forall x \negation A(x) \vdash\ \negation \exists x A(x)
		\end{align}
		が従い,演繹定理より
		\begin{align}
			\vdash \forall x \negation A(x) \rarrow\ \negation \exists x A(x)
		\end{align}
		が得られる.
		\QED
	\end{sketch}
	
	\begin{screen}
		\begin{logicalthm}[量化記号に対する強 De Morgan の法則(2)]
		\label{logicalthm:strong_De_Morgan_law_for_quantifiers_2}
			$A$を$\mathcal{L}$の式とし,$x$を$A$に自由に現れる変項とし,
			また$A$に自由に現れる変項は$x$のみであるとする.このとき
			\begin{align}
				\vdash\ \negation \exists x A(x) \rarrow \forall x \negation A(x).
			\end{align}
		\end{logicalthm}
	\end{screen}
	
	\begin{sketch}
		必要に応じて$A$を$\lang{\varepsilon}$の式に書き換えたものを$\widehat{A}$とする.
		まず存在記号の論理的公理より
		\begin{align}
			\vdash A(\varepsilon x \negation \negation \widehat{A}(x))
			\rarrow \exists x A(x)
		\end{align}
		が成り立つので,対偶を取って
		\begin{align}
			\vdash\ \negation \exists x A(x) 
			\rarrow\ \negation A(\varepsilon x \negation \negation \widehat{A}(x))
		\end{align}
		が成り立ち(論理的定理\ref{logicalthm:introduction_of_contraposition}),
		演繹定理の逆より
		\begin{align}
			\negation \exists x A(x) \vdash\ \negation A(\varepsilon x \negation \negation \widehat{A}(x))
		\end{align}
		が従う.また全称の導出(論理的定理\ref{logicalthm:derivation_of_universal_by_epsilon})より
		\begin{align}
			\vdash\ \negation A(\varepsilon x \negation \negation \widehat{A}(x))
			\rarrow \forall x \negation A(x)
		\end{align}
		が成り立つので,三段論法より
		\begin{align}
			\negation \exists x A(x) \vdash \forall x \negation A(x)
		\end{align}
		が従い,演繹定理より
		\begin{align}
			\vdash\ \negation \exists x A(x) \rarrow \forall x \negation A(x)
		\end{align}
		が得られる.
		\QED
	\end{sketch}
	
	\begin{screen}
		\begin{logicalthm}[含意の論理積への遺伝性]
		\label{logicalthm:heredity_of_implication_to_conjunction}
			$A,B,C$を文とするとき,
			\begin{align}
				\vdash (\, A \rarrow B\, ) \rarrow (\, A \wedge C 
				\rarrow B \wedge C\, ), \\
				\vdash (\, A \rarrow B\, ) \rarrow (\, C \wedge A 
				\rarrow C \wedge B\, ).
			\end{align}
		\end{logicalthm}
	\end{screen}
	
	\begin{sketch}
		論理積の除去より
		\begin{align}
			A \wedge C \vdash A
		\end{align}
		が成り立つので,三段論法より
		\begin{align}
			A \wedge C,\ A \rarrow B \vdash B
		\end{align}
		が従う.再び論理積の除去より
		\begin{align}
			A \wedge C \vdash C
		\end{align}
		が成り立つから,論理積の導入より
		\begin{align}
			A \wedge C,\ A \rarrow B \vdash B \wedge C
		\end{align}
		が従う.そして演繹定理より
		\begin{align}
			\vdash (\, A \rarrow B\, ) \rarrow (\, A \wedge C \rarrow B \rarrow C\, )
		\end{align}
		が得られる.
		\begin{align}
			\vdash (\, A \rarrow B\, ) \rarrow (\, C \wedge A \rarrow C \wedge B\, )
		\end{align}
		も同様に示される.
		\QED
	\end{sketch}
	
	\begin{screen}
		\begin{logicalthm}[含意の含意への遺伝性]
		\label{logicalthm:heredity_of_implication_to_implication}
			$A,B,C$を文とするとき,
			\begin{align}
				\vdash (\, A \rarrow B\, ) \rarrow (\, (\, B \rarrow C\, )
				\rarrow (\, A \rarrow C\, )\, ).
			\end{align}
		\end{logicalthm}
	\end{screen}
	
	\begin{sketch}
		三段論法より
		\begin{align}
			A,\ A \rarrow B \vdash B
		\end{align}
		が成り立つので,再び三段論法より
		\begin{align}
			A,\ A \rarrow B,\ B \rarrow C \vdash C
		\end{align}
		が成り立つ.そして演繹定理より
		\begin{align}
			\vdash (\, A \rarrow B\, ) \rarrow (\, (\, B \rarrow C\, )
			\rarrow (\, A \rarrow C\, )\, )
		\end{align}
		が得られる.
		\QED
	\end{sketch}
	
	次の定理は,量化部分式で束縛される変項の名前替えをしても文の意味が保持されることを保証する.
	\begin{screen}
		\begin{logicalthm}[量化部分式を差し替えても同値]
		\label{logicalthm:equivalence_by_replacing_bound_variables}
			$\varphi$を$\lang{\varepsilon}$の文とし,$\forall y \psi$が$\varphi$の部分式
			として現れたとする.また$z$を$\psi$に自由に現れない変項とし,$\psi$の中で$y$への代入
			について自由であるとする.このとき,$\varphi$の上の$\forall y \psi$のその一か所を
			$\forall z \psi(y/z)$に差し替えた文を$\widehat{\varphi}$とすれば
			\begin{align}
				\vdash \varphi \lrarrow \widehat{\varphi}
			\end{align}
			が成り立つ.$\forall y \psi$ではなく$\exists y \psi$であったとしても同様のことが
			成り立つ.
		\end{logicalthm}
	\end{screen}
	
	\begin{sketch}\mbox{}
		\begin{description}
			\item[step1]
				$\varphi$が$\forall y \psi$であるとき,
				\begin{align}
					\zeta \defeq \varepsilon z \negation \psi(y/z)
				\end{align}
				とおけば全称記号の論理的公理より
				\begin{align}
					\forall y \psi \vdash \psi(y/\zeta)
				\end{align}
				が成り立つが,$z$の選び方より$\psi(y/\zeta)$は$\psi(y/z)(z/\zeta)$と
				同じ式であるから
				\begin{align}
					\forall y \psi \vdash \psi(y/z)(z/\zeta)
				\end{align}
				が成り立つ.そして全称の導出
				(論理的定理\ref{logicalthm:derivation_of_universal_by_epsilon})より
				\begin{align}
					\forall y \psi \vdash \forall z \psi(y/z)
					\label{fom:equivalence_by_replacing_bound_variables_1}
				\end{align}
				が成り立つ.逆に
				\begin{align}
					\zeta \defeq \varepsilon y \negation \psi
				\end{align}
				とおけば
				\begin{align}
					\forall z \psi(y/z) \vdash \psi(y/z)(z/\zeta)
				\end{align}
				が成り立つが,$\psi(y/z)(z/\zeta)$は$\psi(y/\zeta)$と同じ式であるから
				\begin{align}
					\forall z \psi(y/z) \vdash \psi(y/\zeta)
				\end{align}
				が成り立つ.そして全称の導出
				(論理的定理\ref{logicalthm:derivation_of_universal_by_epsilon})より
				\begin{align}
					\forall z \psi(y/z) \vdash \forall y \psi
					\label{fom:equivalence_by_replacing_bound_variables_2}
				\end{align}
				が成り立つ.(\refeq{fom:equivalence_by_replacing_bound_variables_1})と
				(\refeq{fom:equivalence_by_replacing_bound_variables_2})と
				論理積の導入より
				\begin{align}
					\vdash \varphi \lrarrow \widehat{\varphi}
				\end{align}
				が従う.
			
			%\item[step2]
			%	$\varphi$が$\forall y \psi$であり,$\forall y \psi$に変項$x$のみが
			%	自由に現れているとき,
			%	任意の主要$\varepsilon$項$\tau$に対して
			%	\begin{align}
			%		\vdash \varphi(x/\tau) \lrarrow \widehat{\varphi}(x/\tau)
			%	\end{align}
			%	が成り立つことを示す.まず$x$は$\forall y \psi$に自由に現れているので
			%	$x$と$y$は違う変項である.
			%	また$z$は$x$を束縛しないように選ばれているので,$x$と$z$も違う変項である.
			%	ゆえに,$\varphi(x/\tau)$は$\forall y \psi(x/\tau)$と同じ式で,
			%	$\widehat{\varphi}(x/\tau)$は$\forall z \psi(y/z)(x/\tau)$と同じ式である.
			%	\begin{align}
			%		\zeta \defeq \varepsilon z \negation \psi(y/z)(x/\tau)
			%	\end{align}
			%	とおけば,全称記号の論理公理より
			%	\begin{align}
			%		\forall y \psi(x/\tau) \vdash \psi(x/\tau)(y/\zeta)
			%	\end{align}
			%	が成り立つが,$\psi(x/\tau)(y/\zeta)$と$\psi(y/z)(x/\tau)(z/\zeta)$は
			%	同じ式なので
			%	\begin{align}
			%		\forall y \psi(x/\tau) \vdash \psi(y/z)(x/\tau)(z/\zeta)
			%	\end{align}
			%	となり,全称の導出
			%	(論理的定理\ref{logicalthm:derivation_of_universal_by_epsilon})より
			%	\begin{align}
			%		\forall y \psi(x/\tau) \vdash \forall z \psi(y/z)(x/\tau)
			%	\end{align}
			%	が従い,演繹定理より
			%	\begin{align}
			%		\vdash \varphi(x/\tau) \rarrow \widehat{\varphi}(x/\tau)
			%		\label{fom:equivalence_by_replacing_bound_variables_3}
			%	\end{align}
			%	が得られる.逆に
			%	\begin{align}
			%		\zeta \defeq \varepsilon y \negation \psi(x/\tau)
			%	\end{align}
			%	とおけば,全称記号の論理公理より
			%	\begin{align}
			%		\forall z \psi(y/z)(x/\tau) \vdash \psi(y/z)(x/\tau)(z/\zeta)
			%	\end{align}
			%	が成り立つが,$\psi(y/z)(x/\tau)(z/\zeta)$と$\psi(x/\tau)(y/\zeta)$は
			%	同じ式なので
			%	\begin{align}
			%		\forall z \psi(y/z)(x/\tau) \vdash \psi(x/\tau)(y/\zeta)
			%	\end{align}
			%	となり,全称の導出
			%	(論理的定理\ref{logicalthm:derivation_of_universal_by_epsilon})より
			%	\begin{align}
			%		\forall z \psi(y/z)(x/\tau) \vdash \forall y \psi(x/\tau)
			%	\end{align}
			%	が従い,演繹定理より
			%	\begin{align}
			%		\vdash \widehat{\varphi}(x/\tau) \rarrow \varphi(x/\tau)
			%		\label{fom:equivalence_by_replacing_bound_variables_4}
			%	\end{align}
			%	が得られる.(\refeq{fom:equivalence_by_replacing_bound_variables_3})と
			%	(\refeq{fom:equivalence_by_replacing_bound_variables_4})と
			%	論理積の導入より
			%	\begin{align}
			%		\vdash \varphi(x/\tau) \lrarrow \widehat{\varphi}(x/\tau)
			%	\end{align}
			%	が従う.
				
			\item[step2]
				以下では便宜上,$\varphi$に対して行うような部分式の差し替えによって得る式を
				「量化部分式の差し替え」と呼ぶことにする.
				
				\begin{description}
					\item[IH (帰納法\ref{metaaxm:induction_principle_of_L_formulas}の仮定)]
						$\varphi$の任意の真部分式$\eta$に対して,
						$\eta$に自由に現れている変項が$x_{1},\cdots,x_{n}$で全てであるとき,
						任意に主要$\varepsilon$項$\tau_{1},\cdots,\tau_{n}$を取って
						$\eta(x_{1}/\tau_{1})\cdots(x_{n}/\tau_{n})$なる文を
						$\eta^{*}$とする.$\eta$が文なら$\eta^{*}$を$\eta$とする.
						このとき,$\eta^{*}$の任意の量化部分式の差し替え
						$\widehat{\eta^{*}}$に対して
						\begin{align}
							\vdash \eta^{*} \lrarrow \widehat{\eta^{*}}
						\end{align}
						となる
				\end{description}
				と仮定する(帰納法の考え方としては,$\varphi$よりも``前の段階''で作られた
				任意の文に対して量化部分式を差し替えても同値であると仮定している).
				
			\begin{description}
				\item[case1] $\varphi$が
					\begin{align}
						\negation \eta
					\end{align}
					なる文であるとき,$\widehat{\varphi}$とは
					\begin{align}
						\negation \widehat{\eta}
					\end{align}
					なる文であって($\widehat{\eta}$は$\eta$の$\forall y \psi$を
					$\forall z \psi(y/z)$に差し替えた文),(IH)より
					\begin{align}
						\vdash \eta \lrarrow \widehat{\eta}
					\end{align}
					が成り立つ.論理積の除去より
					\begin{align}
						&\vdash \eta \rarrow \widehat{\eta}, \\
						&\vdash \widehat{\eta} \rarrow \eta
					\end{align}
					が成り立ち,対偶律1
					(論理的定理\ref{logicalthm:introduction_of_contraposition})より
					\begin{align}
						&\vdash\ \negation \widehat{\eta} \rarrow\ \negation \eta, \\
						&\vdash\ \negation \eta \rarrow\ \negation \widehat{\eta}
					\end{align}
					が従う.そして論理積の導入より
					\begin{align}
						\vdash \varphi \lrarrow \widehat{\varphi}
					\end{align}
					が得られる.
				
				%\item[case1-2] $\varphi$が
				%	\begin{align}
				%		\negation \eta
				%	\end{align}
				%	なる式で,$\varphi$に$x$のみ自由に現れるとき,
				%	$\widehat{\varphi}(x/\tau)$とは
				%	\begin{align}
				%		\negation \widehat{\eta}(x/\tau)
				%	\end{align}
				%	なる式であって,(IH)より
				%	\begin{align}
				%		\vdash \eta(x/\tau) \lrarrow \widehat{\eta}(x/\tau)
				%	\end{align}
				%	が成り立つので,前段と同様にして
				%	\begin{align}
				%		\vdash\ \negation \eta(x/\tau) \lrarrow\ 
				%		\negation \widehat{\eta}(x/\tau)
				%	\end{align}
				%	が成り立つ.つまり
				%	\begin{align}
				%		\vdash \varphi(x/\tau) \lrarrow \widehat{\varphi}(x/\tau)
				%	\end{align}
				%	が得られる.
					
				\item[case2] 
					$\varphi$が
					\begin{align}
						\eta \vee \chi
					\end{align}
					なる文であるとき,差し替えられる$\forall y \psi$が$\eta$に現れているとすれば,
					$\widehat{\varphi}$は
					\begin{align}
						\widehat{\eta} \vee \chi
					\end{align}
					なる式であって($\widehat{\eta}$は$\eta$の$\forall y \psi$を
					$\forall z \psi(y/z)$に差し替えた文),(IH)より
					\begin{align}
						\vdash \eta \lrarrow \widehat{\eta}
					\end{align}
					が成り立つ.含意の論理和への遺伝性(論理的定理
					\ref{logicalthm:heredity_of_implication_to_disjunction})より
					\begin{align}
						\vdash (\, \eta \lrarrow \widehat{\eta}\, ) 
						\rarrow (\, \eta \vee \chi \lrarrow 
						\widehat{\eta} \vee \chi\, )
					\end{align}
					が成り立つので,三段論法より
					\begin{align}
						\vdash \varphi \lrarrow \widehat{\varphi}
					\end{align}
					が得られる.差し替えられる$\forall y \psi$が$\chi$に現れているときも
					同様であるし,また$\varphi$が$\eta \wedge \chi$や$\eta \rarrow \chi$
					なる文の場合も論理的定理
					\ref{logicalthm:heredity_of_implication_to_conjunction}
					或いは論理的定理
					\ref{logicalthm:heredity_of_implication_to_implication}を使えば
					同様にして$\vdash \varphi \lrarrow \widehat{\varphi}$が示される.
					
				%\item[case2-2] $\varphi$が
				%	\begin{align}
				%		\eta \vee \chi
				%	\end{align}
				%	なる式で,$\varphi$に$x$のみ自由に現れるとき,
				%	$\widehat{\varphi}(x/\tau)$とは
				%	\begin{align}
				%		\widehat{\eta}(x/\tau) \vee \widehat{\chi}(x/\tau)
				%	\end{align}
				%	なる式であって,(IH)より
				%	\begin{align}
				%		&\vdash \eta(x/\tau) \lrarrow \widehat{\eta}(x/\tau), \\
				%		&\vdash \chi(x/\tau) \lrarrow \widehat{\chi}(x/\tau)
				%	\end{align}
				%	が成り立つので,前段と同様にして
				%	\begin{align}
				%		\vdash \eta(x/\tau) \vee \chi(x/\tau) 
				%		\lrarrow \widehat{\eta}(x/\tau) \vee \widehat{\chi}(x/\tau)
				%	\end{align}
				%	が成り立つ.つまり
				%	\begin{align}
				%		\vdash \varphi(x/\tau) \lrarrow \widehat{\varphi}(x/\tau)
				%	\end{align}
				%	が得られる.
					
				\item[case3] $\varphi$が
					\begin{align}
						\exists w \eta
					\end{align}
					なる文のとき,$\widehat{\varphi}$とは
					\begin{align}
						\exists w \widehat{\eta}
					\end{align}
					なる式であって($\widehat{\eta}$は$\eta$の$\forall y \psi$を
					$\forall z \psi(y/z)$に差し替えた文),
					\begin{align}
						\tau \defeq \varepsilon w \widehat{\eta}
					\end{align}
					とおけば存在記号の論理的公理より
					\begin{align}
						\exists w \widehat{\eta} \vdash \widehat{\eta}(w/\tau)
					\end{align}
					が成り立つ.メタ定理
					\ref{metathm:subformula_replacing_and_substitution}より
					$\widehat{\eta}(w/\tau)$は$\eta(w/\tau)$の量化部分式の差し替え
					であるから,(IH)より
					\begin{align}
						\vdash \widehat{\eta}(w/\tau) \rarrow \eta(w/\tau)
					\end{align}
					が成り立ち,三段論法より
					\begin{align}
						\exists w \widehat{\eta} \vdash \eta(w/\tau)
					\end{align}
					が従い,存在記号の論理的公理と演繹定理より
					\begin{align}
						\vdash \exists w \widehat{\eta} \rarrow \exists w \eta
					\end{align}
					が得られる.同様に
					\begin{align}
						\vdash \exists w \eta \rarrow \exists w \widehat{\eta}
					\end{align}
					も成り立つので,論理積の導入より
					\begin{align}
						\vdash \varphi \lrarrow \widehat{\varphi}
					\end{align}
					が得られる.$\varphi$が$\forall y \psi$なる式であっても,全称の導出
					(論理的定理
					\ref{logicalthm:derivation_of_universal_by_epsilon})を利用すれば
					同様にして$\vdash \varphi \lrarrow \widehat{\varphi}$が示される.
					\QED
				%\item[case3-2] $\varphi$が
				%	\begin{align}
				%		\exists w \eta
				%	\end{align}
				%	なる式で,$\varphi$に$x$のみ自由に現れるとき,
				%	$\widehat{\varphi}(x/\tau)$とは
				%	\begin{align}
				%		\exists w \widehat{\eta}(x/\tau)
				%	\end{align}
				%	なる式であって,(IH)より,任意の主要$\varepsilon$項$\sigma$に対して
				%	\begin{align}
				%		\vdash \eta(x/\tau) \lrarrow \widehat{\eta}(x/\tau)
				%	\end{align}
				%	が成り立つ.前段と同様にして
				%	\begin{align}
				%		\vdash \exists w \eta(x/\tau) \lrarrow 
				%		\exists w \widehat{\eta}(x/\tau)
				%	\end{align}
				%	が成り立つ.つまり
				%	\begin{align}
				%		\vdash \varphi(x/\tau) \lrarrow \widehat{\varphi}(x/\tau)
				%	\end{align}
				%	が得られる.
					
				%\item[case4] $\varphi$が
				%	\begin{align}
				%		\forall x \eta
				%	\end{align}
				%	なる式のとき,$\widehat{\varphi}$とは
				%	\begin{align}
				%		\forall x \widehat{\eta}
				%	\end{align}
				%	なる式であって,
				%	\begin{align}
				%		\tau \defeq \varepsilon x \negation \eta
				%	\end{align}
				%	とおけば全称記号の論理的公理より
				%	\begin{align}
				%		\forall x \widehat{\eta} \vdash \widehat{\eta}(x/\tau)
				%	\end{align}
				%	が成り立つ.(IH)より
				%	\begin{align}
				%		\vdash \widehat{\eta}(x/\tau) \rarrow \eta(x/\tau)
				%	\end{align}
				%	が成り立つので三段論法より
				%	\begin{align}
				%		\forall x \widehat{\eta} \vdash \eta(x/\tau)
				%	\end{align}
				%	が従い,全称の導出
				%	(論理的定理\ref{logicalthm:derivation_of_universal_by_epsilon})と
				%	演繹定理より
				%	\begin{align}
				%		\vdash \forall x \widehat{\eta} \rarrow \forall x \eta
				%	\end{align}
				%	が得られる.同様に
				%	\begin{align}
				%		\vdash \forall x \eta \rarrow \forall x \widehat{\eta}
				%	\end{align}
				%	も成り立つので,論理積の導入より
				%	\begin{align}
				%		\vdash \varphi \lrarrow \widehat{\varphi}
				%	\end{align}
				%	が得られる.
			\end{description}
		\end{description}
	\end{sketch}
\section{式の書き換え(没)}
	\begin{itemize}
		\item $x \in y$はそのまま$x \in y$
		\item $x \in \{y|B(y)\}$は$B(x)$
			
			これは公理である.つまり,
			\begin{align}
				\forall x\, \left(\, x \in \{y|B(y)\} \leftrightarrow B(x)\, \right).
			\end{align}
			
		\item $x \in \varepsilon y B(y)$は$\exists t\, \left(\, x \in t \wedge B(t)\, \right)$.ちなみにこれは公理とするべきか:
			\begin{align}
				\forall x\, \left(\, x \in \varepsilon y B(y) \leftrightarrow
				\exists t\, \left(\, x \in t \wedge B(t)\, \right)\, \right).
			\end{align}
			
		\item $\{x|A(x)\} \in y$は$\exists s\, \left(\, s \in y \wedge 
			\forall u\, \left(\, u \in s \leftrightarrow A(u)\, \right)\, \right)$
			
			実はこの両式は同値である.さていま
			\begin{align}
				\{x|A(x)\} \in y \leftrightarrow
				\exists s\, \left(\, s \in y \wedge 
				\forall u\, \left(\, u \in s \leftrightarrow A(u)\, \right)\, \right)
			\end{align}
			という式を$\varphi$とし,これを$\mathcal{L}_{\in}$の式に書き換えたものを$\hat{\varphi}$としよう.そして
			\begin{align}
				\eta = \varepsilon y \rightharpoondown \hat{\varphi}(y)
			\end{align}
			とおこう.ここで証明するのは
			\begin{align}
				\{x|A(x)\} \in \eta \leftrightarrow
				\exists s\, \left(\, s \in \eta \wedge 
				\forall u\, \left(\, u \in s \leftrightarrow A(u)\, \right)\, \right)
			\end{align}
			が成り立つということである.まず
			\begin{align}
				\{x|A(x)\} \in \eta
			\end{align}
			が成り立っているとしよう.すると
			\begin{align}
				\exists s\, \left(\, \{x|A(x)\} = s\, \right)
			\end{align}
			が成り立つのだが,今度も式の書き直し手順によって
			\begin{align}
				\exists s\, \left(\, \forall u\, \left(\, A(u) \leftrightarrow
				u \in s\, \right)\, \right)
			\end{align}
			と書き直される.
			\begin{align}
				\sigma = \varepsilon s\, \left(\, \forall u\, \left(\, A(u) \leftrightarrow
				u \in s\, \right)\, \right)
			\end{align}
			とおくと
			\begin{align}
				\forall u\, \left(\, A(u) \leftrightarrow
				u \in \sigma\, \right)
			\end{align}
			が成り立ち,他方で
			\begin{align}
				\sigma = \{x|A(x)\}
			\end{align}
			が成り立つのだから
			\begin{align}
				\sigma \in \eta
			\end{align}
			も従う.ゆえに
			\begin{align}
				\sigma \in \eta \wedge \forall u\, \left(\, A(u) \leftrightarrow
				u \in \sigma\, \right)
			\end{align}
			が成り立つ.逆に
			\begin{align}
				\exists s\, \left(\, s \in \eta \wedge 
				\forall u\, \left(\, u \in s \leftrightarrow A(u)\, \right)\, \right)
			\end{align}
			が成り立っているとして,
			\begin{align}
				\sigma = \varepsilon s\, \left(\, s \in \eta \wedge 
				\forall u\, \left(\, u \in s \leftrightarrow A(u)\, \right)\, \right)
			\end{align}
			としよう.すると
			\begin{align}
				\sigma \in \eta \wedge 
				\forall u\, \left(\, u \in \sigma \leftrightarrow A(u)\, \right)
			\end{align}
			が成り立つので
			\begin{align}
				\sigma \in \eta
			\end{align}
			かつ
			\begin{align}
				\sigma = \{x|A(x)\}
			\end{align}
			が成立する.ゆえに
			\begin{align}
				\{x|A(x)\} \in \eta
			\end{align}
			が成立する.以上で
			\begin{align}
				\{x|A(x)\} \in \eta \leftrightarrow
				\exists s\, \left(\, s \in \eta \wedge 
				\forall u\, \left(\, u \in s \leftrightarrow A(u)\, \right)\, \right)
			\end{align}
			が得られた.
			
		\item $\{x|A(x)\} \in \{y|B(y)\}$は$\exists s\, \left(\, B(s) \wedge 
			\forall u\, \left(\, u \in s \leftrightarrow A(u)\, \right)\, \right)$
		
		\item $\{x|A(x)\} \in \varepsilon y B(y)$は$\exists s,t\, \left(\, s \in t \wedge 
			\forall u\, \left(\, u \in s \leftrightarrow A(u)\, \right) \wedge B(t)\, \right)$
		
		\item $\varepsilon x A(x) \in y$は$\exists s\, \left(\, s \in y \wedge A(s)\, \right)$
			
			これも公理にしよう:
			\begin{align}
				\forall y\, \left(\, \varepsilon x A(x) \in y \leftrightarrow
				\exists s\, \left(\, s \in y \wedge A(s)\, \right)\, \right).
			\end{align}
			いや,$y$をクラスとした言明の方が良いかも.
			\begin{align}
				\varepsilon x A(x) \in y \leftrightarrow
				\exists s\, \left(\, s \in y \wedge A(s)\, \right).
			\end{align}
		
		\item $\varepsilon x A(x) \in \{y|B(y)\}$は$\exists s\, \left(\, A(s) \wedge B(s)\, \right)$
			
			上の公理からこの式の同値性も導かれます.まず
			\begin{align}
				\exists s\, \left(\, A(s) \wedge B(s)\, \right)
			\end{align}
			が成り立っているとしよう.そして
			\begin{align}
				\sigma = \varepsilon s\, \left(\, A(s) \wedge B(s)\, \right)
			\end{align}
			とおくと,
			\begin{align}
				A(\sigma) \wedge B(\sigma)
			\end{align}
			が成立する.ゆえに
			\begin{align}
				\sigma \in \{y|B(y)\} \wedge A(\sigma)
			\end{align}
			が成立する.ゆえに
			\begin{align}
				\exists s\, \left(\, s \in \{y|B(y)\} \wedge A(s)\, \right)
			\end{align}
			が成り立つ.ゆえに
			\begin{align}
				\varepsilon x A(x) \in \{y|B(y)\}
			\end{align}
			が成り立つ.逆に
			\begin{align}
				\varepsilon x A(x) \in \{y|B(y)\}
			\end{align}
			が成り立っているとしよう.すると
			\begin{align}
				\exists s\, \left(\, s = \varepsilon x A(x)\, \right)
			\end{align}
			が成り立つが,これは$\mathcal{L}_{\in}$の式で
			\begin{align}
				\exists s A(s)
			\end{align}
			であって,
			\begin{align}
				\sigma = \varepsilon s A(s)
			\end{align}
			とおけば
			\begin{align}
				A(\sigma)
			\end{align}
			が成立する.ところで
			\begin{align}
				\sigma \in \{y|B(y)\}
			\end{align}
			なので
			\begin{align}
				B(\sigma)
			\end{align}
			も成り立つ.ゆえに
			\begin{align}
				A(\sigma) \wedge B(\sigma)
			\end{align}
			が成り立つ.ゆえに
			\begin{align}
				\exists s\, \left(\, A(s) \wedge B(s)\, \right)
			\end{align}
			が成り立つ.
			
		\item $\varepsilon x A(x) \in \varepsilon y B(y)$は$\exists s,t\, \left(\, s \in t \wedge A(s) \wedge B(t)\, \right)$
		
			この式の同値性も証明できる.まず
			\begin{align}
				\exists s,t\, \left(\, s \in t \wedge A(s) \wedge B(t)\, \right)
			\end{align}
			が成り立っているとしよう.この式は
			\begin{align}
				\exists s\, \left(\, \exists t\, \left(\, s \in t \wedge A(s) \wedge B(t)\, \right)\, \right)
			\end{align}
			の略記であって,$\exists$の規則より
			\begin{align}
				\sigma = \varepsilon s\, \left(\, \exists t\, \left(\, s \in t \wedge A(s) \wedge B(t)\, \right)\, \right)
			\end{align}
			とおけば
			\begin{align}
				\exists t\, \left(\, \sigma \in t \wedge A(\sigma) \wedge B(t)\, \right)
			\end{align}
			が成立する.$\sigma \in t \wedge A(\sigma) \wedge B(t)$を$\mathcal{L}_{\in}$の式に書き直したものを
			$\varphi(t)$として
			\begin{align}
				\tau \defeq \varepsilon t \varphi(t)
			\end{align}
			とおけば,$\exists$の規則より
			\begin{align}
				\sigma \in \tau \wedge A(\sigma) \wedge B(\tau)
			\end{align}
			が成立する.ゆえに
			\begin{align}
				\exists s\, \left(\, s \in \tau \wedge A(s)\, \right)
			\end{align}
			が成り立つから,公理より
			\begin{align}
				\varepsilon x A(x) \in \tau
			\end{align}
			が成立する.ゆえに
			\begin{align}
				\varepsilon x A(x) \in \tau \wedge B(\tau)
			\end{align}
			が成立する.ゆえに
			\begin{align}
				\exists t\, \left(\, \varepsilon x A(x) \in t \wedge B(t)\, \right)
			\end{align}
			が成立する.公理より
			\begin{align}
				\varepsilon x A(x) \in \varepsilon y B(y)
			\end{align}
			が成立する.逆は容易い.
			\begin{align}
				\varepsilon x A(x) \in \varepsilon y B(y)
			\end{align}
			が成り立っているとすれば公理より
			\begin{align}
				\exists t\, \left(\, \varepsilon x A(x) \in t \wedge B(t)\, \right)
			\end{align}
			が成立する.$\varepsilon x A(x) \in t \wedge B(t)$を$\mathcal{L}_{\in}$の式に書き直したものを$\psi(t)$として
			\begin{align}
				\tau \defeq \varepsilon t \psi(t)
			\end{align}
			とおけば
			\begin{align}
				\varepsilon x A(x) \in \tau \wedge B(\tau)
			\end{align}
			が成立するが,ここで公理より
			\begin{align}
				\exists s\, \left(\, s \in \tau \wedge A(s)\, \right)
			\end{align}
			が成り立つので,$s \in \tau \wedge A(s)$を$\mathcal{L}_{\in}$の式に書き直したものを$\xi(s)$として
			\begin{align}
				\sigma \defeq \varepsilon s \xi(s)
			\end{align}
			とおけば
			\begin{align}
				\sigma \in \tau \wedge A(\sigma)
			\end{align}
			が成立する.以上より
			\begin{align}
				\sigma \in \tau \wedge A(\sigma) \wedge B(\tau)
			\end{align}
			が成立する.ゆえに
			\begin{align}
				\exists t\, \left(\, \sigma \in t \wedge A(\sigma) \wedge B(t)\, \right)
			\end{align}
			が得られる.ゆえに
			\begin{align}
				\exists s\, \left(\, \exists t\, \left(\, \sigma \in t \wedge A(\sigma) \wedge B(t)\, \right)\, \right)
			\end{align}
			が得られる.
	\end{itemize}
	
	\begin{screen}
		\begin{logicalaxm}\mbox{}
			\begin{itemize}
				\item 任意の閉項$\tau$に対して,$A(\tau)$が定理ならば$\exists x A(x)$が成り立つ.
				\item $\exists x A(x)$が定理ならば,$A(\varepsilon x \hat{A}(x))$が成り立つ.
				\item すべての閉項$\tau$に対して$A(\tau)$が定理ならば,$\forall x A(x)$が成り立つ.
				\item $\forall x A(x)$が定理ならば,すべての閉項$\tau$に対して$A(\tau)$が成り立つ.
			\end{itemize}
		\end{logicalaxm}
	\end{screen}
	
	定理として
	\begin{align}
		\forall x A(x) \Longleftrightarrow A(\varepsilon x \rightharpoondown \hat{A}(x))
	\end{align}
	が得られる.アイデアとしてはさあ,$\varepsilon x A(x)$の全体が集合に対応しているのであって,
	いやもちろん集合そのものではないけど,集合は$\varepsilon x A(x)$のどれかに等しい類なわけで,
	だからモデル論に出てくる「宇宙」とかいう得体の知れない集合()は俺の集合論に不要なんだよね.
	俺のノートの「宇宙」はすべて実態が把握できるように,具体的な記号列で書き表せるのが良いよね.
	ちなみにこの「宇宙」は$\{x|x=x\}$とは別ね.
	
	\begin{screen}
		\begin{axm}
			\begin{align}
				\forall x\, \left(\, x \in \{y|B(y)\} \leftrightarrow B(x)\, \right).
			\end{align}
			
			\begin{align}
				\forall x\, \left(\, x \in \varepsilon y B(y) \leftrightarrow
				\left(\, \exists t\, B(t) \rightarrow 
				\exists t\, \left(\, x \in t \wedge B(t)\, \right)\, \right)\, \right).
			\end{align}
			が定理となるために
			\begin{align}
				x \in \varepsilon y B(y) \leftrightarrow
				\exists t\, \left(\, x \in t \wedge B(t) \leftrightarrow \exists y B(y)\, \right)
			\end{align}
			を公理とする.
			
			\begin{align}
				\varepsilon x A(x) \in y \leftrightarrow
				\exists s\, \left(\, s \in y \wedge A(s)\, \right).
			\end{align}
		\end{axm}
	\end{screen}
	
	いや,した二つは公理じゃねえな.定理だ.実際
	\begin{align}
		x \in \varepsilon y B(y)
	\end{align}
	が成り立っているとしよう.$\varepsilon y B(y)$は集合であって
	\begin{align}
		\exists s\, \left(\, s = \varepsilon y B(y)\, \right)
	\end{align}
	が成り立つので,
	\begin{align}
		fff
	\end{align}
	
	\begin{screen}
		\begin{thm}
			\begin{align}
				\exists x A(x) \rightarrow \varepsilon x A(x) \in \{x|A(x)\}.
			\end{align}
		\end{thm}
	\end{screen}
	
	\begin{sketch}
		\begin{align}
			\exists x A(x)
		\end{align}
		が成り立ているとするとき,
		\begin{align}
			\sigma \defeq \varepsilon x A(x)
		\end{align}
		とおけば
		\begin{align}
			A(\sigma)
		\end{align}
		が成り立つので,公理より
		\begin{align}
			\sigma \in \{x|A(x)\}
		\end{align}
		が成立する.ゆえに
		\begin{align}
			\sigma \in \{x|A(x)\} \wedge A(\sigma)
		\end{align}
		が成り立つ.ゆえに
		\begin{align}
			\exists s\, \left(\, s \in \{x|A(x)\} \wedge A(s)\, \right)
		\end{align}
		が成立する.ゆえに公理より
		\begin{align}
			\varepsilon x A(x) \in \{x|A(x)\}
		\end{align}
		が成立する.
		\QED
	\end{sketch}
	
	\begin{itembox}[l]{満たされて欲しいこと}
		\begin{description}
			\item[等号]
				\begin{itemize}
					\item $x = \{y|B(y)\}$と$\forall s\, \left(\, s \in x \leftrightarrow B(s)\, \right)$
					\item $x = \varepsilon y B(y)$と$\exists s\, \left(\, A(s) \wedge \forall u\,
						\left(\, u \in x \leftrightarrow u \in s\, \right)\, \right)$
					\item $\{x|A(x)\} = \{y|B(y)\}$と$\forall s\, \left(\, A(s) \leftrightarrow B(s)\, \right)$
					\item $\{x|A(x)\} = \varepsilon y B(y)$と
						$\exists s\, \left(\, \forall u\,
						\left(\, u \in s \leftrightarrow A(u)\, \right) \wedge B(s)\, \right)$
					\item $\varepsilon x A(x) = \varepsilon y B(y)$と
						$\exists s,t\, \left(\, s = t \wedge A(s) \wedge B(t)\, \right)$
				\end{itemize}
				
			\item[帰属]
				\begin{itemize}
					\item $x \in \{y|B(y)\}$は$B(x)$
					\item $x \in \varepsilon y B(y)$は$\exists t\, \left(\, x \in t \wedge B(t)\, \right)$
					\item $\{x|A(x)\} \in y$は$\exists s\, \left(\, s \in y \wedge \forall u\, \left(\, u \in s \leftrightarrow A(u)\, \right)\, \right)$
					\item $\{x|A(x)\} \in \{y|B(y)\}$は$\exists s\, \left(\, B(s) \wedge \forall u\, \left(\, u \in s \leftrightarrow A(u)\, \right)\, \right)$
					\item $\{x|A(x)\} \in \varepsilon y B(y)$は$\exists s,t\, \left(\, s \in t \wedge \forall u\, \left(\, u \in s \leftrightarrow A(u)\, \right) \wedge B(t)\, \right)$
					\item $\varepsilon x A(x) \in y$は$\exists s\, \left(\, s \in y \wedge A(s)\, \right)$
					\item $\varepsilon x A(x) \in \{y|B(y)\}$は$\exists s\, \left(\, A(s) \wedge B(s)\, \right)$
					\item $\varepsilon x A(x) \in \varepsilon y B(y)$は$\exists s,t\, \left(\, s \in t \wedge A(s) \wedge B(t)\, \right)$
				\end{itemize}
		\end{description}
	\end{itembox}

\section{定理I\hspace{-.1em}I.15.2}
	\begin{description}
		\item[(3)]
			項$\tau$が変項$x$のとき,$\zeta_{\tau}(y)$を
			\begin{align}
				x = y
			\end{align}
			とすれば,
			\begin{align}
				\Sigma' \vdash \forall x\, \exists! y\, (\, x=y\, )
			\end{align}
			つまり
			\begin{align}
				\Sigma' \vdash \forall x\, \exists! y\, \zeta_{\tau}(y)
			\end{align}
			および
			\begin{align}
				\Sigma \vdash \forall x\, (\, x=x\, )
			\end{align}
			つまり
			\begin{align}
				\Sigma \vdash \forall x\, \zeta_{\tau}(\tau)
			\end{align}
			が成り立つ.項$\tau$が
			\begin{align}
				f\tau_{1}\cdots\tau_{n}
			\end{align}
			のとき,$f \in \mathcal{L} \backslash \mathcal{L}_{\in}$ならば$\zeta_{\tau}(y)$を
			\begin{align}
				\exists z_{1}, \cdots, z_{n}\, \left(\, 
				\theta_{f}(z_{1},\cdots,z_{n},y) \wedge \zeta_{\tau_{1}}(z_{1}) \wedge
				\cdots \wedge \zeta_{\tau_{n}}(z_{n})\, \right)
			\end{align}
			とし,$f \in \mathcal{L}_{\in}$ならば
			\begin{align}
				\exists z_{1}, \cdots, z_{n}\, \left(\, 
				f(z_{1},\cdots,z_{n}) = y \wedge \zeta_{\tau_{1}}(z_{1}) \wedge
				\cdots \wedge \zeta_{\tau_{n}}(z_{n})\, \right)
			\end{align}
			とする.仮定より
			\begin{align}
				\Sigma \vdash \exists! y\, \zeta_{\tau_{i}}(y)
			\end{align}
			が成り立つので,
			\begin{align}
				\Sigma \vdash \zeta_{\tau_{i}}(z_{i})
			\end{align}
			を満たす$z_{i}$が取れる.そして定義I\hspace{-.1em}I.15.1より
			\begin{align}
				\Sigma \vdash \exists!y\, \theta_{f}(z_{1},\cdots,z_{n},y)
			\end{align}
			が成り立つので,その$y$を取れば
			\begin{align}
				\Sigma \vdash \exists y\, \zeta_{\tau}(y)
			\end{align}
			が成立する.ただし,$\eta$を
			\begin{align}
				\Sigma \vdash \exists z_{1}, \cdots, z_{n}\, \left(\, 
				\theta_{f}(z_{1},\cdots,z_{n},\eta) \wedge \zeta_{\tau_{1}}(z_{1}) \wedge
				\cdots \wedge \zeta_{\tau_{n}}(z_{n})\, \right)
			\end{align}
			を満たすものとすれば,このとき
			\begin{align}
				\Sigma \vdash \zeta_{\tau_{i}}(w_{i})
			\end{align}
			および
			\begin{align}
				\Sigma \vdash \theta_{f}(w_{1},\cdots,w_{n},\eta)
			\end{align}
			を満たす$w_{i}$が取れるが,$z_{i} = w_{i}$なので
			\begin{align}
				\Sigma \vdash \theta_{f}(z_{1},\cdots,z_{n},\eta)
			\end{align}
			が成り立つことになって,定義I\hspace{-.1em}I.15.1より
			\begin{align}
				y = \eta
			\end{align}
			が成り立つ.ゆえに
			\begin{align}
				\Sigma \vdash \exists! y\, \zeta_{\tau}(y)
			\end{align}
			が成立する.他方で仮定より
			\begin{align}
				\Sigma' \vdash \zeta_{\tau_{i}}(\tau_{i})
			\end{align}
			が成り立ち,かつ定義I\hspace{-.1em}I.15.1より
			\begin{align}
				\Sigma' \vdash \forall x_{1},\cdots,x_{n}\,
				\theta_{f}(x_{1},\cdots,x_{n},f(x_{1},\cdots,x_{n}))
			\end{align}
			が成り立つので
			\begin{align}
				\Sigma' \vdash \theta_{f}(\tau_{1},\cdots,\tau_{n},f(\tau_{1},\cdots,\tau_{n}))
			\end{align}
			が成り立つ.ゆえに
			\begin{align}
				\Sigma' \vdash \theta_{f}(\tau_{1},\cdots,\tau_{n},f(\tau_{1},\cdots,\tau_{n})) \wedge \zeta_{\tau_{1}}(\tau_{1}) \wedge \cdots \wedge \zeta_{\tau_{n}}(\tau_{n})
			\end{align}
			が成り立つ.ゆえに
			\begin{align}
				\Sigma' \vdash \zeta_{\tau}(\tau)
			\end{align}
			が成り立つ.
			
		\item[(2)]
			$\varphi$を$p\tau_{1}\cdots\tau_{n}$なる原子式とするとき,
			$p$が$\mathcal{L} \backslash \mathcal{L}_{\in}$の要素ならば$\hat{\varphi}$を
			\begin{align}
				\exists z_{1},\cdots,z_{n}\, 
				\left(\, \theta_{p}z_{1} \cdots z_{n} \wedge \zeta_{\tau_{1}}(z_{1})
				\wedge \cdots \wedge \zeta_{\tau_{n}}(z_{n})\, \right)
			\end{align}
			とし,$p$が$\mathcal{L}_{\in}$の要素ならば
			\begin{align}
				\exists z_{1},\cdots,z_{n}\, 
				\left(\, pz_{1} \cdots z_{n} \wedge \zeta_{\tau_{1}}(z_{1})
				\wedge \cdots \wedge \zeta_{\tau_{n}}(z_{n})\, \right)
			\end{align}
			とする.$\varphi$が成り立っているとき,仮定より
			\begin{align}
				\Sigma' \vdash \zeta_{\tau_{i}}(\tau_{i})
			\end{align}
			が満たされ,また定義I\hspace{-.1em}I.15.1より($\Delta$は$\Sigma'$に含まれているので)
			\begin{align}
				\Sigma' \cup \{\varphi\} \vdash \theta_{p}\tau_{1} \cdots \tau_{n}
			\end{align}
			も満たされているので
			\begin{align}
				\Sigma' \cup \{\varphi\} \vdash \theta_{p}\tau_{1} \cdots \tau_{n}
				\wedge \zeta_{\tau_{1}}(\tau_{1}) \wedge \cdots \wedge 
				\zeta_{\tau_{n}}(\tau_{n})
			\end{align}
			が成り立つ.すなわち
			\begin{align}
				\Sigma' \cup \{\varphi\} \vdash \hat{\varphi}
			\end{align}
			が成り立つ.つまり
			\begin{align}
				\Sigma' \vdash \varphi \rightarrow \hat{\varphi}
			\end{align}
			が成り立つ.逆に$\hat{\varphi}$が成り立っているとき,
			\begin{align}
				\Sigma' \cup \{\hat{\varphi}\} \vdash \theta_{p}w_{1} \cdots w_{n}
				\wedge \zeta_{\tau_{1}}(w_{1}) \wedge \cdots \wedge 
				\zeta_{\tau_{n}}(w_{n})
			\end{align}
			を満たす$w_{1},\cdots,w_{n}$が取れるが,
			\begin{align}
				\Sigma \vdash \exists! y\, \zeta_{\tau_{i}}(y)
			\end{align}
			かつ
			\begin{align}
				\Sigma' \vdash \zeta_{\tau_{i}}(\tau_{i})
			\end{align}
			なので
			\begin{align}
				w_{i} = \tau_{i}
			\end{align}
			である.ゆえに
			\begin{align}
				\Sigma' \cup \{\hat{\varphi}\} \vdash \theta_{p}\tau_{1} \cdots \tau_{n}
			\end{align}
			が成り立つ.ゆえに
			\begin{align}
				\Sigma' \vdash \hat{\varphi} \rightarrow \varphi
			\end{align}
			が得られる.
			\QED
	\end{description}
	
	\begin{screen}
		$\forall x\, (\, x \notin y\, )$を$\theta_{\emptyset}(y)$とするとき.
	\end{screen}
	
	$\emptyset$の定義$\delta_{\emptyset}$は
	\begin{align}
		\forall x\, (\, x \notin \emptyset\, )
	\end{align}
	である.$\zeta_{\emptyset}(y)$は
	\begin{align}
		\forall x\, (\, x \notin y\, )
	\end{align}
	であって,
	\begin{align}
		\Sigma \vdash \exists! y\, \zeta_{\emptyset}(y)
	\end{align}
	が成り立ち,また$\delta_{\emptyset}$と$\zeta_{\emptyset}(1)$は同じなので
	\begin{align}
		\Sigma \cup \{\delta_{\emptyset}\} = \Sigma' \vdash \zeta_{\emptyset}(\emptyset)
	\end{align}
	が成り立つ.そして,例えば$z$を変項とすれば
	\begin{align}
		z \in \emptyset
	\end{align}
	と
	\begin{align}
		\exists s,t\, \left(\, s \in t \wedge s = z \wedge \forall x\, (\, x \notin t\, )\, \right)
	\end{align}
	が$\Sigma'$の下で同値になる.
	
	\begin{screen}
		$\forall x\, (\, x \cdot y = y \cdot x = x\, )$を$\theta_{1}(y)$とするとき,
	\end{screen}
	
	$1$の定義$\delta_{1}$は
	\begin{align}
		\forall x\, (\, x \cdot 1 = 1 \cdot x = x\, )
	\end{align}
	である.$\zeta_{1}(y)$は
	\begin{align}
		\forall x\, (\, x \cdot y = y \cdot x = x\, )
	\end{align}
	であって,
	\begin{align}
		\Sigma \vdash \exists! y\, \zeta_{1}(y)
	\end{align}
	が成り立ち,また$\delta_{1}$と$\zeta_{1}(1)$は同じなので
	\begin{align}
		\Sigma \cup \{\delta_{1}\} = \Sigma' \vdash \zeta_{1}(1)
	\end{align}
	が成り立つ.そして,例えば$z$を変項とすれば
	\begin{align}
		z \in 1
	\end{align}
	と
	\begin{align}
		\exists s,t\, \left(\, s \in t \wedge s = z \wedge \forall x\, (\, x \cdot t = t \cdot x = x\, )\, \right)
	\end{align}
	が$\Sigma'$の下で同値になる.
	
	\begin{screen}
		$y \cdot (x \cdot x) = x$を$\theta_{i}(x,y)$とするとき,
	\end{screen}
	
	$i$の定義$\delta_{i}$は
	\begin{align}
		\forall x\, \left(\, i(x) \cdot (x \cdot x) = x\, \right)
	\end{align}
	である.$\zeta_{i(x)}(y)$は
	\begin{align}
		\exists x\, \left(\, \theta_{i}(x,y) \wedge x = x\, \right)
	\end{align}
	である.そして,例えば$x,z$を変項とすれば
	\begin{align}
		z \in i(x)
	\end{align}
	と
	\begin{align}
		\exists s,t\, \left(\, s \in t \wedge s = z \wedge \zeta_{i(x)}(t)\, \right)
	\end{align}
	が$\Sigma'$の下で同値になる.
	
\section{菊池誠不完全性定理}
	言語とは定数記号と関数記号と関係記号の全体ということで,
	\begin{itemize}
		\item $\mathcal{L}_{\in}$とは$\{\in,\natural\}$.
		\item $\mathcal{L}$とは$\{\in\}$に加えて閉項の全体.
	\end{itemize}
\section{$\mathcal{L}$の証明の変換}
\label{sec:L_proof_to_L_epsilon_proof}
	この節では 「$\Sigma$から$\psi$への{\bf HE}の証明で$\mathcal{L}$の文の列
	であるものが取れる」ならば 「$\Sigma$から$\psi$への{\bf HE}の証明で
	$\lang{\varepsilon}$の文の列であるものが取れる」ことを示す
	
	\begin{screen}
		\begin{metathm}[$\mathcal{L}$の文の証明は$\lang{\varepsilon}$の文の証明に直せる]
			$\psi$を$\lang{\in}$の文とするとき,
			$\Sigma \provable{\mbox{{\bf HE}},\mathcal{L}} \psi$ならば
			$\Sigma \provable{\mbox{{\bf HE}},\lang{\varepsilon}} \psi$である.
		\end{metathm}
	\end{screen}
	
	\begin{metaprf}
		$\varphi_{1},\cdots,\varphi_{n}$を$\Sigma$から$\psi$への証明で$\mathcal{L}$の
		文の列であるものとし,これらを$\lang{\varepsilon}$の文に書き換えたもの
		(\ref{subsec:formula_rewriting}節参照)を
		$\widehat{\varphi}_{1},\cdots,\widehat{\varphi}_{n}$とする.
		ただし,同じ原子式の書き換えは証明全体で一致するようにしておく
		(書き換え時に用意する変項は$\varphi_{1},\cdots,\varphi_{n}$で使われていないものとする).
		このときメタ定理\ref{metathm:rewritten_formulas_are_of_L_epsilon}と
		メタ定理\ref{metathm:variables_unchanged_after_rewriting}より
		$\widehat{\varphi}_{1},\cdots,\widehat{\varphi}_{n}$はいずれも
		$\lang{\varepsilon}$の文であり,また各$\widehat{\varphi}_{i}$について次が言える:
		\begin{description}
			\item[(1)] $\varphi_{i}$が{\bf HE}の命題論理の公理ならば$\widehat{\varphi}_{i}$も
				{\bf HE}の公理である.
				
			\item[(2)] $\varphi_{i}$が{\bf HE}の量化公理か$\Sigma$の公理ならば$\Sigma 
				\provable{\mbox{{\bf HE}},\lang{\varepsilon}} \widehat{\varphi}_{i}$である.
				
			\item[(3)] $\varphi_{i}$が前の文$\varphi_{j},\varphi_{k}$から三段論法で
				得られている場合は,$\widehat{\varphi}_{i}$は$\widehat{\varphi}_{j}$と
				$\widehat{\varphi}_{k}$から三段論法で得られる.
		\end{description}
		
		(1)については,書き換えても式の結合形式が変わらないため.
		
		(3)については,$\varphi_{k}$を$\varphi_{j} \rarrow \varphi_{i}$なる文とすれば,
		同じ原子式の書き換えは証明全体で一致しているので$\widehat{\varphi_{k}}$は
		$\widehat{\varphi}_{j} \rarrow \widehat{\varphi}_{i}$なる文であり,
		$\widehat{\varphi}_{i}$は$\widehat{\varphi}_{j}$と$\widehat{\varphi}_{k}$から
		三段論法で得られるのである.
		
		(2)について,内包項を含みうる$\Sigma$の公理は外延性,相等性,内包性,要素である.
		これらと{\bf HE}の量化公理について一つずつ示していく.
		\begin{description}
			\item[case1] $\varphi_{i}$が
				\begin{align}
					\exists x \varphi \rarrow \varphi(\varepsilon x \check{\varphi})
				\end{align}
				なる公理であれば($\check{\varphi}$は$\varphi$の書き換え),
				$\widehat{\varphi}_{i}$は
				\begin{align}
					\exists x \widehat{\varphi} \rarrow 
					\widetilde{\varphi(\varepsilon x \check{\varphi})}
				\end{align}
				なる形の式である.ただし$\widehat{\varphi},
				\widetilde{\varphi(\varepsilon x \check{\varphi})}$はそれぞれ
				$\varphi,\varphi(\varepsilon x \check{\varphi})$の書き換えとする.
				まず量化公理と演繹定理の逆より
				\begin{align}
					\exists x \widehat{\varphi} \provable{\mbox{{\bf HE}},\lang{\varepsilon}} \widehat{\varphi}(\varepsilon x \widehat{\varphi})
				\end{align}
				となる.ここで,メタ定理\ref{metathm:substitution_to_rewritten_formula}より
				$\widehat{\varphi}(\varepsilon x \widehat{\varphi})$も
				$\check{\varphi}(\varepsilon x \widehat{\varphi})$も
				$\varphi(\varepsilon x \widehat{\varphi})$の書き換えであるし,
				メタ定理\ref{metathm:formula_rewritings_coincide_except_the_differences_of_bound_variables}
				より$\widehat{\varphi}(\varepsilon x \widehat{\varphi})$から始めて量化部分式を複数回
				差し替えていけば$\check{\varphi}(\varepsilon x \widehat{\varphi})$が得られるので,
				定理\ref{logicalthm:equivalence_by_replacing_bound_variables}より
				\begin{align}
					\provable{\mbox{{\bf HE}},\lang{\varepsilon}} \widehat{\varphi}(\varepsilon x \widehat{\varphi}) \rarrow \check{\varphi}(\varepsilon x \widehat{\varphi})
					\label{thm:L_proof_to_L_epsilon_proof_01}
				\end{align}
				が成り立つ.三段論法より
				\begin{align}
					\exists x \widehat{\varphi}, \provable{\mbox{{\bf HE}},\lang{\varepsilon}} \check{\varphi}(\varepsilon x \widehat{\varphi})
				\end{align}
				となり,存在記号の公理
				\begin{align}
					\provable{\mbox{{\bf HE}},\lang{\varepsilon}} \check{\varphi}(\varepsilon x \widehat{\varphi}) \rarrow \exists x \check{\varphi}
				\end{align}
				と併せて
				\begin{align}
					\exists x \widehat{\varphi} \provable{\mbox{{\bf HE}},\lang{\varepsilon}} \exists x \check{\varphi}
				\end{align}
				が従う.存在記号の公理より
				\begin{align}
					\provable{\mbox{{\bf HE}},\lang{\varepsilon}} \exists x \check{\varphi} \rarrow \check{\varphi}(\varepsilon x \check{\varphi})
				\end{align}
				が成り立つので,再び三段論法より
				\begin{align}
					\exists x \widehat{\varphi} &\provable{\mbox{{\bf HE}},\lang{\varepsilon}} \check{\varphi}(\varepsilon x \check{\varphi})
				\end{align}
				が従う.(\refeq{thm:L_proof_to_L_epsilon_proof_01})と同様の理由で
				%ここで定理\ref{logicalthm:equivalence_by_replacing_bound_variables}より
				\begin{align}
					\provable{\mbox{{\bf HE}},\lang{\varepsilon}} \check{\varphi}(\varepsilon x \check{\varphi}) \rarrow  \widetilde{\varphi(\varepsilon x \check{\varphi})}
				\end{align}
				が成り立つので,三段論法で
				%(定理\ref{metathm:substitution_to_rewritten_formula}より
				%$\check{\varphi}(\varepsilon x \check{\varphi})$も
				%$\varphi(\varepsilon x \check{\varphi})$の書き換え)
				\begin{align}
					\exists x \widehat{\varphi}, \provable{\mbox{{\bf HE}},\lang{\varepsilon}} \widetilde{\varphi(\varepsilon x \check{\varphi})}
				\end{align}
				が得られる.$\varphi_{i}$が$\varphi(\varepsilon x \negation \check{\varphi}) \rarrow \forall x \varphi$なる式の場合も同様である.
				
			\item[case2] $\varphi_{i}$が
				\begin{align}
					\varphi(\tau) \rarrow \exists x \varphi
				\end{align}
				なる公理であれば,$\widehat{\varphi}_{i}$は
				\begin{align}
					\widetilde{\varphi(\tau)} \rarrow \exists x \widehat{\varphi}
				\end{align}
				なる式となる.ただし$\widetilde{\varphi(\tau)},\widehat{\varphi}$は
				それぞれ$\varphi(\tau),\varphi$の書き換えとする.
				これは{\bf HE}から演繹可能である.実際,
				(\refeq{thm:L_proof_to_L_epsilon_proof_01})と同様の理由で
				%定理\ref{logicalthm:equivalence_by_replacing_bound_variables}より
				\begin{align}
					\provable{\mbox{{\bf HE}},\lang{\varepsilon}} 
					\widetilde{\varphi(\tau)} \rarrow \widehat{\varphi}(\tau)
				\end{align}
				が成り立ち,他方で存在記号の公理より
				\begin{align}
					\provable{\mbox{{\bf HE}},\lang{\varepsilon}} 
					\widehat{\varphi}(\tau) \rarrow \exists x \widehat{\varphi}
				\end{align}
				が成り立つので,
				\begin{align}
					\provable{\mbox{{\bf HE}},\lang{\varepsilon}} 
					\widetilde{\varphi(\tau)} \rarrow \exists x \widehat{\varphi}
				\end{align}
				が従う.$\varphi_{i}$が$\forall x \varphi \rarrow \varphi(\tau)$なる
				式の場合も同様である.
				
			%\item[case3] $\varphi_{i}$が
			%	\begin{align}
			%		\forall x \varphi \rarrow \varphi(\tau)
			%	\end{align}
			%	なる公理であれば,$\widehat{\varphi}_{i}$は
			%	\begin{align}
			%		\forall x \widetilde{\varphi} \rarrow \check{\varphi}(\tau)
			%	\end{align}
			%	なる式となる.これは{\bf HE}から演繹可能である.実際,{\bf HE}の量化公理より
			%	\begin{align}
			%		\provable{\mbox{{\bf HE}},\lang{\varepsilon}} \forall x \widetilde{\varphi} \rarrow \widetilde{\varphi}(\tau)
			%	\end{align}
			%	が成り立ち,他方で定理\ref{logicalthm:equivalence_by_replacing_bound_variables}より
			%	\begin{align}
			%		\provable{\mbox{{\bf HE}},\lang{\varepsilon}} \widetilde{\varphi}(\tau) \rarrow \check{\varphi}(\tau)
			%	\end{align}
			%	が成り立つので,
			%	\begin{align}
			%		\provable{\mbox{{\bf HE}},\lang{\varepsilon}} \forall x \widetilde{\varphi} \rarrow \check{\varphi}(\tau)
			%	\end{align}
			%	が従う.
				
			\item[case3] $\varphi_{i}$が外延性公理
				\begin{align}
					\forall x\, (\, x \in a \lrarrow x \in b\, ) \rarrow a = b
				\end{align}
				であるとき,$a,b$が共に主要$\varepsilon$項ならばこれは$\lang{\varepsilon}$
				の文である.
				$a$が$\Set{y}{\varphi(y)}$なる項で$b$が
				主要$\varepsilon$項であるときは,$\widehat{\varphi}_{i}$は
				\begin{align}
					\forall x\, (\, \varphi(x) \lrarrow x \in b\, ) \rarrow 
					\forall z\, (\, \varphi(z) \lrarrow z \in b\, )
				\end{align}
				なる形の文となり,これは${\bf HE}$で証明可能である.実際
				\begin{align}
					\zeta \defeq \varepsilon z \negation (\, \varphi(z) \lrarrow z \in b\, )
				\end{align}
				とおけば
				\begin{align}
					\forall x\, (\, \varphi(x) \lrarrow x \in b\, ) 
					\provable{\mbox{{\bf HE}},\lang{\varepsilon}} \varphi(\zeta) \lrarrow \zeta \in b
				\end{align}
				が成り立つので,全称の導出(論理的定理\ref{logicalthm:derivation_of_universal_by_epsilon})より
				\begin{align}
					\forall x\, (\, \varphi(x) \lrarrow x \in b\, ) 
					\provable{\mbox{{\bf HE}},\lang{\varepsilon}}
					\forall z\, (\, \varphi(z) \lrarrow z \in b\, )
				\end{align}
				となる.$a$が$\Set{y}{\varphi(y)}$なる項で$b$が$\Set{z}{\psi(z)}$なる
				項のときは,$\widehat{\varphi}_{i}$は
				\begin{align}
					\forall x\, (\, \varphi(x) \lrarrow \psi(x)\, ) \rarrow 
					\forall u\, (\, \varphi(u) \lrarrow \psi(u)\, )
				\end{align}
				なる形の文となり,これも${\bf HE}$で証明可能である.
				$a$が主要$\varepsilon$項で$b$が$\Set{z}{\psi(z)}$なる項のときも
				同様に$\widehat{\varphi}_{i}$は{\bf HE}で証明可能である.
			
			\item[case4] $\varphi_{i}$が内包性公理
				\begin{align}
					\forall x\, (\, x \in \Set{y}{\varphi(y)} \lrarrow \varphi(x)\, )
				\end{align}
				なる式であるとき,$\widehat{\varphi}_{i}$は
				\begin{align}
					\forall x\, (\, \varphi(x) \lrarrow \varphi(x)\, )
				\end{align}
				なる式であり,これは{\bf HE}から証明可能である.実際
				\begin{align}
					\tau \defeq \varepsilon x \negation (\, \varphi(x) \lrarrow \varphi(x)\, )
				\end{align}
				とおけば,含意の反射律(論理的定理\ref{logicalthm:reflective_law_of_implication})と論理積の導入より
				\begin{align}
					\provable{\mbox{{\bf HE}},\lang{\varepsilon}} \varphi(\tau) \lrarrow \varphi(\tau)
				\end{align}
				が成り立つので,全称の導出(論理的定理\ref{logicalthm:derivation_of_universal_by_epsilon})より
				\begin{align}
					\provable{\mbox{{\bf HE}},\lang{\varepsilon}} \forall x\, (\, \varphi(x) \lrarrow \varphi(x)\, )
				\end{align}
				となる.
			
			\item[case5] $\varphi_{i}$が要素の公理
				\begin{align}
					a \in b \rarrow \exists x\, (\, a = x\, )
				\end{align}
				なる式であるとき,$a$も$b$も主要$\varepsilon$項ならば
				\begin{align}
					\Sigma \provable{\mbox{{\bf HE}},\lang{\varepsilon}} \exists x\, (\, a = x\, )
				\end{align}
				(定理\ref{thm:critical_epsilon_term_is_set})と含意の導入
				\begin{align}
					\provable{\mbox{{\bf HE}},\lang{\varepsilon}} \exists x\, (\, a = x\, )
					\rarrow (\, a \in b \rarrow \exists x\, (\, a = x\, )\, )
				\end{align}
				から
				\begin{align}
					\Sigma \provable{\mbox{{\bf HE}},\lang{\varepsilon}} a \in b \rarrow \exists x\, (\, a = x\, )
				\end{align}
				が従う.$a$が主要$\varepsilon$項で$b$が$\Set{z}{\psi(z)}$なる項であるとき,
				$\widehat{\varphi}_{i}$は
				\begin{align}
					\psi(a) \rarrow \exists x\, (\, a = x\, )
				\end{align}
				となるが,上と同様にして{\bf HE}で証明できる.
				$a$が$\Set{y}{\varphi(y)}$なる項で$b$が主要$\varepsilon$であるとき,
				$\widehat{\varphi}_{i}$は
				\begin{align}
					\exists s\, (\, \forall u\, (\, \varphi(u) \lrarrow u \in s\, )
					\wedge s \in b\, ) \rarrow \exists x\, \forall v\, (\, \varphi(v) \lrarrow v \in x\, )
				\end{align}
				となるが,これも{\bf HE}で証明可能で,実際
				\begin{align}
					\sigma &\defeq \varepsilon s\, (\, \forall u\, (\, \varphi(u) \lrarrow u \in s\, ), \\
					\tau &\defeq \varepsilon v \negation (\, \varphi(v) \lrarrow v \in \sigma\, )
				\end{align}
				とおけば
				\begin{align}
					\exists s\, (\, \forall u\, (\, \varphi(u) \lrarrow u \in s\, )
					\wedge s \in b\, )
					&\provable{\mbox{{\bf HE}},\lang{\varepsilon}} 
					\forall u\, (\, \varphi(u) \lrarrow u \in \sigma\, )
					\wedge \sigma \in b, \\
					\exists s\, (\, \forall u\, (\, \varphi(u) \lrarrow u \in s\, )
					\wedge s \in b\, )
					&\provable{\mbox{{\bf HE}},\lang{\varepsilon}} 
					\forall u\, (\, \varphi(u) \lrarrow u \in \sigma\, ), \\
					\exists s\, (\, \forall u\, (\, \varphi(u) \lrarrow u \in s\, )
					\wedge s \in b\, )
					&\provable{\mbox{{\bf HE}},\lang{\varepsilon}} \varphi(\tau) \lrarrow \tau \in \sigma, \\
					\exists s\, (\, \forall u\, (\, \varphi(u) \lrarrow u \in s\, )
					\wedge s \in b\, )
					&\provable{\mbox{{\bf HE}},\lang{\varepsilon}} \forall v\, (\, \varphi(v) \lrarrow v \in \sigma\, ), \\
					\exists s\, (\, \forall u\, (\, \varphi(u) \lrarrow u \in s\, )
					\wedge s \in b\, )
					&\provable{\mbox{{\bf HE}},\lang{\varepsilon}} \exists x\, \forall v\, (\, \varphi(v) \lrarrow v \in x\, )
				\end{align}
				が成り立つ.$a$が$\Set{y}{\varphi(y)}$なる項で$b$が$\Set{z}{\psi(z)}$なる項
				であるとき,$\widehat{\varphi}_{i}$は
				\begin{align}
					\exists s\, (\, \forall u\, (\, \varphi(u) \lrarrow u \in s\, )
					\wedge \psi(s)\, ) \rarrow \exists x\, \forall v\, (\, \varphi(v) \lrarrow v \in x\, )
				\end{align}
				となるが,これも同様に{\bf HE}で証明可能である.
				
			\item[case6] $\varphi_{i}$が相等性公理
				\begin{align}
					a = b \rarrow b = a
				\end{align}
				なる式である場合,たとえば$a$が$\Set{y}{\varphi(y)}$なる項で
				$b$が主要$\varepsilon$項であれば,$\widehat{\varphi}_{i}$は
				\begin{align}
					\forall u\, (\, \varphi(u) \lrarrow u \in b\, ) 
					\rarrow \forall v\, (\, v \in b \lrarrow \varphi(v)\, ) 
				\end{align}
				となるが,これは{\bf HE}で証明可能であって,実際
				\begin{align}
					\tau \defeq \varepsilon v \negation (\, v \in b \lrarrow \varphi(v)\, )
				\end{align}
				とおけば
				\begin{align}
					\forall u\, (\, \varphi(u) \lrarrow u \in b\, ) 
					&\provable{\mbox{{\bf HE}},\lang{\varepsilon}} \varphi(\tau) \lrarrow \tau \in b, \\
					\forall u\, (\, \varphi(u) \lrarrow u \in b\, ) 
					&\provable{\mbox{{\bf HE}},\lang{\varepsilon}} \tau \in b \lrarrow \varphi(\tau), \\
					\forall u\, (\, \varphi(u) \lrarrow u \in b\, ) 
					&\provable{\mbox{{\bf HE}},\lang{\varepsilon}} \forall v\, (\, v \in b \lrarrow \varphi(v)\, )
				\end{align}
				が成り立つ.$a$も$b$も内包項である場合や,$a$が主要$\varepsilon$項で
				$b$が内包項である場合も同様のことが言える.
			
			\item[case7] $\varphi_{i}$が相等性公理
				\begin{align}
					a = b \rarrow (\, a \in c \rarrow b \in c\, )
				\end{align}
				なる式である場合,
				\begin{description}
					\item[case(7-1)] $a$と$b$が主要$\varepsilon$項で$c$が$\Set{x}{\xi(x)}$なる項であれば,
						$\widehat{\varphi}_{i}$は
						\begin{align}
							a = b \rarrow (\, \xi(a) \rarrow \xi(b)\, )
						\end{align}
						となるが,代入原理(定理\ref{thm:the_principle_of_substitution})より
						\begin{align}
							\Sigma \provable{\mbox{{\bf HE}},\lang{\varepsilon}} a = b \rarrow (\, \xi(a) \rarrow \xi(b)\, )
						\end{align}
						が成り立つ.
						
					\item[case(7-2)] $a$と$c$が主要$\varepsilon$項で$b$が$\Set{z}{\psi(z)}$なる項であれば,
						$\widehat{\varphi}_{i}$は
						\begin{align}
							\forall u\, (\, u \in a \lrarrow \psi(u)\, ) 
							\rarrow (\, a \in c 
							\rarrow \exists t\, (\, \forall w\, (\, \psi(w) \lrarrow w \in t\, ) \wedge t \in c\, )\, )
						\end{align}
						となるが,
						\begin{align}
							\omega \defeq \varepsilon w \negation (\, \psi(w) \lrarrow w \in a\, )
						\end{align}
						とおけば
						\begin{align}
							\forall u\, (\, u \in a \lrarrow \psi(u)\, )
							\provable{\mbox{{\bf HE}},\lang{\varepsilon}} \psi(\omega) \lrarrow \omega \in a 
						\end{align}
						が成り立つので
						\begin{align}
							\forall u\, (\, u \in a \lrarrow \psi(u)\, )
							\provable{\mbox{{\bf HE}},\lang{\varepsilon}} \forall w\, (\, \psi(w) \lrarrow w \in a\, )
						\end{align}
						が従い,
						\begin{align}
							\forall u\, (\, u \in a \lrarrow \psi(u)\, ),\ a \in c
							\provable{\mbox{{\bf HE}},\lang{\varepsilon}} \forall w\, (\, \psi(w) \lrarrow w \in a\, ) \wedge a \in c
						\end{align}
						より
						\begin{align}
							\forall u\, (\, u \in a \lrarrow \psi(u)\, ),\ a \in c
							\provable{\mbox{{\bf HE}},\lang{\varepsilon}} \exists t\, (\, \forall w\, (\, \psi(w) \lrarrow w \in t\, ) \wedge t \in c\, )\, )
						\end{align}
						となる.
						
					\item[case(7-3)] $a$が主要$\varepsilon$項で$b$が$\Set{z}{\psi(z)}$なる項で
						$c$が$\Set{x}{\xi(x)}$なる項であれば,$\widehat{\varphi}_{i}$は
						\begin{align}
							\forall u\, (\, u \in a \lrarrow \psi(u)\, ) 
							\rarrow (\, \xi(a) 
							\rarrow \exists t\, (\, \forall w\, (\, \psi(w) \lrarrow w \in t\, ) \wedge \xi(t)\, )\, )
						\end{align}
						となるが,
						\begin{align}
							\omega \defeq \varepsilon w \negation (\, \psi(w) \lrarrow w \in t\, )
						\end{align}
						とおけば
						\begin{align}
							\forall u\, (\, u \in a \lrarrow \psi(u)\, )
							\provable{\mbox{{\bf HE}},\lang{\varepsilon}} \psi(\omega) \lrarrow \omega \in a 
						\end{align}
						が成り立つので
						\begin{align}
							\forall u\, (\, u \in a \lrarrow \psi(u)\, )
							\provable{\mbox{{\bf HE}},\lang{\varepsilon}} \forall w\, (\, \psi(w) \lrarrow w \in t\, )
						\end{align}
						が従い,
						\begin{align}
							\forall u\, (\, u \in a \lrarrow \psi(u)\, )\, \xi(a)
							\provable{\mbox{{\bf HE}},\lang{\varepsilon}} \forall w\, (\, \psi(w) \lrarrow w \in t\, ) \wedge \xi(a)
						\end{align}
						より
						\begin{align}
							\forall u\, (\, u \in a \lrarrow \psi(u)\, )\, a \in c
							\provable{\mbox{{\bf HE}},\lang{\varepsilon}} \exists t\, (\, \forall w\, (\, \psi(w) \lrarrow w \in t\, ) \wedge \xi(t)\, )\, )
						\end{align}
						となる.
						
					\item[case(7-4)] $a$が$\Set{y}{\varphi(y)}$なる項で$b$と$z$が主要$\varepsilon$項の場合,
						$\widehat{\varphi}_{i}$は
						\begin{align}
							\forall u\, (\, \varphi(u) \lrarrow u \in b\, ) 
							\rarrow (\, \exists s\, (\, \forall v\, (\, \varphi(v) \lrarrow v \in s\, ) \wedge s \in c\, )
							\rarrow b \in c\, )
						\end{align}
						となるが,
						\begin{align}
							\sigma &\defeq \varepsilon s\, (\, \forall v\, (\, \varphi(v) \lrarrow v \in s\, ) \wedge s \in c\, ), \\
							\delta &\defeq \varepsilon u \negation (\, u \in \sigma \lrarrow u \in b\, )
						\end{align}
						とおけば
						\begin{align}
							\forall u\, (\, \varphi(u) \lrarrow u \in b\, ),\ 
							\exists s\, (\, \forall v\, (\, \varphi(v) \lrarrow v \in s\, ) \wedge s \in c\, )
							&\provable{\mbox{{\bf HE}},\lang{\varepsilon}} \varphi(\delta) \lrarrow \delta \in b, \\
							\forall u\, (\, \varphi(u) \lrarrow u \in b\, ),\ 
							\exists s\, (\, \forall v\, (\, \varphi(v) \lrarrow v \in s\, ) \wedge s \in c\, )
							&\provable{\mbox{{\bf HE}},\lang{\varepsilon}} \varphi(\delta) \lrarrow \delta \in \sigma
						\end{align}
						が成り立つので
						\begin{align}
							\forall u\, (\, \varphi(u) \lrarrow u \in b\, ),\ 
							\exists s\, (\, \forall v\, (\, \varphi(v) \lrarrow v \in s\, ) \wedge s \in c\, )
							\provable{\mbox{{\bf HE}},\lang{\varepsilon}} \delta \in \sigma \lrarrow \delta \in b
						\end{align}
						が従い,
						\begin{align}
							\forall u\, (\, \varphi(u) \lrarrow u \in b\, ),\ 
							\exists s\, (\, \forall v\, (\, \varphi(v) \lrarrow v \in s\, ) \wedge s \in c\, )
							\provable{\mbox{{\bf HE}},\lang{\varepsilon}} \forall u\, (\, u \in \sigma \lrarrow u \in b\, )
						\end{align}
						となり,外延性公理より
						\begin{align}
							\forall u\, (\, \varphi(u) \lrarrow u \in b\, ),\ 
							\exists s\, (\, \forall v\, (\, \varphi(v) \lrarrow v \in s\, ) \wedge s \in c\, ),\ 
							\Sigma
							\provable{\mbox{{\bf HE}},\lang{\varepsilon}} \sigma = b
						\end{align}
						が出る.一方で
						\begin{align}
							\exists s\, (\, \forall v\, (\, \varphi(v) \lrarrow v \in s\, ) \wedge s \in c\, )
							\provable{\mbox{{\bf HE}},\lang{\varepsilon}} \sigma \in c
						\end{align}
						となるので,相等性公理より
						\begin{align}
							\forall u\, (\, \varphi(u) \lrarrow u \in b\, ),\ 
							\exists s\, (\, \forall v\, (\, \varphi(v) \lrarrow v \in s\, ) \wedge s \in c\, ),\ 
							\Sigma
							\provable{\mbox{{\bf HE}},\lang{\varepsilon}} b \in c
						\end{align}
						が成り立つ.
						
					\item[case(7-5)] $a$が$\Set{y}{\varphi(y)}$なる項で$b$が主要$\varepsilon$項で
						$z$が$\Set{x}{\xi(x)}$なる項の場合,$\widehat{\varphi}_{i}$は
						\begin{align}
							\forall u\, (\, \varphi(u) \lrarrow u \in b\, ) 
							\rarrow (\, \exists s\, (\, \forall v\, (\, \varphi(v) \lrarrow v \in s\, ) \wedge \xi(s)\, )
							\rarrow \xi(b)\, )
						\end{align}
						となるが,
						\begin{align}
							\sigma &\defeq \varepsilon s\, (\, \forall v\, (\, \varphi(v) \lrarrow v \in s\, ) \wedge \xi(s)\, ), \\
							\delta &\defeq \varepsilon u \negation (\, u \in \sigma \lrarrow u \in b\, )
						\end{align}
						とおけば
						\begin{align}
							\forall u\, (\, \varphi(u) \lrarrow u \in b\, ),\ 
							\exists s\, (\, \forall v\, (\, \varphi(v) \lrarrow v \in s\, ) \wedge \xi(s)\, )
							&\provable{\mbox{{\bf HE}},\lang{\varepsilon}} \varphi(\delta) \lrarrow \delta \in b, \\
							\forall u\, (\, \varphi(u) \lrarrow u \in b\, ),\ 
							\exists s\, (\, \forall v\, (\, \varphi(v) \lrarrow v \in s\, ) \wedge \xi(s)\, )
							&\provable{\mbox{{\bf HE}},\lang{\varepsilon}} \varphi(\delta) \lrarrow \delta \in \sigma
						\end{align}
						が成り立つので
						\begin{align}
							\forall u\, (\, \varphi(u) \lrarrow u \in b\, ),\ 
							\exists s\, (\, \forall v\, (\, \varphi(v) \lrarrow v \in s\, ) \wedge \xi(s)\, )
							\provable{\mbox{{\bf HE}},\lang{\varepsilon}} \delta \in \sigma \lrarrow \delta \in b
						\end{align}
						が従い,
						\begin{align}
							\forall u\, (\, \varphi(u) \lrarrow u \in b\, ),\ 
							\exists s\, (\, \forall v\, (\, \varphi(v) \lrarrow v \in s\, ) \wedge \xi(s)\, )
							\provable{\mbox{{\bf HE}},\lang{\varepsilon}} \forall u\, (\, u \in \sigma \lrarrow u \in b\, )
						\end{align}
						となり,外延性公理より
						\begin{align}
							\forall u\, (\, \varphi(u) \lrarrow u \in b\, ),\ 
							\exists s\, (\, \forall v\, (\, \varphi(v) \lrarrow v \in s\, ) \wedge \xi(s)\, ),\ 
							\Sigma
							\provable{\mbox{{\bf HE}},\lang{\varepsilon}} \sigma = b
						\end{align}
						が出る.一方で
						\begin{align}
							\exists s\, (\, \forall v\, (\, \varphi(v) \lrarrow v \in s\, ) \wedge \xi(s)\, )
							\provable{\mbox{{\bf HE}},\lang{\varepsilon}} \xi(\sigma)
						\end{align}
						となるので,代入原理(定理\ref{thm:the_principle_of_substitution})より
						\begin{align}
							\forall u\, (\, \varphi(u) \lrarrow u \in b\, ),\ 
							\exists s\, (\, \forall v\, (\, \varphi(v) \lrarrow v \in s\, ) \wedge \xi(s)\, ),\ 
							\Sigma
							\provable{\mbox{{\bf HE}},\lang{\varepsilon}} \xi(b)
						\end{align}
						が成り立つ.
						
					\item[case(7-6)] $a$が$\Set{y}{\varphi(y)}$なる項で$b$が$\Set{z}{\psi(z)}$なる項で
						$c$が主要$\varepsilon$項であれば,$\widehat{\varphi}_{i}$は
						\begin{align}
							\forall u\, (\, \varphi(u) \lrarrow \psi(u)\, ) 
							&\rarrow (\, \exists s\, (\, \forall v\, (\, \varphi(v) \lrarrow v \in s\, ) \wedge s \in c\, ) \\
							&\rarrow \exists t\, (\, \forall w\, (\, \psi(w) \lrarrow w \in t\, ) \wedge t \in c\, )\, )
						\end{align}
						となるが,
						\begin{align}
							\tau &\defeq \varepsilon s\, (\, \forall v\, (\, \varphi(v) \lrarrow v \in s\, ) \wedge s \in c\, ), \\
							\omega &\defeq \varepsilon w \negation (\, \psi(w) \lrarrow w \in t\, )
						\end{align}
						とおけば,まず
						\begin{align}
							\forall u\, (\, \varphi(u) \lrarrow \psi(u)\, ),\ 
							\exists s\, (\, \forall v\, (\, \varphi(v) \lrarrow v \in s\, ) \wedge s \in c\, )
							&\provable{\mbox{{\bf HE}},\lang{\varepsilon}} \varphi(\omega) \lrarrow \psi(\omega), \\
							\forall u\, (\, \varphi(u) \lrarrow \psi(u)\, ),\ 
							\exists s\, (\, \forall v\, (\, \varphi(v) \lrarrow v \in s\, ) \wedge s \in c\, )
							&\provable{\mbox{{\bf HE}},\lang{\varepsilon}} \varphi(\omega) \lrarrow \omega \in \tau
						\end{align}
						が成り立つので
						\begin{align}
							\forall u\, (\, \varphi(u) \lrarrow \psi(u)\, ),\ 
							\exists s\, (\, \forall v\, (\, \varphi(v) \lrarrow v \in s\, ) \wedge s \in c\, )
							\provable{\mbox{{\bf HE}},\lang{\varepsilon}} \psi(\omega) \lrarrow \omega \in \tau
						\end{align}
						が従い
						\begin{align}
							\forall u\, (\, \varphi(u) \lrarrow \psi(u)\, ),\ 
							\exists s\, (\, \forall v\, (\, \varphi(v) \lrarrow v \in s\, ) \wedge s \in c\, )
							\provable{\mbox{{\bf HE}},\lang{\varepsilon}} \forall w\, (\, \psi(w) \lrarrow w \in \tau\, )
						\end{align}
						が出る.他方で
						\begin{align}
							\exists s\, (\, \forall v\, (\, \varphi(v) \lrarrow v \in s\, ) \wedge s \in c\, )
							\provable{\mbox{{\bf HE}},\lang{\varepsilon}} \tau \in c, \\
						\end{align}
						が成り立つので,
						\begin{align}
							\begin{gathered}
								\forall u\, (\, \varphi(u) \lrarrow \psi(u)\, ),\ 
								\exists s\, (\, \forall v\, (\, \varphi(v) \lrarrow v \in s\, ) \wedge s \in c\, ) \\
								\provable{\mbox{{\bf HE}},\lang{\varepsilon}}
								\forall w\, (\, \psi(w) \lrarrow w \in \tau\, ) \wedge \tau \in c
							\end{gathered}
						\end{align}
						となり
						\begin{align}
							\begin{gathered}
								\forall u\, (\, \varphi(u) \lrarrow \psi(u)\, ),\ 
								\exists s\, (\, \forall v\, (\, \varphi(v) \lrarrow v \in s\, ) \wedge s \in c\, ) \\
								\provable{\mbox{{\bf HE}},\lang{\varepsilon}}
								\exists t\, (\, \forall w\, (\, \psi(w) \lrarrow w \in t\, ) \wedge t \in c\, )
							\end{gathered}
						\end{align}
						が得られる.
						
					\item[case(7-7)] $a$が$\Set{y}{\varphi(y)}$なる項で$b$が$\Set{z}{\psi(z)}$なる項で
						$c$が$\Set{x}{\xi(x)}$なる項であれば,$\widehat{\varphi}_{i}$は
						\begin{align}
							\forall u\, (\, \varphi(u) \lrarrow \psi(u)\, ) 
							&\rarrow (\, \exists s\, (\, \forall v\, (\, \varphi(v) \lrarrow v \in s\, ) \wedge \xi(s)\, ) \\
							&\rarrow \exists t\, (\, \forall w\, (\, \psi(w) \lrarrow w \in t\, ) \wedge \xi(t)\, )\, )
						\end{align}
						となるが,
						\begin{align}
							\tau &\defeq \varepsilon s\, (\, \forall v\, (\, \varphi(v) \lrarrow v \in s\, ) \wedge \xi(s)\, ), \\
							\omega &\defeq \varepsilon w \negation (\, \psi(w) \lrarrow w \in t\, )
						\end{align}
						とおけば,まず
						\begin{align}
							\forall u\, (\, \varphi(u) \lrarrow \psi(u)\, ),\ 
							\exists s\, (\, \forall v\, (\, \varphi(v) \lrarrow v \in s\, ) \wedge \xi(s)\, )
							&\provable{\mbox{{\bf HE}},\lang{\varepsilon}} \varphi(\omega) \lrarrow \psi(\omega), \\
							\forall u\, (\, \varphi(u) \lrarrow \psi(u)\, ),\ 
							\exists s\, (\, \forall v\, (\, \varphi(v) \lrarrow v \in s\, ) \wedge \xi(s)\, )
							&\provable{\mbox{{\bf HE}},\lang{\varepsilon}} \varphi(\omega) \lrarrow \omega \in \tau
						\end{align}
						が成り立つので
						\begin{align}
							\forall u\, (\, \varphi(u) \lrarrow \psi(u)\, ),\ 
							\exists s\, (\, \forall v\, (\, \varphi(v) \lrarrow v \in s\, ) \wedge \xi(s)\, )
							\provable{\mbox{{\bf HE}},\lang{\varepsilon}} \psi(\omega) \lrarrow \omega \in \tau
						\end{align}
						が従い
						\begin{align}
							\forall u\, (\, \varphi(u) \lrarrow \psi(u)\, ),\ 
							\exists s\, (\, \forall v\, (\, \varphi(v) \lrarrow v \in s\, ) \wedge \xi(s)\, )
							\provable{\mbox{{\bf HE}},\lang{\varepsilon}} \forall w\, (\, \psi(w) \lrarrow w \in \tau\, )
						\end{align}
						が出る.他方で
						\begin{align}
							\exists s\, (\, \forall v\, (\, \varphi(v) \lrarrow v \in s\, ) \wedge \xi(s)\, )
							\provable{\mbox{{\bf HE}},\lang{\varepsilon}} \xi(\tau), \\
						\end{align}
						が成り立つので,
						\begin{align}
							\begin{gathered}
								\forall u\, (\, \varphi(u) \lrarrow \psi(u)\, ),\ 
								\exists s\, (\, \forall v\, (\, \varphi(v) \lrarrow v \in s\, ) \wedge \xi(s)\, ) \\
								\provable{\mbox{{\bf HE}},\lang{\varepsilon}}
								\forall w\, (\, \psi(w) \lrarrow w \in \tau\, ) \wedge \xi(\tau)
							\end{gathered}
						\end{align}
						となり
						\begin{align}
							\begin{gathered}
								\forall u\, (\, \varphi(u) \lrarrow \psi(u)\, ),\ 
								\exists s\, (\, \forall v\, (\, \varphi(v) \lrarrow v \in s\, ) \wedge \xi(s)\, ) \\
								\provable{\mbox{{\bf HE}},\lang{\varepsilon}}
								\exists t\, (\, \forall w\, (\, \psi(w) \lrarrow w \in t\, ) \wedge \xi(t)\, )
							\end{gathered}
						\end{align}
						が得られる.
				\end{description}
				
			\item[case8] $\varphi_{i}$が相等性公理
				\begin{align}
					a = b \rarrow (\, c \in a \rarrow c \in b\, )
				\end{align}
				なる式である場合,
				\begin{description}
					\item[case(8-1)] $a$と$b$が主要$\varepsilon$項で$c$が$\Set{x}{\xi(x)}$なる項であれば,
						$\widehat{\varphi}_{i}$は
						\begin{align}
							a = b &\rarrow (\, \exists s\, (\, \forall u\, (\, \xi(u) \lrarrow u \in s\, ) \wedge s \in a\, ) \\
							&\rarrow (\, \exists t\, (\, \forall v\, (\, \xi(v) \lrarrow v \in t\, ) \wedge t \in b\, )\, )
						\end{align}
						となるが,ここで
						\begin{align}
							\sigma \defeq \varepsilon s \, (\, \forall u\, (\, \xi(u) \lrarrow u \in s\, ) \wedge s \in a\, )
						\end{align}
						とおけば
						\begin{align}
							\exists s \, (\, \forall u\, (\, \xi(u) \lrarrow u \in s\, ) \wedge s \in a\, )
							\provable{\mbox{{\bf HE}},\lang{\varepsilon}} \sigma \in a
						\end{align}
						が成り立つので,相等性公理より
						\begin{align}
							a = b,\ \exists s \, (\, \forall u\, (\, \xi(u) \lrarrow u \in s\, ) \wedge s \in a\, ),\ \Sigma
							\provable{\mbox{{\bf HE}},\lang{\varepsilon}} \sigma \in b
						\end{align}
						となる.他方で
						\begin{align}
							\exists s \, (\, \forall u\, (\, \xi(u) \lrarrow u \in s\, ) \wedge s \in a\, )
							\provable{\mbox{{\bf HE}},\lang{\varepsilon}} \forall u\, (\, \xi(u) \lrarrow u \in \sigma\, ) 
						\end{align}
						も成り立つので
						\begin{align}
							a = b,\ \exists s \, (\, \forall u\, (\, \xi(u) \lrarrow u \in s\, ) \wedge s \in a\, ),\ \Sigma
							\provable{\mbox{{\bf HE}},\lang{\varepsilon}} \forall u\, (\, \xi(u) \lrarrow u \in \sigma\, ) \wedge \sigma \in b
						\end{align}
						が従い,
						\begin{align}
							a = b,\ \exists s \, (\, \forall u\, (\, \xi(u) \lrarrow u \in s\, ) \wedge s \in a\, ),\ \Sigma
							\provable{\mbox{{\bf HE}},\lang{\varepsilon}} \exists t\, (\, \forall v\, (\, \xi(v) \lrarrow v \in t\, ) \wedge t \in b\, )
						\end{align}
						が出る.
						
					\item[case(8-2)] $a$と$c$が主要$\varepsilon$項で$b$が$\Set{z}{\psi(z)}$なる項であれば,
						$\widehat{\varphi}_{i}$は
						\begin{align}
							\forall u\, (\, u \in a \lrarrow \psi(u)\, ) 
							\rarrow (\, c \in a \rarrow \psi(c)\, )
						\end{align}
						となるが,これは
						\begin{align}
							\forall u\, (\, u \in a \lrarrow \psi(u)\, ) \vdash c \in a \rarrow \psi(c)
						\end{align}
						より直接得られる.
						
					\item[case(8-3)] $a$が主要$\varepsilon$項で$b$が$\Set{z}{\psi(z)}$なる項で
						$c$が$\Set{x}{\xi(x)}$なる項であれば,$\widehat{\varphi}_{i}$は
						\begin{align}
							\forall u\, (\, u \in a \lrarrow \psi(u)\, ) 
							&\rarrow (\, \exists s\, (\, \forall u\, (\, \xi(u) \lrarrow u \in s\, ) \wedge s \in a\, ) \\
							&\rarrow \exists t\, (\, \forall v\, (\, \xi(v) \lrarrow v \in t\, ) \wedge \psi(t)\, )\, )
						\end{align}
						となるが,
						\begin{align}
							\sigma \defeq \varepsilon s\, (\, \forall u\, (\, \xi(u) \lrarrow u \in s\, ) \wedge s \in a\, )
						\end{align}
						とおけば
						\begin{align}
							\exists s\, (\, \forall u\, (\, \xi(u) \lrarrow u \in s\, ) \wedge s \in a\, ) 
							&\provable{\mbox{{\bf HE}},\lang{\varepsilon}} \sigma \in a, \\
							\forall u\, (\, u \in a \lrarrow \psi(u)\, )
							&\provable{\mbox{{\bf HE}},\lang{\varepsilon}} \sigma \in a \rarrow \psi(\sigma)
						\end{align}
						より
						\begin{align}
							\forall u\, (\, u \in a \lrarrow \psi(u)\, ),\ 
							\exists s\, (\, \forall u\, (\, \xi(u) \lrarrow u \in s\, ) \wedge s \in a\, ) 
							\provable{\mbox{{\bf HE}},\lang{\varepsilon}} \psi(\sigma)
						\end{align}
						が成り立つ.他方で
						\begin{align}
							\exists s\, (\, \forall u\, (\, \xi(u) \lrarrow u \in s\, ) \wedge s \in a\, ) 
							\provable{\mbox{{\bf HE}},\lang{\varepsilon}} \forall u\, (\, \xi(u) \lrarrow u \in \sigma\, )
						\end{align}
						も成り立つので,
						\begin{align}
							\omega \defeq \varepsilon v \negation (\, \xi(v) \lrarrow v \in \sigma\, )
						\end{align}
						とおけば
						\begin{align}
							\exists s\, (\, \forall u\, (\, \xi(u) \lrarrow u \in s\, ) \wedge s \in a\, ) 
							\provable{\mbox{{\bf HE}},\lang{\varepsilon}} \xi(\omega) \lrarrow \omega \in \sigma
						\end{align}
						となり
						\begin{align}
							\exists s\, (\, \forall u\, (\, \xi(u) \lrarrow u \in s\, ) \wedge s \in a\, ) 
							\provable{\mbox{{\bf HE}},\lang{\varepsilon}} \forall v\, (\, \xi(v) \lrarrow v \in \sigma\, )
						\end{align}
						が従う.ゆえに
						\begin{align}
							\begin{gathered}
								\forall u\, (\, u \in a \lrarrow \psi(u)\, ),\ 
								\exists s\, (\, \forall u\, (\, \xi(u) \lrarrow u \in s\, ) \wedge s \in a\, ) \\
								\provable{\mbox{{\bf HE}},\lang{\varepsilon}} \forall v\, (\, \xi(v) \lrarrow v \in \sigma\, ) \wedge \psi(\sigma)
							\end{gathered}
						\end{align}
						が従い
						\begin{align}
							\begin{gathered}
								\forall u\, (\, u \in a \lrarrow \psi(u)\, ),\ 
								\exists s\, (\, \forall u\, (\, \xi(u) \lrarrow u \in s\, ) \wedge s \in a\, ) \\
								\provable{\mbox{{\bf HE}},\lang{\varepsilon}} \exists t\, (\, \forall v\, (\, \xi(v) \lrarrow v \in t\, ) \wedge \psi(t)\, )
							\end{gathered}
						\end{align}
						が出る.
						
					\item[case(8-4)] $a$が$\Set{y}{\varphi(y)}$なる項で$b$と$z$が主要$\varepsilon$項の場合,
						$\widehat{\varphi}_{i}$は
						\begin{align}
							\forall u\, (\, \varphi(u) \lrarrow u \in b\, ) 
							\rarrow (\, \varphi(c) \rarrow c \in b\, )
						\end{align}
						となるが,これは
						\begin{align}
							\forall u\, (\, \varphi(u) \lrarrow u \in b\, ) 
							\provable{\mbox{{\bf HE}},\lang{\varepsilon}} 
							\varphi(c) \rarrow c \in b
						\end{align}
						から直接得られる.
						
					\item[case(8-5)] $a$が$\Set{y}{\varphi(y)}$なる項で$b$が主要$\varepsilon$項で
						$z$が$\Set{x}{\xi(x)}$なる項の場合,$\widehat{\varphi}_{i}$は
						\begin{align}
							\forall u\, (\, \varphi(u) \lrarrow u \in b\, ) 
							&\rarrow (\, \exists s\, (\, \forall u\, (\, \xi(u) \lrarrow u \in s\, ) \wedge \varphi(s)\, ) \\
							&\rarrow \exists t\, (\, \forall v\, (\, \xi(v) \lrarrow v \in t\, ) \wedge t \in b\, )\, )
						\end{align}
						となるが,
						\begin{align}
							\sigma \defeq \varepsilon s\, (\, \forall u\, (\, \xi(u) \lrarrow u \in s\, ) \wedge \varphi(s)\, )
						\end{align}
						とおけば,
						\begin{align}
							\exists s\, (\, \forall u\, (\, \xi(u) \lrarrow u \in s\, ) \wedge \varphi(s)\, ) 
							&\provable{\mbox{{\bf HE}},\lang{\varepsilon}} \varphi(\sigma), \\
							\forall u\, (\, \varphi(u) \lrarrow u \in b\, )
							&\provable{\mbox{{\bf HE}},\lang{\varepsilon}} \varphi(\sigma) \rarrow \sigma \in b
						\end{align}
						より
						\begin{align}
							\forall u\, (\, \varphi(u) \lrarrow u \in b\, ),\ 
							\exists s\, (\, \forall u\, (\, \xi(u) \lrarrow u \in s\, ) \wedge \varphi(s)\, ) 
							\provable{\mbox{{\bf HE}},\lang{\varepsilon}} \sigma \in b
						\end{align}
						が成り立つ.他方で
						\begin{align}
							\exists s\, (\, \forall u\, (\, \xi(u) \lrarrow u \in s\, ) \wedge \varphi(s)\, ) 
							\provable{\mbox{{\bf HE}},\lang{\varepsilon}} \forall u\, (\, \xi(u) \lrarrow u \in \sigma\, )
						\end{align}
						も成り立つので,
						\begin{align}
							\omega \defeq \varepsilon v \negation (\, \xi(v) \lrarrow v \in \sigma\, )
						\end{align}
						とおけば
						\begin{align}
							\exists s\, (\, \forall u\, (\, \xi(u) \lrarrow u \in s\, ) \wedge \varphi(s)\, ) 
							\provable{\mbox{{\bf HE}},\lang{\varepsilon}} \xi(\omega) \lrarrow \omega \in \sigma
						\end{align}
						となり
						\begin{align}
							\exists s\, (\, \forall u\, (\, \xi(u) \lrarrow u \in s\, ) \wedge \varphi(s)\, ) 
							\provable{\mbox{{\bf HE}},\lang{\varepsilon}} \forall v\, (\, \xi(v) \lrarrow v \in \sigma\, )
						\end{align}
						が従う.ゆえに
						\begin{align}
							\begin{gathered}
								\forall u\, (\, \varphi(u) \lrarrow u \in b\, ),\ 
								\exists s\, (\, \forall u\, (\, \xi(u) \lrarrow u \in s\, ) \wedge \varphi(s)\, ) \\
								\provable{\mbox{{\bf HE}},\lang{\varepsilon}} \forall v\, (\, \xi(v) \lrarrow v \in \sigma\, ) \wedge \sigma \in b
							\end{gathered}
						\end{align}
						が従い
						\begin{align}
							\begin{gathered}
								\forall u\, (\, \varphi(u) \lrarrow u \in b\, ),\ 
								\exists s\, (\, \forall u\, (\, \xi(u) \lrarrow u \in s\, ) \wedge \varphi(s)\, ) \\
								\provable{\mbox{{\bf HE}},\lang{\varepsilon}} \exists t\, (\, \forall v\, (\, \xi(v) \lrarrow v \in t\, ) \wedge t \in b\, )
							\end{gathered}
						\end{align}
						が出る.
						
					\item[case(8-6)] $a$が$\Set{y}{\varphi(y)}$なる項で$b$が$\Set{z}{\psi(z)}$なる項で
						$c$が主要$\varepsilon$項であれば,$\widehat{\varphi}_{i}$は
						\begin{align}
							\forall u\, (\, \varphi(u) \lrarrow \psi(u)\, ) \rarrow (\, \varphi(c) \rarrow \psi(c)\, )
						\end{align}
						となるが,これは
						\begin{align}
							\forall u\, (\, \varphi(u) \lrarrow \psi(u)\, ) 
							\provable{\mbox{{\bf HE}},\lang{\varepsilon}} 
							\varphi(c) \rarrow \psi(c)
						\end{align}
						から直接得られる.
						
					\item[case(8-7)] $a$が$\Set{y}{\varphi(y)}$なる項で$b$が$\Set{z}{\psi(z)}$なる項で
						$c$が$\Set{x}{\xi(x)}$なる項であれば,$\widehat{\varphi}_{i}$は
						\begin{align}
							\forall u\, (\, \varphi(u) \lrarrow u \in b\, ) 
							&\rarrow (\, \exists s\, (\, \forall u\, (\, \xi(u) \lrarrow u \in s\, ) \wedge \varphi(s)\, ) \\
							&\rarrow \exists t\, (\, \forall v\, (\, \xi(v) \lrarrow v \in t\, ) \wedge \psi(t)\, )\, )
						\end{align}
						となるが,
						\begin{align}
							\sigma \defeq \varepsilon s\, (\, \forall u\, (\, \xi(u) \lrarrow u \in s\, ) \wedge \varphi(s)\, )
						\end{align}
						とおけば,
						\begin{align}
							\exists s\, (\, \forall u\, (\, \xi(u) \lrarrow u \in s\, ) \wedge \varphi(s)\, ) 
							&\provable{\mbox{{\bf HE}},\lang{\varepsilon}} \varphi(\sigma), \\
							\forall u\, (\, \varphi(u) \lrarrow u \in b\, )
							&\provable{\mbox{{\bf HE}},\lang{\varepsilon}} \varphi(\sigma) \rarrow \psi(\sigma)
						\end{align}
						より
						\begin{align}
							\forall u\, (\, \varphi(u) \lrarrow u \in b\, ),\ 
							\exists s\, (\, \forall u\, (\, \xi(u) \lrarrow u \in s\, ) \wedge \varphi(s)\, ) 
							\provable{\mbox{{\bf HE}},\lang{\varepsilon}} \psi(\sigma)
						\end{align}
						が成り立つ.他方で
						\begin{align}
							\exists s\, (\, \forall u\, (\, \xi(u) \lrarrow u \in s\, ) \wedge \varphi(s)\, ) 
							\provable{\mbox{{\bf HE}},\lang{\varepsilon}} \forall u\, (\, \xi(u) \lrarrow u \in \sigma\, )
						\end{align}
						も成り立つので,
						\begin{align}
							\omega \defeq \varepsilon v \negation (\, \xi(v) \lrarrow v \in \sigma\, )
						\end{align}
						とおけば
						\begin{align}
							\exists s\, (\, \forall u\, (\, \xi(u) \lrarrow u \in s\, ) \wedge \varphi(s)\, ) 
							\provable{\mbox{{\bf HE}},\lang{\varepsilon}} \xi(\omega) \lrarrow \omega \in \sigma
						\end{align}
						となり
						\begin{align}
							\exists s\, (\, \forall u\, (\, \xi(u) \lrarrow u \in s\, ) \wedge \varphi(s)\, ) 
							\provable{\mbox{{\bf HE}},\lang{\varepsilon}} \forall v\, (\, \xi(v) \lrarrow v \in \sigma\, )
						\end{align}
						が従う.ゆえに
						\begin{align}
							\begin{gathered}
								\forall u\, (\, \varphi(u) \lrarrow u \in b\, ),\ 
								\exists s\, (\, \forall u\, (\, \xi(u) \lrarrow u \in s\, ) \wedge \varphi(s)\, ) \\
								\provable{\mbox{{\bf HE}},\lang{\varepsilon}} \forall v\, (\, \xi(v) \lrarrow v \in \sigma\, ) \wedge \psi(\sigma)
							\end{gathered}
						\end{align}
						が従い
						\begin{align}
							\begin{gathered}
								\forall u\, (\, \varphi(u) \lrarrow u \in b\, ),\ 
								\exists s\, (\, \forall u\, (\, \xi(u) \lrarrow u \in s\, ) \wedge \varphi(s)\, ) \\
								\provable{\mbox{{\bf HE}},\lang{\varepsilon}} \exists t\, (\, \forall v\, (\, \xi(v) \lrarrow v \in t\, ) \wedge \psi(t)\, )
							\end{gathered}
						\end{align}
						が出る.
						\QED
				\end{description}
		\end{description}
	\end{metaprf}
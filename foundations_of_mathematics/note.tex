\documentclass[a4j,10.5pt,oneside,openany]{jsbook}
%
\usepackage{amsmath,amssymb}
\usepackage{amsthm}
\usepackage{makeidx}
\makeindex
\usepackage{newpxmath,newpxtext}
\usepackage{mathrsfs} %花文字
\usepackage{mathtools} %参照式のみ式番号表示
\usepackage{latexsym} %qed
\usepackage{ascmac}
\usepackage{bussproofs} %証明図
\usepackage{centernot} %\centernot\arrow
\usepackage[dvipdfmx]{graphicx}
\usepackage{tikz} %描画
\usepackage{color}
\usepackage{relsize}
\usepackage{comment}
\usepackage{url}
\usepackage{ulem} %訂正線
\usepackage[dvipdfm,colorlinks=true,linkcolor=blue,filecolor=blue,urlcolor=blue]{hyperref} %文書内リンク
\usepackage{pxjahyper} %%hyperref読み込みの直後に
\setcounter{tocdepth}{3} %table of contents subsection表示
\newtheoremstyle{mystyle}% % Name
	{20pt}%                      % Space above
	{20pt}%                      % Space below
	{\rm}%           % Body font
	{}%                      % Indent amount
	{\gt}%             % Theorem head font
	{.}%                      % Punctuation after theorem head
	{10pt}%                     % Space after theorem head, ' ', or \newline
	{}%                      % Theorem head spec (can be left empty, meaning `normal')
\theoremstyle{mystyle}

\allowdisplaybreaks[1]
\newcommand{\bhline}[1]{\noalign {\hrule height #1}} %表の罫線を太くする.
\newcommand{\bvline}[1]{\vrule width #1} %表の罫線を太くする.
\newcommand{\QED}{% %証明終了
	\relax\ifmmode
		\eqno{%
		\setlength{\fboxsep}{2pt}\setlength{\fboxrule}{0.3pt}
		\fcolorbox{black}{black}{\rule[2pt]{0pt}{1ex}}}
	\else
		\begingroup
		\setlength{\fboxsep}{2pt}\setlength{\fboxrule}{0.3pt}
		\hfill\fcolorbox{black}{black}{\rule[2pt]{0pt}{1ex}}
		\endgroup
	\fi}

\definecolor{DarkMidnightBlue}{rgb}{0.0, 0.2, 0.4}
\definecolor{PakistanGreen}{rgb}{0.0, 0.4, 0.0}
\definecolor{Mahogany}{rgb}{0.65,0.10,0.10}
\definecolor{darkgray}{rgb}{0.21, 0.21, 0.21}
\definecolor{CarolinaBlue}{rgb}{0.6, 0.73, 0.89}

\newtheorem{thm}{\color{DarkMidnightBlue}{定理}}[section]
\newtheorem{dfn}[thm]{\color{PakistanGreen}{定義}}
\newtheorem{axm}[thm]{\color{Mahogany}{公理}}
\newtheorem{schema}[thm]{{公理図式}}
\newtheorem{logicalaxm}[thm]{\color{Mahogany}{推論規則}}
\newtheorem{logicalthm}[thm]{\color{DarkMidnightBlue}{推論法則}}
\newtheorem{metadfn}[thm]{\color{PakistanGreen}{メタ定義}}
\newtheorem{metaaxm}[thm]{\color{Mahogany}{メタ公理}}
\newtheorem{metathm}[thm]{\color{DarkMidnightBlue}{メタ定理}}
\newtheorem{prp}[thm]{命題}
\newtheorem{cor}[thm]{系}
\newtheorem{lem}[thm]{補題}
\newtheorem*{prf}{証明}
\newtheorem*{metaprf}{メタ証明}
\newtheorem*{sketch}{略証}
\newtheorem{rem}[thm]{注意}
\newtheorem{e.g.}[thm]{例}
\newcommand{\defunc}{\mbox{1}\hspace{-0.25em}\mbox{l}} %定義関数
\newcommand*{\sgn}[1]{\operatorname{sgn}\left( #1 \right)} %signal関数
\newcommand{\monologue}[1]{
	{\color{CarolinaBlue}\hspace{-10.5pt}\mask{\hspace{21pt}\vbox{
		\hsize 445pt
		\normalcolor{\vskip 7pt \noindent #1 \vskip 7pt}
	}\hspace{21pt}}{E}}
}

\def\Ddot#1{$\ddot{\mathrm{#1}}$} %文中ddot

%論理
\newcommand{\lang}[1]{\mathcal{L}_{\scalebox{1.2}{$#1$}}} %言語
\newcommand{\Set}[2]{\left\{\, #1 \mid #2\, \right\}} %論理式の対象化
\newcommand{\defeq}{\overset{\mathrm{def}}{=\joinrel=}} %\scalebox{3}[1]{=}}} %定義記号=(=\joinrel=も使える)
\newcommand{\defarrow}{\overset{\mathrm{def}}{\Longleftrightarrow}} %定義記号⇔
\newcommand{\provable}[1]{\vdash_{{\scriptsize #1}}} %証明可能
\newcommand{\rarrow}{\rightarrow} %右矢印
\newcommand{\lrarrow}{\leftrightarrow} %左右矢印

%集合
\newcommand{\EXTAX}{\mbox{{\bf EXT}}} %外延性公理
\newcommand{\EQAX}{\mbox{{\bf EQ}}} %相等性公理
\newcommand{\COMAX}{\mbox{\bf COM}} %内包性公理
\newcommand{\ELEAX}{\mbox{{\bf ELE}}} %要素の公理
\newcommand{\REPAX}{\mbox{{\bf REP}}} %置換公理
\newcommand{\POWAX}{\mbox{{\bf POW}}} %冪集合公理
\newcommand{\PAIAX}{\mbox{{\bf PAI}}} %対集合公理
\newcommand{\INFAX}{\mbox{{\bf INF}}} %無限公理
\newcommand{\REGAX}{\mbox{{\bf REG}}} %正則性公理
\newcommand{\CAX}{\mbox{{\bf C}}} %選択公理

\newcommand{\Univ}{\mathbf{V}} %宇宙
\newcommand{\set}[1]{\operatorname*{set} (#1)} %集合であることの論理式
\newcommand{\power}[1]{\operatorname*{P} (#1)} %冪集合
\newcommand{\rel}[1]{\operatorname*{rel} (#1)} %関係
\newcommand{\dom}[1]{\operatorname*{dom} (#1)} %類の定義域
\newcommand{\ran}[1]{\operatorname*{ran} (#1)} %類の値域
\newcommand{\sing}[1]{\operatorname*{sing} (#1)} %single-valuedの定義式
\newcommand{\fnc}[1]{\operatorname*{fnc} (#1)} %写像の定義式
\newcommand{\fon}{\operatorname*{:on}} %〇上の写像
\newcommand{\inj}{\overset{\mathrm{1:1}}{\longrightarrow}} %単射
\newcommand{\srj}{\overset{\mathrm{onto}}{\longrightarrow}} %全射
\newcommand{\bij}{\underset{\mathrm{onto}}{\overset{\mathrm{1:1}}{\longrightarrow}}} %全単射
\newcommand{\inv}[1]{{#1}^{-1}} %^{\operatorname{inv}}} %集合の反転
\newcommand{\rest}[2]{#1\hspace{-0.25em}\upharpoonright\hspace{-0.25em}{#2}} %制限写像
\newcommand{\tran}[1]{\operatorname*{tran} \left(#1\right)} %推移的類の定義式
\newcommand{\ord}[1]{\operatorname*{ord} \left(#1\right)} %順序数の定義式
\newcommand{\ON}{\mathrm{ON}} %順序数全体
\newcommand{\limo}[1]{\mathrm{lim.o}\left(#1\right)} %極限数の式
%\newcommand{\Natural}{{\boldsymbol \omega}} %自然数全体
\newcommand{\Natural}{\mathbf{N}} %自然数全体
%
%
\setlength{\textwidth}{\fullwidth}
\setlength{\textheight}{40\baselineskip}
\addtolength{\textheight}{\topskip}
%\setlength{\voffset}{-0.55in}
%
%
\title{$\varepsilon$計算とクラスの導入による具体的で直観的な集合論の構築}
\author{関根深澤研修士二年百合川尚学}
\date{\today}

\begin{document}
\mathtoolsset{showonlyrefs = true}
\maketitle
\tableofcontents
\frontmatter
\mainmatter

\begin{comment}
数学について注意深く考え込んでいると,うっかりとんでもない落とし穴にはまってしまうかもしれません.
そのとき,きっと次の事柄に悩まされます.
前提がわからない.明らかなものと明らかでないものとの線引きがわからない.
日本語をどこまで信用してよいのかもわからない.
突き詰めると何も見えなくなる.数学の立脚地は永遠に届かない...
このノートがそういった受難を乗り越える役に立てますように.

\begin{description}
	\item[前提その一] はじめに素朴な数字の概念は持っている.それによってモノを数えることもできる.
		モノの数え方は慣習どおり.
		
	\item[前提その二] 当たり前のことが当たり前であるためには,言葉でそれを保証しなければならない.
	
	\item[] $ZF$集合論では存在という言葉が具体的な意味を持っていない.
		存在したらこうなるであろうという推論規則によってしか存在という概念を表現し得ない.
		$\varepsilon$項は集合である.それも,存在したら``取れる''集合である.
		$\varepsilon$項によって集合を具体的に扱える.
		さらに内包項によって閉じた世界を作ることが出来る.
		$ZF$集合論では集合の宇宙は閉じていないので,存在したらそれに名前を付けて言語を保存拡大するという手法を取る.
		内包項を導入すれば,言語を拡大する必要はなくなる.
		公理とは,既に作られた世界においてどれが集合でどれが集合でないかを選り分ける用に使われる.
\end{description}
\end{comment}

%\section{徒然なるままに支離滅裂}
わからないわからないわからない

基礎論における証明は大抵が直感に頼っているように見えますが,ではその直感が正しいとは誰が保証するのでしょうか.
手元にあるどの本でも保証されていません.もしかしたら神様という超然的な存在を暗黙の裡に認めていて,
直感とは神様が用意した論理であるとして無断で使っているだけなのかもしれませんが,
残念ながら読者はテレパシーを使えないので,筆者の暗黙の了解を推察するなんて困難です.

しかしながら,暗黙の了解を排除しようとすると,その分だけ日本語による明示的な約束が必要になります.
すると新たな問題が生じます.それは日本語で書かれた言明をどこまで信用するか,という問題です.
基礎論の難しさは,その表面上のややこしさよりも日本語に対する認識を揃えることにあるのでしょうか.

論理構造を集合論の結果を用いて解明しようというのならまだしも(こちらは数理論理学と呼ばれる分野で,本来は数学基礎論とは別物だそうです),
集合論を構築することが目的である場合,その土台となる基礎論を集合論の上に展開すると理論が循環することになるでしょう.
基礎論が基礎にしている集合論は「メタ理論」と呼ばれるらしいですが,
その「メタ理論」がどう構成されたのかという点には誰も全く言及していないのですから,
「メタ理論」という言葉は単なる逃げ口上にしか聞こえず,理論の循環を解消できません.
私の考えでは,メタ理論の代わりに絶対的な原理が与えられたとして数学を構築すれば良いのです.
まあ言い方を変えて印象を良く?しようというだけの下らない事情であって,
もったいぶって思想的な立場を主張しても集合論には関係のないことなのですが.

前提:我々は数の概念を持っている.個数の概念を持っている.物の数を数えることが出来る.
数の概念とは?個数の概念とは?
ここで言う数は数学的に構成する数ではなくて,神が用意した概念としての数.
そこまで踏み込むときりがない.

排中律と無矛盾性の違い:
排中律から$\rightharpoondown (A \wedge \rightharpoondown A)$が導かれるが,
$A \wedge \rightharpoondown A$が導かれることを否定しているわけではない.

目的:いかに自然で人工的な世界を作るか.
\chapter{言語$\mathcal{L}_{\in}$}

	本稿の世界を展開するために使用する言語は二つ種類がある.
	一つは自然言語の日本語であり,もう一つは記号のみで作られた人工的な言語である.
	その人工的な言語は記号列が数学の式となるための文法を指定し,
	そこで組み立てられた式のみが考察対象となる.
	日本語は式を解釈したり人工言語を補助するために使われる.
	
	まず,人工的な言語である$\mathcal{L}_{\in}$を設定する.
	以下は$\mathcal{L}_{\in}$を構成する要素である:
	\begin{description}
		\item[矛盾記号] $\bot$
		\item[論理記号] $\rightharpoondown$, $\vee$, $\wedge$, $\Longrightarrow$
		\item[量化子] $\forall$, $\exists$
		\item[述語記号] $=$, $\in$
		\item[使用文字] ローマ字及びギリシア文字.
		\item[接項子] $\natural$
	\end{description}
	
	日本語と同様に,決められた規則に従って並ぶ記号列のみを$\mathcal{L}_{\in}$の単語や文章として扱う.
	$\mathcal{L}_{\in}$において,名詞にあたるものは{\bf 項}\index{こう@項}{\bf (term)}と呼ばれる.
	文字は最もよく使われる項である.述語とは項同士を結ぶものであり,最小単位の文章を形成する.例えば
	\begin{align}
		\in st
	\end{align}
	は$\mathcal{L}_{\in}$の文章となり,日本語には``$s$は$t$の要素である''と翻訳される.
	$\mathcal{L}_{\in}$の文章を{\bf 式}\index{しき@式}{\bf (formula)}或いは
	{\bf 論理式}\index{ろんりしき@論理式}と呼ぶ.論理記号は主に式同士を繋ぐ役割を持つ.
	
	論理学的な言語とは論理記号と変項記号を除く記号をすべて集めたものである.
	本稿で用意した記号で言うと,論理記号とは
	\begin{align}
		\bot,\ \rightharpoondown,\ \vee,\ \wedge,\ \Longrightarrow,\ \forall,\ \exists,\ =
	\end{align}
	であり,変項記号とは文字であって,$\mathcal{L}_{\in}$の語彙は
	\begin{align}
		\in,\ \natural
	\end{align}
	しかない.だが本稿の目的は集合論の構築であって一般の言語について考察するわけではないので,
	論理記号も文字もすべて$\mathcal{L}_{\in}$の一員と見做す方が自然である.
	ついでに記号の分類も主流の論理学とは変えていて,
	\begin{itemize}
		\item $\bot$はそれ単体で式であるので他の記号とは分ける.
		\item 論理記号とは式に作用するものとして$\rightharpoondown,\vee,\wedge,\Longrightarrow$のみとする.
		\item $\forall$と$\exists$は項に作用するものであるから量化子として分類する.
		\item 等号$=$は'等しい'という述語になっているから,論理記号ではなく述語記号に入れる.
	\end{itemize}
	以上の変更点は殆ど無意味であるが,
	いかに``直観的な''集合論を構築するかという目的を勘案すれば良いスタートであるように思える.
	
\section{項}
	
	文字は項として使われるが,文字だけを項とするのは不十分であり,
	例えば$1000$個の相異なる項が必要であるといった場合には異体字まで駆使しても不足する.
	そこで,文字$x$に対して
	\begin{align}
		\natural x
	\end{align}
	もまた項であると約束する.
	また,$\tau$を項とするときに
	\begin{align}
		\natural \tau
	\end{align}
	も項であると約束する.この約束に従えば,文字$x$だけを用いたとしても
	\begin{align}
		x,\quad \natural x, \quad \natural \natural x, \quad \natural \natural \natural x
	\end{align}
	はいずれも項ということになる.極端なことを言えば,「$1000$個の項を用意してくれ」と頼まれたとしても
	$\natural$と$x$だけで$1000$個の項を作り出すことが可能なのだ.
	
	大切なのは,$\natural$を用いれば理屈の上では項に不足しないということであって,
	具体的な数式を扱うときに$\natural$が出てくるかと言えば否である.
	$\natural$が必要になるほどに長い式を読解するのは困難であるから,
	通常は何らかの略記法を導入して複雑なところを覆い隠してしまう.
	
	\begin{itembox}[l]{超記号}
		上で「$\tau$を項とするときに」と書いたが,これは一時的に
		$\tau$を或る項に代用しているだけであって,
		$\tau$が指している項の本来の字面は$x$であるかもしれない.
		この場合の$\tau$を{\bf 超記号}\index{ちょうきごう@超記号}と呼ぶ.
		「$A$を式とする」など式にも超記号が宣言される.
	\end{itembox}
	
	項は形式的には次のよう定義される:
	
	\begin{description}
		\item[項]
			\begin{itemize}
				\item 文字は項である.
				\item $\tau$を項とするとき,$\natural \tau$は項である.
				\item 以上のみが項である.
			\end{itemize}
	\end{description}
	
	上の定義では,はじめに発端を決めて,次に新しい項を作り出す手段を指定している.こういった定義の仕方を
	{\bf 帰納的定義}\index{きのうてきていぎ@帰納的定義}{\bf (inductive definition)}と呼ぶ.
	ただしそれだけでは項の範囲が定まらないので,最後に「以上のみが項である」と加えている.
	
	「以上のみが項である」という約束によって,例えば「$\tau$が項である」という言明が与えられたとき,この言明が
	``$\tau$は或る文字に代用されている''か
	``項$\sigma$が取れて(超記号),$\tau$は$\natural \sigma$に代用されている''
	のどちらか一方にしか解釈され得ないのは,言うまでもない,であろうか.直感的にはそうであっても
	直感を万人が共有している保証はないから,やはりここは明示的に,「$\tau$が項である」という言明の解釈は
	\begin{itemize}
		\item $\tau$は或る文字に代用されている
		\item 項$\sigma$が取れて(超記号),$\tau$は$\natural \sigma$に代用されている
	\end{itemize}
	に限られると決めてしまおう.主張はストレートな方が後々使いやすい.
	
	\begin{itembox}[l]{暗に宣言された超記号}
		上で「項$x$が取れて」と書いたが,この$x$は唐突に出てきたので,
		それが表す文字そのものでしかないのか,或いは超記号であるのか,一見判然しない.
		本来は「項,これを$x$で表す,が取れて」などと書くのが
		良いのかもしれないが,はじめの書き方でも文脈上は超記号として解釈するのが自然であるし,
		何より言い方がまどろこくない.このように見た目の簡潔さのために超記号の宣言を省略する場合もある.
	\end{itembox}
	
\section{式}
	式も項と同様に帰納的に定義される:
	
	\begin{description}
		\item[式]
			\begin{itemize}
				\item $\bot$は式である.
				\item $\sigma$と$\tau$を項とするとき,$\in st$と$=st$は式である.
					これを{\bf 原子式}\index{げんししき@原子式}{\bf (atomic formula)}と呼ぶ.
				\item $\varphi$を式とするとき,$\rightharpoondown \varphi$は式である.
				\item $\varphi$と$\psi$を式とするとき,$\vee \varphi \psi,\ 
					\wedge \varphi \psi,\ \Longrightarrow \varphi \psi$はいずれも式である.
			
				\item $x$を項とし,$\varphi$を式とするとき,$\forall x \varphi$と$\exists x \varphi$は式である.
				
				\item 以上のみが式である.
			\end{itemize}
	\end{description}
	
	例えば「$\varphi$が式である」という言明の解釈は,
	\begin{itemize}
		\item $\varphi$は$\bot$である
		\item 項$s$と項$t$が得られて,$\varphi$は$\in s t$である
		\item 項$s$と項$t$が得られて,$\varphi$は$= s t$である
		\item 式$\psi$が得られて,$\varphi$は$\rightharpoondown \psi$である
		\item 式$\psi$と式$\xi$が得られて,$\varphi$は$\vee \psi \xi$である
		\item 式$\psi$と式$\xi$が得られて,$\varphi$は$\wedge \psi \xi$である
		\item 式$\psi$と式$\xi$が得られて,$\varphi$は$\Longrightarrow \psi \xi$である
		\item 項$x$と式$\psi$が得られて,$\varphi$は$\forall x \psi$である
		\item 項$x$と式$\psi$が得られて,$\varphi$は$\exists x \psi$である
	\end{itemize}
	に限られる.
	
\chapter{量化}
	例えば
	\begin{align}
		\forall x \in x y
	\end{align}
	なる式を考える.中置記法(後述)で
	\begin{align}
		\forall x\, (\, x \in y\, )
	\end{align}
	と書けば若干見やすくなるであろうか.冠頭詞$\forall$は直後の$x$に係って「任意の$x$に対し...」の意味を持ち,
	この式は「任意の$x$に対して$x$は$y$の要素である」と読むのであるが,
	このとき$x$は$\forall x \in x y$で{\bf 束縛されている}{\bf (bound)}や
	或いは{\bf 量化されている}{\bf (quantified)}と言う.
	$\forall$が$\exists$に代わっても,今度は``$x$は$\exists x \in x y$で束縛されている''と言う.
	まあつまり,{\bf 量化子の直後に続く項(量化子が係っている項)は,その量化子から始まる式の中で束縛されている}
	と解釈することになっているのだ.
	
	では
	\begin{align}
		\Longrightarrow \forall x \in x y \in x z
	\end{align}
	という式はどうであるか.$\forall x$の後ろには$x$が二か所に現れているが,
	どちらの$x$も$\forall$によって束縛されているのであろうか?
	結論を言えば$\in x y$の$x$は束縛されていて,$\in x z$の$x$は束縛されていない.
	というのも式の構成法を思い返せば,$\forall x \varphi$が式であると言ったら$\varphi$は式であるはずで,
	今の例で$\forall x$に後続する式は
	\begin{align}
		\in x y
	\end{align}
	しかないのだから,$\forall$から始まる式は
	\begin{align}
		\forall x \in x y
	\end{align}
	しかないのである.$\forall$が係る$x$が束縛されている範囲は
	``$\forall$から始まる式''であるので,$\in x z$の$x$とは
	量化子$\forall$による``束縛''から漏れた``自由な''$x$ということになる.
	
	上の例でみたように,量化はその範囲が重要になる.
	量化子$\forall$が式$\varphi$に現れたとき,
	その$\forall$から始まる$\varphi$の部分式を
	$\forall$の{\bf スコープ}と呼ぶが,
	いつでもスコープが取れることは明白であるとして,
	$\forall$のスコープは唯一つでないと都合が悪いだろう.
	もしも異なるスコープが存在したら,同じ式なのに全く違う解釈に分かれてしまうのだから.
	実際そのような心配は無用であると後で保証するわけだが,
	その準備として{\bf 始切片}という概念について取り掛かる.
	
	ではさらにグレードアップさせて,
	\begin{align}
		\forall x \Longrightarrow \forall x \in x y \in x z
	\end{align}
	なる式における量化はどうであろうか.
	
\section{部分項と部分式}
	\begin{description}
		\item[部分項]
			項から切り取ったひとつづきの部分列で,それ自体が項であるものを元の項に対して
			{\bf 部分項}\index{ぶぶんこう@部分項}{\bf (sub term)}と呼ぶ.
			元の項全体も部分項と捉えるが,自分自身を除く部分項を特に
			{\bf 真部分項}\index{しんぶぶんこう@真部分項}{\bf (proper sub term)}と呼ぶ.
			例えば,文字$x$の部分項は$x$自身のみであって,また$\tau$を項とすると$\tau$は$\natural \tau$の部分項である.
		
		\item[部分式]
			式から切り取ったひとつづきの部分列で,それ自体が式であるものを元の式に対して
			{\bf 部分式}\index{ぶぶんしき@部分式}{\bf (sub formula)}と呼ぶ.
			例えば$\varphi$と$\psi$を式とするとき,$\varphi$と$\psi$は$\vee \varphi \psi$の部分式である.
			元の式全体も部分式と捉えるが,自分自身を除く部分式を特に
			{\bf 真部分式}\index{しんぶぶんしき@真部分式}{\bf (proper sub formula)}と呼ぶ.
	\end{description}
	
\section{始切片}
	$\varphi$を$\mathcal{L}_{\in}$の式とするとき,$\varphi$の左端から切り取るひとつづきの部分列を
	$\varphi$の{\bf 始切片}\index{しせっぺん@始切片}{\bf (initial segment)}と呼ぶ.
	例えば$\varphi$が
	\begin{align}
		\Longrightarrow \forall x \wedge \Longrightarrow \in xy \in xz \Longrightarrow \in xz \in xy = yz
	\end{align}
	である場合,
	\begin{align}
		\textcolor{red}{\Longrightarrow \forall x \wedge \Longrightarrow \in xy \in xz \Longrightarrow \in xz \in x}y = yz
	\end{align}
	や
	\begin{align}
		\textcolor{red}{\Longrightarrow \forall x \wedge \Longrightarrow \in xy} \in xz \Longrightarrow \in xz \in xy = yz
	\end{align}
	など赤字で分けられた部分は$\varphi$の始切片である.また$\varphi$自身も$\varphi$の始切片である.
	
	項についても同様に,項の左端から切り取るひとつづきの部分列をその項の始切片と呼ぶ.
	
	本節の主題は次である.
	\begin{screen}
		\begin{metathm}[始切片の一意性]\label{metathm:initial_segment_L_in}
			$\tau$を$\mathcal{L}_{\in}$の項とするとき,$\tau$の始切片で$\mathcal{L}_{\in}$の項であるものは$\tau$自身に限られる.
			また$\varphi$を$\mathcal{L}_{\in}$の式とするとき,$\varphi$の始切片で$\mathcal{L}_{\in}$の式であるものは$\varphi$自身に限られる.
		\end{metathm}
	\end{screen}
	
	「項の始切片で項であるものはその項自身に限られる.また,式の始切片で式であるものはその式自身に限られる.」という言明を(★)と書くことにする.
	このメタ定理を示すには次の原理を用いる:
	
	\begin{screen}
		\begin{metaaxm}[$\mathcal{L}_{\in}$の項に対する構造的帰納法]
			$\mathcal{L}_{\in}$の項に対する言明Xに対し(Xとは,例えば上の(★)),
			\begin{itemize}
				\item 文字に対してXが言える.
				\item 無作為に与えられた項が与えられたとき,その全ての真部分項に対してXが言えるならば,
					その項に対してもXが言える.
			\end{itemize}
			ならば,いかなる項に対してもXが言える.
		\end{metaaxm}
	\end{screen}
	
	\begin{screen}
		\begin{metaaxm}[$\mathcal{L}_{\in}$の式に対する構造的帰納法]
			$\mathcal{L}_{\in}$の式に対する言明Xに対し(Xとは,例えば上の(★)),
			\begin{itemize}
				\item $\bot$に対してXが言える.
				\item 原子式に対してXが言える.
				\item 無作為に与えられた式が与えられたとき,その全ての真部分式に対してXが言えるならば,
					その式に対してもXが言える.
			\end{itemize}
			ならば,いかなる式に対してもXが言える.
		\end{metaaxm}
	\end{screen}
	
	では定理を示す.
	
	\begin{metaprf}\mbox{}
		\begin{description}
			\item[項について]
				$s$を項とするとき,$s$が文字ならば$s$の始切片は$s$のみである.つまり(★)が言える.
				$s$が文字でないならば,$s$の全ての真部分項に対して(★)が言えると仮定する.
				(項の構成法より)項$t$が取れて$s$は
				\begin{align}
					\natural t
				\end{align}
				と表せる.$u$を$s$の始切片で項であるものとすると
				$u$に対しても(項の構成法より)項$v$が取れて,$u$は
				\begin{align}
					\natural v
				\end{align}
				と表せる.このとき$v$は$t$の始切片であり,
				$t$については(★)が言えるので,$t$と$v$は一致する.
				ゆえに$s$と$u$は一致する.ゆえに$s$に対しても(★)が言える.
				
			\item[式について]
			
	$\bot$については,その始切片は$\bot$に限られる.
	$\in st$なる原子式については,その始切片は
	\begin{align}
		\in, \quad \in s, \quad \in st
	\end{align}
	のいずれかとなるが,このうち式であるものは$\in st$のみである.
	$=st$なる原子式についても,その始切片で式であるものは$=st$に限られる.
	
	いま$\varphi$を任意に与えられた式とし,
	$\varphi$の真部分式に対しては(★)が当てはまっているとする.
	\begin{description}
		\item[ケース1] 式$\psi$が得られて$\varphi$が
			\begin{align}
				\rightharpoondown \psi
			\end{align}
			であるとき,$\psi$は$\varphi$の真部分式であるので(★)は当てはまる.
			$\varphi$の始切片で式であるものは,
			式$\xi$を用いて$\rightharpoondown \xi$と表せるが,
			$\xi$は$\psi$の始切片であるから,帰納法の仮定より$\xi$と$\psi$は一致する.
			ゆえに$\varphi$の始切片で式であるものは$\varphi$自身に限られる.
			
		\item[ケース2] 式$\psi$と$\xi$が得られて$\varphi$が
			\begin{align}
				\vee \psi \xi
			\end{align}
			であるとする.$\varphi$の始切片で式であるものも$\vee$が左端に来るので,
			式$\eta$と式$\zeta$が得られて始切片は
			\begin{align}
				\vee \eta \zeta
			\end{align}
			と表せる.$\psi$と$\eta$,$\xi$と$\zeta$は
			いずれも$\varphi$の真部分式であるので(★)が当てはまる.
			そして$\psi$と$\eta$は一方が他方の始切片であるので,(★)より一致する.
			すると$\xi$と$\zeta$も一方が他方の始切片ということになり,(★)より一致する.
			ゆえに$\vee \psi \xi$と$\vee \eta \zeta$は一致する.
			つまり$\varphi$の始切片で式であるものは$\varphi$自身に限られる.
			$\varphi$が$\wedge \psi \xi$や$\Longrightarrow \psi \xi$である場合も同じである.
			
		\item[ケース3] 項$x$と式$\psi$が得られて,$\varphi$が
			\begin{align}
				\forall x \psi
			\end{align}
			であるとき,$\varphi$の始切片で式であるものは,式$\xi$が取れて
			\begin{align}
				\forall x \xi
			\end{align}
			と表せる.このとき$\xi$は$\psi$の始切片であるし,
			また$\psi$は$\varphi$の真部分式であるから,(★)より$\psi$と$\xi$は一致する.
			ゆえに$\varphi$の始切片で式であるものは$\varphi$自身に限られる.
			$\varphi$が$\forall x \psi$である場合も同じである.
			\QED
	\end{description}

		\end{description}
	\end{metaprf}
	
\section{スコープ}
	$\varphi$を式とし,$s$を「$\natural,\in,\bot,\rightharpoondown,\vee,\wedge,\Longrightarrow,\exists,\forall$」
	のいずれかの記号とし,$\varphi$に$s$が現れたとする.このとき,$s$のその出現位置から始まる$\varphi$の部分式,
	或いは$s$が$\natural$である場合は部分項,を
	$s$の{\bf スコープ}\index{スコープ}{\bf (scope)}と呼ぶ.具体的に,$\varphi$を
	\begin{align}
		\Longrightarrow \forall x \wedge \Longrightarrow \in xy \in xz \Longrightarrow \in xz \in xy = yz
	\end{align}
	としよう.このとき$\varphi$の左から$6$番目に$\in$が現れるが,この$\in$から
	\begin{align}
		\in xy
	\end{align}
	なる原子式が$\varphi$の上に現れている:
	\begin{align}
		\Longrightarrow \forall x \wedge \Longrightarrow \textcolor{red}{\in xy} \in xz \Longrightarrow \in xz \in xy = yz.
	\end{align}
	これは{\bf $\varphi$における左から6番目の$\in$のスコープ}である.他にも,$\varphi$の左から$4$番目に$\wedge$が現れるが,この右側に
	\begin{align}
		\Longrightarrow \in xy \in xz
	\end{align}
	と
	\begin{align}
		\Longrightarrow \in xz \in xy
	\end{align}
	の二つの式が続いていて,$\wedge$を起点に
	\begin{align}
		\wedge \Longrightarrow \in xy \in xz \Longrightarrow \in xz \in xy
	\end{align}
	なる式が$\varphi$の上に現れている:
	\begin{align}
		\Longrightarrow \forall x \textcolor{red}{\wedge \Longrightarrow \in xy \in xz \Longrightarrow \in xz \in xy} = yz.
	\end{align}
	これは{\bf $\varphi$における左から4番目の$\wedge$のスコープ}である.$\varphi$の左から$2$番目には$\forall$が現れて,
	この$\forall$に対して項$x$と
	\begin{align}
		\wedge \Longrightarrow \in xy \in xz \Longrightarrow \in xz \in xy
	\end{align}
	なる式が続き,
	\begin{align}
		\forall x \wedge \Longrightarrow \in xy \in xz \Longrightarrow \in xz \in xy
	\end{align}
	なる式が$\varphi$の上に現れている:
	\begin{align}
		\Longrightarrow \textcolor{red}{\forall x \wedge \Longrightarrow \in xy \in xz \Longrightarrow \in xz \in xy} = yz.
	\end{align}
	
	しかも$\in,\wedge,\forall$のスコープは上にあげた部分式のほかに取りようが無い.
	上の具体例を見れば,直感的に「現れた記号のスコープはただ一つだけ,必ず取ることが出来る」
	が一般の式に対しても当てはまるであるように思えるが,直感を排除してこれを認めるには構造的帰納法の原理が必要になる.
	
	当然ながら$\mathcal{L}_{\in}$の式には同じ記号が何か所にも出現しうるので,
	式$\varphi$に記号$s$が現れたと言ってもそれがどこの$s$を指定しているのかはっきりしない.
	しかし{\bf スコープを考える際には,$\varphi$に複数現れうる$s$のどれか一つを選んで,
	その$s$に終始注目している}のであり,
	「その$s$の...」や「$s$のその出現位置から...」のように限定詞を付けてそのことを示唆することにする.
	
	\begin{screen}
		\begin{metathm}[スコープの存在]\label{metathm:existence_of_scopes_L_in}
		$\varphi$を式とするとき,
		\begin{description}
			\item[(a)] $\natural$が$\varphi$に現れたとき,項$t$が得られて,
				$\natural$のその出現位置から$\natural t$なる項が$\varphi$の上に現れる.
				
			\item[(b)] $\in$が$\varphi$に現れたとき,項$\sigma$と項$\tau$が得られて,
				$\in$のその出現位置から$\in \sigma \tau$なる式が$\varphi$の上に現れる.
				
			\item[(c)] $\rightharpoondown$が$\varphi$に現れたとき,式$\psi$が得られて,
				$\rightharpoondown$のその出現位置から$\rightharpoondown \psi$なる式が
				$\varphi$の上に現れる.
				
			\item[(d)] $\vee$が$\varphi$に現れたとき,式$\psi$と式$\xi$が得られて,
				$\vee$のその出現位置から$\vee \psi \xi$なる式が$\varphi$の上に現れる.
				
			\item[(e)] $\exists$が$\varphi$に現れたとき,項$x$と式$\psi$が得られて,
				$\exists$のその出現位置から$\exists x \psi$なる式が$\varphi$の上に現れる.
		\end{description}
		\end{metathm}
	\end{screen}
	
	(b)では$\in$を$=$に替えたって同じ主張が成り立つし,(d)では$\vee$を$\wedge$や$\Longrightarrow$に替えても同じである.
	(e)では$\exists$を$\forall$に替えても同じことが言える.
	
	\begin{metaprf}\mbox{}
		\begin{description}
			\item[項について]
				「項に$\natural$が現れたとき,項$t$が取れて,
				その$\natural$の出現位置から$\natural t$がその項の部分項として現れる」---(※),を示す.
				$s$を項とするとき,$s$が文字ならば$s$に対して(※)が言える.
				$s$が文字でないとき,$s$の全ての真部分項に対して(※)が言えるとする.
				$s$は文字ではないので,(項の構成法より)項$t$が取れて$s$は
				\begin{align}
					\natural t
				\end{align}
				と表せる.$s$に現れる$\natural$とは$s$の左端のものであるか
				$t$の中に現れるものであるが,$t$は$s$の真部分項であって,
				$t$については(※)が言えるので,結局$s$に対しても(※)が言えるのである.
			
			\item[case1]
				$\bot$に対しては上の言明は当てはまる.
			
			\item[case2]
				$\in s t$なる式に対しては,$\in$のスコープは$\in s t$に他ならない.
				実際,$\in$から始まる$\in s t$の部分式は,項$u,v$が取れて
				\begin{align}
					\in u v
				\end{align}
				と書けるが,このとき$u$と$s$は一方が他方の始切片となっているので,
				メタ定理\ref{metathm:initial_segment_L_in}より$u$と$s$は一致する.
				すると今度は$v$と$t$について一方が他方の始切片となるので,
				メタ定理\ref{metathm:initial_segment_L_in}より$v$と$t$も一致する.
				
				$\in s t$に$\natural$が現れた場合,これが$s$に現れているとすると,
				前段より項$u$が取れて,この$\natural$の出現位置から$\natural u$なる項が$s$の上に現れる.
				また項$v$が取れて,この$\natural$の出現位置から$\natural v$なる項が
				$\in s t$の上に現れているとしても,$u$と$v$は一方が他方の始切片となるから
				メタ定理\ref{metathm:initial_segment_L_in}より
				$u$と$v$は一致する.$\natural$が$t$に現れたときも同じである.
				以上より$\in s t$に対して定理の主張が当てはまる.
					
			\item[case3]
				$\varphi$を任意に与えられた式とし,$\varphi$の全ての真部分式に対しては
				定理の主張が当てはまっているとする.
		
				式$\varphi$と$\psi$に対して上の言明が当てはまるとする.
				式$\rightharpoondown \varphi$に対して,
				$\sigma$が左端の$\rightharpoondown$であるとき
				$\sigma \varphi$は$\rightharpoondown \varphi$の部分式である.
				また$\sigma \psi$が$\sigma$のその出現位置から始まる$\rightharpoondown \varphi$の部分式
				であるとすると,
				$\psi$は$\varphi$の左端から始まる$\varphi$の部分式ということになるので
				メタ定理\ref{metathm:initial_segment_L_in}より$\varphi$と$\psi$は一致する.
				$\sigma$が$\varphi$に現れる記号であれば,帰納法の仮定より
				$\sigma$から始まる$\varphi$の部分式が一意的に得られる.
				その部分式は$\rightharpoondown \varphi$の部分式でもあるし,
				$\rightharpoondown \varphi$の部分式としての一意性は
				メタ定理\ref{metathm:initial_segment_L_in}より従う.
	
				式$\vee \varphi \psi$に対して,
				$\sigma$が左端の$\vee$であるとき,式$\xi$と$\eta$が得られて$\sigma \xi \eta$が
				$\vee \varphi \psi$の部分式となったとすると,
				$\xi$と$\varphi$は左端を同じくし,どちらか一方は他方の部分式である.
				$\xi$が$\varphi$の部分式であるならば,
				メタ定理\ref{metathm:initial_segment_L_in}より$\xi$と$\varphi$は一致する.
				$\varphi$が$\xi$の部分式であるならば,$\xi$と$\psi$が重なるとなると
				$\psi$の左端の記号から始まる$\xi$の部分式と$\psi$は一致しなくてはならない.
				\QED
		\end{description}
	\end{metaprf}
	
	始切片に関する定理からスコープの一意性を示すことが出来る.
	
	\begin{screen}
		\begin{metathm}[スコープの一意性]\label{metathm:uniqueness_of_scopes_L_in}
			$\varphi$を式とし,$s$を
			$\natural,\in,\bot,\rightharpoondown,\vee,\wedge,\Longrightarrow,\exists,\forall$
			のいずれかの記号とし,$\varphi$に$s$が現れたとする.
			このとき$\varphi$におけるその$s$のスコープは唯一つである.
		\end{metathm}
	\end{screen}
	
	\begin{metaprf}\mbox{}
		\begin{description}
			\item[case1]
				$\natural$が$\varphi$に現れた場合,スコープの存在定理\ref{metathm:existence_of_scopes_L_in}
				より項$\tau$が取れて
				\begin{align}
					\natural \tau
				\end{align}
				なる形の項が$\natural$のその出現位置から$\varphi$の上に現れるわけだが,
				\begin{align}
					\natural \sigma
				\end{align}
				なる項も$\natural$のその出現位置から$\varphi$の上に出現しているといった場合,
				$\tau$と$\sigma$は一方が他方の始切片となるわけで,
				始切片のメタ定理\ref{metathm:initial_segment_L_in}より
				$\tau$と$\sigma$は一致する.
			
			\item[case2]
				$\rightharpoondown$が$\varphi$に現れた場合,
				これはcase1において項であったところが式に替わるだけで殆ど同じ証明となる.
			
			\item[case3]
				$\vee$が$\varphi$に現れた場合,定理\ref{metathm:existence_of_scopes_L_in}
				より式$\psi,\xi$が取れて
				\begin{align}
					\vee \psi \xi
				\end{align}
				なる形の式が$\vee$のその出現位置から$\varphi$の上に現れる.ここで
				\begin{align}
					\vee \eta \Gamma
				\end{align}
				なる式も$\vee$のその出現位置から$\varphi$の上に出現しているといった場合,
				まず$\psi$と$\eta$は一方が他方の始切片となるわけで,
				メタ定理\ref{metathm:initial_segment_L_in}より
				$\psi$と$\eta$は一致する.すると今度は$\xi$と$\Gamma$について
				一方が他方の始切片となるので,同様に$\xi$と$\Gamma$も一致する.
				$\wedge$や$\Longrightarrow$のスコープの一意性も同様に示される.
				
			\item[case4]
				$\exists$が$\varphi$に現れた場合,定理\ref{metathm:existence_of_scopes_L_in}
				より項$x$と式$\psi$が取れて
				\begin{align}
					\exists x \psi
				\end{align}
				なる形の式が$\exists$のその出現位置から$\varphi$の上に現れる.ここで
				\begin{align}
					\exists y \xi
				\end{align}
				なる式も$\exists$のその出現位置から$\varphi$の上に出現しているといった場合,
				まず項$x$と項$y$は一方が他方の始切片となるわけで,
				メタ定理\ref{metathm:initial_segment_L_in}より
				$x$と$y$は一致する.すると今度は$\psi$と$\xi$が
				一方が他方の始切片の関係となるので,この両者も一致する.
				$\forall$のスコープの一意性も同様に示される.
				\QED
		\end{description}
	\end{metaprf}
	
	\begin{itembox}[l]{量化}
		$\varphi$に$\forall$が現れるとき,
		その$\forall$に後続する項$x$が取れるが,このとき項$x$は$\forall$のスコープ内で
		{\bf 量化されている}\index{りょうか@量化}{\bf(quantified)}という.
		詳しく言い直せば,項$x$と式$\psi$が取れて,その$\forall$のスコープは
		\begin{align}
			\forall x \psi
		\end{align}
		なる式で表されるが,このとき$x$は$\forall x \psi$において量化されているという.
	\end{itembox}
	
	$A$を式とし,$a$を$A$に現れる項とする.このとき$A$の中の項$a$を全て項$x$に置き換えた式を
	\begin{align}
		(x \mid a)A
	\end{align}
	で表す.特に項$a$と項$x$が同一の項である場合は$(x \mid a)A$は$A$自身に一致する.
	また$A$の中で自由に現れる項が$a$のみであって,かつ$a$が自由に現れる箇所がどれも項$x$の量化スコープではないとき,
	$A$に現れる項$a$のうち,{\bf 自由に現れる箇所}を全て項$x$に置き換えた式を
	\begin{align}
		A(x)
	\end{align}
	と書く.$A$に現れる項$a$が全て自由であるときは$A(a)$は$A$自身に一致する.

\chapter{$\varepsilon$項と内包項}
	通常は集合論の言語には$\lang{\in}$が使われる.
	しかし乍ら,当然集合論と称している以上は「集合」というモノを扱っている筈なのに,
	当の「集合」は$\lang{\in}$では実体を持たない空想でしかない.
	どういう意味かというと,例えば
	\begin{align}	
		\exists x\, \forall y\, (\, y \notin x\, )
	\end{align}
	と書けば「$\forall y\, (\, y \notin x\, )$を満たすような集合$x$が存在する」
	と読むわけだが,その在るべき$x$を$\lang{\in}$では特定できないのである.
	というのも,$\lang{\in}$の``名詞''は{\bf 変項}{\bf (variable)}だけなのだから.
	しかし言語の拡張の仕方によっては,この``空虚な存在''を実在で補強することが可能になる.
	
	\begin{comment}
	...
	考えてみれば愈々不可解である.そもそも集合なるものは我々の想像の中にしかないものであって,
	その想像を紙の上に具象化したはずの``集合論''の世界においてさえ集合が虚構に追いやられているなんて,
	どうして易々と看過できようか.
	この点で,$\lang{\in}$のみで集合論を展開することには感覚的に大きな抵抗があるわけだ.
	そこで,集合を具体的なオブジェクトとして扱えるように言語を拡張しようではないか
	(と意気込んではみるものの,遍く受け入れられている{\bf ZFC}集合論に上手く馴染めない
	偏屈な異分子のたわ言,と一笑に付されるかもしれない.まあこう弱気になることも多々あるが,
	修士号のためには偏執的なこだわりだって岩をも通すのである!).
	
	\end{comment}
	
	言語の拡張は二段階を踏む.
	項$x$が自由に現れる式$A(x)$に対して
	\begin{align}
		\Set{x}{A(x)}
	\end{align}
	なる形の項を導入する.この項の記法は{\bf 内包的記法}\index{ないほうてききほう@内包的記法}
	{\bf (international notation)}と呼ばれる.導入の意図は``$A(x)$を満たす集合$x$の全体''
	という意味を込めた式の対象化であって,実際に後で
	\begin{align}
		\forall u\, \left(\, u \in \Set{x}{A(x)} \lrarrow A(u)\, \right)
	\end{align}
	を保証する(内包性公理).
	
	追加する項はもう一種類ある.$A(x)$を上記のものとするが,この$A(x)$は$x$に関する性質という見方もできる.
	そして``$A(x)$という性質を具えている集合$x$''という意味を込めて
	\begin{align}
		\varepsilon x A(x)
	\end{align}
	なる形の項を導入するのだ.これはHilbertの{\bf $\varepsilon$項}\index{イプシロン項}
	{\bf (epsilon term)}と呼ばれるオブジェクトであるが,
	導入の意図とは裏腹に$\varepsilon x A(x)$は性質$A(x)$を持つとは限らない.
	$\varepsilon x A(x)$が性質$A(x)$を持つのは,$A(x)$を満たす集合$x$が存在するとき,またその時に限られる
	(この点については後述の$\exists$に関する定理によって明らかになる).
	$A(x)$を満たす集合$x$が存在しない場合は,$\varepsilon x A(x)$は正体不明のオブジェクトとなる.
	
\section{$\varepsilon$}
	まずは$\varepsilon$項を項として追加した
	言語$\lang{\varepsilon}$に拡張する.
	$\lang{\varepsilon}$の構成要素は以下である:
	
	\begin{description}
		\item[矛盾記号] $\bot$
		\item[論理記号] $\rightharpoondown,\ \vee,\ \wedge,\ \rarrow$
		\item[量化子] $\forall,\ \exists$
		\item[述語記号] $=,\ \in$
		\item[変項] $\lang{\in}$の項は$\lang{\varepsilon}$の
			{\bf 変項}\index{へんこう@変項}{\bf (variable)}である.またこれらのみが
			$\lang{\varepsilon}$の変項である.
		\item[イプシロン] $\varepsilon$
	\end{description}
	
	$\lang{\in}$からの変更点は,``使用文字''が``変項''に代わったことと
	$\varepsilon$が加わったことである.続いて項と式の定義に移るが,
	帰納のステップは$\lang{\in}$より複雑になる:
	
	\begin{itemize}
		\item $\lang{\varepsilon}$の変項は$\lang{\varepsilon}$の項である.
		\item $\bot$は$\lang{\varepsilon}$の式である.
		\item $\sigma$と$\tau$を$\lang{\varepsilon}$の項とするとき,
			$\in st$と$=st$は$\lang{\varepsilon}$の式である.
		\item $\varphi$を$\lang{\varepsilon}$の式とするとき,
			$\rightharpoondown \varphi$は$\lang{\varepsilon}$の式である.
		\item $\varphi$と$\psi$を$\lang{\varepsilon}$の式とするとき,
			$\vee \varphi \psi,\ \wedge \varphi \psi,\ \rarrow \varphi \psi$は
			いずれも$\lang{\varepsilon}$の式である.
		\item $x$を$\lang{\varepsilon}$の{\bf 変項}とし,$\varphi$を
			$\lang{\varepsilon}$の式とするとき,$\forall x \varphi$と
			$\exists x \varphi$は$\lang{\varepsilon}$の式である.
		\item $x$を$\lang{\varepsilon}$の{\bf 変項}とし,$\varphi$を
			$\lang{\varepsilon}$の式とするとき,$\varepsilon x \varphi$は
			$\lang{\varepsilon}$の項である.
		\item 以上のみが$\lang{\varepsilon}$の項と式である.
	\end{itemize}
	
	$\lang{\in}$に対して行った帰納的定義との大きな違いは,
	{\bf 項と式の定義が循環している}点にある.
	$\lang{\varepsilon}$の式が$\lang{\varepsilon}$の項を用いて
	作られるのは当然ながら,その逆に$\lang{\varepsilon}$の項もまた
	$\lang{\varepsilon}$の式から作られるのである.
	
	定義の循環によって構造が見えづらくなっているが,直感的には次のように捉えることが出来る.
	というよりは,次のように$\lang{\varepsilon}$が作られているとすれば良い.
	
	\begin{enumerate}
		\item $\lang{\in}$の式から$\varepsilon$項を作り,
			その$\varepsilon$項を第$1$世代$\varepsilon$項と呼ぶことにする.
		\item 変項と第$1$世代$\varepsilon$項を項として式を作り,
			これらを第$2$世代の式と呼ぶことにする.
			また第$2$世代の式で作る$\varepsilon$項を第$2$世代$\varepsilon$項と呼ぶことにする.
		\item 第$n$世代の$\varepsilon$項をが出来たら,
			それらと変項を項として第$n+1$世代の式を作り,
			第$n+1$世代$\varepsilon$項を作る.
			
			\begin{itemize}
				\item ちなみに,このように考えると第$n$世代$\varepsilon$項は
					第$n+1$世代$\varepsilon$項でもある.
			\end{itemize}
	\end{enumerate}
	
	こう捉えることで,$\lang{\varepsilon}$における構造的帰納法の原理を規定すれば良い.
	粗く考察してると,項と式に対する言明Xが与えられたとき,
	\begin{enumerate}
		\item まずは$\lang{\in}$の項と式に対してXが言えて,かつ
			第$1$世代の$\varepsilon$項に対してもXが言えることがスタート地点である.
		\item 第$2$世代の式に対してXが言えることと,第$2$世代の$\varepsilon$項に対してXが言えること
			を示す.
			
			$\vdots$
			
		\item 第$n$世代までのすべての式と項に対してXが言えることを仮定して,
			第$n+1$世代の式に対してXが言えることと,第$n+1$世代の$\varepsilon$項に対して
			Xが言えることを示す.
	\end{enumerate}
	の以上が検査出来れば,$\lang{\varepsilon}$のすべての項と式に対してXが言えると
	結論するのは妥当である.ただし第$n$世代だとかいうカテゴライズは直感的考察を補佐するための
	インフォーマルなものであり,更に簡略された手法によってこの操作が実質的に為されることが期される.
	
	\begin{screen}
		\begin{metaaxm}[$\lang{\varepsilon}$の項と式に対する構造的帰納法]
			$\lang{\varepsilon}$の項に対する言明Xと式に対する言明Yに対し,
			\begin{enumerate}
				\item $\lang{\in}$の項と式,および$\lang{\in}$の式
					で作る$\varepsilon$項に対してX及びYが言える.
				\item $\varphi$を任意に与えられた$\lang{\varepsilon}$の式として,
					$\varphi$に現れる全ての項及び真部分式に対して
					X及びYが言えると仮定するとき,
					\begin{itemize}
						\item $\varphi$が$\in \sigma \tau$なる形の原子式であるとき
							$\varphi$に対してYが言える.
						\item $\varphi$が$\rightharpoondown \varphi$なる形の式であるとき
							$\varphi$に対してYが言える.
						\item $\varphi$が$\vee \psi \chi$なる形の式であるとき
							$\varphi$に対してYが言える.
						\item $\varphi$が$\exists x \psi$なる形の式であるとき
							$\varphi$に対してYが言える.
						\item $\varepsilon x \varphi$なる$\varepsilon$項
							に対してXが言える.
					\end{itemize}
			\end{enumerate}
			ならば,いかなる項と式に対してもXが言える.
		\end{metaaxm}
	\end{screen}
	
	次の性質は至極当たり前であるが,
	
	\begin{screen}
		\begin{metathm}
			$A$を$\lang{\varepsilon}$の式としたとき,
			$\varepsilon x A$なる形の$\varepsilon$項は$A$には現れない.
		\end{metathm}
	\end{screen}
	
	もし$A$に$\varepsilon x A$が現れるならば,当然$A$の中の$\varepsilon x A$にも
	$\varepsilon x A$が現れるし,$A$の中の$\varepsilon x A$の中の$\varepsilon x A$にも
	$\varepsilon x A$が現れるといった具合に,この入れ子には終わりがなくなる.
	だが,当然こんなことは起こり得ない.$A$が指す記号列のどの部分を切り取っても
	それは$A$より短い記号列であって,$\varepsilon x A$の現れる余地など無いからである.
	
	しかしながら,やはり全容を把握しきれない世界の話になると,
	何か超然的な力が働いて現世の常識を捻じ曲げうるのではないか,という不安がぬぐえない.
	基礎論の基礎にあるのは,直感や常識の正体の究明ではないのか.
	
	$\varphi$を$\lang{\varepsilon}$の式としたら,$\varphi$の部分式とは,
	$\varphi$から切り取られる一続きの記号列で,それ自身が$\lang{\varepsilon}$の式であるものを指す.
	$\varphi$自身もまた$\varphi$の部分式である.
	
	\begin{screen}
		\begin{metathm}[$\lang{\varepsilon}$の始切片の一意性]
		\label{metathm:initial_segment_L_epsilon}
			$\tau$を$\lang{\varepsilon}$の項とするとき,
			$\tau$の始切片で$\lang{\varepsilon}$の項であるものは$\tau$自身に限られる.
			また$\varphi$を$\lang{\varepsilon}$の式とするとき,
			$\varphi$の始切片で$\lang{\varepsilon}$の式であるものは$\varphi$自身に限られる.
		\end{metathm}
	\end{screen}
	
	\begin{metaprf}\mbox{}
		\begin{description}
			\item[step1]
				$\lang{\in}$の式と項についてはメタ定理\ref{metathm:initial_segment_L_in}より
				当座の定理の主張が従う.また$\varphi$を$\lang{\in}$の式とし,
				$\tau$を$\lang{\varepsilon}$の項とし,また$\tau$は
				\begin{align}
					\varepsilon x \varphi
				\end{align}
				なる$\varepsilon$項の始切片とするとき,$\tau$の左端は$\varepsilon$であるから
				\begin{align}
					\varepsilon y \psi
				\end{align}
				なる形をしているはずである.すると$x$と$y$とは一方が他方の始切片となるので
				メタ定理\ref{metathm:initial_segment_L_in}より$y$は$x$に一致する.
				するとまた$\varphi$と$\psi$はは一方が他方の始切片となるので一致する.
				つまり$\tau$は$\varepsilon x \varphi$そのものである.
				
			\item[step2]
				$\varphi$を$\lang{\varepsilon}$の式とするとき,$\varphi$の
				すべての項や真部分式に対して定理の主張が当たっているなら
				$\varphi$に対しても定理の主張通りのことが満たされる,
				ということはメタ定理\ref{metathm:initial_segment_L_in}と同じように示される.
				もう一度書けば,
				\begin{itembox}[l]{IH (帰納法の仮定)}
					$\varphi$に現れる任意の項$\tau$に対して,その始切片で項であるものは$\tau$
					に限られる.また$\varphi$に現れる任意の真部分式$\psi$に対して,
					その始切片で式であるものは$\psi$に限られる.
				\end{itembox}
				として
				\begin{description}
					\item[case1]
						$\varphi$が
						\begin{align}
							\in s t
						\end{align}
						なる原子式であるとき,$\varphi$の始切片で式であるものもまた
						\begin{align}
							\in u v
						\end{align}
						なる形をしているが,$u$と$s$は一方が他方の始切片となっているので
						(IH)より一致する.すると$v$と$t$も一方が他方の始切片となるので
						(IH)より一致する.ゆえに$\varphi$の始切片で式であるもの
						は$\varphi$自信に限られる.
						
					\item[case2] $\varphi$が
						\begin{align}
							\rightharpoondown \psi
						\end{align}
						なる形の式であるとき,$\varphi$の始切片で式であるももまた
						\begin{align}
							\rightharpoondown \xi
						\end{align}
						なる形をしている.このとき$\xi$は$\psi$の始切片であるから,
						(IH)より$\xi$と$\psi$は一致する.
						ゆえに$\varphi$の始切片で式であるものは$\varphi$自身に限られる.
			
					\item[case3] $\varphi$が
						\begin{align}
							\vee \psi \xi
						\end{align}
						なる形の式であるとき,$\varphi$の始切片で式であるものもまた
						\begin{align}
							\vee \eta \zeta
						\end{align}
						なる形をしている.このとき$\psi$と$\eta$は一方が他方の始切片であるので
						(IH)より一致する.すると$\xi$と$\zeta$も一方が他方の始切片ということに
						なり,(IH)より一致する.ゆえに$\varphi$の始切片で式であるものは
						$\varphi$自身に限られる.
						
					\item[case4] $\varphi$が
						\begin{align}
							\exists x \psi
						\end{align}
						なる形の式であるとき,$\varphi$の始切片で式であるものもまた
						\begin{align}
							\exists y \xi
						\end{align}
						なる形の式である.このとき$x$と$y$は一方が他方の始切片であり,これらは
						変項であるからメタ定理\ref{metathm:initial_segment_L_in}
						より一致する.すると$\psi$と$\chi$も一方が他方の始切片ということに
						なり,(IH)より一致する.ゆえに$\varphi$の始切片で式であるものは
						$\varphi$自身に限られる.
						
					\item[case5] $\varepsilon x \varphi$の始切片で項であるものは
						\begin{align}
							\varepsilon y \psi
						\end{align}
						なる形をしている筈である.このとき,まずメタ定理
						\ref{metathm:initial_segment_L_in}より$x$と$y$は一致する.
						すると$\psi$は$\varphi$の始切片であることになるが,
						前段までの結果から$\varphi$と$\psi$は一致する.
						\QED
				\end{description}
		\end{description}
	\end{metaprf}
	
	\begin{screen}
		\begin{metathm}[$\lang{\varepsilon}$のスコープの存在]
			$\varphi$を$\lang{\varepsilon}$の式,或いは項とするとき,
			\begin{description}
				\item[(a)] $\natural$が$\varphi$に現れたとき,変項$s,t$が得られて,
					$\natural$のその出現位置から$\natural s t$なる変項が$\varphi$の上に現れる.
					
				\item[(b)] $\in$が$\varphi$に現れたとき,$\lang{\varepsilon}$の項$\sigma,\tau$が得られて,
					$\in$のその出現位置から$\in \sigma \tau$なる式が$\varphi$の上に現れる.
				
				\item[(c)] $\rightharpoondown$が$\varphi$に現れたとき,
					$\lang{\varepsilon}$の式$\psi$が得られて,
					$\rightharpoondown$のその出現位置から
					$\rightharpoondown \psi$なる式が$\varphi$の上に現れる.
				
				\item[(d)] $\vee$が$\varphi$に現れたとき,$\lang{\varepsilon}$の式$\psi,\xi$が得られて,
					$\vee$のその出現位置から$\vee \psi \xi$なる式が$\varphi$の上に現れる.
				
				\item[(e)] $\exists$が$\varphi$に現れたとき,変項$x$と$\lang{\varepsilon}$の式$\psi$が得られて,
					$\exists$のその出現位置から$\exists x \psi$なる式が$\varphi$の上に現れる.
			\end{description}
		\end{metathm}
	\end{screen}
	
	(b)では$\in$を$=$に替えたって同じ主張が成り立つし,(d)では$\vee$を$\wedge$や$\lrarrow$に替えても同じである.
	(e)では$\exists$を$\forall$に替えても同じであるのは良いとして,
	$\varepsilon$項の成り立ちから$\exists$を$\varepsilon$に替えても同様の主張が成り立つ.
	
	示すのはスコープの存在だけで良い.一意性は始切片の定理からすぐに従う.実際
	$\varphi$を$\lang{\varepsilon}$の式として,その中に$\varepsilon$が出現したとすると,
	``スコープの存在が保証されていれば!''$\varepsilon$のその出現位置から
	\begin{align}
		\varepsilon x \psi
	\end{align}
	なる$\varepsilon$項が$\varphi$の上に現れるわけだが,他の誰かが「$\varepsilon y \xi$という
	$\varepsilon$項がその$\varepsilon$の出現位置から抜き取れるぞ」と言ってきたとしても,
	当然ながら$x$と$y$は一方が他方の始切片となるので一致する変項であるし(メタ定理\ref{metathm:initial_segment_L_in}),
	すると今度は$\psi$と$\xi$の一方が他方の始切片となるが,そのときもメタ定理\ref{metathm:initial_segment_L_epsilon}より
	両者は一致する.
	
	\begin{metaprf}\mbox{}
		\begin{description}
			\item[step1]
				$\varphi$が$\lang{\in}$の式であるときは,スコープの存在は
				メタ定理\ref{metathm:existence_of_scopes_L_in}で既に示されている.
				また$\lang{\in}$の式$\psi$に対して,
				\begin{align}
					\varepsilon x \psi
				\end{align}
				なる形の$\varepsilon$項に対しても
				(a)から(e)が満たされる.実際,(b)から(e)に関しては,
				$\in,\rightharpoondown,\vee,\exists$は
				$\psi$の中にしか出現し得ないので,スコープの存在は
				メタ定理\ref{metathm:existence_of_scopes_L_in}により保証される.
				(a)については,$\natural$は$\psi$の中に現れる場合と$x$の中に現れる場合があるが,
				いずれの場合もメタ定理\ref{metathm:existence_of_scopes_L_in}より
				スコープは取れる.
			
				ここで$\varphi$を任意に与えられた$\lang{\varepsilon}$の
				式として,次の仮定を置く.
				\begin{itembox}[l]{IH(帰納法の仮定)}
					$\varphi$の全ての部分式,及び
					$\varphi$に現れる全ての$\varepsilon$項の式,つまり
					$\varepsilon x \psi$なる項なら$\psi$のこと,
					に対して(a)から(e)まで言えると仮定する.
				\end{itembox}
				
			\item[step2]
				式$\varphi$が$\in s t$なる形の式であるとき.
				\begin{description}
					\item[case1]
						$\natural$が$\in s t$に現れたとしよう.
						$s$や$t$が変項であれば(a)の成立は見た目通りである.$s$が
						\begin{align}
							\varepsilon x \psi
						\end{align}
						なる形の$\varepsilon$項であって,
						$s$にその$\natural$が現れているとしよう.
						$\natural$が$x$に現れている場合は
						メタ定理\ref{metathm:existence_of_scopes_L_in}に訴えればよい.
						$\natural$が$\psi$に現れている場合は,(a)の成立は(IH)から従う.
						
					\item[case2]
						$\in$が$\in s t$に現れたとしよう.
						それが左端の$\in$であれば,(b)の成立を言うには$s$と$t$を取れば良い.
						$\in$が$s$に現れたとすれば,$s$は$\varepsilon$項であることになり,
						変項$x$と$\lang{\varepsilon}$の式$\psi$が取れて,$s$は
						\begin{align}
							\varepsilon x \psi
						\end{align}
						と表せる.$\in$は$\psi$に現れるので,(IH)より$\lang{\varepsilon}$の項$u,v$が取れて,
						$\in$のその出現位置から$\in s t$なる式が$\psi$の上に現れる.
						$\in$が$t$に現れる場合も同様に(b)の成立が言える.
				
					\item[case3]
						$\in s t$に論理記号($\rightharpoondown,\vee,\wedge,\rarrow,\exists,\forall$のいずれか)
						が現れたとしよう.
						そしてその現れた記号を便宜上$\sigma$と書こう.
						$\sigma$の出現位置が$s$にあるとすれば,そのことは$s$が
						\begin{align}
							\varepsilon x \psi
						\end{align}
						なる形の$\varepsilon$項であることを意味する.当然$\sigma$は$\psi$の中にあるわけで,
						(c)もしくは(d)の成立は(IH)から従う.
						
					\item[case4]
						$\in s t$に$\varepsilon$が現れたとしよう.
						$\varepsilon$の出現位置が$s$にあるとすれば,そのことは$s$が
						\begin{align}
							\varepsilon x \psi
						\end{align}
						なる形の$\varepsilon$項であることを意味する.
						$\varepsilon$の出現位置が$s$の左端である場合,(e)の成立を言うには
						この$x$と$\psi$を取れば良い.
						$\varepsilon$が$\psi$の中にある場合は,
						$(e)$の成立は(IH)から従う.
				\end{description}
				
			\item[step3]
				式$\varphi$が$\rightharpoondown \psi$なる形のとき,
				$\varphi$に現れた記号は左端の$\rightharpoondown$であるか,そうでなければ
				$\psi$の中に現れる.左端の$\rightharpoondown$のスコープは$\varphi$自身である.
				$\psi$に現れた記号のスコープの存在は
				(IH)により保証される.
				
			\item[step4]
				式$\varphi$が$\vee \psi \xi$なる形のとき,
				$\varphi$に現れた記号は左端の$\vee$であるか,そうでなければ
				$\psi \xi$の中に現れる.左端の$\vee$のスコープは$\varphi$自身である.
				$\psi \xi$に現れた記号のスコープの存在は(IH)により保証される.
			
			\item[step5]
				式$\varphi$が$\exists x \psi$なる形のとき,
				$\varphi$に現れた記号は左端の$\exists$であるか,そうでなければ
				$\psi$の中に現れる.左端の$\exists$のスコープは$\varphi$自身である.
				$\psi$に現れた記号のスコープの存在は(IH)により保証される.
				\QED
		\end{description}
	\end{metaprf}
	
\section{言語$\mathcal{L}$}
	本稿における主流の言語は,次に定める$\mathcal{L}$である.$\mathcal{L}$の最大の特徴は
	\begin{align}
		\Set{x}{A}
	\end{align}
	なる形のオブジェクトが``正式に''項として用いられることである.
	他の集合論の本では$\Set{x}{A}$なる項はインフォーマルに導入されるもので,
	しかもこれが常に集合であることを期すために
	$\Set{x \in z}{A}$などのように何らかの$z$を引き出す必要がある.
	$\Set{x}{A}$を正式に項として導入すれば煩雑さをある程度回避することが出来る.
	
	$\mathcal{L}$の構成要素は以下のものである.
	
	\begin{description}
		\item[矛盾記号] $\bot$
		\item[論理記号] $\rightharpoondown,\ \vee,\ \wedge,\ \rarrow$
		\item[量化子] $\forall,\ \exists$
		\item[述語記号] $=,\ \in$
		\item[変項] $\lang{\in}$の項は$\mathcal{L}$の変項である.またこれらのみが
			$\mathcal{L}$の変項である.
		\item[補助記号] $\{,\ |,\ \}$
	\end{description}
	
	$\mathcal{L}$の項と式の構成規則は$\lang{\in}$のものと大差ない.
	
	\begin{description}
		\item[項] 
			\begin{itemize}
				\item $\lang{\varepsilon}$の項は$\mathcal{L}$の項である.
				\item $x$を$\mathcal{L}$の変項とし,$A$を$\lang{\varepsilon}$の式とするとき,
					$\Set{x}{A}$なる記号列は$\mathcal{L}$の項である.
				\item 以上のみが$\mathcal{L}$の項である.
			\end{itemize}
	\end{description}
	
	によって正式に定義される.ここで,$\lang{\in}$の項は$\lang{\varepsilon}$
	の項でもあるから,すなわち$\mathcal{L}$の項でもある.つまり,定義には書いていないが
	{\bf $\mathcal{L}$の変項は$\mathcal{L}$の項である}.
	
	\begin{description}
		\item[式] 
			\begin{itemize}
				\item $\bot$は$\mathcal{L}$の式である.
				\item $\sigma$と$\tau$を$\mathcal{L}$の項とするとき,
					$\in st$と$=st$は$\mathcal{L}$の式である.
				\item $\varphi$を$\mathcal{L}$の式とするとき,
					$\rightharpoondown \varphi$は$\mathcal{L}$の式である.
				\item $\varphi$と$\psi$を$\mathcal{L}$の式とするとき,
					$\vee \varphi \psi,\ \wedge \varphi \psi,\ \rarrow \varphi \psi$は
					いずれも$\mathcal{L}$の式である.
				\item $x$を$\mathcal{L}$の{\bf 変項}とし,$\varphi$を
					$\mathcal{L}$の式とするとき,$\forall x \varphi$と
					$\exists x \varphi$は$\mathcal{L}$の式である.
			\end{itemize}
	\end{description}
	
	言語の拡張の仕方より明らかであるが,次が成り立つ:
	
	\begin{screen}
		\begin{metathm}
			$\lang{\in}$の式は$\lang{\varepsilon}$の式であり,
			また$\lang{\varepsilon}$の式は$\mathcal{L}$の式である.
		\end{metathm}
	\end{screen}
	
	\begin{screen}
		\begin{dfn}[類]
			$A$を$\lang{\in}$の式とし,$x$を$A$に現れる項とし,
			$A$の中で項$x$のみが自由に現れるとき,
			$\Set{x}{A(x)}$及び$\varepsilon x A(x)$を
			{\bf 類}\index{るい@類}{\bf (class)}と呼ぶ.
		\end{dfn}
	\end{screen}
	
	$\varphi$を$\mathcal{L}$の式とし,$s$を$\varphi$に現れる記号とすると,
	\begin{description}
		\item[(1)] $s$は文字である.
		\item[(2)] $s$は$\natural$である.
		\item[(2)] $s$は$\{$である.
		\item[(3)] $s$は$|$である.
		\item[(4)] $s$は$\}$である.
		\item[(5)] $s$は$\bot$である.
		\item[(6)] $s$は$\in$か$=$である.
		\item[(7)] $s$は$\rightharpoondown$である.
		\item[(8)] $s$は$\vee,\wedge,\rightarrow$のいずれかである.
	\end{description}
	
	\begin{screen}
		(★★) いま,$\varphi$を任意に与えられた式としよう.
		\begin{itemize}
			\item $\natural$が$\varphi$に現れたとき,$\lang{\in}$の項$\tau$と$\sigma$が得られて,$\natural \tau \sigma$は
				$\natural$のその出現位置から始まる$\lang{\in}$の項となる.
				また$\natural$のその出現位置から始まる$\lang{\in}$の項は$\natural \tau \sigma$のみである.
				
			\item $\{$が$\varphi$に現れたとき,$\lang{\in}$の変項$x$及び$\lang{\in}$の式$A$が得られて,
				$\{ x|A\}$は$\{$のその出現位置から始まる項となる.
				また$\{$のその出現位置から始まる項は$\{x|A\}$のみである.
				
			\item $|$が$\varphi$に現れたとき,,変項$x$と$\lang{\in}$の式$A$が得られて,
				$\{x|A\}$は$|$のその出現位置から広がる項となる.
				また$|$のその出現位置から広がる項は$\{x|A\}$のみである.
				
			\item $\}$が$\varphi$に現れたとき,変項$x$と式$A$が得られて,
				$\{x|A\}$は$\}$のその出現位置を終点とする項となる.
				また$\}$のその出現位置を終点とする項は$\{x|A\}$のみである.
		\end{itemize}
	\end{screen}
	
	\begin{description}
		\item[$\natural$に対して$\natural \tau \sigma$なる変項$\tau$と$\sigma$が得られること]
			$\natural$が原子項に現れたら,原子項とは文字$x,y$によって
			\begin{align}
				\natural xy
			\end{align}
			と表されるものであるから,$\natural$に対して変項$\tau,\sigma$ (すなわち文字$x,y$)が取れたことになる.
			$\natural$が項に現れたとする.項とは,変項$x,y$によって
			\begin{align}
				\natural xy
			\end{align}
			で表されるものであり,$\natural$は左端の$\natural$であるか,$x$に現れるか,$y$に現れる.
			$\natural$が$x$か$y$に現れるときは帰納法の仮定により,
			$\natural$が左端のものである場合は$x$が$\tau$,$y$が$\sigma$ということになる.
			
		\item[変項の始切片で変項であるものは自分自身のみ]
			$x$が文字である場合はそう.$x$の任意の部分変項が言明を満たしているなら,
			$x$は$\natural st$なる変項である(生成規則)から,$x$の始切片は$\natural uv$なる変項である.
			$s,t,u,v$はいずれも$x$の部分変項なので仮定が適用されている.
			ゆえに$s$と$u$は一方が他方の始切片であり,一致する.すなわち$t$と$v$も一方が他方の始切片であり一致する.
			ゆえに$x$の始切片で変項であるものは$x$自身である.
			
		\item[$\natural$に対して得られる変項の一意性]
			$\natural xy$と$\natural st$が共に変項であるとき,$x$と$s$,$y$と$t$は一致するか.
			$\natural xy$が原子項であるときは明らかである.
			$x$の始切片で変項であるものは$x$自身に限られるので,
			$x$と$s$は一致する.ゆえに$t$は$y$の始切片であり,$t$と$y$も一致する.
		
		\item[生成規則より$x$と$A$が得られるか]
			$\varphi$が原子式であるとき,
			$\{$が現れるとすれば項の中である.項とは$\lang{\in}$の項であるか$\{x|A\}$なるものであるので
			$\{$が現れたならば$\{$とは$\{x|A\}$の$\{$である.
			
			$\varphi$の任意の部分式に対して言明が満たされているとする.
			$\varphi$とは$\rightharpoondown \psi,\vee \psi \xi,...$の形であるから,
			$\varphi$に現れた$\{$とは$\psi$や$\xi$に現れるのである.ゆえに
			仮定より$x$と$A$が取れるわけである.
			
		\item[$\{$に対して]
			項の生成規則より$x$と$A$が得られる.
			$\{y|B\}$もまた$\{$から始まる項である場合,順番に見ていって
			$x$と$y$は一方が他方の始切片という関係になるから一致する.
			すると$A$と$B$は一方が他方の始切片という関係になり,(★)より$A$と$B$は一致する.
			
		\item[$|$について]
			項の生成規則より$x$と$A$が得られる.
			$\{y|B\}$もまた$|$から広がる項である場合,順番に見ていって
			$x$にも$y$にも$\{$という記号は現れないので$x$と$y$は一致する.
			$A$と$B$は一方が他方の始切片という関係になるので(★)より$A$と$B$は一致する.
			
		\item[$\}$について]
			項の生成規則より$x$と$A$が得られる.
			$\{y|B\}$もまた$\}$のその出現位置を終点とする変項である場合,
			$A$と$B$は$\lang{\in}$の式なので$|$という記号は現れない.ゆえに
			$A$と$B$は一致する.すると$x$と$y$は右端で揃うが,
			$x$にも$y$にも$\{$という記号は現れないので$x$と$y$は一致する.
	\end{description}
	
\section{類と集合}
	\begin{screen}
		\begin{dfn}[類と集合]
			$a$を類とするとき,$a$が集合であるという言明を
			\begin{align}
				\set{a} \defarrow \exists x\, (\, x = a\, )
			\end{align}
			で定める.$\set{a}$を満たす類$a$を{\bf 集合}\index{しゅうごう@集合}{\bf (set)}と呼び,
			$\rightharpoondown \set{a}$を満たす類$a$を{\bf 真類}\index{しんるい@真類}{\bf (proper class)}と呼ぶ.
		\end{dfn}
	\end{screen}
	
	ちなみに$\varepsilon x A(x)$は集合である.なぜならば
	\begin{align}
		\varepsilon x A(x) = \varepsilon x A(x)
	\end{align}
	だから
	\begin{align}
		\exists a\, \left(\, a = \varepsilon x A(x)\, \right).
	\end{align}
	また$\Set{x}{A(x)}$が集合であるとき
	\begin{align}
		\exists s\, \left(\, \Set{x}{A(x)} = s\, \right)
	\end{align}
	が成り立つが,量化の規則より
	\begin{align}
		\Set{x}{A(x)} = \varepsilon s \forall u\, (\, u \in s \lrarrow A(u)\, )
	\end{align}
	が得られる.ブルバキや島内では右辺の項で内包表記を導入しているため,
	$\forall u\, (\, u \in s \lrarrow A(s)\, )$を満たす集合$s$が取れなければ
	$\Set{x}{A(x)}$は正体不明の対象となる.一方で本稿では
	内包項の意味は$\varepsilon$項に依らずにはっきり決まっている.
	
\section{式の書き換え}
	$\Set{x}{A(x)}$なる形の項を内包項,$\varepsilon x A(x)$なる形の項を$\varepsilon$項と呼び,
	これらを類と総称することにする.
	また$\varepsilon$項が現れない$\mathcal{L}$の式を甲種式,
	乙種項が現れる$\mathcal{L}$の式を乙種式と呼ぶことにする.
	
	\begin{itembox}[l]{乙種式は書き換えない}
		たとえば,$x \in \varepsilon y B(y)$なる式を$\lang{\in}$の式に書き換えるならば,
		$\varepsilon$項に込められた意味から
		\begin{align}
			\exists t\, (\, x \in t \wedge 
			(\, \exists y B(y) \rarrow B(t)\, )\, )
		\end{align}
		とするのが妥当であるだろう.しかしこうすると集合論では
		\begin{align}
			\forall x\, (\, x \in \varepsilon y\, (\, y=y\, )\, )
		\end{align}
		が成り立ってしまい,これは矛盾を起こす.実際,任意の集合$x$に対して,$t$として$\{x\}$を取れば
		\begin{align}
			\exists t\, (\, x \in t \wedge 
			(\, \exists y B(y) \rarrow B(t)\, )\, )
		\end{align}
		が満たされるので
		\begin{align}
			\forall x\, \exists t\, (\, x \in t \wedge 
			(\, \exists y\, (\, y = y\, ) \rarrow t = t\, )\, )
		\end{align}
		すなわち$\forall x\, (\, x \in \varepsilon y\, (\, y=y\, )\, )$が成り立つ.
		ところが本稿の体系では$\varepsilon y\, (\, y = y\, )$は集合であり,
		その一方で全ての集合を要素に持つ集まりというのは集合ではないから,矛盾が起こる.
		
		他に乙種式を$\lang{\in}$の式に変換する有効な方法が見つかれば話は別だが,
		それが見つからないうちは乙種式は書き換えの対象ではない.
	\end{itembox}
	
	\begin{itemize}
		\item $x \in \Set{y}{B(y)}$は$B(x)$と書き換える.
			
			これは次の公理
			\begin{align}
				\forall x\, \left(\, x \in \Set{y}{B(y)} \leftrightarrow B(x)\, \right)
			\end{align}
			に基づく式の書き換えである.
			
		\item $\Set{x}{A(x)} \in y$は$\exists s\, \left(\, s \in y \wedge 
			\forall u\, (\, u \in s \lrarrow A(s)\, )\, \right)$
			と書き換える.
			これの同値性は
			\begin{align}
				a \in b \rarrow \exists x\, (\, a = x\, )
			\end{align}
			の公理による.
			
	\end{itemize}
	
	量化は$\varepsilon$項についての規則とする.甲種乙種関係なく,式$A(x)$と任意の$\varepsilon$項$\tau$に対して
	\begin{align}
		A(\tau) \vdash \exists x A(x).
	\end{align}
	
	$A(x)$が甲種式であるとき,
	\begin{align}
		\exists x A(x) \vdash A\left(\varepsilon x \mathcal{L}A(x)\right).
	\end{align}
	
	$A(x)$を式とするとき,次の推論規則によって,$\forall x A(x)$とは
	全ての$\varepsilon$項$\tau$で$A(\tau)$が成り立つことを意味するようになる.
	\begin{align}
		\forall x A(x) &\vdash A(\tau). \\
		A(\varepsilon x \rightharpoondown \mathcal{L}A(x)) &\vdash \forall x A(x). 
	\end{align}
	

\chapter{推論}
	\section{証明}
	閉式には,「真」であるか,「偽」であるか,のどちらかのラベルが付けられる.
	「真である」という言明は,「正しい」や「成り立つ」などとも言い換えられる.
	式が真であるか偽であるかは,次の手順に従って発見的に判明していく.
	
	\begin{itemize}
		\item $\Sigma$の閉式は真である.
		\item $A$と$\rightarrow AB$が真であると判明しているならば,$B$は真である.
		\item $\rightarrow \wedge ABA$と$\rightarrow \wedge ABB$は真である.
		\item $A$と$B$が真であると判明しているならば$\wedge AB$と$\wedge BA$は真である.
		\item $\rightarrow A\vee AB$と$\rightarrow B \vee AB$は真である.
		\item $\rightarrow AC$と$\rightarrow BC$が真であると判明しているならば
			$\rightarrow \vee ABC$は真である.
		\item $\rightarrow\wedge A \rightharpoondown A \bot$は真である.
		\item $\rightarrow \rightarrow A \bot \rightharpoondown A$は真である.
		\item $\rightarrow \rightharpoondown\rightharpoondown AA$は真である.
	\end{itemize}
	
	真であると判明している式$\varphi$を起点にして,
	上の推論規則を駆使して閉式$\psi$が真であると判明すれば,
	$\varphi$から始めて$\psi$が真であることに辿り着くまでの手続きは$\psi$の証明と呼ばれ,
	$\psi$は定理と呼ばれる.
	
	証明には真であると判明している式が必要であり,その大元として選ばれた式が$\Sigma$の式である.
	$\Sigma$の式は証明なしに真であると決められているのであり,これらを公理と呼び定理と区別する.
	
	与えられた閉式$\varphi$が証明可能であるとは,
	\begin{itemize}
		\item 閉式$\psi$で,$\psi$と$\psi \rightarrow \varphi$が真であると判明している者が得られる.
		\item 真であると判明している閉式$\psi$と$\xi$が得られて,$\varphi$は$\psi \wedge \xi$である.
		\item 閉式$\psi$と$\xi$で,$\psi \vee \xi$と$\psi \rightarrow \varphi$と$\xi \rightarrow \varphi$が真であると判明しているものが得られる.
	\end{itemize}
	
	のいずれかの場合であり,
	\begin{align}
		\vdash \varphi
	\end{align}
	と書く.
	
	証明された式が真なる式である.では真なる式は
	%\section{式の書き換え(没)}
	\begin{itemize}
		\item $x \in y$はそのまま$x \in y$
		\item $x \in \{y|B(y)\}$は$B(x)$
			
			これは公理である.つまり,
			\begin{align}
				\forall x\, \left(\, x \in \{y|B(y)\} \leftrightarrow B(x)\, \right).
			\end{align}
			
		\item $x \in \varepsilon y B(y)$は$\exists t\, \left(\, x \in t \wedge B(t)\, \right)$.ちなみにこれは公理とするべきか:
			\begin{align}
				\forall x\, \left(\, x \in \varepsilon y B(y) \leftrightarrow
				\exists t\, \left(\, x \in t \wedge B(t)\, \right)\, \right).
			\end{align}
			
		\item $\{x|A(x)\} \in y$は$\exists s\, \left(\, s \in y \wedge 
			\forall u\, \left(\, u \in s \leftrightarrow A(u)\, \right)\, \right)$
			
			実はこの両式は同値である.さていま
			\begin{align}
				\{x|A(x)\} \in y \leftrightarrow
				\exists s\, \left(\, s \in y \wedge 
				\forall u\, \left(\, u \in s \leftrightarrow A(u)\, \right)\, \right)
			\end{align}
			という式を$\varphi$とし,これを$\mathcal{L}_{\in}$の式に書き換えたものを$\hat{\varphi}$としよう.そして
			\begin{align}
				\eta = \varepsilon y \rightharpoondown \hat{\varphi}(y)
			\end{align}
			とおこう.ここで証明するのは
			\begin{align}
				\{x|A(x)\} \in \eta \leftrightarrow
				\exists s\, \left(\, s \in \eta \wedge 
				\forall u\, \left(\, u \in s \leftrightarrow A(u)\, \right)\, \right)
			\end{align}
			が成り立つということである.まず
			\begin{align}
				\{x|A(x)\} \in \eta
			\end{align}
			が成り立っているとしよう.すると
			\begin{align}
				\exists s\, \left(\, \{x|A(x)\} = s\, \right)
			\end{align}
			が成り立つのだが,今度も式の書き直し手順によって
			\begin{align}
				\exists s\, \left(\, \forall u\, \left(\, A(u) \leftrightarrow
				u \in s\, \right)\, \right)
			\end{align}
			と書き直される.
			\begin{align}
				\sigma = \varepsilon s\, \left(\, \forall u\, \left(\, A(u) \leftrightarrow
				u \in s\, \right)\, \right)
			\end{align}
			とおくと
			\begin{align}
				\forall u\, \left(\, A(u) \leftrightarrow
				u \in \sigma\, \right)
			\end{align}
			が成り立ち,他方で
			\begin{align}
				\sigma = \{x|A(x)\}
			\end{align}
			が成り立つのだから
			\begin{align}
				\sigma \in \eta
			\end{align}
			も従う.ゆえに
			\begin{align}
				\sigma \in \eta \wedge \forall u\, \left(\, A(u) \leftrightarrow
				u \in \sigma\, \right)
			\end{align}
			が成り立つ.逆に
			\begin{align}
				\exists s\, \left(\, s \in \eta \wedge 
				\forall u\, \left(\, u \in s \leftrightarrow A(u)\, \right)\, \right)
			\end{align}
			が成り立っているとして,
			\begin{align}
				\sigma = \varepsilon s\, \left(\, s \in \eta \wedge 
				\forall u\, \left(\, u \in s \leftrightarrow A(u)\, \right)\, \right)
			\end{align}
			としよう.すると
			\begin{align}
				\sigma \in \eta \wedge 
				\forall u\, \left(\, u \in \sigma \leftrightarrow A(u)\, \right)
			\end{align}
			が成り立つので
			\begin{align}
				\sigma \in \eta
			\end{align}
			かつ
			\begin{align}
				\sigma = \{x|A(x)\}
			\end{align}
			が成立する.ゆえに
			\begin{align}
				\{x|A(x)\} \in \eta
			\end{align}
			が成立する.以上で
			\begin{align}
				\{x|A(x)\} \in \eta \leftrightarrow
				\exists s\, \left(\, s \in \eta \wedge 
				\forall u\, \left(\, u \in s \leftrightarrow A(u)\, \right)\, \right)
			\end{align}
			が得られた.
			
		\item $\{x|A(x)\} \in \{y|B(y)\}$は$\exists s\, \left(\, B(s) \wedge 
			\forall u\, \left(\, u \in s \leftrightarrow A(u)\, \right)\, \right)$
		
		\item $\{x|A(x)\} \in \varepsilon y B(y)$は$\exists s,t\, \left(\, s \in t \wedge 
			\forall u\, \left(\, u \in s \leftrightarrow A(u)\, \right) \wedge B(t)\, \right)$
		
		\item $\varepsilon x A(x) \in y$は$\exists s\, \left(\, s \in y \wedge A(s)\, \right)$
			
			これも公理にしよう:
			\begin{align}
				\forall y\, \left(\, \varepsilon x A(x) \in y \leftrightarrow
				\exists s\, \left(\, s \in y \wedge A(s)\, \right)\, \right).
			\end{align}
			いや,$y$をクラスとした言明の方が良いかも.
			\begin{align}
				\varepsilon x A(x) \in y \leftrightarrow
				\exists s\, \left(\, s \in y \wedge A(s)\, \right).
			\end{align}
		
		\item $\varepsilon x A(x) \in \{y|B(y)\}$は$\exists s\, \left(\, A(s) \wedge B(s)\, \right)$
			
			上の公理からこの式の同値性も導かれます.まず
			\begin{align}
				\exists s\, \left(\, A(s) \wedge B(s)\, \right)
			\end{align}
			が成り立っているとしよう.そして
			\begin{align}
				\sigma = \varepsilon s\, \left(\, A(s) \wedge B(s)\, \right)
			\end{align}
			とおくと,
			\begin{align}
				A(\sigma) \wedge B(\sigma)
			\end{align}
			が成立する.ゆえに
			\begin{align}
				\sigma \in \{y|B(y)\} \wedge A(\sigma)
			\end{align}
			が成立する.ゆえに
			\begin{align}
				\exists s\, \left(\, s \in \{y|B(y)\} \wedge A(s)\, \right)
			\end{align}
			が成り立つ.ゆえに
			\begin{align}
				\varepsilon x A(x) \in \{y|B(y)\}
			\end{align}
			が成り立つ.逆に
			\begin{align}
				\varepsilon x A(x) \in \{y|B(y)\}
			\end{align}
			が成り立っているとしよう.すると
			\begin{align}
				\exists s\, \left(\, s = \varepsilon x A(x)\, \right)
			\end{align}
			が成り立つが,これは$\mathcal{L}_{\in}$の式で
			\begin{align}
				\exists s A(s)
			\end{align}
			であって,
			\begin{align}
				\sigma = \varepsilon s A(s)
			\end{align}
			とおけば
			\begin{align}
				A(\sigma)
			\end{align}
			が成立する.ところで
			\begin{align}
				\sigma \in \{y|B(y)\}
			\end{align}
			なので
			\begin{align}
				B(\sigma)
			\end{align}
			も成り立つ.ゆえに
			\begin{align}
				A(\sigma) \wedge B(\sigma)
			\end{align}
			が成り立つ.ゆえに
			\begin{align}
				\exists s\, \left(\, A(s) \wedge B(s)\, \right)
			\end{align}
			が成り立つ.
			
		\item $\varepsilon x A(x) \in \varepsilon y B(y)$は$\exists s,t\, \left(\, s \in t \wedge A(s) \wedge B(t)\, \right)$
		
			この式の同値性も証明できる.まず
			\begin{align}
				\exists s,t\, \left(\, s \in t \wedge A(s) \wedge B(t)\, \right)
			\end{align}
			が成り立っているとしよう.この式は
			\begin{align}
				\exists s\, \left(\, \exists t\, \left(\, s \in t \wedge A(s) \wedge B(t)\, \right)\, \right)
			\end{align}
			の略記であって,$\exists$の規則より
			\begin{align}
				\sigma = \varepsilon s\, \left(\, \exists t\, \left(\, s \in t \wedge A(s) \wedge B(t)\, \right)\, \right)
			\end{align}
			とおけば
			\begin{align}
				\exists t\, \left(\, \sigma \in t \wedge A(\sigma) \wedge B(t)\, \right)
			\end{align}
			が成立する.$\sigma \in t \wedge A(\sigma) \wedge B(t)$を$\mathcal{L}_{\in}$の式に書き直したものを
			$\varphi(t)$として
			\begin{align}
				\tau \defeq \varepsilon t \varphi(t)
			\end{align}
			とおけば,$\exists$の規則より
			\begin{align}
				\sigma \in \tau \wedge A(\sigma) \wedge B(\tau)
			\end{align}
			が成立する.ゆえに
			\begin{align}
				\exists s\, \left(\, s \in \tau \wedge A(s)\, \right)
			\end{align}
			が成り立つから,公理より
			\begin{align}
				\varepsilon x A(x) \in \tau
			\end{align}
			が成立する.ゆえに
			\begin{align}
				\varepsilon x A(x) \in \tau \wedge B(\tau)
			\end{align}
			が成立する.ゆえに
			\begin{align}
				\exists t\, \left(\, \varepsilon x A(x) \in t \wedge B(t)\, \right)
			\end{align}
			が成立する.公理より
			\begin{align}
				\varepsilon x A(x) \in \varepsilon y B(y)
			\end{align}
			が成立する.逆は容易い.
			\begin{align}
				\varepsilon x A(x) \in \varepsilon y B(y)
			\end{align}
			が成り立っているとすれば公理より
			\begin{align}
				\exists t\, \left(\, \varepsilon x A(x) \in t \wedge B(t)\, \right)
			\end{align}
			が成立する.$\varepsilon x A(x) \in t \wedge B(t)$を$\mathcal{L}_{\in}$の式に書き直したものを$\psi(t)$として
			\begin{align}
				\tau \defeq \varepsilon t \psi(t)
			\end{align}
			とおけば
			\begin{align}
				\varepsilon x A(x) \in \tau \wedge B(\tau)
			\end{align}
			が成立するが,ここで公理より
			\begin{align}
				\exists s\, \left(\, s \in \tau \wedge A(s)\, \right)
			\end{align}
			が成り立つので,$s \in \tau \wedge A(s)$を$\mathcal{L}_{\in}$の式に書き直したものを$\xi(s)$として
			\begin{align}
				\sigma \defeq \varepsilon s \xi(s)
			\end{align}
			とおけば
			\begin{align}
				\sigma \in \tau \wedge A(\sigma)
			\end{align}
			が成立する.以上より
			\begin{align}
				\sigma \in \tau \wedge A(\sigma) \wedge B(\tau)
			\end{align}
			が成立する.ゆえに
			\begin{align}
				\exists t\, \left(\, \sigma \in t \wedge A(\sigma) \wedge B(t)\, \right)
			\end{align}
			が得られる.ゆえに
			\begin{align}
				\exists s\, \left(\, \exists t\, \left(\, \sigma \in t \wedge A(\sigma) \wedge B(t)\, \right)\, \right)
			\end{align}
			が得られる.
	\end{itemize}
	
	\begin{screen}
		\begin{logicalaxm}\mbox{}
			\begin{itemize}
				\item 任意の閉項$\tau$に対して,$A(\tau)$が定理ならば$\exists x A(x)$が成り立つ.
				\item $\exists x A(x)$が定理ならば,$A(\varepsilon x \hat{A}(x))$が成り立つ.
				\item すべての閉項$\tau$に対して$A(\tau)$が定理ならば,$\forall x A(x)$が成り立つ.
				\item $\forall x A(x)$が定理ならば,すべての閉項$\tau$に対して$A(\tau)$が成り立つ.
			\end{itemize}
		\end{logicalaxm}
	\end{screen}
	
	定理として
	\begin{align}
		\forall x A(x) \Longleftrightarrow A(\varepsilon x \rightharpoondown \hat{A}(x))
	\end{align}
	が得られる.アイデアとしてはさあ,$\varepsilon x A(x)$の全体が集合に対応しているのであって,
	いやもちろん集合そのものではないけど,集合は$\varepsilon x A(x)$のどれかに等しい類なわけで,
	だからモデル論に出てくる「宇宙」とかいう得体の知れない集合()は俺の集合論に不要なんだよね.
	俺のノートの「宇宙」はすべて実態が把握できるように,具体的な記号列で書き表せるのが良いよね.
	ちなみにこの「宇宙」は$\{x|x=x\}$とは別ね.
	
	\begin{screen}
		\begin{axm}
			\begin{align}
				\forall x\, \left(\, x \in \{y|B(y)\} \leftrightarrow B(x)\, \right).
			\end{align}
			
			\begin{align}
				\forall x\, \left(\, x \in \varepsilon y B(y) \leftrightarrow
				\left(\, \exists t\, B(t) \rightarrow 
				\exists t\, \left(\, x \in t \wedge B(t)\, \right)\, \right)\, \right).
			\end{align}
			が定理となるために
			\begin{align}
				x \in \varepsilon y B(y) \leftrightarrow
				\exists t\, \left(\, x \in t \wedge B(t) \leftrightarrow \exists y B(y)\, \right)
			\end{align}
			を公理とする.
			
			\begin{align}
				\varepsilon x A(x) \in y \leftrightarrow
				\exists s\, \left(\, s \in y \wedge A(s)\, \right).
			\end{align}
		\end{axm}
	\end{screen}
	
	いや,した二つは公理じゃねえな.定理だ.実際
	\begin{align}
		x \in \varepsilon y B(y)
	\end{align}
	が成り立っているとしよう.$\varepsilon y B(y)$は集合であって
	\begin{align}
		\exists s\, \left(\, s = \varepsilon y B(y)\, \right)
	\end{align}
	が成り立つので,
	\begin{align}
		fff
	\end{align}
	
	\begin{screen}
		\begin{thm}
			\begin{align}
				\exists x A(x) \rightarrow \varepsilon x A(x) \in \{x|A(x)\}.
			\end{align}
		\end{thm}
	\end{screen}
	
	\begin{sketch}
		\begin{align}
			\exists x A(x)
		\end{align}
		が成り立ているとするとき,
		\begin{align}
			\sigma \defeq \varepsilon x A(x)
		\end{align}
		とおけば
		\begin{align}
			A(\sigma)
		\end{align}
		が成り立つので,公理より
		\begin{align}
			\sigma \in \{x|A(x)\}
		\end{align}
		が成立する.ゆえに
		\begin{align}
			\sigma \in \{x|A(x)\} \wedge A(\sigma)
		\end{align}
		が成り立つ.ゆえに
		\begin{align}
			\exists s\, \left(\, s \in \{x|A(x)\} \wedge A(s)\, \right)
		\end{align}
		が成立する.ゆえに公理より
		\begin{align}
			\varepsilon x A(x) \in \{x|A(x)\}
		\end{align}
		が成立する.
		\QED
	\end{sketch}
	
	\begin{itembox}[l]{満たされて欲しいこと}
		\begin{description}
			\item[等号]
				\begin{itemize}
					\item $x = \{y|B(y)\}$と$\forall s\, \left(\, s \in x \leftrightarrow B(s)\, \right)$
					\item $x = \varepsilon y B(y)$と$\exists s\, \left(\, A(s) \wedge \forall u\,
						\left(\, u \in x \leftrightarrow u \in s\, \right)\, \right)$
					\item $\{x|A(x)\} = \{y|B(y)\}$と$\forall s\, \left(\, A(s) \leftrightarrow B(s)\, \right)$
					\item $\{x|A(x)\} = \varepsilon y B(y)$と
						$\exists s\, \left(\, \forall u\,
						\left(\, u \in s \leftrightarrow A(u)\, \right) \wedge B(s)\, \right)$
					\item $\varepsilon x A(x) = \varepsilon y B(y)$と
						$\exists s,t\, \left(\, s = t \wedge A(s) \wedge B(t)\, \right)$
				\end{itemize}
				
			\item[帰属]
				\begin{itemize}
					\item $x \in \{y|B(y)\}$は$B(x)$
					\item $x \in \varepsilon y B(y)$は$\exists t\, \left(\, x \in t \wedge B(t)\, \right)$
					\item $\{x|A(x)\} \in y$は$\exists s\, \left(\, s \in y \wedge \forall u\, \left(\, u \in s \leftrightarrow A(u)\, \right)\, \right)$
					\item $\{x|A(x)\} \in \{y|B(y)\}$は$\exists s\, \left(\, B(s) \wedge \forall u\, \left(\, u \in s \leftrightarrow A(u)\, \right)\, \right)$
					\item $\{x|A(x)\} \in \varepsilon y B(y)$は$\exists s,t\, \left(\, s \in t \wedge \forall u\, \left(\, u \in s \leftrightarrow A(u)\, \right) \wedge B(t)\, \right)$
					\item $\varepsilon x A(x) \in y$は$\exists s\, \left(\, s \in y \wedge A(s)\, \right)$
					\item $\varepsilon x A(x) \in \{y|B(y)\}$は$\exists s\, \left(\, A(s) \wedge B(s)\, \right)$
					\item $\varepsilon x A(x) \in \varepsilon y B(y)$は$\exists s,t\, \left(\, s \in t \wedge A(s) \wedge B(t)\, \right)$
				\end{itemize}
		\end{description}
	\end{itembox}

\section{定理I\hspace{-.1em}I.15.2}
	\begin{description}
		\item[(3)]
			項$\tau$が変項$x$のとき,$\zeta_{\tau}(y)$を
			\begin{align}
				x = y
			\end{align}
			とすれば,
			\begin{align}
				\Sigma' \vdash \forall x\, \exists! y\, (\, x=y\, )
			\end{align}
			つまり
			\begin{align}
				\Sigma' \vdash \forall x\, \exists! y\, \zeta_{\tau}(y)
			\end{align}
			および
			\begin{align}
				\Sigma \vdash \forall x\, (\, x=x\, )
			\end{align}
			つまり
			\begin{align}
				\Sigma \vdash \forall x\, \zeta_{\tau}(\tau)
			\end{align}
			が成り立つ.項$\tau$が
			\begin{align}
				f\tau_{1}\cdots\tau_{n}
			\end{align}
			のとき,$f \in \mathcal{L} \backslash \mathcal{L}_{\in}$ならば$\zeta_{\tau}(y)$を
			\begin{align}
				\exists z_{1}, \cdots, z_{n}\, \left(\, 
				\theta_{f}(z_{1},\cdots,z_{n},y) \wedge \zeta_{\tau_{1}}(z_{1}) \wedge
				\cdots \wedge \zeta_{\tau_{n}}(z_{n})\, \right)
			\end{align}
			とし,$f \in \mathcal{L}_{\in}$ならば
			\begin{align}
				\exists z_{1}, \cdots, z_{n}\, \left(\, 
				f(z_{1},\cdots,z_{n}) = y \wedge \zeta_{\tau_{1}}(z_{1}) \wedge
				\cdots \wedge \zeta_{\tau_{n}}(z_{n})\, \right)
			\end{align}
			とする.仮定より
			\begin{align}
				\Sigma \vdash \exists! y\, \zeta_{\tau_{i}}(y)
			\end{align}
			が成り立つので,
			\begin{align}
				\Sigma \vdash \zeta_{\tau_{i}}(z_{i})
			\end{align}
			を満たす$z_{i}$が取れる.そして定義I\hspace{-.1em}I.15.1より
			\begin{align}
				\Sigma \vdash \exists!y\, \theta_{f}(z_{1},\cdots,z_{n},y)
			\end{align}
			が成り立つので,その$y$を取れば
			\begin{align}
				\Sigma \vdash \exists y\, \zeta_{\tau}(y)
			\end{align}
			が成立する.ただし,$\eta$を
			\begin{align}
				\Sigma \vdash \exists z_{1}, \cdots, z_{n}\, \left(\, 
				\theta_{f}(z_{1},\cdots,z_{n},\eta) \wedge \zeta_{\tau_{1}}(z_{1}) \wedge
				\cdots \wedge \zeta_{\tau_{n}}(z_{n})\, \right)
			\end{align}
			を満たすものとすれば,このとき
			\begin{align}
				\Sigma \vdash \zeta_{\tau_{i}}(w_{i})
			\end{align}
			および
			\begin{align}
				\Sigma \vdash \theta_{f}(w_{1},\cdots,w_{n},\eta)
			\end{align}
			を満たす$w_{i}$が取れるが,$z_{i} = w_{i}$なので
			\begin{align}
				\Sigma \vdash \theta_{f}(z_{1},\cdots,z_{n},\eta)
			\end{align}
			が成り立つことになって,定義I\hspace{-.1em}I.15.1より
			\begin{align}
				y = \eta
			\end{align}
			が成り立つ.ゆえに
			\begin{align}
				\Sigma \vdash \exists! y\, \zeta_{\tau}(y)
			\end{align}
			が成立する.他方で仮定より
			\begin{align}
				\Sigma' \vdash \zeta_{\tau_{i}}(\tau_{i})
			\end{align}
			が成り立ち,かつ定義I\hspace{-.1em}I.15.1より
			\begin{align}
				\Sigma' \vdash \forall x_{1},\cdots,x_{n}\,
				\theta_{f}(x_{1},\cdots,x_{n},f(x_{1},\cdots,x_{n}))
			\end{align}
			が成り立つので
			\begin{align}
				\Sigma' \vdash \theta_{f}(\tau_{1},\cdots,\tau_{n},f(\tau_{1},\cdots,\tau_{n}))
			\end{align}
			が成り立つ.ゆえに
			\begin{align}
				\Sigma' \vdash \theta_{f}(\tau_{1},\cdots,\tau_{n},f(\tau_{1},\cdots,\tau_{n})) \wedge \zeta_{\tau_{1}}(\tau_{1}) \wedge \cdots \wedge \zeta_{\tau_{n}}(\tau_{n})
			\end{align}
			が成り立つ.ゆえに
			\begin{align}
				\Sigma' \vdash \zeta_{\tau}(\tau)
			\end{align}
			が成り立つ.
			
		\item[(2)]
			$\varphi$を$p\tau_{1}\cdots\tau_{n}$なる原子式とするとき,
			$p$が$\mathcal{L} \backslash \mathcal{L}_{\in}$の要素ならば$\hat{\varphi}$を
			\begin{align}
				\exists z_{1},\cdots,z_{n}\, 
				\left(\, \theta_{p}z_{1} \cdots z_{n} \wedge \zeta_{\tau_{1}}(z_{1})
				\wedge \cdots \wedge \zeta_{\tau_{n}}(z_{n})\, \right)
			\end{align}
			とし,$p$が$\mathcal{L}_{\in}$の要素ならば
			\begin{align}
				\exists z_{1},\cdots,z_{n}\, 
				\left(\, pz_{1} \cdots z_{n} \wedge \zeta_{\tau_{1}}(z_{1})
				\wedge \cdots \wedge \zeta_{\tau_{n}}(z_{n})\, \right)
			\end{align}
			とする.$\varphi$が成り立っているとき,仮定より
			\begin{align}
				\Sigma' \vdash \zeta_{\tau_{i}}(\tau_{i})
			\end{align}
			が満たされ,また定義I\hspace{-.1em}I.15.1より($\Delta$は$\Sigma'$に含まれているので)
			\begin{align}
				\Sigma' \cup \{\varphi\} \vdash \theta_{p}\tau_{1} \cdots \tau_{n}
			\end{align}
			も満たされているので
			\begin{align}
				\Sigma' \cup \{\varphi\} \vdash \theta_{p}\tau_{1} \cdots \tau_{n}
				\wedge \zeta_{\tau_{1}}(\tau_{1}) \wedge \cdots \wedge 
				\zeta_{\tau_{n}}(\tau_{n})
			\end{align}
			が成り立つ.すなわち
			\begin{align}
				\Sigma' \cup \{\varphi\} \vdash \hat{\varphi}
			\end{align}
			が成り立つ.つまり
			\begin{align}
				\Sigma' \vdash \varphi \rightarrow \hat{\varphi}
			\end{align}
			が成り立つ.逆に$\hat{\varphi}$が成り立っているとき,
			\begin{align}
				\Sigma' \cup \{\hat{\varphi}\} \vdash \theta_{p}w_{1} \cdots w_{n}
				\wedge \zeta_{\tau_{1}}(w_{1}) \wedge \cdots \wedge 
				\zeta_{\tau_{n}}(w_{n})
			\end{align}
			を満たす$w_{1},\cdots,w_{n}$が取れるが,
			\begin{align}
				\Sigma \vdash \exists! y\, \zeta_{\tau_{i}}(y)
			\end{align}
			かつ
			\begin{align}
				\Sigma' \vdash \zeta_{\tau_{i}}(\tau_{i})
			\end{align}
			なので
			\begin{align}
				w_{i} = \tau_{i}
			\end{align}
			である.ゆえに
			\begin{align}
				\Sigma' \cup \{\hat{\varphi}\} \vdash \theta_{p}\tau_{1} \cdots \tau_{n}
			\end{align}
			が成り立つ.ゆえに
			\begin{align}
				\Sigma' \vdash \hat{\varphi} \rightarrow \varphi
			\end{align}
			が得られる.
			\QED
	\end{description}
	
	\begin{screen}
		$\forall x\, (\, x \notin y\, )$を$\theta_{\emptyset}(y)$とするとき.
	\end{screen}
	
	$\emptyset$の定義$\delta_{\emptyset}$は
	\begin{align}
		\forall x\, (\, x \notin \emptyset\, )
	\end{align}
	である.$\zeta_{\emptyset}(y)$は
	\begin{align}
		\forall x\, (\, x \notin y\, )
	\end{align}
	であって,
	\begin{align}
		\Sigma \vdash \exists! y\, \zeta_{\emptyset}(y)
	\end{align}
	が成り立ち,また$\delta_{\emptyset}$と$\zeta_{\emptyset}(1)$は同じなので
	\begin{align}
		\Sigma \cup \{\delta_{\emptyset}\} = \Sigma' \vdash \zeta_{\emptyset}(\emptyset)
	\end{align}
	が成り立つ.そして,例えば$z$を変項とすれば
	\begin{align}
		z \in \emptyset
	\end{align}
	と
	\begin{align}
		\exists s,t\, \left(\, s \in t \wedge s = z \wedge \forall x\, (\, x \notin t\, )\, \right)
	\end{align}
	が$\Sigma'$の下で同値になる.
	
	\begin{screen}
		$\forall x\, (\, x \cdot y = y \cdot x = x\, )$を$\theta_{1}(y)$とするとき,
	\end{screen}
	
	$1$の定義$\delta_{1}$は
	\begin{align}
		\forall x\, (\, x \cdot 1 = 1 \cdot x = x\, )
	\end{align}
	である.$\zeta_{1}(y)$は
	\begin{align}
		\forall x\, (\, x \cdot y = y \cdot x = x\, )
	\end{align}
	であって,
	\begin{align}
		\Sigma \vdash \exists! y\, \zeta_{1}(y)
	\end{align}
	が成り立ち,また$\delta_{1}$と$\zeta_{1}(1)$は同じなので
	\begin{align}
		\Sigma \cup \{\delta_{1}\} = \Sigma' \vdash \zeta_{1}(1)
	\end{align}
	が成り立つ.そして,例えば$z$を変項とすれば
	\begin{align}
		z \in 1
	\end{align}
	と
	\begin{align}
		\exists s,t\, \left(\, s \in t \wedge s = z \wedge \forall x\, (\, x \cdot t = t \cdot x = x\, )\, \right)
	\end{align}
	が$\Sigma'$の下で同値になる.
	
	\begin{screen}
		$y \cdot (x \cdot x) = x$を$\theta_{i}(x,y)$とするとき,
	\end{screen}
	
	$i$の定義$\delta_{i}$は
	\begin{align}
		\forall x\, \left(\, i(x) \cdot (x \cdot x) = x\, \right)
	\end{align}
	である.$\zeta_{i(x)}(y)$は
	\begin{align}
		\exists x\, \left(\, \theta_{i}(x,y) \wedge x = x\, \right)
	\end{align}
	である.そして,例えば$x,z$を変項とすれば
	\begin{align}
		z \in i(x)
	\end{align}
	と
	\begin{align}
		\exists s,t\, \left(\, s \in t \wedge s = z \wedge \zeta_{i(x)}(t)\, \right)
	\end{align}
	が$\Sigma'$の下で同値になる.
	
\section{菊池誠不完全性定理}
	言語とは定数記号と関数記号と関係記号の全体ということで,
	\begin{itemize}
		\item $\mathcal{L}_{\in}$とは$\{\in,\natural\}$.
		\item $\mathcal{L}$とは$\{\in\}$に加えて閉項の全体.
	\end{itemize}
	\section{推論}
	本節では,「集合でも真類でもない類は存在しない」と「集合であり真類でもある類は存在しない」の二つの言明の正否の決定を主軸にして
	{\bf 推論規則}\index{すいろんきそく@推論規則}{\bf (rule of inference)}を導入し,基本的な推論法則を導出する.
	
	\begin{screen}
		\begin{logicalaxm}[排中律]
			$A$を任意の文とするとき次は定理である:
			\begin{align}
				A \vee \rightharpoondown A.
			\end{align}
		\end{logicalaxm}
	\end{screen}
	
	排中律の言明は``どんな文でも持ってくれば,その式に対して排中律が適用される''という意味である.
	このように無数に存在し得る定理を一括して表す式は{\bf 公理図式}\index{こうりずしき@公理図式}{\bf (schema)}と呼ばれる.
	
	いま$a,b$を類とするとき,
	\begin{align}
		a \notin b \defarrow\ \rightharpoondown a \in b
	\end{align}
	で$a \notin b$を定める.同様に
	\begin{align}
		a \neq b \defarrow\ \rightharpoondown a = b
	\end{align}
	で$a \neq b$を定める.
	
	\monologue{
		定義記号$\defeq$と同様に,`$A \defarrow B$'とは
		式$B$を記号列$A$で置き換えて良いという意味で使われます.また,式中に記号列$A$が出てくるときは,
		暗黙裡にその$A$を$B$に戻して式を解釈します.
		$\defeq$も$\defarrow$も略記することと同じですね.
	}
	
	\begin{screen}
		\begin{thm}[類は集合であるか真類であるかのいずれかに定まる]
			$a$を類とするとき次は定理である:
			\begin{align}
				\set{a} \vee \rightharpoondown \set{a}.
			\end{align}
		\end{thm}
	\end{screen}
	
	\begin{prf}
		排中律を適用することにより従う.
		\QED
	\end{prf}
	
	排中律をそのまま適用することにより上の定理は導かれたが,``集合であり真類でもある類は存在しない''という主張はまだ得られない.
	以下はこの言明を証明することを目標にしてしばらく推論規則の話が続くが,提示される規則はどれも基本的で直感に反しないため
	通常は無断で使用されてしまうものである.
	
	ここで論理記号の名称を書いておく.
	\begin{itemize}
		\item $\bot$を{\bf 矛盾}\index{むじゅん@矛盾}{\bf (contradiction)}と呼ぶ.
		\item $\vee$を{\bf 論理和}\index{ろんりわ@論理和}{\bf (logical disjunction)}と呼ぶ.
		\item $\wedge$を{\bf 論理積}\index{ろんりせき@論理積}{\bf (logical conjunction)}と呼ぶ.
		\item $\Longrightarrow$を{\bf 含意}\index{がんい@含意}{\bf (implication)}と呼ぶ.
		\item $\rightharpoondown$を{\bf 否定}\index{ひてい@否定}{\bf (negation)}と呼ぶ.
	\end{itemize}
	
	\begin{screen}
		\begin{logicalaxm}[基本的な推論規則]\label{logicalaxm:fundamental_rules_of_inference}
			$A,B,C$を$\mathcal{L}'$の閉式とするとき,次の規則を認める:
			\begin{description}
				\item[三段論法] $A$ならびに$A \Longrightarrow B$が定理なら$B$は定理である.
				\item[演繹法則] $A$を公理に追加した下で$B$が定理であるなら,
					$A$を外した公理系で$A \Longrightarrow B$は定理である.
				\item[論理和の導入イ] $A \Longrightarrow (A \vee B)$は定理である.
				\item[論理和の導入ロ] $A \Longrightarrow (B \vee A)$は定理である.
				\item[論理積の導入] $A,B$が共に定理なら$A \wedge B$は定理である.
				\item[論理積の除去イ] $(A \wedge B) \Longrightarrow A$は定理である.
				\item[論理積の除去ロ] $(A \wedge B) \Longrightarrow B$は定理である.
				\item[場合分け法則] $A \Longrightarrow C$と$B \Longrightarrow C$が共に定理であるとき
					$(A \vee B) \Longrightarrow C$は定理である.
			\end{description}	
		\end{logicalaxm}
	\end{screen}
	
	\monologue{
		演繹法則について,``$A$を公理に追加する''ことを``$A$が成り立っていると仮定する''
		などの言明により示唆することが多いです.
	}
	
	\begin{itembox}[l]{演繹法則の意味}
		我々は公理か,或いは公理図式として,複数の式を選び出し$\mathcal{L}'$の世界において正しいと決める.
		それらは以降小出しに登場させるが,その全体は現段階ですでに決めているのでそれを
		\begin{align}
			\mathscr{S}
		\end{align}
		と呼ぶことにする.本稿で出てくる``正しい式''とは$\mathscr{S}$のみを公理系とした体系において証明される式を指す.
		演繹法則は,``$A$が成り立つとする''などの
		言明により$\mathscr{S}$に式$A$を加えたとき,その新しい公理系$\mathscr{S}'$の下で式$B$が成り立つなら,
		$\mathscr{S}$のみを公理とした体系において
		\begin{align}
			A \Longrightarrow B
		\end{align}
		が成立する,と主張している.複数の式を$\mathscr{S}$に追加する場合もある.
		たとえば$\mathscr{S}'$に式$C$を追加し,その新しい公理体系$\mathscr{S}''$の下で
		式$D$が成り立つ場合,演繹法則に則れば$\mathscr{S}'$の下で
		\begin{align}
			C \Longrightarrow D
		\end{align}
		が成立する.(ちなみに$\mathscr{S}''$から式$A$のみを抜いた公理系の下では
		$A \Longrightarrow D$が正しくなる.)
		このとき$\mathscr{S}$を公理系とした下では,再び演繹法則を適用することにより
		\begin{align}
			A \Longrightarrow (C \Longrightarrow D)
		\end{align}
		が成立するとわかる,が,
		\begin{align}
			C \Longrightarrow D
		\end{align}
		が成り立つ保証は無い.非常に屡々いくつも仮定を重ねたところに演繹法則を運用することがあるが,
		その都度どの段階の公理系を扱っているかを明確に把握しておかないと推論が破綻してしまう恐れがある.
	\end{itembox}
	
	\begin{screen}
		\begin{logicalthm}[含意の反射律]\label{logicalthm:reflective_law_of_implication}
			$A$を文とするとき
			\begin{align}
				\vdash A \Longrightarrow A.
			\end{align}
		\end{logicalthm}
	\end{screen}
	
	\begin{prf}
		$A \vdash A$であるから,演繹法則より$\vdash A \Longrightarrow A$となる.
		\QED
	\end{prf}
	
	\begin{screen}
		\begin{logicalthm}[論理和・論理積の可換律]
		\label{logicalthm:commutative_law_of_disjunction_and_conjunction}
			$A,B$を文とするとき
			\begin{itemize}
				\item $\vdash (A \vee B) \Longrightarrow (B \vee A)$.
				\item $\vdash (A \wedge B) \Longrightarrow (B \wedge A)$.
			\end{itemize}
		\end{logicalthm}
	\end{screen}
	
	\begin{prf}
		$\vee$の導入により
		\begin{align}
			\vdash A \Longrightarrow (B \vee A)
		\end{align}
		と
		\begin{align}
			\vdash B \Longrightarrow (A \vee B)
		\end{align}
		が成り立つので,場合分け法則より
		\begin{align}
			\vdash (A \vee B) \Longrightarrow (B \vee A)
		\end{align}
		が成り立つ.また,$\wedge$の除去より
		\begin{align}
			A \wedge B \vdash A
		\end{align}
		と
		\begin{align}
			A \wedge B \vdash B
		\end{align}
		となるので,$\wedge$の導入により
		\begin{align}
			A \wedge B \vdash B \wedge A
		\end{align}
		が成り立つ.よって演繹法則より
		\begin{align}
			\vdash (A \wedge B) \Longrightarrow (B \wedge A)
		\end{align}
		が成り立つ.
		\QED
	\end{prf}
	
	\begin{screen}
		\begin{logicalthm}[含意の推移律]\label{logicalthm:transitive_law_of_implication}
			$A,B,C$を文とするとき
			\begin{align}
				\vdash ((A \Longrightarrow B) \wedge (B \Longrightarrow C)) 
				\Longrightarrow (A \Longrightarrow C).
			\end{align}
		\end{logicalthm}
	\end{screen}
	
	\begin{prf}
		\begin{align}
			(A \Longrightarrow B) \wedge (B \Longrightarrow C),A \vdash 
			(A \Longrightarrow B) \wedge (B \Longrightarrow C)
		\end{align}
		であるから,$\wedge$の除去より
		\begin{align}
			(A \Longrightarrow B) \wedge (B \Longrightarrow C),A \vdash A \Longrightarrow B
		\end{align}
		となる.また
		\begin{align}
			(A \Longrightarrow B) \wedge (B \Longrightarrow C),A \vdash A
		\end{align}
		でもあるから,三段論法より
		\begin{align}
			(A \Longrightarrow B) \wedge (B \Longrightarrow C),A \vdash B
		\end{align}
		となる.$\wedge$の除去より
		\begin{align}
			(A \Longrightarrow B) \wedge (B \Longrightarrow C),A \vdash B \Longrightarrow C
		\end{align}
		も成り立つから,再び三段論法より
		\begin{align}
			(A \Longrightarrow B) \wedge (B \Longrightarrow C),A \vdash C
		\end{align}
		となる.よって演繹法則より
		\begin{align}
			(A \Longrightarrow B) \wedge (B \Longrightarrow C) \vdash A \Longrightarrow C
		\end{align}
		となり,
		\begin{align}
			\vdash ((A \Longrightarrow B) \wedge (B \Longrightarrow C)) 
			\Longrightarrow (A \Longrightarrow C)
		\end{align}
		を得る.
		\QED
	\end{prf}
	
	\begin{screen}
		\begin{logicalthm}[二式が同時に導かれるならその論理積が導かれる]
		\label{logicalthm:conjunction_of_consequences}
			$A,B,C$を文とするとき
			\begin{align}
				\vdash ((A \Longrightarrow B) \wedge (A \Longrightarrow C))
				\Longrightarrow (A \Longrightarrow (B \wedge C))
			\end{align}
		\end{logicalthm}
	\end{screen}
	
	\begin{prf}
		\begin{align}
			(A \Longrightarrow B) \wedge (A \Longrightarrow C),A \vdash
			(A \Longrightarrow B) \wedge (A \Longrightarrow C)
		\end{align}
		であるから,$\wedge$の除去より
		\begin{align}
			(A \Longrightarrow B) \wedge (A \Longrightarrow C),A \vdash
			A \Longrightarrow B
		\end{align}
		が成り立つ.
		\begin{align}
			(A \Longrightarrow B) \wedge (A \Longrightarrow C),A \vdash A
		\end{align}
		でもあるから
		\begin{align}
			(A \Longrightarrow B) \wedge (A \Longrightarrow C),A \vdash B
		\end{align}
		となる.同様にして
		\begin{align}
			(A \Longrightarrow B) \wedge (A \Longrightarrow C),A \vdash C
		\end{align}
		となるので,$\wedge$の導入により
		\begin{align}
			(A \Longrightarrow B) \wedge (A \Longrightarrow C),A \vdash B \wedge C
		\end{align}
		となり,演繹法則より
		\begin{align}
			(A \Longrightarrow B) \wedge (A \Longrightarrow C) \vdash
			A \Longrightarrow (B \wedge C)
		\end{align}
		が成り立つ.ゆえに
		\begin{align}
			\vdash ((A \Longrightarrow B) \wedge (A \Longrightarrow C))
			\Longrightarrow (A \Longrightarrow (B \wedge C))
		\end{align}
		が得られる.
		\QED
	\end{prf}
	
	\begin{screen}
		\begin{logicalthm}[含意は遺伝する]\label{logicalthm:rule_of_inference_1}
			$A,B,C$を$\mathcal{L}'$の閉式とするとき以下が成り立つ:
			\begin{description}
				\item[(a)] $(A \Longrightarrow B) \Longrightarrow ( (A \vee C) \Longrightarrow (B \vee C) )$.
				
				\item[(b)] $(A \Longrightarrow B) \Longrightarrow ( (A \wedge C) \Longrightarrow (B \wedge C) )$.
				
				\item[(c)] $(A \Longrightarrow B) \Longrightarrow ( (B \Longrightarrow C) \Longrightarrow (A \Longrightarrow C) )$.
				
				\item[(c)] $(A \Longrightarrow B) \Longrightarrow ( (C \Longrightarrow A) \Longrightarrow (C \Longrightarrow B) )$.
			\end{description}
		\end{logicalthm}
	\end{screen}
	
	\begin{prf}\mbox{}
		\begin{description}
			\item[(a)]
				いま$A \Longrightarrow B$が成り立っていると仮定する.
				論理和の導入により
				\begin{align}
					C \Longrightarrow (B \vee C)
				\end{align}
				は定理であるから,含意の推移律より
				\begin{align}
					A \Longrightarrow (B \vee C)
				\end{align}
				が従い,場合分け法則より
				\begin{align}
					(A \vee C) \Longrightarrow (B \vee C)
				\end{align}
				が成立する.ここに演繹法則を適用して
				\begin{align}
					(A \Longrightarrow B) \Longrightarrow 
					( (A \vee C) \Longrightarrow (B \vee C) )
				\end{align}
				が得られる.
				
			\item[(b)]
				いま$A \Longrightarrow B$が成り立っていると仮定する.論理積の除去より
				\begin{align}
					(A \wedge C) \Longrightarrow A
				\end{align}
				は定理であるから,含意の推移律より
				\begin{align}
					(A \wedge C) \Longrightarrow B
				\end{align}
				が従い,他方で論理積の除去より
				\begin{align}
					(A \wedge C) \Longrightarrow C
				\end{align}
				も満たされる.そして推論法則\ref{logicalthm:conjunction_of_consequences}から
				\begin{align}
					(A \wedge C) \Longrightarrow (B \wedge C)
				\end{align}
				が成り立ち,演繹法則より
				\begin{align}
					(A \Longrightarrow B) \Longrightarrow ((A \wedge C) \Longrightarrow (B \wedge C))
				\end{align}
				が得られる.
				
			\item[(c)]
				いま$A \Longrightarrow B$,$B \Longrightarrow C$および
				$A$が成り立っていると仮定する.このとき三段論法より$B$が成り立つので再び三段論法より
				$C$が成立する.ゆえに演繹法則より$A \Longrightarrow B$と$B \Longrightarrow C$が
				成り立っている下で
				\begin{align}
					A \Longrightarrow C
				\end{align}
				が成立し,演繹法則を更に順次適用すれば
				\begin{align}
					(A \Longrightarrow B) \Longrightarrow ( (B \Longrightarrow C) \Longrightarrow (A \Longrightarrow C) )
				\end{align}
				が得られる.
				
			\item[(d)]
				いま$A \Longrightarrow B$,$C \Longrightarrow A$および
				$C$が成り立っていると仮定する.このとき三段論法より$A$が成り立つので再び三段論法より$B$が成立し,
				ここに演繹法則を適用すれば,$A \Longrightarrow B$と$C \Longrightarrow A$が成立している下で
				\begin{align}
					C \Longrightarrow B
				\end{align}
				が成立する.演繹法則を更に順次適用すれば
				\begin{align}
					(A \Longrightarrow B) \Longrightarrow ( (C \Longrightarrow A) \Longrightarrow (C \Longrightarrow B) )
				\end{align}
				が得られる.
				\QED
		\end{description}
	\end{prf}
	
	\begin{screen}
		\begin{logicalthm}[正しい式は仮定を選ばない]\label{logicalthm:rule_of_inference_2}
			$A,B$を$\mathcal{L}'$の閉式とするとき,
			$B \Longrightarrow (A \Longrightarrow B)$は定理である.
		\end{logicalthm}
	\end{screen}
	
	\begin{prf}
		$B$を公理に追加した場合,$A$を公理に追加しても$B$は真であるから,このとき
		\begin{align}
			A \Longrightarrow B
		\end{align}
		は定理となる.従って演繹法則より$B \Longrightarrow (A \Longrightarrow B)$は定理である.
		\QED
	\end{prf}
	
	$A$と$B$を$\mathcal{L}'$の式とするとき,
	\begin{align}
		(A \Longleftrightarrow B) \defarrow
		(A \Longrightarrow B \wedge B \Longrightarrow A)
	\end{align}
	により$\Longleftrightarrow$を定め,式`$A \Longleftrightarrow B$'を
	``$A$と$B$は{\bf 同値である}\index{どうち@同値}{\bf (equivalent)}''と翻訳する.
	
	\begin{screen}
		\begin{logicalthm}[同値記号の可換律]\label{logicalthm:commutative_law_of_equivalence}
			$A$と$B$を$\mathcal{L}'$の閉式とするとき
			\begin{align}
				(A \Longleftrightarrow B) \Longrightarrow (B \Longleftrightarrow A).
			\end{align}
		\end{logicalthm}
	\end{screen}
	
	\begin{sketch}
		$A \Longleftrightarrow B$が成り立っているならば,推論法則\ref{logicalthm:commutative_law_of_disjunction_and_conjunction}より
		\begin{align}
			B \Longrightarrow A \wedge A \Longrightarrow B
		\end{align}
		が成立する.すなわち
		\begin{align}
			B \Longleftrightarrow A
		\end{align}
		が成立する.そして演繹法則から
		\begin{align}
			(A \Longleftrightarrow B) \Longrightarrow (B \Longleftrightarrow A)
		\end{align}
		が成立する.
		\QED
	\end{sketch}
	
	\begin{screen}
		\begin{logicalthm}[同値記号の遺伝性質]\label{logicalthm:hereditary_of_equivalence}
			$A,B,C$を$\mathcal{L}'$の閉式とするとき以下の式が成り立つ:
			\begin{description}
				\item[(a)] $(A \Longleftrightarrow B) \Longrightarrow ((A \vee C) \Longleftrightarrow (B \vee C))$.
				\item[(b)] $(A \Longleftrightarrow B) \Longrightarrow ((A \wedge C) \Longleftrightarrow (B \wedge C))$.
				\item[(c)] $(A \Longleftrightarrow B) \Longrightarrow ((B \Longrightarrow C) \Longleftrightarrow (A \Longrightarrow C))$.
				
				\item[(d)] $(A \Longleftrightarrow B) \Longrightarrow ((C \Longrightarrow A) \Longleftrightarrow (C \Longrightarrow B))$.
			\end{description}
		\end{logicalthm}
	\end{screen}
	
	\begin{prf}
		まず(a)を示す.いま$A \Longleftrightarrow B$が成り立っていると仮定する.このとき$A \Longrightarrow B$と
		$B \Longrightarrow A$が共に成立し,他方で含意の遺伝性質より
		\begin{align}
			&(A \Longrightarrow B) \Longrightarrow ((A \vee C) \Longrightarrow (B \vee C)), \\
			&(B \Longrightarrow A) \Longrightarrow ((B \vee C) \Longrightarrow (A \vee C))
		\end{align}
		が成立するから三段論法より$(A \vee C) \Longrightarrow (B \vee C)$と
		$(B \vee C) \Longrightarrow (A \vee C)$が共に成立する.ここに$\wedge$の導入を適用すれば
		\begin{align}
			(A \vee C) \Longleftrightarrow (B \vee C)
		\end{align}
		が成立し,演繹法則を適用すれば
		\begin{align}
			(A \Longleftrightarrow B) \Longrightarrow ((A \vee C) \Longleftrightarrow (B \vee C))
		\end{align}
		が得られる.(b)(c)(d)も含意の遺伝性を適用すれば得られる.
		\QED
	\end{prf}
	
	\begin{screen}
		\begin{logicalaxm}[矛盾と否定に関する規則]\label{logicalaxm:rules_of_contradiction}
			$A$を$\mathcal{L}'$の閉式とするとき以下の式が成り立つ:
			\begin{description}
				\item[矛盾の発生] 否定が共に成り立つとき矛盾が導かれる:
					\begin{align}
						(A \wedge \rightharpoondown A) \Longrightarrow \bot.
					\end{align}
				\item[否定の導出] 矛盾が導かれるとき否定が成り立つ:
					\begin{align}
						(A \Longrightarrow \bot) \Longrightarrow\ \rightharpoondown A.
					\end{align}
				\item[二重否定の法則] 二重に否定された式は元の式を導く:
					\begin{align}
						\rightharpoondown \rightharpoondown A \Longrightarrow A.
					\end{align}
			\end{description}
		\end{logicalaxm}
	\end{screen}
	
	\monologue{
		$A$を$\mathcal{L}'$の閉式とするとき,式$A \Longrightarrow \bot$を
		``$A$は{\bf 偽である}\index{ぎ@偽}{\bf (false)}''と翻訳します.
	}
	
	否定の導出の逆は定理として得られる.
	\begin{screen}
		\begin{logicalthm}[否定が正しい式は偽である]\label{logicalthm:false_and_negation_are_equivalent}
			$A$を$\mathcal{L}'$の閉式とするとき次が成り立つ:
			\begin{align}
				\rightharpoondown A \Longrightarrow (A \Longrightarrow \bot).
			\end{align}
		\end{logicalthm}
	\end{screen}
	
	\begin{prf}
		$\rightharpoondown A$が成り立っていると仮定する.このとき$A$が成り立っていれば
		推論規則\ref{logicalaxm:rules_of_contradiction}より$\bot$が成立するから,演繹法則より
		\begin{align}
			\rightharpoondown A \Longrightarrow (A \Longrightarrow \bot)
		\end{align}
		が成り立つ.
		\QED
	\end{prf}
	
	\begin{screen}
		\begin{logicalthm}[矛盾からはあらゆる式が導かれる]\label{logicalthm:contradiction_derives_any_formula}
			$A$を$\mathcal{L}'$の閉式とするとき
			\begin{align}
				\bot \Longrightarrow A.
			\end{align}
		\end{logicalthm}
	\end{screen}
	
	\begin{prf}
		推論法則\ref{logicalthm:rule_of_inference_2}より
		\begin{align}
			\bot \Longrightarrow (\rightharpoondown A \Longrightarrow \bot)
		\end{align}
		が成り立つ.また否定の導出より
		\begin{align}
			(\rightharpoondown A \Longrightarrow \bot) \Longrightarrow\ \rightharpoondown \rightharpoondown A
		\end{align}
		も成り立ち,さらに二重否定の法則から
		\begin{align}
			\rightharpoondown \rightharpoondown A \Longrightarrow A
		\end{align}
		も成り立つ.上の式に含意の推移律を適用すれば
		\begin{align}
			\bot \Longrightarrow A
		\end{align}
		が得られる.
		\QED
	\end{prf}
	
	\begin{screen}
		\begin{logicalthm}[背理法の原理]
			$A$を$\mathcal{L}'$の閉式とするとき
			\begin{align}
				(\rightharpoondown A \Longrightarrow \bot) \Longrightarrow A.
			\end{align}
		\end{logicalthm}
	\end{screen}
	
	\begin{prf}
		$\rightharpoondown A \Longrightarrow \bot$が成り立つとき,否定の導出より
		$\rightharpoondown \rightharpoondown A$が成り立つが,二重否定の法則より
		$A$も成立する.
		\QED
	\end{prf}
	
	\begin{screen}
		\begin{logicalthm}[矛盾を導く式はあらゆる式を導く]\label{logicalthm:formula_leading_to_contradiction_derives_any_formula}
			$A,B$を$\mathcal{L}'$の閉式とするとき,次が成り立つ:
			\begin{align}
				(A \Longrightarrow \bot) \Longrightarrow (A \Longrightarrow B).
			\end{align}
		\end{logicalthm}
	\end{screen}
	
	\begin{prf}
		$A \Longrightarrow \bot$が成り立っているとする.推論法則\ref{logicalthm:contradiction_derives_any_formula}より
		\begin{align}
			\bot \Longrightarrow B
		\end{align}
		が満たされるので,含意の推移律より
		\begin{align}
			A \Longrightarrow B
		\end{align}
		が成り立つ.従って演繹法則を適用すれば
		\begin{align}
			(A \Longrightarrow \bot) \Longrightarrow (A \Longrightarrow B)
		\end{align}
		が得られる.
		\QED
	\end{prf}
	
	\begin{screen}
		\begin{logicalthm}[含意は否定と論理和で表せる]\label{logicalthm:rule_of_inference_3}
			$A,B$を$\mathcal{L}'$の閉式とするとき,次が成り立つ:
			\begin{align}
				(A \Longrightarrow B) \Longleftrightarrow (\rightharpoondown A \vee B).
			\end{align}
		\end{logicalthm}
	\end{screen}
	
	\begin{prf}
		$A \Longrightarrow B$が成り立っていると仮定する.含意の遺伝性質より
		\begin{align}
			(A \Longrightarrow B) \Longrightarrow 
			((A \vee \rightharpoondown A) \Longrightarrow (B \vee \rightharpoondown A))
		\end{align}
		が満たされているから三段論法より
		\begin{align}
			(A \vee \rightharpoondown A) \Longrightarrow (B \vee \rightharpoondown A)
		\end{align}
		は定理となり,ここに排中律と三段論法を適用すれば
		\begin{align}
			B \vee \rightharpoondown A
		\end{align}
		が定理となる.
		ここで論理和の可換律より$\rightharpoondown A \vee B$が成り立つので,演繹法則を適用して
		\begin{align}
			(A \Longrightarrow B) \Longrightarrow (\rightharpoondown A \vee B)
		\end{align}
		が得られる.また矛盾に関する推論規則より
		\begin{align}
			\rightharpoondown A \Longrightarrow (A \Longrightarrow \bot)
		\end{align}
		が成り立ち,同時に推論法則\ref{logicalthm:formula_leading_to_contradiction_derives_any_formula}より
		\begin{align}
			(A \Longrightarrow \bot) \Longrightarrow (A \Longrightarrow B)
		\end{align}
		も成り立つので,含意の推移律より
		\begin{align}
			\rightharpoondown A \Longrightarrow (A \Longrightarrow B)
		\end{align}
		が成立する.他方で推論法則\ref{logicalthm:rule_of_inference_2}より
		\begin{align}
			B \Longrightarrow (A \Longrightarrow B)
		\end{align}
		も成り立つから,場合分けの法則より
		\begin{align}
			(\rightharpoondown A \vee B) \Longrightarrow (A \Longrightarrow B)
		\end{align}
		が成り立つ.以上で$(A \Longrightarrow B) \Longleftrightarrow (\rightharpoondown A \vee B)$が得られた.
		\QED
	\end{prf}
	
	\monologue{
		$A,B$を$\mathcal{L}'$の閉式とするとき,$A$が偽であれば$\rightharpoondown A$が成立する
		(推論規則\ref{logicalaxm:rules_of_contradiction})ので
		$\rightharpoondown A \vee B$が成立します(推論規則\ref{logicalaxm:fundamental_rules_of_inference}).
		すなわちこのとき$A \Longrightarrow B$が成り立つのですが,式の解釈としては
		``偽な式からはあらゆる式が導かれる''となりますね.この現象を
		{\bf 空虚な真}\index{くうきょなしん@空虚な真}{\bf (vacuous truth)}と呼びます.
	}
	
	\begin{screen}
		\begin{logicalthm}[二重否定の法則の逆が成り立つ]
			$A$を$\mathcal{L}'$の閉式とするとき,次が成り立つ:
			\begin{align}
				A \Longrightarrow\ \rightharpoondown \rightharpoondown A.
			\end{align}
		\end{logicalthm}
	\end{screen}
	
	\begin{prf}
		排中律より
		\begin{align}
			\rightharpoondown A \vee \rightharpoondown \rightharpoondown A
		\end{align}
		が成立し,また推論法則\ref{logicalthm:rule_of_inference_3}より
		\begin{align}
			(\rightharpoondown A \vee \rightharpoondown \rightharpoondown A)
			\Longrightarrow (A \Longrightarrow\ \rightharpoondown \rightharpoondown A)
		\end{align}
		も成り立つので,三段論法より
		\begin{align}
			A \Longrightarrow\ \rightharpoondown \rightharpoondown A
		\end{align}
		が成立する.
		\QED
	\end{prf}
	
	\begin{screen}
		\begin{logicalthm}[対偶命題は同値]\label{thm:contraposition_is_true}
			$A,B$を$\mathcal{L}'$の閉式とするとき,次が成り立つ:
			\begin{align}
				(A \Longrightarrow B) \Longleftrightarrow (\rightharpoondown B \Longrightarrow\ \rightharpoondown A).
			\end{align}
		\end{logicalthm}
	\end{screen}
	
	\begin{prf}
		推論法則\ref{logicalthm:rule_of_inference_3},論理和の可換律,二重否定の法則(とその逆)を順に用いれば
		\begin{align}
			(A \Longrightarrow B) &\Longleftrightarrow (\rightharpoondown A \vee B) \\
			&\Longleftrightarrow (B \vee \rightharpoondown A) \\
			&\Longleftrightarrow (\rightharpoondown \rightharpoondown B \vee \rightharpoondown A) \\
			&\Longleftrightarrow (\rightharpoondown B \Longrightarrow\ \rightharpoondown A)
		\end{align}
		が成り立つ.
		\QED
	\end{prf}
	
	\monologue{
		対偶命題を述べるときには``対偶を取る''と表現することが多いです.
	}
	
	\begin{screen}
		\begin{logicalthm}[De Morganの法則]
			$A,B$を$\mathcal{L}'$の閉式とするとき,次が成り立つ:
			\begin{itemize}
				\item $\rightharpoondown (A \vee B) \Longleftrightarrow\ \rightharpoondown A \wedge \rightharpoondown B$.
			
				\item $\rightharpoondown (A \wedge B) \Longleftrightarrow\ \rightharpoondown A \vee \rightharpoondown B$.
			\end{itemize}
		\end{logicalthm}
	\end{screen}
	
	\begin{prf}
		$A \Longrightarrow (A \vee B)$は定理であるから,その対偶命題
		\begin{align}
			\rightharpoondown (A \vee B) \Longrightarrow\ \rightharpoondown A
		\end{align}
		も定理となる.同様に$\rightharpoondown (A \vee B) \Longrightarrow\ \rightharpoondown B$は定理となるので,
		$\rightharpoondown (A \vee B)$が成り立っていると仮定すれば$\rightharpoondown A \wedge \rightharpoondown B$が成り立つ.
		ゆえに
		\begin{align}
			\rightharpoondown (A \vee B) \Longrightarrow\ \rightharpoondown A \wedge \rightharpoondown B
		\end{align}
		が得られる.また$A$が成り立っていると仮定すれば,この下で$\rightharpoondown A \wedge \rightharpoondown B$が成り立っているなら
		$A$と$\rightharpoondown A$が同時に成り立つことになるので$\bot$が成立する.つまり
		$A$が成り立っているとき
		\begin{align}
			\rightharpoondown A \wedge \rightharpoondown B \Longrightarrow \bot
		\end{align}
		が成り立つが,このとき$\rightharpoondown(\rightharpoondown A \wedge \rightharpoondown B)$が成り立つので
		\begin{align}
			A \Longrightarrow\ \rightharpoondown(\rightharpoondown A \wedge \rightharpoondown B)
		\end{align}
		が得られる.同様にして
		\begin{align}
			B \Longrightarrow\ \rightharpoondown(\rightharpoondown A \wedge \rightharpoondown B)
		\end{align}
		も得られるから,場合分け法則より
		\begin{align}
			(A \vee B) \Longrightarrow\ \rightharpoondown(\rightharpoondown A \wedge \rightharpoondown B)
		\end{align}
		が成立する.この対偶を取れば
		\begin{align}
			\rightharpoondown A \wedge \rightharpoondown B
			\Longrightarrow\ \rightharpoondown (A \vee B)
		\end{align}
		が出る.以上で一つ目の式が示された.一つ目の式で$A$を$\rightharpoondown A$に,
		$B$を$\rightharpoondown B$に置き換えると
		\begin{align}
			\rightharpoondown \rightharpoondown A \wedge \rightharpoondown \rightharpoondown B
			\Longleftrightarrow\ \rightharpoondown (\rightharpoondown A \vee \rightharpoondown B)
		\end{align}
		が得られるが,このとき二重否定の法則より
		\begin{align}
			A \wedge B
			\Longleftrightarrow\ \rightharpoondown (\rightharpoondown A \vee \rightharpoondown B)
		\end{align}
		が成立し,対偶命題の同値性から
		\begin{align}
			\rightharpoondown (A \wedge B)
			\Longleftrightarrow\ (\rightharpoondown A \vee \rightharpoondown B)
		\end{align}
		は定理となる.
		\QED
	\end{prf}
	
	\monologue{
		以上で``集合であり真類でもある類は存在しない''という言明を証明する準備が整いました.
	}
	
	\begin{screen}
		\begin{thm}[集合であり真類でもある類は存在しない]
			$a$を類とするとき次が成り立つ:
			\begin{align}
				\rightharpoondown (\ \set{a} \wedge \rightharpoondown \set{a}\ ).
			\end{align}
		\end{thm}
	\end{screen}
	
	\begin{prf}
		$a$を類とするとき,排中律より$\set{a} \vee \rightharpoondown \set{a}$
		が成り立ち,論理和の可換律より
		\begin{align}
			\rightharpoondown \set{a} \vee \set{a}
		\end{align}
		も成立する.そしてDe Morganの法則より
		\begin{align}
			\rightharpoondown (\ \rightharpoondown \rightharpoondown \set{a} \wedge \rightharpoondown \set{a}\ )
		\end{align}
		が成り立つが,二重否定の法則より$\rightharpoondown \rightharpoondown \set{a}$と
		$\set{a}$は同値となるので
		\begin{align}
			\rightharpoondown (\ \set{a} \wedge \rightharpoondown \set{a}\ )
		\end{align}
		が成り立つ.
		\QED
	\end{prf}
	
	\monologue{
		次は量化記号が推論操作の上でどのような働きを持つのかを規定しましょう.
	}
	
	\begin{screen}
		\begin{logicalaxm}[量化記号に関する規則]\label{logicalaxm:rules_of_quantifiers}
			$A$を$\mathcal{L}'$の式とし,$x$を$A$に現れる文字とするとき,$x$のみが$A$で量化されていないならば以下を認める:
			\begin{description}
				\item[$\varepsilon$記号の導入] $\varepsilon x A(x)$は$\mathcal{L}$の或る対象に代用される.
				\item[存在記号の規則] $A (\varepsilon x A(x)) \Longleftrightarrow \exists x A(x)$が成り立つ.
				\item[全称記号の規則] $A (\varepsilon x \rightharpoondown A(x)) \Longleftrightarrow \forall x A(x)$が成り立つ.
				\item[存在記号の基本性質] $\tau$を$\mathcal{L}$の対象とするとき
					$A(\tau) \Longrightarrow \exists x A(x)$が成り立つ.
			\end{description}
		\end{logicalaxm}
	\end{screen}
	
	$\varepsilon$記号はHilbertのイプシロン関数と呼ばれるもので,
	量化記号の働きを形式的に表現するには簡便かつ有能である.
	また$\varepsilon$記号が指定する対象を$\mathcal{L}$のものと約束することで,
	$\exists$と$\forall$の作用範囲を$\mathcal{L}$の対象全体に制限している.
	
	\begin{screen}
		\begin{logicalthm}[全称記号と任意性]\label{logicalthm:fundamental_law_of_universal_quantifier}
			$A$を$\mathcal{L}'$の式とし,$x$を$A$に現れる文字とし,$x$のみが$A$で量化されていないとする.このとき
			$\forall x A(x)$が成り立つならば$\mathcal{L}$のいかなる対象$\tau$に対しても$ A(\tau)$が成り立つ.
			逆に,$\mathcal{L}$のいかなる対象$\tau$に対しても$A(\tau)$が成り立てば$\forall x A(x)$が成り立つ.
		\end{logicalthm}
	\end{screen}
	
	\begin{prf}
		$\tau$を$\mathcal{L}$の任意の対象とすれば,存在記号に関する推論規則より
		\begin{align}
			\rightharpoondown A(\tau) \Longrightarrow\ \exists x \rightharpoondown A(x)
		\end{align}
		と
		\begin{align}
			\exists x \rightharpoondown A(x) \Longrightarrow\ \rightharpoondown A
			\left( \varepsilon x \rightharpoondown A(x) \right)
		\end{align}
		が成り立つから,推論法則\ref{logicalthm:transitive_law_of_implication}より
		\begin{align}
			\rightharpoondown A(\tau) \Longrightarrow\ \rightharpoondown A
			\left( \varepsilon x \rightharpoondown A(x) \right)
		\end{align}
		が成り立ち,対偶を取って
		\begin{align}
			A \left( \varepsilon x \rightharpoondown A(x) \right)
			\Longrightarrow A(\tau)
		\end{align}
		が成り立つ.全称記号に関する推論規則より
		\begin{align}
			\forall x A(x) \Longrightarrow A \left( \varepsilon x \rightharpoondown A(x) \right)
		\end{align}
		が満たされているので
		\begin{align}
			\forall x A(x) \Longrightarrow A(\tau)
		\end{align}
		が従う.逆にいかなる対象$\tau$に対しても$A(\tau)$が成り立つとき,特に
		\begin{align}
			A \left( \varepsilon x \rightharpoondown A(x) \right)
		\end{align}
		が成り立つので$\forall x A(x)$も成り立つ.
		\QED
	\end{prf}
	
	\monologue{
		推論法則\ref{logicalthm:fundamental_law_of_universal_quantifier}を根拠にして,
		当面は$\forall x A(x)$という式を``$\mathcal{L}$の任意の対象$x$に対して
		$A(x)$が成立する''と翻訳することにします.また後述する相等性の公理によれば,
		これは``任意の集合$x$に対して$A(x)$が成立する''と翻訳しても同義です.
	}
	
	\begin{screen}
		\begin{logicalthm}[量化記号の性質(イ)]\label{logicalthm:properties_of_quantifiers}
			$A,B$を$\mathcal{L}'$の式とし,$x$を$A,B$に現れる文字とし,$x$のみが$A,B$で量化されていないとする.
			$\mathcal{L}$の任意の対象$\tau$に対して
			\begin{align}
				A(\tau) \Longleftrightarrow B(\tau)
			\end{align}
			が成り立っているとき,
			\begin{align}
				\exists x A(x) \Longleftrightarrow \exists x B(x)
			\end{align}
			および
			\begin{align}
				\forall x A(x) \Longleftrightarrow \forall x B(x)
			\end{align}
			が成り立つ.
		\end{logicalthm}
	\end{screen}
	
	\begin{prf}
		いま,$\mathcal{L}$の任意の対象$\tau$に対して
		\begin{align}
			A(\tau) \Longleftrightarrow B(\tau)
			\label{logicalthm:properties_of_quantifiers_1}
		\end{align}
		が成り立っているとする.
		ここで
		\begin{align}
			\exists x A(x)
		\end{align}
		が成り立っていると仮定すると,
		\begin{align}
			\tau \defeq \varepsilon x A(x)
		\end{align}
		とおけば存在記号に関する規則より
		\begin{align}
			A(\tau)
		\end{align}
		が成立し,(\refeq{logicalthm:properties_of_quantifiers_1})と併せて
		\begin{align}
			B(\tau)
		\end{align}
		が成立する.再び存在記号に関する規則より
		\begin{align}
			\exists x B(x)
		\end{align}
		が成り立つので,演繹法則から
		\begin{align}
			\exists x A(x) \Longrightarrow \exists x B(x)
		\end{align}
		が得られる.$A$と$B$の立場を入れ替えれば
		\begin{align}
			\exists x B(x) \Longrightarrow \exists x A(x)
		\end{align}
		も得られる.今度は
		\begin{align}
			\forall x A(x)
		\end{align}
		が成り立っていると仮定すると,
		推論法則\ref{logicalthm:fundamental_law_of_universal_quantifier}より
		$\mathcal{L}$の任意の対象$\tau$に対して
		\begin{align}
			A(\tau)
		\end{align}
		が成立し,(\refeq{logicalthm:properties_of_quantifiers_1})と併せて
		\begin{align}
			B(\tau)
		\end{align}
		が成立する.$\tau$の任意性と推論法則\ref{logicalthm:fundamental_law_of_universal_quantifier}より
		\begin{align}
			\forall x B(x)
		\end{align}
		が成り立つので,演繹法則から
		\begin{align}
			\forall x A(x) \Longrightarrow \forall x B(x)
		\end{align}
		が得られる.$A$と$B$の立場を入れ替えれば
		\begin{align}
			\forall x B(x) \Longrightarrow \forall x A(x)
		\end{align}
		も得られる.
		\QED
	\end{prf}
	
	\begin{screen}
		\begin{logicalthm}[量化記号に対する De Morgan の法則]\label{logicalthm:De_Morgan_law_for_quantifiers}
			$A$を$\mathcal{L}'$の式とし,$x$を$A$に現れる文字とし,$x$のみが$A$で量化されていないとする.このとき
			\begin{align}
				\exists x \rightharpoondown A(x) \Longleftrightarrow\ \rightharpoondown \forall x A(x)
			\end{align}
			および
			\begin{align}
				\forall x \rightharpoondown A(x) \Longleftrightarrow\ \rightharpoondown \exists x A(x)
			\end{align}
			が成り立つ.
		\end{logicalthm}
	\end{screen}
	
	\begin{sketch}
		推論規則\ref{logicalaxm:rules_of_quantifiers}より
		\begin{align}
			\exists x \rightharpoondown A(x) \Longleftrightarrow\ 
			\rightharpoondown A(\varepsilon x \rightharpoondown A(x))
		\end{align}
		は定理である.他方で推論規則\ref{logicalaxm:rules_of_quantifiers}より
		\begin{align}
			A(\varepsilon x \rightharpoondown A(x)) \Longleftrightarrow \forall x A(x) 
		\end{align}
		もまた定理であり,この対偶を取れば
		\begin{align}
			\rightharpoondown A(\varepsilon x \rightharpoondown A(x)) \Longleftrightarrow\ 
			\rightharpoondown \forall x A(x)
		\end{align}
		が成り立つ.ゆえに
		\begin{align}
			\exists x \rightharpoondown A(x) \Longleftrightarrow\ \rightharpoondown \forall x A(x)
		\end{align}
		が従う.$A$を$\rightharpoondown A$に置き換えれば
		\begin{align}
			\forall x \rightharpoondown A(x) \Longleftrightarrow\ 
			\rightharpoondown \exists x \rightharpoondown \rightharpoondown A(x)
		\end{align}
		が成り立ち,また$\mathcal{L}$の任意の対象$\tau$に対して
		\begin{align}
			A(\tau) \Longleftrightarrow\ \rightharpoondown \rightharpoondown A(\tau)
		\end{align}
		が成り立つので,推論法則\ref{logicalthm:properties_of_quantifiers}より
		\begin{align}
			\exists x \rightharpoondown \rightharpoondown A(x)
			\Longleftrightarrow \exists x A(x)
		\end{align}
		も成り立つ.ゆえに
		\begin{align}
			\forall x \rightharpoondown A(x) \Longleftrightarrow\ 
			\rightharpoondown \exists x A(x)
		\end{align}
		が従う.
		\QED
	\end{sketch}

\chapter{集合}	
	\section{相等性}
	本稿において``等しい''とは項に対する言明であって,$a$と$b$を項とするとき
	\begin{align}
		a = b
	\end{align}
	なる式で表される.この記号
	\begin{align}
		=
	\end{align}
	は{\bf 等号}\index{とうごう@等号}{\bf (equal sign)}と呼ばれるが,
	現時点では述語として導入されているだけで,推論操作における働きはまだ明文化していない.
	本節では,いつ類は等しくなるのか,そして,等しい場合に何が起きるのか,の二つが主題となる.
	
	\begin{screen}
		\begin{axm}[外延性の公理]
			$a,b$を類とするとき,次が成り立つ:
			\begin{align}
				\forall x\, (\, x \in a \Longleftrightarrow x \in b\, )
				\Longrightarrow a=b.
			\end{align}
		\end{axm}
	\end{screen}
	
	\begin{screen}
		\begin{thm}[任意の類は自分自身と等しい]\label{thm:any_class_equals_to_itself}
			$a$を類とするとき次が成り立つ:
			\begin{align}
				a = a.
			\end{align}
		\end{thm}
	\end{screen}
	
	\begin{sketch}
		任意の$\varepsilon$項$\tau$に対して,
		推論法則\ref{logicalthm:reflective_law_of_implication}より
		\begin{align}
			\tau \in a \Longleftrightarrow \tau \in a
		\end{align}
		が成り立つから,$\tau$の任意性より
		\begin{align}
			\forall x\, (\, x \in a  \Longleftrightarrow x \in a\, )
		\end{align}
		が成り立つ.外延性の公理と三段論法より
		\begin{align}
			a = a
		\end{align}が得られる.
		\QED
	\end{sketch}
	
	\begin{screen}
		\begin{thm}[$\varepsilon$項は集合である]
			任意の$\varepsilon$項$\varepsilon x A(x)$に対して
			\begin{align}
				\set{\varepsilon x A(x)}.
			\end{align}
		\end{thm}
	\end{screen}
	
	\begin{sketch}
		定理\ref{thm:any_class_equals_to_itself}より
		\begin{align}
			\varepsilon x A(x) = \varepsilon x A(x)
		\end{align}
		が成立するので,存在記号の推論規則より
		\begin{align}
			\exists y\, \left(\, \varepsilon x A(x) = y\, \right)
		\end{align}
		が成立する.
		\QED
	\end{sketch}
	
	$A$を$\mathcal{L}_{\in}$の式とし,$x$を$A$に現れる変項とし,$x$のみが$A$で自由であるとし,かつ
	\begin{align}
		\set{\Set{x}{A(x)}}
	\end{align}
	が満たされているとする.つまり
	\begin{align}
		\exists y\, \left(\, \Set{x}{A(x)} = y\, \right)
	\end{align}
	が成り立っているということであるが,$\Set{x}{A(x)} = y$を
	\begin{align}
		\forall x\, \left(\, A(x) \Longleftrightarrow x \in y\, \right)
	\end{align}
	と書き換えれば,存在記号の推論規則より
	\begin{align}
		\Set{x}{A(x)} = \varepsilon y \forall x\, \left(\, A(x) \Longleftrightarrow x \in y\, \right)
	\end{align}
	が得られる.
	
	\begin{screen}
		\begin{thm}[集合である内包項は$\varepsilon$項で書ける]
			任意の内包項$\Set{x}{A(x)}$に対して,$\Set{x}{A(x)}$が集合であれば
			\begin{align}
				\Set{x}{A(x)} = \varepsilon y \forall x\, \left(\, A(x) \Longleftrightarrow x \in y\, \right).
			\end{align}
		\end{thm}
	\end{screen}
	
	ブルバキでは$\tau$項を,島内では$\varepsilon$項のみを導入して
	$\varepsilon y \forall x\, \left(\, A(x) \Longleftrightarrow x \in y\, \right)$
	によって$\Set{x}{A(x)}$を定めている.本稿と同じくブルバキの$\tau$項も島内の$\varepsilon$項も
	集合を表すものであるから,
	\begin{align}
		\exists y\, \forall x\, \left(\, A(x) \Longleftrightarrow x \in y\, \right)
	\end{align}
	を満たさないような性質$A$に対しては$\varepsilon y \forall x\, \left(\, A(x) \Longleftrightarrow x \in y\, \right)$
	は不定の集合を指す.本稿では
	
	\begin{screen}
		\begin{axm}[要素の公理]
			要素となりうる類は集合である.つまり,$a,b$を類とするとき
			\begin{align}
				a \in b \Longrightarrow \set{a}.
			\end{align}
		\end{axm}
	\end{screen}
	
	\begin{screen}
		\begin{axm}[内包性公理] 
			$A$を$\mathcal{L}_{\in}$の式とし,$x$を$A$に現れる変項とし,$y$を$A(x)$に現れない変項とし,
			$x$のみが$A$で自由であるとする.このとき
			\begin{align}
				\forall y\, \left(\, y \in \Set{x}{A(x)} \Longleftrightarrow A(y)\, \right).
			\end{align}
		\end{axm}
	\end{screen}
	
	要素の公理で要求していることは{\bf 類を構成できるのは集合に限られる}ということであり,
	内包性公理は{\bf 甲種項はその固有の性質を持つ集合の全体である}という意味を持つ.
	
	
	例えば
	\begin{align}
		a = b
	\end{align}
	と書いてあったら``$a$と$b$は等しい''と読めるわけだが,明らかに$a$は$b$とは違うではないではないか!
	こんなことはしょっちゅう起こることであって,上で述べたように$\Set{x}{A(x)}$が集合なら
	\begin{align}
		\Set{x}{A(x)} = \varepsilon y \forall x\, \left(\, A(x) \Longleftrightarrow x \in y\, \right)
	\end{align}
	が成り立ったりする.そこで``数学的に等しいとは何事か''という疑問が浮かぶのは至極自然であって,
	それに答えるのが次の相等性公理である.
	
	\begin{screen}
		\begin{axm}[相等性公理]
			$A$を$\mathcal{L}'$の式とし,$x$を$A$に現れる文字とし,
			$x$のみが$A$で量化されていないとする.このとき$a,b$を類とすれば次が成り立つ:
			\begin{align}
				a = b \Longrightarrow (\, A(a) \Longleftrightarrow A(b)\, ).
			\end{align}
		\end{axm}
	\end{screen}
	
	\begin{screen}
		\begin{thm}[外延性の公理の逆も成り立つ]\label{thm:axiom_of_extensionality_equivalent}
			$a$と$b$を類とするとき
			\begin{align}
				a=b \Longrightarrow \forall x\, (\, x \in a  \Longleftrightarrow x \in b\, ).
			\end{align}
		\end{thm}
	\end{screen}
	
	\begin{prf}
		$a = b$が成り立っていると仮定すれば,相等性の公理より$\mathcal{L}$の任意の対象$\tau$に対して
		\begin{align}
			\tau \in a \Longleftrightarrow \tau \in b
		\end{align}
		が満たされるから,推論法則\ref{logicalthm:fundamental_law_of_universal_quantifier}より
		\begin{align}
			\forall x\, (\, x \in a  \Longleftrightarrow x \in b\, )
		\end{align}
		が成立する.よって演繹法則より
		\begin{align}
			a = b \Longrightarrow \forall x\, (\, x \in a  \Longleftrightarrow x \in b\, )
		\end{align}
		が成り立つ.
		\QED
	\end{prf}
	
	\monologue{
		等しい類同士は同じ$\mathcal{L}$の対象を要素に持つと示されましたが,
		このとき要素に持つ集合まで一致します.これは相等性の公理から明らかでしょうが,
		詳しくは部分類の箇所で説明いたしましょう.
	}
	
	\begin{screen}
		\begin{thm}[条件を満たす集合は要素である]\label{thm:satisfactory_set_is_an_element}
			$A$を$\mathcal{L}$の式とし,$x$を$A$に現れる文字とし,$t$を$A(x)$に現れない文字とし,
			$x$のみが$A$で量化されていないとする.このとき,$a$を類とすると
			\begin{align}
				A(a) \Longrightarrow 
				\left(\, \set{a} \Longrightarrow a \in \Set{x}{A(x)}\, \right).
			\end{align}
		\end{thm}
	\end{screen}
	
	\begin{sketch}
		いま
		\begin{align}
			A(a)
		\end{align}
		と
		\begin{align}
			\set{a}
		\end{align}
		が成立していると仮定する.このとき要素の公理から
		\begin{align}
			\exists x\, (\, a = x\, )
		\end{align}
		が成立するので,
		\begin{align}
			\tau \defeq \varepsilon x\, (\, a = x\, )
		\end{align}
		とおけば
		\begin{align}
			a = \tau
		\end{align}
		が成り立ち,相等性の公理より
		\begin{align}
			A(\tau)
		\end{align}
		が成立する.よって類の公理より
		\begin{align}
			\tau \in \Set{x}{A(x)}
		\end{align}
		が従い,相等性の公理から
		\begin{align}
			a \in \Set{x}{A(x)}
		\end{align}
		が成立する.
		\QED
	\end{sketch}
	
	\begin{screen}
		\begin{thm}[$\Univ$は集合の全体である]
		\label{thm:V_is_the_whole_of_sets}
			$a$を類とするとき次が成り立つ:
			\begin{align}
				\set{a} \Longleftrightarrow a \in \Univ.
			\end{align}
		\end{thm}
	\end{screen}
	
	\begin{prf}
		$a$を類とするとき,まず要素の公理より
		\begin{align}
			a \in \Univ \Longrightarrow \set{a}
		\end{align}
		が得られる.逆に
		\begin{align}
			\set{a}
		\end{align}
		が成り立っていると仮定する.このとき
		\begin{align}
			\tau \defeq \varepsilon x (\ a = x\ )
		\end{align}
		とおけば,定理\ref{thm:any_class_equals_to_itself}より
		\begin{align}
			\tau = \tau
		\end{align}
		となるので,類の公理より
		\begin{align}
			\tau \in \Univ
		\end{align}
		が成り立つ.そして相等性の公理より
		\begin{align}
			a \in \Univ
		\end{align}
		が従うから
		\begin{align}
			\set{a} \Longrightarrow a \in \Univ
		\end{align}
		も得られる.
		\QED
	\end{prf}
	
	\begin{screen}
		\begin{dfn}[空集合]
			$\emptyset \defeq \Set{x}{x \neq x}$で定める類$\emptyset$を{\bf 空集合}\index{くうしゅうごう@空集合}{\bf (empty set)}と呼ぶ.
		\end{dfn}
	\end{screen}
	
	\begin{screen}
		\begin{axm}[置換公理]
			\begin{align}
				\forall a\, \left[\, \forall x \in a\, \exists!y \varphi(x,y)
				\Longrightarrow \exists z\, \forall y\,
				(\, y \in z \Longleftrightarrow \exists x\, (\, x \in a \wedge 
				\varphi(x,y)\, )\, )\, \right].
			\end{align}
		\end{axm}
	\end{screen}
	
	$\Set{x}{\varphi(x)}$は集合であるとは限らないが,
	集合$a$に対して
	\begin{align}
		a \cap \Set{x}{\varphi(x)}
	\end{align}
	なる類は当然$a$より``小さい集まり''なのだから,集合であってほしいものである.
	置換公理によってそのこと保証され,分出定理として知られている.
	
	\begin{screen}
		\begin{thm}[分出定理]
			\begin{align}
				\forall a\, \exists s\, \forall x\,
				(\, x \in s \Longleftrightarrow x \in a \wedge \varphi(x)\, ).
			\end{align}
		\end{thm}
	\end{screen}
	
	\begin{sketch}
		$x$と$y$が自由に現れる式$\psi(x,y)$を
		\begin{align}
			x = y \wedge \varphi(x)
		\end{align}
		と設定すると,これは
		\begin{align}
			\forall x \in a\, \exists!y \psi(x,y)
		\end{align}
		を満たすので,置換公理より
		\begin{align}
			\forall y\, (\, y \in z \Longleftrightarrow \exists x\, (\, x \in a \wedge 
			\psi(x,y)\, )\, )
		\end{align}
		を満たす集合$z$が取れる.このとき
		\begin{align}
			\Set{y}{y \in a \wedge \varphi(y)} = z
		\end{align}
		が成立する.実際,
		\begin{align}
			y \in z
		\end{align}
		ならば
		\begin{align}
			x \in a \wedge x = y \wedge \varphi(x)
		\end{align}
		を満たす$x$が取れるが,このとき相等性から
		\begin{align}
			y \in a \wedge \varphi(y)
		\end{align}
		が成立する.逆に
		\begin{align}
			y \in a \wedge \varphi(y)
		\end{align}
		であれば
		\begin{align}
			x \defeq y
		\end{align}
		によって
		\begin{align}
			\exists x\, (\, x \in a \wedge \psi(x,y)\, )
		\end{align}
		が成り立つので
		\begin{align}
			y \in z
		\end{align}
		となる.
		\QED
	\end{sketch}
	
	\begin{screen}
		\begin{thm}[$\emptyset$は集合]\label{thm:emptyset_is_a_set}
			$\emptyset$は集合である:
			\begin{align}
				\set{\emptyset}.
			\end{align}
		\end{thm}
	\end{screen}
	
	\begin{sketch}
		分出定理より
		\begin{align}
			\forall z\, \exists y\, \forall x\,
			(\, x \in y \Longleftrightarrow x \in z \wedge x \neq x\, )
			\label{fom:thm_emptyset_is_a_set_1}
		\end{align}
		が成立するが,この式から
		\begin{align}
			\exists y\, \forall x\, (\, x \in y \Longleftrightarrow x \neq x\, )
			\label{fom:thm_emptyset_is_a_set_2}
		\end{align}
		を示せる.これはすなわち$\emptyset$が集合であるということを示唆する.
		$\zeta$を勝手な$\varepsilon$項として,後々の便宜のために
		\begin{align}
			\sigma &\defeq \varepsilon y\, \forall x\,
			(\, x \in y \Longleftrightarrow x \in \zeta \wedge x \neq x\, ), \\
			\tau &\defeq \varepsilon x \rightharpoondown
			(\, x \in \sigma \Longleftrightarrow x \neq x\, )
		\end{align}
		とおけば,(\refeq{fom:thm_emptyset_is_a_set_1})より
		\begin{align}
			\tau \in \sigma \Longleftrightarrow \tau \in \zeta \wedge \tau \neq \tau
		\end{align}
		が成立する.論理和の規則より
		\begin{align}
			\tau \in \zeta \wedge \tau \neq \tau \Longrightarrow \tau \neq \tau
		\end{align}
		が満たされるので,まずは
		\begin{align}
			\tau \in \sigma \Longrightarrow \tau \neq \tau
		\end{align}
		が得られる.また
		\begin{align}
			\tau = \tau
		\end{align}
		は正しいので,
		\begin{align}
			\tau = \tau \Longrightarrow (\, \tau \notin \sigma \Longrightarrow
			\tau = \tau\, )
		\end{align}
		と併せて
		\begin{align}
			\tau \notin \sigma \Longrightarrow \tau = \tau
		\end{align}
		が成り立ち,対偶を取れば
		\begin{align}
			\tau \neq \tau \Longrightarrow \tau \in \sigma
		\end{align}
		も得られる.ゆえに
		\begin{align}
			\forall x\, (\, x \in \sigma \Longleftrightarrow x \neq x\, )
		\end{align}
		が得られ,(\refeq{fom:thm_emptyset_is_a_set_2})が従う.
		\QED
	\end{sketch}
	
	\begin{screen}
		\begin{thm}[空集合は$\mathcal{L}$のいかなる対象も要素に持たない]\label{thm:emptyset_has_nothing}
			\begin{align}
				\forall x\, (\, x \notin \emptyset\, ).
			\end{align}
		\end{thm}
	\end{screen}
	
	\begin{sketch}
		$\tau$を$\mathscr{L}$の対象とするとき,類の公理より
		\begin{align}
			\tau \in \emptyset \Longrightarrow \tau \neq \tau
		\end{align}
		が成り立つから,対偶を取れば
		\begin{align}
			\tau = \tau \Longrightarrow \tau \notin \emptyset
		\end{align}
		が成り立つ(推論法則\ref{thm:contraposition_is_true}).定理\ref{thm:any_class_equals_to_itself}より
		\begin{align}
			\tau = \tau
		\end{align}
		は正しいので,三段論法より
		\begin{align}
			\tau \notin \emptyset
		\end{align}
		が成り立つ.そして$\tau$の任意性より
		\begin{align}
			\forall x\, (\, x \notin \emptyset\, )
		\end{align}
		が得られる.
		\QED
	\end{sketch}
	
	\begin{screen}
		\begin{thm}[$\mathcal{L}$のいかなる対象も要素に持たない類は空集合に等しい]
		\label{thm:uniqueness_of_emptyset}
			$a$を類とするとき次が成り立つ:
			\begin{align}
				\forall x\, (\, x \notin a\, ) \Longleftrightarrow a = \emptyset.
			\end{align}
		\end{thm}
	\end{screen}
	
	\begin{prf}
		$a$を類として$\forall x\, (\, x \notin a\, )$が成り立っていると仮定する.このとき
		$\tau$を$\mathcal{L}$の任意の対象とすれば
		\begin{align}
			\tau \notin a \vee \tau \in \emptyset
		\end{align}
		と
		\begin{align}
			\tau \notin \emptyset \vee \tau \in a
		\end{align}
		が共に成り立つので,推論法則\ref{logicalthm:rule_of_inference_3}より
		\begin{align}
			\tau \in a \Longrightarrow \tau \in \emptyset
		\end{align}
		と
		\begin{align}
			\tau \in \emptyset \Longrightarrow \tau \in a
		\end{align}
		が共に成り立つ.よって
		\begin{align}
			\tau \in a \Longleftrightarrow \tau \in \emptyset
		\end{align}
		が成立し,$\tau$の任意性と推論法則\ref{logicalthm:fundamental_law_of_universal_quantifier}から
		\begin{align}
			\forall x\, (\, x \in a \Longleftrightarrow x \in \emptyset\, )
		\end{align}
		が得られる.ゆえに外延性の公理より
		\begin{align}
			a = \emptyset
		\end{align}
		が成立し,演繹法則より
		\begin{align}
			\forall x\, (\, x \notin a\, ) \Longrightarrow a = \emptyset
		\end{align}
		が得られる.逆に
		\begin{align}
			a = \emptyset
		\end{align}
		が成り立っていると仮定する.ここで$\chi$を$\mathcal{L}$の任意の対象とすれば,
		相等性の公理より
		\begin{align}
			\chi \in a \Longrightarrow \chi \in \emptyset
		\end{align}
		が成立するので,対偶を取れば
		\begin{align}
			\chi \notin \emptyset \Longrightarrow \chi \notin a
		\end{align}
		が成り立つ.定理\ref{thm:emptyset_has_nothing}より
		\begin{align}
			\chi \notin \emptyset
		\end{align}
		が満たされているので,三段論法より
		\begin{align}
			\chi \notin a
		\end{align}
		が成立し,$\chi$の任意性と推論法則\ref{logicalthm:fundamental_law_of_universal_quantifier}より
		\begin{align}
			\forall x\, (\, x \notin a\, )
		\end{align}
		が成立する.ここに演繹法則を適用して
		\begin{align}
			a = \emptyset \Longrightarrow \forall x\, (\, x \notin a\, )
		\end{align}
		も得られる.
		\QED
	\end{prf}
	
	\begin{screen}
		\begin{thm}[空集合はいかなる類も要素に持たない]
		\label{thm:emptyset_does_not_contain_any_class}
			$a,b$を類とするとき次が成り立つ:
			\begin{align}
				b = \emptyset \Longrightarrow a \notin b.
			\end{align}
		\end{thm}
	\end{screen}
	
	\begin{prf}
		いま$a \in b$が成り立っていると仮定する.このとき要素の公理と三段論法より
		\begin{align}
			\set{a}
		\end{align}
		が成立する.ここで
		\begin{align}
			\tau \defeq \varepsilon x\, (\, a = x\, )
		\end{align}
		とおけば,存在記号に関する規則から
		\begin{align}
			a = \tau
		\end{align}
		が成り立つので,相等性の公理より
		\begin{align}
			\tau \in b
		\end{align}
		が従い,存在記号に関する規則より
		\begin{align}
			\exists x\, (\, x \in b\, )
		\end{align}
		が成り立つ.よって演繹法則から
		\begin{align}
			a \in b \Longrightarrow \exists x\, (\, x \in b\, )
		\end{align}
		が成り立つ.この対偶を取り推論法則\ref{logicalthm:De_Morgan_law_for_quantifiers}を適用すれば
		\begin{align}
			\forall x\, (\, x \notin b\, ) \Longrightarrow a \notin b
		\end{align}
		が得られる.定理\ref{thm:uniqueness_of_emptyset}より
		\begin{align}
			b = \emptyset \Longrightarrow \forall x\, (\, x \notin b\, )
		\end{align}
		も正しいので,含意の推移律から
		\begin{align}
			b = \emptyset \Longrightarrow a \notin b
		\end{align}
		が得られる.
		\QED
	\end{prf}
	
	\begin{screen}
		\begin{dfn}[部分類]
			$a,b$を$\mathcal{L}'$の項とするとき,
			\begin{align}
				a \subset b \overset{\mathrm{def}}{\Longleftrightarrow}
				\forall x\ (\ x \in a \Longrightarrow x \in b\ )
			\end{align}
			と定める.式$a \subset b$を``$a$は$b$の{\bf 部分類}\index{ぶぶんるい@部分類}{\bf (subclass)}である''
			と翻訳し,特に$a$が集合である場合は``$a$は$b$の{\bf 部分集合}\index{ぶぶんしゅうごう@部分集合}{\bf (subset)}である''と翻訳する.
			また次の記号も定める:
			\begin{align}
				a \subsetneq b \defarrow a \subset b \wedge a \neq b.
			\end{align}
		\end{dfn}
	\end{screen}
	
	空虚な真の一例として次の結果を得る.
	
	\begin{screen}
		\begin{thm}[空集合は全ての類に含まれる]\label{thm:emptyset_if_a_subset_of_every_class}
			$a$を類とするとき次が成り立つ:
			\begin{align}
				\emptyset \subset a.
			\end{align}
		\end{thm}
	\end{screen}
	
	\begin{prf}
		$a$を類とする.$\tau$を$\mathcal{L}$の任意の対象とすれば
		\begin{align}
			\tau \notin \emptyset
		\end{align}
		が成り立つから,推論規則\ref{logicalaxm:fundamental_rules_of_inference}を適用して
		\begin{align}
			\tau \notin \emptyset \vee \tau \in a
		\end{align}
		が成り立つ.従って
		\begin{align}
			\tau \in \emptyset \Longrightarrow \tau \in a
		\end{align}
		が成り立ち,$\tau$の任意性と推論法則\ref{logicalthm:fundamental_law_of_universal_quantifier}より
		\begin{align}
			\forall x\, (\, x \in \emptyset \Longrightarrow x \in a\, )
		\end{align}
		が成立する.
		\QED
	\end{prf}
	
	$a \subset b$とは$a$に属する全ての``$\mathcal{L}$の対象''は$b$に属するという定義であったが,
	要素となりうる類は集合であるという公理から,$a$に属する全ての``類''もまた$b$に属する.
	
	\begin{screen}
		\begin{thm}[類はその部分類に属する全ての類を要素に持つ]\label{thm:subclass_contains_all_elements}
			$a,b,c$を類とすれば次が成り立つ:
			\begin{align}
				a \subset b \Longrightarrow (\, c \in a \Longrightarrow c \in b\, ).
			\end{align}
		\end{thm}
	\end{screen}
	
	\begin{prf}	
		いま$a \subset b$が成り立っているとする.このとき
		\begin{align}
			c \in a
		\end{align}
		が成り立っていると仮定すれば,要素の公理より
		\begin{align}
			\set{c}
		\end{align}
		が成り立つ.ここで
		\begin{align}
			\tau \defeq \varepsilon x\, (\, c=x\, )
		\end{align}
		とおくと
		\begin{align}
			c = \tau
		\end{align}
		が成り立つので,相等性の公理より
		\begin{align}
			\tau \in a
		\end{align}
		が成り立ち,$a \subset b$と推論法則\ref{logicalthm:fundamental_law_of_universal_quantifier}から
		\begin{align}
			\tau \in b
		\end{align}
		が従う.再び相等性の公理を適用すれば
		\begin{align}
			c \in b
		\end{align}
		が成り立つので,演繹法則より,$a \subset b$が成り立っている下で
		\begin{align}
			c \in a \Longrightarrow c \in b
		\end{align}
		が成立する.再び演繹法則を適用すれば定理の主張が得られる.
		\QED
	\end{prf}
	
	宇宙$\Univ$は類の一つであった.当然のようであるが,それは最大の類である.
	\begin{screen}
		\begin{thm}[$\Univ$は最大の類である]
			$a$を類とするとき次が成り立つ:
			\begin{align}
				a \subset \Univ.
			\end{align}
		\end{thm}
	\end{screen}
	
	\begin{prf}
		$\tau$を$\mathcal{L}$の任意の対象とすれば,定理\ref{thm:any_class_equals_to_itself}と類の公理より
		\begin{align}
			\tau \in \Univ
		\end{align}
		が成立するので,推論規則\ref{logicalaxm:fundamental_rules_of_inference}より
		\begin{align}
			\tau \notin a \vee \tau \in \Univ
		\end{align}
		が成立する.このとき推論法則\ref{logicalthm:rule_of_inference_3}より
		\begin{align}
			\tau \in a \Longrightarrow \tau \in \Univ
		\end{align}
		が成立し,$\tau$の任意性と推論法則\ref{logicalthm:fundamental_law_of_universal_quantifier}から
		\begin{align}
			\forall x\, (\, x \in a \Longrightarrow x \in \Univ\, )
		\end{align}
		が従う.
		\QED
	\end{prf}
	
	\begin{screen}
		\begin{thm}[互いに互いの部分類となる類同士は等しい]\label{thm:mutually_sub_classes_are_equivalent}
			$a,b$を類とするとき次が成り立つ:
			\begin{align}
				a \subset b \wedge b \subset a \Longleftrightarrow a = b.
			\end{align}
		\end{thm}
	\end{screen}
	
	\begin{sketch}
		$a \subset b \wedge b \subset a$が成り立っていると仮定する.
		このとき$\tau$を$\mathcal{L}$の任意の対象とすれば,
		$a \subset b$と推論法則\ref{logicalthm:fundamental_law_of_universal_quantifier}より
		\begin{align}
			\tau \in a \Longrightarrow \tau \in b
		\end{align}
		が成立し,$b \subset a$と推論法則\ref{logicalthm:fundamental_law_of_universal_quantifier}より
		\begin{align}
			\tau \in b \Longrightarrow \tau \in a
		\end{align}
		が成立するので,
		\begin{align}
			\tau \in a \Longleftrightarrow \tau \in b
		\end{align}
		が成り立つ.$\tau$の任意性と推論法則\ref{logicalthm:fundamental_law_of_universal_quantifier}および
		外延性の公理より
		\begin{align}
			a = b
		\end{align}
		が出るので,演繹法則より
		\begin{align}
			a \subset b \wedge b \subset a \Longrightarrow a = b
		\end{align}
		が得られる.逆に$a = b$が満たされていると仮定するとき,$\tau$を$\mathcal{L}$の任意の対象とすれば
		\begin{align}
			\tau \in a \Longrightarrow \tau \in b
		\end{align}
		と
		\begin{align}
			\tau \in b \Longrightarrow \tau \in a
		\end{align}
		が共に成り立つ. よって推論法則\ref{logicalthm:fundamental_law_of_universal_quantifier}より
		\begin{align}
			a \subset b
		\end{align}
		と
		\begin{align}
			b \subset a
		\end{align}
		が共に従う.よって演繹法則より
		\begin{align}
			a = b \Longrightarrow a \subset b \wedge b \subset a
		\end{align}
		も得られる.
		\QED
	\end{sketch}
	
	\monologue{
		定理\ref{thm:subclass_contains_all_elements}と定理\ref{thm:mutually_sub_classes_are_equivalent}より,
			類$a,b$が$a = b$を満たすならば,$a$と$b$は要素に持つ$\mathcal{L}$の対象のみならず,
			要素に持つ類までも一致するのですね.
	}
	
\section{順序型について}
	$(A,R)$を整列集合とするとき,
	\begin{align}
		x \longmapsto 
		\begin{cases}
			\min{A \backslash \ran{x}} & \mbox{if } \ran{x} \subsetneq A \\
			A & \mbox{o.w.} \\
		\end{cases}
	\end{align}
	なる写像$G$に対して
	\begin{align}
		\forall \alpha\, F(\alpha) = G(\rest{F}{\alpha})
	\end{align}
	なる写像$F$を取り
	\begin{align}
		\alpha \defeq \min{\Set{\alpha \in \ON}{F(\alpha) = A}}
	\end{align}
	とおけば,$\alpha$は$(A,R)$の順序型.
	
\section{超限再帰について}
	$\Univ$上の写像$G$が与えられたら,
	\begin{align}
		F \defeq \Set{(\alpha,x)}{\ord{\alpha} \wedge
		\exists f\, \left(\, f \fon \alpha \wedge
		\forall \beta \in \alpha\, \left(\, f(\beta) = G(\rest{f}{\beta})\, \right)
		\wedge x = G(f)\, \right)}
	\end{align}
	により$F$を定めれば
	\begin{align}
		\forall \alpha\, F(\alpha) = G(\rest{F}{\alpha})
	\end{align}
	が成立する.
	
	\begin{screen}
		任意の順序数$\alpha$および$\alpha$上の写像$f$と$g$に対して,
		\begin{align}
			\forall \beta \in \alpha\,
			\left(\, f(\beta) = G(\rest{f}{\beta})\, \right)
		\end{align}
		かつ
		\begin{align}
			\forall \beta \in \alpha\,
			\left(\, g(\beta) = G(\rest{g}{\beta})\, \right)
		\end{align}
		ならば$f = g$である.
	\end{screen}
	
	まず
	\begin{align}
		f(0) = G(\rest{f}{0}) = G(0) = G(\rest{g}{0}) = g(0)
	\end{align}
	が成り立つ.また
	\begin{align}
		\forall \delta \in \beta\, \left(\, 
		\delta \in \alpha \Longrightarrow f(\delta) = g(\delta)\, \right)
	\end{align}
	ならば,$\beta \in \alpha$であるとき
	\begin{align}
		\rest{f}{\beta} = \rest{g}{\beta}
	\end{align}
	となるので
	\begin{align}
		\beta \in \alpha \Longrightarrow f(\beta) = g(\beta)
	\end{align}
	が成り立つ.ゆえに
	\begin{align}
		f = g
	\end{align}
	が得られる.
	
	\begin{screen}
		任意の順序数$\alpha$に対して,$\alpha$上の写像$f$で
		\begin{align}
			\forall \beta \in \alpha\, \left(\, 
			f(\beta) = G(\rest{f}{\beta})\, \right)
		\end{align}
		を満たすものが取れる.
	\end{screen}
	
	$\alpha = 0$のとき$f \defeq 0$とすればよい.$\alpha$の任意の要素$\beta$に対して
	\begin{align}
		g \fon \beta \wedge \forall \gamma\in \beta\, \left(\, 
		g(\gamma) = G(\rest{g}{\gamma})\, \right)
	\end{align}
	なる$g$が存在するとき,
	\begin{align}
		f \defeq \Set{(\beta,x)}{\beta \in \alpha \wedge
		\exists g\, \left(\, g \fon \beta \wedge
		\forall \gamma \in \beta\, \left(\, g(\gamma) = G(\rest{g}{\gamma})\, \right)
		\wedge x = G(g)\, \right)}
	\end{align}
	と定めれば,$f$は$\alpha$上の写像であって
	\begin{align}
		\forall \beta \in \alpha\, \left(\, 
		f(\beta) = G(\rest{f}{\beta})\, \right)
	\end{align}
	を満たす.
	
	\begin{screen}
		任意の順序数$\alpha$に対して$F(\alpha) = G(\rest{F}{\alpha})$が成り立つ.
	\end{screen}
	
	$\alpha = 0$ならば,$0$上の写像は$0$のみなので
	\begin{align}
		F(0) = G(0) = G(\rest{F}{0})
	\end{align}
	である.
	\begin{align}
		\forall \beta \in \alpha\, F(\beta) = G(\rest{F}{\beta})
	\end{align}
	が成り立っているとき,
	\begin{align}
		\forall \beta \in \alpha\, f(\beta) = G(\rest{f}{\beta})
	\end{align}
	を満たす$\alpha$上の写像$f$を取れば,前の一意性より
	\begin{align}
		f = \rest{F}{\alpha}
	\end{align}
	が成立する.よって
	\begin{align}
		F(\alpha) = G(f) = G(\rest{F}{\alpha})
	\end{align}
	となる.
	\QED
	
\section{自然数の全体について}
	$\Natural$を
	\begin{align}
		\Natural \defeq \Set{\beta}{\mbox{$\alpha \leq \beta$である$\alpha$は
		$0$であるか後続型順序数}}
	\end{align}
	によって定めれば,無限公理より
	\begin{align}
		\set{\Natural}
	\end{align}
	である.また$\ord{\Natural}$と$\limo{\Natural}$も証明できるはず.
	$\Natural$が最小の極限数であることは$\Natural$を定義した論理式より従う.

\chapter{イプシロン定理}
	\section{言語}
	\begin{description}
	\item[{\bf EC}]
	{\bf EC}(Elementary calculus)の言語を$L(EC)$と書く.$L(EC)$の構成要素は
	\begin{description}
		\item[矛盾記号] $\bot$
		\item[論理記号] $\rightharpoondown$, $\vee$, $\wedge$, $\rightarrow$
		\item[述語記号] $=$, $\in$
		\item[変項] $x_{0},x_{1},x_{2},\cdots$
	\end{description}
	とする.変項は$L(EC)$の項であって,また$L(EC)$の項は変項だけである.
	$L(EC)$の式は
	\begin{itemize}
		\item 項$s$と式$t$に対して$\in st$と$= st$は式である.
		\item 式$\varphi$と式$\psi$に対して$\rightharpoondown \varphi,
			\vee \varphi \psi,\ \wedge \varphi \psi, \rightarrow \varphi \psi$
			は式である.
		\item 以上のみが$L(EC)$の式である.
	\end{itemize}
	
	\item[{\bf PC}]
	{\bf PC}(Predicate calculus)の言語を$L(PC)$と書く.$L(PC)$の構成要素は
	\begin{description}
		\item[矛盾記号] $\bot$
		\item[論理記号] $\rightharpoondown$, $\vee$, $\wedge$, $\Longrightarrow$
		\item[量化子] $\forall$, $\exists$
		\item[述語記号] $=$, $\in$
		\item[変項] $x_{0},x_{1},x_{2},\cdots$
	\end{description}
	とする.変項は$L(PC)$の項であって,また$L(PC)$の項は変項だけである.
	$L(PC)$の式は
	\begin{itemize}
		\item 項$s$と式$t$に対して$\in st$と$= st$は式である.
		\item 式$\varphi$と式$\psi$に対して$\rightharpoondown \varphi,
			\vee \varphi \psi,\ \wedge \varphi \psi, \rightarrow \varphi \psi$
			は式である.
		\item 式$\varphi$と変項$x$に対して,$\forall x \varphi$と$\exists x \varphi$は式である.
		\item 以上のみが$L(PC)$の式である.
	\end{itemize}
	\end{description}
	
	$L(EC)$と$L(PC)$に$\varepsilon$項を追加した言語をそれぞれ$L(EC_{\varepsilon}),
	L(PC_{\varepsilon})$とする.
	
	\begin{description}
	\item[{\bf EC${}_{\varepsilon}$}]
	言語$L(EC_{\varepsilon})$の構成要素は
	\begin{description}
		\item[矛盾記号] $\bot$
		\item[論理記号] $\rightharpoondown$, $\vee$, $\wedge$, $\rightarrow$
		\item[述語記号] $=$, $\in$
		\item[変項] $x_{0},x_{1},x_{2},\cdots$
		\item[$\varepsilon$記号] $\varepsilon$
	\end{description}
	とする.$L(EC_{\varepsilon})$の項と式は
	\begin{itemize}
		\item 変項は項である.
		\item 項$s$と式$t$に対して$\in st$と$= st$は式である.
		\item 式$\varphi$と式$\psi$に対して$\rightharpoondown \varphi,
			\vee \varphi \psi,\ \wedge \varphi \psi, \rightarrow \varphi \psi$
			は式である.
		\item 式$\varphi$と変項$x$に対して,$\epsilon x \varphi$は項である.
		\item 以上のみが$L(EC_{\varepsilon})$の項と式である.
	\end{itemize}
	
	\item[{\bf PC${}_{\varepsilon}$}]
	言語$L(PC_{\varepsilon})$の構成要素は
	\begin{description}
		\item[矛盾記号] $\bot$
		\item[論理記号] $\rightharpoondown$, $\vee$, $\wedge$, $\Longrightarrow$
		\item[量化子] $\forall$, $\exists$
		\item[述語記号] $=$, $\in$
		\item[変項] $x_{0},x_{1},x_{2},\cdots$
		\item[$\varepsilon$記号] $\varepsilon$
	\end{description}
	とする.$L(PC_{\varepsilon})$の項と式は
	\begin{itemize}
		\item 変項は項である.
		\item 項$s$と式$t$に対して$\in st$と$= st$は式である.
		\item 式$\varphi$と式$\psi$に対して$\rightharpoondown \varphi,
			\vee \varphi \psi,\ \wedge \varphi \psi, \rightarrow \varphi \psi$
			は式である.
		\item 式$\varphi$と変項$x$に対して,$\forall x \varphi$と$\exists x \varphi$は式である.
		\item 式$\varphi$と変項$x$に対して,$\epsilon x \varphi$は項である.
		\item 以上のみが$L(PC_{\varepsilon})$の項と式である.
	\end{itemize}
	\end{description}
	
\section{証明}
	\begin{description}
	\item[{\bf EC}]\mbox{}
	
	\begin{itembox}[l]{{\bf EC}の公理}
		$\varphi$と$\psi$と$\xi$を$L(EC)$の式とするとき,次は{\bf EC}の公理である.
		\begin{description}
			\item[(S)] $(\varphi \rightarrow (\psi \rightarrow \chi)) 
				\rightarrow ((\varphi \rightarrow \psi)
				\rightarrow (\varphi \rightarrow \chi)).$
			\item[(K)] $\varphi \rightarrow (\psi \rightarrow \varphi).$
			\item[(DI1)] $\varphi \rightarrow (\varphi \vee \psi).$
			\item[(DI2)] $\psi \rightarrow (\varphi \vee \psi).$
			\item[(DE)] $(\varphi \rightarrow \chi) \rightarrow 
				((\psi \rightarrow \chi) \rightarrow ((\varphi \vee \psi) \rightarrow \chi)).$
			\item[(CI)] $\varphi \rightarrow (\psi \rightarrow (\varphi \wedge \psi)).$
			\item[(CE1)] $(\varphi \wedge \psi) \rightarrow \varphi.$
			\item[(CE2)] $(\varphi \wedge \psi) \rightarrow \psi.$
				
			\item[(CTI)] $\varphi \rightarrow (\rightharpoondown \varphi \rightarrow \bot).$
			
			\item[(NI)] $(\varphi \rightarrow \bot) \rightarrow\ \rightharpoondown \varphi.$
			\item[(DNE)] $\rightharpoondown \rightharpoondown \varphi \rightarrow \varphi.$
		\end{description}
	\end{itembox}
	
	$\Gamma$を公理系という場合は,$\Gamma$は$L(EC)$の式の集合である.$\Gamma$が空である場合もある.
	$L(EC)$の式$\chi$に対して$\Gamma$から{\bf EC}の証明が存在する(証明可能である)ことを
	\begin{align}
		\Gamma \provable{\mbox{{\bf EC}}} \chi
	\end{align}
	と書くが,{\bf EC}における$\Gamma$から$\chi$への証明とは,
	$L(EC)$の式の列$\varphi_{1},\varphi_{2},
	\cdots,\varphi_{n}$であって,$\varphi_{n}$は$\chi$であり,
	各$i \in \{1,2,\cdots,n\}$に対して
	\begin{itemize}
		\item $\varphi_{i}$は{\bf EC}の公理である.
		\item $\varphi_{i}$は$\Gamma$の公理である.
		\item $\varphi_{i}$は前の式から推論規則を用いて得られる式である.{\bf EC}の推論規則とは,
			\begin{description}
			\item[三段論法]
				$j,k < i$なる$k,j$が取れて,$\varphi_{k}$は
				$\varphi_{j} \rightarrow \varphi_{i}$である.
		\end{description} 
	\end{itemize}
	が満たされているものである.
	
	\item[{\bf PC}]
	$\varphi$をいずれかの言語の式とし,$x$を変項とする.
	$\varphi$に$x$が自由に現れているとき,$\varphi$に自由に現れている
	$x$を変項$t$で置き換えた式を
	\begin{align}
		\varphi(t/x)
	\end{align}
	とする.ただし$t$は$\varphi(t/x)$で{\bf $x$に置き換わった位置で束縛されない}とする.
	このことを{\bf $t$は$\varphi$の中で$x$への代入について自由である}とも言う.
	
	\begin{itembox}[l]{{\bf PC}の公理}
		$\varphi$と$\psi$と$\xi$を$L(PC)$の式とし,$x$と$t$を変項とするとき,
		次は{\bf PC}の公理である.
		\begin{description}
			\item[(S)] $(\varphi \rightarrow (\psi \rightarrow \chi)) 
				\rightarrow ((\varphi \rightarrow \psi)
				\rightarrow (\varphi \rightarrow \chi)).$
			\item[(K)] $\varphi \rightarrow (\psi \rightarrow \varphi).$
			\item[(DI1)] $\varphi \rightarrow (\varphi \vee \psi).$
			\item[(DI2)] $\psi \rightarrow (\varphi \vee \psi).$
			\item[(DE)] $(\varphi \rightarrow \chi) \rightarrow 
				((\psi \rightarrow \chi) \rightarrow ((\varphi \vee \psi) \rightarrow \chi)).$
			\item[(CI)] $\varphi \rightarrow (\psi \rightarrow (\varphi \wedge \psi)).$
			\item[(CE1)] $(\varphi \wedge \psi) \rightarrow \varphi.$
			\item[(CE2)] $(\varphi \wedge \psi) \rightarrow \psi.$
			
			\item[(UE)] $\forall x \varphi \rightarrow \varphi(\tau/x).$
				\\ \textcolor{red}{ただし,$\varphi$には$x$が自由に現れて,
				$t$は$\varphi$の中で$x$への代入について自由である.}
				
			\item[(EI)] $\varphi(\tau/x) \rightarrow \exists x \varphi.$
				\\ \textcolor{red}{ただし,$\varphi$には$x$が自由に現れて,
				$t$は$\varphi$の中で$x$への代入について自由である.}
				
			\item[(CTI)] $\varphi \rightarrow (\rightharpoondown \varphi \rightarrow \bot).$
			
			\item[(NI)] $(\varphi \rightarrow \bot) \rightarrow\ \rightharpoondown \varphi.$
			\item[(DNE)] $\rightharpoondown \rightharpoondown \varphi \rightarrow \varphi.$
		\end{description}
	\end{itembox}
	
	$\Gamma$を公理系という場合は,$\Gamma$は$L(PC)$の文の集合である.$\Gamma$が空である場合もある.
	$L(PC)$の式$\chi$に対して$\Gamma$から{\bf PC}の証明が存在する(証明可能である)ことを
	\begin{align}
		\Gamma \provable{\mbox{{\bf PC}}} \chi
	\end{align}
	と書くが,{\bf PC}における$\Gamma$から$\chi$への証明とは,
	$L(PC)$の式の列$\varphi_{1},\varphi_{2},
	\cdots,\varphi_{n}$であって,$\varphi_{n}$は$\chi$であり,
	各$i \in \{1,2,\cdots,n\}$に対して
	\begin{itemize}
		\item $\varphi_{i}$は{\bf PC}の公理である.
		\item $\varphi_{i}$は$\Gamma$の公理である.
		\item $\varphi_{i}$は前の式から推論規則を用いて得られる式である.{\bf PC}の推論規則とは,
			\begin{description}
			\item[三段論法]
				$j,k < i$なる$k,j$が取れて,$\varphi_{k}$は
				$\varphi_{j} \rightarrow \varphi_{i}$である.
			 	
			\item[存在汎化] 
				$j < i$なる$j$が取れて,$\varphi_{j}$は
				$\varphi(t/x) \rightarrow \psi$なる式であって,
				$\varphi_{i}$は$\exists x \varphi \rightarrow \psi$なる式である.
				\\ \textcolor{red}{ただし,$\varphi$には$x$が自由に現れ,
				$t$は$\varphi$の中で$x$への代入について自由である.
				また$t$は$\varphi$と$\psi$には自由に現れない.}
			
			\item[全称汎化] 
				$j < i$なる$j$が取れて,$\varphi_{j}$は
				$\psi \rightarrow \varphi(t/x)$なる式であって,
				$\varphi_{i}$は$\psi \rightarrow \forall x \varphi$なる式である.
				\\ \textcolor{red}{ただし,$\varphi$には$x$が自由に現れ,
				$t$は$\varphi$の中で$x$への代入について自由である.
				また$t$は$\varphi$と$\psi$には自由に現れない.}
		\end{description} 
	\end{itemize}
	が満たされているものである.
	
	\item[主要論理式]
	{\bf EC}${}_{\varepsilon}$と{\bf PC}${}_{\varepsilon}$の公理には
	\begin{align}
		\varphi(t/x) \rightarrow \varphi(\varepsilon x \varphi/x)
	\end{align}
	なる形の式が追加される.ただし$x$は$\varphi$に自由に現れて,
	$t$は$\varphi$の中で$x$への代入について自由である.
	この形の式を{\bf 主要論理式}{\bf (principal formula)}と呼ぶ.
	
	\item[{\bf EC}${}_{\varepsilon}$]\mbox{}
	
	\begin{itembox}[l]{{\bf EC}${}_{\varepsilon}$の公理}
		$\varphi$と$\psi$と$\xi$を$L(EC_{\varepsilon})$の式とするとき,
		次は{\bf EC}${}_{\varepsilon}$の公理である.
		\begin{description}
			\item[(S)] $(\varphi \rightarrow (\psi \rightarrow \chi)) 
				\rightarrow ((\varphi \rightarrow \psi)
				\rightarrow (\varphi \rightarrow \chi)).$
			\item[(K)] $\varphi \rightarrow (\psi \rightarrow \varphi).$
			\item[(DI1)] $\varphi \rightarrow (\varphi \vee \psi).$
			\item[(DI2)] $\psi \rightarrow (\varphi \vee \psi).$
			\item[(DE)] $(\varphi \rightarrow \chi) \rightarrow 
				((\psi \rightarrow \chi) \rightarrow ((\varphi \vee \psi) \rightarrow \chi)).$
			\item[(CI)] $\varphi \rightarrow (\psi \rightarrow (\varphi \wedge \psi)).$
			\item[(CE1)] $(\varphi \wedge \psi) \rightarrow \varphi.$
			\item[(CE2)] $(\varphi \wedge \psi) \rightarrow \psi.$
				
			\item[(CTI)] $\varphi \rightarrow (\rightharpoondown \varphi \rightarrow \bot).$
			
			\item[(NI)] $(\varphi \rightarrow \bot) \rightarrow\ \rightharpoondown \varphi.$
			\item[(DNE)] $\rightharpoondown \rightharpoondown \varphi \rightarrow \varphi.$
			\item[(PF)] $\varphi(t/x) \rightarrow \varphi(\varepsilon x \varphi/x).$
				\\ \textcolor{red}{ただし,$\varphi$には$x$が自由に現れて,
				$t$は$\varphi$の中で$x$への代入について自由である.}
		\end{description}
	\end{itembox}
	
	$\Gamma$を公理系とする.$L(EC_{\varepsilon})$の式$\chi$に対して$\Gamma$から
	{\bf EC}${}_{\varepsilon}$の証明が存在する(証明可能である)ことを
	\begin{align}
		\Gamma \provable{\mbox{{\bf EC}${}_{\varepsilon}$}} \chi
	\end{align}
	と書くが,{\bf EC}${}_{\varepsilon}$における$\Gamma$から$\chi$への証明とは,
	$L(EC_{\varepsilon})$の式の列$\varphi_{1},\varphi_{2},
	\cdots,\varphi_{n}$であって,$\varphi_{n}$は$\chi$であり,
	各$i \in \{1,2,\cdots,n\}$に対して
	\begin{itemize}
		\item $\varphi_{i}$は{\bf EC}${}_{\varepsilon}$の公理である.
		\item $\varphi_{i}$は$\Gamma$の公理である.
		\item $\varphi_{i}$は前の式から推論規則を用いて得られる式である.
			{\bf EC}${}_{\varepsilon}$の推論規則とは,
			\begin{description}
			\item[三段論法]
				$j,k < i$なる$k,j$が取れて,$\varphi_{k}$は
				$\varphi_{j} \rightarrow \varphi_{i}$である.
		\end{description} 
	\end{itemize}
	が満たされているものである.
	
	\item[{\bf PC}${}_{\varepsilon}$]\mbox{}
	
	\begin{itembox}[l]{{\bf PC}${}_{\varepsilon}$の公理}
		$\varphi$と$\psi$と$\xi$を$L(PC_{\varepsilon})$の式とし,$x$と$t$を変項とするとき,
		次は{\bf PC}${}_{\varepsilon}$の公理である.
		\begin{description}
			\item[(S)] $(\varphi \rightarrow (\psi \rightarrow \chi)) 
				\rightarrow ((\varphi \rightarrow \psi)
				\rightarrow (\varphi \rightarrow \chi)).$
			\item[(K)] $\varphi \rightarrow (\psi \rightarrow \varphi).$
			\item[(DI1)] $\varphi \rightarrow (\varphi \vee \psi).$
			\item[(DI2)] $\psi \rightarrow (\varphi \vee \psi).$
			\item[(DE)] $(\varphi \rightarrow \chi) \rightarrow 
				((\psi \rightarrow \chi) \rightarrow ((\varphi \vee \psi) \rightarrow \chi)).$
			\item[(CI)] $\varphi \rightarrow (\psi \rightarrow (\varphi \wedge \psi)).$
			\item[(CE1)] $(\varphi \wedge \psi) \rightarrow \varphi.$
			\item[(CE2)] $(\varphi \wedge \psi) \rightarrow \psi.$
			
			\item[(UE)] $\forall x \varphi \rightarrow \varphi(\tau/x).$
				\\ \textcolor{red}{ただし,$\varphi$には$x$が自由に現れて,
				$t$は$\varphi$の中で$x$への代入について自由である.}
				
			\item[(EI)] $\varphi(\tau/x) \rightarrow \exists x \varphi.$
				\\ \textcolor{red}{ただし,$\varphi$には$x$が自由に現れて,
				$t$は$\varphi$の中で$x$への代入について自由である.}
				
			\item[(CTI)] $\varphi \rightarrow (\rightharpoondown \varphi \rightarrow \bot).$
			
			\item[(NI)] $(\varphi \rightarrow \bot) \rightarrow\ \rightharpoondown \varphi.$
			\item[(DNE)] $\rightharpoondown \rightharpoondown \varphi \rightarrow \varphi.$
			\item[(PF)] $\varphi(t/x) \rightarrow \varphi(\varepsilon x \varphi/x).$
				\\ \textcolor{red}{ただし,$\varphi$には$x$が自由に現れて,
				$t$は$\varphi$の中で$x$への代入について自由である.}
		\end{description}
	\end{itembox}
	
	$\Gamma$を公理系とする.$L(PC_{\varepsilon})$の式$\chi$に対して
	$\Gamma$から{\bf PC}${}_{\varepsilon}$の証明が存在する(証明可能である)ことを
	\begin{align}
		\Gamma \provable{\mbox{{\bf PC}${}_{\varepsilon}$}} \chi
	\end{align}
	と書くが,{\bf PC}${}_{\varepsilon}$における$\Gamma$から$\chi$への証明とは,
	$L(PC_{\varepsilon})$の式の列$\varphi_{1},\varphi_{2},
	\cdots,\varphi_{n}$であって,$\varphi_{n}$は$\chi$であり,
	各$i \in \{1,2,\cdots,n\}$に対して
	\begin{itemize}
		\item $\varphi_{i}$は{\bf PC}${}_{\varepsilon}$の公理である.
		\item $\varphi_{i}$は$\Gamma$の公理である.
		\item $\varphi_{i}$は前の式から推論規則を用いて得られる式である.
			{\bf PC}${}_{\varepsilon}$の推論規則とは,
			\begin{description}
			\item[三段論法]
				$j,k < i$なる$k,j$が取れて,$\varphi_{k}$は
				$\varphi_{j} \rightarrow \varphi_{i}$である.
			 	
			\item[存在汎化] 
				$j < i$なる$j$が取れて,$\varphi_{j}$は
				$\varphi(t/x) \rightarrow \psi$なる式であって,
				$\varphi_{i}$は$\exists x \varphi \rightarrow \psi$なる式である.
				\\ \textcolor{red}{ただし,$\varphi$には$x$が自由に現れ,
				$t$は$\varphi$の中で$x$への代入について自由である.
				また$t$は$\varphi$と$\psi$には自由に現れない.}
			
			\item[全称汎化] 
				$j < i$なる$j$が取れて,$\varphi_{j}$は
				$\psi \rightarrow \varphi(t/x)$なる式であって,
				$\varphi_{i}$は$\psi \rightarrow \forall x \varphi$なる式である.
				\\ \textcolor{red}{ただし,$\varphi$には$x$が自由に現れ,
				$t$は$\varphi$の中で$x$への代入について自由である.
				また$t$は$\varphi$と$\psi$には自由に現れない.}
		\end{description} 
	\end{itemize}
	が満たされているものである.
	\end{description}
	\section{第一イプシロン定理メモ}
	
	言語$L(EC)$及び$L(EC_{\varepsilon})$を高橋先生の資料と同じものとする.
	{\bf 主要論理式}\index{しゅようろんりしき@主要論理式}{\bf (principal formula)}とは
	\begin{align}
		A(t) \Longrightarrow A(\varepsilon x A)
	\end{align}
	なる形の$L(EC)$の式を指す.ここで$A$とは$L(EC)$の式であって,変項$x$が$A$に自由に現れていて,
	また$A$に自由に出現するのは$x$のみである.$A(t)$とは$A$における$x$の自由な出現を全て閉項$t$に置き換えた式であり,
	$A(\varepsilon x A)$とは$A$における$x$の自由な出現を全て項$\varepsilon x A$に置き換えた式である.
	このとき$\varepsilon x A$は$A(t) \Longrightarrow A(\varepsilon x A)$に{\bf 属している}という.
	
	$EC$の公理とはトートロジーだけである.トートロジーは$EC_{\varepsilon}$の公理でもあるが,
	これに加えて主要論理式も$EC_{\varepsilon}$の公理である.
	
	$\pi = (\varphi_{0},\varphi_{1},\cdots,\varphi_{n})$を$EC_{\varepsilon}$の文の列とするとき,
	{\bf $\pi$の主要論理式}や{\bf $\pi$に現れる主要論理式}とは主要論理式である$\varphi_{i}$を指す.
	また$\pi$の主要論理式に属している$\varepsilon$項を{\bf $\pi$の主要$\varepsilon$項}と呼ぶ.
	
\subsection{埋め込み定理}
	$A$を$L(PC_{\varepsilon})$の式とするとき,$A$を$L(EC_{\varepsilon})$の式に書き換える.
	\begin{align}
		x^{\varepsilon} &\rightarrow x \\
		(\in \tau \sigma)^{\varepsilon} &\rightarrow \in \tau^{\varepsilon} \sigma^{\varepsilon} \\
		(= \tau \sigma)^{\varepsilon} &\rightarrow = \tau^{\varepsilon} \sigma^{\varepsilon} \\
		(\rightharpoondown \varphi)^{\varepsilon} &\rightarrow \rightharpoondown \varphi^{\varepsilon} \\
		(\vee \varphi \psi)^{\varepsilon} &\rightarrow \vee \varphi^{\varepsilon} \psi^{\varepsilon} \\
		(\wedge \varphi \psi)^{\varepsilon} &\rightarrow \wedge \varphi^{\varepsilon} \psi^{\varepsilon} \\
		(\Longrightarrow \varphi \psi)^{\varepsilon} &\rightarrow \Longrightarrow \varphi^{\varepsilon} \psi^{\varepsilon} \\
		(\exists x \varphi)^{\varepsilon} &\rightarrow \varphi^{\varepsilon}(\varepsilon x \varphi^{\varepsilon}) \\
		(\forall x \varphi)^{\varepsilon} &\rightarrow \varphi^{\varepsilon}(\varepsilon x \rightharpoondown \varphi^{\varepsilon}) \\
		(\varepsilon x \psi)^{\varepsilon} &\rightarrow \varepsilon x \varphi^{\varepsilon}
	\end{align}
	
	$A$が$L(PC_{\varepsilon})$の式で,$x$が$A$に自由に現れて,
	かつ$A$に自由に現れているのが$x$のみであるとき,
	$A^{\varepsilon}$にも$x$が自由に現れて,かつ$A^{\varepsilon}$に
	自由に現れているのは$x$のみである.
	
	\begin{align}
		(\varphi[x/\tau])^{\varepsilon} \rightarrow \varphi^{\varepsilon}
		(\varphi^{\varepsilon}[x/\tau^{\varepsilon}]). \\
	\end{align}
	
	\begin{itembox}[c]{$PC_{\varepsilon}$の証明を$EC_{\varepsilon}$の証明に埋め込む}
		$A$を$L(PC_{\varepsilon})$の文とし,$PC_{\varepsilon} \vdash A$であるとする.
		このとき$EC_{\varepsilon} \vdash A^{\varepsilon}$である.
	\end{itembox}
	
	示すべきことは
	\begin{itemize}
		\item $A \in Ax(PC_{\varepsilon})$ならば$\vdash A^{\varepsilon}$であること.
			\begin{itemize}
				\item $\vdash A$ならば$\vdash A^{\varepsilon}$であること.
				\item $A$に$x$が自由に現れて,かつ自由に現れているのが$x$のみであるとき,
					\begin{align}
						\vdash A^{\varepsilon}(t^{\varepsilon}) \Longrightarrow A^{\varepsilon}(\varepsilon x A^{\varepsilon})
					\end{align}
					であること.
				\item $A$に$x$が自由に現れて,かつ自由に現れているのが$x$のみであるとき,
					\begin{align}
						\vdash A^{\varepsilon}(\varepsilon x \rightharpoondown A^{\varepsilon}) \Longrightarrow A^{\varepsilon}(t^{\varepsilon})
					\end{align}
					であること.
			\end{itemize}
		
		\item $PC_{\varepsilon} \vdash B$かつ$PC_{\varepsilon} \vdash B \Longrightarrow A$である$B$が取れるとき,
			$(B \Longrightarrow A)^{\varepsilon}$は$B^{\varepsilon} \Longrightarrow A^{\varepsilon}$なので
			$EC_{\varepsilon} \vdash B^{\varepsilon}$ならば
			$EC_{\varepsilon} \vdash A^{\varepsilon}$となる.
	\end{itemize}

\subsection{階数}
	$B(x,y,z)$を,変項$x,y,z$が,そしてこれらのみが自由に現れる
	$L(EC)$の式とする.このとき
	\begin{align}
		\exists x\, \exists y\, \exists z\, B(x,y,z)
	\end{align}
	に対して,$z$から順に$\varepsilon$項に変換していくと
	\begin{align}
		&\exists x\, \exists y\, B(x,y,\varepsilon z B(x,y,z)), \\
		&\exists x\, \, B(x,\varepsilon y B(x,y,\varepsilon z B(x,y,z)),\varepsilon z B(x,\varepsilon y B(x,y,\varepsilon z B(x,y,z)),z))
	\end{align}
	となるが,最後に$\exists x$を無くすと式が長くなりすぎるのでここで止めておく.
	さて$z$に注目すれば,$B$に自由に現れていた$z$はまず
	\begin{align}
		\varepsilon z B(x,y,z)
	\end{align}
	に置き換えられている.この時点では$x,y$は自由なままであるから,この$\varepsilon$項を
	\begin{align}
		e_{1}[x,y]
	\end{align}
	と略記する.次に$y$は
	\begin{align}
		\varepsilon y B(x,y,\varepsilon z B(x,y,z))
	\end{align}
	に置き換えられるが,$e_{1}[x,y]$を使えば
	\begin{align}
		\varepsilon y B(x,y,e_{1}[x,y])
	\end{align}
	と書ける.この$\varepsilon$項でも$x$は自由なままであるから
	\begin{align}
		e_{2}[x]
	\end{align}
	と略記する.最後に
	\begin{align}
		\exists x\, B\left(x,e_{2}[x],e_{1}[x,e_{2}[x]]\right)
	\end{align}
	から$\exists$を除去するには,$x$を
	\begin{align}
		\varepsilon x B\left(x,e_{2}[x],e_{1}[x,e_{2}[x]]\right)
	\end{align}
	に置き換えれば良い.この$\varepsilon$項を$e_{3}$と書く.以上で
	$\exists x\, \exists y\, \exists z\, B(x,y,z)$は$L(EC_{\varepsilon})$の式
	\begin{align}
		B\left(e_{3},e_{2}[e_{3}],e_{1}[e_{3},e_{2}[e_{3}]]\right)
	\end{align}
	に変換されたわけである.それはさておき,ここで考察するのは{\bf 項間の主従関係}である.
	$e_{2}[x]$は$x$のみによってコントロールされているのだから,
	$x$を司る$e_{3}$を主人だと思えば$e_{2}[x]$は$e_{3}$の直属の子分である.
	$e_{1}[x,y]$は$y$によってもコントロールされているので,
	$e_{1}[x,y]$とは$e_{2}[x]$の子分であり,すなわち$e_{3}$の子分の子分であって,
	この例において一番身分が低いわけである.
	
	$\varepsilon$項を構文解析して,それが何重の子分を従えているかを測った指標を
	{\bf 階数}{\bf (rank)}と呼ぶ.とはいえ直属の子分が複数いることもあり得るので,
	子分の子分の子分の子分...と次々に枝分かれしていく従属関係の中で,最も
	深いものを辿って階数を定めることにする.
	
	\begin{screen}
		\begin{metadfn}[従属]
			$\varepsilon x A$を$L(EC_{\varepsilon})$の$\varepsilon$項とし,
			$e$を$A$に現れる$L(EC_{\varepsilon})$の$\varepsilon$項とするとき,
			$x$が$e$に自由に現れているなら$e$は{\bf $\varepsilon x A$に従属している}
			\index{じゅうぞく@従属}{\bf (subordinate to $\varepsilon x A$)}という.
		\end{metadfn}
	\end{screen}
	
	はじめの例では,$e_{2}[x]$と$e_{1}[x,e_{2}[x]]$は共に$e_{3}$に従属しているし,
	$e_{1}[x,y]$は$e_{2}[x]$に従属している.
	$e_{1}[x,e_{2}[x]]$に従属している$\varepsilon$項は無いし,
	$e_{1}[x,y]$に従属している$\varepsilon$項も無い.
	
	\begin{screen}
		\begin{metadfn}[階数]
			$\theta$を$L(EC_{\varepsilon})$の項または式とするとき,
			$\theta$の{\bf 階数}\index{かいすう@階数}{\bf (rank)}を
			以下の要領で定義する.
			\begin{enumerate}
				\item $\theta$が$\varepsilon$項でなくて,
					$\theta$に$\varepsilon$項が現れないならば,
					$\theta$の階数を$0$とする.
				\item $\theta$が$\varepsilon$項であって,かつ$\theta$に従属している
					$\varepsilon$項が無いならば,$\theta$の階数を$1$とする.
				\item $\theta$が$\varepsilon$項であって,かつ$\theta$に従属している
					$\varepsilon$項があるならば,$\theta$に従属している$\varepsilon$項の
					階数の最大値に$1$を足したものを$\theta$の階数とする.
				\item $\theta$が$\varepsilon$項でなくて,$\theta$に
					$\varepsilon$項が現れるならば,$\theta$に現れる$\varepsilon$項の
					階数の最大値を$\theta$の階数とする.
			\end{enumerate}
			また$\theta$の階数を$rk(\theta)$と書く.
		\end{metadfn}
	\end{screen}
	
	実際に$L(EC_{\varepsilon})$の全ての項および式に対して階数が定まっている.
	(構造的帰納法について準備不足だが,直感的に次の説明は妥当である...)
	\begin{description}
		\item[step1] $\theta$が$L(EC)$の項あるいは式ならば,$\theta$の階数は$0$である.
		
		\item[step2] $\theta$が$L(EC)$の式で作られた$\varepsilon$項ならば
			$\theta$の階数は$1$である.
		
		\item[step3] 項$\tau_{1},\cdots,\tau_{n}$のそれぞれに対して,
			その全ての部分$\varepsilon$項に階数が定まっていれば,
			$f$を$n$項関数として,$f\tau_{1}\cdots\tau_{n}$の階数は
			$rk(\tau_{1}),\cdots,rk(\tau_{n})$の中の最大値である.
			というのも,$f\tau_{1}\cdots\tau_{n}$に現れる$\varepsilon$項は
			$\tau_{1},\cdots,\tau_{n}$のいずれかの部分項になっているためである.
			
		\item[step4] 項$\tau_{1},\cdots,\tau_{n}$のそれぞれに対して,
			その全ての部分$\varepsilon$項に階数が定まっていれば,
			$p$を$n$項述語として,$p\tau_{1}\cdots\tau_{n}$の階数は
			$rk(\tau_{1}),\cdots,rk(\tau_{n})$の中の最大値である.
		
		\item[step5] 式$\varphi$と$\psi$のそれぞれに対して,
			その全ての部分$\varepsilon$項に階数が定まっていれば,
			\begin{align}
				rk(\rightharpoondown \varphi) &\coloneqq rk(\varphi), \\
				rk(\vee \varphi \psi) &\coloneqq \max\{rk(\varphi),rk(\psi)\}, \\
				rk(\wedge \varphi \psi) &\coloneqq \max\{rk(\varphi),rk(\psi)\}, \\
				rk(\Longrightarrow \varphi \psi) 
				&\coloneqq \max\{rk(\varphi),rk(\psi)\}, \\
			\end{align}
			である.というのも,左辺の式に現れる$\varepsilon$項は
			$\varphi$か$\psi$の少なくとも一方に現れているからである.
		
		\item[step6] 式$\varphi$に現れる全ての$\varepsilon$項に対して階数が定まっているならば,
			$\varepsilon x \varphi$の階数は定義通りに定めることが出来る.
	\end{description}
	
	\begin{screen}
		\begin{metathm}[階数定理]
			$\varepsilon x A$を$L(EC_{\varepsilon})$の$\varepsilon$項とし,
			$s$と$t$を,その中に$x$が自由に現れない$L(EC_{\varepsilon})$の項とする.このとき,
			$A$に現れる$s$の一つを$t$に置き換えた式を$A^{t}$とすれば
			\begin{align}
				rk(\varepsilon x A) = rk(\varepsilon x A^{t})
			\end{align}
			が成り立つ.$A$に$e$が現れなければ$A^{t}$は$A$とする.
		\end{metathm}
	\end{screen}
	
	\begin{screen}
		\begin{metathm}[置換定理]
			$\pi$を$L(EC_{\varepsilon})$の証明とし,
			$e$を,$\pi$の主要$\varepsilon$項の中で階数が最大であって,かつ
			階数が最大の$\pi$の主要$\varepsilon$項の中で極大であるものとする.また
			$B(s) \Longrightarrow B(\varepsilon y B)$を$\pi$の主要論理式とし,
			$e$と$\varepsilon y B$は別物であるとする.そして,$B(s) \Longrightarrow B(\varepsilon y B)$に現れる
			$e$を全て閉項$t$に置き換えた式を$C$とする.このとき,
			\begin{description}
				\item[(1)] $C$は主要論理式である.$C$に属する$\varepsilon$項を$e'$と書く.
				\item[(2)] $rk(\varepsilon y B) = rk(e')$が成り立つ.
				\item[(3)] $rk(\varepsilon y B) = rk(e)$ならば$\varepsilon y B$と$e'$は一致する.
			\end{description}
		\end{metathm}
	\end{screen}
	
	\begin{metaprf}\mbox{}
		\begin{description}
			\item[step1]
				$B(s)$ (或いは$B(\varepsilon y B)$)とは,
				$B$で自由に現れる$y$を$s$ (或いは$\varepsilon y B$)で置き換えた式である.
				$y$から代わった$s$ (或いは$\varepsilon y B$)の少なくとも一つを部分項として含む形で
				$e$が$B(s)$ (或いは$B(\varepsilon y B)$)に出現しているとする.
				
				実はこれは起こり得ない.もし起きたとすると,$e$に現れる$s$ (或いは$\varepsilon y B$)
				を元の$y$に戻した項を$e'$とすれば,
				$e'$には$y$が自由に現れるので(そうでないと$y$は$s$
				(或いは$\varepsilon y B$)に置き換えられない),$e'$は
				$y$とは別の変項$x$と適当な式$A$によって
				\begin{align}
					\varepsilon x A
				\end{align}
				なる形をしている.つまり$e'$は$\varepsilon y B$に従属していることになり
				\footnote{
					$e'$が$\varepsilon$項であって$B$に現れることの証明.
				}
				,階数定理と併せて
				\begin{align}
					rk(e) = rk(e') < rk(\varepsilon y B)
				\end{align}
				が成り立ってしまう.しかしこれは$rk(e)$が最大であることに矛盾する.
				
			\item[step2] $rk(\varepsilon y B) = rk(\pi)$ならば$B$に$e$は現れない.なぜならば,
				$e$は階数が$rk(\pi)$である$\pi$の主要$\varepsilon$項の中で極大であるからである.
				$\varepsilon y B$にも$e$は現れず,前段の結果より$B(\varepsilon y B)$に$e$が現れることもない.
				ゆえに,$s$に現れる$e$を$t$に置換した項を$s'$とすれば,$C$は
				\begin{align}
					B(s') \Longrightarrow B(\varepsilon y B)
				\end{align}
				となる.
			
			\item[step3]
				$rk(\varepsilon y B) < rk(\pi)$である場合
				\begin{align}
					rk(\varepsilon y B) = rk(e')
				\end{align}
				が成り立つことを示す.$B$に$e$が現れないならば$e'$は$\varepsilon y B$に一致する.
				$B$に$e$が現れる場合,$B$に現れる$e$を$t$に置き換えた式を$B^{t}$とする.
				このとき階数定理より
				\begin{align}
					rk(B) = rk(B^{t})
				\end{align}
				となる.ゆえに
				\begin{align}
					rk(\varepsilon y B) = rk(B) + 1 = rk(B^{t}) + 1 = rk(\varepsilon y B^{t})
				\end{align}
				となる.
				\QED
		\end{description}
	\end{metaprf}
	
\subsection{アイデア}
	
	\begin{itembox}[l]{第一イプシロン定理の流れ}
		\begin{itemize}
			\item $B$を$EC$の式とし,$B$が$PC_{\varepsilon}$から証明可能であるとする.
			\item このとき$EC_{\varepsilon}から$$B$への証明$\pi$が得られる.
			\item $e$を,$\pi$の主要$\varepsilon$項のうち階数が最大であって,かつ
				その階数を持つ$\pi$の主要$\varepsilon$項の中で次数が最大であるものとする.
			\item $e$が属する$\pi$の主要論理式の一つ$A(t) \Longrightarrow A(e)$を取る.
			\item $\pi$をベースにして,$A(t) \Longrightarrow A(e)$を用いずに
				$EC_{\varepsilon}$から$B$への証明$\pi'$を構成する.このとき以下が満たされる.
				\begin{enumerate}
					\item $A(t) \Longrightarrow A(e)$を除く$\pi$の主要論理式は
						$\pi'$の主要論理式である.
					\item また$e$が属する主要論理式については,それが$\pi'$の主要論理式であるならば$\pi$の主要論理式
						でもある.つまり,直感的に書けば
						\begin{align}
							&\Set{\varphi}{\mbox{$\varphi$は$e$が属する$\pi'$の主要論理式}} \\
							&= \Set{\varphi}{\mbox{$\varphi$は$e$が属する$\pi$の主要論理式}} \backslash \{A(t) \Longrightarrow A(e)\}
						\end{align}
						が成り立つということであって,$e$が属する主要論理式は減る一方である.
						$\pi$の主要論理式で$e$が属しているものが$A(t) \Longrightarrow A(e)$のみ
						であるならば,$\pi'$には$e$が属する主要論理式は現れない.
					
					\item $e$が属する主要論理式が$\pi'$にも残っている場合,
						$e$は$\pi'$の主要$\varepsilon$項の中も階数が最大であって,
						かつその階数を持つ$\pi'$の主要$\varepsilon$項の中で極大である.
					
					\item $\pi'$の主要$\varepsilon$項のうち,階数が$e$と同じであるものは
						$\pi$の主要$\varepsilon$項でもあった($e$と同じ階数の$\varepsilon$項は増えない).
				\end{enumerate}
				
			\item 証明$\pi$の主要$\varepsilon$項の階数の最大値を$rk(\pi)$とする.
				また主要論理式に属する$\varepsilon$項の階数を,その主要論理式の階数と呼ぶことにする.
				前段の操作を続けていけば,まずは階数$rk(\pi)$の主要論理式を全く用いない
				$B$への証明$\pi_{1}$が得られる.このとき$rk(\pi_{1})$は
				$rk(\pi)$よりも小さい.同様にして階数$rk(\pi_{1})$の主要論理式を全く用いない
				$B$への証明$\pi_{2}$が得られる.もちろん$rk(\pi_{2})$は
				$rk(\pi_{1})$よりも小さい.これを繰り返していけば,いずれは主要論理式を全く用いない
				$B$への証明$\pi^{\ast}$が得られる.$\pi^{\ast}$にはトートロジーか
				モーダスポンネスで導かれる式しかない.
				あとは,$\pi^{\ast}$に現れる$\varepsilon$項を$EC$の項に置き換えれば,その式の列は
				$EC$から$B$への証明となっている.
		\end{itemize}
	\end{itembox}
	
	$\pi$を$\varphi_{0},\varphi_{1},\cdots,\varphi_{n}$とし,
	$\varphi_{0},\varphi_{1},\cdots,\varphi_{n}$に現れる$e$を$t$に置き換えた式を
	\begin{align}
		\tilde{\varphi}_{0},\ \tilde{\varphi}_{1},\cdots, \tilde{\varphi}_{n}
	\end{align}
	と書く($e$は,どれかの項の部分項であるときも置き換える?).
	このとき,任意の$0 \leq i \leq n$で
	\begin{enumerate}
		\item $\varphi_{i}$がトートロジーなら$\tilde{\varphi}_{i}$もトートロジーである.
		\item $\varphi_{i}$が主要論理式で,$e$が$\varphi_{i}$の主要項であるならば,
			$\tilde{\varphi}_{i}$は$A(u) \Longrightarrow A(t)$なる形の式である
			\footnotemark.
		\item $\varphi_{i}$が主要論理式で,$e$が$\varphi_{i}$の主要項ではないならば,
			$\tilde{\varphi}_{i}$も主要論理式である.
	\end{enumerate}
	
	\footnotetext{
		$\varepsilon x A$と$\varepsilon y B$が記号列として一致すれば,
		$x$と$y$は一致するし,式$A$と式$B$も一致するので
		$A(\varepsilon x A)$と$B(\varepsilon y B)$も記号列として一致する.
	}
	
	$\varphi$が$A(t) \Longrightarrow A(e)$でない$EC_{\varepsilon}$の公理ならば,
	$\tilde{\varphi}_{i}$と$\tilde{\varphi}_{i+1}$の間に
	\begin{align}
		&\tilde{\varphi}_{i} \Longrightarrow 
		\left( A(t) \Longrightarrow \tilde{\varphi}_{i} \right), \\
		&A(t) \Longrightarrow \tilde{\varphi}_{i}
	\end{align}
	を挿入する.$\varphi_{i}$が$\varphi_{j}$と$\varphi_{k}$からモーダスポンネスで得られる場合は,
	$\tilde{\varphi}_{i}$を
	\begin{align}
		&\left( A(t) \Longrightarrow \tilde{\varphi}_{j} \right)
		\Longrightarrow \left[ \left( A(t) \Longrightarrow 
		\left( \tilde{\varphi}_{j}\Longrightarrow \tilde{\varphi}_{i} \right) \right)
		\Longrightarrow \left( A(t) \Longrightarrow \tilde{\varphi}_{i} \right) \right], \\
		&\left( A(t) \Longrightarrow 
		\left( \tilde{\varphi}_{j}\Longrightarrow \tilde{\varphi}_{i} \right) \right)
		\Longrightarrow \left( A(t) \Longrightarrow \tilde{\varphi}_{i} \right), \\
		&A(t) \Longrightarrow \tilde{\varphi}_{i}
	\end{align}
	で置き換える.すると,$A(t) \Longrightarrow A(e)$を使わない
	$EC_{\varepsilon}$から$A(t) \Longrightarrow B$への証明が得られる.
	$\varphi_{i}$が$e$が属する主要論理式$A(s) \Longrightarrow A(e)$であるときは,
	$\tilde{\varphi}_{i}$とは
	\begin{align}
		A(s') \Longrightarrow A(t)
	\end{align}
	なる形の式であるが
	\footnote{
		$x$を$A$に現れている自由な変項とすれば,$e$とは$\varepsilon x A$のことであるし,
		$A(\varepsilon x A)$とは$A$に自由に現れる$x$を$\varepsilon x A$に置換した式である.
		$A$には$\varepsilon x A$は現れていないので,というのも$\varepsilon x A$が登場するのは
		$A$が作られた後であるからだが,$A(e)$に現れる$e$を$t$に変換した式は
		$A(t)$になる.同様に,$A(s)$に$e$が現れるとすれば,その$e$は$y$に代入された$s$の
		部分項でしかありえない.すなわち,$A(s)$に現れる$e$を$t$で置換した式は,
		$s'$を$s$に現れる$e$を$t$に変換した項として ($s$に$e$が現れなければ$s'$は$s$である)
		$A(s')$となるわけである.
	},$\tilde{\varphi}_{i}$を
	\begin{align}
		&A(t) \Longrightarrow (A(s') \Longrightarrow A(t)), \\
		&A(s') \Longrightarrow A(t)
	\end{align}
	で置き換える.
	
	同様に$A(t) \Longrightarrow A(e)$を使わない$EC_{\varepsilon})$から
	$\rightharpoondown A(t) \Longrightarrow B$への証明を構成する.
	今度は$\pi$に現れる$e$を$t$に置き換える必要はない.
	$\varphi_{i}$が$A(t) \Longrightarrow A(e)$でない$EC_{\varepsilon}$の公理ならば,
	$\varphi_{i}$と$\varphi_{i+1}$の間に
	\begin{align}
		&\varphi_{i} \Longrightarrow (\rightharpoondown A(t) \Longrightarrow \varphi_{i}), \\
		&\rightharpoondown A(t) \Longrightarrow \varphi_{i}
	\end{align}
	を挿入する.$\varphi_{i}$が$\varphi_{j}$と$\varphi_{k}$からモーダスポンネスで得られる場合は,
	$\varphi_{i}$を
	\begin{align}
		&(\rightharpoondown A(t) \Longrightarrow \varphi_{j}) \Longrightarrow
		[(\rightharpoondown A(t) \Longrightarrow 
		(\varphi_{j}\Longrightarrow \varphi_{i}))
		\Longrightarrow (\rightharpoondown A(t) \Longrightarrow \varphi_{i})], \\
		&(\rightharpoondown A(t) \Longrightarrow 
		(\varphi_{j} \Longrightarrow \varphi_{i}))
		\Longrightarrow (\rightharpoondown A(t) \Longrightarrow \varphi_{i}), \\
		&\rightharpoondown A(t) \Longrightarrow \varphi_{i}
	\end{align}
	で置き換える.$\varphi_{i}$が$A(t) \Longrightarrow A(e)$であるときは,$\varphi_{i}$を
	\begin{align}
		\rightharpoondown A(t) \Longrightarrow (A(t) \Longrightarrow A(e))
	\end{align}
	で置き換える.
	
	以上で$A(t) \Longrightarrow B$と$\rightharpoondown A(t) \Longrightarrow B$に対して
	$A(t) \Longrightarrow A(e)$を用いない$EC_{\varepsilon}$からの証明が得られた.後はこれに
	\begin{align}
		&(A(t) \Longrightarrow B) \Longrightarrow
		((\rightharpoondown A(t) \Longrightarrow B) \Longrightarrow
		((A(t) \Longrightarrow B) \wedge (\rightharpoondown A(t) \Longrightarrow B))), \\
		&(\rightharpoondown A(t) \Longrightarrow B) \Longrightarrow
		((A(t) \Longrightarrow B) \wedge (\rightharpoondown A(t) \Longrightarrow B)), \\
		&(A(t) \Longrightarrow B) \wedge (\rightharpoondown A(t) \Longrightarrow B), \\
		&((A(t) \Longrightarrow B) \wedge (\rightharpoondown A(t) \Longrightarrow B))
		\Longrightarrow ((A(t) \vee \rightharpoondown A(t)) \Longrightarrow B), \\
		&(A(t) \vee \rightharpoondown A(t)) \Longrightarrow B, \\
		&A(t) \vee \rightharpoondown A(t), \\
		&B
	\end{align}
	を追加すれば,$A(t) \Longrightarrow A(e)$を用いない$EC_{\varepsilon}$から$B$への証明となる.
	
	\section{第二イプシロン定理}
	$\exists x \forall y \exists z B(x,y,z)$を$L(PC)$の冠頭標準形とする.
	つまり$B(x,y,z)$には量化子が現れないので,$B(x,y,z)$は$L(EC)$の式ということである.
	また
	\begin{align}
		PC_{\varepsilon} \vdash \exists x \forall y \exists z B(x,y,z)
	\end{align}
	であるとする.
	
	$f$を$L(PC)$には無い一変数関数記号とし,
	\begin{align}
		L'(PC) &\defeq L(PC) \cup \{f\}, \\
		L'(EC) &\defeq L(EC) \cup \{f\}, \\
		L'(PC_{\varepsilon}) &\defeq L(PC_{\varepsilon}) \cup \{f\}, \\
		L'(EC_{\varepsilon}) &\defeq L(EC_{\varepsilon}) \cup \{f\}
	\end{align}
	とする.このとき明らかに
	\begin{align}
		{PC'}_{\varepsilon} \vdash \exists x \forall y \exists z B(x,y,z)
	\end{align}
	であるが(ただし${PC'}_{\varepsilon} \vdash$とは$L'(PC_{\varepsilon})$の
	式からなる証明が存在するという意味),
	\begin{align}
		{PC'}_{\varepsilon} &\vdash \exists x \forall y \exists z B(x,y,z), \\
		{PC'}_{\varepsilon} &\vdash \exists x \forall y \exists z B(x,y,z)
		\rightarrow \forall y \exists z B(\tau,y,z), && 
		(\tau \defeq \varepsilon x \forall y \exists z B(x,y,z)) \\
		{PC'}_{\varepsilon} &\vdash \forall y \exists z B(\tau,y,z), \\
		{PC'}_{\varepsilon} &\vdash \forall y \exists z B(\tau,y,z)
		\rightarrow \exists z B(\tau,f(\tau),z), \\
		{PC'}_{\varepsilon} &\vdash \exists z B(\tau,f(\tau),z), \\
		{PC'}_{\varepsilon} &\vdash \exists z B(\tau,f(\tau),z)
		\rightarrow \exists x \exists z B(x,f(x),z), \\
		{PC'}_{\varepsilon} &\vdash \exists x \exists z B(x,f(x),z)
	\end{align}
	が成り立つ.すると拡張第一イプシロン定理より,$p$個の$L'(EC)$の項$r_{i}$
	と,同じく$p$個の$L'(EC)$の項$s_{i}$が取れて,
	\begin{align}
		{EC'}_{\varepsilon} \vdash \bigvee_{i=1}^{p} B(r_{i},f(r_{i}),s_{i})
	\end{align}
	となる.同じ証明で
	\begin{align}
		{PC'}_{\varepsilon} \vdash \bigvee_{i=1}^{p} B(r_{i},f(r_{i}),s_{i})
	\end{align}
	であることも言える.
	\begin{align}
		{PC'}_{\varepsilon} \vdash \bigvee_{i=1}^{p-1} B(r_{i},f(r_{i}),s_{i})
		\vee B(r_{p},f(r_{p}),s_{p})
	\end{align}
	より,まず
	\begin{align}
		{PC'}_{\varepsilon} \vdash \bigvee_{i=1}^{p-1} B(r_{i},f(r_{i}),s_{i})
		\vee \exists z B(r_{p},f(r_{p}),z)
	\end{align}
	となる.続いて,$f(r_{p})$は$\bigvee_{i=1}^{p-1} B(r_{i},f(r_{i}),s_{i})$には現れないので
	\begin{align}
		{PC'}_{\varepsilon} \vdash \bigvee_{i=1}^{p-1} B(r_{i},f(r_{i}),s_{i})
		\vee \forall y \exists z B(r_{p},y,z)
	\end{align}
	となる.最後に
	\begin{align}
		{PC'}_{\varepsilon} \vdash \bigvee_{i=1}^{p-1} B(r_{i},f(r_{i}),s_{i})
		\vee \exists x \forall y \exists z B(x,y,z)
	\end{align}
	となる.これを繰り返せば
	\begin{align}
		{PC'}_{\varepsilon} \vdash \exists x \forall y \exists z B(x,y,z)
		\vee \cdots \vee \exists x \forall y \exists z B(x,y,z)
	\end{align}
	が得られるので
	\begin{align}
		{PC'}_{\varepsilon} \vdash \exists x \forall y \exists z B(x,y,z)
	\end{align}
	となる.最後に,$\exists x \forall y \exists z B(x,y,z)$への証明に残っている
	$f$を含む項を$L(PC)$の項に置き換えれば,$L(PC)$から$\exists x \forall y \exists z B(x,y,z)$
	への証明が得られる.

\chapter{メモ}
	\section{量化再考}
	\begin{enumerate}
		\item $\forall y\, \left(\, \forall x \varphi(x) \Longrightarrow \varphi(y)\, \right)$
		\item $\forall x\, \left(\, \varphi(x) \Longrightarrow \exists y \varphi(y)\, \right)$
		\item $\forall y\, \left(\, \varphi \Longrightarrow \psi(y)\, \right)
			\Longrightarrow \left(\, \varphi \Longrightarrow \forall y \psi(y)\, \right)$
		\item $\forall x\, \left(\, \varphi(x) \Longrightarrow \psi\, \right)
			\Longrightarrow \left(\, \exists x \varphi(x) \Longrightarrow \psi\, \right)$
		\item $\forall x\,  \left(\, \varphi(x) \Longrightarrow \psi(x)\, \right)
			\Longrightarrow \left(\, \forall x \varphi(x) \Longrightarrow \forall x \psi(x)\, \right)$
			
		\item $\forall x \varphi(x) \Longrightarrow \exists x \varphi(x)$
		\item $\forall x \rightharpoondown \varphi(x) \Longleftrightarrow
			\ \rightharpoondown \exists x \varphi(x)$
	\end{enumerate}
	
	\begin{screen}
		\begin{align}
			\forall x \varphi(x) \Longrightarrow \forall y \varphi(y).
		\end{align}
	\end{screen}
	
	\begin{align}
		&\forall y\, \left(\, \forall x \varphi(x) \Longrightarrow \varphi(y)\, \right)
		&& \mbox{(公理1)} \\
		&\forall y\, \left(\, \forall x \varphi(x) \Longrightarrow \varphi(y)\, \right)
		\Longrightarrow \left(\, \forall x \varphi(x) \Longrightarrow \forall y \varphi(y)\, \right),
		&& \mbox{(公理3)} \\
		&\forall x \varphi(x) \Longrightarrow \forall y \varphi(y).
		&& \mbox{(MP)}
	\end{align}
	
	\begin{screen}
		\begin{align}
			\exists x \varphi(x) \Longrightarrow \exists y \varphi(y).
		\end{align}
	\end{screen}
	
	\begin{align}
		&\forall x\, \left(\, \varphi(x) \Longrightarrow \exists y \varphi(y)\, \right)
		&& \mbox{(公理2)} \\
		&\forall x\, \left(\, \varphi(x) \Longrightarrow \exists y \varphi(y)\, \right)
		\Longrightarrow \left(\, \exists x \varphi(x) \Longrightarrow \exists y \varphi(y)\, \right),
		&& \mbox{(公理4)} \\
		&\exists x \varphi(x) \Longrightarrow \exists y \varphi(y).
		&& \mbox{(MP)}
	\end{align}
	\section{置換公理}
	置換公理の二つの形式の同値性をざっくりと.
	\begin{description}
		\item[(T)] $\sing{f} \Longrightarrow \forall a\, \set{f \ast a}.$
		\item[(K)] $\forall a\, \left[\, \forall x \in a\, \exists!y \varphi(x,y)
				\Longrightarrow \exists z\, \forall y\,
				(\, y \in z \Longleftrightarrow \exists x\, (\, x \in a \wedge 
				\varphi(x,y)\, )\, )\, \right].$
	\end{description}
	
	ただし
	\begin{align}
		\sing{f} &\defarrow \forall x,y,z\, (\, (x,y) \in f \wedge (x,z) \in f
		\Longrightarrow y = z\, ), \\
		f \ast a &\defeq \Set{y}{\exists x \in a\, (\, (x,y) \in f\, )}, \\
		\set{s} \defarrow \exists x\, (\, s = x\, )
	\end{align}
	であるし,$\varphi$に自由に現れているのは二つの変項のみで,それらが$s$と$t$とおけば,
	$\varphi$に自由に現れている$s$を全て$x$に,
	$\varphi$に自由に現れている$t$を全て$y$に置き換えた式が
	\begin{align}
		\varphi(x,y)
	\end{align}
	である.またこのとき$x$も$y$も$\varphi(x,y)$で束縛されていないものとする
	($x$と$y$はそのように選ばれた変項であるということである).
	
	\begin{description}
		\item[(T)$\Longrightarrow$(K)]
			$a$を任意の集合とし,
			\begin{align}
				\forall x \in a \exists!y \varphi(x,y)
			\end{align}
			であるとする.
			\begin{align}
				f \defeq \Set{(x,y)}{x \in a \wedge \varphi(x,y)}
			\end{align}
			とおけば$f$は$a$上の写像であって,(T)より
			\begin{align}
				\exists z\, (\, z = f \ast a\, )
			\end{align}
			となる.ところで$f \ast a$とは
			\begin{align}
				\Set{y}{\exists x \in a\, (\, (x,y) \in f\, )}
			\end{align}
			なので
			\begin{align}
				f \ast a = \Set{y}{\exists x \in a \varphi(x,y)}.
			\end{align}
			ゆえに
			\begin{align}
				\exists z\, \forall y\, (\, y \in z \Longleftrightarrow
				\exists x \in a \varphi(x,y)\, )
			\end{align}
			が成り立つ.
			
		\item[(K)$\Longrightarrow$(T)]
			$\sing{f}$とし,$a$を集合とする.
			\begin{align}
				b \defeq a \cap \dom{f}
			\end{align}
			とおけば,(K)からは分出公理が示せるので$b$は集合である.そして
			\begin{align}
				\forall x \in b\, \exists!y\, (\, (x,y) \in f\, )
			\end{align}
			が成り立つのだから,(K)より
			\begin{align}
				z = \Set{y}{\exists x \in b\, (\, (x,y) \in f\, )}
			\end{align}
			が従う.ここで
			\begin{align}
				\Set{y}{\exists x \in b\, (\, (x,y) \in f\, )}
				= f \ast b
				= f \ast a
			\end{align}
			であるから(T)が得られる.
			\QED
	\end{description}

\chapter{Hilbert流証明論}
	\section{Hilbert流証明論メモ}
	参考文献: 戸次大介「数理論理学」
	
	\begin{itembox}[l]{{\bf SK}の公理}
		\begin{description}
			\item[(S)] $(\varphi \rightarrow (\psi \rightarrow \chi)) 
				\rightarrow ((\varphi \rightarrow \psi)
				\rightarrow (\varphi \rightarrow \chi)).$
			
			\item[(K)] $\varphi \rightarrow (\psi \rightarrow \varphi).$
		\end{description}
	\end{itembox}
	
	{\bf SK}から証明可能な式
	\begin{description}
		\item[(I)] $\varphi \rightarrow \varphi$
		\item[(B)] $(\psi \rightarrow \chi) \rightarrow ((\varphi \rightarrow \psi) \rightarrow (\varphi \rightarrow \chi)).$
		\item[(C)] $(\varphi \rightarrow (\psi \rightarrow \chi)) \rightarrow (\psi \rightarrow (\varphi \rightarrow \chi)).$
		\item[(W)] $(\varphi \rightarrow (\varphi \rightarrow \psi)) \rightarrow (\varphi \rightarrow \psi).$
		\item[(B')] $(\varphi \rightarrow \psi) \rightarrow ((\psi \rightarrow \chi) \rightarrow (\varphi \rightarrow \chi)).$
		\item[(C$\ast$)] $\varphi \rightarrow ((\varphi \rightarrow \psi) \rightarrow \psi)$
	\end{description}
	
	\begin{itembox}[l]{否定の追加}
		\begin{description}
			\item[(CTI1)] $\varphi \rightarrow (\rightharpoondown \varphi \rightarrow \bot).$
			
			\item[(CTI2)] $\rightharpoondown \varphi \rightarrow (\varphi \rightarrow \bot).$
			
			\item[(NI)] $(\varphi \rightarrow \bot) \rightarrow\ \rightharpoondown \varphi.$
		\end{description}
	\end{itembox}
	
	このとき証明可能な式
	\begin{description}
		\item[(DNI)] $\varphi \rightarrow\ \rightharpoondown \rightharpoondown \varphi.$
		\item[(CON1)] $(\varphi \rightarrow \psi) \rightarrow (\rightharpoondown \psi \rightarrow\ \rightharpoondown \varphi).$
		\item[(CON2)] $(\varphi \rightarrow\ \rightharpoondown \psi) \rightarrow (\psi \rightarrow\ \rightharpoondown \varphi).$
	\end{description}
	
	\begin{itembox}[l]{{\bf HM}の公理}
		\begin{description}
			\item[(S)] $(\varphi \rightarrow (\psi \rightarrow \chi)) 
				\rightarrow ((\varphi \rightarrow \psi)
				\rightarrow (\varphi \rightarrow \chi)).$
			\item[(K)] $\varphi \rightarrow (\psi \rightarrow \varphi).$
			\item[(DI1)] $\varphi \rightarrow (\varphi \vee \psi).$
			\item[(DI2)] $\psi \rightarrow (\varphi \vee \psi).$
			\item[(DE)] $(\varphi \rightarrow \chi) \rightarrow 
				((\psi \rightarrow \chi) \rightarrow ((\varphi \vee \psi) \rightarrow \chi)).$
			\item[(CI)] $\varphi \rightarrow (\psi \rightarrow (\varphi \wedge \psi)).$
			\item[(CE1)] $(\varphi \wedge \psi) \rightarrow \varphi.$
			\item[(CE2)] $(\varphi \wedge \psi) \rightarrow \psi.$
			\item[(UI)] $\forall \zeta (\psi \rightarrow \varphi[\zeta/\xi]) 
				\rightarrow (\psi \rightarrow \forall \xi \varphi).$
			\item[(UE)] $\forall \xi \varphi \rightarrow \varphi[\tau/\xi].$
			\item[(EI)] $\varphi[\tau/\xi] \rightarrow \exists \xi \varphi.$
			\item[(EE)] $\forall \zeta (\varphi[\zeta/\xi] \rightarrow \psi)
				\rightarrow (\exists \xi \varphi \rightarrow \psi).$
		\end{description}
	\end{itembox}
	
	{\bf HM}から証明可能な式
	\begin{description}
		\item[LNC] $\rightharpoondown (\varphi \wedge \rightharpoondown \varphi).$
		\item[(DIST$\wedge$)] $\varphi \vee (\psi \wedge \chi) 
			\leftrightarrow (\varphi \vee \psi) \wedge (\varphi \vee \chi).$
		\item[(DIST$\vee$)] $\varphi \wedge (\psi \vee \chi) 
			\leftrightarrow (\varphi \wedge \psi) \vee (\varphi \wedge \chi).$
		\item[(DM$\vee$)] $\rightharpoondown (\varphi \vee \psi) \leftrightarrow
			\ \rightharpoondown \varphi \wedge \rightharpoondown \psi.$
	\end{description}
	
	\begin{sketch}[(LNC)]
		\begin{align}
			\varphi \wedge \rightharpoondown \varphi &\provable{\mbox{{\bf HM}}} \varphi, \\
			\varphi \wedge \rightharpoondown \varphi &\provable{\mbox{{\bf HM}}}\ \rightharpoondown \varphi, \\
			\varphi \wedge \rightharpoondown \varphi &\provable{\mbox{{\bf HM}}}
				\varphi \rightarrow (\rightharpoondown \varphi \rightarrow \bot), \\
			\varphi \wedge \rightharpoondown \varphi &\provable{\mbox{{\bf HM}}}\ \rightharpoondown \varphi \rightarrow \bot, \\
			\varphi \wedge \rightharpoondown \varphi &\provable{\mbox{{\bf HM}}} \bot, \\
			&\provable{\mbox{{\bf HM}}} (\varphi \wedge \rightharpoondown \varphi) \rightarrow \bot, \\
			&\provable{\mbox{{\bf HM}}} ((\varphi \wedge \rightharpoondown \varphi) \rightarrow \bot)
				\rightarrow\ \rightharpoondown (\varphi \wedge \rightharpoondown \varphi), \\
			&\provable{\mbox{{\bf HM}}}\ \rightharpoondown (\varphi \wedge \rightharpoondown \varphi).
		\end{align}
		\QED
	\end{sketch}
	
	\begin{sketch}[(DM$\vee$)]
		\begin{align}
			&\provable{\mbox{{\bf HM}}} \varphi \rightarrow (\varphi \vee \psi), && \mbox{(DI1)}\\
			&\provable{\mbox{{\bf HM}}} (\varphi \rightarrow (\varphi \vee \psi))
				\rightarrow (\rightharpoondown (\varphi \vee \psi) \rightarrow\ \rightharpoondown \varphi), 
				&& \mbox{(CON1)}\\
			&\provable{\mbox{{\bf HM}}}\ \rightharpoondown (\varphi \vee \psi) \rightarrow\ \rightharpoondown \varphi, 
				&& \mbox{(MP)}\\
			\rightharpoondown (\varphi \vee \psi) &\provable{\mbox{{\bf HM}}}\ \rightharpoondown \varphi.
				&& \mbox{(DR)}
		\end{align}
		同様に
		\begin{align}
			\rightharpoondown (\varphi \vee \psi) \provable{\mbox{{\bf HM}}}\ \rightharpoondown \psi
		\end{align}
		となり,
		\begin{align}
			\rightharpoondown (\varphi \vee \psi) &\provable{\mbox{{\bf HM}}}\ \rightharpoondown \varphi
				\rightarrow (\rightharpoondown \psi \rightarrow 
				(\rightharpoondown \varphi \wedge \rightharpoondown \psi)), && \mbox{(CI)}\\
			\rightharpoondown (\varphi \vee \psi) &\provable{\mbox{{\bf HM}}}\ 
				\rightharpoondown \psi \rightarrow (\rightharpoondown \varphi \wedge \rightharpoondown \psi), 
				&& \mbox{(MP)}\\
			\rightharpoondown (\varphi \vee \psi) &\provable{\mbox{{\bf HM}}}\ 
				\rightharpoondown \varphi \wedge \rightharpoondown \psi && \mbox{(MP)}
		\end{align}
		が得られる.逆に
		\begin{align}
			\rightharpoondown \varphi \wedge \rightharpoondown \psi &\provable{\mbox{{\bf HM}}}\ \rightharpoondown \varphi, 
				&& \mbox{(CE1)}\\
			\rightharpoondown \varphi \wedge \rightharpoondown \psi &\provable{\mbox{{\bf HM}}}\ 
			\rightharpoondown \varphi \rightarrow (\varphi \rightarrow \bot), && \mbox{(CTI2)}\\
			\rightharpoondown \varphi \wedge \rightharpoondown \psi &\provable{\mbox{{\bf HM}}} \varphi \rightarrow \bot
				&& \mbox{(MP)}
		\end{align}
		となり,同様に
		\begin{align}
			\rightharpoondown \varphi \wedge \rightharpoondown \psi \provable{\mbox{{\bf HM}}} \psi \rightarrow \bot
		\end{align}
		も成り立つ.よって
		\begin{align}
			\rightharpoondown \varphi \wedge \rightharpoondown \psi &\provable{\mbox{{\bf HM}}} 
				(\varphi \rightarrow \bot) \rightarrow ((\psi \rightarrow \bot) 
				\rightarrow ((\varphi \vee \psi) \rightarrow \bot)), && \mbox{(DE)}\\
			\rightharpoondown \varphi \wedge \rightharpoondown \psi &\provable{\mbox{{\bf HM}}} 
				(\psi \rightarrow \bot) \rightarrow ((\varphi \vee \psi) \rightarrow \bot), && \mbox{(MP)}\\
			\rightharpoondown \varphi \wedge \rightharpoondown \psi &\provable{\mbox{{\bf HM}}} 
				(\varphi \vee \psi) \rightarrow \bot, && \mbox{(MP)}\\
			\rightharpoondown \varphi \wedge \rightharpoondown \psi &\provable{\mbox{{\bf HM}}} 
				((\varphi \vee \psi) \rightarrow \bot) \rightarrow\ \rightharpoondown (\varphi \vee \psi), && \mbox{(NI)}\\
			\rightharpoondown \varphi \wedge \rightharpoondown \psi &\provable{\mbox{{\bf HM}}} 
				\ \rightharpoondown (\varphi \vee \psi) && \mbox{(MP)}
		\end{align}
		が得られる.
		\QED
	\end{sketch}

\begin{thebibliography}{数字}
	\bibitem{key1} Moser, G. and Zach, R., ``The Epsilon Calculus and Herbrand Complexity'',
		Studia Logica 82, 133-155 (2006)
	
	\bibitem{key2} 高橋優太, ``1階述語論理に対する$\varepsilon$計算'', \\
		http://www2.kobe-u.ac.jp/~mkikuchi/ss2018files/takahashi1.pdf 
		
	\bibitem{key3} キューネン数学基礎論講義
	
	\bibitem{key5} ブルバキ, 数学原論 集合論 1, 
	
	\bibitem{key4} 竹内外史, 現代集合論入門, 増強版第5刷, 日本評論社, 2016, pp. 138-183, ISBN 978-4-535-60116-1
	
	\bibitem{key6} 島内剛一, 数学の基礎, 第1版第10刷, 日本評論社, 2016, ISBN 978-4-535-60106-2
	
	\bibitem{key7} 戸次大介, 数理論理学, 第2刷, 東京大学出版会, 2016, pp. 148-166, ISBN 978-4-13-062915-7
	
	\bibitem{key8} K. G$\ddot{\mbox{o}}$del, $The\ Consistency\ of\ the\ Continuum\ Hypothesis$, 8th printing, Princeton University Press 1970, p. 3, ISBN 0-691-07927-7.
	
	\bibitem{key9} 菊地誠, 不完全性定理, 初版3刷, 共立出版株式会社, 2017, pp. 86-91, ISBN 978-4-320-11096-0
	
	\bibitem{key10} 前原昭二, 記号論理入門, 新装版第8刷, 日本評論社, 2018, pp. 106-115, ISBN 4-535-60144-5
	
	\bibitem{key11} Kenji Miyamoto and Georg Moser, The Epsilon Calculus with Equality and Herbrand Complexity
\end{thebibliography}

\end{document}
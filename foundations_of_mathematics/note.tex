\documentclass[a4j,10.5pt,oneside,openany]{jsbook}
%
\usepackage{amsmath,amssymb}
\usepackage{amsthm}
\usepackage{newpxmath,newpxtext}
\usepackage{ascmac} %itembox
\usepackage{mathrsfs} %花文字
\usepackage{mathtools} %参照式のみ式番号表示

\usepackage[dvipdfmx]{graphicx} %作画
\usepackage{tikz} %作画

\setlength{\textwidth}{\fullwidth}
\setlength{\textheight}{40\baselineskip}
\addtolength{\textheight}{\topskip}
%\setlength{\voffset}{-0.55in}
%
%
\begin{document}
\mathtoolsset{showonlyrefs = true}

\section{徒然なるままに支離滅裂}

「数学は人間に属するか,神に属するか」という疑問に,私は「数学は人間に属するが,ただし人間は神に属している」と回答しましょう.
ここで言う「神」とは宗教上の絶対的存在を指しているのではなくて,いやもしかすると宗教上の神と同一かもしれませんが,
「三段論法」や「帰納法」や「背理法」など``人間が直感的に信じている論理''を用意した存在の呼び名です.

基礎論における証明は大抵が直感に頼っているように見えますが,ではその直感が正しいとは誰が保証するのでしょうか.
手元にあるどの本でも保証されていません.もしかしたら神様という超然的な存在を暗黙の裡に認めていて,
直感とは神様が用意した論理であるとして無断で使っているだけなのかもしれませんが,
残念ながら読者はテレパシーを使えないので,筆者の暗黙の了解を推察するなんて困難です.

しかしながら,暗黙の了解を排除しようとすると,その分だけ日本語による明示的な約束が必要になります.
すると新たな問題が生じます.それは日本語で書かれた言明をどこまで信用するか,という問題です.
基礎論の難しさは,その表面上のややこしさよりも日本語に対する認識を揃えることにあるのでしょうか.

論理構造を集合論の結果を用いて解明しようというのならまだしも(こちらは数理論理学と呼ばれる分野で,本来は数学基礎論とは別物だそうです),
集合論を構築することが目的である場合,その土台となる基礎論を集合論の上に展開すると理論が循環することになるでしょう.
基礎論が基礎にしている集合論は「メタ理論」と呼ばれるらしいですが,
その「メタ理論」がどう構成されたのかという点には誰も全く言及していないのですから,
「メタ理論」という言葉は単なる逃げ口上にしか聞こえず,理論の循環を解消できません.

\section{言語}
	言語$\mathcal{L}$
	\begin{description}
		\item[矛盾記号] $\bot$
		\item[論理記号]  $\rightharpoondown$, $\vee$, $\wedge$, $\rightarrow$,
			$\forall$, $\exists$
		\item[述語記号] $=$, $\in$
		\item[使用文字] ローマ字及びギリシア文字.
		\item[定項記号] 何らかの記号.
	\end{description}

\section{項}
	
	使用文字のアルファベットは変項記号として使われる.だが,アルファベットだけを変項記号とするのは不十分であり,
	例えば式に$200$個の変項が必要であるといった状況では,小文字や大文字や変体文字まで駆使しても不足する.
	そこで,文字$x$と$y$に対して
	\begin{align}
		\ddagger xy
	\end{align}
	もまた変項であると約束する.これは
	\begin{align}
		\ddagger(x,y)
	\end{align}
	や
	\begin{align}
		x \ddagger y
	\end{align}
	などと書く方が見やすいかもしれないが,当面は始めの記法(前置記法やポーランド記法と呼ばれる)を用いる.
	また,変項$\tau$と$\sigma$に対して
	\begin{align}
		\ddagger \tau \sigma
	\end{align}
	も変項であると約束しよう.この約束に従えば,文字$x$だけを用いたとしても
	\begin{align}
		x,\quad \ddagger xx, \quad \ddagger \ddagger xxx, \quad \ddagger \ddagger \ddagger xxxx
	\end{align}
	はいずれも変項ということになる.しかも,仮に「$1000$個の変項を用意してくれ」と頼まれたとしても
	$\ddagger$と$x$のみを使って$1000$個の変項を作り出すことが出来る.
	
	大切なのは,$\ddagger$を用いれば''理論上は''変項に不足しないということであって,
	具体的な数式を扱うときに$\ddagger$が出てくるかと言えば,否である.
	$\ddagger$が必要になるほどに長い式を人間が解釈するのは困難であるから,
	通常は何らかの略記法を導入して複雑なところを覆い隠してしまう.

	\begin{itembox}[l]{項の定義}
		\begin{itemize}
			\item 文字は変項である.また変項$\tau,\sigma$に対して$\ddagger \tau \sigma$は変項である.
				そしてこれらのみが変項である.
				
			\item 変項は項であり,定項も項である.そしてこれらのみが項である.
		\end{itemize}
	\end{itembox}
	
	言わずもがな,「これらのみが項である」とは,「$\tau$が項である」という言明が与えられたらば以下のいずれかが満たされるということである.
	\begin{itemize}
		\item $\tau$は文字である.
		\item $\tau$は定項である.
		\item 変項$x$と$y$が得られて,$\tau$は$\ddagger xy$である.
	\end{itemize}
	
\section{式}

	\begin{itembox}[l]{式の定義}
		\begin{itemize}
			\item $\bot$は式である.
			\item 項$s$と$t$に対して$\in st$と$=st$は式である.これを原子式と呼ぶ.
			\item $\varphi$を式とするとき$\rightharpoondown \varphi$は式である.
			\item $\varphi$と$\psi$を式とするとき,以下はいずれも式である.
				\begin{align}
					&\vee \varphi \psi, \\
					&\wedge \varphi \psi, \\
					&\rightarrow \varphi \psi.
				\end{align}
			
			\item 変項$x$が式$\varphi$に現れるとき,
				$\forall x \varphi$と$\exists x \varphi$は式である.
			
			\item 以上のみが式である.
		\end{itemize}
	\end{itembox}
	
	項の場合と同様に,「以上のみが式である」とは,「$\varphi$が式である」という言明が与えられたらば
	以下のいずれかが満たされるということである.
	\begin{itemize}
		\item $\varphi$は$\bot$である.
		\item 項$s$と$t$が得られて$\varphi$は$\in s t$である.
		\item 項$s$と$t$が得られて$\varphi$は$= s t$である.
		\item 式$\psi$が得られて$\varphi$は$\rightharpoondown \psi$である.
		\item 式$\psi$と$\xi$が得られて$\varphi$は$\vee \psi \xi$である.
		\item 式$\psi$と$\xi$が得られて$\varphi$は$\wedge \psi \xi$である.
		\item 式$\psi$と$\xi$が得られて$\varphi$は$\rightarrow \psi \xi$である.
		\item 変項$x$が現れる式$\psi$が得られて$\varphi$は$\forall x \psi$である.
		\item 変項$x$が現れる式$\psi$が得られて$\varphi$は$\exists x \psi$である.
	\end{itemize}
	
\section{部分式}
	式は,式同士を組み合わせて作られている.
	式から切り取った一連の部分列で,それ自身が式であるものを元の式に対して部分式と呼ぶ.
	元の式全体も部分式と捉えられるが,自分自身を除く部分式を特に真部分式と呼ぶことにする.
	
\section{始切片}
	$\varphi$を式とするとき,$\varphi$の左端を揃えて切り取る部分列を
	$\varphi$の始切片と呼ぶ.
	$\varphi$とは記号を並べたものであるが,それを下図のようにイメージすれば,
	黒く塗りつぶした部分に相当する記号の並びは$\varphi$の始切片である.
	
	\begin{center}
	\begin{tikzpicture}
		\draw (0,0)--(10,0);
		\draw (0,0)--(0,-1);
		\draw (10,0)--(10,-1);
		\draw (0,-1)--(10,-1);
		\filldraw[fill=black] (0,0)--(7,0)--(7,-1)--(0,-1)--(0,0);
	\end{tikzpicture}
	\end{center}
	
	本節の主題は次である.
	\begin{screen}
		(★) $\varphi$を式とするとき,$\varphi$の始切片で式であるものは$\varphi$自身に限られる.
	\end{screen}
	
	これを示すには次の原理を用いる:
	\begin{itembox}[l]{構成的帰納法}
		式に対する言明に対し,
		\begin{itemize}
			\item $\bot$に対してその言明が妥当である.
			\item 原子式に対してその言明が妥当である.
			\item 式が任意に与えられた\footnotemark
				ときに,その全ての真部分式に対して
				その言明が当てはまるならば,その式自身に対してもその言明が当てはまる.
		\end{itemize}
		ならば,いかなる式に対してもその言明は当てはまる.
	\end{itembox}
	
	\footnotetext{
		``任意に与えられた式''とはどう解釈するべきか.
		どんな式に対しても?
	}
	
	では主題を証明する.
	$\bot$については,その始切片は$\bot$に限られる.
	$\in st$なる原子式については,その始切片は
	\begin{align}
		\in, \quad \in s, \quad \in st
	\end{align}
	のいずれかとなるが,このうち式であるものは$\in st$のみである.
	$=st$なる原子式についても,その始切片で式であるものは$=st$に限られる.
	
	$\varphi$を任意に与えられた式とし,
	$\varphi$の真部分式に対しては(★)が当てはまっているとする.
	\begin{description}
		\item[ケース1] 式$\psi$が得られて$\varphi$が$\rightharpoondown \psi$であるとき,
			$\psi$は$\varphi$の真部分式であるので(★)は当てはまる.
			$\varphi$の始切片で式であるものは,
			式$\xi$を用いて$\rightharpoondown \xi$と表せるが,
			$\xi$は$\psi$の始切片であるから,帰納法の仮定より$\xi$と$\psi$は一致する.
			ゆえに$\varphi$の始切片で式であるものは$\varphi$自身に限られる.
			
		\item[ケース2] 式$\psi$と$\xi$が得られて$\varphi$が$\vee \psi \xi$であるとき,
			$\varphi$の始切片で式であるものは,式$\psi'$と$\xi'$を用いて
			$\vee \psi' \xi'$と表せるが,$\psi$と$\xi$,$\psi'$と$\xi'$は
			いずれも$\varphi$の真部分式であるので(★)は当てはまる.
			このとき$\psi$と$\psi'$は一方が他方の始切片であるので帰納法の仮定より一致する.
			すると$\xi$と$\xi'$も一方が他方の始切片であるので帰納法の仮定より一致する.
			ゆえに$\varphi$の始切片で式であるものは$\varphi$自身に限られる.
			$\varphi$が$\wedge \psi \xi$や$\rightarrow \psi \xi$である場合も同じである.
			
		\item[ケース3] 変項$x$と式$\psi$が得られて$\varphi$が$\forall x \psi$であるとき,
			$\varphi$の始切片で式であるものは,式$\xi$を用いて$\forall x \xi$と表せる.
			このとき$\xi$は$\psi$の始切片であって,$\psi$は$\varphi$の真部分式であって
			(★)が当てはまるので$\psi$と$\xi$は一致する.ゆえに
			$\varphi$の始切片で式であるものは$\varphi$自身に限られる.
			$\varphi$が$\forall x \psi$である場合も同じである.
	\end{description}
	
\section{言語の拡張}
	言語$\mathcal{L}$を言語$\mathcal{L}'$に拡張する.拡張する際に
	\begin{align}
		\{, \quad |, \quad \}
	\end{align}
	の記号を新しく導入する.式の対象化.
	
	\begin{description}
		\item[項] $\mathcal{L}$の項は項である.
			また$A$を$\mathcal{L}$の式とし,変項$x$が$A$に現れて,
			かつ$x$のみが$A$で束縛されていないとき,$\{x|A\}$は項である.
		
		\item[式] $\mathcal{L}$の式と同じ作り方として略.
	\end{description}
	
	$\varphi$を$\mathcal{L}'$の式とし,$s$を$\varphi$に現れる記号とすると,
	\begin{description}
		\item[(0)] $s$は$\mathcal{L}$の定項である.
		\item[(1)] $s$は文字である.
		\item[(2)] $s$は$\ddagger$である.
		\item[(2)] $s$は$\{$である.
		\item[(3)] $s$は$|$である.
		\item[(4)] $s$は$\}$である.
		\item[(5)] $s$は$\bot$である.
		\item[(6)] $s$は$\in$か$=$である.
		\item[(7)] $s$は$\rightharpoondown$である.
		\item[(8)] $s$は$\vee,\wedge,\rightarrow$のいずれかである.
	\end{description}
	
	\begin{screen}
		(★★) いま,$\varphi$を任意に与えられた式としよう.
		\begin{itemize}
			\item $\ddagger$が$\varphi$に現れたとき,$\mathcal{L}$の項$\tau$と$\sigma$が得られて,$\ddagger \tau \sigma$は
				$\ddagger$のその出現位置から始まる$\mathcal{L}$の項となる.
				また$\ddagger$のその出現位置から始まる$\mathcal{L}$の項は$\ddagger \tau \sigma$のみである.
				
			\item $\{$が$\varphi$に現れたとき,$\mathcal{L}$の変項$x$及び$\mathcal{L}$の式$A$が得られて,
				$\{ x|A\}$は$\{$のその出現位置から始まる項となる.
				また$\{$のその出現位置から始まる項は$\{x|A\}$のみである.
				
			\item $|$が$\varphi$に現れたとき,,変項$x$と$\mathcal{L}$の式$A$が得られて,
				$\{x|A\}$は$|$のその出現位置から広がる項となる.
				また$|$のその出現位置から広がる項は$\{x|A\}$のみである.
				
			\item $\}$が$\varphi$に現れたとき,変項$x$と式$A$が得られて,
				$\{x|A\}$は$\}$のその出現位置を終点とする項となる.
				また$\}$のその出現位置を終点とする項は$\{x|A\}$のみである.
		\end{itemize}
	\end{screen}
	
	\begin{description}
		\item[$\ddagger$に対して$\ddagger \tau \sigma$なる変項$\tau$と$\sigma$が得られること]
			$\ddagger$が原子項に現れたら,原子項とは文字$x,y$によって
			\begin{align}
				\ddagger xy
			\end{align}
			と表されるものであるから,$\ddagger$に対して変項$\tau,\sigma$ (すなわち文字$x,y$)が取れたことになる.
			$\ddagger$が項に現れたとする.項とは,変項$x,y$によって
			\begin{align}
				\ddagger xy
			\end{align}
			で表されるものであり,$\ddagger$は左端の$\ddagger$であるか,$x$に現れるか,$y$に現れる.
			$\ddagger$が$x$か$y$に現れるときは帰納法の仮定により,
			$\ddagger$が左端のものである場合は$x$が$\tau$,$y$が$\sigma$ということになる.
			
		\item[変項の始切片で変項であるものは自分自身のみ]
			$x$が文字である場合はそう.$x$の任意の部分変項が言明を満たしているなら,
			$x$は$\ddagger st$なる変項である(生成規則)から,$x$の始切片は$\ddagger uv$なる変項である.
			$s,t,u,v$はいずれも$x$の部分変項なので仮定が適用されている.
			ゆえに$s$と$u$は一方が他方の始切片であり,一致する.すなわち$t$と$v$も一方が他方の始切片であり一致する.
			ゆえに$x$の始切片で変項であるものは$x$自身である.
			
		\item[$\ddagger$に対して得られる変項の一意性]
			$\ddagger xy$と$\ddagger st$が共に変項であるとき,$x$と$s$,$y$と$t$は一致するか.
			$\ddagger xy$が原子項であるときは明らかである.
			$x$の始切片で変項であるものは$x$自身に限られるので,
			$x$と$s$は一致する.ゆえに$t$は$y$の始切片であり,$t$と$y$も一致する.
		
		\item[生成規則より$x$と$A$が得られるか]
			$\varphi$が原子式であるとき,
			$\{$が現れるとすれば項の中である.項とは$\mathcal{L}$の項であるか$\{x|A\}$なるものであるので
			$\{$が現れたならば$\{$とは$\{x|A\}$の$\{$である.
			
			$\varphi$の任意の部分式に対して言明が満たされているとする.
			$\varphi$とは$\rightharpoondown \psi,\vee \psi \xi,...$の形であるから,
			$\varphi$に現れた$\{$とは$\psi$や$\xi$に現れるのである.ゆえに
			仮定より$x$と$A$が取れるわけである.
			
		\item[$\{$に対して]
			項の生成規則より$x$と$A$が得られる.
			$\{y|B\}$もまた$\{$から始まる項である場合,順番に見ていって
			$x$と$y$は一方が他方の始切片という関係になるから一致する.
			すると$A$と$B$は一方が他方の始切片という関係になり,(★)より$A$と$B$は一致する.
			
		\item[$|$について]
			項の生成規則より$x$と$A$が得られる.
			$\{y|B\}$もまた$|$から広がる項である場合,順番に見ていって
			$x$にも$y$にも$\{$という記号は現れないので$x$と$y$は一致する.
			$A$と$B$は一方が他方の始切片という関係になるので(★)より$A$と$B$は一致する.
			
		\item[$\}$について]
			項の生成規則より$x$と$A$が得られる.
			$\{y|B\}$もまた$\}$のその出現位置を終点とする変項である場合,
			$A$と$B$は$\mathcal{L}$の式なので$|$という記号は現れない.ゆえに
			$A$と$B$は一致する.すると$x$と$y$は右端で揃うが,
			$x$にも$y$にも$\{$という記号は現れないので$x$と$y$は一致する.
	\end{description}
	
\section{スコープ 式の解釈はただ一通りであるか}
	$\varphi$を式とする.$\in$が$\varphi$に現れるとき,
	項$s$と$t$が取れて,その$\in$の出現位置から
	\begin{align}
		\in st
	\end{align}
	なる原子式が$\varphi$に現れる.同様に$=$が$\varphi$に現れるとき,
	項$u$と$v$が取れて,その$\in$の出現位置から
	\begin{align}
		\in uv
	\end{align}
	なる原子式が$\varphi$に現れる.これは直感的に正しいことであるが,
	直感を排除してこれを認めるには構成的帰納法の原理が必要になる.
	
	では式に対する次の言明を考察する.
	
	\begin{screen}
		(★★) 記号$s$が$\varphi$に現れたとする.
		\begin{itemize}
			\item $s$が$\in$または$=$であるとき,項$\tau$と$\sigma$が得られて,$s \tau \sigma$は
				$s$のその出現位置から始まる$\varphi$の部分式となる.
				また$s$のその出現位置から始まる$\varphi$の部分式は$s \tau \sigma$のみである.
				
			\item $s$が$\rightharpoondown$であるとき,式$\psi$が得られて,
				$s \psi$は$s$のその出現位置から始まる$\varphi$の部分式となる.
				また$s$のその出現位置から始まる$\varphi$の部分式は$s \psi$のみである.
				
			\item $s$が$\vee,\wedge,\rightarrow$であるとき,式$\psi$と$\xi$が得られて,
				$s \psi \xi$は$s$のその出現位置から始まる$\varphi$の部分式となる.
				また$s$のその出現位置から始まる$\varphi$の部分式は$s \psi \xi$のみである.
				
			\item $\sigma$が$\forall, \exists$であるとき,変項$x$と式$\psi$が得られて,
				$s x \psi$は$s$のその出現位置から始まる$\varphi$の部分式となる.
				また$s$のその出現位置から始まる$\varphi$の部分式は$s x \psi$のみである.
		\end{itemize}
	\end{screen}
	
	$\bot$に対しては上の言明は当てはまる.
	
	$\in \tau \sigma$なる式に対しては,$\in$のスコープは$\in \tau \sigma$に他ならない.
	$= \tau \sigma$なる式についても,$=$のスコープは$= \tau \sigma$に他ならない.
	
	$\varphi$を任意に与えられた式とし,$\varphi$の真部分式に対しては
	(★★)が当てはまっているとする.
	
	\begin{description}
		\item[ケース1] 
	\end{description}
	式$\varphi$と$\psi$に対して上の言明が当てはまるとする.
	式$\rightharpoondown \varphi$に対して,
	$\sigma$が左端の$\rightharpoondown$であるとき
	$\sigma \varphi$は$\rightharpoondown \varphi$の部分式である.
	また$\sigma \psi$が$\sigma$のその出現位置から始まる$\rightharpoondown \varphi$の部分式
	であるとすると,
	$\psi$は$\varphi$の左端から始まる$\varphi$の部分式ということになるので
	帰納法の仮定より$\varphi$と$\psi$は一致する.
	$\sigma$が$\varphi$に現れる記号であれば,帰納法の仮定より
	$\sigma$から始まる$\varphi$の部分式が一意的に得られる.
	その部分式は$\rightharpoondown \varphi$の部分式でもあるし,
	$\rightharpoondown \varphi$の部分式としての一意性は帰納法の仮定より従う.
	
	式$\vee \varphi \psi$に対して,
	$\sigma$が左端の$\vee$であるとき,式$\xi$と$\eta$が得られて$\sigma \xi \eta$が
	$\vee \varphi \psi$の部分式となったとすると,
	$\xi$と$\varphi$は左端を同じくし,どちらか一方は他方の部分式である.
	$\xi$が$\varphi$の部分式であるならば,帰納法の仮定より$\xi$と$\varphi$は一致する.
	$\varphi$が$\xi$の部分式であるならば,$\xi$と$\psi$が重なるとなると
	$\psi$の左端の記号から始まる$\xi$の部分式と$\psi$は一致しなくてはならない.
	
\section{証明}
	閉式には,「真」であるか,「偽」であるか,のどちらかのラベルが付けられる.
	「真である」という言明は,「正しい」や「成り立つ」などとも言い換えられる.
	式が真であるか偽であるかは,次の手順に従って発掘的に判明していく.
	
	\begin{itemize}
		\item $\Sigma$の閉式は真である.
		\item $A$と$\rightarrow AB$が真であると判明しているならば,$B$は真である.
		\item $\rightarrow \wedge ABA$と$\rightarrow \wedge ABB$は真である.
		\item $A$と$B$が真であると判明しているならば$\wedge AB$と$\wedge BA$は真である.
		\item $\rightarrow A\vee AB$と$\rightarrow B \vee AB$は真である.
		\item $\rightarrow AC$と$\rightarrow BC$が真であると判明しているならば
			$\rightarrow \vee ABC$は真である.
		\item $\rightarrow\wedge A \rightharpoondown A \bot$は真である.
		\item $\rightarrow \rightarrow A \bot \rightharpoondown A$は真である.
		\item $\rightarrow \rightharpoondown\rightharpoondown AA$は真である.
	\end{itemize}
	
	真であると判明している式$\varphi$を起点にして,
	上の推論規則を駆使して閉式$\psi$が真であると判明すれば,
	$\varphi$から始めて$\psi$が真であることに辿り着くまでの手続きは$\psi$の証明と呼ばれ,
	$\psi$は定理と呼ばれる.
	
	証明には真であると判明している式が必要であり,その大元として選ばれた式が$\Sigma$の式である.
	$\Sigma$の式は証明なしに真であると決められているのであり,これらを公理と呼び定理と区別する.
	
	与えられた閉式$\varphi$が証明可能であるとは,
	\begin{itemize}
		\item 閉式$\psi$で,$\psi$と$\psi \rightarrow \varphi$が真であると判明している者が得られる.
		\item 真であると判明している閉式$\psi$と$\xi$が得られて,$\varphi$は$\psi \wedge \xi$である.
		\item 閉式$\psi$と$\xi$で,$\psi \vee \xi$と$\psi \rightarrow \varphi$と$\xi \rightarrow \varphi$が真であると判明しているものが得られる.
	\end{itemize}
	
	のいずれかの場合であり,
	\begin{align}
		\vdash \varphi
	\end{align}
	と書く.
	
	証明された式が真なる式である.では真なる式は
\end{document}
\documentclass[a4j,10.5pt,oneside,openany]{jsbook}
%
\usepackage{amsmath,amssymb}
\usepackage{amsthm}
\usepackage{makeidx}
\makeindex
\usepackage{newpxmath,newpxtext}
\usepackage{mathrsfs} %花文字
\usepackage{mathtools} %参照式のみ式番号表示
\usepackage{latexsym} %qed
\usepackage{ascmac}
\usepackage{tabularx}
\usepackage{bussproofs} %証明図
\usepackage{centernot} %\centernot\arrow
\usepackage[dvipdfmx]{graphicx}
\usepackage{tikz} %描画
\usepackage{color}
\usepackage{relsize}
\usepackage{comment}
\usepackage{url}
\usepackage{ulem} %訂正線
\usepackage[dvipdfm,colorlinks=true,linkcolor=blue,filecolor=blue,urlcolor=blue]{hyperref} %文書内リンク
\usepackage{pxjahyper} %%hyperref読み込みの直後に
\setcounter{tocdepth}{3} %table of contents subsection表示
\newtheoremstyle{mystyle}% % Name
	{20pt}%                      % Space above
	{20pt}%                      % Space below
	{\rm}%           % Body font
	{}%                      % Indent amount
	{\gt}%             % Theorem head font
	{.}%                      % Punctuation after theorem head
	{10pt}%                     % Space after theorem head, ' ', or \newline
	{}%                      % Theorem head spec (can be left empty, meaning `normal')
\theoremstyle{mystyle}

\allowdisplaybreaks[1]
\newcommand{\bhline}[1]{\noalign {\hrule height #1}} %表の罫線を太くする.
\newcommand{\bvline}[1]{\vrule width #1} %表の罫線を太くする.
\newcommand{\QED}{% %証明終了
	\relax\ifmmode
		\eqno{%
		\setlength{\fboxsep}{2pt}\setlength{\fboxrule}{0.3pt}
		\fcolorbox{black}{black}{\rule[2pt]{0pt}{1ex}}}
	\else
		\begingroup
		\setlength{\fboxsep}{2pt}\setlength{\fboxrule}{0.3pt}
		\hfill\fcolorbox{black}{black}{\rule[2pt]{0pt}{1ex}}
		\endgroup
	\fi}

\definecolor{DarkMidnightBlue}{rgb}{0.0, 0.2, 0.4}
\definecolor{PakistanGreen}{rgb}{0.0, 0.4, 0.0}
\definecolor{Mahogany}{rgb}{0.65,0.10,0.10}
\definecolor{darkgray}{rgb}{0.21, 0.21, 0.21}
\definecolor{CarolinaBlue}{rgb}{0.6, 0.73, 0.89}

\newtheorem{thm}{\color{DarkMidnightBlue}{定理}}[section]
\newtheorem{dfn}[thm]{\color{PakistanGreen}{定義}}
\newtheorem{axm}[thm]{\color{Mahogany}{公理}}
\newtheorem{schema}[thm]{{公理図式}}
\newtheorem{logicalaxm}[thm]{\color{Mahogany}{推論規則}}
\newtheorem{logicalthm}[thm]{\color{DarkMidnightBlue}{推論法則}}
\newtheorem{metadfn}[thm]{\color{PakistanGreen}{メタ定義}}
\newtheorem{metaaxm}[thm]{\color{Mahogany}{メタ公理}}
\newtheorem{metathm}[thm]{\color{DarkMidnightBlue}{メタ定理}}
\newtheorem{prp}[thm]{命題}
\newtheorem{cor}[thm]{系}
\newtheorem{lem}[thm]{補題}
\newtheorem*{prf}{証明}
\newtheorem*{metaprf}{メタ証明}
\newtheorem*{sketch}{略証}
\newtheorem{rem}[thm]{注意}
\newtheorem{e.g.}[thm]{例}
\newcommand{\defunc}{\mbox{1}\hspace{-0.25em}\mbox{l}} %定義関数
\newcommand*{\sgn}[1]{\operatorname{sgn}\left( #1 \right)} %signal関数
\newcommand{\monologue}[1]{
	{\color{CarolinaBlue}\hspace{-10.5pt}\mask{\hspace{21pt}\vbox{
		\hsize 445pt
		\normalcolor{\vskip 7pt \noindent #1 \vskip 7pt}
	}\hspace{21pt}}{E}}
}

\def\Ddot#1{$\ddot{\mathrm{#1}}$} %文中ddot

%論理
\newcommand{\lang}[1]{\mathcal{L}_{\scalebox{1.2}{$#1$}}} %言語
\newcommand{\Set}[2]{\left\{\, #1 \mid #2\, \right\}} %論理式の対象化
\newcommand{\defeq}{\overset{\mathrm{def}}{=\joinrel=}} %\scalebox{3}[1]{=}}} %定義記号=(=\joinrel=も使える)
\newcommand{\defarrow}{\overset{\mathrm{def}}{\longleftrightarrow}} %定義記号↔
\newcommand{\provable}[1]{\vdash_{{\scriptsize #1}}} %証明可能
\newcommand{\negation}{\rightharpoondown\hspace{-0.25em}} %否定
\newcommand{\rarrow}{\hspace{0.25em}\rightarrow\hspace{0.25em}} %右矢印
\newcommand{\lrarrow}{\hspace{0.25em}\leftrightarrow\hspace{0.25em}} %左右矢印

%集合
\newcommand{\EXTAX}{\mbox{{\bf EXT}}} %外延性公理
\newcommand{\EQAX}{\mbox{{\bf EQ}}} %相等性公理
\newcommand{\EQAXEP}{\mbox{{\bf EQ}}^{\scalebox{1.2}{$\varepsilon$}}} %ε項の相等性公理
\newcommand{\COMAX}{\mbox{\bf COM}} %内包性公理
\newcommand{\ELEAX}{\mbox{{\bf ELE}}} %要素の公理
\newcommand{\REPAX}{\mbox{{\bf REP}}} %置換公理
\newcommand{\POWAX}{\mbox{{\bf POW}}} %冪集合公理
\newcommand{\PAIAX}{\mbox{{\bf PAI}}} %対集合公理
\newcommand{\INFAX}{\mbox{{\bf INF}}} %無限公理
\newcommand{\REGAX}{\mbox{{\bf REG}}} %正則性公理
\newcommand{\CAX}{\mbox{{\bf C}}} %選択公理

\newcommand{\Univ}{\mathbf{V}} %宇宙
\newcommand{\set}[1]{\operatorname*{set} (#1)} %集合であることの論理式
\newcommand{\power}[1]{\operatorname*{P} (#1)} %冪集合
\newcommand{\rel}[1]{\operatorname*{rel} (#1)} %関係
\newcommand{\dom}[1]{\operatorname*{dom} (#1)} %類の定義域
\newcommand{\ran}[1]{\operatorname*{ran} (#1)} %類の値域
\newcommand{\sing}[1]{\operatorname*{sing} (#1)} %single-valuedの定義式
\newcommand{\fnc}[1]{\operatorname*{fnc} (#1)} %写像の定義式
\newcommand{\fon}{\operatorname*{:on}} %〇上の写像
\newcommand{\inj}{\overset{\mathrm{1:1}}{\longrightarrow}} %単射
\newcommand{\srj}{\overset{\mathrm{onto}}{\longrightarrow}} %全射
\newcommand{\bij}{\underset{\mathrm{onto}}{\overset{\mathrm{1:1}}{\longrightarrow}}} %全単射
\newcommand{\inv}[1]{{#1}^{-1}} %^{\operatorname{inv}}} %集合の反転
\newcommand{\rest}[2]{#1\hspace{-0.25em}\upharpoonright\hspace{-0.25em}{#2}} %制限写像
\newcommand{\tran}[1]{\operatorname*{tran} \left(#1\right)} %推移的類の定義式
\newcommand{\ord}[1]{\operatorname*{ord} \left(#1\right)} %順序数の定義式
\newcommand{\ON}{\mathrm{ON}} %順序数全体
\newcommand{\limo}[1]{\mathrm{lim.o}\left(#1\right)} %極限数の式
%\newcommand{\Natural}{{\boldsymbol \omega}} %自然数全体
\newcommand{\Natural}{\mathbf{N}} %自然数全体
%
%
\setlength{\textwidth}{\fullwidth}
\setlength{\textheight}{40\baselineskip}
\addtolength{\textheight}{\topskip}
%\setlength{\voffset}{-0.55in}
%
%
\title{$\varepsilon$計算とクラスの導入による具体的で直観的な集合論の構築}
\author{関根深澤研修士二年百合川尚学}
\date{\today}

\begin{document}
\mathtoolsset{showonlyrefs = true}
\maketitle
\tableofcontents
\frontmatter
\mainmatter

\section{徒然なるままに支離滅裂}
わからないわからないわからない

基礎論における証明は大抵が直感に頼っているように見えますが,ではその直感が正しいとは誰が保証するのでしょうか.
手元にあるどの本でも保証されていません.もしかしたら神様という超然的な存在を暗黙の裡に認めていて,
直感とは神様が用意した論理であるとして無断で使っているだけなのかもしれませんが,
残念ながら読者はテレパシーを使えないので,筆者の暗黙の了解を推察するなんて困難です.

しかしながら,暗黙の了解を排除しようとすると,その分だけ日本語による明示的な約束が必要になります.
すると新たな問題が生じます.それは日本語で書かれた言明をどこまで信用するか,という問題です.
基礎論の難しさは,その表面上のややこしさよりも日本語に対する認識を揃えることにあるのでしょうか.

論理構造を集合論の結果を用いて解明しようというのならまだしも(こちらは数理論理学と呼ばれる分野で,本来は数学基礎論とは別物だそうです),
集合論を構築することが目的である場合,その土台となる基礎論を集合論の上に展開すると理論が循環することになるでしょう.
基礎論が基礎にしている集合論は「メタ理論」と呼ばれるらしいですが,
その「メタ理論」がどう構成されたのかという点には誰も全く言及していないのですから,
「メタ理論」という言葉は単なる逃げ口上にしか聞こえず,理論の循環を解消できません.
私の考えでは,メタ理論の代わりに絶対的な原理が与えられたとして数学を構築すれば良いのです.
まあ言い方を変えて印象を良く?しようというだけの下らない事情であって,
もったいぶって思想的な立場を主張しても集合論には関係のないことなのですが.

前提:我々は数の概念を持っている.個数の概念を持っている.物の数を数えることが出来る.
数の概念とは?個数の概念とは?
ここで言う数は数学的に構成する数ではなくて,神が用意した概念としての数.
そこまで踏み込むときりがない.

排中律と無矛盾性の違い:
排中律から$\negation (A \wedge \negation A)$が導かれるが,
$A \wedge \negation A$が導かれることを否定しているわけではない.

目的:いかに自然で人工的な世界を作るか.

数学について注意深く考え込んでいると,うっかりとんでもない落とし穴にはまってしまうかもしれません.
そのとき,きっと次の事柄に悩まされます.
前提がわからない.明らかなものと明らかでないものとの線引きがわからない.
日本語をどこまで信用してよいのかもわからない.
突き詰めると何も見えなくなる.数学の立脚地は永遠に届かない...
このノートがそういった受難を乗り越える役に立てますように.

\begin{description}
	\item[前提その一] はじめに素朴な数字の概念は持っている.それによってモノを数えることもできる.
		モノの数え方は慣習どおり.
		
	\item[前提その二] 当たり前のことが当たり前であるためには,言葉でそれを保証しなければならない.
	
	\item[] $ZF$集合論では存在という言葉が具体的な意味を持っていない.
		存在したらこうなるであろうという推論規則によってしか存在という概念を表現し得ない.
		$\varepsilon$項は集合である.それも,存在したら``取れる''集合である.
		$\varepsilon$項によって集合を具体的に扱える.
		さらに内包項によって閉じた世界を作ることが出来る.
		$ZF$集合論では集合の宇宙は閉じていないので,存在したらそれに名前を付けて言語を保存拡大するという手法を取る.
		内包項を導入すれば,言語を拡大する必要はなくなる.
		公理とは,既に作られた世界においてどれが集合でどれが集合でないかを選り分ける用に使われる.
\end{description}

\begin{itemize}
	\item $\lang{\varepsilon}$の公理は$\lang{\in}$の公理と同じで良い.
	\item $\mathcal{L}$で示せることは$\lang{\in}$で示せる.実際,
		$\varphi_{1},\cdots,\varphi_{n}$を$\mathcal{L}$の文でできた証明とすれば,
		それを$\lang{\in}$の式に書き換えた式(変項の名前替えは要るかもしれない)の列
		$\psi_{1},\cdots,\psi_{n}$は$\lang{\in}$の文でできた証明となる.
\end{itemize}
\section{導入}
	Hilbertの$\varepsilon$計算とは項を形成するオペレーター$\varepsilon$を用いた
	述語計算の拡張である.$\varepsilon$は式$\varphi(x)$から
	項$\varepsilon x \varphi(x)$を作るものであり,
	この項は次の主要論理式によって制御される:
	\begin{align}
		\varphi(t) \rarrow \varphi(\varepsilon x \varphi(x)).
	\end{align}
	Hilbertが$\varepsilon$を導入したのは述語計算を
	命題計算に埋め込むためであり,その際には$\exists$や$\forall$の付いた式を
	\begin{align}
		\varphi(\varepsilon x \varphi(x)) &\defarrow \exists x \varphi(x), \\
		\varphi(\varepsilon x \negation \varphi(x)) &\defarrow \forall x \varphi(x)
	\end{align}
	と変換する.
	
	この変換は本稿において最も重要な公理の基となるが,
	ただし本稿において$\varepsilon$を導入したのは述語計算を埋め込むためではなく,
	集合を「具体化」するためである.本稿で実践しているのはHilbertの$\varepsilon$計算ではなく
	一種のHenkin拡大であり,先述の主要論理式は本稿では全く不要であって,代わりに
	\begin{align}
		\exists x \varphi(x) \rarrow \varphi(\varepsilon x \varphi(x))
	\end{align}
	が主要な公理となる.「具体化」に対する問題意識の源は,
	通常の公理的集合論においては集合が無定義であるという不可解さである.
	純粋に一階述語論理の言語から構築される集合論を
	``生の''集合論と呼ぶことにすれば,``生の''集合論の言語では
	集合というオブジェクトが用意されていないため「存在」は「実在」を意味しない.たとえば
	\begin{align}
		\exists x\, \forall y\, (\, y \notin x\, )
	\end{align}
	は「空集合は存在する」という定理を表しているが,存在するはずの空集合を実際に取ってくることは
	出来ないのである.それなのに集合論において$\emptyset$が恰も実在するオブジェクトとして扱われているのは,
	一つには$\forall y\, (\, y \notin \emptyset\, )$を
	$\emptyset$の定義式として$\emptyset$を言語に追加している(定義による拡張)のであろうが,
	$\varepsilon$を使えば単に
	\begin{align}
		\varepsilon x\, \forall y\, (\, y \notin x\, )
	\end{align}
	と書くだけで「存在」を「実在」に格上げする効果がある.
	適切な公理と集合の定義によって,
	$\varepsilon$項及び$\varepsilon$項に等しいオブジェクトが全て集合であり,
	かつ集合はこれらに限られるといった体系を構築できるので,
	この意味で$\varepsilon$項によって集合の「具体化」が実現する.
	その他にも$\varepsilon$項を導入することで得られるメリットとして,
	直感的な証明が組み立てやすくなったり,証明で用いる推論規則が三段論法のみで済むといった点がある.
	
	ブルバキ\cite{key5}や島内\cite{key6}でも$\varepsilon$計算を使った集合論を展開している
	(ブルバキ\cite{key5}では$\varepsilon$ではなく$\tau$が使われている).
	ところで,本稿では$\varepsilon$項だけではなく,
	「$\varphi(x)$を満たす集合$x$の全体」の役割を期して
	\begin{align}
		\Set{x}{\varphi(x)}
	\end{align}
	というオブジェクトも取り入れる.ブルバキ\cite{key5}や島内\cite{key6}では
	\begin{align}
		\Set{x}{\varphi(x)} \defeq \varepsilon x\, \forall u\, 
		(\, \varphi(u) \lrarrow u \in x\, )
	\end{align}
	と定めるが,これは欠点がある.
	\begin{align}
		\exists x\, \forall u\, (\, \varphi(u) \lrarrow u \in x\, )
	\end{align}
	が成立しない場合は「$\varphi(x)$を満たす集合$x$の全体」という意味を持たないためである.
	この欠点を解消するには,竹内\cite{key4}に倣って$\varphi$から直接$\Set{x}{\varphi(x)}$
	の形のオブジェクトを作ればよい.
	
	$\Set{x}{\varphi(x)}$なる項は「モノの集まり」という観点からはまさしく「集合」
	なのだが,たとえばRussellのパラドックスが示す通り
	\begin{align}
		\Set{x}{x \notin x}
	\end{align}
	は数学の世界での集合であってはならず,パラドックスを回避するためには
	「モノの集まり」を数学の世界の集合であるものとそうでないものとに分類しなくてはならない.
	数学の世界では単なる「モノの集まり」は類(class)と呼ばれ,
	集合でない類は真類(proper class)と呼ばれる.
	$\varepsilon$項を採用している本稿では
	\begin{align}
		\varepsilon x \varphi(x),\quad \Set{x}{\varphi(x)}
	\end{align}
	の形の項を類と定義し,類$a$が集合であることの判断基準は,竹内\cite{key4}に倣って
	\begin{align}
		\exists x\, (\, a = x\, )
	\end{align}
	が成り立つことであるとする.また$\Set{x}{\varphi(x)}$に対して
	「$\varphi(x)$を満たす集合$x$の全体」の意味を実質的に与えるために,
	\begin{align}
		\forall u\, (\, u \in \Set{x}{\varphi(x)} \lrarrow \varphi(u)\, )
	\end{align}
	と
	\begin{align}
		a \in \Set{x}{\varphi(x)} \rarrow \exists s\, (\, a = s\, )
	\end{align}
	を集合論の公理とする.前者の公理によって$\Set{x}{\varphi(x)}$は
	「$\varphi(x)$を満たす$x$の全体」となり,後者の公理によって
	「$\Set{x}{\varphi(x)}$の要素は全て集合である」ということになる.
	
	本稿で行う集合論の拡張は妥当なものである.
	妥当であるとは本稿の集合論が現代数学で受容可能であるということであり,
	それは``生の''集合論のどの命題に対しても「``生の''集合論で証明可能である」ことと
	「本稿の集合論で証明可能である」ことが同じであるという意味である.
	このような妥当な拡張のことを{\bf 保存拡大}\index{ほぞんかくだい@保存拡大}
	{\bf (conservative extension)}と呼ぶ.
	
\section{章立て}
	第2章では集合論の言語というものを導入し,また構文論的な性質についていくつか述べる.
	その際3つの言語が登場するが,``生の''集合論の言語は$\lang{\in}$と書き,
	それに$\varepsilon$項を追加した言語を$\lang{\varepsilon}$と書き,
	最後に$\Set{x}{\varphi(x)}$なる形の項を追加した言語を$\mathcal{L}$と書く.
	第3章では証明とは何かを規定する.本稿の証明体系は主に古典論理に準じているが,
	$\varepsilon$項を利用するために若干変更を施す.
	第4章では言語$\mathcal{L}$と第3章の証明体系で集合論が展開できることを実演する.
	第5章では,本稿の集合論が``生の''集合論の保存拡大になっていることを示す.
	
	メタ定理とは式や項の形状的な性質に対する主張であって,
	メタ証明はメタ定理の妥当性を日本語によって検証するものである.
	またメタ証明に必要な直感的真理をメタ公理として提示する.
	
	\begin{comment}
	Hilbertの$\varepsilon$計算は,項を形成するオペレーター$\varepsilon$と
	そのような項を含む initial formula による初等的,或いは述語計算の拡張である.
	$\varepsilon$計算の基本的な結果は$\varepsilon$定理と呼ばれ,それらは
	$\varepsilon$除去法によって証明される.$\varepsilon$除去法とは
	$\varepsilon$計算での証明を初等的または述語計算の証明に変換する手法であり,
	具体的には initial formula を除去するのである.主要な結果の一つで,
	BernaysとHilbertにより示されたHerbrandの定理は,拡張$\varepsilon$定理の系として出てくる.
	
	Hilbertの$\varepsilon$計算は$\varepsilon$-オペレーターを用いた述語計算の拡張であり,
	$\varepsilon$は式$A(x)$から項$\varepsilon_{x}A(x)$を作るものである.
	このオペレーターは次の initial formula によって制御される.一つは
	\begin{align}
		A(t) \rarrow A(\varepsilon_{x}A(x))
	\end{align}
	といった形の主要論理式である.ここで$t$は任意の項である.もう一つは
	$\varepsilon$-等号論理式
	\begin{align}
		\vec{u} = \vec{v} \rarrow 
		\varepsilon_{x}B(x,\vec{u}) = \varepsilon_{x}B(x,\vec{v})
	\end{align}
	である.ここで$\vec{u}$と$\vec{v}$は項の列$u_{0},u_{1},\cdots,u_{n-1}$と
	$v_{0},v_{1},\cdots,v_{n-1}$であり,$\vec{u} = \vec{v}$とは
	$u_{0} = v_{0},\ u_{1} = v_{1},\ \cdots,$及び$u_{n-1} = v_{n-1}$のことである.
	また$\varepsilon_{x}B(x,\vec{a})$の真部分項は$\vec{a}$のみである.
	純粋な$\varepsilon$計算は$\varepsilon$オペレーターと主要論理式による初等計算の拡張である.
	$\varepsilon$オペレーターによって存在と全称をエンコード可能である,
	$\exists x A(x) \defeq A(\varepsilon_{x}A(x))$や
	$\forall x A(x) \defeq A(\varepsilon_{x}\negation A(x))$と定義ですれば
	$\varepsilon$計算に埋め込める.
	
	$\varepsilon$計算はHilbertプログラムの文脈で開発された.Gentzen以前の黎明期の証明論は
	$\varepsilon$計算に集中され,$\varepsilon$-除去法,$\varepsilon$-代入法,それから
	それらの業績はBernaysやAckermann,Von Neumannによってもたらされた.
	$\varepsilon$計算を使ったHerbrandの定理の正しい証明は[Bus94]にある.
	通常,定理はオリジナルのものより若干一般性を欠いて以下のように述べられる.
	存在式の冠頭標準形$\exists \vec{x} A(\vec{x})$に対して,
	初等計算における項$\vec{t}_{0},\vec{t}_{1},\cdots,\vec{t}_{k-1}$が取れて
	初等計算で$A(\vec{t}_{0}) \vee A(\vec{t}_{1}) \vee \cdots \vee A(\vec{t}_{k-1})$
	が証明される.しかし$\varepsilon$計算は独立で永続的に惹かれる,また計算機科学や
	証明論的観点でとりわけ価値がある.
	
	$\varepsilon$定理やHerbrandの定理を証明する流れの中で,$\varepsilon$-除去法は,
	$\varepsilon$計算での証明を上で述べた initial formula を用いない証明に証明論的に変形する.
	$\varepsilon$計算において$A(\vec{t})$への証明があったとすると,
	ここで$\vec{t}$とは$\varepsilon$項が現れうる項を含んだ有限列である,
	$\varepsilon$-除去法によって
	$A(\vec{s}_{0}) \vee A(\vec{s}_{1}) \vee \cdots \vee A(\vec{s}_{k-1})$
	への初等的証明が得られる.ここで$\vec{s}_{0},\vec{s}_{1},\cdots,\vec{s}_{k-1}$
	とは$\varepsilon$が無い項である.
	この選言は式$A(\vec{t})$のHerbrand選言と呼ばれるものであり,
	この論文の目的はHerbrand複雑度の解析であり,それは元の式の最短のHerbrand選言の長さ$k$のことである.
	
	Hilbertの$\varepsilon$計算の大元は形式主義にあり,我々は古典的一階論理に焦点を絞る.
	\end{comment}
	

\chapter{言語}
		%この世のはじめに言葉ありきといわれるが,この原則は数学の世界でも同じである.
	本稿の世界を展開するために使用する言語には二つ種類がある.
	一つは自然言語の日本語であり,もう一つは新しくこれから作る言語である.
	その人工的な言語は記号列が数学の式となるための文法を指定し,
	そこで組み立てられた式のみが考察対象となる.
	日本語は式を解釈したり人工言語を補助するために使われる.
	
	さっそく人工的な言語$\lang{\in}$を構築するが,
	これは本論においてはスタンダードな言語ではなく,
	後で$\lang{\in}$をより複雑な言語に拡張するという意味で原始的である.
	以下は$\lang{\in}$の語彙である:
	\begin{description}
		\item[矛盾記号] $\bot$
		\item[論理記号] $\negation,\ \vee,\ \wedge,\ \rarrow$
		\item[量化子] $\forall,\ \exists$
		\item[述語記号] $=,\ \in$
		\item[変項] 後述(第\ref{sec:variables}節).
			
			%\begin{table}
			%	\begin{tabular}{ccccc ccccc ccccc ccccc ccccc c}
			%		$a$ & $b$ & $c$ & $d$ & $e$ & $f$ & $g$ & $h$ & $i$ & $j$ & $k$ & $l$ & $m$ & $n$ & $o$ & $p$ & $q$ & $r$ & $s$ & $t$ & $u$ & $v$ & $w$ & $x$ & $y$ & $z$ \\
			%		$A$ & $B$ & $C$ & $D$ & $E$ & $F$ & $G$ & $H$ & $U$ & $J$ & $K$ & $L$ & $M$ & $N$ & $O$ & $P$ & $Q$ & $R$ & $S$ & $T$ & $U$ & $V$ & $W$ & $X$ & $Y$ & $Z$ \\
			%		$\alpha$ & $\beta$ & $\gamma$ & $\delta$ & $\epsilon$ & $\zeta$ & $\eta$ & $\theta$ & $\iota$ & $\kappa$ & $\lambda$ & $\mu$ & $\nu$ & $\xi$ & & $\pi$ & $\rho$ & $\sigma$ & $\tau$ & $\upsilon$ & $\phi$ & $\chi$ & $\psi$ & $\omega$ & & \\
			%		& & $\Gamma$ & $\Delta$ & & & & $\Theta$ & & & $\Lambda$ & & & $\Xi$ & & $\Pi$ & & $\Sigma$ & & $\Upsilon$ & $\Phi$ & & $\Psi$ & $\Omega$ & &
			%	\end{tabular}
			%\end{table}
			
	\end{description}
	
	日本語と同様に,決められた規則に従って並ぶ記号列のみを$\lang{\in}$の単語や文章として扱う.
	$\lang{\in}$において,名詞にあたるものは{\bf 項}\index{こう@項}{\bf (term)}と呼ばれる.
	文字は最もよく使われる項である.述語とは項同士を結ぶものであり,
	最小の文章を形成する.例えば
	\begin{align}
		\in st
	\end{align}
	は$\lang{\in}$の文章となり,「$s$は$t$の要素である」と読む.
	$\lang{\in}$の文章を$\lang{\in}$の{\bf 式}\index{しき@式}
	{\bf (formula)}或いは$\lang{\in}$の{\bf 論理式}\index{ろんりしき@論理式}と呼ぶ.
	論理記号は主に式同士を繋ぐ役割を持つ.
	
	論理学的な言語の語彙とは論理記号と変項以外の記号をすべて集めたものである.
	本稿で用意した記号で言うと,論理記号とは
	\begin{align}
		\bot,\ \negation,\ \vee,\ \wedge,\ \rarrow,\ \forall,\ \exists,\ =
	\end{align}
	であり,変項記号とは文字であって,$\lang{\in}$の語彙は
	\begin{align}
		\in
	\end{align}
	しかない.{\bf ZF}集合論の言語が$\{\in\}$であるとはこういう訳である.
	しかし,実質的な違いははないが,本稿では論理記号も変項も全て含めて$\lang{\in}$の語彙とする.
	これは,後で量化子に$\varepsilon$を追加するので論理記号を固定したくないためである.
	また等号$=$は論理記号ではなく$\in$と同列の述語記号として扱う.
	
\section{変項}
\label{sec:variables}
	{\bf 変項}\index{へんこう@変項}{\bf (variable)}と呼ばれる最も典型的なものは文字であり,
	本稿では以下の文字を変項として用いる:
	\begin{align}
		&a,b,c,d,e,f,g,h,i,j,k,l,m,n,o,p,q,r,s,t,u,v,w,x,y,z, \\
		&A,B,C,D,E,F,G,H,U,J,K,L,M,N,O,P,Q,R,S,T,U,V,W,X,Y,Z, \\
		&\alpha,\beta,\gamma,\delta,\epsilon,\zeta,\eta,\theta,\iota,
			\kappa,\lambda,\mu,\nu,\xi,\pi,\rho,\sigma,\tau,\upsilon,
			\phi,\chi,\psi,\omega, \\
		&\Gamma,\Delta,\Theta,\Lambda,\Xi,\Pi,\Sigma,\Upsilon,\Phi,\Psi,\Omega, \\
		&\vartheta,\ \varpi,\ \varrho,\ \varsigma,\ \varphi
	\end{align}
	
	だが文字だけを変項とするのは不十分であり,
	例えば$200$個の相異なる変項が必要であるといった場合には上の文字だけでは不足してしまう.
	そこで,文字$x$に対して
	\begin{align}
		\natural x
	\end{align}
	もまた変項であると約束する.
	さらに,$\tau$を変項とするときに
	\footnote{
		「$\tau$を変項とするときに」と書いたが,これは一時的に
		$\tau$を或る変項に代用しているだけであって,
		$\tau$が指している変項の本来の字面は$x$であるかもしれない.
		この場合の$\tau$を{\bf 超記号}\index{ちょうきごう@超記号}{\bf (meta symbol)}
		と呼ぶ.「$A$を式とする」など式にも超記号が宣言される.
	}
	\begin{align}
		\natural \tau
	\end{align}
	も変項であると約束する.この約束に従えば,文字$x$だけを用いたとしても
	\begin{align}
		x,\quad \natural x, \quad \natural \natural x, \quad \natural \natural \natural x
	\end{align}
	はいずれも変項ということになる.極端なことを言えば,$\natural$と$x$だけで
	無数の相異なる変項を作り出せるのである.
	
	大切なのは$\natural$を用いれば理屈の上では変項に不足しないということであって,
	具体的な数式を扱うときに$\natural$が出てくるかと言えば否である.
	$\natural$が必要になるほどに長い式を読解するのは困難であるから,
	通常は何らかの略記法を導入して複雑なところを覆い隠してしまう.
	
	変項は形式的には次のよう定義される:
	
	\begin{screen}
		\begin{metadfn}[変項]
			文字は変項である.また,$\tau$を変項とするとき$\natural \tau$は変項である.
			以上のみが変項である.
		\end{metadfn}
	\end{screen}
	
	上の定義では,はじめに発端を決めて,次に新しい項を作り出す手段を指定している.こういった定義の仕方を
	{\bf 帰納的定義}\index{きのうてきていぎ@帰納的定義}{\bf (inductive definition)}と呼ぶ.
	ただしそれだけでは項の範囲が定まらないので,最後に「以上のみが項である」と付け加えている.
	「以上のみが変項である」という約束によって,例えば「$\tau$が項である」という言明が与えられたとき,
	この言明は
	\begin{itemize}
		\item $\tau$は或る文字に代用されている
		\item 項$\sigma$が取れて\footnotemark,$\tau$は$\natural \sigma$に代用されている
	\end{itemize}
	のどちらか一方にしか解釈され得ない.
	
	%のは,言うまでもない,であろうか.直感的にはそうであっても
	%直感を万人が共有している保証はないから,やはりここは明示的に,「$\tau$が項である」という言明の解釈は
	%\begin{itemize}
	%	\item $\tau$は或る文字に代用されている
	%	\item 項$\sigma$が取れて(超記号),$\tau$は$\natural \sigma$に代用されている
	%\end{itemize}
	%に限られると決めてしまおう.主張はストレートな方が後々使いやすい.
	
	\footnotetext{
		「変項$\sigma$が取れて」と書いたが,この$\sigma$は唐突に出てきたので,
		それが表す文字そのものでしかないのか,或いは超記号であるのか,一見判然しない.
		本来は「変項が取れて,これを$\sigma$で表すと」などと書くのが
		良いのかもしれないが,はじめの書き方でも文脈上は超記号として解釈するのが自然であるし,
		何より言い方がまどろこくない.このように見た目の簡潔さのために超記号の宣言を省略する場合もある.
	}

\section{項と式}
	$\lang{\in}$の{\bf 項}\index{こう@項}{\bf (term)}と
	{\bf 式}\index{しき@式}{\bf (formula)}も変項と同様に帰納的に定義される:
	
	\begin{screen}
		\begin{metadfn}[$\lang{\in}$の項]
			変項は$\lang{\in}$の項であり,またこれらのみが$\lang{\in}$の項である.
		\end{metadfn}
	\end{screen}
	
	\begin{screen}
		\begin{metadfn}[$\lang{\in}$の式]\mbox{}
			\begin{itemize}
				\item $\bot$は式である.
				\item $\sigma$と$\tau$を項とするとき,$\in st$と$=st$は式である.
					これらを{\bf 原子式}\index{げんししき@原子式}{\bf (atomic formula)}と呼ぶ.
				\item $\varphi$を式とするとき,$\negation \varphi$は式である.
				\item $\varphi$と$\psi$を式とするとき,$\vee \varphi \psi,\ 
					\wedge \varphi \psi,\ \rarrow \varphi \psi$はいずれも式である.
			
				\item $x$を項とし,$\varphi$を式とするとき,$\forall x \varphi$と$\exists x \varphi$は式である.
				
				\item 以上のみが式である.
			\end{itemize}
		\end{metadfn}
	\end{screen}
	
	変項と同様に,「$\varphi$が式である」という言明の解釈は
	\begin{itemize}
		\item $\varphi$は$\bot$である
		\item 項$s$と項$t$が得られて,$\varphi$は$\in s t$である
		\item 項$s$と項$t$が得られて,$\varphi$は$= s t$である
		\item 式$\psi$が得られて,$\varphi$は$\negation \psi$である
		\item 式$\psi$と式$\xi$が得られて,$\varphi$は$\vee \psi \xi$である
		\item 式$\psi$と式$\xi$が得られて,$\varphi$は$\wedge \psi \xi$である
		\item 式$\psi$と式$\xi$が得られて,$\varphi$は$\rarrow \psi \xi$である
		\item 項$x$と式$\psi$が得られて,$\varphi$は$\forall x \psi$である
		\item 項$x$と式$\psi$が得られて,$\varphi$は$\exists x \psi$である
	\end{itemize}
	のいずれか一つに限られる.
	
	以下では,$\theta$を$\lang{\in}$の項或いは式とするとき,
	$\theta$から切り取った一続きの記号列$e$のことを
	「{\bf $\theta$に現れる$e$}\index{あらわれる@現れる}」,
	「{\bf $\theta$の上の$e$}\index{うえの@上の}」,
	「{\bf $\theta$の中の$e$}\index{なかの@中の}」などと表現する.
	この慣行は後で登場する拡張言語においても踏襲する.
	
	\begin{screen}
		\begin{metadfn}[$\lang{\in}$の項の部分項]\label{metadfn:L_in_subterm_of_term}
			$\tau$を$\lang{\in}$の項とするとき,
			\begin{itemize}
				\item $\tau$に現れる$\lang{\in}$の項を
					$\tau$の{\bf 部分項}\index{ぶぶんこう@部分項}{\bf (subterm)}と呼ぶ.
				\item $\tau$自身を除く$\tau$の部分項を$\tau$の
					{\bf 真部分項}\index{しんぶぶんこう@真部分項}{\bf (proper subterm)}
					と呼ぶ.
				\item $\tau$が$\natural \sigma$なる項であるとき,$\sigma$を
					$\tau$の{\bf 直部分項}\index{ちょくぶぶんこう@直部分項}
					{\bf (immediate subterm)}と呼ぶ.
			\end{itemize}
		\end{metadfn}
	\end{screen}
	
	たとえば文字$x$の部分項は$x$のみである.また
	$\tau$を変項とするとき,$\tau$は$\natural \tau$の部分項であり,真部分項でもあり,
	直部分項でもある.他方で$\tau$は$\natural \natural \tau$の部分項であり,真部分項でもあるが,
	直部分項ではない.
	
	\begin{screen}
		\begin{metadfn}[$\lang{\in}$の式の項]\label{metadfn:L_in_term_of_formula}
			$\varphi$を$\lang{\in}$の式とするとき,$\varphi$に現れる
			$\lang{\in}$の項のうち,$\varphi$に現れる他の$\lang{\in}$の項の
			いずれにも格納されていないものを「$\varphi$の項」と呼ぶ.
		\end{metadfn}
	\end{screen}
	
	たとえば$\varphi$が$\in \tau \sigma$なる式で,$\tau$が$\natural x$なる項であるとき,
	$x$は$\varphi$の上に現れる$\lang{\in}$の項であるが$\varphi$の項ではない.
	
	\begin{screen}
		\begin{metadfn}[$\lang{\in}$の部分式]\label{metadfn:L_in_subformula}
			$\varphi$を$\lang{\in}$の式とするとき,
			$\varphi$に現れる$\lang{\in}$の式を
			$\varphi$の{\bf 部分式}\index{ぶぶんしき@部分式}{\bf (subformula)}と呼ぶ.
			$\varphi$自身を除く$\varphi$の部分式を特に$\varphi$の
			{\bf 真部分式}\index{しんぶぶんしき@真部分式}{\bf (proper subformula)}と呼ぶ.
		\end{metadfn}
	\end{screen}
	
	\begin{screen}
		\begin{metadfn}[直部分式]
		\label{metadfn:L_in_immediate_subformula}
			$\varphi$を$\lang{\in}$の式とするとき,$\varphi$の{\bf 直部分式}
			\index{ちょくぶぶんしき@直部分式}{\bf (immediate subformula)}を
			\begin{itemize}
				\item $\varphi$が$\negation \psi$なる式ならば$\psi$のこと,
				\item $\varphi$が$\vee \psi \chi,\ \wedge \psi \chi,\ \rarrow \psi \chi$
					なる式ならば$\psi$と$\chi$のこと,
				\item $\varphi$が$\exists x \psi,\ \forall x \psi$なる式ならば$\psi$のこと,
			\end{itemize}
			とする.
		\end{metadfn}
	\end{screen}
	
\section{構造的帰納法}
	まず
	\begin{align}
		\forall x \in x y
	\end{align}
	なる式を考える.中置記法(後述)で
	\begin{align}
		\forall x\, (\, x \in y\, )
	\end{align}
	と書けば若干見やすくなる.冠頭詞$\forall$は直後の$x$に係って「任意の$x$に対し...」の意味を持ち,
	この式は「任意の$x$に対して$x$は$y$の要素である」と読むのであるが,
	このとき$x$は$\forall x \in x y$で{\bf 束縛されている}{\bf (bound)}
	或いは{\bf 量化されている}{\bf (quantified)}と言う.
	$\forall$を$\exists$に替えても同様に「$x$は$\exists x \in x y$で束縛されている」と言う.
	つまり,{\bf 量化子の直後の項(量化子が係っている項)は,その量化子から始まる式の中で束縛されている}
	と解釈することになっている.
	
	では
	\begin{align}
		\rarrow \forall x \in x y \in x z
	\end{align}
	という式はどうであるか.$\forall x$の後ろには$x$が二か所に現れているが,
	どちらの$x$も$\forall$によって束縛されているのか?
	結論を言えば$\in x y$の$x$は束縛されていて,$\in x z$の$x$は束縛されていない.
	というのも式の構成法を思い返せば,$\forall x \varphi$が式であると言ったら$\varphi$は式であるはずで,
	今の例で$\forall x$に後続する式は
	\begin{align}
		\in x y
	\end{align}
	しかないのだから,$\forall$から始まる式は
	\begin{align}
		\forall x \in x y
	\end{align}
	しかないのである.$\forall$が係る$x$が束縛される範囲は
	``$\forall$から始まる式''であるから,$\in x z$の$x$は
	$\forall$の``束縛''から漏れた``自由な''$x$ということになる.
	
	上の例でみたように,量化はその範囲が重要になる.
	量化子$\forall$が式$\varphi$に現れたとき,
	その$\forall$から始まる$\varphi$の部分式を
	$\forall$の{\bf スコープ}と呼ぶが,
	いつでもスコープが取れることは明白であるとして,
	$\forall$のスコープは唯一つでないと都合が悪い.
	もしも異なるスコープが存在したら,同じ式なのに全く違う解釈に分かれてしまうからである.
	実際そのような心配は無用であると後で保証するわけだが,
	その前に{\bf 始切片}という概念を準備しておく必要がある.
	
\subsection{始切片}
	$\varphi$を$\lang{\in}$の式とするとき,$\varphi$の左端から切り取る一続きの記号列を
	$\varphi$の{\bf 始切片}\index{しせっぺん@始切片}{\bf (initial segment)}と呼ぶ.
	例えば$\varphi$が
	\begin{align}
		\rarrow \forall x \wedge \rarrow \in xy \in xz \rarrow \in xz \in xy = yz
	\end{align}
	である場合,
	\begin{align}
		\textcolor{red}{\rarrow \forall x \wedge \rarrow \in xy \in xz \rarrow \in xz \in x}y = yz
	\end{align}
	や
	\begin{align}
		\textcolor{red}{\rarrow \forall x \wedge \rarrow \in xy} \in xz \rarrow \in xz \in xy = yz
	\end{align}
	など赤字で分けられた部分は$\varphi$の始切片である.また$\varphi$自身も$\varphi$の始切片である.
	項についても同様に,項の左端から切り取るひとつづきの部分列をその項の始切片と呼ぶ.
	
	\begin{screen}
		\begin{metadfn}[$\lang{\in}$の始切片]
		\label{metadfn:L_in_initial_segment}
			$\theta$を$\lang{\in}$の項或いは式とするとき,
			$\theta$の左端から切り取る一続きの記号列を$\theta$の
			{\bf 始切片}\index{しせっぺん@始切片}{\bf (initial segment)}と呼ぶ.
		\end{metadfn}
	\end{screen}
	
	本節の主題は次である.
	\begin{screen}
		\begin{metathm}[始切片の一意性]\label{metathm:initial_segment_L_in}
			$\tau$を$\lang{\in}$の項とするとき,$\tau$の始切片で$\lang{\in}$の項であるものは$\tau$自身に限られる.
			また$\varphi$を$\lang{\in}$の式とするとき,$\varphi$の始切片で$\lang{\in}$の式であるものは$\varphi$自身に限られる.
		\end{metathm}
	\end{screen}
	
	\footnote[0]{
		{\bf メタ定理}\index{めたていり@メタ定理}とは式や項或いは証明の形状的な性質
		に対する主張であって,{\bf メタ証明}\index{めたしょうめい@メタ証明}はメタ定理の妥当性を
		日本語によって検証するものである.またメタ証明に必要な直感的真理を{\bf メタ公理}
		\index{めたこうり@メタ公理}として明示する.メタ公理は通常は暗黙の裡に認められている.
		
		本稿では第\ref{chap:set_theory}章から集合論の内側に入る構成になっている.
		だからそれまでの内容は殆ど集合論の外側の話である.
		{\bf メタ定義}\index{めたていぎ@メタ定義}とは既に何の断りもなく出してしまったが,
		集合論の外側で出てくる概念に特別の名前を付ける際にメタ定義として提示する.
	}
	
	「項の始切片で項であるものはその項自身に限られる.また,式の始切片で式であるものはその式自身に限られる.」という言明を(★)と書くことにする.
	このメタ定理を示すには次の原理を用いる:
	
	\begin{screen}
		\begin{metaaxm}[$\lang{\in}$の項に対する構造的帰納法]
			$\lang{\in}$の項に対する言明Xに対し(例えば(★)),
			\begin{itemize}
				\item 文字に対してXが言える.
				\item 無作為に選ばれた項$\tau$について,その直部分項に対してXが言える
					と仮定すれば,$\tau$に対してもXが言える.
			\end{itemize}
			ならば,いかなる項に対してもXが言える.
		\end{metaaxm}
	\end{screen}
	
	\begin{screen}
		\begin{metaaxm}[$\lang{\in}$の式に対する構造的帰納法]
			$\lang{\in}$の式に対する言明Xに対し(例えば(★)),
			\begin{itemize}
				\item 原子式に対してXが言える.
				\item 無作為に選ばれた式$\varphi$について,その任意の真部分式に対してXが言える
					と仮定すれば,$\varphi$に対してもXが言える.
			\end{itemize}
			ならば,いかなる式に対してもXが言える.
		\end{metaaxm}
	\end{screen}
	
	では定理を示す.
	
	\begin{metaprf}\mbox{}
		\begin{description}
			\item[項について]
				$s$を項とするとき,$s$が文字ならば$s$の始切片は$s$のみである.つまり(★)が言える.
				$s$が文字でないとき,
				\begin{itembox}[l]{IH (帰納法の仮定)}
					$s$の直部分項$\tau$に対して,
					$\tau$の始切片で$\lang{\in}$の項であるものは$\tau$自身に限られる.
				\end{itembox}
				と仮定する.(項の構成法より)項$t$が取れて$s$は
				\begin{align}
					\natural t
				\end{align}
				と表せる.$u$を$s$の始切片で項であるものとすると
				$u$に対しても(項の構成法より)項$v$が取れて,$u$は
				\begin{align}
					\natural v
				\end{align}
				と表せる.このとき$v$は$t$の始切片であり,
				$t$については(IH)より(★)が言えるので,$t$と$v$は一致する.
				ゆえに$s$と$u$は一致する.ゆえに$s$に対しても(★)が言える.
				
			\item[式について]
				$\bot$については,その始切片は$\bot$に限られる.
				$\in st$なる原子式については,その始切片は
				\begin{align}
					\in, \quad \in s, \quad \in st
				\end{align}
				のいずれかとなるが,このうち式であるものは$\in st$のみである.
				$=st$なる原子式についても,その始切片で式であるものは$=st$に限られる.
	
				いま$\varphi$を任意に与えられた式とし,
				\begin{itembox}[l]{IH (帰納法の仮定)}
					$\varphi$の任意の真部分式$\psi$に対して,
					$\psi$の始切片で$\lang{\in}$の式であるものは$\psi$自身に限られる.
				\end{itembox}
				と仮定する.このとき
				\begin{description}
					\item[case1] $\varphi$が
						\begin{align}
							\negation \psi
						\end{align}
						なる形の式であるとき,$\varphi$の始切片で式であるものもまた
						\begin{align}
							\negation \xi
						\end{align}
						なる形をしている.このとき$\xi$は$\psi$の始切片であるから,
						(IH)より$\xi$と$\psi$は一致する.
						ゆえに$\varphi$の始切片で式であるものは$\varphi$自身に限られる.
			
					\item[case2] $\varphi$が
						\begin{align}
							\vee \psi \xi
						\end{align}
						なる形の式であるとき,$\varphi$の始切片で式であるものもまた
						\begin{align}
							\vee \eta \zeta
						\end{align}
						なる形をしている.このとき$\psi$と$\eta$は一方が他方の始切片であるので
						(IH)より一致する.すると$\xi$と$\zeta$も一方が他方の始切片ということに
						なり,(IH)より一致する.ゆえに$\varphi$の始切片で式であるものは
						$\varphi$自身に限られる.
						
					\item[case3] $\varphi$が
						\begin{align}
							\exists x \psi
						\end{align}
						なる形の式であるとき,$\varphi$の始切片で式であるものもまた
						\begin{align}
							\exists y \xi
						\end{align}
						なる形の式である.このとき$x$と$y$は一方が他方の始切片であり,これらは
						変項であるから前段の結果より一致する.すると$\psi$と$\chi$も
						一方が他方の始切片ということになり,(IH)より一致する.
						ゆえに$\varphi$の始切片で式であるものは$\varphi$自身に限られる.
						\QED
				\end{description}
		\end{description}
	\end{metaprf}
	
\subsection{スコープ}
	$\varphi$を式とし,$s$を$``\natural,\in,\bot,\negation,\vee,\wedge,
	\rarrow,\exists,\forall''$のいずれかの記号とし,$\varphi$に$s$が現れたとする.このとき,
	$s$のその出現位置から始まる$\varphi$の部分式,
	ただし$s$が$\natural$である場合は部分項,を
	$s$の{\bf スコープ}\index{スコープ}{\bf (scope)}と呼ぶ.具体的に,$\varphi$を
	\begin{align}
		\rarrow \forall x \wedge \rarrow \in xy \in xz \rarrow \in xz \in xy = yz
	\end{align}
	なる式とするとき,$\varphi$の左から$6$番目に$\in$が現れるが,この$\in$から
	\begin{align}
		\in xy
	\end{align}
	なる原子式が$\varphi$の上に現れている:
	\begin{align}
		\rarrow \forall x \wedge \rarrow \textcolor{red}{\in xy} \in xz \rarrow \in xz \in xy = yz.
	\end{align}
	これは{\bf $\varphi$における左から6番目の記号$\in$のスコープ}である.他にも,$\varphi$の左から$4$番目に$\wedge$が現れるが,この右側に
	\begin{align}
		\rarrow \in xy \in xz
	\end{align}
	と
	\begin{align}
		\rarrow \in xz \in xy
	\end{align}
	の二つの式が続いていて,$\wedge$を起点に
	\begin{align}
		\wedge \rarrow \in xy \in xz \rarrow \in xz \in xy
	\end{align}
	なる式が$\varphi$の上に現れている:
	\begin{align}
		\rarrow \forall x \textcolor{red}{\wedge \rarrow \in xy \in xz \rarrow \in xz \in xy} = yz.
	\end{align}
	これは{\bf $\varphi$における左から4番目の記号$\wedge$のスコープ}である.$\varphi$の左から$2$番目には$\forall$が現れて,
	この$\forall$に対して項$x$と
	\begin{align}
		\wedge \rarrow \in xy \in xz \rarrow \in xz \in xy
	\end{align}
	なる式が続き,
	\begin{align}
		\forall x \wedge \rarrow \in xy \in xz \rarrow \in xz \in xy
	\end{align}
	なる式が$\varphi$の上に現れている:
	\begin{align}
		\rarrow \textcolor{red}{\forall x \wedge \rarrow \in xy \in xz \rarrow \in xz \in xy} = yz.
	\end{align}
	
	しかも$\in,\wedge,\forall$のスコープは上にあげた部分式のほかに取りようが無い.
	上の具体例を見れば,直感的に「現れた記号のスコープはただ一つだけ,必ず取ることが出来る」
	ということが一般の式に対しても当てはまるように思えるが,
	直感を排除してこれを認めるには構造的帰納法の原理が必要になる.
	
	当然ながら$\lang{\in}$の式には同じ記号が何か所にも出現しうるので,
	式$\varphi$に記号$s$が現れたと言ってもそれがどこの$s$を指定しているのかはっきりしない.
	しかし{\bf スコープを考える際には,$\varphi$に複数現れうる$s$のどれか一つを選んで,
	その$s$に終始注目している}のであり,
	「その$s$の...」や「$s$のその出現位置から...」のように限定詞を付けてそれを示唆することにする.
	
	\begin{screen}
		\begin{metadfn}[$\lang{\in}$のスコープ]
		\label{metadfn:L_in_scope}
			$\theta$を$\lang{\in}$の項或いは式とし,
			記号$s$が$\theta$に現れたとする($s$は$\natural,\in,=,\negation,\vee,\wedge,
			\rarrow,\forall,\exists$のどれかとする).このとき,
			$s$が$\natural$なら$s$のその位置から$\theta$に現れる$\lang{\in}$の項を,
			$s$が他の記号なら$s$のその位置から$\theta$に現れる$\lang{\in}$の式を,
			$\theta$におけるその$s$の{\bf スコープ}\index{スコープ@スコープ}{\bf (scope)}と呼ぶ.
		\end{metadfn}
	\end{screen}
	
	\begin{screen}
		\begin{metathm}[スコープの存在]\label{metathm:existence_of_scopes_L_in}
		$\varphi$を式,或いは項とするとき,
		\begin{description}
			\item[(a)] $\natural$が$\varphi$に現れたとき,項$t$が得られて,
				$\natural$のその出現位置から$\natural t$なる項が$\varphi$の上に現れる.
				
			\item[(b)] $\in$が$\varphi$に現れたとき,項$\sigma$と項$\tau$が得られて,
				$\in$のその出現位置から$\in \sigma \tau$なる式が$\varphi$の上に現れる.
				
			\item[(c)] $\negation$が$\varphi$に現れたとき,式$\psi$が得られて,
				$\negation$のその出現位置から$\negation \psi$なる式が
				$\varphi$の上に現れる.
				
			\item[(d)] $\vee$が$\varphi$に現れたとき,式$\psi$と式$\xi$が得られて,
				$\vee$のその出現位置から$\vee \psi \xi$なる式が$\varphi$の上に現れる.
				
			\item[(e)] $\exists$が$\varphi$に現れたとき,項$x$と式$\psi$が得られて,
				$\exists$のその出現位置から$\exists x \psi$なる式が$\varphi$の上に現れる.
		\end{description}
		\end{metathm}
	\end{screen}
	
	(b)では$\in$を$=$に替えたって同じ主張が成り立つし,(d)では$\vee$を$\wedge$や$\rarrow$に替えても同じである.
	(e)では$\exists$を$\forall$に替えても同じことが言える.
	
	\begin{metaprf}\mbox{}
		\begin{description}
			\item[case1]
				「項に$\natural$が現れたとき,項$t$が取れて,
				その$\natural$の出現位置から$\natural t$がその項の部分項として現れる」---(※),を示す.
				$s$を項とするとき,$s$が文字ならば$s$に対して(※)が言える.
				$s$が文字でないとき,$s$の直部分項に対して(※)が言えるとする.
				$s$は文字ではないので,(項の構成法より)項$t$が取れて$s$は
				\begin{align}
					\natural t
				\end{align}
				と表せる.$s$に現れる$\natural$とは$s$の左端のものであるか
				$t$の中に現れるものであるが,$t$は$s$の直部分項であって,
				$t$については(※)が言えるので,結局$s$に対しても(※)が言えるのである.
			
			%\item[case2]
			%	$\bot$に対しては上の言明は当てはまる.
			
			\item[case2]
				$\in s t$なる式に対しては,$\in$のスコープは$\in s t$に他ならない.
				実際,$\in$から始まる$\in s t$の部分式は,項$u,v$が取れて
				\begin{align}
					\in u v
				\end{align}
				と書けるが,このとき$u$と$s$は一方が他方の始切片となっているので,
				メタ定理\ref{metathm:initial_segment_L_in}より$u$と$s$は一致する.
				すると今度は$v$と$t$について一方が他方の始切片となるので,
				メタ定理\ref{metathm:initial_segment_L_in}より$v$と$t$も一致する.
				
				$\in s t$に$\natural$が現れた場合,これが$s$に現れているとすると,
				前段より項$u$が取れて,この$\natural$の出現位置から$\natural u$なる項が
				$s$の上に現れる.$\natural$が$t$に現れたときも同じである.
				以上より$\in s t$に対して定理の主張が当てはまる.
					
			\item[case3]
				$\varphi$を任意に与えられた式として
				\begin{itembox}[l]{IH (帰納法の仮定)}
					$\varphi$の直部分式に対しては(a)から(e)の主張が当てはまる
				\end{itembox}
				と仮定する.このとき,
				\begin{itemize}
					\item $\varphi$が
						\begin{align}
							\negation \psi
						\end{align}
						なる形の式であるとき,$\natural,\in,\vee,\exists$が
						$\varphi$に現れたなら,それらは$\psi$の中に現れているのだから
						(IH)よりスコープが取れる.また$\varphi$に$\negation$が現れた場合,
						その$\negation$が$\psi$の中のものならば(IH)に訴えれば良いし,
						$\varphi$の左端の$\negation$を指しているなら
						スコープとして$\varphi$自身を取れば良い.
						
					\item $\varphi$が
						\begin{align}
							\vee \psi \chi
						\end{align}
						なる形の式であるとき,$\natural,\in,\negation,\exists$が
						$\varphi$に現れたなら,それらは$\psi$か$\chi$の中に現れているのだから
						(IH)よりスコープが取れる.また$\varphi$に$\vee$が現れた場合,
						その$\vee$が$\psi,\chi$の中のものならば(IH)に訴えれば良いし,
						$\varphi$の左端の$\vee$を指しているなら
						スコープとして$\varphi$自身を取れば良い.
						
					\item $\varphi$が
						\begin{align}
							\exists x \psi
						\end{align}
						なる形の式であるとき,$\natural,\in,\negation,\vee$が
						$\varphi$に現れたなら,それらは$\psi$の中に現れているのだから
						(IH)よりスコープが取れる.また$\varphi$に$\exists$が現れた場合,
						その$\exists$が$\psi$の中のものならば(IH)に訴えれば良いし,
						$\varphi$の左端の$\exists$を指しているなら
						スコープとして$\varphi$自身を取れば良い.
						\QED
				\end{itemize}
		\end{description}
	\end{metaprf}
	
	始切片に関する定理からスコープの一意性を示すことが出来る.
	
	\begin{screen}
		\begin{metathm}[スコープの一意性]\label{metathm:uniqueness_of_scopes_L_in}
			$\varphi$を式とし,$s$を
			$\natural,\in,\bot,\negation,\vee,\wedge,\rarrow,\exists,\forall$
			のいずれかの記号とし,$\varphi$に$s$が現れたとする.
			このとき$\varphi$におけるその$s$のスコープは唯一つである.
		\end{metathm}
	\end{screen}
	
	\begin{metaprf}\mbox{}
		\begin{description}
			\item[case1]
				$\natural$が$\varphi$に現れた場合,スコープの存在定理\ref{metathm:existence_of_scopes_L_in}
				より項$\tau$が取れて
				\begin{align}
					\natural \tau
				\end{align}
				なる形の項が$\natural$のその出現位置から$\varphi$の上に現れるわけだが,
				\begin{align}
					\natural \sigma
				\end{align}
				なる項も$\natural$のその出現位置から$\varphi$の上に出現しているといった場合,
				$\tau$と$\sigma$は一方が他方の始切片となるわけで,
				始切片のメタ定理\ref{metathm:initial_segment_L_in}より
				$\tau$と$\sigma$は一致する.
			
			\item[case2]
				$\negation$が$\varphi$に現れた場合,
				これはcase1において項であったところが式に替わるだけで殆ど同じ証明となる.
			
			\item[case3]
				$\vee$が$\varphi$に現れた場合,定理\ref{metathm:existence_of_scopes_L_in}
				より式$\psi,\xi$が取れて
				\begin{align}
					\vee \psi \xi
				\end{align}
				なる形の式が$\vee$のその出現位置から$\varphi$の上に現れる.ここで
				\begin{align}
					\vee \eta \Gamma
				\end{align}
				なる式も$\vee$のその出現位置から$\varphi$の上に出現しているといった場合,
				まず$\psi$と$\eta$は一方が他方の始切片となるわけで,
				メタ定理\ref{metathm:initial_segment_L_in}より
				$\psi$と$\eta$は一致する.すると今度は$\xi$と$\Gamma$について
				一方が他方の始切片となるので,同様に$\xi$と$\Gamma$も一致する.
				$\wedge$や$\rarrow$のスコープの一意性も同様に示される.
				
			\item[case4]
				$\exists$が$\varphi$に現れた場合,定理\ref{metathm:existence_of_scopes_L_in}
				より項$x$と式$\psi$が取れて
				\begin{align}
					\exists x \psi
				\end{align}
				なる形の式が$\exists$のその出現位置から$\varphi$の上に現れる.ここで
				\begin{align}
					\exists y \xi
				\end{align}
				なる式も$\exists$のその出現位置から$\varphi$の上に出現しているといった場合,
				まず項$x$と項$y$は一方が他方の始切片となるわけで,
				メタ定理\ref{metathm:initial_segment_L_in}より
				$x$と$y$は一致する.すると今度は$\psi$と$\xi$が
				一方が他方の始切片の関係となるので,この両者も一致する.
				$\forall$のスコープの一意性も同様に示される.
				\QED
		\end{description}
	\end{metaprf}
	\section{拡張}
	通常は集合論の言語には$\lang{\in}$が使われる.
	しかし乍ら,当然集合論と称している以上は「集合」というモノを扱っている筈なのに,
	当の「集合」は$\lang{\in}$では実体を持たない空想でしかない.
	どういう意味かというと,例えば
	\begin{align}	
		\exists x\, \forall y\, (\, y \notin x\, )
	\end{align}
	と書けば「$\forall y\, (\, y \notin x\, )$を満たすような集合$x$が存在する」
	と読むわけだが,その在るべき$x$を$\lang{\in}$では特定できないのである
	($\lang{\in}$の``名詞''は変項だけなので).
	しかし言語の拡張の仕方によっては,この``空虚な存在''を実在で補強することが可能になる.
	
	言語の拡張は二段階を踏む.
	項$x$が自由に現れる式$A(x)$に対して
	\begin{align}
		\Set{x}{A(x)}
	\end{align}
	なる形の項を導入する.この項の記法は{\bf 内包的記法}\index{ないほうてききほう@内包的記法}
	{\bf (intensional notation)}と呼ばれる.導入の意図は``$A(x)$を満たす集合$x$の全体''
	という意味を込めた式の対象化であって,実際に後で
	\begin{align}
		\forall u\, \left(\, u \in \Set{x}{A(x)} \lrarrow A(u)\, \right)
	\end{align}
	を保証する(内包性公理).
	
	追加する項はもう一種類ある.$A(x)$を上記のものとするが,この$A(x)$は$x$に関する性質という見方もできる.
	そして``$A(x)$という性質を具えている集合$x$''という意味を込めて
	\begin{align}
		\varepsilon x A(x)
	\end{align}
	なる形の項を導入するのだ.これはHilbertの{\bf $\varepsilon$項}\index{イプシロン項}
	{\bf (epsilon term)}と呼ばれるオブジェクトであるが,
	導入の意図とは裏腹に$\varepsilon x A(x)$は性質$A(x)$を持つとは限らない.
	$\varepsilon x A(x)$が性質$A(x)$を持つのは,$A(x)$を満たす集合$x$が存在するとき,またその時に限られる
	(この点については後述の$\exists$に関する定理によって明らかになる).
	$A(x)$を満たす集合$x$が存在しない場合は,$\varepsilon x A(x)$は正体不明のオブジェクトとなる.
	
\subsection{$\varepsilon$項}
	まずは$\varepsilon$項を項として追加した
	言語$\lang{\varepsilon}$に拡張する.
	$\lang{\varepsilon}$の構成要素は以下である:
	
	\begin{description}
		\item[矛盾記号] $\bot$
		\item[論理記号] $\negation,\ \vee,\ \wedge,\ \rarrow$
		\item[量化子] $\forall,\ \exists,\ \varepsilon$
		\item[述語記号] $=,\ \in$
		\item[変項] \ref{sec:variables}節のもの.
	\end{description}
	
	$\lang{\in}$からの変更点は,``使用文字''が``変項''に代わったことと
	量化子に$\varepsilon$が加わったことである.続いて項と式の定義に移るが,
	帰納のステップは$\lang{\in}$より複雑になる:
	
	\begin{itemize}
		\item $\lang{\varepsilon}$の変項は$\lang{\varepsilon}$の項である.
		\item $\bot$は$\lang{\varepsilon}$の式である.
		\item $\sigma$と$\tau$を$\lang{\varepsilon}$の項とするとき,
			$\in st$と$=st$は$\lang{\varepsilon}$の式である.
		\item $\varphi$を$\lang{\varepsilon}$の式とするとき,
			$\negation \varphi$は$\lang{\varepsilon}$の式である.
		\item $\varphi$と$\psi$を$\lang{\varepsilon}$の式とするとき,
			$\vee \varphi \psi,\ \wedge \varphi \psi,\ \rarrow \varphi \psi$は
			いずれも$\lang{\varepsilon}$の式である.
		\item $x$を変項とし,$\varphi$を
			$\lang{\varepsilon}$の式とするとき,$\forall x \varphi$と
			$\exists x \varphi$は$\lang{\varepsilon}$の式である.
		\item $x$を変項とし,$\varphi$を
			$\lang{\varepsilon}$の式とするとき,$\varepsilon x \varphi$は
			$\lang{\varepsilon}$の項である.
		\item 以上のみが$\lang{\varepsilon}$の項と式である.
	\end{itemize}
	
	$\lang{\in}$に対して行った帰納的定義との大きな違いは,
	{\bf 項と式の定義が循環している}点にある.
	$\lang{\varepsilon}$の式が$\lang{\varepsilon}$の項を用いて
	作られるのは当然ながら,その逆に$\lang{\varepsilon}$の項もまた
	$\lang{\varepsilon}$の式から作られるのである.
	
	\begin{screen}
		\begin{dfn}[$\varepsilon$項]
			$\varepsilon x \varphi$なる項を
			{\bf ${\boldsymbol \varepsilon}$項}\index{いぷしろんこう@$\varepsilon$項}
			{\bf (epsilon term)}と呼ぶ.ここで$x$は変項であり,$\varphi$は
			$\lang{\varepsilon}$の式である.
		\end{dfn}
	\end{screen}
	
	定義通りなら,式$\varphi$に$x$が自由に現れていない場合でも$\varepsilon x \varphi$は
	$\lang{\varepsilon}$の項である.ただしそのような項は全く無用であるから,
	後で実際に集合論を構築する際には排除してしまう(\ref{sec:restriction_of_formulas}節参照).
	
	\begin{screen}
		\begin{metathm}
			$A$を$\lang{\varepsilon}$の式とするとき,
			$\varepsilon x A$なる形の$\varepsilon$項は$A$には現れない.
		\end{metathm}
	\end{screen}
	
	もし$A$に$\varepsilon x A$が現れるならば,当然$A$の中の$\varepsilon x A$にも
	$\varepsilon x A$が現れるし,$A$の中の$\varepsilon x A$の中の$\varepsilon x A$にも
	$\varepsilon x A$が現れるといった具合に,この入れ子には終わりがなくなる.
	だが,当然こんなことは起こり得ない.
	
	\begin{metaprf}
		$A$が指す記号列のどの部分を切り取っても
		それは$A$より短い記号列であって,$\varepsilon x A$の現れる余地など無いからである.
		\QED
	\end{metaprf}
	
	定義の循環によって構造が見えづらくなっているが,$\lang{\varepsilon}$の項と式は
	次の手順で作られている.
	
	\begin{enumerate}
		\item $\lang{\in}$の式から$\varepsilon$項を作り,
			その$\varepsilon$項を第$1$世代$\varepsilon$項と呼ぶことにする.
		\item 変項と第$1$世代$\varepsilon$項を項として式を作り,
			これらを第$2$世代の式と呼ぶことにする.
			また第$2$世代の式で作る$\varepsilon$項を第$2$世代$\varepsilon$項と呼ぶことにする.
		\item 第$n$世代の$\varepsilon$項をが出来たら,
			それらと変項を項として第$n+1$世代の式を作り,
			第$n+1$世代$\varepsilon$項を作る.
			
			\begin{itemize}
				\item ちなみに,このように考えると第$n$世代$\varepsilon$項は
					第$n+1$世代$\varepsilon$項でもある.
			\end{itemize}
	\end{enumerate}
	
	$\lang{\varepsilon}$の項と式は以上のような帰納的構造を持っているのだから,
	$\lang{\varepsilon}$における構造的帰納法はこれに則ったものになる.
	まずは粗く考察してみると,項と式に対する言明Xが与えられたとき,
	
	\begin{enumerate}
		\item まずは$\lang{\in}$の項と式に対してXが言えて,かつ
			第$1$世代の$\varepsilon$項に対してもXが言えることがスタート地点である.
		\item 第$2$世代の式に対してXが言えることと,第$2$世代の$\varepsilon$項に対してXが言えること
			を示す.
			
			$\vdots$
			
		\item 第$n$世代までのすべての式と項に対してXが言えることを仮定して,
			第$n+1$世代の式に対してXが言えることと,第$n+1$世代の$\varepsilon$項に対して
			Xが言えることを示す.
	\end{enumerate}
	の以上が検査出来れば,$\lang{\varepsilon}$のすべての項と式に対してXが言えると
	結論するのは妥当である.ただし第$n$世代だとかいうカテゴライズは直感的考察を補佐するための
	インフォーマルなものであり,更に簡略されたやり方でこの操作が実質的に為されることが期される.
	
	\begin{screen}
		\begin{metaaxm}[$\lang{\varepsilon}$の項と式に対する構造的帰納法]
			$\lang{\varepsilon}$の項に対する言明Xと式に対する言明Yに対し,
			\begin{enumerate}
				\item $\lang{\in}$の項と式,および$\lang{\in}$の式
					で作る$\varepsilon$項に対してX及びYが言える.
				\item $\varphi$を任意に与えられた$\lang{\varepsilon}$の式として,
					$\varphi$に現れる全ての項及び真部分式に対して
					X及びYが言えると仮定するとき,
					\begin{itemize}
						\item $\varphi$が$\in \sigma \tau$なる形の原子式であるとき
							$\varphi$に対してYが言える.
						\item $\varphi$が$\negation \varphi$なる形の式であるとき
							$\varphi$に対してYが言える.
						\item $\varphi$が$\vee \psi \chi$なる形の式であるとき
							$\varphi$に対してYが言える.
						\item $\varphi$が$\exists x \psi$なる形の式であるとき
							$\varphi$に対してYが言える.
						\item $\varepsilon x \varphi$なる$\varepsilon$項
							に対してXが言える.
					\end{itemize}
			\end{enumerate}
			ならば,いかなる項と式に対してもXが言える.
		\end{metaaxm}
	\end{screen}
	
	$\varphi$を$\lang{\varepsilon}$の式としたら,$\varphi$の部分式とは,
	$\varphi$から切り取られる一続きの記号列で,それ自身が$\lang{\varepsilon}$の式であるものを指す.
	$\varphi$自身もまた$\varphi$の部分式である.
	
	\begin{screen}
		\begin{metathm}[$\lang{\varepsilon}$の始切片の一意性]
		\label{metathm:initial_segment_L_epsilon}
			$\tau$を$\lang{\varepsilon}$の項とするとき,
			$\tau$の始切片で$\lang{\varepsilon}$の項であるものは$\tau$自身に限られる.
			また$\varphi$を$\lang{\varepsilon}$の式とするとき,
			$\varphi$の始切片で$\lang{\varepsilon}$の式であるものは$\varphi$自身に限られる.
		\end{metathm}
	\end{screen}
	
	\begin{metaprf}\mbox{}
		\begin{description}
			\item[step1]
				$\lang{\in}$の式と項についてはメタ定理\ref{metathm:initial_segment_L_in}より
				当座の定理の主張が従う.また$\varphi$を$\lang{\in}$の式とし,
				$\tau$を$\lang{\varepsilon}$の項とし,また$\tau$は
				\begin{align}
					\varepsilon x \varphi
				\end{align}
				なる$\varepsilon$項の始切片とするとき,$\tau$の左端は$\varepsilon$であるから
				\begin{align}
					\varepsilon y \psi
				\end{align}
				なる形をしているはずである.すると$x$と$y$とは一方が他方の始切片となるので
				メタ定理\ref{metathm:initial_segment_L_in}より$y$は$x$に一致する.
				するとまた$\varphi$と$\psi$はは一方が他方の始切片となるので一致する.
				つまり$\tau$は$\varepsilon x \varphi$そのものである.
				
			\item[step2]
				$\varphi$を$\lang{\varepsilon}$の式とするとき,$\varphi$の
				すべての項や真部分式に対して定理の主張が当たっているなら
				$\varphi$に対しても定理の主張通りのことが満たされる,
				ということはメタ定理\ref{metathm:initial_segment_L_in}と同じように示される.
				もう一度書けば,
				\begin{itembox}[l]{IH (帰納法の仮定)}
					$\varphi$に現れる任意の項$\tau$に対して,その始切片で項であるものは$\tau$
					に限られる.また$\varphi$に現れる任意の真部分式$\psi$に対して,
					その始切片で式であるものは$\psi$に限られる.
				\end{itembox}
				として
				\begin{description}
					\item[case1]
						$\varphi$が
						\begin{align}
							\in s t
						\end{align}
						なる原子式であるとき,$\varphi$の始切片で式であるものもまた
						\begin{align}
							\in u v
						\end{align}
						なる形をしているが,$u$と$s$は一方が他方の始切片となっているので
						(IH)より一致する.すると$v$と$t$も一方が他方の始切片となるので
						(IH)より一致する.ゆえに$\varphi$の始切片で式であるもの
						は$\varphi$自信に限られる.
						
					\item[case2] $\varphi$が
						\begin{align}
							\negation \psi
						\end{align}
						なる形の式であるとき,$\varphi$の始切片で式であるももまた
						\begin{align}
							\negation \xi
						\end{align}
						なる形をしている.このとき$\xi$は$\psi$の始切片であるから,
						(IH)より$\xi$と$\psi$は一致する.
						ゆえに$\varphi$の始切片で式であるものは$\varphi$自身に限られる.
			
					\item[case3] $\varphi$が
						\begin{align}
							\vee \psi \xi
						\end{align}
						なる形の式であるとき,$\varphi$の始切片で式であるものもまた
						\begin{align}
							\vee \eta \zeta
						\end{align}
						なる形をしている.このとき$\psi$と$\eta$は一方が他方の始切片であるので
						(IH)より一致する.すると$\xi$と$\zeta$も一方が他方の始切片ということに
						なり,(IH)より一致する.ゆえに$\varphi$の始切片で式であるものは
						$\varphi$自身に限られる.
						
					\item[case4] $\varphi$が
						\begin{align}
							\exists x \psi
						\end{align}
						なる形の式であるとき,$\varphi$の始切片で式であるものもまた
						\begin{align}
							\exists y \xi
						\end{align}
						なる形の式である.このとき$x$と$y$は一方が他方の始切片であり,これらは
						変項であるからメタ定理\ref{metathm:initial_segment_L_in}
						より一致する.すると$\psi$と$\chi$も一方が他方の始切片ということに
						なり,(IH)より一致する.ゆえに$\varphi$の始切片で式であるものは
						$\varphi$自身に限られる.
						
					\item[case5] $\varepsilon x \varphi$の始切片で項であるものは
						\begin{align}
							\varepsilon y \psi
						\end{align}
						なる形をしている筈である.このとき,まずメタ定理
						\ref{metathm:initial_segment_L_in}より$x$と$y$は一致する.
						すると$\psi$は$\varphi$の始切片であることになるが,
						前段までの結果から$\varphi$と$\psi$は一致する.
						\QED
				\end{description}
		\end{description}
	\end{metaprf}
	
	\begin{screen}
		\begin{metathm}[$\lang{\varepsilon}$のスコープの存在]
		\label{metathm:existence_of_scopes_L_epsilon}
			$\varphi$を$\lang{\varepsilon}$の式,或いは項とするとき,
			\begin{description}
				\item[(a)] $\natural$が$\varphi$に現れたとき,変項$t$が得られて,
					$\natural$のその出現位置から$\natural t$なる変項が$\varphi$の上に現れる.
					
				\item[(b)] $\in$が$\varphi$に現れたとき,$\lang{\varepsilon}$の項$\sigma,\tau$が得られて,
					$\in$のその出現位置から$\in \sigma \tau$なる式が$\varphi$の上に現れる.
				
				\item[(c)] $\negation$が$\varphi$に現れたとき,
					$\lang{\varepsilon}$の式$\psi$が得られて,
					$\negation$のその出現位置から
					$\negation \psi$なる式が$\varphi$の上に現れる.
				
				\item[(d)] $\vee$が$\varphi$に現れたとき,$\lang{\varepsilon}$の式$\psi,\xi$が得られて,
					$\vee$のその出現位置から$\vee \psi \xi$なる式が$\varphi$の上に現れる.
				
				\item[(e)] $\exists$が$\varphi$に現れたとき,変項$x$と$\lang{\varepsilon}$の式$\psi$が得られて,
					$\exists$のその出現位置から$\exists x \psi$なる式が$\varphi$の上に現れる.
			\end{description}
		\end{metathm}
	\end{screen}
	
	(b)では$\in$を$=$に替えたって同じ主張が成り立つし,(d)では$\vee$を$\wedge$や$\rarrow$に替えても同じである.
	(e)では$\exists$を$\forall$に替えても同じであるのは良いとして,
	$\varepsilon$項の成り立ちから$\exists$を$\varepsilon$に替えても同様の主張が成り立つ.
	
	示すのはスコープの存在だけで良い.一意性は始切片の定理からすぐに従う.実際
	$\varphi$を$\lang{\varepsilon}$の式として,その中に$\varepsilon$が出現したとすると,
	``スコープの存在が保証されていれば!''$\varepsilon$のその出現位置から
	\begin{align}
		\varepsilon x \psi
	\end{align}
	なる$\varepsilon$項が$\varphi$の上に現れるわけだが,他の誰かが「$\varepsilon y \xi$という
	$\varepsilon$項がその$\varepsilon$の出現位置から抜き取れるぞ」と言ってきたとしても,
	当然ながら$x$と$y$は一方が他方の始切片となるので一致する変項であるし(メタ定理\ref{metathm:initial_segment_L_in}),
	すると今度は$\psi$と$\xi$の一方が他方の始切片となるが,そのときもメタ定理\ref{metathm:initial_segment_L_epsilon}より
	両者は一致する.
	
	\begin{metaprf}\mbox{}
		\begin{description}
			\item[step1]
				$\varphi$が$\lang{\in}$の式であるときは,スコープの存在は
				メタ定理\ref{metathm:existence_of_scopes_L_in}で既に示されている.
				また$\lang{\in}$の式$\psi$に対して,
				\begin{align}
					\varepsilon x \psi
				\end{align}
				なる形の$\varepsilon$項に対しても
				(a)から(e)が満たされる.実際,(b)から(e)に関しては,
				$\in,\negation,\vee,\exists$は
				$\psi$の中にしか出現し得ないので,スコープの存在は
				メタ定理\ref{metathm:existence_of_scopes_L_in}により保証される.
				(a)については,$\natural$は$\psi$の中に現れる場合と$x$の中に現れる場合があるが,
				いずれの場合もメタ定理\ref{metathm:existence_of_scopes_L_in}より
				スコープは取れる.
			
				ここで$\varphi$を任意に与えられた$\lang{\varepsilon}$の
				式として,次の仮定を置く.
				\begin{itembox}[l]{IH(帰納法の仮定)}
					$\varphi$の全ての部分式,及び
					$\varphi$に現れる全ての$\varepsilon$項の式,つまり
					$\varepsilon x \psi$なる項における$\psi$のこと,
					に対して(a)から(e)まで言えると仮定する.
				\end{itembox}
				
			\item[step2]
				式$\varphi$が$\in s t$なる形の式であるとき.
				\begin{description}
					\item[case1]
						$\natural$が$\in s t$に現れたとしよう.
						$s$や$t$が変項であれば(a)の成立は見た目通りである.$s$が
						\begin{align}
							\varepsilon x \psi
						\end{align}
						なる形の$\varepsilon$項であって,
						$s$にその$\natural$が現れているとしよう.
						$\natural$が$x$に現れている場合は
						メタ定理\ref{metathm:existence_of_scopes_L_in}に訴えればよい.
						$\natural$が$\psi$に現れている場合は,(a)の成立は(IH)から従う.
						
					\item[case2]
						$\in$が$\in s t$に現れたとしよう.
						それが左端の$\in$であれば,(b)の成立を言うには$s$と$t$を取れば良い.
						$\in$が$s$に現れたとすれば,$s$は$\varepsilon$項であることになり,
						変項$x$と$\lang{\varepsilon}$の式$\psi$が取れて,$s$は
						\begin{align}
							\varepsilon x \psi
						\end{align}
						と表せる.$\in$は$\psi$に現れるので,(IH)より$\lang{\varepsilon}$の項$u,v$が取れて,
						$\in$のその出現位置から$\in s t$なる式が$\psi$の上に現れる.
						$\in$が$t$に現れる場合も同様に(b)の成立が言える.
				
					\item[case3]
						$\in s t$に論理記号($\negation,\vee,\wedge,\rarrow,\exists,\forall$のいずれか)
						が現れたとしよう.
						そしてその現れた記号を便宜上$\sigma$と書こう.
						$\sigma$の出現位置が$s$にあるとすれば,そのことは$s$が
						\begin{align}
							\varepsilon x \psi
						\end{align}
						なる形の$\varepsilon$項であることを意味する.当然$\sigma$は$\psi$の中にあるわけで,
						(c)もしくは(d)の成立は(IH)から従う.
						
					\item[case4]
						$\in s t$に$\varepsilon$が現れたとしよう.
						$\varepsilon$の出現位置が$s$にあるとすれば,そのことは$s$が
						\begin{align}
							\varepsilon x \psi
						\end{align}
						なる形の$\varepsilon$項であることを意味する.
						$\varepsilon$の出現位置が$s$の左端である場合,(e)の成立を言うには
						この$x$と$\psi$を取れば良い.
						$\varepsilon$が$\psi$の中にある場合は,
						$(e)$の成立は(IH)から従う.
				\end{description}
				
			\item[step3]
				式$\varphi$が$\negation \psi$なる形のとき,
				$\varphi$に現れた記号は左端の$\negation$であるか,そうでなければ
				$\psi$の中に現れる.左端の$\negation$のスコープは$\varphi$自身である.
				$\psi$に現れた記号のスコープの存在は
				(IH)により保証される.
				
			\item[step4]
				式$\varphi$が$\vee \psi \xi$なる形のとき,
				$\varphi$に現れた記号は左端の$\vee$であるか,そうでなければ
				$\psi \xi$の中に現れる.左端の$\vee$のスコープは$\varphi$自身である.
				$\psi \xi$に現れた記号のスコープの存在は(IH)により保証される.
			
			\item[step5]
				式$\varphi$が$\exists x \psi$なる形のとき,
				$\varphi$に現れた記号は左端の$\exists$であるか,そうでなければ
				$\psi$の中に現れる.左端の$\exists$のスコープは$\varphi$自身である.
				$\psi$に現れた記号のスコープの存在は(IH)により保証される.
				\QED
		\end{description}
	\end{metaprf}
	\subsection{内包項}
	本稿における主流の言語は,次に定める$\mathcal{L}$である.$\mathcal{L}$の最大の特徴は
	\begin{align}
		\Set{x}{\varphi(x)}
	\end{align}
	なる形のオブジェクトが``正式に''項として用いられることである.
	他の多くの集合論の本では$\Set{x}{\varphi(x)}$なる項はインフォーマルに導入されるものであるが,
	インフォーマルなものでありながらこの種のオブジェクトはいたるところで堂々と登場するので,
	やはりフォーマルに導入して然るべきである.
	
	$\mathcal{L}$の構成要素は以下のものである.
	
	\begin{description}
		\item[矛盾記号] $\bot$
		\item[論理記号] $\negation,\ \vee,\ \wedge,\ \rarrow$
		\item[量化子] $\forall,\ \exists$
		\item[述語記号] $=,\ \in$
		\item[変項] \ref{sec:variables}節のもの.
		\item[補助記号] $\{,\ |,\ \}$
	\end{description}
	
	$\mathcal{L}$の項と式の構成規則は$\lang{\in}$のものと大差ない.
	
	\begin{description}
		\item[項] 
			\begin{itemize}
				\item 変項は$\mathcal{L}$の項である.
				\item $\lang{\varepsilon}$の項は$\mathcal{L}$の項である.
				\item $x$を$\mathcal{L}$の変項とし,$\varphi$を
					$\lang{\varepsilon}$の式とするとき,
					$\Set{x}{\varphi}$なる記号列は$\mathcal{L}$の項である.
				\item 以上のみが$\mathcal{L}$の項である.
			\end{itemize}
	\end{description}
	
	によって正式に定義される.
	
	\begin{description}
		\item[式] 
			\begin{itemize}
				\item $\bot$は$\mathcal{L}$の式である.
				\item $\sigma$と$\tau$を$\mathcal{L}$の項とするとき,
					$\in st$と$=st$は$\mathcal{L}$の式である.
					これらは$\mathcal{L}$の{\bf 原子式}\index{げんししき@原子式}
					{\bf (atomic formula)}である.
				\item $\varphi$を$\mathcal{L}$の式とするとき,
					$\negation \varphi$は$\mathcal{L}$の式である.
				\item $\varphi$と$\psi$を$\mathcal{L}$の式とするとき,
					$\vee \varphi \psi,\ \wedge \varphi \psi,\ \rarrow \varphi \psi$は
					いずれも$\mathcal{L}$の式である.
				\item $x$を$\mathcal{L}$の{\bf 変項}とし,$\varphi$を
					$\mathcal{L}$の式とするとき,$\forall x \varphi$と
					$\exists x \varphi$は$\mathcal{L}$の式である.
			\end{itemize}
	\end{description}
	
	\begin{screen}
		\begin{dfn}[内包項]
			$\Set{x}{\varphi}$なる項を{\bf 内包項}\index{ないほうこう@内包項}
			と呼ぶ.ここで$x$は変項であり,$\varphi$は$\mathcal{L}$の式である.
		\end{dfn}
	\end{screen}
	
	定義通りなら,$\Set{x}{y=y}$のように式$\varphi$に$x$が自由に現れていない場合でも
	$\Set{x}{\varphi}$は$\mathcal{L}$の項である.ただしそのような項は全く無用であるから,
	後で実際に集合論を構築する際には排除してしまう(\ref{sec:restriction_of_formulas}節参照).
	
	\begin{screen}
		\begin{metathm}
			$\lang{\in}$の式は$\lang{\varepsilon}$の式であり,
			また$\lang{\varepsilon}$の式は$\mathcal{L}$の式である.
		\end{metathm}
	\end{screen}
	
	\begin{metaprf}\mbox{}
		\begin{description}
			\item[step1]
				式の構成法より$\lang{\in}$の原子式は$\lang{\varepsilon}$の式である.
				また$\varphi$を任意に与えられた$\lang{\in}$の式とするとき,
				\begin{itembox}[l]{IH (帰納法の仮定)}
					$\varphi$のすべての真部分式は$\lang{\varepsilon}$の式である
				\end{itembox}
				と仮定すると,$\varphi$が
				\begin{description}
					\item[case1] $\negation \psi$
					\item[case2] $\vee \psi \chi$
					\item[case3] $\exists x \psi$
				\end{description}
				のいずれの形の式であっても,$\psi$も$\chi$も(IH)より$\lang{\varepsilon}$の式
				であるから,式の構成法より$\varphi$自信も$\lang{\varepsilon}$の式である.
				ゆえに$\lang{\in}$の式は$\lang{\varepsilon}$の式である.
				
			\item[step2]
				$\lang{\varepsilon}$の式が$\mathcal{L}$の式であることを示す.
				まず,$\mathcal{L}$の式の構成において使える項を変項に制限すれば
				全ての$\lang{\in}$の式が作られるのだから
				$\lang{\in}$の式は$\mathcal{L}$の式である.
				また$\varphi$を任意に与えられた$\lang{\varepsilon}$の式とするとき,
				\begin{itembox}[l]{IH (帰納法の仮定)}
					$\varphi$のすべての真部分式は$\mathcal{L}$の式である
				\end{itembox}
				と仮定すると(今回は予め$\lang{\varepsilon}$の項は
				$\mathcal{L}$の項とされているので,真部分式に対する仮定のみで十分である),
				\begin{description}
					\item[case1] $\varphi$が$\in \sigma \tau$なる形の原子式であるとき,
						$\sigma$も$\tau$も$\mathcal{L}$の項であるから
						$\in \sigma \tau$は$\mathcal{L}$の式である.
						
					\item[case2] $\varphi$が$\negation \psi$なる形の式であるとき,
						(IH)より$\psi$は$\mathcal{L}$の式であるから
						$\negation \psi$も$\mathcal{L}$の式である.
						
					\item[case3] $\varphi$が$\vee \psi \chi$なる形の式であるとき,
						(IH)より$\psi$も$\chi$も$\mathcal{L}$の式であるから
						$\vee \psi \chi$も$\mathcal{L}$の式である.
						
					\item[case4] $\varphi$が$\exists x \psi$なる形の式であるとき,
						(IH)より$\psi$は$\mathcal{L}$の式であるから
						$\exists x \psi$も$\mathcal{L}$の式である.
				\end{description}
				となる.ゆえに$\lang{\varepsilon}$の式は$\mathcal{L}$の式である.
				\QED
		\end{description}
	\end{metaprf}
	
	\begin{screen}
		\begin{metaaxm}[$\mathcal{L}$の式に対する構造的帰納法]
			$\mathcal{L}$の式に対する言明Xに対し,
			\begin{itemize}
				\item 原子式に対してXが言える.
				\item 無作為に選ばれた式$\varphi$について,その全ての真部分式に対してXが言える
					と仮定すれば,$\varphi$に対してもXが言える.
			\end{itemize}
			ならば,いかなる式に対してもXが言える.
		\end{metaaxm}
	\end{screen}
	
	$\mathcal{L}$の項は帰納的な構成になっていないので構造的帰納法は不要である.
	
	\begin{screen}
		\begin{metathm}[$\mathcal{L}$の始切片の一意性]
		\label{metathm:initial_segment_L}
			$\tau$を$\mathcal{L}$の項とするとき,$\tau$の始切片で$\mathcal{L}$の項であるものは
			$\tau$自信に限られる.また$\varphi$を$\mathcal{L}$の式とするとき,$\varphi$の
			始切片で$\mathcal{L}$の式であるものは$\varphi$自信に限られる.
		\end{metathm}
	\end{screen}
	
	\begin{metaprf}\mbox{}
		\begin{description}
			\item[項について]
				$\tau$を項とするとき,$\tau$が変項ならば
				メタ定理\ref{metathm:initial_segment_L_in}によって,
				$\tau$が$\lang{\varepsilon}$の項ならば
				メタ定理\ref{metathm:initial_segment_L_epsilon}によって,
				$\tau$の始切片で$\mathcal{L}$の項であるものは$\tau$自身に限られる.
				$\tau$が
				\begin{align}
					\Set{x}{\varphi}
				\end{align}
				なる内包項である場合,$\tau$の始切片で項であるものも
				\begin{align}
					\Set{y}{\psi}
				\end{align}
				なる形をしている.メタ定理\ref{metathm:initial_segment_L_in}より
				$x$と$y$が一致し,メタ定理\ref{metathm:initial_segment_L_epsilon}より
				$\varphi$と$\psi$も一致するので,この場合も$\tau$の始切片で項であるものは
				$\tau$自身に限られる.
				
			\item[式について]
				$\in st$なる原子式については,その始切片で式であるものは
				\begin{align}
					\in uv
				\end{align}
				なる形をしているが,前段の結果より$s$と$u$,$t$と$v$は一致する.
				$=st$なる原子式についても,その始切片で$\mathcal{L}$の式であるものは
				$=st$に限られる.
				
				いま$\varphi$を任意に与えられた$\mathcal{L}$の式とし,
				\begin{itembox}[l]{IH (帰納法の仮定)}
					$\varphi$に現れる任意の真部分式$\psi$に対して,
					その始切片で式であるものは$\psi$に限られる.
				\end{itembox}
				と仮定する.このとき
				\begin{description}
					\item[case1] $\varphi$が
						\begin{align}
							\negation \psi
						\end{align}
						なる形の式であるとき,$\varphi$の始切片で式であるものもまた
						\begin{align}
							\negation \xi
						\end{align}
						なる形をしている.このとき$\xi$は$\psi$の始切片であるから,
						(IH)より$\xi$と$\psi$は一致する.
						ゆえに$\varphi$の始切片で式であるものは$\varphi$自身に限られる.
			
					\item[case2] $\varphi$が
						\begin{align}
							\vee \psi \xi
						\end{align}
						なる形の式であるとき,$\varphi$の始切片で式であるものもまた
						\begin{align}
							\vee \eta \zeta
						\end{align}
						なる形をしている.このとき$\psi$と$\eta$は一方が他方の始切片であるので
						(IH)より一致する.すると$\xi$と$\zeta$も一方が他方の始切片ということに
						なり,(IH)より一致する.ゆえに$\varphi$の始切片で式であるものは
						$\varphi$自身に限られる.
						
					\item[case3] $\varphi$が
						\begin{align}
							\exists x \psi
						\end{align}
						なる形の式であるとき,$\varphi$の始切片で式であるものもまた
						\begin{align}
							\exists y \xi
						\end{align}
						なる形の式である.このとき$x$と$y$は一方が他方の始切片であり,これらは
						変項であるからメタ定理\ref{metathm:initial_segment_L_in}より
						一致する.すると$\psi$と$\chi$も一方が他方の始切片ということになり,
						(IH)より一致する.
						ゆえに$\varphi$の始切片で式であるものは$\varphi$自身に限られる.
						\QED
				\end{description}
		\end{description}
	\end{metaprf}
	
	$\varphi$を$\mathcal{L}$の式とし,$s$を
	\begin{align}
		\natural,\ \{,\ \in,\ \negation,\ \vee,
		\ \wedge,\ \rarrow,\ \exists,\ \forall,\ \varepsilon
	\end{align}
	のいずれかの記号とするとき,$s$が$\varphi$に現れたら$s$のその出現位置から始まる$\varphi$の部分式
	(ただし$s$が``$\natural,\{,\varepsilon$''である場合は部分項)を$s$の
	{\bf スコープ}\index{スコープ}{\bf (scope)}と呼ぶ.ところで$\varphi$には
	\begin{align}
		|, \quad \}
	\end{align}
	も現れるので,これらにもスコープを割り当てるために
	\begin{itemize}
		\item $\varphi$に``$|$''が現れたら,``$|$''のその出現位置を跨いで$\varphi$の上に
			現れる内包項$\Set{x}{\psi}$をその``$|$''のスコープと呼ぶ.
			つまり現れた``$|$''とは$\Set{x}{\psi}$の中心線``$|$''のことである.
			
		\item $\varphi$に``$\}$''が現れたら,``$\}$''のその出現位置を右端にして$\varphi$の上に
			現れる内包項$\Set{x}{\psi}$をその``$\}$''のスコープと呼ぶ.
			つまり現れた``$\}$''とは$\Set{x}{\psi}$の右端の``$\}$''のことである.
	\end{itemize}
	と定める.すると,次のメタ定理によって``$\natural,\ \{,\ |,\ \},\ \in,\ \negation,\ \vee,
	\ \wedge,\ \rarrow,\ \exists,\ \forall,\ \varepsilon$''の全ての記号に対して
	スコープが取れることが保証される.
	
	取れるスコープの唯一性はメタ定理\ref{metathm:initial_segment_L}からすぐに従い,
	その証明は$\lang{\in}$や$\lang{\varepsilon}$の場合と殆ど同様であるが,
	``$|$''と``$\}$''のスコープの唯一性について書いておくと
	\begin{itemize}
		\item $\varphi$の中で``$|$''のスコープ$\Set{x}{\psi}$と$\Set{y}{\chi}$が取れたとすれば,
			$\psi$と$\chi$は$\varphi$の中で同じ位置から始まる式であるから
			メタ定理\ref{metathm:initial_segment_L_epsilon}より一致する.
			また$x$と$y$は変項であるからその中に``$\{$''が現れるはずはなく,$x$と$y$も一致すると判る.
			
		\item $\varphi$の中で``$\}$''のスコープ$\Set{x}{\psi}$と$\Set{y}{\chi}$が取れたとすれば,
			$\psi$と$\chi$は$\lang{\varepsilon}$の式であるからその中に``$|$''が現れるはずはなく,
			両者は一致していなくてはならない.すると上と同様に$x$と$y$も一致していなくてはならない.
	\end{itemize}
	
	\begin{screen}
		\begin{metathm}[$\mathcal{L}$のスコープの存在]
		\label{metathm:existence_of_scopes_L}
			$\varphi$を$\mathcal{L}$の式,或いは$\mathcal{L}$の項とするとき,
			\begin{description}
				\item[(a)] $\natural$が$\varphi$に現れたとき,変項$t$が得られて,
					$\natural$のその位置から$\natural t$なる項が$\varphi$の上に現れる.
					
				\item[(b)] $\{$が$\varphi$に現れたとき,変項$x$と$\mathcal{L}$の式$\psi$が得られて,
					$\{$のその出現位置から$\Set{x}{\psi}$なる項が$\varphi$の上に現れる.
					
				\item[(c)] $|$が$\varphi$に現れたとき,変項$x$と$\mathcal{L}$の式$\psi$が得られて,
					$|$のその出現位置を跨いで$\Set{x}{\psi}$なる項が$\varphi$の上に現れる.
					
				\item[(d)] $\}$が$\varphi$に現れたとき,変項$x$と$\mathcal{L}$の式$\psi$が得られて,
					$\}$のその出現位置右端にして$\Set{x}{\psi}$なる項が$\varphi$の上に現れる.
					
				\item[(e)] $\in$が$\varphi$に現れたとき,$\mathcal{L}$の項$\sigma,\tau$が得られて,
					$\in$のその出現位置から$\in \sigma \tau$なる式が$\varphi$の上に現れる.
				
				\item[(f)] $\negation$が$\varphi$に現れたとき,$\mathcal{L}$の式$\psi$が得られて,
					$\negation$のその出現位置から$\negation \psi$なる式が
					$\varphi$の上に現れる.	
				
				\item[(g)] $\vee$が$\varphi$に現れたとき,$\mathcal{L}$の式$\psi,\xi$が得られて,
					$\vee$のその出現位置から$\vee \psi \xi$なる式が$\varphi$の上に現れる.
				
				\item[(h)] $\exists$が$\varphi$に現れたとき,変項$x$と$\mathcal{L}$の式$\psi$が得られて,
					$\exists$のその出現位置から$\exists x \psi$なる式が$\varphi$の上に現れる.
			\end{description}
		\end{metathm}
	\end{screen}
	
	\begin{metaprf}\mbox{}
		\begin{description}
			\item[case1] $\in st$なる原子式に対しては,
				\begin{itemize}
					\item $\natural,\negation,\vee,\exists$が現れたとすれば,
						それらは$s$か$t$の中に現れているのであり,
						メタ定理\ref{metathm:existence_of_scopes_L_in}と
						メタ定理\ref{metathm:existence_of_scopes_L_epsilon}より
						それらのスコープは取れる.仮に$s$と$t$の一方が
						\begin{align}
							\Set{x}{\psi}
						\end{align}
						なる内包項であるとしても,$\natural,\negation,\vee,\exists$が
						現れうるのは$x$或いは$\psi$の中であるから,
						スコープの存在は上記のメタ定理に訴えればよい.
				
					\item $\in st$に$\in$が現れたとすれば,それが$s,t$の中のものならば
						上記の定理によってスコープは取れるし,それが$\in st$の左端の
						$\in$を指しているなら$\in st$自身をスコープとして取れば良い.
						
					\item $\in st$に$\{,\ |,\ \}$が現れたとすれば,$s$と$t$の少なくとも一方は
						\begin{align}
							\Set{x}{\psi}
						\end{align}
						なる項であることになるので,スコープとしてこの内包項を取れば良い.
				\end{itemize}
				
			\item[case2] $\varphi$を任意に与えられた$\mathcal{L}$の式として
				$\varphi$を任意に与えられた式として
				\begin{itembox}[l]{IH (帰納法の仮定)}
					$\varphi$の全ての真部分式に対しては(a)から(h)の主張が当てはまる
				\end{itembox}
				と仮定する.このとき,
				\begin{itemize}
					\item $\varphi$が
						\begin{align}
							\negation \psi
						\end{align}
						なる形の式であるとき,$\natural,\{,|,\},\in,\vee,\exists$が
						$\varphi$に現れたなら,それらは$\psi$の中に現れているのだから
						(IH)よりスコープが取れる.また$\varphi$に$\negation$が現れた場合,
						その$\negation$が$\psi$の中のものならば(IH)に訴えれば良いし,
						$\varphi$の左端の$\negation$を指しているなら
						スコープとして$\varphi$自身を取れば良い.
						
					\item $\varphi$が
						\begin{align}
							\vee \psi \chi
						\end{align}
						なる形の式であるとき,$\natural,\{,|,\},\in,\negation,\exists$が
						$\varphi$に現れたなら,それらは$\psi$か$\chi$の中に現れているのだから
						(IH)よりスコープが取れる.また$\varphi$に$\vee$が現れた場合,
						その$\vee$が$\psi,\chi$の中のものならば(IH)に訴えれば良いし,
						$\varphi$の左端の$\vee$を指しているなら
						スコープとして$\varphi$自身を取れば良い.
						
					\item $\varphi$が
						\begin{align}
							\exists x \psi
						\end{align}
						なる形の式であるとき,$\natural,\{,|,\}\in,\negation,\vee$が
						$\varphi$に現れたなら,それらは$\psi$の中に現れているのだから
						(IH)よりスコープが取れる.また$\varphi$に$\exists$が現れた場合,
						その$\exists$が$\psi$の中のものならば(IH)に訴えれば良いし,
						$\varphi$の左端の$\exists$を指しているなら
						スコープとして$\varphi$自身を取れば良い.
						\QED
				\end{itemize}
		\end{description}
	\end{metaprf}
	
\subsection{量化}
	$\varphi$を$\mathcal{L}$の式とする.もし$\varphi$に$\forall$が現れたら,
	その$\forall$に後続する変項$x$と式$\psi$が取れるが,そのとき$x$は
	\begin{align}
		\forall x \psi
	\end{align}
	の中で{\bf 「量化されている」}\index{りょうか@量化}{\bf(quantified)}や
	{\bf 「束縛されている」}\index{そくばく@束縛}{\bf (bound)}という.
	同様に$\varphi$の中に$\exists$や$\varepsilon$が現れたら,
	その$\exists$ (または$\varepsilon$)の直後にくる変項は,
	「その$\exists$ (または$\varepsilon$)のスコープの中で量化されている」といい,
	また$\varphi$の中に
	\begin{align}
		\Set{x}{\psi}
	\end{align}
	なる内包項が現れたら,$x$は「この内包項の中で量化されている」という.
	他方で$\psi$の中に$x$とは別の変項が現れていても,その変項は
	$\forall x \psi,\ \exists x \psi,\ \varepsilon x \psi,\ \Set{x}{\psi}$
	の中では「量化されていない」と解釈する.
	まとめれば,$\forall,\exists,\varepsilon,$そして$\{$は
	直後に来る変項のみをそのスコープ内で量化しているのである.たとえば
	\begin{align}
		\forall x\, (\, x \in y\, )
	\end{align}
	においては$x$は量化されているし,
	\begin{align}
		\Set{u}{u = z}
	\end{align}
	において$u$は量化されている.量化は二重に行われることもある.例えば
	\begin{align}
		\forall x\, (\, \forall x\, (\, x \in y\, ) \rarrow (\, x \in z\, )\, )
	\end{align}
	なる式においては,$\forall x\, (\, x \in y\, )$にある$x$は
	上式で一番左の$\forall$のスコープ内の$x$でもあるので,これらの$x$は二重に量化されていることになる.
	仮に「何重にも量化されている場合は最も狭いスコープで量化されていることにする」と決めても良いが,
	ただし重要なのは変項が量化されているか否かであって,それが二重でも三重でもどうでも構わない.
	
	上の例では$y$と$z$は量化されていないが,考えている項や式の中で量化されていない変項
	を{\bf 自由な}\index{じゆう@自由}{\bf (free)}変項と呼ぶ.
	現れる変項が自由であるか否かは当然その出現位置に依存しているのであり,たとえば
	\begin{align}
		\forall x\, (\, x \in y\, ) \rarrow (\, x \in z\, )
	\end{align}
	なる式では左の二つの$x$が量化されている一方で右の$x$は自由であるように,
	同じ変項が複数個所に現れる場合はその変項が量化されているか自由であるかは一概には言えない.
	式$\varphi$の中に量化されていない変項が現れている場合は,
	その変項が``その位置''に現れていることを
	{\bf 自由な出現}\index{じゆうなしゅつげん@自由な出現}{\bf (free occurrence)}と呼ぶ.
	
	\begin{screen}
		\begin{metadfn}[文]
			自由な変項が現れない$\mathcal{L}$の式を{\bf 文}\index{ぶん@文}{\bf (sentence)}
			や{\bf 閉式}\index{へいしき@閉式}{\bf (closed formula)}と呼ぶ.
		\end{metadfn}
	\end{screen}
	
\subsection{代入}
	変項とは束縛されうる項であったが,別の項を代入されうる項でもある.
	代入とは別の項で置き換えるということであり,また代入されうるのは式の中で自由な変項のみである.
	ただし,代入には「{\bf 式の中の自由な変項を別の変項に取り替えても式の意味を変えてはならない}」という
	大前提がある.たとえば
	\begin{align}
		\forall u\, (\, u \in x\, )
	\end{align}
	という式で考察すると,この式で$x$は自由であるから別の項を代入して良いのであり,$z$を代入すれば
	\begin{align}
		\forall u\, (\, u \in z\, )
	\end{align}
	となる.そしてこの場合はどちらの式も意味は同じである.意味が同じであるとは
	量化してみれば一目瞭然であって,両式を全称記号で量化すれば
	\begin{align}
		&\forall x\, \forall u\, (\, u \in x\, ), \\
		&\forall z\, \forall u\, (\, u \in z\, )
	\end{align}
	はどちらも「どの集合も,全ての集合を要素に持つ」と解釈され,
	両式を存在記号で量化すれば
	\begin{align}
		&\exists x\, \forall u\, (\, u \in x\, ), \\
		&\exists z\, \forall u\, (\, u \in z\, )
	\end{align}
	はどちらも「或る集合は,全ての集合を要素に持つ」と解釈される.
	ところが$x$に$u$を代入すると
	\begin{align}
		\forall u\, (\, u \in u\, )
	\end{align}
	となり,これは「全ての集合は自分自身を要素に持つ」という意味に変わる.
	つまり先の大前提に立てば,代入する際には{\bf 代入後に束縛されてしまう変項は使ってはいけない}のである.
	
	代入するのは変項だけではない.$\varepsilon$項や内包項だって上の$x$に代入して良い.
	ただし上と同様の注意が必要で,$\varepsilon$項や内包項に$u$が自由に現れている場合と
	そうでない場合では代入後の式の意味が分かれてしまうので,
	代入して良い項は$u$が自由に現れていないものに限る.
	
	以上の考察を一般的な代入規則に敷衍して言えば,
	
	\begin{itembox}[l]{代入可能な項}
		$\varphi$を$\mathcal{L}$の式とし,$x$を$\varphi$に自由に現れる変項とし,
		$\tau$を$\mathcal{L}$の項とする.このとき「$\varphi$に自由に現れる$x$に$\tau$を
		代入する」とは,特筆が無い限り$\varphi$に自由に現れる全ての$x$に
		$\tau$を代入することであって,その際に$\tau$が満たすべき条件は
		\begin{itemize}
			\item $\tau$が変項ならば$\tau$は$\varphi$に代入されたどの箇所でも自由である
			\item $\tau$が$\varepsilon$項や内包項である場合は,
				$\tau$の中に自由に現れる変項があったとしても,
				それらは全て$\tau$が代入されたどの箇所でも束縛されない
		\end{itemize}
		とする.$\tau$がこの条件を満たすとき,
		{\bf 「$\tau$は$\varphi$の中で$x$への代入について自由である」}という.
	\end{itembox}
	
	$\varphi$に自由に現れる$x$に$\tau$を代入した後の式を
	\begin{align}
		\varphi(x/\tau)
	\end{align}
	と書く($x/\tau$は``replace $x$ by $\tau$''の順).
	特に$\varphi$の中に自由に現れている変項が$x$だけである場合は,$\varphi(x/\tau)$を
	\begin{align}
		\varphi(\tau)
	\end{align}
	とも書く.$\tau$が$x$自身である場合は$\varphi(x)$は$\varphi$そのものであるが,
	「$\varphi$に自由に現れているのは$x$だけである」ということを強調するために
	\begin{align}
		\varphi(x)
	\end{align}
	と書くことも多い.$\varphi$に別の変項$y$が現れていて,$y$に項$\sigma$を代入するときは,
	\begin{align}
		\varphi(x/\tau)(y/\sigma)
	\end{align}
	を
	\begin{align}
		\varphi(x/\tau,y/\sigma)
	\end{align}
	とも書く.特に$\varphi$の中に自由に現れている変項が$x$と$y$だけである場合は,
	$\tau$と$\sigma$の代入先がはっきりしていれば
	\begin{align}
		\varphi(\tau,\sigma)
	\end{align}
	とも書くし,「$\varphi$に自由に現れているのは$x$と$y$だけである」ということを強調するために
	\begin{align}
		\varphi(x,y)
	\end{align}
	と書くことも多い.$\varphi$に$x$が自由に現れていない場合でも$\varphi(x/\tau)$などと書かれていたら,
	その式は$\varphi$のことであると理解する.
	
\subsection{類}
	\begin{comment}
	\begin{screen}
		\begin{dfn}[閉項]
			どの変項も自由に現れない$\varepsilon$項を
			{\bf 閉${\boldsymbol \varepsilon}$項}\index{
			へいイプシロンこう@閉$\varepsilon$項}{\bf (closed epsilon term)}と呼び,
			どの変項も自由に現れない内包項を{\bf 閉内包項}\index{
			へいないほうこう@閉内包項}{\bf (closed comprehension term)}と呼ぶ.
			また閉$\varepsilon$項と閉内包項は以上のみである.
		\end{dfn}
	\end{screen}
	\end{comment}
	
	元々の意図としては,例えば$x$のみが自由に現れる式$\varphi(x)$に対して
	``$\varphi(x)$を満たすいずれかの集合$x$''という意味を込めて
	\begin{align}
		\varepsilon x \varphi(x)
	\end{align}
	を作ったのだし,``$\varphi(x)$を満たす集合$x$の全体''という意味を込めて
	\begin{align}
		\Set{x}{\varphi(x)}
	\end{align}
	を作ったのである.つまりこの場合の$\varepsilon x \varphi(x)$と
	$\Set{x}{\varphi(x)}$は``意味を持っている''わけである.
	これが,もし$x$とは別の変項$y$が$\varphi$に自由に現れているとすれば,
	$\varepsilon x \varphi$も$\Set{x}{\varphi}$も$y$に依存してしまい
	意味が定まらなくなる.というのも,変項とは代入可能な項であるから,$y$に代入する項ごとに
	$\varepsilon x \varphi$と$\Set{x}{\varphi}$は別の意味を持ち得るのである.
	また項が閉じていても意味不明な場合がある.たとえば,$\psi$が文であるときに
	\begin{align}
		\varepsilon y \psi
	\end{align}
	や
	\begin{align}
		\Set{y}{\psi}
	\end{align}
	なる項は閉じてはいるが,導入の意図には適っていない.意味不明ながらこういった項が存在しているのは
	導入時にこれらを排除する面倒を避けたからであり,また一旦すべてを作り終えた後で余計なものを捨てる方が
	楽だからである.
	
	とりあえず,導入の意図に適っている項は特別の名前を持っていて然るべきである.
	
	\begin{screen}
		\begin{dfn}[類]
			$\varphi$を$\lang{\varepsilon}$の式とし,$x$を$\varphi$に自由に現れる変項とし,
			$\varphi$に自由に現れる項は$x$のみであるとするとき,$\varepsilon x \varphi$
			と$\Set{x}{\varphi}$を{\bf 類}\index{るい@類}{\bf (class)}と呼ぶ.
			またこれらのみが類である.
		\end{dfn}
	\end{screen}
	
	類には二種類あるので,それらも名前を分けておく.
	\begin{screen}
		\begin{dfn}[主要$\varepsilon$項]
			類である$\varepsilon$項を{\bf 主要${\boldsymbol \varepsilon}$項}
			\index{しゅよういぷしんろんこう@主要$\varepsilon$項}
			{\bf (critical epsilon term)}と呼ぶ.
		\end{dfn}
	\end{screen}
	
	\begin{screen}
		\begin{dfn}[主要内包項]
			類である内包項を{\bf 主要内包項}\index{しゅようないほうこう@主要内包項}と呼ぶ.
		\end{dfn}
	\end{screen}
	
	内包項に関しては便宜上自由な変項の出現も許すことにするが,
	たとえば$\Set{x}{\varphi}$と書いたら少なくとも$x$は$\varphi$に自由に現れているべきであり,
	この意味で性質の良い内包項に対しても特別な名前を付けておく.
	
	\begin{screen}
		\begin{dfn}[正則内包項]
			$\varphi$を$\lang{\varepsilon}$の式とし,$x$を変項とし,
			$\varphi$に$x$が自由に現れているとするとき,
			$\Set{x}{\varphi}$を{\bf 正則内包項}\index{せいそくないほうこう@正則内包項}と呼ぶ.
		\end{dfn}
	\end{screen}
	
\subsection{扱う式の制限}
\label{sec:restriction_of_formulas}
	{\bf 以降では扱う式は,特筆が無い限りそこに現れる$\varepsilon$項は全て
	主要$\varepsilon$項であり,現れる内包項は全て正則内包項であるとする.}

\subsection{式の書き換え}
	$\varepsilon$項を取り入れた目的は{\bf 存在文}\index{そんざいぶん@存在文}
	{\bf (existential sentence)}に対して{\bf 証人}\index{しょうにん@証人}{\bf (witness)}
	を与えることであり,それは
	\begin{align}
		\exists x \varphi(x) \rarrow \varphi(\varepsilon x \varphi(x))
	\end{align}
	なる式を公理とすることで実質的に裏付けされる.
	ただし$\varepsilon$項を作れる式は$\lang{\varepsilon}$の式のみであるから,
	$\varphi$が内包項を含んだ式であると$\varepsilon x \varphi(x)$を使うことが出来ない.
	とはいえ$\mathcal{L}$の式の存在文も往々にして登場するので
	それらに対しても証人を用意できると便利である.
	そこで$\mathcal{L}$の式を``同値''な$\lang{\varepsilon}$の式に書き換えて,
	その書き換えた式で作る$\varepsilon$項を使うことにする.つまり
	$\varphi$が$\mathcal{L}$の式である場合は,$\varphi$を
	``同値''な$\lang{\varepsilon}$の式$\hat{\varphi}$に書き換えてから
	\begin{align}
		\exists x \varphi(x) \rarrow \varphi(\varepsilon x \hat{\varphi}(x))
	\end{align}
	を保証するのである.書き換える必要があるのは内包項を含んでいる式のみであり,
	また原子式だけを書き換えれば十分である.
	書き換えが``同値''というのは後述の\ref{sec:equivalence_of_formula_rewriting}節
	で述べてあるような意味であるが,それは直感的に妥当なものである.原子式の書き換えは次の要領で行う:
	
	\begin{table}[H]
		\begin{center}
		\begin{tabular}{c|c|c}
			元の式 & 書き換え後 & 付記 \\ \hline \hline
			$a = \Set{z}{\psi}$ & $\forall v\, (\, v \in a \lrarrow \psi(z/v)\, )$ & \\ \hline
			$\Set{y}{\varphi} = b$ & $\forall u\, (\, \varphi(y/u) \lrarrow u \in b\, )$ & \\ \hline
			$\Set{y}{\varphi} = \Set{z}{\psi}$ & $\forall u\, (\, \varphi(y/u) \lrarrow \psi(z/u)\, )$ & \\ \hline
			$a \in \Set{z}{\psi}$ & $\psi(z/a)$ & 必要なら束縛変項の名前替えをする\footnotemark \\ \hline
			$\Set{y}{\varphi} \in b$ & $\exists s\, (\, \forall u\, (\, \varphi(y/u) \lrarrow u \in s\, ) \wedge s \in b\, )$ & \\ \hline
			$\Set{y}{\varphi} \in \Set{z}{\psi}$ & $\exists s\, (\, \forall u\, (\, \varphi(y/u) \lrarrow u \in s\, ) \wedge \psi(z/s)\, )$ & \\ \hline
		\end{tabular}
		\end{center}
	\end{table}
	
	ただし上の記号に課している条件は
	\begin{itemize}
		\item $a,b$は$\lang{\varepsilon}$の項である
			(\ref{sec:restriction_of_formulas}節より
			$a,b$は変項か主要$\varepsilon$項).
		\item $\Set{y}{\varphi}$と$\Set{z}{\psi}$を正則内包項である.
		\item $u$は$\varphi$の中で$y$への代入について自由であり,
			$u,v,s$は$\psi$の中で$z$への代入について自由である.
			上の式の書き換えにおいては変項$u,v,s$を追加したが,
			代入について自由である限りどの変項を選んでも構わない.
			従って式の書き換えは一つに決まらないということになるが,
			違う変項を選んでも式の意味は変わらない.
	\end{itemize}
	
	\footnotetext{
			$a$を$\psi$の中の自由な$z$に代入した後で$a$が束縛される場合,
			束縛変項の名前替えをしなくてはならない.たとえば
			\begin{align}
				a \in \Set{z}{\forall a\, (\, z \in a\, )}
			\end{align}
			という式では左辺の$a$は自由であるのに,書き換えの規則をそのまま適用すると
			\begin{align}
				\forall a\, (\, a \in a\, )
			\end{align}
			となり束縛されてしまう.代入後の$a$が束縛されないためには
			\begin{align}
				a \in \Set{z}{\forall b\, (\, z \in b\, )}
			\end{align}
			のように束縛変項$a$を別の変項$b$に替えて
			\begin{align}
				\forall b\, (\, a \in b\, )
			\end{align}
			とすればよい.
	}
	
	原子式に対する書き換えが掲示されたので,$\mathcal{L}$の一般の式$\varphi$から
	$\lang{\varepsilon}$の式$\hat{\varphi}$を得る操作は次の帰納的な要領で行えばよい.
	\begin{description}
		\item[case1] $\varphi$が
			\begin{align}
				\negation \psi
			\end{align}
			なる式であるとき,$\psi$を$\lang{\varepsilon}$の式に書き換えた$\hat{\psi}$を用いて
			\begin{align}
				\negation \hat{\psi}
			\end{align}
			を$\hat{\varphi}$とする.
			
		\item[case2] $\varphi$が
			\begin{align}
				\vee \psi \chi
			\end{align}
			なる式であるとき,$\psi,\chi$を$\lang{\varepsilon}$の式に書き換えた
			$\hat{\psi},\hat{\chi}$を用いて
			\begin{align}
				\vee \hat{\psi} \hat{\chi}
			\end{align}
			を$\hat{\varphi}$とする.$\varphi$が$\wedge \psi \chi$や$\rarrow \psi \chi$
			の形の時も同様にする.
			
		\item[case3] $\varphi$が
			\begin{align}
				\exists x \psi
			\end{align}
			なる式であるとき,$\psi$を$\lang{\varepsilon}$の式に書き換えた
			$\hat{\psi}$を用いて
			\begin{align}
				\exists x \hat{\psi}
			\end{align}
			を$\hat{\varphi}$とする.$\varphi$が$\forall x \psi$の形の時も同様にする.
	\end{description}
	
	もちろん$\varphi$が$\lang{\varepsilon}$の式ならわざわざ書き換える必要は無い.
	
	\begin{screen}
		\begin{metathm}[書き換え後も自由な変項は増減しない]
			$\varphi$を$\mathcal{L}$の式とし,これを$\lang{\varepsilon}$の式に
			書き換えたものを$\hat{\varphi}$とする.このとき
			$\varphi$に自由に現れる変項と$\hat{\varphi}$に自由に現れる変項は一致する.
		\end{metathm}
	\end{screen}
	
	\begin{metaprf}\mbox{}
		\begin{description}
			\item[step1] $\varphi$が原子式であるときは上の書き換え表より一目瞭然である.
			
			\item[step2]
				$\varphi$が一般の式であるとき
				\begin{itembox}[l]{IH (帰納法の仮定)}
					$\varphi$の任意の真部分式$\psi$と,それを$\lang{\varepsilon}$の式
					に書き換えた$\hat{\psi}$は,自由に現れる変項が一致する
				\end{itembox}
				と仮定する.すると
				\begin{description}
					\item[case1] $\varphi$が
						\begin{align}
							\negation \psi
						\end{align}
						なる式の場合,$\varphi$に自由に現れる変項は
						$\psi$に自由に現れる変項と一致するが,それは
						$\hat{\psi}$に自由に現れる変項と一致するので,
						$\negation \hat{\psi}$に自由に現れる変項とも一致する.
						
					\item[case2] $\varphi$が
						\begin{align}
							\vee \psi \chi
						\end{align}
						なる式の場合,$\varphi$に自由に現れる変項は$\psi,\chi$に自由に現れる
						変項と一致するが,それは$\hat{\psi},\hat{\chi}$に
						自由に現れる変項と一致するので,
						$\negation \hat{\psi} \hat{\chi}$に自由に現れる変項とも一致する.
					
					\item $\varphi$が
						\begin{align}
							\exists x \psi
						\end{align}
						なる式の場合,$\varphi$に自由に現れる変項は
						$\psi$に自由に現れる$x$以外の変項と一致するが,それは
						$\hat{\psi}$に自由に現れる$x$以外の変項変項と一致するので,
						$\negation \hat{\psi}$に自由に現れる変項とも一致する.
						\QED
				\end{description}
		\end{description}
	\end{metaprf}
	
\subsection{中置記法}
	たとえば$\in s t$なる原子式は「$s$は$t$の要素である($s$ is in $t$)」と読むのだから,語順通りに,
	或いは$s$が$t$の中にあるというイメージ通りに
	\begin{align}
		s \in t
	\end{align}
	と書きかえる方が見やすくなる.同じように,$\vee \varphi \psi$なる式も
	「$\varphi$または$\psi$」と読むのだから
	\begin{align}
		\varphi \vee \psi
	\end{align}
	と書きかえる方が見やすくなる.$\rarrow \vee \varphi \psi \wedge \chi \xi$のように長い式も,
	上の作法に倣えば
	\begin{align}
		\begin{gathered}
			\rarrow \vee \varphi \psi \wedge \chi \xi \\
			\rarrow \color{red}{\varphi \vee \psi} \color{blue}{\chi \wedge \xi} \\
			\color{red}{\varphi \vee \psi} \color{black}{\rarrow} \color{blue}{\chi \wedge \xi}
		\end{gathered}
	\end{align}
	と書きかえることになるが,一々色分けするわけにもいかないので``(''と``)''を使って
	\begin{align}
		(\varphi \vee \psi) \rarrow (\chi \wedge \xi)
	\end{align}
	と書くようにすれば良い.
	
	\begin{itembox}[l]{{\bf 中置記法}\index{ちゅうちきほう@中置記法}{\bf (infix notation)}}
			$\mathcal{L}$の式は以下の手順で中置記法に書き換える.
			\begin{enumerate}
				\item $\in s t$なる形の原子式は$s \in t$と書きかえる.
					$= s t$も同様に書き換える.
					
				\item $\negation \varphi$なる形の式はそのままにする.
				
				\item $\vee \varphi \psi$なる形の式は$(\varphi \vee \psi)$と書きかえる.
					$\wedge \varphi \psi$と$\rarrow \varphi \psi$の形の式も同様に書き換える.
				
				\item $\exists x \varphi$なる形の式はそのままにする.
					$\forall x \varphi$なる形の式も同様にする.
			\end{enumerate}
	\end{itembox}
	
	上の書き換え法では,たとえば$\rarrow \vee \varphi \psi \wedge \chi \xi$なる式は
	\begin{align}
		((\varphi \vee \psi) \rarrow (\chi \wedge \xi))
	\end{align}
	となるが,括弧はあくまで式の境界の印として使うものであるから,一番外側の括弧は外して
	\begin{align}
		(\varphi \vee \psi) \rarrow (\chi \wedge \xi)
	\end{align}
	と書く方が良い.よって{\bf 中置記法に書き換え終わったときに一番外側にある括弧は外す}ことにする.
	
	$\wedge \vee \exists x \varphi \psi \negation \rarrow \chi \in s t$なる式は
	\begin{align}
		\begin{gathered}
			\wedge \vee \exists x \varphi \psi \negation \rarrow \chi s \in t \\
			\wedge (\exists x \varphi \vee \psi) \negation (\chi \rarrow s \in t) \\
			(\exists x \varphi \vee \psi) \wedge \negation (\chi \rarrow s \in t)
		\end{gathered}
	\end{align}
	となる.
	
	ただしあまり括弧が連なると読みづらくなるので,
	\begin{align}
		(\varphi \vee \psi) \rarrow \chi
	\end{align}
	なる形の式は
	\begin{align}
		\varphi \vee \psi \rarrow \chi
	\end{align}
	に,同様に
	\begin{align}
		\varphi \rarrow (\psi \vee \chi)
	\end{align}
	なる形の式は
	\begin{align}
		\varphi \rarrow \psi \vee \chi
	\end{align}
	とも書く.また$\vee$が$\wedge$であっても同じように括弧を省く.

\chapter{推論}
		第\ref{sec:restriction_of_formulas}節で決めた通り,
	扱う式は全て,そこに現れる$\varepsilon$項は全て主要$\varepsilon$項であり,
	現れる内包項は全て正則内包項であるとする.
	
\section{証明}
	本節では{\bf 推論規則}\index{すいろんきそく@推論規則}{\bf (rule of inference)}を導入し,
	基本的な{\bf 推論法則}\index{すいろんほうそく@推論法則}を導出する.
	推論法則とは,他の本ではそれが用いている証明体系の定理などと呼ばれるが,
	本稿では集合論特有の定理と区別するために推論法則と呼ぶ.
	以下では
	\begin{align}
		\vdash
	\end{align}
	なる記号を用いて,
	\begin{align}
		\varphi \vdash \psi
	\end{align}
	などと書く.$\vdash$の左右にあるのは必ず($\mathcal{L}$の)文であって,
	右側に置かれる文は必ず一本だけであるが,左側には文がいくつあっても良いし,全く無くても良い.
	特に
	\begin{align}
		\vdash \psi
	\end{align}
	を満たす文$\psi$を推論法則と呼ぶ.
	{\bf ``$\vdash$の右の文は,$\vdash$の左の文から証明できる''},と読むが,
	証明とはどのようにされるのだとか,$\vdash \psi$を満たすとは
	どういう意味なのか,とかいったことは後に回して,とりあえず記号のパズルゲームと見立てて
	$\vdash$のルールを定める.
	
	\begin{screen}
		\begin{logicalrule}[演繹規則]\label{logicalrule:deduction_rule}
			$A,B,C,D$を文とするとき,
			\begin{description}
				\item[(a)] $A \vdash D$ならば$\vdash A \rarrow D$が成り立つ.
				\item[(b)] $A,B \vdash D$ならば
					\begin{align}
						B \vdash A \rarrow D,\quad
						A \vdash B \rarrow D
					\end{align}
					が成り立つ.
				\item[(c)] $A,B,C \vdash D$ならば
					\begin{align}
						B,C \vdash A \rarrow D,\quad
						A,C \vdash B \rarrow D,\quad
						A,B \vdash C \rarrow D
					\end{align}
					のいずれも成り立つ.
			\end{description}
		\end{logicalrule}
	\end{screen}
	
	演繹規則においては$\vdash$の左側にせいぜい三つの文しかないのだが,
	実は$\vdash$の左側に不特定多数の文を持ってきても
	演繹規則じみたことが成立する(後述の演繹法則).
	
	では{\bf 証明}\index{しょうめい@証明}{\bf (proof)}とは何かを規定する.
	
	自由な変項が現れない($\mathcal{L}$の)式を{\bf 文}\index{ぶん@文}{\bf (sentence)}や
	{\bf 閉式}\index{へいしき@閉式}{\bf (closed formula)}と呼ぶ.
	証明される式や証明の過程で出てくる式は全て文である.本稿では証明された文を
	{\bf 真な}\index{しん@真}{\bf (true)}文と呼ぶことにするが,
	``証明された''や``真である''という状態は議論が立脚している前提に依存する.
	ここでいう前提とは,推論規則や言語ではなくて
	{\bf 公理系}\index{こうりけい@公理系}{\bf (axioms)}と呼ばれるものを指している.
	公理系とは文の集まりである.$\mathscr{S}$を公理系とするとき,
	$\mathscr{S}$に集められた文を$\mathscr{S}$の{\bf 公理}\index{こうり@公理}{\bf (axiom)}
	と呼ぶ.以下では本稿の集合論が立脚する公理系を$\Sigma$と書くが,
	$\Sigma$に属する文は単に公理と呼んだりもする.
	
	$\Sigma$とは以下の文からなる:
	\begin{description}
		\item[相等性] $a,b,c$を類とするとき
			\begin{align}
				&a = b \rarrow (\, a \in c \rarrow b \in c\, ), \\
				&a = b \rarrow (\, c \in a \rarrow c \in b\, ).
			\end{align}
			
		\item[外延性] $a$と$b$を類とするとき
			\begin{align}
				\forall x\, (\, x \in a \lrarrow x \in b\, ) \rarrow a = b.
			\end{align}
			
		\item[内包性] $\varphi$を$\mathcal{L}$の式とし,$y$を$\varphi$に自由に現れる変項とし,
			$\varphi$に自由に現れる項は$y$のみであるとし,
			$x$は$\varphi$で$y$への代入について自由であるとするとき,
			\begin{align}
				\forall x\, \left(\, x \in \Set{y}{\varphi(y)} \lrarrow \varphi(x)\, \right).
			\end{align}
		
		\item[要素] $a,b$を類とするとき
			\begin{align}
				a \in b \rarrow \exists x\, (\, x = a\, ).
			\end{align}
			
		\item[対] $\forall x\, \forall y\, \exists p\, \forall z\, 
			(\, x = z \vee y = z \lrarrow z \in p\, ).$
			
		\item[合併] $\forall x\, \exists u\, \forall y\, (\, \exists z\, (\, z \in x \wedge y \in z\, ) \lrarrow y \in u\, ).$
			
		\item[冪] $\forall x\, \exists p\, \forall y\, 
			(\, \forall z\, (\, z \in y \rarrow z \in x\, ) \lrarrow y \in p\, ).$
			
		\item[置換] $\varphi$を$\mathcal{L}$の式とし,
			$s,t$を$\varphi$に自由に現れる変項とし,
			$\varphi$に自由に現れる項は$s,t$のみであるとし,
			$x$は$\varphi$で$s$への代入について自由であり,
			$y,z$は$\varphi$で$t$への代入について自由であるとするとき,
			\begin{align}
				\forall x\, \forall y\, \forall z\, 
				(\, \varphi(x,y) \wedge \varphi(x,z)
				\rarrow y = z\, )
				\rarrow \forall a\, \exists z\, \forall y\,
				(\, y \in z \lrarrow \exists x\, (\, x \in a \wedge 
				\varphi(x,y)\, )\, ).
			\end{align}
			
		\item[正則性] $a$を類とするとき,
			\begin{align}
				\exists x\, (\, x \in a\, ) \rarrow
				\exists y\, (\, y \in a \wedge \forall z\, (\, z \in y \rarrow
				z \notin a\, )\, ).
			\end{align}
			
		\item[無限] $\exists x\, (\, 
				\exists s\, (\, \forall t\, (\, t \notin s\, ) \wedge s \in x\, ) 
				\wedge \forall y\, (\, 
				y \in x \rarrow \exists u\, (\, 
				\forall v\, (\, v \in u \lrarrow v \in y \vee v = y\, )
				\wedge u \in x\, )\, )\, ).$
			
		\item[選択]
			
	\end{description}
	
	\begin{screen}
		\begin{metadfn}[証明可能]
			文$\varphi$が公理系$\mathscr{S}$から
			{\bf 証明された}だとか{\bf 証明可能である}\index{しょうめいかのう@証明可能}
			{\bf (provable)}ということは,
			\begin{itemize}
				\item $\varphi$は$\mathscr{S}$の公理である.
				\item $\vdash \varphi$である.
				\item 文$\psi$で,$\psi$と$\psi \rightarrow \varphi$が$\mathscr{S}$から
				証明されているものが取れる({\bf 三段論法}\index{さんだんろんぽう@三段論法}
				{\bf (Modus Pones)}).
			\end{itemize}
			のいずれかが満たされているということである.
		\end{metadfn}
	\end{screen}		
	
	$\varphi$が$\mathscr{S}$から証明可能であることを
	\begin{align}
		\mathscr{S} \vdash \varphi
	\end{align}
	と書く.ただし,{\bf 公理系に変項が生じた場合の証明可能性には
	演繹規則や後述の演繹法則,およびその逆の結果を適用することが出来る}.
	
	たとえばどんな文$\varphi$に対しても
	\begin{align}
		\varphi \vdash \varphi
	\end{align}
	となるし,どんな文$\psi$を追加しても
	\begin{align}
		\varphi,\psi \vdash \varphi
	\end{align}
	となる.これらは最も単純なケースであり,大抵の定理は数多くの複雑なステップを踏まなくては得られない.
	$\mathscr{S}$から証明済みの$\varphi$を起点にして$\mathscr{S} \vdash \psi$であると判明すれば,
	$\varphi$から始めて$\psi$が真であることに辿り着くまでの一連の作業は$\psi$の$\mathscr{S}$からの
	{\bf 証明}\index{しょうめい@証明}{\bf (proof)}と呼ばれ,
	$\psi$は$\mathscr{S}$の{\bf 定理}\index{ていり@定理}{\bf (theorem)}と呼ばれる.
	
	$A,B \vdash \varphi$とは
	$A$と$B$の二つの文のみを公理とした体系において$\varphi$が証明可能であることを表している.
	特に{\bf 推論法則とは公理の無い体系で推論規則だけから導かれる定理}のことである.
	
	ではさっそく演繹法則の証明に進む.ところで,後で見るとおり演繹法則とは
	証明が持つ性質に対する言明であって,つまりメタ視点での定理ということになるので,
	演繹法則の``証明''とは言っても上で規定した証明とは意味が違う.
	メタ定理の``証明''は,本稿では{\bf メタ証明}と呼んで区別する.
	演繹法則を示す前に推論法則を三本用意しなくてはならない.
	
	\begin{screen}
		\begin{logicalthm}[含意の反射律]\label{logicalthm:reflective_law_of_implication}
			$A$を文とするとき
			\begin{align}
				\vdash A \rarrow A.
			\end{align}
		\end{logicalthm}
	\end{screen}
	
	上の言明は``どんな文でも持ってくれば,その式に対して反射律が成立する''という意味である.
	このように無数に存在し得る定理を一括して表す式は{\bf 公理図式}\index{こうりずしき@公理図式}{\bf (schema)}と呼ばれる.
	
	\begin{prf}
		$A \vdash A$であるから,演繹規則より$\vdash A \rarrow A$となる.
		\QED
	\end{prf}
	
	\begin{screen}
		\begin{logicalthm}[含意の導入]
		\label{logicalthm:introduction_of_implication}
			$A,B$を文とするとき
			\begin{align}
				\vdash B \rarrow (\, A \rarrow B\, ).
			\end{align}
		\end{logicalthm}
	\end{screen}
	
	\begin{prf}
		\begin{align}
			A,B \vdash B
		\end{align}
		より演繹規則から
		\begin{align}
			B \vdash A \rarrow B
		\end{align}
		となり,再び演繹規則より
		\begin{align}
			\vdash B \rarrow (\, A \rarrow B\, )
		\end{align}
		が得られる.
		\QED
	\end{prf}
	
	演繹法則を示すための推論法則の導出は次で最後である.
	
	\begin{screen}
		\begin{logicalthm}[含意の分配則]
		\label{logicalthm:distributive_law_of_implication}
			$A,B,C$を文とするとき
			\begin{align}
				\vdash (\, A \rarrow (\, B \rarrow C\, )\, ) 
				\rarrow (\, (\, A \rarrow B\, ) \rarrow (\, A \rarrow C\, )\, ).
			\end{align}
		\end{logicalthm}
	\end{screen}
	
	\begin{prf}
		証明可能性の規則より
		\begin{align}
			A \rarrow (\, B \rarrow C\, ),\ A \rarrow B,\ A
			&\vdash A, \\
			A \rarrow (\, B \rarrow C\, ),\ A \rarrow B,\ A
			&\vdash A \rarrow B
		\end{align}
		となるので
		\begin{align}
			A \rarrow (\, B \rarrow C\, ),\ A \rarrow B,\ A
			\vdash B
		\end{align}
		が成り立つし,同じように
		\begin{align}
			A \rarrow (\, B \rarrow C\, ),\ A \rarrow B,\ A
			&\vdash A, \\
			A \rarrow (\, B \rarrow C\, ),\ A \rarrow B,\ A
			&\vdash A \rarrow (\, B \rarrow C\, )
		\end{align}
		であるから
		\begin{align}
			A \rarrow (\, B \rarrow C\, ),\ A \rarrow B,\ A
			\vdash B \rarrow C
		\end{align}
		も成り立つ.これによって
		\begin{align}
			A \rarrow (\, B \rarrow C\, ),\ A \rarrow B,\ A \vdash C
		\end{align}
		も成り立つから,あとは演繹規則を順次適用すれば
		\begin{align}
			A \rarrow (\, B \rarrow C\, ),\ A \rarrow B
			&\vdash A \rarrow C, \\
			A \rarrow (\, B \rarrow C\, )
			&\vdash (\, A \rarrow B\, ) \rarrow (\, A \rarrow C\, ), \\
			&\vdash (\, A \rarrow (\, B \rarrow C\, )\, ) 
				\rarrow (\, (\, A \rarrow B\, ) \rarrow (\, A \rarrow C\, )\, )
		\end{align}
		となる.
		\QED
	\end{prf}
	
	\begin{screen}
		\begin{metaaxm}[証明に対する構造的帰納法]
			$\mathscr{S}$を公理系とし,Xを文に対する何らかの言明とするとき,
			\begin{itemize}
				\item $\mathscr{S}$の公理に対してXが言える.
				\item 推論法則に対してXが言える.
				\item $\varphi$と$\varphi \rarrow \psi$が$\mathscr{S}$の
					定理であるような文$\varphi$と文$\psi$が取れたとき,
					$\varphi$と$\varphi \rarrow \psi$に対して
					Xが言えるならば,$\psi$に対してXが言える.
			\end{itemize}
			のすべてが満たされていれば,$\mathscr{S}$から証明可能なあらゆる文に対してXが言える.
		\end{metaaxm}
	\end{screen}
	
	公理系$\mathscr{S}$に文$A$を追加した公理系を
	\begin{align}
		A,\ \mathscr{S}
	\end{align}
	や
	\begin{align}
		\mathscr{S},\ A
	\end{align}
	と書く.$A$が既に$\mathscr{S}$の公理であってもこのように表記するが,
	その場合は$\mathscr{S}, A$や$A,\mathscr{S}$とは$\mathscr{S}$そのものである.
	
	\begin{screen}
		\begin{metathm}[演繹法則]\label{metathm:deduction_theorem}
			$\mathscr{S}$を公理系とし,$A$を文とするとき,
			$\mathscr{S}, A$の任意の定理$B$に対して
			\begin{align}
				\mathscr{S} \vdash A \rarrow B
			\end{align}
			が成り立つ.
		\end{metathm}
	\end{screen}
	
	\begin{metaprf}\mbox{}
		\begin{description}
			\item[第一段]
				$B$を$\mathscr{S},A$の公理か或いは推論法則とする.
				$B$が$A$ならば含意の反射律
				(推論法則\ref{logicalthm:reflective_law_of_implication})より
				\begin{align}
					\vdash A \rarrow B
				\end{align}
				が成り立つので
				\begin{align}
					\mathscr{S} \vdash A \rarrow B
				\end{align}
				となる.$B$が$\mathscr{S}$の公理又は推論法則であるとき,まず
				\begin{align}
					\mathscr{S} \vdash B
				\end{align}
				が成り立つが,他方で含意の導入
				(推論法則\ref{logicalthm:introduction_of_implication})より
				\begin{align}
					\mathscr{S} \vdash B \rarrow (\, A \rarrow B\, ) 
				\end{align}
				も成り立つので,証明可能性の定義より
				\begin{align}
					\mathscr{S} \vdash A \rarrow B
				\end{align}
				が従う.
				
			\item[第二段]
				$C$及び$C \rarrow B$が$\mathscr{S}$の定理であるような
				文$C$と文$B$が取れた場合,
				\begin{align}
					\mathscr{S} \vdash A \rarrow (\, C \rarrow B\, )
				\end{align}
				かつ
				\begin{align}
					\mathscr{S} \vdash A \rarrow C
				\end{align}
				であると仮定する.含意の分配則
				(\ref{logicalthm:distributive_law_of_implication})より
				\begin{align}
					\mathscr{S} \vdash 
					(\, A \rarrow (\, C \rarrow B\, )\, ) 
					\rarrow (\, (\, A \rarrow C\, ) \rarrow (\, A \rarrow B\, )\, )
				\end{align}
				が満たされるので,証明可能性の定義の通りに
				\begin{align}
					\mathscr{S} \vdash (\, A \rarrow C\, ) 
					\rarrow (\, A \rarrow B\, )
				\end{align}
				が従い,
				\begin{align}
					\mathscr{S} \vdash A \rarrow B
				\end{align}
				が従う.以上と構造的帰納法より,$\mathscr{S},A$の任意の定理$B$に対して
				\begin{align}
					\mathscr{S} \vdash A \rarrow B
				\end{align}
				が言える.
				\QED
		\end{description}
	\end{metaprf}
	
	演繹法則の逆も得られる.つまり,
	$\mathscr{S}$を公理系とし,$A$と$B$を文とするとき,
	\begin{align}
		\mathscr{S} \vdash A \rarrow B
	\end{align}
	であれば
	\begin{align}
		A,\ \mathscr{S} \vdash B
	\end{align}
	が成り立つ.実際
	\begin{align}
		A,\ \mathscr{S} \vdash A
	\end{align}
	が成り立つのは証明の定義の通りであるし,
	$A \rarrow B$が$\mathscr{S}$の定理ならば
	\begin{align}
		A,\ \mathscr{S} \vdash A \rarrow B
		\label{fom:inversion_of_deduction_theorem}
	\end{align}
	が成り立つので,併せて
	\begin{align}
		A,\ \mathscr{S} \vdash B
	\end{align}
	が従う.ただし(\refeq{fom:inversion_of_deduction_theorem})に
	関しては次のメタ定理を示さなくてはいけない.
	
	\begin{screen}
		\begin{metathm}[公理が増えても証明可能]
			$\mathscr{S}$を公理系とし,$A$を文とするとき,
			$\mathscr{S}$の任意の定理$B$に対して
			\begin{align}
				A,\ \mathscr{S} \vdash B
			\end{align}
			が成り立つ.
		\end{metathm}
	\end{screen}
	
	\begin{metaprf}
		$B$が$\mathscr{S}$の公理であるか推論規則であれば
		\begin{align}
			A,\ \mathscr{S} \vdash B
		\end{align}
		は言える.また
		\begin{align}
			\mathscr{S} &\vdash C, \\
			\mathscr{S} &\vdash C \rarrow B
		\end{align}
		を満たす文$C$が取れるとき,
		\begin{align}
			A,\ \mathscr{S} &\vdash C, \\
			A,\ \mathscr{S} &\vdash C \rarrow B
		\end{align}
		と仮定すれば
		\begin{align}
			A,\ \mathscr{S} \vdash B
		\end{align}
		となる.以上と構造的帰納法より$\mathscr{S}$の任意の定理$B$に対して
		\begin{align}
			A,\ \mathscr{S} \vdash B
		\end{align}
		が成り立つ.
		\QED
	\end{metaprf}
	
	\begin{screen}
		\begin{metathm}[演繹法則の逆]
		\label{metathm:inverse_of_deduction_theorem}
			$\mathscr{S}$を公理系とし,$A$と$B$を文とするとき,
			\begin{align}
				\mathscr{S} \vdash A \rarrow B
			\end{align}
			であれば
			\begin{align}
				A,\ \mathscr{S} \vdash B
			\end{align}
			が成り立つ.
		\end{metathm}
	\end{screen}
	%\section{式の書き換え(没)}
	\begin{itemize}
		\item $x \in y$はそのまま$x \in y$
		\item $x \in \{y|B(y)\}$は$B(x)$
			
			これは公理である.つまり,
			\begin{align}
				\forall x\, \left(\, x \in \{y|B(y)\} \leftrightarrow B(x)\, \right).
			\end{align}
			
		\item $x \in \varepsilon y B(y)$は$\exists t\, \left(\, x \in t \wedge B(t)\, \right)$.ちなみにこれは公理とするべきか:
			\begin{align}
				\forall x\, \left(\, x \in \varepsilon y B(y) \leftrightarrow
				\exists t\, \left(\, x \in t \wedge B(t)\, \right)\, \right).
			\end{align}
			
		\item $\{x|A(x)\} \in y$は$\exists s\, \left(\, s \in y \wedge 
			\forall u\, \left(\, u \in s \leftrightarrow A(u)\, \right)\, \right)$
			
			実はこの両式は同値である.さていま
			\begin{align}
				\{x|A(x)\} \in y \leftrightarrow
				\exists s\, \left(\, s \in y \wedge 
				\forall u\, \left(\, u \in s \leftrightarrow A(u)\, \right)\, \right)
			\end{align}
			という式を$\varphi$とし,これを$\mathcal{L}_{\in}$の式に書き換えたものを$\hat{\varphi}$としよう.そして
			\begin{align}
				\eta = \varepsilon y \rightharpoondown \hat{\varphi}(y)
			\end{align}
			とおこう.ここで証明するのは
			\begin{align}
				\{x|A(x)\} \in \eta \leftrightarrow
				\exists s\, \left(\, s \in \eta \wedge 
				\forall u\, \left(\, u \in s \leftrightarrow A(u)\, \right)\, \right)
			\end{align}
			が成り立つということである.まず
			\begin{align}
				\{x|A(x)\} \in \eta
			\end{align}
			が成り立っているとしよう.すると
			\begin{align}
				\exists s\, \left(\, \{x|A(x)\} = s\, \right)
			\end{align}
			が成り立つのだが,今度も式の書き直し手順によって
			\begin{align}
				\exists s\, \left(\, \forall u\, \left(\, A(u) \leftrightarrow
				u \in s\, \right)\, \right)
			\end{align}
			と書き直される.
			\begin{align}
				\sigma = \varepsilon s\, \left(\, \forall u\, \left(\, A(u) \leftrightarrow
				u \in s\, \right)\, \right)
			\end{align}
			とおくと
			\begin{align}
				\forall u\, \left(\, A(u) \leftrightarrow
				u \in \sigma\, \right)
			\end{align}
			が成り立ち,他方で
			\begin{align}
				\sigma = \{x|A(x)\}
			\end{align}
			が成り立つのだから
			\begin{align}
				\sigma \in \eta
			\end{align}
			も従う.ゆえに
			\begin{align}
				\sigma \in \eta \wedge \forall u\, \left(\, A(u) \leftrightarrow
				u \in \sigma\, \right)
			\end{align}
			が成り立つ.逆に
			\begin{align}
				\exists s\, \left(\, s \in \eta \wedge 
				\forall u\, \left(\, u \in s \leftrightarrow A(u)\, \right)\, \right)
			\end{align}
			が成り立っているとして,
			\begin{align}
				\sigma = \varepsilon s\, \left(\, s \in \eta \wedge 
				\forall u\, \left(\, u \in s \leftrightarrow A(u)\, \right)\, \right)
			\end{align}
			としよう.すると
			\begin{align}
				\sigma \in \eta \wedge 
				\forall u\, \left(\, u \in \sigma \leftrightarrow A(u)\, \right)
			\end{align}
			が成り立つので
			\begin{align}
				\sigma \in \eta
			\end{align}
			かつ
			\begin{align}
				\sigma = \{x|A(x)\}
			\end{align}
			が成立する.ゆえに
			\begin{align}
				\{x|A(x)\} \in \eta
			\end{align}
			が成立する.以上で
			\begin{align}
				\{x|A(x)\} \in \eta \leftrightarrow
				\exists s\, \left(\, s \in \eta \wedge 
				\forall u\, \left(\, u \in s \leftrightarrow A(u)\, \right)\, \right)
			\end{align}
			が得られた.
			
		\item $\{x|A(x)\} \in \{y|B(y)\}$は$\exists s\, \left(\, B(s) \wedge 
			\forall u\, \left(\, u \in s \leftrightarrow A(u)\, \right)\, \right)$
		
		\item $\{x|A(x)\} \in \varepsilon y B(y)$は$\exists s,t\, \left(\, s \in t \wedge 
			\forall u\, \left(\, u \in s \leftrightarrow A(u)\, \right) \wedge B(t)\, \right)$
		
		\item $\varepsilon x A(x) \in y$は$\exists s\, \left(\, s \in y \wedge A(s)\, \right)$
			
			これも公理にしよう:
			\begin{align}
				\forall y\, \left(\, \varepsilon x A(x) \in y \leftrightarrow
				\exists s\, \left(\, s \in y \wedge A(s)\, \right)\, \right).
			\end{align}
			いや,$y$をクラスとした言明の方が良いかも.
			\begin{align}
				\varepsilon x A(x) \in y \leftrightarrow
				\exists s\, \left(\, s \in y \wedge A(s)\, \right).
			\end{align}
		
		\item $\varepsilon x A(x) \in \{y|B(y)\}$は$\exists s\, \left(\, A(s) \wedge B(s)\, \right)$
			
			上の公理からこの式の同値性も導かれます.まず
			\begin{align}
				\exists s\, \left(\, A(s) \wedge B(s)\, \right)
			\end{align}
			が成り立っているとしよう.そして
			\begin{align}
				\sigma = \varepsilon s\, \left(\, A(s) \wedge B(s)\, \right)
			\end{align}
			とおくと,
			\begin{align}
				A(\sigma) \wedge B(\sigma)
			\end{align}
			が成立する.ゆえに
			\begin{align}
				\sigma \in \{y|B(y)\} \wedge A(\sigma)
			\end{align}
			が成立する.ゆえに
			\begin{align}
				\exists s\, \left(\, s \in \{y|B(y)\} \wedge A(s)\, \right)
			\end{align}
			が成り立つ.ゆえに
			\begin{align}
				\varepsilon x A(x) \in \{y|B(y)\}
			\end{align}
			が成り立つ.逆に
			\begin{align}
				\varepsilon x A(x) \in \{y|B(y)\}
			\end{align}
			が成り立っているとしよう.すると
			\begin{align}
				\exists s\, \left(\, s = \varepsilon x A(x)\, \right)
			\end{align}
			が成り立つが,これは$\mathcal{L}_{\in}$の式で
			\begin{align}
				\exists s A(s)
			\end{align}
			であって,
			\begin{align}
				\sigma = \varepsilon s A(s)
			\end{align}
			とおけば
			\begin{align}
				A(\sigma)
			\end{align}
			が成立する.ところで
			\begin{align}
				\sigma \in \{y|B(y)\}
			\end{align}
			なので
			\begin{align}
				B(\sigma)
			\end{align}
			も成り立つ.ゆえに
			\begin{align}
				A(\sigma) \wedge B(\sigma)
			\end{align}
			が成り立つ.ゆえに
			\begin{align}
				\exists s\, \left(\, A(s) \wedge B(s)\, \right)
			\end{align}
			が成り立つ.
			
		\item $\varepsilon x A(x) \in \varepsilon y B(y)$は$\exists s,t\, \left(\, s \in t \wedge A(s) \wedge B(t)\, \right)$
		
			この式の同値性も証明できる.まず
			\begin{align}
				\exists s,t\, \left(\, s \in t \wedge A(s) \wedge B(t)\, \right)
			\end{align}
			が成り立っているとしよう.この式は
			\begin{align}
				\exists s\, \left(\, \exists t\, \left(\, s \in t \wedge A(s) \wedge B(t)\, \right)\, \right)
			\end{align}
			の略記であって,$\exists$の規則より
			\begin{align}
				\sigma = \varepsilon s\, \left(\, \exists t\, \left(\, s \in t \wedge A(s) \wedge B(t)\, \right)\, \right)
			\end{align}
			とおけば
			\begin{align}
				\exists t\, \left(\, \sigma \in t \wedge A(\sigma) \wedge B(t)\, \right)
			\end{align}
			が成立する.$\sigma \in t \wedge A(\sigma) \wedge B(t)$を$\mathcal{L}_{\in}$の式に書き直したものを
			$\varphi(t)$として
			\begin{align}
				\tau \defeq \varepsilon t \varphi(t)
			\end{align}
			とおけば,$\exists$の規則より
			\begin{align}
				\sigma \in \tau \wedge A(\sigma) \wedge B(\tau)
			\end{align}
			が成立する.ゆえに
			\begin{align}
				\exists s\, \left(\, s \in \tau \wedge A(s)\, \right)
			\end{align}
			が成り立つから,公理より
			\begin{align}
				\varepsilon x A(x) \in \tau
			\end{align}
			が成立する.ゆえに
			\begin{align}
				\varepsilon x A(x) \in \tau \wedge B(\tau)
			\end{align}
			が成立する.ゆえに
			\begin{align}
				\exists t\, \left(\, \varepsilon x A(x) \in t \wedge B(t)\, \right)
			\end{align}
			が成立する.公理より
			\begin{align}
				\varepsilon x A(x) \in \varepsilon y B(y)
			\end{align}
			が成立する.逆は容易い.
			\begin{align}
				\varepsilon x A(x) \in \varepsilon y B(y)
			\end{align}
			が成り立っているとすれば公理より
			\begin{align}
				\exists t\, \left(\, \varepsilon x A(x) \in t \wedge B(t)\, \right)
			\end{align}
			が成立する.$\varepsilon x A(x) \in t \wedge B(t)$を$\mathcal{L}_{\in}$の式に書き直したものを$\psi(t)$として
			\begin{align}
				\tau \defeq \varepsilon t \psi(t)
			\end{align}
			とおけば
			\begin{align}
				\varepsilon x A(x) \in \tau \wedge B(\tau)
			\end{align}
			が成立するが,ここで公理より
			\begin{align}
				\exists s\, \left(\, s \in \tau \wedge A(s)\, \right)
			\end{align}
			が成り立つので,$s \in \tau \wedge A(s)$を$\mathcal{L}_{\in}$の式に書き直したものを$\xi(s)$として
			\begin{align}
				\sigma \defeq \varepsilon s \xi(s)
			\end{align}
			とおけば
			\begin{align}
				\sigma \in \tau \wedge A(\sigma)
			\end{align}
			が成立する.以上より
			\begin{align}
				\sigma \in \tau \wedge A(\sigma) \wedge B(\tau)
			\end{align}
			が成立する.ゆえに
			\begin{align}
				\exists t\, \left(\, \sigma \in t \wedge A(\sigma) \wedge B(t)\, \right)
			\end{align}
			が得られる.ゆえに
			\begin{align}
				\exists s\, \left(\, \exists t\, \left(\, \sigma \in t \wedge A(\sigma) \wedge B(t)\, \right)\, \right)
			\end{align}
			が得られる.
	\end{itemize}
	
	\begin{screen}
		\begin{logicalaxm}\mbox{}
			\begin{itemize}
				\item 任意の閉項$\tau$に対して,$A(\tau)$が定理ならば$\exists x A(x)$が成り立つ.
				\item $\exists x A(x)$が定理ならば,$A(\varepsilon x \hat{A}(x))$が成り立つ.
				\item すべての閉項$\tau$に対して$A(\tau)$が定理ならば,$\forall x A(x)$が成り立つ.
				\item $\forall x A(x)$が定理ならば,すべての閉項$\tau$に対して$A(\tau)$が成り立つ.
			\end{itemize}
		\end{logicalaxm}
	\end{screen}
	
	定理として
	\begin{align}
		\forall x A(x) \Longleftrightarrow A(\varepsilon x \rightharpoondown \hat{A}(x))
	\end{align}
	が得られる.アイデアとしてはさあ,$\varepsilon x A(x)$の全体が集合に対応しているのであって,
	いやもちろん集合そのものではないけど,集合は$\varepsilon x A(x)$のどれかに等しい類なわけで,
	だからモデル論に出てくる「宇宙」とかいう得体の知れない集合()は俺の集合論に不要なんだよね.
	俺のノートの「宇宙」はすべて実態が把握できるように,具体的な記号列で書き表せるのが良いよね.
	ちなみにこの「宇宙」は$\{x|x=x\}$とは別ね.
	
	\begin{screen}
		\begin{axm}
			\begin{align}
				\forall x\, \left(\, x \in \{y|B(y)\} \leftrightarrow B(x)\, \right).
			\end{align}
			
			\begin{align}
				\forall x\, \left(\, x \in \varepsilon y B(y) \leftrightarrow
				\left(\, \exists t\, B(t) \rightarrow 
				\exists t\, \left(\, x \in t \wedge B(t)\, \right)\, \right)\, \right).
			\end{align}
			が定理となるために
			\begin{align}
				x \in \varepsilon y B(y) \leftrightarrow
				\exists t\, \left(\, x \in t \wedge B(t) \leftrightarrow \exists y B(y)\, \right)
			\end{align}
			を公理とする.
			
			\begin{align}
				\varepsilon x A(x) \in y \leftrightarrow
				\exists s\, \left(\, s \in y \wedge A(s)\, \right).
			\end{align}
		\end{axm}
	\end{screen}
	
	いや,した二つは公理じゃねえな.定理だ.実際
	\begin{align}
		x \in \varepsilon y B(y)
	\end{align}
	が成り立っているとしよう.$\varepsilon y B(y)$は集合であって
	\begin{align}
		\exists s\, \left(\, s = \varepsilon y B(y)\, \right)
	\end{align}
	が成り立つので,
	\begin{align}
		fff
	\end{align}
	
	\begin{screen}
		\begin{thm}
			\begin{align}
				\exists x A(x) \rightarrow \varepsilon x A(x) \in \{x|A(x)\}.
			\end{align}
		\end{thm}
	\end{screen}
	
	\begin{sketch}
		\begin{align}
			\exists x A(x)
		\end{align}
		が成り立ているとするとき,
		\begin{align}
			\sigma \defeq \varepsilon x A(x)
		\end{align}
		とおけば
		\begin{align}
			A(\sigma)
		\end{align}
		が成り立つので,公理より
		\begin{align}
			\sigma \in \{x|A(x)\}
		\end{align}
		が成立する.ゆえに
		\begin{align}
			\sigma \in \{x|A(x)\} \wedge A(\sigma)
		\end{align}
		が成り立つ.ゆえに
		\begin{align}
			\exists s\, \left(\, s \in \{x|A(x)\} \wedge A(s)\, \right)
		\end{align}
		が成立する.ゆえに公理より
		\begin{align}
			\varepsilon x A(x) \in \{x|A(x)\}
		\end{align}
		が成立する.
		\QED
	\end{sketch}
	
	\begin{itembox}[l]{満たされて欲しいこと}
		\begin{description}
			\item[等号]
				\begin{itemize}
					\item $x = \{y|B(y)\}$と$\forall s\, \left(\, s \in x \leftrightarrow B(s)\, \right)$
					\item $x = \varepsilon y B(y)$と$\exists s\, \left(\, A(s) \wedge \forall u\,
						\left(\, u \in x \leftrightarrow u \in s\, \right)\, \right)$
					\item $\{x|A(x)\} = \{y|B(y)\}$と$\forall s\, \left(\, A(s) \leftrightarrow B(s)\, \right)$
					\item $\{x|A(x)\} = \varepsilon y B(y)$と
						$\exists s\, \left(\, \forall u\,
						\left(\, u \in s \leftrightarrow A(u)\, \right) \wedge B(s)\, \right)$
					\item $\varepsilon x A(x) = \varepsilon y B(y)$と
						$\exists s,t\, \left(\, s = t \wedge A(s) \wedge B(t)\, \right)$
				\end{itemize}
				
			\item[帰属]
				\begin{itemize}
					\item $x \in \{y|B(y)\}$は$B(x)$
					\item $x \in \varepsilon y B(y)$は$\exists t\, \left(\, x \in t \wedge B(t)\, \right)$
					\item $\{x|A(x)\} \in y$は$\exists s\, \left(\, s \in y \wedge \forall u\, \left(\, u \in s \leftrightarrow A(u)\, \right)\, \right)$
					\item $\{x|A(x)\} \in \{y|B(y)\}$は$\exists s\, \left(\, B(s) \wedge \forall u\, \left(\, u \in s \leftrightarrow A(u)\, \right)\, \right)$
					\item $\{x|A(x)\} \in \varepsilon y B(y)$は$\exists s,t\, \left(\, s \in t \wedge \forall u\, \left(\, u \in s \leftrightarrow A(u)\, \right) \wedge B(t)\, \right)$
					\item $\varepsilon x A(x) \in y$は$\exists s\, \left(\, s \in y \wedge A(s)\, \right)$
					\item $\varepsilon x A(x) \in \{y|B(y)\}$は$\exists s\, \left(\, A(s) \wedge B(s)\, \right)$
					\item $\varepsilon x A(x) \in \varepsilon y B(y)$は$\exists s,t\, \left(\, s \in t \wedge A(s) \wedge B(t)\, \right)$
				\end{itemize}
		\end{description}
	\end{itembox}

\section{定理I\hspace{-.1em}I.15.2}
	\begin{description}
		\item[(3)]
			項$\tau$が変項$x$のとき,$\zeta_{\tau}(y)$を
			\begin{align}
				x = y
			\end{align}
			とすれば,
			\begin{align}
				\Sigma' \vdash \forall x\, \exists! y\, (\, x=y\, )
			\end{align}
			つまり
			\begin{align}
				\Sigma' \vdash \forall x\, \exists! y\, \zeta_{\tau}(y)
			\end{align}
			および
			\begin{align}
				\Sigma \vdash \forall x\, (\, x=x\, )
			\end{align}
			つまり
			\begin{align}
				\Sigma \vdash \forall x\, \zeta_{\tau}(\tau)
			\end{align}
			が成り立つ.項$\tau$が
			\begin{align}
				f\tau_{1}\cdots\tau_{n}
			\end{align}
			のとき,$f \in \mathcal{L} \backslash \mathcal{L}_{\in}$ならば$\zeta_{\tau}(y)$を
			\begin{align}
				\exists z_{1}, \cdots, z_{n}\, \left(\, 
				\theta_{f}(z_{1},\cdots,z_{n},y) \wedge \zeta_{\tau_{1}}(z_{1}) \wedge
				\cdots \wedge \zeta_{\tau_{n}}(z_{n})\, \right)
			\end{align}
			とし,$f \in \mathcal{L}_{\in}$ならば
			\begin{align}
				\exists z_{1}, \cdots, z_{n}\, \left(\, 
				f(z_{1},\cdots,z_{n}) = y \wedge \zeta_{\tau_{1}}(z_{1}) \wedge
				\cdots \wedge \zeta_{\tau_{n}}(z_{n})\, \right)
			\end{align}
			とする.仮定より
			\begin{align}
				\Sigma \vdash \exists! y\, \zeta_{\tau_{i}}(y)
			\end{align}
			が成り立つので,
			\begin{align}
				\Sigma \vdash \zeta_{\tau_{i}}(z_{i})
			\end{align}
			を満たす$z_{i}$が取れる.そして定義I\hspace{-.1em}I.15.1より
			\begin{align}
				\Sigma \vdash \exists!y\, \theta_{f}(z_{1},\cdots,z_{n},y)
			\end{align}
			が成り立つので,その$y$を取れば
			\begin{align}
				\Sigma \vdash \exists y\, \zeta_{\tau}(y)
			\end{align}
			が成立する.ただし,$\eta$を
			\begin{align}
				\Sigma \vdash \exists z_{1}, \cdots, z_{n}\, \left(\, 
				\theta_{f}(z_{1},\cdots,z_{n},\eta) \wedge \zeta_{\tau_{1}}(z_{1}) \wedge
				\cdots \wedge \zeta_{\tau_{n}}(z_{n})\, \right)
			\end{align}
			を満たすものとすれば,このとき
			\begin{align}
				\Sigma \vdash \zeta_{\tau_{i}}(w_{i})
			\end{align}
			および
			\begin{align}
				\Sigma \vdash \theta_{f}(w_{1},\cdots,w_{n},\eta)
			\end{align}
			を満たす$w_{i}$が取れるが,$z_{i} = w_{i}$なので
			\begin{align}
				\Sigma \vdash \theta_{f}(z_{1},\cdots,z_{n},\eta)
			\end{align}
			が成り立つことになって,定義I\hspace{-.1em}I.15.1より
			\begin{align}
				y = \eta
			\end{align}
			が成り立つ.ゆえに
			\begin{align}
				\Sigma \vdash \exists! y\, \zeta_{\tau}(y)
			\end{align}
			が成立する.他方で仮定より
			\begin{align}
				\Sigma' \vdash \zeta_{\tau_{i}}(\tau_{i})
			\end{align}
			が成り立ち,かつ定義I\hspace{-.1em}I.15.1より
			\begin{align}
				\Sigma' \vdash \forall x_{1},\cdots,x_{n}\,
				\theta_{f}(x_{1},\cdots,x_{n},f(x_{1},\cdots,x_{n}))
			\end{align}
			が成り立つので
			\begin{align}
				\Sigma' \vdash \theta_{f}(\tau_{1},\cdots,\tau_{n},f(\tau_{1},\cdots,\tau_{n}))
			\end{align}
			が成り立つ.ゆえに
			\begin{align}
				\Sigma' \vdash \theta_{f}(\tau_{1},\cdots,\tau_{n},f(\tau_{1},\cdots,\tau_{n})) \wedge \zeta_{\tau_{1}}(\tau_{1}) \wedge \cdots \wedge \zeta_{\tau_{n}}(\tau_{n})
			\end{align}
			が成り立つ.ゆえに
			\begin{align}
				\Sigma' \vdash \zeta_{\tau}(\tau)
			\end{align}
			が成り立つ.
			
		\item[(2)]
			$\varphi$を$p\tau_{1}\cdots\tau_{n}$なる原子式とするとき,
			$p$が$\mathcal{L} \backslash \mathcal{L}_{\in}$の要素ならば$\hat{\varphi}$を
			\begin{align}
				\exists z_{1},\cdots,z_{n}\, 
				\left(\, \theta_{p}z_{1} \cdots z_{n} \wedge \zeta_{\tau_{1}}(z_{1})
				\wedge \cdots \wedge \zeta_{\tau_{n}}(z_{n})\, \right)
			\end{align}
			とし,$p$が$\mathcal{L}_{\in}$の要素ならば
			\begin{align}
				\exists z_{1},\cdots,z_{n}\, 
				\left(\, pz_{1} \cdots z_{n} \wedge \zeta_{\tau_{1}}(z_{1})
				\wedge \cdots \wedge \zeta_{\tau_{n}}(z_{n})\, \right)
			\end{align}
			とする.$\varphi$が成り立っているとき,仮定より
			\begin{align}
				\Sigma' \vdash \zeta_{\tau_{i}}(\tau_{i})
			\end{align}
			が満たされ,また定義I\hspace{-.1em}I.15.1より($\Delta$は$\Sigma'$に含まれているので)
			\begin{align}
				\Sigma' \cup \{\varphi\} \vdash \theta_{p}\tau_{1} \cdots \tau_{n}
			\end{align}
			も満たされているので
			\begin{align}
				\Sigma' \cup \{\varphi\} \vdash \theta_{p}\tau_{1} \cdots \tau_{n}
				\wedge \zeta_{\tau_{1}}(\tau_{1}) \wedge \cdots \wedge 
				\zeta_{\tau_{n}}(\tau_{n})
			\end{align}
			が成り立つ.すなわち
			\begin{align}
				\Sigma' \cup \{\varphi\} \vdash \hat{\varphi}
			\end{align}
			が成り立つ.つまり
			\begin{align}
				\Sigma' \vdash \varphi \rightarrow \hat{\varphi}
			\end{align}
			が成り立つ.逆に$\hat{\varphi}$が成り立っているとき,
			\begin{align}
				\Sigma' \cup \{\hat{\varphi}\} \vdash \theta_{p}w_{1} \cdots w_{n}
				\wedge \zeta_{\tau_{1}}(w_{1}) \wedge \cdots \wedge 
				\zeta_{\tau_{n}}(w_{n})
			\end{align}
			を満たす$w_{1},\cdots,w_{n}$が取れるが,
			\begin{align}
				\Sigma \vdash \exists! y\, \zeta_{\tau_{i}}(y)
			\end{align}
			かつ
			\begin{align}
				\Sigma' \vdash \zeta_{\tau_{i}}(\tau_{i})
			\end{align}
			なので
			\begin{align}
				w_{i} = \tau_{i}
			\end{align}
			である.ゆえに
			\begin{align}
				\Sigma' \cup \{\hat{\varphi}\} \vdash \theta_{p}\tau_{1} \cdots \tau_{n}
			\end{align}
			が成り立つ.ゆえに
			\begin{align}
				\Sigma' \vdash \hat{\varphi} \rightarrow \varphi
			\end{align}
			が得られる.
			\QED
	\end{description}
	
	\begin{screen}
		$\forall x\, (\, x \notin y\, )$を$\theta_{\emptyset}(y)$とするとき.
	\end{screen}
	
	$\emptyset$の定義$\delta_{\emptyset}$は
	\begin{align}
		\forall x\, (\, x \notin \emptyset\, )
	\end{align}
	である.$\zeta_{\emptyset}(y)$は
	\begin{align}
		\forall x\, (\, x \notin y\, )
	\end{align}
	であって,
	\begin{align}
		\Sigma \vdash \exists! y\, \zeta_{\emptyset}(y)
	\end{align}
	が成り立ち,また$\delta_{\emptyset}$と$\zeta_{\emptyset}(1)$は同じなので
	\begin{align}
		\Sigma \cup \{\delta_{\emptyset}\} = \Sigma' \vdash \zeta_{\emptyset}(\emptyset)
	\end{align}
	が成り立つ.そして,例えば$z$を変項とすれば
	\begin{align}
		z \in \emptyset
	\end{align}
	と
	\begin{align}
		\exists s,t\, \left(\, s \in t \wedge s = z \wedge \forall x\, (\, x \notin t\, )\, \right)
	\end{align}
	が$\Sigma'$の下で同値になる.
	
	\begin{screen}
		$\forall x\, (\, x \cdot y = y \cdot x = x\, )$を$\theta_{1}(y)$とするとき,
	\end{screen}
	
	$1$の定義$\delta_{1}$は
	\begin{align}
		\forall x\, (\, x \cdot 1 = 1 \cdot x = x\, )
	\end{align}
	である.$\zeta_{1}(y)$は
	\begin{align}
		\forall x\, (\, x \cdot y = y \cdot x = x\, )
	\end{align}
	であって,
	\begin{align}
		\Sigma \vdash \exists! y\, \zeta_{1}(y)
	\end{align}
	が成り立ち,また$\delta_{1}$と$\zeta_{1}(1)$は同じなので
	\begin{align}
		\Sigma \cup \{\delta_{1}\} = \Sigma' \vdash \zeta_{1}(1)
	\end{align}
	が成り立つ.そして,例えば$z$を変項とすれば
	\begin{align}
		z \in 1
	\end{align}
	と
	\begin{align}
		\exists s,t\, \left(\, s \in t \wedge s = z \wedge \forall x\, (\, x \cdot t = t \cdot x = x\, )\, \right)
	\end{align}
	が$\Sigma'$の下で同値になる.
	
	\begin{screen}
		$y \cdot (x \cdot x) = x$を$\theta_{i}(x,y)$とするとき,
	\end{screen}
	
	$i$の定義$\delta_{i}$は
	\begin{align}
		\forall x\, \left(\, i(x) \cdot (x \cdot x) = x\, \right)
	\end{align}
	である.$\zeta_{i(x)}(y)$は
	\begin{align}
		\exists x\, \left(\, \theta_{i}(x,y) \wedge x = x\, \right)
	\end{align}
	である.そして,例えば$x,z$を変項とすれば
	\begin{align}
		z \in i(x)
	\end{align}
	と
	\begin{align}
		\exists s,t\, \left(\, s \in t \wedge s = z \wedge \zeta_{i(x)}(t)\, \right)
	\end{align}
	が$\Sigma'$の下で同値になる.
	
\section{菊池誠不完全性定理}
	言語とは定数記号と関数記号と関係記号の全体ということで,
	\begin{itemize}
		\item $\mathcal{L}_{\in}$とは$\{\in,\natural\}$.
		\item $\mathcal{L}$とは$\{\in\}$に加えて閉項の全体.
	\end{itemize}
	\section{推論}
	この節では後の集合論で使ういくつかの演繹定理を導出する.
	%「類は集合であるか真類であるかのいずれかには定まる」と
	%「集合であり真類でもある類は存在しない」の二つの言明を得ることを主軸に
	
	\begin{screen}
		\begin{logicalaxm}[矛盾の導入]
		\label{logicalaxm:introduction_of_contradiction}
			否定が共に成り立つとき矛盾が起きる.$A$を文とするとき
			\begin{align}
				&A \rarrow (\, \negation A \rarrow \bot\, ), \\
				&\negation A \rarrow (\, A \rarrow \bot\, ).
			\end{align}
		\end{logicalaxm}
	\end{screen}
	
	\begin{screen}
		\begin{logicalaxm}[否定の導入]
		\label{logicalaxm:introduction_of_negation}
			矛盾が導かれるとき否定が成り立つ.$A$を文とするとき
			\begin{align}
				(\, A \rarrow \bot\, ) \rarrow\ \negation A.
			\end{align}
		\end{logicalaxm}
	\end{screen}
	
	\begin{comment}
	%%%%%%%%%%%%%%%%%%%%%%%%%%%%コメント%%%%%%%%%%%%%%%%%%%%%%%%%%%%%%
	
	たとえば公理系$\mathscr{S}$の下で
	\begin{align}
		\mathscr{S} \vdash A
		\label{fom:introduction_of_contradiction_1}
	\end{align}
	と
	\begin{align}
		\mathscr{S} \vdash\ \negation A
		\label{fom:introduction_of_contradiction_2}
	\end{align}
	が導かれたとすれば,矛盾の導入は
	\begin{align}
		\mathscr{S} \vdash A \rarrow (\, \negation A \rarrow \bot\, )
	\end{align}
	を満たすので,(\refeq{fom:introduction_of_contradiction_1})との三段論法より
	\begin{align}
		\mathscr{S} \vdash\ \negation A \rarrow \bot
	\end{align}
	となり,(\refeq{fom:introduction_of_contradiction_2})との三段論法より
	\begin{align}
		\mathscr{S} \vdash \bot
	\end{align}
	が従う.これを始めと結論だけ見て直感的に
	\begin{prooftree}
		\AxiomC{$\mathscr{S} \vdash A$}
		\AxiomC{$\mathscr{S} \vdash\ \negation A$}
		\BinaryInfC{$\mathscr{S} \vdash \bot$}
	\end{prooftree}
	と書いてみれば,こちらは矛盾の導入規則そのままの形に似ているので,
	あたかも矛盾の導入規則を``直接''適用したように見える.他の演繹定理も
	実際の証明では演繹定理に直したものを用いるのであるが,
	始めと結論だけ見れば
	\begin{prooftree}
		\AxiomC{$\mathscr{S} \vdash A \rarrow \bot$}
		\UnaryInfC{$\mathscr{S} \vdash\ \negation A$}
	\end{prooftree}
	であったり
	\begin{prooftree}
		\AxiomC{$\mathscr{S} \vdash A$}
		\UnaryInfC{$\mathscr{S} \vdash A \vee B$}
	\end{prooftree}
	であったり
	\begin{prooftree}
		\AxiomC{$\mathscr{S} \vdash A(\tau)$}
		\UnaryInfC{$\mathscr{S} \vdash \exists x A(x)$}
	\end{prooftree}
	であったりが``成り立つ''のである.そしてこれらの水平線が入った規則もどきは
	他の証明体系では正式な演繹定理であったりする.
	どうせすぐに演繹定理に直してしまうものを,どうしてわざわざ演繹定理として導入したのかというと
	(演繹定理の形で推論の公理として導入しても同じである),
	他の証明体系の規則との対応を意識しているのと,
	$\vdash$によって前後に分割してある方が若干見やすいからである.
	
	%%%%%%%%%%%%%%%%%%%%%%%%%%%%コメント%%%%%%%%%%%%%%%%%%%%%%%%%%%%%%
	\end{comment}
	
	$\varphi \rarrow \psi$なる式に対して
	\begin{align}
		\negation \psi \rarrow\ \negation \varphi
	\end{align}
	を$\varphi \rarrow \psi$の{\bf 対偶}\index{たいぐう@対偶}{\bf (contraposition)}と呼ぶ.
	
	\begin{screen}
		\begin{logicalthm}[対偶命題が導かれる]
		\label{logicalthm:introduction_of_contraposition}
			$A$と$B$を文とするとき
			\begin{align}
				\vdash (\, A \rarrow B\, ) 
				\rarrow (\, \negation B \rarrow\ \negation A\, ).
			\end{align}
		\end{logicalthm}
	\end{screen}
	
	\begin{prf}
		証明可能性の定義より
		\begin{align}
			A,\ \negation B,\ A \rarrow B &\vdash A, \\
			A,\ \negation B,\ A \rarrow B &\vdash A \rarrow B
		\end{align}
		となるので,三段論法より
		\begin{align}
			A,\ \negation B,\ A \rarrow B \vdash B
			\label{fom:introduction_of_contraposition_1}
		\end{align}
		が従う.同じく証明可能性の定義より
		\begin{align}
			A,\ \negation B,\ A \rarrow B \vdash\ \negation B
			\label{fom:introduction_of_contraposition_2}
		\end{align}
		も成り立つ.ところで矛盾の導入規則より
		\begin{align}
			\vdash B \rarrow (\, \negation B \rarrow \bot\, )
		\end{align}
		が成り立つので,証明可能性の定義より
		\begin{align}
			A,\ \negation B,\ A \rarrow B \vdash
			B \rarrow (\, \negation B \rarrow \bot\, )
		\end{align}
		となる.これと(\refeq{fom:introduction_of_contraposition_1})との三段論法より
		\begin{align}
			A,\ \negation B,\ A \rarrow B \vdash\ \negation B \rarrow \bot
		\end{align}
		が従い,これと(\refeq{fom:introduction_of_contraposition_2})との三段論法より
		\begin{align}
			A,\ \negation B,\ A \rarrow B \vdash \bot
		\end{align}
		が従う.演繹規則より
		\begin{align}
			\negation B,\ A \rarrow B \vdash A \rarrow \bot
		\end{align}
		となるが,今度は否定の導入規則より
		\begin{align}
			\negation B,\ A \rarrow B \vdash 
			(\, A \rarrow \bot\, ) \rarrow\ \negation A
		\end{align}
		が満たされるので,三段論法より
		\begin{align}
			\negation B,\ A \rarrow B \vdash\ \negation A
		\end{align}
		が出る.そして演繹規則より
		\begin{align}
			A \rarrow B \vdash\ \negation B \rarrow\ \negation A
		\end{align}
		が得られ,再び演繹規則より
		\begin{align}
			\vdash (\, A \rarrow B\, ) \rarrow
			(\, \negation B \rarrow\ \negation A\, )
		\end{align}
		が得られる.
		\QED
	\end{prf}
	
	公理系$\mathscr{S}$の下で$A \rarrow B$が導かれたとすれば,上の演繹定理より
	\begin{align}
		\mathscr{S} \vdash (\, A \rarrow B\, ) 
		\rarrow (\, \negation B \rarrow\ \negation A\, )
	\end{align}
	が成り立つので三段論法より
	\begin{align}
		\mathscr{S} \vdash\ \negation B \rarrow\ \negation A
	\end{align}
	が従う.以下では,$\mathscr{S} \vdash A \rarrow B$であるときに
	{\bf 「対偶を取る」}と宣言して$\mathscr{S} \vdash\ \negation B \rarrow\ \negation A$
	に繋げることもある.
	
	\begin{screen}
		\begin{dfn}[二重否定]
			式$\varphi$に対して,$\negation$を二つ連結させた式
			\begin{align}
				\negation \negation \varphi
			\end{align}
			を$\varphi$の{\bf 二重否定}\index{にじゅうひてい@二重否定}
			{\bf (double negation)}と呼ぶ.
		\end{dfn}
	\end{screen}
		
	\begin{screen}
		\begin{logicalthm}[二重否定の導入]
		\label{logicalthm:introduction_of_double_negation}
			$A$を文とするとき
			\begin{align}
				\vdash A \rarrow \negation \negation A.
			\end{align}
		\end{logicalthm}
	\end{screen}
	
	\begin{prf}
		矛盾の導入規則より
		\begin{align}
			A,\ \negation A \vdash \bot
		\end{align}
		となるので,演繹規則より
		\begin{align}
			A \vdash\ \negation A \rarrow \bot
			\label{fom:introduction_of_double_negation}
		\end{align}
		が従う.また否定の導入規則より
		\begin{align}
			\vdash (\, \negation A \rarrow \bot\, ) \rarrow \negation \negation A
		\end{align}
		が成り立つので,証明可能性の定義より
		\begin{align}
			A \vdash (\, \negation A \rarrow \bot\, ) \rarrow \negation \negation A
		\end{align}
		も成り立ち,(\refeq{fom:introduction_of_double_negation})との三段論法より
		\begin{align}
			A \vdash\ \negation \negation A
		\end{align}
		が従う.そして演繹規則より
		\begin{align}
			\vdash A \rarrow \negation \negation A
		\end{align}
		が得られる.
		\QED
	\end{prf}
	
	\begin{screen}
		\begin{logicalaxm}[論理積の除去]
		\label{logicalaxm:elimination_of_conjunction}
			$A$と$B$を文とするとき
			\begin{align}
				A &\wedge B \vdash A, \\
				A &\wedge B \vdash B.
			\end{align}
		\end{logicalaxm}
	\end{screen}
	
	肯定と否定は両立しない.
	
	\begin{screen}
		\begin{logicalthm}[無矛盾律]
		\label{logicalthm:law_of_noncontradiction}
			$A$を文とするとき
			\begin{align}
				\vdash\ \negation (\, A \wedge \negation A\, ).
			\end{align}
		\end{logicalthm}
	\end{screen}
	
	\begin{prf}
		論理積の除去規則より
		\begin{align}
			A \wedge \negation A &\vdash A, \\
			A \wedge \negation A &\vdash\ \negation A
		\end{align}
		が成り立ち,また矛盾の導入規則より
		\begin{align}
			A \wedge \negation A \vdash A \rarrow (\, \negation A \rarrow \bot\, )
		\end{align}
		が成り立つので,三段論法より
		\begin{align}
			A \wedge \negation A \vdash \bot
		\end{align}
		が従う.ゆえに演繹規則より
		\begin{align}
			\vdash (\, A \wedge \negation A\, ) \rarrow \bot
		\end{align}
		となり,否定の導入規則
		\begin{align}
			\vdash (\, (\, A \wedge \negation A\, ) \rarrow \bot\, )
			\rarrow \negation (\, A \wedge \negation A\, )
		\end{align}
		との三段論法より
		\begin{align}
			\vdash\ \negation (\, A \wedge \negation A\, )
		\end{align}
		が得られる.
		\QED
	\end{prf}
	
	ここで新しい論理記号$\lrarrow$を定めるが,そのときに$\defarrow$なる記号を用いる.
	これは{\bf 定義記号}\index{ていぎきごう@定義記号}と呼ばれ,
	\begin{align}
		P \defarrow \varphi
		\label{fom:defining_arrow}
	\end{align}
	と書けば「式$\varphi$を記号$P$で置き換えて良い」という意味での略記法を導入できる.
	
	\begin{screen}
		\begin{dfn}[同値記号]
			$A$と$B$を$\mathcal{L}$の式とするとき,
			\begin{align}
				A \lrarrow B  \defarrow
				(\, A \rarrow B\, ) \wedge (\, B \rarrow A\, )
			\end{align}
			により$\lrarrow$を定め,式`$A \lrarrow B$'を
			「$A$と$B$は{\bf 同値である}\index{どうち@同値}{\bf (equivalent)}」と読む.
		\end{dfn}
	\end{screen}
	
	以降で{\bf De Morgan の法則}\index{De Morgan の法則}{\bf (De Morgan's laws)}
	\begin{align}
		&\vdash\ \negation (\, \varphi \vee \psi\, ) \lrarrow\ \negation \varphi
		\wedge \negation \psi, \\
		&\vdash\ \negation (\, \varphi \wedge \psi\, ) \lrarrow\ \negation \varphi
		\vee \negation \psi
	\end{align}
	を順番に示していくが,区別するために前者を{\bf 弱 De Morgan の法則}と呼び,
	後者を{\bf 強 De Morgan の法則}と呼ぶ.
	
	\begin{screen}
		\begin{logicalaxm}[論理和の除去]
		\label{logicalaxm:elimination_of_disjunction}
			$A$と$B$と$C$を文とするとき
			\begin{align}
				A \rarrow C,\ B \rarrow C \vdash A \vee B \rarrow C.
			\end{align}
		\end{logicalaxm}
	\end{screen}
	
	{\bf 論理和の除去}とは{\bf 場合分け}\index{ばあいわけ@場合分け}{\bf (proof by case)}
	とも呼ばれる.また場合分け規則に演繹規則を二回適用すれば
	\begin{align}
		\vdash (\, A \rarrow C\, ) 
		\rarrow (\, (\, B \rarrow C\, ) \rarrow (\, A \vee B \rarrow C\, )\, )
	\end{align}
	なる演繹定理が得られる.場合分け規則を実際に用いる際には主にこちらの演繹定理を使う.
	
	\begin{screen}
		\begin{logicalthm}[弱 De Morgan の法則(1)]
		\label{logicalthm:weak_De_Morgan_law_1}
			$A$と$B$を文とするとき
			\begin{align}
				\vdash\ \negation A \wedge \negation B
				\rarrow\ \negation (\, A \vee B\, ).
			\end{align}
		\end{logicalthm}
	\end{screen}
	
	\begin{prf}
		論理積の除去規則より
		\begin{align}
			\negation A \wedge \negation B \vdash\ \negation A
			\label{fom:weak_De_Morgan_law_1_1}
		\end{align}
		となり,また矛盾の導入規則より
		\begin{align}
			\vdash\ \negation A \rarrow (\, A \rarrow \bot\, )
		\end{align}
		が成り立つので
		\begin{align}
			\negation A \wedge \negation B 
			\vdash\ \negation A \rarrow (\, A \rarrow \bot\, )
		\end{align}
		も成り立ち,(\refeq{fom:weak_De_Morgan_law_1_1})との三段論法より
		\begin{align}
			\negation A \wedge \negation B &\vdash A \rarrow \bot
			\label{fom:weak_De_Morgan_law_1_2}
		\end{align}
		が従う.同様に
		\begin{align}
			\negation A \wedge \negation B \vdash B \rarrow \bot
			\label{fom:weak_De_Morgan_law_1_3}
		\end{align}
		も得られる.ところで場合分け規則より
		\begin{align}
			\vdash (\, A \rarrow \bot\, ) \rarrow (\, (\, B \rarrow \bot\, )
			\rarrow (\, A \vee B \rarrow \bot\, )\, )
		\end{align}
		が成り立つので,(\refeq{fom:weak_De_Morgan_law_1_2})と
		(\refeq{fom:weak_De_Morgan_law_1_3})との三段論法より
		\begin{align}
			\negation A \wedge \negation B &\vdash (\, A \rarrow \bot\, ) 
			\rarrow (\, (\, B \rarrow \bot\, ) 
			\rarrow (\, A \vee B \rarrow \bot\, )\, ), \\
			\negation A \wedge \negation B &\vdash (\, B \rarrow \bot\, ) 
			\rarrow (\, A \vee B \rarrow \bot\, ), \\
			\negation A \wedge \negation B &\vdash A \vee B \rarrow \bot
		\end{align}
		となり,否定の導入規則から得られる演繹定理
		\begin{align}
			\vdash (\, A \vee B \rarrow \bot\, ) \rarrow\ \negation (\, A \vee B\, )
		\end{align}
		との三段論法より
		\begin{align}
			\negation A \wedge \negation B \vdash\ \negation (\, A \vee B\, ) 
		\end{align}
		が得られる.そして演繹規則より
		\begin{align}
			\vdash\ \negation A \wedge \negation B
				\rarrow\ \negation (\, A \vee B\, )
		\end{align}
		が出る.
		\QED
	\end{prf}
	
	\begin{screen}
		\begin{logicalthm}[強 De Morgan の法則(1)]
		\label{logicalthm:strong_De_Morgan_law_1}
			$A$と$B$を文とするとき
			\begin{align}
				\vdash\ \negation A \vee \negation B
				\rarrow\ \negation (\, A \wedge B\, ).
			\end{align}
		\end{logicalthm}
	\end{screen}
	
	\begin{prf}
		論理積の除去より
		\begin{align}
			\vdash (\, A \wedge B\, ) \rarrow A
		\end{align}
		が成り立つので,対偶を取れば
		\begin{align}
			\vdash\ \negation A \rarrow\ \negation (\, A \wedge B\, )
			\label{fom:strong_De_Morgan_law_1_1}
		\end{align}
		が成り立つ(演繹定理\ref{logicalthm:introduction_of_contraposition}).
		同様に
		\begin{align}
			\vdash\ \negation B \rarrow\ \negation (\, A \wedge B\, )
			\label{fom:strong_De_Morgan_law_1_2}
		\end{align}
		も得られる.また論理和の除去規則より
		\begin{align}
			\vdash (\, \negation A \rarrow\ \negation (\, A \wedge B\, )\, )
			\rarrow (\, (\, \negation B \rarrow\ \negation (\, A \wedge B\, )\, )
			\rarrow (\, \negation A \vee \negation B 
			\rarrow\ \negation (\, A \wedge B\, )\, )\, )
		\end{align}
		が成り立つので,(\refeq{fom:strong_De_Morgan_law_1_1})との三段論法より
		\begin{align}
			\vdash (\, \negation B \rarrow\ \negation (\, A \wedge B\, )\, )
			\rarrow (\, \negation A \vee \negation B 
			\rarrow\ \negation (\, A \wedge B\, )\, )
		\end{align}
		が従い,(\refeq{fom:strong_De_Morgan_law_1_2})との三段論法より
		\begin{align}
			\vdash\ \negation A \vee \negation B 
			\rarrow\ \negation (\, A \wedge B\, )
		\end{align}
		が得られる.
		\QED
	\end{prf}
	
	\begin{screen}
		\begin{logicalaxm}[論理和の導入]
		\label{logicalaxm:introduction_of_disjunction}
			$A$と$B$を文とするとき
			\begin{align}
				A &\vdash A \vee B, \\
				B &\vdash A \vee B.
			\end{align}
		\end{logicalaxm}
	\end{screen}
	
	\begin{screen}
		\begin{logicalthm}[論理和の可換律]
		\label{logicalthm:commutative_law_of_disjunction}
			$A,B$を文とするとき
			\begin{align}
				\vdash A \vee B \rarrow B \vee A.
			\end{align}
		\end{logicalthm}
	\end{screen}
	
	\begin{prf}
		論理和の導入規則と演繹規則により
		\begin{align}
			\vdash A \rarrow B \vee A
			\label{fom:logicalthm_commutative_law_of_disjunction_1}
		\end{align}
		と
		\begin{align}
			\vdash B \rarrow B \vee A
			\label{fom:logicalthm_commutative_law_of_disjunction_2}
		\end{align}
		が成り立つ.また場合分け規則より
		\begin{align}
			\vdash (\, A \rarrow B \vee A\, ) \rarrow (\, (\, B \rarrow B \vee A\, ) 
			\rarrow (\, A \vee B \rarrow B \vee A\, )\, )
		\end{align}
		が成り立つので,(\refeq{fom:logicalthm_commutative_law_of_disjunction_1})と
		三段論法より
		\begin{align}
			\vdash (\, B \rarrow B \vee A\, ) 
			\rarrow (\, A \vee B \rarrow B \vee A\, )
		\end{align}
		となり,(\refeq{fom:logicalthm_commutative_law_of_disjunction_2})と
		三段論法より
		\begin{align}
			\vdash A \vee B \rarrow B \vee A
		\end{align}
		となる.
		\QED
	\end{prf}
	
	\begin{screen}
		\begin{logicalaxm}[論理積の導入]
		\label{logicalaxm:introduction_of_conjunction}
			$A$と$B$を文とするとき
			\begin{align}
				A,\ B \vdash A \wedge B.
			\end{align}
		\end{logicalaxm}
	\end{screen}
	
	論理積の導入に演繹規則を適用すれば
	\begin{align}
		A,B &\vdash A \wedge B, \\
		A &\vdash B \rarrow A \wedge B, \\
		&\vdash A \rarrow (\, B \rarrow A \wedge B\, )
	\end{align}
	となる.これで
	\begin{align}
		\vdash A \rarrow (\, B \rarrow A \wedge B\, )
	\end{align}
	なる演繹定理が得られた.
	
	\begin{screen}
		\begin{logicalthm}[弱 De Morgan の法則(2)]
		\label{logicalthm:weak_De_Morgan_law_2}
			$A$と$B$を文とするとき
			\begin{align}
				\vdash\ \negation (\, A \vee B\, ) 
				\rarrow\ \negation A \wedge \negation B.
			\end{align}
		\end{logicalthm}
	\end{screen}
	
	\begin{prf}
		論理和の導入規則より
		\begin{align}
			\vdash A \rarrow A \vee B
		\end{align}
		が成り立つが,対偶を取れば
		\begin{align}
			\vdash\ \negation (\, A \vee B\, ) \rarrow\ \negation A
			\label{fom:weak_De_Morgan_law_2_1}
		\end{align}
		となる(演繹定理\ref{logicalthm:introduction_of_contraposition}).
		同じく論理和の導入規則より
		\begin{align}
			\vdash B \rarrow A \vee B
		\end{align}
		が成り立つので
		\begin{align}
			\vdash\ \negation (\, A \vee B\, ) \rarrow\ \negation B
			\label{fom:weak_De_Morgan_law_2_2}
		\end{align}
		も得られる.ここで(\refeq{fom:weak_De_Morgan_law_2_1})と
		(\refeq{fom:weak_De_Morgan_law_2_2})と演繹法則の逆より
		\begin{align}
			\negation (\, A \vee B\, ) &\vdash\ \negation A, 
			\label{fom:weak_De_Morgan_law_2_3} \\
			\negation (\, A \vee B\, ) &\vdash\ \negation B
			\label{fom:weak_De_Morgan_law_2_4}
		\end{align}
		が従う.ところで論理積の導入規則より
		\begin{align}
			\vdash\ \negation A \rarrow (\, \negation B \rarrow\
			\negation A \wedge \negation B\, )
		\end{align}
		が成り立つので,(\refeq{fom:weak_De_Morgan_law_2_3})と
		(\refeq{fom:weak_De_Morgan_law_2_4})との三段論法より
		\begin{align}
			\negation (\, A \vee B\, ) &\vdash\ \negation A \rarrow 
				(\, \negation B \rarrow\ \negation A \wedge \negation B\, ), \\
			\negation (\, A \vee B\, ) &\vdash\ 
				\negation B \rarrow\ \negation A \wedge \negation B, \\
			\negation (\, A \vee B\, ) &\vdash\ \negation A \wedge \negation B
		\end{align}
		が従い,演繹規則より
		\begin{align}
			\vdash\ \negation (\, A \vee B\, ) 
			\rarrow\ \negation A \wedge \negation B
		\end{align}
		が得られる.
		\QED
	\end{prf}
	
	\begin{screen}
		\begin{logicalthm}[論理積の可換律]
		\label{logicalthm:commutative_law_of_conjunction}
			$A,B$を文とするとき
			\begin{align}
				\vdash A \wedge B \rarrow B \wedge A.
			\end{align}
		\end{logicalthm}
	\end{screen}
	
	\begin{prf}
		論理積の除去と演繹規則より
		\begin{align}
			A \wedge B \vdash A
			\label{fom:logicalthm_commutative_law_of_conjunction_1}
		\end{align}
		と
		\begin{align}
			A \wedge B \vdash B
			\label{fom:logicalthm_commutative_law_of_conjunction_2}
		\end{align}
		が成り立つ.また論理積の導入により
		\begin{align}
			\vdash B \rarrow (\, A \rarrow B \wedge A\, )
		\end{align}
		となるので
		\begin{align}
			A \wedge B \vdash B \rarrow (\, A \rarrow B \wedge A\, )
		\end{align}
		も成り立ち,(\refeq{fom:logicalthm_commutative_law_of_conjunction_2})との三段論法より
		\begin{align}
			A \wedge B \vdash A \rarrow B \wedge A
		\end{align}
		となり,(\refeq{fom:logicalthm_commutative_law_of_conjunction_1})との三段論法より
		\begin{align}
			A \wedge B \vdash B \wedge A
		\end{align}
		となり,演繹規則より
		\begin{align}
			\vdash A \wedge B \rarrow B \wedge A
		\end{align}
		が得られる.
		\QED
	\end{prf}

\chapter{集合}	
		$a,b$を$\mathcal{L}$の項とするとき,
	\begin{align}
		a \notin b \defarrow\ \negation a \in b
	\end{align}
	で$a \notin b$を定める.同様に
	\begin{align}
		a \neq b \defarrow\ \negation a = b
	\end{align}
	で$a \neq b$を定める.
	
	類とされた項の多くは集合であるが,{\bf 類が全て集合であると考えると矛盾が起こる}.
	たとえばRussellのパラドックスで有名な
	\begin{align}
		R \defeq \Set{x}{x \notin x}
	\end{align}
	なる類が集合であるとすると($\defarrow$は``式''に対する略記の導入に使ったが,
	$\defeq$とは``類''に対する略記を導入するために使う定義記号である)
	\begin{align}
		\Sigma \vdash R \notin R \lrarrow R \in R
	\end{align}
	が成り立ってしまい(正式な推論は無視してラフに考えれば),これは$\Sigma \vdash \bot$を導く.
	この種の矛盾を回避するために類を導入したのであり,
	集合とは類の中で特定の性質をもつものに限られる.
	
	\begin{screen}
		\begin{dfn}[集合]
			$a$を類とするとき,$a$が集合であるという言明を
			\begin{align}
				\set{a} \defarrow \exists x\, (\, a = x\, )
			\end{align}
			で定める.$\Sigma \vdash \set{a}$を満たす類$a$を
			{\bf 集合}\index{しゅうごう@集合}{\bf (set)}と呼び,
			$\Sigma \vdash\ \negation \set{a}$を満たす類$a$を
			{\bf 真類}\index{しんるい@真類}{\bf (proper class)}と呼ぶ.
		\end{dfn}
	\end{screen}
	
	$\varphi$を$\mathcal{L}$の式とし,$x$を$\varphi$に自由に現れる変項とし,
	$x$のみが$\varphi$で自由であるとする.このとき
	\begin{align}
		\set{\Set{x}{\varphi(x)}} \vdash \set{\Set{x}{\varphi(x)}}
	\end{align}
	が満たされている.つまり
	\begin{align}
		\set{\Set{x}{\varphi(x)}}
		\vdash \exists y\, \left(\, \Set{x}{\varphi(x)} = y\, \right)
	\end{align}
	が成り立っているということであるが,$\Set{x}{\varphi(x)} = y$を
	\begin{align}
		\forall x\, (\, \varphi(x) \lrarrow x \in y\, )
	\end{align}
	と書き換えれば,存在記号の推論規則より
	\begin{align}
		\set{\Set{x}{\varphi(x)}} \vdash \Set{x}{\varphi(x)} = 
		\varepsilon y\, \forall x\, (\, \varphi(x) \lrarrow x \in y\, )
	\end{align}
	が得られる.
	
	\begin{screen}
		\begin{thm}[集合である内包項は$\varepsilon$項で書ける]
		\label{thm:if_a_class_is_a_set_then_equal_to_some_epsilon_term}
			$\varphi$を$\mathcal{L}$の式とし,$x$を$\varphi$に自由に現れる変項とし,
			$x$のみが$\varphi$で自由であるとする.このとき
			\begin{align}
				\set{\Set{x}{\varphi(x)}} \vdash \Set{x}{\varphi(x)} 
				= \varepsilon y\, \forall x\, (\, \varphi(x) \lrarrow x \in y\, ).
			\end{align}
		\end{thm}
	\end{screen}
	
	ブルバキ\cite{key4}では$\tau$項を,島内\cite{key6}では$\varepsilon$項のみを導入して
	$\varepsilon y \forall x\, (\, \varphi(x) \lrarrow x \in y\, )$
	によって$\Set{x}{\varphi(x)}$を定めているが,この定め方には欠点がある.
	というのも,本稿と同じくブルバキ\cite{key4}の$\tau$項も島内\cite{key6}の$\varepsilon$項も
	集合であるから,
	\begin{align}
		\exists y\, \forall x\, (\, \varphi(x) \lrarrow x \in y\, )
	\end{align}
	が成立しない場合は$\varepsilon y \forall x\, (\, \varphi(x) \lrarrow x \in y\, )$
	は正体不明になってしまい,$\Set{x}{\varphi(x)}$が「性質$\varphi$を持つ集合の全体」
	の意味を持たないのである.本稿では内包項と$\varepsilon$項を別々に
	生成しているのでこの欠点は解消される.
	
\section{相等性}
	本稿において``等しい''とは項に対する言明であって,$a$と$b$を項とするとき
	\begin{align}
		a = b
	\end{align}
	なる式で表される.この記号
	\begin{align}
		=
	\end{align}
	は{\bf 等号}\index{とうごう@等号}{\bf (equal sign)}と呼ばれるが,
	現時点では述語として導入されているだけで,推論操作における働きは不明のままである.
	本節では,いつ類は等しくなるのか,そして,等しい場合に何が起きるのか,の二つが主題となる.
	
	\begin{screen}
		\begin{axm}[外延性の公理 (Extensionality)]
			任意の類$a,b$に対して
			\begin{align}
				\EXTAX \defarrow \forall x\, (\, x \in a \lrarrow x \in b\, ) 
				\rarrow a=b.
			\end{align}
		\end{axm}
	\end{screen}
	
	\begin{screen}
		\begin{thm}[任意の類は自分自身と等しい]\label{thm:any_class_equals_to_itself}
			任意の類$\tau$に対して
			\begin{align}
				\EXTAX \vdash \tau = \tau.
			\end{align}
		\end{thm}
	\end{screen}
	
	\begin{sketch}
		いま
		\begin{align}
			\sigma \defeq 
			\varepsilon s \negation (\, s \in \tau \lrarrow s \in \tau\, )
		\end{align}
		とおく.推論法則\ref{logicalthm:reflective_law_of_implication}より
		\begin{align}
			\vdash \sigma \in \tau \lrarrow \sigma \in \tau
		\end{align}
		が成り立つから,全称記号の推論規則より
		\begin{align}
			\vdash \forall s\, (\, s \in \tau  \lrarrow s \in \tau\, )
		\end{align}
		が成り立つ.外延性の公理より
		\begin{align}
			\EXTAX \vdash \forall s\, (\, s \in \tau  \lrarrow s \in \tau\, )
			\rarrow \tau = \tau
		\end{align}
		となるので,三段論法より
		\begin{align}
			\EXTAX \vdash \tau = \tau
		\end{align}
		が得られる.
		\QED
	\end{sketch}
	
	\begin{screen}
		\begin{thm}[主要$\varepsilon$項は集合である]
		\label{thm:critical_epsilon_term_is_set}
			$\tau$を類である$\varepsilon$項とするとき
			\begin{align}
				\EXTAX \vdash \set{\tau}.
			\end{align}
		\end{thm}
	\end{screen}
	
	\begin{sketch}
		定理\ref{thm:any_class_equals_to_itself}より
		\begin{align}
			\EXTAX \vdash \tau = \tau
		\end{align}
		が成立するので,存在記号の推論規則より
		\begin{align}
			\EXTAX \vdash \exists x\, \left(\, \tau = x\, \right)
		\end{align}
		が成立する.
		\QED
	\end{sketch}
	
	例えば
	\begin{align}
		a = b
	\end{align}
	と書いてあったら``$a$と$b$は等しい''と読めるわけだが,明らかに$a$は$b$とは違うではないではないか!
	こんなことはしょっちゅう起こることであって,上で述べたように$\Set{x}{A(x)}$が集合なら
	\begin{align}
		\Set{x}{A(x)} = \varepsilon y \forall x\, \left(\, A(x) \lrarrow x \in y\, \right)
	\end{align}
	が成り立ったりする.そこで``数学的に等しいとは何事か''という疑問が浮かぶのは至極自然であって,
	それに答えるのが次の相等性公理である.
	
	\begin{screen}
		\begin{axm}[相等性公理]
			$a,b,c$を類とするとき
			\begin{align}
				\EQAX \defarrow
				\begin{cases}
					a = b \rarrow b = a, & \\
					a = b \rarrow (\, a \in c \rarrow b \in c\, ), & \\
					a = b \rarrow (\, c \in a \rarrow c \in b\, ). & 
				\end{cases}
			\end{align}
		\end{axm}
	\end{screen}
	
	\begin{screen}
		\begin{thm}[外延性の公理の逆も成り立つ]
		\label{thm:inverse_of_axiom_of_extensionality}
			$a$と$b$を類とするとき
			\begin{align}
				\EQAX \vdash 
				a = b \rarrow \forall x\, (\, x \in a  \lrarrow x \in b\, ).
			\end{align}
		\end{thm}
	\end{screen}
	
	\begin{prf}
		いま
		\begin{align}
			\tau \defeq \varepsilon x \negation (\, x \in a  \lrarrow x \in b\, )
		\end{align}
		とおく.相等性公理より
		\begin{align}
			\EQAX \vdash a = b \rarrow (\, \tau \in a \rarrow \tau \in b\, )
		\end{align}
		となるので,演繹法則の逆より
		\begin{align}
			a = b,\ \EQAX \vdash \tau \in a \rarrow \tau \in b
			\label{fom:inverse_of_axiom_of_extensionality_1}
		\end{align}
		となる.また相等性公理と演繹法則の逆により
		\begin{align}
			a = b,\ \EQAX \vdash b = a
		\end{align}
		が成り立ち,同じく相等性公理より
		\begin{align}
			\EQAX \vdash b = a \rarrow (\, \tau \in b \rarrow \tau \in a\, )
		\end{align}
		も成り立つので,三段論法より
		\begin{align}
			a = b,\ \EQAX \vdash \tau \in b \rarrow \tau \in a
			\label{fom:inverse_of_axiom_of_extensionality_2}
		\end{align}
		も得られる.論理積の導入により
		\begin{align}
			a = b,\ \EQAX \vdash (\, \tau \in a \rarrow \tau \in b\, )
			\rarrow (\, (\, \tau \in b \rarrow \tau \in a\, )
			\rarrow (\, \tau \in a \lrarrow \tau \in b\, )\, )
		\end{align}
		が成り立つので,(\refeq{fom:inverse_of_axiom_of_extensionality_1})との三段論法より
		\begin{align}
			a = b,\ \EQAX \vdash (\, \tau \in b \rarrow \tau \in a\, )
			\rarrow (\, \tau \in a \lrarrow \tau \in b\, )
		\end{align}
		が従い,(\refeq{fom:inverse_of_axiom_of_extensionality_2})との三段論法より
		\begin{align}
			a = b,\ \EQAX \vdash \tau \in a \lrarrow \tau \in b
		\end{align}
		が従う.全称記号の推論規則より
		\begin{align}
			a = b,\ \EQAX \vdash \forall x\, (\, x \in a  \lrarrow x \in b\, )
		\end{align}
		が成立し,演繹法則より
		\begin{align}
			\EQAX \vdash a = b \rarrow \forall x\, (\, x \in a  \lrarrow x \in b\, )
		\end{align}
		が得られる.
		\QED
	\end{prf}
	
	\begin{comment}
	\begin{screen}
		\begin{thm}[(ボツ!!!)等号の対称律]\label{thm:symmetry_of_equality}
			$a,b$を類とするとき
			\begin{align}
				\EXTAX,\EQAX \vdash a = b \rarrow b = a.
			\end{align}
		\end{thm}
	\end{screen}
	
	\begin{prf}
		定理\ref{thm:axiom_of_extensionality_equivalent}より
		\begin{align}
			a=b,\ \EQAX \vdash \forall x\, (\, x \in a  \lrarrow x \in b\, )
		\end{align}
		となるが,ここで類である任意の$\varepsilon$項$\tau$に対して
		\begin{align}
			a=b,\ \EQAX \vdash \tau \in a \lrarrow \tau \in b
		\end{align}
		となるが,他方で推論法則\ref{logicalthm:symmetry_of_equivalence_arrows}より
		\begin{align}
			a=b,\ \EQAX \vdash (\, \tau \in a \lrarrow \tau \in b\, )
				\rarrow (\, \tau \in b \lrarrow \tau \in a\, )
		\end{align}
		が成り立つので,三段論法より
		\begin{align}
			a=b,\ \EQAX \vdash \tau \in b \lrarrow \tau \in a
		\end{align}
		となる.そして$\tau$の任意性より
		\begin{align}
			a=b,\ \EQAX \vdash \forall x\, (\, x \in b  \lrarrow x \in a\, )
		\end{align}
		が成り立つ.外延性の公理より
		\begin{align}
			a=b,\ \EXTAX,\EQAX \vdash \forall x\, (\, x \in b  \lrarrow x \in a\, )
			\rarrow b = a
		\end{align}
		となるので,三段論法より
		\begin{align}
			a=b,\ \EXTAX,\EQAX \vdash b = a
		\end{align}
		となる.最後に演繹法則より
		\begin{align}
			\EXTAX,\EQAX \vdash a = b \rarrow b = a
		\end{align}
		が得られる.
		\QED
	\end{prf}
	\end{comment}
	
	\begin{screen}
		\begin{axm}[内包性公理] 
			$\varphi$を$\mathcal{L}$の式とし,$y$を$\varphi$に自由に現れる変項とし,
			$\varphi$に自由に現れる項は$y$のみであるとし,
			$x$は$\varphi$で$y$への代入について自由であるとするとき,
			\begin{align}
				\COMAX \defarrow \forall x\, (\, x \in \Set{y}{\varphi(y)} \lrarrow \varphi(x)\, ).
			\end{align}
		\end{axm}
	\end{screen}
	
	\begin{screen}
		\begin{thm}[条件を満たす集合は要素である]\label{thm:satisfactory_set_is_an_element}
			$\varphi$を$\mathcal{L}$の式とし,$x$を$\varphi$に自由に現れる変項とし,
			$x$のみが$\varphi$で束縛されていないとする.このとき,任意の類$a$に対して
			\begin{align}
				\EQAX,\COMAX \vdash \varphi(a) \rarrow 
				\left(\, \set{a} \rarrow a \in \Set{x}{\varphi(x)}\, \right).
			\end{align}
		\end{thm}
	\end{screen}
	
	\begin{sketch}
		\begin{align}
			\set{a} \vdash \exists x\, (\, a = x\, )
		\end{align}
		より,
		\begin{align}
			\tau \defeq \varepsilon x\, (\, a = x\, )
		\end{align}
		とおけば
		\begin{align}
			\set{a} \vdash a = \tau
		\end{align}
		となる.相等性の公理より
		\begin{align}
			\set{a},\EQAX \vdash 
			a = \tau \rarrow (\, \varphi(a) \rarrow \varphi(\tau)\, )
		\end{align}
		となるので,三段論法と演繹法則の逆より
		\begin{align}
			\varphi(a),\set{a},\EQAX \vdash \varphi(\tau)
		\end{align}
		となる.内包性公理より
		\begin{align}
			\varphi(a),\set{a},\EQAX,\COMAX \vdash \tau \in \Set{x}{A(x)}
		\end{align}
		が従い,相等性の公理から
		\begin{align}
			\varphi(a),\set{a},\EQAX,\COMAX \vdash a \in \Set{x}{A(x)}
		\end{align}
		が成立する.演繹法則より
		\begin{align}
			\varphi(a),\EQAX,\COMAX &\vdash \set{a} \rarrow a \in \Set{x}{A(x)}, \\
			\EQAX,\COMAX &\vdash \varphi(a) \rarrow 
			\left(\, \set{a} \rarrow a \in \Set{x}{\varphi(x)}\, \right)
		\end{align}
		が従う.
		\QED
	\end{sketch}
	
	\begin{screen}
		\begin{dfn}[宇宙]
			$\Univ \defeq \Set{x}{x=x}$で定める類$\Univ$を{\bf 宇宙}\index{うちゅう@宇宙}
			{\bf (Universe)}と呼ぶ.
		\end{dfn}
	\end{screen}
	
	定理\ref{thm:V_is_the_whole_of_sets}の通り宇宙とは集合の全体を表すが,
	これ自体は集合ではない.また$\Univ$のより具体的な構造ものちに判る.
	ちなみに名前のVとはVon NeumannのVである.
	
	\begin{screen}
		\begin{axm}[要素の公理]
			要素となりうる類は集合である.つまり,$a,b$を類とするとき
			\begin{align}
				\ELEAX \defarrow a \in b \rarrow \set{a}.
			\end{align}
		\end{axm}
	\end{screen}
	
	\begin{screen}
		\begin{thm}[$\Univ$は集合の全体である]
		\label{thm:V_is_the_whole_of_sets}
			$a$を類とするとき次が成り立つ:
			\begin{align}
				\EXTAX,\EQAX,\ELEAX,\COMAX \vdash \set{a} \lrarrow a \in \Univ.
			\end{align}
		\end{thm}
	\end{screen}
	
	\begin{prf}
		$a$を類とするとき,まず要素の公理より
		\begin{align}
			\ELEAX \vdash a \in \Univ \rarrow \set{a}
		\end{align}
		が得られる.逆を示す.いま
		\begin{align}
			\tau \defeq \varepsilon x\, (\, a = x\, )
		\end{align}
		とおくと,
		\begin{align}
			\set{a} \vdash \exists x\, (\, a = x\, )
		\end{align}
		と
		\begin{align}
			\set{a} \vdash \exists x\, (\, a = x\, ) \rarrow a = \tau
		\end{align}
		(存在記号の推論規則)より
		\begin{align}
			\set{a} \vdash a = \tau
			\label{fom:thm_V_is_the_whole_of_sets_1}
		\end{align}
		が成り立つ.他方で定理\ref{thm:any_class_equals_to_itself}と内包性公理より
		\begin{align}
			\EXTAX &\vdash \tau = \tau, \\
			\COMAX &\vdash \tau = \tau \rarrow \tau \in \Univ
		\end{align}
		が成り立つので,三段論法より
		\begin{align}
			\EXTAX,\COMAX \vdash \tau \in \Univ
			\label{fom:thm_V_is_the_whole_of_sets_2}
		\end{align}
		となる.ここで相等性公理より
		\begin{align}
			\EQAX \vdash a = \tau \rarrow \tau = a
		\end{align}
		が成り立つので,(\refeq{fom:thm_V_is_the_whole_of_sets_1})と三段論法より
		\begin{align}
			\set{a},\EQAX \vdash \tau = a
			\label{fom:thm_V_is_the_whole_of_sets_3}
		\end{align}
		となる.同じく相等性公理より
		\begin{align}
			\EQAX \vdash \tau = a \rarrow (\, \tau \in \Univ \rarrow a \in \Univ\, )
		\end{align}
		が成り立つので,(\refeq{fom:thm_V_is_the_whole_of_sets_3})と三段論法より
		\begin{align}
			\set{a},\ \EQAX \vdash \tau \in \Univ \rarrow a \in \Univ
		\end{align}
		となり,(\refeq{fom:thm_V_is_the_whole_of_sets_2})と三段論法より
		\begin{align}
			\set{a},\ \EXTAX,\EQAX,\COMAX \vdash a \in \Univ
		\end{align}
		が成り立つ.最後に演繹法則より
		\begin{align}
			\EXTAX,\EQAX,\COMAX \vdash \set{a} \rarrow a \in \Univ
		\end{align}
		が得られる.
		\QED
	\end{prf}
	
	\begin{screen}
		\begin{logicalthm}[同値関係の可換律]
		\label{logicalthm:commutative_law_of_equivalence_symbol}
			$A,B$を$\mathcal{L}$の文とするとき
			\begin{align}
				\vdash (A \lrarrow B) \rarrow (B \lrarrow A).
			\end{align}
		\end{logicalthm}
	\end{screen}
	
	\begin{sketch}
		論理積の除去規則より
		\begin{align}
			A \lrarrow B &\vdash A \rarrow B, 
			\label{fom:logicalthm_commutative_law_of_equivalence_symbol_1} \\
			A \lrarrow B &\vdash B \rarrow A
			\label{fom:logicalthm_commutative_law_of_equivalence_symbol_2}
		\end{align}
		となる.他方で論理積の導入規則より
		\begin{align}
			\vdash (B \rarrow A) \rarrow ((A \rarrow B) \rarrow (B \lrarrow A))
		\end{align}
		が成り立つので
		\begin{align}
			A \lrarrow B \vdash (B \rarrow A) \rarrow ((A \rarrow B) \rarrow (B \lrarrow A))
		\end{align}
		も成り立つ.これと(\refeq{fom:logicalthm_commutative_law_of_equivalence_symbol_1})
		との三段論法より
		\begin{align}
			A \lrarrow B \vdash (A \rarrow B) \rarrow (B \lrarrow A)
		\end{align}
		となり,(\refeq{fom:logicalthm_commutative_law_of_equivalence_symbol_2})
		との三段論法より
		\begin{align}
			A \lrarrow B \vdash B \lrarrow A
		\end{align}
		が得られる.
		\QED
	\end{sketch}
	
	\begin{screen}
		\begin{logicalthm}[同値関係の推移律]
		\label{logicalthm:transitive_law_of_equivalence_symbol}
			$A,B,C$を$\mathcal{L}$の文とするとき
			\begin{align}
				\vdash (A \lrarrow B) \rarrow ((B \lrarrow C) \rarrow 
				(A \lrarrow C)).
			\end{align}
		\end{logicalthm}
	\end{screen}
	
	\begin{sketch}
		論理積の除去法則より
		\begin{align}
			A \lrarrow B &\vdash A \rarrow B, \\
			A \lrarrow B &\vdash B \rarrow A
		\end{align}
		が成り立つので
		\begin{align}
			A \lrarrow B,\ B \lrarrow C &\vdash A \rarrow B, 
			\label{fom:transitive_law_of_equivalence_symbol_1} \\
			A \lrarrow B,\ B \lrarrow C &\vdash B \rarrow A
		\end{align}
		も成り立つし,対称的に
		\begin{align}
			A \lrarrow B,\ B \lrarrow C &\vdash B \rarrow C, 
			\label{fom:transitive_law_of_equivalence_symbol_2} \\
			A \lrarrow B,\ B \lrarrow C &\vdash C \rarrow B
		\end{align}
		も成り立つ.含意の推移律(推論法則\ref{logicalthm:transitive_law_of_implication})より
		\begin{align}
			\vdash (A \rarrow B) \rarrow ((B \rarrow C) \rarrow (A \rarrow C))
		\end{align}
		となるので,(\refeq{fom:transitive_law_of_equivalence_symbol_1})との三段論法より
		\begin{align}
			A \lrarrow B,\ B \lrarrow C \vdash (B \rarrow C) \rarrow (A \rarrow C)
		\end{align}
		が成り立ち,(\refeq{fom:transitive_law_of_equivalence_symbol_2})との三段論法より
		\begin{align}
			A \lrarrow B,\ B \lrarrow C \vdash A \rarrow C
			\label{fom:transitive_law_of_equivalence_symbol_3}
		\end{align}
		が成り立つ.同様にして
		\begin{align}
			A \lrarrow B,\ B \lrarrow C \vdash C \rarrow A
			\label{fom:transitive_law_of_equivalence_symbol_4}
		\end{align}
		も得られる.論理積の導入規則より
		\begin{align}
			\vdash (A \rarrow C) \rarrow ((C \rarrow A) \rarrow (A \lrarrow C))
		\end{align}
		が成り立つので,(\refeq{fom:transitive_law_of_equivalence_symbol_3})との三段論法より
		\begin{align}
			A \lrarrow B,\ B \lrarrow C \vdash (C \rarrow A) \rarrow (A \lrarrow C)
		\end{align}
		となり,(\refeq{fom:transitive_law_of_equivalence_symbol_4})との三段論法より
		\begin{align}
			A \lrarrow B,\ B \lrarrow C \vdash A \lrarrow C
		\end{align}
		となる.あとは演繹規則を二回適用すれば
		\begin{align}
			\vdash (A \lrarrow B) \rarrow ((B \lrarrow C) \rarrow (A \lrarrow C))
		\end{align}
		が得られる.
		\QED
	\end{sketch}
	
	\begin{screen}
		\begin{thm}[等号の推移律]\label{thm:transitive_law_of_equality}
			$a,b,c$を類とするとき
			\begin{align}
				\EXTAX,\EQAX \vdash a = b \rarrow (\, a = c \rarrow b = c\, ).
			\end{align}
		\end{thm}
	\end{screen}
	
	\begin{sketch}
		まずは
		\begin{align}
			a = b,\ a = c,\ \EQAX \vdash \forall x\, (\, x \in b \lrarrow x \in c\, )
		\end{align}
		を示したいので
		\begin{align}
			\tau \defeq \varepsilon x \negation (\, x \in b \lrarrow x \in c\, )
		\end{align}
		とおく($b,c$が$\lang{\varepsilon}$の項でなければ
		$x \in b \lrarrow x \in c$を書き換える).相等性公理より
		\begin{align}
			a = b,\ a = c,\ \EQAX \vdash a = b \rarrow (\, \tau \in a \rarrow \tau \in b\, )
		\end{align}
		が成り立つので,
		\begin{align}
			a = b,\ a = c,\ \EQAX \vdash a = b
			\label{fom:thm_transitive_law_of_equality_0}
		\end{align}
		との三段論法より
		\begin{align}
			a = b,\ a = c,\ \EQAX \vdash \tau \in a \rarrow \tau \in b
			\label{fom:thm_transitive_law_of_equality_1}
		\end{align}
		となる.同じく相等性公理より
		\begin{align}
			a = b,\ a = c,\ \EQAX \vdash a = b \rarrow b = a, \\
		\end{align}
		が成り立つので,(\refeq{fom:thm_transitive_law_of_equality_0})との三段論法より
		\begin{align}
			a = b,\ a = c,\ \EQAX \vdash b = a
		\end{align}
		となり,同様に相等性公理から
		\begin{align}
			a = b,\ a = c,\ \EQAX \vdash b = a \rarrow (\, \tau \in b \rarrow \tau \in a\, )
		\end{align}
		が成り立つので,三段論法より
		\begin{align}
			a = b,\ a = c,\ \EQAX \vdash \tau \in b \rarrow \tau \in a
			\label{fom:thm_transitive_law_of_equality_2}
		\end{align}
		となる.論理積の導入規則より
		\begin{align}
			a = b,\ a = c,\ \EQAX \vdash (\tau \in a \rarrow \tau \in b)
			\rarrow ((\tau \in b \rarrow \tau \in a) \rarrow 
			(\tau \in a \lrarrow \tau \in b))
		\end{align}
		が成り立つので,(\refeq{fom:thm_transitive_law_of_equality_1})との三段論法より
		\begin{align}
			a = b,\ a = c,\ \EQAX \vdash (\tau \in b \rarrow \tau \in a) \rarrow 
			(\tau \in a \lrarrow \tau \in b)
		\end{align}
		となり,(\refeq{fom:thm_transitive_law_of_equality_2})との三段論法より
		\begin{align}
			a = b,\ a = c,\ \EQAX \vdash \tau \in a \lrarrow \tau \in b
			\label{fom:thm_transitive_law_of_equality_4}
		\end{align}
		となる.対称的に
		\begin{align}
			a = b,\ a = c,\ \EQAX \vdash \tau \in a \lrarrow \tau \in c
			\label{fom:thm_transitive_law_of_equality_3}
		\end{align}
		も得られる.ここで含意の可換律
		(推論法則\ref{logicalthm:commutative_law_of_equivalence_symbol})より
		\begin{align}
			a = b,\ a = c,\ \EQAX \vdash (\, \tau \in a \lrarrow \tau \in b\, )
			\rarrow (\, \tau \in b \lrarrow \tau \in a\, ) 
		\end{align}
		が成り立つので,(\refeq{fom:thm_transitive_law_of_equality_4})との三段論法より
		\begin{align}
			a = b,\ a = c,\ \EQAX \vdash \tau \in b \lrarrow \tau \in a
			\label{fom:thm_transitive_law_of_equality_5}
		\end{align}
		となる.また含意の推移律
		(推論法則\ref{logicalthm:transitive_law_of_equivalence_symbol})より
		\begin{align}
			a = b,\ a = c,\ \EQAX \vdash (\, \tau \in b \lrarrow \tau \in a\, )
			\rarrow ((\, \tau \in a \lrarrow \tau \in c\, )
			\rarrow (\, \tau \in b \lrarrow \tau \in c\, )) 
		\end{align}
		が成り立つので,(\refeq{fom:thm_transitive_law_of_equality_5})との三段論法より
		\begin{align}
			a = b,\ a = c,\ \EQAX \vdash (\, \tau \in a \lrarrow \tau \in c\, )
			\rarrow (\, \tau \in b \lrarrow \tau \in c\, )
		\end{align}
		となり,(\refeq{fom:thm_transitive_law_of_equality_3})との三段論法より
		\begin{align}
			a = b,\ a = c,\ \EQAX \vdash \tau \in b \lrarrow \tau \in c
		\end{align}
		が得られる.全称記号の推論規則より
		\begin{align}
			a = b,\ a = c,\ \EQAX \vdash (\tau \in b \lrarrow \tau \in c)
			\rarrow \forall x\, (\, x \in b \lrarrow x \in c\, )
		\end{align}
		となるので,三段論法より
		\begin{align}
			a = b,\ a = c,\ \EQAX \vdash \forall x\, (\, x \in b \lrarrow x \in c\, )
		\end{align}
		となり,外延性公理より
		\begin{align}
			a = b,\ a = c,\ \EXTAX,\EQAX \vdash \forall x\, (\, x \in b \lrarrow x \in c\, )
			\rarrow b = c
		\end{align}
		となるので,三段論法より
		\begin{align}
			a = b,\ a = c,\ \EXTAX,\EQAX \vdash b = c
		\end{align}
		が得られる.
		\QED
	\end{sketch}
	
	\begin{itembox}[l]{等号の対称律と推移律について}
		本稿では等号の対称律
		\begin{align}
			a = b \rarrow b = a
		\end{align}
		を公理としたが,逆に推移律を公理にすれば
		\begin{align}
			\EXTAX,\EQAX \vdash a = b \rarrow b = a
		\end{align}
		が成立する.実際
		\begin{align}
			a = b,\ \EQAX &\vdash a = a \rarrow b = a, && \\
			\EXTAX &\vdash a = a, 
			&& (\mbox{定理\ref{thm:any_class_equals_to_itself}}), \\
			a = b,\ \EXTAX,\EQAX &\vdash b = a
			&& (\mbox{三段論法})
		\end{align}
		となる.つまり等号の対称律と推移律は外延性公理の下で同値なのである.
	\end{itembox}
	\section{空集合}
	\begin{screen}
		\begin{logicalthm}[分配された論理積の簡約]
		\label{logicalthm:contraction_law_of_distributed_injunctions}
			$A,B,C$を$\mathcal{L}$の文とするとき,
			\begin{align}
				\vdash (A \wedge C) \wedge (B \wedge C) \rarrow A \wedge B.
			\end{align}
		\end{logicalthm}
	\end{screen}
	
	\begin{sketch}
		論理積の除去規則より
		\begin{align}
			(A \wedge C) \wedge (B \wedge C) \vdash A \wedge C
		\end{align}
		となり,また同じく論理積の除去規則より
		\begin{align}
			(A \wedge C) \wedge (B \wedge C) &\vdash A \wedge C \rarrow A
		\end{align}
		となるので,三段論法より
		\begin{align}
			(A \wedge C) \wedge (B \wedge C) &\vdash A, 
			\label{fom:logicalthm_contraction_law_of_injunctions_1}
		\end{align}
		が従う.同様にして
		\begin{align}
			(A \wedge C) \wedge (B \wedge C) \vdash B
			\label{fom:logicalthm_contraction_law_of_injunctions_2}
		\end{align}
		も得られる.ここで論理積の導入規則より
		\begin{align}
			(A \wedge C) \wedge (B \wedge C) \vdash A \rarrow (B \rarrow A \wedge B)
		\end{align}
		が成り立つので,(\refeq{fom:logicalthm_contraction_law_of_injunctions_1})と
		(\refeq{fom:logicalthm_contraction_law_of_injunctions_2})との三段論法より
		\begin{align}
			(A \wedge C) \wedge (B \wedge C) \vdash A \wedge B
		\end{align}
		が出る.
		\QED
	\end{sketch}
	
	\begin{screen}
		\begin{dfn}[空集合]
			$\emptyset \defeq \Set{x}{x \neq x}$で定める類$\emptyset$を{\bf 空集合}\index{くうしゅうごう@空集合}{\bf (empty set)}と呼ぶ.
		\end{dfn}
	\end{screen}
	
	$x$が集合であれば
	\begin{align}
		x = x
	\end{align}
	が成り立つので,$\emptyset$に入る集合など存在しない.
	つまり$\emptyset$は丸っきり``空っぽ''なのである.
	さて,$\emptyset$は集合であるか否か,という問題を考える.
	当然これが``大きすぎる集まり''であるはずはないし,
	そもそも名前に``集合''と付いているのだから
	$\emptyset$は集合であるべきだと思われるのだが,
	実際にこれが集合であることを示すには少し骨が折れる.
	まずは置換公理と分出定理を拵えなくてはならない.
	
	\begin{screen}
		\begin{axm}[置換公理]
			$\varphi$を$\mathcal{L}$の式とし,
			$s,t$を$\varphi$に自由に現れる変項とし,
			$\varphi$に自由に現れる項は$s,t$のみであるとし,
			$x$は$\varphi$で$s$への代入について自由であり,
			$y,z,v$は$\varphi$で$t$への代入について自由であるとするとき,
			\begin{align}
				\REPAX \defarrow \forall x\, \forall y\, \forall z\, 
				(\, \varphi(x,y) \wedge \varphi(x,z)
				\rarrow y = z\, )
				\rarrow \forall a\, \exists u\, \forall v\,
				(\, v \in u \lrarrow \exists x\, (\, x \in a \wedge 
				\varphi(x,v)\, )\, ).
			\end{align}
		\end{axm}
	\end{screen}
	
	$\Set{x}{\varphi(x)}$は集合であるとは限らないが,
	集合$a$に対して
	\begin{align}
		a \cap \Set{x}{\varphi(x)}
	\end{align}
	なる類は当然$a$より``小さい集まり''なのだから,集合であってほしいものである.
	置換公理によってそのこと保証され,分出定理として知られている.
	
	\begin{screen}
		\begin{thm}[分出定理]\label{thm:axiom_of_separation}
			$\varphi$を$\mathcal{L}$の式とし,$x$を$\varphi$に自由に現れる変項とし,
			$\varphi$に自由に現れる項は$x$のみであるとする.このとき
			\begin{align}
				\EXTAX,\EQAX,\EQAXEP,\REPAX \vdash 
				\forall a\, \exists s\, \forall x\,
				(\, x \in s \lrarrow x \in a \wedge \varphi(x)\, ).
				\label{fom:thm_axiom_of_separation_0}
			\end{align}
		\end{thm}
	\end{screen}
	
	\begin{sketch}
		$y$を,$\varphi$の$x$への代入について自由である変項とする.
		そして$x$と$y$が自由に現れる式$\psi(x,y)$を
		\begin{align}
			x = y \wedge \varphi(x)
		\end{align}
		と設定する.
		\begin{description}
			\item[step1]
				まず
				\begin{align}
					\EXTAX,\EQAX \vdash \forall x\, \forall y\, \forall z\, 
					(\, \psi(x,y) \wedge \psi(x,z) \rarrow y = z\, )
					\label{fom:thm_axiom_of_separation_1}
				\end{align}
				が成り立つことを示す.これを見越して
				\begin{align}
					\tau &\defeq \varepsilon x \negation \forall y\, \forall z\, 
					(\, \psi(x,y) \wedge \psi(x,z) \rarrow y = z\, ), \\
					\sigma &\defeq \varepsilon y \negation \forall z\, 
					(\, \psi(\tau,y) \wedge \psi(\tau,z) \rarrow y = z\, ), \\
					\rho &\defeq \varepsilon z \negation 
					(\, \psi(\tau,\sigma) \wedge \psi(\tau,z) \rarrow \sigma = z\, )
				\end{align}
				とおく.$\psi(\tau,\sigma) \wedge \psi(\tau,\rho)$は縮約可能であって(
				推論法則\ref{logicalthm:contraction_law_of_distributed_injunctions})
				\begin{align}
					\vdash (\, \tau = \sigma \wedge \varphi(\tau)\, )
						\wedge (\, \tau = \rho \wedge \varphi(\tau)\, )
						\rarrow \tau = \sigma \wedge \tau = \rho
				\end{align}
				が成り立つので
				\begin{align}
					\psi(\tau,\sigma) \wedge \psi(\tau,\rho) 
					\vdash \tau = \sigma \wedge \tau = \rho
				\end{align}
				がとなり,さらに論理積の除去法則より
				\begin{align}
					\psi(\tau,\sigma) \wedge \psi(\tau,\rho) &\vdash \tau = \sigma, \\
					\psi(\tau,\sigma) \wedge \psi(\tau,\rho) &\vdash \tau = \rho
				\end{align}
				が出る.ここで等号の推移律(定理\ref{thm:transitive_law_of_equality})より
				\begin{align}
					\EXTAX,\EQAX \vdash \tau = \sigma \rarrow 
						(\, \tau = \rho \rarrow \sigma = \rho\, )
				\end{align}
				が成り立つので,三段論法を二回用いれば
				\begin{align}
					\psi(\tau,\sigma) \wedge \psi(\tau,\rho),\ \EXTAX,\EQAX 
					\vdash \sigma = \rho
				\end{align}
				が得られる.ゆえに演繹法則より
				\begin{align}
					\EXTAX,\EQAX \vdash \psi(\tau,\sigma) \wedge \psi(\tau,\rho)
					\rarrow \sigma = \rho
				\end{align}
				となり,全称記号の推論規則より
				\begin{align}
					\EXTAX,\EQAX &\vdash \forall z\, 
						(\, \psi(\tau,\sigma) \wedge \psi(\tau,z) 
						\rarrow \sigma = z\, ), \\
					\EXTAX,\EQAX &\vdash \forall y\, \forall z\, 
						(\, \psi(\tau,y) \wedge \psi(\tau,z) \rarrow y = z\, ), \\
					\EXTAX,\EQAX &\vdash \forall x\, \forall y\, \forall z\, 
						(\, \psi(x,y) \wedge \psi(x,z) \rarrow y = z\, )
				\end{align}
				が従う.
			
			\item[step2]
				置換公理より
				\begin{align}
					\REPAX \vdash \forall x\, \forall y\, \forall z\, 
					(\, \psi(x,y) \wedge \psi(x,z)
					\rarrow y = z\, )
					\rarrow \forall a\, \exists u\, \forall v\,
					(\, v \in u \lrarrow \exists x\, (\, x \in a \wedge 
					\psi(x,v)\, )\, )
				\end{align}
				が成り立つので,(\refeq{fom:thm_axiom_of_separation_1})との三段論法より
				\begin{align}
					\EXTAX,\EQAX,\REPAX \vdash \forall a\, \exists u\, \forall v\,
					(\, v \in u \lrarrow \exists x\, (\, x \in a \wedge 
					\psi(x,v)\, )\, )
					\label{fom:thm_axiom_of_separation_2}
				\end{align}
				が成立する.(\refeq{fom:thm_axiom_of_separation_1})を示したいので
				\begin{align}
					\alpha \defeq \varepsilon a \negation \exists s\, \forall x\,
					(\, x \in s \lrarrow x \in a \wedge \varphi(x)\, )
				\end{align}
				とおくと,全称記号の推論規則より
				\begin{align}
					\EXTAX,\EQAX,\REPAX \vdash &\forall a\, \exists u\, \forall v\,
					(\, v \in u \lrarrow \exists x\, (\, x \in a \wedge 
					\psi(x,v)\, )\, ) \\
					&\rarrow \exists u\, \forall v\,
					(\, v \in u \lrarrow \exists x\, (\, x \in \alpha \wedge 
					\psi(x,v)\, )\, )
				\end{align}
				となるので,(\refeq{fom:thm_axiom_of_separation_2})との三段論法より
				\begin{align}
					\EXTAX,\EQAX,\REPAX \vdash \exists u\, \forall v\,
					(\, v \in u \lrarrow \exists x\, (\, x \in \alpha \wedge 
					\psi(x,v)\, )\, )
					\label{fom:thm_axiom_of_separation_3}
				\end{align}
				が従う.ここで
				\begin{align}
					\zeta \defeq \varepsilon u\, \forall v\,
					(\, v \in u \lrarrow \exists x\, (\, x \in \alpha \wedge 
					\psi(x,v)\, )\, )
				\end{align}
				とおけば,量化子の推論規則と(\refeq{fom:thm_axiom_of_separation_2})との
				三段論法により
				\begin{align}
					\EXTAX,\EQAX,\REPAX \vdash \forall v\,
					(\, v \in \zeta \lrarrow \exists x\, (\, x \in \alpha \wedge 
					\psi(x,v)\, )\, )
					\label{fom:thm_axiom_of_separation_4}
				\end{align}
				が成り立つ.
			
			\item[step3]
				最後に
				\begin{align}
					\EXTAX,\EQAX,\EQAXEP,\REPAX \vdash \forall x\,
					(\, x \in \zeta \lrarrow x \in \alpha \wedge \varphi(x)\, )
					\label{fom:thm_axiom_of_separation_8}
				\end{align}
				となることを示す.いま
				\begin{align}
					\tau \defeq \varepsilon x \negation
					(\, x \in \zeta \lrarrow x \in \alpha \wedge \varphi(x)\, )
				\end{align}
				とおけば,(\refeq{fom:thm_axiom_of_separation_4})と全称記号の推論規則より
				\begin{align}
					\EXTAX,\EQAX,\REPAX \vdash 
					\tau \in \zeta \lrarrow \exists x\, (\, x \in \alpha \wedge 
					\psi(x,\tau)\, )
					\label{fom:thm_axiom_of_separation_5}
				\end{align}
				が従う.ゆえに
				\begin{align}
					\tau \in \zeta,\ \EXTAX,\EQAX,\REPAX \vdash
					\exists x\, (\, x \in \alpha \wedge \psi(x,\tau)\, )
				\end{align}
				となる.ここで
				\begin{align}
					\sigma \defeq \varepsilon x\, (\, x \in \alpha \wedge
					\psi(x,\tau)\, )
				\end{align}
				とおけば
				\begin{align}
					\tau \in \zeta,\ \EXTAX,\EQAX,\REPAX \vdash
					\sigma \in \alpha \wedge \psi(\sigma,\tau)
				\end{align}
				となるので,
				\begin{align}
					\tau \in \zeta,\ \EXTAX,\EQAX,\REPAX &\vdash \sigma \in \alpha, \\
					\tau \in \zeta,\ \EXTAX,\EQAX,\REPAX &\vdash \sigma = \tau, \\
					\tau \in \zeta,\ \EXTAX,\EQAX,\REPAX &\vdash \varphi(\sigma)
				\end{align}
				が従う.ところで相等性公理と代入原理
				(定理\ref{thm:the_principle_of_substitution})より
				\begin{align}
					\tau \in \zeta,\ \EXTAX,\EQAX,\REPAX &\vdash 
						\sigma = \tau \rarrow (\, \sigma \in \alpha \rarrow
						\tau \in \alpha\, ), \\
					\tau \in \zeta,\ \EXTAX,\EQAX,\EQAXEP,\REPAX &\vdash
						\sigma = \tau \rarrow (\, \varphi(\sigma) \rarrow
						\varphi(\tau)\, ), \\
				\end{align}
				が成り立つので,三段論法より
				\begin{align}
					\tau \in \zeta,\ \EXTAX,\EQAX,\REPAX &\vdash \tau \in \alpha, \\
					\tau \in \zeta,\ \EXTAX,\EQAX,\EQAXEP,\REPAX &\vdash \varphi(\tau)
				\end{align}
				が従い
				\begin{align}
					\tau \in \zeta,\ \EXTAX,\EQAX,\EQAXEP,\REPAX \vdash
					\tau \in \alpha \wedge \varphi(\tau)
				\end{align}
				となる.以上で
				\begin{align}
					\EXTAX,\EQAX,\EQAXEP,\REPAX \vdash \tau \in \zeta \rarrow
					\tau \in \alpha \wedge \varphi(\tau)
					\label{fom:thm_axiom_of_separation_6}
				\end{align}
				が得られた.逆に定理\ref{thm:any_class_equals_to_itself}と併せて
				\begin{align}
					\tau \in \alpha \wedge \varphi(\tau),\ 
					\EXTAX,\EQAX,\EQAXEP,\REPAX \vdash
					\tau \in \alpha \wedge (\, \tau = \tau \wedge \varphi(\tau)\, )
				\end{align}
				が成り立つので,存在記号の推論規則より
				\begin{align}
					\tau \in \alpha \wedge \varphi(\tau),\ 
					\EXTAX,\EQAX,\EQAXEP,\REPAX \vdash
					\exists x\, (\, x \in \alpha \wedge \psi(x,\tau)\, )
				\end{align}
				となる.他方で(\refeq{fom:thm_axiom_of_separation_5})より
				\begin{align}
					\EXTAX,\EQAX,\REPAX \vdash 
					\exists x\, (\, x \in \alpha \wedge 
					\psi(x,\tau)\, ) \rarrow \tau \in \zeta
				\end{align}
				が成り立つので,三段論法より
				\begin{align}
					\tau \in \alpha \wedge \varphi(\tau),\ 
					\EXTAX,\EQAX,\EQAXEP,\REPAX \vdash \tau \in \zeta
				\end{align}
				が従う.以上で
				\begin{align}
					\EXTAX,\EQAX,\EQAXEP,\REPAX \vdash 
					\tau \in \alpha \wedge \varphi(\tau) \rarrow \tau \in \zeta
					\label{fom:thm_axiom_of_separation_7}
				\end{align}
				も得られた.(\refeq{fom:thm_axiom_of_separation_6})と
				(\refeq{fom:thm_axiom_of_separation_7})および存在記号の推論規則より
				(\refeq{fom:thm_axiom_of_separation_8})が出る.
				すると存在記号の推論規則より
				\begin{align}
					\EXTAX,\EQAX,\EQAXEP,\REPAX \vdash \exists s\, \forall x\,
					(\, x \in s \lrarrow x \in \alpha \wedge \varphi(x)\, )
				\end{align}
				となり,全称記号の推論規則より
				\begin{align}
					\EXTAX,\EQAX,\EQAXEP,\REPAX \vdash 
					\forall a\, \exists s\, \forall x\,
					(\, x \in s \lrarrow x \in a \wedge \varphi(x)\, )
				\end{align}
				が従う.
				\QED
		\end{description}
	\end{sketch}
	
	\begin{screen}
		\begin{thm}[$\emptyset$は集合]\label{thm:emptyset_is_a_set}
			$\emptyset$は集合である:
			\begin{align}
				\set{\emptyset}.
			\end{align}
		\end{thm}
	\end{screen}
	
	\begin{sketch}
		分出定理より
		\begin{align}
			\forall z\, \exists y\, \forall x\,
			(\, x \in y \lrarrow x \in z \wedge x \neq x\, )
			\label{fom:thm_emptyset_is_a_set_1}
		\end{align}
		が成立するが,この式から
		\begin{align}
			\exists y\, \forall x\, (\, x \in y \lrarrow x \neq x\, )
			\label{fom:thm_emptyset_is_a_set_2}
		\end{align}
		を示せる.これはすなわち$\emptyset$が集合であるということを示唆する.
		$\zeta$を勝手な$\varepsilon$項として,後々の便宜のために
		\begin{align}
			\sigma &\defeq \varepsilon y\, \forall x\,
			(\, x \in y \lrarrow x \in \zeta \wedge x \neq x\, ), \\
			\tau &\defeq \varepsilon x \negation
			(\, x \in \sigma \lrarrow x \neq x\, )
		\end{align}
		とおけば,(\refeq{fom:thm_emptyset_is_a_set_1})より
		\begin{align}
			\tau \in \sigma \lrarrow \tau \in \zeta \wedge \tau \neq \tau
		\end{align}
		が成立する.論理和の規則より
		\begin{align}
			\tau \in \zeta \wedge \tau \neq \tau \rarrow \tau \neq \tau
		\end{align}
		が満たされるので,まずは
		\begin{align}
			\tau \in \sigma \rarrow \tau \neq \tau
		\end{align}
		が得られる.また
		\begin{align}
			\tau = \tau
		\end{align}
		は正しいので,
		\begin{align}
			\tau = \tau \rarrow (\, \tau \notin \sigma \rarrow
			\tau = \tau\, )
		\end{align}
		と併せて
		\begin{align}
			\tau \notin \sigma \rarrow \tau = \tau
		\end{align}
		が成り立ち,対偶を取れば
		\begin{align}
			\tau \neq \tau \rarrow \tau \in \sigma
		\end{align}
		も得られる.ゆえに
		\begin{align}
			\forall x\, (\, x \in \sigma \lrarrow x \neq x\, )
		\end{align}
		が得られ,(\refeq{fom:thm_emptyset_is_a_set_2})が従う.
		\QED
	\end{sketch}
	
	\begin{screen}
		\begin{thm}[空集合は$\mathcal{L}$のいかなる対象も要素に持たない]\label{thm:emptyset_has_nothing}
			\begin{align}
				\forall x\, (\, x \notin \emptyset\, ).
			\end{align}
		\end{thm}
	\end{screen}
	
	\begin{sketch}
		$\tau$を$\mathscr{L}$の対象とするとき,類の公理より
		\begin{align}
			\tau \in \emptyset \rarrow \tau \neq \tau
		\end{align}
		が成り立つから,対偶を取れば
		\begin{align}
			\tau = \tau \rarrow \tau \notin \emptyset
		\end{align}
		が成り立つ(推論法則\ref{thm:contraposition_is_true}).定理\ref{thm:any_class_equals_to_itself}より
		\begin{align}
			\tau = \tau
		\end{align}
		は正しいので,三段論法より
		\begin{align}
			\tau \notin \emptyset
		\end{align}
		が成り立つ.そして$\tau$の任意性より
		\begin{align}
			\forall x\, (\, x \notin \emptyset\, )
		\end{align}
		が得られる.
		\QED
	\end{sketch}
	
	\begin{screen}
		\begin{thm}[$\mathcal{L}$のいかなる対象も要素に持たない類は空集合に等しい]
		\label{thm:uniqueness_of_emptyset}
			$a$を類とするとき次が成り立つ:
			\begin{align}
				\forall x\, (\, x \notin a\, ) \lrarrow a = \emptyset.
			\end{align}
		\end{thm}
	\end{screen}
	
	\begin{prf}
		$a$を類として$\forall x\, (\, x \notin a\, )$が成り立っていると仮定する.このとき
		$\tau$を$\mathcal{L}$の任意の対象とすれば
		\begin{align}
			\tau \notin a \vee \tau \in \emptyset
		\end{align}
		と
		\begin{align}
			\tau \notin \emptyset \vee \tau \in a
		\end{align}
		が共に成り立つので,推論法則\ref{logicalthm:rule_of_inference_3}より
		\begin{align}
			\tau \in a \rarrow \tau \in \emptyset
		\end{align}
		と
		\begin{align}
			\tau \in \emptyset \rarrow \tau \in a
		\end{align}
		が共に成り立つ.よって
		\begin{align}
			\tau \in a \lrarrow \tau \in \emptyset
		\end{align}
		が成立し,$\tau$の任意性と推論法則\ref{logicalthm:fundamental_law_of_universal_quantifier}から
		\begin{align}
			\forall x\, (\, x \in a \lrarrow x \in \emptyset\, )
		\end{align}
		が得られる.ゆえに外延性の公理より
		\begin{align}
			a = \emptyset
		\end{align}
		が成立し,演繹法則より
		\begin{align}
			\forall x\, (\, x \notin a\, ) \rarrow a = \emptyset
		\end{align}
		が得られる.逆に
		\begin{align}
			a = \emptyset
		\end{align}
		が成り立っていると仮定する.ここで$\chi$を$\mathcal{L}$の任意の対象とすれば,
		相等性の公理より
		\begin{align}
			\chi \in a \rarrow \chi \in \emptyset
		\end{align}
		が成立するので,対偶を取れば
		\begin{align}
			\chi \notin \emptyset \rarrow \chi \notin a
		\end{align}
		が成り立つ.定理\ref{thm:emptyset_has_nothing}より
		\begin{align}
			\chi \notin \emptyset
		\end{align}
		が満たされているので,三段論法より
		\begin{align}
			\chi \notin a
		\end{align}
		が成立し,$\chi$の任意性と推論法則\ref{logicalthm:fundamental_law_of_universal_quantifier}より
		\begin{align}
			\forall x\, (\, x \notin a\, )
		\end{align}
		が成立する.ここに演繹法則を適用して
		\begin{align}
			a = \emptyset \rarrow \forall x\, (\, x \notin a\, )
		\end{align}
		も得られる.
		\QED
	\end{prf}
	
	\begin{screen}
		\begin{thm}[空集合はいかなる類も要素に持たない]
		\label{thm:emptyset_does_not_contain_any_class}
			$a,b$を類とするとき次が成り立つ:
			\begin{align}
				b = \emptyset \rarrow a \notin b.
			\end{align}
		\end{thm}
	\end{screen}
	
	\begin{prf}
		いま$a \in b$が成り立っていると仮定する.このとき要素の公理と三段論法より
		\begin{align}
			\set{a}
		\end{align}
		が成立する.ここで
		\begin{align}
			\tau \defeq \varepsilon x\, (\, a = x\, )
		\end{align}
		とおけば,存在記号に関する規則から
		\begin{align}
			a = \tau
		\end{align}
		が成り立つので,相等性の公理より
		\begin{align}
			\tau \in b
		\end{align}
		が従い,存在記号に関する規則より
		\begin{align}
			\exists x\, (\, x \in b\, )
		\end{align}
		が成り立つ.よって演繹法則から
		\begin{align}
			a \in b \rarrow \exists x\, (\, x \in b\, )
		\end{align}
		が成り立つ.この対偶を取り推論法則\ref{logicalthm:De_Morgan_law_for_quantifiers}を適用すれば
		\begin{align}
			\forall x\, (\, x \notin b\, ) \rarrow a \notin b
		\end{align}
		が得られる.定理\ref{thm:uniqueness_of_emptyset}より
		\begin{align}
			b = \emptyset \rarrow \forall x\, (\, x \notin b\, )
		\end{align}
		も正しいので,含意の推移律から
		\begin{align}
			b = \emptyset \rarrow a \notin b
		\end{align}
		が得られる.
		\QED
	\end{prf}
	
	\begin{screen}
		\begin{dfn}[部分類]
			$a,b$を$\mathcal{L}'$の項とするとき,
			\begin{align}
				a \subset b \overset{\mathrm{def}}{\lrarrow}
				\forall x\ (\ x \in a \rarrow x \in b\ )
			\end{align}
			と定める.式$a \subset b$を``$a$は$b$の{\bf 部分類}\index{ぶぶんるい@部分類}{\bf (subclass)}である''
			と翻訳し,特に$a$が集合である場合は``$a$は$b$の{\bf 部分集合}\index{ぶぶんしゅうごう@部分集合}{\bf (subset)}である''と翻訳する.
			また次の記号も定める:
			\begin{align}
				a \subsetneq b \defarrow a \subset b \wedge a \neq b.
			\end{align}
		\end{dfn}
	\end{screen}
	
	空虚な真の一例として次の結果を得る.
	
	\begin{screen}
		\begin{thm}[空集合は全ての類に含まれる]\label{thm:emptyset_if_a_subset_of_every_class}
			$a$を類とするとき次が成り立つ:
			\begin{align}
				\emptyset \subset a.
			\end{align}
		\end{thm}
	\end{screen}
	
	\begin{prf}
		$a$を類とする.$\tau$を$\mathcal{L}$の任意の対象とすれば
		\begin{align}
			\tau \notin \emptyset
		\end{align}
		が成り立つから,推論規則\ref{logicalaxm:fundamental_rules_of_inference}を適用して
		\begin{align}
			\tau \notin \emptyset \vee \tau \in a
		\end{align}
		が成り立つ.従って
		\begin{align}
			\tau \in \emptyset \rarrow \tau \in a
		\end{align}
		が成り立ち,$\tau$の任意性と推論法則\ref{logicalthm:fundamental_law_of_universal_quantifier}より
		\begin{align}
			\forall x\, (\, x \in \emptyset \rarrow x \in a\, )
		\end{align}
		が成立する.
		\QED
	\end{prf}
	
	$a \subset b$とは$a$に属する全ての``$\mathcal{L}$の対象''は$b$に属するという定義であったが,
	要素となりうる類は集合であるという公理から,$a$に属する全ての``類''もまた$b$に属する.
	
	\begin{screen}
		\begin{thm}[類はその部分類に属する全ての類を要素に持つ]\label{thm:subclass_contains_all_elements}
			$a,b,c$を類とすれば次が成り立つ:
			\begin{align}
				a \subset b \rarrow (\, c \in a \rarrow c \in b\, ).
			\end{align}
		\end{thm}
	\end{screen}
	
	\begin{prf}	
		いま$a \subset b$が成り立っているとする.このとき
		\begin{align}
			c \in a
		\end{align}
		が成り立っていると仮定すれば,要素の公理より
		\begin{align}
			\set{c}
		\end{align}
		が成り立つ.ここで
		\begin{align}
			\tau \defeq \varepsilon x\, (\, c=x\, )
		\end{align}
		とおくと
		\begin{align}
			c = \tau
		\end{align}
		が成り立つので,相等性の公理より
		\begin{align}
			\tau \in a
		\end{align}
		が成り立ち,$a \subset b$と推論法則\ref{logicalthm:fundamental_law_of_universal_quantifier}から
		\begin{align}
			\tau \in b
		\end{align}
		が従う.再び相等性の公理を適用すれば
		\begin{align}
			c \in b
		\end{align}
		が成り立つので,演繹法則より,$a \subset b$が成り立っている下で
		\begin{align}
			c \in a \rarrow c \in b
		\end{align}
		が成立する.再び演繹法則を適用すれば定理の主張が得られる.
		\QED
	\end{prf}
	
	宇宙$\Univ$は類の一つであった.当然のようであるが,それは最大の類である.
	\begin{screen}
		\begin{thm}[$\Univ$は最大の類である]
			$a$を類とするとき次が成り立つ:
			\begin{align}
				a \subset \Univ.
			\end{align}
		\end{thm}
	\end{screen}
	
	\begin{prf}
		$\tau$を$\mathcal{L}$の任意の対象とすれば,定理\ref{thm:any_class_equals_to_itself}と類の公理より
		\begin{align}
			\tau \in \Univ
		\end{align}
		が成立するので,推論規則\ref{logicalaxm:fundamental_rules_of_inference}より
		\begin{align}
			\tau \notin a \vee \tau \in \Univ
		\end{align}
		が成立する.このとき推論法則\ref{logicalthm:rule_of_inference_3}より
		\begin{align}
			\tau \in a \rarrow \tau \in \Univ
		\end{align}
		が成立し,$\tau$の任意性と推論法則\ref{logicalthm:fundamental_law_of_universal_quantifier}から
		\begin{align}
			\forall x\, (\, x \in a \rarrow x \in \Univ\, )
		\end{align}
		が従う.
		\QED
	\end{prf}
	
	\begin{screen}
		\begin{thm}[互いに互いの部分類となる類同士は等しい]\label{thm:mutually_sub_classes_are_equivalent}
			$a,b$を類とするとき次が成り立つ:
			\begin{align}
				a \subset b \wedge b \subset a \lrarrow a = b.
			\end{align}
		\end{thm}
	\end{screen}
	
	\begin{sketch}
		$a \subset b \wedge b \subset a$が成り立っていると仮定する.
		このとき$\tau$を$\mathcal{L}$の任意の対象とすれば,
		$a \subset b$と推論法則\ref{logicalthm:fundamental_law_of_universal_quantifier}より
		\begin{align}
			\tau \in a \rarrow \tau \in b
		\end{align}
		が成立し,$b \subset a$と推論法則\ref{logicalthm:fundamental_law_of_universal_quantifier}より
		\begin{align}
			\tau \in b \rarrow \tau \in a
		\end{align}
		が成立するので,
		\begin{align}
			\tau \in a \lrarrow \tau \in b
		\end{align}
		が成り立つ.$\tau$の任意性と推論法則\ref{logicalthm:fundamental_law_of_universal_quantifier}および
		外延性の公理より
		\begin{align}
			a = b
		\end{align}
		が出るので,演繹法則より
		\begin{align}
			a \subset b \wedge b \subset a \rarrow a = b
		\end{align}
		が得られる.逆に$a = b$が満たされていると仮定するとき,$\tau$を$\mathcal{L}$の任意の対象とすれば
		\begin{align}
			\tau \in a \rarrow \tau \in b
		\end{align}
		と
		\begin{align}
			\tau \in b \rarrow \tau \in a
		\end{align}
		が共に成り立つ. よって推論法則\ref{logicalthm:fundamental_law_of_universal_quantifier}より
		\begin{align}
			a \subset b
		\end{align}
		と
		\begin{align}
			b \subset a
		\end{align}
		が共に従う.よって演繹法則より
		\begin{align}
			a = b \rarrow a \subset b \wedge b \subset a
		\end{align}
		も得られる.
		\QED
	\end{sketch}
	
	\monologue{
		定理\ref{thm:subclass_contains_all_elements}と定理\ref{thm:mutually_sub_classes_are_equivalent}より,
			類$a,b$が$a = b$を満たすならば,$a$と$b$は要素に持つ$\mathcal{L}$の対象のみならず,
			要素に持つ類までも一致するのですね.
	}
	
\section{順序型について}
	$(A,R)$を整列集合とするとき,
	\begin{align}
		x \longmapsto 
		\begin{cases}
			\min{A \backslash \ran{x}} & \mbox{if } \ran{x} \subsetneq A \\
			A & \mbox{o.w.} \\
		\end{cases}
	\end{align}
	なる写像$G$に対して
	\begin{align}
		\forall \alpha\, F(\alpha) = G(\rest{F}{\alpha})
	\end{align}
	なる写像$F$を取り
	\begin{align}
		\alpha \defeq \min{\Set{\alpha \in \ON}{F(\alpha) = A}}
	\end{align}
	とおけば,$\alpha$は$(A,R)$の順序型.
	
\section{超限再帰について}
	$\Univ$上の写像$G$が与えられたら,
	\begin{align}
		F \defeq \Set{(\alpha,x)}{\ord{\alpha} \wedge
		\exists f\, \left(\, f \fon \alpha \wedge
		\forall \beta \in \alpha\, \left(\, f(\beta) = G(\rest{f}{\beta})\, \right)
		\wedge x = G(f)\, \right)}
	\end{align}
	により$F$を定めれば
	\begin{align}
		\forall \alpha\, F(\alpha) = G(\rest{F}{\alpha})
	\end{align}
	が成立する.
	
	\begin{screen}
		任意の順序数$\alpha$および$\alpha$上の写像$f$と$g$に対して,
		\begin{align}
			\forall \beta \in \alpha\,
			\left(\, f(\beta) = G(\rest{f}{\beta})\, \right)
		\end{align}
		かつ
		\begin{align}
			\forall \beta \in \alpha\,
			\left(\, g(\beta) = G(\rest{g}{\beta})\, \right)
		\end{align}
		ならば$f = g$である.
	\end{screen}
	
	まず
	\begin{align}
		f(0) = G(\rest{f}{0}) = G(0) = G(\rest{g}{0}) = g(0)
	\end{align}
	が成り立つ.また
	\begin{align}
		\forall \delta \in \beta\, \left(\, 
		\delta \in \alpha \rarrow f(\delta) = g(\delta)\, \right)
	\end{align}
	ならば,$\beta \in \alpha$であるとき
	\begin{align}
		\rest{f}{\beta} = \rest{g}{\beta}
	\end{align}
	となるので
	\begin{align}
		\beta \in \alpha \rarrow f(\beta) = g(\beta)
	\end{align}
	が成り立つ.ゆえに
	\begin{align}
		f = g
	\end{align}
	が得られる.
	
	\begin{screen}
		任意の順序数$\alpha$に対して,$\alpha$上の写像$f$で
		\begin{align}
			\forall \beta \in \alpha\, \left(\, 
			f(\beta) = G(\rest{f}{\beta})\, \right)
		\end{align}
		を満たすものが取れる.
	\end{screen}
	
	$\alpha = 0$のとき$f \defeq 0$とすればよい.$\alpha$の任意の要素$\beta$に対して
	\begin{align}
		g \fon \beta \wedge \forall \gamma\in \beta\, \left(\, 
		g(\gamma) = G(\rest{g}{\gamma})\, \right)
	\end{align}
	なる$g$が存在するとき,
	\begin{align}
		f \defeq \Set{(\beta,x)}{\beta \in \alpha \wedge
		\exists g\, \left(\, g \fon \beta \wedge
		\forall \gamma \in \beta\, \left(\, g(\gamma) = G(\rest{g}{\gamma})\, \right)
		\wedge x = G(g)\, \right)}
	\end{align}
	と定めれば,$f$は$\alpha$上の写像であって
	\begin{align}
		\forall \beta \in \alpha\, \left(\, 
		f(\beta) = G(\rest{f}{\beta})\, \right)
	\end{align}
	を満たす.
	
	\begin{screen}
		任意の順序数$\alpha$に対して$F(\alpha) = G(\rest{F}{\alpha})$が成り立つ.
	\end{screen}
	
	$\alpha = 0$ならば,$0$上の写像は$0$のみなので
	\begin{align}
		F(0) = G(0) = G(\rest{F}{0})
	\end{align}
	である.
	\begin{align}
		\forall \beta \in \alpha\, F(\beta) = G(\rest{F}{\beta})
	\end{align}
	が成り立っているとき,
	\begin{align}
		\forall \beta \in \alpha\, f(\beta) = G(\rest{f}{\beta})
	\end{align}
	を満たす$\alpha$上の写像$f$を取れば,前の一意性より
	\begin{align}
		f = \rest{F}{\alpha}
	\end{align}
	が成立する.よって
	\begin{align}
		F(\alpha) = G(f) = G(\rest{F}{\alpha})
	\end{align}
	となる.
	\QED
	
\section{自然数の全体について}
	$\Natural$を
	\begin{align}
		\Natural \defeq \Set{\beta}{\mbox{$\alpha \leq \beta$である$\alpha$は
		$0$であるか後続型順序数}}
	\end{align}
	によって定めれば,無限公理より
	\begin{align}
		\set{\Natural}
	\end{align}
	である.また$\ord{\Natural}$と$\limo{\Natural}$も証明できるはず.
	$\Natural$が最小の極限数であることは$\Natural$を定義した論理式より従う.
	\section{対}
	$a$と$b$を類とするとき,$a$か$b$の少なくとも一方に等しい集合の全体,つまり
	\begin{align}
		a = x \vee b = x
	\end{align}
	を満たす全ての集合$x$を集めたものを$a$と$b$の対と呼び
	\begin{align}
		\{a,b\}
	\end{align}
	と書く.解釈としては``$a$と$b$のみを要素とする類''のことであり,当然$a$が集合であるならば
	\begin{align}
		a \in \{a,b\}
	\end{align}
	が成立する.しかし$a$と$b$が共に真類であるときは,いかなる集合も$a$にも$b$にも等しくないため
	\begin{align}
		\{a,b\} = \emptyset
	\end{align}
	となる.以上が大雑把な対の説明である.
	
	\begin{screen}
		\begin{dfn}[対]
			$x,y$を$\mathcal{L}$の項とし,$z$を$x$にも$y$にも自由に現れない変項とするとき,
			\begin{align}
				\{x,y\} \defeq \Set{z}{x = z \vee y = z}
			\end{align}
			で$\{x,y\}$を定義し,これを$x$と$y$の{\bf 対}\index{つい@対}{\bf (pair)}と呼ぶ.
			特に$\{x,x\}$を$\{x\}$と書く.
		\end{dfn}
	\end{screen}
	
	上の定義では省略したが,$x$や$y$が内包項である場合は$z = x \vee z = y$を
	$\lang{\varepsilon}$の式に書き換えてから$\{x,y\}$を定めるのである.つまり
	\begin{align}
		\varphi \defarrow x = z \vee y = z
	\end{align}
	とおけば,$\varphi$を$\lang{\varepsilon}$の式に書き換えた式$\hat{\varphi}$によって
	\begin{align}
		\{x,y\} \defeq \Set{z}{\hat{\varphi}(z)}
	\end{align}
	と定めるのである.たとえば$a$や$b$を類として
	\begin{align}
		\varphi \defarrow a = z \vee b = z
	\end{align}
	とおけば,
	\begin{align}
		\COMAX \vdash \forall z\, (\, z \in \{a,b\} \lrarrow \hat{\varphi}(z)\, )
	\end{align}
	が成立するし,同時に定理\ref{thm:equivalent_formula_rewriting_1}と
	定理\ref{thm:equivalent_formula_rewriting_2}より
	\begin{align}
		\EXTAX,\EQAX,\COMAX \vdash 
		\forall z\, (\, \hat{\varphi}(z) \lrarrow \varphi(z)\, )
	\end{align}
	も成り立つので
	\begin{align}
		\EXTAX,\EQAX,\COMAX \vdash 
		\forall z\, (\, z \in \{a,b\} \lrarrow a = z \vee b = z\, )
	\end{align}
	が得られる.
	
	\begin{screen}
		\begin{thm}[対は表示されている要素しか持たない]
		\label{thm:pair_members_are_exactly_the_given_two}
			$a$と$b$を類とするとき次が成立する:
			\begin{align}
				\EXTAX,\EQAX,\COMAX \vdash 
				\forall x\, (\, x \in \{a,b\} \lrarrow a = x \vee b = x\, ).
			\end{align}
		\end{thm}
	\end{screen}
	
	$\ELEAX$を加えれば次が得られる.
	
	\begin{screen}
		\begin{thm}[対の要素は表示されている要素の一方には等しい]
		\label{cor:pair_members_are_exactly_the_given_two}
			$a,b,c$を類とするとき次が成立する:
			\begin{align}
				\EXTAX,\EQAX,\COMAX,\ELEAX \vdash 
				c \in \{a,b\} \rarrow a = c \vee b = c.
			\end{align}
		\end{thm}
	\end{screen}
	
	\begin{sketch}
		要素の公理より
		\begin{align}
			c \in \{a,b\},\ \ELEAX \vdash \set{c}
		\end{align}
		が成り立つので
		\begin{align}
			\tau \defeq \varepsilon s\, (\, c = s\, )
		\end{align}
		とおけば
		\begin{align}
			c \in \{a,b\},\ \ELEAX \vdash c = \tau
			\label{fom:pair_members_are_exactly_the_given_two_2}
		\end{align}
		となる.$\tau$に対しては定理\ref{thm:pair_members_are_exactly_the_given_two}より 
		\begin{align}
			\EXTAX,\EQAX,\COMAX \vdash 
			\tau \in \{a,b\} \rarrow a = \tau \vee b = \tau
		\end{align}
		が成り立つが,ここで(\refeq{fom:pair_members_are_exactly_the_given_two_2})より
		\begin{align}
			c \in \{a,b\},\ \EQAX,\ELEAX \vdash \tau \in \{a,b\}
		\end{align}
		となるので
		\begin{align}
			c \in \{a,b\},\ \EXTAX,\EQAX,\COMAX,\ELEAX \vdash a = \tau \vee b = \tau
		\end{align}
		が従い,代入原理(定理\ref{thm:the_principle_of_substitution})と
		(\refeq{fom:pair_members_are_exactly_the_given_two_2})より
		\begin{align}
			c \in \{a,b\},\ \EXTAX,\EQAX,\COMAX,\ELEAX \vdash a = c \vee b = c
		\end{align}
		が得られる.
		\QED
	\end{sketch}
	
	この逆,つまり
	\begin{align}
		a = c \vee b = c \rarrow c \in \{a,b\}
	\end{align}
	は一般には成立しない.実際$a,b$が共に真類であるときは
	\begin{align}
		\{a,b\} = \emptyset
	\end{align}
	となるためである(定理\ref{thm:pair_of_proper_classes_is_emptyset}).
	
	\begin{screen}
		\begin{thm}[表示の順番を入れ替えても対は等しい]
		\label{thm:commutative_law_of_pairs}
			$a$と$b$を類とするとき
			\begin{align}
				\EXTAX,\EQAX,\COMAX \vdash \{a,b\} = \{b,a\}.
			\end{align}
		\end{thm}
	\end{screen}
	
	\begin{sketch}
		いま
		\begin{align}
			\tau \defeq \varepsilon x \negation (\, x \in \{a,b\} \lrarrow x \in \{b,a\}\, )
		\end{align}
		とおく(必要に応じて$x \in \{a,b\} \lrarrow x \in \{b,a\}$は$\lang{\varepsilon}$の
		式に書き換える).
		定理\ref{thm:pair_members_are_exactly_the_given_two}より
		\begin{align}
			\EXTAX,\EQAX,\COMAX \vdash
			\tau \in \{a,b\} \rarrow a = \tau \vee b = \tau
		\end{align}
		が成り立つので,演繹定理の逆より
		\begin{align}
			\tau \in \{a,b\},\ \EXTAX,\EQAX,\COMAX \vdash a = \tau \vee b = \tau
		\end{align}
		となる.また論理和の可換律
		(推論法則\ref{logicalthm:commutative_law_of_disjunction})より
		\begin{align}
			\tau \in \{a,b\},\ \EXTAX,\EQAX,\COMAX \vdash b = \tau \vee a = \tau
		\end{align}
		が成り立ち,定理\ref{thm:pair_members_are_exactly_the_given_two}より
		\begin{align}
			\tau \in \{a,b\},\ \EXTAX,\EQAX,\COMAX \vdash \tau \in \{b,a\}
		\end{align}
		が従う.そして演繹定理より
		\begin{align}
			\EXTAX,\EQAX,\COMAX \vdash \tau \in \{a,b\} \rarrow \tau \in \{b,a\}
		\end{align}
		が得られる.$a$と$b$を入れ替えれば
		\begin{align}
			\EXTAX,\EQAX,\COMAX \vdash \tau \in \{b,a\} \rarrow \tau \in \{a,b\}
		\end{align}
		が得られるので,論理積の導入より
		\begin{align}
			\EXTAX,\EQAX,\COMAX \vdash \tau \in \{a,b\} \lrarrow \tau \in \{b,a\}
		\end{align}
		が成り立ち,全称の導出(推論法則\ref{logicalthm:derivation_of_universal_by_epsilon})より
		\begin{align}
			\EXTAX,\EQAX,\COMAX \vdash \forall x\, (\, x \in \{a,b\} \lrarrow x \in \{b,a\}\, )
		\end{align}
		となり,外延性公理より
		\begin{align}
			\EXTAX,\EQAX,\COMAX \vdash \{a,b\} = \{b,a\}
		\end{align}
		が従う.
		\QED
	\end{sketch}
		
	\begin{screen}
		\begin{axm}[対の公理] 次の式を$\PAIAX$により参照する:
			\begin{align}
				\forall x\, \forall y\, \exists p\, \forall z\, 
				(\, x = z \vee y = z \lrarrow z \in p\, ).
			\end{align}
		\end{axm}
	\end{screen}
	
	\begin{screen}
		\begin{thm}[集合の対は集合である]
		\label{thm:pair_of_sets_is_a_set}
			$a$と$b$を類とするとき
			\begin{align}
				\EXTAX,\EQAX,\COMAX,\PAIAX \vdash 
				\set{a} \wedge \set{b} \rarrow \set{\{a,b\}}.
			\end{align}
		\end{thm}
	\end{screen}
	
	\begin{sketch}\mbox{}
		\begin{description}
			\item[step1]
				論理積の除去より
				\begin{align}
					\set{a} \wedge \set{b} &\vdash \exists x\, (\, a = x\, ), \\
					\set{a} \wedge \set{b} &\vdash \exists x\, (\, b = x\, )
				\end{align}
				が成り立つので,
				\begin{align}
					\tau &\defeq \varepsilon x\, (\, a = x\, ), \\
					\sigma &\defeq \varepsilon x\, (\, b = x\, )
				\end{align}
				とおけば
				\begin{align}
					\set{a} \wedge \set{b} &\vdash a = \tau, 
					\label{fom:pair_of_sets_is_a_set_1} \\
					\set{a} \wedge \set{b} &\vdash b = \sigma
				\end{align}
				が成り立つ.対の公理より$\tau$と$\sigma$に対しては
				\begin{align}
					\PAIAX \vdash \exists p\, \forall z\, 
						(\, \tau = z \vee \sigma = z \lrarrow z \in p\, )
				\end{align}
				が成り立つので,
				\begin{align}
					\rho \defeq \varepsilon p\, \forall z\, 
						(\, \tau = z \vee \sigma = z \lrarrow z \in p\, )
				\end{align}
				とおけば
				\begin{align}
					\PAIAX \vdash \forall z\, (\, \tau = z \vee \sigma = z \lrarrow z \in \rho\, )
					\label{fom:pair_of_sets_is_a_set_2}
				\end{align}
				となる.
				
			\item[step2]
				次に
				\begin{align}
					\forall z\, (\, z \in \{a,b\} \lrarrow z \in \rho\, )
				\end{align}
				を示すために
				\begin{align}
					\zeta \defeq \varepsilon z \negation (\, z \in \{a,b\} \lrarrow z \in \rho\, )
				\end{align}
				とおく(当然$\lang{\varepsilon}$の式に書き換える).
				等号の推移律(定理\ref{thm:transitive_law_of_equality})より
				\begin{align}
					\EXTAX,\EQAX \vdash a = \tau \rarrow (\, a = \zeta \rarrow \tau = \zeta\, )
				\end{align}
				が成り立つので,(\refeq{fom:pair_of_sets_is_a_set_1})との三段論法より
				\begin{align}
					\set{a} \wedge \set{b},\ \EXTAX,\EQAX \vdash 
					a = \zeta \rarrow \tau = \zeta
				\end{align}
				が成り立ち,論理和の導入より
				\begin{align}
					\set{a} \wedge \set{b},\ \EXTAX,\EQAX \vdash 
					a = \zeta \rarrow \tau = \zeta \vee \sigma = \zeta
				\end{align}
				が従う.同様にして
				\begin{align}
					\set{a} \wedge \set{b},\ \EXTAX,\EQAX \vdash 
					b = \zeta \rarrow \tau = \zeta \vee \sigma = \zeta
				\end{align}
				も成り立つので,論理和の除去より
				\begin{align}
					\set{a} \wedge \set{b},\ \EXTAX,\EQAX \vdash 
					a = \zeta \vee b = \zeta \rarrow \tau = \zeta \vee \sigma = \zeta
				\end{align}
				が得られる.同様に
				\begin{align}
					\set{a} \wedge \set{b},\ \EXTAX,\EQAX \vdash 
					\tau = \zeta \vee \sigma = \zeta \rarrow a = \zeta \vee b = \zeta
				\end{align}
				も得られ,論理積の導入より
				\begin{align}
					\set{a} \wedge \set{b},\ \EXTAX,\EQAX \vdash 
					a = \zeta \vee b = \zeta \lrarrow \tau = \zeta \vee \sigma = \zeta
				\end{align}
				が従う.他方で定理\ref{thm:pair_members_are_exactly_the_given_two}より
				\begin{align}
					\EXTAX,\EQAX,\COMAX \vdash 
					\zeta \in \{a,b\} \lrarrow a = \zeta \vee b = \zeta
				\end{align}
				が成り立ち,また(\refeq{fom:pair_of_sets_is_a_set_2})より
				\begin{align}
					\PAIAX \vdash \tau = \zeta \vee \sigma = \zeta \lrarrow \zeta \in \rho
				\end{align}
				も成り立つので,同値記号の推移律
				(推論法則\ref{logicalthm:transitive_law_of_equivalence_symbol})より
				\begin{align}
					\set{a} \wedge \set{b},\ \EXTAX,\EQAX,\COMAX,\PAIAX \vdash 
					\zeta \in \{a,b\} \lrarrow \zeta \in \rho
				\end{align}
				が従う.そして全称の導出(推論法則\ref{logicalthm:derivation_of_universal_by_epsilon})より
				\begin{align}
					\set{a} \wedge \set{b},\ \EXTAX,\EQAX,\COMAX,\PAIAX \vdash 
					\forall z\, (\, z \in \{a,b\} \lrarrow z \in \rho\, )
				\end{align}
				が成り立ち,外延性公理より
				\begin{align}
					\set{a} \wedge \set{b},\ \EXTAX,\EQAX,\COMAX,\PAIAX \vdash 
					\{a,b\} = \rho
				\end{align}
				が従い,存在記号の推論公理より
				\begin{align}
					\set{a} \wedge \set{b},\ \EXTAX,\EQAX,\COMAX,\PAIAX \vdash 
					\exists p\, (\, \{a,b\} = p\, )
				\end{align}
				が成り立つ.
				\QED
		\end{description}
	\end{sketch}
	
	\begin{screen}
		\begin{thm}[集合は自分自身の対の要素である]
		\label{thm:set_is_an_element_of_its_pair}
			$a$と$b$を類とするとき
			\begin{align}
				\EXTAX,\EQAX,\COMAX &\vdash \set{a} \rarrow a \in \{a,b\}, \\
				\EXTAX,\EQAX,\COMAX &\vdash \set{b} \rarrow b \in \{a,b\}.
			\end{align}
		\end{thm}
	\end{screen}
	
	\begin{sketch}\mbox{}
		\begin{description}
			\item[step1]
				いま
				\begin{align}
					\tau \defeq \varepsilon x\, (\, a = x\, )
				\end{align}
				とおくと
				\begin{align}
					\set{a} \vdash a = \tau
					\label{fom:set_is_an_element_of_its_pair_1}
				\end{align}
				が成り立ち,論理和の導入より
				\begin{align}
					\set{a} \vdash a = \tau \vee b = \tau
				\end{align}
				も成り立つ.定理\ref{thm:pair_members_are_exactly_the_given_two}より
				\begin{align}
					\EXTAX,\EQAX,\COMAX \vdash 
					a = \tau \vee b = \tau \rarrow \tau \in \{a,b\}
				\end{align}
				が成り立つので三段論法より
				\begin{align}
					\set{a},\ \EXTAX,\EQAX,\COMAX \vdash \tau \in \{a,b\}
					\label{fom:set_is_an_element_of_its_pair_2}
				\end{align}
				が従う.また(\refeq{fom:set_is_an_element_of_its_pair_1})と相等性公理より
				\begin{align}
					\set{a},\ \EQAX \vdash \tau = a
				\end{align}
				となり
				\begin{align}
					\set{a},\ \EQAX \vdash \tau \in \{a,b\} \rarrow a \in \{a,b\}
				\end{align}
				となるので,(\refeq{fom:set_is_an_element_of_its_pair_2})と三段論法より
				\begin{align}
					\set{a},\ \EXTAX,\EQAX,\COMAX \vdash a \in \{a,b\}
				\end{align}
				が成立する.
			
			\item[step2]
				前段で$a$と$b$を入れ替えれば
				\begin{align}
					\set{b},\ \EXTAX,\EQAX,\COMAX \vdash b \in \{b,a\}
					\label{fom:set_is_an_element_of_its_pair_3}
				\end{align}
				が成立する.ところで対の対称性(定理\ref{thm:commutative_law_of_pairs})より
				\begin{align}
					\EXTAX,\EQAX,\COMAX \vdash \{b,a\} = \{a,b\}
				\end{align}
				が成立し,また相等性公理より
				\begin{align}
					\EQAX \vdash \{b,a\} = \{a,b\}
					\rarrow (\, b \in \{b,a\} \rarrow b \in \{a,b\}\, )
				\end{align}
				も成り立つので,三段論法より
				\begin{align}
					\EXTAX,\EQAX,\COMAX \vdash b \in \{b,a\} \rarrow b \in \{a,b\}
					\label{fom:set_is_an_element_of_its_pair_4}
				\end{align}
				が従う.(\refeq{fom:set_is_an_element_of_its_pair_3})と
				(\refeq{fom:set_is_an_element_of_its_pair_4})と三段論法より
				\begin{align}
					\set{b},\ \EXTAX,\EQAX,\COMAX \vdash b \in \{a,b\}
				\end{align}
				が得られる.
				\QED
		\end{description}
	\end{sketch}
	
	$a$を集合とすれば対の公理より$\{a\}$も集合となるので,
	定理\ref{thm:set_is_an_element_of_its_pair}より
	\begin{align}
		\EXTAX,\EQAX,\COMAX \vdash \set{a} \rarrow a \in \{a\}
	\end{align}
	が成立する.一方で$a$も$b$も真類であると$\{a,b\}$は空になる.
	
	\begin{screen}
		\begin{thm}[真類同士の対は空]\label{thm:pair_of_proper_classes_is_emptyset}
			$a$と$b$を類とするとき,
			\begin{align}
				\EXTAX,\EQAX,\COMAX \vdash\ 
				\negation \set{a} \wedge \negation \set{b} \rarrow \{a,b\} = \emptyset.
			\end{align}
		\end{thm}
	\end{screen}
	
	\begin{sketch}
		いま
		\begin{align}
			\tau \defeq \varepsilon x \negation (\, x \notin \{a,b\}\, )
		\end{align}
		とおく($x \notin \{a,b\}$は$\lang{\varepsilon}$の式に書き換える).
		\begin{align}
			\negation \set{a} \wedge\ \negation \set{b}
			\vdash\ \negation \exists x\, (\, a = x\, )
		\end{align}
		が成り立ち,De Morgan の法則
		(推論法則\ref{logicalthm:strong_De_Morgan_law_for_quantifiers_2})より
		\begin{align}
			\negation \set{a} \wedge \negation \set{b}
			\vdash \forall x\, (\, a \neq x\, )
		\end{align}
		が従い,全称記号の推論公理より
		\begin{align}
			\negation \set{a} \wedge \negation \set{b} \vdash a \neq \tau
		\end{align}
		となる.同様にして
		\begin{align}
			\negation \set{a} \wedge \negation \set{b} \vdash b \neq \tau
		\end{align}
		も成り立つので,論理積の導入より
		\begin{align}
			\negation \set{a} \wedge \negation \set{b} \vdash
			a \neq \tau \wedge b \neq \tau
		\end{align}
		が成立し,De Morgan の法則(推論法則\ref{logicalthm:weak_De_Morgan_law_1})より
		\begin{align}
			\negation \set{a} \wedge \negation \set{b} \vdash\ 
			\negation (\, a = \tau \vee b = \tau\, )
			\label{fom:pair_of_proper_classes_is_emptyset_1}
		\end{align}
		が従う.ところで定理\ref{thm:pair_members_are_exactly_the_given_two}より
		\begin{align}
			\EXTAX,\EQAX,\COMAX \vdash \tau \in \{a,b\} \rarrow a = \tau \vee b = \tau
		\end{align}
		が成り立つので,対偶を取って
		\begin{align}
			\EXTAX,\EQAX,\COMAX \vdash\ 
			\negation (\, a = \vee b = \tau\, ) \rarrow \tau \notin \{a,b\}
		\end{align}
		が成り立つ(推論法則\ref{logicalthm:introduction_of_contraposition}).
		そして(\refeq{fom:pair_of_proper_classes_is_emptyset_1})との三段論法より
		\begin{align}
			\negation \set{a} \wedge \negation \set{b},\ \EXTAX,\EQAX,\COMAX \vdash
			\tau \notin \{a,b\}
		\end{align}
		が従い,全称の導出(推論法則\ref{logicalthm:derivation_of_universal_by_epsilon})より
		\begin{align}
			\negation \set{a} \wedge \negation \set{b},\ \EXTAX,\EQAX,\COMAX \vdash
			\forall x\, (\, x \notin \{a,b\}\, )
		\end{align}
		が従う.要素を持たない類は空集合である(定理\ref{thm:uniqueness_of_emptyset})ので
		\begin{align}
			\negation \set{a} \wedge \negation \set{b},\ \EXTAX,\EQAX,\COMAX \vdash
			\{a,b\} = \emptyset
		\end{align}
		が得られる.
		\QED
	\end{sketch}
	
	上の定理とは逆に$\{a,b\}$が空ならば$a$も$b$も真類である.
	
	\begin{screen}
		\begin{thm}[空な対に表示されている類は集合ではない]
		\label{thm:classes_displayed_in_empty_pair_are_not_sets}
			$a$と$b$を類とするとき,
			\begin{align}
				\EXTAX,\EQAX,\COMAX \vdash \{a,b\} = \emptyset \rarrow\ \negation \set{a} \wedge \negation \set{b}.
			\end{align}
		\end{thm}
	\end{screen}
	
	\begin{sketch}
		いま
		\begin{align}
			\tau \defeq \varepsilon x\, (\, a = x\, )
		\end{align}
		とおけば
		\begin{align}
			\set{a} \vdash a = \tau
		\end{align}
		が成立し,また定理\ref{thm:set_is_an_element_of_its_pair}より
		\begin{align}
			\set{a},\ \EXTAX,\EQAX,\COMAX \vdash a \in \{a,b\}
		\end{align}
		が成り立つので相等性公理より
		\begin{align}
			\set{a},\ \EXTAX,\EQAX,\COMAX \vdash \tau \in \{a,b\}
		\end{align}
		が従い,存在記号の推論公理より
		\begin{align}
			\set{a},\ \EXTAX,\EQAX,\COMAX \vdash \exists x\, (\, x \in \{a,b\}\, )
		\end{align}
		が成り立つ.演繹定理より
		\begin{align}
			\EXTAX,\EQAX,\COMAX \vdash \set{a} \rarrow \exists x\, (\, x \in \{a,b\}\, )
		\end{align}
		となり,対偶を取れば
		\begin{align}
			\EXTAX,\EQAX,\COMAX \vdash\ \negation \exists x\, (\, x \in \{a,b\}\, )
			\rarrow\ \negation \set{a}
			\label{fom:classes_displayed_in_empty_pair_are_not_sets_1}
		\end{align}
		が得られる(推論法則\ref{logicalthm:introduction_of_contraposition}).
		他方で空の類は要素を持たない(定理\ref{thm:uniqueness_of_emptyset})ので
		\begin{align}
			\EXTAX,\EQAX,\COMAX \vdash \{a,b\} = \emptyset \rarrow \forall x\, (\, x \notin \{a,b\}\, )
			\label{fom:classes_displayed_in_empty_pair_are_not_sets_2}
		\end{align}
		が成り立ち,また De Morgan の法則
		(推論法則\ref{logicalthm:strong_De_Morgan_law_for_quantifiers_1})より
		\begin{align}
			\vdash \forall x\, (\, x \notin \{a,b\}\, ) \rarrow\ \negation \exists x\, (\, x \in \{a,b\}\, )
			\label{fom:classes_displayed_in_empty_pair_are_not_sets_3}
		\end{align}
		も成り立つので,(\refeq{fom:classes_displayed_in_empty_pair_are_not_sets_2})
		(\refeq{fom:classes_displayed_in_empty_pair_are_not_sets_3})
		(\refeq{fom:classes_displayed_in_empty_pair_are_not_sets_1})を併せて
		\begin{align}
			\EXTAX,\EQAX,\COMAX \vdash \{a,b\} = \emptyset \rarrow\ \negation \set{a}
		\end{align}
		が従う.同様にして
		\begin{align}
			\EXTAX,\EQAX,\COMAX \vdash \{a,b\} = \emptyset \rarrow\ \negation \set{b}
		\end{align}
		も成り立ち,論理積の導入より
		\begin{align}
			\EXTAX,\EQAX,\COMAX \vdash \{a,b\} = \emptyset \rarrow\ \negation \set{a} \wedge \negation \set{b}
		\end{align}
		が得られる.
		\QED
	\end{sketch}
	\section{合併}
	$a$を空でない類とするとするとき,$a$の要素もまた空でなければ要素を持つ.
	$a$の要素の要素を全て集めたものを$a$の合併と呼び,その受け皿の意味を込めて
	\begin{align}
		\bigcup a
	\end{align}
	と書く.当然ながら,空の合併は空となる.
	
	\begin{screen}
		\begin{dfn}[合併]
			$x$を$\mathcal{L}$の項とするとき,
			$x$の{\bf 合併}\index{がっぺい@合併}{\bf (union)}を
			\begin{align}
				\bigcup x \defeq \Set{y}{\exists z \in x\, (\, y \in z\, )}
				\label{eq:definition_of_union_1}
			\end{align}
			で定める.
		\end{dfn}
	\end{screen}
	
	\begin{description}
		\item[量化子が付いた式の略記法]
		上の定義で
		\begin{align}
			\exists z \in x\, (\, y \in z\, )
		\end{align}
		という式を書いたが,これは
		\begin{align}
			\exists z \in x\, (\, y \in z\, ) \defarrow 
			\exists z\, (\, z \in x \wedge y \in z\, )
		\end{align}
		により定義される省略形である.同様にして,$\varphi$を式とするとき
		\begin{align}
			\exists z\, \left(\, z \in x \wedge \varphi\, \right)
		\end{align}
		なる式を
		\begin{align}
			\exists z \in x\, \varphi
		\end{align}
		と略記する.また全称記号についても
		\begin{align}
			\forall z\, \left(\, z \in x \rarrow \varphi\, \right)
		\end{align}
		なる式を
		\begin{align}
			\forall z \in x\, \varphi
		\end{align}
		と略記する.
	\end{description}
	
	\begin{screen}
		\begin{axm}[合併の公理]
			次の式を$\POWAX$によって参照する:
			\begin{align}
				\forall x\, \exists u\, \forall y\, (\, \exists z\, (\, z \in x \wedge y \in z\, ) \lrarrow y \in u\, ).
			\end{align}
		\end{axm}
	\end{screen}
	
	\begin{screen}
		\begin{thm}[集合の合併は集合]
		\label{thm:union_of_a_set_is_a_set}
			$a$を類とするとき
			\begin{align}
				\EXTAX,\EQAX,\COMAX,\POWAX \vdash \set{a} \rarrow \set{\bigcup a}.
			\end{align}
		\end{thm}
	\end{screen}
	
	\begin{sketch}\mbox{}
		\begin{description}
			\item[step1]
				まず
				\begin{align}
					\tau \defeq \varepsilon x\, (\, a = x\, )
				\end{align}
				とおけば(必要に応じて$a = x$を$\lang{\varepsilon}$の式に書き換える),
				\begin{align}
					\set{a} \vdash a = \tau
				\end{align}
				が成立する.$\tau$に対して
				\begin{align}
					\POWAX \vdash \exists u\, \forall y\, (\, \exists z\, (\, z \in \tau \wedge y \in z\, ) \lrarrow y \in u\, )
				\end{align}
				が成り立つので,
				\begin{align}
					\upsilon \defeq \varepsilon u\, \forall y\, (\, \exists z\, (\, z \in \tau \wedge y \in z\, ) \lrarrow y \in u\, )
				\end{align}
				とおけば
				\begin{align}
					\POWAX \vdash \forall y\, (\, \exists z\, (\, z \in \tau \wedge y \in z\, ) \lrarrow y \in \upsilon\, )
					\label{fom:union_of_a_set_is_a_set_1}
				\end{align}
				が成立する.次に$\tau$を$a$に置き換えた場合に
				\begin{align}
					\set{a},\ \EQAX,\POWAX \vdash \forall y\, (\, \exists z\, (\, z \in a \wedge y \in z\, ) \lrarrow y \in \upsilon\, )
				\end{align}
				が成立することを示す.
				
			\item[step2]
				いま
				\begin{align}
					\eta \defeq \varepsilon y \negation  (\, \exists z\, (\, z \in a \wedge y \in z\, ) \lrarrow y \in \upsilon\, )
				\end{align}
				とおけば,(\refeq{fom:union_of_a_set_is_a_set_1})より
				\begin{align}
					\POWAX \vdash \exists z\, (\, z \in \tau \wedge \eta \in z\, )
					\lrarrow \eta \in \upsilon
				\end{align}
				が成立する.
				\begin{align}
					\zeta \defeq \varepsilon z\, (\, z \in a \wedge \eta \in z\, )
				\end{align}
				とおけば
				\begin{align}
					\exists z\, (\, z \in a \wedge \eta \in z\, )
					\vdash \zeta \in a \wedge \eta \in \zeta
				\end{align}
				が成り立ち,
				\begin{align}
					\EQAX \vdash a = \tau \rarrow (\, \zeta \in a \rarrow \zeta \in \tau\, )
				\end{align}
				と併せて
				\begin{align}
					\exists z\, (\, z \in a \wedge \eta \in z\, ),\ \set{a},\ \EQAX
					\vdash \zeta \in \tau \wedge \eta \in \zeta
				\end{align}
				が成立する.また(\refeq{fom:union_of_a_set_is_a_set_1})より
				\begin{align}
					\POWAX \vdash (\, \zeta \in \tau \wedge \eta \in \zeta\, )
					\rarrow \eta \in \upsilon
				\end{align}
				が成り立つので
				\begin{align}
					\exists z\, (\, z \in a \wedge \eta \in z\, ),\ \set{a},\ \EQAX,\POWAX \vdash \eta \in \upsilon
				\end{align}
				が従う.ゆえに
				\begin{align}
					\set{a},\ \EQAX,\POWAX \vdash 
					\exists z\, (\, z \in a \wedge \eta \in z\, ) \rarrow \eta \in \upsilon
					\label{fom:union_of_a_set_is_a_set_2}
				\end{align}
				が得られた.
				
			\item[step3]
				逆に(\refeq{fom:union_of_a_set_is_a_set_1})より
				\begin{align}
					\eta \in \upsilon,\ \POWAX \vdash
					\exists z\, (\, z \in \tau \wedge \eta \in z\, )
				\end{align}
				が成り立つので
				\begin{align}
					\eta \in \upsilon,\ \POWAX \vdash
					\zeta \in \tau \wedge \eta \in \zeta
				\end{align}
				が従い,
				\begin{align}
					\set{a},\ \EQAX \vdash
					\zeta \in \tau \rarrow \zeta \in a
				\end{align}
				と併せて
				\begin{align}
					\eta \in \upsilon,\ \set{a},\ \EQAX,\POWAX \vdash
					\zeta \in a \wedge \eta \in \zeta
				\end{align}
				が従い,
				\begin{align}
					\eta \in \upsilon,\ \set{a},\ \EQAX,\POWAX \vdash
					\exists z\, (\, z \in a \wedge \eta \in z\, )
				\end{align}
				が従う.そして演繹規則より
				\begin{align}
					\set{a},\ \EQAX,\POWAX \vdash
					\eta \in \upsilon \rarrow \exists z\, (\, z \in a \wedge \eta \in z\, )
					\label{fom:union_of_a_set_is_a_set_3}
				\end{align}
				も得られる.
				
			\item[step4]
				(\refeq{fom:union_of_a_set_is_a_set_2})と
				(\refeq{fom:union_of_a_set_is_a_set_3})より
				\begin{align}
					\set{a},\ \EQAX,\POWAX \vdash
					\exists z\, (\, z \in a \wedge \eta \in z\, ) \lrarrow \eta \in \upsilon
				\end{align}
				が得られ,全称記号の推論規則より
				\begin{align}
					\set{a},\ \EQAX,\POWAX \vdash
					\forall y\, (\, \exists z\, (\, z \in a \wedge y \in z\, ) \lrarrow y \in \upsilon\, )
				\end{align}
				となり,定理\ref{thm:equivalent_formula_rewriting_4}より
				\begin{align}
					\set{a},\ \EXTAX,\EQAX,\COMAX,\POWAX \vdash
					\Set{z}{\exists z\, (\, z \in a \wedge y \in z\, )} = \upsilon
				\end{align}
				が成り立つ.存在記号の推論規則より
				\begin{align}
					\set{a},\ \EXTAX,\EQAX,\COMAX,\POWAX \vdash
					\exists u\, (\, \Set{z}{\exists z\, (\, z \in a \wedge y \in z\, )} = u\, )
				\end{align}
				が成り立つので,定理が得られた.
				\QED
		\end{description}
	\end{sketch}
	
	\begin{screen}
		\begin{thm}[空集合の合併は空]\label{thm:the_union_of_the_emptyset_is_empty}
			次が成立する:
			\begin{align}
				\bigcup \emptyset = \emptyset.
			\end{align}
		\end{thm}
	\end{screen}
	
	\begin{sketch}
		いま
		\begin{align}
			\zeta &\defeq \varepsilon z \negation (\, z \notin \bigcup \emptyset\, ), \\
			\eta &\defeq \varepsilon y \negation \negation (\, y \in \emptyset \wedge \zeta \in y\, )
		\end{align}
		とおく.定理\ref{thm:emptyset_has_nothing}より
		\begin{align}
			\EXTAX,\COMAX \vdash \eta \notin \emptyset
		\end{align}
		が成り立つので
		\begin{align}
			\EXTAX,\COMAX \vdash \eta \notin \emptyset \vee \zeta \notin \eta
		\end{align}
		も成立し,De Morgan の法則(推論法則\ref{logicalthm:strong_De_Morgan_law_1})より
		\begin{align}
			\EXTAX,\COMAX \vdash\ \negation (\, \eta \in \emptyset \wedge \zeta \in \eta\, )
		\end{align}
		が成立し,全称記号の推論規則より
		\begin{align}
			\EXTAX,\COMAX \vdash \forall y \negation (\, y \in \emptyset \wedge \zeta \in y\, )
		\end{align}
		が成立する.そして量化子の De Morgan の法則
		(推論法則\ref{logicalthm:strong_De_Morgan_law_for_quantifiers_1})より
		\begin{align}
			\EXTAX,\COMAX \vdash\ \negation \exists y\, (\, y \in \emptyset \wedge \zeta \in y\, )
			\label{fom:the_union_of_the_emptyset_is_empty_1}
		\end{align}
		が得られる.他方で
		\begin{align}
			\COMAX \vdash \zeta \in \bigcup \emptyset
			\rarrow \exists y\, (\, y \in \emptyset \wedge \zeta \in y\, )
		\end{align}
		が成り立つので,対偶を取って
		\begin{align}
			\COMAX \vdash\ 
			\negation \exists y\, (\, y \in \emptyset \wedge \zeta \in y\, )
			\rarrow \zeta \notin \bigcup \emptyset
			\label{fom:the_union_of_the_emptyset_is_empty_2}
		\end{align}
		が得られる.(\refeq{fom:the_union_of_the_emptyset_is_empty_1})と
		(\refeq{fom:the_union_of_the_emptyset_is_empty_2})より
		\begin{align}
			\EXTAX,\COMAX \vdash \zeta \notin \bigcup \emptyset
		\end{align}
		が成り立つので,全称記号の推論規則より
		\begin{align}
			\EXTAX,\COMAX \vdash \forall z\, (\, z \notin \bigcup \emptyset\, )
		\end{align}
		が従い,定理\ref{thm:uniqueness_of_emptyset}より
		\begin{align}
			\EXTAX,\COMAX \vdash \bigcup \emptyset = \emptyset
		\end{align}
		が得られる.
		\QED
	\end{sketch}
	
	\begin{screen}
		\begin{thm}[要素の部分集合は合併の部分集合]
		\label{thm:union_is_bigger_than_any_member}
			$a$を類とするとき
			\begin{align}
				\forall x\, \left[\, \exists t \in a\, (\, x \subset t\, ) \rarrow x \subset \bigcup a\, \right].
			\end{align}
		\end{thm}
	\end{screen}
	
	\begin{sketch}
		$\chi$を$\mathcal{L}$の任意の対象として
		\begin{align}
			\exists t \in a\, (\, x \subset t\, )
			\label{fom:thm_union_is_bigger_than_any_member_1}
		\end{align}
		であるとする.ここで
		\begin{align}
			\tau \defeq \varepsilon t\, (\, t \in a \wedge \chi \subset t\, )
		\end{align}
		とおく.$s$を$\mathcal{L}$の任意の対象として
		\begin{align}
			s \in \chi
		\end{align}
		であるとすると,
		\begin{align}
			\chi \subset \tau
		\end{align}
		より
		\begin{align}
			\tau \in a \wedge s \in \tau
		\end{align}
		が成立するので,存在記号の規則より
		\begin{align}
			\exists t\, \left(\, t \in a \wedge s \in t\, \right)
		\end{align}
		が成り立ち
		\begin{align}
			s \in \bigcup a
		\end{align}
		が従う.$s$は任意に与えられていたので,(\refeq{fom:thm_union_is_bigger_than_any_member_1})の下で
		\begin{align}
			\forall s\, (\, s \in \chi \rarrow s \in \bigcup a\, )
		\end{align}
		すなわち
		\begin{align}
			\chi \subset \bigcup a
		\end{align}
		が成り立つ.ゆえに
		\begin{align}
			\exists t \in a\, \left(\, \chi \subset t\, \right) \rarrow \chi \subset \bigcup a
		\end{align}
		が従い,$\chi$も任意に与えられていたので
		\begin{align}
			\forall x\, \left[\, \exists t \in a\, (\, x \subset t\, ) \rarrow x \subset \bigcup a\, \right]
		\end{align}
		が得られる.
		\QED
	\end{sketch}
	
	\begin{screen}
		\begin{thm}[部分集合の合併は部分類]\label{thm:union_of_subsets_is_subclass}
			$a$と$b$を類とするとき
			\begin{align}
				\forall x \in a\, (\, x \subset b\, ) \rarrow \bigcup a \subset b.
			\end{align}
		\end{thm}
	\end{screen}
	
	\begin{sketch}
		いま
		\begin{align}
			\forall x \in a\, (\, x \subset b\, )
			\label{fom:thm_union_of_subsets_is_subclass_1}
		\end{align}
		が成り立っているとする.$\chi$を$\mathcal{L}$の任意の対象とし,
		\begin{align}
			\chi \in \bigcup a
		\end{align}
		であるとする.すると
		\begin{align}
			\exists t\, \left(\, t \in a \wedge \chi \in t\, \right)
		\end{align}
		が成り立つので,
		\begin{align}
			\tau \defeq \varepsilon t\, \left(\, t \in a \wedge \chi \in t\, \right)
		\end{align}
		とおけば
		\begin{align}
			\tau \in a \wedge \chi \in \tau
		\end{align}
		が成立する.ここで(\refeq{fom:thm_union_of_subsets_is_subclass_1})より
		\begin{align}
			\tau \subset b
		\end{align}
		となるから
		\begin{align}
			\chi \in b
		\end{align}
		が従い,演繹法則より(\refeq{fom:thm_union_of_subsets_is_subclass_1})の下で
		\begin{align}
			\chi \in \bigcup a \rarrow \chi \in b
		\end{align}
		が成立する.$\chi$の任意性ゆえに(\refeq{fom:thm_union_of_subsets_is_subclass_1})の下で
		\begin{align}
			\bigcup a \subset b
		\end{align}
		が成立し,演繹法則より
		\begin{align}
			\forall x \in a\, (\, x \subset b\, ) \rarrow \bigcup a \subset b
		\end{align}
		が得られる.
		\QED
	\end{sketch}
	
	\begin{itembox}[l]{対の合併}
		$a,b$を類とするとき,その対の合併を
		\begin{align}
			a \cup b \defeq \bigcup \{a,b\}
		\end{align}
		と書く.
	\end{itembox}
	
	\begin{screen}
		\begin{thm}[対の合併はそれぞれの要素を合わせたもの]\label{thm:union_of_pair_is_union_of_their_elements}
			$a$と$b$を集合とするとき
			\begin{align}
				\forall x\, (\, x \in a \cup b \lrarrow x \in a \vee x \in b\, ).
			\end{align}
		\end{thm}
	\end{screen}
	
	\monologue{
		この定理の主張は,$a$と$b$を類とするとき
		\begin{align}
			\set{a} \wedge \set{b} \rarrow
			\forall x\, (\, x \in a \cup b \lrarrow x \in a \vee x \in b\, )
		\end{align}
		が成り立つということですが,式にまとめてしまうと見づらいのではじめから$a$と$b$を集合としています.
	}
	
	\begin{sketch}
		$\chi$を$\mathcal{L}$の任意の対象とする.
		\begin{align}
			\chi \in a \cup b
		\end{align}
		であるとき,
		\begin{align}
			\exists t\, \left(\, t \in \{a,b\} \wedge \chi \in t\, \right)
		\end{align}
		が成り立つので,
		\begin{align}
			\tau \defeq \varepsilon t\, \left(\, t \in \{a,b\} \wedge \chi \in t\, \right)
		\end{align}
		とおけば
		\begin{align}
			\tau \in \{a,b\} \wedge \chi \in \tau
		\end{align}
		が成立する.
		\begin{align}
			\tau \in \{a,b\}
		\end{align}
		が成り立つので,定理\ref{thm:pair_members_are_exactly_the_given_two}より
		\begin{align}
			\tau = a \vee \tau = b
			\label{fom:thm_union_of_pair_is_union_of_their_elements_1}
		\end{align}
		が従う.ここで相等性の公理より
		\begin{align}
			\tau = a \rarrow \chi \in a
		\end{align}
		が成り立ち,論理和の規則から
		\begin{align}
			\tau = a \rarrow \chi \in a \vee \chi \in b
		\end{align}
		も成り立つ.同様にして
		\begin{align}
			\tau = b \rarrow \chi \in a \vee \chi \in b
		\end{align}
		が成り立つので,場合分け法則より
		\begin{align}
			\tau = a \vee \tau = b \rarrow \chi \in a \vee \chi \in b
		\end{align}
		が成立し,(\refeq{fom:thm_union_of_pair_is_union_of_their_elements_1})と三段論法より
		\begin{align}
			\chi \in a \vee \chi \in b
		\end{align}
		が成立する.ゆえに演繹法則から
		\begin{align}
			\chi \in a \cup b \rarrow \chi \in a \vee \chi \in b
			\label{fom:thm_union_of_pair_is_union_of_their_elements_2}
		\end{align}
		が成立する.逆に
		\begin{align}
			\chi \in a
		\end{align}
		であるとすると,
		\begin{align}
			\tau_a \defeq \varepsilon x\, (\, a = x\, )
		\end{align}
		とおけば定理\ref{thm:set_is_an_element_of_its_pair}より
		\begin{align}
			\tau_a \in \{a,b\} \wedge \chi \in \tau_a
		\end{align}
		が成り立つので,
		\begin{align}
			\exists t\, \left(\, t \in \{a,b\} \wedge \chi \in t\, \right)
		\end{align}
		が成り立ち
		\begin{align}
			\chi \in a \cup b
		\end{align}
		が従う.これでまず
		\begin{align}
			\chi \in a \rarrow \chi \in a \cup b
		\end{align}
		が得られた.同様にして
		\begin{align}
			\chi \in b \rarrow \chi \in a \cup b
		\end{align}
		も得られ,場合分け法則より
		\begin{align}
			\chi \in a \vee \chi \in b \rarrow \chi \in a \cup b
			\label{fom:thm_union_of_pair_is_union_of_their_elements_3}
		\end{align}
		が成立する.以上(\refeq{fom:thm_union_of_pair_is_union_of_their_elements_2})と
		(\refeq{fom:thm_union_of_pair_is_union_of_their_elements_3})から
		\begin{align}
			\chi \in a \cup b \lrarrow \chi \in a \vee \chi \in b
		\end{align}
		が従い,$\chi$の任意性より
		\begin{align}
			\forall x\, (\, x \in a \cup b \lrarrow x \in a \vee x \in b\, ).
		\end{align}
		が出る.
		\QED
	\end{sketch}
	
	\begin{screen}
		\begin{thm}[等しい類の合併は等しい]\label{thm:unions_of_equal_classes_are_equal}
			$a$と$b$を類とするとき
			\begin{align}
				a = b \rarrow \bigcup a = \bigcup b.
			\end{align}
		\end{thm}
	\end{screen}
	
	\begin{sketch}
		いま
		\begin{align}
			a = b
			\label{fom:thm_unions_of_equal_classes_are_equal}
		\end{align}
		が成り立っているとする.$\chi$を$\mathcal{L}$の任意の対象として
		\begin{align}
			\chi \in \bigcup a
		\end{align}
		であるとすれば,
		\begin{align}
			\tau \in a \wedge \chi \in \tau
		\end{align}
		なる$\mathcal{L}$の対象$\tau$が取れる.このとき相等性の公理より
		\begin{align}
			\tau \in b
		\end{align}
		が成り立つから
		\begin{align}
			\tau \in b \wedge \chi \in \tau
		\end{align}
		が従い,ゆえに
		\begin{align}
			\chi \in \bigcup b
		\end{align}
		が従う.ゆえに(\refeq{fom:thm_unions_of_equal_classes_are_equal})の下で
		\begin{align}
			\chi \in \bigcup a \rarrow \chi \in \bigcup b
		\end{align}
		が得られたが,$a$と$b$を入れ替えれば
		\begin{align}
			\chi \in \bigcup b \rarrow \chi \in \bigcup a
		\end{align}
		も得られるので
		\begin{align}
			\chi \in \bigcup a \lrarrow \chi \in \bigcup b
		\end{align}
		が成立する.そして$\chi$の任意性と外延性の公理から
		\begin{align}
			\bigcup a = \bigcup b
		\end{align}
		が成立する.ゆえに演繹法則から
		\begin{align}
			a = b \rarrow \bigcup a = \bigcup b
		\end{align}
		が従う.
		\QED
	\end{sketch}
	
	\begin{screen}
		\begin{thm}[合併の可換律]
			$a$と$b$を類とするとき
			\begin{align}
				a \cup b = b \cup a.
			\end{align}
		\end{thm}
	\end{screen}
	
	\begin{sketch}
		定理\ref{thm:commutative_law_of_pairs}より
		\begin{align}
			\{a,b\} = \{b,a\}
		\end{align}
		が成り立つので,定理\ref{thm:unions_of_equal_classes_are_equal}から
		\begin{align}
			a \cup b = b \cup a
		\end{align}
		が従う.
		\QED
	\end{sketch}
	\section{交叉}
	交叉とは合併の対となる概念である.$a$を類とするとき,$a$の全ての要素が共通して持つ集合の全体を$a$の交叉と呼び,
	合併の記号を上下に反転させて
	\begin{align}
		\bigcap a
	\end{align}
	と書く.またいささか奇妙な結果であるが,空虚な真の為せる業により空の交叉は宇宙に一致する.
	
	\begin{screen}
		\begin{dfn}[交叉]
			$a$を類とするとき,$a$の{\bf 交叉}\index{こうさ@交叉}{\bf (intersection)}を
			\begin{align}
				\bigcap a \defeq \Set{x}{\forall t \in a\, (\, x \in t\, )}
			\end{align}
			で定める.
		\end{dfn}
	\end{screen}
	
	上の定義に現れた
	\begin{align}
		\forall t \in a\, (\, x \in t\, )
	\end{align}
	とは
	\begin{align}
		\forall t\, (\, t \in a \Longrightarrow x \in t\, )
	\end{align}
	を略記した式である.
	
	\begin{screen}
		\begin{thm}[空集合の交叉は宇宙となる]\label{thm:union_of_the_emptyset_is_the_Universe}
			次が成立する:
			\begin{align}
				\bigcap \emptyset = \Univ.
			\end{align}
		\end{thm}
	\end{screen}
	
	\begin{prf}
		$x$を$\mathcal{L}$の任意の対象とするとき,空虚な真より
		\begin{align}
			t \in \emptyset \Longrightarrow x \in t
		\end{align}
		は$\mathcal{L}$のいかなる対象$t$に対してもに真となる.ゆえに
		\begin{align}
			\forall t \in \emptyset\, (\, x \in t\, )
		\end{align}
		が成立し
		\begin{align}
			\forall x\, (\, x \in \bigcap \emptyset\, )
		\end{align}
		が従う.
		\begin{align}
			\forall x\, (\, x \in \Univ\, )
		\end{align}
		も成り立つから
		\begin{align}
			\forall x\, (\, x \in \Univ \Longleftrightarrow x \in \bigcap \emptyset\, )
		\end{align}
		が成立して,外延性の公理より
		\begin{align}
			\bigcap \emptyset = \Univ
		\end{align}
		が従う.
		\QED
	\end{prf}
	
	\begin{screen}
		\begin{thm}[交叉は全ての要素に含まれる]
		\label{thm:intersection_is_obtained_by_all_elements}
			$a$を類とするとき
			\begin{align}
				\forall x\, (\, x \in a \Longrightarrow \bigcap a \subset x\, ).
			\end{align}
		\end{thm}
	\end{screen}
	
	\begin{screen}
		\begin{thm}[全ての要素に共通して含まれる類は交叉にも含まれる]
		\label{thm:if_obtained_by_all_elements_then_obtained_by_intersection}
			$a$と$b$を類とするとき
			\begin{align}
				\forall x \in a\, (\, b \subset x\, ) \Longrightarrow b \subset \bigcap a.
			\end{align}
		\end{thm}
	\end{screen}
	
	\begin{screen}
		\begin{thm}[等しい類の交叉は等しい]\label{thm:intersections_of_equal_classes_are_equal}
			$a$と$b$を類とするとき
			\begin{align}
				a = b \Longrightarrow \bigcap a = \bigcap b.
			\end{align}
		\end{thm}
	\end{screen}
	
	\begin{itembox}[l]{対の交叉}
		$a$と$b$を類とするとき,その対の交叉を
		\begin{align}
			a \cap b \defeq \bigcap \{a,b\}
		\end{align}
		と書く.
	\end{itembox}
	
	\begin{screen}
		\begin{thm}
			\begin{align}
				\forall x\, (\, x \in a \cap b \Longleftrightarrow x \in a \wedge x \in b\, ).
			\end{align}
		\end{thm}
	\end{screen}
	
	\begin{screen}
		\begin{thm}[交叉の可換律]
			\begin{align}
				a \cap b = b \cap a.
			\end{align}
		\end{thm}
	\end{screen}
	
	\begin{screen}
		\begin{thm}[対の交叉が空ならばその構成要素は共通元を持たない]
		\label{thm:if_pair_is_empty_then_its_members_do_not_intersect}
			$a,b$を類とするとき次が成立する:
			\begin{align}
				a \cap b = \emptyset \Longleftrightarrow \forall x\, (\, x \in a \Longrightarrow x \notin b\, ).
			\end{align}
		\end{thm}
	\end{screen}
	
	\begin{sketch}
		定理\ref{thm:uniqueness_of_emptyset}より
		\begin{align}
			a \cap b = \emptyset \Longleftrightarrow \forall x\, \left(\, x \notin a \cap b\, \right)
		\end{align}
		が成立する.また
		\begin{align}
			\forall x\, \left(\, x \notin a \cap b \Longleftrightarrow x \notin a \vee x \notin b\, \right)
		\end{align}
		かつ
		\begin{align}
			\forall x\, \left(\, (\, x \notin a \vee x \notin b\, ) \Longleftrightarrow (\, x \in a \Longrightarrow x \notin b\, )\, \right)
		\end{align}
		が成り立つので
		\begin{align}
			\forall x\, \left(\, x \notin a \cap b \Longleftrightarrow (\, x \in a \Longrightarrow x \notin b\, )\, \right)
		\end{align}
		が成立し,
		\begin{align}
			\forall x\, \left(\, x \notin a \cap b\, \right) \Longleftrightarrow 	
			\forall x\, (\, x \in a \Longrightarrow x \notin b\, )
		\end{align}
		が従う.ゆえに
		\begin{align}
			a \cap b = \emptyset \Longleftrightarrow \forall x\, (\, x \in a \Longrightarrow x \notin b\, ).
		\end{align}
		が得られる.
		\QED
	\end{sketch}
	
	\begin{screen}
		\begin{dfn}[差類]
			$a,b$を類するとき,$a$に属するが$b$には属さない集合の全体を
			$a$から$b$を引いた{\bf 差類}\index{さるい@差類}
			{\bf (class difference)}と呼び,記号は
			\begin{align}
				a \backslash b \defeq \Set{x}{x \in a \wedge x \notin b}
			\end{align}
			で定める.特に$a \backslash b$が集合であるときこれを
			{\bf 差集合}\index{さしゅうごう@差集合}{\bf (set difference)}と呼ぶ.
			また
			\begin{align}
				b \subset a
			\end{align}
			である場合,$a \backslash b$を$a$における$b$の{\bf 補類}\index{ほるい@補類}{\bf (complement)}或いは
			$a \backslash b$が集合であるとき{\bf 補集合}\index{ほしゅうごう@補集合}と呼ぶ.
		\end{dfn}
	\end{screen}
	
	$\set{a} \Longrightarrow \set{a \backslash b}$
	
	\begin{screen}
		\begin{thm}
			$a$と$b$を類とするとき,
			\begin{align}
				b \subset a
			\end{align}
			であれば
			\begin{align}
				\set{a \backslash b} \wedge \set{b} \Longrightarrow \set{a}.
			\end{align}
		\end{thm}
	\end{screen}
	
	\begin{sketch}
		対の公理から
		\begin{align}
			\{a \backslash b,b\}
		\end{align}
		は集合であり,合併の公理と
		\begin{align}
			a = (a \backslash b) \cup b
		\end{align}
		より
		\begin{align}
			\set{a}
		\end{align}
		が従う.
		\QED
	\end{sketch}
	
	\begin{screen}
		\begin{thm}[合併を引いた類は要素の差の交叉で書ける]
		\label{thm:difference_of_union_is_intersection_of_differences_of_elements}
			$a$と$b$を類とするとき,$a$が集合であれば
			\begin{align}
				a \backslash \bigcup b = \bigcap \Set{a \backslash t}{t \in b}.
			\end{align}
		\end{thm}
	\end{screen}
	
	\monologue{
		上の定理の式で
		\begin{align}
			\Set{a \backslash t}{t \in b}
		\end{align}
		と書いていますが,これは
		\begin{align}
			\Set{x}{\exists t \in b\, (\, x=a \backslash t\, )}
		\end{align}
		の略記です.ところがこれもまだ略記されたもので,正しく書くと
		\begin{align}
			\Set{x}{\exists t \in b\, 
			\forall s\, (\, s \in x \Longleftrightarrow s \in a \wedge s \notin t\, )}
		\end{align}
		となります.以降も煩雑さを避けるためにこのように略記します.
	}
	
	\begin{screen}
		\begin{thm}[二つの類の合併の差類は差類同士の交叉]
		\label{thm:difference_of_union_of_two_classes_is_intersection_of_two_differences}
			$a$と$b$と$c$を類とするとき
			\begin{align}
				a \backslash (b \cup c) = (a \backslash b) \cap (a \backslash c).
			\end{align}
		\end{thm}
	\end{screen}
	
	\begin{screen}
		\begin{thm}[交叉を引いた類は要素の差の合併で書ける]
		\label{thm:difference_of_intersection_is_union_of_differences_of_elements}
			$a$と$b$を類とするとき
			\begin{align}
				a \backslash \bigcap b = \bigcup \Set{a \backslash t}{t \in b}.
			\end{align}
		\end{thm}
	\end{screen}
	
	\begin{screen}
		\begin{thm}[二つの類の交叉の差類は差類同士の合併]
		\label{thm:difference_of_intersection_of_two_classes_is_union_of_two_differences}
			$a$と$b$と$c$を類とするとき
			\begin{align}
				a \backslash (b \cap c) = (a \backslash b) \cup (a \backslash c).
			\end{align}
		\end{thm}
	\end{screen}
	
	\begin{screen}
		\begin{thm}
			
		\end{thm}
	\end{screen}
	
	\begin{prf}\mbox{}
		\begin{description}
			\item[(1)] $a^{-1}$の任意の要素$t$に対し或る$V$の要素$x,y$が存在して
				\begin{align}
					(x,y) \in a \wedge t = (y,x)
				\end{align}
				を満たす.$((x,y),(y,x)) \in f$より$((x,y),t) \in f$が成り立つから
				$t \in f \ast a$となる.逆に$f \ast a$の任意の要素$t$に対して
				$a$の或る要素$x$が存在して
				\begin{align}
					x \in a \wedge (x,t) \in f
				\end{align}
				となる.$x$に対し$V$の或る要素$a,b$が存在して$x=(a,b)$となるので
				\begin{align}
					((a,b),t) \in f
				\end{align}
				となり,$V$の或る要素$c,d$が存在して
				\begin{align}
					((a,b),t) = ((c,d),(d,c))
				\end{align}
				となる.$(a,b) = (c,d)$より$a=c$かつ$b=d$となり,
				$t = (d,c)$かつ$(d,c)=(b,a)$より$t=(b,a)$,従って
				$t \in a^{-1}$が成り立つ.
		\end{description}
	\end{prf}
	\section{冪}
	\begin{screen}
		\begin{dfn}[冪]
			$x$を$\mathcal{L}$の項とするとき,
			\begin{align}
				\power{x} \defeq \Set{y}{\forall z\, (\, z \in y \rarrow z \in x\, )}
			\end{align}
			で定める項(必要に応じて$z \in x$は$\lang{\varepsilon}$の式に書き換える)を
			$x$の{\bf 冪}\index{べき@冪}{\bf (power)}と呼ぶ.
		\end{dfn}
	\end{screen}
	
	$x$の冪とはすなわち「$x$の部分集合の全体」である:
	\begin{align}
		\power{x} = \Set{y}{y \subset x}.
	\end{align}
	
	\begin{screen}
		\begin{axm}[冪の公理]
			次の公理を$\POWAX$によって参照する:
			\begin{align}
				\forall x\, \exists p\, \forall y\, 
				(\, \forall z\, (\, z \in y \rarrow z \in x\, ) \lrarrow y \in p\, ).
			\end{align}
		\end{axm}
	\end{screen}
	
	\begin{screen}
		\begin{thm}[集合の冪は集合]
			$a$を類とするとき
			\begin{align}
				\set{a} \rarrow \set{\power{a}}.
			\end{align}
		\end{thm}
	\end{screen}
	\section{関係}
	\begin{screen}
		\begin{dfn}[順序対]
			$x$と$y$を$\mathcal{L}$の項とするとき,
			\begin{align}
				(x,y) \defeq \{\{x\},\{x,y\}\}
			\end{align}
			で定める項$(x,y)$を$x$と$y$の{\bf 順序対}\index{じゅんじょつい@順序対}
			{\bf (ordered pair)}と呼ぶ.
		\end{dfn}
	\end{screen}
	
	\begin{screen}
		\begin{thm}[集合の順序対は集合]
		\label{thm:ordered_pair_of_sets_is_a_set}
			$a$と$b$を類とするとき
			\begin{align}
				\EXTAX,\EQAX,\COMAX,\PAIAX \vdash
				\set{a} \wedge \set{b} \rarrow \set{(a,b)}.
			\end{align}
		\end{thm}
	\end{screen}
	
	\begin{prf}
		集合の対は集合(定理\ref{thm:pair_of_sets_is_a_set})であるから
		\begin{align}
			\set{a},\ \set{b},\ \EXTAX,\EQAX,\COMAX,\PAIAX &\vdash \set{\{a\}}, \\
			\set{a},\ \set{b},\ \EXTAX,\EQAX,\COMAX,\PAIAX &\vdash \set{\{a,b\}}
		\end{align}
		が成り立つので
		\begin{align}
			\set{a},\ \set{b},\ \EXTAX,\EQAX,\COMAX,\PAIAX \vdash 
			\set{\{a\}} \wedge \set{\{a,b\}}
		\end{align}
		が従い,再び定理\ref{thm:pair_of_sets_is_a_set}より
		\begin{align}
			\set{a},\ \set{b},\ \EXTAX,\EQAX,\COMAX,\PAIAX \vdash \set{(a,b)}
		\end{align}
		となる.
		\QED
	\end{prf}
	
	\begin{screen}
		\begin{thm}[順序対の相等性]
		\label{thm:equality_of_ordered_pairs}
			$a,b,c,d$を集合とするとき
			\begin{align}
				(a,b) = (c,d) \rarrow a=c \wedge b=d.
			\end{align}
		\end{thm}
	\end{screen}
	
	\begin{sketch}\mbox{}
		\begin{description}
			\item[step1] 集合は自分自身の対の要素である(定理\ref{thm:set_is_an_element_of_its_pair})から
				\begin{align}
					\set{\{a\}},\ \EXTAX,\EQAX,\COMAX \vdash \{a\} \in (a,b)
				\end{align}
				が成り立つ.従って$(a,b) = (c,d)$と仮定すると,相等性公理より
				\begin{align}
					(a,b) = (c,d),\ \set{\{a\}},\ \EXTAX,\EQAX,\COMAX \vdash \{a\} \in (c,d)
				\end{align}
				が成り立つ.定理\ref{cor:pair_members_are_exactly_the_given_two}
				(対の要素は表示されている要素の一方には等しい)より
				\begin{align}
					\EXTAX,\EQAX,\COMAX,\ELEAX \vdash \{a\} \in (c,d) \rarrow \{a\} = \{c\} \vee \{a\} = \{c,d\}
					\label{fom:equality_of_ordered_pairs_1}
				\end{align}
				となるから,三段論法より
				\begin{align}
					(a,b) = (c,d),\ \set{\{a\}},\ \EXTAX,\EQAX,\COMAX,\ELEAX \vdash \{a\} = \{c\} \vee \{a\} = \{c,d\}
				\end{align}
				が従い,演繹定理より
				\begin{align}
					(a,b) = (c,d),\ \EXTAX,\EQAX,\COMAX,\ELEAX \vdash
					\set{\{a\}} \rarrow \{a\} = \{c\} \vee \{a\} = \{c,d\}
				\end{align}
				が従う.ところで集合の対は集合(定理\ref{thm:pair_of_sets_is_a_set})なので
				\begin{align}
					\set{a},\ \EXTAX,\EQAX,\COMAX,\PAIAX \vdash \set{\{a\}}
				\end{align}
				が成り立つから,三段論法より
				\begin{align}
					(a,b) = (c,d),\ \set{a},\ \EXTAX,\EQAX,\COMAX,\ELEAX \vdash \{a\} = \{c\} \vee \{a\} = \{c,d\}
					\label{fom:equality_of_ordered_pairs_2}
				\end{align}
				が従う.
		
		
			\item[step2] 定理\ref{thm:set_is_an_element_of_its_pair}より
				\begin{align}
					\set{a},\ \EXTAX,\EQAX,\COMAX \vdash a \in \{a\}
				\end{align}
				が成り立つから,相等性公理より
				\begin{align}
					\{a\} = \{c\},\ \set{a},\ \EXTAX,\EQAX,\COMAX \vdash a \in \{c\}
				\end{align}
				となる.また定理\ref{cor:pair_members_are_exactly_the_given_two}より
				\begin{align}
					\EXTAX,\EQAX,\COMAX,\ELEAX \vdash a \in \{c\} \rarrow a = c
				\end{align}
				となるから,三段論法より
				\begin{align}
					\{a\} = \{c\},\ \set{a},\ \EXTAX,\EQAX,\COMAX,\ELEAX \vdash a = c
					\label{fom:equality_of_ordered_pairs_3}
				\end{align}
				が得られる.
				
			\item[step3] 同様に,定理\ref{thm:set_is_an_element_of_its_pair}より
				\begin{align}
					\set{c},\ \EXTAX,\EQAX,\COMAX \vdash c \in \{c,d\}
				\end{align}
				が成り立つので,相等性公理より
				\begin{align}
					\{a\} = \{c,d\},\ \set{c},\ \EXTAX,\EQAX,\COMAX \vdash c \in \{a\}
				\end{align}
				となる.また定理\ref{cor:pair_members_are_exactly_the_given_two}より
				\begin{align}
					\EXTAX,\EQAX,\COMAX,\ELEAX \vdash c \in \{a\} \rarrow a = c
				\end{align}
				となるから,三段論法より
				\begin{align}
					\{a\} = \{c,d\},\ \set{c},\ \EXTAX,\EQAX,\COMAX,\ELEAX \vdash a = c
					\label{fom:equality_of_ordered_pairs_4}
				\end{align}
				が得られる.
				
			\item[step4] (\refeq{fom:equality_of_ordered_pairs_3})と(\refeq{fom:equality_of_ordered_pairs_4})と
				演繹定理より
				\begin{align}
					\set{a},\ \EXTAX,\EQAX,\COMAX,\ELEAX &\vdash \{a\} = \{c\} \rarrow a = c, \\
					\set{c},\ \EXTAX,\EQAX,\COMAX,\ELEAX &\vdash \{a\} = \{c,d\} \rarrow a = c
				\end{align}
				が成り立つので,論理和の除去より
				\begin{align}
					\set{a},\ \set{c},\ \EXTAX,\EQAX,\COMAX,\ELEAX \vdash 
					\{a\} = \{c\} \vee \{a\} = \{c,d\} \rarrow a = c
				\end{align}
				が従い,(\refeq{fom:equality_of_ordered_pairs_2})との三段論法より
				\begin{align}
					(a,b) = (c,d),\ \set{a},\ \set{c},\ \EXTAX,\EQAX,\COMAX,\ELEAX \vdash a = c
				\end{align}
				が出る.
				
			\item[step5]
		ゆえに
		\begin{align}
			\{\{a\},\{a,b\}\} = \{\{a\},\{a,d\}\}
		\end{align}
		である.$(a,b) = (c,d)$に加えて
		\begin{align}
			a = d
		\end{align}
		と仮定すると,
		\begin{align}
			\{a,b\} = \{a\} \vee \{a,b\} = \{a,d\}
		\end{align}
		と
		\begin{align}
			\{a,b\} = \{a\} \rarrow b = a = d
		\end{align}
		となり,
		\begin{align}
			\{a,b\} = \{a,d\} &\rarrow b = a \vee b = d, \\
			b = a &\rarrow b = d, \\
			b = d &\rarrow b = d
		\end{align}
		より
		\begin{align}
			\{a,b\} = \{a,d\} \rarrow b = d
		\end{align}
		も成り立つ.ゆえに
		\begin{align}
			a = d \rarrow b = d
		\end{align}
		である.今度は$(a,b) = (c,d)$に加えて
		\begin{align}
			a \neq d
		\end{align}
		と仮定する.
		\begin{align}
			\{a,d\} = \{a\} \vee \{a,d\} = \{a,b\}
		\end{align}
		と
		\begin{align}
			\{a,d\} \neq \{a\}
		\end{align}
		より
		\begin{align}
			\{a,d\} = \{a,b\}
		\end{align}
		が成り立ち,
		\begin{align}
			d = a \vee d = b
		\end{align}
		が成り立つ.$d \neq a$より
		\begin{align}
			d = b
		\end{align}
		が従う.ゆえに
		\begin{align}
			a \neq d \rarrow b = d
		\end{align}
		でもある.
		\QED
		\end{description}
	\end{sketch}
	\section{順序数}
	$0,1,2,\cdots$で表される数字は,集合論において
	\begin{align}
		0 &\defeq \emptyset, \\
		1 &\defeq \{0\} = \{\emptyset\}, \\
		2 &\defeq \{0,1\} = \{\emptyset,\{\emptyset\}\}, \\
		3 &\defeq \{0,1,2\} = \{\emptyset,\{\emptyset\},\{\emptyset,\{\emptyset\}\}\}
	\end{align}
	といった反復操作で定められる.上の操作を受け継いで``頑張れば手で書き出せる''類を自然数と呼ぶ.
	$0$は集合であり,集合の対は集合であるから$1$もまた集合である.
	更には集合の合併も集合であるから$2,3,4,\cdots$と続く自然数が全て集合であることがわかる.
	自然数の冪も自然数同士の集合演算もその結果は全て集合になるが,
	ここで「集合は$0$に集合演算を施しただけの素姓が明らかなものに限られるか」
	という疑問というか期待が自然に生まれてくる.実際それは正則性公理によって肯定されるわけだが,
	そこでキーになるのは順序数と呼ばれる概念である.
	
	\begin{screen}
		\begin{axm}[正則性公理]
			次の式を$\REGAX$で参照する:
			\begin{align}
				\forall r\, (\, \exists x\, (\, x \in r\, )
				\rarrow \exists y\, (\, y \in r \wedge \forall z\, 
				(\, z \in r \rarrow z \notin y\, )\, )\, ).
			\end{align}
		\end{axm}
	\end{screen}
	
	正則性公理の主張は「空でない集合は自分自身と交わらない要素を持つ」ということである.
	
	\begin{screen}
		\begin{thm}[類は自分自身を要素に持たない]
		\label{thm:no_class_contains_itself}
			$a$を類とするとき
			\begin{align}
				\EXTAX,\EQAX,\COMAX,\ELEAX,\PAIAX,\REGAX \vdash a \notin a.
			\end{align}
			ただし$a$が主要$\varepsilon$項であれば
			\begin{align}
				\EXTAX,\EQAX,\COMAX,\PAIAX,\REGAX \vdash a \notin a.
			\end{align}
		\end{thm}
	\end{screen}
	
	\begin{sketch}
		要素の公理の対偶より
		\begin{align}
			\ELEAX \vdash\ \negation \set{a} \rarrow a \notin a
		\end{align}
		が成り立つので,後は
		\begin{align}
			\EXTAX,\EQAX,\COMAX,\ELEAX,\PAIAX,\REGAX 
			\vdash \set{a} \rarrow a \notin a
		\end{align}
		を示せばよい.集合の対は集合である(定理\ref{thm:pair_of_sets_is_a_set})から
		\begin{align}
			\set{a},\ \EXTAX,\EQAX,\COMAX,\PAIAX \vdash \set{\{a\}}
		\end{align}
		が成り立つ.ここで
		\begin{align}
			\tau \defeq \varepsilon x\, (\, \{a\} = x\, )
		\end{align}
		とおけば,量化記号の論理的公理より
		\begin{align}
			\set{a},\ \EXTAX,\EQAX,\COMAX,\PAIAX \vdash \{a\} = \tau
			\label{fom:no_class_contains_itself_1}
		\end{align}
		と
		\begin{align}
			\REGAX \vdash \exists x\, (\, x \in \tau\, )
			\rarrow \exists y\, (\, y \in \tau \wedge \forall z\, (\, 
			z \in \tau \rarrow z \notin y\, )\, )
			\label{fom:no_class_contains_itself_2}
		\end{align}
		が成り立つ.集合は自分自身の対の要素である
		(定理\ref{thm:set_is_an_element_of_its_pair})から
		\begin{align}
			\set{a},\ \EXTAX,\EQAX,\COMAX \vdash a \in \{a\}
			\label{fom:no_class_contains_itself_3}
		\end{align}
		が成り立ち,(\refeq{fom:no_class_contains_itself_1})と併せて
		\begin{align}
			\set{a},\ \EXTAX,\EQAX,\COMAX,\PAIAX \vdash a \in \tau
		\end{align}
		が成り立つ.これによって
		\begin{align}
			\set{a},\ \EXTAX,\EQAX,\COMAX,\ELEAX,\PAIAX \vdash 
			\exists x\, (\, x \in \tau\, )
		\end{align}
		が従い(定理\ref{thm:emptyset_does_not_contain_any_class}によるが,
		$a$が主要$\varepsilon$項であれば$\ELEAX$は不要),
		(\refeq{fom:no_class_contains_itself_2})との三段論法より
		\begin{align}
			\set{a},\ \EXTAX,\EQAX,\COMAX,\ELEAX,\PAIAX,\REGAX \vdash
			\exists y\, (\, y \in \tau \wedge \forall z\, (\, 
			z \in \tau \rarrow z \notin y\, )\, )
		\end{align}
		となる.
		\begin{align}
			\zeta &\defeq \varepsilon x\, (\, a = x\, ), \\
			\eta &\defeq \varepsilon y\, (\, y \in \tau \wedge \forall z\, (\, 
			z \in \tau \rarrow z \notin y\, )\, )
		\end{align}
		とおけば存在記号の論理的公理より
		\begin{align}
			\set{a} &\vdash a = \zeta, \label{fom:no_class_contains_itself_4} \\
			\set{a},\ \EXTAX,\EQAX,\COMAX,\ELEAX,\PAIAX,\REGAX &\vdash 
			\eta \in \tau, \label{fom:no_class_contains_itself_5} \\
			\set{a},\ \EXTAX,\EQAX,\COMAX,\ELEAX,\PAIAX,\REGAX &\vdash 
			\forall z\, (\, z \in \tau \rarrow z \notin \eta\, )
			\label{fom:no_class_contains_itself_6}
		\end{align}
		が成り立つが,まず(\refeq{fom:no_class_contains_itself_3})と
		(\refeq{fom:no_class_contains_itself_1})と
		(\refeq{fom:no_class_contains_itself_4})より
		\begin{align}
			\set{a},\ \EXTAX,\EQAX,\COMAX \vdash \zeta \in \tau
		\end{align}
		が成り立つので,(\refeq{fom:no_class_contains_itself_6})より
		\begin{align}
			\set{a},\ \EXTAX,\EQAX,\COMAX,\ELEAX,\PAIAX,\REGAX \vdash 
			\zeta \notin \eta
			\label{fom:no_class_contains_itself_7}
		\end{align}
		が従う.また(\refeq{fom:no_class_contains_itself_5})と
		(\refeq{fom:no_class_contains_itself_1})より
		\begin{align}
			\set{a},\ \EXTAX,\EQAX,\COMAX,\ELEAX,\PAIAX,\REGAX \vdash \eta \in \{a\}
		\end{align}
		が成り立つので,定理\ref{thm:pair_members_are_exactly_the_given_two}より
		\begin{align}
			\set{a},\ \EXTAX,\EQAX,\COMAX,\ELEAX,\PAIAX,\REGAX \vdash \eta = a
		\end{align}
		が従い,(\refeq{fom:no_class_contains_itself_7})と
		(\refeq{fom:no_class_contains_itself_4})と併せて
		\begin{align}
			\set{a},\ \EXTAX,\EQAX,\COMAX,\ELEAX,\PAIAX,\REGAX \vdash a \notin a
		\end{align}
		が出る.
		\QED
	\end{sketch}
	
	\begin{screen}
		\begin{logicalthm}[選言三段論法]
		\label{logicalthm:disjunctive_syllogism}
			$A$と$B$を文とするとき
			\begin{align}
				\vdash A \vee B \rarrow (\, \negation A \rarrow B\, ).
			\end{align}
		\end{logicalthm}
	\end{screen}
	
	\begin{sketch}
		まず含意の導入より
		\begin{align}
			\vdash B \rarrow (\, \negation A \rarrow B\, )
		\end{align}
		が成り立つ.また矛盾の導入より
		\begin{align}
			A,\ \negation A \vdash \bot
		\end{align}
		が成り立つが,爆発律(論理的定理\ref{logicalthm:principle_of_explosion})より
		\begin{align}
			A,\ \negation A \vdash B
		\end{align}
		が従い,演繹定理より
		\begin{align}
			\vdash A \rarrow (\, \negation A \rarrow B\, )
		\end{align}
		も得られる.そして論理和の除去より
		\begin{align}
			\vdash A \vee B \rarrow (\, \negation A \rarrow B\, )
		\end{align}
		が出る.
		\QED
	\end{sketch}
	
	\begin{screen}
		\begin{thm}[集合のどの二組も所属関係で堂々巡りしない]
		\label{thm:no_pair_of_sets_go_round}
			\begin{align}
				\EXTAX,\EQAX,\COMAX,\PAIAX,\REGAX \vdash 
				\forall x\, \forall y\, (\, x \in y \rarrow y \notin x\, ).
			\end{align}
		\end{thm}
	\end{screen}
	
	\begin{sketch}
		いま
		\begin{align}
			\chi &\defeq \varepsilon x \negation \forall y\, (\, x \in y \rarrow y \notin x\, ), \\
			\eta &\defeq \varepsilon y \negation (\, \tau \in y \rarrow y \notin \tau\, )
		\end{align}
		とおく.$\chi$と$\eta$は主要$\varepsilon$項であるから
		定理\ref{thm:critical_epsilon_term_is_set}より
		\begin{align}
			\EXTAX &\vdash \set{\chi}, \\
			\EXTAX &\vdash \set{\eta}
		\end{align}
		となり,従って定理\ref{thm:pair_of_sets_is_a_set}より
		\begin{align}
			\EXTAX,\EQAX,\COMAX,\PAIAX \vdash \set{\{\chi,\eta\}}
		\end{align}
		となり,また定理\ref{thm:set_is_an_element_of_its_pair}より
		\begin{align}
			\EXTAX,\EQAX,\COMAX \vdash \chi \in \{\chi,\eta\}
			\label{fom:no_pair_of_sets_go_round_1}
		\end{align}
		となる.ここで
		\begin{align}
			\tau \defeq \varepsilon x\, (\, \{\chi,\eta\} = x\, )
		\end{align}
		とおけば
		\begin{align}
			\EXTAX,\EQAX,\COMAX,\PAIAX \vdash \{\chi,\eta\} = \tau
			\label{fom:no_pair_of_sets_go_round_2}
		\end{align}
		および
		\begin{align}
			\REGAX \vdash \exists x\, (\, x \in \tau\, ) 
			\rarrow \exists y\, (\, y \in \tau \wedge \forall z\, (\, z \in \tau 
			\rarrow z \notin y\, )\, )
			\label{fom:no_pair_of_sets_go_round_3}
		\end{align}
		が成り立つ.(\refeq{fom:no_pair_of_sets_go_round_1})と
		(\refeq{fom:no_pair_of_sets_go_round_2})より
		\begin{align}
			\EXTAX,\EQAX,\COMAX,\PAIAX \vdash \chi \in \tau
			\label{fom:no_pair_of_sets_go_round_6}
		\end{align}
		が成り立つので%定理\ref{thm:emptyset_does_not_contain_any_class}より
		\begin{align}
			\EXTAX,\EQAX,\COMAX,\PAIAX \vdash \exists x\, (\, x \in \tau\, )
		\end{align}
		となり,(\refeq{fom:no_pair_of_sets_go_round_3})より
		\begin{align}
			\EXTAX,\EQAX,\COMAX,\PAIAX,\REGAX \vdash
			\exists y\, (\, y \in \tau \wedge \forall z\, (\, z \in \tau 
			\rarrow z \notin y\, )\, )
		\end{align}
		が従う.ここで
		\begin{align}
			\gamma \defeq \varepsilon y\, (\, y \in \tau \wedge \forall z\, (\, z \in \tau \rarrow z \in y\, )\, )
		\end{align}
		とおけば
		\begin{align}
			\EXTAX,\EQAX,\COMAX,\PAIAX,\REGAX &\vdash \gamma \in \tau, 
			\label{fom:no_pair_of_sets_go_round_4} \\
			\EXTAX,\EQAX,\COMAX,\PAIAX,\REGAX &\vdash \forall z\, (\, z \in \tau \rarrow z \notin \gamma\, )
			\label{fom:no_pair_of_sets_go_round_5}
		\end{align}
		が成り立つ.(\refeq{fom:no_pair_of_sets_go_round_5})より
		\begin{align}
			\EXTAX,\EQAX,\COMAX,\PAIAX,\REGAX \vdash 
			\chi \in \tau \rarrow \chi \notin \gamma
		\end{align}
		となり,(\refeq{fom:no_pair_of_sets_go_round_6})との三段論法より
		\begin{align}
			\EXTAX,\EQAX,\COMAX,\PAIAX,\REGAX \vdash \chi \notin \gamma
			\label{fom:no_pair_of_sets_go_round_8}
		\end{align}
		が成り立つが,
		\begin{align}
			\EQAX \vdash \chi \notin \gamma \rarrow 
			(\, \eta = \gamma \rarrow \chi \notin \eta\, )
		\end{align}
		も成り立つので三段論法より
		\begin{align}
			\EXTAX,\EQAX,\COMAX,\PAIAX,\REGAX \vdash 
			\eta = \gamma \rarrow \chi \notin \eta
		\end{align}
		となり,対偶律2 (論理的定理\ref{logicalthm:contraposition_2})と演繹定理の逆より
		\begin{align}
			\chi \in \eta,\ \EXTAX,\EQAX,\COMAX,\PAIAX,\REGAX \vdash 
			\eta \neq \gamma
			\label{fom:no_pair_of_sets_go_round_7}
		\end{align}
		が従う.他方で(\refeq{fom:no_pair_of_sets_go_round_2})と
		(\refeq{fom:no_pair_of_sets_go_round_4})より
		\begin{align}
			\EXTAX,\EQAX,\COMAX,\PAIAX,\REGAX \vdash \gamma \in \{\chi,\eta\}
		\end{align}
		が成り立つので,定理\ref{thm:pair_members_are_exactly_the_given_two}より
		\begin{align}
			\EXTAX,\EQAX,\COMAX,\PAIAX,\REGAX \vdash \chi = \gamma \vee \eta = \gamma
		\end{align}
		が従う.これと(\refeq{fom:no_pair_of_sets_go_round_7})と
		選言三段論法(論理的定理\ref{logicalthm:disjunctive_syllogism})より
		\begin{align}
			\chi \in \eta,\ \EXTAX,\EQAX,\COMAX,\PAIAX,\REGAX \vdash 
			\chi = \gamma
			\label{fom:no_pair_of_sets_go_round_9}
		\end{align}
		が従う.(\refeq{fom:no_pair_of_sets_go_round_8})を導いたのと同様にして
		\begin{align}
			\EXTAX,\EQAX,\COMAX,\PAIAX,\REGAX \vdash \eta \notin \gamma
		\end{align}
		も成り立つので,(\refeq{fom:no_pair_of_sets_go_round_9})と相等性公理より
		\begin{align}
			\chi \in \eta,\ \EXTAX,\EQAX,\COMAX,\PAIAX,\REGAX \vdash 
			\eta \notin \chi
		\end{align}
		が従う.演繹定理より
		\begin{align}
			\EXTAX,\EQAX,\COMAX,\PAIAX,\REGAX \vdash 
			\chi \in \eta \rarrow \eta \notin \chi
		\end{align}
		が成り立ち,全称の導出
		(論理的定理\ref{logicalthm:derivation_of_universal_by_epsilon})より
		\begin{align}
			\EXTAX,\EQAX,\COMAX,\PAIAX,\REGAX \vdash 
			\forall x\, \forall y\, (\, x \in y \rarrow y \notin x\, )
		\end{align}
		が得られる.
		\QED
	\end{sketch}
	
	\begin{screen}
		\begin{logicalthm}[論理和の結合律]
		\label{logicalthm:associative_law_of_conjunctions}
			$A,B,C$を$\mathcal{L}$の文とするとき
			\begin{align}
				\vdash (\, A \vee B\, ) \vee C \rarrow A \vee (\, B \vee C\, ).
			\end{align}
		\end{logicalthm}
	\end{screen}
	
	\begin{sketch}
		論理和の導入より
		\begin{align}
			\vdash A \rarrow A \vee (\, B \vee C\, )
			\label{fom:associative_law_of_conjunctions_1}
		\end{align}
		となる.同じく論理和の導入より
		\begin{align}
			B \vdash B \vee C
		\end{align}
		となるが,再び論理和の導入より
		\begin{align}
			\vdash B \vee C \rarrow A \vee (\, B \vee C\, )
		\end{align}
		となるので,三段論法より
		\begin{align}
			B \vdash A \vee (\, B \vee C\, )
		\end{align}
		となり,演繹定理より
		\begin{align}
			\vdash B \rarrow A \vee (\, B \vee C\, )
			\label{fom:associative_law_of_conjunctions_2}
		\end{align}
		が従う.(\refeq{fom:associative_law_of_conjunctions_1})と
		(\refeq{fom:associative_law_of_conjunctions_2})と論理和の除去より
		\begin{align}
			\vdash A \vee B \rarrow A \vee (\, B \vee C\, )
		\end{align}
		が得られる.他方で(\refeq{fom:associative_law_of_conjunctions_2})の導出と同様にして
		\begin{align}
			\vdash C \rarrow A \vee (\, B \vee C\, )
		\end{align}
		も得られるから,再び論理和の除去により
		\begin{align}
			\vdash (\, A \vee B\, ) \vee C \rarrow A \vee (\, B \vee C\, )
		\end{align}
		が出る.
		\QED
	\end{sketch}
	
	\begin{screen}
		\begin{thm}[集合のどの三組も所属関係で堂々巡りしない]
		\label{thm:no_three_sets_go_round}
			\begin{align}
				\EXTAX,\EQAX,\COMAX,\PAIAX,\UNIAX,\REGAX \vdash 
				\forall x\, \forall y\, \forall z\, 
				(\, x \in y \wedge y \in z \rarrow z \notin x\, ).
			\end{align}
		\end{thm}
	\end{screen}
	
	\begin{sketch}
		いま
		\begin{align}
			\chi &\defeq \varepsilon x \negation \forall y\, \forall z\, 
				(\, x \in y \wedge y \in z \rarrow z \notin x\, ), \\
			\eta &\defeq \varepsilon y \negation \forall z\, 
				(\, \chi \in y \wedge y \in z \rarrow z \notin \chi\, ), \\
			\zeta &\defeq \varepsilon z \negation 
				(\, \chi \in \eta \wedge \eta \in z \rarrow z \notin \chi\, )
		\end{align}
		とおく.$\chi,\eta,\zeta$は主要$\varepsilon$項であるから,
		定理\ref{thm:critical_epsilon_term_is_set}より
		\begin{align}
			\EXTAX &\vdash \set{\chi}, \label{fom:no_three_sets_go_round_1} \\
			\EXTAX &\vdash \set{\eta}, \label{fom:no_three_sets_go_round_2} \\
			\EXTAX &\vdash \set{\zeta} \label{fom:no_three_sets_go_round_3}
		\end{align}
		が成り立ち,定理\ref{thm:pair_of_sets_is_a_set} (集合の対は集合)より
		\begin{align}
			\EXTAX,\EQAX,\COMAX,\PAIAX \vdash \set{\{\chi,\eta\}}
			\label{fom:no_three_sets_go_round_4}
		\end{align}
		および
		\begin{align}
			\EXTAX,\EQAX,\COMAX,\PAIAX \vdash \set{\{\zeta\}}
			\label{fom:no_three_sets_go_round_5}
		\end{align}
		が成り立ち,定理\ref{thm:set_is_an_element_of_its_pair}
		(集合は自分自身の対の要素)より
		\begin{align}
			\EXTAX,\EQAX,\COMAX \vdash \chi \in \{\chi,\eta\}
			\label{fom:no_three_sets_go_round_6}
		\end{align}
		が成り立つ.定理\ref{thm:union_of_pair_is_union_of_their_elements}より
		\begin{align}
			\EXTAX,\EQAX,\COMAX \vdash 
			\set{\{\chi,\eta\}} \rarrow \forall x\, (\, x \in \{\chi,\eta\}
			\rarrow x \in \{\chi,\eta,\zeta\}\, )
		\end{align}
		が成り立つので,(\refeq{fom:no_three_sets_go_round_4})と
		(\refeq{fom:no_three_sets_go_round_6})との三段論法より
		\begin{align}
			\EXTAX,\EQAX,\COMAX,\PAIAX \vdash \chi \in \{\chi,\eta,\zeta\}
			\label{fom:no_three_sets_go_round_7}
		\end{align}
		となる.同様にして
		\begin{align}
			\EXTAX,\EQAX,\COMAX,\PAIAX \vdash \eta \in \{\chi,\eta,\zeta\},
			\label{fom:no_three_sets_go_round_8}
		\end{align}
		も成り立つ.また(\refeq{fom:no_three_sets_go_round_4})
		と(\refeq{fom:no_three_sets_go_round_5})と
		定理\ref{thm:pair_of_sets_is_a_set} (集合の対は集合)より
		\begin{align}
			\EXTAX,\EQAX,\COMAX,\PAIAX \vdash \set{\{\{\chi,\eta\},\{\zeta\}\}}
		\end{align}
		となるので,定理\ref{thm:union_of_a_set_is_a_set} (集合の合併は集合)と併せて
		\begin{align}
			\EXTAX,\EQAX,\COMAX,\PAIAX,\UNIAX \vdash \set{\{\chi,\eta,\zeta\}}
		\end{align}
		が成立する.ここで
		\begin{align}
			\tau \defeq \varepsilon x\, (\, \{\chi,\eta,\zeta\} = x\, )
			\label{fom:no_three_sets_go_round_9}
		\end{align}
		とおけば
		\begin{align}
			\EXTAX,\EQAX,\COMAX,\PAIAX,\UNIAX \vdash \{\chi,\eta,\zeta\} = \tau
			\label{fom:no_three_sets_go_round_10}
		\end{align}
		が成り立つので,(\refeq{fom:no_three_sets_go_round_7})と相等性公理より
		\begin{align}
			\EXTAX,\EQAX,\COMAX,\PAIAX,\UNIAX \vdash \chi \in \tau
			\label{fom:no_three_sets_go_round_11}
		\end{align}
		が従い,
		\begin{align}
			\EXTAX,\EQAX,\COMAX,\PAIAX,\UNIAX \vdash \exists x\, (\, x \in \tau\, )
			\label{fom:no_three_sets_go_round_12}
		\end{align}
		が従う.
		\begin{align}
			\REGAX \vdash \exists x\, (\, x \in \tau\, )
			\rarrow \exists y\, (\, y \in \tau \wedge \forall z\, (\, z \in \tau \rarrow z \notin y\, )\, )
		\end{align}
		が成り立つので,(\refeq{fom:no_three_sets_go_round_12})との三段論法より
		\begin{align}
			\EXTAX,\EQAX,\COMAX,\PAIAX,\UNIAX,\REGAX \vdash 
			\exists y\, (\, y \in \tau \wedge \forall z\, (\, z \in \tau \rarrow z \notin y\, )\, )
		\end{align}
		が従う.
		\begin{align}
			\gamma \defeq \varepsilon y\, (\, y \in \tau \wedge \forall z\, (\, z \in \tau \rarrow z \notin y\, )\, )
		\end{align}
		とおけば
		\begin{align}
			\EXTAX,\EQAX,\COMAX,\PAIAX,\UNIAX,\REGAX &\vdash \gamma \in \tau, 
			\label{fom:no_three_sets_go_round_13} \\
			\EXTAX,\EQAX,\COMAX,\PAIAX,\UNIAX,\REGAX &\vdash \forall z\, (\, z \in \tau \rarrow z \notin \gamma\, )\, )
			\label{fom:no_three_sets_go_round_14}
		\end{align}
		が成り立つ.特に
		\begin{align}
			\EXTAX,\EQAX,\COMAX,\PAIAX,\UNIAX,\REGAX &\vdash \chi \in \tau \rarrow \chi \notin \gamma, \\
			\EXTAX,\EQAX,\COMAX,\PAIAX,\UNIAX,\REGAX &\vdash \eta \in \tau \rarrow \eta \notin \gamma
		\end{align}
		となるが,(\refeq{fom:no_three_sets_go_round_7})と
		(\refeq{fom:no_three_sets_go_round_8})および
		(\refeq{fom:no_three_sets_go_round_10})より
		\begin{align}
			\EXTAX,\EQAX,\COMAX,\PAIAX,\UNIAX &\vdash \chi \in \tau, \\
			\EXTAX,\EQAX,\COMAX,\PAIAX,\UNIAX &\vdash \eta \in \tau
		\end{align}
		が成り立つので
		\begin{align}
			\EXTAX,\EQAX,\COMAX,\PAIAX,\UNIAX,\REGAX &\vdash \chi \notin \gamma, 
			\label{fom:no_three_sets_go_round_15} \\
			\EXTAX,\EQAX,\COMAX,\PAIAX,\UNIAX,\REGAX &\vdash \eta \notin \gamma
			\label{fom:no_three_sets_go_round_16}
		\end{align}
		が従う.ところで
		\begin{align}
			\EQAX &\vdash \chi \notin \gamma \rarrow (\, \chi \in \eta \rarrow \eta \neq \gamma\, ), \\
			\EQAX &\vdash \eta \notin \gamma \rarrow (\, \eta \in \zeta \rarrow \zeta \neq \gamma\, )
		\end{align}
		が成り立つので,(\refeq{fom:no_three_sets_go_round_15})と
		(\refeq{fom:no_three_sets_go_round_16})から
		\begin{align}
			\EXTAX,\EQAX,\COMAX,\PAIAX,\UNIAX,\REGAX &\vdash \chi \in \eta \rarrow \eta \neq \gamma, 
			\label{fom:no_three_sets_go_round_17} \\
			\EXTAX,\EQAX,\COMAX,\PAIAX,\UNIAX,\REGAX &\vdash \eta \in \zeta \rarrow \zeta \neq \gamma
			\label{fom:no_three_sets_go_round_18}
		\end{align}
		が従い,
		\begin{align}
			\chi \in \eta \wedge \eta \in \zeta,\ 
			\EXTAX,\EQAX,\COMAX,\PAIAX,\UNIAX,\REGAX \vdash \eta \neq \gamma \wedge \zeta \neq \gamma
		\end{align}
		が従い,De Morgan の法則(論理的定理\ref{logicalthm:weak_De_Morgan_law_1})より
		\begin{align}
			\chi \in \eta \wedge \eta \in \zeta,\ 
			\EXTAX,\EQAX,\COMAX,\PAIAX,\UNIAX,\REGAX \vdash\ 
			\negation (\, \eta = \gamma \vee \zeta = \gamma\, )
			\label{fom:no_three_sets_go_round_19}
		\end{align}
		となる.他方で定理\ref{thm:union_of_pair_is_union_of_their_elements}より
		\begin{align}
			\EXTAX,\EQAX,\COMAX \vdash 
			\set{\{\zeta\}} \rarrow \forall x\, (\, x \in \{\zeta\}
			\rarrow x \in \{\zeta\} \cup \{\chi,\eta\}\, )
		\end{align}
		が成り立つので,(\refeq{fom:no_three_sets_go_round_5})と
		\begin{align}
			\EXTAX,\EQAX,\COMAX \vdash \zeta \in \{\zeta\}
		\end{align}
		(定理\ref{thm:set_is_an_element_of_its_pair})との三段論法より
		\begin{align}
			\EXTAX,\EQAX,\COMAX,\PAIAX \vdash \zeta \in \{\zeta\} \cup \{\chi,\eta\}
		\end{align}
		が従い,定理\ref{thm:symmetry_of_union_of_a_pair}
		(合併の対称性)より
		\begin{align}
			\EXTAX,\EQAX,\COMAX,\PAIAX \vdash \zeta \in \{\chi,\eta,\zeta\}
		\end{align}
		が成り立つ.従って(\refeq{fom:no_three_sets_go_round_10})より
		\begin{align}
			\EXTAX,\EQAX,\COMAX,\PAIAX,\UNIAX \vdash \zeta \in \tau
			\label{fom:no_three_sets_go_round_20}
		\end{align}
		となる.(\refeq{fom:no_three_sets_go_round_13})と
		(\refeq{fom:no_three_sets_go_round_10})より
		\begin{align}
			\EXTAX,\EQAX,\COMAX,\PAIAX,\UNIAX,\REGAX \vdash \gamma \in \{\chi,\eta,\zeta\}
		\end{align}
		が成り立つので,定理\ref{thm:triple_members_are_exactly_the_given_three}
		(表示されている要素しか持たない)より
		\begin{align}
			\EXTAX,\EQAX,\COMAX,\PAIAX,\UNIAX,\REGAX \vdash 
			(\, \chi = \gamma \vee \eta = \gamma\, ) \vee \zeta = \gamma
		\end{align}
		となるが,論理和の結合律
		(論理的定理\ref{logicalthm:associative_law_of_conjunctions})より
		\begin{align}
			\EXTAX,\EQAX,\COMAX,\PAIAX,\UNIAX,\REGAX \vdash 
			\chi = \gamma \vee (\, \eta = \gamma \vee \zeta = \gamma\, )
		\end{align}
		となる.
	\end{sketch}
	\section{無限}
	\begin{screen}
		\begin{dfn}[極限数]
			類$\alpha$が{\bf 極限数}\index{きょくげんすう@極限数}{\bf (limit ordinal)}であるということを
			\begin{align}
				\limo{\alpha} \defarrow \alpha \in \ON \wedge \alpha \neq \emptyset
				\wedge \forall \beta \in \ON\, \left(\, \alpha \neq \beta \cup \{\beta\}\, \right)
			\end{align}
			により定める.つまり,極限数とはいずれの順序数の後者でもない$0$を除く順序数のことである.
		\end{dfn}
	\end{screen}
	
	\begin{screen}
		\begin{thm}[全ての要素の後者で閉じていれば極限数]\label{thm:if_closed_for_latter_then_limit_ordinal}
			空でない順序数は,すべての要素の後者について閉じていれば極限数である:
			\begin{align}
				\forall \alpha \in \ON\,
				\left[\, \alpha \neq \emptyset \wedge 
				\forall \beta\, \left(\, \beta \in \alpha \Longrightarrow \beta \cup \{\beta\} \in \alpha\, \right)
				\Longrightarrow \limo{\alpha}\, \right].
			\end{align}
		\end{thm}
	\end{screen}
	
	\begin{sketch}
		$\alpha$を順序数とし,
		\begin{align}
			\alpha \neq \emptyset \wedge 
			\forall \beta\, \left(\, \beta \in \alpha \Longrightarrow \beta \cup \{\beta\} \in \alpha\, \right)
			\label{fom:thm_if_closed_for_latter_then_limit_ordinal_1}
		\end{align}
		が成り立っているとする.ここで$\beta$を順序数とすると
		\begin{align}
			\beta \in \alpha \vee \beta = \alpha \vee \alpha \in \beta
		\end{align}
		が成り立つ.
		\begin{align}
			\beta = \alpha
		\end{align}
		と
		\begin{align}
			\alpha \in \beta
		\end{align}
		の場合はいずれも
		\begin{align}
			\alpha \in \beta \cup \{\beta\}
		\end{align}
		が成り立つので,定理\ref{thm:no_set_is_an_element_of_itself}より
		\begin{align}
			\alpha \neq \beta \cup \{\beta\}
		\end{align}
		が成立する.ゆえに
		\begin{align}
			(\, \beta = \alpha \vee \beta \in \alpha\, ) \Longrightarrow \alpha \neq \beta \cup \{\beta\}
			\label{fom:thm_if_closed_for_latter_then_limit_ordinal_2}
		\end{align}
		が成立する.他方で(\refeq{fom:thm_if_closed_for_latter_then_limit_ordinal_1})より
		\begin{align}
			\beta \in \alpha \Longrightarrow \beta \cup \{\beta\} \in \alpha
		\end{align}
		も満たされて,
		\begin{align}
			\beta \cup \{\beta\} \in \alpha \Longrightarrow \beta \cup \{\beta\} \neq \alpha
		\end{align}
		と併せて
		\begin{align}
			\beta \in \alpha \Longrightarrow \beta \cup \{\beta\} \neq \alpha
			\label{fom:thm_if_closed_for_latter_then_limit_ordinal_3}
		\end{align}
		が成り立つ.そして(\refeq{fom:thm_if_closed_for_latter_then_limit_ordinal_2})と
		(\refeq{fom:thm_if_closed_for_latter_then_limit_ordinal_3})と場合分け法則により
		\begin{align}
			\left(\, \beta \in \alpha \vee \beta = \alpha \vee \alpha \in \beta\, \right)
			\Longrightarrow \alpha \neq \beta \cup \{\beta\}
		\end{align}
		が成立する.ゆえに
		\begin{align}
			\forall \beta \in \ON\, \left(\, \alpha \neq \beta \cup \{\beta\}\, \right)
		\end{align}
		が成立する.ゆえに$\alpha$は極限数である.
		\QED
	\end{sketch}
	
	次の無限公理は極限数の存在を保証する.
	
	\begin{screen}
		\begin{axm}[無限公理]
			空集合を要素に持ち,全ての要素の後者について閉じている集合が存在する:
			\begin{align}
				\exists a\, \left[\, \emptyset \in a
				\wedge \forall x\, \left(\, x \in a \Longrightarrow x \cup \{x\} \in a\, \right)\, \right].
			\end{align}
		\end{axm}
	\end{screen}
	
	\begin{screen}
		\begin{thm}[極限数は存在する]
			\begin{align}
				\exists \alpha \in \ON\, \left(\, \limo{\alpha}\, \right).
			\end{align}
		\end{thm}
	\end{screen}
	
	\begin{prf}
		無限公理より
		\begin{align}
			\emptyset \in a
			\wedge \forall x\, \left(\, x \in a \Longrightarrow x \cup \{x\} \in a\, \right)
		\end{align}
		を満たす集合$a$が取れる.
		\begin{align}
			b \defeq a \cap \ON
		\end{align}
		とおくとき
		\begin{align}
			\bigcup b
		\end{align}
		が極限数となることを示す.まず
		\begin{align}
			\emptyset \in a \cap \ON \wedge \{\emptyset\} \in a \cap \ON
		\end{align}
		が成り立つから
		\begin{align}
			\emptyset \in \bigcup b
		\end{align}
		が成り立つ.ゆえに$\bigcup b$は空ではない.また定理\ref{thm:union_of_set_of_ordinal_numbers_is_ordinal}より
		\begin{align}
			\bigcup b \in \ON
		\end{align}
		が成立する.$\alpha$を$\bigcup b$の要素とすると,
		\begin{align}
			x \in b \wedge \alpha \in x
		\end{align}
		を満たす順序数$x$が取れる.このとき
		\begin{align}
			\alpha \cup \{\alpha\} \in x
		\end{align}
		か
		\begin{align}
			\alpha \cup \{\alpha\} = x
		\end{align}
		が成り立つが,いずれの場合も
		\begin{align}
			\alpha \cup \{\alpha\} \in x \cup \{x\}
		\end{align}
		が成立する.他方で
		\begin{align}
			x \cup \{x\} \in a \cap \ON
		\end{align}
		も成立するから
		\begin{align}
			\alpha \cup \{\alpha\} \in \bigcup b
		\end{align}
		が成立する.ゆえに
		\begin{align}
			\forall \alpha\, \left(\, \alpha \in \bigcup b \Longrightarrow \alpha \cup \{\alpha\} \in \bigcup b\, \right)
		\end{align}
		が成立する.ゆえに定理\ref{thm:if_closed_for_latter_then_limit_ordinal}より$\bigcup b$は極限数である.
		\QED
	\end{prf}
	
	\monologue{
		無限公理から極限数の存在が示されましたが,無限公理の
		代わりに極限数の存在を公理に採用しても無限公理の主張は導かれます.
		すなわち無限公理の主張と極限数が存在するという主張は同値なのです.
		本稿の流れでは極限数の存在を公理とした方が自然に感じられますが,
		しかし無限公理の方が主張が簡単ですし,他の文献ではこちらを公理としているようです.
	}
	
	\begin{screen}
		\begin{thm}[極限数は上限で表せる]
			\begin{align}
				\limo{\alpha} \Longrightarrow \alpha = \bigcup \Set{\beta}{\beta \in \alpha}.
			\end{align}
		\end{thm}
	\end{screen}
	
	\begin{sketch}
		$\alpha$を極限数とする.$x$を$\alpha$の要素とすれば,
		\begin{align}
			x \cup \{x\} \neq \alpha
		\end{align}
		が成り立つから
		\begin{align}
			x \cup \{x\} \in \alpha
		\end{align}
		が成り立ち
		\begin{align}
			x \in \bigcup \Set{\beta}{\beta \in \alpha}
		\end{align}
		が成立する.$x$を$\bigcup \Set{\beta}{\beta \in \alpha}$の要素とすれば
		\begin{align}
			x \in \beta \wedge \beta \in \alpha
		\end{align}
		なる順序数$\beta$が取れて,順序数の推移性より
		\begin{align}
			x \in \alpha
		\end{align}
		が従う.$x$の任意性から
		\begin{align}
			\alpha = \bigcup \Set{\beta}{\beta \in \alpha}
		\end{align}
		が成立する.
		\QED
	\end{sketch}
	
	\begin{screen}
		\begin{dfn}[自然数]
			最小の極限数を
			\begin{align}
				\Natural
			\end{align}
			と書く.また$\Natural$の要素を{\bf 自然数}\index{しぜんすう@自然数}{\bf (natural number)}と呼ぶ.
		\end{dfn}
	\end{screen}
	
	$\Natural$は最小の極限数であるから,その要素である自然数はどれも極限数ではない.
	従って$\emptyset$を除く自然数は必ずいずれかの自然数の後者である.

	\begin{screen}
		\begin{dfn}[無限]\label{def:infinity}
			本稿においては,{\bf 無限}\index{むげん@無限}{\bf (infinity)}を表す記号$\infty$を
			\begin{align}
				\infty \defeq \Natural
			\end{align}
			によって定める.
		\end{dfn}
	\end{screen}
	
	\begin{screen}
		\begin{thm}[超限帰納法]\label{thm:transfinite_induction}
			$A$を$\mathcal{L}'$の式,$\alpha$を$A$に現れる文字,$\beta$を$A$に現れない文字とする.
			このとき,$A$に現れる文字で$\alpha$のみが$A$で量化されていない場合,次が成り立つ:
			\begin{align}
				\forall \alpha \in \ON\, 
				\left(\, \forall \beta \in \alpha\, A(\beta)
				\Longrightarrow A(\alpha)\, \right)
				\Longrightarrow \forall \alpha \in \ON\, A(\alpha).
			\end{align}
		\end{thm}
	\end{screen}
	
	\begin{prf}
		正則性公理と定理\ref{thm:equivalent_condition_of_axiom_of_regularity}より
		\begin{align}
			\forall \alpha\, \left[\, \forall \beta \in \alpha\, (\, \beta \in \ON \Longrightarrow A(\beta)\, )
			\Longrightarrow (\, \alpha \in \ON \Longrightarrow A(\alpha)\, )\, \right]
			\Longrightarrow \forall \alpha\, (\, \alpha \in \ON \Longrightarrow A(\alpha)\, )
		\end{align}
		が成り立つ.このとき$\alpha$を$\mathcal{L}$の任意の対象とすれば,
		\begin{align}
			\begin{gathered}
				\forall \beta \in \alpha\ (\ \beta \in \ON \Longrightarrow A(\beta)\ )
				\Longrightarrow (\ \alpha \in \ON \Longrightarrow A(\alpha)\ ), \\
				\forall \beta \in \alpha\ (\ \beta \in \ON \Longrightarrow A(\beta)\ ) \wedge \alpha \in \ON \Longrightarrow A(\alpha)
			\end{gathered}
		\end{align}
		は同値であり,他方で順序数の要素は順序数である(定理\ref{thm:On_is_transitive})から
		\begin{align}
			\begin{gathered}
				\forall \beta \in \alpha\ (\ \beta \in \ON \Longrightarrow A(\beta)\ ) \wedge \alpha \in \ON, \\
				\alpha \in \ON \wedge \forall \beta \in \alpha\ A(\beta)
			\end{gathered}
		\end{align}
		も同値である.従って
		\begin{align}
			\alpha \in \ON \wedge \forall \beta \in \alpha\ A(\beta)
			\Longrightarrow A(\alpha)
		\end{align}
		が成り立ち,またこれは
		\begin{align}
			\alpha \in \ON \Longrightarrow \left(\ \forall \beta \in \alpha\ A(\beta)
			\Longrightarrow A(\alpha)\ \right)
		\end{align}
		と同値である.$\alpha$の任意性より
		\begin{align}
			\forall \alpha \in \ON\ 
			\left(\ \forall \beta \in \alpha\ A(\beta)
			\Longrightarrow A(\alpha)\ \right)
			\Longrightarrow \forall \alpha \in \ON\ A(\alpha).
		\end{align}
		が得られる.
		\QED
	\end{prf}
	
	\begin{screen}
		\begin{thm}[数学的帰納法の原理]
		\label{thm:the_principle_of_mathematical_induction}
			$\omg$は次の意味で最小の無限集合である:
			\begin{align}
				\forall a\ \left(\ \emptyset \in a \wedge \forall x\ 
				(\ x \in a \Longrightarrow x \cup \{x\} \in a\ ) 
				\Longrightarrow \omg \subset a\ \right).
			\end{align}
		\end{thm}
	\end{screen}
	
	\begin{prf}
		超限帰納法で示す.いま$a$を
		\begin{align}
			\emptyset \in a \wedge \forall x\ 
			(\ x \in a \Longrightarrow x \cup \{x\} \in a\ )
		\end{align}
		を満たす類とし,また$\alpha$を任意に与えられた順序数とする.
		$\alpha = \emptyset$の場合は$\emptyset \in a$より
		\begin{align}
			\emptyset \in \omega \Longrightarrow \emptyset \in a
		\end{align}
		が成立する.$\alpha \neq \emptyset$の場合,$\alpha$の任意の要素$\beta$に対して
		\begin{align}
			\beta \in {\bf \omega} \Longrightarrow \beta \in a
		\end{align}
		が成り立つと仮定する.このとき,$\alpha \in {\bf \omega}$なら
		$\alpha$は極限数でないから$\alpha = \beta \cup \{\beta\}$を満たす順序数$\beta$が取れて,
		仮定より$\beta \in a$となり$\alpha \in a$が従う.以上で
		\begin{align}
			\forall \alpha \in \ON\ (\ \forall \beta \in \alpha\ (\ \beta \in {\bf \omega} \Longrightarrow \beta \in a\ ) \Longrightarrow (\ \alpha \in {\bf \omega} \Longrightarrow \alpha \in a\ )\ )
		\end{align}
		が得られた.超限帰納法により
		\begin{align}
			\forall \alpha \in \ON\ (\ \alpha \in {\bf \omega} \Longrightarrow \alpha \in a\ )
		\end{align}
		となるから$\omega \subset a$が出る.
		\QED
	\end{prf}
	\section{超限帰納法}
	$x$を任意に与えられた集合としたとき,$x$の任意の要素$y$で
	\begin{align}
		A(y)
	\end{align}
	が成り立つならば
	\begin{align}
		A(x)
	\end{align}
	が成り立つとする.すると,なんと$A(x)$は普遍的に成り立つのである.つまり
	\begin{align}
		\forall x\, \left[\, \forall y \in x\, A(y) \Longrightarrow A(x)\, \right]
		\Longrightarrow \forall x A(x)
	\end{align}
	が成り立つわけだが,この事実を本稿では{\bf 集合の帰納法}と呼ぶ.また派生形としては,
	集合を順序数に制限した場合の{\bf 超限帰納法}\index{ちょうげんきのうほう@超限帰納法}{\bf (transfinite induction)}と,
	自然数に制限した場合の{\bf 数学的帰納法}\index{すうがくてききのうほう@数学的帰納法}{\bf (mathematical induction)}がある.
	
	\begin{screen}
		\begin{thm}[集合の帰納法]\label{thm:equivalent_condition_of_axiom_of_regularity}
			$A$を$\mathcal{L}'$の式とし,$x$を$A$に現れる文字とし,$y$を$A$に現れない文字とし,
			$A$に現れる文字で$x$のみが量化されていないとする.このとき
			\begin{align}
				\forall x\, \left[\, \forall y \in x\, A(y) \Longrightarrow A(x)\, \right]
				\Longrightarrow \forall x A(x).
			\end{align}
		\end{thm}
	\end{screen}
	
	\begin{sketch}
		いま
		\begin{align}
			a \defeq \Set{x}{\rightharpoondown A(x)}
		\end{align}
		とおく.正則性公理より
		\begin{align}
			a \neq \emptyset \Longrightarrow 
			\exists x\, (\, x \in a \wedge x \cap a = \emptyset\, )
		\end{align}
		が成り立つので,対偶を取れば
		\begin{align}
			\forall x\, (\, x \notin a \vee x \cap a \neq \emptyset\, )
			\Longrightarrow a = \emptyset
			\label{fom:thm_equivalent_condition_of_axiom_of_regularity_1}
		\end{align}
		が成り立つ.ここで
		\begin{align}
			x \cap a \neq \emptyset \Longleftrightarrow \exists y \in x\, (\, y \in a\, )
		\end{align}
		が成り立つので(\refeq{fom:thm_equivalent_condition_of_axiom_of_regularity_1})から
		\begin{align}
			\forall x\, \left[\, x \notin a \vee \exists y \in x\, (\, y \in a\, )\, \right]
			\Longrightarrow a = \emptyset
			\label{fom:thm_equivalent_condition_of_axiom_of_regularity_2}
		\end{align}
		が従い,そして論理和は否定と含意で書き直せる(推論法則\ref{logicalthm:rule_of_inference_3})から
		\begin{align}
			\forall x\, \left[\, \forall y \in x\, (\, y \notin a\, ) \Longrightarrow x \notin a\, \right]
			\Longrightarrow a = \emptyset
		\end{align}
		が従う.ところで類の公理より
		\begin{align}
			x \notin a \Longleftrightarrow A(x)
		\end{align}
		が成り立つから
		\begin{align}
			\forall x\, \left[\, \forall y \in x\, A(y)
			\Longrightarrow A(x)\, \right]
			\Longrightarrow \forall x A(x)
		\end{align}
		を得る.
		\QED
	\end{sketch}
	
	本稿では正則性公理を認めているが,いまだけは認めないことにして代わりに
	集合の帰納法が正しいと仮定してみると,今度は正則性公理が定理として導かれる.実際,$a$を類とすれば
	\begin{align}
		\forall x\, \left[\, \forall y \in x\, (\, y \notin a\, )
		\Longrightarrow\ x \notin a\, \right]
		\Longrightarrow \forall x\, (\, x \notin a\, )
	\end{align}
	が成立するが,ここで対偶を取れば
	\begin{align}
		\exists x\, (\, x \in a\, ) \Longrightarrow 
		\exists x \in a\, \left[\, \forall y \in x\, (\, y \notin a\, )\, \right]
	\end{align}
	が成立し,
	\begin{align}
		a \neq \emptyset \Longleftrightarrow \exists x\, (\, x \in a\, )
	\end{align}
	と
	\begin{align}
		\forall y \in x\, (\, y \notin a\, ) \Longleftrightarrow x \cap a = \emptyset 
	\end{align}
	が成り立つことを併せれば
	\begin{align}
		a \neq \emptyset \Longrightarrow 
		\exists x \in a\, (\, x \cap a = \emptyset\, )
	\end{align}
	が出る.この意味で正則性公理は{\bf 帰納法の公理}とも呼ばれる.
	
	\begin{screen}
		\begin{thm}[超限帰納法]\label{thm:transfinite_induction}
			$A$を$\mathcal{L}'$の式,$\alpha$を$A$に現れる文字,$\beta$を$A$に現れない文字とする.
			このとき,$A$に現れる文字で$\alpha$のみが$A$で量化されていない場合,次が成り立つ:
			\begin{align}
				\forall \alpha \in \ON\, \left(\, \forall \beta \in \alpha\, A(\beta) \Longrightarrow A(\alpha)\, \right)
				\Longrightarrow \forall \alpha \in \ON\, A(\alpha).
			\end{align}
		\end{thm}
	\end{screen}
	
	\begin{prf}
		定理\ref{thm:equivalent_condition_of_axiom_of_regularity}より
		\begin{align}
			\forall \alpha\, \left[\, \forall \beta \in \alpha\, (\, \beta \in \ON \Longrightarrow A(\beta)\, )
			\Longrightarrow (\, \alpha \in \ON \Longrightarrow A(\alpha)\, )\, \right]
			\Longrightarrow \forall \alpha\, (\, \alpha \in \ON \Longrightarrow A(\alpha)\, )
			\label{fom:thm_transfinite_induction}
		\end{align}
		が成り立つ.いま
		\begin{align}
			\forall \alpha \in \ON\, \left(\, \forall \beta \in \alpha\, A(\beta) \Longrightarrow A(\alpha)\, \right)
			\label{fom:thm_transfinite_induction_1}
		\end{align}
		が成り立っているとする.その上で$\alpha$を集合とし,
		\begin{align}
			\forall \beta \in \alpha\, (\, \beta \in \ON \Longrightarrow A(\beta)\, )
			\label{fom:thm_transfinite_induction_2}
		\end{align}
		が成り立っているとする.さらにその上で
		\begin{align}
			\alpha \in \ON
		\end{align}
		が成り立っているとする.このとき$\beta$を
		\begin{align}
			\beta \in \alpha
		\end{align}
		なる集合とすると,順序数の推移性より
		\begin{align}
			\beta \in \ON
		\end{align}
		が成り立つので,(\refeq{fom:thm_transfinite_induction_2})と併せて
		\begin{align}
			A(\beta)
		\end{align}
		が成り立つ.すなわちいま
		\begin{align}
			\forall \beta \in \alpha\, A(\beta)
		\end{align}
		が成り立つ.また(\refeq{fom:thm_transfinite_induction_1})より
		\begin{align}
			\forall \beta \in \alpha\, A(\beta) \Longrightarrow A(\alpha)
		\end{align}
		が成り立つので,いま
		\begin{align}
			A(\alpha)
		\end{align}
		が成立する.つまり,(\refeq{fom:thm_transfinite_induction_2})までを仮定したときには
		\begin{align}
			\alpha \in \ON \Longrightarrow A(\alpha)
		\end{align}
		が成立する.ゆえに(\refeq{fom:thm_transfinite_induction_1})までを仮定したときには
		\begin{align}
			\forall \beta \in \alpha\, (\, \beta \in \ON \Longrightarrow A(\beta)\, )
			\Longrightarrow (\, \alpha \in \ON \Longrightarrow A(\alpha)\, )
		\end{align}
		が成立し,$\alpha$の任意性から
		\begin{align}
			\forall \alpha\, \left[\, \forall \beta \in \alpha\, (\, \beta \in \ON \Longrightarrow A(\beta)\, )
			\Longrightarrow (\, \alpha \in \ON \Longrightarrow A(\alpha)\, )\, \right]
		\end{align}
		が成立する.ゆえに,何も仮定しなくても
		\begin{align}
			(\refeq{fom:thm_transfinite_induction_1}) \Longrightarrow
			\forall \alpha\, \left[\, \forall \beta \in \alpha\, (\, \beta \in \ON \Longrightarrow A(\beta)\, )
			\Longrightarrow (\, \alpha \in \ON \Longrightarrow A(\alpha)\, )\, \right]
			\label{fom:thm_transfinite_induction_3}
		\end{align}
		が成立する.(\refeq{fom:thm_transfinite_induction})と(\refeq{fom:thm_transfinite_induction_3})と含意の推移性より
		\begin{align}
			\forall \alpha \in \ON\, \left(\, \forall \beta \in \alpha\, A(\beta) \Longrightarrow A(\alpha)\, \right)
			\Longrightarrow \forall \alpha\, (\, \alpha \in \ON \Longrightarrow A(\alpha)\, )
		\end{align}
		が従うが,
		\begin{align}
			\forall \alpha\, (\, \alpha \in \ON \Longrightarrow A(\alpha)\, )
		\end{align}
		を略記したものが
		\begin{align}
			\forall \alpha \in \ON\, A(\alpha)
		\end{align}
		であるから
		\begin{align}
			\forall \alpha \in \ON\, \left(\, \forall \beta \in \alpha\, A(\beta) \Longrightarrow A(\alpha)\, \right)
			\Longrightarrow \forall \alpha \in \ON\, A(\alpha)
		\end{align}
		が成り立つことになる.
		\QED
	\end{prf}
	
	
	以後本稿では超限帰納法を頻繁に扱うので,ここでその{\bf 利用方法}を述べておく.
	順序数に対する何らかの言明$A$が与えられたとき,それがいかなる順序数に対しても真であることを示したいとする.往々にして
	\begin{align}
		\forall \alpha \in \ON\, A(\alpha)
	\end{align}
	をいきなり示すのは難しく,一方で
	\begin{align}
		\forall \beta \in \alpha\, A(\beta)
	\end{align}
	から
	\begin{align}
		A(\alpha)
	\end{align}
	を導くことは容易い.それは順序数の``順番''的な性質の良さによるが,超限帰納法のご利益は
	\begin{align}
		\forall \alpha \in \ON\, \left(\, \forall \beta \in \alpha\, A(\beta) \Longrightarrow A(\alpha)\, \right)
	\end{align}
	が成り立つことさえ示してしまえばいかなる順序数に対しても$A$が真となってくれるところにある.
	
	$\alpha$を任意に与えられた順序数とするとき,
	\begin{align}
		\alpha = 0
	\end{align}
	であると空虚な真によって
	\begin{align}
		\forall \beta \in \alpha\, A(\beta)
	\end{align}
	は必ず真となるから,まずは
	\begin{align}
		A(0)
	\end{align}
	が成り立つことを示さなければならない.$A(0)$が偽であると
	\begin{align}
		\forall \beta \in \alpha\, A(\beta) \wedge \rightharpoondown A(0)
	\end{align}
	が真となって
	\begin{align}
		\forall \beta \in \alpha\, A(\beta) \Longrightarrow A(0)
	\end{align}
	が偽となってしまうからである.$\alpha$が$0$でないときは素直に
	\begin{align}
		\forall \beta \in \alpha\, A(\beta)
	\end{align}
	が成り立つとき
	\begin{align}
		A(\alpha)
	\end{align}
	が成り立つことを示せば良い.以上超限帰納法の利用法をまとめると,
	
	\begin{itembox}[l]{超限帰納法の利用手順}
		順序数に対する何らかの言明$A$が与えられて,それがいかなる順序数に対しても真なることを示したいならば,
		\begin{itemize}
			\item まずは$A(0)$が成り立つことを示し,
			\item 次は$\alpha$を$0$でない順序数として
				$\forall \beta \in \alpha\, A(\beta) \Longrightarrow A(\alpha)$が成り立つことを示す.
		\end{itemize}
	\end{itembox}
	
	\begin{screen}
		\begin{thm}[数学的帰納法の原理]
		\label{thm:the_principle_of_mathematical_induction}
			$\Natural$は次の意味で最小の無限集合である:
			\begin{align}
				\forall a\, \left[\, \emptyset \in a \wedge \forall x\, 
				(\, x \in a \Longrightarrow x \cup \{x\} \in a\, ) 
				\Longrightarrow \Natural \subset a\, \right].
			\end{align}
		\end{thm}
	\end{screen}
	
	\begin{prf}
		$a$を集合とし,
		\begin{align}
			\emptyset \in a \wedge \forall x\, 
			(\, x \in a \Longrightarrow x \cup \{x\} \in a\, )
			\label{fom:thm_the_principle_of_mathematical_induction_1}
		\end{align}
		が成り立っているとする.このとき
		\begin{align}
			\forall \alpha \in \ON\, (\, \alpha \in \Natural \Longrightarrow \alpha \in a\, )
		\end{align}
		が成り立つことを超限帰納法で示す.まずは
		\begin{align}
			0 \in a
		\end{align}
		から
		\begin{align}
			\emptyset \in \Natural \Longrightarrow \emptyset \in a
		\end{align}
		が成立する.次に$\alpha$を任意に与えられた$0$でない順序数とする.
		\begin{align}
			\forall \beta \in \alpha\, (\, \beta \in \Natural \Longrightarrow \beta \in a\, )
			\label{fom:thm_the_principle_of_mathematical_induction_2}
		\end{align}
		が成り立っているとすると,
		\begin{align}
			\alpha \in \Natural
		\end{align}
		なら$\alpha$は極限数でないから
		\begin{align}
			\alpha = \beta \cup \{\beta\}
		\end{align}
		を満たす自然数$\beta$が取れて,(\refeq{fom:thm_the_principle_of_mathematical_induction_2})より
		\begin{align}
			\beta \in a
		\end{align}
		が成り立ち,(\refeq{fom:thm_the_principle_of_mathematical_induction_1})より
		\begin{align}
			\alpha \in a
		\end{align}
		が従う.以上で
		\begin{align}
			\forall \alpha \in \ON\, \left[\ 
				\forall \beta \in \alpha\, (\, \beta \in \Natural \Longrightarrow \beta \in a\, )
				\Longrightarrow (\, \alpha \in \Natural \Longrightarrow \alpha \in a\, )\, \right]
		\end{align}
		が得られた.超限帰納法により
		\begin{align}
			\forall \alpha \in \ON\, (\, \alpha \in \Natural \Longrightarrow \alpha \in a\, )
		\end{align}
		が成り立つから
		\begin{align}
			\Natural \subset a
		\end{align}
		が従う.
		\QED
	\end{prf}
	\section{再帰的定義}
\label{sec:recursive_definition}
	例えば
	\begin{align}
		a_1,\quad a_2,\quad a_3,\quad a_4,\quad \cdots\quad a_n,\quad \cdots
	\end{align}
	なる列が与えられたときに,その$n$重の順序対を
	\begin{align}
		(a_1,a_2,\cdots,a_n)
	\end{align}
	などと書くことがある.まあ
	\begin{align}
		(a_0,a_1)
	\end{align}
	ならば単なる順序対であり,
	\begin{align}
		(a_0,a_1,a_2)
	\end{align}
	も
	\begin{align}
		((a_0,a_1),a_2)
	\end{align}
	で定められ,
	\begin{align}
		(a_0,a_1,a_2,a_3)
	\end{align}
	も
	\begin{align}
		(((a_0,a_1),a_2),a_3)
	\end{align}
	で定められる.このように具体的に全ての要素を書き出せるうちは何も問題は無い.
	ただし,同じ操作を$n$回反復するということを表現するために
	\begin{align}
		\cdots
	\end{align}
	なる不明瞭な記号を無断で用いることは$\mathcal{L}'$において許されない.
	そもそもまだ``$n$回の反復''をどんな式で表現したら良いかもわからないのである.
	次の定理は,このような再帰的な操作が$\mathcal{L}'$で可能であることを保証する.
	
	\begin{screen}
		\begin{thm}[超限帰納法による写像の構成]
			類$G$を$\Univ$上の写像とするとき,
			\begin{align}
				K \defeq \Set{f}{\exists \alpha \in \ON\ \left(\ f:\alpha \longrightarrow V \wedge \forall \beta \in \alpha\ (\ f(\beta) = G(f|_\beta)\ )\ \right)}
			\end{align}
			とおいて
			\begin{align}
				F \defeq \bigcup K
			\end{align}
			と定めると,$F$は$\ON$上の写像であって
			\begin{align}
				\forall \alpha \in \ON\ (\ F(\alpha) = G(F|_\alpha)\ )
			\end{align}
			を満たす.また$\ON$上の写像で上式を満たすのは$F$のみである.
		\end{thm}
	\end{screen}
	
	\begin{prf}\mbox{}
		\begin{description}
			\item[第二段] $F$が写像であることを示す.
				まず$K$の任意の要素は$V \times V$の部分集合であるから
				\begin{align}
					F \subset V \times V
				\end{align}
				となる.$x,y,z$を任意の集合とする.
				$(x,y) \in F$かつ$(x,z) \in F$のとき,
				$K$の或る要素$f$と$g$が存在して
				\begin{align}
					(x,y) \in f \wedge (x,z) \in g
				\end{align}
				を満たすが,ここで$f(x) = g(x)$となることを言うために,
				$\alpha = \operatorname{dom}(f),\ 
				\beta = \operatorname{dom}(g)$とおき,
				\begin{align}
					\forall \gamma \in \ON\ (\ \gamma \in \alpha \wedge \gamma \in \beta \Longrightarrow f(\gamma) = g(\gamma)\ )
					\label{eq:thm_transfinite_recursion_theorem_1}
				\end{align}
				が成り立つことを示す.いま$\gamma$を任意の順序数とする.$\gamma = \emptyset$の場合は
				$f|_\gamma = \emptyset$かつ$g|_\gamma = \emptyset$となるから
				\begin{align}
					f(\gamma) = G(\emptyset) = g(\gamma)
				\end{align}
				が成立する.$\gamma \neq \emptyset$の場合は
				\begin{align}
					\forall \xi \in \gamma\ (\ \xi \in \alpha \wedge \xi \in \beta \Longrightarrow f(\xi) = g(\xi)\ )
				\end{align}
				が成り立っていると仮定する.このとき$\gamma \in \alpha \wedge \gamma \in \beta$ならば
				順序数の推移性より$\gamma$の任意の要素$\xi$は$\xi \in \alpha \wedge \xi \in \beta$を満たすから
				\begin{align}
					\forall \xi \in \gamma\ (\ f(\xi) = g(\xi)\ )
				\end{align}
				が成立する.従って
				\begin{align}
					f|_\gamma = g|_\gamma
				\end{align}
				が成立するので$f(\gamma) = g(\gamma)$が得られる.超限帰納法より
				(\refeq{eq:thm_transfinite_recursion_theorem_1})が得られる.
				以上より
				\begin{align}
					y = f(x) = g(x) = z
				\end{align}
				となるので$F$はsingle-valuedである.
			
			\item[第三段] $\operatorname{dom}(F) \subset \ON$が成り立つことを示す.
				実際
				\begin{align}
					\operatorname{dom}(F) = \bigcup_{f \in K} \operatorname{dom}(f)
				\end{align}
				かつ$\forall f \in K\ (\ \operatorname{dom}(f) \subset \ON\ )$だから
				$\operatorname{dom}(F) \subset \ON$となる.
				
			\item[第四段] $\operatorname{Tran}(\operatorname{dom}(F))$であることを示す.
				実際任意の集合$x,y$について
				\begin{align}
					y \in x \wedge x \in \operatorname{dom}(F)
				\end{align}
				が成り立っているとき,或る$f \in K$で$x \in \operatorname{dom}(f)$
				となり,$\operatorname{dom}(f)$は順序数なので,順序数の推移律から
				\begin{align}
					y \in \operatorname{dom}(f)
				\end{align}
				が従う.ゆえに$y \in \operatorname{dom}(F)$となる.
				
			\item[第五段] $\forall \alpha \in \operatorname{dom}(F)\ (\ F(\alpha) = G(F|_\alpha)\ )$が成り立つことを示す.
				実際,$\alpha \in \operatorname*{dom}(F)$なら
				$K$の或る要素$f$に対して$\alpha \in \operatorname*{dom}(f)$となるが,
				$f \subset F$であるから
				\begin{align}
					f(\alpha) = F(\alpha)
				\end{align}
				が成り立つ.これにより$f|_\alpha = f \cap (\alpha \times V)
				= F \cap (\alpha \times V) = F|_\alpha$より
				\begin{align}
					G(f|_\alpha) = G(F|_\alpha)
				\end{align}
				も成り立つ.$f(\alpha) = G(f|_\alpha)$と併せて
				$F(\alpha) = G(F|_\alpha)$を得る.
			
			\item[第六段] 
				$\alpha$を任意の順序数として
				$\forall \beta \in \alpha\ (\ \beta \in \operatorname{dom}(F)\ )
				\Longrightarrow \alpha \in \operatorname{dom}(F)$が成り立つことを示す.
				$\alpha = \emptyset$の場合は
				\begin{align}
					\forall f \in K\ (\ \operatorname{dom}(f) \neq \emptyset
					\Longrightarrow \emptyset \in \operatorname{dom}(f)\ )
				\end{align}
				が満たされるので$\alpha \in \operatorname{dom}(F)$となる
				(定理\ref{thm:properties_of_ordinal_numbers}).
				$\alpha \neq \emptyset$の場合,
				\begin{align}
					\forall \beta \in \alpha\ (\ \beta \in \operatorname{dom}(F)\ )
				\end{align}
				が成り立っているとして$f = F|_\alpha$とおけば,$f$は$\alpha$上の写像であり,
				$\alpha$の任意の要素$\beta$に対して
				\begin{align}
					f(\beta)
					= F|_\alpha(\beta)
					= F(\beta)
					= G(F|_\beta)
					= G(f|_\beta)
				\end{align}
				を満たすから$f \in K$である.このとき$f' = f \cup \{(\alpha,G(f))\}$も
				$K$に属するので
				\begin{align}	
					\alpha \in \operatorname{dom}(f') \subset
					\operatorname{dom}(F)
				\end{align}
				が成立する.超限帰納法より
				\begin{align}
					\forall \alpha \in \ON\ (\ \alpha \in \operatorname{dom}(F)\ )
				\end{align}
				が成立し,前段の結果と併せて
				\begin{align}
					\ON = \operatorname{dom}(F)
				\end{align}
				を得る.
				
			\item[第七段]
				$F$の一意性を示す.類$H$が
				\begin{align}
					H:\ON \longrightarrow V 
					\wedge \forall \alpha \in \ON\ (\ H(\alpha) = G(H|_\alpha)\ )
				\end{align}
				を満たすとき,$F = H$が成り立つことを示す.
				いま,$\alpha$を任意に与えられた順序数とする.$\alpha = \emptyset$の場合は
				\begin{align}
					F|_\emptyset = \emptyset = H|_\emptyset
				\end{align}
				より$F(\emptyset) = H(\emptyset)$となる.$\alpha \neq \emptyset$の場合,
				\begin{align}
					\forall \beta \in \alpha\ (\ F(\beta) = H(\beta)\ )
				\end{align}
				が成り立っていると仮定すれば
				\begin{align}
					F|_\alpha = H|_\alpha
				\end{align}
				が成り立つから$F(\alpha) = H(\alpha)$となる.以上で
				\begin{align}
					\forall \alpha \in \ON\ \left(\ \forall \beta \in \alpha\ 
					(\ F(\beta) = H(\beta)\ ) \Longrightarrow F(\alpha) = H(\alpha)\ \right)
				\end{align}
				が得られた.超限帰納法より
				\begin{align}
					\forall \alpha \in \ON\ (\ F(\alpha) = H(\alpha)\ )
				\end{align}
				が従い$F = H$が出る.
				\QED
		\end{description}
	\end{prf}
	
	\begin{itembox}[l]{再帰的定義の応用 : 多数の要素からなる順序対}
		$a$を$\Natural$から集合$A$への写像とすると,
		\begin{align}
			a_n \defeq a(n)
		\end{align}
		と書けば
		\begin{align}
			a_0, a_1, a_2, \cdots
		\end{align}
		なる列が作られる.ここでは
		\begin{align}
			(a_0,a_1,\cdots, a_n)
		\end{align}
		のような記法の集合論的意味付けを考察する.
	\end{itembox}
	
		$\Univ$上の写像$G$を
		\begin{align}
			G(x) = 
			\begin{cases}
				a_0 & \mbox{if } \dom{x} = \emptyset \\
				(x(k),a(\dom{x})) & \mbox{if } \dom{x} = k \cup \{k\} \wedge k \in \Natural \\
				\emptyset & \mbox{o.w.}
			\end{cases}
		\end{align}
		によって定めてみると,つまり$G$とは
		\begin{align}
			\{\, (x,y) \mid \quad &\left(\, \dom{x} = \emptyset \Longrightarrow y = a_0\, \right) \\
		&\wedge \forall k \in \Natural\, \left(\, \dom{x} = k \cup \{k\} \Longrightarrow y = (x(k),a(\dom{x}))\, \right) \\
		&\wedge \left[\, \dom{x} \neq \emptyset \wedge \forall k \in \Natural\, \left(\, \dom{x} \neq k \cup \{k\}\, \right)
		\Longrightarrow y = \emptyset\, \right]\, \}
		\end{align}
		のことであるが,$\ON$上の写像$p$で
		\begin{align}
			p(n) =
			\begin{cases}
				a_0 & \mbox{if } (n = 0) \\
				(a_0,a_1) & \mbox{if } (n=1) \\
				((a_0,a_1),a_2) & \mbox{if } (n=2) \\
				(((a_0,a_1),a_2),a_3) & \mbox{if } (n=3)
			\end{cases}
		\end{align}
		を満たすものが取れる.先の
		\begin{align}
			(a_0,a_1,\cdots, a_n)
		\end{align}
		という一見不正確であった記法は,この
		\begin{align}
			p(n)
		\end{align}
		によって定めると決めてしまえば無事解決である.
	
	\subsection{整礎集合}
	いま$\Univ$上の写像$G$を
	\begin{align}
		G(x) = 
		\begin{cases}
			\emptyset & (\operatorname{dom}(x) = \emptyset) \\
			x(\beta) \cup \operatorname{P}(x(\beta)) & (
			\exists \beta \in \ON\ (\ \operatorname{dom}(x) = \beta \cup \{\beta\}\ )) \\
			\bigcup \operatorname{ran}(x) & \mathrm{o.w.}
		\end{cases}
	\end{align}
	で定めると,定理\ref{thm:transfinite_recursion_theorem}より
	\begin{align}
		\forall \alpha \in \ON\ (\ R(\alpha) = G(R|_\alpha)\ )
	\end{align}
	を満たす$\ON$上の写像$R$が唯一つ存在する.以降しばらくはこの$R$が考察対象となる.
	
	\begin{screen}
		\begin{thm}
			\begin{align}\label{thm:R_alpha_plus_1_equals_to_power_of_R_alpha}
				\forall \alpha \in \ON\ 
				\left(\ R(\alpha + 1) = \operatorname{P}(R(\alpha))\ \right)
			\end{align}
		\end{thm}
	\end{screen}
	
	\begin{prf}\mbox{}
		\begin{description}
			\item[第一段] $R(\alpha + 1) = R(\alpha) \cup \operatorname{P}(R(\alpha))$
				となることを示す.
				
			\item[第二段] $\alpha$を任意に与えられた空でない順序数とするとき,
				\begin{align}
					\forall \beta \in \alpha\ 
					\left(\ R(\beta + 1) \subset \operatorname{P}(R(\beta))\ \right)
					\Longrightarrow R(\alpha + 1) \subset \operatorname{P}(R(\alpha))
				\end{align}
				が成り立つことを示す.いま
				\begin{align}
					\forall \beta \in \alpha\ 
					\left(\ R(\beta + 1) \subset \operatorname{P}(R(\beta))\ \right)
					\label{eq:thm_R_alpha_plus_1_equals_to_power_of_R_alpha}
				\end{align}
				が成り立つと仮定する.$x$を$R(\alpha + 1)$の任意の要素とすれば,前段の結果より
				\begin{align}
					x \in R(\alpha) \vee x \subset R(\alpha)
				\end{align}
				となる.$x \in R(\alpha)$であるとき,$\alpha$の或る要素$\beta$に対し
				$x \in R(\beta)$となる.前段の結果より$x \in R(\beta + 1)$となり,
				(\refeq{eq:thm_R_alpha_plus_1_equals_to_power_of_R_alpha})より
				$x \subset R(\beta)$となるが,
				\begin{align}
					x \subset R(\beta) &\Longrightarrow x \subset R(\alpha), \\
					x \subset R(\alpha) &\Longrightarrow x \in \operatorname{P}(R(\alpha))
				\end{align}
				と併せて$x \in \operatorname{P}(R(\alpha))$が成り立つ.
				一方で$x \subset R(\alpha)$であるときも$x \in \operatorname{P}(R(\alpha))$
				となるから
				\begin{align}
					R(\alpha + 1) \subset \operatorname{P}(R(\alpha))
				\end{align}
				が従う.超限帰納法より定理の主張が得られる.
		\end{description}
	\end{prf}
	
	\begin{screen}
		\begin{dfn}[整礎集合]
			$\bigcup_{\alpha \in \ON} R(\alpha)$の要素を{\bf 整礎集合}
			\index{せいそしゅうごう@整礎集合}{\bf (well-founded set)}と呼ぶ.
		\end{dfn}
	\end{screen}
	
	\begin{screen}
		\begin{thm}[すべての集合は整礎的である]\label{thm:every_set_is_well_founded}
			次は定理である:
			\begin{align}
				\Univ = \bigcup_{\alpha \in \ON} R(\alpha).
			\end{align}
		\end{thm}
	\end{screen}
	
	\begin{prf}
		いま,$S$を$\ON$の空でない部分集合として
		\begin{align}
			V \neq \bigcup_{\alpha \in S} R(\alpha)
			\Longrightarrow S \neq \ON
		\end{align}
		が成り立つことを示す.$V \neq \bigcup_{\alpha \in S} R(\alpha)$であれば
		正則性公理より或る集合$a$が存在して
		\begin{align}
			a \in V \backslash \bigcup_{\alpha \in S} R(\alpha)
			\wedge a \cap V \backslash \bigcup_{\alpha \in S} R(\alpha) = \emptyset
		\end{align}
		を満たす.このとき
		\begin{align}
			a \in \bigcup_{\alpha \in S} R(\alpha) \wedge a \subset \bigcup_{\alpha \in S} R(\alpha)
		\end{align}
		となる.ここで
		\begin{align}
			f = \Set{x}{\exists s \in a\ (\ x = (s,\mu \alpha (s \in R(\alpha)))\ )}
		\end{align}
		と定めれば$f:a \longrightarrow \ON$が成り立つ.
		$\beta = \bigcup f(a)$とおけば$\beta$は$\ON$に属する.このとき
		\begin{align}
			\forall t\ (\ t \in a \Longrightarrow t \in R(f(t))
			\Longrightarrow t \in R(\beta)\ )
		\end{align}
		となるから$a \subset R(\beta)$,そして定理\ref{thm:R_alpha_plus_1_equals_to_power_of_R_alpha}
		より$a \in R(\beta + 1)$が従う.
		\begin{align}
			\forall \alpha \in S\ (\ a \notin R(\alpha)\ )
		\end{align}
		であったから$\beta + 1 \notin S$であり,ゆえに$S \neq \ON$となる.
		定理の主張は対偶を取れば得られる.
		\QED
	\end{prf}
	
	\monologue{
		院生「\begin{align}
				\Univ = \bigcup_{\alpha \in \ON} R(\alpha)
			\end{align}
			という美しい式は偶然得られた訳ではありません.John Von Neumann はこの結果を
			予定して正則性公理を導入したのです.
			さて,超限帰納法による写像の構成を応用して
			次は順序数の足し算と掛け算を定義しましょう.」
	}
	
	\begin{screen}
		\begin{thm}[順序数の加法]\label{thm:the_definition_of_addition_of_ordinal_numbers}
			$\alpha$を$\ON$から任意に選ばれた順序数として,$\Univ$上の写像$G_\alpha$を
			\begin{align}
				G_\alpha(x) = 
				\begin{cases}
					\alpha & (\operatorname{dom}(x) = \emptyset) \\
					x(\beta) \cup \{x(\beta)\} & (
					\exists \beta \in \ON\, (\, \operatorname{dom}(x) = \beta \cup \{\beta\}\, )) \\
					\bigcup \operatorname{ran}(x) & \mathrm{o.w.}
				\end{cases}
			\end{align}
			で定めるとき,定理\ref{thm:transfinite_recursion_theorem}より
			\begin{align}
				\forall \beta \in \ON\, (\, A_\alpha(\beta) = G_\alpha(A_\alpha|_\beta)\, )
			\end{align}
			を満たす$\ON$上の写像$A_\alpha$が唯一つ存在する.ここで
			\begin{align}
				\alpha + \beta = A_\alpha (\beta)
			\end{align}
			と書くと,次が成立する:
			\begin{itemize}
				\item $\forall \alpha,\alpha' \in \ON\, \left(\, \alpha = \alpha' \Longrightarrow A_\alpha = A_{\alpha'}\, \right)$.
				\item $\forall \beta \in \ON\, (\, \alpha + \beta \in \ON\, )$.
				\item $\alpha \in {\bf \omega}$のとき,$\forall \beta \in {\bf \omega}\, (\, \alpha + \beta \in {\bf \omega}\, )$.
			\end{itemize}
		\end{thm}
	\end{screen}
	
	\begin{prf}
		いま$\beta$を任意に与えられた順序数とする.このとき,
		\begin{align}
			\forall \gamma \in \beta\ (\ \alpha + \gamma \in \ON\ )
		\end{align}
		が成り立っていると仮定すると,$\beta = \gamma + 1$と表せるとき
		\begin{align}
			\alpha + \beta 
			= G_\alpha (F_\alpha|_\beta)
			= F_\alpha(\gamma) + 1
			= (\alpha + \gamma) + 1 \in \ON
		\end{align}
		となり,$\beta$が極限数のときは
		\begin{align}
			\alpha + \beta = \operatorname*{sup}_{\gamma \in \beta} (\alpha + \gamma)
			= \bigcup \Set{\alpha + \gamma}{\gamma \in \beta}
			\in \ON
		\end{align}
		となるので,
		\begin{align}
			\forall \beta \in \ON\ \left(\ \forall \gamma \in \beta\ (\ \alpha + \gamma \in \ON\ ) \Longrightarrow \alpha + \beta \in \ON\ \right)
		\end{align}
		が得られた.超限帰納法により
		\begin{align}
			\forall \beta \in \ON\ (\ \alpha + \beta \in \ON\ )
		\end{align}
		が成立する.また$\alpha \in {\bf \omega}$のとき,
		\begin{align}
			a = \Set{\beta \in {\bf \omega}}{\alpha + \beta \in {\bf \omega}}
		\end{align}
		とおけば
		\begin{align}
			\emptyset \in a \wedge \forall x\ (\ x \in a \Longrightarrow x \cup \{x\} \in a\ )
		\end{align}
		となるので${\bf \omega} \subset a$が従う.よって
		\begin{align}
			\forall \beta \in {\bf \omega}\ 
			(\ \alpha + \beta \in {\bf \omega}\ )
		\end{align}
		も成り立つ.
		\QED
	\end{prf}
	
	\begin{screen}
		\begin{thm}[加法の性質]
		\label{thm:properties_of_addition_of_ordinal_numbers}
			定理\ref{thm:the_definition_of_addition_of_ordinal_numbers}で定めた
			加法は以下の性質を持つ:
			\begin{itemize}
				\item $\forall \alpha \in \ON\ (\ \alpha + 0 = 0 + \alpha = \alpha\ )$,
				
				\item $\forall \alpha \in \ON\ (\ \alpha + 1 = \alpha \cup \{\alpha\}\ )$,
				
				\item $\forall \alpha,\beta,\gamma \in \ON\ (\ (\alpha + \beta) + \gamma = \alpha + (\beta + \gamma)\ )$,
				
				\item $\forall \alpha,\beta \in {\bf \omega}\ (\ \alpha + \beta = \beta + \alpha\ )$,
				
				\item $\forall \alpha,\beta,\gamma \in \ON\ (\ \beta \in \gamma
					\Longrightarrow \alpha + \beta \in \alpha + \gamma\ )$,
				
				\item $\forall \alpha,\beta \in \beta\ (\ \alpha \in \beta
					\Longrightarrow \exists \gamma \in \ON\ (\ \alpha + \gamma = \beta\ )\ )$.
			\end{itemize}
		\end{thm}
	\end{screen}
	
	\begin{screen}
		\begin{thm}[順序数の乗法]
		\label{thm:the_definition_of_multiplication_of_ordinal_numbers}
			$\alpha$を$\ON$から任意に選ばれた順序数として,$\Univ$上の写像$G_\alpha$を
			\begin{align}
				G_\alpha(x) = 
				\begin{cases}
					0 & (\operatorname{dom}(x) = \emptyset) \\
					x(\beta) + \alpha & (
					\exists \beta \in \ON\ (\ \operatorname{dom}(x) = \beta \cup \{\beta\}\ )) \\
					\bigcup \operatorname{ran}(x) & \mathrm{o.w.}
				\end{cases}
			\end{align}
			で定めるとき,定理\ref{thm:transfinite_recursion_theorem}より
			\begin{align}
				\forall \beta \in \ON\ (\ M_\alpha(\beta) = G_\alpha(M_\alpha|_\beta)\ )
			\end{align}
			を満たす$\ON$上の写像$M_\alpha$が唯一つ存在する.ここで
			\begin{align}
				\alpha \cdot \beta = M_\alpha (\beta)
			\end{align}
			と書くと,次が成立する:
			\begin{itemize}
				\item $\forall \beta \in \ON\ (\ \alpha \cdot \beta \in \ON\ )$.
				\item $\alpha \in {\bf \omega}$のとき,$\forall \beta \in {\bf \omega}\ 
				(\ \alpha \cdot \beta \in {\bf \omega}\ )$.
			\end{itemize}
		\end{thm}
	\end{screen}

\chapter{イプシロン定理}
	\section{言語}
	\begin{description}
	\item[{\bf EC}]
	{\bf EC}(Elementary calculus)の言語を$L(EC)$と書く.$L(EC)$の構成要素は
	\begin{description}
		\item[矛盾記号] $\bot$
		\item[論理記号] $\rightharpoondown$, $\vee$, $\wedge$, $\rightarrow$
		\item[述語記号] $=$, $\in$
		\item[変項] $x_{0},x_{1},x_{2},\cdots$
	\end{description}
	とする.変項は$L(EC)$の項であって,また$L(EC)$の項は変項だけである.
	$L(EC)$の式は
	\begin{itemize}
		\item 項$s$と式$t$に対して$\in st$と$= st$は式である.
		\item 式$\varphi$と式$\psi$に対して$\rightharpoondown \varphi,
			\vee \varphi \psi,\ \wedge \varphi \psi, \rightarrow \varphi \psi$
			は式である.
		\item 以上のみが$L(EC)$の式である.
	\end{itemize}
	
	\item[{\bf PC}]
	{\bf PC}(Predicate calculus)の言語を$L(PC)$と書く.$L(PC)$の構成要素は
	\begin{description}
		\item[矛盾記号] $\bot$
		\item[論理記号] $\rightharpoondown$, $\vee$, $\wedge$, $\Longrightarrow$
		\item[量化子] $\forall$, $\exists$
		\item[述語記号] $=$, $\in$
		\item[変項] $x_{0},x_{1},x_{2},\cdots$
	\end{description}
	とする.変項は$L(PC)$の項であって,また$L(PC)$の項は変項だけである.
	$L(PC)$の式は
	\begin{itemize}
		\item 項$s$と式$t$に対して$\in st$と$= st$は式である.
		\item 式$\varphi$と式$\psi$に対して$\rightharpoondown \varphi,
			\vee \varphi \psi,\ \wedge \varphi \psi, \rightarrow \varphi \psi$
			は式である.
		\item 式$\varphi$と変項$x$に対して,$\forall x \varphi$と$\exists x \varphi$は式である.
		\item 以上のみが$L(PC)$の式である.
	\end{itemize}
	\end{description}
	
	$L(EC)$と$L(PC)$に$\varepsilon$項を追加した言語をそれぞれ$L(EC_{\varepsilon}),
	L(PC_{\varepsilon})$とする.
	
	\begin{description}
	\item[{\bf EC${}_{\varepsilon}$}]
	言語$L(EC_{\varepsilon})$の構成要素は
	\begin{description}
		\item[矛盾記号] $\bot$
		\item[論理記号] $\rightharpoondown$, $\vee$, $\wedge$, $\rightarrow$
		\item[述語記号] $=$, $\in$
		\item[変項] $x_{0},x_{1},x_{2},\cdots$
		\item[$\varepsilon$記号] $\varepsilon$
	\end{description}
	とする.$L(EC_{\varepsilon})$の項と式は
	\begin{itemize}
		\item 変項は項である.
		\item 項$s$と式$t$に対して$\in st$と$= st$は式である.
		\item 式$\varphi$と式$\psi$に対して$\rightharpoondown \varphi,
			\vee \varphi \psi,\ \wedge \varphi \psi, \rightarrow \varphi \psi$
			は式である.
		\item 式$\varphi$と変項$x$に対して,$\epsilon x \varphi$は項である.
		\item 以上のみが$L(EC_{\varepsilon})$の項と式である.
	\end{itemize}
	
	\item[{\bf PC${}_{\varepsilon}$}]
	言語$L(PC_{\varepsilon})$の構成要素は
	\begin{description}
		\item[矛盾記号] $\bot$
		\item[論理記号] $\rightharpoondown$, $\vee$, $\wedge$, $\Longrightarrow$
		\item[量化子] $\forall$, $\exists$
		\item[述語記号] $=$, $\in$
		\item[変項] $x_{0},x_{1},x_{2},\cdots$
		\item[$\varepsilon$記号] $\varepsilon$
	\end{description}
	とする.$L(PC_{\varepsilon})$の項と式は
	\begin{itemize}
		\item 変項は項である.
		\item 項$s$と式$t$に対して$\in st$と$= st$は式である.
		\item 式$\varphi$と式$\psi$に対して$\rightharpoondown \varphi,
			\vee \varphi \psi,\ \wedge \varphi \psi, \rightarrow \varphi \psi$
			は式である.
		\item 式$\varphi$と変項$x$に対して,$\forall x \varphi$と$\exists x \varphi$は式である.
		\item 式$\varphi$と変項$x$に対して,$\epsilon x \varphi$は項である.
		\item 以上のみが$L(PC_{\varepsilon})$の項と式である.
	\end{itemize}
	\end{description}
	
\section{証明}
	\begin{description}
	\item[{\bf EC}]\mbox{}
	
	\begin{itembox}[l]{{\bf EC}の公理}
		$\varphi$と$\psi$と$\xi$を$L(EC)$の式とするとき,次は{\bf EC}の公理である.
		\begin{description}
			\item[(S)] $(\varphi \rightarrow (\psi \rightarrow \chi)) 
				\rightarrow ((\varphi \rightarrow \psi)
				\rightarrow (\varphi \rightarrow \chi)).$
			\item[(K)] $\varphi \rightarrow (\psi \rightarrow \varphi).$
			\item[(DI1)] $\varphi \rightarrow (\varphi \vee \psi).$
			\item[(DI2)] $\psi \rightarrow (\varphi \vee \psi).$
			\item[(DE)] $(\varphi \rightarrow \chi) \rightarrow 
				((\psi \rightarrow \chi) \rightarrow ((\varphi \vee \psi) \rightarrow \chi)).$
			\item[(CI)] $\varphi \rightarrow (\psi \rightarrow (\varphi \wedge \psi)).$
			\item[(CE1)] $(\varphi \wedge \psi) \rightarrow \varphi.$
			\item[(CE2)] $(\varphi \wedge \psi) \rightarrow \psi.$
				
			\item[(CTI)] $\varphi \rightarrow (\rightharpoondown \varphi \rightarrow \bot).$
			
			\item[(NI)] $(\varphi \rightarrow \bot) \rightarrow\ \rightharpoondown \varphi.$
			\item[(DNE)] $\rightharpoondown \rightharpoondown \varphi \rightarrow \varphi.$
		\end{description}
	\end{itembox}
	
	$\Gamma$を公理系という場合は,$\Gamma$は$L(EC)$の式の集合である.$\Gamma$が空である場合もある.
	$L(EC)$の式$\chi$に対して$\Gamma$から{\bf EC}の証明が存在する(証明可能である)ことを
	\begin{align}
		\Gamma \provable{\mbox{{\bf EC}}} \chi
	\end{align}
	と書くが,{\bf EC}における$\Gamma$から$\chi$への証明とは,
	$L(EC)$の式の列$\varphi_{1},\varphi_{2},
	\cdots,\varphi_{n}$であって,$\varphi_{n}$は$\chi$であり,
	各$i \in \{1,2,\cdots,n\}$に対して
	\begin{itemize}
		\item $\varphi_{i}$は{\bf EC}の公理である.
		\item $\varphi_{i}$は$\Gamma$の公理である.
		\item $\varphi_{i}$は前の式から推論規則を用いて得られる式である.{\bf EC}の推論規則とは,
			\begin{description}
			\item[三段論法]
				$j,k < i$なる$k,j$が取れて,$\varphi_{k}$は
				$\varphi_{j} \rightarrow \varphi_{i}$である.
		\end{description} 
	\end{itemize}
	が満たされているものである.
	
	\item[{\bf PC}]
	$\varphi$をいずれかの言語の式とし,$x$を変項とする.
	$\varphi$に$x$が自由に現れているとき,$\varphi$に自由に現れている
	$x$を変項$t$で置き換えた式を
	\begin{align}
		\varphi(t/x)
	\end{align}
	とする.ただし$t$は$\varphi(t/x)$で{\bf $x$に置き換わった位置で束縛されない}とする.
	このことを{\bf $t$は$\varphi$の中で$x$への代入について自由である}とも言う.
	
	\begin{itembox}[l]{{\bf PC}の公理}
		$\varphi$と$\psi$と$\xi$を$L(PC)$の式とし,$x$と$t$を変項とするとき,
		次は{\bf PC}の公理である.
		\begin{description}
			\item[(S)] $(\varphi \rightarrow (\psi \rightarrow \chi)) 
				\rightarrow ((\varphi \rightarrow \psi)
				\rightarrow (\varphi \rightarrow \chi)).$
			\item[(K)] $\varphi \rightarrow (\psi \rightarrow \varphi).$
			\item[(DI1)] $\varphi \rightarrow (\varphi \vee \psi).$
			\item[(DI2)] $\psi \rightarrow (\varphi \vee \psi).$
			\item[(DE)] $(\varphi \rightarrow \chi) \rightarrow 
				((\psi \rightarrow \chi) \rightarrow ((\varphi \vee \psi) \rightarrow \chi)).$
			\item[(CI)] $\varphi \rightarrow (\psi \rightarrow (\varphi \wedge \psi)).$
			\item[(CE1)] $(\varphi \wedge \psi) \rightarrow \varphi.$
			\item[(CE2)] $(\varphi \wedge \psi) \rightarrow \psi.$
			
			\item[(UE)] $\forall x \varphi \rightarrow \varphi(\tau/x).$
				\\ \textcolor{red}{ただし,$\varphi$には$x$が自由に現れて,
				$t$は$\varphi$の中で$x$への代入について自由である.}
				
			\item[(EI)] $\varphi(\tau/x) \rightarrow \exists x \varphi.$
				\\ \textcolor{red}{ただし,$\varphi$には$x$が自由に現れて,
				$t$は$\varphi$の中で$x$への代入について自由である.}
				
			\item[(CTI)] $\varphi \rightarrow (\rightharpoondown \varphi \rightarrow \bot).$
			
			\item[(NI)] $(\varphi \rightarrow \bot) \rightarrow\ \rightharpoondown \varphi.$
			\item[(DNE)] $\rightharpoondown \rightharpoondown \varphi \rightarrow \varphi.$
		\end{description}
	\end{itembox}
	
	$\Gamma$を公理系という場合は,$\Gamma$は$L(PC)$の文の集合である.$\Gamma$が空である場合もある.
	$L(PC)$の式$\chi$に対して$\Gamma$から{\bf PC}の証明が存在する(証明可能である)ことを
	\begin{align}
		\Gamma \provable{\mbox{{\bf PC}}} \chi
	\end{align}
	と書くが,{\bf PC}における$\Gamma$から$\chi$への証明とは,
	$L(PC)$の式の列$\varphi_{1},\varphi_{2},
	\cdots,\varphi_{n}$であって,$\varphi_{n}$は$\chi$であり,
	各$i \in \{1,2,\cdots,n\}$に対して
	\begin{itemize}
		\item $\varphi_{i}$は{\bf PC}の公理である.
		\item $\varphi_{i}$は$\Gamma$の公理である.
		\item $\varphi_{i}$は前の式から推論規則を用いて得られる式である.{\bf PC}の推論規則とは,
			\begin{description}
			\item[三段論法]
				$j,k < i$なる$k,j$が取れて,$\varphi_{k}$は
				$\varphi_{j} \rightarrow \varphi_{i}$である.
			 	
			\item[存在汎化] 
				$j < i$なる$j$が取れて,$\varphi_{j}$は
				$\varphi(t/x) \rightarrow \psi$なる式であって,
				$\varphi_{i}$は$\exists x \varphi \rightarrow \psi$なる式である.
				\\ \textcolor{red}{ただし,$\varphi$には$x$が自由に現れ,
				$t$は$\varphi$の中で$x$への代入について自由である.
				また$t$は$\varphi$と$\psi$には自由に現れない.}
			
			\item[全称汎化] 
				$j < i$なる$j$が取れて,$\varphi_{j}$は
				$\psi \rightarrow \varphi(t/x)$なる式であって,
				$\varphi_{i}$は$\psi \rightarrow \forall x \varphi$なる式である.
				\\ \textcolor{red}{ただし,$\varphi$には$x$が自由に現れ,
				$t$は$\varphi$の中で$x$への代入について自由である.
				また$t$は$\varphi$と$\psi$には自由に現れない.}
		\end{description} 
	\end{itemize}
	が満たされているものである.
	
	\item[主要論理式]
	{\bf EC}${}_{\varepsilon}$と{\bf PC}${}_{\varepsilon}$の公理には
	\begin{align}
		\varphi(t/x) \rightarrow \varphi(\varepsilon x \varphi/x)
	\end{align}
	なる形の式が追加される.ただし$x$は$\varphi$に自由に現れて,
	$t$は$\varphi$の中で$x$への代入について自由である.
	この形の式を{\bf 主要論理式}{\bf (principal formula)}と呼ぶ.
	
	\item[{\bf EC}${}_{\varepsilon}$]\mbox{}
	
	\begin{itembox}[l]{{\bf EC}${}_{\varepsilon}$の公理}
		$\varphi$と$\psi$と$\xi$を$L(EC_{\varepsilon})$の式とするとき,
		次は{\bf EC}${}_{\varepsilon}$の公理である.
		\begin{description}
			\item[(S)] $(\varphi \rightarrow (\psi \rightarrow \chi)) 
				\rightarrow ((\varphi \rightarrow \psi)
				\rightarrow (\varphi \rightarrow \chi)).$
			\item[(K)] $\varphi \rightarrow (\psi \rightarrow \varphi).$
			\item[(DI1)] $\varphi \rightarrow (\varphi \vee \psi).$
			\item[(DI2)] $\psi \rightarrow (\varphi \vee \psi).$
			\item[(DE)] $(\varphi \rightarrow \chi) \rightarrow 
				((\psi \rightarrow \chi) \rightarrow ((\varphi \vee \psi) \rightarrow \chi)).$
			\item[(CI)] $\varphi \rightarrow (\psi \rightarrow (\varphi \wedge \psi)).$
			\item[(CE1)] $(\varphi \wedge \psi) \rightarrow \varphi.$
			\item[(CE2)] $(\varphi \wedge \psi) \rightarrow \psi.$
				
			\item[(CTI)] $\varphi \rightarrow (\rightharpoondown \varphi \rightarrow \bot).$
			
			\item[(NI)] $(\varphi \rightarrow \bot) \rightarrow\ \rightharpoondown \varphi.$
			\item[(DNE)] $\rightharpoondown \rightharpoondown \varphi \rightarrow \varphi.$
			\item[(PF)] $\varphi(t/x) \rightarrow \varphi(\varepsilon x \varphi/x).$
				\\ \textcolor{red}{ただし,$\varphi$には$x$が自由に現れて,
				$t$は$\varphi$の中で$x$への代入について自由である.}
		\end{description}
	\end{itembox}
	
	$\Gamma$を公理系とする.$L(EC_{\varepsilon})$の式$\chi$に対して$\Gamma$から
	{\bf EC}${}_{\varepsilon}$の証明が存在する(証明可能である)ことを
	\begin{align}
		\Gamma \provable{\mbox{{\bf EC}${}_{\varepsilon}$}} \chi
	\end{align}
	と書くが,{\bf EC}${}_{\varepsilon}$における$\Gamma$から$\chi$への証明とは,
	$L(EC_{\varepsilon})$の式の列$\varphi_{1},\varphi_{2},
	\cdots,\varphi_{n}$であって,$\varphi_{n}$は$\chi$であり,
	各$i \in \{1,2,\cdots,n\}$に対して
	\begin{itemize}
		\item $\varphi_{i}$は{\bf EC}${}_{\varepsilon}$の公理である.
		\item $\varphi_{i}$は$\Gamma$の公理である.
		\item $\varphi_{i}$は前の式から推論規則を用いて得られる式である.
			{\bf EC}${}_{\varepsilon}$の推論規則とは,
			\begin{description}
			\item[三段論法]
				$j,k < i$なる$k,j$が取れて,$\varphi_{k}$は
				$\varphi_{j} \rightarrow \varphi_{i}$である.
		\end{description} 
	\end{itemize}
	が満たされているものである.
	
	\item[{\bf PC}${}_{\varepsilon}$]\mbox{}
	
	\begin{itembox}[l]{{\bf PC}${}_{\varepsilon}$の公理}
		$\varphi$と$\psi$と$\xi$を$L(PC_{\varepsilon})$の式とし,$x$と$t$を変項とするとき,
		次は{\bf PC}${}_{\varepsilon}$の公理である.
		\begin{description}
			\item[(S)] $(\varphi \rightarrow (\psi \rightarrow \chi)) 
				\rightarrow ((\varphi \rightarrow \psi)
				\rightarrow (\varphi \rightarrow \chi)).$
			\item[(K)] $\varphi \rightarrow (\psi \rightarrow \varphi).$
			\item[(DI1)] $\varphi \rightarrow (\varphi \vee \psi).$
			\item[(DI2)] $\psi \rightarrow (\varphi \vee \psi).$
			\item[(DE)] $(\varphi \rightarrow \chi) \rightarrow 
				((\psi \rightarrow \chi) \rightarrow ((\varphi \vee \psi) \rightarrow \chi)).$
			\item[(CI)] $\varphi \rightarrow (\psi \rightarrow (\varphi \wedge \psi)).$
			\item[(CE1)] $(\varphi \wedge \psi) \rightarrow \varphi.$
			\item[(CE2)] $(\varphi \wedge \psi) \rightarrow \psi.$
			
			\item[(UE)] $\forall x \varphi \rightarrow \varphi(\tau/x).$
				\\ \textcolor{red}{ただし,$\varphi$には$x$が自由に現れて,
				$t$は$\varphi$の中で$x$への代入について自由である.}
				
			\item[(EI)] $\varphi(\tau/x) \rightarrow \exists x \varphi.$
				\\ \textcolor{red}{ただし,$\varphi$には$x$が自由に現れて,
				$t$は$\varphi$の中で$x$への代入について自由である.}
				
			\item[(CTI)] $\varphi \rightarrow (\rightharpoondown \varphi \rightarrow \bot).$
			
			\item[(NI)] $(\varphi \rightarrow \bot) \rightarrow\ \rightharpoondown \varphi.$
			\item[(DNE)] $\rightharpoondown \rightharpoondown \varphi \rightarrow \varphi.$
			\item[(PF)] $\varphi(t/x) \rightarrow \varphi(\varepsilon x \varphi/x).$
				\\ \textcolor{red}{ただし,$\varphi$には$x$が自由に現れて,
				$t$は$\varphi$の中で$x$への代入について自由である.}
		\end{description}
	\end{itembox}
	
	$\Gamma$を公理系とする.$L(PC_{\varepsilon})$の式$\chi$に対して
	$\Gamma$から{\bf PC}${}_{\varepsilon}$の証明が存在する(証明可能である)ことを
	\begin{align}
		\Gamma \provable{\mbox{{\bf PC}${}_{\varepsilon}$}} \chi
	\end{align}
	と書くが,{\bf PC}${}_{\varepsilon}$における$\Gamma$から$\chi$への証明とは,
	$L(PC_{\varepsilon})$の式の列$\varphi_{1},\varphi_{2},
	\cdots,\varphi_{n}$であって,$\varphi_{n}$は$\chi$であり,
	各$i \in \{1,2,\cdots,n\}$に対して
	\begin{itemize}
		\item $\varphi_{i}$は{\bf PC}${}_{\varepsilon}$の公理である.
		\item $\varphi_{i}$は$\Gamma$の公理である.
		\item $\varphi_{i}$は前の式から推論規則を用いて得られる式である.
			{\bf PC}${}_{\varepsilon}$の推論規則とは,
			\begin{description}
			\item[三段論法]
				$j,k < i$なる$k,j$が取れて,$\varphi_{k}$は
				$\varphi_{j} \rightarrow \varphi_{i}$である.
			 	
			\item[存在汎化] 
				$j < i$なる$j$が取れて,$\varphi_{j}$は
				$\varphi(t/x) \rightarrow \psi$なる式であって,
				$\varphi_{i}$は$\exists x \varphi \rightarrow \psi$なる式である.
				\\ \textcolor{red}{ただし,$\varphi$には$x$が自由に現れ,
				$t$は$\varphi$の中で$x$への代入について自由である.
				また$t$は$\varphi$と$\psi$には自由に現れない.}
			
			\item[全称汎化] 
				$j < i$なる$j$が取れて,$\varphi_{j}$は
				$\psi \rightarrow \varphi(t/x)$なる式であって,
				$\varphi_{i}$は$\psi \rightarrow \forall x \varphi$なる式である.
				\\ \textcolor{red}{ただし,$\varphi$には$x$が自由に現れ,
				$t$は$\varphi$の中で$x$への代入について自由である.
				また$t$は$\varphi$と$\psi$には自由に現れない.}
		\end{description} 
	\end{itemize}
	が満たされているものである.
	\end{description}
	\section{第一イプシロン定理メモ}
	
	言語$L(EC)$及び$L(EC_{\varepsilon})$を高橋先生の資料と同じものとする.
	{\bf 主要論理式}\index{しゅようろんりしき@主要論理式}{\bf (principal formula)}とは
	\begin{align}
		A(t) \Longrightarrow A(\varepsilon x A)
	\end{align}
	なる形の$L(EC)$の式を指す.ここで$A$とは$L(EC)$の式であって,変項$x$が$A$に自由に現れていて,
	また$A$に自由に出現するのは$x$のみである.$A(t)$とは$A$における$x$の自由な出現を全て閉項$t$に置き換えた式であり,
	$A(\varepsilon x A)$とは$A$における$x$の自由な出現を全て項$\varepsilon x A$に置き換えた式である.
	このとき$\varepsilon x A$は$A(t) \Longrightarrow A(\varepsilon x A)$に{\bf 属している}という.
	
	$EC$の公理とはトートロジーだけである.トートロジーは$EC_{\varepsilon}$の公理でもあるが,
	これに加えて主要論理式も$EC_{\varepsilon}$の公理である.
	
	$\pi = (\varphi_{0},\varphi_{1},\cdots,\varphi_{n})$を$EC_{\varepsilon}$の文の列とするとき,
	{\bf $\pi$の主要論理式}や{\bf $\pi$に現れる主要論理式}とは主要論理式である$\varphi_{i}$を指す.
	また$\pi$の主要論理式に属している$\varepsilon$項を{\bf $\pi$の主要$\varepsilon$項}と呼ぶ.
	
\subsection{埋め込み定理}
	$A$を$L(PC_{\varepsilon})$の式とするとき,$A$を$L(EC_{\varepsilon})$の式に書き換える.
	\begin{align}
		x^{\varepsilon} &\rightarrow x \\
		(\in \tau \sigma)^{\varepsilon} &\rightarrow \in \tau^{\varepsilon} \sigma^{\varepsilon} \\
		(= \tau \sigma)^{\varepsilon} &\rightarrow = \tau^{\varepsilon} \sigma^{\varepsilon} \\
		(\rightharpoondown \varphi)^{\varepsilon} &\rightarrow \rightharpoondown \varphi^{\varepsilon} \\
		(\vee \varphi \psi)^{\varepsilon} &\rightarrow \vee \varphi^{\varepsilon} \psi^{\varepsilon} \\
		(\wedge \varphi \psi)^{\varepsilon} &\rightarrow \wedge \varphi^{\varepsilon} \psi^{\varepsilon} \\
		(\Longrightarrow \varphi \psi)^{\varepsilon} &\rightarrow \Longrightarrow \varphi^{\varepsilon} \psi^{\varepsilon} \\
		(\exists x \varphi)^{\varepsilon} &\rightarrow \varphi^{\varepsilon}(\varepsilon x \varphi^{\varepsilon}) \\
		(\forall x \varphi)^{\varepsilon} &\rightarrow \varphi^{\varepsilon}(\varepsilon x \rightharpoondown \varphi^{\varepsilon}) \\
		(\varepsilon x \psi)^{\varepsilon} &\rightarrow \varepsilon x \varphi^{\varepsilon}
	\end{align}
	
	$A$が$L(PC_{\varepsilon})$の式で,$x$が$A$に自由に現れて,
	かつ$A$に自由に現れているのが$x$のみであるとき,
	$A^{\varepsilon}$にも$x$が自由に現れて,かつ$A^{\varepsilon}$に
	自由に現れているのは$x$のみである.
	
	\begin{align}
		(\varphi[x/\tau])^{\varepsilon} \rightarrow \varphi^{\varepsilon}
		(\varphi^{\varepsilon}[x/\tau^{\varepsilon}]). \\
	\end{align}
	
	\begin{itembox}[c]{$PC_{\varepsilon}$の証明を$EC_{\varepsilon}$の証明に埋め込む}
		$A$を$L(PC_{\varepsilon})$の文とし,$PC_{\varepsilon} \vdash A$であるとする.
		このとき$EC_{\varepsilon} \vdash A^{\varepsilon}$である.
	\end{itembox}
	
	示すべきことは
	\begin{itemize}
		\item $A \in Ax(PC_{\varepsilon})$ならば$\vdash A^{\varepsilon}$であること.
			\begin{itemize}
				\item $\vdash A$ならば$\vdash A^{\varepsilon}$であること.
				\item $A$に$x$が自由に現れて,かつ自由に現れているのが$x$のみであるとき,
					\begin{align}
						\vdash A^{\varepsilon}(t^{\varepsilon}) \Longrightarrow A^{\varepsilon}(\varepsilon x A^{\varepsilon})
					\end{align}
					であること.
				\item $A$に$x$が自由に現れて,かつ自由に現れているのが$x$のみであるとき,
					\begin{align}
						\vdash A^{\varepsilon}(\varepsilon x \rightharpoondown A^{\varepsilon}) \Longrightarrow A^{\varepsilon}(t^{\varepsilon})
					\end{align}
					であること.
			\end{itemize}
		
		\item $PC_{\varepsilon} \vdash B$かつ$PC_{\varepsilon} \vdash B \Longrightarrow A$である$B$が取れるとき,
			$(B \Longrightarrow A)^{\varepsilon}$は$B^{\varepsilon} \Longrightarrow A^{\varepsilon}$なので
			$EC_{\varepsilon} \vdash B^{\varepsilon}$ならば
			$EC_{\varepsilon} \vdash A^{\varepsilon}$となる.
	\end{itemize}

\subsection{階数}
	$B(x,y,z)$を,変項$x,y,z$が,そしてこれらのみが自由に現れる
	$L(EC)$の式とする.このとき
	\begin{align}
		\exists x\, \exists y\, \exists z\, B(x,y,z)
	\end{align}
	に対して,$z$から順に$\varepsilon$項に変換していくと
	\begin{align}
		&\exists x\, \exists y\, B(x,y,\varepsilon z B(x,y,z)), \\
		&\exists x\, \, B(x,\varepsilon y B(x,y,\varepsilon z B(x,y,z)),\varepsilon z B(x,\varepsilon y B(x,y,\varepsilon z B(x,y,z)),z))
	\end{align}
	となるが,最後に$\exists x$を無くすと式が長くなりすぎるのでここで止めておく.
	さて$z$に注目すれば,$B$に自由に現れていた$z$はまず
	\begin{align}
		\varepsilon z B(x,y,z)
	\end{align}
	に置き換えられている.この時点では$x,y$は自由なままであるから,この$\varepsilon$項を
	\begin{align}
		e_{1}[x,y]
	\end{align}
	と略記する.次に$y$は
	\begin{align}
		\varepsilon y B(x,y,\varepsilon z B(x,y,z))
	\end{align}
	に置き換えられるが,$e_{1}[x,y]$を使えば
	\begin{align}
		\varepsilon y B(x,y,e_{1}[x,y])
	\end{align}
	と書ける.この$\varepsilon$項でも$x$は自由なままであるから
	\begin{align}
		e_{2}[x]
	\end{align}
	と略記する.最後に
	\begin{align}
		\exists x\, B\left(x,e_{2}[x],e_{1}[x,e_{2}[x]]\right)
	\end{align}
	から$\exists$を除去するには,$x$を
	\begin{align}
		\varepsilon x B\left(x,e_{2}[x],e_{1}[x,e_{2}[x]]\right)
	\end{align}
	に置き換えれば良い.この$\varepsilon$項を$e_{3}$と書く.以上で
	$\exists x\, \exists y\, \exists z\, B(x,y,z)$は$L(EC_{\varepsilon})$の式
	\begin{align}
		B\left(e_{3},e_{2}[e_{3}],e_{1}[e_{3},e_{2}[e_{3}]]\right)
	\end{align}
	に変換されたわけである.それはさておき,ここで考察するのは{\bf 項間の主従関係}である.
	$e_{2}[x]$は$x$のみによってコントロールされているのだから,
	$x$を司る$e_{3}$を主人だと思えば$e_{2}[x]$は$e_{3}$の直属の子分である.
	$e_{1}[x,y]$は$y$によってもコントロールされているので,
	$e_{1}[x,y]$とは$e_{2}[x]$の子分であり,すなわち$e_{3}$の子分の子分であって,
	この例において一番身分が低いわけである.
	
	$\varepsilon$項を構文解析して,それが何重の子分を従えているかを測った指標を
	{\bf 階数}{\bf (rank)}と呼ぶ.とはいえ直属の子分が複数いることもあり得るので,
	子分の子分の子分の子分...と次々に枝分かれしていく従属関係の中で,最も
	深いものを辿って階数を定めることにする.
	
	\begin{screen}
		\begin{metadfn}[従属]
			$\varepsilon x A$を$L(EC_{\varepsilon})$の$\varepsilon$項とし,
			$e$を$A$に現れる$L(EC_{\varepsilon})$の$\varepsilon$項とするとき,
			$x$が$e$に自由に現れているなら$e$は{\bf $\varepsilon x A$に従属している}
			\index{じゅうぞく@従属}{\bf (subordinate to $\varepsilon x A$)}という.
		\end{metadfn}
	\end{screen}
	
	はじめの例では,$e_{2}[x]$と$e_{1}[x,e_{2}[x]]$は共に$e_{3}$に従属しているし,
	$e_{1}[x,y]$は$e_{2}[x]$に従属している.
	$e_{1}[x,e_{2}[x]]$に従属している$\varepsilon$項は無いし,
	$e_{1}[x,y]$に従属している$\varepsilon$項も無い.
	
	\begin{screen}
		\begin{metadfn}[階数]
			$\theta$を$L(EC_{\varepsilon})$の項または式とするとき,
			$\theta$の{\bf 階数}\index{かいすう@階数}{\bf (rank)}を
			以下の要領で定義する.
			\begin{enumerate}
				\item $\theta$が$\varepsilon$項でなくて,
					$\theta$に$\varepsilon$項が現れないならば,
					$\theta$の階数を$0$とする.
				\item $\theta$が$\varepsilon$項であって,かつ$\theta$に従属している
					$\varepsilon$項が無いならば,$\theta$の階数を$1$とする.
				\item $\theta$が$\varepsilon$項であって,かつ$\theta$に従属している
					$\varepsilon$項があるならば,$\theta$に従属している$\varepsilon$項の
					階数の最大値に$1$を足したものを$\theta$の階数とする.
				\item $\theta$が$\varepsilon$項でなくて,$\theta$に
					$\varepsilon$項が現れるならば,$\theta$に現れる$\varepsilon$項の
					階数の最大値を$\theta$の階数とする.
			\end{enumerate}
			また$\theta$の階数を$rk(\theta)$と書く.
		\end{metadfn}
	\end{screen}
	
	実際に$L(EC_{\varepsilon})$の全ての項および式に対して階数が定まっている.
	(構造的帰納法について準備不足だが,直感的に次の説明は妥当である...)
	\begin{description}
		\item[step1] $\theta$が$L(EC)$の項あるいは式ならば,$\theta$の階数は$0$である.
		
		\item[step2] $\theta$が$L(EC)$の式で作られた$\varepsilon$項ならば
			$\theta$の階数は$1$である.
		
		\item[step3] 項$\tau_{1},\cdots,\tau_{n}$のそれぞれに対して,
			その全ての部分$\varepsilon$項に階数が定まっていれば,
			$f$を$n$項関数として,$f\tau_{1}\cdots\tau_{n}$の階数は
			$rk(\tau_{1}),\cdots,rk(\tau_{n})$の中の最大値である.
			というのも,$f\tau_{1}\cdots\tau_{n}$に現れる$\varepsilon$項は
			$\tau_{1},\cdots,\tau_{n}$のいずれかの部分項になっているためである.
			
		\item[step4] 項$\tau_{1},\cdots,\tau_{n}$のそれぞれに対して,
			その全ての部分$\varepsilon$項に階数が定まっていれば,
			$p$を$n$項述語として,$p\tau_{1}\cdots\tau_{n}$の階数は
			$rk(\tau_{1}),\cdots,rk(\tau_{n})$の中の最大値である.
		
		\item[step5] 式$\varphi$と$\psi$のそれぞれに対して,
			その全ての部分$\varepsilon$項に階数が定まっていれば,
			\begin{align}
				rk(\rightharpoondown \varphi) &\coloneqq rk(\varphi), \\
				rk(\vee \varphi \psi) &\coloneqq \max\{rk(\varphi),rk(\psi)\}, \\
				rk(\wedge \varphi \psi) &\coloneqq \max\{rk(\varphi),rk(\psi)\}, \\
				rk(\Longrightarrow \varphi \psi) 
				&\coloneqq \max\{rk(\varphi),rk(\psi)\}, \\
			\end{align}
			である.というのも,左辺の式に現れる$\varepsilon$項は
			$\varphi$か$\psi$の少なくとも一方に現れているからである.
		
		\item[step6] 式$\varphi$に現れる全ての$\varepsilon$項に対して階数が定まっているならば,
			$\varepsilon x \varphi$の階数は定義通りに定めることが出来る.
	\end{description}
	
	\begin{screen}
		\begin{metathm}[階数定理]
			$\varepsilon x A$を$L(EC_{\varepsilon})$の$\varepsilon$項とし,
			$s$と$t$を,その中に$x$が自由に現れない$L(EC_{\varepsilon})$の項とする.このとき,
			$A$に現れる$s$の一つを$t$に置き換えた式を$A^{t}$とすれば
			\begin{align}
				rk(\varepsilon x A) = rk(\varepsilon x A^{t})
			\end{align}
			が成り立つ.$A$に$e$が現れなければ$A^{t}$は$A$とする.
		\end{metathm}
	\end{screen}
	
	\begin{screen}
		\begin{metathm}[置換定理]
			$\pi$を$L(EC_{\varepsilon})$の証明とし,
			$e$を,$\pi$の主要$\varepsilon$項の中で階数が最大であって,かつ
			階数が最大の$\pi$の主要$\varepsilon$項の中で極大であるものとする.また
			$B(s) \Longrightarrow B(\varepsilon y B)$を$\pi$の主要論理式とし,
			$e$と$\varepsilon y B$は別物であるとする.そして,$B(s) \Longrightarrow B(\varepsilon y B)$に現れる
			$e$を全て閉項$t$に置き換えた式を$C$とする.このとき,
			\begin{description}
				\item[(1)] $C$は主要論理式である.$C$に属する$\varepsilon$項を$e'$と書く.
				\item[(2)] $rk(\varepsilon y B) = rk(e')$が成り立つ.
				\item[(3)] $rk(\varepsilon y B) = rk(e)$ならば$\varepsilon y B$と$e'$は一致する.
			\end{description}
		\end{metathm}
	\end{screen}
	
	\begin{metaprf}\mbox{}
		\begin{description}
			\item[step1]
				$B(s)$ (或いは$B(\varepsilon y B)$)とは,
				$B$で自由に現れる$y$を$s$ (或いは$\varepsilon y B$)で置き換えた式である.
				$y$から代わった$s$ (或いは$\varepsilon y B$)の少なくとも一つを部分項として含む形で
				$e$が$B(s)$ (或いは$B(\varepsilon y B)$)に出現しているとする.
				
				実はこれは起こり得ない.もし起きたとすると,$e$に現れる$s$ (或いは$\varepsilon y B$)
				を元の$y$に戻した項を$e'$とすれば,
				$e'$には$y$が自由に現れるので(そうでないと$y$は$s$
				(或いは$\varepsilon y B$)に置き換えられない),$e'$は
				$y$とは別の変項$x$と適当な式$A$によって
				\begin{align}
					\varepsilon x A
				\end{align}
				なる形をしている.つまり$e'$は$\varepsilon y B$に従属していることになり
				\footnote{
					$e'$が$\varepsilon$項であって$B$に現れることの証明.
				}
				,階数定理と併せて
				\begin{align}
					rk(e) = rk(e') < rk(\varepsilon y B)
				\end{align}
				が成り立ってしまう.しかしこれは$rk(e)$が最大であることに矛盾する.
				
			\item[step2] $rk(\varepsilon y B) = rk(\pi)$ならば$B$に$e$は現れない.なぜならば,
				$e$は階数が$rk(\pi)$である$\pi$の主要$\varepsilon$項の中で極大であるからである.
				$\varepsilon y B$にも$e$は現れず,前段の結果より$B(\varepsilon y B)$に$e$が現れることもない.
				ゆえに,$s$に現れる$e$を$t$に置換した項を$s'$とすれば,$C$は
				\begin{align}
					B(s') \Longrightarrow B(\varepsilon y B)
				\end{align}
				となる.
			
			\item[step3]
				$rk(\varepsilon y B) < rk(\pi)$である場合
				\begin{align}
					rk(\varepsilon y B) = rk(e')
				\end{align}
				が成り立つことを示す.$B$に$e$が現れないならば$e'$は$\varepsilon y B$に一致する.
				$B$に$e$が現れる場合,$B$に現れる$e$を$t$に置き換えた式を$B^{t}$とする.
				このとき階数定理より
				\begin{align}
					rk(B) = rk(B^{t})
				\end{align}
				となる.ゆえに
				\begin{align}
					rk(\varepsilon y B) = rk(B) + 1 = rk(B^{t}) + 1 = rk(\varepsilon y B^{t})
				\end{align}
				となる.
				\QED
		\end{description}
	\end{metaprf}
	
\subsection{アイデア}
	
	\begin{itembox}[l]{第一イプシロン定理の流れ}
		\begin{itemize}
			\item $B$を$EC$の式とし,$B$が$PC_{\varepsilon}$から証明可能であるとする.
			\item このとき$EC_{\varepsilon}から$$B$への証明$\pi$が得られる.
			\item $e$を,$\pi$の主要$\varepsilon$項のうち階数が最大であって,かつ
				その階数を持つ$\pi$の主要$\varepsilon$項の中で次数が最大であるものとする.
			\item $e$が属する$\pi$の主要論理式の一つ$A(t) \Longrightarrow A(e)$を取る.
			\item $\pi$をベースにして,$A(t) \Longrightarrow A(e)$を用いずに
				$EC_{\varepsilon}$から$B$への証明$\pi'$を構成する.このとき以下が満たされる.
				\begin{enumerate}
					\item $A(t) \Longrightarrow A(e)$を除く$\pi$の主要論理式は
						$\pi'$の主要論理式である.
					\item また$e$が属する主要論理式については,それが$\pi'$の主要論理式であるならば$\pi$の主要論理式
						でもある.つまり,直感的に書けば
						\begin{align}
							&\Set{\varphi}{\mbox{$\varphi$は$e$が属する$\pi'$の主要論理式}} \\
							&= \Set{\varphi}{\mbox{$\varphi$は$e$が属する$\pi$の主要論理式}} \backslash \{A(t) \Longrightarrow A(e)\}
						\end{align}
						が成り立つということであって,$e$が属する主要論理式は減る一方である.
						$\pi$の主要論理式で$e$が属しているものが$A(t) \Longrightarrow A(e)$のみ
						であるならば,$\pi'$には$e$が属する主要論理式は現れない.
					
					\item $e$が属する主要論理式が$\pi'$にも残っている場合,
						$e$は$\pi'$の主要$\varepsilon$項の中も階数が最大であって,
						かつその階数を持つ$\pi'$の主要$\varepsilon$項の中で極大である.
					
					\item $\pi'$の主要$\varepsilon$項のうち,階数が$e$と同じであるものは
						$\pi$の主要$\varepsilon$項でもあった($e$と同じ階数の$\varepsilon$項は増えない).
				\end{enumerate}
				
			\item 証明$\pi$の主要$\varepsilon$項の階数の最大値を$rk(\pi)$とする.
				また主要論理式に属する$\varepsilon$項の階数を,その主要論理式の階数と呼ぶことにする.
				前段の操作を続けていけば,まずは階数$rk(\pi)$の主要論理式を全く用いない
				$B$への証明$\pi_{1}$が得られる.このとき$rk(\pi_{1})$は
				$rk(\pi)$よりも小さい.同様にして階数$rk(\pi_{1})$の主要論理式を全く用いない
				$B$への証明$\pi_{2}$が得られる.もちろん$rk(\pi_{2})$は
				$rk(\pi_{1})$よりも小さい.これを繰り返していけば,いずれは主要論理式を全く用いない
				$B$への証明$\pi^{\ast}$が得られる.$\pi^{\ast}$にはトートロジーか
				モーダスポンネスで導かれる式しかない.
				あとは,$\pi^{\ast}$に現れる$\varepsilon$項を$EC$の項に置き換えれば,その式の列は
				$EC$から$B$への証明となっている.
		\end{itemize}
	\end{itembox}
	
	$\pi$を$\varphi_{0},\varphi_{1},\cdots,\varphi_{n}$とし,
	$\varphi_{0},\varphi_{1},\cdots,\varphi_{n}$に現れる$e$を$t$に置き換えた式を
	\begin{align}
		\tilde{\varphi}_{0},\ \tilde{\varphi}_{1},\cdots, \tilde{\varphi}_{n}
	\end{align}
	と書く($e$は,どれかの項の部分項であるときも置き換える?).
	このとき,任意の$0 \leq i \leq n$で
	\begin{enumerate}
		\item $\varphi_{i}$がトートロジーなら$\tilde{\varphi}_{i}$もトートロジーである.
		\item $\varphi_{i}$が主要論理式で,$e$が$\varphi_{i}$の主要項であるならば,
			$\tilde{\varphi}_{i}$は$A(u) \Longrightarrow A(t)$なる形の式である
			\footnotemark.
		\item $\varphi_{i}$が主要論理式で,$e$が$\varphi_{i}$の主要項ではないならば,
			$\tilde{\varphi}_{i}$も主要論理式である.
	\end{enumerate}
	
	\footnotetext{
		$\varepsilon x A$と$\varepsilon y B$が記号列として一致すれば,
		$x$と$y$は一致するし,式$A$と式$B$も一致するので
		$A(\varepsilon x A)$と$B(\varepsilon y B)$も記号列として一致する.
	}
	
	$\varphi$が$A(t) \Longrightarrow A(e)$でない$EC_{\varepsilon}$の公理ならば,
	$\tilde{\varphi}_{i}$と$\tilde{\varphi}_{i+1}$の間に
	\begin{align}
		&\tilde{\varphi}_{i} \Longrightarrow 
		\left( A(t) \Longrightarrow \tilde{\varphi}_{i} \right), \\
		&A(t) \Longrightarrow \tilde{\varphi}_{i}
	\end{align}
	を挿入する.$\varphi_{i}$が$\varphi_{j}$と$\varphi_{k}$からモーダスポンネスで得られる場合は,
	$\tilde{\varphi}_{i}$を
	\begin{align}
		&\left( A(t) \Longrightarrow \tilde{\varphi}_{j} \right)
		\Longrightarrow \left[ \left( A(t) \Longrightarrow 
		\left( \tilde{\varphi}_{j}\Longrightarrow \tilde{\varphi}_{i} \right) \right)
		\Longrightarrow \left( A(t) \Longrightarrow \tilde{\varphi}_{i} \right) \right], \\
		&\left( A(t) \Longrightarrow 
		\left( \tilde{\varphi}_{j}\Longrightarrow \tilde{\varphi}_{i} \right) \right)
		\Longrightarrow \left( A(t) \Longrightarrow \tilde{\varphi}_{i} \right), \\
		&A(t) \Longrightarrow \tilde{\varphi}_{i}
	\end{align}
	で置き換える.すると,$A(t) \Longrightarrow A(e)$を使わない
	$EC_{\varepsilon}$から$A(t) \Longrightarrow B$への証明が得られる.
	$\varphi_{i}$が$e$が属する主要論理式$A(s) \Longrightarrow A(e)$であるときは,
	$\tilde{\varphi}_{i}$とは
	\begin{align}
		A(s') \Longrightarrow A(t)
	\end{align}
	なる形の式であるが
	\footnote{
		$x$を$A$に現れている自由な変項とすれば,$e$とは$\varepsilon x A$のことであるし,
		$A(\varepsilon x A)$とは$A$に自由に現れる$x$を$\varepsilon x A$に置換した式である.
		$A$には$\varepsilon x A$は現れていないので,というのも$\varepsilon x A$が登場するのは
		$A$が作られた後であるからだが,$A(e)$に現れる$e$を$t$に変換した式は
		$A(t)$になる.同様に,$A(s)$に$e$が現れるとすれば,その$e$は$y$に代入された$s$の
		部分項でしかありえない.すなわち,$A(s)$に現れる$e$を$t$で置換した式は,
		$s'$を$s$に現れる$e$を$t$に変換した項として ($s$に$e$が現れなければ$s'$は$s$である)
		$A(s')$となるわけである.
	},$\tilde{\varphi}_{i}$を
	\begin{align}
		&A(t) \Longrightarrow (A(s') \Longrightarrow A(t)), \\
		&A(s') \Longrightarrow A(t)
	\end{align}
	で置き換える.
	
	同様に$A(t) \Longrightarrow A(e)$を使わない$EC_{\varepsilon})$から
	$\rightharpoondown A(t) \Longrightarrow B$への証明を構成する.
	今度は$\pi$に現れる$e$を$t$に置き換える必要はない.
	$\varphi_{i}$が$A(t) \Longrightarrow A(e)$でない$EC_{\varepsilon}$の公理ならば,
	$\varphi_{i}$と$\varphi_{i+1}$の間に
	\begin{align}
		&\varphi_{i} \Longrightarrow (\rightharpoondown A(t) \Longrightarrow \varphi_{i}), \\
		&\rightharpoondown A(t) \Longrightarrow \varphi_{i}
	\end{align}
	を挿入する.$\varphi_{i}$が$\varphi_{j}$と$\varphi_{k}$からモーダスポンネスで得られる場合は,
	$\varphi_{i}$を
	\begin{align}
		&(\rightharpoondown A(t) \Longrightarrow \varphi_{j}) \Longrightarrow
		[(\rightharpoondown A(t) \Longrightarrow 
		(\varphi_{j}\Longrightarrow \varphi_{i}))
		\Longrightarrow (\rightharpoondown A(t) \Longrightarrow \varphi_{i})], \\
		&(\rightharpoondown A(t) \Longrightarrow 
		(\varphi_{j} \Longrightarrow \varphi_{i}))
		\Longrightarrow (\rightharpoondown A(t) \Longrightarrow \varphi_{i}), \\
		&\rightharpoondown A(t) \Longrightarrow \varphi_{i}
	\end{align}
	で置き換える.$\varphi_{i}$が$A(t) \Longrightarrow A(e)$であるときは,$\varphi_{i}$を
	\begin{align}
		\rightharpoondown A(t) \Longrightarrow (A(t) \Longrightarrow A(e))
	\end{align}
	で置き換える.
	
	以上で$A(t) \Longrightarrow B$と$\rightharpoondown A(t) \Longrightarrow B$に対して
	$A(t) \Longrightarrow A(e)$を用いない$EC_{\varepsilon}$からの証明が得られた.後はこれに
	\begin{align}
		&(A(t) \Longrightarrow B) \Longrightarrow
		((\rightharpoondown A(t) \Longrightarrow B) \Longrightarrow
		((A(t) \Longrightarrow B) \wedge (\rightharpoondown A(t) \Longrightarrow B))), \\
		&(\rightharpoondown A(t) \Longrightarrow B) \Longrightarrow
		((A(t) \Longrightarrow B) \wedge (\rightharpoondown A(t) \Longrightarrow B)), \\
		&(A(t) \Longrightarrow B) \wedge (\rightharpoondown A(t) \Longrightarrow B), \\
		&((A(t) \Longrightarrow B) \wedge (\rightharpoondown A(t) \Longrightarrow B))
		\Longrightarrow ((A(t) \vee \rightharpoondown A(t)) \Longrightarrow B), \\
		&(A(t) \vee \rightharpoondown A(t)) \Longrightarrow B, \\
		&A(t) \vee \rightharpoondown A(t), \\
		&B
	\end{align}
	を追加すれば,$A(t) \Longrightarrow A(e)$を用いない$EC_{\varepsilon}$から$B$への証明となる.
	
	\section{第二イプシロン定理}
	$\exists x \forall y \exists z B(x,y,z)$を$L(PC)$の冠頭標準形とする.
	つまり$B(x,y,z)$には量化子が現れないので,$B(x,y,z)$は$L(EC)$の式ということである.
	また
	\begin{align}
		PC_{\varepsilon} \vdash \exists x \forall y \exists z B(x,y,z)
	\end{align}
	であるとする.
	
	$f$を$L(PC)$には無い一変数関数記号とし,
	\begin{align}
		L'(PC) &\defeq L(PC) \cup \{f\}, \\
		L'(EC) &\defeq L(EC) \cup \{f\}, \\
		L'(PC_{\varepsilon}) &\defeq L(PC_{\varepsilon}) \cup \{f\}, \\
		L'(EC_{\varepsilon}) &\defeq L(EC_{\varepsilon}) \cup \{f\}
	\end{align}
	とする.このとき明らかに
	\begin{align}
		{PC'}_{\varepsilon} \vdash \exists x \forall y \exists z B(x,y,z)
	\end{align}
	であるが(ただし${PC'}_{\varepsilon} \vdash$とは$L'(PC_{\varepsilon})$の
	式からなる証明が存在するという意味),
	\begin{align}
		{PC'}_{\varepsilon} &\vdash \exists x \forall y \exists z B(x,y,z), \\
		{PC'}_{\varepsilon} &\vdash \exists x \forall y \exists z B(x,y,z)
		\rightarrow \forall y \exists z B(\tau,y,z), && 
		(\tau \defeq \varepsilon x \forall y \exists z B(x,y,z)) \\
		{PC'}_{\varepsilon} &\vdash \forall y \exists z B(\tau,y,z), \\
		{PC'}_{\varepsilon} &\vdash \forall y \exists z B(\tau,y,z)
		\rightarrow \exists z B(\tau,f(\tau),z), \\
		{PC'}_{\varepsilon} &\vdash \exists z B(\tau,f(\tau),z), \\
		{PC'}_{\varepsilon} &\vdash \exists z B(\tau,f(\tau),z)
		\rightarrow \exists x \exists z B(x,f(x),z), \\
		{PC'}_{\varepsilon} &\vdash \exists x \exists z B(x,f(x),z)
	\end{align}
	が成り立つ.すると拡張第一イプシロン定理より,$p$個の$L'(EC)$の項$r_{i}$
	と,同じく$p$個の$L'(EC)$の項$s_{i}$が取れて,
	\begin{align}
		{EC'}_{\varepsilon} \vdash \bigvee_{i=1}^{p} B(r_{i},f(r_{i}),s_{i})
	\end{align}
	となる.同じ証明で
	\begin{align}
		{PC'}_{\varepsilon} \vdash \bigvee_{i=1}^{p} B(r_{i},f(r_{i}),s_{i})
	\end{align}
	であることも言える.
	\begin{align}
		{PC'}_{\varepsilon} \vdash \bigvee_{i=1}^{p-1} B(r_{i},f(r_{i}),s_{i})
		\vee B(r_{p},f(r_{p}),s_{p})
	\end{align}
	より,まず
	\begin{align}
		{PC'}_{\varepsilon} \vdash \bigvee_{i=1}^{p-1} B(r_{i},f(r_{i}),s_{i})
		\vee \exists z B(r_{p},f(r_{p}),z)
	\end{align}
	となる.続いて,$f(r_{p})$は$\bigvee_{i=1}^{p-1} B(r_{i},f(r_{i}),s_{i})$には現れないので
	\begin{align}
		{PC'}_{\varepsilon} \vdash \bigvee_{i=1}^{p-1} B(r_{i},f(r_{i}),s_{i})
		\vee \forall y \exists z B(r_{p},y,z)
	\end{align}
	となる.最後に
	\begin{align}
		{PC'}_{\varepsilon} \vdash \bigvee_{i=1}^{p-1} B(r_{i},f(r_{i}),s_{i})
		\vee \exists x \forall y \exists z B(x,y,z)
	\end{align}
	となる.これを繰り返せば
	\begin{align}
		{PC'}_{\varepsilon} \vdash \exists x \forall y \exists z B(x,y,z)
		\vee \cdots \vee \exists x \forall y \exists z B(x,y,z)
	\end{align}
	が得られるので
	\begin{align}
		{PC'}_{\varepsilon} \vdash \exists x \forall y \exists z B(x,y,z)
	\end{align}
	となる.最後に,$\exists x \forall y \exists z B(x,y,z)$への証明に残っている
	$f$を含む項を$L(PC)$の項に置き換えれば,$L(PC)$から$\exists x \forall y \exists z B(x,y,z)$
	への証明が得られる.

\chapter{メモ}
	\section{量化再考}
	\begin{screen}
		\begin{logicalaxm}[量化の公理]\mbox{}
			\begin{enumerate}
				\item $\forall y\, \left(\, \forall x \varphi(x) \Longrightarrow \varphi(y)\, \right)$
				\item $\forall x\, \left(\, \varphi(x) \Longrightarrow \exists y \varphi(y)\, \right)$
				\item $\forall y\, \left(\, \varphi \Longrightarrow \psi(y)\, \right)
					\Longrightarrow \left(\, \varphi \Longrightarrow \forall y \psi(y)\, \right)$
				\item $\forall x\, \left(\, \varphi(x) \Longrightarrow \psi\, \right)
					\Longrightarrow \left(\, \exists x \varphi(x) \Longrightarrow \psi\, \right)$
				\item $\forall x\,  \left(\, \varphi(x) \Longrightarrow \psi(x)\, \right)
					\Longrightarrow \left(\, \forall x \varphi(x) \Longrightarrow \forall x \psi(x)\, \right)$
			
				\item $\forall x \varphi(x) \Longrightarrow \exists x \varphi(x)$	
			\end{enumerate}
		\end{logicalaxm}
	\end{screen}
	
	\begin{screen}
		$\forall x \varphi(x) \Longrightarrow \forall y \varphi(y).$
	\end{screen}
	
	\begin{align}
		&\forall y\, \left(\, \forall x \varphi(x) \Longrightarrow \varphi(y)\, \right)
		&& \mbox{(公理1)} \\
		&\forall y\, \left(\, \forall x \varphi(x) \Longrightarrow \varphi(y)\, \right)
		\Longrightarrow \left(\, \forall x \varphi(x) \Longrightarrow \forall y \varphi(y)\, \right),
		&& \mbox{(公理3)} \\
		&\forall x \varphi(x) \Longrightarrow \forall y \varphi(y).
		&& \mbox{(MP)}
	\end{align}
	
	\begin{screen}
		$\exists x \varphi(x) \Longrightarrow \exists y \varphi(y).$
	\end{screen}
	
	\begin{align}
		&\forall x\, \left(\, \varphi(x) \Longrightarrow \exists y \varphi(y)\, \right)
		&& \mbox{(公理2)} \\
		&\forall x\, \left(\, \varphi(x) \Longrightarrow \exists y \varphi(y)\, \right)
		\Longrightarrow \left(\, \exists x \varphi(x) \Longrightarrow \exists y \varphi(y)\, \right),
		&& \mbox{(公理4)} \\
		&\exists x \varphi(x) \Longrightarrow \exists y \varphi(y).
		&& \mbox{(MP)}
	\end{align}
	
	\begin{screen}
		$\forall x\,  \left(\, \varphi(x) \Longrightarrow \psi\, \right)
		\Longrightarrow \left(\, \forall x \varphi(x) \Longrightarrow \psi\, \right).$
	\end{screen}
	
	$\forall x\,  \left(\, \varphi(x) \Longrightarrow \psi\, \right)$
	と$\forall x \varphi(x)$からなる文の集合を$\Gamma$とすると,
	\begin{align}
		\Gamma &\vdash \forall x\,  \left(\, \varphi(x) \Longrightarrow \psi\, \right) \\
		\Gamma &\vdash \forall x\,  \left(\, \varphi(x) \Longrightarrow \psi\, \right) \Longrightarrow \left(\, \exists x \varphi(x) \Longrightarrow \psi\, \right) \\
		\Gamma &\vdash \exists x \varphi(x) \Longrightarrow \psi \\
		& \\
		\Gamma &\vdash \forall x \varphi(x) \\
		\Gamma &\vdash \forall x \varphi(x) \Longrightarrow \exists x \varphi \\
		\Gamma &\vdash \exists x \varphi(x) \\
		& \\
		\Gamma &\vdash \psi
	\end{align}
	が成り立つので,演繹法則より
	\begin{align}
		\vdash \forall x\,  \left(\, \varphi(x) \Longrightarrow \psi\, \right)
		\Longrightarrow \left(\, \forall x \varphi(x) \Longrightarrow \psi\, \right)
	\end{align}
	が得られる.
	
	\begin{screen}
		$\tau$を$\mathcal{L}_{\in}$には無い定数記号として,
		$\mathcal{L}_{\in}' = \mathcal{L}_{\in} \cup \{\tau\}$とおく.
		$\varphi$を$\mathcal{L}_{\in}'$の式とし,
		\begin{align}
			\Sigma \vdash_{\mathcal{L}_{\in}'} \varphi
		\end{align}
		であるとする.項$x$を,もし$\tau$が$\varphi$に現れるならば
		$\varphi$の中の$\tau$の出現位置で束縛されない変項とする.
		このとき,$\tau$が$\varphi$に現れるならば
		\begin{align}
			\Sigma \vdash_{\mathcal{L}_{\in}} \forall x \varphi(x/\tau)
		\end{align}
		が成り立つ.$\tau$が$\varphi$に現れなければ
		\begin{align}
			\Sigma \vdash_{\mathcal{L}_{\in}} \varphi
		\end{align}
		が成り立つ.
	\end{screen}
	
	\begin{sketch}
		$\varphi$が$\Sigma$の公理であるときは
		\begin{align}
			\Sigma \vdash_{\mathcal{L}_{\in}} \varphi
		\end{align}
		となるし,$\varphi$が推論法則であるときは,$\varphi$に$\tau$が現れなければ
		\begin{align}
			\Sigma \vdash_{\mathcal{L}_{\in}} \varphi
		\end{align}
		となるし,$\varphi$に$\tau$が現れても
		\begin{align}
			\Sigma \vdash_{\mathcal{L}_{\in}} \forall x \varphi(x/\tau)
		\end{align}
		が成立する.$\varphi$が三段論法によって示されるとき,つまり$\mathcal{L}_{\in}'$の文$\psi$で
		\begin{align}
			\Sigma &\vdash_{\mathcal{L}_{\in}'} \psi, \\
			\Sigma &\vdash_{\mathcal{L}_{\in}'} \psi \Longrightarrow \varphi
		\end{align}
		を満たすものが取れるとき,$y$を$\psi$にも$\varphi$にも表れない変項とする.
		また$\varphi$と$\psi$に$\tau$が現れているかいないかで
		\begin{description}
			\item[case1] $\varphi$にも$\psi$にも$\tau$が現れていないとき,
				\begin{align}
					\Sigma \vdash_{\mathcal{L}_{\in}} \psi
				\end{align}
				かつ
				\begin{align}
					\Sigma \vdash_{\mathcal{L}_{\in}} \psi \Longrightarrow \varphi
				\end{align}
				
			\item[case2] $\varphi$には$\tau$が現れているが,$\psi$には$\tau$が現れていないとき,
				\begin{align}
					\Sigma \vdash_{\mathcal{L}_{\in}} \psi
				\end{align}
				かつ
				\begin{align}
					\Sigma \vdash_{\mathcal{L}_{\in}} 
					\forall y\, (\, \psi \Longrightarrow \varphi(y/\tau)\, )
				\end{align}
				
			\item[case3] $\varphi$には$\tau$が現れていないが,$\psi$には$\tau$が現れているとき,
				\begin{align}
					\Sigma \vdash_{\mathcal{L}_{\in}} \forall y \psi(y/\tau)
				\end{align}
				かつ
				\begin{align}
					\Sigma \vdash_{\mathcal{L}_{\in}} 
					\forall y\, (\, \psi(y/\tau) \Longrightarrow \varphi\, )
				\end{align}
				
			\item[case4] $\varphi$にも$\psi$にも$\tau$が現れているとき,
				\begin{align}
					\Sigma \vdash_{\mathcal{L}_{\in}} \forall y \psi(y/\tau)
				\end{align}
				かつ
				\begin{align}
					\Sigma \vdash_{\mathcal{L}_{\in}} \forall y\, (\, \psi(y/\tau) \Longrightarrow \varphi(y/\tau)\, )
				\end{align}
				
		\end{description}
		のいずれかのケースを一つ仮定する.
		\begin{description}
			\item[case1] 証明可能性の定義より
				\begin{align}
					\Sigma \vdash_{\mathcal{L}_{\in}} \varphi
				\end{align}
				が成り立つ.
				
			\item[case2]
				公理2と併せて
				\begin{align}
					\Sigma \vdash_{\mathcal{L}_{\in}} \psi \Longrightarrow \forall y \varphi(y/\tau)
				\end{align}
				が成り立つので,証明可能性の定義より
				\begin{align}
					\Sigma \vdash_{\mathcal{L}_{\in}} \forall y \varphi(y/\tau)
				\end{align}
				となる.そして
				\begin{align}
					\Sigma \vdash_{\mathcal{L}_{\in}} \forall x \varphi(x/\tau)
				\end{align}
				も成り立つ.
				
			\item[case3]
				\begin{align}
					\Sigma \vdash_{\mathcal{L}_{\in}} \forall y \psi(y/\tau) \Longrightarrow \varphi
				\end{align}
				が成り立つので,,証明可能性の定義より
				\begin{align}
					\Sigma \vdash_{\mathcal{L}_{\in}} \varphi
				\end{align}
				となる.
				
			\item[case4]
				公理5より
				\begin{align}
					\Sigma \vdash_{\mathcal{L}_{\in}} \forall y \psi(y/\tau) \Longrightarrow \forall y \varphi(y/\tau)
				\end{align}
				が成り立つので,証明可能性の定義より
				\begin{align}
					\Sigma \vdash_{\mathcal{L}_{\in}} \forall y \varphi(y/\tau)
				\end{align}
				となる.そして
				\begin{align}
					\Sigma \vdash_{\mathcal{L}_{\in}} \forall x \varphi(x/\tau)
				\end{align}
				も成り立つ.
				\QED
		\end{description}
	\end{sketch}
	
	\begin{screen}
		$\forall x \varphi(x),\ \forall x\, (\, \varphi(x) \Longrightarrow \psi(x)\, )
		\vdash \forall x \psi(x).$
	\end{screen}
	
	公理5より
	\begin{align}
		\forall x\, (\, \varphi(x) \Longrightarrow \psi(x)\, )
		\vdash \forall x \varphi(x) \Longrightarrow \forall x \psi(x)
	\end{align}
	が成り立つので,三段論法より
	\begin{align}
		\forall x \varphi(x),\ \forall x\, (\, \varphi(x) \Longrightarrow \psi(x)\, )
		\vdash \forall x \psi(x)
	\end{align}
	が従う.
	
	\begin{screen}
		$\tau$を定項とし,$\mathcal{L}_{\in}' = \mathcal{L}_{\in} \cup \{\tau\}$とする.
		また$\varphi$を$\mathcal{L}_{\in}$の式とし,項$x$が$\varphi$に自由に現れて,
		また$\varphi$で自由に現れる項は$x$のみであるとする.このとき
		$\vdash_{\mathcal{L}_{\in}'} \varphi(\tau)$なら
		$\vdash_{\mathcal{L}_{\in}} \forall x \varphi(x)$.
	\end{screen}
	
	\begin{screen}
		\begin{logicalthm}[De Morgan 1]
			$\vdash_{\mathcal{L}_{\in}} \forall x \rightharpoondown \varphi(x) 
			\Longrightarrow\ \rightharpoondown \exists x \varphi(x).$
		\end{logicalthm}
	\end{screen}
	
	公理4より
	\begin{align}
		\vdash_{\mathcal{L}_{\in}} \forall x\, (\, \varphi(x) \Longrightarrow\ 
		\rightharpoondown \forall x \rightharpoondown \varphi(x)\, )
		\Longrightarrow (\, \exists x \varphi(x)
		\Longrightarrow\ \rightharpoondown \forall x \rightharpoondown \varphi(x)\, )
	\end{align}
	が成り立ち,また公理1より
	\begin{align}
		\vdash_{\mathcal{L}_{\in}}
		\forall x\, (\, \forall x \rightharpoondown \varphi(x)
		\Longrightarrow \varphi(x)\, )
	\end{align}
	が成り立つので
	\begin{align}
		\vdash_{\mathcal{L}_{\in}}
		\forall x\, (\, \rightharpoondown \varphi(x)
		\Longrightarrow\ \rightharpoondown \forall x \rightharpoondown \varphi(x)\, )
	\end{align}
	も成り立つ.そして三段論法より
	\begin{align}
		\vdash_{\mathcal{L}_{\in}} \exists x \varphi(x)
		\Longrightarrow\ \rightharpoondown \forall x \rightharpoondown \varphi(x)
	\end{align}
	が従う.ゆえに
	\begin{align}
		\vdash_{\mathcal{L}_{\in}} \forall x \rightharpoondown \varphi(x) 
		\Longrightarrow\ \rightharpoondown \exists x \varphi(x)
	\end{align}
	となる.
	
	\begin{screen}
		\begin{logicalthm}[De Morgan 2]
			$\vdash_{\mathcal{L}_{\in}} \rightharpoondown \exists x \varphi(x)
			\Longrightarrow \forall x \rightharpoondown \varphi(x).$
		\end{logicalthm}
	\end{screen}
	
	\begin{align}
		&\forall y\, (\, \varphi(y) \Longrightarrow \exists x \varphi(x)\, )
		&& \mbox{(公理2)} \\
		&\forall y\, (\, \rightharpoondown \exists x \varphi(x) 
		\Longrightarrow\ \rightharpoondown \varphi(y)\, )
		&& \mbox{()} \\
		&\forall y\, (\, \rightharpoondown \exists x \varphi(x) 
		\Longrightarrow\ \rightharpoondown \varphi(y)\, )
		\Longrightarrow (\, \rightharpoondown \exists x \varphi(x) 
		\Longrightarrow \forall y \rightharpoondown \varphi(y)\, )
		&& \mbox{(公理3)} \\
		&\rightharpoondown \exists x \varphi(x)
		\Longrightarrow \forall y \rightharpoondown \varphi(y)
		&& \mbox{(MP)} \\
		&\forall y \rightharpoondown \varphi(y)
		\Longrightarrow \forall x \rightharpoondown \varphi(x)
		&& \mbox{()} \\
		&\rightharpoondown \exists x \varphi(x)
		\Longrightarrow \forall x \rightharpoondown \varphi(x)
		&& \mbox{()}
	\end{align}
	より.
	
	\begin{screen}
		$\rightharpoondown \forall x \varphi(x) \Longrightarrow
		\exists x \rightharpoondown \varphi(x).$
	\end{screen}
	
	公理7より
	\begin{align}
		\vdash \rightharpoondown \forall x 
		\rightharpoondown \rightharpoondown \varphi(x)
		\Longrightarrow \exists x \rightharpoondown \varphi(x)
	\end{align}
	が成り立つ.また
	\begin{align}
		\vdash \forall x\, (\, \rightharpoondown \rightharpoondown \varphi(x)
		\Longrightarrow \varphi(x)\, )
	\end{align}
	と公理5より
	\begin{align}
		\vdash \forall x \rightharpoondown \rightharpoondown \varphi(x)
		\Longrightarrow \forall x \varphi(x)
	\end{align}
	が成り立つので,対偶を取って
	\begin{align}
		\vdash\ \rightharpoondown \forall x \varphi(x) \Longrightarrow\ 
		\rightharpoondown \forall x \rightharpoondown \rightharpoondown \varphi(x)
	\end{align}
	が得られる.よって
	\begin{align}
		\vdash\ \rightharpoondown \forall x \varphi(x) \Longrightarrow
		\exists x \rightharpoondown \varphi(x)
	\end{align}
	となる.
	
	\begin{screen}
		$\vdash \exists x \rightharpoondown \varphi(x) 
		\Longrightarrow\ \rightharpoondown \forall x \varphi(x).$
	\end{screen}
	
	$\vdash_{\mathcal{L}_{\in}'} \varphi(\tau) \Longrightarrow\ 
	\rightharpoondown \rightharpoondown \varphi(\tau)$より
	\begin{align}
		\vdash \forall x \varphi(x) \Longrightarrow
		\forall x \rightharpoondown \rightharpoondown \varphi(x)
	\end{align}
	が成り立ち,さらに公理7より
	\begin{align}
		\vdash \forall x \rightharpoondown \rightharpoondown \varphi(x)
		\Longrightarrow\ \rightharpoondown \exists x \rightharpoondown \varphi(x)
	\end{align}
	が成り立つので,
	\begin{align}
		\vdash \forall x \varphi(x) \Longrightarrow
		\Longrightarrow\ \rightharpoondown \exists x \rightharpoondown \varphi(x)
	\end{align}
	が得られる.
	\section{置換公理}
	置換公理の二つの形式の同値性をざっくりと.
	\begin{description}
		\item[(T)] $\sing{f} \Longrightarrow \forall a\, \set{f \ast a}.$
		\item[(K)] $\forall a\, \left[\, \forall x \in a\, \exists!y \varphi(x,y)
				\Longrightarrow \exists z\, \forall y\,
				(\, y \in z \Longleftrightarrow \exists x\, (\, x \in a \wedge 
				\varphi(x,y)\, )\, )\, \right].$
	\end{description}
	
	ただし
	\begin{align}
		\sing{f} &\defarrow \forall x,y,z\, (\, (x,y) \in f \wedge (x,z) \in f
		\Longrightarrow y = z\, ), \\
		f \ast a &\defeq \Set{y}{\exists x \in a\, (\, (x,y) \in f\, )}, \\
		\set{s} \defarrow \exists x\, (\, s = x\, )
	\end{align}
	であるし,$\varphi$に自由に現れているのは二つの変項のみで,それらが$s$と$t$とおけば,
	$\varphi$に自由に現れている$s$を全て$x$に,
	$\varphi$に自由に現れている$t$を全て$y$に置き換えた式が
	\begin{align}
		\varphi(x,y)
	\end{align}
	である.またこのとき$x$も$y$も$\varphi(x,y)$で束縛されていないものとする
	($x$と$y$はそのように選ばれた変項であるということである).
	
	\begin{description}
		\item[(T)$\Longrightarrow$(K)]
			$a$を任意の集合とし,
			\begin{align}
				\forall x \in a \exists!y \varphi(x,y)
			\end{align}
			であるとする.
			\begin{align}
				f \defeq \Set{(x,y)}{x \in a \wedge \varphi(x,y)}
			\end{align}
			とおけば$f$は$a$上の写像であって,(T)より
			\begin{align}
				\exists z\, (\, z = f \ast a\, )
			\end{align}
			となる.ところで$f \ast a$とは
			\begin{align}
				\Set{y}{\exists x \in a\, (\, (x,y) \in f\, )}
			\end{align}
			なので
			\begin{align}
				f \ast a = \Set{y}{\exists x \in a \varphi(x,y)}.
			\end{align}
			ゆえに
			\begin{align}
				\exists z\, \forall y\, (\, y \in z \Longleftrightarrow
				\exists x \in a \varphi(x,y)\, )
			\end{align}
			が成り立つ.
			
		\item[(K)$\Longrightarrow$(T)]
			$\sing{f}$とし,$a$を集合とする.
			\begin{align}
				b \defeq a \cap \dom{f}
			\end{align}
			とおけば,(K)からは分出公理が示せるので$b$は集合である.そして
			\begin{align}
				\forall x \in b\, \exists!y\, (\, (x,y) \in f\, )
			\end{align}
			が成り立つのだから,(K)より
			\begin{align}
				z = \Set{y}{\exists x \in b\, (\, (x,y) \in f\, )}
			\end{align}
			が従う.ここで
			\begin{align}
				\Set{y}{\exists x \in b\, (\, (x,y) \in f\, )}
				= f \ast b
				= f \ast a
			\end{align}
			であるから(T)が得られる.
			\QED
	\end{description}

\chapter{Hilbert流証明論}
		\begin{flushleft}
		参考文献: 戸次大介「数理論理学」
	\end{flushleft}
	
	この章に出てくる式と項は言語$\mathcal{L}_{\in}$のものとする.
	証明論の奇妙なところは,扱う式が文とは限らないところである.
	式に自由な変項が残ったままであるとその式の意味は定まらない.
	逆に変更が全て束縛されている文は,それが表す意味は非常に判然としている.
	奇妙なのは,$\varphi$が文であって,これが証明されたとしても,その証明の過程には
	文でない式が出現し得る点である.意味が不明瞭な式をもって意味がはっきり
	定まった式を導こうというところが腑に落ちない.と今まで思っていたが,
	どうも勉強しているうちにそれほど不自然には感じなくなってきた.
	
	%たとえば,$\psi$と$\psi \rightarrow \varphi$が証明されれば通常は三段論法から
	%$\varphi$が導かれるわけだが,$\psi$も$\varphi$も文であるとは限らない.

\section{{\bf HK}}
	証明体系には様々な流派があるが,流派の一つHilbert流と呼ばれる証明体系のうちで最も標準的なものが
	{\bf HK}であると,と理解している.
	証明のシステムを概括するために,抽象的に{\bf H}をHilbert流の証明体系とする.
	はじめに,{\bf H}の(論理的)公理と推論規則と言われるものが与えられる.
	また証明の前には公理系と呼ばれる式の集合も与えられる.公理系に属する式をその公理系の公理と呼ぶが,
	公理は意味のはっきりした式であるべきだと思うので{\bf 全て文とする}.
	いま公理系を$\Gamma$とすれば,$\varphi$が$\Gamma$の公理であるか{\bf H}の公理であることを
	\begin{align}
		\Gamma \provable{\mbox{{\bf H}}} \varphi
	\end{align}
	と書く.通常は公理が全くない場合も考察対象であり,その場合は$\varphi$が{\bf H}の公理であることを
	\begin{align}
		\provable{\mbox{{\bf H}}} \varphi
	\end{align}
	と書くのである.$\varphi$が一般の式である場合は,
	$\Gamma \provable{\mbox{{\bf H}}} \varphi$
	なることを「$\varphi$は$\Gamma$の下での定理である」といった趣旨の言い方をする.
	つまり$\Gamma$の公理や{\bf H}の公理は$\Gamma$の下での定理であるわけであるが,
	他の式については,それがすでに既に定理とされた式から{\bf H}の推論規則によって得られている
	ときに限り定理となる.
	
	まずは一番弱い体系の{\bf SK}から始める.以下で$\Gamma$と書いたらそれは公理系を表す.
	
\subsection{{\bf SK}}
	\begin{itembox}[l]{{\bf SK}の公理}
		$\varphi$と$\psi$と$\xi$を式とするとき,次は{\bf SK}の公理である.
		\begin{description}
			\item[(S)] $(\varphi \rightarrow (\psi \rightarrow \chi)) 
				\rightarrow ((\varphi \rightarrow \psi)
				\rightarrow (\varphi \rightarrow \chi)).$
			
			\item[(K)] $\varphi \rightarrow (\psi \rightarrow \varphi).$
		\end{description}
	\end{itembox}
	
	そして
	\begin{align}
		\provable{\mbox{{\bf SK}}} (\varphi \rightarrow (\psi \rightarrow \chi)) 
			\rightarrow ((\varphi \rightarrow \psi)
			\rightarrow (\varphi \rightarrow \chi))
	\end{align}
	および
	\begin{align}
		\provable{\mbox{{\bf SK}}} \varphi \rightarrow (\psi \rightarrow \varphi)
	\end{align}
	と書く.
	
	\begin{itembox}[l]{{\bf SK}の推論規則}
		$\varphi$と$\psi$を式とするとき,次は{\bf SK}の推論規則である.
		\begin{description}
			\item[三段論法] $\Gamma \provable{\mbox{{\bf SK}}} \psi$かつ
				$\Gamma \provable{\mbox{{\bf SK}}} \psi \rightarrow \varphi$ならば
				$\Gamma \provable{\mbox{{\bf SK}}} \varphi$である.
		\end{description}
	\end{itembox}
	
	{\bf SK}から証明可能な式
	\begin{description}
		\item[(I)] $\varphi \rightarrow \varphi$
		\item[(B)] $(\psi \rightarrow \chi) \rightarrow ((\varphi \rightarrow \psi) \rightarrow (\varphi \rightarrow \chi)).$
		\item[(C)] $(\varphi \rightarrow (\psi \rightarrow \chi)) \rightarrow (\psi \rightarrow (\varphi \rightarrow \chi)).$
		\item[(W)] $(\varphi \rightarrow (\varphi \rightarrow \psi)) \rightarrow (\varphi \rightarrow \psi).$
		\item[(B')] $(\varphi \rightarrow \psi) \rightarrow ((\psi \rightarrow \chi) \rightarrow (\varphi \rightarrow \chi)).$
		\item[(C$\ast$)] $\varphi \rightarrow ((\varphi \rightarrow \psi) \rightarrow \psi)$
	\end{description}
	
\subsection{否定}
	{\bf SK}の公理に否定の公理を追加し,推論規則はそのまま据え置いた証明体系を{\bf SK'}とする.
	
	\begin{itembox}[l]{{\bf SK'}で追加された公理}
		$\varphi$と$\psi$と$\xi$を式とするとき,{\bf SK'}の公理は(S)(K)に以下の式を加えたものである.
		\begin{description}
			\item[(CTI1)] $\varphi \rightarrow (\rightharpoondown \varphi \rightarrow \bot).$
			
			\item[(CTI2)] $\rightharpoondown \varphi \rightarrow (\varphi \rightarrow \bot).$
			
			\item[(NI)] $(\varphi \rightarrow \bot) \rightarrow\ \rightharpoondown \varphi.$
		\end{description}
	\end{itembox}
	
	このとき証明可能な式
	\begin{description}
		\item[(DNI)] $\varphi \rightarrow\ \rightharpoondown \rightharpoondown \varphi.$
		\item[(CON1)] $(\varphi \rightarrow \psi) \rightarrow (\rightharpoondown \psi \rightarrow\ \rightharpoondown \varphi).$
		\item[(CON2)] $(\varphi \rightarrow\ \rightharpoondown \psi) \rightarrow (\psi \rightarrow\ \rightharpoondown \varphi).$
	\end{description}
	
\subsection{{\bf HM}}
	{\bf HM}とは最小論理と呼ばれる証明体系である.{\bf HK}においては$\bot$が示されると
	あらゆる式が導かれることになるが(爆発律),{\bf HM}ではそれが起こらないので矛盾許容論理と言われる.
	また背理法が成立しないので「$\negation A$と仮定すると矛盾するので$A$」という論法は使えず,
	証明は矛盾に頼らないという意味で``構成的''になる.
	
	\begin{align}
		\varphi(t/x)
	\end{align}
	とは,式$\varphi$に{\bf 自由に}現れる変項$x$を項$t$で置き換えた式を表す.
	ただし$t$は$\varphi$の中で$x$への代入について自由である.
	
	\begin{itembox}[l]{{\bf HM}の公理}
		$\varphi$と$\psi$と$\xi$を式とし,$x$と$t$を変項とし,$\tau$を項とするとき,
		次は{\bf HM}の公理である.
		\begin{description}
			\item[(S)] $(\varphi \rightarrow (\psi \rightarrow \chi)) 
				\rightarrow ((\varphi \rightarrow \psi)
				\rightarrow (\varphi \rightarrow \chi)).$
			\item[(K)] $\varphi \rightarrow (\psi \rightarrow \varphi).$
			\item[(DI1)] $\varphi \rightarrow (\varphi \vee \psi).$
			\item[(DI2)] $\psi \rightarrow (\varphi \vee \psi).$
			\item[(DE)] $(\varphi \rightarrow \chi) \rightarrow 
				((\psi \rightarrow \chi) \rightarrow ((\varphi \vee \psi) \rightarrow \chi)).$
			\item[(CI)] $\varphi \rightarrow (\psi \rightarrow (\varphi \wedge \psi)).$
			\item[(CE1)] $(\varphi \wedge \psi) \rightarrow \varphi.$
			\item[(CE2)] $(\varphi \wedge \psi) \rightarrow \psi.$
			\item[(UI)] $\forall t (\psi \rightarrow \varphi(t/x)) 
				\rightarrow (\psi \rightarrow \forall x \varphi).$
			\item[(UE)] $\forall x \varphi \rightarrow \varphi(\tau/x).$
			\item[(EI)] $\varphi(\tau/x) \rightarrow \exists x \varphi.$
			\item[(EE)] $\forall t (\varphi(t/x) \rightarrow \psi)
				\rightarrow (\exists x \varphi \rightarrow \psi).$
		\end{description}
	\end{itembox}
	
	{\bf SK}と同様に,上の{\bf HM}の公理は全て$\provable{\mbox{{\bf HM}}} ...$と書かれる.
	
	\begin{itembox}[l]{{\bf HM}の推論規則}
		$\varphi$と$\psi$を式とするとき,次は{\bf HM}の推論規則である.
		\begin{description}
			\item[三段論法] $\Gamma \provable{\mbox{{\bf HM}}} \psi$かつ
				$\Gamma \provable{\mbox{{\bf HM}}} \psi \rightarrow \varphi$ならば
				$\Gamma \provable{\mbox{{\bf HM}}} \varphi$である.
			\item[汎化] $\psi$に変項$x$が自由に現れているとき,
				$\Gamma \provable{\mbox{{\bf HM}}} \psi(y/x)$ならば
				$\Gamma \provable{\mbox{{\bf HM}}} \forall x \psi$である.
				ただし$y$は$\forall x \psi$に自由には現れない変項とする.
		\end{description}
	\end{itembox}
	
	{\bf HM}から証明可能な式
	\begin{description}
		\item[(LNC)] $\rightharpoondown (\varphi \wedge \rightharpoondown \varphi).$
		\item[(DIST$\wedge$)] $\varphi \vee (\psi \wedge \chi) 
			\leftrightarrow (\varphi \vee \psi) \wedge (\varphi \vee \chi).$
		\item[(DIST$\vee$)] $\varphi \wedge (\psi \vee \chi) 
			\leftrightarrow (\varphi \wedge \psi) \vee (\varphi \wedge \chi).$
		\item[(DM$\vee$)] $\rightharpoondown (\varphi \vee \psi) \leftrightarrow
			\ \rightharpoondown \varphi \wedge \rightharpoondown \psi.$
	\end{description}
	
	\begin{sketch}[LNC]
		\begin{align}
			\varphi \wedge \rightharpoondown \varphi &\provable{\mbox{{\bf HM}}} \varphi, \\
			\varphi \wedge \rightharpoondown \varphi &\provable{\mbox{{\bf HM}}}\ \rightharpoondown \varphi, \\
			\varphi \wedge \rightharpoondown \varphi &\provable{\mbox{{\bf HM}}}
				\varphi \rightarrow (\rightharpoondown \varphi \rightarrow \bot), \\
			\varphi \wedge \rightharpoondown \varphi &\provable{\mbox{{\bf HM}}}\ \rightharpoondown \varphi \rightarrow \bot, \\
			\varphi \wedge \rightharpoondown \varphi &\provable{\mbox{{\bf HM}}} \bot, \\
			&\provable{\mbox{{\bf HM}}} (\varphi \wedge \rightharpoondown \varphi) \rightarrow \bot, \\
			&\provable{\mbox{{\bf HM}}} ((\varphi \wedge \rightharpoondown \varphi) \rightarrow \bot)
				\rightarrow\ \rightharpoondown (\varphi \wedge \rightharpoondown \varphi), \\
			&\provable{\mbox{{\bf HM}}}\ \rightharpoondown (\varphi \wedge \rightharpoondown \varphi).
		\end{align}
		\QED
	\end{sketch}
	
	\begin{sketch}[DM$\vee$]
		\begin{align}
			&\provable{\mbox{{\bf HM}}} \varphi \rightarrow (\varphi \vee \psi), && \mbox{(DI1)}\\
			&\provable{\mbox{{\bf HM}}} (\varphi \rightarrow (\varphi \vee \psi))
				\rightarrow (\rightharpoondown (\varphi \vee \psi) \rightarrow\ \rightharpoondown \varphi), 
				&& \mbox{(CON1)}\\
			&\provable{\mbox{{\bf HM}}}\ \rightharpoondown (\varphi \vee \psi) \rightarrow\ \rightharpoondown \varphi, 
				&& \mbox{(MP)}\\
			\rightharpoondown (\varphi \vee \psi) &\provable{\mbox{{\bf HM}}}\ \rightharpoondown \varphi.
				&& \mbox{(DR)}
		\end{align}
		同様に
		\begin{align}
			\rightharpoondown (\varphi \vee \psi) \provable{\mbox{{\bf HM}}}\ \rightharpoondown \psi
		\end{align}
		となり,
		\begin{align}
			\rightharpoondown (\varphi \vee \psi) &\provable{\mbox{{\bf HM}}}\ \rightharpoondown \varphi
				\rightarrow (\rightharpoondown \psi \rightarrow 
				(\rightharpoondown \varphi \wedge \rightharpoondown \psi)), && \mbox{(CI)}\\
			\rightharpoondown (\varphi \vee \psi) &\provable{\mbox{{\bf HM}}}\ 
				\rightharpoondown \psi \rightarrow (\rightharpoondown \varphi \wedge \rightharpoondown \psi), 
				&& \mbox{(MP)}\\
			\rightharpoondown (\varphi \vee \psi) &\provable{\mbox{{\bf HM}}}\ 
				\rightharpoondown \varphi \wedge \rightharpoondown \psi && \mbox{(MP)}
		\end{align}
		が得られる.逆に
		\begin{align}
			\rightharpoondown \varphi \wedge \rightharpoondown \psi &\provable{\mbox{{\bf HM}}}\ \rightharpoondown \varphi, 
				&& \mbox{(CE1)}\\
			\rightharpoondown \varphi \wedge \rightharpoondown \psi &\provable{\mbox{{\bf HM}}}\ 
			\rightharpoondown \varphi \rightarrow (\varphi \rightarrow \bot), && \mbox{(CTI2)}\\
			\rightharpoondown \varphi \wedge \rightharpoondown \psi &\provable{\mbox{{\bf HM}}} \varphi \rightarrow \bot
				&& \mbox{(MP)}
		\end{align}
		となり,同様に
		\begin{align}
			\rightharpoondown \varphi \wedge \rightharpoondown \psi \provable{\mbox{{\bf HM}}} \psi \rightarrow \bot
		\end{align}
		も成り立つ.よって
		\begin{align}
			\rightharpoondown \varphi \wedge \rightharpoondown \psi &\provable{\mbox{{\bf HM}}} 
				(\varphi \rightarrow \bot) \rightarrow ((\psi \rightarrow \bot) 
				\rightarrow ((\varphi \vee \psi) \rightarrow \bot)), && \mbox{(DE)}\\
			\rightharpoondown \varphi \wedge \rightharpoondown \psi &\provable{\mbox{{\bf HM}}} 
				(\psi \rightarrow \bot) \rightarrow ((\varphi \vee \psi) \rightarrow \bot), && \mbox{(MP)}\\
			\rightharpoondown \varphi \wedge \rightharpoondown \psi &\provable{\mbox{{\bf HM}}} 
				(\varphi \vee \psi) \rightarrow \bot, && \mbox{(MP)}\\
			\rightharpoondown \varphi \wedge \rightharpoondown \psi &\provable{\mbox{{\bf HM}}} 
				((\varphi \vee \psi) \rightarrow \bot) \rightarrow\ \rightharpoondown (\varphi \vee \psi), && \mbox{(NI)}\\
			\rightharpoondown \varphi \wedge \rightharpoondown \psi &\provable{\mbox{{\bf HM}}} 
				\ \rightharpoondown (\varphi \vee \psi) && \mbox{(MP)}
		\end{align}
		が得られる.
		\QED
	\end{sketch}
	
\subsection{{\bf HK}}
	{\bf HM}の推論規則はそのままに,公理に{\bf 二重否定除去}を追加すると
	古典論理の証明体系{\bf HK}となる.
	
	\begin{itembox}[l]{{\bf HK}の公理}
		{\bf HM}の公理に次を追加:
		\begin{description}
			\item[(DNE)] $\rightharpoondown \rightharpoondown \varphi \rightarrow \varphi.$
		\end{description}
	\end{itembox}
	
	{\bf HK}から証明可能な式
	\begin{description}
		\item[(CON3)] $(\rightharpoondown \varphi \rightarrow \psi) \rightarrow (\rightharpoondown \psi \rightarrow \varphi).$
		\item[(CON4)] $(\rightharpoondown \varphi \rightarrow\ \rightharpoondown \psi) 
			\rightarrow (\psi \rightarrow \varphi).$
		\item[(RAA)] $(\rightharpoondown \varphi \rightarrow \bot) \rightarrow \varphi.$
		\item[(EFQ)] $\bot \rightarrow \varphi.$
	\end{description}
	
	\begin{sketch}[CON3]
		\begin{align}
			\rightharpoondown \varphi \rightarrow \psi,\ \rightharpoondown \psi &\provable{\mbox{{\bf HK}}}
				\ \rightharpoondown \psi \rightarrow\ \rightharpoondown \rightharpoondown \varphi,
				&& \mbox{(CON1)} \\
			\rightharpoondown \varphi \rightarrow \psi,\ \rightharpoondown \psi &\provable{\mbox{{\bf HK}}}
				\ \rightharpoondown \psi, \\
			\rightharpoondown \varphi \rightarrow \psi,\ \rightharpoondown \psi &\provable{\mbox{{\bf HK}}}
				\ \rightharpoondown \rightharpoondown \varphi, && \mbox{(MP)} \\
			\rightharpoondown \varphi \rightarrow \psi,\ \rightharpoondown \psi &\provable{\mbox{{\bf HK}}}
				\ \rightharpoondown \rightharpoondown \varphi \rightarrow \varphi, && \mbox{(DNE)} \\
			\rightharpoondown \varphi \rightarrow \psi,\ \rightharpoondown \psi &\provable{\mbox{{\bf HK}}}
				\varphi, && \mbox{(MP)} \\
			\rightharpoondown \varphi \rightarrow \psi &\provable{\mbox{{\bf HK}}}
				\ \rightharpoondown \psi \rightarrow \varphi. && \mbox{(DR)}
		\end{align}
		\QED
	\end{sketch}
	
	\begin{sketch}[CON4]
		\begin{align}
			\rightharpoondown \varphi \rightarrow\ \rightharpoondown \psi,\ \psi &\provable{\mbox{{\bf HK}}} \psi, \\
			\rightharpoondown \varphi \rightarrow\ \rightharpoondown \psi,\ \psi &\provable{\mbox{{\bf HK}}} 
				\psi \rightarrow\ \rightharpoondown \rightharpoondown \psi, && \mbox{(DNI)} \\
			\rightharpoondown \varphi \rightarrow\ \rightharpoondown \psi,\ \psi &\provable{\mbox{{\bf HK}}} 
				\ \rightharpoondown \rightharpoondown \psi. && \mbox{(MP)}
		\end{align}
		及び,(CON3)より
		\begin{align}
			\rightharpoondown \varphi \rightarrow\ \rightharpoondown \psi,\ \psi \provable{\mbox{{\bf HK}}} 
				\ \rightharpoondown \rightharpoondown \psi \rightarrow \varphi
		\end{align}
		となるので,(MP)より
		\begin{align}
			\rightharpoondown \varphi \rightarrow\ \rightharpoondown \psi,\ \psi \provable{\mbox{{\bf HK}}} \varphi
		\end{align}
		が成り立つ.よって演繹法則より
		\begin{align}
			\rightharpoondown \varphi \rightarrow\ \rightharpoondown \psi \provable{\mbox{{\bf HK}}} \psi \rightarrow \varphi
		\end{align}
		が得られる.
		\QED
	\end{sketch}
	
	\begin{sketch}[RAA]
		\begin{align}
			&\provable{\mbox{{\bf HK}}} (\rightharpoondown \varphi \rightarrow \bot) \rightarrow\ 
				\rightharpoondown \rightharpoondown \varphi, && \mbox{(NI)} \\
			\rightharpoondown \varphi \rightarrow \bot &\provable{\mbox{{\bf HK}}}\ 
				\rightharpoondown \rightharpoondown \varphi, && \mbox{(DR)} \\
			\rightharpoondown \varphi \rightarrow \bot &\provable{\mbox{{\bf HK}}}\ 
				\rightharpoondown \rightharpoondown \varphi \rightarrow \varphi, && \mbox{(DNE)} \\
			\rightharpoondown \varphi \rightarrow \bot &\provable{\mbox{{\bf HK}}} \varphi, && \mbox{(MP)} \\
			&\provable{\mbox{{\bf HK}}} (\rightharpoondown \varphi \rightarrow \bot) \rightarrow \varphi. && \mbox{(DR)}
		\end{align}
		\QED
	\end{sketch}
	
	\begin{sketch}[EFQ]
		\begin{align}
			&\provable{\mbox{{\bf HK}}} \bot \rightarrow (\rightharpoondown \varphi \rightarrow \bot), && \mbox{(K)} \\
			\bot &\provable{\mbox{{\bf HK}}}\ \rightharpoondown \varphi \rightarrow \bot, && \mbox{(DR)} \\
			\bot &\provable{\mbox{{\bf HK}}} (\rightharpoondown \varphi \rightarrow \bot) \rightarrow \varphi,
				&& \mbox{(RAA)} \\
			\bot &\provable{\mbox{{\bf HK}}} \varphi, && \mbox{(MP)} \\
			&\provable{\mbox{{\bf HK}}} \bot \rightarrow \varphi. && \mbox{(DR)}
		\end{align}
		\QED
	\end{sketch}
	
\section{{\bf HK'}}
	{\bf HK}の量化公理から(UI)と(EE)を取り除き,三段論法に加え
	{\bf 存在汎化}と{\bf 全称汎化}といった推論規則を用いた証明体系を{\bf HK'}とする.
	つまり,
	
	\begin{itembox}[l]{{\bf HK'}の公理}
		$\varphi$と$\psi$と$\xi$を式とし,$x$を変項とし,$\tau$を項とするとき,
		次は{\bf HK'}の公理である.
		\begin{description}
			\item[(S)] $(\varphi \rightarrow (\psi \rightarrow \chi)) 
				\rightarrow ((\varphi \rightarrow \psi)
				\rightarrow (\varphi \rightarrow \chi)).$
			\item[(K)] $\varphi \rightarrow (\psi \rightarrow \varphi).$
			\item[(DI1)] $\varphi \rightarrow (\varphi \vee \psi).$
			\item[(DI2)] $\psi \rightarrow (\varphi \vee \psi).$
			\item[(DE)] $(\varphi \rightarrow \chi) \rightarrow 
				((\psi \rightarrow \chi) \rightarrow ((\varphi \vee \psi) \rightarrow \chi)).$
			\item[(CI)] $\varphi \rightarrow (\psi \rightarrow (\varphi \wedge \psi)).$
			\item[(CE1)] $(\varphi \wedge \psi) \rightarrow \varphi.$
			\item[(CE2)] $(\varphi \wedge \psi) \rightarrow \psi.$
			
			\item[(UE)] $\forall x \varphi \rightarrow \varphi(\tau/x).$
			\item[(EI)] $\varphi(\tau/x) \rightarrow \exists x \varphi.$
			
			\item[(CTI1)] $\varphi \rightarrow (\rightharpoondown \varphi \rightarrow \bot).$
			
			\item[(CTI2)] $\rightharpoondown \varphi \rightarrow (\varphi \rightarrow \bot).$
			
			\item[(NI)] $(\varphi \rightarrow \bot) \rightarrow\ \rightharpoondown \varphi.$
			\item[(DNE)] $\rightharpoondown \rightharpoondown \varphi \rightarrow \varphi.$
		\end{description}
	\end{itembox}
	
	とし,推論規則には,三段論法に加えて
	
	\begin{itembox}[l]{{\bf HK'}の汎化規則}
		$\varphi$と$\psi$を式とし,$x$と$t$を変項とし,$\varphi$に$x$が自由に現れるとし,
		また{\bf $\psi$と$\exists x \varphi$に$t$は自由に現れないとする}.
		\begin{description}
			\item[存在汎化] 
				$\Gamma \provable{\mbox{{\bf HK'}}} \varphi(t/x) \rightarrow \psi$ならば
				$\Gamma \provable{\mbox{{\bf HK'}}} \exists x \varphi \rightarrow \psi$となる.
			
			\item[全称汎化] 
				$\Gamma \provable{\mbox{{\bf HK'}}} \psi \rightarrow \varphi(t/x)$ならば
				$\Gamma \provable{\mbox{{\bf HK'}}} \psi \rightarrow \forall x \varphi$となる.
		\end{description}
	\end{itembox}
	を用いる.規則の前提で太字で強調した文言は{\bf 固有変項条件}と呼ばれる.
	
	\begin{screen}
		\begin{metathm}[{\bf HK}と{\bf HK'}は同値]
			任意の式$\varphi$に対して,$\Gamma \provable{\mbox{{\bf HK}}} \varphi$ならば
			$\Gamma \provable{\mbox{{\bf HK'}}} \varphi$であり,その逆もまた然り.
		\end{metathm}
	\end{screen}
	
	\begin{metaprf}\mbox{}
		\begin{description}
			\item[{\bf HK'}から示されたら{\bf HK}からも証明可能]
			いま
			\begin{align}
				\Gamma \provable{\mbox{{\bf HK'}}} \varphi
			\end{align}
			であるとする.$\varphi$が{\bf HK'}の公理であれば{\bf HK}の公理でもあるし,
			$\Gamma$の公理であれば言わずもがな,これらの場合は
			\begin{align}
				\Gamma \provable{\mbox{{\bf HK}}} \varphi
			\end{align}
			となる.$\varphi$が存在汎化によって得られているとき,つまり,$\varphi$とは
			\begin{align}
				\exists x \psi \rightarrow \chi
			\end{align}
			なる形の式であって,$\psi(t/x) \rightarrow \chi$から存在汎化で得られている場合,
			ここで$t$は$\chi$と$\exists x \psi$に自由に現れない変項であるが,このとき,
			\begin{align}
				\Gamma \provable{\mbox{{\bf HK}}} \psi(t/x) \rightarrow \chi
			\end{align}
			であると仮定すれば,汎化と量化公理(EE)によって
			\begin{align}
				\Gamma &\provable{\mbox{{\bf HK}}} \forall t(\psi(t/x) \rightarrow \chi), \\
				\Gamma &\provable{\mbox{{\bf HK}}} \forall t(\psi(t/x) \rightarrow \chi)
					\rightarrow (\exists x \psi \rightarrow \chi), \\
				\Gamma &\provable{\mbox{{\bf HK}}} \exists x \psi \rightarrow \chi
			\end{align}
			となる.また$\varphi$が全称汎化によって得られているとき,つまり,$\varphi$とは
			\begin{align}
				\chi \rightarrow \forall x \psi
			\end{align}
			なる形の式であって,$\chi \rightarrow \psi(t/x)$から存在汎化で得られている場合,
			ここで$t$は$\chi$と$\forall x \psi$に自由に現れない変項であるが,このとき,
			\begin{align}
				\Gamma \provable{\mbox{{\bf HK}}} \chi \rightarrow \psi(t/x)
			\end{align}
			であると仮定すれば,汎化と量化公理(UI)によって
			\begin{align}
				\Gamma &\provable{\mbox{{\bf HK}}} \forall t(\chi \rightarrow \psi(t/x)), \\
				\Gamma &\provable{\mbox{{\bf HK}}} \forall t(\chi \rightarrow \psi(t/x))
					\rightarrow (\chi \rightarrow \forall x \psi), \\
				\Gamma &\provable{\mbox{{\bf HK}}} \chi \rightarrow \forall x \psi
			\end{align}
			となる.$\varphi$が三段論法で得られている場合,つまり$\psi$と$\psi \rightarrow \varphi$
			なる形の式が{\bf HK'}から示されている場合であるが,
			\begin{align}
				\Gamma &\provable{\mbox{{\bf HK}}} \psi, \\
				\Gamma &\provable{\mbox{{\bf HK}}} \psi \rightarrow \varphi
			\end{align}
			と仮定すれば$\Gamma \provable{\mbox{{\bf HK}}} \varphi$も従う.
	
		\item[{\bf HK}から示されたら{\bf HK'}からも証明可能]
			いま
			\begin{align}
				\Gamma \provable{\mbox{{\bf HK}}} \varphi
			\end{align}
			とする.$\varphi$が量化公理(UI)(EE)以外の{\bf HK}の公理か,$\Gamma$の公理であれば
			\begin{align}
				\Gamma \provable{\mbox{{\bf HK'}}} \varphi
			\end{align}
			である.$\varphi$が
			\begin{align}
				\forall t (\chi \rightarrow \psi(t/x)) 
				\rightarrow (\chi \rightarrow \forall x \psi)
			\end{align}
			なる形の公理であるとき($t$は$\chi$と$\forall x \psi$には自由に現れない),
			\begin{align}
				\Gamma &\provable{\mbox{{\bf HK'}}} 
					\forall t (\chi \rightarrow \psi(t/x)) 
					\rightarrow (\chi \rightarrow \psi(t/x)), \\
				\color{red}\chi \rightarrow \psi(t/x),\ \Gamma &
				\color{red}\provable{\mbox{{\bf HK'}}}
					\chi \rightarrow \psi(t/x), \\
				\color{red}\chi \rightarrow \psi(t/x),\ \Gamma &
				\color{red}\provable{\mbox{{\bf HK'}}}
					\chi \rightarrow \forall x \psi, \\
				\Gamma &\provable{\mbox{{\bf HK'}}} (\chi \rightarrow \psi(t/x)) 
					\rightarrow (\chi \rightarrow \forall x \psi), \\
				\forall t (\chi \rightarrow \psi(t/x)),\ \Gamma
					&\provable{\mbox{{\bf HK'}}} \forall t (\chi \rightarrow \psi(t/x)), \\
				\forall t (\chi \rightarrow \psi(t/x)),\ \Gamma
					&\provable{\mbox{{\bf HK'}}} \forall t (\chi \rightarrow \psi(t/x)) 
					\rightarrow (\chi \rightarrow \psi(t/x)), \\
				\forall t (\chi \rightarrow \psi(t/x)),\ \Gamma
					&\provable{\mbox{{\bf HK'}}} \chi \rightarrow \psi(t/x), \\
				\forall t (\chi \rightarrow \psi(t/x)),\ \Gamma
					&\provable{\mbox{{\bf HK'}}} (\chi \rightarrow \psi(t/x)) 
					\rightarrow (\chi \rightarrow \forall x \psi), \\
				\forall t (\chi \rightarrow \psi(t/x)),\ \Gamma
					&\provable{\mbox{{\bf HK'}}} \chi \rightarrow \forall x \psi, \\
				\Gamma &\provable{\mbox{{\bf HK'}}} \forall t (\chi \rightarrow \psi(t/x)) 
					\rightarrow (\chi \rightarrow \forall x \psi)
			\end{align}
			となる(赤字で{\bf HK'}の全称汎化規則を用いた箇所を示している).
			つまり(UI)は{\bf HK'}の定理である.同様に(EE)も{\bf HK'}の定理である.実際,
			\begin{align}
				\Gamma &\provable{\mbox{{\bf HK'}}} 
					\forall t (\psi(t/x) \rightarrow \chi) 
					\rightarrow (\psi(t/x) \rightarrow \chi), \\
				\color{red}\psi(t/x) \rightarrow \chi,\ \Gamma &
				\color{red}\provable{\mbox{{\bf HK'}}}
					\psi(t/x) \rightarrow \chi, \\
				\color{red}\psi(t/x) \rightarrow \chi,\ \Gamma &
				\color{red}\provable{\mbox{{\bf HK'}}}
					\exists x \psi \rightarrow \chi, \\
				\Gamma &\provable{\mbox{{\bf HK'}}} (\psi(t/x) \rightarrow \chi) 
					\rightarrow (\exists x \psi \rightarrow \chi), \\
				\forall t (\psi(t/x) \rightarrow \chi),\ \Gamma
					&\provable{\mbox{{\bf HK'}}} \forall t (\psi(t/x) \rightarrow \chi), \\
				\forall t (\psi(t/x) \rightarrow \chi),\ \Gamma
					&\provable{\mbox{{\bf HK'}}} \forall t (\psi(t/x) \rightarrow \chi)
					\rightarrow (\psi(t/x) \rightarrow \chi), \\
				\forall t (\psi(t/x) \rightarrow \chi),\ \Gamma
					&\provable{\mbox{{\bf HK'}}} \psi(t/x) \rightarrow \chi, \\
				\forall t (\psi(t/x) \rightarrow \chi),\ \Gamma
					&\provable{\mbox{{\bf HK'}}} (\psi(t/x) \rightarrow \chi) 
					\rightarrow (\exists x \psi \rightarrow \chi), \\
				\forall t (\psi(t/x) \rightarrow \chi),\ \Gamma
					&\provable{\mbox{{\bf HK'}}} \exists x \psi \rightarrow \chi, \\
				\Gamma &\provable{\mbox{{\bf HK'}}} \forall t (\psi(t/x) \rightarrow \chi) 
					\rightarrow (\exists x \psi \rightarrow \chi)
			\end{align}
			となる.$\varphi$が汎化によって導かれているとき,つまり$\varphi$は
			\begin{align}
				\forall x \psi
			\end{align}
			なる式であって,先に$\psi(t/x)$なる式が{\bf HK}から証明されているとき
			($x$は$\psi$に自由に現れ,$t$は$\forall x \psi$に自由に現れない変項である),
			\begin{align}
				\Gamma \provable{\mbox{{\bf HK'}}} \psi(t/x)
			\end{align}
			と仮定したら
			\begin{align}
				\Gamma \provable{\mbox{{\bf HK'}}} \forall x \psi
			\end{align}
			が成り立つ.実際,$\varphi$を$\chi \rightarrow \chi$といった文とすれば
			\begin{align}
				\Gamma \provable{\mbox{{\bf HK'}}} \varphi \rightarrow \psi(t/x)
			\end{align}
			が成り立つので,全称汎化より
			\begin{align}
				\Gamma \provable{\mbox{{\bf HK'}}} \varphi \rightarrow \forall x \psi
			\end{align}
			となり,
			\begin{align}
				\Gamma \provable{\mbox{{\bf HK'}}} \varphi
			\end{align}
			と併せて
			\begin{align}
				\Gamma \provable{\mbox{{\bf HK'}}} \forall x \psi
			\end{align}
			が従う.
			\QED
		\end{description}
	\end{metaprf}
	
\begin{comment}	
\subsection{汎化規則は{\bf HK'}から導かれる}
	{\bf HK}の量化公理に
	\begin{align}
		\forall x (\psi \rightarrow \varphi) \rightarrow
		(\forall x \psi \rightarrow \forall x \varphi),
		\quad \mbox{ただし$\varphi$と$\psi$には$x$が自由に現れる}
	\end{align}
	を追加すれば,一般化規則は導かれる.
	
	\begin{screen}
		$x$と$t$を変項とし,$x$は$\psi$に自由に現れるとし,
		$t$は$\forall x \psi$に自由に現れないとする..
		このとき
		\begin{align}
			\Gamma \provable{\mbox{{\bf HK'}}} \psi(t/x)
		\end{align}
		ならば
		\begin{align}
			\Gamma \provable{\mbox{{\bf HK'}}} \forall x \psi
		\end{align}
		である.
	\end{screen}
	
	\begin{sketch}
		$\provable{\mbox{{\bf HK'}}} \psi(t/x)$であるときは
		\begin{align}
			\provable{\mbox{{\bf HK}}} \forall x \psi
		\end{align}
		が成立する.$\varphi$が三段論法によって示されるとき,つまり式$\psi$で
		\begin{align}
			\Gamma &\provable{\mbox{{\bf HK}}} \psi, \\
			\Sigma &\provable{\mbox{{\bf HK}}} \psi \rightarrow \varphi
		\end{align}
		を満たすものが取れるとき,$\psi$に$x$が自由に現れているかいないかで
		\begin{description}
			\item[case1] $\psi$に$x$が自由に現れていないとき,
				\begin{align}
					\Gamma \provable{\mbox{{\bf HK}}} \psi
				\end{align}
				かつ
				\begin{align}
					\Gamma \provable{\mbox{{\bf HK}}} \forall x (\psi \rightarrow \varphi)
				\end{align}
				
			\item[case2] $\psi$に$x$が自由に現れているとき,
				\begin{align}
					\Gamma \provable{\mbox{{\bf HK}}} \forall x \psi
				\end{align}
				かつ
				\begin{align}
					\Gamma \provable{\mbox{{\bf HK}}} \forall x (\psi \rightarrow \varphi)
				\end{align}
		\end{description}
		のいずれかのケースを一つ仮定する.
		\begin{description}
			\item[case1] 量化公理(UI)より
				\begin{align}
					\Gamma &\provable{\mbox{{\bf HK}}} \forall x (\psi \rightarrow \varphi), \\
					\Gamma &\provable{\mbox{{\bf HK}}} \forall x (\psi \rightarrow \varphi) \rightarrow (\psi \rightarrow \forall x \varphi), \\
					\Gamma &\provable{\mbox{{\bf HK}}} \psi \rightarrow \forall x \varphi
				\end{align}
				が成り立つので,$\Gamma \provable{\mbox{{\bf HK}}} \psi$の仮定と併せて
				\begin{align}
					\Gamma \provable{\mbox{{\bf HK}}} \forall x \varphi
				\end{align}
				が得られる.
				
			\item[case2]
				新しく追加した量化公理を用いれば,
				\begin{align}
					\Gamma &\provable{\mbox{{\bf HK}}} \forall x (\psi \rightarrow \varphi), \\
					\Gamma &\provable{\mbox{{\bf HK}}} \forall x (\psi \rightarrow \varphi) \rightarrow (\forall x \psi \rightarrow \forall x \varphi), \\
					\Gamma &\provable{\mbox{{\bf HK}}} \forall x \psi \rightarrow \forall x \varphi
				\end{align}
				が得られるので,$\Gamma \provable{\mbox{{\bf HK}}} \forall x \psi$の仮定と併せて
				\begin{align}
					\Gamma \provable{\mbox{{\bf HK}}} \forall x \varphi
				\end{align}
				が従う.
				\QED
		\end{description}
	\end{sketch}
	
	逆に,新しく追加した量化公理は一般化規則から導かれる.実際,量化規則(UE)より
	\begin{align}
		\forall x (\psi \rightarrow \varphi),\ \forall x \psi &\provable{\mbox{{\bf HK}}} \forall x (\psi \rightarrow \varphi), \\
		\forall x (\psi \rightarrow \varphi),\ \forall x \psi &\provable{\mbox{{\bf HK}}} \forall x (\psi \rightarrow \varphi)
		\rightarrow (\psi(t/x) \rightarrow \varphi(t/x)), \\
		\forall x (\psi \rightarrow \varphi),\ \forall x \psi &\provable{\mbox{{\bf HK}}} \psi(t/x) \rightarrow \varphi(t/x)
	\end{align}
	となり(ただし$t$は$\psi$と$\varphi$に現れない変項とする),また一方で
	\begin{align}
		\forall x (\psi \rightarrow \varphi),\ \forall x \psi &\provable{\mbox{{\bf HK}}} \forall x \psi, \\
		\forall x (\psi \rightarrow \varphi),\ \forall x \psi &\provable{\mbox{{\bf HK}}} \forall x \psi \rightarrow \psi(t/x), \\
		\forall x (\psi \rightarrow \varphi),\ \forall x \psi &\provable{\mbox{{\bf HK}}} \psi(t/x)
	\end{align}
	も成り立つから,前と併せて三段論法より
	\begin{align}
		\forall x (\psi \rightarrow \varphi),\ \forall x \psi \provable{\mbox{{\bf HK}}} \varphi(t/x)
	\end{align}
	となる.演繹定理より
	\begin{align}
		\forall x (\psi \rightarrow \varphi) \provable{\mbox{{\bf HK}}} \forall x \psi  \rightarrow \varphi(t/x)
	\end{align}
	となり,一般化規則より
	\begin{align}
		\forall x (\psi \rightarrow \varphi) \provable{\mbox{{\bf HK}}} \forall t ( \forall x \psi  \rightarrow \varphi(t/x))
	\end{align}
	となる.量化規則(UI)より
	\begin{align}
		\forall x (\psi \rightarrow \varphi) \provable{\mbox{{\bf HK}}} \forall t ( \forall x \psi  \rightarrow \varphi(t/x)) \rightarrow
		(\forall x \psi \rightarrow \forall x \varphi)
	\end{align}
	が成り立つので,
	\begin{align}
		\forall x (\psi \rightarrow \varphi) \provable{\mbox{{\bf HK}}}
		\forall x \psi \rightarrow \forall x \varphi
	\end{align}
	となる.演繹定理より
	\begin{align}
		\provable{\mbox{{\bf HK}}} \forall x (\psi \rightarrow \varphi)
		\rightarrow (\forall x \psi \rightarrow \forall x \varphi)
	\end{align}
	が得られる.
	
\end{comment}

\section{直観主義と古典論理}
	任意の式$\varphi$に対してその否定翻訳を$\varphi^{N}$と書く.
	$\provable{\mbox{{\bf HM}}} \varphi^{N}$ならば
	$\provable{\mbox{{\bf HK}}} \varphi^{N}$は当たり前.
	逆に$\provable{\mbox{{\bf HK}}} \varphi^{N}$ならば
	$\provable{\mbox{{\bf HK}}} \varphi$を経由して
	$\provable{\mbox{{\bf HM}}} \varphi^{N}$となる.
	{\bf 式を否定翻訳に制限すれば直観主義と古典論理は変わらない.}

\begin{thebibliography}{数字}
	\bibitem{key1} Moser, G. and Zach, R., ``The Epsilon Calculus and Herbrand Complexity'',
		Studia Logica 82, 133-155 (2006)
	
	\bibitem{key2} 高橋優太, ``1階述語論理に対する$\varepsilon$計算'', \\
		http://www2.kobe-u.ac.jp/~mkikuchi/ss2018files/takahashi1.pdf 
		
	\bibitem{key3} キューネン数学基礎論講義
	
	\bibitem{key5} ブルバキ, 数学原論 集合論 1, 
	
	\bibitem{key4} 竹内外史, 現代集合論入門, 増強版第5刷, 日本評論社, 2016, pp. 138-183, ISBN 978-4-535-60116-1
	
	\bibitem{key6} 島内剛一, 数学の基礎, 第1版第10刷, 日本評論社, 2016, ISBN 978-4-535-60106-2
	
	\bibitem{key7} 戸次大介, 数理論理学, 第2刷, 東京大学出版会, 2016, pp. 148-166, ISBN 978-4-13-062915-7
	
	\bibitem{key8} K. G$\ddot{\mbox{o}}$del, $The\ Consistency\ of\ the\ Continuum\ Hypothesis$, 8th printing, Princeton University Press 1970, p. 3, ISBN 0-691-07927-7.
	
	\bibitem{key9} 菊地誠, 不完全性定理, 初版3刷, 共立出版株式会社, 2017, pp. 86-91, ISBN 978-4-320-11096-0
	
	\bibitem{key10} 前原昭二, 記号論理入門, 新装版第8刷, 日本評論社, 2018, pp. 106-115, ISBN 4-535-60144-5
	
	\bibitem{key11} Kenji Miyamoto and Georg Moser, The Epsilon Calculus with Equality and Herbrand Complexity
\end{thebibliography}

\end{document}
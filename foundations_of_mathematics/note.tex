\documentclass[a4j,10.5pt,oneside,openany,report]{jsbook}
%
\usepackage{amsmath,amssymb}
\usepackage{amsthm}
\usepackage{makeidx}
\makeindex
\usepackage{newpxmath,newpxtext}
\usepackage{mathrsfs} %花文字
\usepackage{mathtools} %参照式のみ式番号表示
\usepackage{latexsym} %qed
\usepackage{ascmac}
\usepackage{here} %表を記述位置に強制出力する
\usepackage{tabularx}
\usepackage{bussproofs} %証明図
\usepackage{centernot} %\centernot\arrow
\usepackage[dvipdfmx]{graphicx}
\usepackage{tikz} %描画
\usepackage{color}
\usepackage{relsize}
\usepackage{comment}
\usepackage{url}
\usepackage[normalem]{ulem} %訂正線
\usepackage[dvipdfm,colorlinks=true,linkcolor=blue,filecolor=blue,urlcolor=blue]{hyperref} %文書内リンク
\usepackage{pxjahyper} %%hyperref読み込みの直後に
\setcounter{tocdepth}{3} %table of contents subsection表示
\newtheoremstyle{mystyle}% % Name
	{20pt}%                      % Space above
	{20pt}%                      % Space below
	{\rm}%           % Body font
	{}%                      % Indent amount
	{\gt}%             % Theorem head font
	{.}%                      % Punctuation after theorem head
	{10pt}%                     % Space after theorem head, ' ', or \newline
	{}%                      % Theorem head spec (can be left empty, meaning `normal')
\theoremstyle{mystyle}

\allowdisplaybreaks[1]
\newcommand{\bhline}[1]{\noalign {\hrule height #1}} %表の罫線を太くする.
\newcommand{\bvline}[1]{\vrule width #1} %表の罫線を太くする.
\newcommand{\QED}{% %証明終了
	\relax\ifmmode
		\eqno{%
		\setlength{\fboxsep}{2pt}\setlength{\fboxrule}{0.3pt}
		\fcolorbox{black}{black}{\rule[2pt]{0pt}{1ex}}}
	\else
		\begingroup
		\setlength{\fboxsep}{2pt}\setlength{\fboxrule}{0.3pt}
		\hfill\fcolorbox{black}{black}{\rule[2pt]{0pt}{1ex}}
		\endgroup
	\fi}

\definecolor{DarkMidnightBlue}{rgb}{0.0, 0.2, 0.4}
\definecolor{PakistanGreen}{rgb}{0.0, 0.4, 0.0}
\definecolor{Mahogany}{rgb}{0.65,0.10,0.10}
\definecolor{darkgray}{rgb}{0.21, 0.21, 0.21}
\definecolor{CarolinaBlue}{rgb}{0.6, 0.73, 0.89}

\newtheorem{thm}{\color{DarkMidnightBlue}{定理}}[section]
\newtheorem{dfn}[thm]{\color{PakistanGreen}{定義}}
\newtheorem{axm}[thm]{\color{Mahogany}{公理}}
\newtheorem{schema}[thm]{{公理図式}}
\newtheorem{logicalrule}[thm]{\color{Mahogany}{推論規則}}
\newtheorem{logicalaxm}[thm]{\color{Mahogany}{論理的公理}}
\newtheorem{logicalthm}[thm]{\color{DarkMidnightBlue}{論理的定理}}
\newtheorem{metadfn}[thm]{\color{PakistanGreen}{メタ定義}}
\newtheorem{metaaxm}[thm]{\color{Mahogany}{メタ公理}}
\newtheorem{metathm}[thm]{\color{DarkMidnightBlue}{メタ定理}}
\newtheorem{prp}[thm]{命題}
\newtheorem{cor}[thm]{系}
\newtheorem{lem}[thm]{補題}
\newtheorem*{prf}{証明}
\newtheorem*{metaprf}{メタ証明}
\newtheorem*{sketch}{略証}
\newtheorem{rem}[thm]{\color{Mahogany}{注意}}
\newtheorem{e.g.}[thm]{例}
\newcommand{\defunc}{\mbox{1}\hspace{-0.25em}\mbox{l}} %定義関数
\newcommand*{\sgn}[1]{\operatorname{sgn}\left( #1 \right)} %signal関数
\newcommand{\monologue}[1]{
	{\color{CarolinaBlue}\hspace{-10.5pt}\mask{\hspace{21pt}\vbox{
		\hsize 445pt
		\normalcolor{\vskip 7pt \noindent #1 \vskip 7pt}
	}\hspace{21pt}}{E}}
}

\def\Ddot#1{$\ddot{\mathrm{#1}}$} %文中ddot

%論理
\newcommand{\lang}[1]{\mathcal{L}_{\scalebox{1.2}{$#1$}}} %言語
\newcommand{\Set}[2]{\{\, #1 \mid #2\, \}} %論理式の対象化
\newcommand{\defeq}{\overset{\mathrm{def}}{=\joinrel=}} %\scalebox{3}[1]{=}}} %定義記号=(=\joinrel=も使える)
\newcommand{\defarrow}{\ \overset{\mathrm{def}}{\longleftrightarrow}\ } %定義記号↔
\newcommand{\provable}[1]{\vdash_{{\scriptsize #1}}} %証明可能
\newcommand{\negation}{\rightharpoondown\hspace{-0.25em}} %否定
\newcommand{\rarrow}{\hspace{0.25em}\rightarrow\hspace{0.25em}} %右矢印
\newcommand{\lrarrow}{\hspace{0.25em}\leftrightarrow\hspace{0.25em}} %左右矢印

%集合
\newcommand{\EXTAX}{\mbox{{\bf EXT}}} %外延性公理
\newcommand{\EQAX}{\mbox{{\bf EQ}}} %相等性公理
%\newcommand{\EQAXEP}{\mbox{{\bf EQ}}_{\scalebox{1.2}{$\varepsilon$}}} %ε項の相等性公理
\newcommand{\COMAX}{\mbox{\bf COM}} %内包性公理
\newcommand{\ELEAX}{\mbox{{\bf ELE}}} %要素の公理
\newcommand{\REPAX}{\mbox{{\bf REP}}} %置換公理
\newcommand{\PAIAX}{\mbox{{\bf PAI}}} %対集合公理
\newcommand{\UNIAX}{\mbox{{\bf UNI}}} %合併の公理
\newcommand{\POWAX}{\mbox{{\bf POW}}} %冪集合公理
\newcommand{\INFAX}{\mbox{{\bf INF}}} %無限公理
\newcommand{\REGAX}{\mbox{{\bf REG}}} %正則性公理
\newcommand{\AC}{\mbox{{\bf CHOICE}}} %選択公理

\newcommand{\Univ}{\mathbf{V}} %宇宙
\newcommand{\set}[1]{\operatorname*{set}\hspace{0.15em}(#1)} %集合であることの論理式
\newcommand{\power}[1]{\operatorname*{P}\hspace{0.15em}(#1)} %冪集合
\newcommand{\rel}[1]{\operatorname*{rel}\hspace{0.15em}(#1)} %関係
\newcommand{\dom}[1]{\operatorname*{dom}\hspace{0.15em}(#1)} %類の定義域
\newcommand{\ran}[1]{\operatorname*{ran}\hspace{0.15em}(#1)} %類の値域
\newcommand{\sing}[1]{\operatorname*{sing}\hspace{0.15em}(#1)} %single-valuedの定義式
\newcommand{\fnc}[1]{\operatorname*{fnc}\hspace{0.15em}(#1)} %写像の定義式
\newcommand{\fon}{\operatorname*{:on}} %〇上の写像
\newcommand{\inj}{\overset{\mathrm{1:1}}{\longrightarrow}} %単射
\newcommand{\srj}{\overset{\mathrm{onto}}{\longrightarrow}} %全射
\newcommand{\bij}{\underset{\mathrm{onto}}{\overset{\mathrm{1:1}}{\longrightarrow}}} %全単射
\newcommand{\inv}[1]{{#1}^{-1}} %^{\operatorname{inv}}} %集合の反転
\newcommand{\rest}[2]{#1\hspace{-0.25em}\upharpoonright\hspace{-0.25em}{#2}} %制限写像
\newcommand{\tran}[1]{\operatorname*{tran}\hspace{0.15em}(#1)} %推移的類の定義式
\newcommand{\ord}[1]{\operatorname*{ord}\hspace{0.15em}(#1)} %順序数の定義式
\newcommand{\ON}{\mathrm{ON}} %順序数全体
\newcommand{\limo}[1]{\mathrm{lim.o}\hspace{0.15em}(#1)} %極限数の式
%\newcommand{\Natural}{{\boldsymbol \omega}} %自然数全体
\newcommand{\Natural}{\mathbf{N}} %自然数全体

%基数
\newcommand{\eqp}{\approx} %集合の対等
\newcommand{\card}[1]{\# #1} %濃度
%\newcommand{\card}[1]{\operatorname{card} #1} %濃度
%\newcommand{\card}[1]{\operatorname*{card} \left(#1\right)} %濃度
\newcommand{\CN}{\mathrm{CN}} %基数全体
\newcommand{\InfCN}{\mathrm{ICN}} %無限基数全体
\newcommand{\Fin}[1]{\operatorname*{Fin}\hspace{0.15em}(#1)} %有限集合の定義式
\newcommand{\Inf}[1]{\operatorname*{Inf}\hspace{0.15em}(#1)} %無限集合の定義式
\newcommand{\cof}[2]{\operatorname*{cof}\hspace{0.15em}(#1,#2)} %共終写像が存在する
\newcommand{\cf}[1]{cf(#1)} %共終数
%
%
\setlength{\textwidth}{\fullwidth}
\setlength{\textheight}{40\baselineskip}
\addtolength{\textheight}{\topskip}
%\setlength{\voffset}{-0.55in}
%
%
\title{2019年度大阪大学大学院基礎工学研究科修士論文 \\[1cm]
$\varepsilon$計算とクラスの導入による具体的で直観的な集合論の構築 \\[7cm]}%項を用いたHenkin拡大とクラスの導入による具体的で直観的な集合論の構築}
\author{システム創成専攻社会システム数理領域 \\ 関根・深澤研 百合川尚学 \\ 
学籍番号:29C17095 \\ 指導教員 深澤正彰 教授}
\date{\today}

\begin{document}
\mathtoolsset{showonlyrefs = true}
\maketitle
\tableofcontents
%\frontmatter
%\mainmatter

\chapter{序論}
	\section{導入}
\subsection{$\varepsilon$計算について}
	\begin{itemize}
		\item 量化$\exists,\forall$を使う証明を命題論理の証明に埋め込むためにHilbertが開始.
		
		\vspace{5pt}
		
		\item 式$\varphi(x)$に対して
			\begin{align}
				\varepsilon x \varphi(x)
			\end{align}
			という形のオブジェクトを作り,$\varepsilon$項と呼ぶ.また
			\begin{align}
				\exists x \varphi(x) &\lrarrow \varphi(x/\varepsilon x \varphi(x)), \\
				\forall x \varphi(x) &\lrarrow \varphi(x/\varepsilon \negation x \varphi(x))
			\end{align}
			を公理とする.
			
		\item 命題論理の証明に埋め込む際には$\exists$や$\forall$の付いた式を$\varepsilon$項を
			代入した式に変換すればよい.
			
		\item ただし,今回$\varepsilon$項を導入したのは埋め込むためではなく
			\textcolor{red}{集合を「具体化」}するため.
	
\newpage
		\item ``生の''集合論では\textcolor{red}{集合というオブジェクトが用意されていない}ため,
			「存在」は「実在」ではない.たとえば
			\begin{align}
				\exists x\, (\, x = x\, )
			\end{align}
			は公理であり「集合は存在する」と読むが,集合を``実際に取ってくる''ことはできない.
			
		\item $\varepsilon$項を使えば,$\exists$の公理と集合の存在公理によって
			\begin{align}
				\varepsilon x\, (\, x = x\, ) = \varepsilon x\, (\, x = x\, )
			\end{align}
			が成り立つ.つまり$\varepsilon$項は「存在」を「実在」に変える
			(ある種の$\varepsilon$項は集合である).
	\end{itemize}
	
	\begin{itembox}[l]{$\varepsilon$項のメリット}
		\begin{itemize}
			\item 「存在」と「実在」が同じになる.
			\item ある種の$\varepsilon$項は集合であり,集合を具体的なオブジェクトとして扱える.
			\item 証明で用いる推論規則は三段論法のみで済む.
			\item 証明は全て閉じた式で行える.
		\end{itemize}
	\end{itembox}
	
\newpage
\subsection{クラスについて}
	\begin{itemize}
		\item ブルバキ\cite{}や島内\cite{}でも$\varepsilon$項を使った集合論を展開.
		
		\item ところで,「$\varphi(x)$を満たす集合$x$の全体」の意味の
			\begin{align}
				\Set{x}{\varphi(x)}
			\end{align}
			というオブジェクトも取り入れたい.
		
		\item ``生の''集合論では``インフォーマル''な導入.
		
		\item ブルバキ\cite{}や島内\cite{}では
			\begin{align}
				\Set{x}{\varphi(x)} \defeq \varepsilon x\, \forall u\, 
				(\, \varphi(u) \lrarrow u \in x\, )
			\end{align}
			と定める.これは欠点がある.
			\begin{align}
				\exists x\, \forall u\, (\, \varphi(u) \lrarrow u \in x\, )
			\end{align}
			が成立しない場合は「$\varphi(x)$を満たす集合$x$の全体」という意味を持たない.
			
		\item 式$\varphi$から直接$\Set{x}{\varphi(x)}$の形のオブジェクトを作ればよい.
	\end{itemize}
	
\newpage
	\begin{screen}
		\begin{dfn}[クラス]
			式$\varphi$に$x$が自由に現れていて,かつ自由に現れているのは$x$のみであるとき,
			\begin{align}
				\varepsilon x \varphi(x), \quad \Set{x}{\varphi(x)}
			\end{align}
			の形のオブジェクトを{\bf クラス(class)}と呼ぶ.
		\end{dfn}
	\end{screen}
	
	\begin{itemize}
		\item 集合はクラスである.
		\item クラスである$\varepsilon$項は集合である.
		\item 集合でないクラスもある.たとえば$\Set{x}{x = x}$や$\Set{x}{x \notin x}$
			は集合ではない.
	\end{itemize}
	
	集合の定義は竹内\cite{}に倣う.
	\begin{screen}
		\begin{dfn}[集合]
			\begin{align}
				\exists x\, (\, c = x\, )
			\end{align}
			を満たすクラス$c$を{\bf 集合(set)}と呼ぶ.
		\end{dfn}
	\end{screen}
	
	\begin{description}
		\item[NBG集合論] クラスの概念を取り入れたNBG集合論というものがあるが,
			こちらのクラスは「実在」しない.
	\end{description}

\chapter{言語}
\label{chap:languages}
	\section{導入}
	\begin{itemize}
		\item 集合論の言語$\mathcal{L}_{\in} = \{\in\}$の自然な拡張によりクラスを導入することは容易い:
			\begin{align}
				\Set{x}{\varphi(x)} \quad (\mbox{$\varphi$は$\mathcal{L}_{\in}$の式})
			\end{align}
			なるオブジェクトを取り入れればよい.
			
		\item これが持つ意味は
			\begin{align}
				\forall u\, \left(\, u \in \Set{x}{\varphi(x)}
				\Longleftrightarrow \varphi(u)\, \right)
			\end{align}
			を満たすモノ.
			
\newpage
		\item $\mathcal{L}_{\in}$においては無定義概念であった集合が
			\begin{align}
				\exists x\, (\, x = a\, )
			\end{align}
			を満たすクラス$a$のことであると\textcolor{red}{定義できる}.
		
			\begin{itembox}[l]{留意点}
				\begin{description}
					\item[※] $\varphi(x)$と書いたら,$\varphi$には変項$x$が自由に現れていて,
						また$x$の他に自由な変項は無い.
						
					\item[※] $\varphi(u)$など$x$を他の項で置換する際は,
						項は束縛による障害を受けないように選ばれている.
				\end{description}
			\end{itembox}
			
\newpage
		\item $\exists$とは?
			
		\item $\exists$に形式的な意味を付ける方法として
			\textcolor{red}{Hilbertの$\varepsilon$項}がある:
			
			式$\varphi(x)$に対して
			\begin{align}
				\varepsilon x \varphi(x).
			\end{align}
		
		\item これが持つ意味は
			\begin{align}
				\exists x \varphi(x) \Longleftrightarrow \varphi\left(\varepsilon x \varphi(x)\right)
			\end{align}
			を満たすモノ.
			
		\item 島内では$\varepsilon$項,ブルバキでは$\tau$項.
			
		\item しかし式$\varphi(x)$に対して$\varepsilon x \varphi(x)$なるオブジェクトを項とすると
			\textcolor{red}{項と式の定義が入れ子になってしまう}.
			
\newpage
		\item $\varepsilon$項を活用しつつ入れ子の問題を解消し,
			またクラスの自然な導入により具体的で直観的な集合論を構築.
		
		\item この言語の拡張がZFCの単純な保存拡大ではないのでZFCと厳密にどう関係しているかは未だ不明
			(ZFCで示せることは示せるはず.逆に本稿の集合論で示せることがZFCから示せるかは不明).
	\end{itemize}
	
\section{言語}
	\begin{itemize}
		\item 本稿で使う言語は,論理学的に書けば
			\begin{align}
				\mathcal{L}_{\in} = \{\in,\natural\}
			\end{align}
			及びその拡張言語$\mathcal{L}$.
			
		\item $\natural$とは何か?通常は$\mathcal{L}_{\in} = \{\in\}$.
		
		\item そもそも述語論理では可算個の{\bf 変項}{(variable)}として
			\begin{align}
				v_{0},\ v_{1},\ v_{2},\ \cdots
			\end{align}
			を用意していたりする.集合論の解説書も同様の記号列を変項としている...
			
\newpage
		\item でも実際の式に$v_{0},v_{1},v_{2},\cdots$なんて現れず,通常は文字
			\begin{align}
				a,b,c,\cdots, \quad x,y,z,\cdots, \quad \alpha,\beta,\gamma,\cdots.
			\end{align}
		
		\item \textcolor{red}{文字は項である}と約束する.
			
		\item ただし文字だけだと足りないので,
			
			\begin{itembox}[l]{項の生成規則}
				$\tau$と$\sigma$を項とするとき,
				\begin{align}
					\natural \tau \sigma
				\end{align}
				も項である(ポーランド記法).
			\end{itembox}
	
\newpage
		\item $\natural$を使うことの利点:
			\begin{itemize}
				\item 添え字の数字や「可算個」という言葉を用いることなく,
					\textcolor{red}{実質的に可算無限個の変項を用意できる}.
					\begin{align}
						\natural xx,\ \natural \natural xxx,\ \natural \natural \natural xxxx,\ 
						\natural \natural \natural \natural xxxxx,\ \cdots
					\end{align}
					のように,$\natural$と$x$だけで十分(極端). 
			\end{itemize}
			
			数字や可算の概念は「集合論の中で定義されるもの」と
			「感覚として持っているもの」の二つがあるが,
			字面では同じなのであまり使いたくない.
	\end{itemize}

\newpage
	\begin{itembox}[l]{$\mathcal{L}_{\in}$の項と式の定義}
		\begin{description}
			\item[項] 
				\begin{itemize}
					\item 文字は項である.
					\item 項$\tau$と項$\sigma$に対して
						$\natural \tau \sigma$は項である.
					\item これらのみが項である.
				\end{itemize}
			
			\item[式] 
				\begin{itemize}
					\item 項$\tau$と項$\sigma$に対して
						$\in \tau \sigma$と$= \tau \sigma$は式である.
					\item 式$\varphi$に対して$\rightharpoondown \varphi$は式である.
					\item 式$\varphi$と式$\psi$に対して$\vee \varphi \psi$と
						$\wedge \varphi \psi$と$\Longrightarrow \varphi \psi$
						はいずれも式である.
					\item 式$\varphi$と項$x$に対して$\exists x \varphi$と
						$\forall x \varphi$は式である.
					\item これらのみが式である.
				\end{itemize}
		\end{description}
	\end{itembox}
	
\section{言語の拡張}
	\begin{itembox}[l]{拡張の動機}
		$\mathcal{L}_{\in}$の式$\varphi(x)$に対して
		\begin{align}
			\Set{x}{\varphi(x)}
		\end{align}
		や
		\begin{align}
			\varepsilon x \varphi(x)
		\end{align}
		の形のオブジェクトを\textcolor{red}{項として}式に組み込みたい.
	\end{itembox}
	
\newpage
	$\Set{x}{\varphi(x)}$の形の項を\textcolor{red}{内包項},
	$\varepsilon x \varphi(x)$の形の項を\textcolor{red}{$\varepsilon$項}と呼ぶことにして,
	$\mathcal{L}_{\in}$に内包項と$\varepsilon$項を追加した言語を
	\begin{align}
		\mathcal{L}
	\end{align}
	と名付ける.また$\mathcal{L}_{\in}$の項は\textcolor{red}{($\mathcal{L}$の)変項}と呼ぶ.
	
	
	\begin{itembox}[l]{$\mathcal{L}$の項と式の定義}
		\begin{description}
			\item[項] 内包項と$\varepsilon$項と変項は項である.これらのみが項である.
			\item[式] 式の生成規則は$\mathcal{L}_{\in}$と殆ど同じであるが,
				\begin{itemize}
					\item 式$\varphi$と\textcolor{red}{変項}$x$に対して
						$\exists x \varphi$と$\forall x \varphi$は式である.
				\end{itemize}
				の箇所のみ変える.
		\end{description}
	\end{itembox}
	\chapter{$\varepsilon$項と内包項}
	通常は集合論の言語には$\lang{\in}$が使われる.
	しかし乍ら,当然集合論と称している以上は「集合」というモノを扱っている筈なのに,
	当の「集合」は$\lang{\in}$では実体を持たない空想でしかない.
	どういう意味かというと,例えば
	\begin{align}	
		\exists x\, \forall y\, (\, y \notin x\, )
	\end{align}
	と書けば「$\forall y\, (\, y \notin x\, )$を満たすような集合$x$が存在する」
	と読むわけだが,その在るべき$x$を$\lang{\in}$では特定できないのである.
	というのも,$\lang{\in}$の``名詞''は{\bf 変項}{\bf (variable)}だけなのだから.
	しかし言語の拡張の仕方によっては,この``空虚な存在''を実在で補強することが可能になる.
	
	\begin{comment}
	...
	考えてみれば愈々不可解である.そもそも集合なるものは我々の想像の中にしかないものであって,
	その想像を紙の上に具象化したはずの``集合論''の世界においてさえ集合が虚構に追いやられているなんて,
	どうして易々と看過できようか.
	この点で,$\lang{\in}$のみで集合論を展開することには感覚的に大きな抵抗があるわけだ.
	そこで,集合を具体的なオブジェクトとして扱えるように言語を拡張しようではないか
	(と意気込んではみるものの,遍く受け入れられている{\bf ZFC}集合論に上手く馴染めない
	偏屈な異分子のたわ言,と一笑に付されるかもしれない.まあこう弱気になることも多々あるが,
	修士号のためには偏執的なこだわりだって岩をも通すのである!).
	
	\end{comment}
	
	言語の拡張は二段階を踏む.
	項$x$が自由に現れる式$A(x)$に対して
	\begin{align}
		\Set{x}{A(x)}
	\end{align}
	なる形の項を導入する.この項の記法は{\bf 内包的記法}\index{ないほうてききほう@内包的記法}
	{\bf (international notation)}と呼ばれる.導入の意図は``$A(x)$を満たす集合$x$の全体''
	という意味を込めた式の対象化であって,実際に後で
	\begin{align}
		\forall u\, \left(\, u \in \Set{x}{A(x)} \lrarrow A(u)\, \right)
	\end{align}
	を保証する(内包性公理).
	
	追加する項はもう一種類ある.$A(x)$を上記のものとするが,この$A(x)$は$x$に関する性質という見方もできる.
	そして``$A(x)$という性質を具えている集合$x$''という意味を込めて
	\begin{align}
		\varepsilon x A(x)
	\end{align}
	なる形の項を導入するのだ.これはHilbertの{\bf $\varepsilon$項}\index{イプシロン項}
	{\bf (epsilon term)}と呼ばれるオブジェクトであるが,
	導入の意図とは裏腹に$\varepsilon x A(x)$は性質$A(x)$を持つとは限らない.
	$\varepsilon x A(x)$が性質$A(x)$を持つのは,$A(x)$を満たす集合$x$が存在するとき,またその時に限られる
	(この点については後述の$\exists$に関する定理によって明らかになる).
	$A(x)$を満たす集合$x$が存在しない場合は,$\varepsilon x A(x)$は正体不明のオブジェクトとなる.
	
\section{$\varepsilon$}
	まずは$\varepsilon$項を項として追加した
	言語$\lang{\varepsilon}$に拡張する.
	$\lang{\varepsilon}$の構成要素は以下である:
	
	\begin{description}
		\item[矛盾記号] $\bot$
		\item[論理記号] $\rightharpoondown,\ \vee,\ \wedge,\ \rarrow$
		\item[量化子] $\forall,\ \exists$
		\item[述語記号] $=,\ \in$
		\item[変項] $\lang{\in}$の項は$\lang{\varepsilon}$の
			{\bf 変項}\index{へんこう@変項}{\bf (variable)}である.またこれらのみが
			$\lang{\varepsilon}$の変項である.
		\item[イプシロン] $\varepsilon$
	\end{description}
	
	$\lang{\in}$からの変更点は,``使用文字''が``変項''に代わったことと
	$\varepsilon$が加わったことである.続いて項と式の定義に移るが,
	帰納のステップは$\lang{\in}$より複雑になる:
	
	\begin{itemize}
		\item $\lang{\varepsilon}$の変項は$\lang{\varepsilon}$の項である.
		\item $\bot$は$\lang{\varepsilon}$の式である.
		\item $\sigma$と$\tau$を$\lang{\varepsilon}$の項とするとき,
			$\in st$と$=st$は$\lang{\varepsilon}$の式である.
		\item $\varphi$を$\lang{\varepsilon}$の式とするとき,
			$\rightharpoondown \varphi$は$\lang{\varepsilon}$の式である.
		\item $\varphi$と$\psi$を$\lang{\varepsilon}$の式とするとき,
			$\vee \varphi \psi,\ \wedge \varphi \psi,\ \rarrow \varphi \psi$は
			いずれも$\lang{\varepsilon}$の式である.
		\item $x$を$\lang{\varepsilon}$の{\bf 変項}とし,$\varphi$を
			$\lang{\varepsilon}$の式とするとき,$\forall x \varphi$と
			$\exists x \varphi$は$\lang{\varepsilon}$の式である.
		\item $x$を$\lang{\varepsilon}$の{\bf 変項}とし,$\varphi$を
			$\lang{\varepsilon}$の式とするとき,$\varepsilon x \varphi$は
			$\lang{\varepsilon}$の項である.
		\item 以上のみが$\lang{\varepsilon}$の項と式である.
	\end{itemize}
	
	$\lang{\in}$に対して行った帰納的定義との大きな違いは,
	{\bf 項と式の定義が循環している}点にある.
	$\lang{\varepsilon}$の式が$\lang{\varepsilon}$の項を用いて
	作られるのは当然ながら,その逆に$\lang{\varepsilon}$の項もまた
	$\lang{\varepsilon}$の式から作られるのである.
	
	定義の循環によって構造が見えづらくなっているが,直感的には次のように捉えることが出来る.
	というよりは,次のように$\lang{\varepsilon}$が作られているとすれば良い.
	
	\begin{enumerate}
		\item $\lang{\in}$の式から$\varepsilon$項を作り,
			その$\varepsilon$項を第$1$世代$\varepsilon$項と呼ぶことにする.
		\item 変項と第$1$世代$\varepsilon$項を項として式を作り,
			これらを第$2$世代の式と呼ぶことにする.
			また第$2$世代の式で作る$\varepsilon$項を第$2$世代$\varepsilon$項と呼ぶことにする.
		\item 第$n$世代の$\varepsilon$項をが出来たら,
			それらと変項を項として第$n+1$世代の式を作り,
			第$n+1$世代$\varepsilon$項を作る.
			
			\begin{itemize}
				\item ちなみに,このように考えると第$n$世代$\varepsilon$項は
					第$n+1$世代$\varepsilon$項でもある.
			\end{itemize}
	\end{enumerate}
	
	こう捉えることで,$\lang{\varepsilon}$における構造的帰納法の原理を規定すれば良い.
	粗く考察してると,項と式に対する言明Xが与えられたとき,
	\begin{enumerate}
		\item まずは$\lang{\in}$の項と式に対してXが言えて,かつ
			第$1$世代の$\varepsilon$項に対してもXが言えることがスタート地点である.
		\item 第$2$世代の式に対してXが言えることと,第$2$世代の$\varepsilon$項に対してXが言えること
			を示す.
			
			$\vdots$
			
		\item 第$n$世代までのすべての式と項に対してXが言えることを仮定して,
			第$n+1$世代の式に対してXが言えることと,第$n+1$世代の$\varepsilon$項に対して
			Xが言えることを示す.
	\end{enumerate}
	の以上が検査出来れば,$\lang{\varepsilon}$のすべての項と式に対してXが言えると
	結論するのは妥当である.ただし第$n$世代だとかいうカテゴライズは直感的考察を補佐するための
	インフォーマルなものであり,更に簡略された手法によってこの操作が実質的に為されることが期される.
	
	\begin{screen}
		\begin{metaaxm}[$\lang{\varepsilon}$の項と式に対する構造的帰納法]
			$\lang{\varepsilon}$の項に対する言明Xと式に対する言明Yに対し,
			\begin{enumerate}
				\item $\lang{\in}$の項と式,および$\lang{\in}$の式
					で作る$\varepsilon$項に対してX及びYが言える.
				\item $\varphi$を任意に与えられた$\lang{\varepsilon}$の式として,
					$\varphi$に現れる全ての項及び真部分式に対して
					X及びYが言えると仮定するとき,
					\begin{itemize}
						\item $\varphi$が$\in \sigma \tau$なる形の原子式であるとき
							$\varphi$に対してYが言える.
						\item $\varphi$が$\rightharpoondown \varphi$なる形の式であるとき
							$\varphi$に対してYが言える.
						\item $\varphi$が$\vee \psi \chi$なる形の式であるとき
							$\varphi$に対してYが言える.
						\item $\varphi$が$\exists x \psi$なる形の式であるとき
							$\varphi$に対してYが言える.
						\item $\varepsilon x \varphi$なる$\varepsilon$項
							に対してXが言える.
					\end{itemize}
			\end{enumerate}
			ならば,いかなる項と式に対してもXが言える.
		\end{metaaxm}
	\end{screen}
	
	次の性質は至極当たり前であるが,
	
	\begin{screen}
		\begin{metathm}
			$A$を$\lang{\varepsilon}$の式としたとき,
			$\varepsilon x A$なる形の$\varepsilon$項は$A$には現れない.
		\end{metathm}
	\end{screen}
	
	もし$A$に$\varepsilon x A$が現れるならば,当然$A$の中の$\varepsilon x A$にも
	$\varepsilon x A$が現れるし,$A$の中の$\varepsilon x A$の中の$\varepsilon x A$にも
	$\varepsilon x A$が現れるといった具合に,この入れ子には終わりがなくなる.
	だが,当然こんなことは起こり得ない.$A$が指す記号列のどの部分を切り取っても
	それは$A$より短い記号列であって,$\varepsilon x A$の現れる余地など無いからである.
	
	しかしながら,やはり全容を把握しきれない世界の話になると,
	何か超然的な力が働いて現世の常識を捻じ曲げうるのではないか,という不安がぬぐえない.
	基礎論の基礎にあるのは,直感や常識の正体の究明ではないのか.
	
	$\varphi$を$\lang{\varepsilon}$の式としたら,$\varphi$の部分式とは,
	$\varphi$から切り取られる一続きの記号列で,それ自身が$\lang{\varepsilon}$の式であるものを指す.
	$\varphi$自身もまた$\varphi$の部分式である.
	
	\begin{screen}
		\begin{metathm}[$\lang{\varepsilon}$の始切片の一意性]
		\label{metathm:initial_segment_L_epsilon}
			$\tau$を$\lang{\varepsilon}$の項とするとき,
			$\tau$の始切片で$\lang{\varepsilon}$の項であるものは$\tau$自身に限られる.
			また$\varphi$を$\lang{\varepsilon}$の式とするとき,
			$\varphi$の始切片で$\lang{\varepsilon}$の式であるものは$\varphi$自身に限られる.
		\end{metathm}
	\end{screen}
	
	\begin{metaprf}\mbox{}
		\begin{description}
			\item[step1]
				$\lang{\in}$の式と項についてはメタ定理\ref{metathm:initial_segment_L_in}より
				当座の定理の主張が従う.また$\varphi$を$\lang{\in}$の式とし,
				$\tau$を$\lang{\varepsilon}$の項とし,また$\tau$は
				\begin{align}
					\varepsilon x \varphi
				\end{align}
				なる$\varepsilon$項の始切片とするとき,$\tau$の左端は$\varepsilon$であるから
				\begin{align}
					\varepsilon y \psi
				\end{align}
				なる形をしているはずである.すると$x$と$y$とは一方が他方の始切片となるので
				メタ定理\ref{metathm:initial_segment_L_in}より$y$は$x$に一致する.
				するとまた$\varphi$と$\psi$はは一方が他方の始切片となるので一致する.
				つまり$\tau$は$\varepsilon x \varphi$そのものである.
				
			\item[step2]
				$\varphi$を$\lang{\varepsilon}$の式とするとき,$\varphi$の
				すべての項や真部分式に対して定理の主張が当たっているなら
				$\varphi$に対しても定理の主張通りのことが満たされる,
				ということはメタ定理\ref{metathm:initial_segment_L_in}と同じように示される.
				もう一度書けば,
				\begin{itembox}[l]{IH (帰納法の仮定)}
					$\varphi$に現れる任意の項$\tau$に対して,その始切片で項であるものは$\tau$
					に限られる.また$\varphi$に現れる任意の真部分式$\psi$に対して,
					その始切片で式であるものは$\psi$に限られる.
				\end{itembox}
				として
				\begin{description}
					\item[case1]
						$\varphi$が
						\begin{align}
							\in s t
						\end{align}
						なる原子式であるとき,$\varphi$の始切片で式であるものもまた
						\begin{align}
							\in u v
						\end{align}
						なる形をしているが,$u$と$s$は一方が他方の始切片となっているので
						(IH)より一致する.すると$v$と$t$も一方が他方の始切片となるので
						(IH)より一致する.ゆえに$\varphi$の始切片で式であるもの
						は$\varphi$自信に限られる.
						
					\item[case2] $\varphi$が
						\begin{align}
							\rightharpoondown \psi
						\end{align}
						なる形の式であるとき,$\varphi$の始切片で式であるももまた
						\begin{align}
							\rightharpoondown \xi
						\end{align}
						なる形をしている.このとき$\xi$は$\psi$の始切片であるから,
						(IH)より$\xi$と$\psi$は一致する.
						ゆえに$\varphi$の始切片で式であるものは$\varphi$自身に限られる.
			
					\item[case3] $\varphi$が
						\begin{align}
							\vee \psi \xi
						\end{align}
						なる形の式であるとき,$\varphi$の始切片で式であるものもまた
						\begin{align}
							\vee \eta \zeta
						\end{align}
						なる形をしている.このとき$\psi$と$\eta$は一方が他方の始切片であるので
						(IH)より一致する.すると$\xi$と$\zeta$も一方が他方の始切片ということに
						なり,(IH)より一致する.ゆえに$\varphi$の始切片で式であるものは
						$\varphi$自身に限られる.
						
					\item[case4] $\varphi$が
						\begin{align}
							\exists x \psi
						\end{align}
						なる形の式であるとき,$\varphi$の始切片で式であるものもまた
						\begin{align}
							\exists y \xi
						\end{align}
						なる形の式である.このとき$x$と$y$は一方が他方の始切片であり,これらは
						変項であるからメタ定理\ref{metathm:initial_segment_L_in}
						より一致する.すると$\psi$と$\chi$も一方が他方の始切片ということに
						なり,(IH)より一致する.ゆえに$\varphi$の始切片で式であるものは
						$\varphi$自身に限られる.
						
					\item[case5] $\varepsilon x \varphi$の始切片で項であるものは
						\begin{align}
							\varepsilon y \psi
						\end{align}
						なる形をしている筈である.このとき,まずメタ定理
						\ref{metathm:initial_segment_L_in}より$x$と$y$は一致する.
						すると$\psi$は$\varphi$の始切片であることになるが,
						前段までの結果から$\varphi$と$\psi$は一致する.
						\QED
				\end{description}
		\end{description}
	\end{metaprf}
	
	\begin{screen}
		\begin{metathm}[$\lang{\varepsilon}$のスコープの存在]
			$\varphi$を$\lang{\varepsilon}$の式,或いは項とするとき,
			\begin{description}
				\item[(a)] $\natural$が$\varphi$に現れたとき,変項$s,t$が得られて,
					$\natural$のその出現位置から$\natural s t$なる変項が$\varphi$の上に現れる.
					
				\item[(b)] $\in$が$\varphi$に現れたとき,$\lang{\varepsilon}$の項$\sigma,\tau$が得られて,
					$\in$のその出現位置から$\in \sigma \tau$なる式が$\varphi$の上に現れる.
				
				\item[(c)] $\rightharpoondown$が$\varphi$に現れたとき,
					$\lang{\varepsilon}$の式$\psi$が得られて,
					$\rightharpoondown$のその出現位置から
					$\rightharpoondown \psi$なる式が$\varphi$の上に現れる.
				
				\item[(d)] $\vee$が$\varphi$に現れたとき,$\lang{\varepsilon}$の式$\psi,\xi$が得られて,
					$\vee$のその出現位置から$\vee \psi \xi$なる式が$\varphi$の上に現れる.
				
				\item[(e)] $\exists$が$\varphi$に現れたとき,変項$x$と$\lang{\varepsilon}$の式$\psi$が得られて,
					$\exists$のその出現位置から$\exists x \psi$なる式が$\varphi$の上に現れる.
			\end{description}
		\end{metathm}
	\end{screen}
	
	(b)では$\in$を$=$に替えたって同じ主張が成り立つし,(d)では$\vee$を$\wedge$や$\lrarrow$に替えても同じである.
	(e)では$\exists$を$\forall$に替えても同じであるのは良いとして,
	$\varepsilon$項の成り立ちから$\exists$を$\varepsilon$に替えても同様の主張が成り立つ.
	
	示すのはスコープの存在だけで良い.一意性は始切片の定理からすぐに従う.実際
	$\varphi$を$\lang{\varepsilon}$の式として,その中に$\varepsilon$が出現したとすると,
	``スコープの存在が保証されていれば!''$\varepsilon$のその出現位置から
	\begin{align}
		\varepsilon x \psi
	\end{align}
	なる$\varepsilon$項が$\varphi$の上に現れるわけだが,他の誰かが「$\varepsilon y \xi$という
	$\varepsilon$項がその$\varepsilon$の出現位置から抜き取れるぞ」と言ってきたとしても,
	当然ながら$x$と$y$は一方が他方の始切片となるので一致する変項であるし(メタ定理\ref{metathm:initial_segment_L_in}),
	すると今度は$\psi$と$\xi$の一方が他方の始切片となるが,そのときもメタ定理\ref{metathm:initial_segment_L_epsilon}より
	両者は一致する.
	
	\begin{metaprf}\mbox{}
		\begin{description}
			\item[step1]
				$\varphi$が$\lang{\in}$の式であるときは,スコープの存在は
				メタ定理\ref{metathm:existence_of_scopes_L_in}で既に示されている.
				また$\lang{\in}$の式$\psi$に対して,
				\begin{align}
					\varepsilon x \psi
				\end{align}
				なる形の$\varepsilon$項に対しても
				(a)から(e)が満たされる.実際,(b)から(e)に関しては,
				$\in,\rightharpoondown,\vee,\exists$は
				$\psi$の中にしか出現し得ないので,スコープの存在は
				メタ定理\ref{metathm:existence_of_scopes_L_in}により保証される.
				(a)については,$\natural$は$\psi$の中に現れる場合と$x$の中に現れる場合があるが,
				いずれの場合もメタ定理\ref{metathm:existence_of_scopes_L_in}より
				スコープは取れる.
			
				ここで$\varphi$を任意に与えられた$\lang{\varepsilon}$の
				式として,次の仮定を置く.
				\begin{itembox}[l]{IH(帰納法の仮定)}
					$\varphi$の全ての部分式,及び
					$\varphi$に現れる全ての$\varepsilon$項の式,つまり
					$\varepsilon x \psi$なる項なら$\psi$のこと,
					に対して(a)から(e)まで言えると仮定する.
				\end{itembox}
				
			\item[step2]
				式$\varphi$が$\in s t$なる形の式であるとき.
				\begin{description}
					\item[case1]
						$\natural$が$\in s t$に現れたとしよう.
						$s$や$t$が変項であれば(a)の成立は見た目通りである.$s$が
						\begin{align}
							\varepsilon x \psi
						\end{align}
						なる形の$\varepsilon$項であって,
						$s$にその$\natural$が現れているとしよう.
						$\natural$が$x$に現れている場合は
						メタ定理\ref{metathm:existence_of_scopes_L_in}に訴えればよい.
						$\natural$が$\psi$に現れている場合は,(a)の成立は(IH)から従う.
						
					\item[case2]
						$\in$が$\in s t$に現れたとしよう.
						それが左端の$\in$であれば,(b)の成立を言うには$s$と$t$を取れば良い.
						$\in$が$s$に現れたとすれば,$s$は$\varepsilon$項であることになり,
						変項$x$と$\lang{\varepsilon}$の式$\psi$が取れて,$s$は
						\begin{align}
							\varepsilon x \psi
						\end{align}
						と表せる.$\in$は$\psi$に現れるので,(IH)より$\lang{\varepsilon}$の項$u,v$が取れて,
						$\in$のその出現位置から$\in s t$なる式が$\psi$の上に現れる.
						$\in$が$t$に現れる場合も同様に(b)の成立が言える.
				
					\item[case3]
						$\in s t$に論理記号($\rightharpoondown,\vee,\wedge,\rarrow,\exists,\forall$のいずれか)
						が現れたとしよう.
						そしてその現れた記号を便宜上$\sigma$と書こう.
						$\sigma$の出現位置が$s$にあるとすれば,そのことは$s$が
						\begin{align}
							\varepsilon x \psi
						\end{align}
						なる形の$\varepsilon$項であることを意味する.当然$\sigma$は$\psi$の中にあるわけで,
						(c)もしくは(d)の成立は(IH)から従う.
						
					\item[case4]
						$\in s t$に$\varepsilon$が現れたとしよう.
						$\varepsilon$の出現位置が$s$にあるとすれば,そのことは$s$が
						\begin{align}
							\varepsilon x \psi
						\end{align}
						なる形の$\varepsilon$項であることを意味する.
						$\varepsilon$の出現位置が$s$の左端である場合,(e)の成立を言うには
						この$x$と$\psi$を取れば良い.
						$\varepsilon$が$\psi$の中にある場合は,
						$(e)$の成立は(IH)から従う.
				\end{description}
				
			\item[step3]
				式$\varphi$が$\rightharpoondown \psi$なる形のとき,
				$\varphi$に現れた記号は左端の$\rightharpoondown$であるか,そうでなければ
				$\psi$の中に現れる.左端の$\rightharpoondown$のスコープは$\varphi$自身である.
				$\psi$に現れた記号のスコープの存在は
				(IH)により保証される.
				
			\item[step4]
				式$\varphi$が$\vee \psi \xi$なる形のとき,
				$\varphi$に現れた記号は左端の$\vee$であるか,そうでなければ
				$\psi \xi$の中に現れる.左端の$\vee$のスコープは$\varphi$自身である.
				$\psi \xi$に現れた記号のスコープの存在は(IH)により保証される.
			
			\item[step5]
				式$\varphi$が$\exists x \psi$なる形のとき,
				$\varphi$に現れた記号は左端の$\exists$であるか,そうでなければ
				$\psi$の中に現れる.左端の$\exists$のスコープは$\varphi$自身である.
				$\psi$に現れた記号のスコープの存在は(IH)により保証される.
				\QED
		\end{description}
	\end{metaprf}
	
\section{言語$\mathcal{L}$}
	本稿における主流の言語は,次に定める$\mathcal{L}$である.$\mathcal{L}$の最大の特徴は
	\begin{align}
		\Set{x}{A}
	\end{align}
	なる形のオブジェクトが``正式に''項として用いられることである.
	他の集合論の本では$\Set{x}{A}$なる項はインフォーマルに導入されるもので,
	しかもこれが常に集合であることを期すために
	$\Set{x \in z}{A}$などのように何らかの$z$を引き出す必要がある.
	$\Set{x}{A}$を正式に項として導入すれば煩雑さをある程度回避することが出来る.
	
	$\mathcal{L}$の構成要素は以下のものである.
	
	\begin{description}
		\item[矛盾記号] $\bot$
		\item[論理記号] $\rightharpoondown,\ \vee,\ \wedge,\ \rarrow$
		\item[量化子] $\forall,\ \exists$
		\item[述語記号] $=,\ \in$
		\item[変項] $\lang{\in}$の項は$\mathcal{L}$の変項である.またこれらのみが
			$\mathcal{L}$の変項である.
		\item[補助記号] $\{,\ |,\ \}$
	\end{description}
	
	$\mathcal{L}$の項と式の構成規則は$\lang{\in}$のものと大差ない.
	
	\begin{description}
		\item[項] 
			\begin{itemize}
				\item $\lang{\varepsilon}$の項は$\mathcal{L}$の項である.
				\item $x$を$\mathcal{L}$の変項とし,$A$を$\lang{\varepsilon}$の式とするとき,
					$\Set{x}{A}$なる記号列は$\mathcal{L}$の項である.
				\item 以上のみが$\mathcal{L}$の項である.
			\end{itemize}
	\end{description}
	
	によって正式に定義される.ここで,$\lang{\in}$の項は$\lang{\varepsilon}$
	の項でもあるから,すなわち$\mathcal{L}$の項でもある.つまり,定義には書いていないが
	{\bf $\mathcal{L}$の変項は$\mathcal{L}$の項である}.
	
	\begin{description}
		\item[式] 
			\begin{itemize}
				\item $\bot$は$\mathcal{L}$の式である.
				\item $\sigma$と$\tau$を$\mathcal{L}$の項とするとき,
					$\in st$と$=st$は$\mathcal{L}$の式である.
				\item $\varphi$を$\mathcal{L}$の式とするとき,
					$\rightharpoondown \varphi$は$\mathcal{L}$の式である.
				\item $\varphi$と$\psi$を$\mathcal{L}$の式とするとき,
					$\vee \varphi \psi,\ \wedge \varphi \psi,\ \rarrow \varphi \psi$は
					いずれも$\mathcal{L}$の式である.
				\item $x$を$\mathcal{L}$の{\bf 変項}とし,$\varphi$を
					$\mathcal{L}$の式とするとき,$\forall x \varphi$と
					$\exists x \varphi$は$\mathcal{L}$の式である.
			\end{itemize}
	\end{description}
	
	言語の拡張の仕方より明らかであるが,次が成り立つ:
	
	\begin{screen}
		\begin{metathm}
			$\lang{\in}$の式は$\lang{\varepsilon}$の式であり,
			また$\lang{\varepsilon}$の式は$\mathcal{L}$の式である.
		\end{metathm}
	\end{screen}
	
	\begin{screen}
		\begin{dfn}[類]
			$A$を$\lang{\in}$の式とし,$x$を$A$に現れる項とし,
			$A$の中で項$x$のみが自由に現れるとき,
			$\Set{x}{A(x)}$及び$\varepsilon x A(x)$を
			{\bf 類}\index{るい@類}{\bf (class)}と呼ぶ.
		\end{dfn}
	\end{screen}
	
	$\varphi$を$\mathcal{L}$の式とし,$s$を$\varphi$に現れる記号とすると,
	\begin{description}
		\item[(1)] $s$は文字である.
		\item[(2)] $s$は$\natural$である.
		\item[(2)] $s$は$\{$である.
		\item[(3)] $s$は$|$である.
		\item[(4)] $s$は$\}$である.
		\item[(5)] $s$は$\bot$である.
		\item[(6)] $s$は$\in$か$=$である.
		\item[(7)] $s$は$\rightharpoondown$である.
		\item[(8)] $s$は$\vee,\wedge,\rightarrow$のいずれかである.
	\end{description}
	
	\begin{screen}
		(★★) いま,$\varphi$を任意に与えられた式としよう.
		\begin{itemize}
			\item $\natural$が$\varphi$に現れたとき,$\lang{\in}$の項$\tau$と$\sigma$が得られて,$\natural \tau \sigma$は
				$\natural$のその出現位置から始まる$\lang{\in}$の項となる.
				また$\natural$のその出現位置から始まる$\lang{\in}$の項は$\natural \tau \sigma$のみである.
				
			\item $\{$が$\varphi$に現れたとき,$\lang{\in}$の変項$x$及び$\lang{\in}$の式$A$が得られて,
				$\{ x|A\}$は$\{$のその出現位置から始まる項となる.
				また$\{$のその出現位置から始まる項は$\{x|A\}$のみである.
				
			\item $|$が$\varphi$に現れたとき,,変項$x$と$\lang{\in}$の式$A$が得られて,
				$\{x|A\}$は$|$のその出現位置から広がる項となる.
				また$|$のその出現位置から広がる項は$\{x|A\}$のみである.
				
			\item $\}$が$\varphi$に現れたとき,変項$x$と式$A$が得られて,
				$\{x|A\}$は$\}$のその出現位置を終点とする項となる.
				また$\}$のその出現位置を終点とする項は$\{x|A\}$のみである.
		\end{itemize}
	\end{screen}
	
	\begin{description}
		\item[$\natural$に対して$\natural \tau \sigma$なる変項$\tau$と$\sigma$が得られること]
			$\natural$が原子項に現れたら,原子項とは文字$x,y$によって
			\begin{align}
				\natural xy
			\end{align}
			と表されるものであるから,$\natural$に対して変項$\tau,\sigma$ (すなわち文字$x,y$)が取れたことになる.
			$\natural$が項に現れたとする.項とは,変項$x,y$によって
			\begin{align}
				\natural xy
			\end{align}
			で表されるものであり,$\natural$は左端の$\natural$であるか,$x$に現れるか,$y$に現れる.
			$\natural$が$x$か$y$に現れるときは帰納法の仮定により,
			$\natural$が左端のものである場合は$x$が$\tau$,$y$が$\sigma$ということになる.
			
		\item[変項の始切片で変項であるものは自分自身のみ]
			$x$が文字である場合はそう.$x$の任意の部分変項が言明を満たしているなら,
			$x$は$\natural st$なる変項である(生成規則)から,$x$の始切片は$\natural uv$なる変項である.
			$s,t,u,v$はいずれも$x$の部分変項なので仮定が適用されている.
			ゆえに$s$と$u$は一方が他方の始切片であり,一致する.すなわち$t$と$v$も一方が他方の始切片であり一致する.
			ゆえに$x$の始切片で変項であるものは$x$自身である.
			
		\item[$\natural$に対して得られる変項の一意性]
			$\natural xy$と$\natural st$が共に変項であるとき,$x$と$s$,$y$と$t$は一致するか.
			$\natural xy$が原子項であるときは明らかである.
			$x$の始切片で変項であるものは$x$自身に限られるので,
			$x$と$s$は一致する.ゆえに$t$は$y$の始切片であり,$t$と$y$も一致する.
		
		\item[生成規則より$x$と$A$が得られるか]
			$\varphi$が原子式であるとき,
			$\{$が現れるとすれば項の中である.項とは$\lang{\in}$の項であるか$\{x|A\}$なるものであるので
			$\{$が現れたならば$\{$とは$\{x|A\}$の$\{$である.
			
			$\varphi$の任意の部分式に対して言明が満たされているとする.
			$\varphi$とは$\rightharpoondown \psi,\vee \psi \xi,...$の形であるから,
			$\varphi$に現れた$\{$とは$\psi$や$\xi$に現れるのである.ゆえに
			仮定より$x$と$A$が取れるわけである.
			
		\item[$\{$に対して]
			項の生成規則より$x$と$A$が得られる.
			$\{y|B\}$もまた$\{$から始まる項である場合,順番に見ていって
			$x$と$y$は一方が他方の始切片という関係になるから一致する.
			すると$A$と$B$は一方が他方の始切片という関係になり,(★)より$A$と$B$は一致する.
			
		\item[$|$について]
			項の生成規則より$x$と$A$が得られる.
			$\{y|B\}$もまた$|$から広がる項である場合,順番に見ていって
			$x$にも$y$にも$\{$という記号は現れないので$x$と$y$は一致する.
			$A$と$B$は一方が他方の始切片という関係になるので(★)より$A$と$B$は一致する.
			
		\item[$\}$について]
			項の生成規則より$x$と$A$が得られる.
			$\{y|B\}$もまた$\}$のその出現位置を終点とする変項である場合,
			$A$と$B$は$\lang{\in}$の式なので$|$という記号は現れない.ゆえに
			$A$と$B$は一致する.すると$x$と$y$は右端で揃うが,
			$x$にも$y$にも$\{$という記号は現れないので$x$と$y$は一致する.
	\end{description}
	
\section{類と集合}
	\begin{screen}
		\begin{dfn}[類と集合]
			$a$を類とするとき,$a$が集合であるという言明を
			\begin{align}
				\set{a} \defarrow \exists x\, (\, x = a\, )
			\end{align}
			で定める.$\set{a}$を満たす類$a$を{\bf 集合}\index{しゅうごう@集合}{\bf (set)}と呼び,
			$\rightharpoondown \set{a}$を満たす類$a$を{\bf 真類}\index{しんるい@真類}{\bf (proper class)}と呼ぶ.
		\end{dfn}
	\end{screen}
	
	ちなみに$\varepsilon x A(x)$は集合である.なぜならば
	\begin{align}
		\varepsilon x A(x) = \varepsilon x A(x)
	\end{align}
	だから
	\begin{align}
		\exists a\, \left(\, a = \varepsilon x A(x)\, \right).
	\end{align}
	また$\Set{x}{A(x)}$が集合であるとき
	\begin{align}
		\exists s\, \left(\, \Set{x}{A(x)} = s\, \right)
	\end{align}
	が成り立つが,量化の規則より
	\begin{align}
		\Set{x}{A(x)} = \varepsilon s \forall u\, (\, u \in s \lrarrow A(u)\, )
	\end{align}
	が得られる.ブルバキや島内では右辺の項で内包表記を導入しているため,
	$\forall u\, (\, u \in s \lrarrow A(s)\, )$を満たす集合$s$が取れなければ
	$\Set{x}{A(x)}$は正体不明の対象となる.一方で本稿では
	内包項の意味は$\varepsilon$項に依らずにはっきり決まっている.
	
\section{式の書き換え}
	$\Set{x}{A(x)}$なる形の項を内包項,$\varepsilon x A(x)$なる形の項を$\varepsilon$項と呼び,
	これらを類と総称することにする.
	また$\varepsilon$項が現れない$\mathcal{L}$の式を甲種式,
	乙種項が現れる$\mathcal{L}$の式を乙種式と呼ぶことにする.
	
	\begin{itembox}[l]{乙種式は書き換えない}
		たとえば,$x \in \varepsilon y B(y)$なる式を$\lang{\in}$の式に書き換えるならば,
		$\varepsilon$項に込められた意味から
		\begin{align}
			\exists t\, (\, x \in t \wedge 
			(\, \exists y B(y) \rarrow B(t)\, )\, )
		\end{align}
		とするのが妥当であるだろう.しかしこうすると集合論では
		\begin{align}
			\forall x\, (\, x \in \varepsilon y\, (\, y=y\, )\, )
		\end{align}
		が成り立ってしまい,これは矛盾を起こす.実際,任意の集合$x$に対して,$t$として$\{x\}$を取れば
		\begin{align}
			\exists t\, (\, x \in t \wedge 
			(\, \exists y B(y) \rarrow B(t)\, )\, )
		\end{align}
		が満たされるので
		\begin{align}
			\forall x\, \exists t\, (\, x \in t \wedge 
			(\, \exists y\, (\, y = y\, ) \rarrow t = t\, )\, )
		\end{align}
		すなわち$\forall x\, (\, x \in \varepsilon y\, (\, y=y\, )\, )$が成り立つ.
		ところが本稿の体系では$\varepsilon y\, (\, y = y\, )$は集合であり,
		その一方で全ての集合を要素に持つ集まりというのは集合ではないから,矛盾が起こる.
		
		他に乙種式を$\lang{\in}$の式に変換する有効な方法が見つかれば話は別だが,
		それが見つからないうちは乙種式は書き換えの対象ではない.
	\end{itembox}
	
	\begin{itemize}
		\item $x \in \Set{y}{B(y)}$は$B(x)$と書き換える.
			
			これは次の公理
			\begin{align}
				\forall x\, \left(\, x \in \Set{y}{B(y)} \leftrightarrow B(x)\, \right)
			\end{align}
			に基づく式の書き換えである.
			
		\item $\Set{x}{A(x)} \in y$は$\exists s\, \left(\, s \in y \wedge 
			\forall u\, (\, u \in s \lrarrow A(s)\, )\, \right)$
			と書き換える.
			これの同値性は
			\begin{align}
				a \in b \rarrow \exists x\, (\, a = x\, )
			\end{align}
			の公理による.
			
	\end{itemize}
	
	量化は$\varepsilon$項についての規則とする.甲種乙種関係なく,式$A(x)$と任意の$\varepsilon$項$\tau$に対して
	\begin{align}
		A(\tau) \vdash \exists x A(x).
	\end{align}
	
	$A(x)$が甲種式であるとき,
	\begin{align}
		\exists x A(x) \vdash A\left(\varepsilon x \mathcal{L}A(x)\right).
	\end{align}
	
	$A(x)$を式とするとき,次の推論規則によって,$\forall x A(x)$とは
	全ての$\varepsilon$項$\tau$で$A(\tau)$が成り立つことを意味するようになる.
	\begin{align}
		\forall x A(x) &\vdash A(\tau). \\
		A(\varepsilon x \rightharpoondown \mathcal{L}A(x)) &\vdash \forall x A(x). 
	\end{align}
	
	\subsection{内包項}
	本稿における主流の言語は,次に定める$\mathcal{L}$である.$\mathcal{L}$の最大の特徴は
	\begin{align}
		\Set{x}{\varphi(x)}
	\end{align}
	なる形のオブジェクトが``正式に''項として用いられることである.
	他の多くの集合論の本では$\Set{x}{\varphi(x)}$なる項はインフォーマルに導入されるものであるが,
	インフォーマルなものでありながらこの種のオブジェクトはいたるところで堂々と登場するので,
	やはりフォーマルに導入して然るべきである.
	
	$\mathcal{L}$の構成要素は以下のものである.
	
	\begin{description}
		\item[矛盾記号] $\bot$
		\item[論理記号] $\negation,\ \vee,\ \wedge,\ \rarrow$
		\item[量化子] $\forall,\ \exists$
		\item[述語記号] $=,\ \in$
		\item[変項] \ref{sec:variables}節のもの.
		\item[補助記号] $\{,\ |,\ \}$
	\end{description}
	
	$\mathcal{L}$の項と式の構成規則は$\lang{\in}$のものと大差ない.
	
	\begin{description}
		\item[項] 
			\begin{itemize}
				\item 変項は$\mathcal{L}$の項である.
				\item $\lang{\varepsilon}$の項は$\mathcal{L}$の項である.
				\item $x$を$\mathcal{L}$の変項とし,$\varphi$を
					$\lang{\varepsilon}$の式とするとき,
					$\Set{x}{\varphi}$なる記号列は$\mathcal{L}$の項である.
				\item 以上のみが$\mathcal{L}$の項である.
			\end{itemize}
	\end{description}
	
	によって正式に定義される.
	
	\begin{description}
		\item[式] 
			\begin{itemize}
				\item $\bot$は$\mathcal{L}$の式である.
				\item $\sigma$と$\tau$を$\mathcal{L}$の項とするとき,
					$\in st$と$=st$は$\mathcal{L}$の式である.
					これらは$\mathcal{L}$の{\bf 原子式}\index{げんししき@原子式}
					{\bf (atomic formula)}である.
				\item $\varphi$を$\mathcal{L}$の式とするとき,
					$\negation \varphi$は$\mathcal{L}$の式である.
				\item $\varphi$と$\psi$を$\mathcal{L}$の式とするとき,
					$\vee \varphi \psi,\ \wedge \varphi \psi,\ \rarrow \varphi \psi$は
					いずれも$\mathcal{L}$の式である.
				\item $x$を$\mathcal{L}$の{\bf 変項}とし,$\varphi$を
					$\mathcal{L}$の式とするとき,$\forall x \varphi$と
					$\exists x \varphi$は$\mathcal{L}$の式である.
			\end{itemize}
	\end{description}
	
	\begin{screen}
		\begin{dfn}[内包項]
			$\Set{x}{\varphi}$なる項を{\bf 内包項}\index{ないほうこう@内包項}
			と呼ぶ.ここで$x$は変項であり,$\varphi$は$\mathcal{L}$の式である.
		\end{dfn}
	\end{screen}
	
	定義通りなら,$\Set{x}{y=y}$のように式$\varphi$に$x$が自由に現れていない場合でも
	$\Set{x}{\varphi}$は$\mathcal{L}$の項である.ただしそのような項は全く無用であるから,
	後で実際に集合論を構築する際には排除してしまう(\ref{sec:restriction_of_formulas}節参照).
	
	\begin{screen}
		\begin{metathm}
			$\lang{\in}$の式は$\lang{\varepsilon}$の式であり,
			また$\lang{\varepsilon}$の式は$\mathcal{L}$の式である.
		\end{metathm}
	\end{screen}
	
	\begin{metaprf}\mbox{}
		\begin{description}
			\item[step1]
				式の構成法より$\lang{\in}$の原子式は$\lang{\varepsilon}$の式である.
				また$\varphi$を任意に与えられた$\lang{\in}$の式とするとき,
				\begin{itembox}[l]{IH (帰納法の仮定)}
					$\varphi$のすべての真部分式は$\lang{\varepsilon}$の式である
				\end{itembox}
				と仮定すると,$\varphi$が
				\begin{description}
					\item[case1] $\negation \psi$
					\item[case2] $\vee \psi \chi$
					\item[case3] $\exists x \psi$
				\end{description}
				のいずれの形の式であっても,$\psi$も$\chi$も(IH)より$\lang{\varepsilon}$の式
				であるから,式の構成法より$\varphi$自信も$\lang{\varepsilon}$の式である.
				ゆえに$\lang{\in}$の式は$\lang{\varepsilon}$の式である.
				
			\item[step2]
				$\lang{\varepsilon}$の式が$\mathcal{L}$の式であることを示す.
				まず,$\mathcal{L}$の式の構成において使える項を変項に制限すれば
				全ての$\lang{\in}$の式が作られるのだから
				$\lang{\in}$の式は$\mathcal{L}$の式である.
				また$\varphi$を任意に与えられた$\lang{\varepsilon}$の式とするとき,
				\begin{itembox}[l]{IH (帰納法の仮定)}
					$\varphi$のすべての真部分式は$\mathcal{L}$の式である
				\end{itembox}
				と仮定すると(今回は予め$\lang{\varepsilon}$の項は
				$\mathcal{L}$の項とされているので,真部分式に対する仮定のみで十分である),
				\begin{description}
					\item[case1] $\varphi$が$\in \sigma \tau$なる形の原子式であるとき,
						$\sigma$も$\tau$も$\mathcal{L}$の項であるから
						$\in \sigma \tau$は$\mathcal{L}$の式である.
						
					\item[case2] $\varphi$が$\negation \psi$なる形の式であるとき,
						(IH)より$\psi$は$\mathcal{L}$の式であるから
						$\negation \psi$も$\mathcal{L}$の式である.
						
					\item[case3] $\varphi$が$\vee \psi \chi$なる形の式であるとき,
						(IH)より$\psi$も$\chi$も$\mathcal{L}$の式であるから
						$\vee \psi \chi$も$\mathcal{L}$の式である.
						
					\item[case4] $\varphi$が$\exists x \psi$なる形の式であるとき,
						(IH)より$\psi$は$\mathcal{L}$の式であるから
						$\exists x \psi$も$\mathcal{L}$の式である.
				\end{description}
				となる.ゆえに$\lang{\varepsilon}$の式は$\mathcal{L}$の式である.
				\QED
		\end{description}
	\end{metaprf}
	
	\begin{screen}
		\begin{metaaxm}[$\mathcal{L}$の式に対する構造的帰納法]
			$\mathcal{L}$の式に対する言明Xに対し,
			\begin{itemize}
				\item 原子式に対してXが言える.
				\item 無作為に選ばれた式$\varphi$について,その全ての真部分式に対してXが言える
					と仮定すれば,$\varphi$に対してもXが言える.
			\end{itemize}
			ならば,いかなる式に対してもXが言える.
		\end{metaaxm}
	\end{screen}
	
	$\mathcal{L}$の項は帰納的な構成になっていないので構造的帰納法は不要である.
	
	\begin{screen}
		\begin{metathm}[$\mathcal{L}$の始切片の一意性]
		\label{metathm:initial_segment_L}
			$\tau$を$\mathcal{L}$の項とするとき,$\tau$の始切片で$\mathcal{L}$の項であるものは
			$\tau$自信に限られる.また$\varphi$を$\mathcal{L}$の式とするとき,$\varphi$の
			始切片で$\mathcal{L}$の式であるものは$\varphi$自信に限られる.
		\end{metathm}
	\end{screen}
	
	\begin{metaprf}\mbox{}
		\begin{description}
			\item[項について]
				$\tau$を項とするとき,$\tau$が変項ならば
				メタ定理\ref{metathm:initial_segment_L_in}によって,
				$\tau$が$\lang{\varepsilon}$の項ならば
				メタ定理\ref{metathm:initial_segment_L_epsilon}によって,
				$\tau$の始切片で$\mathcal{L}$の項であるものは$\tau$自身に限られる.
				$\tau$が
				\begin{align}
					\Set{x}{\varphi}
				\end{align}
				なる内包項である場合,$\tau$の始切片で項であるものも
				\begin{align}
					\Set{y}{\psi}
				\end{align}
				なる形をしている.メタ定理\ref{metathm:initial_segment_L_in}より
				$x$と$y$が一致し,メタ定理\ref{metathm:initial_segment_L_epsilon}より
				$\varphi$と$\psi$も一致するので,この場合も$\tau$の始切片で項であるものは
				$\tau$自身に限られる.
				
			\item[式について]
				$\in st$なる原子式については,その始切片で式であるものは
				\begin{align}
					\in uv
				\end{align}
				なる形をしているが,前段の結果より$s$と$u$,$t$と$v$は一致する.
				$=st$なる原子式についても,その始切片で$\mathcal{L}$の式であるものは
				$=st$に限られる.
				
				いま$\varphi$を任意に与えられた$\mathcal{L}$の式とし,
				\begin{itembox}[l]{IH (帰納法の仮定)}
					$\varphi$に現れる任意の真部分式$\psi$に対して,
					その始切片で式であるものは$\psi$に限られる.
				\end{itembox}
				と仮定する.このとき
				\begin{description}
					\item[case1] $\varphi$が
						\begin{align}
							\negation \psi
						\end{align}
						なる形の式であるとき,$\varphi$の始切片で式であるものもまた
						\begin{align}
							\negation \xi
						\end{align}
						なる形をしている.このとき$\xi$は$\psi$の始切片であるから,
						(IH)より$\xi$と$\psi$は一致する.
						ゆえに$\varphi$の始切片で式であるものは$\varphi$自身に限られる.
			
					\item[case2] $\varphi$が
						\begin{align}
							\vee \psi \xi
						\end{align}
						なる形の式であるとき,$\varphi$の始切片で式であるものもまた
						\begin{align}
							\vee \eta \zeta
						\end{align}
						なる形をしている.このとき$\psi$と$\eta$は一方が他方の始切片であるので
						(IH)より一致する.すると$\xi$と$\zeta$も一方が他方の始切片ということに
						なり,(IH)より一致する.ゆえに$\varphi$の始切片で式であるものは
						$\varphi$自身に限られる.
						
					\item[case3] $\varphi$が
						\begin{align}
							\exists x \psi
						\end{align}
						なる形の式であるとき,$\varphi$の始切片で式であるものもまた
						\begin{align}
							\exists y \xi
						\end{align}
						なる形の式である.このとき$x$と$y$は一方が他方の始切片であり,これらは
						変項であるからメタ定理\ref{metathm:initial_segment_L_in}より
						一致する.すると$\psi$と$\chi$も一方が他方の始切片ということになり,
						(IH)より一致する.
						ゆえに$\varphi$の始切片で式であるものは$\varphi$自身に限られる.
						\QED
				\end{description}
		\end{description}
	\end{metaprf}
	
	$\varphi$を$\mathcal{L}$の式とし,$s$を
	\begin{align}
		\natural,\ \{,\ \in,\ \negation,\ \vee,
		\ \wedge,\ \rarrow,\ \exists,\ \forall,\ \varepsilon
	\end{align}
	のいずれかの記号とするとき,$s$が$\varphi$に現れたら$s$のその出現位置から始まる$\varphi$の部分式
	(ただし$s$が``$\natural,\{,\varepsilon$''である場合は部分項)を$s$の
	{\bf スコープ}\index{スコープ}{\bf (scope)}と呼ぶ.ところで$\varphi$には
	\begin{align}
		|, \quad \}
	\end{align}
	も現れるので,これらにもスコープを割り当てるために
	\begin{itemize}
		\item $\varphi$に``$|$''が現れたら,``$|$''のその出現位置を跨いで$\varphi$の上に
			現れる内包項$\Set{x}{\psi}$をその``$|$''のスコープと呼ぶ.
			つまり現れた``$|$''とは$\Set{x}{\psi}$の中心線``$|$''のことである.
			
		\item $\varphi$に``$\}$''が現れたら,``$\}$''のその出現位置を右端にして$\varphi$の上に
			現れる内包項$\Set{x}{\psi}$をその``$\}$''のスコープと呼ぶ.
			つまり現れた``$\}$''とは$\Set{x}{\psi}$の右端の``$\}$''のことである.
	\end{itemize}
	と定める.すると,次のメタ定理によって``$\natural,\ \{,\ |,\ \},\ \in,\ \negation,\ \vee,
	\ \wedge,\ \rarrow,\ \exists,\ \forall,\ \varepsilon$''の全ての記号に対して
	スコープが取れることが保証される.
	
	取れるスコープの唯一性はメタ定理\ref{metathm:initial_segment_L}からすぐに従い,
	その証明は$\lang{\in}$や$\lang{\varepsilon}$の場合と殆ど同様であるが,
	``$|$''と``$\}$''のスコープの唯一性について書いておくと
	\begin{itemize}
		\item $\varphi$の中で``$|$''のスコープ$\Set{x}{\psi}$と$\Set{y}{\chi}$が取れたとすれば,
			$\psi$と$\chi$は$\varphi$の中で同じ位置から始まる式であるから
			メタ定理\ref{metathm:initial_segment_L_epsilon}より一致する.
			また$x$と$y$は変項であるからその中に``$\{$''が現れるはずはなく,$x$と$y$も一致すると判る.
			
		\item $\varphi$の中で``$\}$''のスコープ$\Set{x}{\psi}$と$\Set{y}{\chi}$が取れたとすれば,
			$\psi$と$\chi$は$\lang{\varepsilon}$の式であるからその中に``$|$''が現れるはずはなく,
			両者は一致していなくてはならない.すると上と同様に$x$と$y$も一致していなくてはならない.
	\end{itemize}
	
	\begin{screen}
		\begin{metathm}[$\mathcal{L}$のスコープの存在]
		\label{metathm:existence_of_scopes_L}
			$\varphi$を$\mathcal{L}$の式,或いは$\mathcal{L}$の項とするとき,
			\begin{description}
				\item[(a)] $\natural$が$\varphi$に現れたとき,変項$t$が得られて,
					$\natural$のその位置から$\natural t$なる項が$\varphi$の上に現れる.
					
				\item[(b)] $\{$が$\varphi$に現れたとき,変項$x$と$\mathcal{L}$の式$\psi$が得られて,
					$\{$のその出現位置から$\Set{x}{\psi}$なる項が$\varphi$の上に現れる.
					
				\item[(c)] $|$が$\varphi$に現れたとき,変項$x$と$\mathcal{L}$の式$\psi$が得られて,
					$|$のその出現位置を跨いで$\Set{x}{\psi}$なる項が$\varphi$の上に現れる.
					
				\item[(d)] $\}$が$\varphi$に現れたとき,変項$x$と$\mathcal{L}$の式$\psi$が得られて,
					$\}$のその出現位置右端にして$\Set{x}{\psi}$なる項が$\varphi$の上に現れる.
					
				\item[(e)] $\in$が$\varphi$に現れたとき,$\mathcal{L}$の項$\sigma,\tau$が得られて,
					$\in$のその出現位置から$\in \sigma \tau$なる式が$\varphi$の上に現れる.
				
				\item[(f)] $\negation$が$\varphi$に現れたとき,$\mathcal{L}$の式$\psi$が得られて,
					$\negation$のその出現位置から$\negation \psi$なる式が
					$\varphi$の上に現れる.	
				
				\item[(g)] $\vee$が$\varphi$に現れたとき,$\mathcal{L}$の式$\psi,\xi$が得られて,
					$\vee$のその出現位置から$\vee \psi \xi$なる式が$\varphi$の上に現れる.
				
				\item[(h)] $\exists$が$\varphi$に現れたとき,変項$x$と$\mathcal{L}$の式$\psi$が得られて,
					$\exists$のその出現位置から$\exists x \psi$なる式が$\varphi$の上に現れる.
			\end{description}
		\end{metathm}
	\end{screen}
	
	\begin{metaprf}\mbox{}
		\begin{description}
			\item[case1] $\in st$なる原子式に対しては,
				\begin{itemize}
					\item $\natural,\negation,\vee,\exists$が現れたとすれば,
						それらは$s$か$t$の中に現れているのであり,
						メタ定理\ref{metathm:existence_of_scopes_L_in}と
						メタ定理\ref{metathm:existence_of_scopes_L_epsilon}より
						それらのスコープは取れる.仮に$s$と$t$の一方が
						\begin{align}
							\Set{x}{\psi}
						\end{align}
						なる内包項であるとしても,$\natural,\negation,\vee,\exists$が
						現れうるのは$x$或いは$\psi$の中であるから,
						スコープの存在は上記のメタ定理に訴えればよい.
				
					\item $\in st$に$\in$が現れたとすれば,それが$s,t$の中のものならば
						上記の定理によってスコープは取れるし,それが$\in st$の左端の
						$\in$を指しているなら$\in st$自身をスコープとして取れば良い.
						
					\item $\in st$に$\{,\ |,\ \}$が現れたとすれば,$s$と$t$の少なくとも一方は
						\begin{align}
							\Set{x}{\psi}
						\end{align}
						なる項であることになるので,スコープとしてこの内包項を取れば良い.
				\end{itemize}
				
			\item[case2] $\varphi$を任意に与えられた$\mathcal{L}$の式として
				$\varphi$を任意に与えられた式として
				\begin{itembox}[l]{IH (帰納法の仮定)}
					$\varphi$の全ての真部分式に対しては(a)から(h)の主張が当てはまる
				\end{itembox}
				と仮定する.このとき,
				\begin{itemize}
					\item $\varphi$が
						\begin{align}
							\negation \psi
						\end{align}
						なる形の式であるとき,$\natural,\{,|,\},\in,\vee,\exists$が
						$\varphi$に現れたなら,それらは$\psi$の中に現れているのだから
						(IH)よりスコープが取れる.また$\varphi$に$\negation$が現れた場合,
						その$\negation$が$\psi$の中のものならば(IH)に訴えれば良いし,
						$\varphi$の左端の$\negation$を指しているなら
						スコープとして$\varphi$自身を取れば良い.
						
					\item $\varphi$が
						\begin{align}
							\vee \psi \chi
						\end{align}
						なる形の式であるとき,$\natural,\{,|,\},\in,\negation,\exists$が
						$\varphi$に現れたなら,それらは$\psi$か$\chi$の中に現れているのだから
						(IH)よりスコープが取れる.また$\varphi$に$\vee$が現れた場合,
						その$\vee$が$\psi,\chi$の中のものならば(IH)に訴えれば良いし,
						$\varphi$の左端の$\vee$を指しているなら
						スコープとして$\varphi$自身を取れば良い.
						
					\item $\varphi$が
						\begin{align}
							\exists x \psi
						\end{align}
						なる形の式であるとき,$\natural,\{,|,\}\in,\negation,\vee$が
						$\varphi$に現れたなら,それらは$\psi$の中に現れているのだから
						(IH)よりスコープが取れる.また$\varphi$に$\exists$が現れた場合,
						その$\exists$が$\psi$の中のものならば(IH)に訴えれば良いし,
						$\varphi$の左端の$\exists$を指しているなら
						スコープとして$\varphi$自身を取れば良い.
						\QED
				\end{itemize}
		\end{description}
	\end{metaprf}
	\section{中置記法}
	たとえば$\in s t$なる原子式は「$s$は$t$の要素である($s$ is in $t$)」と読むのだから,語順通りに,
	或いは$s$が$t$の中にあるというイメージ通りに
	\begin{align}
		s \in t
	\end{align}
	と書きかえる方が見やすくなる.同じように,$\vee \varphi \psi$なる式も
	「$\varphi$または$\psi$」と読むのだから
	\begin{align}
		\varphi \vee \psi
	\end{align}
	と書きかえる方が見やすくなる.$\rarrow \vee \varphi \psi \wedge \chi \xi$のように長い式も,
	上の作法に倣えば
	\begin{align}
		\begin{gathered}
			\rarrow \vee \varphi \psi \wedge \chi \xi \\
			\rarrow \color{red}{\varphi \vee \psi} \color{blue}{\chi \wedge \xi} \\
			\color{red}{\varphi \vee \psi} \color{black}{\rarrow} \color{blue}{\chi \wedge \xi}
		\end{gathered}
	\end{align}
	と書きかえることになるが,一々色分けするわけにもいかないので``(''と``)''を使って
	\begin{align}
		(\varphi \vee \psi) \rarrow (\chi \wedge \xi)
	\end{align}
	と書くようにすれば良い.
	
	\begin{itembox}[l]{{\bf 中置記法}\index{ちゅうちきほう@中置記法}{\bf (infix notation)}}
			$\mathcal{L}$の式は以下の手順で中置記法に変換する.
			\begin{enumerate}
				\item $\in s t$なる形の原子式は$s \in t$と書きかえる.
					$= s t$も同様に書き換える.
					
				\item $\negation \varphi$なる形の式は,$\varphi$の中置記法への変換
					$\widehat{\varphi}$を用いて$\negation (\widehat{\varphi})$と変換する.
				
				\item $\vee \varphi \psi$なる形の式は,$\varphi,\psi$の中置記法への変換
					$\widehat{\varphi},\widehat{\psi}$を用いて
					$(\widehat{\varphi}) \vee (\widehat{\psi})$と変換する.
					$\wedge \varphi \psi$と$\rarrow \varphi \psi$の形の式も同様に変換する.
				
				\item $\exists x \varphi$なる形の式は,$\varphi$の中置記法への変換
					$\widehat{\varphi}$を用いて$\exists x (\widehat{\varphi})$と変換する.
					$\forall x \varphi,\varepsilon x \varphi$なる形の式や項も同様にする.
					
				\item $\Set{x}{\varphi}$なる形の項は,$\varphi$の中置記法への変換
					$\widehat{\varphi}$を用いて$\Set{x}{\widehat{\varphi}}$と変換する.
			\end{enumerate}
			中置記法は表示用の記法であって,扱う項や式の``本来の姿''は前節までの{\bf 前置記法}
			\index{ぜんちきほう@前置記法}{\bf (prefix notation)}で書かれたものである.
	\end{itembox}
	
	上の変換法では,たとえば$\rarrow \vee \varphi \psi \wedge \chi \xi$なる式は
	\begin{align}
		(\, (\widehat{\varphi}) \vee (\widehat{\psi})\, ) 
		\rarrow (\, (\widehat{\chi}) \wedge (\widehat{\xi})\, )
	\end{align}
	となるが,括弧はあくまで式の境界の印として使うものであるから,内側の括弧は外して
	\begin{align}
		(\, \widehat{\varphi} \vee \widehat{\psi}\, ) 
		\rarrow (\, \widehat{\chi} \wedge \widehat{\xi}\, )
	\end{align}
	と書く方が良い.
	
	$\wedge \vee \exists x \varphi \psi \negation \rarrow \chi \in s t$なる式を変換すると
	\begin{align}
		(\, (\exists x (\widehat{\varphi})) \vee (\widehat{\psi})\, ) \wedge (\negation (\, (\widehat{\chi}) \rarrow (s \in t)\, ))
	\end{align}
	となるが,これも内側の括弧および$\negation ...$を囲う括弧は外して
	\begin{align}
		(\, \exists x (\widehat{\varphi}) \vee \widehat{\psi}\, ) \wedge \negation (\, \widehat{\chi} \rarrow s \in t\, )
	\end{align}
	と書く.
	
	あまり括弧が連なると読みづらくなるので,
	\begin{align}
		(\, \varphi \vee \psi\, ) \rarrow (\chi)
	\end{align}
	なる形の式は
	\begin{align}
		\varphi \vee \psi \rarrow \chi
	\end{align}
	に,同様に
	\begin{align}
		(\varphi) \rarrow (\, \psi \vee \chi\, )
	\end{align}
	なる形の式は
	\begin{align}
		\varphi \rarrow \psi \vee \chi
	\end{align}
	とも書く.また$\vee$が$\wedge$であっても同じように括弧を省く.
	\begin{align}
		\exists x (\negation (\varphi))
	\end{align}
	なる式は
	\begin{align}
		\exists x \negation (\varphi)
	\end{align}
	とも書き,$\exists$が$\forall$や$\varepsilon$であっても同じように括弧を省く.
	\subsection{量化}
	$\varphi$を$\mathcal{L}$の式とする.もし$\varphi$に$\forall$が現れたら,
	その$\forall$に後続する変項$x$と式$\psi$が取れるが,そのとき$x$は
	\begin{align}
		\forall x \psi
	\end{align}
	の中で{\bf 「量化されている」}\index{りょうか@量化}{\bf(quantified)}や
	{\bf 「束縛されている」}\index{そくばく@束縛}{\bf (bound)}という.
	同様に$\varphi$の中に$\exists$や$\varepsilon$が現れたら,
	その$\exists$ (または$\varepsilon$)の直後にくる変項は,
	「その$\exists$ (または$\varepsilon$)のスコープの中で束縛されている」といい,
	また$\varphi$の中に
	\begin{align}
		\Set{x}{\psi}
	\end{align}
	なる内包項が現れたら,$x$は「この内包項の中で束縛されている」という.
	他方で$\psi$の中に$x$とは別の変項が現れていても,その変項は
	$\forall x \psi,\ \exists x \psi,\ \varepsilon x \psi,\ \Set{x}{\psi}$
	の中では「束縛されていない」と解釈する.
	まとめれば,\underline{$\forall,\exists,\varepsilon,$そして$\{$は
	直後に来る変項のみをそのスコープ内で束縛している}のである.たとえば
	\begin{align}
		\forall x\, (\, x \in y\, )
	\end{align}
	においては$x$は束縛されているし,
	\begin{align}
		\Set{u}{u = z}
	\end{align}
	において$u$は束縛されている.束縛は二重に行われることもある.例えば
	\begin{align}
		\forall x\, (\, \forall x\, (\, x \in y\, ) \rarrow (\, x \in z\, )\, )
	\end{align}
	なる式においては,$\forall x\, (\, x \in y\, )$にある$x$は
	上式で一番左の$\forall$のスコープ内の$x$でもあるので,これらの$x$は二重に束縛されていることになる.
	仮に「何重にも束縛されている場合は最も広いスコープで束縛されていることにする」と決めても良いが,
	ただし重要なのは変項が束縛されているか否かであって,それが二重でも三重でもどうでも構わない.
	
	上の例では$y$と$z$は束縛されていないが,考えている項や式の中で束縛されていない変項
	を{\bf 自由な}\index{じゆう@自由}{\bf (free)}変項と呼ぶ.
	現れる変項が自由であるか否かは当然その出現位置に依存しているのであり,たとえば
	\begin{align}
		\forall x\, (\, x \in y\, ) \rarrow (\, x \in z\, )
	\end{align}
	なる式では左の二つの$x$が束縛されている一方で右の$x$は自由であるように,
	同じ変項が複数個所に現れる場合はその変項が束縛されているか自由であるかは一概には言えない.
	式$\varphi$の中に束縛されていない変項が現れている場合は,
	その変項が``その位置''に現れていることを
	{\bf 自由な出現}\index{じゆうなしゅつげん@自由な出現}{\bf (free occurrence)}と呼ぶ.
	
	\begin{screen}
		\begin{metadfn}[文]
			自由な変項が現れない$\mathcal{L}$の式を{\bf 文}\index{ぶん@文}{\bf (sentence)}
			や{\bf 閉式}\index{へいしき@閉式}{\bf (closed formula)}と呼ぶ.
		\end{metadfn}
	\end{screen}
	\subsection{代入}
	変項とは束縛されうる項であったが,別の項を代入されうる項でもある.
	代入とは別の項で置き換えるということであり,また代入されうるのは式の中で自由な変項のみである.
	ただし,代入には「{\bf 式の中の自由な変項を別の変項に取り替えても式の意味を変えてはならない}」という
	大前提がある.たとえば
	\begin{align}
		\forall u\, (\, u \in x\, )
	\end{align}
	という式で考察すると,この式で$x$は自由であるから別の項を代入して良いのであり,$z$を代入すれば
	\begin{align}
		\forall u\, (\, u \in z\, )
	\end{align}
	となる.そしてこの場合はどちらの式も意味は同じである.意味が同じであるとは
	量化してみれば一目瞭然であって,両式を全称記号で量化すれば
	\begin{align}
		&\forall x\, \forall u\, (\, u \in x\, ), \\
		&\forall z\, \forall u\, (\, u \in z\, )
	\end{align}
	はどちらも「どの集合も,全ての集合を要素に持つ」と解釈され,
	両式を存在記号で量化すれば
	\begin{align}
		&\exists x\, \forall u\, (\, u \in x\, ), \\
		&\exists z\, \forall u\, (\, u \in z\, )
	\end{align}
	はどちらも「或る集合は,全ての集合を要素に持つ」と解釈される.
	ところが$x$に$u$を代入すると
	\begin{align}
		\forall u\, (\, u \in u\, )
	\end{align}
	となり,これは「全ての集合は自分自身を要素に持つ」という意味に変わる.
	つまり先の大前提に立てば,代入する際には{\bf 代入後に束縛されてしまう変項は使ってはいけない}のである.
	
	代入するのは変項だけではない.$\varepsilon$項や内包項だって上の$x$に代入して良い.
	ただし上と同様の注意が必要で,$\varepsilon$項や内包項に$u$が自由に現れている場合と
	そうでない場合では代入後の式の意味が分かれてしまうので,
	代入して良い項は$u$が自由に現れていないものに限る.
	
	以上の考察を一般的な代入規則に敷衍して言えば,
	
	\begin{itembox}[l]{代入可能な項}
		$\varphi$を$\mathcal{L}$の式とし,$x$を$\varphi$に自由に現れる変項とし,
		$\tau$を$\mathcal{L}$の項とする.このとき「$\varphi$に自由に現れる$x$に$\tau$を
		代入する」とは,特筆が無い限り$\varphi$に自由に現れる全ての$x$に
		$\tau$を代入することであって,その際に$\tau$が満たすべき条件は
		\begin{itemize}
			\item $\tau$が変項ならば$\tau$は$\varphi$に代入されたどの箇所でも自由である
			\item $\tau$が$\varepsilon$項や内包項である場合は,
				$\tau$の中に自由に現れる変項があったとしても,
				それらは全て$\tau$が代入されたどの箇所でも束縛されない
		\end{itemize}
		とする.$\tau$がこの条件を満たすとき,
		{\bf 「$\tau$は$\varphi$の中で$x$への代入について自由である」}という.
	\end{itembox}
	
	$\varphi$に自由に現れる$x$に$\tau$を代入した後の式を
	\begin{align}
		\varphi(x/\tau)
	\end{align}
	と書く($x/\tau$は``replace $x$ by $\tau$''の順).
	特に$\varphi$の中に自由に現れている変項が$x$だけである場合は,$\varphi(x/\tau)$を
	\begin{align}
		\varphi(\tau)
	\end{align}
	とも書く.$\tau$が$x$自身である場合は$\varphi(x)$は$\varphi$そのものであるが,
	「$\varphi$に自由に現れているのは$x$だけである」ということを強調するために
	\begin{align}
		\varphi(x)
	\end{align}
	と書くことも多い.$\varphi$に別の変項$y$が現れていて,$y$に項$\sigma$を代入するときは,
	\begin{align}
		\varphi(x/\tau)(y/\sigma)
	\end{align}
	を
	\begin{align}
		\varphi(x/\tau,y/\sigma)
	\end{align}
	とも書く.特に$\varphi$の中に自由に現れている変項が$x$と$y$だけである場合は,
	$\tau$と$\sigma$の代入先がはっきりしていれば
	\begin{align}
		\varphi(\tau,\sigma)
	\end{align}
	とも書くし,「$\varphi$に自由に現れているのは$x$と$y$だけである」ということを強調するために
	\begin{align}
		\varphi(x,y)
	\end{align}
	と書くことも多い.$\varphi$に$x$が自由に現れていない場合でも$\varphi(x/\tau)$などと書かれていたら,
	その式は$\varphi$のことであると理解する.
	\subsection{類}
	\begin{comment}
	\begin{screen}
		\begin{dfn}[閉項]
			どの変項も自由に現れない$\varepsilon$項を
			{\bf 閉${\boldsymbol \varepsilon}$項}\index{
			へいイプシロンこう@閉$\varepsilon$項}{\bf (closed epsilon term)}と呼び,
			どの変項も自由に現れない内包項を{\bf 閉内包項}\index{
			へいないほうこう@閉内包項}{\bf (closed comprehension term)}と呼ぶ.
			また閉$\varepsilon$項と閉内包項は以上のみである.
		\end{dfn}
	\end{screen}
	\end{comment}
	
	元々の意図としては,例えば$x$のみが自由に現れる式$\varphi(x)$に対して
	「$\varphi(x)$を満たすいずれかの集合$x$」という意味を込めて
	\begin{align}
		\varepsilon x \varphi(x)
	\end{align}
	を作ったのだし,「$\varphi(x)$を満たす集合$x$の全体」という意味を込めて
	\begin{align}
		\Set{x}{\varphi(x)}
	\end{align}
	を作ったのである.つまりこの場合の$\varepsilon x \varphi(x)$と
	$\Set{x}{\varphi(x)}$は``意味を持っている''わけである.
	これが,もし$x$とは別の変項$y$が$\varphi$に自由に現れているとすれば,
	$\varepsilon x \varphi$も$\Set{x}{\varphi}$も$y$に依存してしまい
	意味が定まらなくなる.というのも,変項とは代入可能な項であるから,$y$に代入する項ごとに
	$\varepsilon x \varphi$と$\Set{x}{\varphi}$は別の意味を持ち得るのである.
	また項が閉じていても意味不明な場合がある.たとえば
	\begin{align}
		\varepsilon y \forall x\, (\, x = x\, )
	\end{align}
	や
	\begin{align}
		\Set{y}{\forall x\, (\, x = x\, )}
	\end{align}
	なる項は閉じてはいるが,導入の意図には適っていない.意味不明ながらこういった項が存在しているのは
	導入時にこれらを排除する面倒を避けたからであり,また一旦すべてを作り終えた後で余計なものを捨てる方が
	楽だからである.とりあえず,導入の意図に適っている項は特別の名前を持っているべきである.
	
	\begin{screen}
		\begin{dfn}[類]
			$\varphi$を$\lang{\varepsilon}$の式とし,$x$を変項とし,
			$\varphi$には$x$のみ自由に現れているとするとき,$\varepsilon x \varphi(x)$
			と$\Set{x}{\varphi(x)}$を{\bf 類}\index{るい@類}{\bf (class)}と呼ぶ.
			またこれらのみが類である.
		\end{dfn}
	\end{screen}
	
	類には二種類あるので,それらも名前を分けておく.
	\begin{screen}
		\begin{dfn}[主要$\varepsilon$項]
			類である$\varepsilon$項を{\bf 主要${\boldsymbol \varepsilon}$項}
			\index{しゅよういぷしんろんこう@主要$\varepsilon$項}
			{\bf (critical epsilon term)}と呼ぶ.
		\end{dfn}
	\end{screen}
	
	後述することだが,本稿における集合とは,主要$\varepsilon$項か
	主要$\varepsilon$項に等しい類のことである
	(定理\ref{thm:critical_epsilon_term_is_set}).
	
	
	\begin{screen}
		\begin{dfn}[主要内包項]
			類である内包項を{\bf 主要内包項}\index{しゅようないほうこう@主要内包項}と呼ぶ.
		\end{dfn}
	\end{screen}
	
	内包項に関しては便宜上自由な変項の出現も許すことにするが,
	たとえば$\Set{x}{\varphi}$と書いたら少なくとも$x$は$\varphi$に自由に現れているべきであり,
	この意味で性質の良い内包項に対しても特別な名前を付けておく.
	
	\begin{screen}
		\begin{dfn}[正則内包項]
			$\varphi$を$\lang{\varepsilon}$の式とし,$x$を変項とし,
			$\varphi$に$x$が自由に現れているとするとき,
			$\Set{x}{\varphi}$を{\bf 正則内包項}\index{せいそくないほうこう@正則内包項}と呼ぶ.
		\end{dfn}
	\end{screen}
	
\subsection{扱う式の制限}
\label{sec:restriction_of_formulas}
	\begin{itembox}[l]{式の制限}
		以降で扱う$\mathcal{L}$の項と式に対して,特筆が無い限り次が満たされていることを約束する:
		\begin{itemize}
			\item 式に現れる$\varepsilon$項は全て主要$\varepsilon$項である.
			\item 式に現れる内包項は全て正則内包項である.
			\item 項或いは式の上に現れる$\forall x \psi,\exists x \psi$なる形の式は,$\psi$の中に$x$が自由に現れている.
		\end{itemize}
	\end{itembox}
	
	項の中に現れる$\varepsilon$項も,項の中の項の中に現れる$\varepsilon$項も,
	現れうる$\varepsilon$項は全て主要$\varepsilon$項である.
	
	\begin{screen}
		\begin{metathm}[$\lang{\varepsilon}$の式に代入後も$\lang{\varepsilon}$の式]
		\label{metathm:substitutions_into_L_epsilon_formulas}
			$\varphi$を$\lang{\varepsilon}$の式とし,$x$を$\varphi$に自由に現れる変項とし,
			$a$を$\varphi$の中で$x$への代入について自由である$\lang{\varepsilon}$の項
			(変項または主要$\varepsilon$項)とする.
			このとき$\varphi(x/a)$は$\lang{\varepsilon}$の式である.
		\end{metathm}
	\end{screen}
	
	\begin{metaprf}\mbox{}
		\begin{description}
			\item[step1] $\varphi$が原子式であるとき,例えば$\varphi$が
				\begin{align}
					x \in b
				\end{align}
				なる式であれば,$\varphi(x/a)$は
				\begin{align}
					a \in b
				\end{align}
				なる式である($b$が$x$ならば$\varphi(x/a)$は$a \in a$となる.$b$が$x$でない
				ならば$b$は変項か主要$\varepsilon$項なので$\varphi(x/a)$は$a \in b$となる).
				他の場合も同様に$\varphi(x/a)$が$\lang{\in}$の式であると判る.
			
			\item[step2] $\varphi$が原子式でないとき,
				\begin{itembox}[l]{IH (帰納法の仮定)}
					$\varphi$の任意の真部分式$\psi$に対して
					$\psi(x/a)$は$\lang{\varepsilon}$の式である
				\end{itembox}
				と仮定する.
				\begin{description}
					\item[case1] $\varphi$が
						\begin{align}
							\negation \psi
						\end{align}
						なる式のとき,$\varphi(x/a)$は
						\begin{align}
							\negation \psi(x/a)
						\end{align}
						なる式であって,(IH)より$\psi(x/a)$は$\lang{\varepsilon}$の式であるから
						$\varphi(x/a)$は$\lang{\varepsilon}$の式である.
						
					\item[case2] $\varphi$が
						\begin{align}
							\vee \psi \xi
						\end{align}
						なる式のとき,$\varphi(x/a)$は
						\begin{align}
							\vee \psi(x/a) \xi(x/a)
						\end{align}
						なる式であって,(IH)より$\psi(x/a),\xi(x/a)$は$\lang{\varepsilon}$の式
						であるから$\varphi(x/a)$は$\lang{\varepsilon}$の式である.
					
					\item[case3] $\varphi$が
						\begin{align}
							\exists z \psi
						\end{align}
						なる式のとき,$\varphi(x/a)$は
						\begin{align}
							\exists z \psi(x/a)
						\end{align}
						なる式であって,(IH)より$\psi(x/a)$は$\lang{\varepsilon}$の式であるから
						$\varphi(x/a)$は$\lang{\varepsilon}$の式である.
						\QED
				\end{description}
		\end{description}
	\end{metaprf}
	\subsection{式の書き換え}
\label{subsec:formula_rewriting}
	$\varepsilon$項を取り入れたのは{\bf 存在文}\index{そんざいぶん@存在文}
	{\bf (existential sentence)}に対して{\bf 証人}\index{しょうにん@証人}
	{\bf (witnessing term)}を与えるためであり,それは
	\begin{align}
		\exists x \varphi(x) \rarrow \varphi(\varepsilon x \varphi(x))
	\end{align}
	なる式を公理とすることで裏付けされる.ただし$\varepsilon$項を作れるのは$\lang{\varepsilon}$
	の式のみであるから,$\varphi$が内包項を含んだ式であると$\varepsilon x \varphi(x)$を
	使うことが出来ない.とはいえ内包項を含んだ存在文も往々にして登場するので,それらに対しても
	証人を用意できると便利である.そこで$\varphi$を内包項を含んだ$\mathcal{L}$の式とするとき,
	$\varphi$を``同値''な$\lang{\varepsilon}$の式$\hat{\varphi}$に書き換えて
	\begin{align}
		\exists x \varphi(x) \rarrow \varphi(\varepsilon x \hat{\varphi}(x))
	\end{align}
	を公理とする(量化公理\ref{logicalaxm:rules_of_quantifiers}).
	注意点は\underline{同値な書き換えはいくらでも作れる}ということであり,
	$\check{\varphi}$も$\varphi$の書き換えならば
	\begin{align}
		\exists x \varphi(x) \rarrow \varphi(\varepsilon x \check{\varphi}(x))
	\end{align}
	も公理とする.書き換える必要があるのは内包項を含んでいる式のみであり,
	またそのような式の原子式の部分,つまり$\in$と$=$のスコープ,だけを書き換えれば十分である.
	書き換えが``同値''というのは後述の\ref{sec:equivalence_of_formula_rewriting}節
	で述べてあるような意味であるが,直感的に妥当な範囲でしかない.原子式の書き換えは次の要領で行う:
	
	\begin{table}[H]
		\begin{center}
		\begin{tabular}{c|c|c}
			元の式 & 書き換え後 & 付記 \\ \hline \hline
			$a = \Set{z}{\psi}$ & $\forall v\, (\, v \in a \lrarrow \psi(z/v)\, )$ & \\ \hline
			$\Set{y}{\varphi} = b$ & $\forall u\, (\, \varphi(y/u) \lrarrow u \in b\, )$ & \\ \hline
			$\Set{y}{\varphi} = \Set{z}{\psi}$ & $\forall u\, (\, \varphi(y/u) \lrarrow \psi(z/u)\, )$ & \\ \hline
			$a \in \Set{z}{\psi}$ & $\psi(z/a)$ & 必要なら束縛変項の名前替えをする \\ \hline
			$\Set{y}{\varphi} \in b$ & $\exists s\, (\, \forall u\, (\, \varphi(y/u) \lrarrow u \in s\, ) \wedge s \in b\, )$ & \\ \hline
			$\Set{y}{\varphi} \in \Set{z}{\psi}$ & $\exists s\, (\, \forall u\, (\, \varphi(y/u) \lrarrow u \in s\, ) \wedge \psi(z/s)\, )$ & \\ \hline
		\end{tabular}
		\end{center}
	\end{table}
	
	ただし上の記号に課している条件は
	\begin{itemize}
		\item $a,b$は$\lang{\varepsilon}$の項である
			(\ref{sec:restriction_of_formulas}節より
			$a,b$は変項か主要$\varepsilon$項).
		
		\item $\Set{y}{\varphi}$と$\Set{z}{\psi}$を正則内包項である.
		
		\item $u$は$\varphi$の中で$y$への代入について自由であり,
			$u,v,s$は$\psi$の中で$z$への代入について自由である.
			上の式の書き換えにおいては変項$u,v,s$を追加したが,
			代入について自由である限りどの変項を選んでも構わない.
			従って\underline{式の書き換えは一つに決まらない}ということになるが,
			違う変項を選んでも式の意味は変わらない.
			
		\item 付記「束縛変項の名前替え」について.
			$a$を$\psi$の中の自由な$z$に代入した後で$a$が束縛される場合,
			束縛変項の名前替えをしなくてはならない.たとえば
			\begin{align}
				a \in \Set{z}{\forall a\, (\, z \in a\, )}
			\end{align}
			という式では左辺の$a$は自由であるのに,書き換えの規則をそのまま適用すると
			\begin{align}
				\forall a\, (\, a \in a\, )
			\end{align}
			となり束縛されてしまう.代入後の$a$が束縛されないためには
			\begin{align}
				a \in \Set{z}{\forall b\, (\, z \in b\, )}
			\end{align}
			のように束縛変項$a$を別の変項$b$に替えて
			\begin{align}
				\forall b\, (\, a \in b\, )
			\end{align}
			とすればよい.
	\end{itemize}
	
	\begin{screen}
		\begin{metadfn}[式の書き換え]
			$\varphi$を$\lang{\varepsilon}$の式ではない$\mathcal{L}$の式とするとき,
			$\varphi$の原子式の部分,つまり$\in$と$=$のスコープを全て上表に従って直した式を
			$\varphi$の{\bf 書き換え}と呼ぶ.
		\end{metadfn}
	\end{screen}
	
	\begin{screen}
		\begin{metathm}[書き換えは$\lang{\varepsilon}$の式]
		\label{metathm:rewritten_formulas_are_of_L_epsilon}
			$\varphi$を$\lang{\varepsilon}$の式ではない$\mathcal{L}$の式とし,
			$\hat{\varphi}$を$\varphi$の書き換えとするとき,
			$\hat{\varphi}$は$\lang{\varepsilon}$の式である.
		\end{metathm}
	\end{screen}
	
	\begin{metaprf}\mbox{}
		\begin{description}
			\item[step1] $\varphi$が原子式なら,表の通り$\hat{\varphi}$は
				$\lang{\varepsilon}$の式である.
			
			\item[step2] $\varphi$が原子式ないとき,
				\begin{itembox}[l]{IH (帰納法の仮定)}
					$\varphi$の任意の真部分式は,それが$\lang{\varepsilon}$の式でない場合
					その書き換えは$\lang{\varepsilon}$の式である.
				\end{itembox}
				と仮定する.
				\begin{description}
					\item[case1] $\varphi$が
						\begin{align}
							\negation \psi
						\end{align}
						なる式のとき,$\varphi$の$\in,=$のスコープはいずれも
						$\psi$の部分原子式であり,逆に$\psi$の$\in,=$のどのスコープも
						$\varphi$の部分原子式であるから,$\varphi$の原子式の部分を
						全て書き換えるということは$\psi$の原子式の部分を全て書き換える
						ということになる.$\hat{\varphi}$は
						\begin{align}
							\negation \hat{\psi}
						\end{align}
						なる形の式であるが,$\hat{\psi}$は$\psi$の書き換えであり,
						(IH)より$\lang{\varepsilon}$の式である.ゆえに
						$\hat{\varphi}$も$\lang{\varepsilon}$の式である.
						
					\item[case2] $\varphi$が
						\begin{align}
							\vee \psi \chi
						\end{align}
						なる式のとき,$\varphi$の$\in,=$のスコープはいずれも
						$\psi$か$\chi$の一方の部分原子式であり(始切片の一意性
						のメタ定理\ref{metathm:initial_segment_L}より
						$\varphi$の真部分式が$\psi$と$\chi$の境を跨ぐことはない),
						逆に$\psi,\chi$の$\in,=$のどのスコープも
						$\varphi$の部分原子式であるから,$\varphi$の原子式の部分を
						全て書き換えるということは$\psi$と$\chi$の原子式の部分を全て書き換える
						ということになる.$\hat{\varphi}$は
						\begin{align}
							\vee \hat{\psi} \hat{\chi}
						\end{align}
						なる形の式であるが,$\hat{\psi},\hat{\chi}$はそれぞれ$\psi,\chi$の
						書き換えであり(もしくは,$\psi,\chi$の一方は元から
						$\lang{\varepsilon}$の式かもしれない),(IH)よりどちらも
						$\lang{\varepsilon}$の式である.ゆえに
						$\hat{\varphi}$も$\lang{\varepsilon}$の式である.
						
					\item[case3] $\varphi$が
						\begin{align}
							\exists x \psi
						\end{align}
						なる式のとき,case1 と同様の理由で$\varphi$の原子式の部分を
						全て書き換えるということは$\psi$の原子式の部分を全て書き換える
						ということになる.$\hat{\varphi}$は
						\begin{align}
							\exists x \hat{\psi}
						\end{align}
						なる形の式であるが,$\hat{\psi}$は$\psi$の書き換えであり,
						(IH)より$\lang{\varepsilon}$の式である.ゆえに
						$\hat{\varphi}$も$\lang{\varepsilon}$の式である.
						\QED
				\end{description}
		\end{description}
	\end{metaprf}
	
	\begin{screen}
		\begin{metathm}[部分式の書き換えとの関係]
		\label{metathm:relation_to_subformula_rewriting}
			$\varphi$を$\lang{\varepsilon}$の式ではない$\mathcal{L}$の式とするとき,
			\begin{description}
				\item[case1] $\varphi$が$\negation \psi$なる式のとき,
					$\varphi$の書き換え$\hat{\varphi}$は$\negation \hat{\psi}$
					なる形の式であるが,このとき$\hat{\psi}$は$\psi$の書き換えである.
					逆に$\check{\psi}$を$\psi$の書き換えとすれば$\negation \check{\psi}$
					は$\varphi$の書き換えである.
					
				\item[case2] $\varphi$が$\vee \psi \chi$なる式のとき,
					$\varphi$の書き換え$\hat{\varphi}$は$\vee \hat{\psi} \hat{\chi}$
					なる形の式であるが,このとき$\hat{\psi},\hat{\chi}$はそれぞれ$\psi,\chi$の
					書き換えである.逆に$\check{\psi},\check{\chi}$をそれぞれ$\psi,\chi$の
					書き換えとすれば$\vee \check{\psi} \check{\chi}$は$\varphi$の
					書き換えである.なお,$\psi,\chi$の一方は元から
					$\lang{\varepsilon}$の式かもしれないが,たとえば$\psi$がそうなら
					$\hat{\psi}$も$\check{\psi}$も$\psi$であるとする.
					
				\item[case3] $\varphi$が$\exists x \psi$なる式のとき,
					$\varphi$の書き換え$\hat{\varphi}$は$\exists x \hat{\psi}$
					なる形の式であるが,このとき$\hat{\psi}$は$\psi$の書き換えである.
					逆に$\check{\psi}$を$\psi$の書き換えとすれば$\exists x \check{\psi}$
					は$\varphi$の書き換えである.
			\end{description}
		\end{metathm}
	\end{screen}
	
	\begin{metaprf}
		証明は前定理の説明と大方被ってしまうがもう一度載せて置く.
		\begin{description}
			\item[case1] $\varphi$の$\in,=$のスコープはいずれも
				$\psi$の部分原子式であり,逆に$\psi$の$\in,=$のどのスコープも
				$\varphi$の部分原子式であるから,$\varphi$の原子式の部分を
				全て書き換えることと$\psi$の原子式の部分を全て書き換えることは同じである.
				従って$\hat{\psi}$は$\psi$の書き換えであり,$\negation \check{\psi}$は
				$\varphi$の書き換えである.
			
			\item[case2] $\varphi$の$\in,=$のスコープはいずれも
				$\psi$か$\chi$の一方の部分原子式であり(始切片の一意性
				のメタ定理\ref{metathm:initial_segment_L}より
				$\varphi$の真部分式が$\psi$と$\chi$の境を跨ぐことはない),
				逆に$\psi,\chi$の$\in,=$のどのスコープも
				$\varphi$の部分原子式であるから,$\varphi$の原子式の部分を
				全て書き換えることと$\psi$と$\chi$の原子式の部分を全て書き換えることは同じである.
				従って$\hat{\psi},\hat{\chi}$はそれぞれ$\psi,\chi$の書き換えであり,
				$\negation \check{\psi} \check{\chi}$は$\varphi$の書き換えである.
				
			\item[case3] case1 と同じ理由によって,$\hat{\psi}$は$\psi$の書き換えであり,
				$\exists x \check{\psi}$は$\varphi$の書き換えである.
				\QED
		\end{description}
	\end{metaprf}
	
	\begin{screen}
		\begin{metathm}[書き換え後も自由な変項は増減しない]
		\label{metathm:variables_unchanged_after_rewriting}
			$\varphi$を$\lang{\varepsilon}$の式ではない$\mathcal{L}$の式とし,
			この書き換えを$\hat{\varphi}$とする.このとき
			$\varphi$に自由に現れる変項は$\hat{\varphi}$にも自由に現れ,
			逆に$\hat{\varphi}$に自由に現れる変項は$\varphi$にも自由に現れる.
			特に,$\varphi$が文ならば$\hat{\varphi}$も文である.
		\end{metathm}
	\end{screen}
	
	\begin{metaprf}\mbox{}
		\begin{description}
			\item[step1] $\varphi$が原子式であるときは上の書き換え表より一目瞭然である.
			
			\item[step2]
				$\varphi$が一般の式であるとき
				\begin{itembox}[l]{IH (帰納法の仮定)}
					$\varphi$の任意の真部分式$\psi$に対し,その書き換えを
					$\hat{\psi}$とすれば($\psi$が$\lang{\varepsilon}$の式ならば
					$\hat{\psi}$は$\psi$とする),
					$\psi$に自由に現れる変項は$\hat{\psi}$にも自由に現れ,
					逆に$\hat{\psi}$に自由に現れる変項は$\psi$にも自由に現れる.
				\end{itembox}
				と仮定する.すると
				\begin{description}
					\item[case1] $\varphi$が
						\begin{align}
							\negation \psi
						\end{align}
						なる式の場合,$\hat{\varphi}$は
						\begin{align}
							\negation \hat{\psi}
						\end{align}
						なる形の式であるが,
						メタ定理\ref{metathm:relation_to_subformula_rewriting}より
						$\hat{\psi}$は$\psi$の書き換えである.
						$\varphi$に自由に現れる変項は$\psi$に自由に現れる変項と一致するが,
						(IH)よりそれは$\hat{\psi}$に自由に現れる変項と一致するので,
						$\hat{\varphi}$に自由に現れる変項とも一致する.
						
					\item[case2] $\varphi$が
						\begin{align}
							\vee \psi \chi
						\end{align}
						なる式の場合,$\hat{\varphi}$は
						\begin{align}
							\vee \hat{\psi} \hat{\chi}
						\end{align}
						なる形の式であるが,
						メタ定理\ref{metathm:relation_to_subformula_rewriting}より
						$\hat{\psi},\hat{\chi}$はそれぞれ$\psi,\chi$の書き換えである
						($\psi,\chi$の一方は元から$\lang{\varepsilon}$の式かもしれないが,
						たとえば$\psi$がそうなら$\hat{\psi}$も$\check{\psi}$も$\psi$
						であるとする).
						$\varphi$に自由に現れる変項は$\psi,\chi$に自由に現れる変項と
						一致するが,(IH)よりそれは$\hat{\psi},\hat{\chi}$に自由に現れる
						変項と一致するので,$\hat{\varphi}$に自由に現れる変項とも一致する.
					
					\item[case3] $\varphi$が
						\begin{align}
							\exists x \psi
						\end{align}
						なる式の場合,$\hat{\varphi}$は
						\begin{align}
							\exists x \hat{\psi}
						\end{align}
						なる形の式であるが,
						メタ定理\ref{metathm:relation_to_subformula_rewriting}より
						$\hat{\psi}$は$\psi$の書き換えである.
						$\varphi$に自由に現れる変項は$\psi$に自由に現れる$x$以外の
						変項と一致するが,(IH)よりそれは$\hat{\psi}$に自由に現れる$x$以外の
						変項と一致するので,$\hat{\varphi}$に自由に現れる変項とも一致する.
						\QED
				\end{description}
		\end{description}
	\end{metaprf}
		\begin{screen}
		\begin{metathm}[部分式の差し替えと代入]
		\label{metathm:subformula_replacing_and_substitution}
			$\varphi$を$\lang{\varepsilon}$の式とし,$x$を$\varphi$に自由に現れる変項とする.
			また$\varphi$に$\forall z \xi$ (resp. $\exists z \xi$)の
			形の部分式が現れているとし,$y$を$\xi$に自由に現れない変項で$\xi$の中で$z$への
			代入について自由であるものとし,$\varphi$の$\forall z \xi$ 
			(resp. $\exists z \xi$)の部分を一か所だけ$\forall y \xi(z/y)$ 
			(resp. $\exists y \xi(z/y)$)に差し替えた式を$\widetilde{\varphi}$とする.
			それから$\tau$を主要$\varepsilon$項とする.このとき,
			\begin{description}
				\item[(1)] $\varphi$における$x$の自由な出現が,差し替えられる$\forall z \xi$ 
					(resp. $\exists z \xi$)の中にある場合\footnotemark,
					$\widetilde{\varphi}(x/\tau)$は$\varphi(x/\tau)$の部分式$\forall z \xi(x/\tau)$ 
					(resp. $\exists z \xi(x/\tau)$)を$\forall y \xi(x/\tau)(z/y)$ 
					(resp. $\exists y \xi(x/\tau)(z/y)$)に差し替えた式である.
					
				\item[(2)] $\varphi$における$x$の自由な出現が,差し替えられる$\forall z \xi$ 
					(resp. $\exists z \xi$)の中に無い場合,$\widetilde{\varphi}(x/\tau)$
					は$\varphi(x/\tau)$の部分式$\forall z \xi$ 
					(resp. $\exists z \xi$)を$\forall y \xi(z/y)$ 
					(resp. $\exists y \xi(z/y)$)に差し替えた式である.
			\end{description}
		\end{metathm}
	\end{screen}
	
	\footnotetext{
		$\varphi$における$x$の自由な出現が差し替えられる$\forall z \xi$ (resp. $\exists z \xi$)の中にある場合,
		$\xi$は$\forall x$或いは$\exists x$から始まる$\varphi$の部分式の内部には無い.
		従って$\xi$に自由に現れる$x$は全て$\varphi$にも自由に現れる.
	}
	
	\begin{metaprf} 差し替えられる$\varphi$の部分式が$\forall z \xi$だとして示すが,
		$\exists z \xi$に替えても同じである.
		\begin{description}
			\item[step1] $\varphi$が$\forall z \xi$なる式である場合,
				$x$は$\varphi$に自由に現れているので$x$は$z$ではない.
				$\widetilde{\varphi}$とは$\forall y \xi(z/y)$なる式であるが,
				$y$の選び方より$x$は$y$でもない.
				すなわち$x$は$\widetilde{\varphi}$にも自由に現れている.
				%また代入条件より$\tau$もまた$z$でも$y$でもない.
				$\widetilde{\varphi}(x/\tau)$とは
				\begin{align}
					\forall y \xi(z/y)(x/\tau)
				\end{align}
				なる式であるが,いま$\xi(z/y)(x/\tau)$と$\xi(x/\tau)(z/y)$は同じ式なので,
				$\widetilde{\varphi}(x/\tau)$は
				\begin{align}
					\forall y \xi(x/\tau)(z/y)
				\end{align}
				と同じ式である.いまの場合$\varphi(x/\tau)$は$\forall z \xi(x/\tau)$であるから
				(1)の主張が成り立つ.
				
			\item[step2]\mbox{}
				\begin{itembox}[l]{IH (帰納法の仮定)}
					$\varphi$の任意の真部分式$\psi$に対して,
					差し替えられる$\forall z \xi$が$\psi$に部分式として現れているとき,
					$\psi$の$\forall z \xi$を$\forall y \xi(z/y)$ 
					に差し替えた式を$\widetilde{\psi}$とする.このとき
					\begin{description}
						\item[(1)] $\varphi$における$x$の自由な出現が,
							差し替えられる$\forall z \xi$の中にある場合,
							$\widetilde{\psi}(x/\tau)$は$\psi(x/\tau)$の部分式
							$\forall z \xi(x/\tau)$ 
							を$\forall y \xi(x/\tau)(z/y)$に差し替えた式である.
						
						\item[(2)] $\varphi$における$x$の自由な出現が,
							差し替えられる$\forall z \xi$の中に無い場合,
							$\widetilde{\psi}(x/\tau)$は$\psi(x/\tau)$の部分式
							$\forall z \xi$ を$\forall y \xi(z/y)$に差し替えた式である.
					\end{description}
				\end{itembox}
				と仮定する.このとき,
				\begin{description}
					\item[case1] $\varphi$が
						\begin{align}
							\negation \psi
						\end{align}
						なる式である場合,差し替えられる$\forall z \xi$は$\psi$に現れる.
						$\varphi(x/\tau)$は
						\begin{align}
							\negation \psi(x/\tau)
						\end{align}
						なる式であり,$\widetilde{\varphi}$は
						\begin{align}
							\negation \widetilde{\psi}
						\end{align}
						なる式であり,$\widetilde{\varphi}(x/\tau)$は
						\begin{align}
							\negation \widetilde{\psi}(x/\tau)
						\end{align}
						なる式である.
						\begin{description}
							\item[(1)] $\varphi$における$x$の自由な出現が
								差し替えられる$\forall z \xi$の中にある場合,
								$\varphi(x/\tau)$の部分式
								$\forall z \xi(x/\tau)$を
								$\forall y \xi(x/\tau)(z/y)$に差し替えた式は,
								(IH)より
								\begin{align}
									\negation \widetilde{\psi}(x/\tau)
								\end{align}
								と同じ式である.従ってその式は
								$\widetilde{\varphi}(x/\tau)$とも同じ式である.
								
							\item[(2)] $\varphi$における$x$の自由な出現が
								差し替えられる$\forall z \xi$の中に無い場合,
								$\varphi(x/\tau)$の部分式$\forall z \xi$を
								$\forall y \xi(z/y)$に差し替えた式は,(IH)より
								\begin{align}
									\negation \widetilde{\psi}(x/\tau)
								\end{align}
								と同じ式である.従ってその式は
								$\widetilde{\varphi}(x/\tau)$とも同じ式である.
						\end{description}
						
					\item[case2] $\varphi$が
						\begin{align}
							\vee \psi \chi
						\end{align}
						なる式である場合,差し替えられる$\forall z \xi$は$\psi$か$\chi$
						のどちらか一方に現れるが,ここでは
						$\psi$の側に現れているとする.$\varphi(x/\tau)$は
						\begin{align}
							\vee \psi(x/\tau) \chi(x/\tau)
						\end{align}
						なる式であり,$\widetilde{\varphi}$は
						\begin{align}
							\vee \widetilde{\psi} \chi
						\end{align}
						なる式であり,$\widetilde{\varphi}(x/\tau)$は
						\begin{align}
							\vee \widetilde{\psi}(x/\tau) \chi(x/\tau)
						\end{align}
						なる式である.
						\begin{description}
							\item[(1)] $\varphi$における$x$の自由な出現が
								差し替えられる$\forall z \xi$の中にある場合,
								$\varphi(x/\tau)$の部分式
								$\forall z \xi(x/\tau)$を
								$\forall y \xi(x/\tau)(z/y)$に差し替えた式は,
								(IH)より
								\begin{align}
									\vee \widetilde{\psi}(x/\tau) \chi(x/\tau)
								\end{align}
								と同じ式である.従ってその式は
								$\widetilde{\varphi}(x/\tau)$とも同じ式である.
								
							\item[(2)] $\varphi$における$x$の自由な出現が
								差し替えられる$\forall z \xi$の中に無い場合,
								$\varphi(x/\tau)$の部分式$\forall z \xi$を
								$\forall y \xi(z/y)$に差し替えた式は,(IH)より
								\begin{align}
									\vee \widetilde{\psi}(x/\tau) \chi(x/\tau)
								\end{align}
								と同じ式である.従ってその式は
								$\widetilde{\varphi}(x/\tau)$とも同じ式である.
						\end{description}
						
					\item[case3] $\varphi$が
						\begin{align}
							\exists w \psi
						\end{align}
						なる式である場合,差し替えられる$\forall z \xi$は$\psi$に現れる.
						$\varphi(x/\tau)$は
						\begin{align}
							\exists w \psi(x/\tau)
						\end{align}
						なる式であり,$\widetilde{\varphi}$は
						\begin{align}
							\exists w \widetilde{\psi}
						\end{align}
						なる式であり,$\widetilde{\varphi}(x/\tau)$は
						\begin{align}
							\exists w \widetilde{\psi}(x/\tau)
						\end{align}
						なる式である.
						\begin{description}
							\item[(1)] $\varphi$における$x$の自由な出現が
								差し替えられる$\forall z \xi$の中にある場合,
								$\varphi(x/\tau)$の部分式
								$\forall z \xi(x/\tau)$を
								$\forall y \xi(x/\tau)(z/y)$に差し替えた式は,
								(IH)より
								\begin{align}
									\exists w \widetilde{\psi}(x/\tau)
								\end{align}
								と同じ式である.従ってその式は
								$\widetilde{\varphi}(x/\tau)$とも同じ式である.
								
							\item[(2)] $\varphi$における$x$の自由な出現が
								差し替えられる$\forall z \xi$の中に無い場合,
								$\varphi(x/\tau)$の部分式$\forall z \xi$を
								$\forall y \xi(z/y)$に差し替えた式は,(IH)より
								\begin{align}
									\exists w \widetilde{\psi}(x/\tau)
								\end{align}
								と同じ式である.従ってその式は
								$\widetilde{\varphi}(x/\tau)$とも同じ式である.
								\QED
						\end{description}
				\end{description}
		\end{description}
	\end{metaprf}
	
	%\footnotetext{
	%	「任意の真部分式に対し,それが直部分式ならば...」と書き改めれば構造的帰納法の原理を適用できる
	%	が,始めから直部分式と限定しても実質的には変わらない.
	%}
	
	\begin{screen}
		\begin{metathm}[書き換えへの代入は代入した式の書き換え]
		\label{metathm:substitution_to_rewritten_formula}
			$\varphi$を$\lang{\varepsilon}$の式ではない$\mathcal{L}$の式とし,
			$\varphi$には変項$x$が自由に現れているとし,$\tau$を主要$\varepsilon$項とし,
			$\widehat{\varphi}$を$\varphi$の書き換えとする\footnotemark
			.このとき$\widehat{\varphi}(x/\tau)$は$\varphi(x/\tau)$の書き換えである.
		\end{metathm}
	\end{screen}
	
	\footnotetext{
		定理\ref{metathm:variables_unchanged_after_rewriting}より
		$\widehat{\varphi}$にも$x$は自由に現れている.
	}
	
	証明が長いので第一証明と第二証明に分割する.第一証明では$\widehat{\varphi}$が$\varphi$の
	部分式で原子式であるものを全て表\ref{tab:formula_rewriting}の通りに直した式である場合を扱い,
	第二証明では「式の書き換えによる構造的帰納法」のセカンドステップを扱う.
	
	\begin{metaprf}[第一] $\widehat{\varphi}$が$\varphi$の部分式で原子式であるものを全て
		表\ref{tab:formula_rewriting}の通りに直した式であるとき,$\widehat{\varphi}(x/\tau)$
		が$\varphi(x/\tau)$の書き換えであることを示す.
		\begin{description}
			\item[step1] $\varphi$が原子式であるとする.
				\begin{description}
					\item[case1] $\varphi$が
						\begin{align}
							x = \Set{z}{\psi}
						\end{align}
						なる式のとき,$\widehat{\varphi}$は
						\begin{align}
							\forall v\, (\, v \in x \lrarrow \psi(z/v)\, )
						\end{align}
						なる式である.
						\begin{itemize}
							\item $x$と$z$が同じであるとする.
								このとき$\widehat{\varphi}(x/\tau)$は
								\begin{align}
									\forall v\, (\, v \in \tau \lrarrow \psi(z/v)\, )
								\end{align}
								となる.他方で$\varphi(x/\tau)$は
								\begin{align}
									\tau = \Set{z}{\psi}
								\end{align}
								であるから$\widehat{\varphi}(x/\tau)$は
								$\varphi(x/\tau)$の書き換えである.
								
							\item $x$と$z$が違うとする.このとき
								\begin{itemize}
									\item $x$が$\Set{z}{\psi}$に自由に現れている場合,
										$\widehat{\varphi}(x/\tau)$は
										\begin{align}
											\forall v\, (\, v \in \tau \lrarrow \psi(z/v)(x/\tau)\, )
										\end{align}
										となるが,書き換えの変項条件より$x$は$v$とも違うので,
										%代入条件より$\tau$もまた$z$とも$v$とも違うので,
										$\psi(z/v)(x/\tau)$と$\psi(x/\tau)(z/v)$は
										同じ式である.従って$\widehat{\varphi}(x/\tau)$は
										\begin{align}
											\forall v\, (\, v \in \tau \lrarrow \psi(x/\tau)(z/v)\, )
										\end{align}
										と同じ式である.他方で$\varphi(x/\tau)$は
										\begin{align}
											\tau = \Set{z}{\psi(x/\tau)}
										\end{align}
										であるから,この場合は
										$\widehat{\varphi}(x/\tau)$は
										$\varphi(x/\tau)$の書き換えである.
										
									\item $x$が$\Set{z}{\psi}$に自由に現れていない場合,
										$\widehat{\varphi}(x/\tau)$は
										\begin{align}
											\forall v\, (\, v \in \tau \lrarrow \psi(z/v)\, )
										\end{align}
										となるが,$\varphi(x/\tau)$は
										\begin{align}
											\tau = \Set{z}{\psi}
										\end{align}
										であるからこの場合も
										$\widehat{\varphi}(x/\tau)$は
										$\varphi(x/\tau)$の書き換えである.
								\end{itemize}
						\end{itemize}
						
					\item[case2] $\varphi$が
						\begin{align}
							a = \Set{z}{\psi}
						\end{align}
						なる式のとき($a$と$x$は違う$\lang{\varepsilon}$の項),
						$\widehat{\varphi}$は
						\begin{align}
							\forall v\, (\, v \in a \lrarrow \psi(z/v)\, )
						\end{align}
						なる式である.$\varphi$には$x$が自由に現れているので,つまり
						$x$は$z$ではなく,また$\Set{z}{\psi}$に自由に現れている.従って
						$\widehat{\varphi}(x/\tau)$は
						\begin{align}
							\forall v\, (\, v \in a \lrarrow \psi(z/v)(x/\tau)\, )
						\end{align}
						となるが,書き換えの変項条件より$x$は$v$とも違うので,
						%代入条件より$\tau$もまた$z$とも$v$とも違うので,
						$\psi(z/v)(x/\tau)$と$\psi(x/\tau)(z/v)$は
						同じ式である.従って$\widehat{\varphi}(x/\tau)$は
						\begin{align}
							\forall v\, (\, v \in a \lrarrow \psi(x/\tau)(z/v)\, )
						\end{align}
						と同じ式である.他方で$\varphi(x/\tau)$は
						\begin{align}
							a = \Set{z}{\psi(x/\tau)}
						\end{align}
						であるから$\widehat{\varphi}(x/\tau)$は
						$\varphi(x/\tau)$の書き換えである.
					
					\item[case3] $\varphi$が
						\begin{align}
							\Set{y}{\xi} = x
						\end{align}
						なる式のとき,$\widehat{\varphi}$は
						\begin{align}
							\forall u\, (\, \xi(y/u) \lrarrow u \in x\, )
						\end{align}
						なる式である.
						\begin{itemize}
							\item $x$と$y$が同じであるとする.このとき
								$\widehat{\varphi}(x/\tau)$は
								\begin{align}
									\forall u\, (\, \xi(y/u) \lrarrow u \in \tau\, )
								\end{align}
								となる.他方で$\varphi(x/\tau)$は
								\begin{align}
									\Set{y}{\xi} = \tau
								\end{align}
								であるから$\widehat{\varphi}(x/\tau)$は
								$\varphi(x/\tau)$の書き換えである.
								
							\item $x$と$y$が違うとする.このとき
								\begin{itemize}
									\item $x$が$\Set{y}{\xi}$に自由に現れていれば,
										$\widehat{\varphi}(x/\tau)$は
										\begin{align}
											\forall u\, (\, \xi(y/u)(x/\tau) \lrarrow u \in \tau\, )
										\end{align}
										となるが,書き換えの変項条件より$x$は$u$とも違うので,
										%代入条件より$\tau$もまた$y$とも$u$とも違うので,
										$\xi(y/u)(x/\tau)$と$\xi(x/\tau)(y/u)$は
										同じ式である.従って$\widehat{\varphi}(x/\tau)$は
										\begin{align}
											\forall u\, (\, \xi(x/\tau)(y/u) \lrarrow u \in \tau\, )
										\end{align}
										と同じ式である.他方で$\varphi(x/\tau)$は
										\begin{align}
											\Set{y}{\xi(x/\tau)} = \tau
										\end{align}
										であるから,この場合は
										$\widehat{\varphi}(x/\tau)$は
										$\varphi(x/\tau)$の書き換えである.
								
									\item $x$が$\Set{y}{\xi}$に自由に現れていない場合,
										$\widehat{\varphi}(x/\tau)$は
										\begin{align}
											\forall u\, (\, \xi(y/u) \lrarrow u \in \tau\, )
										\end{align}
										となるが,$\varphi(x/\tau)$は
										\begin{align}
											\Set{y}{\xi} = \tau
										\end{align}
										であるからこの場合も$\widehat{\varphi}(x/\tau)$は
										$\varphi(x/\tau)$の書き換えである.
								\end{itemize}
						\end{itemize}
						
					\item[case4] $\varphi$が
						\begin{align}
							\Set{y}{\xi} = b
						\end{align}
						なる式のとき($b$は$x$と違う$\lang{\varepsilon}$の項),
						$\widehat{\varphi}$は
						\begin{align}
							\forall u\, (\, \xi(y/u) \lrarrow u \in b\, )
						\end{align}
						なる式である.$\varphi$には$x$が自由に現れているので,つまり
						$x$は$y$ではなく,また$\Set{y}{\xi}$に自由に現れている.
						従って$\widehat{\varphi}(x/\tau)$は
						\begin{align}
							\forall u\, (\, \xi(y/u)(x/\tau) \lrarrow u \in b\, )
						\end{align}
						となるが,書き換えの変項条件より$x$は$u$とも違うので,
						%代入条件より$\tau$もまた$y$とも$u$とも違うので,
						$\xi(y/u)(x/\tau)$と$\xi(x/\tau)(y/u)$は
						同じ式である.従って$\widehat{\varphi}(x/\tau)$は
						\begin{align}
							\forall u\, (\, \xi(x/\tau)(y/u) \lrarrow u \in b\, )
						\end{align}
						と同じ式である.他方で$\varphi(x/\tau)$は
						\begin{align}
							\Set{y}{\xi(x/\tau)} = b
						\end{align}
						であるから$\widehat{\varphi}(x/\tau)$は
						$\varphi(x/\tau)$の書き換えである.
					
					\item[case5] $\varphi$が
						\begin{align}
							\Set{y}{\xi} = \Set{z}{\psi}
						\end{align}
						なる式のとき,$\widehat{\varphi}$は
						\begin{align}
							\forall u\, (\, \xi(y/u) \lrarrow \psi(z/u)\, )
						\end{align}
						なる式である.
						\begin{itemize}
							\item $x$と$y$が同じであるとする.このとき
								$x$は$\Set{y}{\xi}$には自由に
								現れないので,$x$が$\varphi$に自由に現れている以上
								$\Set{z}{\psi}$に自由に現れることになる.
								すなわち$x$と$z$は違う項である.
								このとき$\widehat{\varphi}(x/\tau)$は
								\begin{align}
									\forall u\, (\, \xi(y/u) \lrarrow \psi(z/u)(x/\tau)\, )
								\end{align}
								となるが,書き換えの変項条件より$x$は$u$とも違うので,
								%代入条件より$\tau$もまた$z$とも$u$とも違うので,
								$\psi(z/u)(x/\tau)$と$\psi(x/\tau)(z/u)$は
								同じ式である.従って$\widehat{\varphi}(x/\tau)$は
								\begin{align}
									\forall u\, (\, \xi(y/u) \lrarrow \psi(x/\tau)(z/u)\, )
								\end{align}
								と同じ式である.他方で$\varphi(x/\tau)$は
								\begin{align}
									\Set{y}{\xi} = \Set{z}{\psi(x/\tau)}
								\end{align}
								であるから$\widehat{\varphi}(x/\tau)$は
								$\varphi(x/\tau)$の書き換えである.
								
							\item $x$と$y$と違い,$x$と$z$が同じであるとする.
								$x$が$\varphi$に自由に現れている以上
								$x$は$\Set{y}{\xi}$に自由に現れることになるから,
								$\widehat{\varphi}(x/\tau)$は
								\begin{align}
									\forall u\, (\, \xi(y/u)(x/\tau) \lrarrow \psi(z/u)\, )
								\end{align}
								となるが,書き換えの変項条件より$x$は$u$とも違うので,
								%代入条件より$\tau$もまた$y$とも$u$とも違うので,
								$\xi(y/u)(x/\tau)$と$\xi(x/\tau)(y/u)$は
								同じ式である.従って$\widehat{\varphi}(x/\tau)$は
								\begin{align}
									\forall u\, (\, \xi(x/\tau)(y/u) \lrarrow \psi(z/u)\, )
								\end{align}
								と同じ式である.他方で$\varphi(x/\tau)$は
								\begin{align}
									\Set{y}{\xi(x/\tau)} = \Set{z}{\psi}
								\end{align}
								であるから$\widehat{\varphi}(x/\tau)$は
								$\varphi(x/\tau)$の書き換えである.
							
							\item $x$が$y$とも$z$とも違うとする,このとき
								$x$は$\Set{y}{\xi}$か$\Set{z}{\psi}$の少なくとも
								一方には自由に現れている.
								このとき$\widehat{\varphi}(x/\tau)$は
								\begin{align}
									\forall u\, (\, \xi(y/u)(x/\tau) \lrarrow \psi(z/u)(x/\tau)\, )
								\end{align}
								となるが,書き換えの変項条件より$x$は$u$とも違うので,
								$\widehat{\varphi}(x/\tau)$は
								\begin{align}
									\forall u\, (\, \xi(x/\tau)(y/u) \lrarrow \psi(x/\tau)(z/u)\, )
								\end{align}
								と同じ式である.他方で$\varphi(x/\tau)$は
								\begin{align}
									\Set{y}{\xi(x/\tau)} = \Set{z}{\psi(x/\tau)}
								\end{align}
								であるから$\widehat{\varphi}(x/\tau)$は
								$\varphi(x/\tau)$の書き換えである.
						\end{itemize}
						
					\item[case6] $\varphi$が
						\begin{align}
							x \in \Set{z}{\psi}
						\end{align}
						なる式のとき,$\widehat{\varphi}$を得るために必要ならば$\psi$の変項の
						名前替えをしたものを$\widetilde{\psi}$とする.ただし
						名前替えをしなかったら$\widetilde{\psi}$は$\psi$とする.
						$\widehat{\varphi}$は$\widetilde{\psi}(z/x)$なる式であり,
						$\widehat{\varphi}(x/\tau)$は$\widetilde{\psi}(z/x)(x/\tau)$
						となる.
						\begin{itemize}
							\item $x$と$z$が同じであるとする.
								このときは$\psi$の変項の名前替えは必要ない.
								$\widehat{\varphi}$とは$\psi$そのものであり,
								$\widehat{\varphi}(x/\tau)$は$\psi(z/\tau)$と同じ式である.
								他方で$\varphi(x/\tau)$は
								\begin{align}
									\tau \in \Set{z}{\psi}
								\end{align}
								となるから,$\psi(z/\tau)$は$\varphi(x/\tau)$の書き換えである.
								従って$\widehat{\varphi}(x/\tau)$は$\varphi(x/\tau)$の書き換えである.
								
							\item $x$と$z$が違うとする.このとき
								\begin{itemize}
									\item $x$が$\Set{z}{\psi}$に自由に現れている場合.
										$\widetilde{\psi}(z/x)(x/\tau)$は
										$\widetilde{\psi}(x/\tau)(z/\tau)$と同じ式である.
										他方で$\varphi(x/\tau)$は
										\begin{align}
											\tau \in \Set{z}{\psi(x/\tau)}
										\end{align}
										となるから,$\varphi(x/\tau)$の書き換えは$\psi(x/\tau)(z/\tau)$となる.
										
										ここで$\widetilde{\psi}(x/\tau)(z/\tau)$が
										$\psi(x/\tau)(z/\tau)$の量化部分式を
										(ゼロ回乃至数回だけ)差し替えた式であることを示す.
										ゼロ回というのは$\widetilde{\psi}$が$\psi$であるということだから,
										既に$\widehat{\varphi}(x/\tau)$が
										$\varphi(x/\tau)$の書き換えであると判ってしまう.
										以下では$\widetilde{\psi}$は$\psi$ではないと仮定して話を進める.
										そもそも$\widetilde{\psi}$とはどのように出来ていたかというと,
										表\ref{tab:formula_rewriting}の下の変項条件で書いたように,
										$\psi$の量化部分式を一回乃至数回差し替えているのである.
										それが$n$回あったとして,$\psi$から$\widetilde{\psi}$に至るまでの
										差し替えの履歴を
										\begin{align}
											\psi_{1},\ \psi_{2},\ \cdots,\ \psi_{n}
										\end{align}
										としよう.$\psi_{n}$とは$\widetilde{\psi}$のことである.
										メタ定理\ref{metathm:subformula_replacing_and_substitution}より,
										$\psi_{1}(x/\tau)$とは$\psi(x/\tau)$の量化部分式を一つ差し替えた式である.
										すると再びメタ定理\ref{metathm:subformula_replacing_and_substitution}より
										$\psi_{1}(x/\tau)(z/\tau)$は$\psi(x/\tau)(z/\tau)$
										の量化部分式を一つ差し替えた式となる.
										同様に$\psi_{i+1}(x/\tau)(z/\tau)$は$\psi_{i}(x/\tau)(z/\tau)$の
										量化部分式を一つ差し替えた式であるから,$\widetilde{\psi}(x/\tau)(z/\tau)$
										は$\psi(x/\tau)(z/\tau)$の量化部分式を$n$回差し替えた式なのである.
										$\psi(x/\tau)(z/\tau)$とは$\varphi(x/\tau)$の書き換えなのだから,
										書き換えの定義によって$\widetilde{\psi}(x/\tau)(z/\tau)$は
										$\varphi(x/\tau)$の書き換えである.つまり$\widehat{\varphi}(x/\tau)$は
										$\varphi(x/\tau)$の書き換えである.
										
										
									\item $x$が$\Set{z}{\psi}$に自由に現れていない場合.
										$\psi$に$x$は自由に現れないので
										$\widetilde{\psi}$にも
										$x$は自由に現れない($\widetilde{\psi}$は
										$\psi$に自由に現れる変項に影響しないように
										部分式を差し替えて作られているため).
										従って$\widetilde{\psi}(z/x)(x/\tau)$は
										$\widetilde{\psi}(z/\tau)$と同じ式である.
										他方で$\varphi(x/\tau)$は
										\begin{align}
											\tau \in \Set{z}{\psi}
										\end{align}
										となるから,$\psi(z/\tau)$は$\varphi(x/\tau)$の書き換えとなる.
										先と同じ論法で$\widetilde{\psi}(z/\tau)$が
										$\psi(z/\tau)$の量化部分式をゼロ回乃至数回だけ差し替えた
										式であると判るので,書き換えの定義より
										$\widehat{\varphi}(x/\tau)$は$\varphi(x/\tau)$の書き換えである.
								\end{itemize}
						\end{itemize}
						
					\item[case7] $\varphi$が
						\begin{align}
							a \in \Set{z}{\psi}
						\end{align}
						なる式のとき($a$は$x$とは違う$\lang{\varepsilon}$の項),
						$x$は$\varphi$に自由に現れているので,$x$は$z$とは違う変項であり,
						$\psi$に自由に現れている.$\widehat{\varphi}$を得るために
						必要ならば$\psi$の変項の名前替えをして$\widetilde{\psi}$を作る.ただし
						名前替えをしなかったら$\widetilde{\psi}$は$\psi$とする.
						$\widehat{\varphi}$は$\widetilde{\psi}(z/a)$なる式であり,
						$\widehat{\varphi}(x/\tau)$は$\widetilde{\psi}(z/a)(x/\tau)$
						となるが,$x$は$z$とも$a$とも違うので$\widehat{\varphi}(x/\tau)$は
						\begin{align}
							\widetilde{\psi}(x/\tau)(z/a)
						\end{align}
						と一致する.他方で$\varphi(x/\tau)$は
						\begin{align}
							a \in \Set{z}{\psi(x/\tau)}
						\end{align}
						となる.ところで,$\widetilde{\psi}$は$\psi$の量化部分式を
						或る$n$回だけ差し替えた式であるから,
						メタ定理\ref{metathm:subformula_replacing_and_substitution}
						とcase6の説明より,$\widetilde{\psi}(x/\tau)$もまた
						$\psi(x/\tau)$の量化部分式を$n$回だけ差し替えた式である.
						すなわち$\widetilde{\psi}(x/\tau)(z/a)$は
						$\varphi(x/\tau)$の書き換えとなっている.ゆえに
						$\widehat{\varphi}(x/\tau)$は$\varphi(x/\tau)$の書き換えである.
						
					\item[case8] $\varphi$が
						\begin{align}
							\Set{y}{\xi} \in x
						\end{align}
						なる式のとき,$\widehat{\varphi}$は
						\begin{align}
							\exists s\, (\, \forall u\, (\, \xi(y/u) \lrarrow u \in s\, ) \wedge s \in x\, )
						\end{align}
						なる式である.
						\begin{itemize}
							\item $x$と$y$が同じであるとする.このとき
								$\widehat{\varphi}(x/\tau)$は
								\begin{align}
									\exists s\, (\, \forall u\, (\, \xi(y/u) \lrarrow u \in s\, ) \wedge s \in \tau\, )
								\end{align}
								となる.他方で$\varphi(x/\tau)$は
								\begin{align}
									\Set{y}{\xi} \in \tau
								\end{align}
								であるから$\widehat{\varphi}(x/\tau)$は
								$\varphi(x/\tau)$の書き換えである.
								
							\item $x$と$y$が違うとする.このとき
								\begin{itemize}
									\item $x$が$\Set{y}{\xi}$に自由に現れているならば,
										$\widehat{\varphi}(x/\tau)$は
										\begin{align}
											\exists s\, (\, \forall u\, (\, \xi(y/u)(x/\tau) \lrarrow u \in s\, ) \wedge s \in \tau\, )
										\end{align}
										となるが,書き換えの変項条件より$x$は$u$とも違うので,
										%代入条件より$\tau$もまた$y$とも$u$とも違うので,
										$\xi(y/u)(x/\tau)$と$\xi(x/\tau)(y/u)$は
										同じ式である.従って$\widehat{\varphi}(x/\tau)$は
										\begin{align}
											\exists s\, (\, \forall u\, (\, \xi(x/\tau)(y/u) \lrarrow u \in s\, ) \wedge s \in \tau\, )
										\end{align}
										と同じ式である.他方で$\varphi(x/\tau)$は
										\begin{align}
											\Set{y}{\xi(x/\tau)} \in \tau
										\end{align}
										であるから,この場合は
										$\widehat{\varphi}(x/\tau)$は
										$\varphi(x/\tau)$の書き換えである.
										
									\item $x$が$\Set{y}{\xi}$に自由に現れていない場合,
										$\widehat{\varphi}(x/\tau)$は
										\begin{align}
											\exists s\, (\, \forall u\, (\, \xi(y/u) \lrarrow u \in s\, ) \wedge s \in \tau\, )
										\end{align}
										となり,$\varphi(x/\tau)$は
										\begin{align}
											\Set{y}{\xi} \in \tau
										\end{align}
										であるからこの場合も
										$\widehat{\varphi}(x/\tau)$は
										$\varphi(x/\tau)$の書き換えである.
								\end{itemize}
						\end{itemize}
					
					\item[case9] $\varphi$が
						\begin{align}
							\Set{y}{\xi} \in b
						\end{align}
						なる式のとき($b$は$x$と違う$\lang{\varepsilon}$の項),
						$\widehat{\varphi}$は
						\begin{align}
							\exists s\, (\, \forall u\, (\, \xi(y/u) \lrarrow u \in s\, ) \wedge s \in b\, )
						\end{align}
						なる式である.$\varphi$には$x$が自由に現れているので,つまり
						$x$は$y$ではなく,また$\xi$に自由に現れている.
						従って$\widehat{\varphi}(x/\tau)$は
						\begin{align}
							\exists s\, (\, \forall u\, (\, \xi(y/u)(x/\tau) \lrarrow u \in s\, ) \wedge s \in b\, )
						\end{align}
						となるが,書き換えの変項条件より$x$は$u$とも違うので,
						%代入条件より$\tau$もまた$y$とも$u$とも違うので,
						$\xi(y/u)(x/\tau)$と$\xi(x/\tau)(y/u)$は
						同じ式である.従って$\widehat{\varphi}(x/\tau)$は
						\begin{align}
							\exists s\, (\, \forall u\, (\, \xi(x/\tau)(y/u) \lrarrow u \in s\, ) \wedge s \in b\, )
						\end{align}
						と同じ式である.他方で$\varphi(x/\tau)$は
						\begin{align}
							\Set{y}{\xi(x/\tau)} \in b
						\end{align}
						であるから,$\widehat{\varphi}(x/\tau)$は
						$\varphi(x/\tau)$の書き換えである.
						
					\item[case10] $\varphi$が
						\begin{align}
							\Set{y}{\xi} \in \Set{z}{\psi}
						\end{align}
						なる式のとき,$\widehat{\varphi}$は
						\begin{align}
							\exists s\, (\, \forall u\, (\, \xi(y/u) \lrarrow u \in s\, ) \wedge \psi(z/s)\, )
						\end{align}
						なる式である.
						\begin{itemize}
							\item $x$と$y$が同じであるとする.このとき
								$x$は$\Set{y}{\xi}$には自由に
								現れないので,$x$が$\varphi$に自由に現れている以上
								$\Set{z}{\psi}$に自由に現れることになる.
								すなわち$x$と$z$は違う項である.
								このとき$\widehat{\varphi}(x/\tau)$は
								\begin{align}
									\exists s\, (\, \forall u\, (\, \xi(y/u) \lrarrow u \in s\, ) \wedge \psi(z/s)(x/\tau)\, )
								\end{align}
								となるが,書き換えの変項条件より$x$は$s$とも違うので,
								%代入条件より$\tau$もまた$z$とも$s$とも違うので,
								$\psi(z/s)(x/\tau)$と$\psi(x/\tau)(z/s)$は
								同じ式である.従って$\widehat{\varphi}(x/\tau)$は
								\begin{align}
									\exists s\, (\, \forall u\, (\, \xi(y/u) \lrarrow u \in s\, ) \wedge \psi(x/\tau)(z/s)\, )
								\end{align}
								と同じ式である.他方で$\varphi(x/\tau)$は
								\begin{align}
									\Set{y}{\xi} \in \Set{z}{\psi(x/\tau)}
								\end{align}
								であるから,$\widehat{\varphi}(x/\tau)$は
								$\varphi(x/\tau)$の書き換えである.
								
							\item $x$と$y$が違い,$x$と$z$が同じであるとする.
								$x$は$\Set{z}{\psi}$には自由に
								現れないので,$x$が$\varphi$に自由に現れている以上
								$\Set{y}{\xi}$に自由に現れることになる.
								このとき$\widehat{\varphi}(x/\tau)$は
								\begin{align}
									\exists s\, (\, \forall u\, (\, \xi(y/u)(x/\tau) \lrarrow u \in s\, ) \wedge \psi(z/s)\, )
								\end{align}
								となるが,書き換えの変項条件より$x$は$u$とも違うので,
								%代入条件より$\tau$もまた$y$とも$u$とも違うので,
								$\xi(y/u)(x/\tau)$と$\xi(x/\tau)(y/u)$は
								同じ式である.従って$\widehat{\varphi}(x/\tau)$は
								\begin{align}
									\exists s\, (\, \forall u\, (\, \xi(x/\tau)(y/u) \lrarrow u \in s\, ) \wedge \psi(z/s)\, )
								\end{align}
								と同じ式である.他方で$\varphi(x/\tau)$は
								\begin{align}
									\Set{y}{\xi(x/\tau)} \in \Set{z}{\psi}
								\end{align}
								であるから,$\widehat{\varphi}(x/\tau)$は
								$\varphi(x/\tau)$の書き換えである.
								
							\item $x$が$y$とも$z$とも違うとする.このとき$x$は
								$\Set{y}{\xi}$か$\Set{z}{\psi}$の
								少なくとも一方にはには自由に現れている.
								このとき$\widehat{\varphi}(x/\tau)$は
								\begin{align}
									\exists s\, (\, \forall u\, (\, \xi(y/u)(x/\tau) \lrarrow u \in s\, ) \wedge \psi(z/s)(x/\tau)\, )
								\end{align}
								となるが,書き換えの変項条件より$x$は$u$とも$s$とも違うので,
								$\widehat{\varphi}(x/\tau)$は
								\begin{align}
									\exists s\, (\, \forall u\, (\, \xi(x/\tau)(y/u) \lrarrow u \in s\, ) \wedge \psi(x/\tau)(z/s)\, )
								\end{align}
								と同じ式である.他方で$\varphi(x/\tau)$は
								\begin{align}
									\Set{y}{\xi(x/\tau)} \in \Set{z}{\psi(x/\tau)}
								\end{align}
								であるから,$\widehat{\varphi}(x/\tau)$は
								$\varphi(x/\tau)$の書き換えである.
						\end{itemize}
				\end{description}
			
			\item[step2] $\varphi$が一般の式であるとき,
				\begin{itembox}[l]{IH (帰納法の仮定)}
					$\varphi$の任意の真部分式$\psi$に対し,$\widehat{\psi}$が
					$\psi$の部分式で原子式であるものを全て
					表\ref{tab:formula_rewriting}の通りに直した式
					であるとすれば($\psi$が$\lang{\varepsilon}$の
					式ならば$\widehat{\psi}$は$\psi$とする),
					$\widehat{\psi}(x/\tau)$は$\psi(x/\tau)$の書き換えである.
				\end{itembox}
				と仮定する
				\footnote{
					メタ定理\ref{metathm:variables_unchanged_after_rewriting}より
					$\psi$に$x$が自由に現れていなければ$\widehat{\psi}$にも
					$x$は自由に現れないので,$\psi$に$x$が自由に現れていない場合は
					$\psi(x/\tau)$は$\psi$であり,$\widehat{\psi}(x/\tau)$は
					$\widehat{\psi}$である.
				}.
				
				\begin{description}
					\item[case1] $\varphi$が
						\begin{align}
							\negation \psi
						\end{align}
						なる式である場合,メタ定理\ref{metathm:relation_to_subformula_rewriting_1}より$\widehat{\varphi}$は
						\begin{align}
							\negation \widehat{\psi}
						\end{align}
						なる形で書けて,$\widehat{\psi}$は$\psi$の書き換えである.
						(IH)より$\widehat{\psi}(x/\tau)$は
						$\psi(x/\tau)$の書き換えであるから,
						再びメタ定理\ref{metathm:relation_to_subformula_rewriting_1}より
						$\negation \widehat{\psi}(x/\tau)$は
						$\negation \psi(x/\tau)$の書き換えである.
						$\negation \widehat{\psi}(x/\tau)$とは
						$\widehat{\varphi}(x/\tau)$のことであり,
						$\negation \psi(x/\tau)$とは$\varphi(x/\tau)$のことであるから,
						$\widehat{\varphi}(x/\tau)$は$\varphi(x/\tau)$の書き換えである.
					
					\item[case2] $\varphi$が
						\begin{align}
							\vee \psi \xi
						\end{align}
						なる式である場合,メタ定理\ref{metathm:relation_to_subformula_rewriting_2}より$\widehat{\varphi}$は
						\begin{align}
							\vee \widehat{\psi} \widehat{\xi}
						\end{align}
						なる形で書けて,$\widehat{\psi}$は$\psi$の書き換えであり,
						$\widehat{\xi}$は$\xi$の書き換えである.
						(IH)より$\widehat{\psi}(x/\tau)$は
						$\psi(x/\tau)$の書き換えであり,また$\widehat{\xi}(x/\tau)$は
						$\xi(x/\tau)$の書き換えであるから,
						再びメタ定理\ref{metathm:relation_to_subformula_rewriting_2}より
						$\vee \widehat{\psi}(x/\tau)\widehat{\xi}(x/\tau)$は
						$\vee \psi(x/\tau)\xi(x/\tau)$の書き換えである..
						$\vee \widehat{\psi}(x/\tau)\widehat{\xi}(x/\tau)$とは
						$\widehat{\varphi}(x/\tau)$のことであり,
						$\vee \psi(x/\tau)\xi(x/\tau)$とは
						$\varphi(x/\tau)$のことであるから,
						$\widehat{\varphi}(x/\tau)$は$\varphi(x/\tau)$の書き換えである.
					
					\item[case3] $\varphi$が
						\begin{align}
							\exists y \psi
						\end{align}
						なる式である場合,メタ定理\ref{metathm:relation_to_subformula_rewriting_3}より$\widehat{\varphi}$は
						\begin{align}
							\exists y \widehat{\psi}
						\end{align}
						なる形で書けて,$\widehat{\psi}$は$\psi$の書き換えである.
						(IH)より$\widehat{\psi}(x/\tau)$は
						$\psi(x/\tau)$の書き換えであるから,
						再びメタ定理\ref{metathm:relation_to_subformula_rewriting_3}より
						$\exists y \widehat{\psi}(x/\tau)$は
						$\exists y \psi(x/\tau)$の書き換えである.
						$\exists y \widehat{\psi}(x/\tau)$とは
						$\widehat{\varphi}(x/\tau)$のことであり,
						$\exists y \psi(x/\tau)$とは$\varphi(x/\tau)$のことであるから,
						$\widehat{\varphi}(x/\tau)$は$\varphi(x/\tau)$の書き換えである.
						\QED
				\end{description}
		\end{description}
	\end{metaprf}
	
	\begin{metaprf}[第二]
		$\widehat{\varphi}$を$\varphi$の書き換えとし,
		\begin{itembox}[l]{IH (帰納法の仮定)}
			$\widehat{\varphi}(x/\tau)$は$\varphi(x/\tau)$の書き換えである
		\end{itembox}
		と仮定する.このとき,$\widehat{\varphi}$に$\forall z \xi$ (resp. $\exists z \xi$)の
		形の部分式が現れているとし,$y$を$\xi$に自由に現れない変項で$\xi$の中で$z$への代入について
		自由であるものとし,$\widehat{\varphi}$の$\forall z \xi$ (resp. $\exists z \xi$)
		の部分を一か所だけ$\forall y \xi(z/y)$ (resp. $\exists y \xi(z/y)$)
		に差し替えた式を$\widetilde{\varphi}$とする(つまり$\widetilde{\varphi}$も$\varphi$の書き換えである).
		メタ定理\ref{metathm:subformula_replacing_and_substitution}より
		$\widetilde{\varphi}(x/\tau)$とは$\widehat{\varphi}(x/\tau)$の部分式
		$\forall z \xi(x/\tau)$ (resp. $\exists z \xi(x/\tau)$)を
		$\forall y \xi(x/\tau)(z/y)$ (resp. $\exists y \xi(x/\tau)(z/y)$)に差し替えた式であり,
		(IH)より$\widehat{\varphi}(x/\tau)$は$\varphi(x/\tau)$の書き換えであるから,
		$\widetilde{\varphi}(x/\tau)$もまた$\varphi(x/\tau)$の書き換えである.
		\QED
	\end{metaprf}

\chapter{推論}
\label{chap:inference}
	\section{証明}
	閉式には,「真」であるか,「偽」であるか,のどちらかのラベルが付けられる.
	「真である」という言明は,「正しい」や「成り立つ」などとも言い換えられる.
	式が真であるか偽であるかは,次の手順に従って発見的に判明していく.
	
	\begin{itemize}
		\item $\Sigma$の閉式は真である.
		\item $A$と$\rightarrow AB$が真であると判明しているならば,$B$は真である.
		\item $\rightarrow \wedge ABA$と$\rightarrow \wedge ABB$は真である.
		\item $A$と$B$が真であると判明しているならば$\wedge AB$と$\wedge BA$は真である.
		\item $\rightarrow A\vee AB$と$\rightarrow B \vee AB$は真である.
		\item $\rightarrow AC$と$\rightarrow BC$が真であると判明しているならば
			$\rightarrow \vee ABC$は真である.
		\item $\rightarrow\wedge A \rightharpoondown A \bot$は真である.
		\item $\rightarrow \rightarrow A \bot \rightharpoondown A$は真である.
		\item $\rightarrow \rightharpoondown\rightharpoondown AA$は真である.
	\end{itemize}
	
	真であると判明している式$\varphi$を起点にして,
	上の推論規則を駆使して閉式$\psi$が真であると判明すれば,
	$\varphi$から始めて$\psi$が真であることに辿り着くまでの手続きは$\psi$の証明と呼ばれ,
	$\psi$は定理と呼ばれる.
	
	証明には真であると判明している式が必要であり,その大元として選ばれた式が$\Sigma$の式である.
	$\Sigma$の式は証明なしに真であると決められているのであり,これらを公理と呼び定理と区別する.
	
	与えられた閉式$\varphi$が証明可能であるとは,
	\begin{itemize}
		\item 閉式$\psi$で,$\psi$と$\psi \rightarrow \varphi$が真であると判明している者が得られる.
		\item 真であると判明している閉式$\psi$と$\xi$が得られて,$\varphi$は$\psi \wedge \xi$である.
		\item 閉式$\psi$と$\xi$で,$\psi \vee \xi$と$\psi \rightarrow \varphi$と$\xi \rightarrow \varphi$が真であると判明しているものが得られる.
	\end{itemize}
	
	のいずれかの場合であり,
	\begin{align}
		\vdash \varphi
	\end{align}
	と書く.
	
	証明された式が真なる式である.では真なる式は
	%\section{式の書き換え(没)}
	\begin{itemize}
		\item $x \in y$はそのまま$x \in y$
		\item $x \in \{y|B(y)\}$は$B(x)$
			
			これは公理である.つまり,
			\begin{align}
				\forall x\, \left(\, x \in \{y|B(y)\} \leftrightarrow B(x)\, \right).
			\end{align}
			
		\item $x \in \varepsilon y B(y)$は$\exists t\, \left(\, x \in t \wedge B(t)\, \right)$.ちなみにこれは公理とするべきか:
			\begin{align}
				\forall x\, \left(\, x \in \varepsilon y B(y) \leftrightarrow
				\exists t\, \left(\, x \in t \wedge B(t)\, \right)\, \right).
			\end{align}
			
		\item $\{x|A(x)\} \in y$は$\exists s\, \left(\, s \in y \wedge 
			\forall u\, \left(\, u \in s \leftrightarrow A(u)\, \right)\, \right)$
			
			実はこの両式は同値である.さていま
			\begin{align}
				\{x|A(x)\} \in y \leftrightarrow
				\exists s\, \left(\, s \in y \wedge 
				\forall u\, \left(\, u \in s \leftrightarrow A(u)\, \right)\, \right)
			\end{align}
			という式を$\varphi$とし,これを$\mathcal{L}_{\in}$の式に書き換えたものを$\hat{\varphi}$としよう.そして
			\begin{align}
				\eta = \varepsilon y \rightharpoondown \hat{\varphi}(y)
			\end{align}
			とおこう.ここで証明するのは
			\begin{align}
				\{x|A(x)\} \in \eta \leftrightarrow
				\exists s\, \left(\, s \in \eta \wedge 
				\forall u\, \left(\, u \in s \leftrightarrow A(u)\, \right)\, \right)
			\end{align}
			が成り立つということである.まず
			\begin{align}
				\{x|A(x)\} \in \eta
			\end{align}
			が成り立っているとしよう.すると
			\begin{align}
				\exists s\, \left(\, \{x|A(x)\} = s\, \right)
			\end{align}
			が成り立つのだが,今度も式の書き直し手順によって
			\begin{align}
				\exists s\, \left(\, \forall u\, \left(\, A(u) \leftrightarrow
				u \in s\, \right)\, \right)
			\end{align}
			と書き直される.
			\begin{align}
				\sigma = \varepsilon s\, \left(\, \forall u\, \left(\, A(u) \leftrightarrow
				u \in s\, \right)\, \right)
			\end{align}
			とおくと
			\begin{align}
				\forall u\, \left(\, A(u) \leftrightarrow
				u \in \sigma\, \right)
			\end{align}
			が成り立ち,他方で
			\begin{align}
				\sigma = \{x|A(x)\}
			\end{align}
			が成り立つのだから
			\begin{align}
				\sigma \in \eta
			\end{align}
			も従う.ゆえに
			\begin{align}
				\sigma \in \eta \wedge \forall u\, \left(\, A(u) \leftrightarrow
				u \in \sigma\, \right)
			\end{align}
			が成り立つ.逆に
			\begin{align}
				\exists s\, \left(\, s \in \eta \wedge 
				\forall u\, \left(\, u \in s \leftrightarrow A(u)\, \right)\, \right)
			\end{align}
			が成り立っているとして,
			\begin{align}
				\sigma = \varepsilon s\, \left(\, s \in \eta \wedge 
				\forall u\, \left(\, u \in s \leftrightarrow A(u)\, \right)\, \right)
			\end{align}
			としよう.すると
			\begin{align}
				\sigma \in \eta \wedge 
				\forall u\, \left(\, u \in \sigma \leftrightarrow A(u)\, \right)
			\end{align}
			が成り立つので
			\begin{align}
				\sigma \in \eta
			\end{align}
			かつ
			\begin{align}
				\sigma = \{x|A(x)\}
			\end{align}
			が成立する.ゆえに
			\begin{align}
				\{x|A(x)\} \in \eta
			\end{align}
			が成立する.以上で
			\begin{align}
				\{x|A(x)\} \in \eta \leftrightarrow
				\exists s\, \left(\, s \in \eta \wedge 
				\forall u\, \left(\, u \in s \leftrightarrow A(u)\, \right)\, \right)
			\end{align}
			が得られた.
			
		\item $\{x|A(x)\} \in \{y|B(y)\}$は$\exists s\, \left(\, B(s) \wedge 
			\forall u\, \left(\, u \in s \leftrightarrow A(u)\, \right)\, \right)$
		
		\item $\{x|A(x)\} \in \varepsilon y B(y)$は$\exists s,t\, \left(\, s \in t \wedge 
			\forall u\, \left(\, u \in s \leftrightarrow A(u)\, \right) \wedge B(t)\, \right)$
		
		\item $\varepsilon x A(x) \in y$は$\exists s\, \left(\, s \in y \wedge A(s)\, \right)$
			
			これも公理にしよう:
			\begin{align}
				\forall y\, \left(\, \varepsilon x A(x) \in y \leftrightarrow
				\exists s\, \left(\, s \in y \wedge A(s)\, \right)\, \right).
			\end{align}
			いや,$y$をクラスとした言明の方が良いかも.
			\begin{align}
				\varepsilon x A(x) \in y \leftrightarrow
				\exists s\, \left(\, s \in y \wedge A(s)\, \right).
			\end{align}
		
		\item $\varepsilon x A(x) \in \{y|B(y)\}$は$\exists s\, \left(\, A(s) \wedge B(s)\, \right)$
			
			上の公理からこの式の同値性も導かれます.まず
			\begin{align}
				\exists s\, \left(\, A(s) \wedge B(s)\, \right)
			\end{align}
			が成り立っているとしよう.そして
			\begin{align}
				\sigma = \varepsilon s\, \left(\, A(s) \wedge B(s)\, \right)
			\end{align}
			とおくと,
			\begin{align}
				A(\sigma) \wedge B(\sigma)
			\end{align}
			が成立する.ゆえに
			\begin{align}
				\sigma \in \{y|B(y)\} \wedge A(\sigma)
			\end{align}
			が成立する.ゆえに
			\begin{align}
				\exists s\, \left(\, s \in \{y|B(y)\} \wedge A(s)\, \right)
			\end{align}
			が成り立つ.ゆえに
			\begin{align}
				\varepsilon x A(x) \in \{y|B(y)\}
			\end{align}
			が成り立つ.逆に
			\begin{align}
				\varepsilon x A(x) \in \{y|B(y)\}
			\end{align}
			が成り立っているとしよう.すると
			\begin{align}
				\exists s\, \left(\, s = \varepsilon x A(x)\, \right)
			\end{align}
			が成り立つが,これは$\mathcal{L}_{\in}$の式で
			\begin{align}
				\exists s A(s)
			\end{align}
			であって,
			\begin{align}
				\sigma = \varepsilon s A(s)
			\end{align}
			とおけば
			\begin{align}
				A(\sigma)
			\end{align}
			が成立する.ところで
			\begin{align}
				\sigma \in \{y|B(y)\}
			\end{align}
			なので
			\begin{align}
				B(\sigma)
			\end{align}
			も成り立つ.ゆえに
			\begin{align}
				A(\sigma) \wedge B(\sigma)
			\end{align}
			が成り立つ.ゆえに
			\begin{align}
				\exists s\, \left(\, A(s) \wedge B(s)\, \right)
			\end{align}
			が成り立つ.
			
		\item $\varepsilon x A(x) \in \varepsilon y B(y)$は$\exists s,t\, \left(\, s \in t \wedge A(s) \wedge B(t)\, \right)$
		
			この式の同値性も証明できる.まず
			\begin{align}
				\exists s,t\, \left(\, s \in t \wedge A(s) \wedge B(t)\, \right)
			\end{align}
			が成り立っているとしよう.この式は
			\begin{align}
				\exists s\, \left(\, \exists t\, \left(\, s \in t \wedge A(s) \wedge B(t)\, \right)\, \right)
			\end{align}
			の略記であって,$\exists$の規則より
			\begin{align}
				\sigma = \varepsilon s\, \left(\, \exists t\, \left(\, s \in t \wedge A(s) \wedge B(t)\, \right)\, \right)
			\end{align}
			とおけば
			\begin{align}
				\exists t\, \left(\, \sigma \in t \wedge A(\sigma) \wedge B(t)\, \right)
			\end{align}
			が成立する.$\sigma \in t \wedge A(\sigma) \wedge B(t)$を$\mathcal{L}_{\in}$の式に書き直したものを
			$\varphi(t)$として
			\begin{align}
				\tau \defeq \varepsilon t \varphi(t)
			\end{align}
			とおけば,$\exists$の規則より
			\begin{align}
				\sigma \in \tau \wedge A(\sigma) \wedge B(\tau)
			\end{align}
			が成立する.ゆえに
			\begin{align}
				\exists s\, \left(\, s \in \tau \wedge A(s)\, \right)
			\end{align}
			が成り立つから,公理より
			\begin{align}
				\varepsilon x A(x) \in \tau
			\end{align}
			が成立する.ゆえに
			\begin{align}
				\varepsilon x A(x) \in \tau \wedge B(\tau)
			\end{align}
			が成立する.ゆえに
			\begin{align}
				\exists t\, \left(\, \varepsilon x A(x) \in t \wedge B(t)\, \right)
			\end{align}
			が成立する.公理より
			\begin{align}
				\varepsilon x A(x) \in \varepsilon y B(y)
			\end{align}
			が成立する.逆は容易い.
			\begin{align}
				\varepsilon x A(x) \in \varepsilon y B(y)
			\end{align}
			が成り立っているとすれば公理より
			\begin{align}
				\exists t\, \left(\, \varepsilon x A(x) \in t \wedge B(t)\, \right)
			\end{align}
			が成立する.$\varepsilon x A(x) \in t \wedge B(t)$を$\mathcal{L}_{\in}$の式に書き直したものを$\psi(t)$として
			\begin{align}
				\tau \defeq \varepsilon t \psi(t)
			\end{align}
			とおけば
			\begin{align}
				\varepsilon x A(x) \in \tau \wedge B(\tau)
			\end{align}
			が成立するが,ここで公理より
			\begin{align}
				\exists s\, \left(\, s \in \tau \wedge A(s)\, \right)
			\end{align}
			が成り立つので,$s \in \tau \wedge A(s)$を$\mathcal{L}_{\in}$の式に書き直したものを$\xi(s)$として
			\begin{align}
				\sigma \defeq \varepsilon s \xi(s)
			\end{align}
			とおけば
			\begin{align}
				\sigma \in \tau \wedge A(\sigma)
			\end{align}
			が成立する.以上より
			\begin{align}
				\sigma \in \tau \wedge A(\sigma) \wedge B(\tau)
			\end{align}
			が成立する.ゆえに
			\begin{align}
				\exists t\, \left(\, \sigma \in t \wedge A(\sigma) \wedge B(t)\, \right)
			\end{align}
			が得られる.ゆえに
			\begin{align}
				\exists s\, \left(\, \exists t\, \left(\, \sigma \in t \wedge A(\sigma) \wedge B(t)\, \right)\, \right)
			\end{align}
			が得られる.
	\end{itemize}
	
	\begin{screen}
		\begin{logicalaxm}\mbox{}
			\begin{itemize}
				\item 任意の閉項$\tau$に対して,$A(\tau)$が定理ならば$\exists x A(x)$が成り立つ.
				\item $\exists x A(x)$が定理ならば,$A(\varepsilon x \hat{A}(x))$が成り立つ.
				\item すべての閉項$\tau$に対して$A(\tau)$が定理ならば,$\forall x A(x)$が成り立つ.
				\item $\forall x A(x)$が定理ならば,すべての閉項$\tau$に対して$A(\tau)$が成り立つ.
			\end{itemize}
		\end{logicalaxm}
	\end{screen}
	
	定理として
	\begin{align}
		\forall x A(x) \Longleftrightarrow A(\varepsilon x \rightharpoondown \hat{A}(x))
	\end{align}
	が得られる.アイデアとしてはさあ,$\varepsilon x A(x)$の全体が集合に対応しているのであって,
	いやもちろん集合そのものではないけど,集合は$\varepsilon x A(x)$のどれかに等しい類なわけで,
	だからモデル論に出てくる「宇宙」とかいう得体の知れない集合()は俺の集合論に不要なんだよね.
	俺のノートの「宇宙」はすべて実態が把握できるように,具体的な記号列で書き表せるのが良いよね.
	ちなみにこの「宇宙」は$\{x|x=x\}$とは別ね.
	
	\begin{screen}
		\begin{axm}
			\begin{align}
				\forall x\, \left(\, x \in \{y|B(y)\} \leftrightarrow B(x)\, \right).
			\end{align}
			
			\begin{align}
				\forall x\, \left(\, x \in \varepsilon y B(y) \leftrightarrow
				\left(\, \exists t\, B(t) \rightarrow 
				\exists t\, \left(\, x \in t \wedge B(t)\, \right)\, \right)\, \right).
			\end{align}
			が定理となるために
			\begin{align}
				x \in \varepsilon y B(y) \leftrightarrow
				\exists t\, \left(\, x \in t \wedge B(t) \leftrightarrow \exists y B(y)\, \right)
			\end{align}
			を公理とする.
			
			\begin{align}
				\varepsilon x A(x) \in y \leftrightarrow
				\exists s\, \left(\, s \in y \wedge A(s)\, \right).
			\end{align}
		\end{axm}
	\end{screen}
	
	いや,した二つは公理じゃねえな.定理だ.実際
	\begin{align}
		x \in \varepsilon y B(y)
	\end{align}
	が成り立っているとしよう.$\varepsilon y B(y)$は集合であって
	\begin{align}
		\exists s\, \left(\, s = \varepsilon y B(y)\, \right)
	\end{align}
	が成り立つので,
	\begin{align}
		fff
	\end{align}
	
	\begin{screen}
		\begin{thm}
			\begin{align}
				\exists x A(x) \rightarrow \varepsilon x A(x) \in \{x|A(x)\}.
			\end{align}
		\end{thm}
	\end{screen}
	
	\begin{sketch}
		\begin{align}
			\exists x A(x)
		\end{align}
		が成り立ているとするとき,
		\begin{align}
			\sigma \defeq \varepsilon x A(x)
		\end{align}
		とおけば
		\begin{align}
			A(\sigma)
		\end{align}
		が成り立つので,公理より
		\begin{align}
			\sigma \in \{x|A(x)\}
		\end{align}
		が成立する.ゆえに
		\begin{align}
			\sigma \in \{x|A(x)\} \wedge A(\sigma)
		\end{align}
		が成り立つ.ゆえに
		\begin{align}
			\exists s\, \left(\, s \in \{x|A(x)\} \wedge A(s)\, \right)
		\end{align}
		が成立する.ゆえに公理より
		\begin{align}
			\varepsilon x A(x) \in \{x|A(x)\}
		\end{align}
		が成立する.
		\QED
	\end{sketch}
	
	\begin{itembox}[l]{満たされて欲しいこと}
		\begin{description}
			\item[等号]
				\begin{itemize}
					\item $x = \{y|B(y)\}$と$\forall s\, \left(\, s \in x \leftrightarrow B(s)\, \right)$
					\item $x = \varepsilon y B(y)$と$\exists s\, \left(\, A(s) \wedge \forall u\,
						\left(\, u \in x \leftrightarrow u \in s\, \right)\, \right)$
					\item $\{x|A(x)\} = \{y|B(y)\}$と$\forall s\, \left(\, A(s) \leftrightarrow B(s)\, \right)$
					\item $\{x|A(x)\} = \varepsilon y B(y)$と
						$\exists s\, \left(\, \forall u\,
						\left(\, u \in s \leftrightarrow A(u)\, \right) \wedge B(s)\, \right)$
					\item $\varepsilon x A(x) = \varepsilon y B(y)$と
						$\exists s,t\, \left(\, s = t \wedge A(s) \wedge B(t)\, \right)$
				\end{itemize}
				
			\item[帰属]
				\begin{itemize}
					\item $x \in \{y|B(y)\}$は$B(x)$
					\item $x \in \varepsilon y B(y)$は$\exists t\, \left(\, x \in t \wedge B(t)\, \right)$
					\item $\{x|A(x)\} \in y$は$\exists s\, \left(\, s \in y \wedge \forall u\, \left(\, u \in s \leftrightarrow A(u)\, \right)\, \right)$
					\item $\{x|A(x)\} \in \{y|B(y)\}$は$\exists s\, \left(\, B(s) \wedge \forall u\, \left(\, u \in s \leftrightarrow A(u)\, \right)\, \right)$
					\item $\{x|A(x)\} \in \varepsilon y B(y)$は$\exists s,t\, \left(\, s \in t \wedge \forall u\, \left(\, u \in s \leftrightarrow A(u)\, \right) \wedge B(t)\, \right)$
					\item $\varepsilon x A(x) \in y$は$\exists s\, \left(\, s \in y \wedge A(s)\, \right)$
					\item $\varepsilon x A(x) \in \{y|B(y)\}$は$\exists s\, \left(\, A(s) \wedge B(s)\, \right)$
					\item $\varepsilon x A(x) \in \varepsilon y B(y)$は$\exists s,t\, \left(\, s \in t \wedge A(s) \wedge B(t)\, \right)$
				\end{itemize}
		\end{description}
	\end{itembox}

\section{定理I\hspace{-.1em}I.15.2}
	\begin{description}
		\item[(3)]
			項$\tau$が変項$x$のとき,$\zeta_{\tau}(y)$を
			\begin{align}
				x = y
			\end{align}
			とすれば,
			\begin{align}
				\Sigma' \vdash \forall x\, \exists! y\, (\, x=y\, )
			\end{align}
			つまり
			\begin{align}
				\Sigma' \vdash \forall x\, \exists! y\, \zeta_{\tau}(y)
			\end{align}
			および
			\begin{align}
				\Sigma \vdash \forall x\, (\, x=x\, )
			\end{align}
			つまり
			\begin{align}
				\Sigma \vdash \forall x\, \zeta_{\tau}(\tau)
			\end{align}
			が成り立つ.項$\tau$が
			\begin{align}
				f\tau_{1}\cdots\tau_{n}
			\end{align}
			のとき,$f \in \mathcal{L} \backslash \mathcal{L}_{\in}$ならば$\zeta_{\tau}(y)$を
			\begin{align}
				\exists z_{1}, \cdots, z_{n}\, \left(\, 
				\theta_{f}(z_{1},\cdots,z_{n},y) \wedge \zeta_{\tau_{1}}(z_{1}) \wedge
				\cdots \wedge \zeta_{\tau_{n}}(z_{n})\, \right)
			\end{align}
			とし,$f \in \mathcal{L}_{\in}$ならば
			\begin{align}
				\exists z_{1}, \cdots, z_{n}\, \left(\, 
				f(z_{1},\cdots,z_{n}) = y \wedge \zeta_{\tau_{1}}(z_{1}) \wedge
				\cdots \wedge \zeta_{\tau_{n}}(z_{n})\, \right)
			\end{align}
			とする.仮定より
			\begin{align}
				\Sigma \vdash \exists! y\, \zeta_{\tau_{i}}(y)
			\end{align}
			が成り立つので,
			\begin{align}
				\Sigma \vdash \zeta_{\tau_{i}}(z_{i})
			\end{align}
			を満たす$z_{i}$が取れる.そして定義I\hspace{-.1em}I.15.1より
			\begin{align}
				\Sigma \vdash \exists!y\, \theta_{f}(z_{1},\cdots,z_{n},y)
			\end{align}
			が成り立つので,その$y$を取れば
			\begin{align}
				\Sigma \vdash \exists y\, \zeta_{\tau}(y)
			\end{align}
			が成立する.ただし,$\eta$を
			\begin{align}
				\Sigma \vdash \exists z_{1}, \cdots, z_{n}\, \left(\, 
				\theta_{f}(z_{1},\cdots,z_{n},\eta) \wedge \zeta_{\tau_{1}}(z_{1}) \wedge
				\cdots \wedge \zeta_{\tau_{n}}(z_{n})\, \right)
			\end{align}
			を満たすものとすれば,このとき
			\begin{align}
				\Sigma \vdash \zeta_{\tau_{i}}(w_{i})
			\end{align}
			および
			\begin{align}
				\Sigma \vdash \theta_{f}(w_{1},\cdots,w_{n},\eta)
			\end{align}
			を満たす$w_{i}$が取れるが,$z_{i} = w_{i}$なので
			\begin{align}
				\Sigma \vdash \theta_{f}(z_{1},\cdots,z_{n},\eta)
			\end{align}
			が成り立つことになって,定義I\hspace{-.1em}I.15.1より
			\begin{align}
				y = \eta
			\end{align}
			が成り立つ.ゆえに
			\begin{align}
				\Sigma \vdash \exists! y\, \zeta_{\tau}(y)
			\end{align}
			が成立する.他方で仮定より
			\begin{align}
				\Sigma' \vdash \zeta_{\tau_{i}}(\tau_{i})
			\end{align}
			が成り立ち,かつ定義I\hspace{-.1em}I.15.1より
			\begin{align}
				\Sigma' \vdash \forall x_{1},\cdots,x_{n}\,
				\theta_{f}(x_{1},\cdots,x_{n},f(x_{1},\cdots,x_{n}))
			\end{align}
			が成り立つので
			\begin{align}
				\Sigma' \vdash \theta_{f}(\tau_{1},\cdots,\tau_{n},f(\tau_{1},\cdots,\tau_{n}))
			\end{align}
			が成り立つ.ゆえに
			\begin{align}
				\Sigma' \vdash \theta_{f}(\tau_{1},\cdots,\tau_{n},f(\tau_{1},\cdots,\tau_{n})) \wedge \zeta_{\tau_{1}}(\tau_{1}) \wedge \cdots \wedge \zeta_{\tau_{n}}(\tau_{n})
			\end{align}
			が成り立つ.ゆえに
			\begin{align}
				\Sigma' \vdash \zeta_{\tau}(\tau)
			\end{align}
			が成り立つ.
			
		\item[(2)]
			$\varphi$を$p\tau_{1}\cdots\tau_{n}$なる原子式とするとき,
			$p$が$\mathcal{L} \backslash \mathcal{L}_{\in}$の要素ならば$\hat{\varphi}$を
			\begin{align}
				\exists z_{1},\cdots,z_{n}\, 
				\left(\, \theta_{p}z_{1} \cdots z_{n} \wedge \zeta_{\tau_{1}}(z_{1})
				\wedge \cdots \wedge \zeta_{\tau_{n}}(z_{n})\, \right)
			\end{align}
			とし,$p$が$\mathcal{L}_{\in}$の要素ならば
			\begin{align}
				\exists z_{1},\cdots,z_{n}\, 
				\left(\, pz_{1} \cdots z_{n} \wedge \zeta_{\tau_{1}}(z_{1})
				\wedge \cdots \wedge \zeta_{\tau_{n}}(z_{n})\, \right)
			\end{align}
			とする.$\varphi$が成り立っているとき,仮定より
			\begin{align}
				\Sigma' \vdash \zeta_{\tau_{i}}(\tau_{i})
			\end{align}
			が満たされ,また定義I\hspace{-.1em}I.15.1より($\Delta$は$\Sigma'$に含まれているので)
			\begin{align}
				\Sigma' \cup \{\varphi\} \vdash \theta_{p}\tau_{1} \cdots \tau_{n}
			\end{align}
			も満たされているので
			\begin{align}
				\Sigma' \cup \{\varphi\} \vdash \theta_{p}\tau_{1} \cdots \tau_{n}
				\wedge \zeta_{\tau_{1}}(\tau_{1}) \wedge \cdots \wedge 
				\zeta_{\tau_{n}}(\tau_{n})
			\end{align}
			が成り立つ.すなわち
			\begin{align}
				\Sigma' \cup \{\varphi\} \vdash \hat{\varphi}
			\end{align}
			が成り立つ.つまり
			\begin{align}
				\Sigma' \vdash \varphi \rightarrow \hat{\varphi}
			\end{align}
			が成り立つ.逆に$\hat{\varphi}$が成り立っているとき,
			\begin{align}
				\Sigma' \cup \{\hat{\varphi}\} \vdash \theta_{p}w_{1} \cdots w_{n}
				\wedge \zeta_{\tau_{1}}(w_{1}) \wedge \cdots \wedge 
				\zeta_{\tau_{n}}(w_{n})
			\end{align}
			を満たす$w_{1},\cdots,w_{n}$が取れるが,
			\begin{align}
				\Sigma \vdash \exists! y\, \zeta_{\tau_{i}}(y)
			\end{align}
			かつ
			\begin{align}
				\Sigma' \vdash \zeta_{\tau_{i}}(\tau_{i})
			\end{align}
			なので
			\begin{align}
				w_{i} = \tau_{i}
			\end{align}
			である.ゆえに
			\begin{align}
				\Sigma' \cup \{\hat{\varphi}\} \vdash \theta_{p}\tau_{1} \cdots \tau_{n}
			\end{align}
			が成り立つ.ゆえに
			\begin{align}
				\Sigma' \vdash \hat{\varphi} \rightarrow \varphi
			\end{align}
			が得られる.
			\QED
	\end{description}
	
	\begin{screen}
		$\forall x\, (\, x \notin y\, )$を$\theta_{\emptyset}(y)$とするとき.
	\end{screen}
	
	$\emptyset$の定義$\delta_{\emptyset}$は
	\begin{align}
		\forall x\, (\, x \notin \emptyset\, )
	\end{align}
	である.$\zeta_{\emptyset}(y)$は
	\begin{align}
		\forall x\, (\, x \notin y\, )
	\end{align}
	であって,
	\begin{align}
		\Sigma \vdash \exists! y\, \zeta_{\emptyset}(y)
	\end{align}
	が成り立ち,また$\delta_{\emptyset}$と$\zeta_{\emptyset}(1)$は同じなので
	\begin{align}
		\Sigma \cup \{\delta_{\emptyset}\} = \Sigma' \vdash \zeta_{\emptyset}(\emptyset)
	\end{align}
	が成り立つ.そして,例えば$z$を変項とすれば
	\begin{align}
		z \in \emptyset
	\end{align}
	と
	\begin{align}
		\exists s,t\, \left(\, s \in t \wedge s = z \wedge \forall x\, (\, x \notin t\, )\, \right)
	\end{align}
	が$\Sigma'$の下で同値になる.
	
	\begin{screen}
		$\forall x\, (\, x \cdot y = y \cdot x = x\, )$を$\theta_{1}(y)$とするとき,
	\end{screen}
	
	$1$の定義$\delta_{1}$は
	\begin{align}
		\forall x\, (\, x \cdot 1 = 1 \cdot x = x\, )
	\end{align}
	である.$\zeta_{1}(y)$は
	\begin{align}
		\forall x\, (\, x \cdot y = y \cdot x = x\, )
	\end{align}
	であって,
	\begin{align}
		\Sigma \vdash \exists! y\, \zeta_{1}(y)
	\end{align}
	が成り立ち,また$\delta_{1}$と$\zeta_{1}(1)$は同じなので
	\begin{align}
		\Sigma \cup \{\delta_{1}\} = \Sigma' \vdash \zeta_{1}(1)
	\end{align}
	が成り立つ.そして,例えば$z$を変項とすれば
	\begin{align}
		z \in 1
	\end{align}
	と
	\begin{align}
		\exists s,t\, \left(\, s \in t \wedge s = z \wedge \forall x\, (\, x \cdot t = t \cdot x = x\, )\, \right)
	\end{align}
	が$\Sigma'$の下で同値になる.
	
	\begin{screen}
		$y \cdot (x \cdot x) = x$を$\theta_{i}(x,y)$とするとき,
	\end{screen}
	
	$i$の定義$\delta_{i}$は
	\begin{align}
		\forall x\, \left(\, i(x) \cdot (x \cdot x) = x\, \right)
	\end{align}
	である.$\zeta_{i(x)}(y)$は
	\begin{align}
		\exists x\, \left(\, \theta_{i}(x,y) \wedge x = x\, \right)
	\end{align}
	である.そして,例えば$x,z$を変項とすれば
	\begin{align}
		z \in i(x)
	\end{align}
	と
	\begin{align}
		\exists s,t\, \left(\, s \in t \wedge s = z \wedge \zeta_{i(x)}(t)\, \right)
	\end{align}
	が$\Sigma'$の下で同値になる.
	
\section{菊池誠不完全性定理}
	言語とは定数記号と関数記号と関係記号の全体ということで,
	\begin{itemize}
		\item $\mathcal{L}_{\in}$とは$\{\in,\natural\}$.
		\item $\mathcal{L}$とは$\{\in\}$に加えて閉項の全体.
	\end{itemize}
	\section{推論}
	本節では,「集合でも真類でもない類は存在しない」と「集合であり真類でもある類は存在しない」の二つの言明の正否の決定を主軸にして
	{\bf 推論規則}\index{すいろんきそく@推論規則}{\bf (rule of inference)}を導入し,基本的な推論法則を導出する.
	
	\begin{screen}
		\begin{logicalaxm}[排中律]
			$A$を任意の文とするとき次は定理である:
			\begin{align}
				A \vee \rightharpoondown A.
			\end{align}
		\end{logicalaxm}
	\end{screen}
	
	排中律の言明は``どんな文でも持ってくれば,その式に対して排中律が適用される''という意味である.
	このように無数に存在し得る定理を一括して表す式は{\bf 公理図式}\index{こうりずしき@公理図式}{\bf (schema)}と呼ばれる.
	
	いま$a,b$を類とするとき,
	\begin{align}
		a \notin b \defarrow\ \rightharpoondown a \in b
	\end{align}
	で$a \notin b$を定める.同様に
	\begin{align}
		a \neq b \defarrow\ \rightharpoondown a = b
	\end{align}
	で$a \neq b$を定める.
	
	\monologue{
		定義記号$\defeq$と同様に,`$A \defarrow B$'とは
		式$B$を記号列$A$で置き換えて良いという意味で使われます.また,式中に記号列$A$が出てくるときは,
		暗黙裡にその$A$を$B$に戻して式を解釈します.
		$\defeq$も$\defarrow$も略記することと同じですね.
	}
	
	\begin{screen}
		\begin{thm}[類は集合であるか真類であるかのいずれかに定まる]
			$a$を類とするとき次は定理である:
			\begin{align}
				\set{a} \vee \rightharpoondown \set{a}.
			\end{align}
		\end{thm}
	\end{screen}
	
	\begin{prf}
		排中律を適用することにより従う.
		\QED
	\end{prf}
	
	排中律をそのまま適用することにより上の定理は導かれたが,``集合であり真類でもある類は存在しない''という主張はまだ得られない.
	以下はこの言明を証明することを目標にしてしばらく推論規則の話が続くが,提示される規則はどれも基本的で直感に反しないため
	通常は無断で使用されてしまうものである.
	
	ここで論理記号の名称を書いておく.
	\begin{itemize}
		\item $\bot$を{\bf 矛盾}\index{むじゅん@矛盾}{\bf (contradiction)}と呼ぶ.
		\item $\vee$を{\bf 論理和}\index{ろんりわ@論理和}{\bf (logical disjunction)}と呼ぶ.
		\item $\wedge$を{\bf 論理積}\index{ろんりせき@論理積}{\bf (logical conjunction)}と呼ぶ.
		\item $\Longrightarrow$を{\bf 含意}\index{がんい@含意}{\bf (implication)}と呼ぶ.
		\item $\rightharpoondown$を{\bf 否定}\index{ひてい@否定}{\bf (negation)}と呼ぶ.
	\end{itemize}
	
	\begin{screen}
		\begin{logicalaxm}[基本的な推論規則]\label{logicalaxm:fundamental_rules_of_inference}
			$A,B,C$を$\mathcal{L}'$の閉式とするとき,次の規則を認める:
			\begin{description}
				\item[三段論法] $A$ならびに$A \Longrightarrow B$が定理なら$B$は定理である.
				\item[演繹法則] $A$を公理に追加した下で$B$が定理であるなら,
					$A$を外した公理系で$A \Longrightarrow B$は定理である.
				\item[論理和の導入イ] $A \Longrightarrow (A \vee B)$は定理である.
				\item[論理和の導入ロ] $A \Longrightarrow (B \vee A)$は定理である.
				\item[論理積の導入] $A,B$が共に定理なら$A \wedge B$は定理である.
				\item[論理積の除去イ] $(A \wedge B) \Longrightarrow A$は定理である.
				\item[論理積の除去ロ] $(A \wedge B) \Longrightarrow B$は定理である.
				\item[場合分け法則] $A \Longrightarrow C$と$B \Longrightarrow C$が共に定理であるとき
					$(A \vee B) \Longrightarrow C$は定理である.
			\end{description}	
		\end{logicalaxm}
	\end{screen}
	
	\monologue{
		演繹法則について,``$A$を公理に追加する''ことを``$A$が成り立っていると仮定する''
		などの言明により示唆することが多いです.
	}
	
	\begin{itembox}[l]{演繹法則の意味}
		我々は公理か,或いは公理図式として,複数の式を選び出し$\mathcal{L}'$の世界において正しいと決める.
		それらは以降小出しに登場させるが,その全体は現段階ですでに決めているのでそれを
		\begin{align}
			\mathscr{S}
		\end{align}
		と呼ぶことにする.本稿で出てくる``正しい式''とは$\mathscr{S}$のみを公理系とした体系において証明される式を指す.
		演繹法則は,``$A$が成り立つとする''などの
		言明により$\mathscr{S}$に式$A$を加えたとき,その新しい公理系$\mathscr{S}'$の下で式$B$が成り立つなら,
		$\mathscr{S}$のみを公理とした体系において
		\begin{align}
			A \Longrightarrow B
		\end{align}
		が成立する,と主張している.複数の式を$\mathscr{S}$に追加する場合もある.
		たとえば$\mathscr{S}'$に式$C$を追加し,その新しい公理体系$\mathscr{S}''$の下で
		式$D$が成り立つ場合,演繹法則に則れば$\mathscr{S}'$の下で
		\begin{align}
			C \Longrightarrow D
		\end{align}
		が成立する.(ちなみに$\mathscr{S}''$から式$A$のみを抜いた公理系の下では
		$A \Longrightarrow D$が正しくなる.)
		このとき$\mathscr{S}$を公理系とした下では,再び演繹法則を適用することにより
		\begin{align}
			A \Longrightarrow (C \Longrightarrow D)
		\end{align}
		が成立するとわかる,が,
		\begin{align}
			C \Longrightarrow D
		\end{align}
		が成り立つ保証は無い.非常に屡々いくつも仮定を重ねたところに演繹法則を運用することがあるが,
		その都度どの段階の公理系を扱っているかを明確に把握しておかないと推論が破綻してしまう恐れがある.
	\end{itembox}
	
	\begin{screen}
		\begin{logicalthm}[含意の反射律]\label{logicalthm:reflective_law_of_implication}
			$A$を文とするとき
			\begin{align}
				\vdash A \Longrightarrow A.
			\end{align}
		\end{logicalthm}
	\end{screen}
	
	\begin{prf}
		$A \vdash A$であるから,演繹法則より$\vdash A \Longrightarrow A$となる.
		\QED
	\end{prf}
	
	\begin{screen}
		\begin{logicalthm}[論理和・論理積の可換律]
		\label{logicalthm:commutative_law_of_disjunction_and_conjunction}
			$A,B$を文とするとき
			\begin{itemize}
				\item $\vdash (A \vee B) \Longrightarrow (B \vee A)$.
				\item $\vdash (A \wedge B) \Longrightarrow (B \wedge A)$.
			\end{itemize}
		\end{logicalthm}
	\end{screen}
	
	\begin{prf}
		$\vee$の導入により
		\begin{align}
			\vdash A \Longrightarrow (B \vee A)
		\end{align}
		と
		\begin{align}
			\vdash B \Longrightarrow (A \vee B)
		\end{align}
		が成り立つので,場合分け法則より
		\begin{align}
			\vdash (A \vee B) \Longrightarrow (B \vee A)
		\end{align}
		が成り立つ.また,$\wedge$の除去より
		\begin{align}
			A \wedge B \vdash A
		\end{align}
		と
		\begin{align}
			A \wedge B \vdash B
		\end{align}
		となるので,$\wedge$の導入により
		\begin{align}
			A \wedge B \vdash B \wedge A
		\end{align}
		が成り立つ.よって演繹法則より
		\begin{align}
			\vdash (A \wedge B) \Longrightarrow (B \wedge A)
		\end{align}
		が成り立つ.
		\QED
	\end{prf}
	
	\begin{screen}
		\begin{logicalthm}[含意の推移律]\label{logicalthm:transitive_law_of_implication}
			$A,B,C$を文とするとき
			\begin{align}
				\vdash ((A \Longrightarrow B) \wedge (B \Longrightarrow C)) 
				\Longrightarrow (A \Longrightarrow C).
			\end{align}
		\end{logicalthm}
	\end{screen}
	
	\begin{prf}
		\begin{align}
			(A \Longrightarrow B) \wedge (B \Longrightarrow C),A \vdash 
			(A \Longrightarrow B) \wedge (B \Longrightarrow C)
		\end{align}
		であるから,$\wedge$の除去より
		\begin{align}
			(A \Longrightarrow B) \wedge (B \Longrightarrow C),A \vdash A \Longrightarrow B
		\end{align}
		となる.また
		\begin{align}
			(A \Longrightarrow B) \wedge (B \Longrightarrow C),A \vdash A
		\end{align}
		でもあるから,三段論法より
		\begin{align}
			(A \Longrightarrow B) \wedge (B \Longrightarrow C),A \vdash B
		\end{align}
		となる.$\wedge$の除去より
		\begin{align}
			(A \Longrightarrow B) \wedge (B \Longrightarrow C),A \vdash B \Longrightarrow C
		\end{align}
		も成り立つから,再び三段論法より
		\begin{align}
			(A \Longrightarrow B) \wedge (B \Longrightarrow C),A \vdash C
		\end{align}
		となる.よって演繹法則より
		\begin{align}
			(A \Longrightarrow B) \wedge (B \Longrightarrow C) \vdash A \Longrightarrow C
		\end{align}
		となり,
		\begin{align}
			\vdash ((A \Longrightarrow B) \wedge (B \Longrightarrow C)) 
			\Longrightarrow (A \Longrightarrow C)
		\end{align}
		を得る.
		\QED
	\end{prf}
	
	\begin{screen}
		\begin{logicalthm}[二式が同時に導かれるならその論理積が導かれる]
		\label{logicalthm:conjunction_of_consequences}
			$A,B,C$を文とするとき
			\begin{align}
				\vdash ((A \Longrightarrow B) \wedge (A \Longrightarrow C))
				\Longrightarrow (A \Longrightarrow (B \wedge C))
			\end{align}
		\end{logicalthm}
	\end{screen}
	
	\begin{prf}
		\begin{align}
			(A \Longrightarrow B) \wedge (A \Longrightarrow C),A \vdash
			(A \Longrightarrow B) \wedge (A \Longrightarrow C)
		\end{align}
		であるから,$\wedge$の除去より
		\begin{align}
			(A \Longrightarrow B) \wedge (A \Longrightarrow C),A \vdash
			A \Longrightarrow B
		\end{align}
		が成り立つ.
		\begin{align}
			(A \Longrightarrow B) \wedge (A \Longrightarrow C),A \vdash A
		\end{align}
		でもあるから
		\begin{align}
			(A \Longrightarrow B) \wedge (A \Longrightarrow C),A \vdash B
		\end{align}
		となる.同様にして
		\begin{align}
			(A \Longrightarrow B) \wedge (A \Longrightarrow C),A \vdash C
		\end{align}
		となるので,$\wedge$の導入により
		\begin{align}
			(A \Longrightarrow B) \wedge (A \Longrightarrow C),A \vdash B \wedge C
		\end{align}
		となり,演繹法則より
		\begin{align}
			(A \Longrightarrow B) \wedge (A \Longrightarrow C) \vdash
			A \Longrightarrow (B \wedge C)
		\end{align}
		が成り立つ.ゆえに
		\begin{align}
			\vdash ((A \Longrightarrow B) \wedge (A \Longrightarrow C))
			\Longrightarrow (A \Longrightarrow (B \wedge C))
		\end{align}
		が得られる.
		\QED
	\end{prf}
	
	\begin{screen}
		\begin{logicalthm}[含意は遺伝する]\label{logicalthm:rule_of_inference_1}
			$A,B,C$を$\mathcal{L}'$の閉式とするとき以下が成り立つ:
			\begin{description}
				\item[(a)] $(A \Longrightarrow B) \Longrightarrow ( (A \vee C) \Longrightarrow (B \vee C) )$.
				
				\item[(b)] $(A \Longrightarrow B) \Longrightarrow ( (A \wedge C) \Longrightarrow (B \wedge C) )$.
				
				\item[(c)] $(A \Longrightarrow B) \Longrightarrow ( (B \Longrightarrow C) \Longrightarrow (A \Longrightarrow C) )$.
				
				\item[(c)] $(A \Longrightarrow B) \Longrightarrow ( (C \Longrightarrow A) \Longrightarrow (C \Longrightarrow B) )$.
			\end{description}
		\end{logicalthm}
	\end{screen}
	
	\begin{prf}\mbox{}
		\begin{description}
			\item[(a)]
				いま$A \Longrightarrow B$が成り立っていると仮定する.
				論理和の導入により
				\begin{align}
					C \Longrightarrow (B \vee C)
				\end{align}
				は定理であるから,含意の推移律より
				\begin{align}
					A \Longrightarrow (B \vee C)
				\end{align}
				が従い,場合分け法則より
				\begin{align}
					(A \vee C) \Longrightarrow (B \vee C)
				\end{align}
				が成立する.ここに演繹法則を適用して
				\begin{align}
					(A \Longrightarrow B) \Longrightarrow 
					( (A \vee C) \Longrightarrow (B \vee C) )
				\end{align}
				が得られる.
				
			\item[(b)]
				いま$A \Longrightarrow B$が成り立っていると仮定する.論理積の除去より
				\begin{align}
					(A \wedge C) \Longrightarrow A
				\end{align}
				は定理であるから,含意の推移律より
				\begin{align}
					(A \wedge C) \Longrightarrow B
				\end{align}
				が従い,他方で論理積の除去より
				\begin{align}
					(A \wedge C) \Longrightarrow C
				\end{align}
				も満たされる.そして推論法則\ref{logicalthm:conjunction_of_consequences}から
				\begin{align}
					(A \wedge C) \Longrightarrow (B \wedge C)
				\end{align}
				が成り立ち,演繹法則より
				\begin{align}
					(A \Longrightarrow B) \Longrightarrow ((A \wedge C) \Longrightarrow (B \wedge C))
				\end{align}
				が得られる.
				
			\item[(c)]
				いま$A \Longrightarrow B$,$B \Longrightarrow C$および
				$A$が成り立っていると仮定する.このとき三段論法より$B$が成り立つので再び三段論法より
				$C$が成立する.ゆえに演繹法則より$A \Longrightarrow B$と$B \Longrightarrow C$が
				成り立っている下で
				\begin{align}
					A \Longrightarrow C
				\end{align}
				が成立し,演繹法則を更に順次適用すれば
				\begin{align}
					(A \Longrightarrow B) \Longrightarrow ( (B \Longrightarrow C) \Longrightarrow (A \Longrightarrow C) )
				\end{align}
				が得られる.
				
			\item[(d)]
				いま$A \Longrightarrow B$,$C \Longrightarrow A$および
				$C$が成り立っていると仮定する.このとき三段論法より$A$が成り立つので再び三段論法より$B$が成立し,
				ここに演繹法則を適用すれば,$A \Longrightarrow B$と$C \Longrightarrow A$が成立している下で
				\begin{align}
					C \Longrightarrow B
				\end{align}
				が成立する.演繹法則を更に順次適用すれば
				\begin{align}
					(A \Longrightarrow B) \Longrightarrow ( (C \Longrightarrow A) \Longrightarrow (C \Longrightarrow B) )
				\end{align}
				が得られる.
				\QED
		\end{description}
	\end{prf}
	
	\begin{screen}
		\begin{logicalthm}[正しい式は仮定を選ばない]\label{logicalthm:rule_of_inference_2}
			$A,B$を$\mathcal{L}'$の閉式とするとき,
			$B \Longrightarrow (A \Longrightarrow B)$は定理である.
		\end{logicalthm}
	\end{screen}
	
	\begin{prf}
		$B$を公理に追加した場合,$A$を公理に追加しても$B$は真であるから,このとき
		\begin{align}
			A \Longrightarrow B
		\end{align}
		は定理となる.従って演繹法則より$B \Longrightarrow (A \Longrightarrow B)$は定理である.
		\QED
	\end{prf}
	
	$A$と$B$を$\mathcal{L}'$の式とするとき,
	\begin{align}
		(A \Longleftrightarrow B) \defarrow
		(A \Longrightarrow B \wedge B \Longrightarrow A)
	\end{align}
	により$\Longleftrightarrow$を定め,式`$A \Longleftrightarrow B$'を
	``$A$と$B$は{\bf 同値である}\index{どうち@同値}{\bf (equivalent)}''と翻訳する.
	
	\begin{screen}
		\begin{logicalthm}[同値記号の可換律]\label{logicalthm:commutative_law_of_equivalence}
			$A$と$B$を$\mathcal{L}'$の閉式とするとき
			\begin{align}
				(A \Longleftrightarrow B) \Longrightarrow (B \Longleftrightarrow A).
			\end{align}
		\end{logicalthm}
	\end{screen}
	
	\begin{sketch}
		$A \Longleftrightarrow B$が成り立っているならば,推論法則\ref{logicalthm:commutative_law_of_disjunction_and_conjunction}より
		\begin{align}
			B \Longrightarrow A \wedge A \Longrightarrow B
		\end{align}
		が成立する.すなわち
		\begin{align}
			B \Longleftrightarrow A
		\end{align}
		が成立する.そして演繹法則から
		\begin{align}
			(A \Longleftrightarrow B) \Longrightarrow (B \Longleftrightarrow A)
		\end{align}
		が成立する.
		\QED
	\end{sketch}
	
	\begin{screen}
		\begin{logicalthm}[同値記号の遺伝性質]\label{logicalthm:hereditary_of_equivalence}
			$A,B,C$を$\mathcal{L}'$の閉式とするとき以下の式が成り立つ:
			\begin{description}
				\item[(a)] $(A \Longleftrightarrow B) \Longrightarrow ((A \vee C) \Longleftrightarrow (B \vee C))$.
				\item[(b)] $(A \Longleftrightarrow B) \Longrightarrow ((A \wedge C) \Longleftrightarrow (B \wedge C))$.
				\item[(c)] $(A \Longleftrightarrow B) \Longrightarrow ((B \Longrightarrow C) \Longleftrightarrow (A \Longrightarrow C))$.
				
				\item[(d)] $(A \Longleftrightarrow B) \Longrightarrow ((C \Longrightarrow A) \Longleftrightarrow (C \Longrightarrow B))$.
			\end{description}
		\end{logicalthm}
	\end{screen}
	
	\begin{prf}
		まず(a)を示す.いま$A \Longleftrightarrow B$が成り立っていると仮定する.このとき$A \Longrightarrow B$と
		$B \Longrightarrow A$が共に成立し,他方で含意の遺伝性質より
		\begin{align}
			&(A \Longrightarrow B) \Longrightarrow ((A \vee C) \Longrightarrow (B \vee C)), \\
			&(B \Longrightarrow A) \Longrightarrow ((B \vee C) \Longrightarrow (A \vee C))
		\end{align}
		が成立するから三段論法より$(A \vee C) \Longrightarrow (B \vee C)$と
		$(B \vee C) \Longrightarrow (A \vee C)$が共に成立する.ここに$\wedge$の導入を適用すれば
		\begin{align}
			(A \vee C) \Longleftrightarrow (B \vee C)
		\end{align}
		が成立し,演繹法則を適用すれば
		\begin{align}
			(A \Longleftrightarrow B) \Longrightarrow ((A \vee C) \Longleftrightarrow (B \vee C))
		\end{align}
		が得られる.(b)(c)(d)も含意の遺伝性を適用すれば得られる.
		\QED
	\end{prf}
	
	\begin{screen}
		\begin{logicalaxm}[矛盾と否定に関する規則]\label{logicalaxm:rules_of_contradiction}
			$A$を$\mathcal{L}'$の閉式とするとき以下の式が成り立つ:
			\begin{description}
				\item[矛盾の発生] 否定が共に成り立つとき矛盾が導かれる:
					\begin{align}
						(A \wedge \rightharpoondown A) \Longrightarrow \bot.
					\end{align}
				\item[否定の導出] 矛盾が導かれるとき否定が成り立つ:
					\begin{align}
						(A \Longrightarrow \bot) \Longrightarrow\ \rightharpoondown A.
					\end{align}
				\item[二重否定の法則] 二重に否定された式は元の式を導く:
					\begin{align}
						\rightharpoondown \rightharpoondown A \Longrightarrow A.
					\end{align}
			\end{description}
		\end{logicalaxm}
	\end{screen}
	
	\monologue{
		$A$を$\mathcal{L}'$の閉式とするとき,式$A \Longrightarrow \bot$を
		``$A$は{\bf 偽である}\index{ぎ@偽}{\bf (false)}''と翻訳します.
	}
	
	否定の導出の逆は定理として得られる.
	\begin{screen}
		\begin{logicalthm}[否定が正しい式は偽である]\label{logicalthm:false_and_negation_are_equivalent}
			$A$を$\mathcal{L}'$の閉式とするとき次が成り立つ:
			\begin{align}
				\rightharpoondown A \Longrightarrow (A \Longrightarrow \bot).
			\end{align}
		\end{logicalthm}
	\end{screen}
	
	\begin{prf}
		$\rightharpoondown A$が成り立っていると仮定する.このとき$A$が成り立っていれば
		推論規則\ref{logicalaxm:rules_of_contradiction}より$\bot$が成立するから,演繹法則より
		\begin{align}
			\rightharpoondown A \Longrightarrow (A \Longrightarrow \bot)
		\end{align}
		が成り立つ.
		\QED
	\end{prf}
	
	\begin{screen}
		\begin{logicalthm}[矛盾からはあらゆる式が導かれる]\label{logicalthm:contradiction_derives_any_formula}
			$A$を$\mathcal{L}'$の閉式とするとき
			\begin{align}
				\bot \Longrightarrow A.
			\end{align}
		\end{logicalthm}
	\end{screen}
	
	\begin{prf}
		推論法則\ref{logicalthm:rule_of_inference_2}より
		\begin{align}
			\bot \Longrightarrow (\rightharpoondown A \Longrightarrow \bot)
		\end{align}
		が成り立つ.また否定の導出より
		\begin{align}
			(\rightharpoondown A \Longrightarrow \bot) \Longrightarrow\ \rightharpoondown \rightharpoondown A
		\end{align}
		も成り立ち,さらに二重否定の法則から
		\begin{align}
			\rightharpoondown \rightharpoondown A \Longrightarrow A
		\end{align}
		も成り立つ.上の式に含意の推移律を適用すれば
		\begin{align}
			\bot \Longrightarrow A
		\end{align}
		が得られる.
		\QED
	\end{prf}
	
	\begin{screen}
		\begin{logicalthm}[背理法の原理]
			$A$を$\mathcal{L}'$の閉式とするとき
			\begin{align}
				(\rightharpoondown A \Longrightarrow \bot) \Longrightarrow A.
			\end{align}
		\end{logicalthm}
	\end{screen}
	
	\begin{prf}
		$\rightharpoondown A \Longrightarrow \bot$が成り立つとき,否定の導出より
		$\rightharpoondown \rightharpoondown A$が成り立つが,二重否定の法則より
		$A$も成立する.
		\QED
	\end{prf}
	
	\begin{screen}
		\begin{logicalthm}[矛盾を導く式はあらゆる式を導く]\label{logicalthm:formula_leading_to_contradiction_derives_any_formula}
			$A,B$を$\mathcal{L}'$の閉式とするとき,次が成り立つ:
			\begin{align}
				(A \Longrightarrow \bot) \Longrightarrow (A \Longrightarrow B).
			\end{align}
		\end{logicalthm}
	\end{screen}
	
	\begin{prf}
		$A \Longrightarrow \bot$が成り立っているとする.推論法則\ref{logicalthm:contradiction_derives_any_formula}より
		\begin{align}
			\bot \Longrightarrow B
		\end{align}
		が満たされるので,含意の推移律より
		\begin{align}
			A \Longrightarrow B
		\end{align}
		が成り立つ.従って演繹法則を適用すれば
		\begin{align}
			(A \Longrightarrow \bot) \Longrightarrow (A \Longrightarrow B)
		\end{align}
		が得られる.
		\QED
	\end{prf}
	
	\begin{screen}
		\begin{logicalthm}[含意は否定と論理和で表せる]\label{logicalthm:rule_of_inference_3}
			$A,B$を$\mathcal{L}'$の閉式とするとき,次が成り立つ:
			\begin{align}
				(A \Longrightarrow B) \Longleftrightarrow (\rightharpoondown A \vee B).
			\end{align}
		\end{logicalthm}
	\end{screen}
	
	\begin{prf}
		$A \Longrightarrow B$が成り立っていると仮定する.含意の遺伝性質より
		\begin{align}
			(A \Longrightarrow B) \Longrightarrow 
			((A \vee \rightharpoondown A) \Longrightarrow (B \vee \rightharpoondown A))
		\end{align}
		が満たされているから三段論法より
		\begin{align}
			(A \vee \rightharpoondown A) \Longrightarrow (B \vee \rightharpoondown A)
		\end{align}
		は定理となり,ここに排中律と三段論法を適用すれば
		\begin{align}
			B \vee \rightharpoondown A
		\end{align}
		が定理となる.
		ここで論理和の可換律より$\rightharpoondown A \vee B$が成り立つので,演繹法則を適用して
		\begin{align}
			(A \Longrightarrow B) \Longrightarrow (\rightharpoondown A \vee B)
		\end{align}
		が得られる.また矛盾に関する推論規則より
		\begin{align}
			\rightharpoondown A \Longrightarrow (A \Longrightarrow \bot)
		\end{align}
		が成り立ち,同時に推論法則\ref{logicalthm:formula_leading_to_contradiction_derives_any_formula}より
		\begin{align}
			(A \Longrightarrow \bot) \Longrightarrow (A \Longrightarrow B)
		\end{align}
		も成り立つので,含意の推移律より
		\begin{align}
			\rightharpoondown A \Longrightarrow (A \Longrightarrow B)
		\end{align}
		が成立する.他方で推論法則\ref{logicalthm:rule_of_inference_2}より
		\begin{align}
			B \Longrightarrow (A \Longrightarrow B)
		\end{align}
		も成り立つから,場合分けの法則より
		\begin{align}
			(\rightharpoondown A \vee B) \Longrightarrow (A \Longrightarrow B)
		\end{align}
		が成り立つ.以上で$(A \Longrightarrow B) \Longleftrightarrow (\rightharpoondown A \vee B)$が得られた.
		\QED
	\end{prf}
	
	\monologue{
		$A,B$を$\mathcal{L}'$の閉式とするとき,$A$が偽であれば$\rightharpoondown A$が成立する
		(推論規則\ref{logicalaxm:rules_of_contradiction})ので
		$\rightharpoondown A \vee B$が成立します(推論規則\ref{logicalaxm:fundamental_rules_of_inference}).
		すなわちこのとき$A \Longrightarrow B$が成り立つのですが,式の解釈としては
		``偽な式からはあらゆる式が導かれる''となりますね.この現象を
		{\bf 空虚な真}\index{くうきょなしん@空虚な真}{\bf (vacuous truth)}と呼びます.
	}
	
	\begin{screen}
		\begin{logicalthm}[二重否定の法則の逆が成り立つ]
			$A$を$\mathcal{L}'$の閉式とするとき,次が成り立つ:
			\begin{align}
				A \Longrightarrow\ \rightharpoondown \rightharpoondown A.
			\end{align}
		\end{logicalthm}
	\end{screen}
	
	\begin{prf}
		排中律より
		\begin{align}
			\rightharpoondown A \vee \rightharpoondown \rightharpoondown A
		\end{align}
		が成立し,また推論法則\ref{logicalthm:rule_of_inference_3}より
		\begin{align}
			(\rightharpoondown A \vee \rightharpoondown \rightharpoondown A)
			\Longrightarrow (A \Longrightarrow\ \rightharpoondown \rightharpoondown A)
		\end{align}
		も成り立つので,三段論法より
		\begin{align}
			A \Longrightarrow\ \rightharpoondown \rightharpoondown A
		\end{align}
		が成立する.
		\QED
	\end{prf}
	
	\begin{screen}
		\begin{logicalthm}[対偶命題は同値]\label{thm:contraposition_is_true}
			$A,B$を$\mathcal{L}'$の閉式とするとき,次が成り立つ:
			\begin{align}
				(A \Longrightarrow B) \Longleftrightarrow (\rightharpoondown B \Longrightarrow\ \rightharpoondown A).
			\end{align}
		\end{logicalthm}
	\end{screen}
	
	\begin{prf}
		推論法則\ref{logicalthm:rule_of_inference_3},論理和の可換律,二重否定の法則(とその逆)を順に用いれば
		\begin{align}
			(A \Longrightarrow B) &\Longleftrightarrow (\rightharpoondown A \vee B) \\
			&\Longleftrightarrow (B \vee \rightharpoondown A) \\
			&\Longleftrightarrow (\rightharpoondown \rightharpoondown B \vee \rightharpoondown A) \\
			&\Longleftrightarrow (\rightharpoondown B \Longrightarrow\ \rightharpoondown A)
		\end{align}
		が成り立つ.
		\QED
	\end{prf}
	
	\monologue{
		対偶命題を述べるときには``対偶を取る''と表現することが多いです.
	}
	
	\begin{screen}
		\begin{logicalthm}[De Morganの法則]
			$A,B$を$\mathcal{L}'$の閉式とするとき,次が成り立つ:
			\begin{itemize}
				\item $\rightharpoondown (A \vee B) \Longleftrightarrow\ \rightharpoondown A \wedge \rightharpoondown B$.
			
				\item $\rightharpoondown (A \wedge B) \Longleftrightarrow\ \rightharpoondown A \vee \rightharpoondown B$.
			\end{itemize}
		\end{logicalthm}
	\end{screen}
	
	\begin{prf}
		$A \Longrightarrow (A \vee B)$は定理であるから,その対偶命題
		\begin{align}
			\rightharpoondown (A \vee B) \Longrightarrow\ \rightharpoondown A
		\end{align}
		も定理となる.同様に$\rightharpoondown (A \vee B) \Longrightarrow\ \rightharpoondown B$は定理となるので,
		$\rightharpoondown (A \vee B)$が成り立っていると仮定すれば$\rightharpoondown A \wedge \rightharpoondown B$が成り立つ.
		ゆえに
		\begin{align}
			\rightharpoondown (A \vee B) \Longrightarrow\ \rightharpoondown A \wedge \rightharpoondown B
		\end{align}
		が得られる.また$A$が成り立っていると仮定すれば,この下で$\rightharpoondown A \wedge \rightharpoondown B$が成り立っているなら
		$A$と$\rightharpoondown A$が同時に成り立つことになるので$\bot$が成立する.つまり
		$A$が成り立っているとき
		\begin{align}
			\rightharpoondown A \wedge \rightharpoondown B \Longrightarrow \bot
		\end{align}
		が成り立つが,このとき$\rightharpoondown(\rightharpoondown A \wedge \rightharpoondown B)$が成り立つので
		\begin{align}
			A \Longrightarrow\ \rightharpoondown(\rightharpoondown A \wedge \rightharpoondown B)
		\end{align}
		が得られる.同様にして
		\begin{align}
			B \Longrightarrow\ \rightharpoondown(\rightharpoondown A \wedge \rightharpoondown B)
		\end{align}
		も得られるから,場合分け法則より
		\begin{align}
			(A \vee B) \Longrightarrow\ \rightharpoondown(\rightharpoondown A \wedge \rightharpoondown B)
		\end{align}
		が成立する.この対偶を取れば
		\begin{align}
			\rightharpoondown A \wedge \rightharpoondown B
			\Longrightarrow\ \rightharpoondown (A \vee B)
		\end{align}
		が出る.以上で一つ目の式が示された.一つ目の式で$A$を$\rightharpoondown A$に,
		$B$を$\rightharpoondown B$に置き換えると
		\begin{align}
			\rightharpoondown \rightharpoondown A \wedge \rightharpoondown \rightharpoondown B
			\Longleftrightarrow\ \rightharpoondown (\rightharpoondown A \vee \rightharpoondown B)
		\end{align}
		が得られるが,このとき二重否定の法則より
		\begin{align}
			A \wedge B
			\Longleftrightarrow\ \rightharpoondown (\rightharpoondown A \vee \rightharpoondown B)
		\end{align}
		が成立し,対偶命題の同値性から
		\begin{align}
			\rightharpoondown (A \wedge B)
			\Longleftrightarrow\ (\rightharpoondown A \vee \rightharpoondown B)
		\end{align}
		は定理となる.
		\QED
	\end{prf}
	
	\monologue{
		以上で``集合であり真類でもある類は存在しない''という言明を証明する準備が整いました.
	}
	
	\begin{screen}
		\begin{thm}[集合であり真類でもある類は存在しない]
			$a$を類とするとき次が成り立つ:
			\begin{align}
				\rightharpoondown (\ \set{a} \wedge \rightharpoondown \set{a}\ ).
			\end{align}
		\end{thm}
	\end{screen}
	
	\begin{prf}
		$a$を類とするとき,排中律より$\set{a} \vee \rightharpoondown \set{a}$
		が成り立ち,論理和の可換律より
		\begin{align}
			\rightharpoondown \set{a} \vee \set{a}
		\end{align}
		も成立する.そしてDe Morganの法則より
		\begin{align}
			\rightharpoondown (\ \rightharpoondown \rightharpoondown \set{a} \wedge \rightharpoondown \set{a}\ )
		\end{align}
		が成り立つが,二重否定の法則より$\rightharpoondown \rightharpoondown \set{a}$と
		$\set{a}$は同値となるので
		\begin{align}
			\rightharpoondown (\ \set{a} \wedge \rightharpoondown \set{a}\ )
		\end{align}
		が成り立つ.
		\QED
	\end{prf}
	
	\monologue{
		次は量化記号が推論操作の上でどのような働きを持つのかを規定しましょう.
	}
	
	\begin{screen}
		\begin{logicalaxm}[量化記号に関する規則]\label{logicalaxm:rules_of_quantifiers}
			$A$を$\mathcal{L}'$の式とし,$x$を$A$に現れる文字とするとき,$x$のみが$A$で量化されていないならば以下を認める:
			\begin{description}
				\item[$\varepsilon$記号の導入] $\varepsilon x A(x)$は$\mathcal{L}$の或る対象に代用される.
				\item[存在記号の規則] $A (\varepsilon x A(x)) \Longleftrightarrow \exists x A(x)$が成り立つ.
				\item[全称記号の規則] $A (\varepsilon x \rightharpoondown A(x)) \Longleftrightarrow \forall x A(x)$が成り立つ.
				\item[存在記号の基本性質] $\tau$を$\mathcal{L}$の対象とするとき
					$A(\tau) \Longrightarrow \exists x A(x)$が成り立つ.
			\end{description}
		\end{logicalaxm}
	\end{screen}
	
	$\varepsilon$記号はHilbertのイプシロン関数と呼ばれるもので,
	量化記号の働きを形式的に表現するには簡便かつ有能である.
	また$\varepsilon$記号が指定する対象を$\mathcal{L}$のものと約束することで,
	$\exists$と$\forall$の作用範囲を$\mathcal{L}$の対象全体に制限している.
	
	\begin{screen}
		\begin{logicalthm}[全称記号と任意性]\label{logicalthm:fundamental_law_of_universal_quantifier}
			$A$を$\mathcal{L}'$の式とし,$x$を$A$に現れる文字とし,$x$のみが$A$で量化されていないとする.このとき
			$\forall x A(x)$が成り立つならば$\mathcal{L}$のいかなる対象$\tau$に対しても$ A(\tau)$が成り立つ.
			逆に,$\mathcal{L}$のいかなる対象$\tau$に対しても$A(\tau)$が成り立てば$\forall x A(x)$が成り立つ.
		\end{logicalthm}
	\end{screen}
	
	\begin{prf}
		$\tau$を$\mathcal{L}$の任意の対象とすれば,存在記号に関する推論規則より
		\begin{align}
			\rightharpoondown A(\tau) \Longrightarrow\ \exists x \rightharpoondown A(x)
		\end{align}
		と
		\begin{align}
			\exists x \rightharpoondown A(x) \Longrightarrow\ \rightharpoondown A
			\left( \varepsilon x \rightharpoondown A(x) \right)
		\end{align}
		が成り立つから,推論法則\ref{logicalthm:transitive_law_of_implication}より
		\begin{align}
			\rightharpoondown A(\tau) \Longrightarrow\ \rightharpoondown A
			\left( \varepsilon x \rightharpoondown A(x) \right)
		\end{align}
		が成り立ち,対偶を取って
		\begin{align}
			A \left( \varepsilon x \rightharpoondown A(x) \right)
			\Longrightarrow A(\tau)
		\end{align}
		が成り立つ.全称記号に関する推論規則より
		\begin{align}
			\forall x A(x) \Longrightarrow A \left( \varepsilon x \rightharpoondown A(x) \right)
		\end{align}
		が満たされているので
		\begin{align}
			\forall x A(x) \Longrightarrow A(\tau)
		\end{align}
		が従う.逆にいかなる対象$\tau$に対しても$A(\tau)$が成り立つとき,特に
		\begin{align}
			A \left( \varepsilon x \rightharpoondown A(x) \right)
		\end{align}
		が成り立つので$\forall x A(x)$も成り立つ.
		\QED
	\end{prf}
	
	\monologue{
		推論法則\ref{logicalthm:fundamental_law_of_universal_quantifier}を根拠にして,
		当面は$\forall x A(x)$という式を``$\mathcal{L}$の任意の対象$x$に対して
		$A(x)$が成立する''と翻訳することにします.また後述する相等性の公理によれば,
		これは``任意の集合$x$に対して$A(x)$が成立する''と翻訳しても同義です.
	}
	
	\begin{screen}
		\begin{logicalthm}[量化記号の性質(イ)]\label{logicalthm:properties_of_quantifiers}
			$A,B$を$\mathcal{L}'$の式とし,$x$を$A,B$に現れる文字とし,$x$のみが$A,B$で量化されていないとする.
			$\mathcal{L}$の任意の対象$\tau$に対して
			\begin{align}
				A(\tau) \Longleftrightarrow B(\tau)
			\end{align}
			が成り立っているとき,
			\begin{align}
				\exists x A(x) \Longleftrightarrow \exists x B(x)
			\end{align}
			および
			\begin{align}
				\forall x A(x) \Longleftrightarrow \forall x B(x)
			\end{align}
			が成り立つ.
		\end{logicalthm}
	\end{screen}
	
	\begin{prf}
		いま,$\mathcal{L}$の任意の対象$\tau$に対して
		\begin{align}
			A(\tau) \Longleftrightarrow B(\tau)
			\label{logicalthm:properties_of_quantifiers_1}
		\end{align}
		が成り立っているとする.
		ここで
		\begin{align}
			\exists x A(x)
		\end{align}
		が成り立っていると仮定すると,
		\begin{align}
			\tau \defeq \varepsilon x A(x)
		\end{align}
		とおけば存在記号に関する規則より
		\begin{align}
			A(\tau)
		\end{align}
		が成立し,(\refeq{logicalthm:properties_of_quantifiers_1})と併せて
		\begin{align}
			B(\tau)
		\end{align}
		が成立する.再び存在記号に関する規則より
		\begin{align}
			\exists x B(x)
		\end{align}
		が成り立つので,演繹法則から
		\begin{align}
			\exists x A(x) \Longrightarrow \exists x B(x)
		\end{align}
		が得られる.$A$と$B$の立場を入れ替えれば
		\begin{align}
			\exists x B(x) \Longrightarrow \exists x A(x)
		\end{align}
		も得られる.今度は
		\begin{align}
			\forall x A(x)
		\end{align}
		が成り立っていると仮定すると,
		推論法則\ref{logicalthm:fundamental_law_of_universal_quantifier}より
		$\mathcal{L}$の任意の対象$\tau$に対して
		\begin{align}
			A(\tau)
		\end{align}
		が成立し,(\refeq{logicalthm:properties_of_quantifiers_1})と併せて
		\begin{align}
			B(\tau)
		\end{align}
		が成立する.$\tau$の任意性と推論法則\ref{logicalthm:fundamental_law_of_universal_quantifier}より
		\begin{align}
			\forall x B(x)
		\end{align}
		が成り立つので,演繹法則から
		\begin{align}
			\forall x A(x) \Longrightarrow \forall x B(x)
		\end{align}
		が得られる.$A$と$B$の立場を入れ替えれば
		\begin{align}
			\forall x B(x) \Longrightarrow \forall x A(x)
		\end{align}
		も得られる.
		\QED
	\end{prf}
	
	\begin{screen}
		\begin{logicalthm}[量化記号に対する De Morgan の法則]\label{logicalthm:De_Morgan_law_for_quantifiers}
			$A$を$\mathcal{L}'$の式とし,$x$を$A$に現れる文字とし,$x$のみが$A$で量化されていないとする.このとき
			\begin{align}
				\exists x \rightharpoondown A(x) \Longleftrightarrow\ \rightharpoondown \forall x A(x)
			\end{align}
			および
			\begin{align}
				\forall x \rightharpoondown A(x) \Longleftrightarrow\ \rightharpoondown \exists x A(x)
			\end{align}
			が成り立つ.
		\end{logicalthm}
	\end{screen}
	
	\begin{sketch}
		推論規則\ref{logicalaxm:rules_of_quantifiers}より
		\begin{align}
			\exists x \rightharpoondown A(x) \Longleftrightarrow\ 
			\rightharpoondown A(\varepsilon x \rightharpoondown A(x))
		\end{align}
		は定理である.他方で推論規則\ref{logicalaxm:rules_of_quantifiers}より
		\begin{align}
			A(\varepsilon x \rightharpoondown A(x)) \Longleftrightarrow \forall x A(x) 
		\end{align}
		もまた定理であり,この対偶を取れば
		\begin{align}
			\rightharpoondown A(\varepsilon x \rightharpoondown A(x)) \Longleftrightarrow\ 
			\rightharpoondown \forall x A(x)
		\end{align}
		が成り立つ.ゆえに
		\begin{align}
			\exists x \rightharpoondown A(x) \Longleftrightarrow\ \rightharpoondown \forall x A(x)
		\end{align}
		が従う.$A$を$\rightharpoondown A$に置き換えれば
		\begin{align}
			\forall x \rightharpoondown A(x) \Longleftrightarrow\ 
			\rightharpoondown \exists x \rightharpoondown \rightharpoondown A(x)
		\end{align}
		が成り立ち,また$\mathcal{L}$の任意の対象$\tau$に対して
		\begin{align}
			A(\tau) \Longleftrightarrow\ \rightharpoondown \rightharpoondown A(\tau)
		\end{align}
		が成り立つので,推論法則\ref{logicalthm:properties_of_quantifiers}より
		\begin{align}
			\exists x \rightharpoondown \rightharpoondown A(x)
			\Longleftrightarrow \exists x A(x)
		\end{align}
		も成り立つ.ゆえに
		\begin{align}
			\forall x \rightharpoondown A(x) \Longleftrightarrow\ 
			\rightharpoondown \exists x A(x)
		\end{align}
		が従う.
		\QED
	\end{sketch}
		\begin{screen}
		\begin{logicalaxm}[二重否定の除去]
		\label{logicalaxm:elimination_of_double_negation}
			$A$を文とするとき以下が成り立つ:
			\begin{align}
				\negation \negation A \rarrow A.
			\end{align}
		\end{logicalaxm}
	\end{screen}
	
	\begin{screen}
		\begin{logicalthm}[対偶律3]\label{logicalthm:contraposition_3}
			$A$と$B$を文とするとき
			\begin{align}
				\vdash (\, \negation A \rarrow B\, )
				\rarrow (\, \negation B \rarrow A\, ).
			\end{align}
		\end{logicalthm}
	\end{screen}
	
	\begin{sketch}
		対偶律1 (論理的定理\ref{logicalthm:introduction_of_contraposition})より
		\begin{align}
			\negation A \rarrow B \vdash\ \negation B \rarrow\ \negation \negation A
		\end{align}
		が成り立つので,演繹定理の逆より
		\begin{align}
			\negation B,\ \negation A \rarrow B \vdash\ \negation \negation A
		\end{align}
		となる.二重否定の除去より
		\begin{align}
			\vdash \negation \negation A \rarrow A
		\end{align}
		が成り立つので,三段論法より
		\begin{align}
			\negation B,\ \negation A \rarrow B \vdash A
		\end{align}
		が従い,演繹定理より
		\begin{align}
			\negation A \rarrow B \vdash\ \negation B \rarrow A
		\end{align}
		が得られる.
		\QED
	\end{sketch}
	
	\begin{screen}
		\begin{logicalthm}[対偶律4]
		\label{logicalthm:proof_by_contraposition}
			$A$と$B$を文とするとき
			\begin{align}
				\vdash (\, \negation B \rarrow\ \negation A\, )
				\rarrow (\, A \rarrow B\, ).
			\end{align}
		\end{logicalthm}
	\end{screen}
	
	\begin{prf}
		二重否定の導入(論理的定理\ref{logicalthm:introduction_of_double_negation})より
		\begin{align}
			\negation B \rarrow\ \negation A \vdash 
			A \rarrow\ \negation \negation A
		\end{align}
		が成り立つので,演繹定理の逆より
		\begin{align}
			A,\ \negation B \rarrow\ \negation A \vdash\ \negation \negation A
		\end{align}
		となる.また$\negation B \rarrow\ \negation A$の対偶を取れば
		\begin{align}
			A,\ \negation B \rarrow\ \negation A \vdash\ 
			\negation \negation A \rarrow\ \negation \negation B
		\end{align}
		が成り立つので(論理的定理\ref{logicalthm:introduction_of_contraposition}),
		三段論法より
		\begin{align}
			A,\ \negation B \rarrow\ \negation A \vdash\ \negation \negation B
		\end{align}
		となる.ここで二重否定の除去より
		\begin{align}
			A,\ \negation B \rarrow\ \negation A \vdash\ 
			\negation \negation B \rarrow B
		\end{align}
		となるので,三段論法より
		\begin{align}
			A,\ \negation B \rarrow\ \negation A \vdash B
		\end{align}
		が従い,演繹定理より
		\begin{align}
			\negation B \rarrow\ \negation A &\vdash A \rarrow B, \\
			&\vdash (\, \negation B \rarrow\ \negation A\, ) 
			\rarrow (\, A \rarrow B\, )
		\end{align}
		が得られる.
		\QED
	\end{prf}
	
	\begin{screen}
		\begin{logicalthm}[背理法の原理]
		\label{logicalthm:proof_by_contradiction}
			$A$を文とするとき
			\begin{align}
				\vdash (\, \negation A \rarrow \bot\, ) \rarrow A.
			\end{align}
		\end{logicalthm}
	\end{screen}
	
	\begin{prf}
		否定の導入より
		\begin{align}
			\negation A \rarrow \bot \vdash\ \negation \negation A
		\end{align}
		が成り立ち,二重否定の法則より
		\begin{align}
			\negation A \rarrow \bot \vdash\ \negation \negation A \rarrow A
		\end{align}
		が成り立つので,三段論法より
		\begin{align}
			\negation A \rarrow \bot \vdash A
		\end{align}
		となる.そして演繹定理より
		\begin{align}
			\vdash (\, \negation A \rarrow \bot\, ) \rarrow A
		\end{align}
		が得られる.
		\QED
	\end{prf}
	
	次の{\bf 爆発律}\index{ばくはつりつ@爆発律}{\bf (principle of explosion)}
	とは「矛盾からはあらゆる式が導かれる」ことを表している.
	またなぜ$\bot$が「矛盾」と呼ばれるのかが明確になる.
	実際,公理系$\mathscr{S}$からひとたび$\bot$が導かれれば,爆発律との三段論法によって
	どんな式でも$\mathscr{S}$の定理となる.すると$\mathscr{S}$においては
	$A$とその否定$\negation A$など食い違う結論が共に定理となってしまい,
	まさしく``矛盾''が引き起こされるのである.
	
	\begin{screen}
		\begin{logicalthm}[爆発律]
		\label{logicalthm:principle_of_explosion}
			$A$を文とするとき
			\begin{align}
				\vdash \bot \rarrow A.
			\end{align}
		\end{logicalthm}
	\end{screen}
	
	\begin{prf}
		含意の導入より
		\begin{align}
			\vdash \bot \rarrow (\, \negation A \rarrow \bot\, )
		\end{align}
		が成り立つので,演繹定理の逆より
		\begin{align}
			\bot \vdash\ \negation A \rarrow \bot
		\end{align}
		となる.また背理法の原理(論理的定理\ref{logicalthm:proof_by_contradiction})より
		\begin{align}
			\bot \vdash (\, \negation A \rarrow \bot\, ) \rarrow A
		\end{align}
		が成り立つので,三段論法より
		\begin{align}
			\bot \vdash A
		\end{align}
		が従い,演繹定理より
		\begin{align}
			\vdash \bot \rarrow A
		\end{align}
		が得られる.
		\QED
	\end{prf}
	
	\begin{screen}
		\begin{logicalthm}[否定の論理和は含意で書ける]
		\label{logicalthm:disjunction_of_negation_rewritable_by_implication}
			$A$と$B$を文とするとき
			\begin{align}
				\vdash (\, \negation A \vee B\, ) \rarrow (\, A \rarrow B\, ).
			\end{align}
		\end{logicalthm}
	\end{screen}
	
	\begin{prf}
		矛盾の導入より
		\begin{align}
			A,\ \negation A \vdash \bot
		\end{align}
		が成り立ち,爆発律(論理的定理\ref{logicalthm:principle_of_explosion})より
		\begin{align}
			A,\ \negation A \vdash \bot \rarrow B
		\end{align}
		が成り立つので,三段論法より
		\begin{align}
			A,\ \negation A \vdash B
		\end{align}
		が従い,演繹定理より
		\begin{align}
			\vdash\ \negation A \rarrow (\, A \rarrow B\, )
			\label{fom:disjunction_of_negation_rewritable_by_implication_1}
		\end{align}
		が得られる.また含意の導入より
		\begin{align}
			\vdash B \rarrow (\, A \rarrow B\, )
			\label{fom:disjunction_of_negation_rewritable_by_implication_2}
		\end{align}
		も得られる.ところで論理和の除去より
		\begin{align}
			\vdash (\, \negation A \rarrow (\, A \rarrow B\, )\, )
			\rarrow (\, (\, B \rarrow (\, A \rarrow B\, )\, )
			\rarrow (\, \negation A \vee B \rarrow (\, A \rarrow B\, )\, )\, )
		\end{align}
		が成り立つので,(\refeq{fom:disjunction_of_negation_rewritable_by_implication_1})
		との三段論法より
		\begin{align}
			\vdash (\, B \rarrow (\, A \rarrow B\, )\, )
			\rarrow (\, \negation A \vee B \rarrow (\, A \rarrow B\, )\, )
		\end{align}
		となり,(\refeq{fom:disjunction_of_negation_rewritable_by_implication_2})
		との三段論法より
		\begin{align}
			\vdash\ \negation A \vee B \rarrow (\, A \rarrow B\, )
		\end{align}
		が得られる.
		\QED
	\end{prf}
	
	\begin{screen}
		\begin{thm}[驚嘆すべき帰結]\label{logicalthm:consequentia_mirabilis}
			$A$を文とするとき
			\begin{align}
				\vdash (\, \negation A \rarrow A\, ) \rarrow A.
			\end{align}
		\end{thm}
	\end{screen}
	
	\begin{sketch}
		三段論法より
		\begin{align}
			\negation A,\ \negation A \rarrow A \vdash A
		\end{align}
		が成り立ち,他方で矛盾の導入(CTD1)より
		\begin{align}
			\negation A,\ \negation A \rarrow A
			\vdash A \rarrow (\, \negation A \rarrow \bot\, )
		\end{align}
		も成り立つので,三段論法より
		\begin{align}
			\negation A,\ \negation A \rarrow A \vdash\ \negation A \rarrow \bot
		\end{align}
		が従う.
		\begin{align}
			\negation A,\ \negation A \rarrow A \vdash\ \negation A
		\end{align}
		との三段論法より
		\begin{align}
			\negation A,\ \negation A \rarrow A \vdash \bot
		\end{align}
		となり,演繹定理より
		\begin{align}
			\negation A \rarrow A \vdash\ \negation A \rarrow \bot
		\end{align}
		が従う.背理法の原理(論理的定理\ref{logicalthm:proof_by_contradiction})より
		\begin{align}
			\negation A \rarrow A \vdash (\, \negation A \rarrow \bot\, )
			\rarrow A
		\end{align}
		が成り立つので三段論法より
		\begin{align}
			\negation A \rarrow A \vdash A
		\end{align}
		となり,演繹定理より
		\begin{align}
			\vdash (\, \negation A \rarrow A\, ) \rarrow A
		\end{align}
		が得られる.
		\QED
	\end{sketch}
	
	\begin{screen}
		\begin{logicalthm}[排中律]\label{logicalthm:law_of_excluded_middle}
			$A$を文とするとき
			\begin{align}
				\vdash A \vee \negation A.
			\end{align}
		\end{logicalthm}
	\end{screen}
	
	\begin{prf}
		論理和の導入より
		\begin{align}
			A \vdash A \vee \negation A
		\end{align}
		となり,他方で矛盾の導入より
		\begin{align}
			A \vdash (\, A \vee \negation A\, )
			\rarrow (\, \negation (\, A \vee \negation A\, ) \rarrow \bot\, )
		\end{align}
		も成り立つので三段論法より
		\begin{align}
			A \vdash\ \negation (\, A \vee \negation A\, ) \rarrow \bot
		\end{align}
		が従う.演繹定理の逆より
		\begin{align}
			\negation (\, A \vee \negation A\, ),\ A \vdash \bot
		\end{align}
		となり,演繹定理より
		\begin{align}
			\negation (\, A \vee \negation A\, ) \vdash A \rarrow \bot
		\end{align}
		となる.否定の導入より
		\begin{align}
			\negation (\, A \vee \negation A\, ) \vdash (\, A \rarrow \bot\, )
			\rarrow\ \negation A
		\end{align}
		が成り立つので三段論法より
		\begin{align}
			\negation (\, A \vee \negation A\, ) \vdash\ \negation A
		\end{align}
		が従う.論理和の導入より
		\begin{align}
			\negation (\, A \vee \negation A\, ) \vdash\ \negation A
			\rarrow A \vee \negation A
		\end{align}
		が成り立つので三段論法より
		\begin{align}
			\negation (\, A \vee \negation A\, ) \vdash A \vee \negation A
		\end{align}
		が従い,演繹定理より
		\begin{align}
			\vdash\ \negation (\, A \vee \negation A\, ) \rarrow A \vee \negation A
		\end{align}
		が成り立つ.驚嘆すべき帰結(論理的定理\ref{logicalthm:consequentia_mirabilis})より
		\begin{align}
			\vdash
			(\, \negation (\, A \vee \negation A\, ) \rarrow A \vee \negation A\, )
			\rarrow A \vee \negation A
		\end{align}
		が成り立つので三段論法より
		\begin{align}
			\vdash A \vee \negation A
		\end{align}
		が出る.
		\QED
	\end{prf}
	
	排中律の言明は「いかなる文も肯定か否定の一方は成り立つ」と読めるが,
	肯定と否定のどちらか一方が証明可能であるということを保証しているわけではない.
	無矛盾律についても似たようなことが言える.無矛盾律とは「肯定と否定は両立しない」と読めるわけだが,
	もしかすると,或る公理系$\mathscr{S}$の下では或る文$A$に対して
	\begin{align}
		\mathscr{S} \vdash A \wedge \negation A
	\end{align}
	が導かれるかもしれない.この場合$\mathscr{S}$は矛盾することになるが,
	予め$\mathscr{S}$が無矛盾であることが判っていない限りはこの事態が起こらないとは言い切れない
	(極端な例では,矛盾$\bot$が公理であっても無矛盾律は定理である).
	
	\begin{screen}
		\begin{logicalthm}[含意の論理和への遺伝性]
		\label{logicalthm:heredity_of_implication_to_disjunction}
			$A,B,C$を文とするとき
			\begin{align}
				\vdash (\, A \rarrow B\, ) \rarrow (\, A \vee C \rarrow B \vee C\, ), \\
				\vdash (\, A \rarrow B\, ) \rarrow (\, C \vee A \rarrow C \vee B\, ).
			\end{align}
		\end{logicalthm}
	\end{screen}
	
	\begin{sketch}
		三段論法より
		\begin{align}
			A,\ A \rarrow B \vdash B
		\end{align}
		が成り立ち,また論理和の導入より
		\begin{align}
			\vdash B \rarrow B \vee C
		\end{align}
		も成り立つので,三段論法より
		\begin{align}
			A,\ A \rarrow B \vdash B \vee C
		\end{align}
		となり,演繹定理より
		\begin{align}
			A \rarrow B \vdash A \rarrow B \vee C
			\label{fom:heredity_of_implication_to_disjunction_1}
		\end{align}
		が得られる.論理和の導入より
		\begin{align}
			\vdash C \rarrow B \vee C
			\label{fom:heredity_of_implication_to_disjunction_2}
		\end{align}
		も満たされている.ところで論理和の除去より
		\begin{align}
			A \rarrow B \vdash (\, A \rarrow B \vee C\, )
			\rarrow (\, (\, C \rarrow B \vee C\, )
			\rarrow (\, A \vee C \rarrow B \vee C\, )\, )
		\end{align}
		が成り立つので,(\refeq{fom:heredity_of_implication_to_disjunction_1})との三段論法より
		\begin{align}
			A \rarrow B \vdash (\, C \rarrow B \vee C\, )
			\rarrow (\, A \vee C \rarrow B \vee C\, )
		\end{align}
		となり,(\refeq{fom:heredity_of_implication_to_disjunction_2})との三段論法より
		\begin{align}
			A \rarrow B \vdash A \vee C \rarrow B \vee C
		\end{align}
		が従う.
		\begin{align}
			\vdash (\, A \rarrow B\, ) \rarrow (\, C \vee A \rarrow C \vee B\, )
		\end{align}
		も同様に示される.
		\QED
	\end{sketch}
	
	\begin{screen}
		\begin{logicalthm}[含意は否定と論理和で表せる]
		\label{logicalthm:implication_rewritable_by_disjunction_of_negation}
			$A$と$B$を文とするとき
			\begin{align}
				\vdash (\, A \rarrow B\, ) \rarrow (\, \negation A \vee B\, ).
			\end{align}
		\end{logicalthm}
	\end{screen}
	
	\begin{prf}
		含意の論理和への遺伝性
		(論理的定理\ref{logicalthm:heredity_of_implication_to_disjunction})より
		\begin{align}
			\vdash (\, A \rarrow B\, ) 
			\rarrow (\, A \vee \negation A \rarrow B \vee \negation A\, )
		\end{align}
		が成り立つので,演繹定理の逆より
		\begin{align}
			A \rarrow B \vdash A \vee \negation A \rarrow B \vee \negation A
		\end{align}
		が成り立つ.また排中律(論理的定理\ref{logicalthm:law_of_excluded_middle})より
		\begin{align}
			A \rarrow B \vdash A \vee \negation A
		\end{align}
		も成り立つので,三段論法より
		\begin{align}
			A \rarrow B \vdash B \vee \negation A
		\end{align}
		となる.論理和の可換性(論理的定理\ref{logicalthm:commutative_law_of_disjunction})より
		\begin{align}
			A \rarrow B \vdash B \vee \negation A \rarrow\ \negation A \vee B
		\end{align}
		が成り立つので,三段論法より
		\begin{align}
			A \rarrow B \vdash\ \negation A \vee B
		\end{align}
		が従い,演繹定理より
		\begin{align}
			\vdash (\, A \rarrow B\, ) \rarrow (\, \negation A \vee B\, )
		\end{align}
		が得られる.
		\QED
	\end{prf}
	
	\begin{screen}
		\begin{logicalthm}[強 De Morgan の法則(2)]
		\label{logicalthm:strong_De_Morgan_law_2}
			$A$と$B$を文とするとき
			\begin{align}
				\vdash\ \negation (\, A \wedge B\, )
				\rarrow\ \negation A \vee \negation B.
			\end{align}
		\end{logicalthm}
	\end{screen}
	
	\begin{prf}
		論理積の導入より
		\begin{align}
			A \vdash B \rarrow A \wedge B
		\end{align}
		が成り立つので,これの対偶を取って
		\begin{align}
			A \vdash\ \negation (\, A \wedge B\, ) \rarrow\ \negation B
		\end{align}
		を得る(論理的定理\ref{logicalthm:introduction_of_contraposition}).
		そして演繹定理の逆より
		\begin{align}
			A,\ \negation (\, A \wedge B\, ) \vdash\ \negation B
		\end{align}
		が成立し,演繹定理より
		\begin{align}
			\negation (\, A \wedge B\, ) \vdash A \rarrow\ \negation B
		\end{align}
		となる.論理的定理\ref{logicalthm:implication_rewritable_by_disjunction_of_negation}より
		\begin{align}
			\negation (\, A \wedge B\, ) \vdash (\, A \rarrow\ \negation B\, )
			\rarrow (\, \negation A \vee \negation B\, )
		\end{align}
		が成り立つので三段論法より
		\begin{align}
			\negation (\, A \wedge B\, ) \vdash\ \negation A \vee \negation B
		\end{align}
		が従う.そして演繹定理より
		\begin{align}
			\vdash\ \negation (\, A \wedge B\, )
			\rarrow\ \negation A \vee \negation B.
		\end{align}
		を得る.
		\QED
	\end{prf}
		\begin{screen}
		\begin{logicalrule}[量化記号に関する規則]
		\label{logicalrule:rules_of_quantifiers}
			$A$を$\mathcal{L}$の式とし,$x$を$A$に自由に現れる変項とし,
			$A$に自由に現れる項が$x$のみであるとする.
			また$\tau$を主要$\varepsilon$項とする.このとき以下を推論規則とする.
			\begin{align}
				A(\tau) &\vdash \exists x A(x), \\
				\exists x A(x) &\vdash A(\varepsilon x \hat{A}(x)), \\
				\forall x A(x) &\vdash A(\tau), \\
				\negation \forall x A(x) &\vdash \exists x \negation A(x).
			\end{align}
			ただし$\hat{A}$とは必要に応じて$A$を$\lang{\varepsilon}$の式に書き直したものである.
		\end{logicalrule}
	\end{screen}
	
	存在記号の推論規則より
	\begin{align}
		\vdash \exists x \negation A \rarrow\ \negation A(\varepsilon x \widehat{\negation A})
	\end{align}
	が成り立つが,ここで$\widehat{\negation A}$と$\negation \hat{A}$は同一の記号列なので
	(もとより$A$が$\lang{\varepsilon}$の式ならばどちらも$\negation A$である)
	\begin{align}
		\vdash \exists x \negation A \rarrow\ \negation A(\varepsilon x \negation \hat{A})
	\end{align}
	となり,量化の四番目の推論規則との三段論法で
	\begin{align}
		\negation \forall x A \vdash\ \negation A(\varepsilon x \negation \hat{A})
	\end{align}
	が従う.そして対偶論法(推論法則\ref{logicalthm:proof_by_contraposition})より
	\begin{align}
		\vdash A(\varepsilon x \negation \hat{A}) \rarrow \forall x A
	\end{align}
	が得られる.これは非常に有用な結果であるから一つの定理として述べておく.
	
	\begin{screen}
		\begin{logicalthm}[$\varepsilon$項による全称の導出]
		\label{logicalthm:derivation_of_universal_by_epsilon}
			$A$を$\mathcal{L}$の式とし,$x$を$A$に自由に現れる変項とし,
			$A$に自由に現れる項が$x$のみであるとする.このとき
			\begin{align}
				\vdash A(\varepsilon x \negation \hat{A}(x)) \rarrow \forall x A(x).
			\end{align}
			ただし$\hat{A}$とは必要に応じて$A$を$\lang{\varepsilon}$の式に書き直したものである.
		\end{logicalthm}
	\end{screen}
	
	どれでも一つ,$A(\tau)$を成り立たせるような主要$\varepsilon$項$\tau$が取れれば
	$\exists x A(x)$が成り立つのだし,逆に$\exists x A(x)$が成り立つならば
	$\varepsilon x A(x)$なる$\epsilon$項が$A(\varepsilon x A(x))$を満たすのである.
	そして主要$\varepsilon$項は集合であるから(定理\ref{thm:critical_epsilon_term_is_set}),
	「$A(x)$を満たす集合$x$が存在する」ということと
	「$A(x)$を満たす集合$x$が{\bf ``実際に取れる''}」ということが同じ意味になる.
	
	$\forall x A(x)$が成り立つならばいかなる主要$\varepsilon$項$\tau$も$A(\tau)$を満たすし,
	逆にいかなる主要$\varepsilon$項$\tau$も$A(\tau)$を満たすならば,
	特に$\varepsilon x \negation A(x)$なる$\varepsilon$項も
	$A(\varepsilon x \negation A(x))$を満たすのだから$\forall x A(x)$が成立する.
	つまり,「$\forall x A(x)$が成り立つ」ということと
	「任意の主要$\varepsilon$項$\tau$が$A(\tau)$を満たす」ということは同じ意味になる.
	
	後述することであるが,主要$\varepsilon$項はどれも集合であって
	(定理\ref{thm:critical_epsilon_term_is_set}),また集合である類は
	いずれかの主要$\varepsilon$項と等しい
	(定理\ref{thm:if_a_class_is_a_set_then_equal_to_some_epsilon_term}).
	ゆえに,{\bf 量化子の亘る範囲は集合に制限される}のである.
	
	量化記号についても De Morgan の法則があり,それを
	\begin{description}
		\item[弱 De Morgan の法則] $\exists x \negation A(x) \lrarrow\ \negation \forall x A(x)$,
		\item[強 De Morgan の法則] $\forall x \negation A(x) \lrarrow\ \negation \exists x A(x)$,
	\end{description}
	と呼ぶことにする.
	
	\begin{screen}
		\begin{logicalthm}[量化記号に対する弱 De Morgan の法則(1)]
		\label{logicalthm:weak_De_Morgan_law_for_quantifiers_1}
			$A$を$\mathcal{L}$の式とし,$x$を$A$に自由に現れる変項とし,
			また$A$に自由に現れる変項は$x$のみであるとする.このとき
			\begin{align}
				\vdash \exists x \negation A(x) \rarrow\ \negation \forall x A(x).
			\end{align}
		\end{logicalthm}
	\end{screen}
	
	\begin{sketch}
		必要に応じて$A$を$\lang{\varepsilon}$の式に書き換えたものを$\hat{A}$とする.
		存在記号の推論規則より
		\begin{align}
			\exists x \negation A(x) \vdash\ \negation A(\varepsilon x \negation \hat{A}(x))
			\label{fom:weak_De_Morgan_law_for_quantifiers_1_1}
		\end{align}
		となる.また全称記号の推論規則より
		\begin{align}
			\vdash \forall x A(x) \rarrow A(\varepsilon x \negation \hat{A}(x))
		\end{align}
		が成り立つので,対偶を取って
		\begin{align}
			\vdash\ \negation A(\varepsilon x \negation \hat{A}(x)) \rarrow\ \negation \forall x A(x)
			\label{fom:weak_De_Morgan_law_for_quantifiers_1_2}
		\end{align}
		となる(推論法則\ref{logicalthm:introduction_of_contraposition}).
		(\refeq{fom:weak_De_Morgan_law_for_quantifiers_1_1})と
		(\refeq{fom:weak_De_Morgan_law_for_quantifiers_1_2})の三段論法より
		\begin{align}
			\exists x \negation A(x) \vdash\ \negation \forall x A(x)
		\end{align}
		が従い,演繹規則より
		\begin{align}
			\vdash \exists x \negation A(x) \rarrow\ \negation \forall x A(x)
		\end{align}
		が得られる.
		\QED
	\end{sketch}
	
	\begin{screen}
		\begin{logicalthm}[量化記号に対する弱 De Morgan の法則(2)]
		\label{logicalthm:weak_De_Morgan_law_for_quantifiers_2}
			$A$を$\mathcal{L}$の式とし,$x$を$A$に自由に現れる変項とし,
			また$A$に自由に現れる変項は$x$のみであるとする.このとき
			\begin{align}
				\vdash\ \negation \forall x A(x) \rarrow \exists x \negation A(x).
			\end{align}
		\end{logicalthm}
	\end{screen}
	
	\begin{sketch}
		推論規則
		\begin{align}
			\negation \forall x A(x) \vdash \exists x \negation A(x)
		\end{align}
		に演繹規則を適用して得られる.
		\QED
	\end{sketch}
	
	\begin{comment}
		必要に応じて$A$を$\lang{\varepsilon}$の式に書き換えたものを$\hat{A}$とする.
		存在記号の推論規則より
		\begin{align}
			\vdash\ \negation A(\varepsilon x \negation \hat{A}(x)) \rarrow \exists x \negation A(x)
		\end{align}
		が成り立つので,対偶を取って
		\begin{align}
			\vdash\ \negation \exists x \negation A(x) \rarrow\ \negation\negation A(\varepsilon x \negation \hat{A}(x))
		\end{align}
		が従い(推論法則\ref{logicalthm:introduction_of_contraposition}),演繹法則の逆より
		\begin{align}
			\negation \exists x \negation A(x) \vdash\ \negation\negation A(\varepsilon x \negation \hat{A}(x))
		\end{align}
		となる.そして二重否定の除去規則より
		\begin{align}
			\negation \exists x \negation A(x) \vdash A(\varepsilon x \negation \hat{A}(x))
		\end{align}
		が成立し,全称の導出(推論法則\ref{derivation_of_universal_by_epsilon})より
		\begin{align}
			\negation \exists x \negation A(x) \vdash \forall x A(x)
		\end{align}
		が従う.演繹規則より
		\begin{align}
			\vdash\ \negation \exists x \negation A(x) \rarrow \forall x A(x)
		\end{align}
		となり,対偶を取れば
		\begin{align}
			\vdash\ \negation \forall x A(x) \rarrow\ \negation\negation \exists x \negation A(x)
		\end{align}
		となるが,先と同様に二重否定の除去によって
		\begin{align}
			\vdash\ \negation \forall x A(x) \rarrow \exists x \negation A(x)
		\end{align}
		が得られる.
		\QED
	\end{comment}
	
	\begin{screen}
		\begin{logicalthm}[量化記号に対する強 De Morgan の法則(1)]
		\label{logicalthm:strong_De_Morgan_law_for_quantifiers_1}
			$A$を$\mathcal{L}$の式とし,$x$を$A$に自由に現れる変項とし,
			また$A$に自由に現れる変項は$x$のみであるとする.このとき
			\begin{align}
				\vdash \forall x \negation A(x) \rarrow\ \negation \exists x A(x).
			\end{align}
		\end{logicalthm}
	\end{screen}
	
	\begin{sketch}
		必要に応じて$A$を$\lang{\varepsilon}$の式に書き換えたものを$\hat{A}$とする.
		まず存在記号の推論規則より
		\begin{align}
			\vdash \exists x A(x) \rarrow A(\varepsilon x \hat{A}(x))
		\end{align}
		が成り立つので,対偶を取って
		\begin{align}
			\vdash\ \negation A(\varepsilon x \hat{A}(x)) 
			\rarrow\ \negation \exists x A(x)
		\end{align}
		が成り立つ(推論法則\ref{logicalthm:introduction_of_contraposition}).また
		全称記号の推論規則より
		\begin{align}
			\forall x \negation A(x) \vdash\ \negation A(\varepsilon x \hat{A}(x))
		\end{align}
		が成り立つので,三段論法より
		\begin{align}
			\forall x \negation A(x) \vdash\ \negation \exists x A(x)
		\end{align}
		が従い,演繹規則より
		\begin{align}
			\vdash \forall x \negation A(x) \rarrow\ \negation \exists x A(x)
		\end{align}
		が得られる.
		\QED
	\end{sketch}
	
	\begin{screen}
		\begin{logicalthm}[量化記号に対する強 De Morgan の法則(2)]
		\label{logicalthm:strong_De_Morgan_law_for_quantifiers_2}
			$A$を$\mathcal{L}$の式とし,$x$を$A$に自由に現れる変項とし,
			また$A$に自由に現れる変項は$x$のみであるとする.このとき
			\begin{align}
				\vdash\ \negation \exists x A(x) \rarrow \forall x \negation A(x).
			\end{align}
		\end{logicalthm}
	\end{screen}
	
	\begin{sketch}
		必要に応じて$A$を$\lang{\varepsilon}$の式に書き換えたものを$\hat{A}$とする.
		まず存在記号の推論規則より
		\begin{align}
			\vdash A(\varepsilon x \negation \negation \hat{A}(x))
			\rarrow \exists x A(x)
		\end{align}
		が成り立つので,対偶を取って
		\begin{align}
			\vdash\ \negation \exists x A(x) 
			\rarrow\ \negation A(\varepsilon x \negation \negation \hat{A}(x))
		\end{align}
		が成り立ち(推論法則\ref{logicalthm:introduction_of_contraposition}),
		演繹法則の逆より
		\begin{align}
			\negation \exists x A(x) \vdash\ \negation A(\varepsilon x \negation \negation \hat{A}(x))
		\end{align}
		が従う.また全称の導出(推論法則\ref{logicalthm:derivation_of_universal_by_epsilon})より
		\begin{align}
			\vdash \negation A(\varepsilon x \negation \negation \hat{A}(x))
			\rarrow \forall x \negation A(x)
		\end{align}
		が成り立つので,三段論法より
		\begin{align}
			\negation \exists x A(x) \vdash \forall x \negation A(x)
		\end{align}
		が従い,演繹規則より
		\begin{align}
			\vdash\ \negation \exists x A(x) \rarrow \forall x \negation A(x)
		\end{align}
		が得られる.
		\QED
	\end{sketch}
	%\section{その他の推論規則}
	\begin{screen}
		\begin{logicalthm}[含意の推移律]
		\label{logicalthm:transitive_law_of_implication}
			$A,B,C$を文とするとき
			\begin{align}
				\vdash (A \rarrow B) \rarrow ((B \rarrow C) \rarrow (A \rarrow C)).
			\end{align}
		\end{logicalthm}
	\end{screen}
	
	\begin{prf}
		\begin{align}
			A \rarrow B,\ B \rarrow C,\ A &\vdash A, \\
			A \rarrow B,\ B \rarrow C,\ A &\vdash A \rarrow B, \\
		\end{align}
		と三段論法より
		\begin{align}
			A \rarrow B,\ B \rarrow C,\ A \vdash B
		\end{align}
		となる.これと
		\begin{align}
			A \rarrow B,\ B \rarrow C,\ A \vdash B \rarrow C
		\end{align}
		と三段論法より
		\begin{align}
			A \rarrow B,\ B \rarrow C,\ A \vdash C
		\end{align}
		となる.これに演繹規則を三回適用すれば
		\begin{align}
			A \rarrow B,\ B \rarrow C &\vdash A \rarrow C, \\
			A \rarrow B &\vdash (B \rarrow C) \rarrow (A \rarrow C), \\
			&\vdash (A \rarrow B) \rarrow ((B \rarrow C) \rarrow (A \rarrow C))
		\end{align}
		が得られる.
		\QED
	\end{prf}
	
	\begin{screen}
		\begin{logicalthm}[二式が同時に導かれるならその論理積が導かれる]
		\label{logicalthm:conjunction_of_consequences}
			$A,B,C$を文とするとき
			\begin{align}
				\vdash (A \rarrow B) \rarrow ((A \rarrow C) 
				\rarrow (A \rarrow B \wedge C)).
			\end{align}
		\end{logicalthm}
	\end{screen}
	
	\begin{prf}
		\begin{align}
			A \rarrow B,\ A \rarrow C,\ A &\vdash A, \\
			A \rarrow B,\ A \rarrow C,\ A &\vdash A \rarrow B
		\end{align}
		と三段論法より
		\begin{align}
			A \rarrow B,\ A \rarrow C,\ A \vdash B
			\label{fom:logicalthm_conjunction_of_consequences_1}
		\end{align}
		が得られ,同様に
		\begin{align}
			A \rarrow B,\ A \rarrow C,\ A &\vdash A, \\
			A \rarrow B,\ A \rarrow C,\ A &\vdash A \rarrow C
		\end{align}
		と三段論法より
		\begin{align}
			A \rarrow B,\ A \rarrow C,\ A \vdash C
			\label{fom:logicalthm_conjunction_of_consequences_2}
		\end{align}
		が得られる.ここで論理積の導入より
		\begin{align}
			\vdash B \rarrow (C \rarrow B \wedge C)
		\end{align}
		が成り立つので
		\begin{align}
			A \rarrow B,\ A \rarrow C,\ A \vdash B \rarrow (C \rarrow B \wedge C)
		\end{align}
		も成り立つ.これと(\refeq{fom:logicalthm_conjunction_of_consequences_1})との
		三段論法より
		\begin{align}
			A \rarrow B,\ A \rarrow C,\ A \vdash C \rarrow B \wedge C
		\end{align}
		となり,(\refeq{fom:logicalthm_conjunction_of_consequences_2})との三段論法より
		\begin{align}
			A \rarrow B,\ A \rarrow C,\ A \vdash B \wedge C
		\end{align}
		となる.あとは演繹規則を三回適用すれば
		\begin{align}
			A \rarrow B,\ A \rarrow C &\vdash A \rarrow B \wedge C, \\
			A \rarrow B &\vdash (A \rarrow C) \rarrow (A \rarrow B \wedge C), \\
			&\vdash (A \rarrow B) \rarrow ((A \rarrow C) \rarrow (A \rarrow B \wedge C))
		\end{align}
		が得られる.
		\QED
	\end{prf}
	
	\begin{screen}
		\begin{logicalthm}[含意は遺伝する]\label{logicalthm:rule_of_inference_1}
			$A,B,C$を$\mathcal{L}'$の閉式とするとき以下が成り立つ:
			\begin{description}
				\item[(a)] $(A \rarrow B) \rarrow ( (A \vee C) \rarrow (B \vee C) )$.
				
				\item[(b)] $(A \rarrow B) \rarrow ( (A \wedge C) \rarrow (B \wedge C) )$.
				
				\item[(c)] $(A \rarrow B) \rarrow ( (B \rarrow C) \rarrow (A \rarrow C) )$.
				
				\item[(c)] $(A \rarrow B) \rarrow ( (C \rarrow A) \rarrow (C \rarrow B) )$.
			\end{description}
		\end{logicalthm}
	\end{screen}
	
	\begin{prf}\mbox{}
		\begin{description}
			\item[(a)]
				いま$A \rarrow B$が成り立っていると仮定する.
				論理和の導入により
				\begin{align}
					C \rarrow (B \vee C)
				\end{align}
				は定理であるから,含意の推移律より
				\begin{align}
					A \rarrow (B \vee C)
				\end{align}
				が従い,場合分け法則より
				\begin{align}
					(A \vee C) \rarrow (B \vee C)
				\end{align}
				が成立する.ここに演繹法則を適用して
				\begin{align}
					(A \rarrow B) \rarrow 
					( (A \vee C) \rarrow (B \vee C) )
				\end{align}
				が得られる.
				
			\item[(b)]
				いま$A \rarrow B$が成り立っていると仮定する.論理積の除去より
				\begin{align}
					(A \wedge C) \rarrow A
				\end{align}
				は定理であるから,含意の推移律より
				\begin{align}
					(A \wedge C) \rarrow B
				\end{align}
				が従い,他方で論理積の除去より
				\begin{align}
					(A \wedge C) \rarrow C
				\end{align}
				も満たされる.そして推論法則\ref{logicalthm:conjunction_of_consequences}から
				\begin{align}
					(A \wedge C) \rarrow (B \wedge C)
				\end{align}
				が成り立ち,演繹法則より
				\begin{align}
					(A \rarrow B) \rarrow ((A \wedge C) \rarrow (B \wedge C))
				\end{align}
				が得られる.
				
			\item[(c)]
				いま$A \rarrow B$,$B \rarrow C$および
				$A$が成り立っていると仮定する.このとき三段論法より$B$が成り立つので再び三段論法より
				$C$が成立する.ゆえに演繹法則より$A \rarrow B$と$B \rarrow C$が
				成り立っている下で
				\begin{align}
					A \rarrow C
				\end{align}
				が成立し,演繹法則を更に順次適用すれば
				\begin{align}
					(A \rarrow B) \rarrow ( (B \rarrow C) \rarrow (A \rarrow C) )
				\end{align}
				が得られる.
				
			\item[(d)]
				いま$A \rarrow B$,$C \rarrow A$および
				$C$が成り立っていると仮定する.このとき三段論法より$A$が成り立つので再び三段論法より$B$が成立し,
				ここに演繹法則を適用すれば,$A \rarrow B$と$C \rarrow A$が成立している下で
				\begin{align}
					C \rarrow B
				\end{align}
				が成立する.演繹法則を更に順次適用すれば
				\begin{align}
					(A \rarrow B) \rarrow ( (C \rarrow A) \rarrow (C \rarrow B) )
				\end{align}
				が得られる.
				\QED
		\end{description}
	\end{prf}
	
	\begin{screen}
		\begin{logicalthm}[同値記号の対称律]
		\label{logicalthm:symmetry_of_equivalence_arrows}
			$A,B$を$\mathcal{L}$の文とするとき
			\begin{align}
				\vdash (A \lrarrow B) \rarrow (B \lrarrow A).
			\end{align}
		\end{logicalthm}
	\end{screen}
	
	\begin{prf}
		$\wedge$の除去(推論規則\ref{logicalaxm:elimination_of_conjunction})より
		\begin{align}
			A \lrarrow B &\vdash A \rarrow B, \\
			A \lrarrow B &\vdash B \rarrow A
		\end{align}
		となる.他方で論理積の導入(推論規則\ref{logicalaxm:introduction_of_conjunction})より
		\begin{align}
			\vdash (B \rarrow A) \rarrow ((A \rarrow B) \rarrow 
			(B \rarrow A) \wedge (A \rarrow B))
		\end{align}
		が成り立つので,三段論法を二回適用すれば
		\begin{align}
			A \lrarrow B \vdash (B \rarrow A) \wedge (A \rarrow B)
		\end{align}
		となる.つまり
		\begin{align}
			A \lrarrow B \vdash B \lrarrow A
		\end{align}
		が得られた.
		\QED
	\end{prf}
	
	\begin{screen}
		\begin{logicalthm}[同値記号の遺伝性質]\label{logicalthm:hereditary_of_equivalence}
			$A,B,C$を$\mathcal{L}'$の閉式とするとき以下の式が成り立つ:
			\begin{description}
				\item[(a)] $(A \lrarrow B) \rarrow ((A \vee C) \lrarrow (B \vee C))$.
				\item[(b)] $(A \lrarrow B) \rarrow ((A \wedge C) \lrarrow (B \wedge C))$.
				\item[(c)] $(A \lrarrow B) \rarrow ((B \rarrow C) \lrarrow (A \rarrow C))$.
				
				\item[(d)] $(A \lrarrow B) \rarrow ((C \rarrow A) \lrarrow (C \rarrow B))$.
			\end{description}
		\end{logicalthm}
	\end{screen}
	
	\begin{prf}
		まず(a)を示す.いま$A \lrarrow B$が成り立っていると仮定する.このとき$A \rarrow B$と
		$B \rarrow A$が共に成立し,他方で含意の遺伝性質より
		\begin{align}
			&(A \rarrow B) \rarrow ((A \vee C) \rarrow (B \vee C)), \\
			&(B \rarrow A) \rarrow ((B \vee C) \rarrow (A \vee C))
		\end{align}
		が成立するから三段論法より$(A \vee C) \rarrow (B \vee C)$と
		$(B \vee C) \rarrow (A \vee C)$が共に成立する.ここに$\wedge$の導入を適用すれば
		\begin{align}
			(A \vee C) \lrarrow (B \vee C)
		\end{align}
		が成立し,演繹法則を適用すれば
		\begin{align}
			(A \lrarrow B) \rarrow ((A \vee C) \lrarrow (B \vee C))
		\end{align}
		が得られる.(b)(c)(d)も含意の遺伝性を適用すれば得られる.
		\QED
	\end{prf}
	
	文$A$が$\mathscr{S} \vdash\ \negation A$を満たすとき,
	$A$は公理系$\mathscr{S}$において{\bf 偽である}\index{ぎ@偽}{\bf (false)}という.
	
	\begin{screen}
		\begin{logicalthm}[偽な式は矛盾を導く]\label{logicalthm:false_and_negation_are_equivalent}
			$A$を文とするとき
			\begin{align}
				\vdash\ \negation A \rarrow (A \rarrow \bot).
			\end{align}
		\end{logicalthm}
	\end{screen}
	
	\begin{prf}
		矛盾の規則より
		\begin{align}
			A,\negation A \vdash \bot
		\end{align}
		である.演繹法則より
		\begin{align}
			\negation A \vdash A \rarrow \bot
		\end{align}
		が成り立ち,再び演繹法則より
		\begin{align}
			\vdash\ \negation A \rarrow (A \rarrow \bot)
		\end{align}
		が得られる.
		\QED
	\end{prf}
	
	$A,B \vdash C$と$A \wedge B \vdash C$は同値である.$A,B \vdash C$が成り立っているとすると,
	$\vdash A \rarrow (B \rarrow C)$が成り立つ.
	$A \wedge B \vdash A$と三段論法より
	$A \wedge B \vdash B \rarrow C$が成り立ち,
	$A \wedge B \vdash B$と三段論法より
	$A \wedge B \vdash C$が成り立つ.逆に$A \wedge B \vdash C$が成り立っているとすると
	$\vdash (A \wedge B) \rarrow C$が成り立つ.
	$A,B \vdash A \wedge B$と三段論法より
	$A, B \vdash C$が成り立つ.
	
	\begin{screen}
		\begin{thm}[類は集合であるか真類であるかのいずれかに定まる]
			$a$を類とするとき
			\begin{align}
				\vdash \set{a} \vee \negation \set{a}.
			\end{align}
		\end{thm}
	\end{screen}
	
	\begin{prf}
		排中律を適用することにより従う.
		\QED
	\end{prf}
	
	\begin{screen}
		\begin{logicalthm}[矛盾を導く式はあらゆる式を導く]\label{logicalthm:formula_leading_to_contradiction_derives_any_formula}
			$A,B$を文とするとき
			\begin{align}
				\vdash (A \rarrow \bot) \rarrow (A \rarrow B).
			\end{align}
		\end{logicalthm}
	\end{screen}
	
	\begin{prf}
		推論法則\ref{logicalthm:principle_of_explosion}より
		\begin{align}
			\vdash \bot \rarrow B
		\end{align}
		が成り立つので
		\begin{align}
			A \rarrow \bot \vdash \bot \rarrow B
		\end{align}
		が成り立つ.
		\begin{align}
			A \rarrow \bot \vdash A \rarrow \bot
		\end{align}
		も成り立つので,含意の推移律(推論法則\ref{logicalthm:transitive_law_of_implication})より
		\begin{align}
			A \rarrow \bot \vdash A \rarrow B
		\end{align}
		が成り立つ.そして演繹法則より
		\begin{align}
			\vdash (A \rarrow \bot) \rarrow (A \rarrow B)
		\end{align}
		が得られる.
		\QED
	\end{prf}
	
	\begin{itembox}[l]{空虚な真}
		$A,B$を文とするとき,偽な式は矛盾を導くので(推論法則\ref{logicalthm:false_and_negation_are_equivalent})
		\begin{align}
			\vdash\ \negation A \rarrow (A \rarrow \bot)
		\end{align}
		が成り立ち,矛盾を導く式はあらゆる式を導くから(推論法則\ref{logicalthm:formula_leading_to_contradiction_derives_any_formula})
		\begin{align}
			\vdash (A \rarrow \bot) \rarrow (A \rarrow B)
		\end{align}
		が成り立つ.以上と含意の推移律より
		\begin{align}
			\vdash\ \negation A \rarrow (A \rarrow B)
		\end{align}
		が得られる.つまり``偽な式はあらゆる式を導く''のであり,この現象を
		{\bf 空虚な真}\index{くうきょなしん@空虚な真}{\bf (vacuous truth)}と呼ぶ.
	\end{itembox}
	
	\begin{screen}
		\begin{logicalthm}[含意は否定と論理和で表せる]\label{logicalthm:rule_of_inference_3}
			$A,B$を文とするとき
			\begin{align}
				\vdash (A \rarrow B) \lrarrow (\negation A \vee B).
			\end{align}
		\end{logicalthm}
	\end{screen}
	
	\begin{prf}\mbox{}
		\begin{description}
			\item[第一段]
				含意の遺伝性質より
				\begin{align}
					\vdash (A \rarrow B) \rarrow 
					((A \vee \negation A) \rarrow (B \vee \negation A))
				\end{align}
				が成り立つので
				\begin{align}
					A \rarrow B \vdash (A \vee \negation A) \rarrow (B \vee \negation A)
				\end{align}
				となる.排中律より
				\begin{align}
					A \rarrow B \vdash A \vee \negation A
				\end{align}
				も成り立つので,三段論法より
				\begin{align}
					A \rarrow B \vdash B \vee \negation A
				\end{align}
				が成り立ち,論理和の可換律(推論法則\ref{logicalthm:commutative_law_of_disjunction})より
				\begin{align}
					A \rarrow B \vdash\ \negation A \vee B
				\end{align}
				が得られ,演繹法則より
				\begin{align}
					\vdash (A \rarrow B) \rarrow (\negation A \vee B)
				\end{align}
				が得られる.
				
			\item[第二段]
				推論法則\ref{logicalthm:false_and_negation_are_equivalent}より
				\begin{align}
					\vdash\ \negation A \rarrow (A \rarrow \bot)
				\end{align}
				が成り立ち,一方で推論法則\ref{logicalthm:formula_leading_to_contradiction_derives_any_formula}より
				\begin{align}
					\vdash (A \rarrow \bot) \rarrow (A \rarrow B)
				\end{align}
				も成り立つので,含意の推移律より
				\begin{align}
					\vdash\ \negation A \rarrow (A \rarrow B)
				\end{align}
				が成立する.推論法則\ref{logicalthm:introduction_of_implication}より
				\begin{align}
					\vdash B \rarrow (A \rarrow B)
				\end{align}
				も成り立つから,場合分けの法則より
				\begin{align}
					\vdash (\negation A \vee B) \rarrow (A \rarrow B)
				\end{align}
				が成り立つ.
				\QED
		\end{description}
	\end{prf}
	
	\begin{screen}
		\begin{logicalthm}[二重否定の法則の逆が成り立つ]\label{logicalthm:converse_of_law_of_double_negative}
			$A$を文とするとき
			\begin{align}
				\vdash A \rarrow \negation \negation A.
			\end{align}
		\end{logicalthm}
	\end{screen}
	
	\begin{prf}
		排中律より
		\begin{align}
			\vdash\ \negation A \vee \negation \negation A
		\end{align}
		が成立し,また推論法則\ref{logicalthm:rule_of_inference_3}より
		\begin{align}
			\vdash (\negation A \vee \negation \negation A)
			\rarrow (A \rarrow \negation \negation A)
		\end{align}
		も成り立つので,三段論法より
		\begin{align}
			\vdash A \rarrow \negation \negation A
		\end{align}
		が成立する.
		\QED
	\end{prf}
	
	\begin{screen}
		\begin{logicalthm}[対偶命題は同値]\label{thm:contraposition_is_true}
			$A,B$を文とするとき
			\begin{align}
				\vdash (A \rarrow B) \lrarrow (\negation B \rarrow \negation A).
			\end{align}
		\end{logicalthm}
	\end{screen}
	
	\begin{prf}\mbox{}
		\begin{description}
			\item[第一段]
				含意は否定と論理和で表せるので(推論法則\ref{logicalthm:rule_of_inference_3})
				\begin{align}
					\vdash (A \rarrow B) \rarrow (\negation A \vee B)
					\label{eq:thm_contraposition_is_true_1}
				\end{align}
				が成り立つ.また論理和は可換であるから(推論法則\ref{logicalthm:commutative_law_of_disjunction})
				\begin{align}
					\vdash (\negation A \vee B) \rarrow (B \vee \negation A)
					\label{eq:thm_contraposition_is_true_2}
				\end{align}
				が成り立つ.ところで二重否定の法則の逆(推論法則\ref{logicalthm:converse_of_law_of_double_negative})より
				\begin{align}
					\vdash B \rarrow \negation \negation B
				\end{align}
				が成り立ち,また含意の遺伝性質より
				\begin{align}
					\vdash (B \rarrow \negation \negation B)
					\rarrow ((B \vee \negation A) 
					\rarrow (\negation \negation B \vee \negation A))
				\end{align}
				も成り立つから,三段論法より
				\begin{align}
					\vdash (B \vee \negation A) 
					\rarrow (\negation \negation B \vee \negation A)
					\label{eq:thm_contraposition_is_true_3}
				\end{align}
				が成立する.再び推論法則\ref{logicalthm:rule_of_inference_3}によって
				\begin{align}
					\vdash (\negation \negation B \vee \negation A)
					\rarrow (\negation B \rarrow \negation A)
					\label{eq:thm_contraposition_is_true_4}
				\end{align}
				が成り立つ.(\refeq{eq:thm_contraposition_is_true_1})と(\refeq{eq:thm_contraposition_is_true_2})と
				(\refeq{eq:thm_contraposition_is_true_3})と(\refeq{eq:thm_contraposition_is_true_4})に含意の推移律を適用すれば
				\begin{align}
					\vdash (A \rarrow B) \rarrow (\negation B \rarrow \negation A)
				\end{align}
				が得られる.
				
			\item[第二段]
				含意は否定と論理和で表せるので(推論法則\ref{logicalthm:rule_of_inference_3})
				\begin{align}
					\vdash (\negation B \rarrow \negation A)
					\rarrow (\negation \negation B \vee \negation A)
					\label{eq:thm_contraposition_is_true_5}
				\end{align}
				が成り立つ.ところで二重否定の法則より
				\begin{align}
					\vdash\ \negation \negation B \rarrow B
				\end{align}
				が成り立ち,また含意の遺伝性質より
				\begin{align}
					\vdash (\negation \negation B \rarrow B)
					\rarrow ((\negation \negation B \vee \negation A)
					\rarrow (B \vee \negation A))
				\end{align}
				も成り立つから,三段論法より
				\begin{align}
					\vdash (\negation \negation B \vee \negation A)
					\rarrow (B \vee \negation A)
					\label{eq:thm_contraposition_is_true_6}
				\end{align}
				が成立する.論理和は可換であるから(推論法則\ref{logicalthm:commutative_law_of_disjunction})
				\begin{align}
					\vdash (B \vee \negation A) \rarrow (\negation A \vee B)
					\label{eq:thm_contraposition_is_true_7}
				\end{align}
				が成り立つ.再び推論法則\ref{logicalthm:rule_of_inference_3}によって
				\begin{align}
					\vdash (\negation A \vee B)
					\rarrow (A \rarrow B)
					\label{eq:thm_contraposition_is_true_8}
				\end{align}
				が成り立つ.(\refeq{eq:thm_contraposition_is_true_5})と(\refeq{eq:thm_contraposition_is_true_6})と
				(\refeq{eq:thm_contraposition_is_true_7})と(\refeq{eq:thm_contraposition_is_true_8})に含意の推移律を適用すれば
				\begin{align}
					\vdash (\negation B \rarrow \negation A) \rarrow (A \rarrow B)
				\end{align}
				が得られる.
				\QED
		\end{description}
	\end{prf}
	
	上の証明は簡単に書けば
	\begin{align}
		(A \rarrow B) &\lrarrow (\negation A \vee B) \\
		&\lrarrow (B \vee \negation A) \\
		&\lrarrow (\negation \negation B \vee \negation A) \\
		&\lrarrow (\negation B \rarrow \negation A)
	\end{align}
	で足りる.
	
	\begin{screen}
		\begin{logicalthm}[De Morganの法則]
			$A,B$を文とするとき
			\begin{itemize}
				\item $\vdash\ \negation (A \vee B) \lrarrow \negation A \wedge \negation B$.
			
				\item $\vdash\ \negation (A \wedge B) \lrarrow \negation A \vee \negation B$.
			\end{itemize}
		\end{logicalthm}
	\end{screen}
	
	\begin{prf}\mbox{}
		\begin{description}
			\item[第一段]	論理和の導入の対偶を取れば
				\begin{align}
					\vdash\ \negation (A \vee B) \rarrow \negation A
				\end{align}
				と
				\begin{align}
					\vdash\ \negation (A \vee B) \rarrow \negation B
				\end{align}
				が成り立つ(推論法則\ref{logicalthm:introduction_of_contraposition}).
				二式が同時に導かれるならその論理積も導かれるので(推論法則\ref{logicalthm:conjunction_of_consequences})
				\begin{align}
					\vdash\ \negation (A \vee B) \rarrow (\negation A \wedge \negation B)
				\end{align}
				が得られる.また
				\begin{align}
					A, \negation A \wedge \negation B \vdash A
				\end{align}
				かつ
				\begin{align}
					A, \negation A \wedge \negation B \vdash\ \negation A
				\end{align}
				より
				\begin{align}
					A, \negation A \wedge \negation B \vdash \bot
				\end{align}
				が成り立つので,演繹法則より
				\begin{align}
					A \vdash (\negation A \wedge \negation B) \rarrow \bot
				\end{align}
				が従い,否定の導入により
				\begin{align}
					A \vdash\ \negation (\negation A \wedge \negation B)
				\end{align}
				が成り立つ.同様にして
				\begin{align}
					B \vdash\ \negation (\negation A \wedge \negation B)
				\end{align}
				も成り立つので,場合分け法則より
				\begin{align}
					\vdash (A \vee B) \rarrow \negation (\negation A \wedge \negation B)
				\end{align}
				が成立する.この対偶を取れば
				\begin{align}
					\vdash (\negation A \wedge \negation B) \rarrow \negation (A \vee B)
				\end{align}
				が得られる(推論法則\ref{logicalthm:introduction_of_contraposition}).
				
			\item[第二段]
				前段の結果より
				\begin{align}
					\vdash (\negation \negation A \wedge \negation \negation B)
					\lrarrow \negation (\negation A \vee \negation B)
				\end{align}
				が成り立つ.ところで二重否定の法則とその逆(推論法則\ref{logicalthm:converse_of_law_of_double_negative})より
				\begin{align}
					\vdash (\negation \negation A \wedge \negation \negation B)
					\lrarrow (A \wedge B)
				\end{align}
				が成り立つので
				\begin{align}
					\vdash (A \wedge B) 
					\lrarrow \negation (\negation A \vee \negation B)
				\end{align}
				が成り立つ.対偶命題の同値性(推論法則\ref{logicalthm:introduction_of_contraposition})から
				\begin{align}
					\vdash\ \negation (A \wedge B)
					\lrarrow (\negation A \vee \negation B)
				\end{align}
				が得られる.
				\QED
		\end{description}
	\end{prf}
	
	\monologue{
		以上で``集合であり真類でもある類は存在しない''という言明を証明する準備が整いました.
	}
	
	\begin{screen}
		\begin{thm}[集合であり真類でもある類は存在しない]
			$a$を類とするとき
			\begin{align}
				\vdash\ \negation (\ \set{a} \wedge \negation \set{a}\ ).
			\end{align}
		\end{thm}
	\end{screen}
	
	\begin{prf}
		$a$を類とするとき,排中律より
		\begin{align}
			\vdash \set{a} \vee \negation \set{a}
		\end{align}
		が成り立ち,論理和の可換律より
		\begin{align}
			\vdash\ \negation \set{a} \vee \set{a}
		\end{align}
		も成立する.そしてDe Morganの法則より
		\begin{align}
			\vdash\ \negation (\ \negation \negation \set{a} \wedge \negation \set{a}\ )
		\end{align}
		が成り立つが,二重否定の法則より$\negation \negation \set{a}$と
		$\set{a}$は同値となるので
		\begin{align}
			\vdash\ \negation (\ \set{a} \wedge \negation \set{a}\ )
		\end{align}
		が成り立つ.
		\QED
	\end{prf}
	
	「集合であり真類でもある類は存在しない」とは言ったものの,それはあくまで
	\begin{align}
		\Sigma \vdash\ \negation (\ \set{a} \wedge \negation \set{a}\ )
	\end{align}
	を翻訳したに過ぎないのであって,もしかすると
	\begin{align}
		\Sigma \vdash \set{a} \wedge \negation \set{a}
	\end{align}
	も導かれるかもしれない.この場合$\Sigma$は矛盾することになるが,$\Sigma$の無矛盾性が不明であるため
	この事態が起こらないとは言えない.

\chapter{集合}
\label{chap:set_theory}
		$a,b$を$\mathcal{L}$の項とするとき,
	\begin{align}
		a \notin b \defarrow\ \negation a \in b
	\end{align}
	で$a \notin b$を定める.同様に
	\begin{align}
		a \neq b \defarrow\ \negation a = b
	\end{align}
	で$a \neq b$を定める.
	
	類とされた項の多くは集合であるが,{\bf 類が全て集合であると考えると矛盾が起こる}.
	たとえばRussellのパラドックスで有名な
	\begin{align}
		R \defeq \Set{x}{x \notin x}
	\end{align}
	なる類が集合であるとすると($\defarrow$は``式''に対する略記の導入に使ったが,
	$\defeq$とは``類''に対する略記を導入するために使う定義記号である)
	\begin{align}
		\Sigma \vdash R \notin R \lrarrow R \in R
	\end{align}
	が成り立ってしまい(正式な推論は無視してラフに考えれば),これは$\Sigma \vdash \bot$を導く.
	この種の矛盾を回避するために類を導入したのであり,
	集合とは類の中で特定の性質をもつものに限られる.
	
	\begin{screen}
		\begin{dfn}[集合]
			$a$を類とするとき,$a$が集合であるという言明を
			\begin{align}
				\set{a} \defarrow \exists x\, (\, a = x\, )
			\end{align}
			で定める.$\Sigma \vdash \set{a}$を満たす類$a$を
			{\bf 集合}\index{しゅうごう@集合}{\bf (set)}と呼び,
			$\Sigma \vdash\ \negation \set{a}$を満たす類$a$を
			{\bf 真類}\index{しんるい@真類}{\bf (proper class)}と呼ぶ.
		\end{dfn}
	\end{screen}
	
	$\varphi$を$\mathcal{L}$の式とし,$x$を$\varphi$に自由に現れる変項とし,
	$x$のみが$\varphi$で自由であるとする.このとき
	\begin{align}
		\set{\Set{x}{\varphi(x)}} \vdash \set{\Set{x}{\varphi(x)}}
	\end{align}
	が満たされている.つまり
	\begin{align}
		\set{\Set{x}{\varphi(x)}}
		\vdash \exists y\, \left(\, \Set{x}{\varphi(x)} = y\, \right)
	\end{align}
	が成り立っているということであるが,$\Set{x}{\varphi(x)} = y$を
	\begin{align}
		\forall x\, (\, \varphi(x) \lrarrow x \in y\, )
	\end{align}
	と書き換えれば,存在記号の推論規則より
	\begin{align}
		\set{\Set{x}{\varphi(x)}} \vdash \Set{x}{\varphi(x)} = 
		\varepsilon y\, \forall x\, (\, \varphi(x) \lrarrow x \in y\, )
	\end{align}
	が得られる.
	
	\begin{screen}
		\begin{thm}[集合である内包項は$\varepsilon$項で書ける]
		\label{thm:if_a_class_is_a_set_then_equal_to_some_epsilon_term}
			$\varphi$を$\mathcal{L}$の式とし,$x$を$\varphi$に自由に現れる変項とし,
			$x$のみが$\varphi$で自由であるとする.このとき
			\begin{align}
				\set{\Set{x}{\varphi(x)}} \vdash \Set{x}{\varphi(x)} 
				= \varepsilon y\, \forall x\, (\, \varphi(x) \lrarrow x \in y\, ).
			\end{align}
		\end{thm}
	\end{screen}
	
	ブルバキ\cite{key4}では$\tau$項を,島内\cite{key6}では$\varepsilon$項のみを導入して
	$\varepsilon y \forall x\, (\, \varphi(x) \lrarrow x \in y\, )$
	によって$\Set{x}{\varphi(x)}$を定めているが,この定め方には欠点がある.
	というのも,本稿と同じくブルバキ\cite{key4}の$\tau$項も島内\cite{key6}の$\varepsilon$項も
	集合であるから,
	\begin{align}
		\exists y\, \forall x\, (\, \varphi(x) \lrarrow x \in y\, )
	\end{align}
	が成立しない場合は$\varepsilon y \forall x\, (\, \varphi(x) \lrarrow x \in y\, )$
	は正体不明になってしまい,$\Set{x}{\varphi(x)}$が「性質$\varphi$を持つ集合の全体」
	の意味を持たないのである.本稿では内包項と$\varepsilon$項を別々に
	生成しているのでこの欠点は解消される.
	
\section{相等性}
	本稿において``等しい''とは項に対する言明であって,$a$と$b$を項とするとき
	\begin{align}
		a = b
	\end{align}
	なる式で表される.この記号
	\begin{align}
		=
	\end{align}
	は{\bf 等号}\index{とうごう@等号}{\bf (equal sign)}と呼ばれるが,
	現時点では述語として導入されているだけで,推論操作における働きは不明のままである.
	本節では,いつ類は等しくなるのか,そして,等しい場合に何が起きるのか,の二つが主題となる.
	
	\begin{screen}
		\begin{axm}[外延性の公理 (Extensionality)]
			任意の類$a,b$に対して
			\begin{align}
				\EXTAX \defarrow \forall x\, (\, x \in a \lrarrow x \in b\, ) 
				\rarrow a=b.
			\end{align}
		\end{axm}
	\end{screen}
	
	\begin{screen}
		\begin{thm}[任意の類は自分自身と等しい]\label{thm:any_class_equals_to_itself}
			任意の類$\tau$に対して
			\begin{align}
				\EXTAX \vdash \tau = \tau.
			\end{align}
		\end{thm}
	\end{screen}
	
	\begin{sketch}
		いま
		\begin{align}
			\sigma \defeq 
			\varepsilon s \negation (\, s \in \tau \lrarrow s \in \tau\, )
		\end{align}
		とおく.推論法則\ref{logicalthm:reflective_law_of_implication}より
		\begin{align}
			\vdash \sigma \in \tau \lrarrow \sigma \in \tau
		\end{align}
		が成り立つから,全称記号の推論規則より
		\begin{align}
			\vdash \forall s\, (\, s \in \tau  \lrarrow s \in \tau\, )
		\end{align}
		が成り立つ.外延性の公理より
		\begin{align}
			\EXTAX \vdash \forall s\, (\, s \in \tau  \lrarrow s \in \tau\, )
			\rarrow \tau = \tau
		\end{align}
		となるので,三段論法より
		\begin{align}
			\EXTAX \vdash \tau = \tau
		\end{align}
		が得られる.
		\QED
	\end{sketch}
	
	\begin{screen}
		\begin{thm}[主要$\varepsilon$項は集合である]
		\label{thm:critical_epsilon_term_is_set}
			$\tau$を類である$\varepsilon$項とするとき
			\begin{align}
				\EXTAX \vdash \set{\tau}.
			\end{align}
		\end{thm}
	\end{screen}
	
	\begin{sketch}
		定理\ref{thm:any_class_equals_to_itself}より
		\begin{align}
			\EXTAX \vdash \tau = \tau
		\end{align}
		が成立するので,存在記号の推論規則より
		\begin{align}
			\EXTAX \vdash \exists x\, \left(\, \tau = x\, \right)
		\end{align}
		が成立する.
		\QED
	\end{sketch}
	
	例えば
	\begin{align}
		a = b
	\end{align}
	と書いてあったら``$a$と$b$は等しい''と読めるわけだが,明らかに$a$は$b$とは違うではないではないか!
	こんなことはしょっちゅう起こることであって,上で述べたように$\Set{x}{A(x)}$が集合なら
	\begin{align}
		\Set{x}{A(x)} = \varepsilon y \forall x\, \left(\, A(x) \lrarrow x \in y\, \right)
	\end{align}
	が成り立ったりする.そこで``数学的に等しいとは何事か''という疑問が浮かぶのは至極自然であって,
	それに答えるのが次の相等性公理である.
	
	\begin{screen}
		\begin{axm}[相等性公理]
			$a,b,c$を類とするとき
			\begin{align}
				\EQAX \defarrow
				\begin{cases}
					a = b \rarrow b = a, & \\
					a = b \rarrow (\, a \in c \rarrow b \in c\, ), & \\
					a = b \rarrow (\, c \in a \rarrow c \in b\, ). & 
				\end{cases}
			\end{align}
		\end{axm}
	\end{screen}
	
	\begin{screen}
		\begin{thm}[外延性の公理の逆も成り立つ]
		\label{thm:inverse_of_axiom_of_extensionality}
			$a$と$b$を類とするとき
			\begin{align}
				\EQAX \vdash 
				a = b \rarrow \forall x\, (\, x \in a  \lrarrow x \in b\, ).
			\end{align}
		\end{thm}
	\end{screen}
	
	\begin{prf}
		いま
		\begin{align}
			\tau \defeq \varepsilon x \negation (\, x \in a  \lrarrow x \in b\, )
		\end{align}
		とおく.相等性公理より
		\begin{align}
			\EQAX \vdash a = b \rarrow (\, \tau \in a \rarrow \tau \in b\, )
		\end{align}
		となるので,演繹法則の逆より
		\begin{align}
			a = b,\ \EQAX \vdash \tau \in a \rarrow \tau \in b
			\label{fom:inverse_of_axiom_of_extensionality_1}
		\end{align}
		となる.また相等性公理と演繹法則の逆により
		\begin{align}
			a = b,\ \EQAX \vdash b = a
		\end{align}
		が成り立ち,同じく相等性公理より
		\begin{align}
			\EQAX \vdash b = a \rarrow (\, \tau \in b \rarrow \tau \in a\, )
		\end{align}
		も成り立つので,三段論法より
		\begin{align}
			a = b,\ \EQAX \vdash \tau \in b \rarrow \tau \in a
			\label{fom:inverse_of_axiom_of_extensionality_2}
		\end{align}
		も得られる.論理積の導入により
		\begin{align}
			a = b,\ \EQAX \vdash (\, \tau \in a \rarrow \tau \in b\, )
			\rarrow (\, (\, \tau \in b \rarrow \tau \in a\, )
			\rarrow (\, \tau \in a \lrarrow \tau \in b\, )\, )
		\end{align}
		が成り立つので,(\refeq{fom:inverse_of_axiom_of_extensionality_1})との三段論法より
		\begin{align}
			a = b,\ \EQAX \vdash (\, \tau \in b \rarrow \tau \in a\, )
			\rarrow (\, \tau \in a \lrarrow \tau \in b\, )
		\end{align}
		が従い,(\refeq{fom:inverse_of_axiom_of_extensionality_2})との三段論法より
		\begin{align}
			a = b,\ \EQAX \vdash \tau \in a \lrarrow \tau \in b
		\end{align}
		が従う.全称記号の推論規則より
		\begin{align}
			a = b,\ \EQAX \vdash \forall x\, (\, x \in a  \lrarrow x \in b\, )
		\end{align}
		が成立し,演繹法則より
		\begin{align}
			\EQAX \vdash a = b \rarrow \forall x\, (\, x \in a  \lrarrow x \in b\, )
		\end{align}
		が得られる.
		\QED
	\end{prf}
	
	\begin{comment}
	\begin{screen}
		\begin{thm}[(ボツ!!!)等号の対称律]\label{thm:symmetry_of_equality}
			$a,b$を類とするとき
			\begin{align}
				\EXTAX,\EQAX \vdash a = b \rarrow b = a.
			\end{align}
		\end{thm}
	\end{screen}
	
	\begin{prf}
		定理\ref{thm:axiom_of_extensionality_equivalent}より
		\begin{align}
			a=b,\ \EQAX \vdash \forall x\, (\, x \in a  \lrarrow x \in b\, )
		\end{align}
		となるが,ここで類である任意の$\varepsilon$項$\tau$に対して
		\begin{align}
			a=b,\ \EQAX \vdash \tau \in a \lrarrow \tau \in b
		\end{align}
		となるが,他方で推論法則\ref{logicalthm:symmetry_of_equivalence_arrows}より
		\begin{align}
			a=b,\ \EQAX \vdash (\, \tau \in a \lrarrow \tau \in b\, )
				\rarrow (\, \tau \in b \lrarrow \tau \in a\, )
		\end{align}
		が成り立つので,三段論法より
		\begin{align}
			a=b,\ \EQAX \vdash \tau \in b \lrarrow \tau \in a
		\end{align}
		となる.そして$\tau$の任意性より
		\begin{align}
			a=b,\ \EQAX \vdash \forall x\, (\, x \in b  \lrarrow x \in a\, )
		\end{align}
		が成り立つ.外延性の公理より
		\begin{align}
			a=b,\ \EXTAX,\EQAX \vdash \forall x\, (\, x \in b  \lrarrow x \in a\, )
			\rarrow b = a
		\end{align}
		となるので,三段論法より
		\begin{align}
			a=b,\ \EXTAX,\EQAX \vdash b = a
		\end{align}
		となる.最後に演繹法則より
		\begin{align}
			\EXTAX,\EQAX \vdash a = b \rarrow b = a
		\end{align}
		が得られる.
		\QED
	\end{prf}
	\end{comment}
	
	\begin{screen}
		\begin{axm}[内包性公理] 
			$\varphi$を$\mathcal{L}$の式とし,$y$を$\varphi$に自由に現れる変項とし,
			$\varphi$に自由に現れる項は$y$のみであるとし,
			$x$は$\varphi$で$y$への代入について自由であるとするとき,
			\begin{align}
				\COMAX \defarrow \forall x\, (\, x \in \Set{y}{\varphi(y)} \lrarrow \varphi(x)\, ).
			\end{align}
		\end{axm}
	\end{screen}
	
	\begin{screen}
		\begin{thm}[条件を満たす集合は要素である]\label{thm:satisfactory_set_is_an_element}
			$\varphi$を$\mathcal{L}$の式とし,$x$を$\varphi$に自由に現れる変項とし,
			$x$のみが$\varphi$で束縛されていないとする.このとき,任意の類$a$に対して
			\begin{align}
				\EQAX,\COMAX \vdash \varphi(a) \rarrow 
				\left(\, \set{a} \rarrow a \in \Set{x}{\varphi(x)}\, \right).
			\end{align}
		\end{thm}
	\end{screen}
	
	\begin{sketch}
		\begin{align}
			\set{a} \vdash \exists x\, (\, a = x\, )
		\end{align}
		より,
		\begin{align}
			\tau \defeq \varepsilon x\, (\, a = x\, )
		\end{align}
		とおけば
		\begin{align}
			\set{a} \vdash a = \tau
		\end{align}
		となる.相等性の公理より
		\begin{align}
			\set{a},\EQAX \vdash 
			a = \tau \rarrow (\, \varphi(a) \rarrow \varphi(\tau)\, )
		\end{align}
		となるので,三段論法と演繹法則の逆より
		\begin{align}
			\varphi(a),\set{a},\EQAX \vdash \varphi(\tau)
		\end{align}
		となる.内包性公理より
		\begin{align}
			\varphi(a),\set{a},\EQAX,\COMAX \vdash \tau \in \Set{x}{A(x)}
		\end{align}
		が従い,相等性の公理から
		\begin{align}
			\varphi(a),\set{a},\EQAX,\COMAX \vdash a \in \Set{x}{A(x)}
		\end{align}
		が成立する.演繹法則より
		\begin{align}
			\varphi(a),\EQAX,\COMAX &\vdash \set{a} \rarrow a \in \Set{x}{A(x)}, \\
			\EQAX,\COMAX &\vdash \varphi(a) \rarrow 
			\left(\, \set{a} \rarrow a \in \Set{x}{\varphi(x)}\, \right)
		\end{align}
		が従う.
		\QED
	\end{sketch}
	
	\begin{screen}
		\begin{dfn}[宇宙]
			$\Univ \defeq \Set{x}{x=x}$で定める類$\Univ$を{\bf 宇宙}\index{うちゅう@宇宙}
			{\bf (Universe)}と呼ぶ.
		\end{dfn}
	\end{screen}
	
	定理\ref{thm:V_is_the_whole_of_sets}の通り宇宙とは集合の全体を表すが,
	これ自体は集合ではない.また$\Univ$のより具体的な構造ものちに判る.
	ちなみに名前のVとはVon NeumannのVである.
	
	\begin{screen}
		\begin{axm}[要素の公理]
			要素となりうる類は集合である.つまり,$a,b$を類とするとき
			\begin{align}
				\ELEAX \defarrow a \in b \rarrow \set{a}.
			\end{align}
		\end{axm}
	\end{screen}
	
	\begin{screen}
		\begin{thm}[$\Univ$は集合の全体である]
		\label{thm:V_is_the_whole_of_sets}
			$a$を類とするとき次が成り立つ:
			\begin{align}
				\EXTAX,\EQAX,\ELEAX,\COMAX \vdash \set{a} \lrarrow a \in \Univ.
			\end{align}
		\end{thm}
	\end{screen}
	
	\begin{prf}
		$a$を類とするとき,まず要素の公理より
		\begin{align}
			\ELEAX \vdash a \in \Univ \rarrow \set{a}
		\end{align}
		が得られる.逆を示す.いま
		\begin{align}
			\tau \defeq \varepsilon x\, (\, a = x\, )
		\end{align}
		とおくと,
		\begin{align}
			\set{a} \vdash \exists x\, (\, a = x\, )
		\end{align}
		と
		\begin{align}
			\set{a} \vdash \exists x\, (\, a = x\, ) \rarrow a = \tau
		\end{align}
		(存在記号の推論規則)より
		\begin{align}
			\set{a} \vdash a = \tau
			\label{fom:thm_V_is_the_whole_of_sets_1}
		\end{align}
		が成り立つ.他方で定理\ref{thm:any_class_equals_to_itself}と内包性公理より
		\begin{align}
			\EXTAX &\vdash \tau = \tau, \\
			\COMAX &\vdash \tau = \tau \rarrow \tau \in \Univ
		\end{align}
		が成り立つので,三段論法より
		\begin{align}
			\EXTAX,\COMAX \vdash \tau \in \Univ
			\label{fom:thm_V_is_the_whole_of_sets_2}
		\end{align}
		となる.ここで相等性公理より
		\begin{align}
			\EQAX \vdash a = \tau \rarrow \tau = a
		\end{align}
		が成り立つので,(\refeq{fom:thm_V_is_the_whole_of_sets_1})と三段論法より
		\begin{align}
			\set{a},\EQAX \vdash \tau = a
			\label{fom:thm_V_is_the_whole_of_sets_3}
		\end{align}
		となる.同じく相等性公理より
		\begin{align}
			\EQAX \vdash \tau = a \rarrow (\, \tau \in \Univ \rarrow a \in \Univ\, )
		\end{align}
		が成り立つので,(\refeq{fom:thm_V_is_the_whole_of_sets_3})と三段論法より
		\begin{align}
			\set{a},\ \EQAX \vdash \tau \in \Univ \rarrow a \in \Univ
		\end{align}
		となり,(\refeq{fom:thm_V_is_the_whole_of_sets_2})と三段論法より
		\begin{align}
			\set{a},\ \EXTAX,\EQAX,\COMAX \vdash a \in \Univ
		\end{align}
		が成り立つ.最後に演繹法則より
		\begin{align}
			\EXTAX,\EQAX,\COMAX \vdash \set{a} \rarrow a \in \Univ
		\end{align}
		が得られる.
		\QED
	\end{prf}
	
	\begin{screen}
		\begin{logicalthm}[同値関係の可換律]
		\label{logicalthm:commutative_law_of_equivalence_symbol}
			$A,B$を$\mathcal{L}$の文とするとき
			\begin{align}
				\vdash (A \lrarrow B) \rarrow (B \lrarrow A).
			\end{align}
		\end{logicalthm}
	\end{screen}
	
	\begin{sketch}
		論理積の除去規則より
		\begin{align}
			A \lrarrow B &\vdash A \rarrow B, 
			\label{fom:logicalthm_commutative_law_of_equivalence_symbol_1} \\
			A \lrarrow B &\vdash B \rarrow A
			\label{fom:logicalthm_commutative_law_of_equivalence_symbol_2}
		\end{align}
		となる.他方で論理積の導入規則より
		\begin{align}
			\vdash (B \rarrow A) \rarrow ((A \rarrow B) \rarrow (B \lrarrow A))
		\end{align}
		が成り立つので
		\begin{align}
			A \lrarrow B \vdash (B \rarrow A) \rarrow ((A \rarrow B) \rarrow (B \lrarrow A))
		\end{align}
		も成り立つ.これと(\refeq{fom:logicalthm_commutative_law_of_equivalence_symbol_1})
		との三段論法より
		\begin{align}
			A \lrarrow B \vdash (A \rarrow B) \rarrow (B \lrarrow A)
		\end{align}
		となり,(\refeq{fom:logicalthm_commutative_law_of_equivalence_symbol_2})
		との三段論法より
		\begin{align}
			A \lrarrow B \vdash B \lrarrow A
		\end{align}
		が得られる.
		\QED
	\end{sketch}
	
	\begin{screen}
		\begin{logicalthm}[同値関係の推移律]
		\label{logicalthm:transitive_law_of_equivalence_symbol}
			$A,B,C$を$\mathcal{L}$の文とするとき
			\begin{align}
				\vdash (A \lrarrow B) \rarrow ((B \lrarrow C) \rarrow 
				(A \lrarrow C)).
			\end{align}
		\end{logicalthm}
	\end{screen}
	
	\begin{sketch}
		論理積の除去法則より
		\begin{align}
			A \lrarrow B &\vdash A \rarrow B, \\
			A \lrarrow B &\vdash B \rarrow A
		\end{align}
		が成り立つので
		\begin{align}
			A \lrarrow B,\ B \lrarrow C &\vdash A \rarrow B, 
			\label{fom:transitive_law_of_equivalence_symbol_1} \\
			A \lrarrow B,\ B \lrarrow C &\vdash B \rarrow A
		\end{align}
		も成り立つし,対称的に
		\begin{align}
			A \lrarrow B,\ B \lrarrow C &\vdash B \rarrow C, 
			\label{fom:transitive_law_of_equivalence_symbol_2} \\
			A \lrarrow B,\ B \lrarrow C &\vdash C \rarrow B
		\end{align}
		も成り立つ.含意の推移律(推論法則\ref{logicalthm:transitive_law_of_implication})より
		\begin{align}
			\vdash (A \rarrow B) \rarrow ((B \rarrow C) \rarrow (A \rarrow C))
		\end{align}
		となるので,(\refeq{fom:transitive_law_of_equivalence_symbol_1})との三段論法より
		\begin{align}
			A \lrarrow B,\ B \lrarrow C \vdash (B \rarrow C) \rarrow (A \rarrow C)
		\end{align}
		が成り立ち,(\refeq{fom:transitive_law_of_equivalence_symbol_2})との三段論法より
		\begin{align}
			A \lrarrow B,\ B \lrarrow C \vdash A \rarrow C
			\label{fom:transitive_law_of_equivalence_symbol_3}
		\end{align}
		が成り立つ.同様にして
		\begin{align}
			A \lrarrow B,\ B \lrarrow C \vdash C \rarrow A
			\label{fom:transitive_law_of_equivalence_symbol_4}
		\end{align}
		も得られる.論理積の導入規則より
		\begin{align}
			\vdash (A \rarrow C) \rarrow ((C \rarrow A) \rarrow (A \lrarrow C))
		\end{align}
		が成り立つので,(\refeq{fom:transitive_law_of_equivalence_symbol_3})との三段論法より
		\begin{align}
			A \lrarrow B,\ B \lrarrow C \vdash (C \rarrow A) \rarrow (A \lrarrow C)
		\end{align}
		となり,(\refeq{fom:transitive_law_of_equivalence_symbol_4})との三段論法より
		\begin{align}
			A \lrarrow B,\ B \lrarrow C \vdash A \lrarrow C
		\end{align}
		となる.あとは演繹規則を二回適用すれば
		\begin{align}
			\vdash (A \lrarrow B) \rarrow ((B \lrarrow C) \rarrow (A \lrarrow C))
		\end{align}
		が得られる.
		\QED
	\end{sketch}
	
	\begin{screen}
		\begin{thm}[等号の推移律]\label{thm:transitive_law_of_equality}
			$a,b,c$を類とするとき
			\begin{align}
				\EXTAX,\EQAX \vdash a = b \rarrow (\, a = c \rarrow b = c\, ).
			\end{align}
		\end{thm}
	\end{screen}
	
	\begin{sketch}
		まずは
		\begin{align}
			a = b,\ a = c,\ \EQAX \vdash \forall x\, (\, x \in b \lrarrow x \in c\, )
		\end{align}
		を示したいので
		\begin{align}
			\tau \defeq \varepsilon x \negation (\, x \in b \lrarrow x \in c\, )
		\end{align}
		とおく($b,c$が$\lang{\varepsilon}$の項でなければ
		$x \in b \lrarrow x \in c$を書き換える).相等性公理より
		\begin{align}
			a = b,\ a = c,\ \EQAX \vdash a = b \rarrow (\, \tau \in a \rarrow \tau \in b\, )
		\end{align}
		が成り立つので,
		\begin{align}
			a = b,\ a = c,\ \EQAX \vdash a = b
			\label{fom:thm_transitive_law_of_equality_0}
		\end{align}
		との三段論法より
		\begin{align}
			a = b,\ a = c,\ \EQAX \vdash \tau \in a \rarrow \tau \in b
			\label{fom:thm_transitive_law_of_equality_1}
		\end{align}
		となる.同じく相等性公理より
		\begin{align}
			a = b,\ a = c,\ \EQAX \vdash a = b \rarrow b = a, \\
		\end{align}
		が成り立つので,(\refeq{fom:thm_transitive_law_of_equality_0})との三段論法より
		\begin{align}
			a = b,\ a = c,\ \EQAX \vdash b = a
		\end{align}
		となり,同様に相等性公理から
		\begin{align}
			a = b,\ a = c,\ \EQAX \vdash b = a \rarrow (\, \tau \in b \rarrow \tau \in a\, )
		\end{align}
		が成り立つので,三段論法より
		\begin{align}
			a = b,\ a = c,\ \EQAX \vdash \tau \in b \rarrow \tau \in a
			\label{fom:thm_transitive_law_of_equality_2}
		\end{align}
		となる.論理積の導入規則より
		\begin{align}
			a = b,\ a = c,\ \EQAX \vdash (\tau \in a \rarrow \tau \in b)
			\rarrow ((\tau \in b \rarrow \tau \in a) \rarrow 
			(\tau \in a \lrarrow \tau \in b))
		\end{align}
		が成り立つので,(\refeq{fom:thm_transitive_law_of_equality_1})との三段論法より
		\begin{align}
			a = b,\ a = c,\ \EQAX \vdash (\tau \in b \rarrow \tau \in a) \rarrow 
			(\tau \in a \lrarrow \tau \in b)
		\end{align}
		となり,(\refeq{fom:thm_transitive_law_of_equality_2})との三段論法より
		\begin{align}
			a = b,\ a = c,\ \EQAX \vdash \tau \in a \lrarrow \tau \in b
			\label{fom:thm_transitive_law_of_equality_4}
		\end{align}
		となる.対称的に
		\begin{align}
			a = b,\ a = c,\ \EQAX \vdash \tau \in a \lrarrow \tau \in c
			\label{fom:thm_transitive_law_of_equality_3}
		\end{align}
		も得られる.ここで含意の可換律
		(推論法則\ref{logicalthm:commutative_law_of_equivalence_symbol})より
		\begin{align}
			a = b,\ a = c,\ \EQAX \vdash (\, \tau \in a \lrarrow \tau \in b\, )
			\rarrow (\, \tau \in b \lrarrow \tau \in a\, ) 
		\end{align}
		が成り立つので,(\refeq{fom:thm_transitive_law_of_equality_4})との三段論法より
		\begin{align}
			a = b,\ a = c,\ \EQAX \vdash \tau \in b \lrarrow \tau \in a
			\label{fom:thm_transitive_law_of_equality_5}
		\end{align}
		となる.また含意の推移律
		(推論法則\ref{logicalthm:transitive_law_of_equivalence_symbol})より
		\begin{align}
			a = b,\ a = c,\ \EQAX \vdash (\, \tau \in b \lrarrow \tau \in a\, )
			\rarrow ((\, \tau \in a \lrarrow \tau \in c\, )
			\rarrow (\, \tau \in b \lrarrow \tau \in c\, )) 
		\end{align}
		が成り立つので,(\refeq{fom:thm_transitive_law_of_equality_5})との三段論法より
		\begin{align}
			a = b,\ a = c,\ \EQAX \vdash (\, \tau \in a \lrarrow \tau \in c\, )
			\rarrow (\, \tau \in b \lrarrow \tau \in c\, )
		\end{align}
		となり,(\refeq{fom:thm_transitive_law_of_equality_3})との三段論法より
		\begin{align}
			a = b,\ a = c,\ \EQAX \vdash \tau \in b \lrarrow \tau \in c
		\end{align}
		が得られる.全称記号の推論規則より
		\begin{align}
			a = b,\ a = c,\ \EQAX \vdash (\tau \in b \lrarrow \tau \in c)
			\rarrow \forall x\, (\, x \in b \lrarrow x \in c\, )
		\end{align}
		となるので,三段論法より
		\begin{align}
			a = b,\ a = c,\ \EQAX \vdash \forall x\, (\, x \in b \lrarrow x \in c\, )
		\end{align}
		となり,外延性公理より
		\begin{align}
			a = b,\ a = c,\ \EXTAX,\EQAX \vdash \forall x\, (\, x \in b \lrarrow x \in c\, )
			\rarrow b = c
		\end{align}
		となるので,三段論法より
		\begin{align}
			a = b,\ a = c,\ \EXTAX,\EQAX \vdash b = c
		\end{align}
		が得られる.
		\QED
	\end{sketch}
	
	\begin{itembox}[l]{等号の対称律と推移律について}
		本稿では等号の対称律
		\begin{align}
			a = b \rarrow b = a
		\end{align}
		を公理としたが,逆に推移律を公理にすれば
		\begin{align}
			\EXTAX,\EQAX \vdash a = b \rarrow b = a
		\end{align}
		が成立する.実際
		\begin{align}
			a = b,\ \EQAX &\vdash a = a \rarrow b = a, && \\
			\EXTAX &\vdash a = a, 
			&& (\mbox{定理\ref{thm:any_class_equals_to_itself}}), \\
			a = b,\ \EXTAX,\EQAX &\vdash b = a
			&& (\mbox{三段論法})
		\end{align}
		となる.つまり等号の対称律と推移律は外延性公理の下で同値なのである.
	\end{itembox}
	\section{変換の同値性}
	\begin{screen}
		\begin{logicalthm}[同値記号の対称律]
		\label{logicalthm:symmetry_of_equivalence_arrows}
			$A,B$を$\mathcal{L}$の文とするとき
			\begin{align}
				\vdash (A \lrarrow B) \rarrow (B \lrarrow A).
			\end{align}
		\end{logicalthm}
	\end{screen}
	
	\begin{prf}
		$\wedge$の除去(推論規則\ref{logicalaxm:elimination_of_conjunction})より
		\begin{align}
			A \lrarrow B &\vdash A \rarrow B, \\
			A \lrarrow B &\vdash B \rarrow A
		\end{align}
		となる.他方で論理積の導入(推論規則\ref{logicalaxm:introduction_of_conjunction})より
		\begin{align}
			\vdash (B \rarrow A) \rarrow ((A \rarrow B) \rarrow 
			(B \rarrow A) \wedge (A \rarrow B))
		\end{align}
		が成り立つので,三段論法を二回適用すれば
		\begin{align}
			A \lrarrow B \vdash (B \rarrow A) \wedge (A \rarrow B)
		\end{align}
		となる.つまり
		\begin{align}
			A \lrarrow B \vdash B \lrarrow A
		\end{align}
		が得られた.
		\QED
	\end{prf}
	
	\begin{itemize}
		\item $a = \Set{z}{\psi(z)} \rarrow \forall v\, (\, v \in a \lrarrow \psi(v)\, )$
		\item $\Set{y}{\varphi(y)} = b \rarrow \forall u\, (\, \varphi(u) \lrarrow u \in b\, )$
		\item $\Set{y}{\varphi(y)} = \Set{z}{\psi(z)} \rarrow \forall u\, (\, \varphi(u) \lrarrow \psi(u)\, )$
		\item $a \in \Set{z}{\psi(z)} \rarrow \psi(a)$
		\item $\Set{y}{\varphi(y)} \in b
			\rarrow \exists s\, (\, \forall u\, (\, \varphi(u) \lrarrow u \in s\, )
			\wedge s \in b\, )$
		\item $\Set{y}{\varphi(y)} \in \Set{z}{\psi(z)} 
			\rarrow \exists s\, (\, \forall u\, (\, \varphi(u) \lrarrow u \in s\, )
			\wedge \psi(s)\, )$
	\end{itemize}
	
	\begin{screen}
		\begin{thm}
		\label{thm:equivalent_formula_rewriting_1}
			$a$を主要$\varepsilon$項とし,$\psi$を$\lang{\varepsilon}$の式とし,
			$z$を$\psi$に自由に現れる変項とし,$\psi$に自由に現れる変項は$z$のみであるとする.このとき
			\begin{align}
				\EQAX,\COMAX \vdash a = \Set{z}{\psi(z)} 
				\rarrow \forall v\, (\, v \in a \lrarrow \psi(v)\, ).
			\end{align}
		\end{thm}
	\end{screen}
	
	\begin{sketch}
		いま
		\begin{align}
			\tau \defeq \varepsilon v \negation (\, v \in a \lrarrow \psi(v)\, )
		\end{align}
		とおく.外延性公理の逆(定理\ref{thm:inverse_of_axiom_of_extensionality})より
		\begin{align}
			a = \Set{z}{\psi(z)},\ \EQAX \vdash 
			\tau \in a \lrarrow \tau \in \Set{z}{\psi(z)}
		\end{align}
		が成り立ち,他方で内包性公理より
		\begin{align}
			\COMAX \vdash \tau \in \Set{z}{\psi(z)} \lrarrow \psi(\tau)
		\end{align}
		が成り立つので,同値記号の推移律
		(推論法則\ref{logicalthm:transitive_law_of_equivalence_symbol})より
		\begin{align}
			a = \Set{z}{\psi(z)},\ \EQAX,\COMAX \vdash \tau \in a \lrarrow \psi(\tau)
		\end{align}
		が従う.そして全称記号の推論規則より
		\begin{align}
			a = \Set{z}{\psi(z)},\ \EQAX,\COMAX \vdash 
			\forall v\, (\, v \in a \lrarrow \psi(v)\, )
		\end{align}
		が得られる.
		\QED
	\end{sketch}
	
	\begin{screen}
		\begin{thm}
		\label{thm:equivalent_formula_rewriting_3}
			$b$を主要$\varepsilon$項とし,$\varphi$を$\lang{\varepsilon}$の式とし,
			$y$を$\varphi$に自由に現れる変項とし,$\varphi$に自由に現れる変項は$y$のみ
			であるとする.このとき
			\begin{align}
				\EQAX,\COMAX \vdash \Set{y}{\varphi(y)} = b 
				\rarrow \forall u\, (\, \varphi(u) \lrarrow u \in b\, ).
			\end{align}
		\end{thm}
	\end{screen}
	
	\begin{sketch}
		いま
		\begin{align}
			\tau \defeq \varepsilon u \negation (\, \varphi(u) \lrarrow u \in b\, )
		\end{align}
		とおけば,まず外延性公理の逆(定理\ref{thm:inverse_of_axiom_of_extensionality})より
		\begin{align}
			\Set{y}{\varphi(y)} = b,\ \EQAX \vdash 
			\tau \in \Set{z}{\psi(z)} \lrarrow \tau \in b
			\label{fom:equivalent_formula_rewriting_3_1}
		\end{align}
		が成り立つ.他方で内包性公理より
		\begin{align}
			\COMAX \vdash \tau \in \Set{y}{\varphi(y)} \lrarrow \varphi(\tau)
		\end{align}
		となり,同値記号の対称律(\ref{logicalthm:symmetry_of_equivalence_arrows})より
		\begin{align}
			\COMAX \vdash \varphi(\tau) \lrarrow \tau \in \Set{y}{\varphi(y)}
			\label{fom:equivalent_formula_rewriting_3_2}
		\end{align}
		が成り立つ.(\refeq{fom:equivalent_formula_rewriting_3_1})と
		(\refeq{fom:equivalent_formula_rewriting_3_2})と同値記号の推移律
		(推論法則\ref{logicalthm:transitive_law_of_equivalence_symbol})より
		\begin{align}
			\Set{y}{\varphi(y)} = b,\ \EQAX,\COMAX \vdash 
			\varphi(\tau) \lrarrow \tau \in b 
		\end{align}
		が成り立ち,全称記号の推論規則より
		\begin{align}
			\Set{y}{\varphi(y)} = b,\ \EQAX,\COMAX \vdash 
			\forall u\, (\, \varphi(u) \lrarrow u \in b\, )
		\end{align}
		が得られる.
		\QED
	\end{sketch}
	
	\begin{screen}
		\begin{thm}
			$a$を主要$\varepsilon$項とし,$\psi$を$\lang{\varepsilon}$の式とし,
			$z$を$\psi$に自由に現れる変項とし,$\psi$に自由に現れる変項は$z$のみであるとする.このとき
			\begin{align}
				\COMAX \vdash a \in \Set{z}{\psi(z)} \rarrow \psi(a).
			\end{align}
		\end{thm}
	\end{screen}
	
	\begin{sketch}
		$a$は主要$\varepsilon$項であるから,内包性公理より
		\begin{align}
			\COMAX \vdash a \in \Set{z}{\psi(z)} \rarrow \psi(a)
		\end{align}
		が成り立つ.
		\QED
	\end{sketch}
	
	\begin{screen}
		\begin{thm}
			$a$を主要$\varepsilon$項とし,$\psi$を$\lang{\varepsilon}$の式とし,
			$z$を$\psi$に自由に現れる変項とし,$\psi$に自由に現れる変項は$z$のみであるとする.このとき
			\begin{align}
				\COMAX \vdash \psi(a) \rarrow a \in \Set{z}{\psi(z)}.
			\end{align}
		\end{thm}
	\end{screen}
	
	\begin{sketch}
		$a$は主要$\varepsilon$項であるから,内包性公理より
		\begin{align}
			\COMAX \vdash \psi(a) \rarrow a \in \Set{z}{\psi(z)}
		\end{align}
		が成り立つ.
		\QED
	\end{sketch}
	
	\begin{screen}
		\begin{thm}
		\label{thm:equivalent_formula_rewriting_9}
			$b$を主要$\varepsilon$項とし,$\varphi$を$\lang{\varepsilon}$の式とし,
			$y$を$\varphi$に自由に現れる変項とし,
			$\varphi$に自由に現れる変項は$y$のみであるとする.このとき
			\begin{align}
				\EQAX,\COMAX,\ELEAX \vdash \Set{y}{\varphi(y)} \in b
				\rarrow \exists s\, (\, 
				\forall u\, (\, \varphi(u) \lrarrow u \in s\, )
				\wedge s \in b\, ).
			\end{align}
		\end{thm}
	\end{screen}
	
	\begin{sketch}
		要素の公理より
		\begin{align}
			\Set{y}{\varphi(y)} \in b,\ \ELEAX \vdash 
			\exists s\, (\, \Set{y}{\varphi(y)} = s\, )
		\end{align}
		が成り立つので,
		\begin{align}
			\sigma \defeq 
			\varepsilon s\, \forall u\, (\, \varphi(u) \lrarrow u \in s\, )
		\end{align}
		とおけば存在記号の推論規則より
		\begin{align}
			\Set{y}{\varphi(y)} \in b,\ \ELEAX \vdash \Set{y}{\varphi(y)} = \sigma
			\label{fom:equivalent_formula_rewriting_9_1}
		\end{align}
		となる.ここで相等性公理より
		\begin{align}
			\EQAX \vdash \Set{y}{\varphi(y)} = \sigma
			\rarrow (\, \Set{y}{\varphi(y)} \in b \rarrow \sigma \in b\, )
		\end{align}
		が成り立つので,(\refeq{fom:equivalent_formula_rewriting_9_1})と三段論法より
		\begin{align}
			\Set{y}{\varphi(y)} \in b,\ \EQAX,\ELEAX \vdash \sigma \in b
			\label{fom:equivalent_formula_rewriting_9_2}
		\end{align}
		が得られる.他方で定理\ref{thm:equivalent_formula_rewriting_3}より
		\begin{align}
			\EQAX,\COMAX \vdash \Set{y}{\varphi(y)} = \sigma
			\rarrow \forall u\, (\, \varphi(u) \lrarrow u \in \sigma\, )
		\end{align}
		が成り立つので,(\refeq{fom:equivalent_formula_rewriting_9_1})と三段論法より
		\begin{align}
			\Set{y}{\varphi(y)} \in b,\ \EQAX,\COMAX,\ELEAX \vdash
			\forall u\, (\, \varphi(u) \lrarrow u \in \sigma\, )
			\label{fom:equivalent_formula_rewriting_9_3}
		\end{align}
		も得られる.(\refeq{fom:equivalent_formula_rewriting_9_2})と
		(\refeq{fom:equivalent_formula_rewriting_9_3})と論理積の導入規則より
		\begin{align}
			\Set{y}{\varphi(y)} \in b,\ \EQAX,\COMAX,\ELEAX \vdash
			\forall u\, (\, \varphi(u) \lrarrow u \in \sigma\, ) \wedge \sigma \in b
		\end{align}
		が成り立つので,存在記号の推論規則より
		\begin{align}
			\Set{y}{\varphi(y)} \in b,\ \EQAX,\COMAX,\ELEAX \vdash
			\exists s\, (\, \forall u\, (\, \varphi(u) \lrarrow u \in s\, ) \wedge s \in b\, )
		\end{align}
		が得られる.
		\QED
	\end{sketch}
	
	\begin{screen}
		\begin{thm}
		\label{thm:equivalent_formula_rewriting_11}
			$\varphi$と$\psi$を$\lang{\varepsilon}$の式とし,
			$y$を$\varphi$に自由に現れる変項とし,
			$z$を$\psi$に自由に現れる変項とし,
			$\varphi$に自由に現れる変項は$y$のみであるとし,
			$\psi$に自由に現れる変項は$z$のみであるとし,する.このとき
			\begin{align}
				\EQAX,\COMAX,\ELEAX \vdash \Set{y}{\varphi(y)} \in \Set{z}{\psi(z)}
				\rarrow \exists s\, (\, 
				\forall u\, (\, \varphi(u) \lrarrow u \in s\, )
				\wedge \psi(s)\, ).
			\end{align}
		\end{thm}
	\end{screen}
	
	\begin{sketch}
		まず(\refeq{fom:equivalent_formula_rewriting_9_1})と
		(\refeq{fom:equivalent_formula_rewriting_9_3})と同様に,
		\begin{align}
			\sigma \defeq 
			\varepsilon s\, \forall u\, (\, \varphi(u) \lrarrow u \in s\, )
		\end{align}
		とおけば
		\begin{align}
			\Set{y}{\varphi(y)} \in \Set{z}{\psi(z)},\ \ELEAX \vdash 
			\Set{y}{\varphi(y)} = \sigma
			\label{fom:equivalent_formula_rewriting_11_1}
		\end{align}
		と
		\begin{align}
			\Set{y}{\varphi(y)} \in \Set{z}{\psi(z)},\ \EQAX,\COMAX,\ELEAX \vdash
			\forall u\, (\, \varphi(u) \lrarrow u \in \sigma\, )
			\label{fom:equivalent_formula_rewriting_11_2}
		\end{align}
		が成り立つ.また相等性公理より
		\begin{align}
			\EQAX \vdash \Set{y}{\varphi(y)} = \sigma
			\rarrow (\, \Set{y}{\varphi(y)} \in \Set{z}{\psi(z)}
			\rarrow \sigma \in \Set{z}{\psi(z)}\, )
		\end{align}
		となるので,(\refeq{fom:equivalent_formula_rewriting_11_1})との三段論法より
		\begin{align}
			\Set{y}{\varphi(y)} \in \Set{z}{\psi(z)},\ \EQAX,\ELEAX \vdash 
			\sigma \in \Set{z}{\psi(z)}
		\end{align}
		が成り立ち,内包性公理より
		\begin{align}
			\COMAX \vdash \sigma \in \Set{z}{\psi(z)} \rarrow \psi(\sigma)
		\end{align}
		が成り立つので
		\begin{align}
			\Set{y}{\varphi(y)} \in \Set{z}{\psi(z)},\ \EQAX,\ELEAX \vdash 
			\psi(\sigma)
			\label{fom:equivalent_formula_rewriting_11_3}
		\end{align}
		が得られる.(\refeq{fom:equivalent_formula_rewriting_11_2})と
		(\refeq{fom:equivalent_formula_rewriting_11_3})と論理積の導入規則より
		\begin{align}
			\Set{y}{\varphi(y)} \in \Set{z}{\psi(z)},\ \EQAX,\COMAX,\ELEAX \vdash
			\forall u\, (\, \varphi(u) \lrarrow u \in \sigma\, ) \wedge \psi(\sigma)
		\end{align}
		が成り立ち,存在記号の推論規則より
		\begin{align}
			\Set{y}{\varphi(y)} \in \Set{z}{\psi(z)},\ \EQAX,\COMAX,\ELEAX \vdash
			\exists s\, (\forall u\, (\, \varphi(u) \lrarrow u \in x\, ) \wedge \psi(x)\, )
		\end{align}
		が得られる.
		\QED
	\end{sketch}
	\section{代入原理}
	$a$と$b$を類とし,$\varphi$を$x$のみが自由に現れる式とするとき,
	\begin{align}
		a = b
	\end{align}
	ならば$a$と$b$をそれぞれ$\varphi$の自由な$x$に代入しても
	\begin{align}
		\varphi(a) \lrarrow \varphi(b)
	\end{align}
	が成立するというのは{\bf 代入原理}\index{だいにゅうげんり@代入原理}
	{\bf (the principle of substitution)}と呼ばれる.
	この原理の証明は相等性公理に負うところが多いが,
	本稿では$\varepsilon$項という厄介なものを抱え込んでいるため
	$\EQAX$だけでは不十分であり,次に追加する公理が必要になる.
	
	\begin{screen}
		\begin{axm}[$\varepsilon$項に対する相等性公理]
			$a,b$を類とし,$\varphi$を$\lang{\varepsilon}$の式とし,$\varphi$には変項$x,y$が
			自由に現れ,また$\varphi$に自由に現れる変項はこれらのみであるとする.このとき
			\begin{align}
				\EQAXEP \defarrow
				a = b \rarrow \varepsilon x \varphi(x,a) = \varepsilon x \varphi(x,b).
			\end{align}
		\end{axm}
	\end{screen}
	
	代入原理を示すには構造的帰納法の原理が必要になるので,証明はメタなものとなる.
	
	\begin{screen}
		\begin{thm}[代入原理]\label{thm:the_principle_of_substitution}
			$a,b$を類とし,$\varphi$を$\mathcal{L}$の式とし,$x$を$\varphi$に自由に現れる変項
			とし,$\varphi$に自由に現れる変項は$x$のみであるとする.このとき
			\begin{align}
				\EXTAX,\EQAX,\EQAXEP \vdash a = b \rarrow 
				(\, \varphi(a) \lrarrow \varphi(b)\, ).
			\end{align}
		\end{thm}
	\end{screen}
	
	\begin{sketch}\mbox{}
		\begin{description}
			\item[step1]
				$c$を類として,$\varphi$が
				\begin{align}
					x \in c
				\end{align}
				なる式であるとき,
				\begin{align}
					a = b,\ \EQAX \vdash a \in c \rarrow b \in c
				\end{align}
				となる.また
				\begin{align}
					a = b,\ \EXTAX,\EQAX \vdash b = a
				\end{align}
				より
				\begin{align}
					a = b,\ \EXTAX,\EQAX \vdash b \in c \rarrow a \in c
				\end{align}
				も成り立つ.ゆえに
				\begin{align}
					a = b,\ \EXTAX,\EQAX \vdash a \in c \lrarrow b \in c
				\end{align}
				が得られる.同様に
				\begin{align}
					a = b,\ \EXTAX,\EQAX \vdash c \in a \lrarrow c \in b
				\end{align}
				も得られる.
				
			\item[step2]
				$\varphi$が
				\begin{align}
					x \in \varepsilon y\, R(x,y)
				\end{align}
				なる式であるとき,
				\begin{align}
					a = b,\ \EQAXEP \vdash \varepsilon y\, R(a,y) = \varepsilon y\, R(b,y)
				\end{align}
				となる.
				\begin{align}
					a = b,\ \EXTAX,\EQAX 
					\vdash (\, \varepsilon y\, R(a,y) = \varepsilon y\, R(b,y)\, )
					\rarrow (\, a \in \varepsilon y\, R(a,y) \lrarrow a \in \varepsilon y\, R(b,y)\, )
				\end{align}
				なので
				\begin{align}
					a = b,\ \EXTAX,\EQAX,\EQAXEP \vdash 
					a \in \varepsilon y\, R(a,y) \lrarrow a \in \varepsilon y\, R(b,y)
				\end{align}
				となる.
				\begin{align}
					a=b,\ \EXTAX,\EQAX \vdash 
					a \in \varepsilon y\, R(b,y) \lrarrow b \in \varepsilon y\, R(b,y)
				\end{align}
				も成り立つので
				\begin{align}
					a=b,\ \EXTAX,\EQAX,\EQAXEP \vdash 
					a \in \varepsilon y\, R(a,y) \lrarrow b \in \varepsilon y\, R(b,y)
				\end{align}
				が得られる.
				
			\item[step3]
				$\varphi$が
				\begin{align}
					\varepsilon y R(x,y) \in \varepsilon z T(x,z)
				\end{align}
				なる形のとき,
				\begin{align}
					a = b,\ \EQAXEP \vdash 
					\varepsilon y R(a,y) = \varepsilon y R(b,y)
				\end{align}
				と
				\begin{align}
					a = b,\ \EQAX, \EQAXEP \vdash 
					(\, \varepsilon y R(a,y) = \varepsilon y R(b,y)\, )
					\rarrow (\, \varepsilon y R(a,y) \in \varepsilon z T(a,z)
					\lrarrow \varepsilon y R(b,y) \in \varepsilon z T(a,z)\, )
				\end{align}
				より
				\begin{align}
					a = b,\ \EQAX, \EQAXEP \vdash 
					\varepsilon y R(a,y) \in \varepsilon z T(a,z)
					\lrarrow \varepsilon y R(b,y) \in \varepsilon z T(a,z)
				\end{align}
				が成り立つ.他方で
				\begin{align}
					a = b,\ \EQAXEP \vdash 
					\varepsilon z T(a,z) = \varepsilon z T(b,z)
				\end{align}
				と
				\begin{align}
					a = b,\ \EQAX, \EQAXEP \vdash 
					(\, \varepsilon z T(a,z) = \varepsilon z T(b,z)\, )
					\rarrow (\, \varepsilon y R(b,y) \in \varepsilon z T(a,z)
					\lrarrow \varepsilon y R(b,y) \in \varepsilon z T(b,z)\, )
				\end{align}
				より
				\begin{align}
					a = b,\ \EQAX, \EQAXEP \vdash 
					\varepsilon y R(b,y) \in \varepsilon z T(a,z)
					\lrarrow \varepsilon y R(b,y) \in \varepsilon z T(b,z)
				\end{align}
				が成り立つ.同値関係の推移律
				(\ref{logicalthm:transitive_law_of_equivalence_symbol})より
				\begin{align}
					a = b,\ \EQAX, \EQAXEP \vdash 
					\varepsilon y R(a,y) \in \varepsilon z T(a,z)
					\lrarrow \varepsilon y R(b,y) \in \varepsilon z T(b,z)
				\end{align}
				が成立する.
		\end{description}
	\end{sketch}
	\section{空集合}
	\begin{screen}
		\begin{logicalthm}[分配された論理積の簡約]
		\label{logicalthm:contraction_law_of_distributed_injunctions}
			$A,B,C$を$\mathcal{L}$の文とするとき,
			\begin{align}
				\vdash (A \wedge C) \wedge (B \wedge C) \rarrow A \wedge B.
			\end{align}
		\end{logicalthm}
	\end{screen}
	
	\begin{sketch}
		論理積の除去規則より
		\begin{align}
			(A \wedge C) \wedge (B \wedge C) \vdash A \wedge C
		\end{align}
		となり,また同じく論理積の除去規則より
		\begin{align}
			(A \wedge C) \wedge (B \wedge C) &\vdash A \wedge C \rarrow A
		\end{align}
		となるので,三段論法より
		\begin{align}
			(A \wedge C) \wedge (B \wedge C) &\vdash A, 
			\label{fom:logicalthm_contraction_law_of_injunctions_1}
		\end{align}
		が従う.同様にして
		\begin{align}
			(A \wedge C) \wedge (B \wedge C) \vdash B
			\label{fom:logicalthm_contraction_law_of_injunctions_2}
		\end{align}
		も得られる.ここで論理積の導入規則より
		\begin{align}
			(A \wedge C) \wedge (B \wedge C) \vdash A \rarrow (B \rarrow A \wedge B)
		\end{align}
		が成り立つので,(\refeq{fom:logicalthm_contraction_law_of_injunctions_1})と
		(\refeq{fom:logicalthm_contraction_law_of_injunctions_2})との三段論法より
		\begin{align}
			(A \wedge C) \wedge (B \wedge C) \vdash A \wedge B
		\end{align}
		が出る.
		\QED
	\end{sketch}
	
	\begin{screen}
		\begin{dfn}[空集合]
			$\emptyset \defeq \Set{x}{x \neq x}$で定める類$\emptyset$を{\bf 空集合}\index{くうしゅうごう@空集合}{\bf (empty set)}と呼ぶ.
		\end{dfn}
	\end{screen}
	
	$x$が集合であれば
	\begin{align}
		x = x
	\end{align}
	が成り立つので,$\emptyset$に入る集合など存在しない.
	つまり$\emptyset$は丸っきり``空っぽ''なのである.
	さて,$\emptyset$は集合であるか否か,という問題を考える.
	当然これが``大きすぎる集まり''であるはずはないし,
	そもそも名前に``集合''と付いているのだから
	$\emptyset$は集合であるべきだと思われるのだが,
	実際にこれが集合であることを示すには少し骨が折れる.
	まずは置換公理と分出定理を拵えなくてはならない.
	
	\begin{screen}
		\begin{axm}[置換公理]
			$\varphi$を$\mathcal{L}$の式とし,
			$s,t$を$\varphi$に自由に現れる変項とし,
			$\varphi$に自由に現れる項は$s,t$のみであるとし,
			$x$は$\varphi$で$s$への代入について自由であり,
			$y,z,v$は$\varphi$で$t$への代入について自由であるとするとき,
			\begin{align}
				\REPAX \defarrow \forall x\, \forall y\, \forall z\, 
				(\, \varphi(x,y) \wedge \varphi(x,z)
				\rarrow y = z\, )
				\rarrow \forall a\, \exists u\, \forall v\,
				(\, v \in u \lrarrow \exists x\, (\, x \in a \wedge 
				\varphi(x,v)\, )\, ).
			\end{align}
		\end{axm}
	\end{screen}
	
	$\Set{x}{\varphi(x)}$は集合であるとは限らないが,
	集合$a$に対して
	\begin{align}
		a \cap \Set{x}{\varphi(x)}
	\end{align}
	なる類は当然$a$より``小さい集まり''なのだから,集合であってほしいものである.
	置換公理によってそのこと保証され,分出定理として知られている.
	
	\begin{screen}
		\begin{thm}[分出定理]\label{thm:axiom_of_separation}
			$\varphi$を$\mathcal{L}$の式とし,$x$を$\varphi$に自由に現れる変項とし,
			$\varphi$に自由に現れる項は$x$のみであるとする.このとき
			\begin{align}
				\EXTAX,\EQAX,\EQAXEP,\REPAX \vdash 
				\forall a\, \exists s\, \forall x\,
				(\, x \in s \lrarrow x \in a \wedge \varphi(x)\, ).
				\label{fom:thm_axiom_of_separation_0}
			\end{align}
		\end{thm}
	\end{screen}
	
	\begin{sketch}
		$y$を,$\varphi$の$x$への代入について自由である変項とする.
		そして$x$と$y$が自由に現れる式$\psi(x,y)$を
		\begin{align}
			x = y \wedge \varphi(x)
		\end{align}
		と設定する.
		\begin{description}
			\item[step1]
				まず
				\begin{align}
					\EXTAX,\EQAX \vdash \forall x\, \forall y\, \forall z\, 
					(\, \psi(x,y) \wedge \psi(x,z) \rarrow y = z\, )
					\label{fom:thm_axiom_of_separation_1}
				\end{align}
				が成り立つことを示す.これを見越して
				\begin{align}
					\tau &\defeq \varepsilon x \negation \forall y\, \forall z\, 
					(\, \psi(x,y) \wedge \psi(x,z) \rarrow y = z\, ), \\
					\sigma &\defeq \varepsilon y \negation \forall z\, 
					(\, \psi(\tau,y) \wedge \psi(\tau,z) \rarrow y = z\, ), \\
					\rho &\defeq \varepsilon z \negation 
					(\, \psi(\tau,\sigma) \wedge \psi(\tau,z) \rarrow \sigma = z\, )
				\end{align}
				とおく.$\psi(\tau,\sigma) \wedge \psi(\tau,\rho)$は縮約可能であって(
				推論法則\ref{logicalthm:contraction_law_of_distributed_injunctions})
				\begin{align}
					\vdash (\, \tau = \sigma \wedge \varphi(\tau)\, )
						\wedge (\, \tau = \rho \wedge \varphi(\tau)\, )
						\rarrow \tau = \sigma \wedge \tau = \rho
				\end{align}
				が成り立つので
				\begin{align}
					\psi(\tau,\sigma) \wedge \psi(\tau,\rho) 
					\vdash \tau = \sigma \wedge \tau = \rho
				\end{align}
				がとなり,さらに論理積の除去法則より
				\begin{align}
					\psi(\tau,\sigma) \wedge \psi(\tau,\rho) &\vdash \tau = \sigma, \\
					\psi(\tau,\sigma) \wedge \psi(\tau,\rho) &\vdash \tau = \rho
				\end{align}
				が出る.ここで等号の推移律(定理\ref{thm:transitive_law_of_equality})より
				\begin{align}
					\EXTAX,\EQAX \vdash \tau = \sigma \rarrow 
						(\, \tau = \rho \rarrow \sigma = \rho\, )
				\end{align}
				が成り立つので,三段論法を二回用いれば
				\begin{align}
					\psi(\tau,\sigma) \wedge \psi(\tau,\rho),\ \EXTAX,\EQAX 
					\vdash \sigma = \rho
				\end{align}
				が得られる.ゆえに演繹法則より
				\begin{align}
					\EXTAX,\EQAX \vdash \psi(\tau,\sigma) \wedge \psi(\tau,\rho)
					\rarrow \sigma = \rho
				\end{align}
				となり,全称記号の推論規則より
				\begin{align}
					\EXTAX,\EQAX &\vdash \forall z\, 
						(\, \psi(\tau,\sigma) \wedge \psi(\tau,z) 
						\rarrow \sigma = z\, ), \\
					\EXTAX,\EQAX &\vdash \forall y\, \forall z\, 
						(\, \psi(\tau,y) \wedge \psi(\tau,z) \rarrow y = z\, ), \\
					\EXTAX,\EQAX &\vdash \forall x\, \forall y\, \forall z\, 
						(\, \psi(x,y) \wedge \psi(x,z) \rarrow y = z\, )
				\end{align}
				が従う.
			
			\item[step2]
				置換公理より
				\begin{align}
					\REPAX \vdash \forall x\, \forall y\, \forall z\, 
					(\, \psi(x,y) \wedge \psi(x,z)
					\rarrow y = z\, )
					\rarrow \forall a\, \exists u\, \forall v\,
					(\, v \in u \lrarrow \exists x\, (\, x \in a \wedge 
					\psi(x,v)\, )\, )
				\end{align}
				が成り立つので,(\refeq{fom:thm_axiom_of_separation_1})との三段論法より
				\begin{align}
					\EXTAX,\EQAX,\REPAX \vdash \forall a\, \exists u\, \forall v\,
					(\, v \in u \lrarrow \exists x\, (\, x \in a \wedge 
					\psi(x,v)\, )\, )
					\label{fom:thm_axiom_of_separation_2}
				\end{align}
				が成立する.(\refeq{fom:thm_axiom_of_separation_1})を示したいので
				\begin{align}
					\alpha \defeq \varepsilon a \negation \exists s\, \forall x\,
					(\, x \in s \lrarrow x \in a \wedge \varphi(x)\, )
				\end{align}
				とおくと,全称記号の推論規則より
				\begin{align}
					\EXTAX,\EQAX,\REPAX \vdash &\forall a\, \exists u\, \forall v\,
					(\, v \in u \lrarrow \exists x\, (\, x \in a \wedge 
					\psi(x,v)\, )\, ) \\
					&\rarrow \exists u\, \forall v\,
					(\, v \in u \lrarrow \exists x\, (\, x \in \alpha \wedge 
					\psi(x,v)\, )\, )
				\end{align}
				となるので,(\refeq{fom:thm_axiom_of_separation_2})との三段論法より
				\begin{align}
					\EXTAX,\EQAX,\REPAX \vdash \exists u\, \forall v\,
					(\, v \in u \lrarrow \exists x\, (\, x \in \alpha \wedge 
					\psi(x,v)\, )\, )
					\label{fom:thm_axiom_of_separation_3}
				\end{align}
				が従う.ここで
				\begin{align}
					\zeta \defeq \varepsilon u\, \forall v\,
					(\, v \in u \lrarrow \exists x\, (\, x \in \alpha \wedge 
					\psi(x,v)\, )\, )
				\end{align}
				とおけば,量化子の推論規則と(\refeq{fom:thm_axiom_of_separation_2})との
				三段論法により
				\begin{align}
					\EXTAX,\EQAX,\REPAX \vdash \forall v\,
					(\, v \in \zeta \lrarrow \exists x\, (\, x \in \alpha \wedge 
					\psi(x,v)\, )\, )
					\label{fom:thm_axiom_of_separation_4}
				\end{align}
				が成り立つ.
			
			\item[step3]
				最後に
				\begin{align}
					\EXTAX,\EQAX,\EQAXEP,\REPAX \vdash \forall x\,
					(\, x \in \zeta \lrarrow x \in \alpha \wedge \varphi(x)\, )
					\label{fom:thm_axiom_of_separation_8}
				\end{align}
				となることを示す.いま
				\begin{align}
					\tau \defeq \varepsilon x \negation
					(\, x \in \zeta \lrarrow x \in \alpha \wedge \varphi(x)\, )
				\end{align}
				とおけば,(\refeq{fom:thm_axiom_of_separation_4})と全称記号の推論規則より
				\begin{align}
					\EXTAX,\EQAX,\REPAX \vdash 
					\tau \in \zeta \lrarrow \exists x\, (\, x \in \alpha \wedge 
					\psi(x,\tau)\, )
					\label{fom:thm_axiom_of_separation_5}
				\end{align}
				が従う.ゆえに
				\begin{align}
					\tau \in \zeta,\ \EXTAX,\EQAX,\REPAX \vdash
					\exists x\, (\, x \in \alpha \wedge \psi(x,\tau)\, )
				\end{align}
				となる.ここで
				\begin{align}
					\sigma \defeq \varepsilon x\, (\, x \in \alpha \wedge
					\psi(x,\tau)\, )
				\end{align}
				とおけば
				\begin{align}
					\tau \in \zeta,\ \EXTAX,\EQAX,\REPAX \vdash
					\sigma \in \alpha \wedge \psi(\sigma,\tau)
				\end{align}
				となるので,
				\begin{align}
					\tau \in \zeta,\ \EXTAX,\EQAX,\REPAX &\vdash \sigma \in \alpha, \\
					\tau \in \zeta,\ \EXTAX,\EQAX,\REPAX &\vdash \sigma = \tau, \\
					\tau \in \zeta,\ \EXTAX,\EQAX,\REPAX &\vdash \varphi(\sigma)
				\end{align}
				が従う.ところで相等性公理と代入原理
				(定理\ref{thm:the_principle_of_substitution})より
				\begin{align}
					\tau \in \zeta,\ \EXTAX,\EQAX,\REPAX &\vdash 
						\sigma = \tau \rarrow (\, \sigma \in \alpha \rarrow
						\tau \in \alpha\, ), \\
					\tau \in \zeta,\ \EXTAX,\EQAX,\EQAXEP,\REPAX &\vdash
						\sigma = \tau \rarrow (\, \varphi(\sigma) \rarrow
						\varphi(\tau)\, ), \\
				\end{align}
				が成り立つので,三段論法より
				\begin{align}
					\tau \in \zeta,\ \EXTAX,\EQAX,\REPAX &\vdash \tau \in \alpha, \\
					\tau \in \zeta,\ \EXTAX,\EQAX,\EQAXEP,\REPAX &\vdash \varphi(\tau)
				\end{align}
				が従い
				\begin{align}
					\tau \in \zeta,\ \EXTAX,\EQAX,\EQAXEP,\REPAX \vdash
					\tau \in \alpha \wedge \varphi(\tau)
				\end{align}
				となる.以上で
				\begin{align}
					\EXTAX,\EQAX,\EQAXEP,\REPAX \vdash \tau \in \zeta \rarrow
					\tau \in \alpha \wedge \varphi(\tau)
					\label{fom:thm_axiom_of_separation_6}
				\end{align}
				が得られた.逆に定理\ref{thm:any_class_equals_to_itself}と併せて
				\begin{align}
					\tau \in \alpha \wedge \varphi(\tau),\ 
					\EXTAX,\EQAX,\EQAXEP,\REPAX \vdash
					\tau \in \alpha \wedge (\, \tau = \tau \wedge \varphi(\tau)\, )
				\end{align}
				が成り立つので,存在記号の推論規則より
				\begin{align}
					\tau \in \alpha \wedge \varphi(\tau),\ 
					\EXTAX,\EQAX,\EQAXEP,\REPAX \vdash
					\exists x\, (\, x \in \alpha \wedge \psi(x,\tau)\, )
				\end{align}
				となる.他方で(\refeq{fom:thm_axiom_of_separation_5})より
				\begin{align}
					\EXTAX,\EQAX,\REPAX \vdash 
					\exists x\, (\, x \in \alpha \wedge 
					\psi(x,\tau)\, ) \rarrow \tau \in \zeta
				\end{align}
				が成り立つので,三段論法より
				\begin{align}
					\tau \in \alpha \wedge \varphi(\tau),\ 
					\EXTAX,\EQAX,\EQAXEP,\REPAX \vdash \tau \in \zeta
				\end{align}
				が従う.以上で
				\begin{align}
					\EXTAX,\EQAX,\EQAXEP,\REPAX \vdash 
					\tau \in \alpha \wedge \varphi(\tau) \rarrow \tau \in \zeta
					\label{fom:thm_axiom_of_separation_7}
				\end{align}
				も得られた.(\refeq{fom:thm_axiom_of_separation_6})と
				(\refeq{fom:thm_axiom_of_separation_7})および存在記号の推論規則より
				(\refeq{fom:thm_axiom_of_separation_8})が出る.
				すると存在記号の推論規則より
				\begin{align}
					\EXTAX,\EQAX,\EQAXEP,\REPAX \vdash \exists s\, \forall x\,
					(\, x \in s \lrarrow x \in \alpha \wedge \varphi(x)\, )
				\end{align}
				となり,全称記号の推論規則より
				\begin{align}
					\EXTAX,\EQAX,\EQAXEP,\REPAX \vdash 
					\forall a\, \exists s\, \forall x\,
					(\, x \in s \lrarrow x \in a \wedge \varphi(x)\, )
				\end{align}
				が従う.
				\QED
		\end{description}
	\end{sketch}
	
	\begin{screen}
		\begin{thm}[$\emptyset$は集合]\label{thm:emptyset_is_a_set}
			$\emptyset$は集合である:
			\begin{align}
				\set{\emptyset}.
			\end{align}
		\end{thm}
	\end{screen}
	
	\begin{sketch}
		分出定理より
		\begin{align}
			\forall z\, \exists y\, \forall x\,
			(\, x \in y \lrarrow x \in z \wedge x \neq x\, )
			\label{fom:thm_emptyset_is_a_set_1}
		\end{align}
		が成立するが,この式から
		\begin{align}
			\exists y\, \forall x\, (\, x \in y \lrarrow x \neq x\, )
			\label{fom:thm_emptyset_is_a_set_2}
		\end{align}
		を示せる.これはすなわち$\emptyset$が集合であるということを示唆する.
		$\zeta$を勝手な$\varepsilon$項として,後々の便宜のために
		\begin{align}
			\sigma &\defeq \varepsilon y\, \forall x\,
			(\, x \in y \lrarrow x \in \zeta \wedge x \neq x\, ), \\
			\tau &\defeq \varepsilon x \negation
			(\, x \in \sigma \lrarrow x \neq x\, )
		\end{align}
		とおけば,(\refeq{fom:thm_emptyset_is_a_set_1})より
		\begin{align}
			\tau \in \sigma \lrarrow \tau \in \zeta \wedge \tau \neq \tau
		\end{align}
		が成立する.論理和の規則より
		\begin{align}
			\tau \in \zeta \wedge \tau \neq \tau \rarrow \tau \neq \tau
		\end{align}
		が満たされるので,まずは
		\begin{align}
			\tau \in \sigma \rarrow \tau \neq \tau
		\end{align}
		が得られる.また
		\begin{align}
			\tau = \tau
		\end{align}
		は正しいので,
		\begin{align}
			\tau = \tau \rarrow (\, \tau \notin \sigma \rarrow
			\tau = \tau\, )
		\end{align}
		と併せて
		\begin{align}
			\tau \notin \sigma \rarrow \tau = \tau
		\end{align}
		が成り立ち,対偶を取れば
		\begin{align}
			\tau \neq \tau \rarrow \tau \in \sigma
		\end{align}
		も得られる.ゆえに
		\begin{align}
			\forall x\, (\, x \in \sigma \lrarrow x \neq x\, )
		\end{align}
		が得られ,(\refeq{fom:thm_emptyset_is_a_set_2})が従う.
		\QED
	\end{sketch}
	
	\begin{screen}
		\begin{thm}[空集合は$\mathcal{L}$のいかなる対象も要素に持たない]\label{thm:emptyset_has_nothing}
			\begin{align}
				\forall x\, (\, x \notin \emptyset\, ).
			\end{align}
		\end{thm}
	\end{screen}
	
	\begin{sketch}
		$\tau$を$\mathscr{L}$の対象とするとき,類の公理より
		\begin{align}
			\tau \in \emptyset \rarrow \tau \neq \tau
		\end{align}
		が成り立つから,対偶を取れば
		\begin{align}
			\tau = \tau \rarrow \tau \notin \emptyset
		\end{align}
		が成り立つ(推論法則\ref{thm:contraposition_is_true}).定理\ref{thm:any_class_equals_to_itself}より
		\begin{align}
			\tau = \tau
		\end{align}
		は正しいので,三段論法より
		\begin{align}
			\tau \notin \emptyset
		\end{align}
		が成り立つ.そして$\tau$の任意性より
		\begin{align}
			\forall x\, (\, x \notin \emptyset\, )
		\end{align}
		が得られる.
		\QED
	\end{sketch}
	
	\begin{screen}
		\begin{thm}[$\mathcal{L}$のいかなる対象も要素に持たない類は空集合に等しい]
		\label{thm:uniqueness_of_emptyset}
			$a$を類とするとき次が成り立つ:
			\begin{align}
				\forall x\, (\, x \notin a\, ) \lrarrow a = \emptyset.
			\end{align}
		\end{thm}
	\end{screen}
	
	\begin{prf}
		$a$を類として$\forall x\, (\, x \notin a\, )$が成り立っていると仮定する.このとき
		$\tau$を$\mathcal{L}$の任意の対象とすれば
		\begin{align}
			\tau \notin a \vee \tau \in \emptyset
		\end{align}
		と
		\begin{align}
			\tau \notin \emptyset \vee \tau \in a
		\end{align}
		が共に成り立つので,推論法則\ref{logicalthm:rule_of_inference_3}より
		\begin{align}
			\tau \in a \rarrow \tau \in \emptyset
		\end{align}
		と
		\begin{align}
			\tau \in \emptyset \rarrow \tau \in a
		\end{align}
		が共に成り立つ.よって
		\begin{align}
			\tau \in a \lrarrow \tau \in \emptyset
		\end{align}
		が成立し,$\tau$の任意性と推論法則\ref{logicalthm:fundamental_law_of_universal_quantifier}から
		\begin{align}
			\forall x\, (\, x \in a \lrarrow x \in \emptyset\, )
		\end{align}
		が得られる.ゆえに外延性の公理より
		\begin{align}
			a = \emptyset
		\end{align}
		が成立し,演繹法則より
		\begin{align}
			\forall x\, (\, x \notin a\, ) \rarrow a = \emptyset
		\end{align}
		が得られる.逆に
		\begin{align}
			a = \emptyset
		\end{align}
		が成り立っていると仮定する.ここで$\chi$を$\mathcal{L}$の任意の対象とすれば,
		相等性の公理より
		\begin{align}
			\chi \in a \rarrow \chi \in \emptyset
		\end{align}
		が成立するので,対偶を取れば
		\begin{align}
			\chi \notin \emptyset \rarrow \chi \notin a
		\end{align}
		が成り立つ.定理\ref{thm:emptyset_has_nothing}より
		\begin{align}
			\chi \notin \emptyset
		\end{align}
		が満たされているので,三段論法より
		\begin{align}
			\chi \notin a
		\end{align}
		が成立し,$\chi$の任意性と推論法則\ref{logicalthm:fundamental_law_of_universal_quantifier}より
		\begin{align}
			\forall x\, (\, x \notin a\, )
		\end{align}
		が成立する.ここに演繹法則を適用して
		\begin{align}
			a = \emptyset \rarrow \forall x\, (\, x \notin a\, )
		\end{align}
		も得られる.
		\QED
	\end{prf}
	
	\begin{screen}
		\begin{thm}[空集合はいかなる類も要素に持たない]
		\label{thm:emptyset_does_not_contain_any_class}
			$a,b$を類とするとき次が成り立つ:
			\begin{align}
				b = \emptyset \rarrow a \notin b.
			\end{align}
		\end{thm}
	\end{screen}
	
	\begin{prf}
		いま$a \in b$が成り立っていると仮定する.このとき要素の公理と三段論法より
		\begin{align}
			\set{a}
		\end{align}
		が成立する.ここで
		\begin{align}
			\tau \defeq \varepsilon x\, (\, a = x\, )
		\end{align}
		とおけば,存在記号に関する規則から
		\begin{align}
			a = \tau
		\end{align}
		が成り立つので,相等性の公理より
		\begin{align}
			\tau \in b
		\end{align}
		が従い,存在記号に関する規則より
		\begin{align}
			\exists x\, (\, x \in b\, )
		\end{align}
		が成り立つ.よって演繹法則から
		\begin{align}
			a \in b \rarrow \exists x\, (\, x \in b\, )
		\end{align}
		が成り立つ.この対偶を取り推論法則\ref{logicalthm:De_Morgan_law_for_quantifiers}を適用すれば
		\begin{align}
			\forall x\, (\, x \notin b\, ) \rarrow a \notin b
		\end{align}
		が得られる.定理\ref{thm:uniqueness_of_emptyset}より
		\begin{align}
			b = \emptyset \rarrow \forall x\, (\, x \notin b\, )
		\end{align}
		も正しいので,含意の推移律から
		\begin{align}
			b = \emptyset \rarrow a \notin b
		\end{align}
		が得られる.
		\QED
	\end{prf}
	
	\begin{screen}
		\begin{dfn}[部分類]
			$a,b$を$\mathcal{L}'$の項とするとき,
			\begin{align}
				a \subset b \overset{\mathrm{def}}{\lrarrow}
				\forall x\ (\ x \in a \rarrow x \in b\ )
			\end{align}
			と定める.式$a \subset b$を``$a$は$b$の{\bf 部分類}\index{ぶぶんるい@部分類}{\bf (subclass)}である''
			と翻訳し,特に$a$が集合である場合は``$a$は$b$の{\bf 部分集合}\index{ぶぶんしゅうごう@部分集合}{\bf (subset)}である''と翻訳する.
			また次の記号も定める:
			\begin{align}
				a \subsetneq b \defarrow a \subset b \wedge a \neq b.
			\end{align}
		\end{dfn}
	\end{screen}
	
	空虚な真の一例として次の結果を得る.
	
	\begin{screen}
		\begin{thm}[空集合は全ての類に含まれる]\label{thm:emptyset_if_a_subset_of_every_class}
			$a$を類とするとき次が成り立つ:
			\begin{align}
				\emptyset \subset a.
			\end{align}
		\end{thm}
	\end{screen}
	
	\begin{prf}
		$a$を類とする.$\tau$を$\mathcal{L}$の任意の対象とすれば
		\begin{align}
			\tau \notin \emptyset
		\end{align}
		が成り立つから,推論規則\ref{logicalaxm:fundamental_rules_of_inference}を適用して
		\begin{align}
			\tau \notin \emptyset \vee \tau \in a
		\end{align}
		が成り立つ.従って
		\begin{align}
			\tau \in \emptyset \rarrow \tau \in a
		\end{align}
		が成り立ち,$\tau$の任意性と推論法則\ref{logicalthm:fundamental_law_of_universal_quantifier}より
		\begin{align}
			\forall x\, (\, x \in \emptyset \rarrow x \in a\, )
		\end{align}
		が成立する.
		\QED
	\end{prf}
	
	$a \subset b$とは$a$に属する全ての``$\mathcal{L}$の対象''は$b$に属するという定義であったが,
	要素となりうる類は集合であるという公理から,$a$に属する全ての``類''もまた$b$に属する.
	
	\begin{screen}
		\begin{thm}[類はその部分類に属する全ての類を要素に持つ]\label{thm:subclass_contains_all_elements}
			$a,b,c$を類とすれば次が成り立つ:
			\begin{align}
				a \subset b \rarrow (\, c \in a \rarrow c \in b\, ).
			\end{align}
		\end{thm}
	\end{screen}
	
	\begin{prf}	
		いま$a \subset b$が成り立っているとする.このとき
		\begin{align}
			c \in a
		\end{align}
		が成り立っていると仮定すれば,要素の公理より
		\begin{align}
			\set{c}
		\end{align}
		が成り立つ.ここで
		\begin{align}
			\tau \defeq \varepsilon x\, (\, c=x\, )
		\end{align}
		とおくと
		\begin{align}
			c = \tau
		\end{align}
		が成り立つので,相等性の公理より
		\begin{align}
			\tau \in a
		\end{align}
		が成り立ち,$a \subset b$と推論法則\ref{logicalthm:fundamental_law_of_universal_quantifier}から
		\begin{align}
			\tau \in b
		\end{align}
		が従う.再び相等性の公理を適用すれば
		\begin{align}
			c \in b
		\end{align}
		が成り立つので,演繹法則より,$a \subset b$が成り立っている下で
		\begin{align}
			c \in a \rarrow c \in b
		\end{align}
		が成立する.再び演繹法則を適用すれば定理の主張が得られる.
		\QED
	\end{prf}
	
	宇宙$\Univ$は類の一つであった.当然のようであるが,それは最大の類である.
	\begin{screen}
		\begin{thm}[$\Univ$は最大の類である]
			$a$を類とするとき次が成り立つ:
			\begin{align}
				a \subset \Univ.
			\end{align}
		\end{thm}
	\end{screen}
	
	\begin{prf}
		$\tau$を$\mathcal{L}$の任意の対象とすれば,定理\ref{thm:any_class_equals_to_itself}と類の公理より
		\begin{align}
			\tau \in \Univ
		\end{align}
		が成立するので,推論規則\ref{logicalaxm:fundamental_rules_of_inference}より
		\begin{align}
			\tau \notin a \vee \tau \in \Univ
		\end{align}
		が成立する.このとき推論法則\ref{logicalthm:rule_of_inference_3}より
		\begin{align}
			\tau \in a \rarrow \tau \in \Univ
		\end{align}
		が成立し,$\tau$の任意性と推論法則\ref{logicalthm:fundamental_law_of_universal_quantifier}から
		\begin{align}
			\forall x\, (\, x \in a \rarrow x \in \Univ\, )
		\end{align}
		が従う.
		\QED
	\end{prf}
	
	\begin{screen}
		\begin{thm}[互いに互いの部分類となる類同士は等しい]\label{thm:mutually_sub_classes_are_equivalent}
			$a,b$を類とするとき次が成り立つ:
			\begin{align}
				a \subset b \wedge b \subset a \lrarrow a = b.
			\end{align}
		\end{thm}
	\end{screen}
	
	\begin{sketch}
		$a \subset b \wedge b \subset a$が成り立っていると仮定する.
		このとき$\tau$を$\mathcal{L}$の任意の対象とすれば,
		$a \subset b$と推論法則\ref{logicalthm:fundamental_law_of_universal_quantifier}より
		\begin{align}
			\tau \in a \rarrow \tau \in b
		\end{align}
		が成立し,$b \subset a$と推論法則\ref{logicalthm:fundamental_law_of_universal_quantifier}より
		\begin{align}
			\tau \in b \rarrow \tau \in a
		\end{align}
		が成立するので,
		\begin{align}
			\tau \in a \lrarrow \tau \in b
		\end{align}
		が成り立つ.$\tau$の任意性と推論法則\ref{logicalthm:fundamental_law_of_universal_quantifier}および
		外延性の公理より
		\begin{align}
			a = b
		\end{align}
		が出るので,演繹法則より
		\begin{align}
			a \subset b \wedge b \subset a \rarrow a = b
		\end{align}
		が得られる.逆に$a = b$が満たされていると仮定するとき,$\tau$を$\mathcal{L}$の任意の対象とすれば
		\begin{align}
			\tau \in a \rarrow \tau \in b
		\end{align}
		と
		\begin{align}
			\tau \in b \rarrow \tau \in a
		\end{align}
		が共に成り立つ. よって推論法則\ref{logicalthm:fundamental_law_of_universal_quantifier}より
		\begin{align}
			a \subset b
		\end{align}
		と
		\begin{align}
			b \subset a
		\end{align}
		が共に従う.よって演繹法則より
		\begin{align}
			a = b \rarrow a \subset b \wedge b \subset a
		\end{align}
		も得られる.
		\QED
	\end{sketch}
	
	\monologue{
		定理\ref{thm:subclass_contains_all_elements}と定理\ref{thm:mutually_sub_classes_are_equivalent}より,
			類$a,b$が$a = b$を満たすならば,$a$と$b$は要素に持つ$\mathcal{L}$の対象のみならず,
			要素に持つ類までも一致するのですね.
	}
	
\section{順序型について}
	$(A,R)$を整列集合とするとき,
	\begin{align}
		x \longmapsto 
		\begin{cases}
			\min{A \backslash \ran{x}} & \mbox{if } \ran{x} \subsetneq A \\
			A & \mbox{o.w.} \\
		\end{cases}
	\end{align}
	なる写像$G$に対して
	\begin{align}
		\forall \alpha\, F(\alpha) = G(\rest{F}{\alpha})
	\end{align}
	なる写像$F$を取り
	\begin{align}
		\alpha \defeq \min{\Set{\alpha \in \ON}{F(\alpha) = A}}
	\end{align}
	とおけば,$\alpha$は$(A,R)$の順序型.
	
\section{超限再帰について}
	$\Univ$上の写像$G$が与えられたら,
	\begin{align}
		F \defeq \Set{(\alpha,x)}{\ord{\alpha} \wedge
		\exists f\, \left(\, f \fon \alpha \wedge
		\forall \beta \in \alpha\, \left(\, f(\beta) = G(\rest{f}{\beta})\, \right)
		\wedge x = G(f)\, \right)}
	\end{align}
	により$F$を定めれば
	\begin{align}
		\forall \alpha\, F(\alpha) = G(\rest{F}{\alpha})
	\end{align}
	が成立する.
	
	\begin{screen}
		任意の順序数$\alpha$および$\alpha$上の写像$f$と$g$に対して,
		\begin{align}
			\forall \beta \in \alpha\,
			\left(\, f(\beta) = G(\rest{f}{\beta})\, \right)
		\end{align}
		かつ
		\begin{align}
			\forall \beta \in \alpha\,
			\left(\, g(\beta) = G(\rest{g}{\beta})\, \right)
		\end{align}
		ならば$f = g$である.
	\end{screen}
	
	まず
	\begin{align}
		f(0) = G(\rest{f}{0}) = G(0) = G(\rest{g}{0}) = g(0)
	\end{align}
	が成り立つ.また
	\begin{align}
		\forall \delta \in \beta\, \left(\, 
		\delta \in \alpha \rarrow f(\delta) = g(\delta)\, \right)
	\end{align}
	ならば,$\beta \in \alpha$であるとき
	\begin{align}
		\rest{f}{\beta} = \rest{g}{\beta}
	\end{align}
	となるので
	\begin{align}
		\beta \in \alpha \rarrow f(\beta) = g(\beta)
	\end{align}
	が成り立つ.ゆえに
	\begin{align}
		f = g
	\end{align}
	が得られる.
	
	\begin{screen}
		任意の順序数$\alpha$に対して,$\alpha$上の写像$f$で
		\begin{align}
			\forall \beta \in \alpha\, \left(\, 
			f(\beta) = G(\rest{f}{\beta})\, \right)
		\end{align}
		を満たすものが取れる.
	\end{screen}
	
	$\alpha = 0$のとき$f \defeq 0$とすればよい.$\alpha$の任意の要素$\beta$に対して
	\begin{align}
		g \fon \beta \wedge \forall \gamma\in \beta\, \left(\, 
		g(\gamma) = G(\rest{g}{\gamma})\, \right)
	\end{align}
	なる$g$が存在するとき,
	\begin{align}
		f \defeq \Set{(\beta,x)}{\beta \in \alpha \wedge
		\exists g\, \left(\, g \fon \beta \wedge
		\forall \gamma \in \beta\, \left(\, g(\gamma) = G(\rest{g}{\gamma})\, \right)
		\wedge x = G(g)\, \right)}
	\end{align}
	と定めれば,$f$は$\alpha$上の写像であって
	\begin{align}
		\forall \beta \in \alpha\, \left(\, 
		f(\beta) = G(\rest{f}{\beta})\, \right)
	\end{align}
	を満たす.
	
	\begin{screen}
		任意の順序数$\alpha$に対して$F(\alpha) = G(\rest{F}{\alpha})$が成り立つ.
	\end{screen}
	
	$\alpha = 0$ならば,$0$上の写像は$0$のみなので
	\begin{align}
		F(0) = G(0) = G(\rest{F}{0})
	\end{align}
	である.
	\begin{align}
		\forall \beta \in \alpha\, F(\beta) = G(\rest{F}{\beta})
	\end{align}
	が成り立っているとき,
	\begin{align}
		\forall \beta \in \alpha\, f(\beta) = G(\rest{f}{\beta})
	\end{align}
	を満たす$\alpha$上の写像$f$を取れば,前の一意性より
	\begin{align}
		f = \rest{F}{\alpha}
	\end{align}
	が成立する.よって
	\begin{align}
		F(\alpha) = G(f) = G(\rest{F}{\alpha})
	\end{align}
	となる.
	\QED
	
\section{自然数の全体について}
	$\Natural$を
	\begin{align}
		\Natural \defeq \Set{\beta}{\mbox{$\alpha \leq \beta$である$\alpha$は
		$0$であるか後続型順序数}}
	\end{align}
	によって定めれば,無限公理より
	\begin{align}
		\set{\Natural}
	\end{align}
	である.また$\ord{\Natural}$と$\limo{\Natural}$も証明できるはず.
	$\Natural$が最小の極限数であることは$\Natural$を定義した論理式より従う.
	\section{対}
	$a$と$b$を類とするとき,$a$か$b$の少なくとも一方に等しい集合の全体,つまり
	\begin{align}
		a = x \vee b = x
	\end{align}
	を満たす全ての集合$x$を集めたものを$a$と$b$の対と呼び
	\begin{align}
		\{a,b\}
	\end{align}
	と書く.解釈としては``$a$と$b$のみを要素とする類''のことであり,当然$a$が集合であるならば
	\begin{align}
		a \in \{a,b\}
	\end{align}
	が成立する.しかし$a$と$b$が共に真類であるときは,いかなる集合も$a$にも$b$にも等しくないため
	\begin{align}
		\{a,b\} = \emptyset
	\end{align}
	となる.以上が大雑把な対の説明である.
	
	\begin{screen}
		\begin{dfn}[対]
			$x,y$を$\mathcal{L}$の項とし,$z$を$x$にも$y$にも自由に現れない変項とするとき,
			\begin{align}
				\{x,y\} \defeq \Set{z}{x = z \vee y = z}
			\end{align}
			で$\{x,y\}$を定義し,これを$x$と$y$の{\bf 対}\index{つい@対}{\bf (pair)}と呼ぶ.
			特に$\{x,x\}$を$\{x\}$と書く.
		\end{dfn}
	\end{screen}
	
	上の定義では省略したが,$x$や$y$が内包項である場合は$z = x \vee z = y$を
	$\lang{\varepsilon}$の式に書き換えてから$\{x,y\}$を定めるのである.つまり
	\begin{align}
		\varphi \defarrow x = z \vee y = z
	\end{align}
	とおけば,$\varphi$を$\lang{\varepsilon}$の式に書き換えた式$\hat{\varphi}$によって
	\begin{align}
		\{x,y\} \defeq \Set{z}{\hat{\varphi}(z)}
	\end{align}
	と定めるのである.たとえば$a$や$b$を類として
	\begin{align}
		\varphi \defarrow a = z \vee b = z
	\end{align}
	とおけば,
	\begin{align}
		\COMAX \vdash \forall z\, (\, z \in \{a,b\} \lrarrow \hat{\varphi}(z)\, )
	\end{align}
	が成立するし,同時に定理\ref{thm:equivalent_formula_rewriting_1}と
	定理\ref{thm:equivalent_formula_rewriting_2}より
	\begin{align}
		\EXTAX,\EQAX,\COMAX \vdash 
		\forall z\, (\, \hat{\varphi}(z) \lrarrow \varphi(z)\, )
	\end{align}
	も成り立つので
	\begin{align}
		\EXTAX,\EQAX,\COMAX \vdash 
		\forall z\, (\, z \in \{a,b\} \lrarrow a = z \vee b = z\, )
	\end{align}
	が得られる.
	
	\begin{screen}
		\begin{thm}[対は表示されている要素しか持たない]
		\label{thm:pair_members_are_exactly_the_given_two}
			$a$と$b$を類とするとき次が成立する:
			\begin{align}
				\EXTAX,\EQAX,\COMAX \vdash 
				\forall x\, (\, x \in \{a,b\} \lrarrow a = x \vee b = x\, ).
			\end{align}
		\end{thm}
	\end{screen}
	
	$\ELEAX$を加えれば次が得られる.
	
	\begin{screen}
		\begin{thm}[対の要素は表示されている要素の一方には等しい]
		\label{cor:pair_members_are_exactly_the_given_two}
			$a,b,c$を類とするとき次が成立する:
			\begin{align}
				\EXTAX,\EQAX,\COMAX,\ELEAX \vdash 
				c \in \{a,b\} \rarrow a = c \vee b = c.
			\end{align}
		\end{thm}
	\end{screen}
	
	\begin{sketch}
		要素の公理より
		\begin{align}
			c \in \{a,b\},\ \ELEAX \vdash \set{c}
		\end{align}
		が成り立つので
		\begin{align}
			\tau \defeq \varepsilon s\, (\, c = s\, )
		\end{align}
		とおけば
		\begin{align}
			c \in \{a,b\},\ \ELEAX \vdash c = \tau
			\label{fom:pair_members_are_exactly_the_given_two_2}
		\end{align}
		となる.$\tau$に対しては定理\ref{thm:pair_members_are_exactly_the_given_two}より 
		\begin{align}
			\EXTAX,\EQAX,\COMAX \vdash 
			\tau \in \{a,b\} \rarrow a = \tau \vee b = \tau
		\end{align}
		が成り立つが,ここで(\refeq{fom:pair_members_are_exactly_the_given_two_2})より
		\begin{align}
			c \in \{a,b\},\ \EQAX,\ELEAX \vdash \tau \in \{a,b\}
		\end{align}
		となるので
		\begin{align}
			c \in \{a,b\},\ \EXTAX,\EQAX,\COMAX,\ELEAX \vdash a = \tau \vee b = \tau
		\end{align}
		が従い,代入原理(定理\ref{thm:the_principle_of_substitution})と
		(\refeq{fom:pair_members_are_exactly_the_given_two_2})より
		\begin{align}
			c \in \{a,b\},\ \EXTAX,\EQAX,\COMAX,\ELEAX \vdash a = c \vee b = c
		\end{align}
		が得られる.
		\QED
	\end{sketch}
	
	この逆,つまり
	\begin{align}
		a = c \vee b = c \rarrow c \in \{a,b\}
	\end{align}
	は一般には成立しない.実際$a,b$が共に真類であるときは
	\begin{align}
		\{a,b\} = \emptyset
	\end{align}
	となるためである(定理\ref{thm:pair_of_proper_classes_is_emptyset}).
	
	\begin{screen}
		\begin{thm}[表示の順番を入れ替えても対は等しい]
		\label{thm:commutative_law_of_pairs}
			$a$と$b$を類とするとき
			\begin{align}
				\EXTAX,\EQAX,\COMAX \vdash \{a,b\} = \{b,a\}.
			\end{align}
		\end{thm}
	\end{screen}
	
	\begin{sketch}
		いま
		\begin{align}
			\tau \defeq \varepsilon x \negation (\, x \in \{a,b\} \lrarrow x \in \{b,a\}\, )
		\end{align}
		とおく(必要に応じて$x \in \{a,b\} \lrarrow x \in \{b,a\}$は$\lang{\varepsilon}$の
		式に書き換える).
		定理\ref{thm:pair_members_are_exactly_the_given_two}より
		\begin{align}
			\EXTAX,\EQAX,\COMAX \vdash
			\tau \in \{a,b\} \rarrow a = \tau \vee b = \tau
		\end{align}
		が成り立つので,演繹定理の逆より
		\begin{align}
			\tau \in \{a,b\},\ \EXTAX,\EQAX,\COMAX \vdash a = \tau \vee b = \tau
		\end{align}
		となる.また論理和の可換律
		(推論法則\ref{logicalthm:commutative_law_of_disjunction})より
		\begin{align}
			\tau \in \{a,b\},\ \EXTAX,\EQAX,\COMAX \vdash b = \tau \vee a = \tau
		\end{align}
		が成り立ち,定理\ref{thm:pair_members_are_exactly_the_given_two}より
		\begin{align}
			\tau \in \{a,b\},\ \EXTAX,\EQAX,\COMAX \vdash \tau \in \{b,a\}
		\end{align}
		が従う.そして演繹定理より
		\begin{align}
			\EXTAX,\EQAX,\COMAX \vdash \tau \in \{a,b\} \rarrow \tau \in \{b,a\}
		\end{align}
		が得られる.$a$と$b$を入れ替えれば
		\begin{align}
			\EXTAX,\EQAX,\COMAX \vdash \tau \in \{b,a\} \rarrow \tau \in \{a,b\}
		\end{align}
		が得られるので,論理積の導入より
		\begin{align}
			\EXTAX,\EQAX,\COMAX \vdash \tau \in \{a,b\} \lrarrow \tau \in \{b,a\}
		\end{align}
		が成り立ち,全称の導出(推論法則\ref{logicalthm:derivation_of_universal_by_epsilon})より
		\begin{align}
			\EXTAX,\EQAX,\COMAX \vdash \forall x\, (\, x \in \{a,b\} \lrarrow x \in \{b,a\}\, )
		\end{align}
		となり,外延性公理より
		\begin{align}
			\EXTAX,\EQAX,\COMAX \vdash \{a,b\} = \{b,a\}
		\end{align}
		が従う.
		\QED
	\end{sketch}
		
	\begin{screen}
		\begin{axm}[対の公理] 次の式を$\PAIAX$により参照する:
			\begin{align}
				\forall x\, \forall y\, \exists p\, \forall z\, 
				(\, x = z \vee y = z \lrarrow z \in p\, ).
			\end{align}
		\end{axm}
	\end{screen}
	
	\begin{screen}
		\begin{thm}[集合の対は集合である]
		\label{thm:pair_of_sets_is_a_set}
			$a$と$b$を類とするとき
			\begin{align}
				\EXTAX,\EQAX,\COMAX,\PAIAX \vdash 
				\set{a} \wedge \set{b} \rarrow \set{\{a,b\}}.
			\end{align}
		\end{thm}
	\end{screen}
	
	\begin{sketch}\mbox{}
		\begin{description}
			\item[step1]
				論理積の除去より
				\begin{align}
					\set{a} \wedge \set{b} &\vdash \exists x\, (\, a = x\, ), \\
					\set{a} \wedge \set{b} &\vdash \exists x\, (\, b = x\, )
				\end{align}
				が成り立つので,
				\begin{align}
					\tau &\defeq \varepsilon x\, (\, a = x\, ), \\
					\sigma &\defeq \varepsilon x\, (\, b = x\, )
				\end{align}
				とおけば
				\begin{align}
					\set{a} \wedge \set{b} &\vdash a = \tau, 
					\label{fom:pair_of_sets_is_a_set_1} \\
					\set{a} \wedge \set{b} &\vdash b = \sigma
				\end{align}
				が成り立つ.対の公理より$\tau$と$\sigma$に対しては
				\begin{align}
					\PAIAX \vdash \exists p\, \forall z\, 
						(\, \tau = z \vee \sigma = z \lrarrow z \in p\, )
				\end{align}
				が成り立つので,
				\begin{align}
					\rho \defeq \varepsilon p\, \forall z\, 
						(\, \tau = z \vee \sigma = z \lrarrow z \in p\, )
				\end{align}
				とおけば
				\begin{align}
					\PAIAX \vdash \forall z\, (\, \tau = z \vee \sigma = z \lrarrow z \in \rho\, )
					\label{fom:pair_of_sets_is_a_set_2}
				\end{align}
				となる.
				
			\item[step2]
				次に
				\begin{align}
					\forall z\, (\, z \in \{a,b\} \lrarrow z \in \rho\, )
				\end{align}
				を示すために
				\begin{align}
					\zeta \defeq \varepsilon z \negation (\, z \in \{a,b\} \lrarrow z \in \rho\, )
				\end{align}
				とおく(当然$\lang{\varepsilon}$の式に書き換える).
				等号の推移律(定理\ref{thm:transitive_law_of_equality})より
				\begin{align}
					\EXTAX,\EQAX \vdash a = \tau \rarrow (\, a = \zeta \rarrow \tau = \zeta\, )
				\end{align}
				が成り立つので,(\refeq{fom:pair_of_sets_is_a_set_1})との三段論法より
				\begin{align}
					\set{a} \wedge \set{b},\ \EXTAX,\EQAX \vdash 
					a = \zeta \rarrow \tau = \zeta
				\end{align}
				が成り立ち,論理和の導入より
				\begin{align}
					\set{a} \wedge \set{b},\ \EXTAX,\EQAX \vdash 
					a = \zeta \rarrow \tau = \zeta \vee \sigma = \zeta
				\end{align}
				が従う.同様にして
				\begin{align}
					\set{a} \wedge \set{b},\ \EXTAX,\EQAX \vdash 
					b = \zeta \rarrow \tau = \zeta \vee \sigma = \zeta
				\end{align}
				も成り立つので,論理和の除去より
				\begin{align}
					\set{a} \wedge \set{b},\ \EXTAX,\EQAX \vdash 
					a = \zeta \vee b = \zeta \rarrow \tau = \zeta \vee \sigma = \zeta
				\end{align}
				が得られる.同様に
				\begin{align}
					\set{a} \wedge \set{b},\ \EXTAX,\EQAX \vdash 
					\tau = \zeta \vee \sigma = \zeta \rarrow a = \zeta \vee b = \zeta
				\end{align}
				も得られ,論理積の導入より
				\begin{align}
					\set{a} \wedge \set{b},\ \EXTAX,\EQAX \vdash 
					a = \zeta \vee b = \zeta \lrarrow \tau = \zeta \vee \sigma = \zeta
				\end{align}
				が従う.他方で定理\ref{thm:pair_members_are_exactly_the_given_two}より
				\begin{align}
					\EXTAX,\EQAX,\COMAX \vdash 
					\zeta \in \{a,b\} \lrarrow a = \zeta \vee b = \zeta
				\end{align}
				が成り立ち,また(\refeq{fom:pair_of_sets_is_a_set_2})より
				\begin{align}
					\PAIAX \vdash \tau = \zeta \vee \sigma = \zeta \lrarrow \zeta \in \rho
				\end{align}
				も成り立つので,同値記号の推移律
				(推論法則\ref{logicalthm:transitive_law_of_equivalence_symbol})より
				\begin{align}
					\set{a} \wedge \set{b},\ \EXTAX,\EQAX,\COMAX,\PAIAX \vdash 
					\zeta \in \{a,b\} \lrarrow \zeta \in \rho
				\end{align}
				が従う.そして全称の導出(推論法則\ref{logicalthm:derivation_of_universal_by_epsilon})より
				\begin{align}
					\set{a} \wedge \set{b},\ \EXTAX,\EQAX,\COMAX,\PAIAX \vdash 
					\forall z\, (\, z \in \{a,b\} \lrarrow z \in \rho\, )
				\end{align}
				が成り立ち,外延性公理より
				\begin{align}
					\set{a} \wedge \set{b},\ \EXTAX,\EQAX,\COMAX,\PAIAX \vdash 
					\{a,b\} = \rho
				\end{align}
				が従い,存在記号の推論公理より
				\begin{align}
					\set{a} \wedge \set{b},\ \EXTAX,\EQAX,\COMAX,\PAIAX \vdash 
					\exists p\, (\, \{a,b\} = p\, )
				\end{align}
				が成り立つ.
				\QED
		\end{description}
	\end{sketch}
	
	\begin{screen}
		\begin{thm}[集合は自分自身の対の要素である]
		\label{thm:set_is_an_element_of_its_pair}
			$a$と$b$を類とするとき
			\begin{align}
				\EXTAX,\EQAX,\COMAX &\vdash \set{a} \rarrow a \in \{a,b\}, \\
				\EXTAX,\EQAX,\COMAX &\vdash \set{b} \rarrow b \in \{a,b\}.
			\end{align}
		\end{thm}
	\end{screen}
	
	\begin{sketch}\mbox{}
		\begin{description}
			\item[step1]
				いま
				\begin{align}
					\tau \defeq \varepsilon x\, (\, a = x\, )
				\end{align}
				とおくと
				\begin{align}
					\set{a} \vdash a = \tau
					\label{fom:set_is_an_element_of_its_pair_1}
				\end{align}
				が成り立ち,論理和の導入より
				\begin{align}
					\set{a} \vdash a = \tau \vee b = \tau
				\end{align}
				も成り立つ.定理\ref{thm:pair_members_are_exactly_the_given_two}より
				\begin{align}
					\EXTAX,\EQAX,\COMAX \vdash 
					a = \tau \vee b = \tau \rarrow \tau \in \{a,b\}
				\end{align}
				が成り立つので三段論法より
				\begin{align}
					\set{a},\ \EXTAX,\EQAX,\COMAX \vdash \tau \in \{a,b\}
					\label{fom:set_is_an_element_of_its_pair_2}
				\end{align}
				が従う.また(\refeq{fom:set_is_an_element_of_its_pair_1})と相等性公理より
				\begin{align}
					\set{a},\ \EQAX \vdash \tau = a
				\end{align}
				となり
				\begin{align}
					\set{a},\ \EQAX \vdash \tau \in \{a,b\} \rarrow a \in \{a,b\}
				\end{align}
				となるので,(\refeq{fom:set_is_an_element_of_its_pair_2})と三段論法より
				\begin{align}
					\set{a},\ \EXTAX,\EQAX,\COMAX \vdash a \in \{a,b\}
				\end{align}
				が成立する.
			
			\item[step2]
				前段で$a$と$b$を入れ替えれば
				\begin{align}
					\set{b},\ \EXTAX,\EQAX,\COMAX \vdash b \in \{b,a\}
					\label{fom:set_is_an_element_of_its_pair_3}
				\end{align}
				が成立する.ところで対の対称性(定理\ref{thm:commutative_law_of_pairs})より
				\begin{align}
					\EXTAX,\EQAX,\COMAX \vdash \{b,a\} = \{a,b\}
				\end{align}
				が成立し,また相等性公理より
				\begin{align}
					\EQAX \vdash \{b,a\} = \{a,b\}
					\rarrow (\, b \in \{b,a\} \rarrow b \in \{a,b\}\, )
				\end{align}
				も成り立つので,三段論法より
				\begin{align}
					\EXTAX,\EQAX,\COMAX \vdash b \in \{b,a\} \rarrow b \in \{a,b\}
					\label{fom:set_is_an_element_of_its_pair_4}
				\end{align}
				が従う.(\refeq{fom:set_is_an_element_of_its_pair_3})と
				(\refeq{fom:set_is_an_element_of_its_pair_4})と三段論法より
				\begin{align}
					\set{b},\ \EXTAX,\EQAX,\COMAX \vdash b \in \{a,b\}
				\end{align}
				が得られる.
				\QED
		\end{description}
	\end{sketch}
	
	$a$を集合とすれば対の公理より$\{a\}$も集合となるので,
	定理\ref{thm:set_is_an_element_of_its_pair}より
	\begin{align}
		\EXTAX,\EQAX,\COMAX \vdash \set{a} \rarrow a \in \{a\}
	\end{align}
	が成立する.一方で$a$も$b$も真類であると$\{a,b\}$は空になる.
	
	\begin{screen}
		\begin{thm}[真類同士の対は空]\label{thm:pair_of_proper_classes_is_emptyset}
			$a$と$b$を類とするとき,
			\begin{align}
				\EXTAX,\EQAX,\COMAX \vdash\ 
				\negation \set{a} \wedge \negation \set{b} \rarrow \{a,b\} = \emptyset.
			\end{align}
		\end{thm}
	\end{screen}
	
	\begin{sketch}
		いま
		\begin{align}
			\tau \defeq \varepsilon x \negation (\, x \notin \{a,b\}\, )
		\end{align}
		とおく($x \notin \{a,b\}$は$\lang{\varepsilon}$の式に書き換える).
		\begin{align}
			\negation \set{a} \wedge\ \negation \set{b}
			\vdash\ \negation \exists x\, (\, a = x\, )
		\end{align}
		が成り立ち,De Morgan の法則
		(推論法則\ref{logicalthm:strong_De_Morgan_law_for_quantifiers_2})より
		\begin{align}
			\negation \set{a} \wedge \negation \set{b}
			\vdash \forall x\, (\, a \neq x\, )
		\end{align}
		が従い,全称記号の推論公理より
		\begin{align}
			\negation \set{a} \wedge \negation \set{b} \vdash a \neq \tau
		\end{align}
		となる.同様にして
		\begin{align}
			\negation \set{a} \wedge \negation \set{b} \vdash b \neq \tau
		\end{align}
		も成り立つので,論理積の導入より
		\begin{align}
			\negation \set{a} \wedge \negation \set{b} \vdash
			a \neq \tau \wedge b \neq \tau
		\end{align}
		が成立し,De Morgan の法則(推論法則\ref{logicalthm:weak_De_Morgan_law_1})より
		\begin{align}
			\negation \set{a} \wedge \negation \set{b} \vdash\ 
			\negation (\, a = \tau \vee b = \tau\, )
			\label{fom:pair_of_proper_classes_is_emptyset_1}
		\end{align}
		が従う.ところで定理\ref{thm:pair_members_are_exactly_the_given_two}より
		\begin{align}
			\EXTAX,\EQAX,\COMAX \vdash \tau \in \{a,b\} \rarrow a = \tau \vee b = \tau
		\end{align}
		が成り立つので,対偶を取って
		\begin{align}
			\EXTAX,\EQAX,\COMAX \vdash\ 
			\negation (\, a = \vee b = \tau\, ) \rarrow \tau \notin \{a,b\}
		\end{align}
		が成り立つ(推論法則\ref{logicalthm:introduction_of_contraposition}).
		そして(\refeq{fom:pair_of_proper_classes_is_emptyset_1})との三段論法より
		\begin{align}
			\negation \set{a} \wedge \negation \set{b},\ \EXTAX,\EQAX,\COMAX \vdash
			\tau \notin \{a,b\}
		\end{align}
		が従い,全称の導出(推論法則\ref{logicalthm:derivation_of_universal_by_epsilon})より
		\begin{align}
			\negation \set{a} \wedge \negation \set{b},\ \EXTAX,\EQAX,\COMAX \vdash
			\forall x\, (\, x \notin \{a,b\}\, )
		\end{align}
		が従う.要素を持たない類は空集合である(定理\ref{thm:uniqueness_of_emptyset})ので
		\begin{align}
			\negation \set{a} \wedge \negation \set{b},\ \EXTAX,\EQAX,\COMAX \vdash
			\{a,b\} = \emptyset
		\end{align}
		が得られる.
		\QED
	\end{sketch}
	
	上の定理とは逆に$\{a,b\}$が空ならば$a$も$b$も真類である.
	
	\begin{screen}
		\begin{thm}[空な対に表示されている類は集合ではない]
		\label{thm:classes_displayed_in_empty_pair_are_not_sets}
			$a$と$b$を類とするとき,
			\begin{align}
				\EXTAX,\EQAX,\COMAX \vdash \{a,b\} = \emptyset \rarrow\ \negation \set{a} \wedge \negation \set{b}.
			\end{align}
		\end{thm}
	\end{screen}
	
	\begin{sketch}
		いま
		\begin{align}
			\tau \defeq \varepsilon x\, (\, a = x\, )
		\end{align}
		とおけば
		\begin{align}
			\set{a} \vdash a = \tau
		\end{align}
		が成立し,また定理\ref{thm:set_is_an_element_of_its_pair}より
		\begin{align}
			\set{a},\ \EXTAX,\EQAX,\COMAX \vdash a \in \{a,b\}
		\end{align}
		が成り立つので相等性公理より
		\begin{align}
			\set{a},\ \EXTAX,\EQAX,\COMAX \vdash \tau \in \{a,b\}
		\end{align}
		が従い,存在記号の推論公理より
		\begin{align}
			\set{a},\ \EXTAX,\EQAX,\COMAX \vdash \exists x\, (\, x \in \{a,b\}\, )
		\end{align}
		が成り立つ.演繹定理より
		\begin{align}
			\EXTAX,\EQAX,\COMAX \vdash \set{a} \rarrow \exists x\, (\, x \in \{a,b\}\, )
		\end{align}
		となり,対偶を取れば
		\begin{align}
			\EXTAX,\EQAX,\COMAX \vdash\ \negation \exists x\, (\, x \in \{a,b\}\, )
			\rarrow\ \negation \set{a}
			\label{fom:classes_displayed_in_empty_pair_are_not_sets_1}
		\end{align}
		が得られる(推論法則\ref{logicalthm:introduction_of_contraposition}).
		他方で空の類は要素を持たない(定理\ref{thm:uniqueness_of_emptyset})ので
		\begin{align}
			\EXTAX,\EQAX,\COMAX \vdash \{a,b\} = \emptyset \rarrow \forall x\, (\, x \notin \{a,b\}\, )
			\label{fom:classes_displayed_in_empty_pair_are_not_sets_2}
		\end{align}
		が成り立ち,また De Morgan の法則
		(推論法則\ref{logicalthm:strong_De_Morgan_law_for_quantifiers_1})より
		\begin{align}
			\vdash \forall x\, (\, x \notin \{a,b\}\, ) \rarrow\ \negation \exists x\, (\, x \in \{a,b\}\, )
			\label{fom:classes_displayed_in_empty_pair_are_not_sets_3}
		\end{align}
		も成り立つので,(\refeq{fom:classes_displayed_in_empty_pair_are_not_sets_2})
		(\refeq{fom:classes_displayed_in_empty_pair_are_not_sets_3})
		(\refeq{fom:classes_displayed_in_empty_pair_are_not_sets_1})を併せて
		\begin{align}
			\EXTAX,\EQAX,\COMAX \vdash \{a,b\} = \emptyset \rarrow\ \negation \set{a}
		\end{align}
		が従う.同様にして
		\begin{align}
			\EXTAX,\EQAX,\COMAX \vdash \{a,b\} = \emptyset \rarrow\ \negation \set{b}
		\end{align}
		も成り立ち,論理積の導入より
		\begin{align}
			\EXTAX,\EQAX,\COMAX \vdash \{a,b\} = \emptyset \rarrow\ \negation \set{a} \wedge \negation \set{b}
		\end{align}
		が得られる.
		\QED
	\end{sketch}
	\section{合併}
	\begin{screen}
		\begin{dfn}[合併]
			$a$を類とするとき,$a$の{\bf 合併}\index{がっぺい@合併}{\bf (union)}を
			\begin{align}
				\bigcup a \defeq \Set{x}{\exists t \in a\, (\, x \in t\, )}
				\label{eq:definition_of_union_1}
			\end{align}
			で定める.
		\end{dfn}
	\end{screen}
	
	\begin{screen}
		\begin{axm}[合併の公理]
			集合の合併は集合である.つまり,$a$を類とするとき次が成り立つ:
			\begin{align}
				\set{a} \Longrightarrow \set{\bigcup a}.
			\end{align}
		\end{axm}
	\end{screen}
	
	\begin{screen}
		\begin{thm}[空集合の合併は空]\label{thm:the_union_of_the_emptyset_is_empty}
			次が成立する:
			\begin{align}
				\bigcup \emptyset = \emptyset.
			\end{align}
		\end{thm}
	\end{screen}
	
	\begin{prf}
		$\chi$と$\tau$を$\mathcal{L}$の任意の対象とすれば,定理\ref{thm:emptyset_has_nothing}より
		\begin{align}
			\chi \notin \emptyset
		\end{align}
		が成り立つので
		\begin{align}
			\chi \notin \emptyset \vee \tau \notin \chi
		\end{align}
		が成立し,$\chi$の任意性と推論法則\ref{logicalthm:fundamental_law_of_universal_quantifier}より
		\begin{align}
			\forall x\, (\, x \notin \emptyset \vee \tau \notin x\, )
		\end{align}
		が成り立つ.ここで推論法則\ref{logicalthm:properties_of_quantifiers}より
		\begin{align}
			\forall x\, (\, x \notin \emptyset \vee \tau \notin x\, )
			&\Longleftrightarrow \forall x\, \rightharpoondown (\, x \in \emptyset \wedge \tau \in x\, ) \\
			&\Longleftrightarrow\, \rightharpoondown \exists x\, (\, x \in \emptyset \wedge \tau \in x\, )
		\end{align}
		が成立するので,三段論法より
		\begin{align}
			\rightharpoondown \exists x\, (\, x \in \emptyset \wedge \tau \in x\, )
		\end{align}
		が従う.他方で合併の定義から
		\begin{align}
			\rightharpoondown \exists x\, (\, x \in \emptyset \wedge \tau \in x\, )
			\Longleftrightarrow \tau \notin \bigcup \emptyset
		\end{align}
		が満たされているので,再び三段論法より
		\begin{align}
			\tau \notin \bigcup \emptyset
		\end{align}
		が従う.$\tau$の任意性と推論法則\ref{logicalthm:fundamental_law_of_universal_quantifier}より
		\begin{align}
			\forall t\, (\, t \notin \bigcup \emptyset\, )
		\end{align}
		が成立し,定理\ref{thm:uniqueness_of_emptyset}より
		\begin{align}
			\bigcup \emptyset = \emptyset
		\end{align}
		が従う.
		\QED
	\end{prf}
	
	\begin{screen}
		\begin{thm}[合併は任意の要素より大きい]\label{thm:union_is_bigger_than_any_member}
			$a$を類とするとき次が成立する:
			\begin{align}
				\forall x\, (\, x \in a \Longrightarrow x \subset \bigcup a\, ).
			\end{align}
		\end{thm}
	\end{screen}
	
	\begin{sketch}
		$\chi$を$\mathcal{L}$の任意の対象として
		\begin{align}
			\chi \in a
			\label{fom:thm_union_is_bigger_than_any_member_1}
		\end{align}
		であるとする.また$\tau$も$\mathcal{L}$の任意の対象として
		\begin{align}
			\tau \in \chi
		\end{align}
		であるとする.このとき
		\begin{align}
			\chi \in a \wedge \tau \in \chi
		\end{align}
		が成立するので,存在記号の規則より
		\begin{align}
			\exists x\, \left(\, x \in a \wedge \tau \in x\, \right)
		\end{align}
		が成り立ち
		\begin{align}
			\tau \in \bigcup a
		\end{align}
		が従う.$\tau$は任意に与えられていたので,(\refeq{fom:thm_union_is_bigger_than_any_member_1})の下で
		\begin{align}
			\forall t\, (\, t \in \chi \Longrightarrow t \in \bigcup a\, )
		\end{align}
		すなわち
		\begin{align}
			\chi \subset \bigcup a
		\end{align}
		が成り立つ.ゆえに
		\begin{align}
			\chi \in a \Longrightarrow \chi \subset \bigcup a
		\end{align}
		が従い,$\chi$も任意に与えられていたので
		\begin{align}
			\forall x\, (\, x \in a \Longrightarrow x \subset \bigcup a\, )
		\end{align}
		が得られる.
		\QED
	\end{sketch}
	
	$a,b$を類とするとき,その対の合併を
	\begin{align}
		a \cup b \defeq \bigcup \{a,b\}
	\end{align}
	と書く.
	
	\begin{screen}
		\begin{thm}[対の合併はそれぞれの要素を合わせたもの]\label{thm:union_of_pair_is_union_of_their_elements}
			$a$と$b$を集合とするとき
			\begin{align}
				\forall x\, (\, x \in a \cup b \Longleftrightarrow x \in a \vee x \in b\, ).
			\end{align}
		\end{thm}
	\end{screen}
	
	\monologue{
		この定理の主張は,$a$と$b$を類とするとき
		\begin{align}
			\set{a} \wedge \set{b} \Longrightarrow
			\forall x\, (\, x \in a \cup b \Longleftrightarrow x \in a \vee x \in b\, )
		\end{align}
		が成り立つということですが,式にまとめてしまうと見づらいのではじめから$a$と$b$を集合としています.
	}
	
	\begin{sketch}
		$\chi$を$\mathcal{L}$の任意の対象とする.
		\begin{align}
			\chi \in a \cup b
		\end{align}
		であるとき,
		\begin{align}
			\exists t\, \left(\, t \in \{a,b\} \wedge \chi \in t\, \right)
		\end{align}
		が成り立つので,
		\begin{align}
			\tau \defeq \varepsilon t\, \left(\, t \in \{a,b\} \wedge \chi \in t\, \right)
		\end{align}
		とおけば
		\begin{align}
			\tau \in \{a,b\} \wedge \chi \in \tau
		\end{align}
		が成立する.
		\begin{align}
			\tau \in \{a,b\}
		\end{align}
		が成り立つので,定理\ref{thm:pair_members_are_exactly_the_given_two}より
		\begin{align}
			\tau = a \vee \tau = b
			\label{fom:thm_union_of_pair_is_union_of_their_elements_1}
		\end{align}
		が従う.ここで相等性の公理より
		\begin{align}
			\tau = a \Longrightarrow \chi \in a
		\end{align}
		が成り立ち,論理和の規則から
		\begin{align}
			\tau = a \Longrightarrow \chi \in a \vee \chi \in b
		\end{align}
		も成り立つ.同様にして
		\begin{align}
			\tau = b \Longrightarrow \chi \in a \vee \chi \in b
		\end{align}
		が成り立つので,場合分け法則より
		\begin{align}
			\tau = a \vee \tau = b \Longrightarrow \chi \in a \vee \chi \in b
		\end{align}
		が成立し,(\refeq{fom:thm_union_of_pair_is_union_of_their_elements_1})と三段論法より
		\begin{align}
			\chi \in a \vee \chi \in b
		\end{align}
		が成立する.ゆえに演繹法則から
		\begin{align}
			\chi \in a \cup b \Longrightarrow \chi \in a \vee \chi \in b
			\label{fom:thm_union_of_pair_is_union_of_their_elements_2}
		\end{align}
		が成立する.逆に
		\begin{align}
			\chi \in a
		\end{align}
		であるとすると,
		\begin{align}
			\tau_a \defeq \varepsilon x\, (\, a = x\, )
		\end{align}
		とおけば定理\ref{thm:set_is_an_element_of_its_pair}より
		\begin{align}
			\tau_a \in \{a,b\} \wedge \chi \in \tau_a
		\end{align}
		が成り立つので,
		\begin{align}
			\exists t\, \left(\, t \in \{a,b\} \wedge \chi \in t\, \right)
		\end{align}
		が成り立ち
		\begin{align}
			\chi \in a \cup b
		\end{align}
		が従う.これでまず
		\begin{align}
			\chi \in a \Longrightarrow \chi \in a \cup b
		\end{align}
		が得られた.同様にして
		\begin{align}
			\chi \in b \Longrightarrow \chi \in a \cup b
		\end{align}
		も得られ,場合分け法則より
		\begin{align}
			\chi \in a \vee \chi \in b \Longrightarrow \chi \in a \cup b
			\label{fom:thm_union_of_pair_is_union_of_their_elements_3}
		\end{align}
		が成立する.以上(\refeq{fom:thm_union_of_pair_is_union_of_their_elements_2})と
		(\refeq{fom:thm_union_of_pair_is_union_of_their_elements_3})から
		\begin{align}
			\chi \in a \cup b \Longleftrightarrow \chi \in a \vee \chi \in b
		\end{align}
		が従い,$\chi$の任意性より
		\begin{align}
			\forall x\, (\, x \in a \cup b \Longleftrightarrow x \in a \vee x \in b\, ).
		\end{align}
		が出る.
		\QED
	\end{sketch}
	
	\begin{screen}
		\begin{thm}[等しい類の合併は等しい]\label{thm:unions_of_equal_classes_are_equal}
			$a$と$b$を類とするとき
			\begin{align}
				a = b \Longrightarrow \bigcup a = \bigcup b.
			\end{align}
		\end{thm}
	\end{screen}
	
	\begin{sketch}
		いま
		\begin{align}
			a = b
			\label{fom:thm_unions_of_equal_classes_are_equal}
		\end{align}
		が成り立っているとする.$\chi$を$\mathcal{L}$の任意の対象として
		\begin{align}
			\chi \in \bigcup a
		\end{align}
		であるとすれば,
		\begin{align}
			\tau \in a \wedge \chi \in \tau
		\end{align}
		なる$\mathcal{L}$の対象$\tau$が取れる.このとき相等性の公理より
		\begin{align}
			\tau \in b
		\end{align}
		が成り立つから
		\begin{align}
			\tau \in b \wedge \chi \in \tau
		\end{align}
		が従い,ゆえに
		\begin{align}
			\chi \in \bigcup b
		\end{align}
		が従う.ゆえに(\refeq{fom:thm_unions_of_equal_classes_are_equal})の下で
		\begin{align}
			\chi \in \bigcup a \Longrightarrow \chi \in \bigcup b
		\end{align}
		が得られたが,$a$と$b$を入れ替えれば
		\begin{align}
			\chi \in \bigcup b \Longrightarrow \chi \in \bigcup a
		\end{align}
		も得られるので
		\begin{align}
			\chi \in \bigcup a \Longleftrightarrow \chi \in \bigcup b
		\end{align}
		が成立する.そして$\chi$の任意性と外延性の公理から
		\begin{align}
			\bigcup a = \bigcup b
		\end{align}
		が成立する.ゆえに演繹法則から
		\begin{align}
			a = b \Longrightarrow \bigcup a = \bigcup b
		\end{align}
		が従う.
		\QED
	\end{sketch}
	
	\begin{screen}
		\begin{thm}[合併の可換律]
			$a$と$b$を類とするとき
			\begin{align}
				a \cup b = b \cup a.
			\end{align}
		\end{thm}
	\end{screen}
	
	\begin{sketch}
		定理\ref{thm:commutative_law_of_pairs}より
		\begin{align}
			\{a,b\} = \{b,a\}
		\end{align}
		が成り立つので,定理\ref{thm:unions_of_equal_classes_are_equal}から
		\begin{align}
			a \cup b = b \cup a
		\end{align}
		が従う.
		\QED
	\end{sketch}
	%\section{交叉}
	交叉とは合併の対となる概念である.$a$を類とするとき,$a$の全ての要素が共通して持つ集合の全体を$a$の交叉と呼び,
	合併の記号を上下に反転させて
	\begin{align}
		\bigcap a
	\end{align}
	と書く.またいささか奇妙な結果であるが,空虚な真の為せる業により空の交叉は宇宙に一致する.
	
	\begin{screen}
		\begin{dfn}[交叉]
			$a$を類とするとき,$a$の{\bf 交叉}\index{こうさ@交叉}{\bf (intersection)}を
			\begin{align}
				\bigcap a \defeq \Set{x}{\forall t \in a\, (\, x \in t\, )}
			\end{align}
			で定める.
		\end{dfn}
	\end{screen}
	
	上の定義に現れた
	\begin{align}
		\forall t \in a\, (\, x \in t\, )
	\end{align}
	とは
	\begin{align}
		\forall t\, (\, t \in a \Longrightarrow x \in t\, )
	\end{align}
	を略記した式である.
	
	\begin{screen}
		\begin{thm}[空集合の交叉は宇宙となる]\label{thm:union_of_the_emptyset_is_the_Universe}
			次が成立する:
			\begin{align}
				\bigcap \emptyset = \Univ.
			\end{align}
		\end{thm}
	\end{screen}
	
	\begin{prf}
		$x$を$\mathcal{L}$の任意の対象とするとき,空虚な真より
		\begin{align}
			t \in \emptyset \Longrightarrow x \in t
		\end{align}
		は$\mathcal{L}$のいかなる対象$t$に対してもに真となる.ゆえに
		\begin{align}
			\forall t \in \emptyset\, (\, x \in t\, )
		\end{align}
		が成立し
		\begin{align}
			\forall x\, (\, x \in \bigcap \emptyset\, )
		\end{align}
		が従う.
		\begin{align}
			\forall x\, (\, x \in \Univ\, )
		\end{align}
		も成り立つから
		\begin{align}
			\forall x\, (\, x \in \Univ \Longleftrightarrow x \in \bigcap \emptyset\, )
		\end{align}
		が成立して,外延性の公理より
		\begin{align}
			\bigcap \emptyset = \Univ
		\end{align}
		が従う.
		\QED
	\end{prf}
	
	\begin{screen}
		\begin{thm}[交叉は全ての要素に含まれる]
		\label{thm:intersection_is_obtained_by_all_elements}
			$a$を類とするとき
			\begin{align}
				\forall x\, (\, x \in a \Longrightarrow \bigcap a \subset x\, ).
			\end{align}
		\end{thm}
	\end{screen}
	
	\begin{screen}
		\begin{thm}[全ての要素に共通して含まれる類は交叉にも含まれる]
		\label{thm:if_obtained_by_all_elements_then_obtained_by_intersection}
			$a$と$b$を類とするとき
			\begin{align}
				\forall x \in a\, (\, b \subset x\, ) \Longrightarrow b \subset \bigcap a.
			\end{align}
		\end{thm}
	\end{screen}
	
	\begin{screen}
		\begin{thm}[等しい類の交叉は等しい]\label{thm:intersections_of_equal_classes_are_equal}
			$a$と$b$を類とするとき
			\begin{align}
				a = b \Longrightarrow \bigcap a = \bigcap b.
			\end{align}
		\end{thm}
	\end{screen}
	
	\begin{itembox}[l]{対の交叉}
		$a$と$b$を類とするとき,その対の交叉を
		\begin{align}
			a \cap b \defeq \bigcap \{a,b\}
		\end{align}
		と書く.
	\end{itembox}
	
	\begin{screen}
		\begin{thm}
			\begin{align}
				\forall x\, (\, x \in a \cap b \Longleftrightarrow x \in a \wedge x \in b\, ).
			\end{align}
		\end{thm}
	\end{screen}
	
	\begin{screen}
		\begin{thm}[交叉の可換律]
			\begin{align}
				a \cap b = b \cap a.
			\end{align}
		\end{thm}
	\end{screen}
	
	\begin{screen}
		\begin{thm}[対の交叉が空ならばその構成要素は共通元を持たない]
		\label{thm:if_pair_is_empty_then_its_members_do_not_intersect}
			$a,b$を類とするとき次が成立する:
			\begin{align}
				a \cap b = \emptyset \Longleftrightarrow \forall x\, (\, x \in a \Longrightarrow x \notin b\, ).
			\end{align}
		\end{thm}
	\end{screen}
	
	\begin{sketch}
		定理\ref{thm:uniqueness_of_emptyset}より
		\begin{align}
			a \cap b = \emptyset \Longleftrightarrow \forall x\, \left(\, x \notin a \cap b\, \right)
		\end{align}
		が成立する.また
		\begin{align}
			\forall x\, \left(\, x \notin a \cap b \Longleftrightarrow x \notin a \vee x \notin b\, \right)
		\end{align}
		かつ
		\begin{align}
			\forall x\, \left(\, (\, x \notin a \vee x \notin b\, ) \Longleftrightarrow (\, x \in a \Longrightarrow x \notin b\, )\, \right)
		\end{align}
		が成り立つので
		\begin{align}
			\forall x\, \left(\, x \notin a \cap b \Longleftrightarrow (\, x \in a \Longrightarrow x \notin b\, )\, \right)
		\end{align}
		が成立し,
		\begin{align}
			\forall x\, \left(\, x \notin a \cap b\, \right) \Longleftrightarrow 	
			\forall x\, (\, x \in a \Longrightarrow x \notin b\, )
		\end{align}
		が従う.ゆえに
		\begin{align}
			a \cap b = \emptyset \Longleftrightarrow \forall x\, (\, x \in a \Longrightarrow x \notin b\, ).
		\end{align}
		が得られる.
		\QED
	\end{sketch}
	
	\begin{screen}
		\begin{dfn}[差類]
			$a,b$を類するとき,$a$に属するが$b$には属さない集合の全体を
			$a$から$b$を引いた{\bf 差類}\index{さるい@差類}
			{\bf (class difference)}と呼び,記号は
			\begin{align}
				a \backslash b \defeq \Set{x}{x \in a \wedge x \notin b}
			\end{align}
			で定める.特に$a \backslash b$が集合であるときこれを
			{\bf 差集合}\index{さしゅうごう@差集合}{\bf (set difference)}と呼ぶ.
			また
			\begin{align}
				b \subset a
			\end{align}
			である場合,$a \backslash b$を$a$における$b$の{\bf 補類}\index{ほるい@補類}{\bf (complement)}或いは
			$a \backslash b$が集合であるとき{\bf 補集合}\index{ほしゅうごう@補集合}と呼ぶ.
		\end{dfn}
	\end{screen}
	
	$\set{a} \Longrightarrow \set{a \backslash b}$
	
	\begin{screen}
		\begin{thm}
			$a$と$b$を類とするとき,
			\begin{align}
				b \subset a
			\end{align}
			であれば
			\begin{align}
				\set{a \backslash b} \wedge \set{b} \Longrightarrow \set{a}.
			\end{align}
		\end{thm}
	\end{screen}
	
	\begin{sketch}
		対の公理から
		\begin{align}
			\{a \backslash b,b\}
		\end{align}
		は集合であり,合併の公理と
		\begin{align}
			a = (a \backslash b) \cup b
		\end{align}
		より
		\begin{align}
			\set{a}
		\end{align}
		が従う.
		\QED
	\end{sketch}
	
	\begin{screen}
		\begin{thm}[合併を引いた類は要素の差の交叉で書ける]
		\label{thm:difference_of_union_is_intersection_of_differences_of_elements}
			$a$と$b$を類とするとき,$a$が集合であれば
			\begin{align}
				a \backslash \bigcup b = \bigcap \Set{a \backslash t}{t \in b}.
			\end{align}
		\end{thm}
	\end{screen}
	
	\monologue{
		上の定理の式で
		\begin{align}
			\Set{a \backslash t}{t \in b}
		\end{align}
		と書いていますが,これは
		\begin{align}
			\Set{x}{\exists t \in b\, (\, x=a \backslash t\, )}
		\end{align}
		の略記です.ところがこれもまだ略記されたもので,正しく書くと
		\begin{align}
			\Set{x}{\exists t \in b\, 
			\forall s\, (\, s \in x \Longleftrightarrow s \in a \wedge s \notin t\, )}
		\end{align}
		となります.以降も煩雑さを避けるためにこのように略記します.
	}
	
	\begin{screen}
		\begin{thm}[二つの類の合併の差類は差類同士の交叉]
		\label{thm:difference_of_union_of_two_classes_is_intersection_of_two_differences}
			$a$と$b$と$c$を類とするとき
			\begin{align}
				a \backslash (b \cup c) = (a \backslash b) \cap (a \backslash c).
			\end{align}
		\end{thm}
	\end{screen}
	
	\begin{screen}
		\begin{thm}[交叉を引いた類は要素の差の合併で書ける]
		\label{thm:difference_of_intersection_is_union_of_differences_of_elements}
			$a$と$b$を類とするとき
			\begin{align}
				a \backslash \bigcap b = \bigcup \Set{a \backslash t}{t \in b}.
			\end{align}
		\end{thm}
	\end{screen}
	
	\begin{screen}
		\begin{thm}[二つの類の交叉の差類は差類同士の合併]
		\label{thm:difference_of_intersection_of_two_classes_is_union_of_two_differences}
			$a$と$b$と$c$を類とするとき
			\begin{align}
				a \backslash (b \cap c) = (a \backslash b) \cup (a \backslash c).
			\end{align}
		\end{thm}
	\end{screen}
	
	\begin{screen}
		\begin{thm}
			
		\end{thm}
	\end{screen}
	
	\begin{prf}\mbox{}
		\begin{description}
			\item[(1)] $a^{-1}$の任意の要素$t$に対し或る$V$の要素$x,y$が存在して
				\begin{align}
					(x,y) \in a \wedge t = (y,x)
				\end{align}
				を満たす.$((x,y),(y,x)) \in f$より$((x,y),t) \in f$が成り立つから
				$t \in f \ast a$となる.逆に$f \ast a$の任意の要素$t$に対して
				$a$の或る要素$x$が存在して
				\begin{align}
					x \in a \wedge (x,t) \in f
				\end{align}
				となる.$x$に対し$V$の或る要素$a,b$が存在して$x=(a,b)$となるので
				\begin{align}
					((a,b),t) \in f
				\end{align}
				となり,$V$の或る要素$c,d$が存在して
				\begin{align}
					((a,b),t) = ((c,d),(d,c))
				\end{align}
				となる.$(a,b) = (c,d)$より$a=c$かつ$b=d$となり,
				$t = (d,c)$かつ$(d,c)=(b,a)$より$t=(b,a)$,従って
				$t \in a^{-1}$が成り立つ.
		\end{description}
	\end{prf}
	\section{冪}
	\begin{screen}
		\begin{dfn}[冪]
			$a$を類とするとき,
			\begin{align}
				\power{a} \defeq \Set{x}{x \subset a}
			\end{align}
			で定義される類$\power{a}$を$a$の{\bf 冪}\index{べき@冪}{\bf (power)}と呼ぶ.
		\end{dfn}
	\end{screen}
	
	\begin{screen}
		\begin{axm}[冪の公理]
			集合の冪は集合である.つまり,$a$を類とするとき
			\begin{align}
				\set{a} \Longrightarrow \set{\power{a}}.
			\end{align}
		\end{axm}
	\end{screen}
	\subsection{関係}
	\begin{screen}
		\begin{dfn}[対集合]
			$a,b$を集合とするとき,$\mathcal{L}$の或る対象$x,y$が存在して$a = x$かつ$b = y$を満たすが,このとき
			\begin{align}
				\{a,b\} \coloneqq \Set{t}{t \in x \wedge t \in y}
			\end{align}
			で$\{a,b\}$を定義し,これを$a$と$b$の{\bf 対集合}と呼ぶ.
		\end{dfn}
	\end{screen}
	
	\monologue{
		院生「一般に集合$a,b$に対して
			\begin{align}
				\{a,b\} \coloneqq \Set{t}{t \in a \wedge t \in b}
			\end{align}
			と定めることは出来ません.$a,b$が集合であっても$\mathcal{L}$の対象ではない場合,そもそも
			\begin{align}
				\Set{t}{t \in a \wedge t \in b}
			\end{align}
			が$\mathcal{L}'$の対象でないのです.」
	}
	
	\begin{screen}
		\begin{axm}[対集合の公理]
			$a,b$を集合とするとき次が成り立つ:
			\begin{align}
				\{a,b\} \in \Univ.
			\end{align}
		\end{axm}
	\end{screen}
	
	\begin{screen}
		\begin{dfn}[順序対]
			集合$x,y$に対し,
			\begin{align}
				(x,y) = \{\{x\},\{x,y\}\}
			\end{align}
			で定義される類$(x,y)$を$x$と$y$の{\bf 順序対}\index{じゅんじょつい@順序対}
			{\bf (ordered pair)}と呼ぶ.
		\end{dfn}
	\end{screen}
	
	\begin{screen}
		\begin{thm}[順序対は集合]\mbox{}
			\begin{description}
				\item[(1)] $\forall x,y\ \left(\ (x,y) \in \Univ\ \right)$.
				\item[(2)] $\forall x,y,s,t\ 
					\left(\ (x,y)=(s,t) \Longleftrightarrow x=s \wedge y=t\ \right)$.
			\end{description}
		\end{thm}
	\end{screen}
	
	\begin{screen}
		\begin{dfn}[Cartesian積]
			類$a,b$に対し,$a \times b$を
			\begin{align}
				a \times b = \Set{x}{\exists s \in a\ \exists t \in b\ (\ x=(s,t)\ )}
			\end{align}
			で定め,これを$a$と$b$の{\bf Cartesian 積}\index{Cartesian せき@Cartesian 積}
			{\bf (Cartesian product)}と呼ぶ.
		\end{dfn}
	\end{screen}
	
	\monologue{
		院生「類$a$と類$b$のCartesian 積は
			\begin{align}
				a \times b = \Set{(s,t)}{s \in a \wedge t \in b} 
			\end{align}
			と簡略して書かれることも多いです.
	}
	
	\begin{comment}
	\monologue{
		院生「類$a$と類$b$のCartesian 積は
			\begin{align}
				a \times b = \Set{(s,t)}{s \in a \wedge t \in b} 
			\end{align}
			と簡略して書かれることも多いです.ところで他の本やネットなどを見ていると
			Cartesian 積を直積とも呼んでいるそうです.本稿でも後で直積というものを定義いたしますが,
			本稿ではCartesian 積と直積を明確に区別いたします.
			これは巷にあふれる直積の定義の不自然さを解消するためです.
			どういう点が不自然であるか簡単に説明いたしましょう.
			まだ有限とか数だとか定義していませんが,説明の便宜のために使用いたします.
			よく見る直積の定義だと,有限か有限でないかで直積の定め方が変わります.
			\begin{align}
				I_1 \times I_2 \times \cdots \times I_n 
				= \Set{(x_1,x_2,\cdots,x_n)}{x_1 \in I_1 \wedge x_2 \in I_2 \wedge
				\cdots \wedge x_n \in I_n}
			\end{align}
			そして
			\begin{align}
				I_1 \times I_2 \times \cdots \times I_n 
				= \prod_{i=1}^n I_i
			\end{align}
			と書いている.ここで
			$\prod_{i=1}^n I_i$は$\prod_{i\in\{1,2,\cdots,n\}} I_i$の別の記法です.
			他方$I$を$\{1,2,\cdots,n\}$から$V$への写像と見ることもできますから
			\begin{align}
				\prod_{i=1}^n I_i = \Set{f}{f:\{1,2,\cdots,n\} \longrightarrow V \wedge \forall i \in \{1,2,\cdots,n\}\ (\ f(i) \in I_i\ )}
			\end{align}
			となるはずです.食い違います.
			」
	}
	\end{comment}
	
	\begin{screen}
		\begin{dfn}[関係]
			$V \times V$の部分類を{\bf 関係}\index{かんけい@関係}{\bf (relation)}と呼ぶ.
		\end{dfn}
	\end{screen}
	
	いま,関係$E$を
	\begin{align}
		E = \Set{x}{\exists s,t\ (\ x=(s,t) \wedge s = t\ )}
	\end{align}
	と定めてみる.このとき$E$は次の性質を満たす:
	\begin{description}
		\item[(a)] $\forall x\ (\ (x,x) \in E\ )$.
		\item[(b)] $\forall x,y\ (\ (x,y) \in E \Longrightarrow (y,x) \in E\ )$.
		\item[(c)] $\forall x,y,z\ (\ (x,y) \in E \wedge (y,z) \in E \Longrightarrow (x,z) \in E\ )$.
	\end{description}
	性質(a)を反射律と呼ぶ.性質(b)を対称律と呼ぶ.性質(c)を推移律と呼ぶ.
	
	\begin{screen}
		\begin{dfn}[同値関係]
			$a$を集合とするとき,
			\begin{description}
				\item[反射律] $\forall x \in a\ (\ (x,x) \in R\ )$.
				\item[対称律] $\forall x,y \in a\ (\ (x,y) \in R \Longrightarrow (y,x) \in R\ )$.
				\item[推移律] $\forall x,y,z \in a\ (\ (x,y) \in R \wedge (y,z) \in R \Longrightarrow (x,z) \in R\ )$.
			\end{description}
			を満たす関係$R$を$a$上の{\bf 同値関係}\index{どうちかんけい@同値関係}
			{\bf (equivalence relation)}と呼ぶ.
		\end{dfn}
	\end{screen}
	
	\monologue{
		院生「集合$a$に対して$R = E \cap (a \times a)$とおけば$R$は$a$上の同値関係となりますね.」
	}
	
	$E$とは別の関係$O$を
	\begin{align}
		O = \Set{x}{\exists s,t\ (\ x=(s,t) \wedge s \subset t\ )}
	\end{align}
	により定めてみる.このとき$O$は次の性質を満たす:
	\begin{description}
		\item[(a)] $\forall x\ (\ (x,x) \in O\ )$.
		\item[(b')] $\forall x,y\ (\ (x,y) \in O \wedge (y,x) \in O \Longrightarrow x=y\ )$.
		\item[(c)] $\forall x,y,z\ (\ (x,y) \in O \wedge (y,z) \in O \Longrightarrow (x,z) \in O\ )$.
	\end{description}
	性質(b')を反対称律と呼ぶ.
	
	\begin{screen}
		\begin{dfn}[順序関係]
			$a$を類とする.$a$に対し或る関係$R$が存在して
			\begin{description}
				\item[反射律] $\forall x \in a\ (\ (x,x) \in R\ )$.
				\item[反対称律] $\forall x,y \in a\ (\ (x,y) \in R \wedge (y,x) \in R \Longrightarrow x=y\ )$.
				\item[推移律] $\forall x,y,z \in a\ (\ (x,y) \in R \wedge (y,z) \in R \Longrightarrow (x,z) \in R\ )$.
			\end{description}
			が満たされているとき,$R$を$a$上の{\bf 順序}\index{じゅんじょ@順序}{\bf (order)}と呼ぶ.
			$a,R$が共に集合であるときは対$(a,R)$を{\bf 順序集合}\index{じゅんじょしゅうごう@順序集合}
			{\bf (ordered set)}と呼ぶ.特に
			\begin{align}
				\forall x,y \in a\ (\ (x,y) \in R \vee (y,x) \in R\ )
			\end{align}
			が成り立つとき,$R$を$a$上の{\bf 全順序}\index{ぜんじゅんじょ@全順序}
			{\bf (total order)}と呼ぶ.			
		\end{dfn}
	\end{screen}
	
	\monologue{
		院生「反射律と推移律のみを満たす関係を{\bf 前順序}\index{ぜんじゅんじょ@前順序}
			{\bf (preorder)}と呼びます.また全順序は{\bf 線型順序}
			\index{せんけいじゅんじょ@線型順序}{\bf (linear order)}とも呼ばれます.
			また表記上の問題ですが,集合$R$を集合$a$上の順序関係として
			\begin{align}
				x \leq y \Longleftrightarrow (x,y) \in R
			\end{align}
			で記号$\leq$を定めるとき,$(a,\leq)$と順序対の形で表して
			これを順序集合と呼ぶこともあります.」
	}
	
	\begin{screen}
		\begin{dfn}[整列集合]
			$x$が{\bf 整列集合}\index{せいれつしゅうごう@整列集合}{\bf (wellordered set)}
			であるとは,$x$が集合$a$と$a$上の順序$R$の対$(a,R)$に等しく,
			かつ$a$の空でない任意の部分集合が$R$に関する最小元を持つことをいう.
			またこのときの$R$を{\bf 整列順序}\index{せいれつじゅんじょ@整列順序}
			{\bf (wellorder)}と呼ぶ.
		\end{dfn}
	\end{screen}
	
	\begin{screen}
		\begin{thm}[整列順序は全順序]
		\end{thm}
	\end{screen}
	%\input{chapters/sets/relation_2}
	\section{順序数}
	$0,1,2,\cdots$で表される数字は,集合論において
	\begin{align}
		0 &\defeq \emptyset, \\
		1 &\defeq \{0\} = \{\emptyset\}, \\
		2 &\defeq \{0,1\} = \{\emptyset,\{\emptyset\}\}, \\
		3 &\defeq \{0,1,2\} = \{\emptyset,\{\emptyset\},\{\emptyset,\{\emptyset\}\}\}
	\end{align}
	といった反復操作で定められる.上の操作を受け継いで``頑張れば手で書き出せる''類を自然数と呼ぶ.
	$0$は集合であり,対集合の公理から$1$もまた集合である.
	そして和集合の公理から$2$が集合であること,更には$3,4,\cdots$と続く自然数が全て集合であることがわかる.
	自然数の冪も自然数同士の集合演算もその結果は全て集合になるが,
	ここで
	\begin{align}
		\mbox{集合は$0$に集合演算を施しただけの素姓が明らかなものに限られるか}
	\end{align}
	という疑問というか期待が自然に生まれてくる.実際それは正則性公理によって肯定されるわけだが,
	そこでキーになるのは順序数と呼ばれる概念である.
	
	\begin{screen}
		\begin{logicalthm}[論理和・論理積の結合律]\label{logicalthm:associative_law}
			$A,B,C$を$\mathcal{L}'$の閉式とするとき次が成り立つ:
			\begin{description}
				\item[(イ)] $(A \vee B) \vee C \Longleftrightarrow A \vee (B \vee C)$.
				\item[(ロ)] $(A \wedge B) \wedge C \Longleftrightarrow A \wedge (B \wedge C)$.
			\end{description}
		\end{logicalthm}
	\end{screen}
	
	\begin{screen}
		\begin{logicalthm}[論理和・論理積の分配律]\label{logicalthm:distributive_law}
			$A,B,C$を$\mathcal{L}'$の閉式とするとき次が成り立つ:
			\begin{description}
				\item[(イ)] $(A \vee B) \wedge C \Longleftrightarrow (A \wedge C) \vee (B \wedge C)$.
				\item[(ロ)] $(A \wedge B) \vee C \Longleftrightarrow (A \vee C) \wedge (B \vee C)$.
			\end{description}
		\end{logicalthm}
	\end{screen}
	
	\begin{prf}\mbox{}
		\begin{description}
			\item[(イ)] いま$(A \vee B) \wedge C$が成立していると仮定する.
				このとき論理積の除去により$A \vee B$と$C$が同時に成り立つ.ここで$A$が成り立っているとすれば,
				論理積の導入により
				\begin{align}
					A \wedge C
				\end{align}
				が成り立つので演繹法則より
				\begin{align}
					A \Longrightarrow (A \wedge C)
				\end{align}
				が成立する.他方で論理和の導入より
				\begin{align}
					(A \wedge C) \Longrightarrow (A \wedge C) \vee (B \wedge C)
				\end{align}
				も成り立つので,含意の推移律から
				\begin{align}
					A \Longrightarrow (A \wedge C) \vee (B \wedge C)
				\end{align}
				が従う.$A$と$B$を入れ替えれば
				\begin{align}
					B \Longrightarrow (B \wedge C) \vee (A \wedge C)
				\end{align}
				が成り立つが,論理和の可換律より
				\begin{align}
					(B \wedge C) \vee (A \wedge C) \Longrightarrow (A \wedge C) \vee (B \wedge C)
				\end{align}
				が成り立つので
				\begin{align}
					B \Longrightarrow (A \wedge C) \vee (B \wedge C)
				\end{align}
				が従う.よって場合分け法則から
				\begin{align}
					(A \vee B) \Longrightarrow (A \wedge C) \vee (B \wedge C)
				\end{align}
				が成立するが,いま$A \vee B$は満たされているので三段論法より
				\begin{align}
					(A \wedge C) \vee (B \wedge C)
				\end{align}
				が成立する.ここに演繹法則を適用すれば
				\begin{align}
					(A \vee B) \wedge C \Longrightarrow (A \wedge C) \vee (B \wedge C)
				\end{align}
				が得られる.次に$A \wedge C$が成り立っていると仮定する.このとき
				$A$が成り立つので$A \vee B$も成立し,同時に$C$も成り立つので
				$(A \vee B) \wedge C$が成立する.すなわち
				\begin{align}
					A \wedge C \Longrightarrow (A \vee B) \wedge C
				\end{align}
				が成立する.$A$と$B$を入れ替えれば
				\begin{align}
					B \wedge C \Longrightarrow (A \vee B) \wedge C
				\end{align}
				も成立するので
				\begin{align}
					(A \wedge C) \vee (B \wedge C) \Longrightarrow (A \vee B) \wedge C
				\end{align}
				が得られる.
				
			\item[(ロ)]
				(イ)の結果を$\rightharpoondown A,\rightharpoondown B,\rightharpoondown C$に適用すれば
				\begin{align}
					(\rightharpoondown A \vee \rightharpoondown B) \wedge \rightharpoondown C
					\Longleftrightarrow (\rightharpoondown A \wedge \rightharpoondown C) 
						\vee (\rightharpoondown B \wedge \rightharpoondown C)
				\end{align}
				が得られる.ここでDe Morganの法則と同値記号の遺伝性質から
				\begin{align}
					(\rightharpoondown A \vee \rightharpoondown B) \wedge \rightharpoondown C
					&\Longleftrightarrow\ \rightharpoondown (A \wedge B) \wedge \rightharpoondown C \\
					&\Longleftrightarrow\ \rightharpoondown ((A \wedge B) \vee C)
				\end{align}
				が成立し,一方で
				\begin{align}
					(\rightharpoondown A \wedge \rightharpoondown C) 
						\vee (\rightharpoondown B \wedge \rightharpoondown C)
					&\Longleftrightarrow\ \rightharpoondown (A \vee C) \vee \rightharpoondown (B \vee C) \\
					&\Longleftrightarrow\ \rightharpoondown ((A \vee C) \wedge (B \vee C))
				\end{align}
				も成立するから,含意の推移律より
				\begin{align}
					\rightharpoondown ((A \wedge B) \vee C)
					\Longleftrightarrow\ \rightharpoondown ((A \vee C) \wedge (B \vee C))
				\end{align}
				が従う.最後に対偶を取れば
				\begin{align}
					(A \wedge B) \vee C \Longleftrightarrow (A \vee C) \wedge (B \vee C)
				\end{align}
				が得られる.
				\QED
		\end{description}
	\end{prf}
	
	\begin{screen}
		\begin{logicalthm}[選言三段論法]\label{logicalthm:disjunctive_syllogism}
			$A,B,C$を$\mathcal{L}'$の閉式とするとき次が成り立つ:
			\begin{align}
				(A \vee B) \wedge \rightharpoondown B \Longrightarrow A.
			\end{align}
		\end{logicalthm}
	\end{screen}
	
	\begin{prf}
		分配律(推論法則\ref{logicalthm:distributive_law})より
		\begin{align}
			(A \vee B) \wedge \rightharpoondown B
			\Longrightarrow (A \wedge \rightharpoondown B) \vee (B \wedge \rightharpoondown B)
		\end{align}
		が成立する.ここで矛盾に関する規則から
		\begin{align}
			B \wedge \rightharpoondown B \Longrightarrow \bot
		\end{align}
		が満たされるので
		\begin{align}
			(A \wedge \rightharpoondown B) \vee (B \wedge \rightharpoondown B)
			\Longrightarrow (A \wedge \rightharpoondown B) \vee \bot
		\end{align}
		が従う.また,論理積の除去より
		\begin{align}
			(A \wedge \rightharpoondown B) \Longrightarrow A
		\end{align}
		が成り立ち,他方で矛盾に関する規則より
		\begin{align}
			\bot \Longrightarrow A
		\end{align}
		も成り立つから,場合分け法則より
		\begin{align}
			(A \wedge \rightharpoondown B) \vee \bot \Longrightarrow A
		\end{align}
		が従う.以上の式と含意の推移律から
		\begin{align}
			(A \vee B) \wedge \rightharpoondown B \Longrightarrow A
		\end{align}
		が得られる.
		\QED
	\end{prf}
	
	\begin{screen}
		\begin{axm}[正則性公理]
			$a$を類とするとき,$a$は空でなければ自分自身と交わらない要素を持つ:
			\begin{align}
				a \neq \emptyset \Longrightarrow 
				\exists x \in a\, (\, x \cap a = \emptyset\, ).
			\end{align}
		\end{axm}
	\end{screen}
	
	\begin{screen}
		\begin{thm}[いかなる類も自分自身を要素に持たない]
		\label{thm:no_set_is_an_element_of_itself}
			$a,b,c$を類とするとき次が成り立つ:
			\begin{description}
				\item[(イ)] $a \notin a$.
				
				\item[(ロ)] $a \notin b \vee b \notin a$.
				
				\item[(ハ)] $a \notin b \vee b \notin c \vee c \notin a$.
			\end{description}
		\end{thm}
	\end{screen}
	
	\begin{prf}\mbox{}
		\begin{description}
			\item[(イ)] $a$を類とする.まず要素の公理の対偶より
				\begin{align}
					\rightharpoondown \set{a} \Longrightarrow a \notin a
				\end{align}
				が満たされる.次に$a$が集合であるとする.
				このとき定理\ref{thm:pair_of_proper_classes_is_emptyset}より
				\begin{align}
					a \in \{a\}
				\end{align}
				が成り立つから,正則性公理より
				\begin{align}
					\exists x\, \left(\, x \in \{z\} \wedge x \cap \{a\} = \emptyset\, \right)
				\end{align}
				が従う.ここで$\chi \coloneqq \varepsilon x\, \left(\, x \in \{a\} \wedge x \cap \{a\} = \emptyset\, \right)$とおけば
				\begin{align}
					\chi = a
				\end{align}
				となるので,相等性の公理より
				\begin{align}
					a \cap \{a\} = \emptyset
				\end{align}
				が成り立つ.$a \in \{a\}$であるから
				定理\ref{thm:if_pair_is_empty_then_its_members_do_not_intersect}より
				$a \notin a$が従い,演繹法則から
				\begin{align}
					\set{a} \Longrightarrow a \notin a
				\end{align}
				が得られる.そして場合分け法則から
				\begin{align}
					\set{a} \vee \rightharpoondown \set{a} \Longrightarrow a \notin a
				\end{align}
				が成立し,排中律と三段論法から
				\begin{align}
					a \notin a
				\end{align}
				が出る.
			
			\item[(ロ)]
				要素の公理より
				\begin{align}
					a \in b \Longrightarrow \set{a}
				\end{align}
				となり,定理\ref{thm:pair_of_proper_classes_is_emptyset}より
				\begin{align}
					\set{a} \Longrightarrow a \in \{a,b\}
				\end{align}
				となるので,
				\begin{align}
					a \in b \Longrightarrow a \in \{a,b\}
				\end{align}
				が成立する.また定理\ref{thm:if_pair_is_empty_then_its_members_do_not_intersect}より
				\begin{align}
					a \in b \wedge a \in \{a,b\} 
					&\Longrightarrow \exists x\, \left(\, x \in b \wedge x \in \{a,b\}\, \right) \\
					&\Longrightarrow b \cap \{a,b\} \neq \emptyset
				\end{align}
				が成立する.他方で正則性公理より
				\begin{align}
					a \in \{a,b\} &\Longrightarrow \exists x\, \left(\, x \in \{a,b\}\, \right) \\
					&\Longrightarrow \{a,b\} \neq \emptyset \\
					&\Longrightarrow \exists x\, \left(\, x \in \{a,b\} \wedge x \cap \{a,b\} = \emptyset\, \right)
				\end{align}
				も成立する.以上を踏まえて$a \in b$が成り立っていると仮定する.このとき
				\begin{align}
					a \in \{a,b\}
				\end{align}
				が成立するので
				\begin{align}
					b \cap \{a,b\} \neq \emptyset
				\end{align}
				も成り立ち,さらに
				\begin{align}
					\exists x\, \left(\, x \in \{a,b\} \wedge x \cap \{a,b\} = \emptyset\, \right)
				\end{align}
				も満たされる.ここで
				\begin{align}
					\chi \coloneqq \varepsilon x\, \left(\, x \in \{a,b\} \wedge x \cap \{a,b\} = \emptyset\, \right)
				\end{align}
				とおけば$\chi \in \{a,b\}$から
				\begin{align}
					\chi = a \vee \chi = b
				\end{align}
				が従うが,相等性の公理より
				\begin{align}
					\chi = b \Longrightarrow b \cap \{a,b\} = \emptyset
				\end{align}
				となるので,$b \cap \{a,b\} \neq \emptyset$と併せて
				\begin{align}
					\chi \neq b
				\end{align}
				が成立する.選言三段論法(推論法則\ref{logicalthm:disjunctive_syllogism})より
				\begin{align}
					(\chi = a \vee \chi = b) \wedge \chi \neq b \Longrightarrow \chi = a
				\end{align}
				となるから
				\begin{align}
					\chi = a
				\end{align}
				が従い,相等性の公理より
				\begin{align}
					a \cap \{a,b\} = \emptyset
				\end{align}
				が成立する.いま要素の公理より
				\begin{align}
					\rightharpoondown \set{b} \Longrightarrow b \notin a
				\end{align}
				が満たされ,他方で定理\ref{thm:pair_of_proper_classes_is_emptyset}より
				\begin{align}
					\set{b} \Longrightarrow b \in \{a,b\},
				\end{align}
				$a \cap \{a,b\}$の仮定と定理\ref{thm:if_pair_is_empty_then_its_members_do_not_intersect}より
				\begin{align}
					b \in \{a,b\} = \emptyset \Longrightarrow b \notin a
				\end{align}
				が満たされるので
				\begin{align}
					\set{b} \Longrightarrow b \notin a
				\end{align}
				が成立する.従って
				\begin{align}
					b \notin a
				\end{align}
				が従い,演繹法則より
				\begin{align}
					a \in b \Longrightarrow b \notin a
				\end{align}
				が得られる.これは$a \notin b \vee b \notin a$と同値である.
				
			\item[(ハ)]
				$a \in b \wedge b \in c$が満たされていると仮定すれば,$a,b$は集合であるから
				\begin{align}
					a,b \in \{a,b,c\}
				\end{align}
				が成立する.ゆえに$b \cap \{a,b,c\} \neq \emptyset$と$c \cap \{a,b,c\} \neq \emptyset$が従う.
				他方,正則性公理より
				\begin{align}
					\tau \in \{a,b,c\} \wedge \tau \cap \{a,b,c\} = \emptyset
				\end{align}
				を満たす$\mathcal{L}$の対象$\tau$が取れる.ここで$\tau \in \{a,b,c\}$より
				\begin{align}
					\tau = a \vee \tau = b \vee \tau = a
				\end{align}
				が成り立つが,$b \cap \{a,b,c\} \neq \emptyset$と$c \cap \{a,b,c\} \neq \emptyset$
				より$\tau \neq b$かつ$\tau \neq c$となる.よって$\tau = a$となり
				\begin{align}
					a \cap \{a,b,c\} = \emptyset
				\end{align}
				が従う.$c$が真類ならば要素の公理より$c \notin a$となり,$c$が集合ならば$c \in \{a,b,c\}$となるので,
				いずれにせよ
				\begin{align}
					c \notin a
				\end{align}
				が成立する.以上で
				\begin{align}
					a \in b \wedge b \in c \Longrightarrow c \notin a
				\end{align}
				が得られる.
				\QED
		\end{description}
	\end{prf}
	
	\begin{screen}
		\begin{dfn}[順序数]
			類$a$に対して,$a$が{\bf 推移的類}\index{すいいてきるい@推移的類}{\bf (transitive class)}であるということを
			\begin{align}
				\tran{a} \defarrow
				\forall s\, (\, s \in a \Longrightarrow s \subset a\, )
			\end{align}
			で定める.また$a$が(集合であるならば){\bf 順序数}\index{じゅんじょすう@順序数}{\bf (ordinal number)}であるということを
			\begin{align}
				\ord{a} \defarrow
				\tran{a} \wedge \forall t,u \in a\, (\, t \in u \vee t = u \vee u \in t\, )
			\end{align}
			で定める.順序数の全体を
			\begin{align}
				\ON \defeq \Set{x}{\ord{x}}
			\end{align}
			とおく.
		\end{dfn}
	\end{screen}
	
	空虚な真の一例であるが,例えば$0$は順序数の性質を満たす.
	ここに一つの順序数が得られたが,いま仮に$\alpha$を順序数とすれば
	\begin{align}
		\alpha \cup \{\alpha\}
	\end{align}
	もまた順序数となることが直ちに判明する.数字の定め方から
	\begin{align}
		1 &= 0 \cup \{0\}, \\
		2 &= 1 \cup \{1\}, \\
		3 &= 2 \cup \{2\}, \\
		&\vdots
	\end{align}
	が成り立つから,数字は全て順序数である.
	
	いま$\ON$上の関係を
	\begin{align}
		\leq\ \defeq \Set{x}{\exists \alpha,\beta \in \ON\, 
		(\, x=(\alpha,\beta) \wedge \alpha \subset \beta\, )}
	\end{align}
	と定める.
		
	\begin{itembox}[l]{中置記法について}
		$x$と$y$を項とするとき,
		\begin{align}
			(x,y) \in\ \leq
		\end{align}
		なることを往々にして
		\begin{align}
			x \leq y
		\end{align}
		とも書くが,このような書き方を{\bf 中置記法}\index{ちゅうちきほう@中置記法}{\bf (infix notation)}と呼ぶ.
		同様にして,
		\begin{align}
			(x,y) \in\ \leq \wedge x \neq y
		\end{align}
		なることを
		\begin{align}
			x < y
		\end{align}
		とも書く.
	\end{itembox}
	
	以下順序数の性質を列挙するが,長いので主張だけ先に述べておく.
	\begin{itemize}
		\item $\ON$は推移的類である.
		\item $\leq$は$\ON$において整列順序となる.
		\item $a$を$a \subset \ON$なる集合とすると,$\bigcup a$は$a$の$\leq$に関する上限となる.
		\item $\ON$は集合ではない.
	\end{itemize}
	
	\begin{screen}
		\begin{thm}[推移的で$\in$が全順序となる類は$\ON$に含まれる]
		\label{thm:transitive_totally_ordered_class_is_contained_in_ON}
			$S$を類とするとき
			\begin{align}
				\ord{S} \Longrightarrow S \subset \ON.
			\end{align}
		\end{thm}
	\end{screen}
	
	\begin{sketch}
		$x$を$S$の要素とする.まず
		\begin{align}
			\forall s,t \in x\, (\, s \in t \vee s = t \vee t \in s\, )
			\label{fom:thm_transitive_totally_ordered_class_is_contained_in_ON_1}
		\end{align}
		が成り立つことを示す.実際$S$の推移性より
		\begin{align}
			x \subset S
		\end{align}
		が成り立つので,$x$の要素は全て$S$の要素となり
		(\refeq{fom:thm_transitive_totally_ordered_class_is_contained_in_ON_1})が満たされる.次に
		\begin{align}
			\tran{x}
		\end{align}
		が成り立つことを示す.$y$を$x$の要素とする.また$z$を$y$の要素とする.このとき
		\begin{align}
			x \subset S
		\end{align}
		から
		\begin{align}
			y \in S
		\end{align}
		が成り立つので
		\begin{align}
			y \subset S
		\end{align}
		が成り立ち,ゆえに
		\begin{align}
			z \in S
		\end{align}
		となる.従って
		\begin{align}
			z \in x \vee z = x \vee x \in z
			\label{fom:thm_transitive_totally_ordered_class_is_contained_in_ON_2}
		\end{align}
		が成立する.ところで定理\ref{thm:no_set_is_an_element_of_itself}より
		\begin{align}
			z \in y \Longrightarrow y \notin z
		\end{align}
		が成り立つから
		\begin{align}
			y \notin z
			\label{fom:thm_transitive_totally_ordered_class_is_contained_in_ON_3}
		\end{align}
		が成立する.また相当性の公理から
		\begin{align}
			z = x \vee y \in x \Longrightarrow y \in z
		\end{align}
		が成り立つので,その対偶と(\refeq{fom:thm_transitive_totally_ordered_class_is_contained_in_ON_2})から
		\begin{align}
			z \neq x \vee y \notin x
		\end{align}
		も満たされる.いま
		\begin{align}
			y \in x
		\end{align}
		が成り立っていて,さらに選言三段論法より
		\begin{align}
			(\, z \neq x \vee y \notin x\, ) \wedge y \in x \Longrightarrow z \neq x
		\end{align}
		も成り立つから,
		\begin{align}
			z \neq x
		\end{align}
		が成立する.他方で定理\ref{thm:no_set_is_an_element_of_itself}より
		\begin{align}
			z \in y \wedge y \in x \Longrightarrow x \notin z
		\end{align}
		が成立するから,ゆえにいま
		\begin{align}
			z \neq x \wedge x \notin z
		\end{align}
		が,つまり
		\begin{align}
			\rightharpoondown (\, z = x \vee x \in z\, )
			\label{fom:thm_transitive_totally_ordered_class_is_contained_in_ON_4}
		\end{align}
		が成立している.ここで選言三段論法より
		\begin{align}
			(\, z \in x \vee z = x \vee x \in z\, ) \wedge 
			\rightharpoondown (\, z = x \vee x \in z\, )
			\Longrightarrow z \in x
		\end{align}
		も成り立つので,(\refeq{fom:thm_transitive_totally_ordered_class_is_contained_in_ON_3})と
		(\refeq{fom:thm_transitive_totally_ordered_class_is_contained_in_ON_4})と併せて
		\begin{align}
			z \in x
		\end{align}
		が従う.以上より,$y$を$x$の要素とすれば
		\begin{align}
			\forall z \in y\, (\, z \in y \Longrightarrow z \in x\, )
		\end{align}
		が成り立ち,ゆえに
		\begin{align}
			y \subset x
		\end{align}
		が成り立つ.ゆえに$x$は推移的である.ゆえに
		\begin{align}
			\ord{x}
		\end{align}
		が成立し
		\begin{align}
			x \in \ON
		\end{align}
		となる.$x$の任意性より
		\begin{align}
			S \subset \ON
		\end{align}
		が得られる.
		\QED
	\end{sketch}
	
	\begin{screen}
		\begin{thm}[$\ON$は推移的]\label{thm:On_is_transitive}
			$\tran{\ON}$が成立する.
		\end{thm}
	\end{screen}
	
	\begin{prf} 
		$x$を順序数とすると
		\begin{align}
			\ord{x}
		\end{align}
		が成り立つので,定理\ref{thm:transitive_totally_ordered_class_is_contained_in_ON}から
		\begin{align}
			x \subset \ON
		\end{align}
		が成立する.ゆえに$\ON$は推移的である.
		\QED
	\end{prf}
	
	\begin{screen}
		\begin{thm}[$\ON$において$\in$と$<$は同義]
		\label{thm:element_and_proper_subset_correspond_between_ordinal_numbers}
			\begin{align}
				\forall \alpha,\beta \in \ON\,
				(\, \alpha \in \beta \Longleftrightarrow \alpha < \beta\, ).
			\end{align}
		\end{thm}
	\end{screen}
	
	\begin{prf}
		$\alpha,\beta$を任意に与えられた順序数とする.
		\begin{align}
			\alpha \in \beta
		\end{align}
		が成り立っているとすると,順序数の推移性より
		\begin{align}
			\alpha \subset \beta
		\end{align}
		が成り立つ.定理\ref{thm:no_set_is_an_element_of_itself}より
		\begin{align}
			\alpha \neq \beta
		\end{align}
		も成り立つから
		\begin{align}
			\alpha < \beta
		\end{align}
		が成り立つ.ゆえに
		\begin{align}
			\alpha \in \beta \Longrightarrow \alpha < \beta
		\end{align}
		が成立する.逆に
		\begin{align}
			\alpha < \beta
		\end{align}
		が成り立っているとすると
		\begin{align}
			\beta \backslash \alpha \neq \emptyset
		\end{align}
		が成り立つので,正則性公理より
		\begin{align}
			\gamma \in \beta \backslash \alpha \wedge \gamma \cap (\beta \backslash \alpha) = \emptyset
		\end{align}
		を満たす$\gamma$が取れる.このとき
		\begin{align}
			\alpha = \gamma
		\end{align}
		が成り立つことを示す.$x$を$\alpha$の任意の要素とすれば,
		$x,\gamma$は共に$\beta$に属するから
		\begin{align}
			x \in \gamma \vee x = \gamma \vee \gamma \in x
			\label{eq:thm_element_and_proper_subset_correspond_between_ordinal_numbers_1}
		\end{align}
		が成り立つ.ところで相等性の公理から
		\begin{align}
			x = \gamma \wedge x \in \alpha \Longrightarrow \gamma \in \alpha
		\end{align}
		が成り立ち,$\alpha$の推移性から
		\begin{align}
			\gamma \in x \wedge x \in \alpha \Longrightarrow \gamma \in \alpha
		\end{align}
		が成り立つから,それぞれ対偶を取れば
		\begin{align}
			\gamma \notin \alpha \Longrightarrow x \neq \gamma \vee x \notin \alpha
		\end{align}
		と
		\begin{align}
			\gamma \notin \alpha \Longrightarrow \gamma \notin x \vee x \notin \alpha
		\end{align}
		が成立する.いま
		\begin{align}
			\gamma \notin \alpha
		\end{align}
		が成り立っているので
		\begin{align}
			x \neq \gamma \vee x \notin \alpha
		\end{align}
		と
		\begin{align}
			\gamma \notin x \vee x \notin \alpha
		\end{align}
		が共に成り立ち,また
		\begin{align}
			x \in \alpha
		\end{align}
		でもあるから選言三段論法より
		\begin{align}
			x \neq \gamma
		\end{align}
		と
		\begin{align}
			\gamma \notin x
		\end{align}
		が共に成立する.そして(\refeq{eq:thm_element_and_proper_subset_correspond_between_ordinal_numbers_1})と
		選言三段論法より
		\begin{align}
			x \in \gamma
		\end{align}
		が従うので
		\begin{align}
			\alpha \subset \gamma
		\end{align}
		が得られる.逆に$x$を$\gamma$に任意の要素とすると
		\begin{align}
			x \in \beta \wedge x \notin \beta \backslash \alpha
		\end{align}
		が成り立つから,すなわち
		\begin{align}
			x \in \beta \wedge (\, x \notin \beta \vee x \in \alpha\, )
		\end{align}
		が成立する.ゆえに選言三段論法より
		\begin{align}
			x \in \alpha
		\end{align}
		が成り立ち,$x$の任意性より
		\begin{align}
			\gamma \subset \alpha
		\end{align}
		となる.従って
		\begin{align}
			\gamma = \alpha
		\end{align}
		が成立し,
		\begin{align}
			\gamma \in \beta
		\end{align}
		なので
		\begin{align}
			\alpha \in \beta
		\end{align}
		が成り立つ.以上で
		\begin{align}
			\alpha < \beta \Longrightarrow \alpha \in \beta
		\end{align}
		も得られた.
		\QED
	\end{prf}
	
	\begin{screen}
		\begin{thm}[$\ON$の整列性]\label{thm:On_is_wellordered}
			$\leq$は$\ON$上の整列順序である.また次が成り立つ.
			\begin{align}
				\forall \alpha,\beta \in \ON\,
				\left(\, \alpha \in \beta \vee \alpha = \beta \vee \beta \in \alpha\, \right).
			\end{align}
		\end{thm}
	\end{screen}
	
	\begin{prf}\mbox{}
		\begin{description}
			\item[第一段]
				$\alpha,\beta,\gamma$を順序数とすれば
				\begin{align}
					\alpha \subset \alpha
				\end{align}
				かつ
				\begin{align}
					\alpha \subset \beta \wedge \beta \subset \alpha \Longrightarrow \alpha = \beta
				\end{align}
				かつ
				\begin{align}
					\alpha \subset \beta \wedge \beta \subset \gamma \Longrightarrow \alpha \subset \gamma
				\end{align}
				が成り立つ.ゆえに$\leq$は$\ON$上の順序である.
				
			\item[第二段]
				$\leq$が全順序であることを示す.$\alpha$と$\beta$を順序数とする.このとき
				\begin{align}
					\ord{\alpha \cap \beta}
				\end{align}
				が成り立ち,他方で定理\ref{thm:no_set_is_an_element_of_itself}より
				\begin{align}
					\alpha \cap \beta \notin \alpha \cap \beta
				\end{align}
				が満たされるので
				\begin{align}
					\alpha \cap \beta \notin \alpha \vee \alpha \cap \beta \notin \beta
					\label{eq:thm_On_is_wellordered_5}
				\end{align}
				が成立する.ところで
				\begin{align}
					\alpha \cap \beta \subset \alpha
				\end{align}
				は正しいので定理\ref{thm:element_and_proper_subset_correspond_between_ordinal_numbers}から
				\begin{align}
					\alpha \cap \beta \in \alpha \vee \alpha \cap \beta = \alpha
				\end{align}
				が成立する.従って
				\begin{align}
					\alpha \cap \beta \notin \alpha \Longrightarrow 
					(\alpha \cap \beta \in \alpha \vee \alpha \cap \beta = \alpha) \wedge \alpha \cap \beta \notin \alpha
					\label{eq:thm_On_is_wellordered_2}
				\end{align}
				が成り立ち,他方で選言三段論法より
				\begin{align}
					(\alpha \cap \beta \in \alpha \vee \alpha \cap \beta = \alpha) \wedge \alpha \cap \beta \notin \alpha
					\Longrightarrow \alpha \cap \beta = \alpha
					\label{eq:thm_On_is_wellordered_3}
				\end{align}
				も成り立ち,かつ
				\begin{align}
					\alpha \cap \beta = \alpha \Longrightarrow \alpha \subset \beta
					\label{eq:thm_On_is_wellordered_4}
				\end{align}
				も成り立つので,(\refeq{eq:thm_On_is_wellordered_2})と(\refeq{eq:thm_On_is_wellordered_3})と
				(\refeq{eq:thm_On_is_wellordered_4})から
				\begin{align}
					\alpha \cap \beta \notin \alpha \Longrightarrow \alpha \subset \beta
				\end{align}
				が得られる.同様にして
				\begin{align}
					\alpha \cap \beta \notin \beta \Longrightarrow \beta \subset \alpha
				\end{align}
				も得られる.さらに論理和の規則から
				\begin{align}
					\alpha \cap \beta \notin \alpha \Longrightarrow \alpha \subset \beta \vee \beta \subset \alpha
				\end{align}
				と
				\begin{align}
					\alpha \cap \beta \notin \beta \Longrightarrow \alpha \subset \beta \vee \beta \subset \alpha
				\end{align}
				が従うので,(\refeq{eq:thm_On_is_wellordered_5})と場合分け法則より
				\begin{align}
					\alpha \subset \beta \vee \beta \subset \alpha
				\end{align}
				が成立して
				\begin{align}
					(\alpha,\beta) \in\ \leq \vee (\beta,\alpha) \in\ \leq
				\end{align}
				が成立する.ゆえに$\leq$は全順序である.
			
			\item[第三段]
				$\leq$が整列順序であることを示す.$a$を$\ON$の空でない部分集合とする.このとき正則性公理より
				\begin{align}
					x \in a \wedge x \cap a = \emptyset
				\end{align}
				を満たす集合$x$が取れるが,この$x$が$a$の最小限である.実際,任意に$a$から要素$y$を取ると
				\begin{align}
					x \cap a = \emptyset
				\end{align}
				から
				\begin{align}
					y \notin x
				\end{align}
				が従い,また前段の結果より
				\begin{align}
					x \in y \vee x = y \vee y \in x
				\end{align}
				も成り立つので,選言三段論法より
				\begin{align}
					x \in y \vee x = y
					\label{eq:thm_On_is_wellordered_6}
				\end{align}
				が成り立つ.$y$は推移的であるから
				\begin{align}
					x \in y \Longrightarrow x \subset y
				\end{align}
				が成立して,また
				\begin{align}
					x = y \Longrightarrow x \subset y
				\end{align}
				も成り立つから,(\refeq{eq:thm_On_is_wellordered_6})と場合分け法則から
				\begin{align}
					(x,y) \in\ \leq
				\end{align}
				が従う.$y$の任意性より
				\begin{align}
					\forall y \in a\, \left[\, (x,y) \in\ \leq\, \right]
				\end{align}
				が成立するので$x$は$a$の最小限である.
				\QED
		\end{description}
	\end{prf}
	
	\begin{screen}
		\begin{thm}[$\ON$の部分集合の合併は順序数となる]\label{thm:union_of_set_of_ordinal_numbers_is_ordinal}
			\begin{align}
				\forall a\,
				\left(\, a \subset \ON \Longrightarrow \bigcup a \in \ON\, \right).
			\end{align}
		\end{thm}
	\end{screen}
	
	\begin{prf}
		和集合の公理より$\bigcup a \in \Univ$となる.また順序数の推移性より
		$\bigcup a$の任意の要素は順序数であるから,定理\ref{thm:On_is_wellordered}より
		\begin{align}
			\forall x,y \in \bigcup a\ (\ x \in y \vee x = y \vee y \in x\ )
		\end{align}
		も成り立つ.最後に$\operatorname{Tran}(\bigcup a)$が成り立つことを示す.
		$b$を$\bigcup a$の任意の要素とすれば,$a$の或る要素$x$に対して
		\begin{align}
			b \in x
		\end{align}
		となるが,$x$の推移性より$b \subset x$となり,$x \subset \bigcup a$と併せて
		\begin{align}
			b \subset \bigcup a
		\end{align}
		が従う.
		\QED
	\end{prf}
	
	\begin{screen}
		\begin{thm}[Burali-Forti]\label{thm:Burali_Forti}
			順序数の全体は集合ではない.
			\begin{align}
				\rightharpoondown \set{\ON}.
			\end{align}
		\end{thm}
	\end{screen}
	
	\begin{prf}
		$a$を類とするとき,定理\ref{thm:satisfactory_set_is_an_element}より
		\begin{align}
			\ord{a} \Longrightarrow \left(\, \set{a} \Longrightarrow a \in \ON\, \right)
		\end{align}
		が成り立つ.定理\ref{thm:On_is_transitive}と定理\ref{thm:On_is_wellordered}より
		\begin{align}
			\ord{\ON}
		\end{align}
		が成り立つから
		\begin{align}
			\set{\ON} \Longrightarrow \ON \in \ON
			\label{eq:Burali_Forti_1}
		\end{align}
		が従い,また定理\ref{thm:no_set_is_an_element_of_itself}より
		\begin{align}
			\ON \notin \ON
		\end{align}
		も成り立つので,(\refeq{eq:Burali_Forti_1})の対偶から
		\begin{align}
			\rightharpoondown \set{\ON}
		\end{align}
		が成立する.
		\QED
	\end{prf}
	
	\begin{screen}
		\begin{thm}[順序数は自分自身との合併が後者となる]\label{thm:latter_element_is_ordinal}
			$\alpha$が順序数であるということと $\alpha \cup \{\alpha\}$が順序数であるということは同値である.
			\begin{align}
				\forall \alpha\ (\ \alpha \in \ON \Longleftrightarrow \alpha \cup \{\alpha\} \in \ON\ ).
			\end{align}
		\end{thm}
	\end{screen}
	
	\begin{screen}
		\begin{thm}[順序数は自分自身との合併が後者となる]
			$\alpha$を順序数とすれば,$\ON$において$\alpha \cup \{\alpha\}$は$\alpha$の後者である:
			\begin{align}
				\forall \alpha \in \ON\ 
				\left(\ \forall \beta \in \ON\ (\ \alpha < \beta 
				\Longrightarrow \alpha \cup \{\alpha\} \leq \beta\ )
				\ \right).
			\end{align}
		\end{thm}
	\end{screen}
		\begin{screen}
		\begin{dfn}[推移的クラス]
			$\mathcal{L}$の項$x$に対して,$x$が{\bf 推移的}\index{すいいてき@推移的}
			{\bf (transitive)}であるということを
			\begin{align}
				\tran{x} \defarrow
				\forall s\, (\, s \in x \rarrow s \subset x\, )
			\end{align}
			で定める.
		\end{dfn}
	\end{screen}
	
	$x$が推移的であるとは,「$x$の要素の要素が$x$の要素となる」という意味である.
	
	\begin{screen}
		\begin{dfn}[順序数]
			$\mathcal{L}$の項$x$に対して
			\begin{align}
				\ord{x} \defarrow \tran{x} \wedge 
				\forall t,u \in x\, (\, t \in u \vee t = u \vee u \in t\, )
			\end{align}
			と定め(ただし$t \in u \vee t = u \vee u \in t$は
			$(\, t \in u \vee t = u\, ) \vee u \in t$の略記とする),
			\begin{align}
				\ON \defeq \Set{x}{\ord{x}}
			\end{align}
			とおく.$\ON$の要素を{\bf 順序数}\index{じゅんじょすう@順序数}
			{\bf (ordinal number)}と呼ぶ.
		\end{dfn}
	\end{screen}
	
	空虚な真の一例であるが,例えば$0$は順序数の性質を満たす.
	ここに一つの順序数が得られたが,いま仮に$\alpha$を順序数とすれば
	\begin{align}
		\alpha \cup \{\alpha\}
	\end{align}
	もまた順序数となることが直ちに判明する.数字の定め方から
	\begin{align}
		1 &= 0 \cup \{0\}, \\
		2 &= 1 \cup \{1\}, \\
		3 &= 2 \cup \{2\}, \\
		&\vdots
	\end{align}
	が成り立つから,数字は全て順序数である.
	
	いま関係を
	\begin{align}
		\leq\ \defeq \Set{x}{\exists \alpha,\beta\, 
		(\, x=(\alpha,\beta) \wedge \alpha \subset \beta\, )}
	\end{align}
	と定める.そして
	\begin{align}
		x \leq y &\defarrow (x,y) \in\ \leq, \\
		x < y &\defarrow x \leq y \wedge x \neq y
	\end{align}
	と書く(中置記法).
	
	以下順序数の性質を列挙するが,長いので主張だけ先に述べておく.
	\begin{itemize}
		\item $\ON$は推移的クラスである.
		\item $\ON$上で$\in$と$<$は同義になる.
		\item $\leq$は$\ON$において整列順序となる.
		%\item $a$を$a \subset \ON$なる集合とすると,$\bigcup a$は$a$の$\leq$に関する上限となる.
		\item $\ON$は集合ではない.
	\end{itemize}
	
	\begin{screen}
		\begin{thm}[推移的で$\in$が全順序となるクラスは$\ON$に含まれる]
		\label{thm:transitive_totally_ordered_class}
			$a$をクラスとするとき
			\begin{align}
				\EXTAX,\EQAX,\COMAX,\PAIAX,\UNIAX,\REGAX \vdash 
				\ord{a} \rarrow a \subset \ON.
			\end{align}
		\end{thm}
	\end{screen}
	
	\begin{sketch}
		いま
		\begin{align}
			\chi \defeq \varepsilon x \negation 
			(\, x \in a \rarrow x \in \ON\, )
		\end{align}
		とおく.
		\begin{description}
			\item[step1] まず
				\begin{align}
					\chi \in a,\ \ord{a} \vdash 
					\forall s,t \in \chi\, (\, s \in t \vee s = t \vee t \in s\, )
					\label{fom:thm_transitive_totally_ordered_class_1}
				\end{align}
				を示す.$a$の推移性より
				\begin{align}
					\chi \in a,\ \ord{a} \vdash \chi \subset a
				\end{align}
				が成り立つから,
				\begin{align}
					\sigma &\defeq \varepsilon s \negation 
					(\, s \in \chi \rarrow \forall t\, (\, t \in \chi \rarrow 
					(\, s \in t \vee s = t \vee t \in s\, )\, )\, ), \\
					\tau &\defeq \varepsilon t \negation (\, t \in \chi \rarrow 
					(\, \sigma \in t \vee \sigma = t \vee t \in \sigma\, )\, )
				\end{align}
				とおけば,
				\begin{align}
					\tau \in \chi,\ \sigma \in \chi,\ \chi \in a,\ \ord{a} 
					&\vdash \sigma \in a, 
					\label{fom:thm_transitive_totally_ordered_class_2} \\
					\tau \in \chi,\ \sigma \in \chi,\ \chi \in a,\ \ord{a} 
					&\vdash \tau \in a
					\label{fom:thm_transitive_totally_ordered_class_3}
				\end{align}
				となる.他方で$\ord{a}$の定義式より
				\begin{align}
					\ord{a} \vdash 
					\forall s\, (\, s \in a \rarrow 
					\forall t\, (\, t \in a \rarrow (\, \sigma \in t \vee \sigma = t \vee t \in \sigma\, )\, )\, )
					\label{fom:thm_transitive_totally_ordered_class_4}
				\end{align}
				が成り立つので,全称記号の論理的公理および
				(\refeq{fom:thm_transitive_totally_ordered_class_2})と
				(\refeq{fom:thm_transitive_totally_ordered_class_3})との三段論法により
				\begin{align}
					\tau \in \chi,\ \sigma \in \chi,\ \chi \in a,\ \ord{a} \vdash 
					\sigma \in \tau \vee \sigma = \tau \vee \tau \in \sigma
				\end{align}
				が従う.演繹定理より
				\begin{align}
					\sigma \in \chi,\ \chi \in a,\ \ord{a} \vdash 
					\tau \in \chi \rarrow (\, \sigma \in \tau \vee \sigma = \tau \vee \tau \in \sigma\, )
				\end{align}
				が成り立つので,全称の導出
				(論理的定理\ref{logicalthm:derivation_of_universal_by_epsilon})より
				\begin{align}
					\sigma \in \chi,\ \chi \in a,\ \ord{a} \vdash 
					\forall t\, (\, t \in \chi \rarrow (\, \sigma \in t \vee \sigma = t \vee t \in \sigma\, )
				\end{align}
				となり,再び演繹定理と全称の導出によって
				\begin{align}
					\chi \in a,\ \ord{a} \vdash 
					\forall s\, (\, s \in \chi \rarrow \forall t\, (\, t \in \chi \rarrow (\, s \in t \vee s = t \vee t \in s\, )\, )
				\end{align}
				が得られる.
				
			\item[step2] 次に
				\begin{align}
					\chi \in a,\ \ord{a},\ 
					\EXTAX,\EQAX,\COMAX,\PAIAX,\UNIAX,\REGAX \vdash \tran{\chi}
				\end{align}
				を示す.いま
				\begin{align}
					\eta &\defeq \varepsilon y \negation (\, y \in \chi \rarrow y \subset \chi\, ), \\
					\zeta &\defeq \varepsilon z \negation (\, z \in \eta \rarrow z \in \chi\, )
				\end{align}
				とおく.
				\begin{description}
					\item[step2-1]
						この段では
						\begin{align}
							\zeta \in \eta,\ \eta \in \chi,\ \chi \in a,\ \ord{a} 
							\vdash 
							\zeta \in \chi \vee \zeta = \chi \vee \chi \in \zeta
						\end{align}
						を示す.$a$の推移性より
						\begin{align}
							\chi \in a,\ \ord{a} \vdash \chi \subset a
						\end{align}
						が成り立つので,
						\begin{align}
							\eta \in \chi,\ \chi \in a,\ \ord{a} \vdash \eta \in a
						\end{align}
						が成り立ち,再び$a$の推移性より
						\begin{align}
							\eta \in \chi,\ \chi \in a,\ \ord{a} \vdash 
							\eta \subset a
						\end{align}
						となる.従って
						\begin{align}
							\zeta \in \eta,\ \eta \in \chi,\ \chi \in a,\ \ord{a} 
							\vdash \zeta \in a
						\end{align}
						も成り立ち,
						(\refeq{fom:thm_transitive_totally_ordered_class_4})と
						全称記号の論理的公理より
						\begin{align}
							\zeta \in \eta,\ \eta \in \chi,\ \chi \in a,\ \ord{a} 
							\vdash 
							\zeta \in \chi \vee \zeta = \chi \vee \chi \in \zeta
							\label{fom:thm_transitive_totally_ordered_class_5}
						\end{align}
						が得られる.
						
					\item[step2-2] この段では
						\begin{align}
							\zeta \in \eta,\ \eta \in \chi,\ 
							\EXTAX,\EQAX,\COMAX,\PAIAX,\REGAX 
							\vdash \zeta \neq \chi
						\end{align}
						を示す.定理\ref{thm:no_pair_of_sets_go_round}
						(所属関係で堂々巡りしない)より
						\begin{align}
							\EXTAX,\EQAX,\COMAX,\PAIAX,\REGAX \vdash 
							\zeta \in \eta \rarrow \eta \notin \zeta
						\end{align}
						が成り立つので,演繹定理の逆より
						\begin{align}
							\zeta \in \eta,\ \EXTAX,\EQAX,\COMAX,\PAIAX,\REGAX 
							\vdash \eta \notin \zeta
						\end{align}
						となる.他方で
						\begin{align}
							\EQAX \vdash \eta \notin \zeta 
							\rarrow \zeta \neq \chi \vee \eta \notin \chi
						\end{align}
						が成り立つので($\zeta = \chi \wedge \eta \in \chi 
						\rarrow \eta \in \zeta$の対偶を取ってDe Morganの法則)
						\begin{align}
							\zeta \in \eta,\ \EXTAX,\EQAX,\COMAX,\PAIAX,\REGAX 
							\vdash \zeta \neq \chi \vee \eta \notin \chi
						\end{align}
						が従い,論理的定理\ref{logicalthm:disjunction_of_negation_rewritable_by_implication}
						(含意の論理和は否定で書ける)より
						\begin{align}
							\zeta \in \eta,\ \EXTAX,\EQAX,\COMAX,\PAIAX,\REGAX 
							\vdash \eta \in \chi
							\rarrow \zeta \neq \chi
						\end{align}
						となる.そして演繹定理の逆より
						\begin{align}
							\zeta \in \eta,\ \eta \in \chi,\ 
							\EXTAX,\EQAX,\COMAX,\PAIAX,\REGAX 
							\vdash \zeta \neq \chi
							\label{fom:thm_transitive_totally_ordered_class_6}
						\end{align}
						が得られる.
						
					\item[step2-3] この段では
						\begin{align}
							\zeta \in \eta,\ \eta \in \chi,\ \chi \in a,\ \ord{a},\ 
							\EXTAX,\EQAX,\COMAX,\PAIAX,\UNIAX,\REGAX \vdash 
							\zeta \in \chi
						\end{align}
						を示す.定理\ref{thm:no_three_sets_go_round}
						(所属関係で堂々巡りしない)より
						\begin{align}
							\EXTAX,\EQAX,\COMAX,\PAIAX,\UNIAX,\REGAX \vdash 
							\zeta \in \eta \wedge \eta \in \chi \rarrow 
							\chi \notin \zeta
						\end{align}
						が成り立つので,
						\begin{align}
							\zeta \in \eta,\ \eta \in \chi,\ 
							\EXTAX,\EQAX,\COMAX,\PAIAX,\UNIAX,\REGAX \vdash 
							\chi \notin \zeta
						\end{align}
						が従う.ここで
						(\refeq{fom:thm_transitive_totally_ordered_class_6})
						と論理積の導入より
						\begin{align}
							\zeta \in \eta,\ \eta \in \chi,\ 
							\EXTAX,\EQAX,\COMAX,\PAIAX,\UNIAX,\REGAX \vdash 
							\zeta \neq \chi \wedge \chi \notin \zeta
						\end{align}
						となり,De Morganの法則
						(論理的定理\ref{logicalthm:weak_De_Morgan_law_1})より
						\begin{align}
							\zeta \in \eta,\ \eta \in \chi,\ 
							\EXTAX,\EQAX,\COMAX,\PAIAX,\UNIAX,\REGAX \vdash\ 
							\negation (\, \zeta = \chi \vee \chi \in \zeta\, )
							\label{fom:thm_transitive_totally_ordered_class_7}
						\end{align}
						が成り立つ.ところで
						(\refeq{fom:thm_transitive_totally_ordered_class_5})と
						論理和の結合律
						(論理的定理\ref{logicalthm:associative_law_of_disjunctions})
						より
						\begin{align}
							\zeta \in \eta,\ \eta \in \chi,\ \chi \in a,\ \ord{a} 
							\vdash 
							\zeta \in \chi \vee (\, \zeta = \chi \vee \chi \in \zeta\, )
						\end{align}
						が成り立つので,
						(\refeq{fom:thm_transitive_totally_ordered_class_7})と
						選言三段論法
						(論理的定理\ref{logicalthm:disjunctive_syllogism})より
						\begin{align}
							\zeta \in \eta,\ \eta \in \chi,\ \chi \in a,\ \ord{a},\ 
							\EXTAX,\EQAX,\COMAX,\PAIAX,\UNIAX,\REGAX \vdash 
							\zeta \in \chi
							\label{fom:thm_transitive_totally_ordered_class_8}
						\end{align}
						が出る.
				\end{description}
				(\refeq{fom:thm_transitive_totally_ordered_class_8})と演繹定理より
				\begin{align}
					\eta \in \chi,\ \chi \in a,\ \ord{a},\ 
					\EXTAX,\EQAX,\COMAX,\PAIAX,\UNIAX,\REGAX \vdash 
					\zeta \in \eta \rarrow \zeta \in \chi
				\end{align}
				が成り立つので,全称の導出
				(論理的定理\ref{logicalthm:derivation_of_universal_by_epsilon})より
				\begin{align}
					\eta \in \chi,\ \chi \in a,\ \ord{a},\ 
					\EXTAX,\EQAX,\COMAX,\PAIAX,\UNIAX,\REGAX \vdash 
					\eta \subset \chi
				\end{align}
				が得られ,再び演繹定理と全称の導出により
				\begin{align}
					\chi \in a,\ \ord{a},\ 
					\EXTAX,\EQAX,\COMAX,\PAIAX,\UNIAX,\REGAX \vdash 
					\forall y\, (\, y \in \chi \rarrow y \subset \chi\, )
				\end{align}
				が得られる.すなわち
				\begin{align}
					\chi \in a,\ \ord{a},\ 
					\EXTAX,\EQAX,\COMAX,\PAIAX,\UNIAX,\REGAX \vdash \tran{\chi}
				\end{align}
				が成立する.
				
			\item[step3] step1 と step2 の結果を併せれば
				\begin{align}
					\chi \in a,\ \ord{a},\ 
					\EXTAX,\EQAX,\COMAX,\PAIAX,\UNIAX,\REGAX \vdash \ord{\chi}
				\end{align}
				が成り立つので,演繹定理より
				\begin{align}
					\ord{a},\ \EXTAX,\EQAX,\COMAX,\PAIAX,\UNIAX,\REGAX \vdash 
					\chi \in a \rarrow \chi \in \ON
				\end{align}
				となり,全称の導出
				(論理的定理\ref{logicalthm:derivation_of_universal_by_epsilon})より
				\begin{align}
					\ord{a},\ \EXTAX,\EQAX,\COMAX,\PAIAX,\UNIAX,\REGAX \vdash 
					a \subset \ON
				\end{align}
				が出る.
				\QED
		\end{description}
	\end{sketch}
	
	\begin{screen}
		\begin{thm}[$\ON$は推移的]\label{thm:On_is_transitive}
			\begin{align}
				\EXTAX,\EQAX,\COMAX,\PAIAX,\UNIAX,\REGAX \vdash \tran{\ON}.
			\end{align}
		\end{thm}
	\end{screen}
	
	\begin{prf}
		いま
		\begin{align}
			\chi \defeq \varepsilon x \negation 
			(\, x \in \ON \rarrow x \subset \ON\, )
		\end{align}
		とおけば,定理\ref{thm:transitive_totally_ordered_class}より
		\begin{align}
			\EXTAX,\EQAX,\COMAX,\PAIAX,\UNIAX,\REGAX \vdash 
			\chi \in \ON \rarrow \chi \subset \ON
		\end{align}
		が成り立つので,全称の導出
		(論理的定理\ref{logicalthm:derivation_of_universal_by_epsilon})より
		\begin{align}
			\EXTAX,\EQAX,\COMAX,\PAIAX,\UNIAX,\REGAX \vdash 
			\forall x\, (\, x \in \ON \rarrow x \subset \ON\, )
		\end{align}
		が従う.
		\QED
	\end{prf}
	
	\begin{screen}
		\begin{dfn}[クラスの差]
			$x$と$y$を$\mathcal{L}$の項とするとき,
			\begin{align}
				x \backslash y \defeq \Set{z}{z \in x \wedge z \notin y}
			\end{align}
			と定める.この$x \backslash y$を$x$と$y$の{\bf 差}\index{さ@差}{\bf (difference)}と呼び,
			$x$と$y$が集合であれば$x \backslash y$を{\bf 差集合}\index{さしゅうごう@差集合}{\bf (set difference)}と呼ぶ.
		\end{dfn}
	\end{screen}
	
	$a$と$b$をクラスとするとき,任意の主要$\varepsilon$項$\tau$に対して
	\begin{align}
		\EXTAX,\EQAX,\COMAX,\ELEAX \vdash \tau \in b \backslash a \lrarrow \tau \in b \wedge \tau \notin a
	\end{align}
	が成り立つので(注意\ref{rem:epsilon_terms_of_not_L_epsilon_formula}),
	\begin{align}
		\EXTAX,\EQAX,\COMAX,\ELEAX \vdash b \backslash a \subset b
	\end{align}
	は常に満たされる.
	
	\begin{screen}
		\begin{thm}[差集合は集合]\label{thm:set_difference_is_set}
			$a$と$b$を主要$\varepsilon$項とするとき
			\begin{align}
				\EXTAX,\EQAX,\COMAX,\REPAX \vdash \set{b \backslash a}.
			\end{align}
		\end{thm}
	\end{screen}
	
	\begin{sketch}
		分出定理(定理\ref{thm:axiom_of_separation})より
		\begin{align}
			\EXTAX,\EQAX,\REPAX \vdash \exists s\, \forall x\, (\, x \in s \lrarrow x \in b \wedge (\, x \in b \wedge x \notin a\, )\, )
		\end{align}
		が成り立つ.ここで
		\begin{align}
			\sigma &\defeq \varepsilon s\, \forall x\, (\, x \in s \lrarrow x \in b \wedge (\, x \in b \wedge x \notin a\, )\, ), \\
			\tau &\defeq \varepsilon x\, (\, x \in \sigma \lrarrow x \in b \backslash a\, )
		\end{align}
		とおけば,
		\begin{align}
			\EXTAX,\EQAX,\REPAX \vdash \tau \in \sigma \lrarrow \tau \in b \wedge (\, \tau \in b \wedge \tau \notin a\, )
		\end{align}
		が成り立つ.ところで
		\begin{align}
			\vdash \tau \in b \wedge (\, \tau \in b \wedge \tau \notin a\, ) \lrarrow \tau \in b \wedge \tau \notin a
		\end{align}
		であるから,同値関係の推移律(論理的定理\ref{logicalthm:transitive_law_of_equivalence_symbol})より
		\begin{align}
			\EXTAX,\EQAX,\REPAX \vdash \tau \in \sigma \lrarrow \tau \in b \wedge \tau \notin a
		\end{align}
		が従う.また
		\begin{align}
			\COMAX \vdash \tau \in b \wedge \tau \notin a \lrarrow \tau \in b \backslash a
		\end{align}
		も成り立つので,再び同値関係の推移律によって
		\begin{align}
			\EXTAX,\EQAX,\COMAX,\REPAX \vdash \tau \in \sigma \lrarrow \tau \in b \backslash a
		\end{align}
		となり,全称の導出(論理的定理\ref{logicalthm:derivation_of_universal_by_epsilon})より
		\begin{align}
			\EXTAX,\EQAX,\COMAX,\REPAX \vdash \forall x\, (\, x \in \sigma \lrarrow x \in b \backslash a\, )
		\end{align}
		が従い,外延性公理と相等性公理(等号の対称性)より
		\begin{align}
			\EXTAX,\EQAX,\COMAX,\REPAX \vdash b \backslash a = \sigma
		\end{align}
		が出る.従って存在記号の論理的公理より
		\begin{align}
			\EXTAX,\EQAX,\COMAX,\REPAX \vdash \exists s\, (\, b \backslash a = s\, )
		\end{align}
		が得られる.
		\QED
	\end{sketch}
	
	\begin{screen}
		\begin{thm}[$\ON$において$\in$と$<$は同義]
		\label{thm:element_and_proper_subset_correspond}
			\begin{align}
				\EXTAX,\EQAX,\COMAX,\ELEAX,\REPAX,\PAIAX,\REGAX 
				\vdash \forall \alpha,\beta \in \ON\, (\, \alpha \in \beta \lrarrow \alpha < \beta\, ).
			\end{align}
		\end{thm}
	\end{screen}
	
	\begin{prf}
		いま
		\begin{align}
			a &\defeq \varepsilon \alpha \negation 
			(\, \alpha \in \ON \rarrow \forall \beta\, (\, \beta \in \ON \rarrow 
			(\, \alpha \in \beta \lrarrow \alpha < \beta\, )\,) \,), \\
			b &\defeq \varepsilon \beta \negation (\, \beta \in \ON \rarrow 
			(\, a \in \beta \lrarrow a < \beta\, )\,)
		\end{align}
		とおく.
		\begin{description}
			\item[step1] まず
				\begin{align}
					\ord{b},\ \EXTAX,\EQAX,\COMAX,\ELEAX,\PAIAX,\REGAX \vdash 
					a \in b \rarrow a < b
				\end{align}
				を示す.定理\ref{thm:critical_epsilon_term_is_set} (主要$\varepsilon$項は集合)と
				定理\ref{thm:ordered_pair_of_sets_is_a_set} (集合の順序対は集合)より
				\begin{align}
					\EXTAX,\EQAX,\COMAX,\PAIAX \vdash \set{(a,b)}
				\end{align}
				が成り立つので,
				\begin{align}
					\tau \defeq \varepsilon x\, (\, (a,b) = x\, )
				\end{align}
				とおけば
				\begin{align}
					\EXTAX,\EQAX,\COMAX,\PAIAX \vdash (a,b) = \tau
					\label{fom:element_and_proper_subset_correspond_1}
				\end{align}
				となる.ところで順序数の推移性より
				\begin{align}
					a \in b,\ \ord{b} \vdash a \subset b
				\end{align}
				が成り立つから,
				\begin{align}
					a \in b,\ \ord{b},\ \EXTAX,\EQAX,\COMAX,\PAIAX 
					\vdash \tau = (a,b) \wedge a \subset b
				\end{align}
				となり,存在記号の論理的公理より
				\begin{align}
					a \in b,\ \ord{b},\ \EXTAX,\EQAX,\COMAX,\PAIAX 
					\vdash \exists \alpha\, \exists \beta\, (\, \tau = (\alpha,\beta) \wedge \alpha \subset \beta\, )
				\end{align}
				が従う.よって注意\ref{rem:epsilon_terms_of_not_L_epsilon_formula}より
				\begin{align}
					a \in b,\ \ord{b},\ \EXTAX,\EQAX,\COMAX,\ELEAX,\PAIAX \vdash \tau \in\ \leq
					\label{fom:element_and_proper_subset_correspond_2}
				\end{align}
				となり,
				%\begin{align}
				%	a \in b,\ \ord{b},\ \EQAX,\COMAX 
				%	\vdash a \leq b
				%\end{align}
				%ところで定理\ref{thm:transitive_totally_ordered_class}より
				%\begin{align}
				%	\ord{b},\ \EXTAX,\EQAX,\COMAX,\PAIAX,\UNIAX,\REGAX \vdash 
				%	b \subset \ON
				%\end{align}
				%が成り立つので,
				%\begin{align}
				%	a \in b,\ \ord{b},\ \EXTAX,\EQAX,\COMAX,\PAIAX,\UNIAX,\REGAX 
				%	\vdash a \in \ON
				%\end{align}
				%となり,(\refeq{fom:element_and_proper_subset_correspond_2})と
				%論理積の導入より
				%\begin{align}
				%	&a \in b,\ \ord{b},\ \EXTAX,\EQAX,\COMAX,\PAIAX,\UNIAX,\REGAX \\
				%	&\vdash a \in \ON \wedge \exists \beta\, (\, \beta \in \ON \wedge 
				%	(\, \tau = (a,\beta) \wedge a \subset \beta\, )\, )
				%\end{align}
				%が成り立ち,存在記号の論理的公理より
				%\begin{align}
				%	&a \in b,\ \ord{b},\ \EXTAX,\EQAX,\COMAX,\PAIAX,\UNIAX,\REGAX \\
				%	&\vdash \exists \alpha\, (\, \alpha \in \ON \wedge 
				%	\exists \beta\, (\, \beta \in \ON \wedge 
				%	(\, \tau = (\alpha,\beta) \wedge \alpha \subset \beta\, )\, )\, )
				%\end{align}
				%が成り立つ.ゆえに
				%\begin{align}
				%	a \in b,\ \ord{b},\ \EXTAX,\EQAX,\COMAX,\PAIAX,\UNIAX,\REGAX 
				%	\vdash \tau \in\ \leq
				%\end{align}
				%となり,
				(\refeq{fom:element_and_proper_subset_correspond_1})と相等性公理より
				\begin{align}
					a \in b,\ \ord{b},\ \EXTAX,\EQAX,\COMAX,\ELEAX,\PAIAX 
					\vdash a \leq b
					\label{fom:element_and_proper_subset_correspond_3}
				\end{align}
				が従う.他方で
				\begin{align}
					a \in b,\ \EQAX \vdash a \notin a \rarrow a \neq b
				\end{align}
				が成り立つので,定理\ref{thm:no_class_contains_itself}
				(自分自身は要素ではない)と併せて
				\begin{align}
					a \in b,\ \EXTAX,\EQAX,\COMAX,\PAIAX,\REGAX \vdash 
					a \neq b
					\label{fom:element_and_proper_subset_correspond_4}
				\end{align}
				が従う.(\refeq{fom:element_and_proper_subset_correspond_3})と
				(\refeq{fom:element_and_proper_subset_correspond_4})と論理積の導入より
				\begin{align}
					a \in b,\ \ord{b},\ \EXTAX,\EQAX,\COMAX,\ELEAX,\PAIAX,\REGAX 
					\vdash a \leq b \wedge a \neq b
				\end{align}
				となるので,
				\begin{align}
					a \in b,\ \ord{b},\ \EXTAX,\EQAX,\COMAX,\ELEAX,\PAIAX,\REGAX \vdash a < b
					\label{fom:element_and_proper_subset_correspond_21}
				\end{align}
				が得られる.
				
			\item[step2]
				外延性公理と対偶律1 
				(論理的定理\ref{logicalthm:introduction_of_contraposition})より
				\begin{align}
					a \neq b,\ \EXTAX \vdash\ \negation\forall x\, (\, x \in a \lrarrow x \in b\, )
				\end{align}
				となり,量化子の論理的公理より
				\begin{align}
					a \neq b,\ \EXTAX \vdash \exists x \negation (\, x \in a \lrarrow x \in b\, )
				\end{align}
				が成り立つので,
				\begin{align}
					\tau \defeq \varepsilon x \negation (\, x \in a \lrarrow x \in b\, )
				\end{align}
				とおけば,存在記号の論理的公理より
				\begin{align}
					a \neq b,\ \EXTAX \vdash\ \negation (\, \tau \in a \lrarrow \tau \in b\, )
				\end{align}
				が成り立ち,De Morganの法則(論理的定理\ref{logicalthm:strong_De_Morgan_law_2})より
				\begin{align}
					a \neq b,\ \EXTAX \vdash\ \negation (\, \tau \in a \rarrow \tau \in b\, ) \vee
					\negation (\, \tau \in b \rarrow \tau \in a\, )
				\end{align}
				が従う.よって論理的定理\ref{logicalthm:disjunction_of_negation_rewritable_by_implication}
				(否定の論理和は含意で書ける)より
				\begin{align}
					a \neq b,\ \EXTAX \vdash (\, \tau \in a \rarrow \tau \in b\, )
					\rarrow\ \negation (\, \tau \in b \rarrow \tau \in a\, )
					\label{fom:element_and_proper_subset_correspond_22}
				\end{align}
				が成り立つ.他方で
				\begin{align}
					a < b \vdash a \neq b
				\end{align}
				であるから,(\refeq{fom:element_and_proper_subset_correspond_22})より
				\begin{align}
					a < b,\ \EXTAX \vdash (\, \tau \in a \rarrow \tau \in b\, )
					\rarrow\ \negation (\, \tau \in b \rarrow \tau \in a\, )
					\label{fom:element_and_proper_subset_correspond_6}
				\end{align}
				が成り立ち,また
				\begin{align}
					a < b \vdash \tau \in a \rarrow \tau \in b
				\end{align}
				も成り立つので,(\refeq{fom:element_and_proper_subset_correspond_6})との三段論法より
				\begin{align}
					a < b,\ \EXTAX \vdash\ \negation (\, \tau \in b \rarrow \tau \in a\, )
					\label{fom:element_and_proper_subset_correspond_7}
				\end{align}
				となる.ところで論理的定理\ref{logicalthm:disjunction_of_negation_rewritable_by_implication}
				(否定の論理和は含意で書ける)と対偶律1 (論理的定理\ref{logicalthm:introduction_of_contraposition})より
				\begin{align}
					\vdash\ \negation (\, \tau \in b \rarrow \tau \in a\, ) \rarrow\ 
					\negation (\, \tau \notin b \vee \tau \in a\, )
				\end{align}
				が成り立つので,(\refeq{fom:element_and_proper_subset_correspond_7})との三段論法より
				\begin{align}
					a < b,\ \EXTAX \vdash\ \negation (\, \tau \notin b \vee \tau \in a\, )
				\end{align}
				が従い,De Morganの法則(論理的定理\ref{logicalthm:weak_De_Morgan_law_2})と二重否定の除去より
				\begin{align}
					a < b,\ \EXTAX \vdash \tau \in b \wedge \tau \notin a
				\end{align}
				が従う.そして
				\begin{align}
					a < b,\ \EXTAX,\COMAX \vdash \tau \in b \backslash a
					\label{fom:element_and_proper_subset_correspond_5}
				\end{align}
				が得られる.
				
			\item[step3] 定理\ref{thm:set_difference_is_set}より
				\begin{align}
					\EXTAX,\EQAX,\COMAX,\REPAX \vdash \set{b \backslash a}
				\end{align}
				が成り立つので,
				\begin{align}
					\sigma \defeq \varepsilon s\, (\, b \backslash a = s\, )
				\end{align}
				とおけば
				\begin{align}
					\EXTAX,\EQAX,\COMAX,\REPAX \vdash b \backslash a = \sigma
					\label{fom:element_and_proper_subset_correspond_8}
				\end{align}
				となる.(\refeq{fom:element_and_proper_subset_correspond_5})と併せて
				\begin{align}
					a < b,\ \EXTAX,\EQAX,\COMAX,\REPAX \vdash \tau \in \sigma
				\end{align}
				となり,存在記号の論理的公理より
				\begin{align}
					a < b,\ \EXTAX,\EQAX,\COMAX,\REPAX \vdash \exists x\, (\, x \in \sigma\, )
				\end{align}
				となるが,正則性公理より
				\begin{align}
					\REGAX \vdash \exists x\, (\, x \in \sigma\, ) 
					\rarrow \exists y\, (\, y \in \sigma \wedge \forall z\, (\, z \in \sigma \rarrow z \notin y\, )\, )
				\end{align}
				が成り立つので
				\begin{align}
					a < b,\ \EXTAX,\EQAX,\COMAX,\REPAX,\REGAX \vdash 
					\exists y\, (\, y \in \sigma \wedge \forall z\, (\, z \in \sigma \rarrow z \notin y\, )\, )
				\end{align}
				が従う.ここで
				\begin{align}
					\eta \defeq \varepsilon y\, (\, y \in \sigma \wedge 
					\forall z\, (\, z \in \sigma \rarrow z \notin y\, )\, )
				\end{align}
				とおけば
				\begin{align}
					a < b,\ \EXTAX,\EQAX,\COMAX,\REPAX,\REGAX &\vdash \eta \in \sigma, 
					\label{fom:element_and_proper_subset_correspond_9} \\
					a < b,\ \EXTAX,\EQAX,\COMAX,\REPAX,\REGAX &\vdash 
					\forall z\, (\, z \in \sigma \rarrow z \notin \eta\, )
					\label{fom:element_and_proper_subset_correspond_10}
				\end{align}
				が成り立つ.正則性公理の式の意味を考えれば,ここで取られた$\eta$は
				「$b \backslash a$の要素であり,$b \backslash a$とは交わらない」
				という性質を持っている.$a$と$b$が順序数であれば$\eta$は$a$に等しくなる
				ことが示されるが,それは次段以降で解説する.
				
			\item[step4] この段では
				\begin{align}
					a < b,\ \ord{a},\ \ord{b},\ \EXTAX,\EQAX,\COMAX,\REPAX,\REGAX \vdash a \subset \eta 
				\end{align}
				を示す.いま
				\begin{align}
					\chi \defeq \varepsilon x \negation (\, x \in a \rarrow x \in \eta\, )
				\end{align}
				とおく.(\refeq{fom:element_and_proper_subset_correspond_8})と
				(\refeq{fom:element_and_proper_subset_correspond_9})より
				\begin{align}
					a < b,\ \EXTAX,\EQAX,\COMAX,\REPAX,\REGAX &\vdash \eta \in b, 
					\label{fom:element_and_proper_subset_correspond_11} \\
					a < b,\ \EXTAX,\EQAX,\COMAX,\REPAX,\REGAX &\vdash \eta \notin a
					\label{fom:element_and_proper_subset_correspond_12}
				\end{align}
				となり,他方で
				\begin{align}
					\chi \in a,\ a < b \vdash \chi \in b
				\end{align}
				となるから,$\ord{b}$を公理に追加すれば
				\begin{align}
					\chi \in a,\ a < b,\ \ord{b},\ \EXTAX,\EQAX,\COMAX,\REPAX,\REGAX 
					\vdash \chi \in \eta \vee \chi = \eta \vee \eta \in \chi
					\label{fom:element_and_proper_subset_correspond_13}
				\end{align}
				が成り立つ.ところで
				\begin{align}
					\EQAX \vdash \chi = \eta \wedge \chi \in a \rarrow \eta \in a
				\end{align}
				が成り立つので,対偶律1 (論理的定理\ref{logicalthm:introduction_of_contraposition})より
				\begin{align}
					\EQAX \vdash \eta \notin a \rarrow \chi \neq \eta \vee \chi \notin a
				\end{align}
				となる.また順序数の推移性より
				\begin{align}
					\ord{a} \vdash \eta \in \chi \wedge \chi \in a \rarrow \eta \in a
				\end{align}
				が成り立つので,対偶律1 (論理的定理\ref{logicalthm:introduction_of_contraposition})より
				\begin{align}
					\ord{a} \vdash \eta \notin a \rarrow \eta \notin \chi \vee \chi \notin a
				\end{align}
				となる.よって,(\refeq{fom:element_and_proper_subset_correspond_12})と併せて
				\begin{align}
					a < b,\ \EXTAX,\EQAX,\COMAX,\REPAX,\REGAX &\vdash \chi \neq \eta \vee \chi \notin a, \\
					a < b,\ \ord{a},\ \EXTAX,\EQAX,\COMAX,\REPAX,\REGAX &\vdash \eta \notin \chi \vee \chi \notin a
				\end{align}
				が成り立ち,論理的定理\ref{logicalthm:disjunction_of_negation_rewritable_by_implication}
				(否定の論理和は含意で書ける)より
				\begin{align}
					a < b,\ \EXTAX,\EQAX,\COMAX,\REPAX,\REGAX 
					&\vdash \chi \in a \rarrow \chi \neq \eta, \\
					a < b,\ \ord{a},\ \EXTAX,\EQAX,\COMAX,\REPAX,\REGAX 
					&\vdash \chi \in a \rarrow \eta \notin \chi
				\end{align}
				が成り立ち,演繹定理の逆により
				\begin{align}
					\chi \in a,\ a < b,\ \EXTAX,\EQAX,\COMAX,\REPAX,\REGAX &\vdash \chi \neq \eta,
					\label{fom:element_and_proper_subset_correspond_14} \\
					\chi \in a,\ a < b,\ \ord{a},\ \EXTAX,\EQAX,\COMAX,\REPAX,\REGAX &\vdash \eta \notin \chi
					\label{fom:element_and_proper_subset_correspond_15}
				\end{align}
				となり,論理積の導入とDe Morganの法則(論理的定理\ref{logicalthm:weak_De_Morgan_law_1})より
				\begin{align}
					\chi \in a,\ a < b,\ \ord{a},\ \EXTAX,\EQAX,\COMAX,\REPAX,\REGAX \vdash\ 
					\negation (\, \chi = \eta \vee \eta \in \chi\, )
				\end{align}
				が従う.これと(\refeq{fom:element_and_proper_subset_correspond_13})と
				選言三段論法(論理的定理\ref{logicalthm:disjunctive_syllogism})より
				\begin{align}
					\chi \in a,\ a < b,\ \ord{a},\ \ord{b},\ \EXTAX,\EQAX,\COMAX,\REPAX,\REGAX \vdash \chi \in \eta
				\end{align}
				が成立するので,演繹定理より
				\begin{align}
					a < b,\ \ord{a},\ \ord{b},\ \EXTAX,\EQAX,\COMAX,\REPAX,\REGAX \vdash 
					\chi \in a \rarrow \chi \in \eta
				\end{align}
				となり,全称の導出(論理的定理\ref{logicalthm:derivation_of_universal_by_epsilon})より
				\begin{align}
					a < b,\ \ord{a},\ \ord{b},\ \EXTAX,\EQAX,\COMAX,\REPAX,\REGAX \vdash a \subset \eta
					\label{fom:element_and_proper_subset_correspond_16}
				\end{align}
				が出る.
			
			\item[step5] この段では
				\begin{align}
					a < b,\ \ord{b},\ \EXTAX,\EQAX,\COMAX,\REPAX,\REGAX \vdash \eta \subset a
				\end{align}
				を示す.いま
				\begin{align}
					\chi \defeq \varepsilon x \negation (\, x \in \eta \rarrow x \in a\, )
				\end{align}
				とおく.(\refeq{fom:element_and_proper_subset_correspond_11})と順序数の推移性より
				\begin{align}
					\chi \in \eta,\ a < b,\ \ord{b},\ \EXTAX,\EQAX,\COMAX,\REPAX,\REGAX \vdash \chi \in b
					\label{fom:element_and_proper_subset_correspond_17}
				\end{align}
				となる.他方で(\refeq{fom:element_and_proper_subset_correspond_10})と
				対偶律2 (論理的定理\ref{logicalthm:contraposition_2})より
				\begin{align}
					a < b,\ \EXTAX,\EQAX,\COMAX,\REPAX,\REGAX \vdash 
					\chi \in \eta \rarrow \chi \notin \sigma
				\end{align}
				となり,演繹定理の逆より
				\begin{align}
					\chi \in \eta,\ a < b,\ \EXTAX,\EQAX,\COMAX,\REPAX,\REGAX \vdash \chi \notin \sigma
				\end{align}
				となるが,(\refeq{fom:element_and_proper_subset_correspond_8})より
				\begin{align}
					\chi \in \eta,\ a < b,\ \EXTAX,\EQAX,\COMAX,\REPAX,\REGAX \vdash \chi \notin b \backslash a
					\label{fom:element_and_proper_subset_correspond_18}
				\end{align}
				が従う.ところで
				\begin{align}
					\COMAX \vdash \chi \in b \backslash a \lrarrow \chi \in b \wedge \chi \notin a
				\end{align}
				が成り立つので,対偶を取って
				\begin{align}
					\COMAX \vdash \chi \notin b \backslash a \lrarrow \chi \notin b \vee \chi \in a
				\end{align}
				となるから,(\refeq{fom:element_and_proper_subset_correspond_18})より
				\begin{align}
					\chi \in \eta,\ a < b,\ \EXTAX,\EQAX,\COMAX,\REPAX,\REGAX \vdash \chi \notin b \vee \chi \in a
				\end{align}
				が従い,論理的定理\ref{logicalthm:disjunction_of_negation_rewritable_by_implication}
				(否定の論理和は含意で書ける)より
				\begin{align}
					\chi \in \eta,\ a < b,\ \EXTAX,\EQAX,\COMAX,\REPAX,\REGAX \vdash \chi \in b \rarrow \chi \in a
				\end{align}
				が成り立つ.これと(\refeq{fom:element_and_proper_subset_correspond_17})との三段論法より
				\begin{align}
					\chi \in \eta,\ a < b,\ \ord{b},\ \EXTAX,\EQAX,\COMAX,\REPAX,\REGAX \vdash \chi \in a
				\end{align}
				が成り立ち,演繹定理と全称の導出(論理的定理\ref{logicalthm:derivation_of_universal_by_epsilon})より
				\begin{align}
					a < b,\ \ord{b},\ \EXTAX,\EQAX,\COMAX,\REPAX,\REGAX \vdash \eta \subset a
					\label{fom:element_and_proper_subset_correspond_19}
				\end{align}
				が出る.
			
			\item[step6] (\refeq{fom:element_and_proper_subset_correspond_16})と
				(\refeq{fom:element_and_proper_subset_correspond_19})および
				定理\ref{thm:mutually_including_classes_are_equivalent} (互いに相手を包含するクラス同士は等しい)より
				\begin{align}
					a < b,\ \ord{a},\ \ord{b},\ \EXTAX,\EQAX,\COMAX,\REPAX,\REGAX \vdash \eta = a
				\end{align}
				が成り立ち,(\refeq{fom:element_and_proper_subset_correspond_11})と相等性公理より
				\begin{align}
					a < b,\ \ord{a},\ \ord{b},\ \EXTAX,\EQAX,\COMAX,\REPAX,\REGAX \vdash a \in b
					\label{fom:element_and_proper_subset_correspond_20}
				\end{align}
				が従う.(\refeq{fom:element_and_proper_subset_correspond_21})と
				(\refeq{fom:element_and_proper_subset_correspond_20})と演繹定理および論理積の導入より
				\begin{align}
					\ord{a},\ \EXTAX,\EQAX,\COMAX,\ELEAX,\REPAX,\PAIAX,\REGAX \vdash 
					\ord{b} \rarrow (\, a \in b \lrarrow a < b\, )
				\end{align}
				が成り立つが,ここで全称の導出(論理的定理\ref{logicalthm:derivation_of_universal_by_epsilon})より
				\begin{align}
					&\ord{a},\ \EXTAX,\EQAX,\COMAX,\ELEAX,\REPAX,\PAIAX,\REGAX \\
					&\vdash \forall \beta\, (\, \beta \in \ON \rarrow (\, a \in \beta \lrarrow a < \beta\, )\, )
				\end{align}
				が従い,同様にして
				\begin{align}
					&\EXTAX,\EQAX,\COMAX,\ELEAX,\REPAX,\PAIAX,\REGAX \\
					&\vdash \forall \alpha\, (\, \alpha \in \ON \rarrow
					\forall \beta\, (\, \beta \in \ON \rarrow (\, \alpha \in \beta \lrarrow \alpha < \beta\, )\, )\, )
				\end{align}
				が得られる.
				\QED
		\end{description}
	\end{prf}
	
	\begin{screen}
		\begin{thm}[$\leq$は$\ON$の全順序]
		\label{thm:ON_is_totally_ordered}
			\begin{align}
				\EXTAX,\EQAX,\COMAX,\ELEAX,\REPAX,\PAIAX,\REGAX \vdash 
				\forall \alpha,\beta \in \ON\,
				(\, \alpha \in \beta \vee \alpha = \beta \vee \beta \in \alpha\, ).
			\end{align}
		\end{thm}
	\end{screen}
	
	\begin{sketch}
		いま
		\begin{align}
			a &\defeq \varepsilon \alpha \negation 
			(\, \alpha \in \ON \rarrow \forall \beta\, (\, \beta \in \ON \rarrow 
			(\, \alpha \in \beta \vee \alpha = \beta \vee \beta \in \alpha\, )\,) \,), \\
			b &\defeq \varepsilon \beta \negation (\, \beta \in \ON \rarrow 
			(\, a \in \beta \vee a = \beta \vee \beta \in a\, )\,)
		\end{align}
		とおく.
		\begin{description}
			\item[step1] この段では
				\begin{align}
					\ord{a},\ \ord{b} \vdash \ord{a \cap b}
					\label{fom:ON_is_totally_ordered_1}
				\end{align}
				を示す.実際,
				\begin{align}
					\chi &\defeq \varepsilon x \negation 
					(\, x \in a \cap b \rarrow \forall y\, (\, 
					y \in a \cap b \rarrow (\, x \in y \vee x = y \vee y \in x\, )\, )\, ), \\
					\eta &\defeq \varepsilon y \negation (\, 
					y \in a \cap b \rarrow (\, \chi \in y \vee \chi = y \vee y \in \chi\, )\, )
				\end{align}
				とおけば,
				\begin{align}
					\chi \in a \cap b \vdash \chi \in a
				\end{align}
				および
				\begin{align}
					\eta \in a \cap b \vdash \eta \in a
				\end{align}
				および
				\begin{align}
					\ord{a} \vdash \chi \in a \rarrow (\, 
					\eta \in a \rarrow (\, \chi \in \eta \vee \chi = \eta \vee \eta \in \chi\, )\, )
				\end{align}
				より
				\begin{align}
					\eta \in a \cap b,\ \chi \in a \cap b,\ \ord{a} 
					\vdash \chi \in \eta \vee \chi = \eta \vee \eta \in \chi
				\end{align}
				が成り立つので,演繹定理と全称の導出
				(論理的定理\ref{logicalthm:derivation_of_universal_by_epsilon})より
				\begin{align}
					\ord{a} \vdash \forall x,y \in a \cap b\, 
					(\, x \in y \vee x = y \vee y \in x\, )
				\end{align}
				が得られる.今度は
				\begin{align}
					\chi \defeq \varepsilon x \negation 
					(\, x \in a \cap b \rarrow x \subset a \cap b\, )
				\end{align}
				とおき直せば,順序数の推移性より
				\begin{align}
					\chi \in a \cap b,\ \ord{a} &\vdash \chi \subset a, \\
					\chi \in a \cap b,\ \ord{b} &\vdash \chi \subset b
				\end{align}
				が成り立つので
				\begin{align}
					\chi \in a \cap b,\ \ord{a},\ \ord{b} \vdash \chi \subset a \cap b
				\end{align}
				となり,演繹定理と全称の導出
				(論理的定理\ref{logicalthm:derivation_of_universal_by_epsilon})より
				\begin{align}
					\ord{a},\ \ord{b} \vdash 
					\forall x\, (\, x \in a \cap b \rarrow x \subset a \cap b\, )
				\end{align}
				も得られる.
				
			\item[step2] 
				%\ELEAX \vdash a \cap b \in a \rarrow \exists x\, (\, a \cap b = x\, )
				%\tau \defeq \varepsilon x\, (\, a \cap b = x\, )
				%a \cap b \in a,\ \ELEAX \vdash a \cap b = \tau
				%a \cap b \in a,\ \ELEAX,\EQAX \vdash a \cap b = \tau \rarrow (\, a \cap b \in a \rarrow \tau \in a\, )
				%a \cap b \in a,\ \ELEAX,\EQAX \vdash \tau \in a
				%a \cap b \in b,\ \ELEAX,\EQAX \vdash \tau \in b
				%a \cap b \in a,\ a \cap b \in b,\ \ELEAX,\EQAX \vdash \tau \in a \wedge \tau \in b
				%a \cap b \in a,\ a \cap b \in b,\ \ELEAX,\EQAX,\COMAX \vdash \tau \in a \cap b
				%a \cap b \in a,\ a \cap b \in b,\ \ELEAX,\EQAX,\COMAX \vdash a \cap b \in a \cap b
				%\ELEAX,\EQAX,\COMAX \vdash a \cap b \in a \wedge a \cap b \in b \rarrow a \cap b \in a \cap b
				定理\ref{thm:no_class_contains_itself} (自分自身は要素に持たない)より
				\begin{align}
					\EXTAX,\EQAX,\COMAX,\PAIAX,\REGAX \vdash a \cap b \notin a \cap b
				\end{align}
				が成り立ち,他方で
				\begin{align}
					\EQAX,\COMAX,\ELEAX \vdash a \cap b \notin a \cap b
					\rarrow a \cap b \notin a \vee a \cap b \notin b
				\end{align}
				も成り立つので,三段論法より
				\begin{align}
					\EXTAX,\EQAX,\COMAX,\ELEAX,\PAIAX,\REGAX \vdash 
					a \cap b \notin a \vee a \cap b \notin b
					\label{fom:ON_is_totally_ordered_2}
				\end{align}
				が従う.ところで
				\begin{align}
					\COMAX \vdash a \cap b \subset a
				\end{align}
				と定理\ref{thm:element_and_proper_subset_correspond}
				および(\refeq{fom:ON_is_totally_ordered_1})より
				\begin{align}
					\ord{a},\ \ord{b},\ \EXTAX,\EQAX,\COMAX,\ELEAX,\REPAX,\PAIAX,\REGAX \vdash 
					a \cap b \in a \vee a \cap b = a
				\end{align}
				が成り立つので,選言三段論法
				(論理的定理\ref{logicalthm:disjunctive_syllogism})より
				\begin{align}
					\ord{a},\ \ord{b},\ \EXTAX,\EQAX,\COMAX,\ELEAX,\REPAX,\PAIAX,\REGAX \vdash 
					a \cap b \notin a \rarrow a \cap b = a
				\end{align}
				が成り立つ.演繹定理の逆より
				\begin{align}
					a \cap b \notin a,\ \ord{a},\ \ord{b},\ \EXTAX,\EQAX,\COMAX,\ELEAX,\REPAX,\PAIAX,\REGAX \vdash 
					a \cap b = a
				\end{align}
				となるが,ここで
				\begin{align}
					a \cap b = a,\ \COMAX \vdash a \subset b
				\end{align}
				が成り立つので
				\begin{align}
					a \cap b \notin a,\ \ord{a},\ \ord{b},\ \EXTAX,\EQAX,\COMAX,\ELEAX,\REPAX,\PAIAX,\REGAX \vdash 
					a \subset b
				\end{align}
				が従い,定理\ref{thm:element_and_proper_subset_correspond}より
				\begin{align}
					a \cap b \notin a,\ \ord{a},\ \ord{b},\ \EXTAX,\EQAX,\COMAX,\ELEAX,\REPAX,\PAIAX,\REGAX \vdash 
					a \in b \vee a = b
				\end{align}
				が成り立つ.論理和の導入より
				\begin{align}
					&a \cap b \notin a,\ \ord{a},\ \ord{b},\ \EXTAX,\EQAX,\COMAX,\ELEAX,\REPAX,\PAIAX,\REGAX \\
					&\vdash a \in b \vee a = b \vee b \in a
				\end{align}
				となり,演繹定理より
				\begin{align}
					&\ord{a},\ \ord{b},\ \EXTAX,\EQAX,\COMAX,\ELEAX,\REPAX,\PAIAX,\REGAX \\
					&\vdash a \cap b \notin a \rarrow a \in b \vee a = b \vee b \in a
				\end{align}
				が従う.同様にして
				\begin{align}
					&\ord{a},\ \ord{b},\ \EXTAX,\EQAX,\COMAX,\ELEAX,\REPAX,\PAIAX,\REGAX \\
					&\vdash a \cap b \notin b \rarrow a \in b \vee a = b \vee b \in a
				\end{align}
				も成り立つので,論理和の除去より
				\begin{align}
					&\ord{a},\ \ord{b},\ \EXTAX,\EQAX,\COMAX,\ELEAX,\REPAX,\PAIAX,\REGAX \\
					&\vdash 
					a \cap b \notin a \vee a \cap b \notin b
					\rarrow a \in b \vee a = b \vee b \in a
				\end{align}
				が成り立ち,(\refeq{fom:ON_is_totally_ordered_2})との三段論法より
				\begin{align}
					\ord{a},\ \ord{b},\ \EXTAX,\EQAX,\COMAX,\ELEAX,\REPAX,\PAIAX,\REGAX \vdash 
					\rarrow a \in b \vee a = b \vee b \in a
				\end{align}
				が従う.あとは演繹定理と全称の導出
				(論理的定理\ref{logicalthm:derivation_of_universal_by_epsilon})より
				\begin{align}
					\EXTAX,\EQAX,\COMAX,\ELEAX,\REPAX,\PAIAX,\REGAX \vdash 
					\forall \alpha,\beta \in \ON\, (\, \alpha \in \beta \vee \alpha = \beta \vee \beta \in \alpha\, )
				\end{align}
				が出る.
				\QED
		\end{description}
	\end{sketch}
	
	\begin{screen}
		\begin{thm}[Burali-Forti]\label{thm:Burali_Forti}
			順序数の全体は集合ではない.
			\begin{align}
				\EXTAX,\EQAX,\COMAX,\ELEAX,\REPAX,\PAIAX,\UNIAX,\REGAX
				\vdash\ \negation \set{\ON}.
			\end{align}
		\end{thm}
	\end{screen}
	
	\begin{prf}
		定理\ref{thm:satisfactory_set_is_an_element}より
		\begin{align}
			\EQAX,\COMAX \vdash \ord{\ON} \rarrow (\, \set{\ON} \rarrow \ON \in \ON\, )
			\label{eq:Burali_Forti_1}
		\end{align}
		が成り立つ.定理\ref{thm:On_is_transitive}と定理\ref{thm:ON_is_totally_ordered}より
		\begin{align}
			\EXTAX,\EQAX,\COMAX,\ELEAX,\REPAX,\PAIAX,\UNIAX,\REGAX
			\vdash \ord{\ON}
		\end{align}
		が成り立つから,(\refeq{eq:Burali_Forti_1})との三段論法より
		\begin{align}
			\EXTAX,\EQAX,\COMAX,\ELEAX,\REPAX,\PAIAX,\UNIAX,\REGAX
			\vdash \set{\ON} \rarrow \ON \in \ON
		\end{align}
		となり,対偶律1 (論理的定理\ref{logicalthm:introduction_of_contraposition})より
		\begin{align}
			\EXTAX,\EQAX,\COMAX,\ELEAX,\REPAX,\PAIAX,\UNIAX,\REGAX
			\vdash \ON \notin \ON \rarrow\ \negation \set{\ON}
			\label{eq:Burali_Forti_2}
		\end{align}
		が従う.他方で定理\ref{thm:no_class_contains_itself} (自分自身は要素に持たない)より
		\begin{align}
			\EXTAX,\EQAX,\COMAX,\ELEAX,\PAIAX,\REGAX \vdash \ON \notin \ON
		\end{align}
		も成り立つので,(\refeq{eq:Burali_Forti_2})との三段論法から
		\begin{align}
			\EXTAX,\EQAX,\COMAX,\ELEAX,\REPAX,\PAIAX,\UNIAX,\REGAX
			\vdash\ \negation \set{\ON}
		\end{align}
		が得られる.
		\QED
	\end{prf}
	%	\begin{screen}
		\begin{thm}[$\ON$の整列性]\label{thm:On_is_wellordered}
			$\leq$は$\ON$上の整列順序である.また次が成り立つ.
			\begin{align}
				\forall \alpha,\beta \in \ON\,
				\left(\, \alpha \in \beta \vee \alpha = \beta \vee \beta \in \alpha\, \right).
			\end{align}
		\end{thm}
	\end{screen}
	
	\begin{prf}\mbox{}
		\begin{description}
			\item[第一段]
				$\alpha,\beta,\gamma$を順序数とすれば
				\begin{align}
					\alpha \subset \alpha
				\end{align}
				かつ
				\begin{align}
					\alpha \subset \beta \wedge \beta \subset \alpha \Longrightarrow \alpha = \beta
				\end{align}
				かつ
				\begin{align}
					\alpha \subset \beta \wedge \beta \subset \gamma \Longrightarrow \alpha \subset \gamma
				\end{align}
				が成り立つ.ゆえに$\leq$は$\ON$上の順序である.
				
			\item[第二段]
				
			
			\item[第三段]
				$\leq$が整列順序であることを示す.$a$を$\ON$の空でない部分集合とする.このとき正則性公理より
				\begin{align}
					x \in a \wedge x \cap a = \emptyset
				\end{align}
				を満たす集合$x$が取れるが,この$x$が$a$の最小限である.実際,任意に$a$から要素$y$を取ると
				\begin{align}
					x \cap a = \emptyset
				\end{align}
				から
				\begin{align}
					y \notin x
				\end{align}
				が従い,また前段の結果より
				\begin{align}
					x \in y \vee x = y \vee y \in x
				\end{align}
				も成り立つので,選言三段論法より
				\begin{align}
					x \in y \vee x = y
					\label{eq:thm_On_is_wellordered_6}
				\end{align}
				が成り立つ.$y$は推移的であるから
				\begin{align}
					x \in y \Longrightarrow x \subset y
				\end{align}
				が成立して,また
				\begin{align}
					x = y \Longrightarrow x \subset y
				\end{align}
				も成り立つから,(\refeq{eq:thm_On_is_wellordered_6})と場合分け法則から
				\begin{align}
					(x,y) \in\ \leq
				\end{align}
				が従う.$y$の任意性より
				\begin{align}
					\forall y \in a\, \left[\, (x,y) \in\ \leq\, \right]
				\end{align}
				が成立するので$x$は$a$の最小限である.
				\QED
		\end{description}
	\end{prf}
	
	\begin{screen}
		\begin{thm}[$\ON$の部分集合の合併は順序数となる]\label{thm:union_of_set_of_ordinal_numbers_is_ordinal}
			\begin{align}
				\forall a\,
				\left(\, a \subset \ON \Longrightarrow \bigcup a \in \ON\, \right).
			\end{align}
		\end{thm}
	\end{screen}
	
	\begin{prf}
		和集合の公理より$\bigcup a \in \Univ$となる.また順序数の推移性より
		$\bigcup a$の任意の要素は順序数であるから,定理\ref{thm:On_is_wellordered}より
		\begin{align}
			\forall x,y \in \bigcup a\ (\ x \in y \vee x = y \vee y \in x\ )
		\end{align}
		も成り立つ.最後に$\operatorname{Tran}(\bigcup a)$が成り立つことを示す.
		$b$を$\bigcup a$の任意の要素とすれば,$a$の或る要素$x$に対して
		\begin{align}
			b \in x
		\end{align}
		となるが,$x$の推移性より$b \subset x$となり,$x \subset \bigcup a$と併せて
		\begin{align}
			b \subset \bigcup a
		\end{align}
		が従う.
		\QED
	\end{prf}
	
	\begin{screen}
		\begin{dfn}[後者]
			$x$を集合とするとき,
			\begin{align}
				x \cup \{x\}
			\end{align}
			を$x$の{\bf 後者}\index{こうしゃ@後者}{\bf (latter)}と呼ぶ.
		\end{dfn}
	\end{screen}
	
	\begin{screen}
		\begin{thm}[順序数の後者は順序数である]\label{thm:latter_element_is_ordinal}
			$\alpha$が順序数であるということと$\alpha \cup \{\alpha\}$が順序数であるということは同値である.
			\begin{align}
				\forall \alpha\, \left(\, \alpha \in \ON \Longleftrightarrow \alpha \cup \{\alpha\} \in \ON\, \right).
			\end{align}
		\end{thm}
	\end{screen}
	
	\begin{sketch}\mbox{}
		\begin{description}
			\item[第一段]
				$\alpha$を順序数とする.そして$x$を
				\begin{align}
					x \in \alpha \cup \{\alpha\}
					\label{fom:thm_latter_element_is_ordinal_3}
				\end{align}
				なる任意の集合とすると,
				\begin{align}
					y \in x
				\end{align}
				なる任意の集合$y$に対して定理\ref{thm:union_of_pair_is_union_of_their_elements}より
				\begin{align}
					y \in \alpha \vee y \in \{\alpha\}
					\label{fom:thm_latter_element_is_ordinal_5}
				\end{align}
				が成立する.$\alpha$が順序数であるから
				\begin{align}
					y \in \alpha \Longrightarrow y \subset \alpha
					\label{fom:thm_latter_element_is_ordinal_1}
				\end{align}
				が成立する.他方で定理\ref{thm:pair_members_are_exactly_the_given_two}より
				\begin{align}
					y \in \{\alpha\} \Longrightarrow y = \alpha
				\end{align}
				が成立し,
				\begin{align}
					y = \alpha \Longrightarrow y \subset \alpha
				\end{align}
				であるから
				\begin{align}
					y \in \{\alpha\} \Longrightarrow y \subset \alpha
					\label{fom:thm_latter_element_is_ordinal_2}
				\end{align}
				が従う.定理\ref{thm:union_is_bigger_than_any_member}より
				\begin{align}
					y \subset \alpha \Longrightarrow y \subset \alpha \cup \{\alpha\}
				\end{align}
				が成り立つので,(\refeq{fom:thm_latter_element_is_ordinal_1})と
				(\refeq{fom:thm_latter_element_is_ordinal_2})と併せて
				\begin{align}
					y \in \alpha \Longrightarrow y \subset \alpha \cup \{\alpha\}
				\end{align}
				かつ
				\begin{align}
					y \in \{\alpha\} \Longrightarrow y \subset \alpha \cup \{\alpha\}
				\end{align}
				が成立し,場合分け法則より
				\begin{align}
					y \in \alpha \vee y \in \{\alpha\} \Longrightarrow y \subset \alpha \cup \{\alpha\}
				\end{align}
				が従う.そして(\refeq{fom:thm_latter_element_is_ordinal_5})と併せて
				\begin{align}
					y \subset \alpha \cup \{\alpha\}
				\end{align}
				が成立する.$y$の任意性ゆえに(\refeq{fom:thm_latter_element_is_ordinal_3})の下で
				\begin{align}
					\forall y\, \left(\, y \in x \Longrightarrow y \subset \alpha \cup \{\alpha\}\, \right)
				\end{align}
				が成り立ち,演繹法則と$x$の任意性から
				\begin{align}
					\forall x\, \left(\, x \in \alpha \cup \{\alpha\} \Longrightarrow x \subset \alpha \cup \{\alpha\}\, \right)
				\end{align}
				が従う.ゆえにいま
				\begin{align}
					\tran{\alpha \cup \{\alpha\}}
					\label{fom:thm_latter_element_is_ordinal_4}
				\end{align}
				が得られた.また$s$と$t$を$\alpha \cup \{\alpha\}$の任意の要素とすると
				\begin{align}
					s \in \alpha \vee s = \alpha
				\end{align}
				と
				\begin{align}
					t \in \alpha \vee t = \alpha
				\end{align}
				が成り立つが,
				\begin{align}
					s \in \alpha \Longrightarrow s \in \ON
				\end{align}
				かつ
				\begin{align}
					s = \alpha \Longrightarrow s \in \ON
				\end{align}
				から
				\begin{align}
					s \in \alpha \vee s = \alpha \Longrightarrow s \in \ON
				\end{align}
				が従い,同様にして
				\begin{align}
					t \in \alpha \vee t = \alpha \Longrightarrow t \in \ON
				\end{align}
				も成り立つので,
				\begin{align}
					s \in \ON
				\end{align}
				かつ
				\begin{align}
					t \in \ON
				\end{align}
				となる.このとき定理\ref{thm:On_is_wellordered}より
				\begin{align}
					s \in t \vee s = t \vee t \in s
				\end{align}
				が成り立つので,$s$および$t$の任意性より
				\begin{align}
					\forall s,t \in \alpha \cup \{\alpha\}\,
					\left(\, s \in t \vee s = t \vee t \in s\, \right)
				\end{align}
				が得られた.(\refeq{fom:thm_latter_element_is_ordinal_4})と併せて
				\begin{align}
					\ord{\alpha \cup \{\alpha\}}
				\end{align}
				が従い,演繹法則より
				\begin{align}
					\alpha \in \ON \Longrightarrow \alpha \cup \{\alpha\} \in \ON
				\end{align}
				を得る.
				
			\item[第二段]
		\end{description}
	\end{sketch}
	
	\begin{screen}
		\begin{thm}[順序数は後者が直後の数となる]
			$\alpha$を順序数とすれば,$\ON$において$\alpha \cup \{\alpha\}$は$\alpha$の直後の数である:
			\begin{align}
				\forall \alpha \in \ON\, 
				\left[\, \forall \beta \in \ON\, (\, \alpha < \beta 
				\Longrightarrow \alpha \cup \{\alpha\} \leq \beta\, )
				\, \right].
			\end{align}
		\end{thm}
	\end{screen}
	
	\begin{sketch}
		$\alpha$と$\beta$を任意に与えられた順序数とし,
		\begin{align}
			\alpha < \beta
		\end{align}
		であるとする.定理\ref{thm:element_and_proper_subset_correspond_between_ordinal_numbers}より,このとき
		\begin{align}
			\alpha \in \beta
		\end{align}
		が成り立ち,$\leq$の定義より
		\begin{align}
			\alpha \subset \beta
		\end{align}
		も成り立つ.ところで,いま$t$を任意の集合とすると
		\begin{align}
			t \in \{\alpha\} \Longrightarrow t = \alpha
		\end{align}
		かつ
		\begin{align}
			t = \alpha \Longrightarrow t \in \beta
		\end{align}
		が成り立つので,
		\begin{align}
			\{\alpha\} \subset \beta
		\end{align}
		が成り立つ.ゆえに
		\begin{align}
			\forall x\, \left(\, x \in \left\{ \alpha, \{\alpha\} \right\} \Longrightarrow x \subset \beta\, \right)
		\end{align}
		が成り立つ.ゆえに定理\ref{thm:union_of_subsets_is_subclass}より
		\begin{align}
			\alpha \cup \{\alpha\} \subset \beta.
		\end{align}
		すなわち
		\begin{align}
			\alpha \cup \{\alpha\} \leq \beta
		\end{align}
		が成り立つ.
		\QED
	\end{sketch}
	%\section{無限}
	\begin{screen}
		\begin{dfn}[極限数]
			類$\alpha$が{\bf 極限数}\index{きょくげんすう@極限数}{\bf (limit ordinal)}であるということを
			\begin{align}
				\limo{\alpha} \defarrow \alpha \in \ON \wedge \alpha \neq \emptyset
				\wedge \forall \beta \in \ON\, \left(\, \alpha \neq \beta \cup \{\beta\}\, \right)
			\end{align}
			により定める.つまり,極限数とはいずれの順序数の後者でもない$0$を除く順序数のことである.
		\end{dfn}
	\end{screen}
	
	\begin{screen}
		\begin{thm}[全ての要素の後者で閉じていれば極限数]\label{thm:if_closed_for_latter_then_limit_ordinal}
			空でない順序数は,すべての要素の後者について閉じていれば極限数である:
			\begin{align}
				\forall \alpha \in \ON\,
				\left[\, \alpha \neq \emptyset \wedge 
				\forall \beta\, \left(\, \beta \in \alpha \Longrightarrow \beta \cup \{\beta\} \in \alpha\, \right)
				\Longrightarrow \limo{\alpha}\, \right].
			\end{align}
		\end{thm}
	\end{screen}
	
	\begin{sketch}
		$\alpha$を順序数とし,
		\begin{align}
			\alpha \neq \emptyset \wedge 
			\forall \beta\, \left(\, \beta \in \alpha \Longrightarrow \beta \cup \{\beta\} \in \alpha\, \right)
			\label{fom:thm_if_closed_for_latter_then_limit_ordinal_1}
		\end{align}
		が成り立っているとする.ここで$\beta$を順序数とすると
		\begin{align}
			\beta \in \alpha \vee \beta = \alpha \vee \alpha \in \beta
		\end{align}
		が成り立つ.
		\begin{align}
			\beta = \alpha
		\end{align}
		と
		\begin{align}
			\alpha \in \beta
		\end{align}
		の場合はいずれも
		\begin{align}
			\alpha \in \beta \cup \{\beta\}
		\end{align}
		が成り立つので,定理\ref{thm:no_set_is_an_element_of_itself}より
		\begin{align}
			\alpha \neq \beta \cup \{\beta\}
		\end{align}
		が成立する.ゆえに
		\begin{align}
			(\, \beta = \alpha \vee \beta \in \alpha\, ) \Longrightarrow \alpha \neq \beta \cup \{\beta\}
			\label{fom:thm_if_closed_for_latter_then_limit_ordinal_2}
		\end{align}
		が成立する.他方で(\refeq{fom:thm_if_closed_for_latter_then_limit_ordinal_1})より
		\begin{align}
			\beta \in \alpha \Longrightarrow \beta \cup \{\beta\} \in \alpha
		\end{align}
		も満たされて,
		\begin{align}
			\beta \cup \{\beta\} \in \alpha \Longrightarrow \beta \cup \{\beta\} \neq \alpha
		\end{align}
		と併せて
		\begin{align}
			\beta \in \alpha \Longrightarrow \beta \cup \{\beta\} \neq \alpha
			\label{fom:thm_if_closed_for_latter_then_limit_ordinal_3}
		\end{align}
		が成り立つ.そして(\refeq{fom:thm_if_closed_for_latter_then_limit_ordinal_2})と
		(\refeq{fom:thm_if_closed_for_latter_then_limit_ordinal_3})と場合分け法則により
		\begin{align}
			\left(\, \beta \in \alpha \vee \beta = \alpha \vee \alpha \in \beta\, \right)
			\Longrightarrow \alpha \neq \beta \cup \{\beta\}
		\end{align}
		が成立する.ゆえに
		\begin{align}
			\forall \beta \in \ON\, \left(\, \alpha \neq \beta \cup \{\beta\}\, \right)
		\end{align}
		が成立する.ゆえに$\alpha$は極限数である.
		\QED
	\end{sketch}
	
	次の無限公理は極限数の存在を保証する.
	
	\begin{screen}
		\begin{axm}[無限公理]
			空集合を要素に持ち,全ての要素の後者について閉じている集合が存在する:
			\begin{align}
				\exists a\, \left[\, \emptyset \in a
				\wedge \forall x\, \left(\, x \in a \Longrightarrow x \cup \{x\} \in a\, \right)\, \right].
			\end{align}
		\end{axm}
	\end{screen}
	
	\begin{screen}
		\begin{thm}[極限数は存在する]
			\begin{align}
				\exists \alpha \in \ON\, \left(\, \limo{\alpha}\, \right).
			\end{align}
		\end{thm}
	\end{screen}
	
	\begin{prf}
		無限公理より
		\begin{align}
			\emptyset \in a
			\wedge \forall x\, \left(\, x \in a \Longrightarrow x \cup \{x\} \in a\, \right)
		\end{align}
		を満たす集合$a$が取れる.
		\begin{align}
			b \defeq a \cap \ON
		\end{align}
		とおくとき
		\begin{align}
			\bigcup b
		\end{align}
		が極限数となることを示す.まず
		\begin{align}
			\emptyset \in a \cap \ON \wedge \{\emptyset\} \in a \cap \ON
		\end{align}
		が成り立つから
		\begin{align}
			\emptyset \in \bigcup b
		\end{align}
		が成り立つ.ゆえに$\bigcup b$は空ではない.また定理\ref{thm:union_of_set_of_ordinal_numbers_is_ordinal}より
		\begin{align}
			\bigcup b \in \ON
		\end{align}
		が成立する.$\alpha$を$\bigcup b$の要素とすると,
		\begin{align}
			x \in b \wedge \alpha \in x
		\end{align}
		を満たす順序数$x$が取れる.このとき
		\begin{align}
			\alpha \cup \{\alpha\} \in x
		\end{align}
		か
		\begin{align}
			\alpha \cup \{\alpha\} = x
		\end{align}
		が成り立つが,いずれの場合も
		\begin{align}
			\alpha \cup \{\alpha\} \in x \cup \{x\}
		\end{align}
		が成立する.他方で
		\begin{align}
			x \cup \{x\} \in a \cap \ON
		\end{align}
		も成立するから
		\begin{align}
			\alpha \cup \{\alpha\} \in \bigcup b
		\end{align}
		が成立する.ゆえに
		\begin{align}
			\forall \alpha\, \left(\, \alpha \in \bigcup b \Longrightarrow \alpha \cup \{\alpha\} \in \bigcup b\, \right)
		\end{align}
		が成立する.ゆえに定理\ref{thm:if_closed_for_latter_then_limit_ordinal}より$\bigcup b$は極限数である.
		\QED
	\end{prf}
	
	\monologue{
		無限公理から極限数の存在が示されましたが,無限公理の
		代わりに極限数の存在を公理に採用しても無限公理の主張は導かれます.
		すなわち無限公理の主張と極限数が存在するという主張は同値なのです.
		本稿の流れでは極限数の存在を公理とした方が自然に感じられますが,
		しかし無限公理の方が主張が簡単ですし,他の文献ではこちらを公理としているようです.
	}
	
	\begin{screen}
		\begin{thm}[極限数は上限で表せる]
			\begin{align}
				\limo{\alpha} \Longrightarrow \alpha = \bigcup \Set{\beta}{\beta \in \alpha}.
			\end{align}
		\end{thm}
	\end{screen}
	
	\begin{sketch}
		$\alpha$を極限数とする.$x$を$\alpha$の要素とすれば,
		\begin{align}
			x \cup \{x\} \neq \alpha
		\end{align}
		が成り立つから
		\begin{align}
			x \cup \{x\} \in \alpha
		\end{align}
		が成り立ち
		\begin{align}
			x \in \bigcup \Set{\beta}{\beta \in \alpha}
		\end{align}
		が成立する.$x$を$\bigcup \Set{\beta}{\beta \in \alpha}$の要素とすれば
		\begin{align}
			x \in \beta \wedge \beta \in \alpha
		\end{align}
		なる順序数$\beta$が取れて,順序数の推移性より
		\begin{align}
			x \in \alpha
		\end{align}
		が従う.$x$の任意性から
		\begin{align}
			\alpha = \bigcup \Set{\beta}{\beta \in \alpha}
		\end{align}
		が成立する.
		\QED
	\end{sketch}
	
	\begin{screen}
		\begin{dfn}[自然数]
			最小の極限数を
			\begin{align}
				\Natural
			\end{align}
			と書く.また$\Natural$の要素を{\bf 自然数}\index{しぜんすう@自然数}{\bf (natural number)}と呼ぶ.
		\end{dfn}
	\end{screen}
	
	$\Natural$は最小の極限数であるから,その要素である自然数はどれも極限数ではない.
	従って$\emptyset$を除く自然数は必ずいずれかの自然数の後者である.

	\begin{screen}
		\begin{dfn}[無限]\label{def:infinity}
			本稿においては,{\bf 無限}\index{むげん@無限}{\bf (infinity)}を表す記号$\infty$を
			\begin{align}
				\infty \defeq \Natural
			\end{align}
			によって定める.
		\end{dfn}
	\end{screen}
	
	\begin{screen}
		\begin{thm}[超限帰納法]\label{thm:transfinite_induction}
			$A$を$\mathcal{L}'$の式,$\alpha$を$A$に現れる文字,$\beta$を$A$に現れない文字とする.
			このとき,$A$に現れる文字で$\alpha$のみが$A$で量化されていない場合,次が成り立つ:
			\begin{align}
				\forall \alpha \in \ON\, 
				\left(\, \forall \beta \in \alpha\, A(\beta)
				\Longrightarrow A(\alpha)\, \right)
				\Longrightarrow \forall \alpha \in \ON\, A(\alpha).
			\end{align}
		\end{thm}
	\end{screen}
	
	\begin{prf}
		正則性公理と定理\ref{thm:equivalent_condition_of_axiom_of_regularity}より
		\begin{align}
			\forall \alpha\, \left[\, \forall \beta \in \alpha\, (\, \beta \in \ON \Longrightarrow A(\beta)\, )
			\Longrightarrow (\, \alpha \in \ON \Longrightarrow A(\alpha)\, )\, \right]
			\Longrightarrow \forall \alpha\, (\, \alpha \in \ON \Longrightarrow A(\alpha)\, )
		\end{align}
		が成り立つ.このとき$\alpha$を$\mathcal{L}$の任意の対象とすれば,
		\begin{align}
			\begin{gathered}
				\forall \beta \in \alpha\ (\ \beta \in \ON \Longrightarrow A(\beta)\ )
				\Longrightarrow (\ \alpha \in \ON \Longrightarrow A(\alpha)\ ), \\
				\forall \beta \in \alpha\ (\ \beta \in \ON \Longrightarrow A(\beta)\ ) \wedge \alpha \in \ON \Longrightarrow A(\alpha)
			\end{gathered}
		\end{align}
		は同値であり,他方で順序数の要素は順序数である(定理\ref{thm:On_is_transitive})から
		\begin{align}
			\begin{gathered}
				\forall \beta \in \alpha\ (\ \beta \in \ON \Longrightarrow A(\beta)\ ) \wedge \alpha \in \ON, \\
				\alpha \in \ON \wedge \forall \beta \in \alpha\ A(\beta)
			\end{gathered}
		\end{align}
		も同値である.従って
		\begin{align}
			\alpha \in \ON \wedge \forall \beta \in \alpha\ A(\beta)
			\Longrightarrow A(\alpha)
		\end{align}
		が成り立ち,またこれは
		\begin{align}
			\alpha \in \ON \Longrightarrow \left(\ \forall \beta \in \alpha\ A(\beta)
			\Longrightarrow A(\alpha)\ \right)
		\end{align}
		と同値である.$\alpha$の任意性より
		\begin{align}
			\forall \alpha \in \ON\ 
			\left(\ \forall \beta \in \alpha\ A(\beta)
			\Longrightarrow A(\alpha)\ \right)
			\Longrightarrow \forall \alpha \in \ON\ A(\alpha).
		\end{align}
		が得られる.
		\QED
	\end{prf}
	
	\begin{screen}
		\begin{thm}[数学的帰納法の原理]
		\label{thm:the_principle_of_mathematical_induction}
			$\omg$は次の意味で最小の無限集合である:
			\begin{align}
				\forall a\ \left(\ \emptyset \in a \wedge \forall x\ 
				(\ x \in a \Longrightarrow x \cup \{x\} \in a\ ) 
				\Longrightarrow \omg \subset a\ \right).
			\end{align}
		\end{thm}
	\end{screen}
	
	\begin{prf}
		超限帰納法で示す.いま$a$を
		\begin{align}
			\emptyset \in a \wedge \forall x\ 
			(\ x \in a \Longrightarrow x \cup \{x\} \in a\ )
		\end{align}
		を満たす類とし,また$\alpha$を任意に与えられた順序数とする.
		$\alpha = \emptyset$の場合は$\emptyset \in a$より
		\begin{align}
			\emptyset \in \omega \Longrightarrow \emptyset \in a
		\end{align}
		が成立する.$\alpha \neq \emptyset$の場合,$\alpha$の任意の要素$\beta$に対して
		\begin{align}
			\beta \in {\bf \omega} \Longrightarrow \beta \in a
		\end{align}
		が成り立つと仮定する.このとき,$\alpha \in {\bf \omega}$なら
		$\alpha$は極限数でないから$\alpha = \beta \cup \{\beta\}$を満たす順序数$\beta$が取れて,
		仮定より$\beta \in a$となり$\alpha \in a$が従う.以上で
		\begin{align}
			\forall \alpha \in \ON\ (\ \forall \beta \in \alpha\ (\ \beta \in {\bf \omega} \Longrightarrow \beta \in a\ ) \Longrightarrow (\ \alpha \in {\bf \omega} \Longrightarrow \alpha \in a\ )\ )
		\end{align}
		が得られた.超限帰納法により
		\begin{align}
			\forall \alpha \in \ON\ (\ \alpha \in {\bf \omega} \Longrightarrow \alpha \in a\ )
		\end{align}
		となるから$\omega \subset a$が出る.
		\QED
	\end{prf}
	%\section{超限帰納法}
	$x$を任意に与えられた集合としたとき,$x$の任意の要素$y$で
	\begin{align}
		A(y)
	\end{align}
	が成り立つならば
	\begin{align}
		A(x)
	\end{align}
	が成り立つとする.すると,なんと$A(x)$は普遍的に成り立つのである.つまり
	\begin{align}
		\forall x\, \left[\, \forall y \in x\, A(y) \Longrightarrow A(x)\, \right]
		\Longrightarrow \forall x A(x)
	\end{align}
	が成り立つわけだが,この事実を本稿では{\bf 集合の帰納法}と呼ぶ.また派生形としては,
	集合を順序数に制限した場合の{\bf 超限帰納法}\index{ちょうげんきのうほう@超限帰納法}{\bf (transfinite induction)}と,
	自然数に制限した場合の{\bf 数学的帰納法}\index{すうがくてききのうほう@数学的帰納法}{\bf (mathematical induction)}がある.
	
	\begin{screen}
		\begin{thm}[集合の帰納法]\label{thm:equivalent_condition_of_axiom_of_regularity}
			$A$を$\mathcal{L}'$の式とし,$x$を$A$に現れる文字とし,$y$を$A$に現れない文字とし,
			$A$に現れる文字で$x$のみが量化されていないとする.このとき
			\begin{align}
				\forall x\, \left[\, \forall y \in x\, A(y) \Longrightarrow A(x)\, \right]
				\Longrightarrow \forall x A(x).
			\end{align}
		\end{thm}
	\end{screen}
	
	\begin{sketch}
		いま
		\begin{align}
			a \defeq \Set{x}{\rightharpoondown A(x)}
		\end{align}
		とおく.正則性公理より
		\begin{align}
			a \neq \emptyset \Longrightarrow 
			\exists x\, (\, x \in a \wedge x \cap a = \emptyset\, )
		\end{align}
		が成り立つので,対偶を取れば
		\begin{align}
			\forall x\, (\, x \notin a \vee x \cap a \neq \emptyset\, )
			\Longrightarrow a = \emptyset
			\label{fom:thm_equivalent_condition_of_axiom_of_regularity_1}
		\end{align}
		が成り立つ.ここで
		\begin{align}
			x \cap a \neq \emptyset \Longleftrightarrow \exists y \in x\, (\, y \in a\, )
		\end{align}
		が成り立つので(\refeq{fom:thm_equivalent_condition_of_axiom_of_regularity_1})から
		\begin{align}
			\forall x\, \left[\, x \notin a \vee \exists y \in x\, (\, y \in a\, )\, \right]
			\Longrightarrow a = \emptyset
			\label{fom:thm_equivalent_condition_of_axiom_of_regularity_2}
		\end{align}
		が従い,そして論理和は否定と含意で書き直せる(推論法則\ref{logicalthm:rule_of_inference_3})から
		\begin{align}
			\forall x\, \left[\, \forall y \in x\, (\, y \notin a\, ) \Longrightarrow x \notin a\, \right]
			\Longrightarrow a = \emptyset
		\end{align}
		が従う.ところで類の公理より
		\begin{align}
			x \notin a \Longleftrightarrow A(x)
		\end{align}
		が成り立つから
		\begin{align}
			\forall x\, \left[\, \forall y \in x\, A(y)
			\Longrightarrow A(x)\, \right]
			\Longrightarrow \forall x A(x)
		\end{align}
		を得る.
		\QED
	\end{sketch}
	
	本稿では正則性公理を認めているが,いまだけは認めないことにして代わりに
	集合の帰納法が正しいと仮定してみると,今度は正則性公理が定理として導かれる.実際,$a$を類とすれば
	\begin{align}
		\forall x\, \left[\, \forall y \in x\, (\, y \notin a\, )
		\Longrightarrow\ x \notin a\, \right]
		\Longrightarrow \forall x\, (\, x \notin a\, )
	\end{align}
	が成立するが,ここで対偶を取れば
	\begin{align}
		\exists x\, (\, x \in a\, ) \Longrightarrow 
		\exists x \in a\, \left[\, \forall y \in x\, (\, y \notin a\, )\, \right]
	\end{align}
	が成立し,
	\begin{align}
		a \neq \emptyset \Longleftrightarrow \exists x\, (\, x \in a\, )
	\end{align}
	と
	\begin{align}
		\forall y \in x\, (\, y \notin a\, ) \Longleftrightarrow x \cap a = \emptyset 
	\end{align}
	が成り立つことを併せれば
	\begin{align}
		a \neq \emptyset \Longrightarrow 
		\exists x \in a\, (\, x \cap a = \emptyset\, )
	\end{align}
	が出る.この意味で正則性公理は{\bf 帰納法の公理}とも呼ばれる.
	
	\begin{screen}
		\begin{thm}[超限帰納法]\label{thm:transfinite_induction}
			$A$を$\mathcal{L}'$の式,$\alpha$を$A$に現れる文字,$\beta$を$A$に現れない文字とする.
			このとき,$A$に現れる文字で$\alpha$のみが$A$で量化されていない場合,次が成り立つ:
			\begin{align}
				\forall \alpha \in \ON\, \left(\, \forall \beta \in \alpha\, A(\beta) \Longrightarrow A(\alpha)\, \right)
				\Longrightarrow \forall \alpha \in \ON\, A(\alpha).
			\end{align}
		\end{thm}
	\end{screen}
	
	\begin{prf}
		定理\ref{thm:equivalent_condition_of_axiom_of_regularity}より
		\begin{align}
			\forall \alpha\, \left[\, \forall \beta \in \alpha\, (\, \beta \in \ON \Longrightarrow A(\beta)\, )
			\Longrightarrow (\, \alpha \in \ON \Longrightarrow A(\alpha)\, )\, \right]
			\Longrightarrow \forall \alpha\, (\, \alpha \in \ON \Longrightarrow A(\alpha)\, )
			\label{fom:thm_transfinite_induction}
		\end{align}
		が成り立つ.いま
		\begin{align}
			\forall \alpha \in \ON\, \left(\, \forall \beta \in \alpha\, A(\beta) \Longrightarrow A(\alpha)\, \right)
			\label{fom:thm_transfinite_induction_1}
		\end{align}
		が成り立っているとする.その上で$\alpha$を集合とし,
		\begin{align}
			\forall \beta \in \alpha\, (\, \beta \in \ON \Longrightarrow A(\beta)\, )
			\label{fom:thm_transfinite_induction_2}
		\end{align}
		が成り立っているとする.さらにその上で
		\begin{align}
			\alpha \in \ON
		\end{align}
		が成り立っているとする.このとき$\beta$を
		\begin{align}
			\beta \in \alpha
		\end{align}
		なる集合とすると,順序数の推移性より
		\begin{align}
			\beta \in \ON
		\end{align}
		が成り立つので,(\refeq{fom:thm_transfinite_induction_2})と併せて
		\begin{align}
			A(\beta)
		\end{align}
		が成り立つ.すなわちいま
		\begin{align}
			\forall \beta \in \alpha\, A(\beta)
		\end{align}
		が成り立つ.また(\refeq{fom:thm_transfinite_induction_1})より
		\begin{align}
			\forall \beta \in \alpha\, A(\beta) \Longrightarrow A(\alpha)
		\end{align}
		が成り立つので,いま
		\begin{align}
			A(\alpha)
		\end{align}
		が成立する.つまり,(\refeq{fom:thm_transfinite_induction_2})までを仮定したときには
		\begin{align}
			\alpha \in \ON \Longrightarrow A(\alpha)
		\end{align}
		が成立する.ゆえに(\refeq{fom:thm_transfinite_induction_1})までを仮定したときには
		\begin{align}
			\forall \beta \in \alpha\, (\, \beta \in \ON \Longrightarrow A(\beta)\, )
			\Longrightarrow (\, \alpha \in \ON \Longrightarrow A(\alpha)\, )
		\end{align}
		が成立し,$\alpha$の任意性から
		\begin{align}
			\forall \alpha\, \left[\, \forall \beta \in \alpha\, (\, \beta \in \ON \Longrightarrow A(\beta)\, )
			\Longrightarrow (\, \alpha \in \ON \Longrightarrow A(\alpha)\, )\, \right]
		\end{align}
		が成立する.ゆえに,何も仮定しなくても
		\begin{align}
			(\refeq{fom:thm_transfinite_induction_1}) \Longrightarrow
			\forall \alpha\, \left[\, \forall \beta \in \alpha\, (\, \beta \in \ON \Longrightarrow A(\beta)\, )
			\Longrightarrow (\, \alpha \in \ON \Longrightarrow A(\alpha)\, )\, \right]
			\label{fom:thm_transfinite_induction_3}
		\end{align}
		が成立する.(\refeq{fom:thm_transfinite_induction})と(\refeq{fom:thm_transfinite_induction_3})と含意の推移性より
		\begin{align}
			\forall \alpha \in \ON\, \left(\, \forall \beta \in \alpha\, A(\beta) \Longrightarrow A(\alpha)\, \right)
			\Longrightarrow \forall \alpha\, (\, \alpha \in \ON \Longrightarrow A(\alpha)\, )
		\end{align}
		が従うが,
		\begin{align}
			\forall \alpha\, (\, \alpha \in \ON \Longrightarrow A(\alpha)\, )
		\end{align}
		を略記したものが
		\begin{align}
			\forall \alpha \in \ON\, A(\alpha)
		\end{align}
		であるから
		\begin{align}
			\forall \alpha \in \ON\, \left(\, \forall \beta \in \alpha\, A(\beta) \Longrightarrow A(\alpha)\, \right)
			\Longrightarrow \forall \alpha \in \ON\, A(\alpha)
		\end{align}
		が成り立つことになる.
		\QED
	\end{prf}
	
	
	以後本稿では超限帰納法を頻繁に扱うので,ここでその{\bf 利用方法}を述べておく.
	順序数に対する何らかの言明$A$が与えられたとき,それがいかなる順序数に対しても真であることを示したいとする.往々にして
	\begin{align}
		\forall \alpha \in \ON\, A(\alpha)
	\end{align}
	をいきなり示すのは難しく,一方で
	\begin{align}
		\forall \beta \in \alpha\, A(\beta)
	\end{align}
	から
	\begin{align}
		A(\alpha)
	\end{align}
	を導くことは容易い.それは順序数の``順番''的な性質の良さによるが,超限帰納法のご利益は
	\begin{align}
		\forall \alpha \in \ON\, \left(\, \forall \beta \in \alpha\, A(\beta) \Longrightarrow A(\alpha)\, \right)
	\end{align}
	が成り立つことさえ示してしまえばいかなる順序数に対しても$A$が真となってくれるところにある.
	
	$\alpha$を任意に与えられた順序数とするとき,
	\begin{align}
		\alpha = 0
	\end{align}
	であると空虚な真によって
	\begin{align}
		\forall \beta \in \alpha\, A(\beta)
	\end{align}
	は必ず真となるから,まずは
	\begin{align}
		A(0)
	\end{align}
	が成り立つことを示さなければならない.$A(0)$が偽であると
	\begin{align}
		\forall \beta \in \alpha\, A(\beta) \wedge \rightharpoondown A(0)
	\end{align}
	が真となって
	\begin{align}
		\forall \beta \in \alpha\, A(\beta) \Longrightarrow A(0)
	\end{align}
	が偽となってしまうからである.$\alpha$が$0$でないときは素直に
	\begin{align}
		\forall \beta \in \alpha\, A(\beta)
	\end{align}
	が成り立つとき
	\begin{align}
		A(\alpha)
	\end{align}
	が成り立つことを示せば良い.以上超限帰納法の利用法をまとめると,
	
	\begin{itembox}[l]{超限帰納法の利用手順}
		順序数に対する何らかの言明$A$が与えられて,それがいかなる順序数に対しても真なることを示したいならば,
		\begin{itemize}
			\item まずは$A(0)$が成り立つことを示し,
			\item 次は$\alpha$を$0$でない順序数として
				$\forall \beta \in \alpha\, A(\beta) \Longrightarrow A(\alpha)$が成り立つことを示す.
		\end{itemize}
	\end{itembox}
	
	\begin{screen}
		\begin{thm}[数学的帰納法の原理]
		\label{thm:the_principle_of_mathematical_induction}
			$\Natural$は次の意味で最小の無限集合である:
			\begin{align}
				\forall a\, \left[\, \emptyset \in a \wedge \forall x\, 
				(\, x \in a \Longrightarrow x \cup \{x\} \in a\, ) 
				\Longrightarrow \Natural \subset a\, \right].
			\end{align}
		\end{thm}
	\end{screen}
	
	\begin{prf}
		$a$を集合とし,
		\begin{align}
			\emptyset \in a \wedge \forall x\, 
			(\, x \in a \Longrightarrow x \cup \{x\} \in a\, )
			\label{fom:thm_the_principle_of_mathematical_induction_1}
		\end{align}
		が成り立っているとする.このとき
		\begin{align}
			\forall \alpha \in \ON\, (\, \alpha \in \Natural \Longrightarrow \alpha \in a\, )
		\end{align}
		が成り立つことを超限帰納法で示す.まずは
		\begin{align}
			0 \in a
		\end{align}
		から
		\begin{align}
			\emptyset \in \Natural \Longrightarrow \emptyset \in a
		\end{align}
		が成立する.次に$\alpha$を任意に与えられた$0$でない順序数とする.
		\begin{align}
			\forall \beta \in \alpha\, (\, \beta \in \Natural \Longrightarrow \beta \in a\, )
			\label{fom:thm_the_principle_of_mathematical_induction_2}
		\end{align}
		が成り立っているとすると,
		\begin{align}
			\alpha \in \Natural
		\end{align}
		なら$\alpha$は極限数でないから
		\begin{align}
			\alpha = \beta \cup \{\beta\}
		\end{align}
		を満たす自然数$\beta$が取れて,(\refeq{fom:thm_the_principle_of_mathematical_induction_2})より
		\begin{align}
			\beta \in a
		\end{align}
		が成り立ち,(\refeq{fom:thm_the_principle_of_mathematical_induction_1})より
		\begin{align}
			\alpha \in a
		\end{align}
		が従う.以上で
		\begin{align}
			\forall \alpha \in \ON\, \left[\ 
				\forall \beta \in \alpha\, (\, \beta \in \Natural \Longrightarrow \beta \in a\, )
				\Longrightarrow (\, \alpha \in \Natural \Longrightarrow \alpha \in a\, )\, \right]
		\end{align}
		が得られた.超限帰納法により
		\begin{align}
			\forall \alpha \in \ON\, (\, \alpha \in \Natural \Longrightarrow \alpha \in a\, )
		\end{align}
		が成り立つから
		\begin{align}
			\Natural \subset a
		\end{align}
		が従う.
		\QED
	\end{prf}
	%\section{再帰的定義}
\label{sec:recursive_definition}
	例えば
	\begin{align}
		a_1,\quad a_2,\quad a_3,\quad a_4,\quad \cdots\quad a_n,\quad \cdots
	\end{align}
	なる列が与えられたときに,その$n$重の順序対を
	\begin{align}
		(a_1,a_2,\cdots,a_n)
	\end{align}
	などと書くことがある.まあ
	\begin{align}
		(a_0,a_1)
	\end{align}
	ならば単なる順序対であり,
	\begin{align}
		(a_0,a_1,a_2)
	\end{align}
	も
	\begin{align}
		((a_0,a_1),a_2)
	\end{align}
	で定められ,
	\begin{align}
		(a_0,a_1,a_2,a_3)
	\end{align}
	も
	\begin{align}
		(((a_0,a_1),a_2),a_3)
	\end{align}
	で定められる.このように具体的に全ての要素を書き出せるうちは何も問題は無い.
	ただし,同じ操作を$n$回反復するということを表現するために
	\begin{align}
		\cdots
	\end{align}
	なる不明瞭な記号を無断で用いることは$\mathcal{L}'$において許されない.
	そもそもまだ``$n$回の反復''をどんな式で表現したら良いかもわからないのである.
	次の定理は,このような再帰的な操作が$\mathcal{L}'$で可能であることを保証する.
	
	\begin{screen}
		\begin{thm}[超限帰納法による写像の構成]
			類$G$を$\Univ$上の写像とするとき,
			\begin{align}
				K \defeq \Set{f}{\exists \alpha \in \ON\ \left(\ f:\alpha \longrightarrow V \wedge \forall \beta \in \alpha\ (\ f(\beta) = G(f|_\beta)\ )\ \right)}
			\end{align}
			とおいて
			\begin{align}
				F \defeq \bigcup K
			\end{align}
			と定めると,$F$は$\ON$上の写像であって
			\begin{align}
				\forall \alpha \in \ON\ (\ F(\alpha) = G(F|_\alpha)\ )
			\end{align}
			を満たす.また$\ON$上の写像で上式を満たすのは$F$のみである.
		\end{thm}
	\end{screen}
	
	\begin{prf}\mbox{}
		\begin{description}
			\item[第二段] $F$が写像であることを示す.
				まず$K$の任意の要素は$V \times V$の部分集合であるから
				\begin{align}
					F \subset V \times V
				\end{align}
				となる.$x,y,z$を任意の集合とする.
				$(x,y) \in F$かつ$(x,z) \in F$のとき,
				$K$の或る要素$f$と$g$が存在して
				\begin{align}
					(x,y) \in f \wedge (x,z) \in g
				\end{align}
				を満たすが,ここで$f(x) = g(x)$となることを言うために,
				$\alpha = \operatorname{dom}(f),\ 
				\beta = \operatorname{dom}(g)$とおき,
				\begin{align}
					\forall \gamma \in \ON\ (\ \gamma \in \alpha \wedge \gamma \in \beta \Longrightarrow f(\gamma) = g(\gamma)\ )
					\label{eq:thm_transfinite_recursion_theorem_1}
				\end{align}
				が成り立つことを示す.いま$\gamma$を任意の順序数とする.$\gamma = \emptyset$の場合は
				$f|_\gamma = \emptyset$かつ$g|_\gamma = \emptyset$となるから
				\begin{align}
					f(\gamma) = G(\emptyset) = g(\gamma)
				\end{align}
				が成立する.$\gamma \neq \emptyset$の場合は
				\begin{align}
					\forall \xi \in \gamma\ (\ \xi \in \alpha \wedge \xi \in \beta \Longrightarrow f(\xi) = g(\xi)\ )
				\end{align}
				が成り立っていると仮定する.このとき$\gamma \in \alpha \wedge \gamma \in \beta$ならば
				順序数の推移性より$\gamma$の任意の要素$\xi$は$\xi \in \alpha \wedge \xi \in \beta$を満たすから
				\begin{align}
					\forall \xi \in \gamma\ (\ f(\xi) = g(\xi)\ )
				\end{align}
				が成立する.従って
				\begin{align}
					f|_\gamma = g|_\gamma
				\end{align}
				が成立するので$f(\gamma) = g(\gamma)$が得られる.超限帰納法より
				(\refeq{eq:thm_transfinite_recursion_theorem_1})が得られる.
				以上より
				\begin{align}
					y = f(x) = g(x) = z
				\end{align}
				となるので$F$はsingle-valuedである.
			
			\item[第三段] $\operatorname{dom}(F) \subset \ON$が成り立つことを示す.
				実際
				\begin{align}
					\operatorname{dom}(F) = \bigcup_{f \in K} \operatorname{dom}(f)
				\end{align}
				かつ$\forall f \in K\ (\ \operatorname{dom}(f) \subset \ON\ )$だから
				$\operatorname{dom}(F) \subset \ON$となる.
				
			\item[第四段] $\operatorname{Tran}(\operatorname{dom}(F))$であることを示す.
				実際任意の集合$x,y$について
				\begin{align}
					y \in x \wedge x \in \operatorname{dom}(F)
				\end{align}
				が成り立っているとき,或る$f \in K$で$x \in \operatorname{dom}(f)$
				となり,$\operatorname{dom}(f)$は順序数なので,順序数の推移律から
				\begin{align}
					y \in \operatorname{dom}(f)
				\end{align}
				が従う.ゆえに$y \in \operatorname{dom}(F)$となる.
				
			\item[第五段] $\forall \alpha \in \operatorname{dom}(F)\ (\ F(\alpha) = G(F|_\alpha)\ )$が成り立つことを示す.
				実際,$\alpha \in \operatorname*{dom}(F)$なら
				$K$の或る要素$f$に対して$\alpha \in \operatorname*{dom}(f)$となるが,
				$f \subset F$であるから
				\begin{align}
					f(\alpha) = F(\alpha)
				\end{align}
				が成り立つ.これにより$f|_\alpha = f \cap (\alpha \times V)
				= F \cap (\alpha \times V) = F|_\alpha$より
				\begin{align}
					G(f|_\alpha) = G(F|_\alpha)
				\end{align}
				も成り立つ.$f(\alpha) = G(f|_\alpha)$と併せて
				$F(\alpha) = G(F|_\alpha)$を得る.
			
			\item[第六段] 
				$\alpha$を任意の順序数として
				$\forall \beta \in \alpha\ (\ \beta \in \operatorname{dom}(F)\ )
				\Longrightarrow \alpha \in \operatorname{dom}(F)$が成り立つことを示す.
				$\alpha = \emptyset$の場合は
				\begin{align}
					\forall f \in K\ (\ \operatorname{dom}(f) \neq \emptyset
					\Longrightarrow \emptyset \in \operatorname{dom}(f)\ )
				\end{align}
				が満たされるので$\alpha \in \operatorname{dom}(F)$となる
				(定理\ref{thm:properties_of_ordinal_numbers}).
				$\alpha \neq \emptyset$の場合,
				\begin{align}
					\forall \beta \in \alpha\ (\ \beta \in \operatorname{dom}(F)\ )
				\end{align}
				が成り立っているとして$f = F|_\alpha$とおけば,$f$は$\alpha$上の写像であり,
				$\alpha$の任意の要素$\beta$に対して
				\begin{align}
					f(\beta)
					= F|_\alpha(\beta)
					= F(\beta)
					= G(F|_\beta)
					= G(f|_\beta)
				\end{align}
				を満たすから$f \in K$である.このとき$f' = f \cup \{(\alpha,G(f))\}$も
				$K$に属するので
				\begin{align}	
					\alpha \in \operatorname{dom}(f') \subset
					\operatorname{dom}(F)
				\end{align}
				が成立する.超限帰納法より
				\begin{align}
					\forall \alpha \in \ON\ (\ \alpha \in \operatorname{dom}(F)\ )
				\end{align}
				が成立し,前段の結果と併せて
				\begin{align}
					\ON = \operatorname{dom}(F)
				\end{align}
				を得る.
				
			\item[第七段]
				$F$の一意性を示す.類$H$が
				\begin{align}
					H:\ON \longrightarrow V 
					\wedge \forall \alpha \in \ON\ (\ H(\alpha) = G(H|_\alpha)\ )
				\end{align}
				を満たすとき,$F = H$が成り立つことを示す.
				いま,$\alpha$を任意に与えられた順序数とする.$\alpha = \emptyset$の場合は
				\begin{align}
					F|_\emptyset = \emptyset = H|_\emptyset
				\end{align}
				より$F(\emptyset) = H(\emptyset)$となる.$\alpha \neq \emptyset$の場合,
				\begin{align}
					\forall \beta \in \alpha\ (\ F(\beta) = H(\beta)\ )
				\end{align}
				が成り立っていると仮定すれば
				\begin{align}
					F|_\alpha = H|_\alpha
				\end{align}
				が成り立つから$F(\alpha) = H(\alpha)$となる.以上で
				\begin{align}
					\forall \alpha \in \ON\ \left(\ \forall \beta \in \alpha\ 
					(\ F(\beta) = H(\beta)\ ) \Longrightarrow F(\alpha) = H(\alpha)\ \right)
				\end{align}
				が得られた.超限帰納法より
				\begin{align}
					\forall \alpha \in \ON\ (\ F(\alpha) = H(\alpha)\ )
				\end{align}
				が従い$F = H$が出る.
				\QED
		\end{description}
	\end{prf}
	
	\begin{itembox}[l]{再帰的定義の応用 : 多数の要素からなる順序対}
		$a$を$\Natural$から集合$A$への写像とすると,
		\begin{align}
			a_n \defeq a(n)
		\end{align}
		と書けば
		\begin{align}
			a_0, a_1, a_2, \cdots
		\end{align}
		なる列が作られる.ここでは
		\begin{align}
			(a_0,a_1,\cdots, a_n)
		\end{align}
		のような記法の集合論的意味付けを考察する.
	\end{itembox}
	
		$\Univ$上の写像$G$を
		\begin{align}
			G(x) = 
			\begin{cases}
				a_0 & \mbox{if } \dom{x} = \emptyset \\
				(x(k),a(\dom{x})) & \mbox{if } \dom{x} = k \cup \{k\} \wedge k \in \Natural \\
				\emptyset & \mbox{o.w.}
			\end{cases}
		\end{align}
		によって定めてみると,つまり$G$とは
		\begin{align}
			\{\, (x,y) \mid \quad &\left(\, \dom{x} = \emptyset \Longrightarrow y = a_0\, \right) \\
		&\wedge \forall k \in \Natural\, \left(\, \dom{x} = k \cup \{k\} \Longrightarrow y = (x(k),a(\dom{x}))\, \right) \\
		&\wedge \left[\, \dom{x} \neq \emptyset \wedge \forall k \in \Natural\, \left(\, \dom{x} \neq k \cup \{k\}\, \right)
		\Longrightarrow y = \emptyset\, \right]\, \}
		\end{align}
		のことであるが,$\ON$上の写像$p$で
		\begin{align}
			p(n) =
			\begin{cases}
				a_0 & \mbox{if } (n = 0) \\
				(a_0,a_1) & \mbox{if } (n=1) \\
				((a_0,a_1),a_2) & \mbox{if } (n=2) \\
				(((a_0,a_1),a_2),a_3) & \mbox{if } (n=3)
			\end{cases}
		\end{align}
		を満たすものが取れる.先の
		\begin{align}
			(a_0,a_1,\cdots, a_n)
		\end{align}
		という一見不正確であった記法は,この
		\begin{align}
			p(n)
		\end{align}
		によって定めると決めてしまえば無事解決である.
	
	%\subsection{整礎集合}
	いま$\Univ$上の写像$G$を
	\begin{align}
		G(x) = 
		\begin{cases}
			\emptyset & (\operatorname{dom}(x) = \emptyset) \\
			x(\beta) \cup \operatorname{P}(x(\beta)) & (
			\exists \beta \in \ON\ (\ \operatorname{dom}(x) = \beta \cup \{\beta\}\ )) \\
			\bigcup \operatorname{ran}(x) & \mathrm{o.w.}
		\end{cases}
	\end{align}
	で定めると,定理\ref{thm:transfinite_recursion_theorem}より
	\begin{align}
		\forall \alpha \in \ON\ (\ R(\alpha) = G(R|_\alpha)\ )
	\end{align}
	を満たす$\ON$上の写像$R$が唯一つ存在する.以降しばらくはこの$R$が考察対象となる.
	
	\begin{screen}
		\begin{thm}
			\begin{align}\label{thm:R_alpha_plus_1_equals_to_power_of_R_alpha}
				\forall \alpha \in \ON\ 
				\left(\ R(\alpha + 1) = \operatorname{P}(R(\alpha))\ \right)
			\end{align}
		\end{thm}
	\end{screen}
	
	\begin{prf}\mbox{}
		\begin{description}
			\item[第一段] $R(\alpha + 1) = R(\alpha) \cup \operatorname{P}(R(\alpha))$
				となることを示す.
				
			\item[第二段] $\alpha$を任意に与えられた空でない順序数とするとき,
				\begin{align}
					\forall \beta \in \alpha\ 
					\left(\ R(\beta + 1) \subset \operatorname{P}(R(\beta))\ \right)
					\Longrightarrow R(\alpha + 1) \subset \operatorname{P}(R(\alpha))
				\end{align}
				が成り立つことを示す.いま
				\begin{align}
					\forall \beta \in \alpha\ 
					\left(\ R(\beta + 1) \subset \operatorname{P}(R(\beta))\ \right)
					\label{eq:thm_R_alpha_plus_1_equals_to_power_of_R_alpha}
				\end{align}
				が成り立つと仮定する.$x$を$R(\alpha + 1)$の任意の要素とすれば,前段の結果より
				\begin{align}
					x \in R(\alpha) \vee x \subset R(\alpha)
				\end{align}
				となる.$x \in R(\alpha)$であるとき,$\alpha$の或る要素$\beta$に対し
				$x \in R(\beta)$となる.前段の結果より$x \in R(\beta + 1)$となり,
				(\refeq{eq:thm_R_alpha_plus_1_equals_to_power_of_R_alpha})より
				$x \subset R(\beta)$となるが,
				\begin{align}
					x \subset R(\beta) &\Longrightarrow x \subset R(\alpha), \\
					x \subset R(\alpha) &\Longrightarrow x \in \operatorname{P}(R(\alpha))
				\end{align}
				と併せて$x \in \operatorname{P}(R(\alpha))$が成り立つ.
				一方で$x \subset R(\alpha)$であるときも$x \in \operatorname{P}(R(\alpha))$
				となるから
				\begin{align}
					R(\alpha + 1) \subset \operatorname{P}(R(\alpha))
				\end{align}
				が従う.超限帰納法より定理の主張が得られる.
		\end{description}
	\end{prf}
	
	\begin{screen}
		\begin{dfn}[整礎集合]
			$\bigcup_{\alpha \in \ON} R(\alpha)$の要素を{\bf 整礎集合}
			\index{せいそしゅうごう@整礎集合}{\bf (well-founded set)}と呼ぶ.
		\end{dfn}
	\end{screen}
	
	\begin{screen}
		\begin{thm}[すべての集合は整礎的である]\label{thm:every_set_is_well_founded}
			次は定理である:
			\begin{align}
				\Univ = \bigcup_{\alpha \in \ON} R(\alpha).
			\end{align}
		\end{thm}
	\end{screen}
	
	\begin{prf}
		いま,$S$を$\ON$の空でない部分集合として
		\begin{align}
			V \neq \bigcup_{\alpha \in S} R(\alpha)
			\Longrightarrow S \neq \ON
		\end{align}
		が成り立つことを示す.$V \neq \bigcup_{\alpha \in S} R(\alpha)$であれば
		正則性公理より或る集合$a$が存在して
		\begin{align}
			a \in V \backslash \bigcup_{\alpha \in S} R(\alpha)
			\wedge a \cap V \backslash \bigcup_{\alpha \in S} R(\alpha) = \emptyset
		\end{align}
		を満たす.このとき
		\begin{align}
			a \in \bigcup_{\alpha \in S} R(\alpha) \wedge a \subset \bigcup_{\alpha \in S} R(\alpha)
		\end{align}
		となる.ここで
		\begin{align}
			f = \Set{x}{\exists s \in a\ (\ x = (s,\mu \alpha (s \in R(\alpha)))\ )}
		\end{align}
		と定めれば$f:a \longrightarrow \ON$が成り立つ.
		$\beta = \bigcup f(a)$とおけば$\beta$は$\ON$に属する.このとき
		\begin{align}
			\forall t\ (\ t \in a \Longrightarrow t \in R(f(t))
			\Longrightarrow t \in R(\beta)\ )
		\end{align}
		となるから$a \subset R(\beta)$,そして定理\ref{thm:R_alpha_plus_1_equals_to_power_of_R_alpha}
		より$a \in R(\beta + 1)$が従う.
		\begin{align}
			\forall \alpha \in S\ (\ a \notin R(\alpha)\ )
		\end{align}
		であったから$\beta + 1 \notin S$であり,ゆえに$S \neq \ON$となる.
		定理の主張は対偶を取れば得られる.
		\QED
	\end{prf}
	
	\monologue{
		院生「\begin{align}
				\Univ = \bigcup_{\alpha \in \ON} R(\alpha)
			\end{align}
			という美しい式は偶然得られた訳ではありません.John Von Neumann はこの結果を
			予定して正則性公理を導入したのです.
			さて,超限帰納法による写像の構成を応用して
			次は順序数の足し算と掛け算を定義しましょう.」
	}
	
	\begin{screen}
		\begin{thm}[順序数の加法]\label{thm:the_definition_of_addition_of_ordinal_numbers}
			$\alpha$を$\ON$から任意に選ばれた順序数として,$\Univ$上の写像$G_\alpha$を
			\begin{align}
				G_\alpha(x) = 
				\begin{cases}
					\alpha & (\operatorname{dom}(x) = \emptyset) \\
					x(\beta) \cup \{x(\beta)\} & (
					\exists \beta \in \ON\, (\, \operatorname{dom}(x) = \beta \cup \{\beta\}\, )) \\
					\bigcup \operatorname{ran}(x) & \mathrm{o.w.}
				\end{cases}
			\end{align}
			で定めるとき,定理\ref{thm:transfinite_recursion_theorem}より
			\begin{align}
				\forall \beta \in \ON\, (\, A_\alpha(\beta) = G_\alpha(A_\alpha|_\beta)\, )
			\end{align}
			を満たす$\ON$上の写像$A_\alpha$が唯一つ存在する.ここで
			\begin{align}
				\alpha + \beta = A_\alpha (\beta)
			\end{align}
			と書くと,次が成立する:
			\begin{itemize}
				\item $\forall \alpha,\alpha' \in \ON\, \left(\, \alpha = \alpha' \Longrightarrow A_\alpha = A_{\alpha'}\, \right)$.
				\item $\forall \beta \in \ON\, (\, \alpha + \beta \in \ON\, )$.
				\item $\alpha \in {\bf \omega}$のとき,$\forall \beta \in {\bf \omega}\, (\, \alpha + \beta \in {\bf \omega}\, )$.
			\end{itemize}
		\end{thm}
	\end{screen}
	
	\begin{prf}
		いま$\beta$を任意に与えられた順序数とする.このとき,
		\begin{align}
			\forall \gamma \in \beta\ (\ \alpha + \gamma \in \ON\ )
		\end{align}
		が成り立っていると仮定すると,$\beta = \gamma + 1$と表せるとき
		\begin{align}
			\alpha + \beta 
			= G_\alpha (F_\alpha|_\beta)
			= F_\alpha(\gamma) + 1
			= (\alpha + \gamma) + 1 \in \ON
		\end{align}
		となり,$\beta$が極限数のときは
		\begin{align}
			\alpha + \beta = \operatorname*{sup}_{\gamma \in \beta} (\alpha + \gamma)
			= \bigcup \Set{\alpha + \gamma}{\gamma \in \beta}
			\in \ON
		\end{align}
		となるので,
		\begin{align}
			\forall \beta \in \ON\ \left(\ \forall \gamma \in \beta\ (\ \alpha + \gamma \in \ON\ ) \Longrightarrow \alpha + \beta \in \ON\ \right)
		\end{align}
		が得られた.超限帰納法により
		\begin{align}
			\forall \beta \in \ON\ (\ \alpha + \beta \in \ON\ )
		\end{align}
		が成立する.また$\alpha \in {\bf \omega}$のとき,
		\begin{align}
			a = \Set{\beta \in {\bf \omega}}{\alpha + \beta \in {\bf \omega}}
		\end{align}
		とおけば
		\begin{align}
			\emptyset \in a \wedge \forall x\ (\ x \in a \Longrightarrow x \cup \{x\} \in a\ )
		\end{align}
		となるので${\bf \omega} \subset a$が従う.よって
		\begin{align}
			\forall \beta \in {\bf \omega}\ 
			(\ \alpha + \beta \in {\bf \omega}\ )
		\end{align}
		も成り立つ.
		\QED
	\end{prf}
	
	\begin{screen}
		\begin{thm}[加法の性質]
		\label{thm:properties_of_addition_of_ordinal_numbers}
			定理\ref{thm:the_definition_of_addition_of_ordinal_numbers}で定めた
			加法は以下の性質を持つ:
			\begin{itemize}
				\item $\forall \alpha \in \ON\ (\ \alpha + 0 = 0 + \alpha = \alpha\ )$,
				
				\item $\forall \alpha \in \ON\ (\ \alpha + 1 = \alpha \cup \{\alpha\}\ )$,
				
				\item $\forall \alpha,\beta,\gamma \in \ON\ (\ (\alpha + \beta) + \gamma = \alpha + (\beta + \gamma)\ )$,
				
				\item $\forall \alpha,\beta \in {\bf \omega}\ (\ \alpha + \beta = \beta + \alpha\ )$,
				
				\item $\forall \alpha,\beta,\gamma \in \ON\ (\ \beta \in \gamma
					\Longrightarrow \alpha + \beta \in \alpha + \gamma\ )$,
				
				\item $\forall \alpha,\beta \in \beta\ (\ \alpha \in \beta
					\Longrightarrow \exists \gamma \in \ON\ (\ \alpha + \gamma = \beta\ )\ )$.
			\end{itemize}
		\end{thm}
	\end{screen}
	
	\begin{screen}
		\begin{thm}[順序数の乗法]
		\label{thm:the_definition_of_multiplication_of_ordinal_numbers}
			$\alpha$を$\ON$から任意に選ばれた順序数として,$\Univ$上の写像$G_\alpha$を
			\begin{align}
				G_\alpha(x) = 
				\begin{cases}
					0 & (\operatorname{dom}(x) = \emptyset) \\
					x(\beta) + \alpha & (
					\exists \beta \in \ON\ (\ \operatorname{dom}(x) = \beta \cup \{\beta\}\ )) \\
					\bigcup \operatorname{ran}(x) & \mathrm{o.w.}
				\end{cases}
			\end{align}
			で定めるとき,定理\ref{thm:transfinite_recursion_theorem}より
			\begin{align}
				\forall \beta \in \ON\ (\ M_\alpha(\beta) = G_\alpha(M_\alpha|_\beta)\ )
			\end{align}
			を満たす$\ON$上の写像$M_\alpha$が唯一つ存在する.ここで
			\begin{align}
				\alpha \cdot \beta = M_\alpha (\beta)
			\end{align}
			と書くと,次が成立する:
			\begin{itemize}
				\item $\forall \beta \in \ON\ (\ \alpha \cdot \beta \in \ON\ )$.
				\item $\alpha \in {\bf \omega}$のとき,$\forall \beta \in {\bf \omega}\ 
				(\ \alpha \cdot \beta \in {\bf \omega}\ )$.
			\end{itemize}
		\end{thm}
	\end{screen}
	%	
	\begin{screen}
		\begin{dfn}[選択関数]
			$a$を集合とするとき,
			\begin{align}
				f \fon a \wedge \forall x \in a\, (\, x \neq \emptyset \Longrightarrow f(x) \in x\, )
			\end{align}
			を満たす写像$f$を$a$上の{\bf 選択関数}\index{せんたくかんすう@選択関数}{\bf (choice function)}と呼ぶ.
		\end{dfn}
	\end{screen}
	
	$a = \emptyset$ならば空写像が$a$上の選択関数となる.
	選択公理とは,空集合に限らずどの集合の上にも選択関数が存在することを保証する.
	
	\begin{screen}
		\begin{axm}[選択公理]
			いかなる集合の上にも選択関数が存在する:
			\begin{align}
				\forall a\, \exists f\ \left[\ 
				f \fon a \wedge \forall x \in a\ 
				(\ x \neq \emptyset \Longrightarrow f(x) \in x\ )\ \right]. 
			\end{align}
		\end{axm}
	\end{screen}
	
	後述する直積は選択公理と密接な関係がある.いま$a$を集合として
	\begin{align}
		b \defeq a \backslash \{\emptyset\}
	\end{align}
	とおき,$h$を$b$上の恒等写像とすると
	\begin{align}
		\forall x \in b\, (\, h(x) \neq \emptyset\, )
	\end{align}
	が成り立つが,ここで
	\begin{align}
		\forall x \in b\, (\, g(x) \in h(x)\, )
	\end{align}
	を満たす$b$上の写像$g$が取れるとする.この主張は``$h$の直積は空ではない''という意味なのだが,このとき
	\begin{align}
		f \defeq 
		\begin{cases}
			g \cup \{(\emptyset,\emptyset)\} & \mbox{if } \emptyset \in a \\
			g & \mbox{if } \emptyset \notin a
		\end{cases}
	\end{align}
	により$a$上の写像$f$を定めれば
	\begin{align}
		\forall x \in a\, (\, x \neq \emptyset \Longrightarrow f(x) \in x\, )
	\end{align}
	が成立する.つまり空な値を取らない写像の直積は空でないという主張を真とすれば選択公理が導かれる.
	本稿では選択公理はその名の通り公理であるから上の内容は無意味であるが,今度は
	{\bf 空な値を取らない写像の直積は空でない}という主張が定理として得られることになる.
	
	\begin{screen}
		\begin{dfn}[直積]
			$a$を類とし,$h$を$a$上の写像とするとき,
			\begin{align}
				\prod_{x \in a} h(x) \defeq
				\Set{f}{f \fon a \wedge \forall x \in a\, (\, f(x) \in h(x)\, )} 
			\end{align}
			で定める類を$h$の{\bf 直積}\index{ちょくせき@直積}{\bf (direct product)}と呼ぶ.
		\end{dfn}
	\end{screen}
	
	$a$を集合とし,$h$を$a$上の写像とし,
	\begin{align}
		h(x) = \emptyset
	\end{align}
	なる$a$の要素$x$が取れるとする.このとき
	\begin{align}
		\prod_{x \in a} h(x) = \emptyset
	\end{align}
	が成立する.実際,$a$上の任意の写像$f$に対して必ず
	\begin{align}
		f(x) \notin h(x)
	\end{align}
	が成り立つので
	\begin{align}
		f \notin \prod_{x \in a} h(x)
	\end{align}
	が成立する.つまり任意の集合$a$及び$a$上の写像$h$に対して
	\begin{align}
		\exists x \in a\, (\, h(x) = \emptyset\, )
		\Longrightarrow \prod_{x \in a} h(x) = \emptyset
	\end{align}
	が成り立つわけだが,選択公理から演繹すればこの逆の主張も得られる.
	
	\begin{screen}
		\begin{thm}[空な値を取らない写像の直積は空でない]
		\label{thm:direct_product_of_non_empty_sets_is_not_empty}
			\begin{align}
			\forall a\, \forall h\, \left(\, h \fon a \wedge \forall x \in a\, (\, h(x) \neq \emptyset\, )
				\Longrightarrow \prod_{x \in a} h(x) \neq \emptyset\, \right).
			\end{align}
		\end{thm}
	\end{screen}
	
	\begin{sketch}
		いま$a$を集合とし,$h$を$a$上の写像とし,
		\begin{align}
			\forall x \in a\, (\, h(x) \neq \emptyset\, )
			\label{fom:thm_direct_product_of_non_empty_sets_is_not_empty_1}
		\end{align}
		が成り立っているとする.このとき
		\begin{align}
			b \defeq h \ast a
		\end{align}
		とおけば,選択公理より$b$上の写像$g$で
		\begin{align}
			\forall t \in b\, \left(\, t \neq \emptyset \Longrightarrow g(t) \in t\, \right)
			\label{fom:thm_direct_product_of_non_empty_sets_is_not_empty_2}
		\end{align}
		を満たすものが取れる.ここで
		\begin{align}
			f \defeq \Set{t}{\exists x \in a\, \left[\, t = (x,g(h(x)))\, \right]}
		\end{align}
		と定めれば,$f$は$a$上の写像であって
		\begin{align}
			\forall x \in a\, \left(\, f(x) \in h(x)\, \right)
		\end{align}
		が成立する.実際,任意の集合$s,t,u$に対して
		\begin{align}
			(s,t) \in f \wedge (s,u) \in f
		\end{align}
		であるとすれば
		\begin{align}
			(s,t) = (x,g(h(x)))
		\end{align}
		を満たす$a$の要素$x$と
		\begin{align}
			(s,u) = (y,g(h(y)))
		\end{align}
		を満たす$a$の要素$y$が取れるが,このとき
		\begin{align}
			x = s = y
		\end{align}
		が成り立つので
		\begin{align}
			t = g(h(x)) = g(h(y)) = u
		\end{align}
		が従う.また$x$を$a$の要素とすれば
		\begin{align}
			(x,g(h(x))) \in f
		\end{align}
		が成り立つので
		\begin{align}
			x \in \dom{f}
		\end{align}
		となり,逆に$x$を$\dom{f}$の要素とすれば
		\begin{align}
			(x,y) \in f
		\end{align}
		を満たす集合$y$が取れるが,このとき
		\begin{align}
			(x,y) = (z,g(h(z)))
		\end{align}
		を満たす$a$の要素$z$が取れるので
		\begin{align}
			x \in a
		\end{align}
		が従う.ゆえに$f$は$a$上の写像である.そして$x$を$a$の要素とすれば
		\begin{align}
			f(x) = g(h(x))
		\end{align}
		が成り立つが,(\refeq{fom:thm_direct_product_of_non_empty_sets_is_not_empty_1})より
		\begin{align}
			h(x) \neq \emptyset
		\end{align}
		が満たされるので,(\refeq{fom:thm_direct_product_of_non_empty_sets_is_not_empty_2})より
		\begin{align}
			f(x) \in h(x)
		\end{align}
		が成立する.
		\QED
	\end{sketch}
	
	\monologue{
		整列可能定理の証明は幾分技巧的で見通しが悪いですから,はじめに直感的な解説をしておきます.
		定理の主張は集合$a$に対して順序数$\alpha$と写像$f$で
		\begin{align}
			f:\alpha \bij a
		\end{align}
		を満たすものが取れるというものです.順序数は
		\begin{align}
			0,1,2,3,\cdots
		\end{align}
		と順番に並んでいますから,まず$0$に対して$a$の何らかの要素$x_0$を対応させます.
		次は$1$に対して$a \backslash \{x_0\}$の何らかの要素$x_1$を対応させ,
		その次は$2$に対して$a \backslash \{x_0,x_1\}$の要素を対応させ...と,同様の操作を
		$a$の要素が尽きるまで繰り返します.操作が終了した時点で,それまでに使われなかった順序数のうちで
		最小のものを$\alpha$とすれば,写像
		\begin{align}
			f:\alpha \ni \beta \longmapsto x_\beta \in a
		\end{align}
		が得られるという寸法です.`$a \backslash \{\cdots\}$の何らかの要素を対応させる'
		という不明瞭な操作を$\mathcal{L}'$のことばで表現する際に選択公理が使われますから
		整列可能定理は選択公理から導かれると言えますが,
		逆に整列可能定理が真であると仮定すれば選択公理の主張が導かれます.
		つまり(\ref{sec:logic_and_set_theory}節で登場した公理体系の下で)
		選択公理と整列可能定理は同値な主張となります.
	}
	
	
	\begin{screen}
		\begin{thm}[整列可能定理]\label{thm:well_ordering_theorem}
			任意の集合は,或る順序数との間に全単射を持つ:
			\begin{align}
				\forall a\ \exists \alpha \in \ON\ 
				\exists f\, \left(\, f:\alpha \bij a\, \right).
			\end{align}
		\end{thm}
	\end{screen}
	
	次の主張は整列可能定理と証明が殆ど被るのでまとめて述べておく.
	\begin{screen}
		\begin{thm}[整列集合は唯一つの順序数に順序同型である]\label{thm:existence_of_order_type}
			$(a,O_W)$を整列集合とするとき,或るただ一つの順序数$\alpha$と
			$\alpha$から$a$への全単射$f$が存在して
			\begin{align}
				\gamma \leq \delta \Longrightarrow (f(\gamma),f(\delta)) \in O_W
			\end{align}
			を満たす.
		\end{thm}
	\end{screen}
	
	\begin{prf} $\chi$を任意に与えられた$\mathcal{L}$の対象とする.
		\begin{description}
			\item[第一段]
				$\chi = \emptyset$の場合,
				\begin{align}
					\emptyset: \emptyset \bij \chi
				\end{align}
				が満たされるから
				\begin{align}
					\exists f\, \left(\, f:\emptyset \bij \chi\, \right)
				\end{align}
				が成立し,$\emptyset$は$\ON$の要素であるから
				\begin{align}
					\exists \alpha \in \ON\, \exists f\, \left(\, f:\alpha \bij \chi\, \right)
				\end{align}
				が従う.以上より
				\begin{align}
					\chi = \emptyset \Longrightarrow \exists \alpha \in \ON\, 
					\exists f\, \left(\, f:\alpha \bij \chi\, \right)
				\end{align}
				が成り立つ.
				
			\item[第二段]
				$\chi \neq \emptyset$の場合,
				\begin{align}
					P \coloneqq \dirpro{\chi} \backslash \{\chi\}
				\end{align}
				とおけば
				\begin{align}
					\forall p \in P\, (\, \chi \backslash p \neq \emptyset\, )
				\end{align}
				が満たされるので,選択公理より
				\begin{align}
					g \fon P \wedge \forall p \in P\, (\, g(p) \in \chi \backslash p\, ) 
				\end{align}
				を満たす写像$g$が存在する.
				\begin{align}
					G \coloneqq \Set{z}{\exists s\, \left(\, 
						\left(\, \ran{s} \in P \Longrightarrow z=(s,g(\ran{s}))\, \right) 
						\wedge \left(\, \ran{s} \notin P \Longrightarrow z=(s,\emptyset)\, \right)\, \right)}
				\end{align}
				で$\Univ$上の写像$G$を定めれば
				\begin{align}
					\forall \alpha \in \ON\, \left(\, F(\alpha) = G(F|_\alpha)\, \right)
				\end{align}
				を満たす類$F$が存在して,$G$の定め方より
				\begin{align}
					\alpha \in \ON \Longrightarrow F(\alpha) = 
					\begin{cases}
						g(F \ast \alpha) & (F \ast \alpha \subsetneq \chi) \\
						\emptyset & (F \ast \alpha = a \vee F \ast \alpha \not\subset \chi)
					\end{cases}
				\end{align}
				が成立する.
				\begin{align}
					\forall \alpha \in \ON\, \left(\, 
					F \ast \alpha \subsetneq \chi \Longrightarrow g(F \ast \alpha) \in \chi\, \right)
				\end{align}
				が満たさるので
				\begin{align}
					F:\ON \longrightarrow \chi \cup \{\emptyset\}
				\end{align}
				が成立することに注意しておく.以下,適当な順序数$\gamma$を選べば
				\begin{align}
					F|_\gamma
				\end{align}
				が$\gamma$から$\chi$への全単射となることを示す.
			
			\item[第三段]
				$S$を類とするとき
				\begin{align}
					\ord{S} \wedge \forall \alpha \in S\, \left(\, F \ast \alpha \neq \chi\, \right)
					\Longrightarrow \set{F \ast S} \wedge
					F|_S:S \bij F \ast S \wedge \set{S}
					\label{eq:thm_well_ordering_theorem_1}
				\end{align}
				が成り立つことを示す.いま
				\begin{align}
					\ord{S} \wedge \forall \alpha \in S\, \left(\, F \ast \alpha \neq \chi\, \right)
				\end{align}
				が成り立っているとする.このとき
				\begin{align}
					F(\emptyset) = g(\emptyset) \in \chi
				\end{align}
				が成立し,また$\alpha$を任意の順序数とすれば,
				\begin{align}
					\forall \beta \in \alpha\, \left(\, \beta \in S \Longrightarrow F(\beta) \in \chi\, \right)
				\end{align}
				が満たされているとき
				\begin{align}
					\alpha \in S &\Longrightarrow \alpha \subset S \\
					&\Longrightarrow \forall \beta \in \alpha\, (\, \beta \in S\, ) \\
					&\Longrightarrow \forall \beta \in \alpha\, (\, F(\beta) \in \chi\, ) \\
					&\Longrightarrow F \ast \alpha \subset \chi, \\
					\alpha \in S &\Longrightarrow F \ast \alpha \neq \chi
				\end{align}
				より
				\begin{align}
					\alpha \in S \Longrightarrow F(\alpha) \in \chi
				\end{align}
				が成立する.よって超限帰納法より
				\begin{align}
					\forall \alpha \in S\, (\, F(\alpha) \in \chi\, )
				\end{align}
				となる.従って
				\begin{align}
					F \ast S \subset \chi
				\end{align}
				が得られる.そして$\chi$は集合であるから
				\begin{align}
					\set{F \ast S}
				\end{align}
				が出る.次に$F|_S$が単射であることを示す.$\ord{S}$から
				\begin{align}
					S \subset \ON
				\end{align}
				が満たされるので,$\beta,\gamma \in S$に対して$\beta \neq \gamma$ならば
				\begin{align}
					\beta \in \gamma \vee \gamma \in \beta
				\end{align}
				が成り立つ.$\beta \in \gamma$の場合
				\begin{align}
					F(\gamma) = g(F \ast \gamma) \in \chi \backslash (F \ast \gamma)
				\end{align}
				が成り立つので
				\begin{align}
					F(\gamma) \notin F \ast \gamma
				\end{align}
				が従う.他方で
				\begin{align}
					F(\beta) \in F \ast \gamma
				\end{align}
				が満たされるので
				\begin{align}
					F(\gamma) \neq F(\beta)
				\end{align}
				が満たされる.よって
				\begin{align}
					\beta \in \gamma \Longrightarrow F(\gamma) \neq F(\beta)
				\end{align}
				が成立する.$\beta$と$\gamma$を入れ替えれば
				\begin{align}
					\gamma \in \beta \Longrightarrow F(\gamma) \neq F(\beta)
				\end{align}
				も得られるので,場合分け法則より
				\begin{align}
					\beta \neq \gamma \Longrightarrow F(\gamma) \neq F(\beta)
				\end{align}
				が成り立つ.よって$F|_S$は単射である.このとき
				\begin{align}
					F|_S:S \bij F \ast S
				\end{align}
				となり
				\begin{align}
					S = {F|_S}^{-1}(F \ast S)
				\end{align}
				が成り立つので,置換公理より
				\begin{align}
					\set{S}
				\end{align}
				が出る.
				
			\item[第四段]
				Burali-Fortiの定理より
				\begin{align}
					\rightharpoondown \set{\ON}
				\end{align}
				が成り立つので,式(\refeq{eq:thm_well_ordering_theorem_1})の対偶から
				\begin{align}
					\rightharpoondown \ord{\ON} \vee 
					\exists \alpha \in \ON\, \left(\, F \ast \alpha = \chi\, \right)
				\end{align}
				が従う.一方で
				\begin{align}
					\ord{\ON}
				\end{align}
				は正しいので,選言三段論法より
				\begin{align}
					\exists \alpha \in \ON\, \left(\, F \ast \alpha = \chi\, \right)
				\end{align}
				が成立する.$\gamma$を
				\begin{align}
					F \ast \alpha = \chi
				\end{align}
				を満たす順序数$\alpha$のうちで最小のものとすれば,式(\refeq{eq:thm_well_ordering_theorem_1})より
				\begin{align}
					F|_\gamma:\gamma \bij \chi
				\end{align}
				が成り立つので,
				\begin{align}
					\chi \neq \emptyset \Longrightarrow \exists \alpha \in \ON\ 
					\exists f\, \left(\, f:\alpha \bij \chi\, \right)
				\end{align}
				も得られた.場合分け法則より
				\begin{align}
					\chi = \emptyset \vee \chi \neq \emptyset \Longrightarrow \exists \alpha \in \ON\ 
					\exists f\, \left(\, f:\alpha \bij \chi\, \right)
				\end{align}
				が成立し,排中律から
				\begin{align}
					\exists f\, \left(\, f:\alpha \bij \chi\, \right)
				\end{align}
				は真となる.そして$\chi$の任意性より定理の主張が出る.
				\QED
		\end{description}
	\end{prf}
	
	\monologue{
		整列定理により{\bf いかなる集合の上にも整列順序が定められます}.実際,$a$を集合として
		\begin{align}
			g \coloneqq \varepsilon f\ \left( f:\alpha \bij a \right)
		\end{align}
		とおき,
		\begin{align}
			R \coloneqq \Set{x}{\exists s,t \in a\, \left(\, g^{-1}(s) \subset g^{-1}(t) \wedge x = (s,t)\, \right)}
		\end{align}
		で$a$上の関係を定めれば,$R$は$a$上の整列順序となります.まさしく`整列可能'なのですね.
	}
	%
	\begin{screen}
		\begin{axm}[選択公理]
			$a$を類とするとき,$a$上の写像$f$で,
			$a$の空でない要素$x$から$f(x)$を選択するもの
			(これを{\bf 選択関数}\index{せんたくかんすう@選択関数}{\bf (choice function)}と呼ぶ)
			が存在する:
			\begin{align}
				\exists f\ \left(\ 
				f:a \longrightarrow V \wedge \forall x \in a\ 
				(\ x \neq \emptyset \Longrightarrow f(x) \in x\ )\ \right). 
			\end{align}
		\end{axm}
	\end{screen}
	
	\begin{screen}
		\begin{thm}[整列可能定理]
			任意の集合は,或る順序数と全単射で結ばれる:
			\begin{align}
				\forall a\ \exists \alpha \in \ON\ 
				\exists f\ \left( f:\alpha \bij a \right).
			\end{align}
		\end{thm}
	\end{screen}
	
	\begin{screen}
		\begin{dfn}[有限・可算・無限]
			
		\end{dfn}
	\end{screen}
	
	\begin{screen}
		\begin{thm}[任意の無限集合は可算集合を含む]
			\begin{align}
				\forall a\ \left(\ \exists \alpha \in \ON \backslash {\bf \omega}\ (\  \alpha \simeq a\ )
				\Longrightarrow \exists b\ (\ b \subset a \wedge {\bf \omega} \simeq b\ )\ \right).
			\end{align}
		\end{thm}
	\end{screen}
	%\section{アレフ数}
	有限基数を抜いた基数の全体を
	\begin{align}
		\InfCN \defeq \CN \backslash \Natural
	\end{align}
	とおく.`I'を付けたのは,これが無限濃度の全体を表しているからである.
	$\InfCN$は$\ON$の部分集合であるが,$\ON$から$\InfCN$への順序同型写像を取ることが出来る.
	それは
	\begin{align}
		\aleph
	\end{align}
	と書かれ,アレフ数と呼ばれる. 
	
	\begin{screen}
		\begin{dfn}[無限基数割り当て写像$\aleph$]
			$\Univ$上の写像$G$を
			\begin{align}
				G \defeq \Set{x}{\exists s\, \exists \alpha \in \ON\, 
				\left[\, x=(s,\alpha) \wedge \alpha \in \InfCN \backslash \ran{s} \wedge
				\forall \beta \in \ON\, \left(\, \beta \in \InfCN \backslash \ran{s}
				\Longrightarrow \beta \leq \alpha\, \right)\, \right]}
			\end{align}
			で定めるとき,超限帰納法による写像の構成から
			\begin{align}
				\forall \beta \in \ON\, 
				\left(\, \aleph(\beta) = \mu \alpha\, (\, \alpha \in \InfCN \backslash \aleph \ast \beta\, )\, \right)
			\end{align}
			を満たす$\ON$上の写像$\aleph$が取れる.$\alpha$を順序数とするとき,
			\begin{align}
				\aleph(\alpha)
			\end{align}
			のことを
			\begin{align}
				\aleph_\alpha
			\end{align}
			と書く.
		\end{dfn}
	\end{screen}
	
	次の定理は{\bf 異なる無限がいくらでも存在する}ことを主張している.
	
	\begin{screen}
		\begin{thm}[$\aleph$は$\ON$と順序同型]
			$\aleph$は$\ON$から$\InfCN$への順序同型となる.つまり
			\begin{align}
				\aleph:\ON \bij \InfCN \wedge \forall \gamma, \delta \in \ON\, \left(\, \gamma < \delta
				\Longrightarrow \aleph_\gamma < \aleph_\delta\, \right).
			\end{align}
		\end{thm}
	\end{screen}
	
	\begin{sketch}
		いま$\gamma,\delta$を$\ON$の要素として
		\begin{align}
			\gamma < \delta
		\end{align}
		であると仮定する.このとき
		\begin{align}
			F \ast \gamma \subset F \ast \delta
		\end{align}
		かつ
		\begin{align}
			F(\delta) \in \InfCN \backslash F \ast \delta
		\end{align}
		が満たされるので
		\begin{align}
			F(\delta) \in \InfCN \backslash F \ast \gamma
		\end{align}
		が成立する.従って
		\begin{align}
			F(\gamma) \leq F(\delta)
		\end{align}
		が成立する.一方で
		\begin{align}
			F(\gamma) \in F \ast \delta \wedge
			F(\delta) \in \InfCN \backslash F \ast \delta
		\end{align}
		から
		\begin{align}
			F(\gamma) \neq F(\delta)
		\end{align}
		も満たされるので
		\begin{align}
			F(\gamma) < F(\delta)
		\end{align}
		が従う.以上より
		\begin{align}
			\forall \gamma, \delta \in \ON\, (\, \gamma < \delta \Longrightarrow F(\gamma) < F(\delta)\, )
		\end{align}
		が得られる.またこの結果より$F$が単射であることも従う.
	\end{sketch}
	
	\begin{screen}
		\begin{thm}
			$\alpha$が極限数であるならば
			\begin{align}
				\bigcup_{\beta \in \alpha} \aleph_{\beta} = \aleph_{\alpha}.
			\end{align}
		\end{thm}
	\end{screen}
	
	\begin{sketch}
		$\gamma$を
		\begin{align}
			\Natural \leq \gamma < \aleph_{\alpha}
		\end{align}
		を満たす順序数とすると
		\begin{align}
			\card{\gamma} = \aleph_{\beta}
		\end{align}
		を満たす$\alpha$の要素$\beta$が取れる.このとき
		\begin{align}
			\card{\gamma} < \aleph_{\beta+1}
		\end{align}
		であるから
		\begin{align}
			\gamma < \aleph_{\beta+1}
		\end{align}
		が成り立つ.なぜならば,任意の基数$\delta$に対して
		\begin{align}
			\delta \leq \gamma \Longrightarrow 
			\delta = \card{\delta} \leq \card{\gamma}
		\end{align}
		が成り立つからである.以上より
		\begin{align}
			\gamma < \aleph_{\beta+1} < \aleph_{\alpha}
		\end{align}
		が従う.ゆえに
		\begin{align}
			\bigcup_{\beta \in \alpha} \aleph_{\beta} = \aleph_{\alpha}
		\end{align}
		を得た.
		\QED
	\end{sketch}
	
	\begin{screen}
		\begin{thm}[順序数がそのアレフ数を超えることはない]
		\label{thm:no_ordinal_number_is_bigger_than_its_aleph_number}
			任意の順序数$\alpha$に対して
			\begin{align}
				\alpha \leq \aleph_{\alpha}.
			\end{align}
		\end{thm}
	\end{screen}
	
	\begin{sketch}
		超限帰納法により示す.まず
		\begin{align}
			0 \leq \aleph_{0}.
		\end{align}
		次に$\alpha$を$0$でない順序数とし,$\alpha$の任意の要素$\beta$に対して
		\begin{align}
			\beta \leq \aleph_{\beta}
		\end{align}
		が成り立っているとする.
		\begin{align}
			\alpha = \beta + 1
		\end{align}
		を満たす順序数$\beta$が取れるとき,
		\begin{align}
			\beta < \aleph_{\alpha}
		\end{align}
		より
		\begin{align}
			\alpha = \beta + 1 \leq \aleph_{\alpha}
		\end{align}
		が従う.$\alpha$が極限数であるとき,
		\begin{align}
			\alpha = \bigcup_{\beta \in \alpha} \beta
			\leq \bigcup_{\beta \in \alpha} \aleph_{\beta}
			= \aleph_{\alpha}
		\end{align}
		が成立する.ゆえに超限帰納法より,任意の順序数$\alpha$に対して
		\begin{align}
			\alpha \leq \aleph_{\alpha}
		\end{align}
		が成立する.
		\QED
	\end{sketch}
	
	\begin{screen}
		\begin{thm}[無限順序数は後続数と濃度が等しい]
			\begin{align}
				\forall \alpha \in \ON\, (\,
				\Natural \leq \alpha \rarrow \card{\alpha} = \card{(\alpha + 1)}\, ).
			\end{align}
		\end{thm}
	\end{screen}
	
	\begin{sketch}
		$\alpha + 1$から$\alpha$への全単射を
		\begin{align}
			\alpha + 1 \ni x \longmapsto
			\begin{cases}
				0 & \mbox{if } x = \alpha, \\
				1 + x & \mbox{if } x < \Natural, \\
				x & \mbox{if } x \neq \alpha \wedge \Natural \leq x
			\end{cases}
		\end{align}
		によって定めることが出来るので
		\begin{align}
			\card{\alpha} = \card{(\alpha + 1)}
		\end{align}
		となる.
		\QED
	\end{sketch}
	
	\begin{screen}
		\begin{thm}[無限基数は極限数]
			\begin{align}
				\forall \alpha \in \ON\, \limo{\aleph_{\alpha}}.
			\end{align}
		\end{thm}
	\end{screen}
	
	\begin{sketch}
		$\beta$を$\aleph_{\alpha}$の要素とすれば,
		\begin{align}
			\beta < \Natural
		\end{align}
		ならば
		\begin{align}
			\beta < \beta + 1 < \Natural \leq \aleph_{\alpha}
		\end{align}
		となる.他方で
		\begin{align}
			\Natural \leq \beta
		\end{align}
		ならば
		\begin{align}
			\beta + 1 \leq \aleph_{\alpha}
		\end{align}
		かつ
		\begin{align}
			\card{\beta} = \card{(\beta + 1)} \leq \beta < \aleph_{\alpha}
		\end{align}
		となるので
		\begin{align}
			\beta + 1 < \aleph_{\alpha}
		\end{align}
		となる.以上で
		\begin{align}
			\forall \beta \in \aleph_{\alpha}\, (\, \beta + 1 < \aleph_{\alpha}\, )
		\end{align}
		が得られ
		\begin{align}
			\limo{\aleph_{\alpha}}
		\end{align}
		が出る.
		\QED
	\end{sketch}
	%\section{極大原理}
	いま,$P$が集合で,$O$が$P$上の順序関係であるとする.$c$が$P$の部分集合で
	\begin{align}
		\forall x,y \in c\, (\, (x,y) \in O \vee (y,x) \in O\, )
	\end{align}
	を満たすとき,つまり$c$のどの$2$要素も比較可能であるということだが,
	$c$を順序集合$(P,O)$の{\bf 鎖}\index{さ@鎖}{\bf (chain)}と呼ぶ.
	ここで$(P,O)$の鎖の全体を
	\begin{align}
		\mathscr{C}
	\end{align}
	とする.つまり$\mathscr{C}$とは
	\begin{align}
		\mathscr{C} \defeq \Set{c}{c \subset P \wedge \forall x,y \in c\, (\, (x,y) \in O \vee (y,x) \in O\, )}
	\end{align}	
	で定められた集合である.$\mathscr{C}$の要素$c$で
	\begin{align}
		\forall u \in \mathscr{C}\, (\, c \subset u \Longrightarrow c=u\, )
	\end{align}
	を満たすものを$(P,O)$の{\bf 極大鎖}\index{きょくだいさ@極大鎖}{\bf (maximal chain)}と呼ぶ.
	つまり,$c$が$(P,O)$の極大鎖であるとは$c$を含む鎖が$c$の他に無いということである.
	
	\begin{screen}
		\begin{thm}[極大鎖は必ず存在する]
		\label{thm:existence_of_maximal_chain}
			$P$を集合とし,$O$を$P$上の順序関係とするとき,$(P,O)$の極大鎖を取ることが出来る.
		\end{thm}
	\end{screen}
	
	$P$が空ならば$O$は当然空であるから(空虚な真により$\emptyset$は$P$上の順序関係である),
	\begin{align}
		\mathscr{C} = \{\emptyset\}
	\end{align}
	が成り立つ.従って$(P,O)$の極大鎖は$\emptyset$ということになる.以下では$P$は空でないとする.
	つまり$P$の要素$x$が取れるわけだが,
	\begin{align}
		\{x\}
	\end{align}
	はこれで一つの鎖をなすので$\mathscr{C}$は空ではない.
	
	\begin{sketch}\mbox{}
		\begin{description}
			\item[第一段]
				始めに全体のあらすじを書いておく.$\mathscr{C}$上の写像$h$を
				\begin{align}
					\mathscr{C} \ni c \longmapsto
					\begin{cases}
						\Set{u \in \mathscr{C}}{ c \subset u \wedge c \neq u} 
						& \mbox{if } \exists u \in \mathscr{C}\, (\, c \subset u \wedge c \neq u\, ) \\
						\{c\} &\mbox{if } \forall u \in \mathscr{C}\, (\, \rightharpoondown c \subset u \vee c=u\, )
					\end{cases}
				\end{align}
				なる関係により定めると,定理(\ref{thm:direct_product_of_non_empty_sets_is_not_empty})より
				$\mathscr{C}$上の写像$f$で,$\mathscr{C}$の任意の要素$c$に対して
				\begin{align}
					\begin{cases}
						c \subset f(c) \wedge c \neq f(c) 
						& \mbox{if } \exists u \in \mathscr{C}\, (\, c \subset u \wedge c \neq u\, ) \\
						c=f(c) & \mbox{if } \forall u \in \mathscr{C}\, (\, \rightharpoondown c \subset u \vee c=u\, )
					\end{cases}
				\end{align}
				を満たすものが取れる.ここで$\mathscr{C}$から要素$c$を選び,また$\Univ$の上の写像$G$を
				\begin{align}
					x \longmapsto
					\begin{cases}
						c & \mbox{if } \dom{x} = \emptyset \\
						f(x(\beta)) & \mbox{if } \beta \in \ON \wedge \dom{x} = \beta+1 \wedge x(\beta) \in \mathscr{C} \\
						\bigcup \ran{x} & \mbox{otherwise}
					\end{cases}
				\end{align}
				なる関係により定めると,$\ON$上の写像$F$で
				\begin{itemize}
					\item $F(0) = c$
					\item 任意の順序数$\alpha$に対して,$F(\alpha) \in \mathscr{C}$ならば$F(\alpha+1)=f(F(\alpha))$
					\item $\alpha$が極限数ならば$F(\alpha)=\bigcup_{\beta \in \alpha} F(\beta)$
				\end{itemize}
				を満たすものが取れる.以降はこの$F$をメインに考察する.
				
				まず,次の第二段では,任意の順序数$\alpha$に対して$F(\alpha)$が$(P,O)$の鎖であること,つまり
				\begin{align}
					\forall \alpha \in \ON\, (\, F(\alpha) \in \mathscr{C}\, )
				\end{align}
				が成り立つことを示す.
				
				第三段では,$\alpha$を極限数としたときに,$F(\alpha)$が極大鎖でないならば
				\begin{align}
					\card{\alpha} \leq \card{F(\alpha)}
				\end{align}
				が成り立つことを示す.
				
				第四段では,任意の順序数$\beta$に対して,
				\begin{align}
					\card{\beta} \leq \card{F(\beta)}
				\end{align}
				が満たされていてかつ$F(\beta)$が極大鎖ではないときに
				\begin{align}
					\card{(\beta+1)} \leq \card{F(\beta+1)}
				\end{align}
				が成り立つこと,つまり
				\begin{align}
					\card{\beta} \leq \card{F(\beta)} \wedge
					\exists u \in \mathscr{C}\, \left(\, F(\beta) \subset u \wedge F(\beta) \neq u\, \right)
					\Longrightarrow \card{(\beta+1)} \leq \card{F(\beta+1)}
				\end{align}
				が成り立つことを示す.
				
				以上を踏まえて,いま
				\begin{align}
					\delta \defeq \card{P}
				\end{align}
				とおけば,定理\ref{thm:no_ordinal_number_is_bigger_than_its_aleph_number}より
				\begin{align}
					\card{F(\aleph_{\delta+1})} \leq \card{P}
					=\delta < \aleph_{\delta+1}
				\end{align}
				が成り立つから,
				\begin{align}
					\card{F(\alpha)} < \card{\alpha}
				\end{align}
				を満たす最小の順序数$\alpha$を取ることが出来る.
				\begin{align}
					0 \leq \card{F(0)}
				\end{align}
				なので$\alpha$は$0$ではない.
				$\alpha$が極限数であるとき,第三段の対偶より$F(\alpha)$は極大鎖である.
				$\alpha$が極限数でないとき,つまり
				\begin{align}
					\alpha = \beta+1
				\end{align}
				を満たす順序数$\beta$が取れるとき,$\alpha$の取り方より
				\begin{align}
					\card{\beta} \leq \card{F(\beta)}
				\end{align}
				が成り立つので,第四段の対偶と選言三段論法(P. \pageref{logicalthm:disjunctive_syllogism})より
				\begin{align}
					\rightharpoondown \exists u \in \mathscr{C}\, (\, F(\beta) \subset u \wedge F(\beta) \neq u\, )
				\end{align}
				が従う.つまり$F(\beta)$は$(P,O)$の極大鎖である.以上より,いずれの場合も極大鎖は取れる.
			
			\item[第二段]
				 始めに
				 \begin{align}
					\forall \beta,\gamma \in \ON\,
					\left(\, \beta < \gamma \wedge \forall \delta \in \gamma\, (\, F(\delta) \in \mathscr{C}\, )
					\Longrightarrow F(\beta) \subset F(\gamma)\, \right)
				\end{align}
				が成り立つことを示す.これは,$\beta$を任意の順序数として
				\begin{align}
					\forall \gamma \in \ON\, \left(\, \beta < \gamma \wedge
					\forall \delta \in \gamma\, (\, F(\delta) \in \mathscr{C}\, )
					\Longrightarrow F(\beta) \subset F(\gamma)\, \right)
				\end{align}
				が成り立つことを超限帰納法で示せばよい.
				\begin{align}
					\gamma = 0
				\end{align}
				の場合は
				\begin{align}
					\beta < \gamma \wedge \forall \delta \in \gamma\, (\, F(\delta) \in \mathscr{C}\, )
				\end{align}
				の仮定が偽になるので
				\begin{align}
					\beta < \gamma \wedge \forall \delta \in \gamma\, (\, F(\delta) \in \mathscr{C}\, )
					\Longrightarrow F(\beta) \subset F(\gamma)
				\end{align}
				は成立する.次に$\gamma$を$0$でない順序数として,$\gamma$の任意の要素$\rho$に対して
				\begin{align}
					\beta < \rho \wedge \forall \delta \in \rho\, (\, F(\delta) \in \mathscr{C}\, )
					\Longrightarrow F(\beta) \subset F(\rho)
					\label{fom:thm_existence_of_maximal_chain_1}
				\end{align}
				が成り立っているとし,この下で
				\begin{align}
					\beta < \gamma \wedge \forall \delta \in \gamma\, (\, F(\delta) \in \mathscr{C}\, )
				\end{align}
				が満たされているとする.
				\begin{align}
					\gamma = \rho + 1
				\end{align}
				を満たす順序数$\rho$が取れるとき,
				\begin{align}
					\beta = \rho
				\end{align}
				ならば
				\begin{align}
					F(\beta) \subset f(F(\beta)) = F(\gamma)
				\end{align}
				が成り立ち,
				\begin{align}
					\beta < \rho
				\end{align}
				ならば(\refeq{fom:thm_existence_of_maximal_chain_1})より
				\begin{align}
					F(\beta) \subset F(\rho) \subset f(F(\rho)) = F(\gamma)
				\end{align}
				が成り立つ.$\gamma$が極限数であるとき,
				\begin{align}
					F(\beta) \subset \bigcup_{\delta \in \gamma} F(\delta) = F(\gamma)
				\end{align}
				が成り立つ.以上と超限帰納法より
				\begin{align}
					\forall \gamma \in \ON\, \left(\, \beta < \gamma \wedge
					\forall \delta \in \gamma\, (\, F(\delta) \in \mathscr{C}\, )
					\Longrightarrow F(\beta) \subset F(\gamma)\, \right)
				\end{align}
				を得る.
				
				では,任意の順序数$\alpha$に対して$F(\alpha)$が鎖であることを超限帰納法により示す.まず
				\begin{align}
					F(0) = c
				\end{align}
				なので
				\begin{align}
					F(0) \in \mathscr{C}
				\end{align}
				が成り立つ.次に$\alpha$を$0$でない任意の順序数として
				\begin{align}
					\forall \beta \in \alpha\, (\, F(\beta) \in \mathscr{C}\, )
				\end{align}
				が満たされているとする.このとき
				\begin{align}
					\alpha = \beta + 1
				\end{align}
				を満たす順序数$\beta$が取れるならば
				\begin{align}
					F(\alpha) = f(F(\beta))
				\end{align}
				が成り立つので$F(\alpha)$は鎖である.
				$\alpha$が極限数である場合,$x$と$y$を$F(\alpha)$の任意の要素とすれば,
				\begin{align}
					F(\alpha) = \bigcup_{\beta \in \alpha} F(\beta)
				\end{align}
				より
				\begin{align}
					x \in F(\beta)
				\end{align}
				を満たす$\alpha$の要素$\beta$と
				\begin{align}
					y \in F(\gamma)
				\end{align}
				を満たす$\alpha$の要素$\gamma$が取れて,
				\begin{align}
					F(\beta) \subset F(\gamma)
				\end{align}
				または
				\begin{align}
					F(\gamma) \subset F(\beta)
				\end{align}
				が成り立つが,いずれの場合も
				\begin{align}
					(x,y) \in O \vee (y,x) \in O
				\end{align}
				が従う.つまりこの場合も$F(\alpha)$は鎖である.よって超限帰納法より
				\begin{align}
					\forall \alpha \in \ON\, (\, F(\alpha) \in \mathscr{C}\, )
				\end{align}
				が成立する.
				
			\item[第三段]
				$\alpha$が極限数であるとし,$F(\alpha)$が極大鎖でないとする.$\alpha$の任意の要素$\beta$に対して
				\begin{align}
					F(\beta) \subset F(\alpha)
				\end{align}
				が成り立つので$F(\beta)$もまた極大鎖ではなく,ゆえに
				\begin{align}
					F(\beta) \subset F(\beta + 1) \wedge F(\beta) \neq F(\beta + 1)
				\end{align}
				が成立する.定理(\ref{thm:direct_product_of_non_empty_sets_is_not_empty})より
				$\alpha$上の写像$g$で,$\alpha$の任意の要素$\beta$に対して
				\begin{align}
					g(\beta) \in F(\beta+1) \wedge g(\beta) \notin F(\beta)
				\end{align}
				を満たすものが取れる.この$g$は単射である.実際,$\alpha$の二要素$\beta$と$\gamma$に対して,
				\begin{align}
					\beta < \gamma
				\end{align}
				ならば
				\begin{align}
					F(\beta+1) \subset F(\gamma)
				\end{align}
				が成り立つが,
				\begin{align}
					g(\beta) \in F(\beta + 1)
				\end{align}
				かつ
				\begin{align}
					g(\gamma) \notin F(\gamma)
				\end{align}
				が満たされているので
				\begin{align}
					g(\beta) \neq g(\gamma)
				\end{align}
				が従う.$\alpha$から$F(\alpha)$への単射が得られたので
				\begin{align}
					\card{\alpha} \leq \card{F(\alpha)}
				\end{align}
				が成立する.
			
			\item[第四段]
				いま$\beta$を順序数として
				\begin{align}
					\card{\beta} \leq \card{F(\beta)}
				\end{align}
				が満たされていて,かつ$F(\beta)$は極大鎖ではないとする.
				ここで$\beta$から$\card{\beta}$への全単射を$p$とし,
				$\card{F(\beta)}$から$F(\beta)$への全単射を$q$とすると,
				$q \circ p$は$\beta$から$F(\beta)$への単射である.
				いま$F(\beta)$は極大鎖ではないので
				\begin{align}
					x \notin F(\beta) \wedge x \in F(\beta + 1)
				\end{align}
				を満たす$x$が取れて,
				\begin{align}
					g \defeq (q \circ p) \cup \{(\beta,x)\}
				\end{align}
				と$g$を定めると,$g$は$\beta+1$から$F(\beta+1)$への単射となる.すなわち
				\begin{align}
					\card{(\beta+1)} \leq \card{F(\beta+1)}
				\end{align}
				が成立する.
				\QED
		\end{description}
	\end{sketch}
	%\section{共終数}
	順序数$\alpha,\beta$に対して,$\alpha$から$\beta$への写像$f$で
	\begin{align}
		\forall \delta,\zeta \in \alpha\, (\, \delta < \zeta \rarrow f(\delta) < f(\zeta)\, )
	\end{align}
	と
	\begin{align}
		\forall \gamma \in \beta\, \exists \delta \in \alpha\, (\, \gamma < f(\delta)\, )
	\end{align}
	を満たすもの({\bf 共終写像}\index{きょうしゅうしゃぞう@共終写像}{\bf (cofinal function)})
	が存在することを
	\begin{align}
		\cof{\alpha}{\beta}
	\end{align}
	と書く.つまり
	\begin{align}
		\cof{\alpha}{\beta} \defarrow
		\exists f\, (\, &f:\alpha \rarrow \beta \\
		&\wedge \forall \delta,\zeta \in \alpha\, (\, \delta < \zeta \rarrow f(\delta) < f(\zeta)\, ) \\
		&\wedge \forall \gamma \in \beta\, \exists \delta \in \alpha\, (\, \gamma < f(\delta)\, )
		 \, )
	\end{align}
	である.また$\beta$に対して$\cof{\alpha}{\beta}$を満たす最小の順序数$\alpha$を
	\begin{align}
		\cf{\beta}
	\end{align}
	と書き,$\beta$の{\bf 共終数}\index{きょうしゅうすう@共終数}{\bf (cofinality)}と呼ぶ.
	
	\begin{screen}
		\begin{thm}[共終の概念は極限数のみに使われる]
			\begin{align}
				&\forall \alpha,\beta \in \ON\, 
				(\, \cof{\alpha}{\beta} \rarrow \limo{\beta}\, ), \\
				&\forall \alpha,\beta \in \ON\, 
				(\, \cof{\alpha}{\beta} \rarrow \limo{\alpha}\, )
			\end{align}
		\end{thm}
	\end{screen}
	
	\begin{sketch}
		$\cof{\alpha}{\beta}$であるとし,$f$を$\alpha$から$\beta$への共終写像とする.
		$\beta$の任意の要素$\gamma$に対して
		\begin{align}
			\gamma < f(\delta) < \beta
		\end{align}
		なる$\alpha$の要素$\delta$が取れるので,
		\begin{align}
			\forall \gamma \in \beta\, (\, \gamma + 1 \neq \beta\, )
		\end{align}
		である.ゆえに$\beta$は極限数である.
		$\alpha$の任意の要素$\delta$に対して
		\begin{align}
			f(\delta) < f(\zeta)
		\end{align}
		なる$\alpha$の要素$\zeta$が取れて,$f$の単調増大性より
		\begin{align}
			\delta < \zeta
		\end{align}
		となるので,
		\begin{align}
			\forall \delta \in \alpha\, (\, \delta+ 1 \neq \alpha\, )
		\end{align}
		である.ゆえに$\alpha$も極限数である.
		\QED
	\end{sketch}
	
	\begin{screen}
		\begin{thm}[極限数は自分自身と共終]
			\begin{align}
				\forall \alpha \in \ON\, 
				(\, \limo{\alpha} \rarrow \cof{\alpha}{\alpha}\, )
			\end{align}
		\end{thm}
	\end{screen}
	
	\begin{sketch}
		$\alpha$上の恒等写像が共終写像となる.
	\end{sketch}
	
	\begin{screen}
		\begin{thm}[共終なら共終数は同じ]
			\begin{align}
				\forall \alpha \in \ON\, 
				(\, \limo{\alpha} \wedge \limo{\beta} \wedge \cof{\alpha}{\beta}
				 \rarrow \cf{\alpha} = \cf{\alpha}\, ).
			\end{align}
		\end{thm}
	\end{screen}
	
	\begin{sketch}
		$\cof{\cf{\alpha}}{\alpha}$と$\cof{\alpha}{\beta}$より
		$\cof{\cf{\alpha}}{\beta}$が成り立つので
		\begin{align}
			\cf{\beta} \leq \cf{\alpha}
		\end{align}
		となる.$f$を$\alpha$から$\beta$への共終写像とし,$g$を
		$\cf{\beta}$から$\beta$への共終写像とし,$\cf{\beta}$から$\alpha$への写像$h$を
		\begin{align}
			\xi \longmapsto \min{\Set{\eta \in \alpha}{g(\xi) < f(\eta)}}
		\end{align}
		なるものとして定める.このとき$\alpha$の任意の要素$\delta$に対して
		\begin{align}
			f(\delta) < g(\xi)
		\end{align}
		なる$\xi$が取れて,
		\begin{align}
			g(\xi) < f(\eta)
		\end{align}
		なる$\eta$が取れるから,
		\begin{align}
			f(\delta) < g(\xi) < f(h(\xi))
		\end{align}
		が成り立つ.ゆえに
		\begin{align}
			\delta < h(\xi)
		\end{align}
		である.ゆえに
		\begin{align}
			\forall \delta \in \alpha\, \exists \xi \in \cf{\beta}\, 
			(\, \delta < h(\xi)\, )
		\end{align}
		が成り立つ.今度は$\cf{\beta}$から$\alpha$への写像$H$を
		\begin{align}
			\xi \longmapsto \max{\{g(\xi),\sup{\Set{H(\kappa)+1}{\kappa < \xi}}\}}
		\end{align}
		なるものとして再帰的に定義する.このとき
		\begin{align}
			\xi_{1} < \xi_{2}
		\end{align}
		ならば
		\begin{align}
			H(\xi_{1}) < H(\xi_{1}) + 1 \leq \sup{\Set{H(\kappa)+1}{\kappa < \xi_{2}}}
			\leq H(\xi_{2})
		\end{align}
		が成り立つし,また$\alpha$の任意の要素$\delta$に対して
		\begin{align}
			\delta < g(\xi)
		\end{align}
		なる$\xi$が取れるが,
		\begin{align}
			\delta < g(\xi) \leq H(\xi)
		\end{align}
		となる.ゆえに$H$は$\cf{\beta}$から$\alpha$への共終写像である.従って
		\begin{align}
			\cof{\cf{\beta}}{\alpha}
		\end{align}
		となり,
		\begin{align}
			\cf{\alpha} \leq \cf{\beta}
		\end{align}
		が出る.
		\QED
	\end{sketch}
	
	\begin{screen}
		\begin{thm}[共終数は基数である]
			\begin{align}
				\forall \alpha \in \ON\, (\, \limo{\alpha} \rarrow \cf{\alpha} \in \CN\, ).
			\end{align}
		\end{thm}
	\end{screen}
	
	\begin{screen}
		\begin{dfn}[正則基数]
			$\cf{\alpha} = \alpha$を満たす基数$\alpha$を{\bf 正則基数}
			\index{せいそくきすう@正則基数}{\bf (regular cardinal)}と呼ぶ.
			そうでない基数を{\bf 特異基数}\index{とくいきすう@特異基数}
			{\bf (singular cardinal)}と呼ぶ.
		\end{dfn}
	\end{screen}
		
	\begin{screen}
		\begin{thm}[$\Natural$は正則である]
			\begin{align}
				\cf{\Natural} = \Natural.
			\end{align}
		\end{thm}
	\end{screen}
	
	\begin{sketch}
		$\limo{\Natural}$より
		\begin{align}
			\cof{\Natural}{\Natural}
		\end{align}
		が成り立つので
		\begin{align}
			\cf{\Natural} \leq \Natural
		\end{align}
		となる.他方で
		\begin{align}
			\limo{\cf{\Natural}}
		\end{align}
		より
		\begin{align}
			\Natural \leq \cf{\Natural}
		\end{align}
		である.
		\QED
	\end{sketch}
	
	\begin{screen}
		\begin{thm}[共終数は正則である]
			\begin{align}
				\forall \alpha \in \ON\, 
				(\, \limo{\alpha} \rarrow \cf{\cf{\alpha}} = \cf{\alpha}\, ).
			\end{align}
		\end{thm}
	\end{screen}
	
	\begin{sketch}
		\begin{align}
			\cof{\cf{\alpha}}{\cf{\alpha}}
		\end{align}
		より
		\begin{align}
			\cf{\cf{\alpha}} \leq \cf{\alpha}
		\end{align}
		となる.また
		\begin{align}
			&\cof{\cf{\cf{\alpha}}}{\cf{\alpha}}, \\
			&\cof{\cf{\alpha}}{\alpha}
		\end{align}
		より
		\begin{align}
			\cof{\cf{\cf{\alpha}}}{\alpha}
		\end{align}
		が従い,
		\begin{align}
			\cf{\cf{\alpha}} \leq \cf{\alpha}
		\end{align}
		も得られる.
		\QED
	\end{sketch}
	

%\chapter{イプシロン定理}
	%\section{言語}
	\begin{description}
	\item[{\bf EC}]
	{\bf EC}(Elementary calculus)の言語を$L(EC)$と書く.$L(EC)$の構成要素は
	\begin{description}
		\item[矛盾記号] $\bot$
		\item[論理記号] $\rightharpoondown$, $\vee$, $\wedge$, $\rightarrow$
		\item[述語記号] $=$, $\in$
		\item[変項] $x_{0},x_{1},x_{2},\cdots$
	\end{description}
	とする.変項は$L(EC)$の項であって,また$L(EC)$の項は変項だけである.
	$L(EC)$の式は
	\begin{itemize}
		\item 項$s$と式$t$に対して$\in st$と$= st$は式である.
		\item 式$\varphi$と式$\psi$に対して$\rightharpoondown \varphi,
			\vee \varphi \psi,\ \wedge \varphi \psi, \rightarrow \varphi \psi$
			は式である.
		\item 以上のみが$L(EC)$の式である.
	\end{itemize}
	
	\item[{\bf PC}]
	{\bf PC}(Predicate calculus)の言語を$L(PC)$と書く.$L(PC)$の構成要素は
	\begin{description}
		\item[矛盾記号] $\bot$
		\item[論理記号] $\rightharpoondown$, $\vee$, $\wedge$, $\Longrightarrow$
		\item[量化子] $\forall$, $\exists$
		\item[述語記号] $=$, $\in$
		\item[変項] $x_{0},x_{1},x_{2},\cdots$
	\end{description}
	とする.変項は$L(PC)$の項であって,また$L(PC)$の項は変項だけである.
	$L(PC)$の式は
	\begin{itemize}
		\item 項$s$と式$t$に対して$\in st$と$= st$は式である.
		\item 式$\varphi$と式$\psi$に対して$\rightharpoondown \varphi,
			\vee \varphi \psi,\ \wedge \varphi \psi, \rightarrow \varphi \psi$
			は式である.
		\item 式$\varphi$と変項$x$に対して,$\forall x \varphi$と$\exists x \varphi$は式である.
		\item 以上のみが$L(PC)$の式である.
	\end{itemize}
	\end{description}
	
	$L(EC)$と$L(PC)$に$\varepsilon$項を追加した言語をそれぞれ$L(EC_{\varepsilon}),
	L(PC_{\varepsilon})$とする.
	
	\begin{description}
	\item[{\bf EC${}_{\varepsilon}$}]
	言語$L(EC_{\varepsilon})$の構成要素は
	\begin{description}
		\item[矛盾記号] $\bot$
		\item[論理記号] $\rightharpoondown$, $\vee$, $\wedge$, $\rightarrow$
		\item[述語記号] $=$, $\in$
		\item[変項] $x_{0},x_{1},x_{2},\cdots$
		\item[$\varepsilon$記号] $\varepsilon$
	\end{description}
	とする.$L(EC_{\varepsilon})$の項と式は
	\begin{itemize}
		\item 変項は項である.
		\item 項$s$と式$t$に対して$\in st$と$= st$は式である.
		\item 式$\varphi$と式$\psi$に対して$\rightharpoondown \varphi,
			\vee \varphi \psi,\ \wedge \varphi \psi, \rightarrow \varphi \psi$
			は式である.
		\item 式$\varphi$と変項$x$に対して,$\epsilon x \varphi$は項である.
		\item 以上のみが$L(EC_{\varepsilon})$の項と式である.
	\end{itemize}
	
	\item[{\bf PC${}_{\varepsilon}$}]
	言語$L(PC_{\varepsilon})$の構成要素は
	\begin{description}
		\item[矛盾記号] $\bot$
		\item[論理記号] $\rightharpoondown$, $\vee$, $\wedge$, $\Longrightarrow$
		\item[量化子] $\forall$, $\exists$
		\item[述語記号] $=$, $\in$
		\item[変項] $x_{0},x_{1},x_{2},\cdots$
		\item[$\varepsilon$記号] $\varepsilon$
	\end{description}
	とする.$L(PC_{\varepsilon})$の項と式は
	\begin{itemize}
		\item 変項は項である.
		\item 項$s$と式$t$に対して$\in st$と$= st$は式である.
		\item 式$\varphi$と式$\psi$に対して$\rightharpoondown \varphi,
			\vee \varphi \psi,\ \wedge \varphi \psi, \rightarrow \varphi \psi$
			は式である.
		\item 式$\varphi$と変項$x$に対して,$\forall x \varphi$と$\exists x \varphi$は式である.
		\item 式$\varphi$と変項$x$に対して,$\epsilon x \varphi$は項である.
		\item 以上のみが$L(PC_{\varepsilon})$の項と式である.
	\end{itemize}
	\end{description}
	
\section{証明}
	\begin{description}
	\item[{\bf EC}]\mbox{}
	
	\begin{itembox}[l]{{\bf EC}の公理}
		$\varphi$と$\psi$と$\xi$を$L(EC)$の式とするとき,次は{\bf EC}の公理である.
		\begin{description}
			\item[(S)] $(\varphi \rightarrow (\psi \rightarrow \chi)) 
				\rightarrow ((\varphi \rightarrow \psi)
				\rightarrow (\varphi \rightarrow \chi)).$
			\item[(K)] $\varphi \rightarrow (\psi \rightarrow \varphi).$
			\item[(DI1)] $\varphi \rightarrow (\varphi \vee \psi).$
			\item[(DI2)] $\psi \rightarrow (\varphi \vee \psi).$
			\item[(DE)] $(\varphi \rightarrow \chi) \rightarrow 
				((\psi \rightarrow \chi) \rightarrow ((\varphi \vee \psi) \rightarrow \chi)).$
			\item[(CI)] $\varphi \rightarrow (\psi \rightarrow (\varphi \wedge \psi)).$
			\item[(CE1)] $(\varphi \wedge \psi) \rightarrow \varphi.$
			\item[(CE2)] $(\varphi \wedge \psi) \rightarrow \psi.$
				
			\item[(CTI)] $\varphi \rightarrow (\rightharpoondown \varphi \rightarrow \bot).$
			
			\item[(NI)] $(\varphi \rightarrow \bot) \rightarrow\ \rightharpoondown \varphi.$
			\item[(DNE)] $\rightharpoondown \rightharpoondown \varphi \rightarrow \varphi.$
		\end{description}
	\end{itembox}
	
	$\Gamma$を公理系という場合は,$\Gamma$は$L(EC)$の式の集合である.$\Gamma$が空である場合もある.
	$L(EC)$の式$\chi$に対して$\Gamma$から{\bf EC}の証明が存在する(証明可能である)ことを
	\begin{align}
		\Gamma \provable{\mbox{{\bf EC}}} \chi
	\end{align}
	と書くが,{\bf EC}における$\Gamma$から$\chi$への証明とは,
	$L(EC)$の式の列$\varphi_{1},\varphi_{2},
	\cdots,\varphi_{n}$であって,$\varphi_{n}$は$\chi$であり,
	各$i \in \{1,2,\cdots,n\}$に対して
	\begin{itemize}
		\item $\varphi_{i}$は{\bf EC}の公理である.
		\item $\varphi_{i}$は$\Gamma$の公理である.
		\item $\varphi_{i}$は前の式から推論規則を用いて得られる式である.{\bf EC}の推論規則とは,
			\begin{description}
			\item[三段論法]
				$j,k < i$なる$k,j$が取れて,$\varphi_{k}$は
				$\varphi_{j} \rightarrow \varphi_{i}$である.
		\end{description} 
	\end{itemize}
	が満たされているものである.
	
	\item[{\bf PC}]
	$\varphi$をいずれかの言語の式とし,$x$を変項とする.
	$\varphi$に$x$が自由に現れているとき,$\varphi$に自由に現れている
	$x$を変項$t$で置き換えた式を
	\begin{align}
		\varphi(t/x)
	\end{align}
	とする.ただし$t$は$\varphi(t/x)$で{\bf $x$に置き換わった位置で束縛されない}とする.
	このことを{\bf $t$は$\varphi$の中で$x$への代入について自由である}とも言う.
	
	\begin{itembox}[l]{{\bf PC}の公理}
		$\varphi$と$\psi$と$\xi$を$L(PC)$の式とし,$x$と$t$を変項とするとき,
		次は{\bf PC}の公理である.
		\begin{description}
			\item[(S)] $(\varphi \rightarrow (\psi \rightarrow \chi)) 
				\rightarrow ((\varphi \rightarrow \psi)
				\rightarrow (\varphi \rightarrow \chi)).$
			\item[(K)] $\varphi \rightarrow (\psi \rightarrow \varphi).$
			\item[(DI1)] $\varphi \rightarrow (\varphi \vee \psi).$
			\item[(DI2)] $\psi \rightarrow (\varphi \vee \psi).$
			\item[(DE)] $(\varphi \rightarrow \chi) \rightarrow 
				((\psi \rightarrow \chi) \rightarrow ((\varphi \vee \psi) \rightarrow \chi)).$
			\item[(CI)] $\varphi \rightarrow (\psi \rightarrow (\varphi \wedge \psi)).$
			\item[(CE1)] $(\varphi \wedge \psi) \rightarrow \varphi.$
			\item[(CE2)] $(\varphi \wedge \psi) \rightarrow \psi.$
			
			\item[(UE)] $\forall x \varphi \rightarrow \varphi(\tau/x).$
				\\ \textcolor{red}{ただし,$\varphi$には$x$が自由に現れて,
				$t$は$\varphi$の中で$x$への代入について自由である.}
				
			\item[(EI)] $\varphi(\tau/x) \rightarrow \exists x \varphi.$
				\\ \textcolor{red}{ただし,$\varphi$には$x$が自由に現れて,
				$t$は$\varphi$の中で$x$への代入について自由である.}
				
			\item[(CTI)] $\varphi \rightarrow (\rightharpoondown \varphi \rightarrow \bot).$
			
			\item[(NI)] $(\varphi \rightarrow \bot) \rightarrow\ \rightharpoondown \varphi.$
			\item[(DNE)] $\rightharpoondown \rightharpoondown \varphi \rightarrow \varphi.$
		\end{description}
	\end{itembox}
	
	$\Gamma$を公理系という場合は,$\Gamma$は$L(PC)$の文の集合である.$\Gamma$が空である場合もある.
	$L(PC)$の式$\chi$に対して$\Gamma$から{\bf PC}の証明が存在する(証明可能である)ことを
	\begin{align}
		\Gamma \provable{\mbox{{\bf PC}}} \chi
	\end{align}
	と書くが,{\bf PC}における$\Gamma$から$\chi$への証明とは,
	$L(PC)$の式の列$\varphi_{1},\varphi_{2},
	\cdots,\varphi_{n}$であって,$\varphi_{n}$は$\chi$であり,
	各$i \in \{1,2,\cdots,n\}$に対して
	\begin{itemize}
		\item $\varphi_{i}$は{\bf PC}の公理である.
		\item $\varphi_{i}$は$\Gamma$の公理である.
		\item $\varphi_{i}$は前の式から推論規則を用いて得られる式である.{\bf PC}の推論規則とは,
			\begin{description}
			\item[三段論法]
				$j,k < i$なる$k,j$が取れて,$\varphi_{k}$は
				$\varphi_{j} \rightarrow \varphi_{i}$である.
			 	
			\item[存在汎化] 
				$j < i$なる$j$が取れて,$\varphi_{j}$は
				$\varphi(t/x) \rightarrow \psi$なる式であって,
				$\varphi_{i}$は$\exists x \varphi \rightarrow \psi$なる式である.
				\\ \textcolor{red}{ただし,$\varphi$には$x$が自由に現れ,
				$t$は$\varphi$の中で$x$への代入について自由である.
				また$t$は$\varphi$と$\psi$には自由に現れない.}
			
			\item[全称汎化] 
				$j < i$なる$j$が取れて,$\varphi_{j}$は
				$\psi \rightarrow \varphi(t/x)$なる式であって,
				$\varphi_{i}$は$\psi \rightarrow \forall x \varphi$なる式である.
				\\ \textcolor{red}{ただし,$\varphi$には$x$が自由に現れ,
				$t$は$\varphi$の中で$x$への代入について自由である.
				また$t$は$\varphi$と$\psi$には自由に現れない.}
		\end{description} 
	\end{itemize}
	が満たされているものである.
	
	\item[主要論理式]
	{\bf EC}${}_{\varepsilon}$と{\bf PC}${}_{\varepsilon}$の公理には
	\begin{align}
		\varphi(t/x) \rightarrow \varphi(\varepsilon x \varphi/x)
	\end{align}
	なる形の式が追加される.ただし$x$は$\varphi$に自由に現れて,
	$t$は$\varphi$の中で$x$への代入について自由である.
	この形の式を{\bf 主要論理式}{\bf (principal formula)}と呼ぶ.
	
	\item[{\bf EC}${}_{\varepsilon}$]\mbox{}
	
	\begin{itembox}[l]{{\bf EC}${}_{\varepsilon}$の公理}
		$\varphi$と$\psi$と$\xi$を$L(EC_{\varepsilon})$の式とするとき,
		次は{\bf EC}${}_{\varepsilon}$の公理である.
		\begin{description}
			\item[(S)] $(\varphi \rightarrow (\psi \rightarrow \chi)) 
				\rightarrow ((\varphi \rightarrow \psi)
				\rightarrow (\varphi \rightarrow \chi)).$
			\item[(K)] $\varphi \rightarrow (\psi \rightarrow \varphi).$
			\item[(DI1)] $\varphi \rightarrow (\varphi \vee \psi).$
			\item[(DI2)] $\psi \rightarrow (\varphi \vee \psi).$
			\item[(DE)] $(\varphi \rightarrow \chi) \rightarrow 
				((\psi \rightarrow \chi) \rightarrow ((\varphi \vee \psi) \rightarrow \chi)).$
			\item[(CI)] $\varphi \rightarrow (\psi \rightarrow (\varphi \wedge \psi)).$
			\item[(CE1)] $(\varphi \wedge \psi) \rightarrow \varphi.$
			\item[(CE2)] $(\varphi \wedge \psi) \rightarrow \psi.$
				
			\item[(CTI)] $\varphi \rightarrow (\rightharpoondown \varphi \rightarrow \bot).$
			
			\item[(NI)] $(\varphi \rightarrow \bot) \rightarrow\ \rightharpoondown \varphi.$
			\item[(DNE)] $\rightharpoondown \rightharpoondown \varphi \rightarrow \varphi.$
			\item[(PF)] $\varphi(t/x) \rightarrow \varphi(\varepsilon x \varphi/x).$
				\\ \textcolor{red}{ただし,$\varphi$には$x$が自由に現れて,
				$t$は$\varphi$の中で$x$への代入について自由である.}
		\end{description}
	\end{itembox}
	
	$\Gamma$を公理系とする.$L(EC_{\varepsilon})$の式$\chi$に対して$\Gamma$から
	{\bf EC}${}_{\varepsilon}$の証明が存在する(証明可能である)ことを
	\begin{align}
		\Gamma \provable{\mbox{{\bf EC}${}_{\varepsilon}$}} \chi
	\end{align}
	と書くが,{\bf EC}${}_{\varepsilon}$における$\Gamma$から$\chi$への証明とは,
	$L(EC_{\varepsilon})$の式の列$\varphi_{1},\varphi_{2},
	\cdots,\varphi_{n}$であって,$\varphi_{n}$は$\chi$であり,
	各$i \in \{1,2,\cdots,n\}$に対して
	\begin{itemize}
		\item $\varphi_{i}$は{\bf EC}${}_{\varepsilon}$の公理である.
		\item $\varphi_{i}$は$\Gamma$の公理である.
		\item $\varphi_{i}$は前の式から推論規則を用いて得られる式である.
			{\bf EC}${}_{\varepsilon}$の推論規則とは,
			\begin{description}
			\item[三段論法]
				$j,k < i$なる$k,j$が取れて,$\varphi_{k}$は
				$\varphi_{j} \rightarrow \varphi_{i}$である.
		\end{description} 
	\end{itemize}
	が満たされているものである.
	
	\item[{\bf PC}${}_{\varepsilon}$]\mbox{}
	
	\begin{itembox}[l]{{\bf PC}${}_{\varepsilon}$の公理}
		$\varphi$と$\psi$と$\xi$を$L(PC_{\varepsilon})$の式とし,$x$と$t$を変項とするとき,
		次は{\bf PC}${}_{\varepsilon}$の公理である.
		\begin{description}
			\item[(S)] $(\varphi \rightarrow (\psi \rightarrow \chi)) 
				\rightarrow ((\varphi \rightarrow \psi)
				\rightarrow (\varphi \rightarrow \chi)).$
			\item[(K)] $\varphi \rightarrow (\psi \rightarrow \varphi).$
			\item[(DI1)] $\varphi \rightarrow (\varphi \vee \psi).$
			\item[(DI2)] $\psi \rightarrow (\varphi \vee \psi).$
			\item[(DE)] $(\varphi \rightarrow \chi) \rightarrow 
				((\psi \rightarrow \chi) \rightarrow ((\varphi \vee \psi) \rightarrow \chi)).$
			\item[(CI)] $\varphi \rightarrow (\psi \rightarrow (\varphi \wedge \psi)).$
			\item[(CE1)] $(\varphi \wedge \psi) \rightarrow \varphi.$
			\item[(CE2)] $(\varphi \wedge \psi) \rightarrow \psi.$
			
			\item[(UE)] $\forall x \varphi \rightarrow \varphi(\tau/x).$
				\\ \textcolor{red}{ただし,$\varphi$には$x$が自由に現れて,
				$t$は$\varphi$の中で$x$への代入について自由である.}
				
			\item[(EI)] $\varphi(\tau/x) \rightarrow \exists x \varphi.$
				\\ \textcolor{red}{ただし,$\varphi$には$x$が自由に現れて,
				$t$は$\varphi$の中で$x$への代入について自由である.}
				
			\item[(CTI)] $\varphi \rightarrow (\rightharpoondown \varphi \rightarrow \bot).$
			
			\item[(NI)] $(\varphi \rightarrow \bot) \rightarrow\ \rightharpoondown \varphi.$
			\item[(DNE)] $\rightharpoondown \rightharpoondown \varphi \rightarrow \varphi.$
			\item[(PF)] $\varphi(t/x) \rightarrow \varphi(\varepsilon x \varphi/x).$
				\\ \textcolor{red}{ただし,$\varphi$には$x$が自由に現れて,
				$t$は$\varphi$の中で$x$への代入について自由である.}
		\end{description}
	\end{itembox}
	
	$\Gamma$を公理系とする.$L(PC_{\varepsilon})$の式$\chi$に対して
	$\Gamma$から{\bf PC}${}_{\varepsilon}$の証明が存在する(証明可能である)ことを
	\begin{align}
		\Gamma \provable{\mbox{{\bf PC}${}_{\varepsilon}$}} \chi
	\end{align}
	と書くが,{\bf PC}${}_{\varepsilon}$における$\Gamma$から$\chi$への証明とは,
	$L(PC_{\varepsilon})$の式の列$\varphi_{1},\varphi_{2},
	\cdots,\varphi_{n}$であって,$\varphi_{n}$は$\chi$であり,
	各$i \in \{1,2,\cdots,n\}$に対して
	\begin{itemize}
		\item $\varphi_{i}$は{\bf PC}${}_{\varepsilon}$の公理である.
		\item $\varphi_{i}$は$\Gamma$の公理である.
		\item $\varphi_{i}$は前の式から推論規則を用いて得られる式である.
			{\bf PC}${}_{\varepsilon}$の推論規則とは,
			\begin{description}
			\item[三段論法]
				$j,k < i$なる$k,j$が取れて,$\varphi_{k}$は
				$\varphi_{j} \rightarrow \varphi_{i}$である.
			 	
			\item[存在汎化] 
				$j < i$なる$j$が取れて,$\varphi_{j}$は
				$\varphi(t/x) \rightarrow \psi$なる式であって,
				$\varphi_{i}$は$\exists x \varphi \rightarrow \psi$なる式である.
				\\ \textcolor{red}{ただし,$\varphi$には$x$が自由に現れ,
				$t$は$\varphi$の中で$x$への代入について自由である.
				また$t$は$\varphi$と$\psi$には自由に現れない.}
			
			\item[全称汎化] 
				$j < i$なる$j$が取れて,$\varphi_{j}$は
				$\psi \rightarrow \varphi(t/x)$なる式であって,
				$\varphi_{i}$は$\psi \rightarrow \forall x \varphi$なる式である.
				\\ \textcolor{red}{ただし,$\varphi$には$x$が自由に現れ,
				$t$は$\varphi$の中で$x$への代入について自由である.
				また$t$は$\varphi$と$\psi$には自由に現れない.}
		\end{description} 
	\end{itemize}
	が満たされているものである.
	\end{description}
	%\section{第一イプシロン定理メモ}
	
	言語$L(EC)$及び$L(EC_{\varepsilon})$を高橋先生の資料と同じものとする.
	{\bf 主要論理式}\index{しゅようろんりしき@主要論理式}{\bf (principal formula)}とは
	\begin{align}
		A(t) \Longrightarrow A(\varepsilon x A)
	\end{align}
	なる形の$L(EC)$の式を指す.ここで$A$とは$L(EC)$の式であって,変項$x$が$A$に自由に現れていて,
	また$A$に自由に出現するのは$x$のみである.$A(t)$とは$A$における$x$の自由な出現を全て閉項$t$に置き換えた式であり,
	$A(\varepsilon x A)$とは$A$における$x$の自由な出現を全て項$\varepsilon x A$に置き換えた式である.
	このとき$\varepsilon x A$は$A(t) \Longrightarrow A(\varepsilon x A)$に{\bf 属している}という.
	
	$EC$の公理とはトートロジーだけである.トートロジーは$EC_{\varepsilon}$の公理でもあるが,
	これに加えて主要論理式も$EC_{\varepsilon}$の公理である.
	
	$\pi = (\varphi_{0},\varphi_{1},\cdots,\varphi_{n})$を$EC_{\varepsilon}$の文の列とするとき,
	{\bf $\pi$の主要論理式}や{\bf $\pi$に現れる主要論理式}とは主要論理式である$\varphi_{i}$を指す.
	また$\pi$の主要論理式に属している$\varepsilon$項を{\bf $\pi$の主要$\varepsilon$項}と呼ぶ.
	
\subsection{埋め込み定理}
	$A$を$L(PC_{\varepsilon})$の式とするとき,$A$を$L(EC_{\varepsilon})$の式に書き換える.
	\begin{align}
		x^{\varepsilon} &\rightarrow x \\
		(\in \tau \sigma)^{\varepsilon} &\rightarrow \in \tau^{\varepsilon} \sigma^{\varepsilon} \\
		(= \tau \sigma)^{\varepsilon} &\rightarrow = \tau^{\varepsilon} \sigma^{\varepsilon} \\
		(\rightharpoondown \varphi)^{\varepsilon} &\rightarrow \rightharpoondown \varphi^{\varepsilon} \\
		(\vee \varphi \psi)^{\varepsilon} &\rightarrow \vee \varphi^{\varepsilon} \psi^{\varepsilon} \\
		(\wedge \varphi \psi)^{\varepsilon} &\rightarrow \wedge \varphi^{\varepsilon} \psi^{\varepsilon} \\
		(\Longrightarrow \varphi \psi)^{\varepsilon} &\rightarrow \Longrightarrow \varphi^{\varepsilon} \psi^{\varepsilon} \\
		(\exists x \varphi)^{\varepsilon} &\rightarrow \varphi^{\varepsilon}(\varepsilon x \varphi^{\varepsilon}) \\
		(\forall x \varphi)^{\varepsilon} &\rightarrow \varphi^{\varepsilon}(\varepsilon x \rightharpoondown \varphi^{\varepsilon}) \\
		(\varepsilon x \psi)^{\varepsilon} &\rightarrow \varepsilon x \varphi^{\varepsilon}
	\end{align}
	
	$A$が$L(PC_{\varepsilon})$の式で,$x$が$A$に自由に現れて,
	かつ$A$に自由に現れているのが$x$のみであるとき,
	$A^{\varepsilon}$にも$x$が自由に現れて,かつ$A^{\varepsilon}$に
	自由に現れているのは$x$のみである.
	
	\begin{align}
		(\varphi[x/\tau])^{\varepsilon} \rightarrow \varphi^{\varepsilon}
		(\varphi^{\varepsilon}[x/\tau^{\varepsilon}]). \\
	\end{align}
	
	\begin{itembox}[c]{$PC_{\varepsilon}$の証明を$EC_{\varepsilon}$の証明に埋め込む}
		$A$を$L(PC_{\varepsilon})$の文とし,$PC_{\varepsilon} \vdash A$であるとする.
		このとき$EC_{\varepsilon} \vdash A^{\varepsilon}$である.
	\end{itembox}
	
	示すべきことは
	\begin{itemize}
		\item $A \in Ax(PC_{\varepsilon})$ならば$\vdash A^{\varepsilon}$であること.
			\begin{itemize}
				\item $\vdash A$ならば$\vdash A^{\varepsilon}$であること.
				\item $A$に$x$が自由に現れて,かつ自由に現れているのが$x$のみであるとき,
					\begin{align}
						\vdash A^{\varepsilon}(t^{\varepsilon}) \Longrightarrow A^{\varepsilon}(\varepsilon x A^{\varepsilon})
					\end{align}
					であること.
				\item $A$に$x$が自由に現れて,かつ自由に現れているのが$x$のみであるとき,
					\begin{align}
						\vdash A^{\varepsilon}(\varepsilon x \rightharpoondown A^{\varepsilon}) \Longrightarrow A^{\varepsilon}(t^{\varepsilon})
					\end{align}
					であること.
			\end{itemize}
		
		\item $PC_{\varepsilon} \vdash B$かつ$PC_{\varepsilon} \vdash B \Longrightarrow A$である$B$が取れるとき,
			$(B \Longrightarrow A)^{\varepsilon}$は$B^{\varepsilon} \Longrightarrow A^{\varepsilon}$なので
			$EC_{\varepsilon} \vdash B^{\varepsilon}$ならば
			$EC_{\varepsilon} \vdash A^{\varepsilon}$となる.
	\end{itemize}

\subsection{階数}
	$B(x,y,z)$を,変項$x,y,z$が,そしてこれらのみが自由に現れる
	$L(EC)$の式とする.このとき
	\begin{align}
		\exists x\, \exists y\, \exists z\, B(x,y,z)
	\end{align}
	に対して,$z$から順に$\varepsilon$項に変換していくと
	\begin{align}
		&\exists x\, \exists y\, B(x,y,\varepsilon z B(x,y,z)), \\
		&\exists x\, \, B(x,\varepsilon y B(x,y,\varepsilon z B(x,y,z)),\varepsilon z B(x,\varepsilon y B(x,y,\varepsilon z B(x,y,z)),z))
	\end{align}
	となるが,最後に$\exists x$を無くすと式が長くなりすぎるのでここで止めておく.
	さて$z$に注目すれば,$B$に自由に現れていた$z$はまず
	\begin{align}
		\varepsilon z B(x,y,z)
	\end{align}
	に置き換えられている.この時点では$x,y$は自由なままであるから,この$\varepsilon$項を
	\begin{align}
		e_{1}[x,y]
	\end{align}
	と略記する.次に$y$は
	\begin{align}
		\varepsilon y B(x,y,\varepsilon z B(x,y,z))
	\end{align}
	に置き換えられるが,$e_{1}[x,y]$を使えば
	\begin{align}
		\varepsilon y B(x,y,e_{1}[x,y])
	\end{align}
	と書ける.この$\varepsilon$項でも$x$は自由なままであるから
	\begin{align}
		e_{2}[x]
	\end{align}
	と略記する.最後に
	\begin{align}
		\exists x\, B\left(x,e_{2}[x],e_{1}[x,e_{2}[x]]\right)
	\end{align}
	から$\exists$を除去するには,$x$を
	\begin{align}
		\varepsilon x B\left(x,e_{2}[x],e_{1}[x,e_{2}[x]]\right)
	\end{align}
	に置き換えれば良い.この$\varepsilon$項を$e_{3}$と書く.以上で
	$\exists x\, \exists y\, \exists z\, B(x,y,z)$は$L(EC_{\varepsilon})$の式
	\begin{align}
		B\left(e_{3},e_{2}[e_{3}],e_{1}[e_{3},e_{2}[e_{3}]]\right)
	\end{align}
	に変換されたわけである.それはさておき,ここで考察するのは{\bf 項間の主従関係}である.
	$e_{2}[x]$は$x$のみによってコントロールされているのだから,
	$x$を司る$e_{3}$を主人だと思えば$e_{2}[x]$は$e_{3}$の直属の子分である.
	$e_{1}[x,y]$は$y$によってもコントロールされているので,
	$e_{1}[x,y]$とは$e_{2}[x]$の子分であり,すなわち$e_{3}$の子分の子分であって,
	この例において一番身分が低いわけである.
	
	$\varepsilon$項を構文解析して,それが何重の子分を従えているかを測った指標を
	{\bf 階数}{\bf (rank)}と呼ぶ.とはいえ直属の子分が複数いることもあり得るので,
	子分の子分の子分の子分...と次々に枝分かれしていく従属関係の中で,最も
	深いものを辿って階数を定めることにする.
	
	\begin{screen}
		\begin{metadfn}[従属]
			$\varepsilon x A$を$L(EC_{\varepsilon})$の$\varepsilon$項とし,
			$e$を$A$に現れる$L(EC_{\varepsilon})$の$\varepsilon$項とするとき,
			$x$が$e$に自由に現れているなら$e$は{\bf $\varepsilon x A$に従属している}
			\index{じゅうぞく@従属}{\bf (subordinate to $\varepsilon x A$)}という.
		\end{metadfn}
	\end{screen}
	
	はじめの例では,$e_{2}[x]$と$e_{1}[x,e_{2}[x]]$は共に$e_{3}$に従属しているし,
	$e_{1}[x,y]$は$e_{2}[x]$に従属している.
	$e_{1}[x,e_{2}[x]]$に従属している$\varepsilon$項は無いし,
	$e_{1}[x,y]$に従属している$\varepsilon$項も無い.
	
	\begin{screen}
		\begin{metadfn}[階数]
			$\theta$を$L(EC_{\varepsilon})$の項または式とするとき,
			$\theta$の{\bf 階数}\index{かいすう@階数}{\bf (rank)}を
			以下の要領で定義する.
			\begin{enumerate}
				\item $\theta$が$\varepsilon$項でなくて,
					$\theta$に$\varepsilon$項が現れないならば,
					$\theta$の階数を$0$とする.
				\item $\theta$が$\varepsilon$項であって,かつ$\theta$に従属している
					$\varepsilon$項が無いならば,$\theta$の階数を$1$とする.
				\item $\theta$が$\varepsilon$項であって,かつ$\theta$に従属している
					$\varepsilon$項があるならば,$\theta$に従属している$\varepsilon$項の
					階数の最大値に$1$を足したものを$\theta$の階数とする.
				\item $\theta$が$\varepsilon$項でなくて,$\theta$に
					$\varepsilon$項が現れるならば,$\theta$に現れる$\varepsilon$項の
					階数の最大値を$\theta$の階数とする.
			\end{enumerate}
			また$\theta$の階数を$rk(\theta)$と書く.
		\end{metadfn}
	\end{screen}
	
	実際に$L(EC_{\varepsilon})$の全ての項および式に対して階数が定まっている.
	(構造的帰納法について準備不足だが,直感的に次の説明は妥当である...)
	\begin{description}
		\item[step1] $\theta$が$L(EC)$の項あるいは式ならば,$\theta$の階数は$0$である.
		
		\item[step2] $\theta$が$L(EC)$の式で作られた$\varepsilon$項ならば
			$\theta$の階数は$1$である.
		
		\item[step3] 項$\tau_{1},\cdots,\tau_{n}$のそれぞれに対して,
			その全ての部分$\varepsilon$項に階数が定まっていれば,
			$f$を$n$項関数として,$f\tau_{1}\cdots\tau_{n}$の階数は
			$rk(\tau_{1}),\cdots,rk(\tau_{n})$の中の最大値である.
			というのも,$f\tau_{1}\cdots\tau_{n}$に現れる$\varepsilon$項は
			$\tau_{1},\cdots,\tau_{n}$のいずれかの部分項になっているためである.
			
		\item[step4] 項$\tau_{1},\cdots,\tau_{n}$のそれぞれに対して,
			その全ての部分$\varepsilon$項に階数が定まっていれば,
			$p$を$n$項述語として,$p\tau_{1}\cdots\tau_{n}$の階数は
			$rk(\tau_{1}),\cdots,rk(\tau_{n})$の中の最大値である.
		
		\item[step5] 式$\varphi$と$\psi$のそれぞれに対して,
			その全ての部分$\varepsilon$項に階数が定まっていれば,
			\begin{align}
				rk(\rightharpoondown \varphi) &\coloneqq rk(\varphi), \\
				rk(\vee \varphi \psi) &\coloneqq \max\{rk(\varphi),rk(\psi)\}, \\
				rk(\wedge \varphi \psi) &\coloneqq \max\{rk(\varphi),rk(\psi)\}, \\
				rk(\Longrightarrow \varphi \psi) 
				&\coloneqq \max\{rk(\varphi),rk(\psi)\}, \\
			\end{align}
			である.というのも,左辺の式に現れる$\varepsilon$項は
			$\varphi$か$\psi$の少なくとも一方に現れているからである.
		
		\item[step6] 式$\varphi$に現れる全ての$\varepsilon$項に対して階数が定まっているならば,
			$\varepsilon x \varphi$の階数は定義通りに定めることが出来る.
	\end{description}
	
	\begin{screen}
		\begin{metathm}[階数定理]
			$\varepsilon x A$を$L(EC_{\varepsilon})$の$\varepsilon$項とし,
			$s$と$t$を,その中に$x$が自由に現れない$L(EC_{\varepsilon})$の項とする.このとき,
			$A$に現れる$s$の一つを$t$に置き換えた式を$A^{t}$とすれば
			\begin{align}
				rk(\varepsilon x A) = rk(\varepsilon x A^{t})
			\end{align}
			が成り立つ.$A$に$e$が現れなければ$A^{t}$は$A$とする.
		\end{metathm}
	\end{screen}
	
	\begin{screen}
		\begin{metathm}[置換定理]
			$\pi$を$L(EC_{\varepsilon})$の証明とし,
			$e$を,$\pi$の主要$\varepsilon$項の中で階数が最大であって,かつ
			階数が最大の$\pi$の主要$\varepsilon$項の中で極大であるものとする.また
			$B(s) \Longrightarrow B(\varepsilon y B)$を$\pi$の主要論理式とし,
			$e$と$\varepsilon y B$は別物であるとする.そして,$B(s) \Longrightarrow B(\varepsilon y B)$に現れる
			$e$を全て閉項$t$に置き換えた式を$C$とする.このとき,
			\begin{description}
				\item[(1)] $C$は主要論理式である.$C$に属する$\varepsilon$項を$e'$と書く.
				\item[(2)] $rk(\varepsilon y B) = rk(e')$が成り立つ.
				\item[(3)] $rk(\varepsilon y B) = rk(e)$ならば$\varepsilon y B$と$e'$は一致する.
			\end{description}
		\end{metathm}
	\end{screen}
	
	\begin{metaprf}\mbox{}
		\begin{description}
			\item[step1]
				$B(s)$ (或いは$B(\varepsilon y B)$)とは,
				$B$で自由に現れる$y$を$s$ (或いは$\varepsilon y B$)で置き換えた式である.
				$y$から代わった$s$ (或いは$\varepsilon y B$)の少なくとも一つを部分項として含む形で
				$e$が$B(s)$ (或いは$B(\varepsilon y B)$)に出現しているとする.
				
				実はこれは起こり得ない.もし起きたとすると,$e$に現れる$s$ (或いは$\varepsilon y B$)
				を元の$y$に戻した項を$e'$とすれば,
				$e'$には$y$が自由に現れるので(そうでないと$y$は$s$
				(或いは$\varepsilon y B$)に置き換えられない),$e'$は
				$y$とは別の変項$x$と適当な式$A$によって
				\begin{align}
					\varepsilon x A
				\end{align}
				なる形をしている.つまり$e'$は$\varepsilon y B$に従属していることになり
				\footnote{
					$e'$が$\varepsilon$項であって$B$に現れることの証明.
				}
				,階数定理と併せて
				\begin{align}
					rk(e) = rk(e') < rk(\varepsilon y B)
				\end{align}
				が成り立ってしまう.しかしこれは$rk(e)$が最大であることに矛盾する.
				
			\item[step2] $rk(\varepsilon y B) = rk(\pi)$ならば$B$に$e$は現れない.なぜならば,
				$e$は階数が$rk(\pi)$である$\pi$の主要$\varepsilon$項の中で極大であるからである.
				$\varepsilon y B$にも$e$は現れず,前段の結果より$B(\varepsilon y B)$に$e$が現れることもない.
				ゆえに,$s$に現れる$e$を$t$に置換した項を$s'$とすれば,$C$は
				\begin{align}
					B(s') \Longrightarrow B(\varepsilon y B)
				\end{align}
				となる.
			
			\item[step3]
				$rk(\varepsilon y B) < rk(\pi)$である場合
				\begin{align}
					rk(\varepsilon y B) = rk(e')
				\end{align}
				が成り立つことを示す.$B$に$e$が現れないならば$e'$は$\varepsilon y B$に一致する.
				$B$に$e$が現れる場合,$B$に現れる$e$を$t$に置き換えた式を$B^{t}$とする.
				このとき階数定理より
				\begin{align}
					rk(B) = rk(B^{t})
				\end{align}
				となる.ゆえに
				\begin{align}
					rk(\varepsilon y B) = rk(B) + 1 = rk(B^{t}) + 1 = rk(\varepsilon y B^{t})
				\end{align}
				となる.
				\QED
		\end{description}
	\end{metaprf}
	
\subsection{アイデア}
	
	\begin{itembox}[l]{第一イプシロン定理の流れ}
		\begin{itemize}
			\item $B$を$EC$の式とし,$B$が$PC_{\varepsilon}$から証明可能であるとする.
			\item このとき$EC_{\varepsilon}から$$B$への証明$\pi$が得られる.
			\item $e$を,$\pi$の主要$\varepsilon$項のうち階数が最大であって,かつ
				その階数を持つ$\pi$の主要$\varepsilon$項の中で次数が最大であるものとする.
			\item $e$が属する$\pi$の主要論理式の一つ$A(t) \Longrightarrow A(e)$を取る.
			\item $\pi$をベースにして,$A(t) \Longrightarrow A(e)$を用いずに
				$EC_{\varepsilon}$から$B$への証明$\pi'$を構成する.このとき以下が満たされる.
				\begin{enumerate}
					\item $A(t) \Longrightarrow A(e)$を除く$\pi$の主要論理式は
						$\pi'$の主要論理式である.
					\item また$e$が属する主要論理式については,それが$\pi'$の主要論理式であるならば$\pi$の主要論理式
						でもある.つまり,直感的に書けば
						\begin{align}
							&\Set{\varphi}{\mbox{$\varphi$は$e$が属する$\pi'$の主要論理式}} \\
							&= \Set{\varphi}{\mbox{$\varphi$は$e$が属する$\pi$の主要論理式}} \backslash \{A(t) \Longrightarrow A(e)\}
						\end{align}
						が成り立つということであって,$e$が属する主要論理式は減る一方である.
						$\pi$の主要論理式で$e$が属しているものが$A(t) \Longrightarrow A(e)$のみ
						であるならば,$\pi'$には$e$が属する主要論理式は現れない.
					
					\item $e$が属する主要論理式が$\pi'$にも残っている場合,
						$e$は$\pi'$の主要$\varepsilon$項の中も階数が最大であって,
						かつその階数を持つ$\pi'$の主要$\varepsilon$項の中で極大である.
					
					\item $\pi'$の主要$\varepsilon$項のうち,階数が$e$と同じであるものは
						$\pi$の主要$\varepsilon$項でもあった($e$と同じ階数の$\varepsilon$項は増えない).
				\end{enumerate}
				
			\item 証明$\pi$の主要$\varepsilon$項の階数の最大値を$rk(\pi)$とする.
				また主要論理式に属する$\varepsilon$項の階数を,その主要論理式の階数と呼ぶことにする.
				前段の操作を続けていけば,まずは階数$rk(\pi)$の主要論理式を全く用いない
				$B$への証明$\pi_{1}$が得られる.このとき$rk(\pi_{1})$は
				$rk(\pi)$よりも小さい.同様にして階数$rk(\pi_{1})$の主要論理式を全く用いない
				$B$への証明$\pi_{2}$が得られる.もちろん$rk(\pi_{2})$は
				$rk(\pi_{1})$よりも小さい.これを繰り返していけば,いずれは主要論理式を全く用いない
				$B$への証明$\pi^{\ast}$が得られる.$\pi^{\ast}$にはトートロジーか
				モーダスポンネスで導かれる式しかない.
				あとは,$\pi^{\ast}$に現れる$\varepsilon$項を$EC$の項に置き換えれば,その式の列は
				$EC$から$B$への証明となっている.
		\end{itemize}
	\end{itembox}
	
	$\pi$を$\varphi_{0},\varphi_{1},\cdots,\varphi_{n}$とし,
	$\varphi_{0},\varphi_{1},\cdots,\varphi_{n}$に現れる$e$を$t$に置き換えた式を
	\begin{align}
		\tilde{\varphi}_{0},\ \tilde{\varphi}_{1},\cdots, \tilde{\varphi}_{n}
	\end{align}
	と書く($e$は,どれかの項の部分項であるときも置き換える?).
	このとき,任意の$0 \leq i \leq n$で
	\begin{enumerate}
		\item $\varphi_{i}$がトートロジーなら$\tilde{\varphi}_{i}$もトートロジーである.
		\item $\varphi_{i}$が主要論理式で,$e$が$\varphi_{i}$の主要項であるならば,
			$\tilde{\varphi}_{i}$は$A(u) \Longrightarrow A(t)$なる形の式である
			\footnotemark.
		\item $\varphi_{i}$が主要論理式で,$e$が$\varphi_{i}$の主要項ではないならば,
			$\tilde{\varphi}_{i}$も主要論理式である.
	\end{enumerate}
	
	\footnotetext{
		$\varepsilon x A$と$\varepsilon y B$が記号列として一致すれば,
		$x$と$y$は一致するし,式$A$と式$B$も一致するので
		$A(\varepsilon x A)$と$B(\varepsilon y B)$も記号列として一致する.
	}
	
	$\varphi$が$A(t) \Longrightarrow A(e)$でない$EC_{\varepsilon}$の公理ならば,
	$\tilde{\varphi}_{i}$と$\tilde{\varphi}_{i+1}$の間に
	\begin{align}
		&\tilde{\varphi}_{i} \Longrightarrow 
		\left( A(t) \Longrightarrow \tilde{\varphi}_{i} \right), \\
		&A(t) \Longrightarrow \tilde{\varphi}_{i}
	\end{align}
	を挿入する.$\varphi_{i}$が$\varphi_{j}$と$\varphi_{k}$からモーダスポンネスで得られる場合は,
	$\tilde{\varphi}_{i}$を
	\begin{align}
		&\left( A(t) \Longrightarrow \tilde{\varphi}_{j} \right)
		\Longrightarrow \left[ \left( A(t) \Longrightarrow 
		\left( \tilde{\varphi}_{j}\Longrightarrow \tilde{\varphi}_{i} \right) \right)
		\Longrightarrow \left( A(t) \Longrightarrow \tilde{\varphi}_{i} \right) \right], \\
		&\left( A(t) \Longrightarrow 
		\left( \tilde{\varphi}_{j}\Longrightarrow \tilde{\varphi}_{i} \right) \right)
		\Longrightarrow \left( A(t) \Longrightarrow \tilde{\varphi}_{i} \right), \\
		&A(t) \Longrightarrow \tilde{\varphi}_{i}
	\end{align}
	で置き換える.すると,$A(t) \Longrightarrow A(e)$を使わない
	$EC_{\varepsilon}$から$A(t) \Longrightarrow B$への証明が得られる.
	$\varphi_{i}$が$e$が属する主要論理式$A(s) \Longrightarrow A(e)$であるときは,
	$\tilde{\varphi}_{i}$とは
	\begin{align}
		A(s') \Longrightarrow A(t)
	\end{align}
	なる形の式であるが
	\footnote{
		$x$を$A$に現れている自由な変項とすれば,$e$とは$\varepsilon x A$のことであるし,
		$A(\varepsilon x A)$とは$A$に自由に現れる$x$を$\varepsilon x A$に置換した式である.
		$A$には$\varepsilon x A$は現れていないので,というのも$\varepsilon x A$が登場するのは
		$A$が作られた後であるからだが,$A(e)$に現れる$e$を$t$に変換した式は
		$A(t)$になる.同様に,$A(s)$に$e$が現れるとすれば,その$e$は$y$に代入された$s$の
		部分項でしかありえない.すなわち,$A(s)$に現れる$e$を$t$で置換した式は,
		$s'$を$s$に現れる$e$を$t$に変換した項として ($s$に$e$が現れなければ$s'$は$s$である)
		$A(s')$となるわけである.
	},$\tilde{\varphi}_{i}$を
	\begin{align}
		&A(t) \Longrightarrow (A(s') \Longrightarrow A(t)), \\
		&A(s') \Longrightarrow A(t)
	\end{align}
	で置き換える.
	
	同様に$A(t) \Longrightarrow A(e)$を使わない$EC_{\varepsilon})$から
	$\rightharpoondown A(t) \Longrightarrow B$への証明を構成する.
	今度は$\pi$に現れる$e$を$t$に置き換える必要はない.
	$\varphi_{i}$が$A(t) \Longrightarrow A(e)$でない$EC_{\varepsilon}$の公理ならば,
	$\varphi_{i}$と$\varphi_{i+1}$の間に
	\begin{align}
		&\varphi_{i} \Longrightarrow (\rightharpoondown A(t) \Longrightarrow \varphi_{i}), \\
		&\rightharpoondown A(t) \Longrightarrow \varphi_{i}
	\end{align}
	を挿入する.$\varphi_{i}$が$\varphi_{j}$と$\varphi_{k}$からモーダスポンネスで得られる場合は,
	$\varphi_{i}$を
	\begin{align}
		&(\rightharpoondown A(t) \Longrightarrow \varphi_{j}) \Longrightarrow
		[(\rightharpoondown A(t) \Longrightarrow 
		(\varphi_{j}\Longrightarrow \varphi_{i}))
		\Longrightarrow (\rightharpoondown A(t) \Longrightarrow \varphi_{i})], \\
		&(\rightharpoondown A(t) \Longrightarrow 
		(\varphi_{j} \Longrightarrow \varphi_{i}))
		\Longrightarrow (\rightharpoondown A(t) \Longrightarrow \varphi_{i}), \\
		&\rightharpoondown A(t) \Longrightarrow \varphi_{i}
	\end{align}
	で置き換える.$\varphi_{i}$が$A(t) \Longrightarrow A(e)$であるときは,$\varphi_{i}$を
	\begin{align}
		\rightharpoondown A(t) \Longrightarrow (A(t) \Longrightarrow A(e))
	\end{align}
	で置き換える.
	
	以上で$A(t) \Longrightarrow B$と$\rightharpoondown A(t) \Longrightarrow B$に対して
	$A(t) \Longrightarrow A(e)$を用いない$EC_{\varepsilon}$からの証明が得られた.後はこれに
	\begin{align}
		&(A(t) \Longrightarrow B) \Longrightarrow
		((\rightharpoondown A(t) \Longrightarrow B) \Longrightarrow
		((A(t) \Longrightarrow B) \wedge (\rightharpoondown A(t) \Longrightarrow B))), \\
		&(\rightharpoondown A(t) \Longrightarrow B) \Longrightarrow
		((A(t) \Longrightarrow B) \wedge (\rightharpoondown A(t) \Longrightarrow B)), \\
		&(A(t) \Longrightarrow B) \wedge (\rightharpoondown A(t) \Longrightarrow B), \\
		&((A(t) \Longrightarrow B) \wedge (\rightharpoondown A(t) \Longrightarrow B))
		\Longrightarrow ((A(t) \vee \rightharpoondown A(t)) \Longrightarrow B), \\
		&(A(t) \vee \rightharpoondown A(t)) \Longrightarrow B, \\
		&A(t) \vee \rightharpoondown A(t), \\
		&B
	\end{align}
	を追加すれば,$A(t) \Longrightarrow A(e)$を用いない$EC_{\varepsilon}$から$B$への証明となる.
	
	%\section{第二イプシロン定理}
	$\exists x \forall y \exists z B(x,y,z)$を$L(PC)$の冠頭標準形とする.
	つまり$B(x,y,z)$には量化子が現れないので,$B(x,y,z)$は$L(EC)$の式ということである.
	また
	\begin{align}
		PC_{\varepsilon} \vdash \exists x \forall y \exists z B(x,y,z)
	\end{align}
	であるとする.
	
	$f$を$L(PC)$には無い一変数関数記号とし,
	\begin{align}
		L'(PC) &\defeq L(PC) \cup \{f\}, \\
		L'(EC) &\defeq L(EC) \cup \{f\}, \\
		L'(PC_{\varepsilon}) &\defeq L(PC_{\varepsilon}) \cup \{f\}, \\
		L'(EC_{\varepsilon}) &\defeq L(EC_{\varepsilon}) \cup \{f\}
	\end{align}
	とする.このとき明らかに
	\begin{align}
		{PC'}_{\varepsilon} \vdash \exists x \forall y \exists z B(x,y,z)
	\end{align}
	であるが(ただし${PC'}_{\varepsilon} \vdash$とは$L'(PC_{\varepsilon})$の
	式からなる証明が存在するという意味),
	\begin{align}
		{PC'}_{\varepsilon} &\vdash \exists x \forall y \exists z B(x,y,z), \\
		{PC'}_{\varepsilon} &\vdash \exists x \forall y \exists z B(x,y,z)
		\rightarrow \forall y \exists z B(\tau,y,z), && 
		(\tau \defeq \varepsilon x \forall y \exists z B(x,y,z)) \\
		{PC'}_{\varepsilon} &\vdash \forall y \exists z B(\tau,y,z), \\
		{PC'}_{\varepsilon} &\vdash \forall y \exists z B(\tau,y,z)
		\rightarrow \exists z B(\tau,f(\tau),z), \\
		{PC'}_{\varepsilon} &\vdash \exists z B(\tau,f(\tau),z), \\
		{PC'}_{\varepsilon} &\vdash \exists z B(\tau,f(\tau),z)
		\rightarrow \exists x \exists z B(x,f(x),z), \\
		{PC'}_{\varepsilon} &\vdash \exists x \exists z B(x,f(x),z)
	\end{align}
	が成り立つ.すると拡張第一イプシロン定理より,$p$個の$L'(EC)$の項$r_{i}$
	と,同じく$p$個の$L'(EC)$の項$s_{i}$が取れて,
	\begin{align}
		{EC'}_{\varepsilon} \vdash \bigvee_{i=1}^{p} B(r_{i},f(r_{i}),s_{i})
	\end{align}
	となる.同じ証明で
	\begin{align}
		{PC'}_{\varepsilon} \vdash \bigvee_{i=1}^{p} B(r_{i},f(r_{i}),s_{i})
	\end{align}
	であることも言える.
	\begin{align}
		{PC'}_{\varepsilon} \vdash \bigvee_{i=1}^{p-1} B(r_{i},f(r_{i}),s_{i})
		\vee B(r_{p},f(r_{p}),s_{p})
	\end{align}
	より,まず
	\begin{align}
		{PC'}_{\varepsilon} \vdash \bigvee_{i=1}^{p-1} B(r_{i},f(r_{i}),s_{i})
		\vee \exists z B(r_{p},f(r_{p}),z)
	\end{align}
	となる.続いて,$f(r_{p})$は$\bigvee_{i=1}^{p-1} B(r_{i},f(r_{i}),s_{i})$には現れないので
	\begin{align}
		{PC'}_{\varepsilon} \vdash \bigvee_{i=1}^{p-1} B(r_{i},f(r_{i}),s_{i})
		\vee \forall y \exists z B(r_{p},y,z)
	\end{align}
	となる.最後に
	\begin{align}
		{PC'}_{\varepsilon} \vdash \bigvee_{i=1}^{p-1} B(r_{i},f(r_{i}),s_{i})
		\vee \exists x \forall y \exists z B(x,y,z)
	\end{align}
	となる.これを繰り返せば
	\begin{align}
		{PC'}_{\varepsilon} \vdash \exists x \forall y \exists z B(x,y,z)
		\vee \cdots \vee \exists x \forall y \exists z B(x,y,z)
	\end{align}
	が得られるので
	\begin{align}
		{PC'}_{\varepsilon} \vdash \exists x \forall y \exists z B(x,y,z)
	\end{align}
	となる.最後に,$\exists x \forall y \exists z B(x,y,z)$への証明に残っている
	$f$を含む項を$L(PC)$の項に置き換えれば,$L(PC)$から$\exists x \forall y \exists z B(x,y,z)$
	への証明が得られる.

%\chapter{メモ}
	%\subsection{量化}
	$\varphi$を$\mathcal{L}$の式とする.もし$\varphi$に$\forall$が現れたら,
	その$\forall$に後続する変項$x$と式$\psi$が取れるが,そのとき$x$は
	\begin{align}
		\forall x \psi
	\end{align}
	の中で{\bf 「量化されている」}\index{りょうか@量化}{\bf(quantified)}や
	{\bf 「束縛されている」}\index{そくばく@束縛}{\bf (bound)}という.
	同様に$\varphi$の中に$\exists$や$\varepsilon$が現れたら,
	その$\exists$ (または$\varepsilon$)の直後にくる変項は,
	「その$\exists$ (または$\varepsilon$)のスコープの中で束縛されている」といい,
	また$\varphi$の中に
	\begin{align}
		\Set{x}{\psi}
	\end{align}
	なる内包項が現れたら,$x$は「この内包項の中で束縛されている」という.
	他方で$\psi$の中に$x$とは別の変項が現れていても,その変項は
	$\forall x \psi,\ \exists x \psi,\ \varepsilon x \psi,\ \Set{x}{\psi}$
	の中では「束縛されていない」と解釈する.
	まとめれば,\underline{$\forall,\exists,\varepsilon,$そして$\{$は
	直後に来る変項のみをそのスコープ内で束縛している}のである.たとえば
	\begin{align}
		\forall x\, (\, x \in y\, )
	\end{align}
	においては$x$は束縛されているし,
	\begin{align}
		\Set{u}{u = z}
	\end{align}
	において$u$は束縛されている.束縛は二重に行われることもある.例えば
	\begin{align}
		\forall x\, (\, \forall x\, (\, x \in y\, ) \rarrow (\, x \in z\, )\, )
	\end{align}
	なる式においては,$\forall x\, (\, x \in y\, )$にある$x$は
	上式で一番左の$\forall$のスコープ内の$x$でもあるので,これらの$x$は二重に束縛されていることになる.
	仮に「何重にも束縛されている場合は最も広いスコープで束縛されていることにする」と決めても良いが,
	ただし重要なのは変項が束縛されているか否かであって,それが二重でも三重でもどうでも構わない.
	
	上の例では$y$と$z$は束縛されていないが,考えている項や式の中で束縛されていない変項
	を{\bf 自由な}\index{じゆう@自由}{\bf (free)}変項と呼ぶ.
	現れる変項が自由であるか否かは当然その出現位置に依存しているのであり,たとえば
	\begin{align}
		\forall x\, (\, x \in y\, ) \rarrow (\, x \in z\, )
	\end{align}
	なる式では左の二つの$x$が束縛されている一方で右の$x$は自由であるように,
	同じ変項が複数個所に現れる場合はその変項が束縛されているか自由であるかは一概には言えない.
	式$\varphi$の中に束縛されていない変項が現れている場合は,
	その変項が``その位置''に現れていることを
	{\bf 自由な出現}\index{じゆうなしゅつげん@自由な出現}{\bf (free occurrence)}と呼ぶ.
	
	\begin{screen}
		\begin{metadfn}[文]
			自由な変項が現れない$\mathcal{L}$の式を{\bf 文}\index{ぶん@文}{\bf (sentence)}
			や{\bf 閉式}\index{へいしき@閉式}{\bf (closed formula)}と呼ぶ.
		\end{metadfn}
	\end{screen}
	%\section{置換公理}
	置換公理の二つの形式の同値性をざっくりと.
	\begin{description}
		\item[(T)] $\sing{f} \Longrightarrow \forall a\, \set{f \ast a}.$
		\item[(K)] $\forall a\, \left[\, \forall x \in a\, \exists!y \varphi(x,y)
				\Longrightarrow \exists z\, \forall y\,
				(\, y \in z \Longleftrightarrow \exists x\, (\, x \in a \wedge 
				\varphi(x,y)\, )\, )\, \right].$
	\end{description}
	
	ただし
	\begin{align}
		\sing{f} &\defarrow \forall x,y,z\, (\, (x,y) \in f \wedge (x,z) \in f
		\Longrightarrow y = z\, ), \\
		f \ast a &\defeq \Set{y}{\exists x \in a\, (\, (x,y) \in f\, )}, \\
		\set{s} \defarrow \exists x\, (\, s = x\, )
	\end{align}
	であるし,$\varphi$に自由に現れているのは二つの変項のみで,それらが$s$と$t$とおけば,
	$\varphi$に自由に現れている$s$を全て$x$に,
	$\varphi$に自由に現れている$t$を全て$y$に置き換えた式が
	\begin{align}
		\varphi(x,y)
	\end{align}
	である.またこのとき$x$も$y$も$\varphi(x,y)$で束縛されていないものとする
	($x$と$y$はそのように選ばれた変項であるということである).
	
	\begin{description}
		\item[(T)$\Longrightarrow$(K)]
			$a$を任意の集合とし,
			\begin{align}
				\forall x \in a \exists!y \varphi(x,y)
			\end{align}
			であるとする.
			\begin{align}
				f \defeq \Set{(x,y)}{x \in a \wedge \varphi(x,y)}
			\end{align}
			とおけば$f$は$a$上の写像であって,(T)より
			\begin{align}
				\exists z\, (\, z = f \ast a\, )
			\end{align}
			となる.ところで$f \ast a$とは
			\begin{align}
				\Set{y}{\exists x \in a\, (\, (x,y) \in f\, )}
			\end{align}
			なので
			\begin{align}
				f \ast a = \Set{y}{\exists x \in a \varphi(x,y)}.
			\end{align}
			ゆえに
			\begin{align}
				\exists z\, \forall y\, (\, y \in z \Longleftrightarrow
				\exists x \in a \varphi(x,y)\, )
			\end{align}
			が成り立つ.
			
		\item[(K)$\Longrightarrow$(T)]
			$\sing{f}$とし,$a$を集合とする.
			\begin{align}
				b \defeq a \cap \dom{f}
			\end{align}
			とおけば,(K)からは分出公理が示せるので$b$は集合である.そして
			\begin{align}
				\forall x \in b\, \exists!y\, (\, (x,y) \in f\, )
			\end{align}
			が成り立つのだから,(K)より
			\begin{align}
				z = \Set{y}{\exists x \in b\, (\, (x,y) \in f\, )}
			\end{align}
			が従う.ここで
			\begin{align}
				\Set{y}{\exists x \in b\, (\, (x,y) \in f\, )}
				= f \ast b
				= f \ast a
			\end{align}
			であるから(T)が得られる.
			\QED
	\end{description}
	%\chapter{定理参照メモ}

\section{証明}
	\begin{screen}
		\begin{logicalaxm}[演繹規則]\ref{logicalaxm:deduction_rule}
			$A,B,C,D$を文とするとき,
			\begin{description}
				\item[(a)] $A \vdash D$ならば$\vdash A \rarrow D$が成り立つ.
				\item[(b)] $A,B \vdash D$ならば
					\begin{align}
						B \vdash A \rarrow D,\quad
						A \vdash B \rarrow D
					\end{align}
					が成り立つ.
				\item[(c)] $A,B,C \vdash D$ならば
					\begin{align}
						B,C \vdash A \rarrow D,\quad
						A,C \vdash B \rarrow D,\quad
						A,B \vdash C \rarrow D
					\end{align}
					のいずれも成り立つ.
			\end{description}
		\end{logicalaxm}
	\end{screen}
	
	\begin{screen}
		\begin{metadfn}[証明可能]
			文$\varphi$が公理系$\mathscr{S}$から
			{\bf 証明された}だとか{\bf 証明可能である}\index{しょうめいかのう@証明可能}
			{\bf (provable)}ということは,
			\begin{itemize}
				\item $\varphi$は$\mathscr{S}$の公理である.
				\item $\vdash \varphi$である.
				\item 文$\psi$で,$\psi$と$\psi \rightarrow \varphi$が$\mathscr{S}$から
				証明されているものが取れる({\bf 三段論法}\index{さんだんろんぽう@三段論法}
				{\bf (Modus Pones)}).
			\end{itemize}
			のいずれかが満たされているということであり,$\varphi$が$\mathscr{S}$から証明可能であることを
			\begin{align}
				\mathscr{S} \vdash \varphi
			\end{align}
			と書く.ただし,公理系に変項が生じた場合の証明可能性には
			演繹規則や後述の演繹法則,およびその逆の結果を適用することが出来る.
		\end{metadfn}
	\end{screen}
	
	\begin{screen}
		\begin{logicalthm}[含意の反射律]
		\ref{logicalthm:reflective_law_of_implication}
			$A$を文とするとき
			\begin{align}
				\vdash A \rarrow A.
			\end{align}
		\end{logicalthm}
	\end{screen}
	
	\begin{screen}
		\begin{logicalthm}[含意の導入]
		\ref{logicalthm:introduction_of_implication}
			$A,B$を文とするとき
			\begin{align}
				\vdash B \rarrow (A \rarrow B).
			\end{align}
		\end{logicalthm}
	\end{screen}
	
	\begin{screen}
		\begin{logicalthm}[含意の分配則]
		\ref{logicalthm:distributive_law_of_implication}
			$A,B,C$を文とするとき
			\begin{align}
				\vdash (A \rarrow (B \rarrow C)) \rarrow ((A \rarrow B) \rarrow (A \rarrow C)).
			\end{align}
		\end{logicalthm}
	\end{screen}
	
	\begin{screen}
		\begin{metaaxm}[証明に対する構造的帰納法]
			$\mathscr{S}$を公理系とし,Xを文に対する何らかの言明とするとき,
			\begin{itemize}
				\item $\mathscr{S}$の公理に対してXが言える.
				\item 推論法則に対してXが言える.
				\item $\varphi$と$\varphi \rarrow \psi$が$\mathscr{S}$の
					定理であるような文$\varphi$と文$\psi$が取れたとき,
					$\varphi$と$\varphi \rarrow \psi$に対して
					Xが言えるならば,$\psi$に対してXが言える.
			\end{itemize}
			のすべてが満たされていれば,$\mathscr{S}$から証明可能なあらゆる文に対してXが言える.
		\end{metaaxm}
	\end{screen}
	
	\begin{screen}
		\begin{metathm}[演繹法則]\ref{metathm:deduction_theorem}
			$\mathscr{S}$を公理系とし,$A$を文とするとき,
			$\mathscr{S}, A$の任意の定理$B$に対して
			\begin{align}
				\mathscr{S} \vdash A \rarrow B
			\end{align}
			が成り立つ.
		\end{metathm}
	\end{screen}
	
	\begin{screen}
		\begin{metathm}[公理が増えても証明可能]
			$\mathscr{S}$を公理系とし,$A$を文とするとき,
			$\mathscr{S}$の任意の定理$B$に対して
			\begin{align}
				\mathscr{S}, A \vdash B
			\end{align}
			が成り立つ.
		\end{metathm}
	\end{screen}
	
	\begin{screen}
		\begin{metathm}[演繹法則の逆]
		\ref{metathm:inverse_of_deduction_theorem}
			$\mathscr{S}$を公理系とし,$A$と$B$を文とするとき,
			\begin{align}
				\mathscr{S} \vdash A \rarrow B
			\end{align}
			であれば
			\begin{align}
				A,\ \mathscr{S} \vdash B
			\end{align}
			が成り立つ.
		\end{metathm}
	\end{screen}
	
\section{推論}
	\begin{screen}
		\begin{logicalaxm}[否定と矛盾に関する規則]
		\ref{logicalaxm:rules_of_negation_and_contradiction}
			$A$を文とするとき以下が成り立つ:
			\begin{description}
				\item[矛盾の導入] 否定が共に成り立つとき矛盾が起きる:
					\begin{align}
						A,\ \negation A \vdash \bot.
					\end{align}
				\item[否定の導入] 矛盾が導かれるとき否定が成り立つ:
					\begin{align}
						A \rarrow \bot \vdash\ \negation A.
					\end{align}
			\end{description}
		\end{logicalaxm}
	\end{screen}
	
	\begin{screen}
		\begin{dfn}[対偶]
			$\varphi \rarrow \psi$なる式に対して
			\begin{align}
				\negation \psi \rarrow\ \negation \varphi
			\end{align}
			を$\varphi \rarrow \psi$の{\bf 対偶}\index{たいぐう@対偶}
			{\bf (contraposition)}と呼ぶ.
		\end{dfn}
	\end{screen}
	
	\begin{screen}
		\begin{logicalthm}[対偶命題が導かれる]
		\ref{logicalthm:introduction_of_contraposition}
			$A$と$B$を文とするとき
			\begin{align}
				\vdash (\, A \rarrow B\, ) 
				\rarrow (\, \negation B \rarrow \negation A\, ).
			\end{align}
		\end{logicalthm}
	\end{screen}
	
	\begin{screen}
		\begin{dfn}[二重否定]
			式$\varphi$に対して,$\negation$を二つ連結させた式
			\begin{align}
				\negation \negation \varphi
			\end{align}
			を$\varphi$の{\bf 二重否定}\index{にじゅうひてい@二重否定}
			{\bf (double negation)}と呼ぶ.
		\end{dfn}
	\end{screen}
	
	\begin{screen}
		\begin{logicalthm}[二重否定の導入]
		\ref{logicalthm:introduction_of_double_negation}
			$A$を文とするとき
			\begin{align}
				\vdash A \rarrow \negation \negation A.
			\end{align}
		\end{logicalthm}
	\end{screen}
	
	\begin{screen}
		\begin{logicalaxm}[論理積の除去]
		\ref{logicalaxm:elimination_of_conjunction}
			$A$と$B$を文とするとき
			\begin{align}
				A &\wedge B \vdash A, \\
				A &\wedge B \vdash B.
			\end{align}
		\end{logicalaxm}
	\end{screen}
	
	\begin{screen}
		\begin{logicalthm}[無矛盾律]
		\ref{logicalthm:law_of_noncontradiction}
			$A$を文とするとき
			\begin{align}
				\vdash\ \negation (\, A \wedge \negation A\, ).
			\end{align}
		\end{logicalthm}
	\end{screen}
	
	\begin{screen}
		\begin{dfn}[同値記号]
			$A$と$B$を$\mathcal{L}$の式とするとき,
			\begin{align}
				A \lrarrow B  \defarrow
				(\, A \rarrow B\, ) \wedge (\, B \rarrow A\, )
			\end{align}
			により$\lrarrow$を定め,式`$A \lrarrow B$'を
			「$A$と$B$は{\bf 同値である}\index{どうち@同値}{\bf (equivalent)}」と読む.
		\end{dfn}
	\end{screen}
	
	\begin{screen}
		\begin{logicalaxm}[場合分け規則]
		\ref{logicalaxm:elimination_of_disjunction}
			$A$と$B$と$C$を文とするとき
			\begin{align}
				A \rarrow C,\ B \rarrow C \vdash A \vee B \rarrow C.
			\end{align}
		\end{logicalaxm}
	\end{screen}
	
	\begin{screen}
		\begin{logicalthm}[弱 De Morgan の法則(1)]
		\ref{logicalthm:weak_De_Morgan_law_1}
			$A$と$B$を文とするとき
			\begin{align}
				\vdash\ \negation A \wedge \negation B
				\rarrow\ \negation (\, A \vee B\, ).
			\end{align}
		\end{logicalthm}
	\end{screen}
	
	\begin{screen}
		\begin{logicalthm}[強 De Morgan の法則(1)]
		\ref{logicalthm:strong_De_Morgan_law_1}
			$A$と$B$を文とするとき
			\begin{align}
				\vdash\ \negation A \vee \negation B
				\rarrow\ \negation (\, A \wedge B\, ).
			\end{align}
		\end{logicalthm}
	\end{screen}
	
	\begin{screen}
		\begin{logicalaxm}[論理和の導入]
		\ref{logicalaxm:introduction_of_disjunction}
			$A$と$B$を文とするとき
			\begin{align}
				A &\vdash A \vee B, \\
				B &\vdash A \vee B.
			\end{align}
		\end{logicalaxm}
	\end{screen}
	
	\begin{screen}
		\begin{logicalthm}[論理和の可換律]
		\ref{logicalthm:commutative_law_of_disjunction}
			$A,B$を文とするとき
			\begin{align}
				\vdash A \vee B \rarrow B \vee A.
			\end{align}
		\end{logicalthm}
	\end{screen}
	
	\begin{screen}
		\begin{logicalaxm}[論理積の導入]
		\ref{logicalaxm:introduction_of_conjunction}
			$A$と$B$を文とするとき
			\begin{align}
				A,\ B \vdash A \wedge B.
			\end{align}
		\end{logicalaxm}
	\end{screen}
	
	\begin{screen}
		\begin{logicalthm}[弱 De Morgan の法則(2)]
		\ref{logicalthm:weak_De_Morgan_law_2}
			$A$と$B$を文とするとき
			\begin{align}
				\vdash\ \negation (\, A \vee B\, ) 
				\rarrow\ \negation A \wedge \negation B.
			\end{align}
		\end{logicalthm}
	\end{screen}
	
	\begin{screen}
		\begin{logicalthm}[論理積の可換律]
		\ref{logicalthm:commutative_law_of_conjunction}
			$A,B$を文とするとき
			\begin{align}
				\vdash A \wedge B \rarrow B \wedge A.
			\end{align}
		\end{logicalthm}
	\end{screen}
	
	\begin{screen}
		\begin{logicalaxm}[二重否定の除去]
		\ref{logicalaxm:elimination_of_double_negation}
			$A$を文とするとき以下が成り立つ:
			\begin{align}
				\negation \negation A \vdash A.
			\end{align}
		\end{logicalaxm}
	\end{screen}
	
	\begin{screen}
		\begin{logicalthm}[対偶論法の原理]
		\ref{logicalthm:proof_by_contraposition}
			$A$と$B$を文とするとき
			\begin{align}
				\vdash (\, \negation B \rarrow\ \negation A\, )
				\rarrow (\, A \rarrow B\, ).
			\end{align}
		\end{logicalthm}
	\end{screen}
	
	\begin{screen}
		\begin{logicalthm}[背理法の原理]
		\ref{logicalthm:proof_by_contradiction}
			$A$を文とするとき
			\begin{align}
				\vdash (\, \negation A \rarrow \bot\, ) \rarrow A.
			\end{align}
		\end{logicalthm}
	\end{screen}
	
	\begin{screen}
		\begin{logicalthm}[爆発律]
		\ref{logicalthm:principle_of_explosion}
			$A$を文とするとき
			\begin{align}
				\vdash \bot \rarrow A.
			\end{align}
		\end{logicalthm}
	\end{screen}
	
	\begin{screen}
		\begin{logicalthm}[否定の論理和は含意で書ける]
		\ref{logicalthm:disjunction_of_negation_rewritable_by_implication}
			$A$と$B$を文とするとき
			\begin{align}
				\vdash\ \negation A \vee B \rarrow (\, A \rarrow B\, ).
			\end{align}
		\end{logicalthm}
	\end{screen}
	
	\begin{screen}
		\begin{logicalthm}[排中律]\ref{logicalthm:law_of_excluded_middle}
			$A$を文とするとき
			\begin{align}
				\vdash A \vee \negation A.
			\end{align}
		\end{logicalthm}
	\end{screen}
	
	\begin{screen}
		\begin{logicalthm}[含意の論理和への遺伝性]
		\ref{logicalthm:heredity_of_implication_to_disjunction}
			$A,B,C$を文とするとき
			\begin{align}
				\vdash (\, A \rarrow B\, ) \rarrow (\, A \vee C \rarrow B \vee C\, ).
			\end{align}
		\end{logicalthm}
	\end{screen}
	
	\begin{screen}
		\begin{logicalthm}[含意は否定と論理和で表せる]
		\ref{logicalthm:implication_rewritable_by_disjunction_of_negation}
			$A$と$B$を文とするとき
			\begin{align}
				\vdash (\, A \rarrow B\, ) \rarrow (\, \negation A \vee B\, ).
			\end{align}
		\end{logicalthm}
	\end{screen}
	
	\begin{screen}
		\begin{logicalthm}[強 De Morgan の法則(2)]
		\ref{logicalthm:strong_De_Morgan_law_2}
			$A$と$B$を文とするとき
			\begin{align}
				\vdash\ \negation (\, A \wedge B\, )
				\rarrow\ \negation A \vee \negation B.
			\end{align}
		\end{logicalthm}
	\end{screen}
	
	\begin{screen}
		\begin{logicalaxm}[量化記号に関する規則]
		\ref{logicalaxm:rules_of_quantifiers}
			$A$を$\mathcal{L}$の式とし,$x$を$A$に自由に現れる変項とし,
			$A$に自由に現れる項が$x$のみであるとする.
			また$\tau$を任意の$\varepsilon$項とする.このとき以下を推論規則とする.
			\begin{align}
				A(\tau) &\vdash \exists x A(x), \\
				\exists x A(x) &\vdash A(\varepsilon x A(x)), \\
				\forall x A(x) &\vdash A(\tau), \\
				A(\varepsilon x \negation A(x)) &\vdash \forall x A(x).
			\end{align}
		\end{logicalaxm}
	\end{screen}
	
	\begin{screen}
		\begin{logicalthm}[量化記号に対する弱 De Morgan の法則(1)]
		\label{logicalthm:weak_De_Morgan_law_for_quantifiers_1}
			$A$を$\mathcal{L}$の式とし,$x$を$A$に自由に現れる変項とし,
			また$A$に自由に現れる変項は$x$のみであるとする.このとき
			\begin{align}
				\vdash \exists x \negation A(x) \rarrow\ \negation \forall x A(x).
			\end{align}
		\end{logicalthm}
	\end{screen}
	
	\begin{screen}
		\begin{logicalthm}[量化記号に対する弱 De Morgan の法則(2)]
		\label{logicalthm:weak_De_Morgan_law_for_quantifiers_2}
			$A$を$\mathcal{L}$の式とし,$x$を$A$に自由に現れる変項とし,
			また$A$に自由に現れる変項は$x$のみであるとする.このとき
			\begin{align}
				\vdash\ \negation \forall x A(x) \rarrow \exists x \negation A(x).
			\end{align}
		\end{logicalthm}
	\end{screen}
	
	\begin{screen}
		\begin{logicalthm}[量化記号に対する強 De Morgan の法則(1)]
		\label{logicalthm:strong_De_Morgan_law_for_quantifiers_1}
			$A$を$\mathcal{L}$の式とし,$x$を$A$に自由に現れる変項とし,
			また$A$に自由に現れる変項は$x$のみであるとする.このとき
			\begin{align}
				\vdash \forall x \negation A(x) \rarrow\ \negation \exists x A(x).
			\end{align}
		\end{logicalthm}
	\end{screen}
	
	\begin{screen}
		\begin{logicalthm}[量化記号に対する強 De Morgan の法則(2)]
		\label{logicalthm:strong_De_Morgan_law_for_quantifiers_2}
			$A$を$\mathcal{L}$の式とし,$x$を$A$に自由に現れる変項とし,
			また$A$に自由に現れる変項は$x$のみであるとする.このとき
			\begin{align}
				\vdash\ \negation \exists x A(x) \rarrow \forall x \negation A(x).
			\end{align}
		\end{logicalthm}
	\end{screen}
	
\section{その他の推論法則}
	\begin{screen}
		\begin{logicalthm}[含意の推移律]
		\ref{logicalthm:transitive_law_of_implication}
			$A,B,C$を文とするとき
			\begin{align}
				\vdash (A \rarrow B) \rarrow ((B \rarrow C) \rarrow (A \rarrow C)).
			\end{align}
		\end{logicalthm}
	\end{screen}
	
	\begin{screen}
		\begin{logicalthm}[二式が同時に導かれるならその論理積が導かれる]
		\ref{logicalthm:conjunction_of_consequences}
			$A,B,C$を文とするとき
			\begin{align}
				\vdash (A \rarrow B) \rarrow ((A \rarrow C) 
				\rarrow (A \rarrow B \wedge C)).
			\end{align}
		\end{logicalthm}
	\end{screen}
	
	\begin{screen}
		\begin{logicalthm}[含意は遺伝する]
		\ref{logicalthm:rule_of_inference_1}
			$A,B,C$を$\mathcal{L}'$の閉式とするとき以下が成り立つ:
			\begin{description}
				\item[(a)] $(A \rarrow B) \rarrow ( (A \vee C) \rarrow (B \vee C) )$.
				
				\item[(b)] $(A \rarrow B) \rarrow ( (A \wedge C) \rarrow (B \wedge C) )$.
				
				\item[(c)] $(A \rarrow B) \rarrow ( (B \rarrow C) \rarrow (A \rarrow C) )$.
				
				\item[(c)] $(A \rarrow B) \rarrow ( (C \rarrow A) \rarrow (C \rarrow B) )$.
			\end{description}
		\end{logicalthm}
	\end{screen}
	
	\begin{screen}
		\begin{logicalthm}[同値記号の遺伝性質]
		\ref{logicalthm:hereditary_of_equivalence}
			$A,B,C$を$\mathcal{L}'$の閉式とするとき以下の式が成り立つ:
			\begin{description}
				\item[(a)] $(A \lrarrow B) \rarrow ((A \vee C) \lrarrow (B \vee C))$.
				\item[(b)] $(A \lrarrow B) \rarrow ((A \wedge C) \lrarrow (B \wedge C))$.
				\item[(c)] $(A \lrarrow B) \rarrow ((B \rarrow C) \lrarrow (A \rarrow C))$.
				
				\item[(d)] $(A \lrarrow B) \rarrow ((C \rarrow A) \lrarrow (C \rarrow B))$.
			\end{description}
		\end{logicalthm}
	\end{screen}
	
	\begin{screen}
		\begin{logicalthm}[偽な式は矛盾を導く]
		\ref{logicalthm:false_and_negation_are_equivalent}
			$A$を文とするとき
			\begin{align}
				\vdash\ \negation A \rarrow (A \rarrow \bot).
			\end{align}
		\end{logicalthm}
	\end{screen}
	
	\begin{screen}
		\begin{thm}[類は集合であるか真類であるかのいずれかに定まる]
			$a$を類とするとき
			\begin{align}
				\vdash \set{a} \vee \negation \set{a}.
			\end{align}
		\end{thm}
	\end{screen}
	
	\begin{screen}
		\begin{logicalthm}[矛盾を導く式はあらゆる式を導く]
		\ref{logicalthm:formula_leading_to_contradiction_derives_any_formula}
			$A,B$を文とするとき
			\begin{align}
				\vdash (A \rarrow \bot) \rarrow (A \rarrow B).
			\end{align}
		\end{logicalthm}
	\end{screen}
	
	\begin{screen}
		\begin{logicalthm}[含意は否定と論理和で表せる]
		\ref{logicalthm:rule_of_inference_3}
			$A,B$を文とするとき
			\begin{align}
				\vdash (A \rarrow B) \lrarrow (\negation A \vee B).
			\end{align}
		\end{logicalthm}
	\end{screen}
	
	\begin{screen}
		\begin{logicalthm}[二重否定の法則の逆が成り立つ]
		\ref{logicalthm:converse_of_law_of_double_negative}
			$A$を文とするとき
			\begin{align}
				\vdash A \rarrow \negation \negation A.
			\end{align}
		\end{logicalthm}
	\end{screen}
	
	\begin{screen}
		\begin{logicalthm}[対偶命題は同値]\ref{thm:contraposition_is_true}
			$A,B$を文とするとき
			\begin{align}
				\vdash (A \rarrow B) \lrarrow (\negation B \rarrow \negation A).
			\end{align}
		\end{logicalthm}
	\end{screen}
	
	\begin{screen}
		\begin{logicalthm}[De Morganの法則]
			$A,B$を文とするとき
			\begin{itemize}
				\item $\vdash\ \negation (A \vee B) \lrarrow \negation A \wedge \negation B$.
			
				\item $\vdash\ \negation (A \wedge B) \lrarrow \negation A \vee \negation B$.
			\end{itemize}
		\end{logicalthm}
	\end{screen}
	
	\begin{screen}
		\begin{thm}[集合であり真類でもある類は存在しない]
			$a$を類とするとき
			\begin{align}
				\vdash\ \negation (\ \set{a} \wedge \negation \set{a}\ ).
			\end{align}
		\end{thm}
	\end{screen}
	
	\begin{screen}
		\begin{logicalthm}[量化記号の性質(ロ)]
		\ref{logicalthm:properties_of_quantifiers_2}
			$A,B$を$\mathcal{L}'$の式とし,$x$を$A,B$に現れる文字とするとき,$x$のみが$A,B$で量化されていないならば以下は定理である:
			\begin{description}
				\item[(a)] $\exists x ( A(x) \vee B(x) ) \lrarrow \exists x A(x) \vee \exists x B(x)$.
				
				\item[(b)] $\forall x ( A(x) \wedge B(x) ) \lrarrow \forall x A(x) \wedge \forall x B(x)$.
			\end{description}
		\end{logicalthm}
	\end{screen}
	
	\begin{screen}
		\begin{logicalthm}[量化記号の性質(イ)]
		\ref{logicalthm:properties_of_quantifiers}
			$A,B$を$\mathcal{L}'$の式とし,$x$を$A,B$に現れる文字とし,$x$のみが$A,B$で量化されていないとする.
			$\mathcal{L}$の任意の対象$\tau$に対して
			\begin{align}
				A(\tau) \lrarrow B(\tau)
			\end{align}
			が成り立っているとき,
			\begin{align}
				\exists x A(x) \lrarrow \exists x B(x)
			\end{align}
			および
			\begin{align}
				\forall x A(x) \lrarrow \forall x B(x)
			\end{align}
			が成り立つ.
		\end{logicalthm}
	\end{screen}
	
	\begin{screen}
		\begin{logicalthm}[量化記号に対する De Morgan の法則]
		\ref{logicalthm:De_Morgan_law_for_quantifiers}
			$A$を$\mathcal{L}'$の式とし,$x$を$A$に現れる文字とし,$x$のみが$A$で量化されていないとする.このとき
			\begin{align}
				\exists x \negation A(x) \lrarrow \negation \forall x A(x)
			\end{align}
			および
			\begin{align}
				\forall x \negation A(x) \lrarrow \negation \exists x A(x)
			\end{align}
			が成り立つ.
		\end{logicalthm}
	\end{screen}

\section{集合}
	\begin{screen}
		\begin{dfn}[集合]
			$a$を類とするとき,$a$が集合であるという言明を
			\begin{align}
				\set{a} \defarrow \exists x\, (\, a = x\, )
			\end{align}
			で定める.$\Sigma \vdash \set{a}$を満たす類$a$を
			{\bf 集合}\index{しゅうごう@集合}{\bf (set)}と呼び,
			$\Sigma \vdash\ \negation \set{a}$を満たす類$a$を
			{\bf 真類}\index{しんるい@真類}{\bf (proper class)}と呼ぶ.
		\end{dfn}
	\end{screen}
	
	\begin{screen}
		\begin{thm}[集合である内包項は$\varepsilon$項で書ける]
			$\varphi$を$\mathcal{L}$の式とし,$x$を$\varphi$に自由に現れる変項とし,
			$x$のみが$\varphi$で自由であるとする.このとき
			\begin{align}
				\set{\Set{x}{\varphi(x)}} \vdash \Set{x}{\varphi(x)} 
				= \varepsilon y\, \forall x\, (\, \varphi(x) \lrarrow x \in y\, ).
			\end{align}
		\end{thm}
	\end{screen}
	
\section{相等性}
	\begin{screen}
		\begin{axm}[外延性の公理 (Extensionality)]
			任意の類$a,b$に対して
			\begin{align}
				\EXTAX \defarrow \forall x\, (\, x \in a \lrarrow x \in b\, ) 
				\rarrow a=b.
			\end{align}
		\end{axm}
	\end{screen}
	
	\begin{screen}
		\begin{thm}[任意の類は自分自身と等しい]\ref{thm:any_class_equals_to_itself}
			任意の類$\tau$に対して
			\begin{align}
				\EXTAX \vdash \tau = \tau.
			\end{align}
		\end{thm}
	\end{screen}
	
	\begin{screen}
		\begin{thm}[類である$\varepsilon$項は集合である]
			$\tau$を類である$\varepsilon$項とするとき
			\begin{align}
				\EXTAX \vdash \set{\tau}.
			\end{align}
		\end{thm}
	\end{screen}
	
	\begin{screen}
		\begin{axm}[相等性公理]
			$a,b,c$を類とするとき
			\begin{align}
				\EQAX \defarrow
				\begin{cases}
					a = b \rarrow b = a, & \\
					a = b \rarrow (\, a \in c \rarrow b \in c\, ), & \\
					a = b \rarrow (\, c \in a \rarrow c \in b\, ). & 
				\end{cases}
			\end{align}
		\end{axm}
	\end{screen}
	
	\begin{screen}
		\begin{thm}[外延性の公理の逆も成り立つ]
		\ref{thm:inverse_of_axiom_of_extensionality}
			$a$と$b$を類とするとき
			\begin{align}
				\EQAX \vdash 
				a = b \rarrow \forall x\, (\, x \in a  \lrarrow x \in b\, ).
			\end{align}
		\end{thm}
	\end{screen}
	
	\begin{screen}
		\begin{axm}[内包性公理] 
			$\varphi$を$\mathcal{L}$の式とし,$y$を$\varphi$に自由に現れる変項とし,
			$\varphi$に自由に現れる項は$y$のみであるとし,
			$x$は$\varphi$で$y$への代入について自由であるとするとき,
			\begin{align}
				\COMAX \defarrow \forall x\, (\, x \in \Set{y}{\varphi(y)} \lrarrow \varphi(x)\, ).
			\end{align}
		\end{axm}
	\end{screen}
	
	\begin{screen}
		\begin{thm}[条件を満たす集合は要素である]\ref{thm:satisfactory_set_is_an_element}
			$\varphi$を$\mathcal{L}$の式とし,$x$を$\varphi$に自由に現れる変項とし,
			$x$のみが$\varphi$で束縛されていないとする.このとき,任意の類$a$に対して
			\begin{align}
				\EQAX,\COMAX \vdash \varphi(a) \rarrow 
				\left(\, \set{a} \rarrow a \in \Set{x}{\varphi(x)}\, \right).
			\end{align}
		\end{thm}
	\end{screen}
	
	\begin{screen}
		\begin{dfn}[宇宙]
			$\Univ \defeq \Set{x}{x=x}$で定める類$\Univ$を{\bf 宇宙}\index{うちゅう@宇宙}
			{\bf (Universe)}と呼ぶ.
		\end{dfn}
	\end{screen}
	
	\begin{screen}
		\begin{axm}[要素の公理]
			要素となりうる類は集合である.つまり,$a,b$を類とするとき
			\begin{align}
				\ELEAX \defarrow a \in b \rarrow \set{a}.
			\end{align}
		\end{axm}
	\end{screen}
	
	\begin{screen}
		\begin{thm}[$\Univ$は集合の全体である]
		\ref{thm:V_is_the_whole_of_sets}
			$a$を類とするとき次が成り立つ:
			\begin{align}
				\EXTAX,\EQAX,\ELEAX,\COMAX \vdash \set{a} \lrarrow a \in \Univ.
			\end{align}
		\end{thm}
	\end{screen}
	
	\begin{screen}
		\begin{logicalthm}[同値関係の可換律]
		\ref{logicalthm:commutative_law_of_equivalence_symbol}
			$A,B$を$\mathcal{L}$の文とするとき
			\begin{align}
				\vdash (A \lrarrow B) \rarrow (B \lrarrow A).
			\end{align}
		\end{logicalthm}
	\end{screen}
	
	\begin{screen}
		\begin{logicalthm}[同値関係の推移律]
		\ref{logicalthm:transitive_law_of_equivalence_symbol}
			$A,B,C$を$\mathcal{L}$の文とするとき
			\begin{align}
				\vdash (A \lrarrow B) \rarrow ((B \lrarrow C) \rarrow 
				(A \lrarrow C)).
			\end{align}
		\end{logicalthm}
	\end{screen}
	
	\begin{screen}
		\begin{thm}[等号の推移律]\ref{thm:transitive_law_of_equality}
			$a,b,c$を類とするとき
			\begin{align}
				\EXTAX,\EQAX \vdash a = b \rarrow (\, a = c \rarrow b = c\, ).
			\end{align}
		\end{thm}
	\end{screen}
	
\section{代入原理}
	\begin{screen}
		\begin{axm}[$\varepsilon$項に対する相等性公理]
			$a,b$を類とし,$\varphi$を$\lang{\varepsilon}$の式とし,$\varphi$には変項$x,y$が
			自由に現れ,また$\varphi$に自由に現れる変項はこれらのみであるとする.このとき
			\begin{align}
				\EQAXEP \defarrow
				a = b \rarrow \varepsilon x \varphi(x,a) = \varepsilon x \varphi(x,b).
			\end{align}
		\end{axm}
	\end{screen}
	
	\begin{screen}
		\begin{thm}[代入原理]\ref{thm:the_principle_of_substitution}
			$a,b$を類とし,$\varphi$を$\mathcal{L}$の式とし,$x$を$\varphi$に自由に現れる変項
			とし,$\varphi$に自由に現れる変項は$x$のみであるとする.このとき
			\begin{align}
				\EXTAX,\EQAX,\EQAXEP \vdash a = b \rarrow 
				(\, \varphi(a) \lrarrow \varphi(b)\, ).
			\end{align}
		\end{thm}
	\end{screen}

\section{空集合}
	\begin{screen}
		\begin{logicalthm}[分配された論理積の簡約]
		\ref{logicalthm:contraction_law_of_distributed_injunctions}
			$A,B,C$を$\mathcal{L}$の文とするとき,
			\begin{align}
				\vdash (A \wedge C) \wedge (B \wedge C) \rarrow A \wedge B.
			\end{align}
		\end{logicalthm}
	\end{screen}
	
	\begin{screen}
		\begin{dfn}[空集合]
			$\emptyset \defeq \Set{x}{x \neq x}$で定める類$\emptyset$を{\bf 空集合}\index{くうしゅうごう@空集合}{\bf (empty set)}と呼ぶ.
		\end{dfn}
	\end{screen}
	
	\begin{screen}
		\begin{axm}[置換公理]
			$\varphi$を$\mathcal{L}$の式とし,
			$s,t$を$\varphi$に自由に現れる変項とし,
			$\varphi$に自由に現れる項は$s,t$のみであるとし,
			$x$は$\varphi$で$s$への代入について自由であり,
			$y,z,v$は$\varphi$で$t$への代入について自由であるとするとき,
			\begin{align}
				\REPAX \defarrow \forall x\, \forall y\, \forall z\, 
				(\, \varphi(x,y) \wedge \varphi(x,z)
				\rarrow y = z\, )
				\rarrow \forall a\, \exists u\, \forall v\,
				(\, v \in u \lrarrow \exists x\, (\, x \in a \wedge 
				\varphi(x,v)\, )\, ).
			\end{align}
		\end{axm}
	\end{screen}
	
	\begin{screen}
		\begin{thm}[分出定理]\ref{thm:axiom_of_separation}
			$\varphi$を$\mathcal{L}$の式とし,$x$を$\varphi$に自由に現れる変項とし,
			$\varphi$に自由に現れる項は$x$のみであるとする.このとき
			\begin{align}
				\EXTAX,\EQAX,\EQAXEP,\REPAX \vdash 
				\forall a\, \exists s\, \forall x\,
				(\, x \in s \lrarrow x \in a \wedge \varphi(x)\, ).
			\end{align}
		\end{thm}
	\end{screen}
	
	\begin{screen}
		\begin{thm}[$\emptyset$は集合]\ref{thm:emptyset_is_a_set}
			\begin{align}
				\EXTAX,\EQAX,\COMAX,\REPAX \vdash \set{\emptyset}.
			\end{align}
		\end{thm}
	\end{screen}
	
	\begin{screen}
		\begin{thm}[空集合はいかなる集合も持たない]\ref{thm:emptyset_has_nothing}
			\begin{align}
				\EXTAX,\COMAX \vdash \forall x\, (\, x \notin \emptyset\, ).
			\end{align}
		\end{thm}
	\end{screen}
	
	\begin{screen}
		\begin{thm}[空の類は空集合に等しい]\ref{thm:uniqueness_of_emptyset}
			$a$を類とするとき
			\begin{align}
				\EXTAX,\COMAX &\vdash \forall x\, (\, x \notin a\, ) \rarrow a = \emptyset, \\
				\EXTAX,\EQAX,\COMAX &\vdash a = \emptyset \rarrow \forall x\, (\, x \notin a\, ).
			\end{align}
		\end{thm}
	\end{screen}
	
	\begin{screen}
		\begin{thm}[類を要素として持てば空ではない]
		\ref{thm:emptyset_does_not_contain_any_class}
			$a,b$を類とするとき
			\begin{align}
				\EQAX,\ELEAX \vdash a \in b \rarrow \exists x\, (\, x \in b\, ).
			\end{align}
		\end{thm}
	\end{screen}
	
	\begin{screen}
		\begin{dfn}[部分類]
			$x,y$を$\mathcal{L}$の項とするとき,
			\begin{align}
				x \subset y \defarrow
				\forall z\, (\, z \in x \rarrow z \in y\, )
			\end{align}
			と定める.式$z \subset y$を「$x$は$y$の{\bf 部分類}\index{ぶぶんるい@部分類}
			{\bf (subclass)}である」や「$x$は$y$に含まれる」などと翻訳し,特に$x$が集合である場合は
			「$x$は$y$の{\bf 部分集合}\index{ぶぶんしゅうごう@部分集合}{\bf (subset)}である」
			と翻訳する.また
			\begin{align}
				x \subsetneq y \defarrow x \subset y \wedge x \neq y
			\end{align}
			と定め,これを「$x$は$y$に{\bf 真に含まれる}」と翻訳する.
		\end{dfn}
	\end{screen}
	
	\begin{screen}
		\begin{thm}[空集合は全ての類に含まれる]
		\ref{thm:emptyset_if_a_subset_of_every_class}
			$a$を類とするとき
			\begin{align}
				\EXTAX,\COMAX \vdash \emptyset \subset a.
			\end{align}
		\end{thm}
	\end{screen}
	
	\begin{screen}
		\begin{thm}[類はその部分類に属する全ての類を要素に持つ]
		\ref{thm:subclass_contains_all_elements}
			$a,b,c$を類とするとき
			\begin{align}
				\EQAX,\ELEAX \vdash 
				a \subset b \rarrow (\, c \in a \rarrow c \in b\, ).
			\end{align}
		\end{thm}
	\end{screen}
	
	\begin{screen}
		\begin{thm}[$\Univ$は最大の類である]
			$a$を類とするとき
			\begin{align}
				\EXTAX,\COMAX \vdash a \subset \Univ.
			\end{align}
		\end{thm}
	\end{screen}
	
	\begin{screen}
		\begin{thm}[等しい類は相手を包含する]
		\ref{thm:equivalent_classes_includes_the_other}
			$a,b$を類とするとき
			\begin{align}
				\EQAX \vdash a = b \rarrow a \subset b \wedge b \subset a.
			\end{align}
		\end{thm}
	\end{screen}
	
	\begin{screen}
		\begin{thm}[互いに相手を包含する類同士は等しい]
		\ref{thm:mutually_including_classes_are_equivalent}
			$a,b$を類とするとき
			\begin{align}
				\EXTAX \vdash a \subset b \wedge b \subset a \rarrow a = b.
			\end{align}
		\end{thm}
	\end{screen}

\section{変換の同値性}
	\begin{screen}
		\begin{logicalthm}[同値記号の対称律]
		\ref{logicalthm:symmetry_of_equivalence_arrows}
			$A,B$を$\mathcal{L}$の文とするとき
			\begin{align}
				\vdash (A \lrarrow B) \rarrow (B \lrarrow A).
			\end{align}
		\end{logicalthm}
	\end{screen}
	
	\begin{screen}
		\begin{thm}
		\ref{thm:equivalent_formula_rewriting_1}
			$a$を主要$\varepsilon$項とし,$\psi$を$\lang{\varepsilon}$の式とし,
			$z$を$\psi$に自由に現れる変項とし,$\psi$に自由に現れる変項は$z$のみであるとする.このとき
			\begin{align}
				\EQAX,\COMAX \vdash a = \Set{z}{\psi(z)} 
				\rarrow \forall v\, (\, v \in a \lrarrow \psi(v)\, ).
			\end{align}
		\end{thm}
	\end{screen}
	
	\begin{screen}
		\begin{thm}
		\ref{thm:equivalent_formula_rewriting_2}
			$a$を主要$\varepsilon$項とし,$\psi$を$\lang{\varepsilon}$の式とし,
			$z$を$\psi$に自由に現れる変項とし,$\psi$に自由に現れる変項は$z$のみであるとする.このとき
			\begin{align}
				\EXTAX,\COMAX \vdash \forall v\, (\, v \in a \lrarrow \psi(v)\, )
				\rarrow a = \Set{z}{\psi(z)}.
			\end{align}
		\end{thm}
	\end{screen}
	
	\begin{screen}
		\begin{thm}
		\ref{thm:equivalent_formula_rewriting_3}
			$b$を主要$\varepsilon$項とし,$\varphi$を$\lang{\varepsilon}$の式とし,
			$y$を$\varphi$に自由に現れる変項とし,$\varphi$に自由に現れる変項は$y$のみ
			であるとする.このとき
			\begin{align}
				\EQAX,\COMAX \vdash \Set{y}{\varphi(y)} = b 
				\rarrow \forall u\, (\, \varphi(u) \lrarrow u \in b\, ).
			\end{align}
		\end{thm}
	\end{screen}
	
	\begin{screen}
		\begin{thm}
		\ref{thm:equivalent_formula_rewriting_4}
			$b$を主要$\varepsilon$項とし,$\varphi$を$\lang{\varepsilon}$の式とし,
			$y$を$\varphi$に自由に現れる変項とし,$\varphi$に自由に現れる変項は$y$のみ
			であるとする.このとき
			\begin{align}
				\EXTAX,\COMAX \vdash \forall u\, (\, \varphi(u) \lrarrow u \in b\, )
				\rarrow \Set{y}{\varphi(y)} = b.
			\end{align}
		\end{thm}
	\end{screen}
	
	\begin{screen}
		\begin{thm}
		\ref{thm:equivalent_formula_rewriting_5}
			$\varphi$と$\psi$を$\lang{\varepsilon}$の式とし,
			$y$を$\varphi$に自由に現れる変項とし,
			$z$を$\psi$に自由に現れる変項とし,
			$\varphi$に自由に現れる変項は$y$のみであるとし,
			$\psi$に自由に現れる変項は$z$のみであるとし,する.このとき
			\begin{align}
				\EQAX,\COMAX \vdash \Set{y}{\varphi(y)} = \Set{z}{\psi(z)}
				\rarrow \forall u\, (\, \varphi(u) \lrarrow \psi(u)\, ).
			\end{align}
		\end{thm}
	\end{screen}
	
	\begin{screen}
		\begin{thm}
		\ref{thm:equivalent_formula_rewriting_6}
			$\varphi$と$\psi$を$\lang{\varepsilon}$の式とし,
			$y$を$\varphi$に自由に現れる変項とし,
			$z$を$\psi$に自由に現れる変項とし,
			$\varphi$に自由に現れる変項は$y$のみであるとし,
			$\psi$に自由に現れる変項は$z$のみであるとし,する.このとき
			\begin{align}
				\EXTAX,\COMAX \vdash \forall u\, (\, \varphi(u) \lrarrow \psi(u)\, )
				\rarrow \Set{y}{\varphi(y)} = \Set{z}{\psi(z)}.
			\end{align}
		\end{thm}
	\end{screen}
	
	\begin{screen}
		\begin{thm}
		\ref{thm:equivalent_formula_rewriting_7}
			$a$を主要$\varepsilon$項とし,$\psi$を$\lang{\varepsilon}$の式とし,
			$z$を$\psi$に自由に現れる変項とし,$\psi$に自由に現れる変項は$z$のみであるとする.このとき
			\begin{align}
				\COMAX \vdash a \in \Set{z}{\psi(z)} \rarrow \psi(a).
			\end{align}
		\end{thm}
	\end{screen}
	
	\begin{screen}
		\begin{thm}
		\ref{thm:equivalent_formula_rewriting_8}
			$a$を主要$\varepsilon$項とし,$\psi$を$\lang{\varepsilon}$の式とし,
			$z$を$\psi$に自由に現れる変項とし,$\psi$に自由に現れる変項は$z$のみであるとする.このとき
			\begin{align}
				\COMAX \vdash \psi(a) \rarrow a \in \Set{z}{\psi(z)}.
			\end{align}
		\end{thm}
	\end{screen}
	
	\begin{screen}
		\begin{thm}
		\ref{thm:equivalent_formula_rewriting_9}
			$b$を主要$\varepsilon$項とし,$\varphi$を$\lang{\varepsilon}$の式とし,
			$y$を$\varphi$に自由に現れる変項とし,
			$\varphi$に自由に現れる変項は$y$のみであるとする.このとき
			\begin{align}
				\EQAX,\COMAX,\ELEAX \vdash \Set{y}{\varphi(y)} \in b
				\rarrow \exists s\, (\, 
				\forall u\, (\, \varphi(u) \lrarrow u \in s\, )
				\wedge s \in b\, ).
			\end{align}
		\end{thm}
	\end{screen}
	
	\begin{screen}
		\begin{thm}
		\ref{thm:equivalent_formula_rewriting_10}
			$b$を主要$\varepsilon$項とし,$\varphi$を$\lang{\varepsilon}$の式とし,
			$y$を$\varphi$に自由に現れる変項とし,
			$\varphi$に自由に現れる変項は$y$のみであるとする.このとき
			\begin{align}
				\EXTAX,\EQAX,\COMAX \vdash \exists s\, (\, \forall u\, (\, \varphi(u) \lrarrow u \in s\, ) \wedge s \in b\, ) \rarrow \Set{y}{\varphi(y)} \in b.
			\end{align}
		\end{thm}
	\end{screen}
	
	\begin{screen}
		\begin{thm}
		\ref{thm:equivalent_formula_rewriting_11}
			$\varphi$と$\psi$を$\lang{\varepsilon}$の式とし,
			$y$を$\varphi$に自由に現れる変項とし,
			$z$を$\psi$に自由に現れる変項とし,
			$\varphi$に自由に現れる変項は$y$のみであるとし,
			$\psi$に自由に現れる変項は$z$のみであるとし,する.このとき
			\begin{align}
				\EQAX,\COMAX,\ELEAX \vdash \Set{y}{\varphi(y)} \in \Set{z}{\psi(z)}
				\rarrow \exists s\, (\, 
				\forall u\, (\, \varphi(u) \lrarrow u \in s\, )
				\wedge \psi(s)\, ).
			\end{align}
		\end{thm}
	\end{screen}
	
	\begin{screen}
		\begin{thm}
		\ref{thm:equivalent_formula_rewriting_12}
			$\varphi$と$\psi$を$\lang{\varepsilon}$の式とし,
			$y$を$\varphi$に自由に現れる変項とし,
			$z$を$\psi$に自由に現れる変項とし,
			$\varphi$に自由に現れる変項は$y$のみであるとし,
			$\psi$に自由に現れる変項は$z$のみであるとし,する.このとき
			\begin{align}
				\EXTAX,\EQAX,\COMAX \vdash \exists s\, (\, \forall u\, (\, \varphi(u) \lrarrow u \in s\, ) \wedge \psi(s)\, ) \rarrow \Set{y}{\varphi(y)} \in \Set{z}{\psi(z)}.
			\end{align}
		\end{thm}
	\end{screen}
	
\section{対}
	\begin{screen}
		\begin{dfn}[対]
			$x,y$を$\mathcal{L}$の項とし,$z$を$x$にも$y$にも自由に現れない変項とするとき,
			\begin{align}
				\{x,y\} \defeq \Set{z}{x = z \vee y = z}
			\end{align}
			で$\{x,y\}$を定義し,これを$x$と$y$の{\bf 対}\index{つい@対}{\bf (pair)}と呼ぶ.
			特に$\{x,x\}$を$\{x\}$と書く.
		\end{dfn}
	\end{screen}
	
	\begin{screen}
		\begin{thm}[対は表示されている要素しか持たない]
		\ref{thm:pair_members_are_exactly_the_given_two}
			$a$と$b$を類とするとき次が成立する:
			\begin{align}
				\EXTAX,\EQAX,\COMAX \vdash 
				\forall x\, (\, x \in \{a,b\} \lrarrow a = x \vee b = x\, ).
			\end{align}
		\end{thm}
	\end{screen}
	
	\begin{screen}
		\begin{thm}[対の対称性]
		\ref{thm:commutative_law_of_pairs}
			$a$と$b$を類とするとき
			\begin{align}
				\EXTAX,\EQAX,\COMAX \vdash \{a,b\} = \{b,a\}.
			\end{align}
		\end{thm}
	\end{screen}
	
	\begin{screen}
		\begin{axm}[対の公理]
			\begin{align}
				\PAIAX \defarrow \forall x\, \forall y\, \exists p\, \forall z\, 
				(\, x = z \vee y = z \lrarrow z \in p\, ).
			\end{align}
		\end{axm}
	\end{screen}
	
	\begin{screen}
		\begin{thm}[集合の対は集合である]
		\ref{thm:pair_of_sets_is_a_set}
			$a$と$b$を類とするとき
			\begin{align}
				\EXTAX,\EQAX,\COMAX,\PAIAX \vdash 
				\set{a} \wedge \set{b} \rarrow \set{\{a,b\}}.
			\end{align}
		\end{thm}
	\end{screen}
	
	\begin{screen}
		\begin{thm}[集合は対の要素たりうる]\ref{thm:set_is_an_element_of_its_pair}
			$a$と$b$を類とするとき
			\begin{align}
				\EXTAX,\EQAX,\COMAX \vdash \set{a} \rarrow a \in \{a,b\}.
			\end{align}
		\end{thm}
	\end{screen}
	
	\begin{screen}
		\begin{thm}[真類同士の対は空]\ref{thm:pair_of_proper_classes_is_emptyset}
			$a$と$b$を類とするとき,
			\begin{align}
				\EXTAX,\EQAX,\COMAX \vdash\ 
				\negation \set{a} \wedge \negation \set{b} \rarrow \{a,b\} = \emptyset.
			\end{align}
		\end{thm}
	\end{screen}

%\chapter{Hilbert流証明論}
	%\section{Hilbert流証明論メモ}
	参考文献: 戸次大介「数理論理学」
	
	\begin{itembox}[l]{{\bf SK}の公理}
		\begin{description}
			\item[(S)] $(\varphi \rightarrow (\psi \rightarrow \chi)) 
				\rightarrow ((\varphi \rightarrow \psi)
				\rightarrow (\varphi \rightarrow \chi)).$
			
			\item[(K)] $\varphi \rightarrow (\psi \rightarrow \varphi).$
		\end{description}
	\end{itembox}
	
	{\bf SK}から証明可能な式
	\begin{description}
		\item[(I)] $\varphi \rightarrow \varphi$
		\item[(B)] $(\psi \rightarrow \chi) \rightarrow ((\varphi \rightarrow \psi) \rightarrow (\varphi \rightarrow \chi)).$
		\item[(C)] $(\varphi \rightarrow (\psi \rightarrow \chi)) \rightarrow (\psi \rightarrow (\varphi \rightarrow \chi)).$
		\item[(W)] $(\varphi \rightarrow (\varphi \rightarrow \psi)) \rightarrow (\varphi \rightarrow \psi).$
		\item[(B')] $(\varphi \rightarrow \psi) \rightarrow ((\psi \rightarrow \chi) \rightarrow (\varphi \rightarrow \chi)).$
		\item[(C$\ast$)] $\varphi \rightarrow ((\varphi \rightarrow \psi) \rightarrow \psi)$
	\end{description}
	
	\begin{itembox}[l]{否定の追加}
		\begin{description}
			\item[(CTI1)] $\varphi \rightarrow (\rightharpoondown \varphi \rightarrow \bot).$
			
			\item[(CTI2)] $\rightharpoondown \varphi \rightarrow (\varphi \rightarrow \bot).$
			
			\item[(NI)] $(\varphi \rightarrow \bot) \rightarrow\ \rightharpoondown \varphi.$
		\end{description}
	\end{itembox}
	
	このとき証明可能な式
	\begin{description}
		\item[(DNI)] $\varphi \rightarrow\ \rightharpoondown \rightharpoondown \varphi.$
		\item[(CON1)] $(\varphi \rightarrow \psi) \rightarrow (\rightharpoondown \psi \rightarrow\ \rightharpoondown \varphi).$
		\item[(CON2)] $(\varphi \rightarrow\ \rightharpoondown \psi) \rightarrow (\psi \rightarrow\ \rightharpoondown \varphi).$
	\end{description}
	
	\begin{itembox}[l]{{\bf HM}の公理}
		\begin{description}
			\item[(S)] $(\varphi \rightarrow (\psi \rightarrow \chi)) 
				\rightarrow ((\varphi \rightarrow \psi)
				\rightarrow (\varphi \rightarrow \chi)).$
			\item[(K)] $\varphi \rightarrow (\psi \rightarrow \varphi).$
			\item[(DI1)] $\varphi \rightarrow (\varphi \vee \psi).$
			\item[(DI2)] $\psi \rightarrow (\varphi \vee \psi).$
			\item[(DE)] $(\varphi \rightarrow \chi) \rightarrow 
				((\psi \rightarrow \chi) \rightarrow ((\varphi \vee \psi) \rightarrow \chi)).$
			\item[(CI)] $\varphi \rightarrow (\psi \rightarrow (\varphi \wedge \psi)).$
			\item[(CE1)] $(\varphi \wedge \psi) \rightarrow \varphi.$
			\item[(CE2)] $(\varphi \wedge \psi) \rightarrow \psi.$
			\item[(UI)] $\forall \zeta (\psi \rightarrow \varphi[\zeta/\xi]) 
				\rightarrow (\psi \rightarrow \forall \xi \varphi).$
			\item[(UE)] $\forall \xi \varphi \rightarrow \varphi[\tau/\xi].$
			\item[(EI)] $\varphi[\tau/\xi] \rightarrow \exists \xi \varphi.$
			\item[(EE)] $\forall \zeta (\varphi[\zeta/\xi] \rightarrow \psi)
				\rightarrow (\exists \xi \varphi \rightarrow \psi).$
		\end{description}
	\end{itembox}
	
	{\bf HM}から証明可能な式
	\begin{description}
		\item[LNC] $\rightharpoondown (\varphi \wedge \rightharpoondown \varphi).$
		\item[(DIST$\wedge$)] $\varphi \vee (\psi \wedge \chi) 
			\leftrightarrow (\varphi \vee \psi) \wedge (\varphi \vee \chi).$
		\item[(DIST$\vee$)] $\varphi \wedge (\psi \vee \chi) 
			\leftrightarrow (\varphi \wedge \psi) \vee (\varphi \wedge \chi).$
		\item[(DM$\vee$)] $\rightharpoondown (\varphi \vee \psi) \leftrightarrow
			\ \rightharpoondown \varphi \wedge \rightharpoondown \psi.$
	\end{description}
	
	\begin{sketch}[(LNC)]
		\begin{align}
			\varphi \wedge \rightharpoondown \varphi &\provable{\mbox{{\bf HM}}} \varphi, \\
			\varphi \wedge \rightharpoondown \varphi &\provable{\mbox{{\bf HM}}}\ \rightharpoondown \varphi, \\
			\varphi \wedge \rightharpoondown \varphi &\provable{\mbox{{\bf HM}}}
				\varphi \rightarrow (\rightharpoondown \varphi \rightarrow \bot), \\
			\varphi \wedge \rightharpoondown \varphi &\provable{\mbox{{\bf HM}}}\ \rightharpoondown \varphi \rightarrow \bot, \\
			\varphi \wedge \rightharpoondown \varphi &\provable{\mbox{{\bf HM}}} \bot, \\
			&\provable{\mbox{{\bf HM}}} (\varphi \wedge \rightharpoondown \varphi) \rightarrow \bot, \\
			&\provable{\mbox{{\bf HM}}} ((\varphi \wedge \rightharpoondown \varphi) \rightarrow \bot)
				\rightarrow\ \rightharpoondown (\varphi \wedge \rightharpoondown \varphi), \\
			&\provable{\mbox{{\bf HM}}}\ \rightharpoondown (\varphi \wedge \rightharpoondown \varphi).
		\end{align}
		\QED
	\end{sketch}
	
	\begin{sketch}[(DM$\vee$)]
		\begin{align}
			&\provable{\mbox{{\bf HM}}} \varphi \rightarrow (\varphi \vee \psi), && \mbox{(DI1)}\\
			&\provable{\mbox{{\bf HM}}} (\varphi \rightarrow (\varphi \vee \psi))
				\rightarrow (\rightharpoondown (\varphi \vee \psi) \rightarrow\ \rightharpoondown \varphi), 
				&& \mbox{(CON1)}\\
			&\provable{\mbox{{\bf HM}}}\ \rightharpoondown (\varphi \vee \psi) \rightarrow\ \rightharpoondown \varphi, 
				&& \mbox{(MP)}\\
			\rightharpoondown (\varphi \vee \psi) &\provable{\mbox{{\bf HM}}}\ \rightharpoondown \varphi.
				&& \mbox{(DR)}
		\end{align}
		同様に
		\begin{align}
			\rightharpoondown (\varphi \vee \psi) \provable{\mbox{{\bf HM}}}\ \rightharpoondown \psi
		\end{align}
		となり,
		\begin{align}
			\rightharpoondown (\varphi \vee \psi) &\provable{\mbox{{\bf HM}}}\ \rightharpoondown \varphi
				\rightarrow (\rightharpoondown \psi \rightarrow 
				(\rightharpoondown \varphi \wedge \rightharpoondown \psi)), && \mbox{(CI)}\\
			\rightharpoondown (\varphi \vee \psi) &\provable{\mbox{{\bf HM}}}\ 
				\rightharpoondown \psi \rightarrow (\rightharpoondown \varphi \wedge \rightharpoondown \psi), 
				&& \mbox{(MP)}\\
			\rightharpoondown (\varphi \vee \psi) &\provable{\mbox{{\bf HM}}}\ 
				\rightharpoondown \varphi \wedge \rightharpoondown \psi && \mbox{(MP)}
		\end{align}
		が得られる.逆に
		\begin{align}
			\rightharpoondown \varphi \wedge \rightharpoondown \psi &\provable{\mbox{{\bf HM}}}\ \rightharpoondown \varphi, 
				&& \mbox{(CE1)}\\
			\rightharpoondown \varphi \wedge \rightharpoondown \psi &\provable{\mbox{{\bf HM}}}\ 
			\rightharpoondown \varphi \rightarrow (\varphi \rightarrow \bot), && \mbox{(CTI2)}\\
			\rightharpoondown \varphi \wedge \rightharpoondown \psi &\provable{\mbox{{\bf HM}}} \varphi \rightarrow \bot
				&& \mbox{(MP)}
		\end{align}
		となり,同様に
		\begin{align}
			\rightharpoondown \varphi \wedge \rightharpoondown \psi \provable{\mbox{{\bf HM}}} \psi \rightarrow \bot
		\end{align}
		も成り立つ.よって
		\begin{align}
			\rightharpoondown \varphi \wedge \rightharpoondown \psi &\provable{\mbox{{\bf HM}}} 
				(\varphi \rightarrow \bot) \rightarrow ((\psi \rightarrow \bot) 
				\rightarrow ((\varphi \vee \psi) \rightarrow \bot)), && \mbox{(DE)}\\
			\rightharpoondown \varphi \wedge \rightharpoondown \psi &\provable{\mbox{{\bf HM}}} 
				(\psi \rightarrow \bot) \rightarrow ((\varphi \vee \psi) \rightarrow \bot), && \mbox{(MP)}\\
			\rightharpoondown \varphi \wedge \rightharpoondown \psi &\provable{\mbox{{\bf HM}}} 
				(\varphi \vee \psi) \rightarrow \bot, && \mbox{(MP)}\\
			\rightharpoondown \varphi \wedge \rightharpoondown \psi &\provable{\mbox{{\bf HM}}} 
				((\varphi \vee \psi) \rightarrow \bot) \rightarrow\ \rightharpoondown (\varphi \vee \psi), && \mbox{(NI)}\\
			\rightharpoondown \varphi \wedge \rightharpoondown \psi &\provable{\mbox{{\bf HM}}} 
				\ \rightharpoondown (\varphi \vee \psi) && \mbox{(MP)}
		\end{align}
		が得られる.
		\QED
	\end{sketch}

\chapter{保存拡大}
\label{chap:conservative_extension}
	\section{古典論理}
	\begin{screen}
		\begin{logicalaxm}[{\bf HK}の公理(命題論理)]
			$\varphi$と$\psi$と$\xi$を式とするとき,次は{\bf HK}の公理である.
			\begin{description}
				\item[(S)] $(\, \varphi \rarrow (\, \psi \rarrow \chi\, )\, ) 
					\rarrow (\, (\, \varphi \rarrow \psi\, )
					\rarrow (\, \varphi \rarrow \chi\, )\, ).$
				\item[(K)] $\varphi \rarrow (\, \psi \rarrow \varphi\, ).$
				\item[(CTD1)] $\varphi \rarrow (\, \negation \varphi \rarrow \bot\, ).$
				\item[(CTD2)] $\negation \varphi \rarrow (\, \varphi \rarrow \bot\, ).$
				\item[(DI)] $(\, \varphi \rarrow \bot\, ) \rarrow\ \negation \varphi.$
				\item[(DI1)] $\varphi \rarrow \varphi \vee \psi.$
				\item[(DI2)] $\psi \rarrow \varphi \vee \psi.$
				\item[(DE)] $(\, \varphi \rarrow \chi\, ) \rarrow 
					(\, (\, \psi \rarrow \chi\, ) 
					\rarrow (\, \varphi \vee \psi \rarrow \chi\, )\, ).$
				\item[(CI)] $\varphi \rarrow (\, \psi \rarrow (\, \varphi \wedge \psi\, )\, ).$
				\item[(CE1)] $\varphi \wedge \psi \rarrow \varphi.$
				\item[(CE2)] $\varphi \wedge \psi \rarrow \psi.$
				\item[(DNE)] $\negation \negation \varphi \rarrow \varphi$.
			\end{description}
		\end{logicalaxm}
	\end{screen}
	
	\begin{screen}
		\begin{logicalaxm}[{\bf HK}の公理(量化)]
			$\varphi$と$\psi$と$\xi$を式とし,$x$と$y$を変項とし,$t$を項とする.また
			$y$は$\psi$には自由に現れず,$\varphi$には$x$が自由に現れ,
			$y$と$t$は$\varphi$の中で$x$への代入について自由であるとする.このとき
			次は{\bf HK}の公理である.
			\begin{description}
				\item[(UI)] $\forall y\, (\, \psi \rarrow \varphi(x/y)\, ) 
					\rarrow (\, \psi \rarrow \forall x \varphi\, ).$
				
				\item[(UE)] $\forall x \varphi \rarrow \varphi(x/t).$
				
				\item[(EI)] $\varphi(x/t) \rarrow \exists x \varphi.$
				
				\item[(EE)] $\forall y\, (\, \varphi(x/y) \rarrow \psi\, )
						\rarrow (\, \exists x \varphi \rarrow \psi\, ).$
			\end{description}
		\end{logicalaxm}
	\end{screen}
	
	古典論理で証明可能なことを$\provable{\mbox{{\bf HK}}}$と書く.
	
	\begin{screen}
		\begin{metadfn}[{\bf HK}における証明可能性]
			式$\varphi$が公理系$\mathscr{S}$から
			{\bf 証明された}だとか{\bf 証明可能である}\index{しょうめいかのう@証明可能}
			{\bf (provable)}ということは,
			\begin{itemize}
				\item $\varphi$は$\mathscr{S}$の公理である.
				\item $\varphi$は{\bf HK}の公理である.
				\item 式$\psi$で,$\psi$と$\psi \rightarrow \varphi$が$\mathscr{S}$から
				証明されているものが取れる({\bf 三段論法}\index{さんだんろんぽう@三段論法}
				{\bf (Modus Pones)}).
				\item 式$\psi$と変項$a$が取れて,$\psi$には$x$が自由に現れていて,
				$a$は$\varphi$の中で$x$への代入について自由であり,
				また$\mathscr{S}$のどの公理の中にも$a$は自由に現れないとする.
				そして$\mathscr{S}$から$\psi(x/a)$が証明されていて,
				$\varphi$とは$\forall x \psi$なる形の式である
				({\bf 汎化}\index{はんか@汎化}{\bf (generalization)}).
			\end{itemize}
			のいずれかが満たされているということである.
		\end{metadfn}
	\end{screen}
	
	\begin{screen}
		\begin{thm}
			$\varphi$を$\lang{\varepsilon}$の式とし,
			変項$x$が$\varphi$に自由に現れるとき,
			\begin{align}
				\provable{\mbox{{\bf HK}}}
				\ \negation \exists x \varphi \rarrow \forall x \negation \varphi.
			\end{align}
		\end{thm}
	\end{screen}
	
	\begin{sketch}
		$a$を,$\varphi$には現れず,かつ$\varphi$の中で$x$への代入について自由である変項とすると,
		存在記号の導入規則より
		\begin{align}
			\provable{\mbox{{\bf HK}}} \varphi(x/a) \rarrow \exists x \varphi
		\end{align}
		が成り立つので,対偶を取って
		\begin{align}
			\negation \exists x \varphi \provable{\mbox{{\bf HK}}}
			\ \negation \varphi(x/a) 
		\end{align}
		となる.汎化により
		\begin{align}
			\negation \exists x \varphi \provable{\mbox{{\bf HK}}}
			\forall x \negation \varphi 
		\end{align}
		が成り立つ.
		\QED
	\end{sketch}
	
	\begin{screen}
		\begin{thm}
			$\varphi$を$\lang{\varepsilon}$の式とし,
			変項$x$が$\varphi$に自由に現れるとき,
			\begin{align}
				\provable{\mbox{{\bf HK}}} \forall x \negation \varphi
				\rarrow\ \negation \exists x \varphi.
			\end{align}
		\end{thm}
	\end{screen}
	
	\begin{screen}
		\begin{thm}
			$\varphi$を$\lang{\varepsilon}$の式とし,
			変項$x$が$\varphi$に自由に現れるとき,
			\begin{align}
				\provable{\mbox{{\bf HK}}}\ \negation \forall x \varphi \rarrow 
				\exists x \negation \varphi.
			\end{align}
		\end{thm}
	\end{screen}
	
	\begin{screen}
		\begin{thm}
			$\varphi$を$\lang{\varepsilon}$の式とし,
			変項$x$が$\varphi$に自由に現れるとき,
			\begin{align}
				\provable{\mbox{{\bf HK}}}
				\exists x \negation \varphi \rarrow\ \negation \forall x \varphi.
			\end{align}
		\end{thm}
	\end{screen}
	
	\begin{sketch}
	\end{sketch}
	
	\begin{screen}
		\begin{thm}
			$\varphi$と$\psi$に変項$x$が自由に現れるとき,
			\begin{align}
				\provable{\mbox{{\bf HK}}} \forall x\, (\, \varphi \rarrow \psi\, )
				\rarrow (\, \exists x \varphi \rarrow \exists x \psi\, ).
			\end{align}
		\end{thm}
	\end{screen}
	
	\begin{sketch}
		全称記号の除去より
		\begin{align}
			\forall x\, (\, \varphi \rarrow \psi\, ) \provable{\mbox{{\bf HK}}}
			\varphi \rarrow \psi
		\end{align}
		となるので,
		\begin{align}
			\varphi,\ \forall x\, (\, \varphi \rarrow \psi\, ) 
			\provable{\mbox{{\bf HK}}} \psi
		\end{align}
		が成り立ち,存在記号の導入より
		\begin{align}
			\varphi,\ \forall x\, (\, \varphi \rarrow \psi\, ) 
			\provable{\mbox{{\bf HK}}} \exists x \psi
		\end{align}
		が成り立ち,演繹法則より
		\begin{align}
			\forall x\, (\, \varphi \rarrow \psi\, ) 
			\provable{\mbox{{\bf HK}}} \varphi \rarrow \exists x \psi
		\end{align}
		が従う.汎化によって
		\begin{align}
			\forall x\, (\, \varphi \rarrow \psi\, ) \provable{\mbox{{\bf HK}}} 
			\forall x\, (\, \varphi \rarrow \exists x \psi\, )
		\end{align}
		となり,存在記号の除去より
		\begin{align}
			\forall x\, (\, \varphi \rarrow \psi\, ) \provable{\mbox{{\bf HK}}} 
			\exists x \varphi \rarrow \exists x \psi
		\end{align}
		が従う.
	\end{sketch}
	
	\begin{screen}
		\begin{thm}
			$\varphi$と$\psi$を$\lang{\varepsilon}$の式とし,
			$\psi$には$x$が自由に現れて,$\varphi$には$x$が自由に現れないとき,
			\begin{align}
				\provable{\mbox{{\bf HK}}} (\, \varphi \rarrow \exists x \psi\, ) 
				\rarrow \exists x\, (\, \varphi \rarrow \psi\, ).
			\end{align}
		\end{thm}
	\end{screen}
	
	\begin{sketch}
		\begin{align}
			(\, \varphi \rarrow \exists x \psi\, ) &\rarrow 
				(\, \negation \varphi \vee \exists x \psi\, ), \\
			(\, \negation \varphi \vee \exists x \psi\, ) &\rarrow
				\ \negation (\, \varphi \wedge \negation \exists x \psi\, ), \\
			\negation (\, \varphi \wedge \negation \exists x \psi\, ) &\rarrow 
				\ \negation (\, \varphi \wedge \forall x \negation \psi\, ), \\
			\negation (\, \varphi \wedge \forall x \negation \psi\, ) &\rarrow 
				\ \negation \forall x\, (\, \varphi \wedge \negation \psi\, ), \\
			\negation \forall x\, (\, \varphi \wedge \negation \psi\, ) &\rarrow 
				\exists x \negation (\, \varphi \wedge \negation \psi\, ), \\
			\exists x \negation (\, \varphi \wedge \negation \psi\, ) &\rarrow 
				\exists x\, (\, \negation \varphi \vee \psi\, ), \\
			\exists x\, (\, \negation \varphi \vee \psi\, ) &\rarrow 
				\exists x\, (\, \varphi \rarrow \psi\, )
		\end{align}
	\end{sketch}
	
	\subsection{最小論理}
	{\bf HK}の公理から二重否定除去(DNE)を抜いた体系を{\bf 最小論理}
	\index{さいしょうろんり@最小論理}{\bf (minimal logic)}と呼ぶ.
	この体系では背理法が成り立たないので「~と仮定すると矛盾するので…」といった論法は使えない.
	
	\begin{screen}
		\begin{thm}[対偶律$1$]\label{classic:contraposition_1}
			$\varphi$と$\psi$を式とするとき
			\begin{align}
				\provable{\mbox{{\bf HK}}} (\, \varphi \rarrow \psi\, )
				\rarrow (\, \negation \psi \rarrow\ \negation \varphi\, ).
			\end{align}
		\end{thm}
	\end{screen}
	
	\begin{sketch}
		$\varphi$と$\varphi \rarrow \psi$の三段論法から
		\begin{align}
			\varphi,\ \negation \psi,\ \varphi \rarrow \psi
			\provable{\mbox{{\bf HK}}} \psi
		\end{align}
		が成り立ち,
		\begin{align}
			\varphi,\ \negation \psi,\ \varphi \rarrow \psi
			\provable{\mbox{{\bf HK}}}\ \negation \psi
		\end{align}
		も成り立つので,矛盾の規則(DTC1)より
		\begin{align}
			\varphi,\ \negation \psi,\ \varphi \rarrow \psi
			\provable{\mbox{{\bf HK}}} \bot
		\end{align}
		が従う.演繹定理より
		\begin{align}
			\negation \psi,\ \varphi \rarrow \psi
			\provable{\mbox{{\bf HK}}} \varphi \rarrow \bot
		\end{align}
		となり,否定の導入(NI)より
		\begin{align}
			\negation \psi,\ \varphi \rarrow \psi
			\provable{\mbox{{\bf HK}}}\ \negation \varphi
		\end{align}
		が従う.そして演繹定理より
		\begin{align}
			\varphi \rarrow \psi
			\provable{\mbox{{\bf HK}}}\ \negation \psi \rarrow\ \negation \varphi
		\end{align}
		が得られる.
		\QED
	\end{sketch}
	
	\begin{screen}
		\begin{thm}[弱 De Morgan の法則$1$]
		\label{classic:weak_De_Morgan_law_for_quantifier_1}
			$\varphi$を式とし,
			変項$x$が$\varphi$に自由に現れるとするとき,
			\begin{align}
				\provable{\mbox{{\bf HK}}}
				\ \negation \exists x \varphi \rarrow \forall x \negation \varphi.
			\end{align}
		\end{thm}
	\end{screen}
	
	\begin{sketch}
		$y$を$\varphi$には現れない変項とすると,存在記号の導入規則より
		\begin{align}
			\provable{\mbox{{\bf HK}}} \varphi(x/y) \rarrow \exists x \varphi
		\end{align}
		が成り立ち,対偶律$1$ (定理\ref{classic:contraposition_1})より
		\begin{align}
			\provable{\mbox{{\bf HK}}}\ 
			\negation \exists x \varphi \rarrow\ \negation \varphi(x/y) 
		\end{align}
		となる.汎化により
		\begin{align}
			\provable{\mbox{{\bf HK}}} \forall y\, (\, \negation \exists x \varphi \rarrow\ \negation \varphi(x/y) \, ) 
		\end{align}
		が成り立つので,量化の公理(UI)との三段論法より
		\begin{align}
			\provable{\mbox{{\bf HK}}}\ 
			\negation \exists x \varphi \rarrow \forall x \negation \varphi 
		\end{align}
		が得られる.
		\QED
	\end{sketch}
	
	\begin{screen}
		\begin{thm}[強 De Morgan の法則$1$]
		\label{classic:strong_De_Morgan_law_for_quantifier_1}
			$\varphi$を式とし,
			変項$x$が$\varphi$に自由に現れるとするとき,
			\begin{align}
				\provable{\mbox{{\bf HK}}}
				\exists x \negation \varphi \rarrow\ \negation \forall x \varphi.
			\end{align}
		\end{thm}
	\end{screen}
	
	\begin{sketch}
		$y$を$\varphi$に現れない変項とすれば,量化の公理(UE)より
		\begin{align}
			\provable{\mbox{{\bf HK}}} \forall x \varphi \rarrow \varphi(x/y)
		\end{align}
		が成り立ち,対偶律1 (定理\ref{classic:contraposition_1})より
		\begin{align}
			\provable{\mbox{{\bf HK}}}\ \negation \varphi(x/y) \rarrow\ \negation \forall x \varphi
		\end{align}
		となる.汎化によって
		\begin{align}
			\provable{\mbox{{\bf HK}}} \forall y\, (\, \negation \varphi(x/y) \rarrow\ \negation \forall x \varphi\, )
		\end{align}
		が成り立ち,量化の公理(EE)より
		\begin{align}
			\provable{\mbox{{\bf HK}}} \exists x \negation \varphi \rarrow\ \negation \forall x \varphi
		\end{align}
		が得られる.
		\QED
	\end{sketch}
	
	\begin{screen}
		\begin{thm}[二重否定の導入]
		\label{classic:introduction_of_double_negation}
			$\varphi$を式とするとき
			\begin{align}
				\provable{\mbox{{\bf HK}}} \varphi \rarrow\ \negation \negation \varphi.
			\end{align}
		\end{thm}
	\end{screen}
	
	\begin{sketch}
		矛盾の導入(CTD1)より
		\begin{align}
			\varphi \provable{\mbox{{\bf HK}}}\ \negation \varphi \rarrow \bot
		\end{align}
		が成り立ち,否定の導入(NI)より
		\begin{align}
			\varphi \provable{\mbox{{\bf HK}}}\ \negation \negation \varphi
		\end{align}
		が従う.
		\QED
	\end{sketch}
	
	\begin{screen}
		\begin{thm}[対偶律$2$]\label{classic:contraposition_2}
			$\varphi$と$\psi$を式とするとき
			\begin{align}
				\provable{\mbox{{\bf HK}}} (\, \varphi \rarrow\ \negation \psi\, )
				\rarrow (\, \psi \rarrow\ \negation \varphi\, ).
			\end{align}
		\end{thm}
	\end{screen}
	
	\begin{sketch}
		対偶律$1$ (定理\ref{classic:contraposition_1})より
		\begin{align}
			\varphi \rarrow\ \negation \psi \provable{\mbox{{\bf HK}}}\ 
			\negation \negation \psi \rarrow\ \negation \varphi
		\end{align}
		が成り立ち,他方で二重否定の導入(定理\ref{classic:introduction_of_double_negation})より
		\begin{align}
			\psi \provable{\mbox{{\bf HK}}}\ \negation \negation \psi
		\end{align}
		が成り立つので,三段論法より
		\begin{align}
			\psi,\ \varphi \rarrow\ \negation \psi \provable{\mbox{{\bf HK}}}\ 
			\negation \varphi
		\end{align}
		が従い,演繹定理より
		\begin{align}
			\varphi \rarrow\ \negation \psi \provable{\mbox{{\bf HK}}}
			\psi \rarrow\ \negation \varphi
		\end{align}
		が得られる.
		\QED
	\end{sketch}
	
	\begin{screen}
		\begin{thm}[弱 De Morgan の法則$2$]
		\label{classic:weak_De_Morgan_law_for_quantifier_2}
			$\varphi$を式とし,
			変項$x$が$\varphi$に自由に現れるとするとき,
			\begin{align}
				\provable{\mbox{{\bf HK}}} \forall x \negation \varphi
				\rarrow\ \negation \exists x \varphi.
			\end{align}
		\end{thm}
	\end{screen}
	
	\begin{sketch}
		$y$を$\varphi$に現れない変項とすれば,量化の公理(UE)より
		\begin{align}
			\provable{\mbox{{\bf HK}}} \forall x \negation \varphi \rarrow\ \negation \varphi(x/y)
		\end{align}
		となるので,対偶律$2$ (定理\ref{classic:contraposition_2})より
		\begin{align}
			\provable{\mbox{{\bf HK}}} \varphi(x/y) \rarrow\ \negation \forall x \negation \varphi
		\end{align}
		となる.汎化によって
		\begin{align}
			\provable{\mbox{{\bf HK}}}\ \forall y\, (\, \varphi(x/y) \rarrow\ \negation \forall x \negation \varphi\, )
		\end{align}
		が成り立ち,量化の公理(EE)によって
		\begin{align}
			\provable{\mbox{{\bf HK}}}\ \exists x \negation \varphi \rarrow\ \negation \forall x \negation \varphi
		\end{align}
		が従い,再び対偶律$2$ (定理\ref{classic:contraposition_2})より
		\begin{align}
			\provable{\mbox{{\bf HK}}} \forall x \negation \varphi \rarrow\ \negation \exists x \varphi
		\end{align}
		が得られる.
		\QED
	\end{sketch}
	
	\begin{screen}
		\begin{thm}[De Morgan の法則1]
		\label{classic:De_Morgan_law_1}
			$\varphi$と$\psi$を式とするとき
			\begin{align}
				\provable{\mbox{{\bf HK}}} (\, \negation \varphi \vee \psi\, ) 
				\rarrow\ \negation (\, \varphi \wedge \negation \psi\, ).
			\end{align}
		\end{thm}
	\end{screen}
	
	\begin{sketch}
		論理積の除去(CE1)(CE2)より
		\begin{align}
			\varphi \wedge \negation \psi &\provable{\mbox{{\bf HK}}}\ \negation \varphi, \\
			\varphi \wedge \negation \psi &\provable{\mbox{{\bf HK}}} \psi
		\end{align}
		が成り立つので,矛盾の導入(CTD1)(CTD2)より
		\begin{align}
			\varphi \wedge \negation \psi &\provable{\mbox{{\bf HK}}}
			\varphi \rarrow \bot, \\
			\varphi \wedge \negation \psi &\provable{\mbox{{\bf HK}}}\ 
			\negation \psi \rarrow \bot
		\end{align}
		となり,論理和の除去(DE)より
		\begin{align}
			\varphi \wedge \negation \psi &\provable{\mbox{{\bf HK}}}
			\varphi \vee \negation \psi \rarrow \bot
		\end{align}
		が従い,否定の導入(NI)より
		\begin{align}
			\varphi \wedge \negation \psi &\provable{\mbox{{\bf HK}}}\ 
			\negation (\, \varphi \vee \negation \psi\, )
		\end{align}
		が得られる.
		\QED
	\end{sketch}
	
	\begin{screen}
		\begin{thm}[論理和の対称律]
		\label{classic:symmetry_of_disjunction}
			$\varphi$と$\psi$を式とするとき,
			\begin{align}
				\provable{\mbox{{\bf HK}}} 
				\varphi \vee \psi \rarrow \psi \vee \varphi.
			\end{align}
		\end{thm}
	\end{screen}
	
	\begin{sketch}
		論理和の導入(DI1)(DI2)より
		\begin{align}
			&\provable{\mbox{{\bf HK}}} \varphi \rarrow \psi \vee \varphi, \\
			&\provable{\mbox{{\bf HK}}} \psi \rarrow \psi \vee \varphi
		\end{align}
		が成り立つので,論理和の除去(DE)より
		\begin{align}
			\provable{\mbox{{\bf HK}}} \varphi \vee \psi \rarrow \psi \vee \varphi
		\end{align}
		が従う.
		\QED
	\end{sketch}
	
	\begin{screen}
		\begin{thm}[含意の論理和への遺伝性]
		\label{classic:heredity_of_implication_to_disjunction}
			$\varphi$と$\psi$と$\chi$を式とするとき,
			\begin{align}
				\provable{\mbox{{\bf HK}}} (\, \varphi \rarrow \psi\, )
				\rarrow (\, \varphi \wedge \chi \rarrow \psi \wedge \chi\, ).
			\end{align}
		\end{thm}
	\end{screen}
	
	\begin{sketch}
		三段論法より
		\begin{align}
			\varphi,\ \varphi \rarrow \psi \provable{\mbox{{\bf HK}}} \psi
		\end{align}
		が成り立ち,論理和の導入(DI1)より
		\begin{align}
			\varphi,\ \varphi \rarrow \psi \provable{\mbox{{\bf HK}}} \psi \vee \chi
		\end{align}
		が従い,演繹定理より
		\begin{align}
			\varphi \rarrow \psi \provable{\mbox{{\bf HK}}} 
			\varphi \rarrow \psi \vee \chi
		\end{align}
		が得られる.他方で論理和の導入(DI2)より
		\begin{align}
			\varphi \rarrow \psi \provable{\mbox{{\bf HK}}} 
			\chi \rarrow \psi \vee \chi
		\end{align}
		も成り立つので,論理和の除去(DE)より
		\begin{align}
			\varphi \rarrow \psi \provable{\mbox{{\bf HK}}} 
			\varphi \vee \chi \rarrow \psi \vee \chi
		\end{align}
		が得られる.
		\QED
	\end{sketch}
	
	\begin{screen}
		\begin{thm}[含意の論理積への遺伝性]
		\label{classic:heredity_of_implication_to_conjunction}
			$\varphi$と$\psi$と$\chi$を式とするとき,
			\begin{align}
				\provable{\mbox{{\bf HK}}} (\, \psi \rarrow \chi\, )
				\rarrow (\, \varphi \wedge \psi \rarrow \varphi \wedge \chi\, ).
			\end{align}
		\end{thm}
	\end{screen}
	
	\begin{sketch}
		論理積の除去(CE1)(CE2)及び三段論法より
		\begin{align}
			\psi \rarrow \chi,\ \varphi \wedge \psi &\provable{\mbox{{\bf HK}}} \varphi, \\
			\psi \rarrow \chi,\ \varphi \wedge \psi &\provable{\mbox{{\bf HK}}} \psi,
		\end{align}
		そして
		\begin{align}
			\psi \rarrow \chi,\ \varphi \wedge \psi \provable{\mbox{{\bf HK}}} \chi
		\end{align}
		が成り立つので,論理積の導入(CI)より
		\begin{align}
			\psi \rarrow \chi,\ \varphi \wedge \psi \provable{\mbox{{\bf HK}}} \varphi \wedge \chi
		\end{align}
		が従う.
		\QED
	\end{sketch}
	
	\begin{screen}
		\begin{thm}[論理積と全称の交換]
		\label{classic:commutation_of_conjunction_and_universal_quantifier}
			$\varphi$と$\psi$を式とし,
			$\psi$には変項$x$が自由に現れるとするとき,
			\begin{align}
				\provable{\mbox{{\bf HK}}} \forall x\, (\, \varphi \wedge \psi\, )
				\rarrow \varphi \wedge \forall x \psi.
			\end{align}
		\end{thm}
	\end{screen}
	
	\begin{sketch}
		量化の公理(UE)より
		\begin{align}
			\forall x\, (\, \varphi \wedge \psi\, ) \provable{\mbox{{\bf HK}}} 
			\varphi \wedge \psi
		\end{align}
		が成り立ち,論理積の除去(CE1)(CE2)より
		\begin{align}
			\forall x\, (\, \varphi \wedge \psi\, ) &\provable{\mbox{{\bf HK}}} \varphi, \\
			\forall x\, (\, \varphi \wedge \psi\, ) &\provable{\mbox{{\bf HK}}} \psi
		\end{align}
		となる.汎化によって
		\begin{align}
			\forall x\, (\, \varphi \wedge \psi\, ) \provable{\mbox{{\bf HK}}} 
			\forall x \psi
		\end{align}
		が成り立ち,論理積の導入(CI)によって
		\begin{align}
			\forall x\, (\, \varphi \wedge \psi\, ) \provable{\mbox{{\bf HK}}} 
			\varphi \wedge \forall x \psi
		\end{align}
		が得られる.
		\QED
	\end{sketch}
	
	\begin{screen}
		\begin{thm}
		\label{classic:no_description_1}
			$\varphi$と$\psi$に変項$x$が自由に現れるとき,
			\begin{align}
				\provable{\mbox{{\bf HK}}} \forall x\, (\, \varphi \rarrow \psi\, )
				\rarrow (\, \exists x \varphi \rarrow \exists x \psi\, ).
			\end{align}
		\end{thm}
	\end{screen}
	
	\begin{sketch}
		量化の公理(UE)より
		\begin{align}
			\forall x\, (\, \varphi \rarrow \psi\, ) \provable{\mbox{{\bf HK}}}
			\varphi \rarrow \psi
		\end{align}
		となるので,演繹定理より
		\begin{align}
			\varphi,\ \forall x\, (\, \varphi \rarrow \psi\, ) 
			\provable{\mbox{{\bf HK}}} \psi
		\end{align}
		が成り立つ.量化の公理(EI)より
		\begin{align}
			\varphi,\ \forall x\, (\, \varphi \rarrow \psi\, ) 
			\provable{\mbox{{\bf HK}}} \exists x \psi
		\end{align}
		が成り立ち,演繹定理より
		\begin{align}
			\forall x\, (\, \varphi \rarrow \psi\, ) 
			\provable{\mbox{{\bf HK}}} \varphi \rarrow \exists x \psi
		\end{align}
		が従う.汎化によって
		\begin{align}
			\forall x\, (\, \varphi \rarrow \psi\, ) \provable{\mbox{{\bf HK}}} 
			\forall x\, (\, \varphi \rarrow \exists x \psi\, )
		\end{align}
		となり,量化の公理(EE)より
		\begin{align}
			\forall x\, (\, \varphi \rarrow \psi\, ) \provable{\mbox{{\bf HK}}} 
			\exists x \varphi \rarrow \exists x \psi
		\end{align}
		が従う.
		\QED
	\end{sketch}
	\input{chapters/conservative_extension/double_negation}
	\section{Henkin拡大}
\label{sec:Henkin_expansion}
	この節では「$\Sigma$から$\psi$への{\bf HE}の証明で$\lang{\varepsilon}$の文の列
	であるものが取れる」ならば「$\Gamma$から$\psi$への{\bf HK}の証明で$\lang{\in}$の
	式の列であるものが取れる」ことを示す.{\bf HE}の証明に使われる式は全て\underline{文}である.
	
	\begin{screen}
		\begin{logicalaxm}[{\bf HE}の公理(命題論理)]
			$\varphi$と$\psi$と$\xi$を文とするとき
			\begin{description}
				\item[(S)] $(\, \varphi \rarrow (\, \psi \rarrow \chi\, )\, ) 
					\rarrow (\, (\, \varphi \rarrow \psi\, )
					\rarrow (\, \varphi \rarrow \chi\, )\, ).$
				\item[(K)] $\varphi \rarrow (\, \psi \rarrow \varphi\, ).$
				\item[(CTD1)] $\varphi \rarrow (\, \negation \varphi \rarrow \bot\, ).$
				\item[(CTD2)] $\negation \varphi \rarrow (\, \varphi \rarrow \bot\, ).$
				\item[(DI)] $(\, \varphi \rarrow \bot\, ) \rarrow\ \negation \varphi.$
				\item[(DI1)] $\varphi \rarrow (\, \varphi \vee \psi\, ).$
				\item[(DI2)] $\psi \rarrow (\, \varphi \vee \psi\, ).$
				\item[(DE)] $(\, \varphi \rarrow \chi\, ) \rarrow 
					(\, (\, \psi \rarrow \chi\, ) 
					\rarrow (\, (\, \varphi \vee \psi) \rarrow \chi\, )\, ).$
				\item[(CI)] $\varphi \rarrow (\, \psi \rarrow (\, \varphi \wedge \psi\, )\, ).$
				\item[(CE1)] $(\, \varphi \wedge \psi\, ) \rarrow \varphi.$
				\item[(CE2)] $(\, \varphi \wedge \psi\, ) \rarrow \psi.$
				\item[(DNE)] $\negation \negation \varphi \rarrow \varphi$.
			\end{description}
		\end{logicalaxm}
	\end{screen}
	
	\begin{screen}
		\begin{logicalaxm}[{\bf HE}の公理(量化)]
			$\varphi$を式とし,$\tau$を主要$\varepsilon$項とし,
			$x$を変項とし,$\varphi$には$x$のみが自由に現れているとするとき
			\begin{description}
				\item[(DM)] $\negation \forall x \varphi
					\rarrow \exists x \negation \varphi.$
				
				\item[(UE)] $\forall x \varphi \rarrow \varphi(x/\tau).$
				
				\item[(EI)] $\varphi(x/\tau) \rarrow \exists x \varphi.$
				
				\item[(EE)] $\hat{\varphi}$を,$\varphi$が$\lang{\varepsilon}$の式でないときは
					$\varphi$を$\lang{\varepsilon}$の式に書き直したものとし,
					$\varphi$が$\lang{\varepsilon}$の式であるときは$\varphi$そのものとする.このとき
					\begin{align}
						\exists x \varphi \rarrow \varphi(x/\varepsilon x \hat{\varphi}).
					\end{align}
			\end{description}
		\end{logicalaxm}
	\end{screen}
	
	第\ref{chap:inference}章での証明可能性の定義を列の概念を用いて書き直しておく.
	
	\begin{screen}
		\begin{metadfn}[{\bf HE}における証明]
			$\mathscr{S}$を文からなる公理系とする.このとき文の列$\varphi_{1},\varphi_{2},\cdots,
			\varphi_{n}$が$\mathscr{S}$から$\varphi_{n}$への{\bf HE}の証明であるとは,
			各$\varphi_{i}$が次のいずれかであるということである:
			\begin{itemize}
				\item $\varphi_{i}$は{\bf HE}の公理である.
				\item $\varphi_{i}$は$\mathscr{S}$の公理である.
				\item $\varphi_{i}$は,これより前の式$\varphi_{j}$と$\varphi_{k}$の
					三段論法で得られる.
			\end{itemize}
		\end{metadfn}
	\end{screen}
	
	$\psi$を文とし,$\mathscr{S}$を公理系とするとき,$\mathscr{S}$から$\psi$への
	{\bf HE}の証明で$\lang{\varepsilon}$の文の列であるものが取れることを
	\begin{align}
		\mathscr{S} \provable{\mbox{{\bf HE}},\lang{\varepsilon}} \psi
	\end{align}
	と書く.他方で$\mathscr{S}$から$\psi$への{\bf HE}の証明で$\mathcal{L}$の文の列であるものが
	取れることは,第\ref{chap:inference}章の証明可能性と同義であるから
	\begin{align}
		\mathscr{S} \vdash \psi
	\end{align}
	と書く.
	
	いま{\bf HK}の公理に{\bf HE}の(EE)を追加した証明体系を{\bf HK$\varepsilon$}とする.
	$\mathscr{T}$を公理系とするとき,$\mathscr{T}$に
	{\bf HE}の(EE)を追加した公理系を$\mathscr{T}$の{\bf Henkin拡大}
	\index{Henkinかくだい@Henkin拡大}{\bf (Henkin extension)}と呼ぶが,
	今回は{\bf HK}の公理に追加しているのでHenkin拡大の一種と見ることが出来る.
	Henkin拡大とはすなわち,全ての存在文に証人を付けるための拡大である.
	公理系$\mathscr{S}$から文$\psi$への{\bf HK$\varepsilon$}の証明で
	$\lang{\varepsilon}$の式の列であるものが取れることを
	\begin{align}
		\mathscr{S} \provable{\mbox{{\bf HK}$\varepsilon$},\lang{\varepsilon}} \psi
	\end{align}
	と書く.
	
	\begin{screen}
		\begin{metathm}[{\bf HE}で証明可能なら{\bf HK$\varepsilon$}でも証明可能]
		\label{metathm:Henkin_expansion_1}
			$\mathscr{S}$を$\lang{\in}$の文からなる公理系とし,$\psi$を$\lang{\in}$の文とする.
			このとき$\mathscr{S} \provable{\mbox{{\bf HE}},\lang{\varepsilon}} \psi$ならば
			$\mathscr{S} \provable{\mbox{{\bf HK}$\varepsilon$},\lang{\varepsilon}} \psi$
			である.
		\end{metathm}
	\end{screen}
	
	\begin{metaprf}
		{\bf HE}の公理で{\bf HK$\varepsilon$}の公理でないものは
		\begin{align}
			\negation \forall x \varphi \rarrow \exists x \negation \varphi
		\end{align}
		だけであるが,これはDe Morgan の法則
		(定理\ref{classic:weak_De_Morgan_law_for_quantifier_1})より導かれる.
		従って,証明の中に{\bf HE}の公理(DM)があれば,それより前の列に(DM)への
		{\bf HK}の証明を挿入すれば,$\mathscr{S}$から$\psi$への{\bf HK$\varepsilon$}の証明になる.
		\QED
	\end{metaprf}
	
	\begin{screen}
		\begin{metathm}[$\varepsilon$項を変項に取り替えても証明]
		\label{metathm:Henkin_expansion_2_lemma}
			$\mathscr{S}$を$\lang{\in}$の文からなる公理系とし,$\psi$を$\lang{\in}$の文とし,
			$\varphi_{1},\cdots,\varphi_{n}$を$\mathscr{S}$から$\psi$への
			{\bf HK}の証明で$\lang{\varepsilon}$の式の列であるものとする.
			また$e$をこの列の式に現れる主要$\varepsilon$項とし,$y$を
			この証明に現れない変項とする.そして各$\varphi_{i}$に対して
			そこに現れる$e$を全て$y$に取り替えた式を$\hat{\varphi}_{i}$と書く.
			ただし取り替えるのは他の項の真部分項になっていない$e$のみであり,
			$\varphi_{i}$に$e$が現れなければ$\hat{\varphi}_{i}$は$\varphi_{i}$とする.
			このとき
			\begin{align}
				\hat{\varphi}_{1},\ \cdots,\ \hat{\varphi}_{n}
			\end{align}
			は$\mathscr{S}$から$\psi$への{\bf HK}の証明である.
		\end{metathm}
	\end{screen}
	
	\begin{metaprf}\mbox{}
		\begin{description}
			\item[case1] $\varphi_{i}$が{\bf HK}の命題論理の公理であるとき,
				たとえば$\varphi_{i}$が
				\begin{align}
					\varphi \rarrow (\, \chi \rarrow \varphi\, )
				\end{align}
				なる式なら,$\hat{\varphi}_{i}$も
				\begin{align}
					\hat{\varphi} \rarrow (\, \hat{\chi} \rarrow \hat{\varphi}\, )
				\end{align}
				なる形の式となるので{\bf HK}の公理である.他の場合も同様である.
				
			\item[case2] $\varphi_{i}$が{\bf HK}の量化公理であるとき,つまり
				\begin{align}
					&\forall z\, (\, \chi \rarrow \varphi(x/z)\, ) 
						\rarrow (\, \chi \rarrow \forall x \varphi\, ), \\
					&\forall x \varphi \rarrow \varphi(x/t), \\
					&\varphi(x/t) \rarrow \exists x \varphi, \\
					&\forall z\, (\, \varphi(x/z) \rarrow \chi\, )
						\rarrow (\, \exists x \varphi \rarrow \chi\, )
				\end{align}
				のいずれかであるとき,$\hat{\varphi}_{i}$は
				\begin{align}
					&\forall z\, (\, \hat{\chi} \rarrow \hat{\varphi}(x/z)\, ) 
						\rarrow (\, \hat{\chi} \rarrow \forall x \hat{\varphi}\, ), \\
					&\forall x \hat{\varphi} \rarrow \hat{\varphi}(x/t), \\
					&\hat{\varphi}(x/t) \rarrow \exists x \hat{\varphi}, \\
					&\forall z\, (\, \hat{\varphi}(x/z) \rarrow \hat{\chi}\, )
						\rarrow (\, \exists x \hat{\varphi} \rarrow \hat{\chi}\, )
				\end{align}
				なる形の式となり,{\bf HK}の公理であるための変項の条件も満たされる.
				ここで注意しておくと,$e$が$\varphi(x/t)$に現れる場合,
				$\varphi$で$x$に代入された$t$を含むようには$e$は現れない.もし
				そのような$t$が$e$に現れたら,$e$のその$t$を$x$に置き換えた$\varepsilon$項$e'$は,
				$\varphi$すなわち$\exists x \varphi$の中に現れることになるが,$e'$は
				主要$\varepsilon$項ではないので第\ref{sec:restriction_of_formulas}節の
				約束に違反してしまう.従って$\varphi(x/t)$に現れる$e$を$y$に置き換えた式は
				$\hat{\varphi}(x/t)$なる形で書けるのである.
				
			\item[case3] $\varphi_{i}$が$\mathscr{S}$の公理であるときは
				$\lang{\in}$の文なので,$\hat{\varphi}_{i}$は$\varphi_{i}$である.
				
			\item[case4] $\varphi_{i}$が前の式$\varphi_{j},\varphi_{k}$から
				三段論法で得られているとき,$\varphi_{k}$が$\varphi_{j} \rarrow \varphi_{i}$
				なる式なら$\hat{\varphi}_{k}$は
				\begin{align}
					\hat{\varphi}_{j} \rarrow \hat{\varphi}_{i}
				\end{align}
				なる式であるから,$\hat{\varphi}_{i}$は$\hat{\varphi}_{j}$と
				$\hat{\varphi}_{k}$から三段論法で得られる.
				
			\item[case5] $\varphi_{i}$が前の式$\varphi_{j}$から
				汎化で得られているとき,つまり変項$a,x$と$x$が自由に現れる式$\chi$が取れて,
				$\varphi_{j}$が$\chi(x/a)$で$\varphi_{i}$が$\forall x \chi$であるとき,
				$\hat{\varphi}_{j}$は
				\begin{align}
					\hat{\chi}(x/a)
				\end{align}
				なる式で($e$は$x$に代入された$a$を含むようには現れない)
				$\hat{\varphi}_{i}$は
				\begin{align}
					\forall x \hat{\chi}
				\end{align}
				なる式であるから,$\hat{\varphi}_{i}$は$\hat{\varphi}_{j}$から汎化で得られる.
				\QED
		\end{description}
	\end{metaprf}
	
	\begin{screen}
		\begin{metathm}[{\bf HK$\varepsilon$}で証明可能なら{\bf HK}でも証明可能]
		\label{metathm:Henkin_expansion_2}
			$\mathscr{S}$を$\lang{\in}$の文からなる公理系とし,$\psi$を$\lang{\in}$の文とするとき,
			$\mathscr{S} \provable{\mbox{{\bf HK}$\varepsilon$},\lang{\varepsilon}} \psi$
			ならば$\mathscr{S} \provable{\mbox{{\bf HK}},\lang{\varepsilon}} \psi$である.
		\end{metathm}
	\end{screen}
	
	\begin{sketch}
		$\varphi_{1},\cdots,\varphi_{n}$の中から{\bf HE}の(EE)であるものを全て取り出して
		$\varphi_{i_{1}},\cdots\varphi_{i_{m}}$と並べれば,
		$\varphi_{1},\cdots,\varphi_{n}$は公理系
		$\varphi_{i_{1}},\cdots,\varphi_{i_{m}},\mathscr{S}$から$\psi$への
		{\bf HK}の証明となる.各$\varphi_{i_{j}}$は
		\begin{align}
			\exists x_{j} F_{j}(x_{j}) \rarrow F_{j}(\varepsilon x_{j} F_{j})
		\end{align}
		なる形をしている.ここで$\varepsilon x_{m} F_{m}$は$F_{1},\cdots,F_{m-1}$の中には
		現れないとすると
		\footnote{
			このような項$\varepsilon x_{m} F_{m}$は必ず取れる.たとえば
			$\varepsilon x_{i} F_{i}$が$F_{j}$に現れたら,
			$\varepsilon x_{j} F_{j}$は$F_{i}$には現れない.
			実際$F_{i}$に現れたら$\varepsilon x_{i} F_{i}$が$F_{i}$に現れることになるが,
			$F_{i}$より長い$\varepsilon x_{i} F_{i}$が$F_{i}$に現れることなどあり得ない.
			同様に,$\varepsilon x_{j} F_{j}$が$F_{k}$に現れたら,
			$\varepsilon x_{k} F_{k}$は$F_{i}$と$F_{j}$には現れない.この確認を繰り返せばよい.
		}
		\begin{align}
			\varphi_{i_{1}},\cdots,\varphi_{i_{m-1}},\mathscr{S} 
			\provable{\mbox{{\bf HK}}} \psi
		\end{align}
		が示される.実際,{\bf HK}の演繹定理より
		\begin{align}
			\varphi_{i_{1}},\cdots,\varphi_{i_{m-1}},\mathscr{S} 
			\provable{\mbox{{\bf HK}}} 
			(\, \exists x_{m} F_{m}(x_{m}) \rarrow F_{m}(\varepsilon x_{m} F_{m})\, ) \rarrow \psi
		\end{align}
		が成り立つが,このときの$\varphi_{i_{1}},\cdots,\varphi_{i_{m-1}},\mathscr{S}$から
		$(\, \exists x_{m} F_{m}(x_{m}) \rarrow F_{m}(\varepsilon x_{m} F_{m})\, ) 
		\rarrow \psi$への証明に現れる$\varepsilon x_{m} F_{m}$を,
		その証明に使われていない変項$y$に置き換えれば
		\footnote{
			置き換える$\varepsilon x_{m} F_{m}$は他の項の真部分項になっていない
			箇所のものだけである.また$\varepsilon x_{m} F_{m}$は
			$F_{1},\cdots,F_{m-1}$の中には現れないので
			$\varphi_{i_{1}},\cdots,\varphi_{i_{m-1}}$の中にも現れない.ここで注意しておくと,
			たとえば$\varepsilon x_{m} F_{m}$が$F_{1}(\varepsilon x_{1} F_{1})$に
			現れたとすると,その$\varepsilon x_{m} F_{m}$の中の$\varepsilon x_{1} F_{1}$を
			$x_{1}$に置き換えた$\varepsilon$項が$F_{1}$に現れることになる.しかし
			その$\varepsilon$項は主要$\varepsilon$項ではなく,
			第\ref{sec:restriction_of_formulas}節の約束に違反してしまう.
			ゆえに$\varepsilon x_{m} F_{m}$は$\varphi_{i_{1}},\cdots,\varphi_{i_{m-1}}$
			の中に現れず,これらの式は置換による影響を受けない.
			$\mathscr{S}$の公理も$\lang{\in}$の文であるから置換による影響を受けない.
		},それで得られる式の列は$\varphi_{i_{1}},\cdots,\varphi_{i_{m-1}},\mathscr{S}$
		から$(\, \exists x_{m} F_{m}(x_{m}) \rarrow F_{m}(y)\, ) \rarrow \psi$への
		{\bf HK}の証明となる(メタ定理\ref{metathm:Henkin_expansion_2_lemma}).つまり
		\begin{align}
			\varphi_{i_{1}},\cdots,\varphi_{i_{m-1}},\mathscr{S} 
			\provable{\mbox{{\bf HK}}} 
			(\, \exists x_{m} F_{m}(x_{m}) \rarrow F_{m}(y)\, ) \rarrow \psi
		\end{align}
		が成り立つ.すると汎化により
		\begin{align}
			\varphi_{i_{1}},\cdots,\varphi_{i_{m-1}},\mathscr{S} 
			\provable{\mbox{{\bf HK}}} 
			\forall y\, (\, (\, \exists x_{m} F_{m}(x_{m}) \rarrow F_{m}(y)\, ) \rarrow \psi\, )
		\end{align}
		となり,{\bf HK}の公理(EE)により
		\begin{align}
			\varphi_{i_{1}},\cdots,\varphi_{i_{m-1}},\mathscr{S} 
			\provable{\mbox{{\bf HK}}} 
			\exists y\, (\, \exists x_{m} F_{m}(x_{m}) \rarrow F_{m}(y)\, ) \rarrow \psi
		\end{align}
		が従う.定理\ref{classic:lemma_for_Henkin_expansion}より
		\begin{align}
			\provable{\mbox{{\bf HK}}} 
			\exists y\, (\, \exists x_{m} F_{m}(x_{m}) \rarrow F_{m}(y)\, )
		\end{align}
		が成り立つので,三段論法より
		\begin{align}
			\varphi_{i_{1}},\cdots,\varphi_{i_{m-1}},\mathscr{S} 
			\provable{\mbox{{\bf HK}}} \psi
		\end{align}
		が従う.以降も同様にして{\bf HK}の公理(EE)を一本ずつ削除していけば,
		$\mathscr{S}$から$\psi$への{\bf HK}の証明で
		$\lang{\varepsilon}$の式の列であるものが得られる.
		\QED
	\end{sketch}
	
	第\ref{chap:inference}章の$\Sigma$の公理は$\lang{\varepsilon}$の文の集まりであったが,
	それらを$\lang{\in}$の文に直した公理体系を$\Gamma$と書く.$\Gamma$は次の文からなる.
	\begin{description}
	\label{axioms_of_Gamma}
		\item[集合の存在]
			\begin{align}
				\exists x\, (\, x = x\, ).
			\end{align}
		
		\item[外延性]
			\begin{align}
				\forall x\, \forall y\, (\, \forall z\, 
				(\, z \in x \lrarrow z \in y\, ) \rarrow x = y\, ).
			\end{align}
			
		\item[相等性] 
			\begin{align}
				&\forall x\, \forall y\, (\, x = y \rarrow y = x\, ), \\
				&\forall x\, \forall y\, \forall z\, 
				(\, x = y \rarrow (\, x \in z \rarrow y \in z\, )\, ), \\
				&\forall x\, \forall y\, \forall z\, 
				(\, x = y \rarrow (\, z \in x \rarrow z \in y\, )\, ).
			\end{align}
		
		\item[置換] $\varphi$を$\lang{\in}$の式とし,
			$s,t$を$\varphi$に自由に現れる変項とし,
			$x$は$\varphi$で$s$への代入について自由であり,
			$y,z$は$\varphi$で$t$への代入について自由であるとするとき,
			次の式の全称閉包\footnotemark は公理である:
			\begin{align}
				\forall x\, \forall y\, \forall z\, 
				(\, \varphi(x,y) \wedge \varphi(x,z)
				\rarrow y = z\, )
				\rarrow \forall a\, \exists z\, \forall y\,
				(\, y \in z \lrarrow \exists x\, (\, x \in a \wedge 
				\varphi(x,y)\, )\, ).
			\end{align}
			
		\item[対] 
			\begin{align}
				\forall x\, \forall y\, \exists p\, \forall z\, 
				(\, x = z \vee y = z \lrarrow z \in p\, ).
			\end{align}
			
		\item[合併] 
			\begin{align}
				\forall x\, \exists u\, \forall y\, (\, \exists z\, (\, z \in x \wedge y \in z\, ) \lrarrow y \in u\, ).
			\end{align}
			
		\item[冪] 
			\begin{align}
				\forall x\, \exists p\, \forall y\, 
				(\, \forall z\, (\, z \in y \rarrow z \in x\, ) \lrarrow y \in p\, ).
			\end{align}
			
		\item[正則性] 
			\begin{align}
				\forall r\, (\, \exists x\, (\, x \in r\, ) \rarrow
				\exists y\, (\, y \in r \wedge \forall z\, (\, z \in r \rarrow
				z \notin y\, )\, )\, ).
			\end{align}
			
		\item[無限] 
			\begin{align}
				\exists x\, (\, 
				\exists s\, (\, \forall t\, (\, t \notin s\, ) \wedge s \in x\, ) 
				\wedge \forall y\, (\, 
				y \in x \rarrow \exists u\, (\, 
				\forall v\, (\, v \in u \lrarrow v \in y \vee v = y\, )
				\wedge u \in x\, )\, )\, ).
			\end{align}
	\end{description}
	
	\footnotetext{
		$\varphi$を$\lang{\varepsilon}$の式とするとき,$\varphi$の{\bf 全称閉包}
		\index{ぜんしょうへいほう@全称閉包}{\bf (universal closure)}とは
		\begin{align}
			\forall x_{1}\, \cdots \forall x_{n} \varphi
		\end{align}
		なる形の文を指す.ただし$x_{1},\cdots,x_{n}$は$\varphi$に自由に現れる変項であって,
		また$\varphi$に自由に現れる変項はこれらのみであるとする.$\varphi$が文であるときは
		$\varphi$自身を全称閉包とする.
	}
	
	\begin{screen}
		\begin{metathm}[$\Sigma$の定理は$\Gamma$の定理]
		\label{metathm:Henkin_expansion_3}
			$\psi$を$\lang{\in}$の文とするとき,
			$\Sigma \provable{\mbox{{\bf HE}},\lang{\varepsilon}} \psi$ならば
			$\Gamma \provable{\mbox{{\bf HE}},\lang{\varepsilon}} \psi$である.
		\end{metathm}
	\end{screen}
	
	\begin{sketch}
		$\Sigma$の公理が$\lang{\varepsilon}$の文であるときに$\Gamma$から証明できることを示せばよい.
		$\Sigma$と$\Gamma$で形が違う公理は外延性,相等性,要素,置換である
		(内包性公理は$\lang{\varepsilon}$の式ではありえないので今回は対象外).
		\begin{description}
			\item[外延性]	$a$と$b$を主要$\varepsilon$項とするとき,{\bf HE}の公理(UE)によって
				\begin{align}
					\Gamma &\provable{\mbox{{\bf HE}},\lang{\varepsilon}} \forall x\, \forall y\, (\, \forall z\, 
						(\, z \in x \lrarrow z \in y\, ) \rarrow x = y\, ), \\
					\Gamma &\provable{\mbox{{\bf HE}},\lang{\varepsilon}} \forall y\, (\, \forall z\, 
						(\, z \in a \lrarrow z \in y\, ) \rarrow a = y\, ), \\
					\Gamma &\provable{\mbox{{\bf HE}},\lang{\varepsilon}} \forall z\, 
						(\, z \in a \lrarrow z \in b\, ) \rarrow a = b
				\end{align}
				となる.$\Sigma$の相等性の公理も同様に導かれる.
				
			\item[要素] $a$と$b$を主要$\varepsilon$項とするとき,
				定理\ref{thm:any_class_equals_to_itself}と
				定理\ref{thm:critical_epsilon_term_is_set}の証明をもう一度おさらいすれば
				\begin{align}
					\sigma &\defeq \varepsilon z \negation (\, z \in a \lrarrow z \in a\, ), \\
					&\provable{\mbox{{\bf HE}},\lang{\varepsilon}} \sigma \in a \lrarrow \sigma \in a, 
						&& \mbox{含意の反射律(推論法則\ref{logicalthm:reflective_law_of_implication})と論理積の導入} \\
					&\provable{\mbox{{\bf HE}},\lang{\varepsilon}} \forall z\, (\, z \in a \lrarrow z \in a\, ), 
						&& \mbox{全称の導出(推論法則\ref{logicalthm:derivation_of_universal_by_epsilon})} \\
					\Gamma &\provable{\mbox{{\bf HE}},\lang{\varepsilon}} \forall z\, (\, z \in a \lrarrow z \in a\, ) \rarrow a = a, 
						&& \mbox{前段の結果} \\
					\Gamma &\provable{\mbox{{\bf HE}},\lang{\varepsilon}} a = a, 
						&& \mbox{三段論法} \\
					\Gamma &\provable{\mbox{{\bf HE}},\lang{\varepsilon}} \exists x\, (\, a = x\, )
						&& \mbox{{\bf HE}の公理(EI)}
				\end{align}
				となる.含意の導入(K)より
				\begin{align}
					\provable{\mbox{{\bf HE}},\lang{\varepsilon}} \exists x\, (\, a = x\, ) \rarrow
					(\, a \in b \rarrow \exists x\, (\, a = x\, )\, )
				\end{align}
				が成り立つので,三段論法より
				\begin{align}
					\Gamma \provable{\mbox{{\bf HE}},\lang{\varepsilon}} a \in b \rarrow \exists x\, (\, a = x\, )
				\end{align}
				が従う.
				
			\item[置換] $\psi$が
				\begin{align}
					\forall x\, \forall y\, \forall z\, 
					(\, \varphi(x,y) \wedge \varphi(x,z)
					\rarrow y = z\, )
					\rarrow \forall a\, \exists z\, \forall y\,
					(\, y \in z \lrarrow \exists x\, (\, x \in a \wedge 
					\varphi(x,y)\, )\, )
				\end{align}
				なる文であるとき,$\varphi$に現れる主要$\varepsilon$項で
				$\varphi$の中で極大であるもの(他の項の真部分項になっていないもの)
				を$\psi$に現れない変項で置き換える.その際主要$\varepsilon$項ごとに
				違う変項を用いるが,同じ主要$\varepsilon$項は同じ変項で置き換える.
				そうして得られた式を$\tilde{\psi}$とし,新しく追加した変項を
				$x_{1},\cdots,x_{n}$とすれば
				\begin{align}
					\forall x_{1} \cdots \forall x_{n} \tilde{\psi}(x_{1},\cdots,x_{n})
				\end{align}
				は$\Gamma$の置換公理となる.$x_{1},\cdots,x_{n}$によって置き換えられた
				主要$\varepsilon$項を$\tau_{1},\cdots,\tau_{n}$とすれば,
				{\bf HE}の公理(UE)より
				\begin{align}
					\Gamma &\provable{\mbox{{\bf HE}},\lang{\varepsilon}} \forall x_{1} \cdots \forall x_{n} \tilde{\psi}(x_{1},\cdots,x_{n}), \\
					\Gamma &\provable{\mbox{{\bf HE}},\lang{\varepsilon}} \forall x_{2} \cdots \forall x_{n} \tilde{\psi}(\tau_{1},x_{2},\cdots,x_{n}), \\
					\Gamma &\provable{\mbox{{\bf HE}},\lang{\varepsilon}} \forall x_{3} \cdots \forall x_{n} \tilde{\psi}(\tau_{1},\tau_{2},x_{3},\cdots,x_{n}), \\
					&\vdots \\
					\Gamma &\provable{\mbox{{\bf HE}},\lang{\varepsilon}} \tilde{\psi}(\tau_{1},\cdots,\tau_{n}), \\
				\end{align}
				が得られる.$\tilde{\psi}(\tau_{1},\cdots,\tau_{n})$とは$\psi$のことであるから
				$\psi$は$\Gamma$の定理である.
				\QED
		\end{description}
	\end{sketch}
	
	\begin{screen}
		\begin{metathm}[{\bf HE}で証明可能なら{\bf HK}でも証明可能]
		\label{metathm:Henkin_expansion_HE_to_HK}
			$\psi$を$\lang{\in}$の文とするとき,$\Sigma \provable{\mbox{{\bf HE}},\lang{\varepsilon}} \psi$ならば
			$\Gamma \provable{\mbox{{\bf HK}},\lang{\in}} \psi$である.
		\end{metathm}
	\end{screen}
	
	\begin{sketch}
		$\Sigma \provable{\mbox{{\bf HE}},\lang{\varepsilon}} \psi$ならば,
		メタ定理\ref{metathm:Henkin_expansion_3}より
		\begin{align}
			\Gamma \provable{\mbox{{\bf HE}},\lang{\varepsilon}} \psi
		\end{align}
		となり,メタ定理\ref{metathm:Henkin_expansion_1}より
		\begin{align}
			\Gamma \provable{\mbox{{\bf HK$\varepsilon$}},\lang{\varepsilon}} \psi
		\end{align}
		となり,メタ定理\ref{metathm:Henkin_expansion_2}より
		\begin{align}
			\Gamma \provable{\mbox{{\bf HK}},\lang{\varepsilon}} \psi
		\end{align}
		となる.最後の{\bf HK}の証明を$\varphi_{1},\cdots,\varphi_{n}$とする.
		$\varphi_{1},\cdots,\varphi_{n}$には主要$\varepsilon$項が残っている場合,
		$\varphi_{1},\cdots,\varphi_{n}$の中で極大に現れる(他の項の真部分項ではない)
		主要$\varepsilon$項が$e_{1},\cdots,e_{m}$で全てであるなら,
		$\varphi_{1},\cdots,\varphi_{n}$に現れない相異なる変項$y_{1},\cdots,y_{m}$
		を用意して,$e_{1}$を$y_{1}$に,$e_{2}$を$y_{2}$に,…,$e_{m}$を$y_{m}$に置き換える.
		全て置き換え終わった後の式の列はメタ定理\ref{metathm:Henkin_expansion_2_lemma}より
		{\bf HK}の証明であるし,式自体は$\lang{\in}$のものとなる.
		\QED
	\end{sketch}
	\section{正則証明}
	今度は逆に,$\lang{\in}$の式からなる{\bf HK}の証明から
	$\lang{\varepsilon}$の文からなる{\bf HE}の証明を構成する.
	{\bf HK}の証明の中で汎化が使われている場合,その固有変項は
	適当な主要$\varepsilon$項に置き換えることになる.たとえば
	\begin{align}
		\psi(x/a)
	\end{align}
	から($\psi$は$x$のみ自由に現れる式とする)
	\begin{align}
		\forall x \psi
	\end{align}
	が汎化で導かれる場合,$a$を$\varepsilon x \negation \psi$に置き換えれば
	\begin{align}
		\psi(x/\varepsilon x \negation \psi), 
		\quad \psi(x/\varepsilon x \negation \psi) \rarrow \forall x \psi
	\end{align}
	から三段論法で$\forall x \psi$が出てくる.ここで注意しておくと,汎化の固有変項の条件より
	$a$は$\forall x \psi$に自由に現れないので,$a$は$\psi$にも自由に現れず,
	\begin{align}
		\psi(x/a)(a/\varepsilon x \negation \psi)
	\end{align}
	と
	\begin{align}
		\psi(x/\varepsilon x \negation \psi)
	\end{align}
	は一致しているのである.固有変項の置き換えは証明全体で一斉に行うので,
	二つの汎化に対して同じ固有変項が使われている場合は
	代入する主要$\varepsilon$項をうまく選ぶことが出来ない.
	従って,どの固有変項も一度の汎化にしか用いられないように証明を直す必要がある.
	
	\begin{screen}
		\begin{metadfn}[正則証明]
			{\bf 正則証明}\index{せいそくしょうめい@正則証明}{\bf (regular proof)}とは,
			その証明の中に現れるどの固有変項も一度の汎化にしか用いられていないものである.
		\end{metadfn}
	\end{screen}
	
	\begin{screen}
		\begin{metathm}[証明の変項は取り替えても可能]
			$\varphi_{1},\cdots,\varphi_{n}$を$\Gamma$から$\varphi_{n}$への{\bf HK}の
			証明とし,$a$をこの証明に自由に現れる変項とし,$b$をこの証明に現れない変項とする.このとき
			\begin{align}
				\varphi_{1}(a/b),\ \varphi_{2}(a/b),\ \cdots,\ \varphi_{n}(a/b)
			\end{align}
			は$\Gamma$から$\varphi_{n}(a/b)$への{\bf HK}の証明となる.
		\end{metathm}
	\end{screen}
	
	\begin{metaprf}
		式の列が証明であるための条件に照合していく.各$\varphi_{i}$に対して
		\begin{description}
			\item[case1] $\varphi_{i}$が{\bf HK}の公理である場合,たとえば$\varphi_{i}$が
				\begin{align}
					\varphi \rarrow (\, \psi \rarrow \varphi\, )
				\end{align}
				なる形の公理ならば,$\varphi_{i}(a/b)$は
				\begin{align}
					\varphi(a/b) \rarrow (\, \psi(a/b) \rarrow \varphi(a/b)\, )
				\end{align}
				なる式であるから{\bf HK}の公理である.$\varphi_{i}$が
				\begin{align}
					\forall y\, (\, \psi \rarrow \varphi(x/y)\, )
					\rarrow (\, \psi \rarrow \forall x \varphi\, )
				\end{align}
				なる形の公理ならば,$\varphi_{i}(a/b)$は
				\begin{align}
					\forall y\, (\, \psi \rarrow \varphi(x/y)\, )
					\rarrow (\, \psi \rarrow \forall x \varphi\, )
				\end{align}
				
			\item[case2] $\varphi_{i}$が$\Gamma$の公理である場合,
				$\varphi_{i}$は文なので$\varphi_{i}(a/b)$は$\varphi_{i}$である.
			
			\item[case3] $\varphi_{i}$が前の式$\varphi_{j},\varphi_{k}$から
				三段論法で得られるとき,$\varphi_{k}$が$\varphi_{j} \rarrow \varphi_{i}$
				なる形の式であるとすれば,$\varphi_{k}(a/b)$は
				\begin{align}
					\varphi_{j}(a/b) \rarrow \varphi_{i}(a/b)
				\end{align}
				となる.つまり$\varphi_{i}(a/b)$は$\varphi_{j}(a/b)$と$\varphi_{k}(a/b)$から
				三段論法で得られる.
				
			\item[case4] $\varphi_{i}$が前の式$\varphi_{j}$から汎化で得られるとき,
				変項$e,x$と$x$が自由に現れる式$\psi$が取れて,$\varphi_{j}$は$\psi(x/e)$,
				$\varphi_{i}$は$\forall x \psi$なる式である.また$e$は$\psi$の中で$x$への代入
				について自由であり,$e$は$\forall x \psi$に自由に現れない.
				このとき$\varphi_{j}(a/b)$は
				\begin{align}
					\psi(a/b)(x/e)
				\end{align}
				となり,$\varphi_{i}(a/b)$は
				\begin{align}
					\forall x \psi(a/b)
				\end{align}
				となるので,$\varphi_{i}(a/b)$は$\varphi_{j}(a/b)$から汎化で得られる.
				\QED
		\end{description}
	\end{metaprf}
	
	\begin{screen}
		\begin{metathm}[どんな証明も正則化できる]
			$\varphi_{1},\cdots,\varphi_{n}$を$\Gamma$から$\varphi_{n}$への{\bf HK}の
			証明とするとき,$\Gamma$から$\varphi_{n}$への{\bf HK}の正則証明が得られる.
		\end{metathm}
	\end{screen}
	
	\begin{metaprf}
		$\varphi_{1},\cdots,\varphi_{n}$の中で汎化が使われていなければ
		これ自体が正則証明である.汎化が使われている場合,
		使われている箇所を
		\begin{align}
			\varphi_{i_{1}} \quad &\mbox{から} \quad \varphi_{j_{1}}, \\
			\varphi_{i_{2}} \quad &\mbox{から} \quad \varphi_{j_{2}}, \\
			&\vdots \\
			\varphi_{i_{\ell}} \quad &\mbox{から} \quad \varphi_{j_{\ell}}
		\end{align}
		とすべて列挙し(ただし$i_{1} < i_{2} < \cdots < i_{\ell}$),
		$a_{1},\cdots,a_{\ell}$をそれぞれの固有変項とする.
		ただし,もしかすると$a_{2}$と$a_{5}$は同じ文字$x$であるかもしれない.
		いま想定しているのはこのような状況であり,これから正則証明を構成するのである.
		
		$\varphi_{i_{1}}$の直後に$\varphi_{j_{1}}$を移動し,
		$\varphi_{i_{2}}$の直後に$\varphi_{j_{2}}$に移動し,…,
		$\varphi_{i_{\ell}}$の直後に$\varphi_{j_{\ell}}$を移動することによって
		$\varphi_{1},\cdots,\varphi_{n}$を並べ替えたものを
		\begin{align}
			\psi_{1},\cdots,\psi_{n}
		\end{align}
		と書けば,これもまた{\bf HK}の証明となっている.なぜならこの並び替えでは
		三段論法と汎化の順番が崩れないためである.並び替えによって
		$\varphi_{i_{1}},\cdots,\varphi_{i_{\ell}}$の位置も変動しうるが,
		動いた先をそれぞれ$\psi_{k_{1}},\cdots,\psi_{k_{\ell}}$とする.
		そして$b_{1},\cdots,b_{\ell}$を$\varphi_{i_{1}},\cdots,\varphi_{i_{\ell}}$に
		現れない相異なる変項とする.このとき
		\begin{align}
			&\psi_{1}(a_{1}/b_{1}),\ \cdots,\ \textcolor{red}{\psi_{k_{1}+1}(a_{1}/b_{1})}, \\
			&\psi_{1}(a_{2}/b_{2}),\ \cdots,\ \psi_{k_{1}+1}(a_{2}/b_{2}),\ \cdots,\ \textcolor{red}{\psi_{k_{2}+1}(a_{2}/b_{2})}, \\
			&\psi_{1}(a_{3}/b_{3}),\ \cdots,\ \psi_{k_{1}+2}(a_{3}/b_{3}),\ \cdots,\ \psi_{k_{2}+1}(a_{3}/b_{3}),\ \cdots,\ \textcolor{red}{\psi_{k_{3}+1}(a_{3}/b_{3})}, \\
			&\vdots \\
			&\psi_{1}(a_{\ell}/b_{\ell}),\ \cdots,\ \psi_{k_{1}+1}(a_{\ell}/b_{\ell}),\ \cdots,\ \psi_{k_{2}+1}(a_{\ell}/b_{\ell}),\ \cdots,\ 
			\psi_{k_{3}+1}(a_{\ell}/b_{\ell}), \cdots,\ \textcolor{red}{\psi_{k_{\ell}+1}(a_{\ell}/b_{\ell})}, \\
			&\psi_{k_{\ell}+2}(a_{\ell}/b_{\ell}),\ \cdots,\ \psi_{n}(a_{\ell}/b_{\ell})
		\end{align}
		は{\bf HK}の証明となっている.ところで,固有変項の条件より$a_{1}$は$\psi_{k_{1}+1}$に
		自由に現れないので$\psi_{k_{1}+1}(a_{1}/b_{1})$は$\psi_{k_{1}+1}$に一致する.
		同様に,赤字の$\psi_{k_{2}+1}(a_{2}/b_{2}),\ \psi_{k_{3}+1}(a_{3}/a_{3}),\ 
		\cdots,\ \psi_{k_{\ell}+1}(a_{\ell}/b_{\ell})$はそれぞれ
		$\psi_{k_{2}+1},\ \psi_{k_{3}+1},\ \cdots,\ \psi_{k_{\ell}+1}$と同じ式である.つまり
		\begin{align}
			&\psi_{1}(a_{1}/b_{1}),\ \cdots,\ \textcolor{red}{\psi_{k_{1}+1}}, \\
			&\psi_{1}(a_{2}/b_{2}),\ \cdots,\ \psi_{k_{1}+1}(a_{2}/b_{2}),\ \cdots,\ \textcolor{red}{\psi_{k_{2}+1}}, \\
			&\psi_{1}(a_{3}/b_{3}),\ \cdots,\ \psi_{k_{1}+2}(a_{3}/b_{3}),\ \cdots,\ \psi_{k_{2}+1}(a_{3}/b_{3}),\ \cdots,\ \textcolor{red}{\psi_{k_{3}+1}}, \\
			&\vdots \\
			&\psi_{1}(a_{\ell}/b_{\ell}),\ \cdots,\ \psi_{k_{1}+1}(a_{\ell}/b_{\ell}),\ \cdots,\ \psi_{k_{2}+1}(a_{\ell}/b_{\ell}),\ \cdots,\ 
			\psi_{k_{3}+1}(a_{\ell}/b_{\ell}), \cdots,\ \textcolor{red}{\psi_{k_{\ell}+1}}, \\
			&\psi_{k_{\ell}+2}(a_{\ell}/b_{\ell}),\ \cdots,\ \psi_{n}(a_{\ell}/b_{\ell})
		\end{align}
		は{\bf HK}の証明である.
		
		$\varphi$を$\lang{\in}$の文とし,$\varphi_{1},\cdots,\varphi_{n}$を
		$\lang{\in}$の式からなる$\varphi$への{\bf HK}の証明とする.
		$\varphi_{i}$から$\varphi_{j}$にかけて汎化が用いられ(固有変項$a$),
		$\varphi_{k}$から$\varphi_{\ell}$にかけて汎化が用いられているとき(固有変項$a$),
		$\varphi_{1},\cdots,\varphi_{n}$に自由に現れる$a$を$b$に置き換えたものを
		$\hat{\varphi}_{1},\cdots,\hat{\varphi}_{n}$と書けば,
		\begin{align}
			\varphi_{1},\cdots,\varphi_{j},
			\hat{\varphi}_{1},\cdots,\hat{\varphi}_{j-1},\hat{\varphi}_{j+1},
			\cdots,\hat{\varphi}_{n}
		\end{align}
		は$\varphi$への正則証明になっている.
		\QED
	\end{metaprf}
	
	\begin{screen}
		\begin{metathm}[{\bf HK}の定理は{\bf HE}の定理]
		\label{metathm:theorems_in_HK_provable_in_HE}
			$\mathscr{S}$を$\lang{\varepsilon}$の文からなる公理系とし,
			$\psi$を$\lang{\in}$の文とするとき,
			$\mathscr{S} \provable{\mbox{{\bf HK}}} \psi$ならば
			$\mathscr{S} \provable{\mbox{{\bf HE}}} \psi$である.
			ただし$\psi$への{\bf HK}の証明は$\lang{\in}$の式だけからなるものとする.
		\end{metathm}
	\end{screen}
	
	\begin{sketch}
		$\lang{\in}$の式の列$\varphi_{1},\cdots,\varphi_{n}$を
		$\mathscr{S}$から$\psi$への{\bf HK}の正則な証明とし,また
		\begin{align}
			a_{1},\cdots,a_{m}
		\end{align}
		をこの証明に使われる固有変項とし,$a_{1},a_{2},\cdots$の順番に汎化に用いられるとする.
	
		\begin{description}
			\item[step1]
				まず$\varphi_{1},\cdots,\varphi_{n}$の中に
				自由に現れる変項のうち,$a_{1},\cdots,a_{m}$以外のものをすべて相異なる
				主要$\varepsilon$項に置き換える.たとえば$x$が$\varphi_{1},\cdots,\varphi_{n}$
				のいずれかの式の中に自由に現れているなら,主要$\varepsilon$項$\tau$を取ってきて,
				$\varphi_{1},\cdots,\varphi_{n}$
				に自由に現れている$x$を一斉に$\tau$に置き換えるといった要領である.
				$a_{1},\cdots,a_{m}$以外の自由な変項を全て主要$\varepsilon$項に
				置き換え終わった式の列を
				\begin{align}
					\hat{\varphi}_{1}, \cdots, \hat{\varphi}_{n}
				\end{align}
				と書く.このとき,
				\begin{itemize}
					\item $\varphi_{i}$が(UI)と(EE)以外の{\bf HK}の公理ならば
						$\hat{\varphi}_{i}$は第\ref{chap:inference}章の推論法則である.
					\item $\varphi_{i}$が(UI)か(EE)ならば
						$\hat{\varphi}_{i}$は{\bf HE}から証明可能である(step3).
					\item $\varphi_{i}$が$\mathscr{S}$の公理ならば,
						$\varphi_{i}$は変項の置換による影響を受けないので
						$\hat{\varphi}_{i}$は$\varphi_{i}$と同一である.
					\item $\varphi_{i}$が前の式$\varphi_{j},\varphi_{k}$から
						三段論法で得られているならば,$\hat{\varphi}_{i}$も
						$\hat{\varphi}_{j},\hat{\varphi}_{k}$から三段論法で得られる.
					\item $\varphi_{i}$が前の式$\varphi_{j}$から
						汎化で得られているならば,$\hat{\varphi}_{i}$も
						$\hat{\varphi}_{j}$から汎化で得られる.
				\end{itemize}
			
			\item[step2]
				次に$a_{m},a_{m-1},\cdots$の順に固有変項を置き換える.$a_{m}$が
				\begin{align}
					F(x/a_{m})
				\end{align}
				から
				\begin{align}
					\forall x F
				\end{align}
				への汎化に使われているならば,$\hat{\varphi}_{1}, \cdots, \hat{\varphi}_{n}$に
				自由に現れる$a_{m}$を全て$\varepsilon x \negation F$に置き換えて,
				$\forall x F$の前の列に
				\begin{align}
					F(x/\varepsilon x \negation F) \rarrow \forall x F
				\end{align}
				への{\bf HE}の証明を挿入すればよい
				(推論法則\ref{logicalthm:derivation_of_universal_by_epsilon}).
				同様にして$a_{m-1},\cdots,a_{1}$も主要$\varepsilon$項に置き換えていく.
				
			\item[step3]
				step2の終了後に得られる式の列を
				$\tilde{\varphi}_{1},\cdots,\tilde{\varphi}_{m}$とする.
				これらは全て$\lang{\varepsilon}$の文であるが,
				この列の中に{\bf HK}の公理(UI)と(EE)の形の式が残っている場合は
				まだ{\bf HE}の証明ではない.とはいえ下で示す通り(UI)と(EE)は{\bf HE}で証明できるから,
				$\tilde{\varphi}_{1},\cdots,\tilde{\varphi}_{m}$の中で
				(UI)または(EE)の形の式があれば,その式の前の列にその式への{\bf HE}の
				証明を挿入すればよい.
			
				最後に(UI)と(EE)が{\bf HE}で証明可能であることを示す.
				\begin{description}
					\item[(UI)]
						$\forall y\, (\, \psi \rarrow \varphi(x/y)\, ) 
						\rarrow (\, \psi \rarrow \forall x \varphi\, )$を示す.
						{\bf HE}の公理(UE)より
						\begin{align}
							\forall y\, (\, \psi \rarrow \varphi(x/y)\, ) \provable{\mbox{{\bf HE}}} 
							\psi \rarrow \varphi(x/\varepsilon x \negation \varphi)
						\end{align}
						が成り立つので
						\begin{align}
							\psi,\ \forall y\, (\, \psi \rarrow \varphi(x/y)\, ) \provable{\mbox{{\bf HE}}} 
							\varphi(x/\varepsilon x \negation \varphi)
						\end{align}
						となり,全称の導出
						(推論法則\ref{logicalthm:derivation_of_universal_by_epsilon})
						\begin{align}
							\provable{\mbox{{\bf HE}}} \varphi(x/\varepsilon x \negation \varphi)
							\rarrow \forall x \varphi
						\end{align}
						との三段論法より
						\begin{align}
							\psi,\ \forall y\, (\, \psi \rarrow \varphi(x/y)\, ) \provable{\mbox{{\bf HE}}}
							\forall x \varphi
						\end{align}
						が従う.よって演繹定理より
						\begin{align}
							\provable{\mbox{{\bf HE}}} \forall y\, (\, \psi \rarrow \varphi(x/y)\, )
							\rarrow (\, \psi \rarrow \forall x \varphi\, )
						\end{align}
						が得られる.
					
					\item[(EE)]
						$\forall y\, (\, \varphi(x/y) \rarrow \psi\, ) 
						\rarrow (\, \exists x \varphi \rarrow \psi\, )$を示す.
						{\bf HE}の公理(UE)より
						\begin{align}
							\forall y\, (\, \varphi(x/y) \rarrow \psi\, ) \provable{\mbox{{\bf HE}}}
							\varphi(x/\varepsilon x \varphi) \rarrow \psi
						\end{align}
						が成り立ち,他方で{\bf HE}の公理(EE)より
						\begin{align}
							\exists x \varphi \provable{\mbox{{\bf HE}}} \varphi(x/\varepsilon x \varphi)
						\end{align}
						も成り立つので,三段論法より
						\begin{align}
							\exists x \varphi,\ \forall y\, (\, \varphi(x/y) \rarrow \psi\, ) \provable{\mbox{{\bf HE}}} \psi
						\end{align}
						が成り立つ.よって演繹定理より
						\begin{align}
							\provable{\mbox{{\bf HE}}} \forall y\, (\, \varphi(x/y) \rarrow \psi\, ) 
							\rarrow (\, \exists x \varphi \rarrow \psi\, )
						\end{align}
						が得られる.
						\QED
				\end{description}
		\end{description}
	\end{sketch}
	
	\begin{screen}
		\begin{metathm}[$\Gamma$の定理は$\Sigma$の定理]
			$\psi$を$\lang{\in}$の文とするとき,
			$\Gamma \provable{\mbox{{\bf HK}}} \psi$ならば
			$\Sigma \provable{\mbox{{\bf HE}}} \psi$である.
			ただし$\psi$への{\bf HK}の証明は$\lang{\in}$の式だけからなるものとする.
		\end{metathm}
	\end{screen}
	
	\begin{sketch}
		$\Gamma \provable{\mbox{{\bf HK}}} \psi$ならば
		メタ定理\ref{metathm:theorems_in_HK_provable_in_HE}より
		$\Gamma \provable{\mbox{{\bf HE}}} \psi$であるから,あとは
		$\Gamma$の公理が$\Sigma$から証明可能であることを示せばよい.
		$\Sigma$のものと違う$\Gamma$の公理は外延性,相等性,置換であるが,
		たとえば外延性公理
		\begin{align}
			\forall x\, \forall y\, (\, \forall z\, 
			(\, z \in x \lrarrow z \in y\, ) \rarrow x = y\, )
		\end{align}
		については
		\begin{align}
			a &\defeq \varepsilon x \negation \forall y\, (\, \forall z\, 
			(\, z \in x \lrarrow z \in y\, ) \rarrow x = y\, ), \\
			b &\defeq \varepsilon y \negation (\, \forall z\, 
			(\, z \in a \lrarrow z \in y\, ) \rarrow a = y\, ),
		\end{align}
		とおけば
		\begin{align}
			\Sigma \vdash \forall z\, (\, z \in a \lrarrow z \in b\, ) \rarrow a = b
		\end{align}
		が成り立つので,全称の導出
		(推論法則\ref{logicalthm:derivation_of_universal_by_epsilon})より
		\begin{align}
			\Sigma &\vdash \forall y\, (\, \forall z\, 
			(\, z \in a \lrarrow z \in y\, ) \rarrow a = y\, ), \\
			\Sigma &\vdash \forall x\, \forall y\, (\, \forall z\, 
			(\, z \in x \lrarrow z \in y\, ) \rarrow x = y\, )
		\end{align}
		が従う.相等性と置換の公理も同様にして導かれる.
		\QED
	\end{sketch}
	
\section{$\mathcal{L}$の証明の変換}
	$\lang{\varepsilon}$の証明は$\mathcal{L}$の証明でもあるが,逆に
	$\mathcal{L}$の証明を$\lang{\varepsilon}$の証明にっ変換することも出来る.
	
	いま$\varphi$を$\lang{\varepsilon}$の文とし,$\varphi_{1},\cdots,\varphi_{n}$を
	$\varphi$への$\mathcal{L}$の証明とする.
	そして$\varphi_{i}$を$\lang{\varepsilon}$の式に書き直し,$\hat{\varphi}_{i}$と書く.
	一般に式の書き換えは新しく用意する変項の違いで一意性を欠くが,
	同じ式を書き換える際に変項を揃えれば解決できる.
	たとえば,$\mathcal{L}$の文の列
	\begin{align}
		\varphi,\quad \varphi \rarrow \psi,\quad \psi
	\end{align}
	を$\lang{\varepsilon}$の文に書き換えるときは,
	左の$\varphi$を$\hat{\varphi}$に書き換えたならば,
	真ん中の$\varphi \rarrow \psi$は$\hat{\varphi} \rarrow \tilde{\psi}$に書き換えて,
	右の$\psi$は$\tilde{\psi}$に書き換えればよい.また
	\begin{align}
		\exists x G(x) \rarrow G(\varepsilon x \hat{G}(x))
	\end{align}
	なる$\mathcal{L}$の文については,
	\section{$\mathcal{L}$の証明の変換}
\label{sec:L_proof_to_L_epsilon_proof}
	この節では 「$\Sigma$から$\psi$への{\bf HE}の証明で$\mathcal{L}$の文の列
	であるものが取れる」ならば 「$\Sigma$から$\psi$への{\bf HE}の証明で
	$\lang{\varepsilon}$の文の列であるものが取れる」ことを示す
	
	\begin{screen}
		\begin{metathm}[$\mathcal{L}$の文の証明は$\lang{\varepsilon}$の文の証明に直せる]
			$\psi$を$\lang{\in}$の文とするとき,
			$\Sigma \provable{\mbox{{\bf HE}},\mathcal{L}} \psi$ならば
			$\Sigma \provable{\mbox{{\bf HE}},\lang{\varepsilon}} \psi$である.
		\end{metathm}
	\end{screen}
	
	\begin{metaprf}
		$\varphi_{1},\cdots,\varphi_{n}$を$\Sigma$から$\psi$への証明で$\mathcal{L}$の
		文の列であるものとし,これらを$\lang{\varepsilon}$の文に書き換えたもの
		(\ref{subsec:formula_rewriting}節参照)を
		$\widehat{\varphi}_{1},\cdots,\widehat{\varphi}_{n}$とする.
		ただし,同じ原子式の書き換えは証明全体で一致するようにしておく
		(書き換え時に用意する変項は$\varphi_{1},\cdots,\varphi_{n}$で使われていないものとする).
		このときメタ定理\ref{metathm:rewritten_formulas_are_of_L_epsilon}と
		メタ定理\ref{metathm:variables_unchanged_after_rewriting}より
		$\widehat{\varphi}_{1},\cdots,\widehat{\varphi}_{n}$はいずれも
		$\lang{\varepsilon}$の文であり,また各$\widehat{\varphi}_{i}$について次が言える:
		\begin{description}
			\item[(1)] $\varphi_{i}$が{\bf HE}の命題論理の公理ならば$\widehat{\varphi}_{i}$も
				{\bf HE}の公理である.
				
			\item[(2)] $\varphi_{i}$が{\bf HE}の量化公理か$\Sigma$の公理ならば$\Sigma 
				\provable{\mbox{{\bf HE}},\lang{\varepsilon}} \widehat{\varphi}_{i}$である.
				
			\item[(3)] $\varphi_{i}$が前の文$\varphi_{j},\varphi_{k}$から三段論法で
				得られている場合は,$\widehat{\varphi}_{i}$は$\widehat{\varphi}_{j}$と
				$\widehat{\varphi}_{k}$から三段論法で得られる.
		\end{description}
		
		(1)については,書き換えても式の結合形式が変わらないため.
		
		(3)については,$\varphi_{k}$を$\varphi_{j} \rarrow \varphi_{i}$なる文とすれば,
		同じ原子式の書き換えは証明全体で一致しているので$\widehat{\varphi_{k}}$は
		$\widehat{\varphi}_{j} \rarrow \widehat{\varphi}_{i}$なる文であり,
		$\widehat{\varphi}_{i}$は$\widehat{\varphi}_{j}$と$\widehat{\varphi}_{k}$から
		三段論法で得られるのである.
		
		(2)について,内包項を含みうる$\Sigma$の公理は外延性,相等性,内包性,要素である.
		これらと{\bf HE}の量化公理について一つずつ示していく.
		\begin{description}
			\item[case1] $\varphi_{i}$が
				\begin{align}
					\exists x \varphi \rarrow \varphi(\varepsilon x \check{\varphi})
				\end{align}
				なる公理であれば($\check{\varphi}$は$\varphi$の書き換え),
				$\widehat{\varphi}_{i}$は
				\begin{align}
					\exists x \widehat{\varphi} \rarrow 
					\widetilde{\varphi(\varepsilon x \check{\varphi})}
				\end{align}
				なる形の式である.ただし$\widehat{\varphi},
				\widetilde{\varphi(\varepsilon x \check{\varphi})}$はそれぞれ
				$\varphi,\varphi(\varepsilon x \check{\varphi})$の書き換えとする.
				まず量化公理と演繹定理の逆より
				\begin{align}
					\exists x \widehat{\varphi} \provable{\mbox{{\bf HE}},\lang{\varepsilon}} \widehat{\varphi}(\varepsilon x \widehat{\varphi})
				\end{align}
				となる.ここで,メタ定理\ref{metathm:substitution_to_rewritten_formula}より
				$\widehat{\varphi}(\varepsilon x \widehat{\varphi})$も
				$\check{\varphi}(\varepsilon x \widehat{\varphi})$も
				$\varphi(\varepsilon x \widehat{\varphi})$の書き換えであるし,
				メタ定理\ref{metathm:formula_rewritings_coincide_except_the_differences_of_bound_variables}
				より$\widehat{\varphi}(\varepsilon x \widehat{\varphi})$から始めて量化部分式を複数回
				差し替えていけば$\check{\varphi}(\varepsilon x \widehat{\varphi})$が得られるので,
				定理\ref{logicalthm:equivalence_by_replacing_bound_variables}より
				\begin{align}
					\provable{\mbox{{\bf HE}},\lang{\varepsilon}} \widehat{\varphi}(\varepsilon x \widehat{\varphi}) \rarrow \check{\varphi}(\varepsilon x \widehat{\varphi})
					\label{thm:L_proof_to_L_epsilon_proof_01}
				\end{align}
				が成り立つ.三段論法より
				\begin{align}
					\exists x \widehat{\varphi}, \provable{\mbox{{\bf HE}},\lang{\varepsilon}} \check{\varphi}(\varepsilon x \widehat{\varphi})
				\end{align}
				となり,存在記号の公理
				\begin{align}
					\provable{\mbox{{\bf HE}},\lang{\varepsilon}} \check{\varphi}(\varepsilon x \widehat{\varphi}) \rarrow \exists x \check{\varphi}
				\end{align}
				と併せて
				\begin{align}
					\exists x \widehat{\varphi} \provable{\mbox{{\bf HE}},\lang{\varepsilon}} \exists x \check{\varphi}
				\end{align}
				が従う.存在記号の公理より
				\begin{align}
					\provable{\mbox{{\bf HE}},\lang{\varepsilon}} \exists x \check{\varphi} \rarrow \check{\varphi}(\varepsilon x \check{\varphi})
				\end{align}
				が成り立つので,再び三段論法より
				\begin{align}
					\exists x \widehat{\varphi} &\provable{\mbox{{\bf HE}},\lang{\varepsilon}} \check{\varphi}(\varepsilon x \check{\varphi})
				\end{align}
				が従う.(\refeq{thm:L_proof_to_L_epsilon_proof_01})と同様の理由で
				%ここで定理\ref{logicalthm:equivalence_by_replacing_bound_variables}より
				\begin{align}
					\provable{\mbox{{\bf HE}},\lang{\varepsilon}} \check{\varphi}(\varepsilon x \check{\varphi}) \rarrow  \widetilde{\varphi(\varepsilon x \check{\varphi})}
				\end{align}
				が成り立つので,三段論法で
				%(定理\ref{metathm:substitution_to_rewritten_formula}より
				%$\check{\varphi}(\varepsilon x \check{\varphi})$も
				%$\varphi(\varepsilon x \check{\varphi})$の書き換え)
				\begin{align}
					\exists x \widehat{\varphi}, \provable{\mbox{{\bf HE}},\lang{\varepsilon}} \widetilde{\varphi(\varepsilon x \check{\varphi})}
				\end{align}
				が得られる.$\varphi_{i}$が$\varphi(\varepsilon x \negation \check{\varphi}) \rarrow \forall x \varphi$なる式の場合も同様である.
				
			\item[case2] $\varphi_{i}$が
				\begin{align}
					\varphi(\tau) \rarrow \exists x \varphi
				\end{align}
				なる公理であれば,$\widehat{\varphi}_{i}$は
				\begin{align}
					\widetilde{\varphi(\tau)} \rarrow \exists x \widehat{\varphi}
				\end{align}
				なる式となる.ただし$\widetilde{\varphi(\tau)},\widehat{\varphi}$は
				それぞれ$\varphi(\tau),\varphi$の書き換えとする.
				これは{\bf HE}から演繹可能である.実際,
				(\refeq{thm:L_proof_to_L_epsilon_proof_01})と同様の理由で
				%定理\ref{logicalthm:equivalence_by_replacing_bound_variables}より
				\begin{align}
					\provable{\mbox{{\bf HE}},\lang{\varepsilon}} 
					\widetilde{\varphi(\tau)} \rarrow \widehat{\varphi}(\tau)
				\end{align}
				が成り立ち,他方で存在記号の公理より
				\begin{align}
					\provable{\mbox{{\bf HE}},\lang{\varepsilon}} 
					\widehat{\varphi}(\tau) \rarrow \exists x \widehat{\varphi}
				\end{align}
				が成り立つので,
				\begin{align}
					\provable{\mbox{{\bf HE}},\lang{\varepsilon}} 
					\widetilde{\varphi(\tau)} \rarrow \exists x \widehat{\varphi}
				\end{align}
				が従う.$\varphi_{i}$が$\forall x \varphi \rarrow \varphi(\tau)$なる
				式の場合も同様である.
				
			%\item[case3] $\varphi_{i}$が
			%	\begin{align}
			%		\forall x \varphi \rarrow \varphi(\tau)
			%	\end{align}
			%	なる公理であれば,$\widehat{\varphi}_{i}$は
			%	\begin{align}
			%		\forall x \widetilde{\varphi} \rarrow \check{\varphi}(\tau)
			%	\end{align}
			%	なる式となる.これは{\bf HE}から演繹可能である.実際,{\bf HE}の量化公理より
			%	\begin{align}
			%		\provable{\mbox{{\bf HE}},\lang{\varepsilon}} \forall x \widetilde{\varphi} \rarrow \widetilde{\varphi}(\tau)
			%	\end{align}
			%	が成り立ち,他方で定理\ref{logicalthm:equivalence_by_replacing_bound_variables}より
			%	\begin{align}
			%		\provable{\mbox{{\bf HE}},\lang{\varepsilon}} \widetilde{\varphi}(\tau) \rarrow \check{\varphi}(\tau)
			%	\end{align}
			%	が成り立つので,
			%	\begin{align}
			%		\provable{\mbox{{\bf HE}},\lang{\varepsilon}} \forall x \widetilde{\varphi} \rarrow \check{\varphi}(\tau)
			%	\end{align}
			%	が従う.
				
			\item[case3] $\varphi_{i}$が外延性公理
				\begin{align}
					\forall x\, (\, x \in a \lrarrow x \in b\, ) \rarrow a = b
				\end{align}
				であるとき,$a,b$が共に主要$\varepsilon$項ならばこれは$\lang{\varepsilon}$
				の文である.
				$a$が$\Set{y}{\varphi(y)}$なる項で$b$が
				主要$\varepsilon$項であるときは,$\widehat{\varphi}_{i}$は
				\begin{align}
					\forall x\, (\, \varphi(x) \lrarrow x \in b\, ) \rarrow 
					\forall z\, (\, \varphi(z) \lrarrow z \in b\, )
				\end{align}
				なる形の文となり,これは${\bf HE}$で証明可能である.実際
				\begin{align}
					\zeta \defeq \varepsilon z \negation (\, \varphi(z) \lrarrow z \in b\, )
				\end{align}
				とおけば
				\begin{align}
					\forall x\, (\, \varphi(x) \lrarrow x \in b\, ) 
					\provable{\mbox{{\bf HE}},\lang{\varepsilon}} \varphi(\zeta) \lrarrow \zeta \in b
				\end{align}
				が成り立つので,全称の導出(論理的定理\ref{logicalthm:derivation_of_universal_by_epsilon})より
				\begin{align}
					\forall x\, (\, \varphi(x) \lrarrow x \in b\, ) 
					\provable{\mbox{{\bf HE}},\lang{\varepsilon}}
					\forall z\, (\, \varphi(z) \lrarrow z \in b\, )
				\end{align}
				となる.$a$が$\Set{y}{\varphi(y)}$なる項で$b$が$\Set{z}{\psi(z)}$なる
				項のときは,$\widehat{\varphi}_{i}$は
				\begin{align}
					\forall x\, (\, \varphi(x) \lrarrow \psi(x)\, ) \rarrow 
					\forall u\, (\, \varphi(u) \lrarrow \psi(u)\, )
				\end{align}
				なる形の文となり,これも${\bf HE}$で証明可能である.
				$a$が主要$\varepsilon$項で$b$が$\Set{z}{\psi(z)}$なる項のときも
				同様に$\widehat{\varphi}_{i}$は{\bf HE}で証明可能である.
			
			\item[case4] $\varphi_{i}$が内包性公理
				\begin{align}
					\forall x\, (\, x \in \Set{y}{\varphi(y)} \lrarrow \varphi(x)\, )
				\end{align}
				なる式であるとき,$\widehat{\varphi}_{i}$は
				\begin{align}
					\forall x\, (\, \varphi(x) \lrarrow \varphi(x)\, )
				\end{align}
				なる式であり,これは{\bf HE}から証明可能である.実際
				\begin{align}
					\tau \defeq \varepsilon x \negation (\, \varphi(x) \lrarrow \varphi(x)\, )
				\end{align}
				とおけば,含意の反射律(論理的定理\ref{logicalthm:reflective_law_of_implication})と論理積の導入より
				\begin{align}
					\provable{\mbox{{\bf HE}},\lang{\varepsilon}} \varphi(\tau) \lrarrow \varphi(\tau)
				\end{align}
				が成り立つので,全称の導出(論理的定理\ref{logicalthm:derivation_of_universal_by_epsilon})より
				\begin{align}
					\provable{\mbox{{\bf HE}},\lang{\varepsilon}} \forall x\, (\, \varphi(x) \lrarrow \varphi(x)\, )
				\end{align}
				となる.
			
			\item[case5] $\varphi_{i}$が要素の公理
				\begin{align}
					a \in b \rarrow \exists x\, (\, a = x\, )
				\end{align}
				なる式であるとき,$a$も$b$も主要$\varepsilon$項ならば
				\begin{align}
					\Sigma \provable{\mbox{{\bf HE}},\lang{\varepsilon}} \exists x\, (\, a = x\, )
				\end{align}
				(定理\ref{thm:critical_epsilon_term_is_set})と含意の導入
				\begin{align}
					\provable{\mbox{{\bf HE}},\lang{\varepsilon}} \exists x\, (\, a = x\, )
					\rarrow (\, a \in b \rarrow \exists x\, (\, a = x\, )\, )
				\end{align}
				から
				\begin{align}
					\Sigma \provable{\mbox{{\bf HE}},\lang{\varepsilon}} a \in b \rarrow \exists x\, (\, a = x\, )
				\end{align}
				が従う.$a$が主要$\varepsilon$項で$b$が$\Set{z}{\psi(z)}$なる項であるとき,
				$\widehat{\varphi}_{i}$は
				\begin{align}
					\psi(a) \rarrow \exists x\, (\, a = x\, )
				\end{align}
				となるが,上と同様にして{\bf HE}で証明できる.
				$a$が$\Set{y}{\varphi(y)}$なる項で$b$が主要$\varepsilon$であるとき,
				$\widehat{\varphi}_{i}$は
				\begin{align}
					\exists s\, (\, \forall u\, (\, \varphi(u) \lrarrow u \in s\, )
					\wedge s \in b\, ) \rarrow \exists x\, \forall v\, (\, \varphi(v) \lrarrow v \in x\, )
				\end{align}
				となるが,これも{\bf HE}で証明可能で,実際
				\begin{align}
					\sigma &\defeq \varepsilon s\, (\, \forall u\, (\, \varphi(u) \lrarrow u \in s\, ), \\
					\tau &\defeq \varepsilon v \negation (\, \varphi(v) \lrarrow v \in \sigma\, )
				\end{align}
				とおけば
				\begin{align}
					\exists s\, (\, \forall u\, (\, \varphi(u) \lrarrow u \in s\, )
					\wedge s \in b\, )
					&\provable{\mbox{{\bf HE}},\lang{\varepsilon}} 
					\forall u\, (\, \varphi(u) \lrarrow u \in \sigma\, )
					\wedge \sigma \in b, \\
					\exists s\, (\, \forall u\, (\, \varphi(u) \lrarrow u \in s\, )
					\wedge s \in b\, )
					&\provable{\mbox{{\bf HE}},\lang{\varepsilon}} 
					\forall u\, (\, \varphi(u) \lrarrow u \in \sigma\, ), \\
					\exists s\, (\, \forall u\, (\, \varphi(u) \lrarrow u \in s\, )
					\wedge s \in b\, )
					&\provable{\mbox{{\bf HE}},\lang{\varepsilon}} \varphi(\tau) \lrarrow \tau \in \sigma, \\
					\exists s\, (\, \forall u\, (\, \varphi(u) \lrarrow u \in s\, )
					\wedge s \in b\, )
					&\provable{\mbox{{\bf HE}},\lang{\varepsilon}} \forall v\, (\, \varphi(v) \lrarrow v \in \sigma\, ), \\
					\exists s\, (\, \forall u\, (\, \varphi(u) \lrarrow u \in s\, )
					\wedge s \in b\, )
					&\provable{\mbox{{\bf HE}},\lang{\varepsilon}} \exists x\, \forall v\, (\, \varphi(v) \lrarrow v \in x\, )
				\end{align}
				が成り立つ.$a$が$\Set{y}{\varphi(y)}$なる項で$b$が$\Set{z}{\psi(z)}$なる項
				であるとき,$\widehat{\varphi}_{i}$は
				\begin{align}
					\exists s\, (\, \forall u\, (\, \varphi(u) \lrarrow u \in s\, )
					\wedge \psi(s)\, ) \rarrow \exists x\, \forall v\, (\, \varphi(v) \lrarrow v \in x\, )
				\end{align}
				となるが,これも同様に{\bf HE}で証明可能である.
				
			\item[case6] $\varphi_{i}$が相等性公理
				\begin{align}
					a = b \rarrow b = a
				\end{align}
				なる式である場合,たとえば$a$が$\Set{y}{\varphi(y)}$なる項で
				$b$が主要$\varepsilon$項であれば,$\widehat{\varphi}_{i}$は
				\begin{align}
					\forall u\, (\, \varphi(u) \lrarrow u \in b\, ) 
					\rarrow \forall v\, (\, v \in b \lrarrow \varphi(v)\, ) 
				\end{align}
				となるが,これは{\bf HE}で証明可能であって,実際
				\begin{align}
					\tau \defeq \varepsilon v \negation (\, v \in b \lrarrow \varphi(v)\, )
				\end{align}
				とおけば
				\begin{align}
					\forall u\, (\, \varphi(u) \lrarrow u \in b\, ) 
					&\provable{\mbox{{\bf HE}},\lang{\varepsilon}} \varphi(\tau) \lrarrow \tau \in b, \\
					\forall u\, (\, \varphi(u) \lrarrow u \in b\, ) 
					&\provable{\mbox{{\bf HE}},\lang{\varepsilon}} \tau \in b \lrarrow \varphi(\tau), \\
					\forall u\, (\, \varphi(u) \lrarrow u \in b\, ) 
					&\provable{\mbox{{\bf HE}},\lang{\varepsilon}} \forall v\, (\, v \in b \lrarrow \varphi(v)\, )
				\end{align}
				が成り立つ.$a$も$b$も内包項である場合や,$a$が主要$\varepsilon$項で
				$b$が内包項である場合も同様のことが言える.
			
			\item[case7] $\varphi_{i}$が相等性公理
				\begin{align}
					a = b \rarrow (\, a \in c \rarrow b \in c\, )
				\end{align}
				なる式である場合,
				\begin{description}
					\item[case(7-1)] $a$と$b$が主要$\varepsilon$項で$c$が$\Set{x}{\xi(x)}$なる項であれば,
						$\widehat{\varphi}_{i}$は
						\begin{align}
							a = b \rarrow (\, \xi(a) \rarrow \xi(b)\, )
						\end{align}
						となるが,代入原理(定理\ref{thm:the_principle_of_substitution})より
						\begin{align}
							\Sigma \provable{\mbox{{\bf HE}},\lang{\varepsilon}} a = b \rarrow (\, \xi(a) \rarrow \xi(b)\, )
						\end{align}
						が成り立つ.
						
					\item[case(7-2)] $a$と$c$が主要$\varepsilon$項で$b$が$\Set{z}{\psi(z)}$なる項であれば,
						$\widehat{\varphi}_{i}$は
						\begin{align}
							\forall u\, (\, u \in a \lrarrow \psi(u)\, ) 
							\rarrow (\, a \in c 
							\rarrow \exists t\, (\, \forall w\, (\, \psi(w) \lrarrow w \in t\, ) \wedge t \in c\, )\, )
						\end{align}
						となるが,
						\begin{align}
							\omega \defeq \varepsilon w \negation (\, \psi(w) \lrarrow w \in a\, )
						\end{align}
						とおけば
						\begin{align}
							\forall u\, (\, u \in a \lrarrow \psi(u)\, )
							\provable{\mbox{{\bf HE}},\lang{\varepsilon}} \psi(\omega) \lrarrow \omega \in a 
						\end{align}
						が成り立つので
						\begin{align}
							\forall u\, (\, u \in a \lrarrow \psi(u)\, )
							\provable{\mbox{{\bf HE}},\lang{\varepsilon}} \forall w\, (\, \psi(w) \lrarrow w \in a\, )
						\end{align}
						が従い,
						\begin{align}
							\forall u\, (\, u \in a \lrarrow \psi(u)\, ),\ a \in c
							\provable{\mbox{{\bf HE}},\lang{\varepsilon}} \forall w\, (\, \psi(w) \lrarrow w \in a\, ) \wedge a \in c
						\end{align}
						より
						\begin{align}
							\forall u\, (\, u \in a \lrarrow \psi(u)\, ),\ a \in c
							\provable{\mbox{{\bf HE}},\lang{\varepsilon}} \exists t\, (\, \forall w\, (\, \psi(w) \lrarrow w \in t\, ) \wedge t \in c\, )\, )
						\end{align}
						となる.
						
					\item[case(7-3)] $a$が主要$\varepsilon$項で$b$が$\Set{z}{\psi(z)}$なる項で
						$c$が$\Set{x}{\xi(x)}$なる項であれば,$\widehat{\varphi}_{i}$は
						\begin{align}
							\forall u\, (\, u \in a \lrarrow \psi(u)\, ) 
							\rarrow (\, \xi(a) 
							\rarrow \exists t\, (\, \forall w\, (\, \psi(w) \lrarrow w \in t\, ) \wedge \xi(t)\, )\, )
						\end{align}
						となるが,
						\begin{align}
							\omega \defeq \varepsilon w \negation (\, \psi(w) \lrarrow w \in t\, )
						\end{align}
						とおけば
						\begin{align}
							\forall u\, (\, u \in a \lrarrow \psi(u)\, )
							\provable{\mbox{{\bf HE}},\lang{\varepsilon}} \psi(\omega) \lrarrow \omega \in a 
						\end{align}
						が成り立つので
						\begin{align}
							\forall u\, (\, u \in a \lrarrow \psi(u)\, )
							\provable{\mbox{{\bf HE}},\lang{\varepsilon}} \forall w\, (\, \psi(w) \lrarrow w \in t\, )
						\end{align}
						が従い,
						\begin{align}
							\forall u\, (\, u \in a \lrarrow \psi(u)\, )\, \xi(a)
							\provable{\mbox{{\bf HE}},\lang{\varepsilon}} \forall w\, (\, \psi(w) \lrarrow w \in t\, ) \wedge \xi(a)
						\end{align}
						より
						\begin{align}
							\forall u\, (\, u \in a \lrarrow \psi(u)\, )\, a \in c
							\provable{\mbox{{\bf HE}},\lang{\varepsilon}} \exists t\, (\, \forall w\, (\, \psi(w) \lrarrow w \in t\, ) \wedge \xi(t)\, )\, )
						\end{align}
						となる.
						
					\item[case(7-4)] $a$が$\Set{y}{\varphi(y)}$なる項で$b$と$z$が主要$\varepsilon$項の場合,
						$\widehat{\varphi}_{i}$は
						\begin{align}
							\forall u\, (\, \varphi(u) \lrarrow u \in b\, ) 
							\rarrow (\, \exists s\, (\, \forall v\, (\, \varphi(v) \lrarrow v \in s\, ) \wedge s \in c\, )
							\rarrow b \in c\, )
						\end{align}
						となるが,
						\begin{align}
							\sigma &\defeq \varepsilon s\, (\, \forall v\, (\, \varphi(v) \lrarrow v \in s\, ) \wedge s \in c\, ), \\
							\delta &\defeq \varepsilon u \negation (\, u \in \sigma \lrarrow u \in b\, )
						\end{align}
						とおけば
						\begin{align}
							\forall u\, (\, \varphi(u) \lrarrow u \in b\, ),\ 
							\exists s\, (\, \forall v\, (\, \varphi(v) \lrarrow v \in s\, ) \wedge s \in c\, )
							&\provable{\mbox{{\bf HE}},\lang{\varepsilon}} \varphi(\delta) \lrarrow \delta \in b, \\
							\forall u\, (\, \varphi(u) \lrarrow u \in b\, ),\ 
							\exists s\, (\, \forall v\, (\, \varphi(v) \lrarrow v \in s\, ) \wedge s \in c\, )
							&\provable{\mbox{{\bf HE}},\lang{\varepsilon}} \varphi(\delta) \lrarrow \delta \in \sigma
						\end{align}
						が成り立つので
						\begin{align}
							\forall u\, (\, \varphi(u) \lrarrow u \in b\, ),\ 
							\exists s\, (\, \forall v\, (\, \varphi(v) \lrarrow v \in s\, ) \wedge s \in c\, )
							\provable{\mbox{{\bf HE}},\lang{\varepsilon}} \delta \in \sigma \lrarrow \delta \in b
						\end{align}
						が従い,
						\begin{align}
							\forall u\, (\, \varphi(u) \lrarrow u \in b\, ),\ 
							\exists s\, (\, \forall v\, (\, \varphi(v) \lrarrow v \in s\, ) \wedge s \in c\, )
							\provable{\mbox{{\bf HE}},\lang{\varepsilon}} \forall u\, (\, u \in \sigma \lrarrow u \in b\, )
						\end{align}
						となり,外延性公理より
						\begin{align}
							\forall u\, (\, \varphi(u) \lrarrow u \in b\, ),\ 
							\exists s\, (\, \forall v\, (\, \varphi(v) \lrarrow v \in s\, ) \wedge s \in c\, ),\ 
							\Sigma
							\provable{\mbox{{\bf HE}},\lang{\varepsilon}} \sigma = b
						\end{align}
						が出る.一方で
						\begin{align}
							\exists s\, (\, \forall v\, (\, \varphi(v) \lrarrow v \in s\, ) \wedge s \in c\, )
							\provable{\mbox{{\bf HE}},\lang{\varepsilon}} \sigma \in c
						\end{align}
						となるので,相等性公理より
						\begin{align}
							\forall u\, (\, \varphi(u) \lrarrow u \in b\, ),\ 
							\exists s\, (\, \forall v\, (\, \varphi(v) \lrarrow v \in s\, ) \wedge s \in c\, ),\ 
							\Sigma
							\provable{\mbox{{\bf HE}},\lang{\varepsilon}} b \in c
						\end{align}
						が成り立つ.
						
					\item[case(7-5)] $a$が$\Set{y}{\varphi(y)}$なる項で$b$が主要$\varepsilon$項で
						$z$が$\Set{x}{\xi(x)}$なる項の場合,$\widehat{\varphi}_{i}$は
						\begin{align}
							\forall u\, (\, \varphi(u) \lrarrow u \in b\, ) 
							\rarrow (\, \exists s\, (\, \forall v\, (\, \varphi(v) \lrarrow v \in s\, ) \wedge \xi(s)\, )
							\rarrow \xi(b)\, )
						\end{align}
						となるが,
						\begin{align}
							\sigma &\defeq \varepsilon s\, (\, \forall v\, (\, \varphi(v) \lrarrow v \in s\, ) \wedge \xi(s)\, ), \\
							\delta &\defeq \varepsilon u \negation (\, u \in \sigma \lrarrow u \in b\, )
						\end{align}
						とおけば
						\begin{align}
							\forall u\, (\, \varphi(u) \lrarrow u \in b\, ),\ 
							\exists s\, (\, \forall v\, (\, \varphi(v) \lrarrow v \in s\, ) \wedge \xi(s)\, )
							&\provable{\mbox{{\bf HE}},\lang{\varepsilon}} \varphi(\delta) \lrarrow \delta \in b, \\
							\forall u\, (\, \varphi(u) \lrarrow u \in b\, ),\ 
							\exists s\, (\, \forall v\, (\, \varphi(v) \lrarrow v \in s\, ) \wedge \xi(s)\, )
							&\provable{\mbox{{\bf HE}},\lang{\varepsilon}} \varphi(\delta) \lrarrow \delta \in \sigma
						\end{align}
						が成り立つので
						\begin{align}
							\forall u\, (\, \varphi(u) \lrarrow u \in b\, ),\ 
							\exists s\, (\, \forall v\, (\, \varphi(v) \lrarrow v \in s\, ) \wedge \xi(s)\, )
							\provable{\mbox{{\bf HE}},\lang{\varepsilon}} \delta \in \sigma \lrarrow \delta \in b
						\end{align}
						が従い,
						\begin{align}
							\forall u\, (\, \varphi(u) \lrarrow u \in b\, ),\ 
							\exists s\, (\, \forall v\, (\, \varphi(v) \lrarrow v \in s\, ) \wedge \xi(s)\, )
							\provable{\mbox{{\bf HE}},\lang{\varepsilon}} \forall u\, (\, u \in \sigma \lrarrow u \in b\, )
						\end{align}
						となり,外延性公理より
						\begin{align}
							\forall u\, (\, \varphi(u) \lrarrow u \in b\, ),\ 
							\exists s\, (\, \forall v\, (\, \varphi(v) \lrarrow v \in s\, ) \wedge \xi(s)\, ),\ 
							\Sigma
							\provable{\mbox{{\bf HE}},\lang{\varepsilon}} \sigma = b
						\end{align}
						が出る.一方で
						\begin{align}
							\exists s\, (\, \forall v\, (\, \varphi(v) \lrarrow v \in s\, ) \wedge \xi(s)\, )
							\provable{\mbox{{\bf HE}},\lang{\varepsilon}} \xi(\sigma)
						\end{align}
						となるので,代入原理(定理\ref{thm:the_principle_of_substitution})より
						\begin{align}
							\forall u\, (\, \varphi(u) \lrarrow u \in b\, ),\ 
							\exists s\, (\, \forall v\, (\, \varphi(v) \lrarrow v \in s\, ) \wedge \xi(s)\, ),\ 
							\Sigma
							\provable{\mbox{{\bf HE}},\lang{\varepsilon}} \xi(b)
						\end{align}
						が成り立つ.
						
					\item[case(7-6)] $a$が$\Set{y}{\varphi(y)}$なる項で$b$が$\Set{z}{\psi(z)}$なる項で
						$c$が主要$\varepsilon$項であれば,$\widehat{\varphi}_{i}$は
						\begin{align}
							\forall u\, (\, \varphi(u) \lrarrow \psi(u)\, ) 
							&\rarrow (\, \exists s\, (\, \forall v\, (\, \varphi(v) \lrarrow v \in s\, ) \wedge s \in c\, ) \\
							&\rarrow \exists t\, (\, \forall w\, (\, \psi(w) \lrarrow w \in t\, ) \wedge t \in c\, )\, )
						\end{align}
						となるが,
						\begin{align}
							\tau &\defeq \varepsilon s\, (\, \forall v\, (\, \varphi(v) \lrarrow v \in s\, ) \wedge s \in c\, ), \\
							\omega &\defeq \varepsilon w \negation (\, \psi(w) \lrarrow w \in t\, )
						\end{align}
						とおけば,まず
						\begin{align}
							\forall u\, (\, \varphi(u) \lrarrow \psi(u)\, ),\ 
							\exists s\, (\, \forall v\, (\, \varphi(v) \lrarrow v \in s\, ) \wedge s \in c\, )
							&\provable{\mbox{{\bf HE}},\lang{\varepsilon}} \varphi(\omega) \lrarrow \psi(\omega), \\
							\forall u\, (\, \varphi(u) \lrarrow \psi(u)\, ),\ 
							\exists s\, (\, \forall v\, (\, \varphi(v) \lrarrow v \in s\, ) \wedge s \in c\, )
							&\provable{\mbox{{\bf HE}},\lang{\varepsilon}} \varphi(\omega) \lrarrow \omega \in \tau
						\end{align}
						が成り立つので
						\begin{align}
							\forall u\, (\, \varphi(u) \lrarrow \psi(u)\, ),\ 
							\exists s\, (\, \forall v\, (\, \varphi(v) \lrarrow v \in s\, ) \wedge s \in c\, )
							\provable{\mbox{{\bf HE}},\lang{\varepsilon}} \psi(\omega) \lrarrow \omega \in \tau
						\end{align}
						が従い
						\begin{align}
							\forall u\, (\, \varphi(u) \lrarrow \psi(u)\, ),\ 
							\exists s\, (\, \forall v\, (\, \varphi(v) \lrarrow v \in s\, ) \wedge s \in c\, )
							\provable{\mbox{{\bf HE}},\lang{\varepsilon}} \forall w\, (\, \psi(w) \lrarrow w \in \tau\, )
						\end{align}
						が出る.他方で
						\begin{align}
							\exists s\, (\, \forall v\, (\, \varphi(v) \lrarrow v \in s\, ) \wedge s \in c\, )
							\provable{\mbox{{\bf HE}},\lang{\varepsilon}} \tau \in c, \\
						\end{align}
						が成り立つので,
						\begin{align}
							\begin{gathered}
								\forall u\, (\, \varphi(u) \lrarrow \psi(u)\, ),\ 
								\exists s\, (\, \forall v\, (\, \varphi(v) \lrarrow v \in s\, ) \wedge s \in c\, ) \\
								\provable{\mbox{{\bf HE}},\lang{\varepsilon}}
								\forall w\, (\, \psi(w) \lrarrow w \in \tau\, ) \wedge \tau \in c
							\end{gathered}
						\end{align}
						となり
						\begin{align}
							\begin{gathered}
								\forall u\, (\, \varphi(u) \lrarrow \psi(u)\, ),\ 
								\exists s\, (\, \forall v\, (\, \varphi(v) \lrarrow v \in s\, ) \wedge s \in c\, ) \\
								\provable{\mbox{{\bf HE}},\lang{\varepsilon}}
								\exists t\, (\, \forall w\, (\, \psi(w) \lrarrow w \in t\, ) \wedge t \in c\, )
							\end{gathered}
						\end{align}
						が得られる.
						
					\item[case(7-7)] $a$が$\Set{y}{\varphi(y)}$なる項で$b$が$\Set{z}{\psi(z)}$なる項で
						$c$が$\Set{x}{\xi(x)}$なる項であれば,$\widehat{\varphi}_{i}$は
						\begin{align}
							\forall u\, (\, \varphi(u) \lrarrow \psi(u)\, ) 
							&\rarrow (\, \exists s\, (\, \forall v\, (\, \varphi(v) \lrarrow v \in s\, ) \wedge \xi(s)\, ) \\
							&\rarrow \exists t\, (\, \forall w\, (\, \psi(w) \lrarrow w \in t\, ) \wedge \xi(t)\, )\, )
						\end{align}
						となるが,
						\begin{align}
							\tau &\defeq \varepsilon s\, (\, \forall v\, (\, \varphi(v) \lrarrow v \in s\, ) \wedge \xi(s)\, ), \\
							\omega &\defeq \varepsilon w \negation (\, \psi(w) \lrarrow w \in t\, )
						\end{align}
						とおけば,まず
						\begin{align}
							\forall u\, (\, \varphi(u) \lrarrow \psi(u)\, ),\ 
							\exists s\, (\, \forall v\, (\, \varphi(v) \lrarrow v \in s\, ) \wedge \xi(s)\, )
							&\provable{\mbox{{\bf HE}},\lang{\varepsilon}} \varphi(\omega) \lrarrow \psi(\omega), \\
							\forall u\, (\, \varphi(u) \lrarrow \psi(u)\, ),\ 
							\exists s\, (\, \forall v\, (\, \varphi(v) \lrarrow v \in s\, ) \wedge \xi(s)\, )
							&\provable{\mbox{{\bf HE}},\lang{\varepsilon}} \varphi(\omega) \lrarrow \omega \in \tau
						\end{align}
						が成り立つので
						\begin{align}
							\forall u\, (\, \varphi(u) \lrarrow \psi(u)\, ),\ 
							\exists s\, (\, \forall v\, (\, \varphi(v) \lrarrow v \in s\, ) \wedge \xi(s)\, )
							\provable{\mbox{{\bf HE}},\lang{\varepsilon}} \psi(\omega) \lrarrow \omega \in \tau
						\end{align}
						が従い
						\begin{align}
							\forall u\, (\, \varphi(u) \lrarrow \psi(u)\, ),\ 
							\exists s\, (\, \forall v\, (\, \varphi(v) \lrarrow v \in s\, ) \wedge \xi(s)\, )
							\provable{\mbox{{\bf HE}},\lang{\varepsilon}} \forall w\, (\, \psi(w) \lrarrow w \in \tau\, )
						\end{align}
						が出る.他方で
						\begin{align}
							\exists s\, (\, \forall v\, (\, \varphi(v) \lrarrow v \in s\, ) \wedge \xi(s)\, )
							\provable{\mbox{{\bf HE}},\lang{\varepsilon}} \xi(\tau), \\
						\end{align}
						が成り立つので,
						\begin{align}
							\begin{gathered}
								\forall u\, (\, \varphi(u) \lrarrow \psi(u)\, ),\ 
								\exists s\, (\, \forall v\, (\, \varphi(v) \lrarrow v \in s\, ) \wedge \xi(s)\, ) \\
								\provable{\mbox{{\bf HE}},\lang{\varepsilon}}
								\forall w\, (\, \psi(w) \lrarrow w \in \tau\, ) \wedge \xi(\tau)
							\end{gathered}
						\end{align}
						となり
						\begin{align}
							\begin{gathered}
								\forall u\, (\, \varphi(u) \lrarrow \psi(u)\, ),\ 
								\exists s\, (\, \forall v\, (\, \varphi(v) \lrarrow v \in s\, ) \wedge \xi(s)\, ) \\
								\provable{\mbox{{\bf HE}},\lang{\varepsilon}}
								\exists t\, (\, \forall w\, (\, \psi(w) \lrarrow w \in t\, ) \wedge \xi(t)\, )
							\end{gathered}
						\end{align}
						が得られる.
				\end{description}
				
			\item[case8] $\varphi_{i}$が相等性公理
				\begin{align}
					a = b \rarrow (\, c \in a \rarrow c \in b\, )
				\end{align}
				なる式である場合,
				\begin{description}
					\item[case(8-1)] $a$と$b$が主要$\varepsilon$項で$c$が$\Set{x}{\xi(x)}$なる項であれば,
						$\widehat{\varphi}_{i}$は
						\begin{align}
							a = b &\rarrow (\, \exists s\, (\, \forall u\, (\, \xi(u) \lrarrow u \in s\, ) \wedge s \in a\, ) \\
							&\rarrow (\, \exists t\, (\, \forall v\, (\, \xi(v) \lrarrow v \in t\, ) \wedge t \in b\, )\, )
						\end{align}
						となるが,ここで
						\begin{align}
							\sigma \defeq \varepsilon s \, (\, \forall u\, (\, \xi(u) \lrarrow u \in s\, ) \wedge s \in a\, )
						\end{align}
						とおけば
						\begin{align}
							\exists s \, (\, \forall u\, (\, \xi(u) \lrarrow u \in s\, ) \wedge s \in a\, )
							\provable{\mbox{{\bf HE}},\lang{\varepsilon}} \sigma \in a
						\end{align}
						が成り立つので,相等性公理より
						\begin{align}
							a = b,\ \exists s \, (\, \forall u\, (\, \xi(u) \lrarrow u \in s\, ) \wedge s \in a\, ),\ \Sigma
							\provable{\mbox{{\bf HE}},\lang{\varepsilon}} \sigma \in b
						\end{align}
						となる.他方で
						\begin{align}
							\exists s \, (\, \forall u\, (\, \xi(u) \lrarrow u \in s\, ) \wedge s \in a\, )
							\provable{\mbox{{\bf HE}},\lang{\varepsilon}} \forall u\, (\, \xi(u) \lrarrow u \in \sigma\, ) 
						\end{align}
						も成り立つので
						\begin{align}
							a = b,\ \exists s \, (\, \forall u\, (\, \xi(u) \lrarrow u \in s\, ) \wedge s \in a\, ),\ \Sigma
							\provable{\mbox{{\bf HE}},\lang{\varepsilon}} \forall u\, (\, \xi(u) \lrarrow u \in \sigma\, ) \wedge \sigma \in b
						\end{align}
						が従い,
						\begin{align}
							a = b,\ \exists s \, (\, \forall u\, (\, \xi(u) \lrarrow u \in s\, ) \wedge s \in a\, ),\ \Sigma
							\provable{\mbox{{\bf HE}},\lang{\varepsilon}} \exists t\, (\, \forall v\, (\, \xi(v) \lrarrow v \in t\, ) \wedge t \in b\, )
						\end{align}
						が出る.
						
					\item[case(8-2)] $a$と$c$が主要$\varepsilon$項で$b$が$\Set{z}{\psi(z)}$なる項であれば,
						$\widehat{\varphi}_{i}$は
						\begin{align}
							\forall u\, (\, u \in a \lrarrow \psi(u)\, ) 
							\rarrow (\, c \in a \rarrow \psi(c)\, )
						\end{align}
						となるが,これは
						\begin{align}
							\forall u\, (\, u \in a \lrarrow \psi(u)\, ) \vdash c \in a \rarrow \psi(c)
						\end{align}
						より直接得られる.
						
					\item[case(8-3)] $a$が主要$\varepsilon$項で$b$が$\Set{z}{\psi(z)}$なる項で
						$c$が$\Set{x}{\xi(x)}$なる項であれば,$\widehat{\varphi}_{i}$は
						\begin{align}
							\forall u\, (\, u \in a \lrarrow \psi(u)\, ) 
							&\rarrow (\, \exists s\, (\, \forall u\, (\, \xi(u) \lrarrow u \in s\, ) \wedge s \in a\, ) \\
							&\rarrow \exists t\, (\, \forall v\, (\, \xi(v) \lrarrow v \in t\, ) \wedge \psi(t)\, )\, )
						\end{align}
						となるが,
						\begin{align}
							\sigma \defeq \varepsilon s\, (\, \forall u\, (\, \xi(u) \lrarrow u \in s\, ) \wedge s \in a\, )
						\end{align}
						とおけば
						\begin{align}
							\exists s\, (\, \forall u\, (\, \xi(u) \lrarrow u \in s\, ) \wedge s \in a\, ) 
							&\provable{\mbox{{\bf HE}},\lang{\varepsilon}} \sigma \in a, \\
							\forall u\, (\, u \in a \lrarrow \psi(u)\, )
							&\provable{\mbox{{\bf HE}},\lang{\varepsilon}} \sigma \in a \rarrow \psi(\sigma)
						\end{align}
						より
						\begin{align}
							\forall u\, (\, u \in a \lrarrow \psi(u)\, ),\ 
							\exists s\, (\, \forall u\, (\, \xi(u) \lrarrow u \in s\, ) \wedge s \in a\, ) 
							\provable{\mbox{{\bf HE}},\lang{\varepsilon}} \psi(\sigma)
						\end{align}
						が成り立つ.他方で
						\begin{align}
							\exists s\, (\, \forall u\, (\, \xi(u) \lrarrow u \in s\, ) \wedge s \in a\, ) 
							\provable{\mbox{{\bf HE}},\lang{\varepsilon}} \forall u\, (\, \xi(u) \lrarrow u \in \sigma\, )
						\end{align}
						も成り立つので,
						\begin{align}
							\omega \defeq \varepsilon v \negation (\, \xi(v) \lrarrow v \in \sigma\, )
						\end{align}
						とおけば
						\begin{align}
							\exists s\, (\, \forall u\, (\, \xi(u) \lrarrow u \in s\, ) \wedge s \in a\, ) 
							\provable{\mbox{{\bf HE}},\lang{\varepsilon}} \xi(\omega) \lrarrow \omega \in \sigma
						\end{align}
						となり
						\begin{align}
							\exists s\, (\, \forall u\, (\, \xi(u) \lrarrow u \in s\, ) \wedge s \in a\, ) 
							\provable{\mbox{{\bf HE}},\lang{\varepsilon}} \forall v\, (\, \xi(v) \lrarrow v \in \sigma\, )
						\end{align}
						が従う.ゆえに
						\begin{align}
							\begin{gathered}
								\forall u\, (\, u \in a \lrarrow \psi(u)\, ),\ 
								\exists s\, (\, \forall u\, (\, \xi(u) \lrarrow u \in s\, ) \wedge s \in a\, ) \\
								\provable{\mbox{{\bf HE}},\lang{\varepsilon}} \forall v\, (\, \xi(v) \lrarrow v \in \sigma\, ) \wedge \psi(\sigma)
							\end{gathered}
						\end{align}
						が従い
						\begin{align}
							\begin{gathered}
								\forall u\, (\, u \in a \lrarrow \psi(u)\, ),\ 
								\exists s\, (\, \forall u\, (\, \xi(u) \lrarrow u \in s\, ) \wedge s \in a\, ) \\
								\provable{\mbox{{\bf HE}},\lang{\varepsilon}} \exists t\, (\, \forall v\, (\, \xi(v) \lrarrow v \in t\, ) \wedge \psi(t)\, )
							\end{gathered}
						\end{align}
						が出る.
						
					\item[case(8-4)] $a$が$\Set{y}{\varphi(y)}$なる項で$b$と$z$が主要$\varepsilon$項の場合,
						$\widehat{\varphi}_{i}$は
						\begin{align}
							\forall u\, (\, \varphi(u) \lrarrow u \in b\, ) 
							\rarrow (\, \varphi(c) \rarrow c \in b\, )
						\end{align}
						となるが,これは
						\begin{align}
							\forall u\, (\, \varphi(u) \lrarrow u \in b\, ) 
							\provable{\mbox{{\bf HE}},\lang{\varepsilon}} 
							\varphi(c) \rarrow c \in b
						\end{align}
						から直接得られる.
						
					\item[case(8-5)] $a$が$\Set{y}{\varphi(y)}$なる項で$b$が主要$\varepsilon$項で
						$z$が$\Set{x}{\xi(x)}$なる項の場合,$\widehat{\varphi}_{i}$は
						\begin{align}
							\forall u\, (\, \varphi(u) \lrarrow u \in b\, ) 
							&\rarrow (\, \exists s\, (\, \forall u\, (\, \xi(u) \lrarrow u \in s\, ) \wedge \varphi(s)\, ) \\
							&\rarrow \exists t\, (\, \forall v\, (\, \xi(v) \lrarrow v \in t\, ) \wedge t \in b\, )\, )
						\end{align}
						となるが,
						\begin{align}
							\sigma \defeq \varepsilon s\, (\, \forall u\, (\, \xi(u) \lrarrow u \in s\, ) \wedge \varphi(s)\, )
						\end{align}
						とおけば,
						\begin{align}
							\exists s\, (\, \forall u\, (\, \xi(u) \lrarrow u \in s\, ) \wedge \varphi(s)\, ) 
							&\provable{\mbox{{\bf HE}},\lang{\varepsilon}} \varphi(\sigma), \\
							\forall u\, (\, \varphi(u) \lrarrow u \in b\, )
							&\provable{\mbox{{\bf HE}},\lang{\varepsilon}} \varphi(\sigma) \rarrow \sigma \in b
						\end{align}
						より
						\begin{align}
							\forall u\, (\, \varphi(u) \lrarrow u \in b\, ),\ 
							\exists s\, (\, \forall u\, (\, \xi(u) \lrarrow u \in s\, ) \wedge \varphi(s)\, ) 
							\provable{\mbox{{\bf HE}},\lang{\varepsilon}} \sigma \in b
						\end{align}
						が成り立つ.他方で
						\begin{align}
							\exists s\, (\, \forall u\, (\, \xi(u) \lrarrow u \in s\, ) \wedge \varphi(s)\, ) 
							\provable{\mbox{{\bf HE}},\lang{\varepsilon}} \forall u\, (\, \xi(u) \lrarrow u \in \sigma\, )
						\end{align}
						も成り立つので,
						\begin{align}
							\omega \defeq \varepsilon v \negation (\, \xi(v) \lrarrow v \in \sigma\, )
						\end{align}
						とおけば
						\begin{align}
							\exists s\, (\, \forall u\, (\, \xi(u) \lrarrow u \in s\, ) \wedge \varphi(s)\, ) 
							\provable{\mbox{{\bf HE}},\lang{\varepsilon}} \xi(\omega) \lrarrow \omega \in \sigma
						\end{align}
						となり
						\begin{align}
							\exists s\, (\, \forall u\, (\, \xi(u) \lrarrow u \in s\, ) \wedge \varphi(s)\, ) 
							\provable{\mbox{{\bf HE}},\lang{\varepsilon}} \forall v\, (\, \xi(v) \lrarrow v \in \sigma\, )
						\end{align}
						が従う.ゆえに
						\begin{align}
							\begin{gathered}
								\forall u\, (\, \varphi(u) \lrarrow u \in b\, ),\ 
								\exists s\, (\, \forall u\, (\, \xi(u) \lrarrow u \in s\, ) \wedge \varphi(s)\, ) \\
								\provable{\mbox{{\bf HE}},\lang{\varepsilon}} \forall v\, (\, \xi(v) \lrarrow v \in \sigma\, ) \wedge \sigma \in b
							\end{gathered}
						\end{align}
						が従い
						\begin{align}
							\begin{gathered}
								\forall u\, (\, \varphi(u) \lrarrow u \in b\, ),\ 
								\exists s\, (\, \forall u\, (\, \xi(u) \lrarrow u \in s\, ) \wedge \varphi(s)\, ) \\
								\provable{\mbox{{\bf HE}},\lang{\varepsilon}} \exists t\, (\, \forall v\, (\, \xi(v) \lrarrow v \in t\, ) \wedge t \in b\, )
							\end{gathered}
						\end{align}
						が出る.
						
					\item[case(8-6)] $a$が$\Set{y}{\varphi(y)}$なる項で$b$が$\Set{z}{\psi(z)}$なる項で
						$c$が主要$\varepsilon$項であれば,$\widehat{\varphi}_{i}$は
						\begin{align}
							\forall u\, (\, \varphi(u) \lrarrow \psi(u)\, ) \rarrow (\, \varphi(c) \rarrow \psi(c)\, )
						\end{align}
						となるが,これは
						\begin{align}
							\forall u\, (\, \varphi(u) \lrarrow \psi(u)\, ) 
							\provable{\mbox{{\bf HE}},\lang{\varepsilon}} 
							\varphi(c) \rarrow \psi(c)
						\end{align}
						から直接得られる.
						
					\item[case(8-7)] $a$が$\Set{y}{\varphi(y)}$なる項で$b$が$\Set{z}{\psi(z)}$なる項で
						$c$が$\Set{x}{\xi(x)}$なる項であれば,$\widehat{\varphi}_{i}$は
						\begin{align}
							\forall u\, (\, \varphi(u) \lrarrow u \in b\, ) 
							&\rarrow (\, \exists s\, (\, \forall u\, (\, \xi(u) \lrarrow u \in s\, ) \wedge \varphi(s)\, ) \\
							&\rarrow \exists t\, (\, \forall v\, (\, \xi(v) \lrarrow v \in t\, ) \wedge \psi(t)\, )\, )
						\end{align}
						となるが,
						\begin{align}
							\sigma \defeq \varepsilon s\, (\, \forall u\, (\, \xi(u) \lrarrow u \in s\, ) \wedge \varphi(s)\, )
						\end{align}
						とおけば,
						\begin{align}
							\exists s\, (\, \forall u\, (\, \xi(u) \lrarrow u \in s\, ) \wedge \varphi(s)\, ) 
							&\provable{\mbox{{\bf HE}},\lang{\varepsilon}} \varphi(\sigma), \\
							\forall u\, (\, \varphi(u) \lrarrow u \in b\, )
							&\provable{\mbox{{\bf HE}},\lang{\varepsilon}} \varphi(\sigma) \rarrow \psi(\sigma)
						\end{align}
						より
						\begin{align}
							\forall u\, (\, \varphi(u) \lrarrow u \in b\, ),\ 
							\exists s\, (\, \forall u\, (\, \xi(u) \lrarrow u \in s\, ) \wedge \varphi(s)\, ) 
							\provable{\mbox{{\bf HE}},\lang{\varepsilon}} \psi(\sigma)
						\end{align}
						が成り立つ.他方で
						\begin{align}
							\exists s\, (\, \forall u\, (\, \xi(u) \lrarrow u \in s\, ) \wedge \varphi(s)\, ) 
							\provable{\mbox{{\bf HE}},\lang{\varepsilon}} \forall u\, (\, \xi(u) \lrarrow u \in \sigma\, )
						\end{align}
						も成り立つので,
						\begin{align}
							\omega \defeq \varepsilon v \negation (\, \xi(v) \lrarrow v \in \sigma\, )
						\end{align}
						とおけば
						\begin{align}
							\exists s\, (\, \forall u\, (\, \xi(u) \lrarrow u \in s\, ) \wedge \varphi(s)\, ) 
							\provable{\mbox{{\bf HE}},\lang{\varepsilon}} \xi(\omega) \lrarrow \omega \in \sigma
						\end{align}
						となり
						\begin{align}
							\exists s\, (\, \forall u\, (\, \xi(u) \lrarrow u \in s\, ) \wedge \varphi(s)\, ) 
							\provable{\mbox{{\bf HE}},\lang{\varepsilon}} \forall v\, (\, \xi(v) \lrarrow v \in \sigma\, )
						\end{align}
						が従う.ゆえに
						\begin{align}
							\begin{gathered}
								\forall u\, (\, \varphi(u) \lrarrow u \in b\, ),\ 
								\exists s\, (\, \forall u\, (\, \xi(u) \lrarrow u \in s\, ) \wedge \varphi(s)\, ) \\
								\provable{\mbox{{\bf HE}},\lang{\varepsilon}} \forall v\, (\, \xi(v) \lrarrow v \in \sigma\, ) \wedge \psi(\sigma)
							\end{gathered}
						\end{align}
						が従い
						\begin{align}
							\begin{gathered}
								\forall u\, (\, \varphi(u) \lrarrow u \in b\, ),\ 
								\exists s\, (\, \forall u\, (\, \xi(u) \lrarrow u \in s\, ) \wedge \varphi(s)\, ) \\
								\provable{\mbox{{\bf HE}},\lang{\varepsilon}} \exists t\, (\, \forall v\, (\, \xi(v) \lrarrow v \in t\, ) \wedge \psi(t)\, )
							\end{gathered}
						\end{align}
						が出る.
						\QED
				\end{description}
		\end{description}
	\end{metaprf}

\chapter{結論}
	本論文ではクラス(漢字で「類」とも書かれる)を扱うための{\bf ZF}集合論の一つの拡張を提示した.
	それは主要$\varepsilon$項を集合の基準系とし,内包的記法によって書かれた記号列を
	正式に項として採用することによって為されたし,拡張に合わせて集合論の公理や
	証明の規則を変形してもその証明力が{\bf ZF}集合論の保存拡大になっていることも示した.
	$\varepsilon$項の中でも主要$\varepsilon$項のみを抽出して他は切り捨てるという点が
	Hilbertの$\varepsilon$計算との大きな違いであるが,それは目的が違うから成し得たことであり,
	本論文ではあくまでも集合の実在化,さらに言えば量化の亘る範囲の具体化に必要な分だけを取ったのである.
	本論文の集合論では集合もクラスも全て記号列で書けるモノであるし,
	使う記号は論理記号,文字,及び$\varepsilon$や$\natural$など数え切れる程しかないので,
	序論でも述べた通り集合もクラスも可算個しかない.これでは様々な無限を扱う集合論には不釣合に思えるが,
	G$\ddot{\mbox{o}}$delの完全性定理によると{\bf ZF}集合論が無矛盾であれば
	可算集合のモデルが作れるので別におかしい話ではない.
	
	しかし乍ら,実際に集合論を組み立てていく中で,本論文の集合論には大きな難点があると判明した.
	証明が冗長になる点である.{\bf ZF}集合論の証明では変項がむき出しであるまま使えるところを,
	本論文では証明は全て文で行うことにしてしまったために,一々主要$\varepsilon$項を用意して
	その項の代用のための文字が増えてしまうし,また汎化の代わりに
	\begin{align}
		\varphi(\varepsilon x \negation \varphi(x)) \rarrow \forall x \varphi(x)
	\end{align}
	を用いるため,証明の初めには然るべき主要$\varepsilon$を一々宣言しなくてはならない.
	とはいえ若干冗長さを取り除く論法があって,たとえば$\forall x \varphi(x)$を示したいのなら,
	証明の始めに「$\tau$を$\varepsilon x \negation \varphi(x)$とする」と書くのではなく
	「任意の集合$\tau$に対して」と書けばよい.なぜなら
	$\tau$が集合であればそれは何らかの主要$\varepsilon$項に等しいわけだし,
	また任意の$\tau$で言えることは$\varepsilon x \negation \varphi(x)$に対しても言えるからである.
	同様に$\exists x \psi(x)$が導かれたら,「$\sigma$を$\varepsilon x \psi(x)$とおけば
	$\psi(\sigma)$が成り立つ」と書く代わりに「$\psi(\sigma)$を満たす集合$\sigma$が取れる」と
	書けばよい.
\chapter*{謝辞}
	本論文の内容は確率論とは程遠い集合論の基礎であるにもかかわらず,
	終始ご指導下さった指導教員の深澤正彰先生に厚く御礼申し上げます.
	また時おり研究室で近況をたずねて下さった永沼伸顕先生には精神面で支えられました.厚く御礼申し上げます.
\begin{thebibliography}{数字}
	\bibitem{key1} Moser, G. and Zach, R., ``The Epsilon Calculus and Herbrand Complexity'',
		Studia Logica 82, 133-155 (2006)
	
	\bibitem{key2} 高橋優太, ``1階述語論理に対する$\varepsilon$計算'', \\
		http://www2.kobe-u.ac.jp/~mkikuchi/ss2018files/takahashi1.pdf 
		
	\bibitem{key3} キューネン数学基礎論講義
	
	\bibitem{key5} ブルバキ, 数学原論 集合論 1, 
	
	\bibitem{key4} 竹内外史, 現代集合論入門, 増強版第5刷, 日本評論社, 2016, pp. 138-183, ISBN 978-4-535-60116-1
	
	\bibitem{key6} 島内剛一, 数学の基礎, 第1版第10刷, 日本評論社, 2016, ISBN 978-4-535-60106-2
	
	\bibitem{key7} 戸次大介, 数理論理学, 第2刷, 東京大学出版会, 2016, pp. 148-166, ISBN 978-4-13-062915-7
	
	\bibitem{key8} K. G$\ddot{\mbox{o}}$del, $The\ Consistency\ of\ the\ Continuum\ Hypothesis$, 8th printing, Princeton University Press 1970, p. 3, ISBN 0-691-07927-7.
	
	\bibitem{key9} 菊地誠, 不完全性定理, 初版3刷, 共立出版株式会社, 2017, pp. 86-91, ISBN 978-4-320-11096-0
	
	\bibitem{key10} 前原昭二, 記号論理入門, 新装版第8刷, 日本評論社, 2018, pp. 106-115, ISBN 4-535-60144-5
	
	\bibitem{key11} Kenji Miyamoto and Georg Moser, The Epsilon Calculus with Equality and Herbrand Complexity
\end{thebibliography}
\printindex
\end{document}
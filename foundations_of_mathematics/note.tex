\documentclass[a4j,10.5pt,oneside,openany]{jsbook}
%
\usepackage{amsmath,amssymb}
\usepackage{amsthm}
\usepackage{makeidx}
\makeindex
\usepackage{newpxmath,newpxtext}
\usepackage{mathrsfs} %花文字
\usepackage{mathtools} %参照式のみ式番号表示
\usepackage{latexsym} %qed
\usepackage{ascmac}
\usepackage{centernot} %\centernot\arrow
\usepackage[dvipdfmx]{graphicx}
\usepackage{tikz} %描画
\usepackage{color}
\usepackage{relsize}
\usepackage{comment}
\usepackage{url}
\usepackage{ulem} %訂正線
\usepackage[dvipdfm,colorlinks=true,linkcolor=blue,filecolor=blue,urlcolor=blue]{hyperref} %文書内リンク
\usepackage{pxjahyper} %%hyperref読み込みの直後に
\setcounter{tocdepth}{3} %table of contents subsection表示
\newtheoremstyle{mystyle}% % Name
	{20pt}%                      % Space above
	{20pt}%                      % Space below
	{\rm}%           % Body font
	{}%                      % Indent amount
	{\gt}%             % Theorem head font
	{.}%                      % Punctuation after theorem head
	{10pt}%                     % Space after theorem head, ' ', or \newline
	{}%                      % Theorem head spec (can be left empty, meaning `normal')
\theoremstyle{mystyle}

\allowdisplaybreaks[1]
\newcommand{\bhline}[1]{\noalign {\hrule height #1}} %表の罫線を太くする.
\newcommand{\bvline}[1]{\vrule width #1} %表の罫線を太くする.
\newcommand{\QED}{% %証明終了
	\relax\ifmmode
		\eqno{%
		\setlength{\fboxsep}{2pt}\setlength{\fboxrule}{0.3pt}
		\fcolorbox{black}{black}{\rule[2pt]{0pt}{1ex}}}
	\else
		\begingroup
		\setlength{\fboxsep}{2pt}\setlength{\fboxrule}{0.3pt}
		\hfill\fcolorbox{black}{black}{\rule[2pt]{0pt}{1ex}}
		\endgroup
	\fi}

\definecolor{DarkMidnightBlue}{rgb}{0.0, 0.2, 0.4}
\definecolor{PakistanGreen}{rgb}{0.0, 0.4, 0.0}
\definecolor{Mahogany}{rgb}{0.65,0.10,0.10}
\definecolor{darkgray}{rgb}{0.21, 0.21, 0.21}
\definecolor{CarolinaBlue}{rgb}{0.6, 0.73, 0.89}

\newtheorem{thm}{\color{DarkMidnightBlue}{定理}}[section]
\newtheorem{dfn}[thm]{\color{PakistanGreen}{定義}}
\newtheorem{axm}[thm]{\color{Mahogany}{公理}}
\newtheorem{schema}[thm]{{公理図式}}
\newtheorem{logicalaxm}[thm]{\color{Mahogany}{推論規則}}
\newtheorem{logicalthm}[thm]{\color{DarkMidnightBlue}{推論法則}}
\newtheorem{metaaxm}[thm]{\color{Mahogany}{メタ公理}}
\newtheorem{metathm}[thm]{\color{DarkMidnightBlue}{メタ定理}}
\newtheorem{prp}[thm]{命題}
\newtheorem{cor}[thm]{系}
\newtheorem{lem}[thm]{補題}
\newtheorem*{prf}{証明}
\newtheorem*{sketch}{略証}
\newtheorem{rem}[thm]{注意}
\newtheorem{e.g.}[thm]{例}
\newcommand{\defunc}{\mbox{1}\hspace{-0.25em}\mbox{l}} %定義関数
\newcommand*{\sgn}[1]{\operatorname{sgn}\left( #1 \right)} %signal関数
\newcommand{\monologue}[1]{
	{\color{CarolinaBlue}\hspace{-10.5pt}\mask{\hspace{21pt}\vbox{
		\hsize 445pt
		\normalcolor{\vskip 7pt \noindent #1 \vskip 7pt}
	}\hspace{21pt}}{E}}
}

\def\Ddot#1{$\ddot{\mathrm{#1}}$} %文中ddot

%集合論理
\newcommand{\Set}[2]{\left\{\, #1 \mid \quad #2\, \right\}} %論理式の対象化
\newcommand{\defeq}{\overset{\mathrm{def}}{=\joinrel=}} %\scalebox{3}[1]{=}}} %定義記号=(=\joinrel=も使える)
\newcommand{\defarrow}{\overset{\mathrm{def}}{\Longleftrightarrow}} %定義記号⇔
\newcommand{\Univ}{\mathbf{V}} %宇宙
\newcommand{\set}[1]{\operatorname*{set} (#1)} %集合であることの論理式
\newcommand{\power}[1]{\operatorname*{P} (#1)} %冪集合
\newcommand{\rel}[1]{\operatorname*{rel} (#1)} %関係
\newcommand{\dom}[1]{\operatorname*{dom} (#1)} %類の定義域
\newcommand{\ran}[1]{\operatorname*{ran} (#1)} %類の値域
\newcommand{\sing}[1]{\operatorname*{sing} (#1)} %single-valuedの定義式
\newcommand{\fnc}[1]{\operatorname*{fnc} (#1)} %写像の定義式
\newcommand{\fon}{\operatorname*{:on}} %〇上の写像
\newcommand{\inj}{\overset{\mathrm{1:1}}{\longrightarrow}} %単射
\newcommand{\srj}{\overset{\mathrm{onto}}{\longrightarrow}} %全射
\newcommand{\bij}{\underset{\mathrm{onto}}{\overset{\mathrm{1:1}}{\longrightarrow}}} %全単射
\newcommand{\inv}[1]{{#1}^{-1}} %^{\operatorname{inv}}} %集合の反転
\newcommand{\rest}[2]{#1\hspace{-0.41em}\upharpoonright_{#2}} %制限写像
\newcommand{\tran}[1]{\operatorname*{tran} \left(#1\right)} %推移的類の定義式
\newcommand{\ord}[1]{\operatorname*{ord} \left(#1\right)} %順序数の定義式
\newcommand{\ON}{\mathrm{ON}} %順序数全体
\newcommand{\limo}[1]{\mathrm{lim.o}\left(#1\right)} %極限数の式
%\newcommand{\Natural}{{\boldsymbol \omega}} %自然数全体
\newcommand{\Natural}{\mathbf{N}} %自然数全体
%
%
\setlength{\textwidth}{\fullwidth}
\setlength{\textheight}{40\baselineskip}
\addtolength{\textheight}{\topskip}
%\setlength{\voffset}{-0.55in}
%
%
\begin{document}
\mathtoolsset{showonlyrefs = true}

\section{徒然なるままに支離滅裂}
わからないわからないわからない

基礎論における証明は大抵が直感に頼っているように見えますが,ではその直感が正しいとは誰が保証するのでしょうか.
手元にあるどの本でも保証されていません.もしかしたら神様という超然的な存在を暗黙の裡に認めていて,
直感とは神様が用意した論理であるとして無断で使っているだけなのかもしれませんが,
残念ながら読者はテレパシーを使えないので,筆者の暗黙の了解を推察するなんて困難です.

しかしながら,暗黙の了解を排除しようとすると,その分だけ日本語による明示的な約束が必要になります.
すると新たな問題が生じます.それは日本語で書かれた言明をどこまで信用するか,という問題です.
基礎論の難しさは,その表面上のややこしさよりも日本語に対する認識を揃えることにあるのでしょうか.

論理構造を集合論の結果を用いて解明しようというのならまだしも(こちらは数理論理学と呼ばれる分野で,本来は数学基礎論とは別物だそうです),
集合論を構築することが目的である場合,その土台となる基礎論を集合論の上に展開すると理論が循環することになるでしょう.
基礎論が基礎にしている集合論は「メタ理論」と呼ばれるらしいですが,
その「メタ理論」がどう構成されたのかという点には誰も全く言及していないのですから,
「メタ理論」という言葉は単なる逃げ口上にしか聞こえず,理論の循環を解消できません.
私の考えでは,メタ理論の代わりに絶対的な原理が与えられたとして数学を構築すれば良いのです.
まあ言い方を変えて印象を良く?しようというだけの下らない事情であって,
もったいぶって思想的な立場を主張しても集合論には関係のないことなのですが.

前提:我々は数の概念を持っている.個数の概念を持っている.物の数を数えることが出来る.
数の概念とは?個数の概念とは?
ここで言う数は数学的に構成する数ではなくて,神が用意した概念としての数.
そこまで踏み込むときりがない.

\section{言語}
	本稿の世界を展開するために使用する言語は二つある.
	一つは自然言語の日本語であり,もう一つは記号のみで作られた人工的な言語である.
	その人工的な言語は記号列が数学の式となるための文法を指定し,
	そこで組み立てられた式のみが考察対象となる.
	日本語は式を解釈したり人工言語を補助するために使われる.
	
	まず,人工的な言語である$\mathcal{L}_{0}$を設定する.
	以下は$\mathcal{L}_{0}$を構成する要素である:
	\begin{description}
		\item[矛盾記号] $\bot$
		\item[論理記号] $\rightharpoondown$, $\vee$, $\wedge$, $\rightarrow$, $\forall$, $\exists$
		\item[述語記号] $=$, $\in$
		\item[使用文字] ローマ字及びギリシア文字.
		\item[接項記号] $\natural$
	\end{description}
	
	日本語と同様に,決められた規則に従って並ぶ記号列のみを$\mathcal{L}_{0}$の単語や文章として扱う.
	$\mathcal{L}_{0}$において,名詞にあたるものは{\bf 項}\index{こう@項}{\bf (term)}と呼ばれる.
	文字は最もよく使われる項である.述語とは項同士を結ぶものであり,最小単位の文章を形成する.例えば
	\begin{align}
		\in st
	\end{align}
	は$\mathcal{L}_{0}$の文章となり,日本語には``$s$は$t$の要素である''と翻訳される.
	$\mathcal{L}_{0}$の文章を{\bf 式}\index{しき@式}{\bf (formula)}或いは
	{\bf 論理式}\index{ろんりしき@論理式}と呼ぶ.論理記号は主に式同士を繋ぐ役割を持つ.
	
\section{項}
	
	文字は項として使われるが,文字だけを項とするのは不十分であり,
	例えば$1000$個の相異なる項が必要であるといった場合には異体字まで駆使しても不足する.
	そこで,文字$x$と$y$に対して
	\begin{align}
		\natural xy
	\end{align}
	もまた項であると約束する.これは
	\begin{align}
		\natural(x,y)
	\end{align}
	や
	\begin{align}
		x \natural y
	\end{align}
	などと書く方が見やすいかもしれないが,当面は始めの記法(前置記法やポーランド記法と呼ばれる)を用いる.
	また,$\tau$と$\sigma$を項とするときに
	\begin{align}
		\natural \tau \sigma
	\end{align}
	も項であると約束する.この約束に従えば,文字$x$だけを用いたとしても
	\begin{align}
		x,\quad \natural xx, \quad \natural \natural xxx, \quad \natural \natural \natural xxxx
	\end{align}
	はいずれも項ということになる.極端なことを言えば,「$1000$個の項を用意してくれ」と頼まれたとしても
	$\natural$と$x$のみを使って$1000$個の項を作り出すことが出来るのだ.
	
	大切なのは,$\natural$を用いれば理屈の上では項に不足しないということであって,
	具体的な数式を扱うときに$\natural$が出てくるかと言えば否である.
	$\natural$が必要になるほどに長い式を読解するのは困難であるから,
	通常は何らかの略記法を導入して複雑なところを覆い隠してしまう.
	
	\begin{itembox}[l]{超記号}
		上で「$\tau$と$\sigma$を項とするときに」と書いたが,これは一時的に
		$\tau$と$\sigma$をそれぞれ或る項に代用しているだけであって,
		この場合の$\tau$や$\sigma$を{\bf 超記号}\index{ちょうきごう@超記号}と呼ぶ.
		要は便宜上の代用記号であって,「$A$を式とする」など式にも超記号が宣言される.
	\end{itembox}
	
	項は形式的には次のよう定義される:
	
	\begin{description}
		\item[項]
			\begin{itemize}
				\item 文字は項である.
				\item $\tau$と$\sigma$を項とするとき,$\natural \tau \sigma$は項である.
				\item 以上のみが項である.
			\end{itemize}
	\end{description}
	
	上の定義では,はじめに発端を決めて,次に新しい項を作り出す手段を指定している.こういった定義の仕方を
	{\bf 帰納的定義}\index{きのうてきていぎ@帰納的定義}{\bf (inductive definition)}と呼ぶ.
	ただしそれだけでは項の範囲が定まらないので,最後に「以上のみが項である」と加えている.
	
	「以上のみが項である」という約束によって,例えば「$\tau$が項である」という言明が与えられたとき,この言明が
	``$\tau$は或る文字に代用されている''か
	``項$x$と項$y$が取れて(いずれも超記号),$\tau$は$\natural xy$に代用されている''
	のどちらか一方にしか解釈され得ないのは,言うまでもない,であろうか.直感的にはそうであっても
	直感を万人が共有している保証はないから,やはりここは明示的に,「$\tau$が項である」という言明の解釈は
	\begin{itemize}
		\item $\tau$は或る文字に代用されている
		\item 項$x$と項$y$が取れて(いずれも超記号),$\tau$は$\natural xy$に代用されている
	\end{itemize}
	に限られると決めてしまおう.こちらの方が誤解を生まない.
	
	\begin{itembox}[l]{暗に宣言された超記号}
		上で「項$x$と項$y$が取れて」と書いたが,この$x$と$y$は唐突に出てきたので,
		それが表す文字そのものでしかないのか,或いは超記号であるのか,一見判然しない.
		本来は「二つの項(これをそれぞれ$x$と$y$で表す)が取れて」などと書くのが
		良いのかもしれないが,はじめの書き方でも文脈上は超記号として解釈するのが自然であるし,
		何より言い方がまどろこくない.このように見た目の簡潔さのために超記号の宣言を省略する場合もある.
	\end{itembox}
	
\section{式}
	式も項と同様に帰納的に定義される:
	
	\begin{description}
		\item[式]
			\begin{itemize}
				\item $\bot$は式である.
				\item $\sigma$と$\tau$を項とするとき,$\in st$と$=st$は式である.
					これを{\bf 原子式}\index{げんししき@原子式}{\bf (atomic formula)}と呼ぶ.
				\item $\varphi$を式とするとき,$\rightharpoondown \varphi$は式である.
				\item $\varphi$と$\psi$を式とするとき,$\vee \varphi \psi,\ 
					\wedge \varphi \psi,\ \Longrightarrow \varphi \psi$はいずれも式である.
			
				\item $x$を項とし,$\varphi$を式とするとき,$\forall x \varphi$と$\exists x \varphi$は式である.
				
				\item 以上のみが式である.
			\end{itemize}
	\end{description}
	
	例えば「$\varphi$が式である」という言明の解釈は,
	\begin{itemize}
		\item $\varphi$は$\bot$である
		\item 項$s$と項$t$が得られて,$\varphi$は$\in s t$である
		\item 項$s$と項$t$が得られて,$\varphi$は$= s t$である
		\item 式$\psi$が得られて,$\varphi$は$\rightharpoondown \psi$である
		\item 式$\psi$と式$\xi$が得られて,$\varphi$は$\vee \psi \xi$である
		\item 式$\psi$と式$\xi$が得られて,$\varphi$は$\wedge \psi \xi$である
		\item 式$\psi$と式$\xi$が得られて,$\varphi$は$\Longrightarrow \psi \xi$である
		\item 項$x$と式$\psi$が得られて,$\varphi$は$\forall x \psi$である
		\item 項$x$と式$\psi$が得られて,$\varphi$は$\exists x \psi$である
	\end{itemize}
	に限られる.
	
\section{部分式}
	式から切り取ったひとつづきの部分列で,それ自身が式であるものを元の式に対して
	{\bf 部分式}\index{ぶぶんしき@部分式}{\bf (sub formula)}と呼ぶ.
	例えば$\varphi$と$\psi$を式とするとき,$\varphi$と$\psi$は$\vee \varphi \psi$の部分式である.
	元の式全体も部分式と捉えることにするが,自分自身を除く部分式を特に
	{\bf 真部分式}\index{しんぶぶんしき@真部分式}{\bf (proper sub formula)}と呼ぶことにする.
	
\section{始切片}
	$\varphi$を式とするとき,$\varphi$の左端から切り取るひとつづきの部分列を
	$\varphi$の{\bf 始切片}\index{しせっぺん@始切片}{\bf (initial segment)}と呼ぶ.
	例えば$\varphi$が
	\begin{align}
		\Longrightarrow \forall x \wedge \Longrightarrow \in xy \in xz \Longrightarrow \in xz \in xy = yz
	\end{align}
	である場合,
	\begin{align}
		\textcolor{red}{\Longrightarrow \forall x \wedge \Longrightarrow \in xy \in xz \Longrightarrow \in xz \in x}y = yz
	\end{align}
	や
	\begin{align}
		\textcolor{red}{\Longrightarrow \forall x \wedge \Longrightarrow \in xy} \in xz \Longrightarrow \in xz \in xy = yz
	\end{align}
	など赤字で分けられた部分は$\varphi$の始切片である.また$\varphi$自身も$\varphi$の始切片である.
	
	本節の主題は次である.
	\begin{screen}
		\begin{metathm}
			(★) $\varphi$を式とするとき,$\varphi$の始切片で式であるものは$\varphi$自身に限られる.
		\end{metathm}
	\end{screen}
	
	これを示すには次の原理を用いる:
	\begin{screen}
		\begin{metaaxm}[式に対する構造的帰納法]
			式に対する言明に対し,
			\begin{itemize}
				\item $\bot$に対してその言明が当てはまる.
				\item 原子式に対してその言明が当てはまる.
				\item 式が任意に与えられた\footnotemark
					ときに,その全ての真部分式に対して
					その言明が当てはまるならば,その式自身に対してもその言明が当てはまる.
			\end{itemize}
			ならば,いかなる式に対してもその言明は当てはまる.
		\end{metaaxm}
	\end{screen}
	
	\footnotetext{
		``任意に与えられた式''とはどう解釈するべきか.
		どんな式に対しても?
	}
	
	では定理を示す.
	$\bot$については,その始切片は$\bot$に限られる.
	$\in st$なる原子式については,その始切片は
	\begin{align}
		\in, \quad \in s, \quad \in st
	\end{align}
	のいずれかとなるが,このうち式であるものは$\in st$のみである.
	$=st$なる原子式についても,その始切片で式であるものは$=st$に限られる.
	
	いま$\varphi$を任意に与えられた式とし,
	$\varphi$の真部分式に対しては(★)が当てはまっているとする.
	\begin{description}
		\item[ケース1] 式$\psi$が得られて$\varphi$が
			\begin{align}
				\rightharpoondown \psi
			\end{align}
			であるとき,$\psi$は$\varphi$の真部分式であるので(★)は当てはまる.
			$\varphi$の始切片で式であるものは,
			式$\xi$を用いて$\rightharpoondown \xi$と表せるが,
			$\xi$は$\psi$の始切片であるから,帰納法の仮定より$\xi$と$\psi$は一致する.
			ゆえに$\varphi$の始切片で式であるものは$\varphi$自身に限られる.
			
		\item[ケース2] 式$\psi$と$\xi$が得られて$\varphi$が
			\begin{align}
				\vee \psi \xi
			\end{align}
			であるとする.$\varphi$の始切片で式であるものも$\vee$が左端に来るので,
			式$\eta$と式$\zeta$が得られて始切片は
			\begin{align}
				\vee \eta \zeta
			\end{align}
			と表せる.$\psi$と$\eta$,$\xi$と$\zeta$は
			いずれも$\varphi$の真部分式であるので(★)が当てはまる.
			そして$\psi$と$\eta$は一方が他方の始切片であるので,(★)より一致する.
			すると$\xi$と$\zeta$も一方が他方の始切片ということになり,(★)より一致する.
			ゆえに$\vee \psi \xi$と$\vee \eta \zeta$は一致する.
			つまり$\varphi$の始切片で式であるものは$\varphi$自身に限られる.
			$\varphi$が$\wedge \psi \xi$や$\Longrightarrow \psi \xi$である場合も同じである.
			
		\item[ケース3] 項$x$と式$\psi$が得られて,$\varphi$が
			\begin{align}
				\forall x \psi
			\end{align}
			であるとき,$\varphi$の始切片で式であるものは,式$\xi$が取れて
			\begin{align}
				\forall x \xi
			\end{align}
			と表せる.このとき$\xi$は$\psi$の始切片であるし,
			また$\psi$は$\varphi$の真部分式であるから,(★)より$\psi$と$\xi$は一致する.
			ゆえに$\varphi$の始切片で式であるものは$\varphi$自身に限られる.
			$\varphi$が$\forall x \psi$である場合も同じである.
			\QED
	\end{description}

\section{スコープ}
	$\varphi$を式とし,$s$を$\varphi$に現れた記号とするとき,$s$のその出現位置から始まる$\varphi$の部分式を
	$s$の{\bf スコープ}\index{スコープ}{\bf (scope)}と呼ぶ.具体的に,$\varphi$を
	\begin{align}
		\Longrightarrow \forall x \wedge \Longrightarrow \in xy \in xz \Longrightarrow \in xz \in xy = yz
	\end{align}
	としよう.このとき$\varphi$の左から$6$番目に$\in$が現れるが,この$\in$から
	\begin{align}
		\in xy
	\end{align}
	なる原子式が$\varphi$の上に現れている:
	\begin{align}
		\Longrightarrow \forall x \wedge \Longrightarrow \textcolor{red}{\in xy} \in xz \Longrightarrow \in xz \in xy = yz.
	\end{align}
	他にも,$\varphi$の左から$4$番目に$\wedge$が現れるが,この右側に
	\begin{align}
		\Longrightarrow \in xy \in xz
	\end{align}
	と
	\begin{align}
		\Longrightarrow \in xz \in xy
	\end{align}
	の二つの式が続いていて,$\wedge$を起点に
	\begin{align}
		\wedge \Longrightarrow \in xy \in xz \Longrightarrow \in xz \in xy
	\end{align}
	なる式が$\varphi$の上に現れている:
	\begin{align}
		\Longrightarrow \forall x \textcolor{red}{\wedge \Longrightarrow \in xy \in xz \Longrightarrow \in xz \in xy} = yz.
	\end{align}
	$\varphi$の左から$2$番目には$\forall$が現れて,
	この$\forall$に対して項$x$と
	\begin{align}
		\wedge \Longrightarrow \in xy \in xz \Longrightarrow \in xz \in xy
	\end{align}
	なる式が続き,
	\begin{align}
		\forall x \wedge \Longrightarrow \in xy \in xz \Longrightarrow \in xz \in xy
	\end{align}
	なる式が$\varphi$の上に現れている:
	\begin{align}
		\Longrightarrow \textcolor{red}{\forall x \wedge \Longrightarrow \in xy \in xz \Longrightarrow \in xz \in xy} = yz.
	\end{align}
	
	しかも$\in,\wedge,\forall$のスコープは上にあげた部分式のほかに取りようが無い.
	上の具体例を見れば,直感的に「現れた記号のスコープはただ一つだけ,必ず取ることが出来る」
	が一般の式に対して当てはまるであるように思えるが,直感を排除してこれを認めるには構造的帰納法の原理が必要になる.
	
	\begin{screen}
		\begin{metathm}
		(★★) $\varphi$を式とするとき,
		\begin{itemize}
			\item $\natural$が$\varphi$に現れたとき,項$\sigma$と項$\tau$が得られて,
				$\natural$のその出現位置から$\natural \sigma \tau$なる項が$\varphi$の上に現れる.
				また$\natural$のその出現位置から始まる$\varphi$上の項は$\natural \sigma \tau$に限られる.
				
			\item $\in$が$\varphi$に現れたとき,項$\sigma$と項$\tau$が得られて,
				$\in$のその出現位置から$\in \sigma \tau$なる式が$\varphi$の上に現れる.
				また$\in$のその出現位置から始まる$\varphi$の部分式は$\in \sigma \tau$に限られる.
				$\in$が$=$であっても同じ主張が成り立つ.
				
			\item $\rightharpoondown$が$\varphi$に現れたとき,式$\psi$が得られて,
				$\rightharpoondown$のその出現位置から$\rightharpoondown \psi$なる式が
				$\varphi$の上に現れる.
				また$\rightharpoondown$のその出現位置から始まる$\varphi$の部分式は$\rightharpoondown \psi$に限られる.
				
			\item $\vee$が$\varphi$に現れたとき,式$\psi$と式$\xi$が得られて,
				$\vee$のその出現位置から$\vee \psi \xi$なる式が$\varphi$の上に現れる.
				また$\vee$のその出現位置から始まる$\varphi$の部分式は$\vee \psi \xi$に限られる.
				$\vee$が$\wedge$や$\Longrightarrow$であっても同じ主張が成り立つ.
				
			\item $\exists$が$\varphi$に現れたとき,項$x$と式$\psi$が得られて,
				$\exists$のその出現位置から$\exists x \psi$なる式が$\varphi$の上に現れる.
				また$\exists$のその出現位置から始まる$\varphi$の部分式は$\exists x \psi$に限られる.
				$\exists$が$\forall$であっても同じ主張が成り立つ.
		\end{itemize}
		\end{metathm}
	\end{screen}
	
	$\bot$に対しては上の言明は当てはまる.
	
	$\in \tau \sigma$なる式に対しては,$\in$のスコープは$\in \tau \sigma$に他ならない.
	$= \tau \sigma$なる式についても,$=$のスコープは$= \tau \sigma$に他ならない.
	
	$\varphi$を任意に与えられた式とし,$\varphi$の真部分式に対しては
	(★★)が当てはまっているとする.
	
	\begin{description}
		\item[ケース1] 
	\end{description}
	式$\varphi$と$\psi$に対して上の言明が当てはまるとする.
	式$\rightharpoondown \varphi$に対して,
	$\sigma$が左端の$\rightharpoondown$であるとき
	$\sigma \varphi$は$\rightharpoondown \varphi$の部分式である.
	また$\sigma \psi$が$\sigma$のその出現位置から始まる$\rightharpoondown \varphi$の部分式
	であるとすると,
	$\psi$は$\varphi$の左端から始まる$\varphi$の部分式ということになるので
	帰納法の仮定より$\varphi$と$\psi$は一致する.
	$\sigma$が$\varphi$に現れる記号であれば,帰納法の仮定より
	$\sigma$から始まる$\varphi$の部分式が一意的に得られる.
	その部分式は$\rightharpoondown \varphi$の部分式でもあるし,
	$\rightharpoondown \varphi$の部分式としての一意性は帰納法の仮定より従う.
	
	式$\vee \varphi \psi$に対して,
	$\sigma$が左端の$\vee$であるとき,式$\xi$と$\eta$が得られて$\sigma \xi \eta$が
	$\vee \varphi \psi$の部分式となったとすると,
	$\xi$と$\varphi$は左端を同じくし,どちらか一方は他方の部分式である.
	$\xi$が$\varphi$の部分式であるならば,帰納法の仮定より$\xi$と$\varphi$は一致する.
	$\varphi$が$\xi$の部分式であるならば,$\xi$と$\psi$が重なるとなると
	$\psi$の左端の記号から始まる$\xi$の部分式と$\psi$は一致しなくてはならない.
	
	\begin{itembox}[l]{量化}
		$\varphi$に$\forall$が現れるとき,
		その$\forall$に後続する項$x$が取れるが,このとき項$x$は$\forall$のスコープ内で
		{\bf 量化されている}\index{りょうか@量化}{\bf(quantified)}という.
		詳しく言い直せば,項$x$と式$\psi$が取れて,その$\forall$のスコープは
		\begin{align}
			\forall x \psi
		\end{align}
		なる式で表されるが,このとき$x$は$\forall x \psi$において量化されているという.
	\end{itembox}

\section{言語の拡張}
	式の対象化を図るため,言語$\mathcal{L}_{0}$を言語$\mathcal{L}$に拡張する.拡張する際に
	\begin{align}
		\{, \quad |, \quad \}
	\end{align}
	の記号を新しく導入する.
	
	ここで$\mathcal{L}_{0}$の項を変項と呼ぶことにしよう.
	つまり変項とは,文字と$\natural$によって構成された記号列である.
	
	\begin{description}
		\item[項] 変項は項である.また$A$を$\mathcal{L}_{0}$の式とし,変項$x$が$A$に現れて,
			かつ$x$のみが$A$で自由であるとき,$\{x|A\}$と$\varepsilon x A$は項である.
			変項に対して,新しく導入された項のことを閉項と呼ぶことにしよう.
			
		\item[式] 
			\begin{itemize}
				\item $\bot$は式である.
				\item 項$s$と$t$に対して$\in st$と$=st$は式である.これを原子式と呼ぶ.
				\item $\varphi$を式とするとき$\rightharpoondown \varphi$は式である.
				\item $\varphi$と$\psi$を式とするとき,以下はいずれも式である.
					\begin{align}
						&\vee \varphi \psi, \\
						&\wedge \varphi \psi, \\
						&\rightarrow \varphi \psi.
					\end{align}
			
				\item 変項$x$が式$\varphi$に現れるとき,$\forall x \varphi$と$\exists x \varphi$は式である.
			
				\item 以上のみが式である.
		\end{itemize}
	\end{description}
	
	$\varphi$を$\mathcal{L}$の式とし,$s$を$\varphi$に現れる記号とすると,
	\begin{description}
		\item[(1)] $s$は文字である.
		\item[(2)] $s$は$\natural$である.
		\item[(2)] $s$は$\{$である.
		\item[(3)] $s$は$|$である.
		\item[(4)] $s$は$\}$である.
		\item[(5)] $s$は$\bot$である.
		\item[(6)] $s$は$\in$か$=$である.
		\item[(7)] $s$は$\rightharpoondown$である.
		\item[(8)] $s$は$\vee,\wedge,\rightarrow$のいずれかである.
	\end{description}
	
	\begin{screen}
		(★★) いま,$\varphi$を任意に与えられた式としよう.
		\begin{itemize}
			\item $\natural$が$\varphi$に現れたとき,$\mathcal{L}_{0}$の項$\tau$と$\sigma$が得られて,$\natural \tau \sigma$は
				$\natural$のその出現位置から始まる$\mathcal{L}_{0}$の項となる.
				また$\natural$のその出現位置から始まる$\mathcal{L}_{0}$の項は$\natural \tau \sigma$のみである.
				
			\item $\{$が$\varphi$に現れたとき,$\mathcal{L}_{0}$の変項$x$及び$\mathcal{L}_{0}$の式$A$が得られて,
				$\{ x|A\}$は$\{$のその出現位置から始まる項となる.
				また$\{$のその出現位置から始まる項は$\{x|A\}$のみである.
				
			\item $|$が$\varphi$に現れたとき,,変項$x$と$\mathcal{L}_{0}$の式$A$が得られて,
				$\{x|A\}$は$|$のその出現位置から広がる項となる.
				また$|$のその出現位置から広がる項は$\{x|A\}$のみである.
				
			\item $\}$が$\varphi$に現れたとき,変項$x$と式$A$が得られて,
				$\{x|A\}$は$\}$のその出現位置を終点とする項となる.
				また$\}$のその出現位置を終点とする項は$\{x|A\}$のみである.
		\end{itemize}
	\end{screen}
	
	\begin{description}
		\item[$\natural$に対して$\natural \tau \sigma$なる変項$\tau$と$\sigma$が得られること]
			$\natural$が原子項に現れたら,原子項とは文字$x,y$によって
			\begin{align}
				\natural xy
			\end{align}
			と表されるものであるから,$\natural$に対して変項$\tau,\sigma$ (すなわち文字$x,y$)が取れたことになる.
			$\natural$が項に現れたとする.項とは,変項$x,y$によって
			\begin{align}
				\natural xy
			\end{align}
			で表されるものであり,$\natural$は左端の$\natural$であるか,$x$に現れるか,$y$に現れる.
			$\natural$が$x$か$y$に現れるときは帰納法の仮定により,
			$\natural$が左端のものである場合は$x$が$\tau$,$y$が$\sigma$ということになる.
			
		\item[変項の始切片で変項であるものは自分自身のみ]
			$x$が文字である場合はそう.$x$の任意の部分変項が言明を満たしているなら,
			$x$は$\natural st$なる変項である(生成規則)から,$x$の始切片は$\natural uv$なる変項である.
			$s,t,u,v$はいずれも$x$の部分変項なので仮定が適用されている.
			ゆえに$s$と$u$は一方が他方の始切片であり,一致する.すなわち$t$と$v$も一方が他方の始切片であり一致する.
			ゆえに$x$の始切片で変項であるものは$x$自身である.
			
		\item[$\natural$に対して得られる変項の一意性]
			$\natural xy$と$\natural st$が共に変項であるとき,$x$と$s$,$y$と$t$は一致するか.
			$\natural xy$が原子項であるときは明らかである.
			$x$の始切片で変項であるものは$x$自身に限られるので,
			$x$と$s$は一致する.ゆえに$t$は$y$の始切片であり,$t$と$y$も一致する.
		
		\item[生成規則より$x$と$A$が得られるか]
			$\varphi$が原子式であるとき,
			$\{$が現れるとすれば項の中である.項とは$\mathcal{L}_{0}$の項であるか$\{x|A\}$なるものであるので
			$\{$が現れたならば$\{$とは$\{x|A\}$の$\{$である.
			
			$\varphi$の任意の部分式に対して言明が満たされているとする.
			$\varphi$とは$\rightharpoondown \psi,\vee \psi \xi,...$の形であるから,
			$\varphi$に現れた$\{$とは$\psi$や$\xi$に現れるのである.ゆえに
			仮定より$x$と$A$が取れるわけである.
			
		\item[$\{$に対して]
			項の生成規則より$x$と$A$が得られる.
			$\{y|B\}$もまた$\{$から始まる項である場合,順番に見ていって
			$x$と$y$は一方が他方の始切片という関係になるから一致する.
			すると$A$と$B$は一方が他方の始切片という関係になり,(★)より$A$と$B$は一致する.
			
		\item[$|$について]
			項の生成規則より$x$と$A$が得られる.
			$\{y|B\}$もまた$|$から広がる項である場合,順番に見ていって
			$x$にも$y$にも$\{$という記号は現れないので$x$と$y$は一致する.
			$A$と$B$は一方が他方の始切片という関係になるので(★)より$A$と$B$は一致する.
			
		\item[$\}$について]
			項の生成規則より$x$と$A$が得られる.
			$\{y|B\}$もまた$\}$のその出現位置を終点とする変項である場合,
			$A$と$B$は$\mathcal{L}_{0}$の式なので$|$という記号は現れない.ゆえに
			$A$と$B$は一致する.すると$x$と$y$は右端で揃うが,
			$x$にも$y$にも$\{$という記号は現れないので$x$と$y$は一致する.
	\end{description}

\section{証明}
	閉式には,「真」であるか,「偽」であるか,のどちらかのラベルが付けられる.
	「真である」という言明は,「正しい」や「成り立つ」などとも言い換えられる.
	式が真であるか偽であるかは,次の手順に従って発見的に判明していく.
	
	\begin{itemize}
		\item $\Sigma$の閉式は真である.
		\item $A$と$\rightarrow AB$が真であると判明しているならば,$B$は真である.
		\item $\rightarrow \wedge ABA$と$\rightarrow \wedge ABB$は真である.
		\item $A$と$B$が真であると判明しているならば$\wedge AB$と$\wedge BA$は真である.
		\item $\rightarrow A\vee AB$と$\rightarrow B \vee AB$は真である.
		\item $\rightarrow AC$と$\rightarrow BC$が真であると判明しているならば
			$\rightarrow \vee ABC$は真である.
		\item $\rightarrow\wedge A \rightharpoondown A \bot$は真である.
		\item $\rightarrow \rightarrow A \bot \rightharpoondown A$は真である.
		\item $\rightarrow \rightharpoondown\rightharpoondown AA$は真である.
	\end{itemize}
	
	真であると判明している式$\varphi$を起点にして,
	上の推論規則を駆使して閉式$\psi$が真であると判明すれば,
	$\varphi$から始めて$\psi$が真であることに辿り着くまでの手続きは$\psi$の証明と呼ばれ,
	$\psi$は定理と呼ばれる.
	
	証明には真であると判明している式が必要であり,その大元として選ばれた式が$\Sigma$の式である.
	$\Sigma$の式は証明なしに真であると決められているのであり,これらを公理と呼び定理と区別する.
	
	与えられた閉式$\varphi$が証明可能であるとは,
	\begin{itemize}
		\item 閉式$\psi$で,$\psi$と$\psi \rightarrow \varphi$が真であると判明している者が得られる.
		\item 真であると判明している閉式$\psi$と$\xi$が得られて,$\varphi$は$\psi \wedge \xi$である.
		\item 閉式$\psi$と$\xi$で,$\psi \vee \xi$と$\psi \rightarrow \varphi$と$\xi \rightarrow \varphi$が真であると判明しているものが得られる.
	\end{itemize}
	
	のいずれかの場合であり,
	\begin{align}
		\vdash \varphi
	\end{align}
	と書く.
	
	証明された式が真なる式である.では真なる式は

\section{類と集合}
	\begin{screen}
		\begin{dfn}[類と集合]
			$\mathcal{L}$の閉項を{\bf 類}\index{るい@類}{\bf (class)}と呼ぶ.また
			\begin{align}
				\exists x\, (\, x = a\, )
			\end{align}
			を満たす類$a$を{\bf 集合}\index{しゅうごう@集合}{\bf (set)}と呼び,
			\begin{align}
				\rightharpoondown \exists x\, (\, x = a\, )
			\end{align}
			を満たす類$a$を{\bf 真類}\index{しんるい@真類}{\bf (proper class)}と呼ぶ.
		\end{dfn}
	\end{screen}
	
	ちなみに$\varepsilon x A(x)$は集合である.なぜならば
	\begin{align}
		\varepsilon x A(x) = \varepsilon x A(x)
	\end{align}
	だから
	\begin{align}
		\exists a\, \left(\, a = \varepsilon x A(x)\, \right).
	\end{align}
	
\section{中置記法}
	
	
\section{式の書き換え}
	\begin{itemize}
		\item $x \in y$はそのまま$x \in y$
		\item $x \in \{y|B(y)\}$は$B(x)$
			
			これは公理である.つまり,
			\begin{align}
				\forall x\, \left(\, x \in \{y|B(y)\} \leftrightarrow B(x)\, \right).
			\end{align}
			
		\item $x \in \varepsilon y B(y)$は$\exists t\, \left(\, x \in t \wedge B(t)\, \right)$.ちなみにこれは公理とするべきか:
			\begin{align}
				\forall x\, \left(\, x \in \varepsilon y B(y) \leftrightarrow
				\exists t\, \left(\, x \in t \wedge B(t)\, \right)\, \right).
			\end{align}
			
		\item $\{x|A(x)\} \in y$は$\exists s\, \left(\, s \in y \wedge 
			\forall u\, \left(\, u \in s \leftrightarrow A(u)\, \right)\, \right)$
			
			実はこの両式は同値である.さていま
			\begin{align}
				\{x|A(x)\} \in y \leftrightarrow
				\exists s\, \left(\, s \in y \wedge 
				\forall u\, \left(\, u \in s \leftrightarrow A(u)\, \right)\, \right)
			\end{align}
			という式を$\varphi$とし,これを$\mathcal{L}_{0}$の式に書き換えたものを$\hat{\varphi}$としよう.そして
			\begin{align}
				\eta = \varepsilon y \rightharpoondown \hat{\varphi}(y)
			\end{align}
			とおこう.ここで証明するのは
			\begin{align}
				\{x|A(x)\} \in \eta \leftrightarrow
				\exists s\, \left(\, s \in \eta \wedge 
				\forall u\, \left(\, u \in s \leftrightarrow A(u)\, \right)\, \right)
			\end{align}
			が成り立つということである.まず
			\begin{align}
				\{x|A(x)\} \in \eta
			\end{align}
			が成り立っているとしよう.すると
			\begin{align}
				\exists s\, \left(\, \{x|A(x)\} = s\, \right)
			\end{align}
			が成り立つのだが,今度も式の書き直し手順によって
			\begin{align}
				\exists s\, \left(\, \forall u\, \left(\, A(u) \leftrightarrow
				u \in s\, \right)\, \right)
			\end{align}
			と書き直される.
			\begin{align}
				\sigma = \varepsilon s\, \left(\, \forall u\, \left(\, A(u) \leftrightarrow
				u \in s\, \right)\, \right)
			\end{align}
			とおくと
			\begin{align}
				\forall u\, \left(\, A(u) \leftrightarrow
				u \in \sigma\, \right)
			\end{align}
			が成り立ち,他方で
			\begin{align}
				\sigma = \{x|A(x)\}
			\end{align}
			が成り立つのだから
			\begin{align}
				\sigma \in \eta
			\end{align}
			も従う.ゆえに
			\begin{align}
				\sigma \in \eta \wedge \forall u\, \left(\, A(u) \leftrightarrow
				u \in \sigma\, \right)
			\end{align}
			が成り立つ.逆に
			\begin{align}
				\exists s\, \left(\, s \in \eta \wedge 
				\forall u\, \left(\, u \in s \leftrightarrow A(u)\, \right)\, \right)
			\end{align}
			が成り立っているとして,
			\begin{align}
				\sigma = \varepsilon s\, \left(\, s \in \eta \wedge 
				\forall u\, \left(\, u \in s \leftrightarrow A(u)\, \right)\, \right)
			\end{align}
			としよう.すると
			\begin{align}
				\sigma \in \eta \wedge 
				\forall u\, \left(\, u \in \sigma \leftrightarrow A(u)\, \right)
			\end{align}
			が成り立つので
			\begin{align}
				\sigma \in \eta
			\end{align}
			かつ
			\begin{align}
				\sigma = \{x|A(x)\}
			\end{align}
			が成立する.ゆえに
			\begin{align}
				\{x|A(x)\} \in \eta
			\end{align}
			が成立する.以上で
			\begin{align}
				\{x|A(x)\} \in \eta \leftrightarrow
				\exists s\, \left(\, s \in \eta \wedge 
				\forall u\, \left(\, u \in s \leftrightarrow A(u)\, \right)\, \right)
			\end{align}
			が得られた.
			
		\item $\{x|A(x)\} \in \{y|B(y)\}$は$\exists s\, \left(\, B(s) \wedge 
			\forall u\, \left(\, u \in s \leftrightarrow A(u)\, \right)\, \right)$
		
		\item $\{x|A(x)\} \in \varepsilon y B(y)$は$\exists s,t\, \left(\, s \in t \wedge 
			\forall u\, \left(\, u \in s \leftrightarrow A(u)\, \right) \wedge B(t)\, \right)$
		
		\item $\varepsilon x A(x) \in y$は$\exists s\, \left(\, s \in y \wedge A(s)\, \right)$
			
			これも公理にしよう:
			\begin{align}
				\forall y\, \left(\, \varepsilon x A(x) \in y \leftrightarrow
				\exists s\, \left(\, s \in y \wedge A(s)\, \right)\, \right).
			\end{align}
			いや,$y$をクラスとした言明の方が良いかも.
			\begin{align}
				\varepsilon x A(x) \in y \leftrightarrow
				\exists s\, \left(\, s \in y \wedge A(s)\, \right).
			\end{align}
		
		\item $\varepsilon x A(x) \in \{y|B(y)\}$は$\exists s\, \left(\, A(s) \wedge B(s)\, \right)$
			
			上の公理からこの式の同値性も導かれます.まず
			\begin{align}
				\exists s\, \left(\, A(s) \wedge B(s)\, \right)
			\end{align}
			が成り立っているとしよう.そして
			\begin{align}
				\sigma = \varepsilon s\, \left(\, A(s) \wedge B(s)\, \right)
			\end{align}
			とおくと,
			\begin{align}
				A(\sigma) \wedge B(\sigma)
			\end{align}
			が成立する.ゆえに
			\begin{align}
				\sigma \in \{y|B(y)\} \wedge A(\sigma)
			\end{align}
			が成立する.ゆえに
			\begin{align}
				\exists s\, \left(\, s \in \{y|B(y)\} \wedge A(s)\, \right)
			\end{align}
			が成り立つ.ゆえに
			\begin{align}
				\varepsilon x A(x) \in \{y|B(y)\}
			\end{align}
			が成り立つ.逆に
			\begin{align}
				\varepsilon x A(x) \in \{y|B(y)\}
			\end{align}
			が成り立っているとしよう.すると
			\begin{align}
				\exists s\, \left(\, s = \varepsilon x A(x)\, \right)
			\end{align}
			が成り立つが,これは$\mathcal{L}_{0}$の式で
			\begin{align}
				\exists s A(s)
			\end{align}
			であって,
			\begin{align}
				\sigma = \varepsilon s A(s)
			\end{align}
			とおけば
			\begin{align}
				A(\sigma)
			\end{align}
			が成立する.ところで
			\begin{align}
				\sigma \in \{y|B(y)\}
			\end{align}
			なので
			\begin{align}
				B(\sigma)
			\end{align}
			も成り立つ.ゆえに
			\begin{align}
				A(\sigma) \wedge B(\sigma)
			\end{align}
			が成り立つ.ゆえに
			\begin{align}
				\exists s\, \left(\, A(s) \wedge B(s)\, \right)
			\end{align}
			が成り立つ.
			
		\item $\varepsilon x A(x) \in \varepsilon y B(y)$は$\exists s,t\, \left(\, s \in t \wedge A(s) \wedge B(t)\, \right)$
		
			この式の同値性も証明できる.まず
			\begin{align}
				\exists s,t\, \left(\, s \in t \wedge A(s) \wedge B(t)\, \right)
			\end{align}
			が成り立っているとしよう.この式は
			\begin{align}
				\exists s\, \left(\, \exists t\, \left(\, s \in t \wedge A(s) \wedge B(t)\, \right)\, \right)
			\end{align}
			の略記であって,$\exists$の規則より
			\begin{align}
				\sigma = \varepsilon s\, \left(\, \exists t\, \left(\, s \in t \wedge A(s) \wedge B(t)\, \right)\, \right)
			\end{align}
			とおけば
			\begin{align}
				\exists t\, \left(\, \sigma \in t \wedge A(\sigma) \wedge B(t)\, \right)
			\end{align}
			が成立する.$\sigma \in t \wedge A(\sigma) \wedge B(t)$を$\mathcal{L}_{0}$の式に書き直したものを
			$\varphi(t)$として
			\begin{align}
				\tau \defeq \varepsilon t \varphi(t)
			\end{align}
			とおけば,$\exists$の規則より
			\begin{align}
				\sigma \in \tau \wedge A(\sigma) \wedge B(\tau)
			\end{align}
			が成立する.ゆえに
			\begin{align}
				\exists s\, \left(\, s \in \tau \wedge A(s)\, \right)
			\end{align}
			が成り立つから,公理より
			\begin{align}
				\varepsilon x A(x) \in \tau
			\end{align}
			が成立する.ゆえに
			\begin{align}
				\varepsilon x A(x) \in \tau \wedge B(\tau)
			\end{align}
			が成立する.ゆえに
			\begin{align}
				\exists t\, \left(\, \varepsilon x A(x) \in t \wedge B(t)\, \right)
			\end{align}
			が成立する.公理より
			\begin{align}
				\varepsilon x A(x) \in \varepsilon y B(y)
			\end{align}
			が成立する.逆は容易い.
			\begin{align}
				\varepsilon x A(x) \in \varepsilon y B(y)
			\end{align}
			が成り立っているとすれば公理より
			\begin{align}
				\exists t\, \left(\, \varepsilon x A(x) \in t \wedge B(t)\, \right)
			\end{align}
			が成立する.$\varepsilon x A(x) \in t \wedge B(t)$を$\mathcal{L}_{0}$の式に書き直したものを$\psi(t)$として
			\begin{align}
				\tau \defeq \varepsilon t \psi(t)
			\end{align}
			とおけば
			\begin{align}
				\varepsilon x A(x) \in \tau \wedge B(\tau)
			\end{align}
			が成立するが,ここで公理より
			\begin{align}
				\exists s\, \left(\, s \in \tau \wedge A(s)\, \right)
			\end{align}
			が成り立つので,$s \in \tau \wedge A(s)$を$\mathcal{L}_{0}$の式に書き直したものを$\xi(s)$として
			\begin{align}
				\sigma \defeq \varepsilon s \xi(s)
			\end{align}
			とおけば
			\begin{align}
				\sigma \in \tau \wedge A(\sigma)
			\end{align}
			が成立する.以上より
			\begin{align}
				\sigma \in \tau \wedge A(\sigma) \wedge B(\tau)
			\end{align}
			が成立する.ゆえに
			\begin{align}
				\exists t\, \left(\, \sigma \in t \wedge A(\sigma) \wedge B(t)\, \right)
			\end{align}
			が得られる.ゆえに
			\begin{align}
				\exists s\, \left(\, \exists t\, \left(\, \sigma \in t \wedge A(\sigma) \wedge B(t)\, \right)\, \right)
			\end{align}
			が得られる.
	\end{itemize}
	
	\begin{screen}
		\begin{logicalaxm}\mbox{}
			\begin{itemize}
				\item 任意の閉項$\tau$に対して,$A(\tau)$が定理ならば$\exists x A(x)$が成り立つ.
				\item $\exists x A(x)$が定理ならば,$A(\varepsilon x \hat{A}(x))$が成り立つ.
				\item すべての閉項$\tau$に対して$A(\tau)$が定理ならば,$\forall x A(x)$が成り立つ.
				\item $\forall x A(x)$が定理ならば,すべての閉項$\tau$に対して$A(\tau)$が成り立つ.
			\end{itemize}
		\end{logicalaxm}
	\end{screen}
	
	定理として
	\begin{align}
		\forall x A(x) \Longleftrightarrow A(\varepsilon x \rightharpoondown \hat{A}(x))
	\end{align}
	が得られる.アイデアとしてはさあ,$\varepsilon x A(x)$の全体が集合に対応しているのであって,
	いやもちろん集合そのものではないけど,集合は$\varepsilon x A(x)$のどれかに等しい類なわけで,
	だからモデル論に出てくる「宇宙」とかいう得体の知れない集合()は俺の集合論に不要なんだよね.
	俺のノートの「宇宙」はすべて実態が把握できるように,具体的な記号列で書き表せるのが良いよね.
	ちなみにこの「宇宙」は$\{x|x=x\}$とは別ね.
	
	\begin{screen}
		\begin{axm}
			\begin{align}
				\forall x\, \left(\, x \in \{y|B(y)\} \leftrightarrow B(x)\, \right).
			\end{align}
			
			\begin{align}
				\forall x\, \left(\, x \in \varepsilon y B(y) \leftrightarrow
				\left(\, \exists t\, B(t) \rightarrow 
				\exists t\, \left(\, x \in t \wedge B(t)\, \right)\, \right)\, \right).
			\end{align}
			が定理となるために
			\begin{align}
				\exists t\, \left(\, x \in t \wedge 
				\left(\, \exists u\, \forall v\, \left(\, B(v) \leftrightarrow v \in u\, \right)\,
				\rightarrow B(t)\, \right)\, \right)
				\rightarrow x \in \varepsilon y B(y)
			\end{align}
			を公理とする.
			
			\begin{align}
				\varepsilon x A(x) \in y \leftrightarrow
				\exists s\, \left(\, s \in y \wedge A(s)\, \right).
			\end{align}
		\end{axm}
	\end{screen}
	
	いや,した二つは公理じゃねえな.定理だ.実際
	\begin{align}
		x \in \varepsilon y B(y)
	\end{align}
	が成り立っているとしよう.$\varepsilon y B(y)$は集合であって
	\begin{align}
		\exists s\, \left(\, s = \varepsilon y B(y)\, \right)
	\end{align}
	が成り立つので,
	\begin{align}
		fff
	\end{align}
	
	\begin{screen}
		\begin{thm}
			\begin{align}
				\exists x A(x) \rightarrow \varepsilon x A(x) \in \{x|A(x)\}.
			\end{align}
		\end{thm}
	\end{screen}
	
	\begin{sketch}
		\begin{align}
			\exists x A(x)
		\end{align}
		が成り立ているとするとき,
		\begin{align}
			\sigma \defeq \varepsilon x A(x)
		\end{align}
		とおけば
		\begin{align}
			A(\sigma)
		\end{align}
		が成り立つので,公理より
		\begin{align}
			\sigma \in \{x|A(x)\}
		\end{align}
		が成立する.ゆえに
		\begin{align}
			\sigma \in \{x|A(x)\} \wedge A(\sigma)
		\end{align}
		が成り立つ.ゆえに
		\begin{align}
			\exists s\, \left(\, s \in \{x|A(x)\} \wedge A(s)\, \right)
		\end{align}
		が成立する.ゆえに公理より
		\begin{align}
			\varepsilon x A(x) \in \{x|A(x)\}
		\end{align}
		が成立する.
		\QED
	\end{sketch}
	
	\begin{itembox}[l]{満たされて欲しいこと}
		\begin{description}
			\item[等号]
				\begin{itemize}
					\item $x = \{y|B(y)\}$と$\forall s\, \left(\, s \in x \leftrightarrow B(s)\, \right)$
					\item $x = \varepsilon y B(y)$と$\exists s\, \left(\, A(s) \wedge \forall u\,
						\left(\, u \in x \leftrightarrow u \in s\, \right)\, \right)$
					\item $\{x|A(x)\} = \{y|B(y)\}$と$\forall s\, \left(\, A(s) \leftrightarrow B(s)\, \right)$
					\item $\{x|A(x)\} = \varepsilon y B(y)$と
						$\exists s\, \left(\, \forall u\,
						\left(\, u \in s \leftrightarrow A(u)\, \right) \wedge B(s)\, \right)$
					\item $\varepsilon x A(x) = \varepsilon y B(y)$と
						$\exists s,t\, \left(\, s = t \wedge A(s) \wedge B(t)\, \right)$
				\end{itemize}
				
			\item[帰属]
				\begin{itemize}
					\item $x \in \{y|B(y)\}$は$B(x)$
					\item $x \in \varepsilon y B(y)$は$\exists t\, \left(\, x \in t \wedge B(t)\, \right)$
					\item $\{x|A(x)\} \in y$は$\exists s\, \left(\, s \in y \wedge \forall u\, \left(\, u \in s \leftrightarrow A(u)\, \right)\, \right)$
					\item $\{x|A(x)\} \in \{y|B(y)\}$は$\exists s\, \left(\, B(s) \wedge \forall u\, \left(\, u \in s \leftrightarrow A(u)\, \right)\, \right)$
					\item $\{x|A(x)\} \in \varepsilon y B(y)$は$\exists s,t\, \left(\, s \in t \wedge \forall u\, \left(\, u \in s \leftrightarrow A(u)\, \right) \wedge B(t)\, \right)$
					\item $\varepsilon x A(x) \in y$は$\exists s\, \left(\, s \in y \wedge A(s)\, \right)$
					\item $\varepsilon x A(x) \in \{y|B(y)\}$は$\exists s\, \left(\, A(s) \wedge B(s)\, \right)$
					\item $\varepsilon x A(x) \in \varepsilon y B(y)$は$\exists s,t\, \left(\, s \in t \wedge A(s) \wedge B(t)\, \right)$
				\end{itemize}
		\end{description}
	\end{itembox}

\section{定理I\hspace{-.1em}I.15.2}
	\begin{description}
		\item[(3)]
			項$\tau$が変項$x$のとき,$\zeta_{\tau}(y)$を
			\begin{align}
				x = y
			\end{align}
			とすれば,
			\begin{align}
				\Sigma' \vdash \forall x\, \exists! y\, (\, x=y\, )
			\end{align}
			つまり
			\begin{align}
				\Sigma' \vdash \forall x\, \exists! y\, \zeta_{\tau}(y)
			\end{align}
			および
			\begin{align}
				\Sigma \vdash \forall x\, (\, x=x\, )
			\end{align}
			つまり
			\begin{align}
				\Sigma \vdash \forall x\, \zeta_{\tau}(\tau)
			\end{align}
			が成り立つ.項$\tau$が
			\begin{align}
				f\tau_{1}\cdots\tau_{n}
			\end{align}
			のとき,$f \in \mathcal{L} \backslash \mathcal{L}_{0}$ならば$\zeta_{\tau}(y)$を
			\begin{align}
				\exists z_{1}, \cdots, z_{n}\, \left(\, 
				\theta_{f}(z_{1},\cdots,z_{n},y) \wedge \zeta_{\tau_{1}}(z_{1}) \wedge
				\cdots \wedge \zeta_{\tau_{n}}(z_{n})\, \right)
			\end{align}
			とし,$f \in \mathcal{L}_{0}$ならば
			\begin{align}
				\exists z_{1}, \cdots, z_{n}\, \left(\, 
				f(z_{1},\cdots,z_{n}) = y \wedge \zeta_{\tau_{1}}(z_{1}) \wedge
				\cdots \wedge \zeta_{\tau_{n}}(z_{n})\, \right)
			\end{align}
			とする.仮定より
			\begin{align}
				\Sigma \vdash \exists! y\, \zeta_{\tau_{i}}(y)
			\end{align}
			が成り立つので,
			\begin{align}
				\Sigma \vdash \zeta_{\tau_{i}}(z_{i})
			\end{align}
			を満たす$z_{i}$が取れる.そして定義I\hspace{-.1em}I.15.1より
			\begin{align}
				\Sigma \vdash \exists!y\, \theta_{f}(z_{1},\cdots,z_{n},y)
			\end{align}
			が成り立つので,その$y$を取れば
			\begin{align}
				\Sigma \vdash \exists y\, \zeta_{\tau}(y)
			\end{align}
			が成立する.ただし,$\eta$を
			\begin{align}
				\Sigma \vdash \exists z_{1}, \cdots, z_{n}\, \left(\, 
				\theta_{f}(z_{1},\cdots,z_{n},\eta) \wedge \zeta_{\tau_{1}}(z_{1}) \wedge
				\cdots \wedge \zeta_{\tau_{n}}(z_{n})\, \right)
			\end{align}
			を満たすものとすれば,このとき
			\begin{align}
				\Sigma \vdash \zeta_{\tau_{i}}(w_{i})
			\end{align}
			および
			\begin{align}
				\Sigma \vdash \theta_{f}(w_{1},\cdots,w_{n},\eta)
			\end{align}
			を満たす$w_{i}$が取れるが,$z_{i} = w_{i}$なので
			\begin{align}
				\Sigma \vdash \theta_{f}(z_{1},\cdots,z_{n},\eta)
			\end{align}
			が成り立つことになって,定義I\hspace{-.1em}I.15.1より
			\begin{align}
				y = \eta
			\end{align}
			が成り立つ.ゆえに
			\begin{align}
				\Sigma \vdash \exists! y\, \zeta_{\tau}(y)
			\end{align}
			が成立する.他方で仮定より
			\begin{align}
				\Sigma' \vdash \zeta_{\tau_{i}}(\tau_{i})
			\end{align}
			が成り立ち,かつ定義I\hspace{-.1em}I.15.1より
			\begin{align}
				\Sigma' \vdash \forall x_{1},\cdots,x_{n}\,
				\theta_{f}(x_{1},\cdots,x_{n},f(x_{1},\cdots,x_{n}))
			\end{align}
			が成り立つので
			\begin{align}
				\Sigma' \vdash \theta_{f}(\tau_{1},\cdots,\tau_{n},f(\tau_{1},\cdots,\tau_{n}))
			\end{align}
			が成り立つ.ゆえに
			\begin{align}
				\Sigma' \vdash \theta_{f}(\tau_{1},\cdots,\tau_{n},f(\tau_{1},\cdots,\tau_{n})) \wedge \zeta_{\tau_{1}}(\tau_{1}) \wedge \cdots \wedge \zeta_{\tau_{n}}(\tau_{n})
			\end{align}
			が成り立つ.ゆえに
			\begin{align}
				\Sigma' \vdash \zeta_{\tau}(\tau)
			\end{align}
			が成り立つ.
			
		\item[(2)]
			$\varphi$を$p\tau_{1}\cdots\tau_{n}$なる原子式とするとき,
			$p$が$\mathcal{L} \backslash \mathcal{L}_{0}$の要素ならば$\hat{\varphi}$を
			\begin{align}
				\exists z_{1},\cdots,z_{n}\, 
				\left(\, \theta_{p}z_{1} \cdots z_{n} \wedge \zeta_{\tau_{1}}(z_{1})
				\wedge \cdots \wedge \zeta_{\tau_{n}}(z_{n})\, \right)
			\end{align}
			とし,$p$が$\mathcal{L}_{0}$の要素ならば
			\begin{align}
				\exists z_{1},\cdots,z_{n}\, 
				\left(\, pz_{1} \cdots z_{n} \wedge \zeta_{\tau_{1}}(z_{1})
				\wedge \cdots \wedge \zeta_{\tau_{n}}(z_{n})\, \right)
			\end{align}
			とする.$\varphi$が成り立っているとき,仮定より
			\begin{align}
				\Sigma' \vdash \zeta_{\tau_{i}}(\tau_{i})
			\end{align}
			が満たされ,また定義I\hspace{-.1em}I.15.1より($\Delta$は$\Sigma'$に含まれているので)
			\begin{align}
				\Sigma' \cup \{\varphi\} \vdash \theta_{p}\tau_{1} \cdots \tau_{n}
			\end{align}
			も満たされているので
			\begin{align}
				\Sigma' \cup \{\varphi\} \vdash \theta_{p}\tau_{1} \cdots \tau_{n}
				\wedge \zeta_{\tau_{1}}(\tau_{1}) \wedge \cdots \wedge 
				\zeta_{\tau_{n}}(\tau_{n})
			\end{align}
			が成り立つ.すなわち
			\begin{align}
				\Sigma' \cup \{\varphi\} \vdash \hat{\varphi}
			\end{align}
			が成り立つ.つまり
			\begin{align}
				\Sigma' \vdash \varphi \rightarrow \hat{\varphi}
			\end{align}
			が成り立つ.逆に$\hat{\varphi}$が成り立っているとき,
			\begin{align}
				\Sigma' \cup \{\hat{\varphi}\} \vdash \theta_{p}w_{1} \cdots w_{n}
				\wedge \zeta_{\tau_{1}}(w_{1}) \wedge \cdots \wedge 
				\zeta_{\tau_{n}}(w_{n})
			\end{align}
			を満たす$w_{1},\cdots,w_{n}$が取れるが,
			\begin{align}
				\Sigma \vdash \exists! y\, \zeta_{\tau_{i}}(y)
			\end{align}
			かつ
			\begin{align}
				\Sigma' \vdash \zeta_{\tau_{i}}(\tau_{i})
			\end{align}
			なので
			\begin{align}
				w_{i} = \tau_{i}
			\end{align}
			である.ゆえに
			\begin{align}
				\Sigma' \cup \{\hat{\varphi}\} \vdash \theta_{p}\tau_{1} \cdots \tau_{n}
			\end{align}
			が成り立つ.ゆえに
			\begin{align}
				\Sigma' \vdash \hat{\varphi} \rightarrow \varphi
			\end{align}
			が得られる.
			\QED
	\end{description}
	
	\begin{screen}
		$\forall x\, (\, x \notin y\, )$を$\theta_{\emptyset}(y)$とするとき.
	\end{screen}
	
	$\emptyset$の定義$\delta_{\emptyset}$は
	\begin{align}
		\forall x\, (\, x \notin \emptyset\, )
	\end{align}
	である.$\zeta_{\emptyset}(y)$は
	\begin{align}
		\forall x\, (\, x \notin y\, )
	\end{align}
	であって,
	\begin{align}
		\Sigma \vdash \exists! y\, \zeta_{\emptyset}(y)
	\end{align}
	が成り立ち,また$\delta_{\emptyset}$と$\zeta_{\emptyset}(1)$は同じなので
	\begin{align}
		\Sigma \cup \{\delta_{\emptyset}\} = \Sigma' \vdash \zeta_{\emptyset}(\emptyset)
	\end{align}
	が成り立つ.そして,例えば$z$を変項とすれば
	\begin{align}
		z \in \emptyset
	\end{align}
	と
	\begin{align}
		\exists s,t\, \left(\, s \in t \wedge s = z \wedge \forall x\, (\, x \notin t\, )\, \right)
	\end{align}
	が$\Sigma'$の下で同値になる.
	
	\begin{screen}
		$\forall x\, (\, x \cdot y = y \cdot x = x\, )$を$\theta_{1}(y)$とするとき,
	\end{screen}
	
	$1$の定義$\delta_{1}$は
	\begin{align}
		\forall x\, (\, x \cdot 1 = 1 \cdot x = x\, )
	\end{align}
	である.$\zeta_{1}(y)$は
	\begin{align}
		\forall x\, (\, x \cdot y = y \cdot x = x\, )
	\end{align}
	であって,
	\begin{align}
		\Sigma \vdash \exists! y\, \zeta_{1}(y)
	\end{align}
	が成り立ち,また$\delta_{1}$と$\zeta_{1}(1)$は同じなので
	\begin{align}
		\Sigma \cup \{\delta_{1}\} = \Sigma' \vdash \zeta_{1}(1)
	\end{align}
	が成り立つ.そして,例えば$z$を変項とすれば
	\begin{align}
		z \in 1
	\end{align}
	と
	\begin{align}
		\exists s,t\, \left(\, s \in t \wedge s = z \wedge \forall x\, (\, x \cdot t = t \cdot x = x\, )\, \right)
	\end{align}
	が$\Sigma'$の下で同値になる.
	
	\begin{screen}
		$y \cdot (x \cdot x) = x$を$\theta_{i}(x,y)$とするとき,
	\end{screen}
	
	$i$の定義$\delta_{i}$は
	\begin{align}
		\forall x\, \left(\, i(x) \cdot (x \cdot x) = x\, \right)
	\end{align}
	である.$\zeta_{i(x)}(y)$は
	\begin{align}
		\exists x\, \left(\, \theta_{i}(x,y) \wedge x = x\, \right)
	\end{align}
	である.そして,例えば$x,z$を変項とすれば
	\begin{align}
		z \in i(x)
	\end{align}
	と
	\begin{align}
		\exists s,t\, \left(\, s \in t \wedge s = z \wedge \zeta_{i(x)}(t)\, \right)
	\end{align}
	が$\Sigma'$の下で同値になる.
	
\section{菊池誠不完全性定理}
	言語とは定数記号と関数記号と関係記号の全体ということで,
	\begin{itemize}
		\item $\mathcal{L}_{0}$とは$\{\in,\natural\}$.
		\item $\mathcal{L}$とは$\{\in\}$に加えて閉項の全体.
	\end{itemize}
\end{document}
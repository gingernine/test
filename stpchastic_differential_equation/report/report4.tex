\section{10/25}
	確率空間を$(\Omega,\mathcal{F},\operatorname{P})$とし,以下でマルチンゲールを定義する.集合$I$によって確率過程の時点を表現し,
	以降でこれは$[0,\infty)$や$\{0,,1,\cdots,n\}$など実数の区間や高々可算集合を指すものと考え,
	$I$が高々可算集合の場合は離散位相,$\R$の区間の場合は相対位相を考える.また扱う確率変数は全て実数値で考える.
	\begin{itembox}[l]{}
		\begin{dfn}[フィルトレーション]
			$\mathcal{F}$の部分$\sigma$-加法族の部分系$\left\{\ \mathcal{F}_\alpha\quad |\quad \alpha \in I\ \right\}$
			がフィルトレーション(filtration)であるとは,任意の$\alpha,\beta \in I$に対して$\alpha \leq \beta$ならば
			$\mathcal{F}_\alpha \subset \mathcal{F}_\beta$の関係をもつことで定義する.
		\end{dfn}
	\end{itembox}
	
	マルチンゲールを定義する前に同値類に対して順序を定める.
	$(\Omega,\mathcal{F},\operatorname{P})$上の可測$\mathcal{F}/\borel{\R}$関数の全体を$\mathcal{M}$,
	$\mathcal{M}$をa.s.で等しいものにまとめた商空間を$M$と表す.$[f],[g] \in M$に対して
	\begin{align}
		\mbox{$f \leq g\ $P-a.s.かつそのときに限り$[f] \leq [g]$}
	\end{align}
	として関係''$\leq$''(記号は$\mathcal{M}$におけるものと同じであるが)を定義すればこれは$M$において順序となる.
	この定義がwell-defined,つまり代表元の取り方に依存しないことは
	$(f' > g') \subset (f \neq f') \cup (f > g) \cup (g \neq g')$
	\footnote{$(f > g) \coloneqq \left\{\ x \in \Omega\quad |\quad f(x) > g(x)\ \right\}$}
	かつ右辺が零集合であることにより明確であるが,順序関係としての定義を満たしていることは以下で判る.
	\begin{itemize}
		\item $f=f$により$[f] \leq [f]$,
		\item $[f] \leq [g]$かつ$[g] \leq [f]$なら$(f > g)$と$(g > f)$はP-零集合だから$[f]=[g]$,
		\item $[f] \leq [g]$かつ$[g] \leq [h]$なら$(f > h) \subset (f > g) \cup (g > h)$により$[f]\leq[h]$.
	\end{itemize}
	
	次にマルチンゲールを定義する.確率変数とその同値類の表記は区別しないが,大体は文脈から判断するべきことであると留意しておく.
	\begin{itembox}[l]{}
		\begin{dfn}[マルチンゲール]
			$(\Omega,\mathcal{F},\operatorname{P})$上の実確率変数の族
			$(M_\alpha)_{\alpha \in I} \subset \semiLp{p}{\mathcal{F},\operatorname{P}}\ (p \geq 1)$が
			次の四条件を満たすとき,これを$\mathrm{L}^p$-劣マルチンゲール($\mathrm{L}^p$-submartingale)という.
			\begin{description}
				\item[(M.1)] $\forall \alpha \in I$に対し$M_\alpha$は可測$\mathcal{F}_\alpha/\borel{\R}$である.
				\item[(M.2)] 任意の$\alpha \leq \beta\ (\alpha,\beta \in I)$に対し$\cexp{M_\beta}{\mathcal{F}_\alpha} \geq M_\alpha\ $({\scriptsize 同値類に対する順序関係})が成り立つ.
				\item[(M.3)] 各$\omega \in \Omega$において,任意の$\alpha \in I$で左極限が存在する:$\exists \lim\limits_{\beta \uparrow \alpha} M_\beta(\omega) \in \R.$
				\item[(M.4)] 各$\omega \in \Omega$において,任意の$\alpha \in I$で右連続である:$M_\alpha(\omega) = \lim\limits_{\beta \downarrow \alpha} M_\beta(\omega).$
			\end{description}
			条件(M.2)の不等号が逆向き''$\leq$''の場合,$(M_\alpha)_{\alpha \in I}$を$\mathrm{L}^p$-優マルチンゲール($\mathrm{L}^p$-supermartingale)といい,
			劣かつ優マルチンゲールであるものをマルチンゲールという.
			\label{dfn:martingale}
		\end{dfn}
	\end{itembox}
	
	\begin{itembox}[l]{}
		\begin{dfn}[停止時刻]
			$\Omega$上の関数で次を満たすものを($(\mathcal{F}_\alpha)$-)停止時刻(stopping time)という:
			\begin{align}
				\tau:\Omega \longrightarrow I\quad \mathrm{s.t.}\quad \forall \alpha \in I,\ (\tau \leq \alpha) \in \mathcal{F}_\alpha.
			\end{align}
		\end{dfn}
	\end{itembox}
	
	\begin{itembox}[l]{}
		\begin{rem}[停止時刻は可測]
			上で定義した$\tau$は可測$\mathcal{F}/\borel{I}$である.
		\end{rem}
	\end{itembox}
	\begin{prf}\mbox{}
		\begin{description}
			\item[$I$が$\R$の区間である場合]
				任意の$\alpha \in I$に対して$I_\alpha \coloneqq (-\infty,\alpha) \cap I$は$I$における(相対の)開集合であり
				$\tau^{-1}(I_\alpha) = (\tau \leq \alpha) \in \mathcal{F}_\alpha \subset \mathcal{F}$が成り立つ.
				つまり
				\begin{align}
					\left\{\ I_\alpha\quad |\quad \alpha \in I\ \right\} \subset \left\{\ A \in \borel{I}\quad |\quad \tau^{-1}(A) \in \mathcal{F}\ \right\}
					\label{eq:stopping_time_mble}
				\end{align}
				が成り立ち,左辺の$I_\alpha$の形の全体は$\borel{I}$を生成するから$\tau$の可測性が証明された.
				
			\item[$I$が高々可算集合である場合]
				先ず$\alpha \in I$に対して$(\tau < \alpha)$が$\mathcal{F}_\alpha$に属することを示す.
				$\alpha$に対して直前の元$\beta \in I$が存在するか$\alpha$が$I$の最小限である場合,前者なら$(\tau < \alpha) = (\tau \leq \beta)$
				となり後者なら$(\tau < \alpha) = \emptyset$となるからどちらも$\mathcal{F}_\alpha$に属する.
				そうでない場合は$\alpha - 1/n < x < \alpha$を満たす点列$x_n \in I\ (n=1,2,3,\cdots)$を取れば,
				$(\tau < \alpha) = \cap_{n=1}^{\infty}(\tau \leq \alpha - 1/n)$
				により$(\tau < \alpha) \in \mathcal{F}_\alpha$が判る.以上の準備の下で
				任意の$\alpha \in I$に対して$\tau^{-1}(\{\alpha\}) = (\tau \leq \alpha) - (\tau < \alpha) \in \mathcal{F}_\alpha$が成り立ち,
				更に可算集合$I$には離散位相が入っているから任意の$A \in \borel{I}$は
				一点集合の可算和で表現できて,$\tau^{-1}(A) \in \mathcal{F}$であると証明された.		
		\end{description}
		\QED
	\end{prf}
	
	\begin{itembox}[l]{}
		\begin{dfn}[停止時刻の再定義]
			今$\tau$の終集合は$I$であるが,$I \rightarrow \R$の単射$i$を用いて$\tau^* \coloneqq i \circ \tau$とすれば,
			\begin{align}
				\borel{I}=\left\{\ A \cap I\quad |\quad A \in \borel{\R}\ \right\} = \left\{\ i^{-1}(A)\quad |\quad A \in \borel{\R}\ \right\}
			\end{align}
			により$i$が可測$\borel{I}/\borel{\R}$であるから合成写像$\tau^*$は可測$\mathcal{F}/\borel{\R}$となる.以降は
			この$\tau^*$を停止時刻$\tau$と表記して扱うことにする.
		\end{dfn}
	\end{itembox}
	
	定数関数は停止時刻となる.$\tau$が$\Omega$上の定数関数なら$(\tau \leq \alpha)$は空集合か全体集合にしかならないからである.
	また$\sigma,\tau$を$I$に値を取る停止時刻とすると
	$\sigma \vee \tau$と$\sigma \wedge \tau$も停止時刻となる.
	\begin{align}
		\begin{cases}
			(\sigma \wedge \tau \leq \alpha) = (\sigma \leq \alpha) \cup (\tau \leq \alpha), \\
			(\sigma \vee \tau \leq \alpha) = (\sigma \leq \alpha) \cap (\tau \leq \alpha)
		\end{cases}
		\quad ,(\forall \alpha \in I)
	\end{align}
	が成り立つからである.
	
	\begin{itembox}[l]{}
		\begin{dfn}[停止時刻の前に決まっている事象系]
			$\tau$を$I$に値を取る停止時刻とする.$\tau$に対し次の集合系を定義する.
			\begin{align}
				\mathcal{F}_\tau \coloneqq \left\{\ A \in \mathcal{F}\quad |\quad (\tau \leq \alpha) \cap A \in \mathcal{F}_\alpha,\ \forall \alpha \in I\ \right\}.
			\end{align}
		\end{dfn}
	\end{itembox}
	\begin{itembox}[l]{}
		\begin{prp}[停止時刻の性質]
			$\sigma, \tau$を$I$に値を取る停止時刻であるとする.
			\begin{description}
				\item[(1)] $\mathcal{F}_\tau$は$\sigma$-加法族である.
				\item[(2)] 或る$\alpha \in I$に対して$\tau(\omega) = \alpha\ (\forall \omega \in \Omega)$なら$\mathcal{F}_\alpha = \mathcal{F}_\tau$.
				\item[(3)] $\sigma(\omega) \leq \tau(\omega)\ (\forall \omega \in \Omega)$ならば$\mathcal{F}_\sigma \subset \mathcal{F}_\tau$.
				\item[(4)] $\mathcal{F}_{\sigma \wedge \tau} = \mathcal{F}_\sigma \cap \mathcal{F}_\tau$.
				\item[(5)] $\mathcal{F}_{\sigma \vee \tau} = \mathcal{F}_\sigma \vee \mathcal{F}_\tau$.
			\end{description}
		\end{prp}
	\end{itembox}
	
	\begin{prf}\mbox{}
		\begin{description}
			\item[(1)] 停止時刻の定義より$\Omega \in \mathcal{F}_\tau$である.また$A \in \mathcal{F}_\tau$なら
				$A^c \cap (\tau \leq \alpha) = (\tau \leq \alpha) - A \cap (\tau \leq \alpha) \in \mathcal{F}_\alpha$より
				$A^c \in \mathcal{F}_\tau$となる.可算個の$A_n \in \mathcal{F}_\tau$については
				$\cup_{n=1}^{\infty} A_n \cap (\tau \leq \alpha) = \cup_{n=1}^{\infty} \left(A_n \cap (\tau \leq \alpha)\right) \in \mathcal{F}_\alpha$
				により$\cup_{n=1}^{\infty} A_n \in \mathcal{F}_\tau$が成り立つ.
			
			\item[(2)] $A \in \mathcal{F}_\alpha$なら任意の$\beta \in I$に対して
				\begin{align}
					A \cap (\tau \leq \beta) =
					\begin{cases}
						A & \alpha \leq \beta \\
						\emptyset & \alpha > \beta
					\end{cases}
				\end{align}
				が成り立つから,いずれの場合も$A \in \mathcal{F}_\beta$となり$A \subset \mathcal{F}_\tau$が成り立つ.
				逆に$A \in \mathcal{F}_\tau$のとき,$A = A \cap (\tau \leq \alpha) \in \mathcal{F}_\alpha$
				が成り立ち$\mathcal{F}_\alpha = \mathcal{F}_\tau$が示された.
				
			\item[(3)] $A \in \mathcal{F}_\sigma$なら任意の$\alpha \in I$に対して
				\begin{align}
					A \cap (\tau \leq \alpha) = A \cap (\sigma \leq \alpha) \cap (\tau \leq \alpha) \in \mathcal{F}_\alpha
				\end{align}
				が成り立つから$A \in \mathcal{F}_\tau$となる.
			
			\item[(4)] $\sigma \wedge \tau$が停止時刻であることと(3)より
				$\mathcal{F}_{\sigma \wedge \tau} \subset \mathcal{F}_\sigma$と$\mathcal{F}_{\sigma \wedge \tau} \subset \mathcal{F}_\tau$が判る.
				また$A \in \mathcal{F}_\sigma \cap \mathcal{F}_\tau$に対し
				\begin{align}
					A \cap (\sigma \wedge \tau \leq \alpha) = [A \cap (\sigma \leq \alpha)] \bigcup [A \cap (\tau \leq \alpha)] \in \mathcal{F}_\alpha \quad (\forall \alpha \in I)
				\end{align}
				より$A \in \mathcal{F}_{\sigma \wedge \tau}$も成り立つ.
			
			\item[(5)] 
				先ず$\sigma \vee \tau$が停止時刻であることと(3)より
				$\mathcal{F}_\sigma \subset \mathcal{F}_{\sigma \wedge \tau}$と$\mathcal{F}_\tau \subset \mathcal{F}_{\sigma \wedge \tau}$が判る.
				逆に$A \in \mathcal{F}_{\sigma \wedge \tau}$に対して
				
		\end{description}
	\end{prf}
	
	\begin{itembox}[l]{}
		\begin{prp}[停止時刻と条件付き期待値]
			$X \in \Lp{1}{\mathcal{F},\operatorname{P}}$と
			$I$に値を取る停止時刻$\sigma, \tau$に対し以下が成立する.
			\begin{description}
				\item[(1)] $\cexp{\defunc_{(\sigma > \tau)} X}{\mathcal{F}_\tau} = \cexp{\defunc_{(\sigma > \tau)} X}{\mathcal{F}_{\sigma \wedge \tau}}$.
				\item[(2)] $\cexp{\defunc_{(\sigma \geq \tau)} X}{\mathcal{F}_\tau} = \cexp{\defunc_{(\sigma \geq \tau)} X}{\mathcal{F}_{\sigma \wedge \tau}}$.
				\item[(3)] $\cexp{\cexp{X}{\mathcal{F}_\tau}}{\mathcal{F}_\sigma} = \cexp{X}{\mathcal{F}_{\sigma \wedge \tau}}$.
			\end{description}
		\end{prp}
	\end{itembox}
	
	\begin{prf}\mbox{}
		\begin{description}
			\item[第一段] $\defunc_{(\sigma > \tau)}$が可測$\mathcal{F}_{\sigma \wedge \tau}/\borel{\R}$であることを示す.
				$A \in \borel{\R}$に対し
				\begin{align}
					\defunc_{(\sigma > \tau)}^{-1}(A) = 
					\begin{cases}
						\Omega & (0 \in A,\ 1 \in A) \\
						(\sigma > \tau) & (0 \notin A,\ 1 \in A) \\
						(\sigma > \tau)^c & (0 \in A,\ 1 \notin A) \\
						\emptyset & (0 \notin A,\ 1 \notin A)
					\end{cases}
				\end{align}
				と表現できるから,示すことは任意の$\alpha \in I$に対して
				\begin{align}
					(\sigma > \tau) \cap (\sigma \wedge \tau \leq \alpha) \in \mathcal{F}_\alpha
				\end{align}
				が成立することである.これが示されれば
				\begin{align}
					(\sigma > \tau)^c \cap (\sigma \wedge \tau \leq \alpha)
					= (\sigma \wedge \tau \leq \alpha) \backslash \left[(\sigma > \tau) \cap (\sigma \wedge \tau \leq \alpha)\right] \in \mathcal{F}_\alpha
				\end{align}
				も成り立ち,更に$(\sigma > \tau)^c = (\sigma \leq \tau)$であることと$\sigma,\tau$の対等性により$\defunc_{(\sigma \geq \tau)}$もまた
				可測$\mathcal{F}_{\sigma \wedge \tau}/\borel{\R}$であることが判る.目的の式は次が成り立つことにより示される.
				\begin{align}
					(\sigma > \tau) \cap (\sigma \wedge \tau \leq \alpha)
					&= (\sigma > \tau) \cap (\sigma \leq \alpha) + (\sigma > \tau) \cap (\sigma > \alpha) \cap (\tau \leq \alpha) \\
					&= \left[\bigcup_{\substack{\beta \in \Q \cap I \\ \beta \leq \alpha}} (\sigma > \beta)\cap(\tau \leq \beta)\right]\cap(\sigma \leq \alpha) + (\sigma > \alpha) \cap (\tau \leq \alpha)
					\label{eq:stopping_time_conditional_expectation_1} \\
					&\in \mathcal{F}_\alpha.
				\end{align}
			
			\item[第二段] 一般の実確率変数$Y$と停止時刻$\tau$に対して
				\begin{itemize}
					\item $Y$が可測$\mathcal{F}_\tau/\borel{\R}$$\quad \Leftrightarrow \quad$任意の$\alpha \in I$に対し$Y \defunc_{\tau \leq \alpha}$が可測$\mathcal{F}_\alpha/\borel{\R}$
				\end{itemize}
				が成り立つことを示す.
				\begin{description}
					\item[$\Rightarrow$について]
						$Y$の単関数近似列$(Y_n)_{n=1}^{\infty}$の一つ一つは$Y_n = \sum_{j=1}^{N_n}a_{j,n}\defunc_{A_{j,n}}\ (A_{j,n} \in \mathcal{F}_\tau)$
						の形で表現できる.$\alpha \in I$と$A \in \mathcal{F}_\tau$の指示関数$\defunc_{A}$に対し
						\begin{align}
							\left( \defunc_A\defunc_{(\tau \leq \alpha)} \right)^{-1}(E) =
							\begin{cases}
								\Omega & (0 \in E,\ 1 \in E) \\
								A \cap (\tau \leq \alpha) & (0 \notin E,\ 1 \in E) \\
								[A \cap (\tau \leq \alpha)]^c & (0 \in E,\ 1 \notin E) \\
								\emptyset & (0 \notin E,\ 1 \notin E)
							\end{cases}
							\quad (\forall E \in \borel{\R})
						\end{align}
						となり,$A \cap (\tau \leq \alpha) \in \mathcal{F}_\alpha$より$\defunc_A\defunc_{(\tau \leq \alpha)}$が
						可測$\mathcal{F}_\alpha/\borel{\R}$であると判る.$(Y_n)_{n=1}^{\infty}$は$Y$に各点収束していくから
						$Y$も可測$\mathcal{F}_\alpha/\borel{\R}$となり,$\alpha \in I$の任意性から''$\Rightarrow$''が示された.
						
					\item[$\Leftarrow$について]
						任意の$E \in \borel{\R}$に対して
						\begin{align}
							\left\{\ \omega \in \Omega\quad |\quad Y(\omega)\defunc_{(\tau \leq \alpha)}(\omega) \in E\ \right\}
							= \begin{cases}
								Y^{-1}(E) \cap (\tau \leq \alpha) & (0 \notin E) \\
								Y^{-1}(E) \cap (\tau \leq \alpha) + (\tau \leq \alpha)^c & (0 \in E)
							\end{cases}
						\end{align}
						がいずれも$\mathcal{F}_\alpha$に属する.特に下段について$(\tau \leq \alpha)^c \in \mathcal{F}_\alpha$
						より$Y^{-1}(E) \cap (\tau \leq \alpha) \in \mathcal{F}_\alpha$となるから,結局
						$Y^{-1}(E) \cap (\tau \leq \alpha) \in \mathcal{F}_\alpha \ (\forall E \in \borel{\R})$が成り立つ.
						$\alpha \in I$の任意性から$Y^{-1}(E) \in \mathcal{F}_\tau\ (\forall E \in \borel{\R})$が示された.
				\end{description}
			
			\item[第三段]
				(1)の式を示す.第一段と性質$\tilde{\mathrm{C}}$5より
				\begin{align}
					\cexp{\defunc_{(\sigma > \tau)}X}{\mathcal{F}_\tau} = \defunc_{(\sigma > \tau)} \cexp{X}{\mathcal{F}_\tau}
				\end{align}
				が成り立つから,あとは右辺が(関数とみて)可測$\mathcal{F}_{\sigma \wedge \tau}/\borel{\R}$であればよく,このためには
				第二段の結果より任意の$\alpha \in I$に対して$\cexp{X}{\mathcal{F}_\tau}\defunc_{(\sigma > \tau)}\defunc_{(\sigma \wedge \tau \leq \alpha)}$が
				可測$\mathcal{F}_\alpha/\borel{\R}$であることを示せばよい.
				式(\refeq{eq:stopping_time_conditional_expectation_1})を使えば
				\begin{align}
					\defunc_{(\sigma > \tau)}\defunc_{(\sigma \wedge \tau \leq \alpha)}
					= \sup{\substack{\beta \in \Q \cap I \\ \beta \leq \alpha}}{\defunc_{(\sigma > \beta)}\defunc_{(\tau \leq \beta)}\defunc_{(\sigma \leq \alpha)}}
						+ \defunc_{(\sigma > \alpha)} \defunc_{(\tau \leq \alpha)}
				\end{align}
				が成り立つ.$\beta \leq \alpha$ならば,$\cexp{X}{\mathcal{F}_\tau}$が可測$\mathcal{F}_\tau/\borel{\R}$であることと第二段の結果より
				$\cexp{X}{\mathcal{F}_\tau}\defunc_{(\tau \leq \beta)}$が可測$\mathcal{F}_\beta/\borel{\R}$すなわち可測$\mathcal{F}_\alpha/\borel{\R}$
				となるから,これで$\cexp{X}{\mathcal{F}_\tau}\defunc_{(\sigma > \tau)}\defunc_{(\sigma \wedge \tau \leq \alpha)}$が可測$\mathcal{F}_\alpha/\borel{\R}$
				であると判り$\defunc_{(\sigma > \tau)} \cexp{X}{\mathcal{F}_\tau}$が可測$\mathcal{F}_{\sigma \wedge \tau}/\borel{\R}$であることが示された.
				以上で
				\begin{align}
					\cexp{\cexp{\defunc_{(\sigma > \tau)}X}{\mathcal{F}_\tau}}{\mathcal{F}_{\sigma \wedge \tau}}
					= \cexp{\defunc_{(\sigma > \tau)} \cexp{X}{\mathcal{F}_\tau}}{\mathcal{F}_{\sigma \wedge \tau}}
					= \defunc_{(\sigma > \tau)} \cexp{X}{\mathcal{F}_\tau}
					= \cexp{\defunc_{(\sigma > \tau)}X}{\mathcal{F}_\tau}
				\end{align}
				が成り立ち,
				\begin{align}
					\cexp{\cexp{\defunc_{(\sigma > \tau)}X}{\mathcal{F}_\tau}}{\mathcal{F}_{\sigma \wedge \tau}}
					= \defunc_{(\sigma > \tau)} \cexp{X}{\mathcal{F}_{\sigma \wedge \tau}}
					= \cexp{\defunc_{(\sigma > \tau)}X}{\mathcal{F}_{\sigma \wedge \tau}}
				\end{align}
				と併せて(1)の式を得る.(2)の式も以上と同じ理由で成り立つ.
				
			\item[第四段]
				(3)の式を示す.
				\begin{align}
					\cexp{\cexp{X}{\mathcal{F}_\tau}}{\mathcal{F}_\sigma}
					&= \cexp{\cexp{X}{\mathcal{F}_\tau}\defunc_{(\sigma > \tau)}}{\mathcal{F}_\sigma}
						+ \cexp{\cexp{X}{\mathcal{F}_\tau}\defunc_{(\sigma \leq \tau)}}{\mathcal{F}_\sigma} \\
					&= \cexp{\cexp{X}{\mathcal{F}_{\sigma \wedge \tau}}\defunc_{(\sigma > \tau)}}{\mathcal{F}_\sigma}
						+ \cexp{\cexp{X}{\mathcal{F}_\tau}\defunc_{(\sigma \leq \tau)}}{\mathcal{F}_\sigma} && (\scriptsize\because\mbox{(1)}) \\
					&= \cexp{X}{\mathcal{F}_{\sigma \wedge \tau}}\defunc_{(\sigma > \tau)}
						+ \cexp{\cexp{X}{\mathcal{F}_\tau}\defunc_{(\sigma \leq \tau)}}{\mathcal{F}_{\sigma \wedge \tau}} && (\scriptsize\because\mbox{(2)}) \\
					&= \cexp{X}{\mathcal{F}_{\sigma \wedge \tau}}\defunc_{(\sigma > \tau)} + \cexp{X}{\mathcal{F}_{\sigma \wedge \tau}}\defunc_{(\sigma \leq \tau)} \\
					&= \cexp{X}{\mathcal{F}_{\sigma \wedge \tau}}.
				\end{align}
				(2)式を使った箇所では$X$を$\cexp{X}{\mathcal{F}_\tau}$に置き換え$\tau$と$\sigma$を入れ替えて適用した.
		\end{description}
		\QED
	\end{prf}
	
	$\tau$を停止時刻とし,$\tau(\Omega)$が高々可算集合である場合,実確率変数の族$(M_\alpha)_{\alpha \in I}$
	に対して
	\begin{align}
		M_\tau \coloneqq \sum_{\alpha \in \tau(\Omega)}M_\alpha
	\end{align}
	とおく.全ての$\alpha \in \tau(\Omega)$について$M_\alpha$が可測$\mathcal{F}_\alpha/\borel{\R}$であるとき,
	$M_\tau$は可測$\mathcal{F}_\tau/\borel{\R}$となる.なぜならば任意の$\alpha \in I$と$A \in \borel{\R}$に対して
	\begin{align}
		(M_\tau \in A) \cap (\tau \leq \alpha)
		= \bigcup_{\substack{\beta \in \tau(\Omega) \\ \beta \leq \alpha}}(M_\beta \in A) \cap (\tau = \beta) \in \mathcal{F}_\alpha 
	\end{align}
	が成り立つからである.$M_\tau$の可測性を確認したところで次の定理を証明する.
	
	\begin{itembox}[l]{}
		\begin{thm}[任意抽出定理(その1)]
			$I = \{\ 1,2,\cdots,n\ \}$とし,実確率変数の族$(M_\alpha)_{\alpha \in I}$が$\mathrm{L}^p$-劣マルチンゲールであるとする.
			このとき$I$に値を取る停止時刻$\sigma$と$\tau$について次が成立する:
			\begin{align}
				\cexp{M_\tau}{\mathcal{F}_\sigma} \geq M_{\sigma \wedge \tau}.
			\end{align}
		\end{thm}
	\end{itembox}
	
	\begin{prf}\mbox{}
		\begin{description}
			\item[$\sigma \leq \tau$の場合]
				$F_\alpha \coloneqq \defunc_{\sigma < \alpha \leq \tau}\ (\alpha \in I)$とおくと,
				\begin{align}
					(\sigma < \alpha \leq \tau) = (\sigma < \alpha) \cap (\alpha \leq \tau) = (\sigma \leq \alpha-1) \cap (\tau \leq \alpha-1)^c
				\end{align}
				より$F_\alpha$は可測$\mathcal{F}_{\alpha-1}/\borel{\R}$となる.$F_\alpha$を用いて
				\begin{align}
					N_\beta \coloneqq \sum_{\alpha=0}^{\beta-1} F_{\alpha+1}(M_{\alpha+1} - M_\alpha) \quad (\beta \in I)
				\end{align}
				として$(N_\beta)_{\beta \in I}$を定義すれば,これもまた$\mathrm{L}^p$-劣マルチンゲールとなる.今$I$は有限集合であるから
				定義\ref{dfn:martingale}の条件(M.1)(M.2)を満たすことを確認すればよい.
				\begin{description}
					\item[(M.1)] 先ず$N_\beta$が可測$\mathcal{F}_\beta/\borel{\R}$であることを示す.
						$N_\beta$を構成する級数の項のうち最も可測性が厳しいものは最終項$F_{\beta}(M_{\beta} - M_{\beta-1})$であり,
						$M_\beta$も$F_{\beta}$も可測$\mathcal{F}_\beta/\borel{\R}$であるから$N_\beta$の可測性も判明する.可積分性については,
						$M_\alpha\ (\alpha \in I)$が$p$乗可積分であるからその有限個の結合で表現される$(N_\beta)_{\beta \in I}$もまた$p$乗可積分となる.
					\item[(M.2)]	
						$\alpha \leq \beta\ (\alpha,\beta \in I)$に対して
						\begin{align}
							N_{\beta} - N_{\alpha} = \sum_{\gamma=\alpha}^{\beta-1} F_{\gamma+1}(M_{\gamma+1} - M_\gamma)
						\end{align}
						と表せるから,(関数の同値類を同様に表記して)$\mathcal{F}_{\alpha}$で条件付ければ,
						性質$\tilde{\mathrm{C}}$5,$\tilde{\mathrm{C}}$6と$F_\alpha$の可測性事情,そして$(M_\alpha)_{\alpha \in I}$が劣マルチンゲールであることにより
						\begin{align}
							\cexp{N_{\beta} - N_{\alpha}}{\mathcal{F}_\alpha}
							&= \sum_{\gamma=\alpha}^{\beta-1} \cexp{F_{\gamma+1}(M_{\gamma+1} - M_\gamma)}{\mathcal{F}_\alpha} \\
							&= \sum_{\gamma=\alpha}^{\beta-1} \cexp{\cexp{F_{\gamma+1}(M_{\gamma+1} - M_\gamma)}{\mathcal{F}_\gamma}}{\mathcal{F}_\alpha} \\
							&= \sum_{\gamma=\alpha}^{\beta-1} \cexp{F_{\gamma+1}\cexp{M_{\gamma+1} - M_\gamma}{\mathcal{F}_\gamma}}{\mathcal{F}_\alpha}
							\geq 0\quad (\mbox{P-a.s.})
						\end{align}
						が成り立つ\footnote{同値類ではなく代表元の関数と見做している.}.
						従って$\cexp{N_{\beta}}{\mathcal{F}_\alpha} \geq \cexp{N_{\alpha}}{\mathcal{F}_\alpha} = N_\alpha$
						\footnote{こちらは同値類に対する順序記号を使っている.等号は性質$\tilde{\mathrm{C}}$5による.}が成り立つ.
				\end{description}
				
		\end{description}
	\end{prf}
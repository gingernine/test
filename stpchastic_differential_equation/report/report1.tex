\section{10/4講義ノート}
\begin{qst}
係数体を$\K$,$\K = \R$或は$\K = \C$と考える.測度空間を$(X,\mathcal{F},m)$とし,
可測$\mathcal{F}/\borel{\K}$関数$f$に対して
\begin{align}
	\Norm{f}{\mathscr{L}^p} \coloneqq
	\begin{cases}
		\inf{}{\{\ r \in \R\quad |\quad |f(x)| \leq r,\ \mathrm{a.e.}x \in X\ \}} & (p = \infty) \\
		\left( \int_{X} |f(x)|^p\ m(dx) \right)^{\frac{1}{p}} & (0 < p < \infty)
	\end{cases}
\end{align}
と定め,
\begin{align}
	\semiLp{p}{X,\mathcal{F},m} \coloneqq \{\ f:X \rightarrow \K \quad |\quad f:\mbox{可測}\mathcal{F}/\borel{\K},\ \Norm{f}{\mathscr{L}^p} < \infty \ \} \quad (1 \leq p \leq \infty)
\end{align}
として空間$\semiLp{p}{X,\mathcal{F},m}$を定義する.この空間は$\K$上の線形空間となるが,そのことを保証するために
次の二つの不等式が成り立つことを証明する.
\begin{thm}[H\Ddot{o}lderの不等式]
	$1 \leq p, q \leq \infty$,$p + q = pq\ (p = \infty$なら$q = 1)$とする.このとき
	任意の可測$\mathcal{F}/\borel{\K}$関数$f,g$に対して次が成り立つ:
	\begin{align}
		\int_{X} |f(x)g(x)|\ m(dx) \leq \Norm{f}{\mathscr{L}^p} \Norm{g}{\mathscr{L}^q}. \label{ineq:holder}
	\end{align} 
\end{thm}
\begin{prf}
	まず次の補助定理を証明する.
	\begin{lem}
		$f \in \semiLp{\infty}{X,\mathcal{F},m}$ならば
		\begin{align}
			|f(x)| \leq \Norm{f}{\mathscr{L}^\infty} \quad (\mathrm{a.e.}x \in X).
		\end{align}
	\end{lem}
	\begin{prf}
		$\semiLp{\infty}{X,\mathcal{F},m}$の定義により,任意の実数$\alpha > \Norm{f}{\mathscr{L}^\infty}$に対して
		\begin{align}
			m(\{\ x \in X\quad |\quad |f(x)| > \alpha\ \}) = 0
		\end{align}
		である.これにより
		\begin{align}
			\{\ x \in X\quad |\quad |f(x)| > \Norm{f}{\mathscr{L}^\infty}\ \} = \bigcup_{n =1}^{\infty} \{\ x \in X\quad |\quad |f(x)| > \Norm{f}{\mathscr{L}^\infty} + 1/n\ \}
		\end{align}
		の右辺は$m$-零集合となり補題が証明された.
		\QED
	\end{prf}
	
	定理の証明に入る.\mbox{}\\
	\begin{description}
		\item[$p = \infty,\ q = 1$の場合]
			$\Norm{f}{\mathscr{L}^\infty} = \infty$又は$\Norm{g}{\mathscr{L}^1} = \infty$の場合は明らかに不等式(\refeq{ineq:holder})
			が成り立つから,$\Norm{f}{\mathscr{L}^\infty} < \infty$かつ$\Norm{g}{\mathscr{L}^1} < \infty$の場合を考える.
			補助定理により,或る$m$-零集合$A \in \mathcal{F}$を除いて$|f(x)| \leq \Norm{f}{\mathscr{L}^\infty}$が成り立つから,
			\begin{align}
				|f(x)g(x)| \leq \Norm{f}{\mathscr{L}^\infty}|g(x)| \quad (\forall x \in X \backslash A).
			\end{align}
			従って
			\begin{align}
				\int_{X} |f(x)g(x)|\ m(dx) = \int_{X \backslash A} |f(x)g(x)|\ m(dx) \leq \Norm{f}{\mathscr{L}^\infty} \int_{X \backslash A} |g(x)|\ m(dx) 
				= \Norm{f}{\mathscr{L}^\infty} \Norm{g}{\mathscr{L}^1}
			\end{align}
			となり不等式(\refeq{ineq:holder})が成り立つ.
		
		\item[$1 < p,q < \infty$の場合]
			$\Norm{f}{\mathscr{L}^p} = \infty$又は$\Norm{g}{\mathscr{L}^q} = \infty$の場合は明らかに不等式(\refeq{ineq:holder})
			が成り立つから,$\Norm{f}{\mathscr{L}^p} < \infty$かつ$\Norm{g}{\mathscr{L}^q} < \infty$の場合を考える.
			$\Norm{f}{\mathscr{L}^p} = 0$であるとすると
			\begin{align}
				B \coloneqq \{\ x \in X\quad |\quad |f(x)| > 0\ \}
			\end{align}
			は$m$-零集合となるから,
			\begin{align}
				\int_{X} |f(x)g(x)|\ m(dx) = \int_{B} |f(x)g(x)|\ m(dx) + \int_{X \backslash B} |f(x)g(x)|\ m(dx) = 0
			\end{align}
			となり不等式(\refeq{ineq:holder})が成り立つ.$\Norm{g}{\mathscr{L}^q} = 0$の場合も同じである.
			
			最後に$0 < \Norm{f}{\mathscr{L}^p},\Norm{g}{\mathscr{L}^q} < \infty$の場合を示す.
			$-\Log{t} \quad (t > 0)$は凸関数であるから,$1/p + 1/q = 1$に対して
			\begin{align}
				-\Log{\left( \frac{s}{p} + \frac{t}{q} \right)} \leq \frac{1}{p}(-\Log{s}) + \frac{1}{q}(-\Log{t}) \quad (\forall s,t > 0)
			\end{align}
			が成り立ち,従って
			\begin{align}
				s^{1/p}t^{1/q} \leq \frac{s}{p} + \frac{t}{q} \quad (\forall s,t > 0)
			\end{align}
			が成り立つ.この不等式を用いれば
			\begin{align}
				F(x) \coloneqq |f(x)|^p/ \Norm{f}{\mathscr{L}^p}^p,\quad G(x) \coloneqq |g(x)|^q/ \Norm{g}{\mathscr{L}^q}^q \quad (\forall x \in X)
			\end{align}
			とした$F,G$に対し
			\begin{align}
				F(x)^{1/p}G(x)^{1/q} \leq \frac{1}{p}F(x) + \frac{1}{q}G(x) \quad (\forall x \in X)
			\end{align}
			となり,両辺を積分して
			\begin{align}
				\int_{X} F(x)^{1/p}G(x)^{1/q}\ m(dx) &\leq \frac{1}{p} \int_{X} F(x)\ m(dx) + \frac{1}{q} \int_{X} G(x)\ m(dx) \\
				&= \frac{1}{p} \frac{1}{\Norm{f}{\mathscr{L}^p}^p} \int_{X} |f(x)|^p\ m(dx) + \frac{1}{q} \frac{1}{\Norm{g}{\mathscr{L}^q}^q} \int_{X} |g(x)|^q\ m(dx) \\
				&= \frac{1}{p} + \frac{1}{q} = 1
			\end{align}
			が成り立つ.最左辺と最右辺を比べて
			\begin{align}
				1 \geq \int_{X} F(x)^{1/p}G(x)^{1/q}\ m(dx) = \int_{X} \frac{|f(x)|}{\Norm{f}{\mathscr{L}^p}} \frac{|g(x)|}{\Norm{g}{\mathscr{L}^q}}\ m(dx)
			\end{align}
			から不等式
			\begin{align}
				\int_{X} |f(x)g(x)|\ m(dx) \leq \Norm{f}{\mathscr{L}^p}\Norm{g}{\mathscr{L}^q}
			\end{align}
			が示された.
			\QED
	\end{description}
\end{prf}

\begin{thm}[Minkowskiの不等式]
	$1 \leq p \leq \infty$とする.このとき
	任意の可測$\mathcal{F}/\borel{\K}$関数$f,g$に対して次が成り立つ:
	\begin{align}
		\Norm{f+g}{\mathscr{L}^p} \leq \Norm{f}{\mathscr{L}^p} + \Norm{g}{\mathscr{L}^p}. \label{ineq:minkowski}
	\end{align}
\end{thm}
\begin{prf}
	\begin{description}\mbox{}\\
		\item[$p = \infty$の場合]
			\begin{align}
				|f(x) + g(x)| \leq |f(x)| + |g(x)| \quad (\forall x \in X)
			\end{align}
			である.従って$\Norm{f}{\mathscr{L}^\infty} = \infty$又は$\Norm{g}{\mathscr{L}^\infty} = \infty$の場合に不等式
			(\refeq{ineq:minkowski})が成り立つことは明らかである.$\Norm{f}{\mathscr{L}^\infty} < \infty$かつ$\Norm{g}{\mathscr{L}^\infty} < \infty$
			の場合は
			\begin{align}
				C \coloneqq \{\ x \in X\quad |\quad |f(x)| > \Norm{f}{\mathscr{L}^\infty}\ \} \bigcup \{\ x \in X\quad |\quad |g(x)| > \Norm{g}{\mathscr{L}^\infty}\ \}
			\end{align}
			が$m$-零集合となり,$\Norm{\cdot}{\mathscr{L}^\infty}$の定義と
			\begin{align}
				|f(x) + g(x)| \leq \Norm{f}{\mathscr{L}^\infty} + \Norm{g}{\mathscr{L}^\infty} \quad (\forall x \in X \backslash C)
			\end{align}
			の関係により不等式(\refeq{ineq:minkowski})が成り立つ.
		
		\item[$p = 1$の場合]
			\begin{align}
				|f(x) + g(x)| \leq |f(x)| + |g(x)| \quad (\forall x \in X)
			\end{align}
			の両辺を積分することにより不等式(\refeq{ineq:minkowski})が成り立つ.
		
		\item[$1 < p < \infty$の場合]
			$p + q = pq$が成り立つように$q > 1$を取る.
			\begin{align}
				|f(x) + g(x)|^p = |f(x) + g(x)||f(x) + g(x)|^{p-1} \leq |f(x)||f(x) + g(x)|^{p-1} + |g(x)||f(x) + g(x)|^{p-1}
			\end{align}
			の両辺を積分すれば,H\Ddot{o}lderの不等式により
			\begin{align}
				\Norm{f+g}{\mathscr{L}^p}^p &= \int_{X} |f(x) + g(x)|^p\ m(dx) \\
				&\leq \int_{X} |f(x)||f(x) + g(x)|^{p-1}\ m(dx) + \int_{X} |g(x)||f(x) + g(x)|^{p-1}\ m(dx) \\
				&\leq \left( \int_{X} |f(x)|^p\ m(dx) \right)^{1/p} \left( \int_{X} |f(x) + g(x)|^{q(p-1)}\ m(dx) \right)^{1/q} \\
					&\qquad + \left( \int_{X} |g(x)|^p\ m(dx) \right)^{1/p} \left( \int_{X} |f(x) + g(x)|^{q(p-1)}\ m(dx) \right)^{1/q} \\
				&= \left( \int_{X} |f(x)|^p\ m(dx) \right)^{1/p} \left( \int_{X} |f(x) + g(x)|^p\ m(dx) \right)^{1/q} \\
					&\qquad + \left( \int_{X} |g(x)|^p\ m(dx) \right)^{1/p} \left( \int_{X} |f(x) + g(x)|^p\ m(dx) \right)^{1/q} \\
				&= \Norm{f}{\mathscr{L}^p}\Norm{f+g}{\mathscr{L}^p}^{p/q} + \Norm{g}{\mathscr{L}^p}\Norm{f+g}{\mathscr{L}^p}^{p/q} \\
				&= \Norm{f}{\mathscr{L}^p}\Norm{f+g}{\mathscr{L}^p}^{p-1} + \Norm{g}{\mathscr{L}^p}\Norm{f+g}{\mathscr{L}^p}^{p-1}
			\end{align}
			が成り立つ.$\Norm{f+g}{\mathscr{L}^p} = 0$の場合は明らかに不等式(\refeq{ineq:minkowski})が成り立つ.
			$\Norm{f+g}{\mathscr{L}^p} = \infty$の場合,
			\begin{align}
				|f(x) + g(x)| \leq |f(x)| + |g(x)| \leq 2 \max{}{(|f(x)|,|g(x)|)} \quad (\forall x \in X)
			\end{align}
			より
			\begin{align}
				|f(x) + g(x)|^p \leq 2^p \max{}{\left( |f(x)|^p,|g(x)|^p \right)} \leq 2^p \left( |f(x)|^p + |g(x)|^p \right) \quad (\forall x \in X)
			\end{align}
			から両辺を積分して
			\begin{align}
				\Norm{f+g}{\mathscr{L}^p}^p \leq 2^p \left( \Norm{f}{\mathscr{L}^p}^p + \Norm{g}{\mathscr{L}^p}^p \right)
			\end{align}
			という関係が出るから,上式右辺も$\infty$となり不等式(\refeq{ineq:minkowski})が成り立つ.
			$0 < \Norm{f+g}{\mathscr{L}^p} < \infty$の場合,$\Norm{f}{\mathscr{L}^p} + \Norm{g}{\mathscr{L}^p} = \infty$
			なら不等式(\refeq{ineq:minkowski})は明らかに成り立ち,$\Norm{f}{\mathscr{L}^p} + \Norm{g}{\mathscr{L}^p} < \infty$
			の場合は
			\begin{align}
				\Norm{f+g}{\mathscr{L}^p}^p \leq \Norm{f}{\mathscr{L}^p}\Norm{f+g}{\mathscr{L}^p}^{p-1} + \Norm{g}{\mathscr{L}^p}\Norm{f+g}{\mathscr{L}^p}^{p-1}
			\end{align}
			の両辺を$\Norm{f+g}{\mathscr{L}^p}^{p-1}$で割って不等式(\refeq{ineq:minkowski})が成り立つと判る.
			\QED
	\end{description}
\end{prf}

$\Norm{\cdot}{\mathscr{L}^p}$は線形空間$\semiLp{p}{X,\mathcal{F},m}$においてセミノルムとなる.
\begin{bcs}\mbox{}\\
	\begin{description}
	\item[正値性] これは明らかである.
	\item[同次性] 
		\begin{align}
			\left( \int_{X} |\alpha f(x)|^p\ m(dx) \right)^{1/p} = \left( |\alpha|^p \int_{X} |f(x)|^p\ m(dx) \right)^{1/p} 
			= |\alpha| \left( \int_{X} |f(x)|^p\ m(dx) \right)^{1/p} \quad (1 \leq p < \infty)
		\end{align}
		と
		\begin{align}
			\inf{}{\{\ r \in \R\quad |\quad |\alpha f(x)| \leq r,\ \mathrm{a.e.}x \in X\ \}} = |\alpha|\inf{}{\{\ r \in \R\quad |\quad |f(x)| \leq r,\ \mathrm{a.e.}x \in X\ \}}
		\end{align}
		により,任意の$\alpha \in \K$と任意の$f \in \semiLp{p}{X,\mathcal{F},m}\ (1 \leq p \leq \infty)$に対して
		\begin{align}
			\Norm{\alpha f}{\mathscr{L}^p} = |\alpha|\Norm{f}{\mathscr{L}^p}
		\end{align}
		が成り立つ.
	\item[三角不等式] Minkowskiの不等式による.
	\end{description}
	\QED
\end{bcs}
しかし$\Norm{\cdot}{\mathscr{L}^p}$は$\semiLp{p}{X,\mathcal{F},m}$のノルムとはならない.$\Norm{f}{\mathscr{L}^p} = 0$であっても
$f(x) = 0 \ (\forall x \in X)$とは限らず,$m$-零集合の上で
$1 \in \K$を取るような関数$g$でも$\Norm{g}{\mathscr{L}^p} = 0$を満たすからである.
ここで次のものを考える.可測関数の集合を
\begin{align}
	\mathcal{M} \coloneqq \{\ f:X \rightarrow \K\quad |\quad f:\mbox{可測}\mathcal{F}/\borel{\K} \}
\end{align}
と表すことにする.
\begin{align}
	f,g \in \mathcal{M},\quad f \sim g \DEF f(x) = g(x)\quad \mathrm{a.e.}x \in X
\end{align}
と定義した関係$\sim$は$\mathcal{M}$における同値関係となり,この関係で$\mathcal{M}$を割った商を$ M \coloneqq \mathcal{M}/\sim$と表す.
$M$の元を$[f]\ $($f$は同値類の代表元)と表し,$M$における加法とスカラ倍を次のように定義すれば$M$は$\K$上の線形空間となる:
\begin{align}
	&[f] + [g] \coloneqq [f+g] && (\forall [f],[g] \in M),\\
	&\alpha [f] \coloneqq [\alpha f] && (\forall [f] \in M,\ \alpha \in \K).
\end{align}
そしてこの表現はwell-definedである.つまり代表元に依らずに値がただ一つに定まる.
\begin{bcs}\mbox{}\\
	任意の$f' \in [f]$と$g' \in [g]$に対して,$[f'] = [f],\ [g'] = [g]$
	であるから
	\begin{align}
		[f + g] = [f' + g'],\quad [\alpha f'] = [\alpha f]
	\end{align}
	をいえばよい.
	\begin{align}
		(f \neq g) \coloneqq \{\ x \in X\quad |\quad f(x) \neq g(x)\ \}
	\end{align}
	と簡略した表記を使えば
	\begin{align}
		&(f+g \neq f'+g') \subset (f \neq f') \cup (g \neq g'), \\
		&(\alpha f \neq \alpha f') = (f \neq f')
	\end{align}
	であり,どちらも右辺は$m$-零集合であるから$[f + g] = [f' + g'],\ [\alpha f'] = [\alpha f]$である.
	\QED
\end{bcs}

次に商空間$M$におけるノルムを定義する.
\begin{align}
	\Norm{[f]}{\mathrm{L}^p} \coloneqq \Norm{f}{\mathscr{L}^p} \quad (1 \leq p \leq \infty)
\end{align}
として$\Norm{\cdot}{\mathrm{L}^p}$を定義すればこれはwell-definedである.つまり代表元に依らずに値がただ一つに定まる.
\begin{bcs}\mbox{}\\
	$f \in \semiLp{p}{X,\mathcal{F},m}$とし,任意に$g \in [f]$で$f \neq g$となるものを選ぶ.
	示すことは$\Norm{f}{\mathscr{L}^p}^p = \Norm{g}{\mathscr{L}^p}^p$が成り立つことである.
	\begin{align}
		A \coloneqq \{\ x \in X\quad |\quad f(x) \neq g(x)\ \} \quad \in \mathcal{F}
	\end{align}
	とおけば,$f,g$は同じ同値類の元同士であるから$m(A)=0$である.
	\begin{description}
		\item[$p = \infty$の場合]
			$A^c$の上で$f(x)=g(x)$となるから
			\begin{align}
				\left\{\ x \in X\quad |\quad |g(x)| > \Norm{f}{\mathscr{L}^\infty}\ \right\} 
				&\subset A + A^c \cap \left\{\ x \in X\quad |\quad |g(x)| > \Norm{f}{\mathscr{L}^\infty}\ \right\} \\
				&= A + A^c \cap \left\{\ x \in X\quad |\quad |f(x)| > \Norm{f}{\mathscr{L}^\infty}\ \right\} \\
				&\subset A + \left\{\ x \in X\quad |\quad |f(x)| > \Norm{f}{\mathscr{L}^\infty}\ \right\}
			\end{align}
			が成り立ち,最右辺は2項とも$m$-零集合であるから最左辺も$m$-零集合となる.すなわち$\Norm{g}{\mathscr{L}^\infty} \leq \Norm{f}{\mathscr{L}^\infty}$が示された.
			逆向きの不等号も同様に示されるから$\Norm{g}{\mathscr{L}^\infty} = \Norm{f}{\mathscr{L}^\infty}$となる.
		\item[$1 \leq p < \infty$の場合]
			$m(A)=0$により
			\begin{align}
				\Norm{f}{\mathscr{L}^p}^p = \int_X f(x)\ m(dx) = \int_{A^c} f(x)\ m(dx) = \int_{A^c} g(x)\ m(dx) = \int_X g(x)\ m(dx) = \Norm{g}{\mathscr{L}^p}^p
			\end{align}
			が成り立つ.
	\end{description}
	\QED
\end{bcs}

$\semiLp{p}{X,\mathcal{F},m}$に代わって
\begin{align}
	\Lp{p}{X,\mathcal{F},m} \coloneqq \{\ [f] \in M \quad |\quad \Norm{[f]}{\mathrm{L}^p} < \infty\ \} \quad (1 \leq p \leq \infty)
\end{align}
を定義すると,$\Lp{p}{X,\mathcal{F},m}$は$\Norm{\cdot}{\mathrm{L}^p}$をノルムとしてノルム空間となる.
\begin{bcs}\mbox{}\\
	任意の$[f],[g] \in \Lp{p}{X,\mathcal{F},m}$と$\alpha \in \K$に対して,
	$\Norm{[f]}{\mathrm{L}^p} \geq 0$であることは$\Norm{\cdot}{\mathscr{L}^p}$の正値性による.
	また関数が0でない$x \in X$の集合の測度が正となるとノルムは正となるから,
	$\Norm{[f]}{\mathrm{L}^p} = 0$であるなら$[f]$は零写像(これを0と表す)の同値類(線形空間の零元),
	つまり$[f] = [0]$である.逆に$[f] = [0]$なら$\Norm{[f]}{\mathrm{L}^p} = 0$である.
	$\Norm{\cdot}{\mathscr{L}^p}$の同次性とMinkowskiの不等式から
	\begin{align}
		&\Norm{\alpha[f]}{\mathrm{L}^p} = \Norm{[\alpha f]}{\mathrm{L}^p} = \Norm{\alpha f}{\mathscr{L}^p} = |\alpha|\Norm{f}{\mathscr{L}^p} = |\alpha|\Norm{[f]}{\mathrm{L}^p} \\
		&\Norm{[f] + [g]}{\mathrm{L}^p} = \Norm{[f + g]}{\mathrm{L}^p} = \Norm{f + g}{\mathscr{L}^p} \leq \Norm{f}{\mathscr{L}^p} + \Norm{g}{\mathscr{L}^p} = \Norm{[f]}{\mathrm{L}^p} + \Norm{[g]}{\mathrm{L}^p}
	\end{align}
	も成り立つ.以上より$\Norm{\cdot}{\mathrm{L}^p}$は$\Lp{p}{X,\mathcal{F},m}$におけるノルムとなる.
	\QED
\end{bcs}

\begin{prp}[$\mathrm{L}^p$の完備性]
	上で定義したノルム空間$\Lp{p}{X,\mathcal{F},m}$はBanach空間である.$(1 \leq p \leq \infty)$
\end{prp}
\begin{prf}
	随意に$\Lp{p}{X,\mathcal{F},m}$のCauchy列$[f_n] \in \Lp{p}{X,\mathcal{F},m}\ (n=1,2,3,\cdots)$を取る.
	Cauchy列であるから$1/2$に対して或る$N_1 \in \N$が取れて,$n>m \geq N_1$ならば
	$\Norm{[f_n]-[f_m]}{\mathrm{L}^p} = \Norm{[f_n - f_m]}{\mathrm{L}^p} < 1/2$となる.
	ここで$m = n_1$と表記することにする.
	同様に$1/2^2$に対して或る$N_2 \in \N\ (N_2 > N_1)$が取れて,$n'>m' \geq N_2$ならば
	$\Norm{[f_{n'} - f_{m'}]}{\mathrm{L}^p} < 1/2^2$となる.
	先ほどの$n$について,$n > N_2$となるように取れるからこれを$n = n_2$と表記し,更に$m' = n_2$ともしておく.今のところ
	\begin{align}
		\Norm{[f_{n_2} - f_{n_1}]}{\mathrm{L}^p} < 1/2
	\end{align}
	と表示できる.再び同様に$1/2^3$に対して或る$N_3 \in \N\ (N_3 > N_2)$が取れて,$n''>m'' \geq N_2$ならば
	$\Norm{[f_{n''} - f_{m''}]}{\mathrm{L}^p} < 1/2^3$となる.
	先ほどの$n'$について$n' > N_3$となるように取れるからこれを$n' = n_3$と表記し,更に$m'' = n_3$ともしておく.今までのところで
	\begin{align}
		&\Norm{[f_{n_2} - f_{n_1}]}{\mathrm{L}^p} < 1/2 \\
		&\Norm{[f_{n_3} - f_{n_2}]}{\mathrm{L}^p} < 1/2^2
	\end{align}
	が成り立っている.数学的帰納法により
	\begin{align}
		\Norm{[f_{n_{k+1}} - f_{n_k}]}{\mathrm{L}^p} < 1/2^k \quad (n_{k+1} > n_k,\ k=1,2,3,\cdots) \label{ineq:Lp_banach_2}
	\end{align}
	が成り立つように自然数の部分列$(n_k)_{k=1}^{\infty}$を取ることができる.
	\begin{description}
		\item[$p = \infty$の場合]\mbox{}\\
			$[f_n]$の代表元$f_n$について,
			\begin{align}
				A_n \coloneqq \left\{\ x \in X\quad |\quad |f_n(x)| > \Norm{f_n}{\mathscr{L}^\infty} \right\}
			\end{align}
			とおけばH\Ddot{o}lderの不等式の証明中の補助定理より$m(A_n) = 0$であり,
			\begin{align}
				A \coloneqq \bigcup_{n=1}^{\infty} A_n
			\end{align}
			として$m$-零集合を定め
			\begin{align}
				\hat{f}_n(x) =
				\begin{cases}
					f_n(x) & (x \notin A) \\
					0 & (x \in A)
				\end{cases}
				\quad (\forall x \in X)
			\end{align}
			と定義した$\hat{f}_n$もまた$[f_n]$の元となる.代表元を$f_n$に替えて$\hat{f}_n$とすれば,
			$\hat{f}_n$は$X$上の有界可測関数であり
			\begin{align}
				\Norm{[f_{n_{k+1}} - f_{n_k}]}{\mathrm{L}^\infty} = \Norm{[\hat{f}_{n_{k+1}} - \hat{f}_{n_k}]}{\mathrm{L}^\infty}
				= \Norm{\hat{f}_{n_{k+1}} - \hat{f}_{n_k}}{\mathscr{L}^\infty} < 1/2^k \quad (k=1,2,3,\cdots) \label{ineq:Lp_banach_1}
			\end{align}
			を満たしている.すなわち
			\begin{align}
				\sup{x \in X}{|\hat{f}_{n_{k+1}}(x) - \hat{f}_{n_k}(x)|} < 1/2^k \quad (k=1,2,3,\cdots) 
			\end{align}
			が成り立っていることになるから,各点$x \in X$で$\left( \hat{f}_{n_k}(x) \right)_{k=1}^{\infty}$は$\K$のCauchy列となる.
			(これは$\sum_{k > N} 1/2^k = 1/2^N \longrightarrow 0\ (N \longrightarrow \infty)$による.)従って各点$x \in X$
			で極限が存在するからこれを$\hat{f}(x)$として表す.一般に距離空間に値を取る可測関数列の各点収束の極限関数は可測関数であるから
			$\hat{f}$もまた可測$\mathcal{F}/\borel{\K}$である.また$\hat{f}$は有界である.これは次のように示される.
			式(\refeq{ineq:Lp_banach_1})から
			\begin{align}
				\left|\ \Norm{\hat{f}_{n_{k+1}}}{\mathscr{L}^\infty} - \Norm{\hat{f}_{n_k}}{\mathscr{L}^\infty} \right| 
				\leq \Norm{\hat{f}_{n_{k+1}} - \hat{f}_{n_k}}{\mathscr{L}^\infty} < 1/2^k \quad (k=1,2,3,\cdots)
			\end{align}
			となるから,$\left( \Norm{\hat{f}_{n_k}}{\mathscr{L}^\infty} \right)_{k=1}^{\infty}$は実数のCauchy列となり有界となる.
			\begin{align}
				\sup{k \in \N}{\Norm{\hat{f}_{n_k}}{\mathscr{L}^\infty}} < \alpha
			\end{align}
			となるように実数$\alpha > 0$を取れば,各点$x \in X$で収束数列$\left( \hat{f}_{n_k}(x) \right)_{k=1}^{\infty}$は
			開区間$(-\alpha,\alpha)$の範囲内に収まっているのだから,その収束先は$\hat{f}(x) \in (-\alpha,\alpha)$であり$\hat{f}$が$X$上で有界であると判る.
			以上で$\hat{f}$が有界可測関数であることが示された.
			$\hat{f}$を代表元とする$[\hat{f}] \in \Lp{\infty}{X,\mathcal{F},m}$に対し
			\begin{align}
				\Norm{[f_{n_k}] - [\hat{f}]}{\mathrm{L}^\infty} = \Norm{\hat{f}_{n_k} - \hat{f}}{\mathscr{L}^\infty} = \sup{x \in X}{|\hat{f}_{n_k}(x) - \hat{f}(x)|}
				\longrightarrow 0 \ (k \longrightarrow \infty)
			\end{align}
			が成り立つから,Cauchy列$\left( [f_{n_k}] \right)_{k=1}^{\infty}$が$[\hat{f}]$に収束すると示されて
			$\Lp{\infty}{X,\mathcal{F},m}$がBanach空間であると判明した.
			
		\item[$1 \leq p < \infty$の場合]\mbox{}\\
			$k=1,2,3,\cdots$に対して
			\begin{align}	
				f_{n_k}(x) &\coloneqq f_{n_1}(x) + \sum_{j=1}^{k}(f_{n_j}(x) - f_{n_{j-1}}(x)) \\
			\end{align}
			と表現できるから,これに対して
			\begin{align}
				g_k(x) &\coloneqq |f_{n_1}(x)| + \sum_{j=1}^{k}|f_{n_j}(x) - f_{n_{j-1}}(x)|
			\end{align}
			として可測関数列$(g_k)_{k=1}^{\infty}$を用意する.Minkowskiの不等式と式(\refeq{ineq:Lp_banach_2})より
			\begin{align}
				\Norm{g_k}{\mathscr{L}^p} \leq \Norm{f_{n_1}}{\mathscr{L}^p} + \sum_{j=1}^{k}\Norm{f_{n_j} - f_{n_{j-1}}}{\mathscr{L}^p}
				< \Norm{f_{n_1}}{\mathscr{L}^p} + \sum_{j=1}^{k} 1/2^j < \Norm{f_{n_1}}{\mathscr{L}^p} + 1 < \infty
			\end{align}
			が成り立つ.各点$x \in X$で$g_k(x)$は$k$について単調増大であるから,単調収束定理より
			\begin{align}
				\Norm{g}{\mathscr{L}^p}^p = \lim_{k \to \infty} \Norm{g_k}{\mathscr{L}^p}^p < \Norm{f_{n_1}}{\mathscr{L}^p} + 1 < \infty
			\end{align}
			となるので$g \in \Lp{p}{X,\mathcal{F},m}$であると判る.
	\end{description}
\end{prf}

\end{qst}
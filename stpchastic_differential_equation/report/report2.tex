\section{10/11}
	基礎におく確率空間を$(\Omega,\mathcal{F},\mu)$とする.係数体を$\R$として考えると,
	ノルム空間$\Lp{2}{\Omega,\mathcal{F},\mu}$は
	\begin{align}
		\inprod<[f],[g]>_{\Lp{2}{\mathcal{F},\mu}} \coloneqq \int_{\Omega} f(x)g(x)\ \mu(dx) \quad \left( [f],[g] \in \Lp{2}{\Omega,\mathcal{F},\mu} \right)
	\end{align}
	を内積としてHilbert空間となる.これは次のように示される.まず左辺は代表元の選び方に依らない.任意に$f' \in [f]$と$g' \in [g]$を取っても,
	\begin{align}
		E &\coloneqq \{\ x \in \Omega\quad |\quad f(x) \neq f'(x)\ \}, \\
		F &\coloneqq \{\ x \in \Omega\quad |\quad g(x) \neq g'(x)\ \}
	\end{align}
	とした$E,F$は$\mu$-零集合であって
	\begin{align}
		\int_{\Omega} f(x)g(x)\ \mu(dx) = \int_{\Omega \backslash (E \cup F)} f(x)g(x)\ \mu(dx)
		= \int_{\Omega \backslash (E \cup F)} f'(x)g'(x)\ \mu(dx) = \int_{\Omega} f'(x)g'(x)\ \mu(dx)
	\end{align}
	が成り立つからである.二乗可積分な関数を扱っているから上式中の積分は全て有限確定であり,つまり$\inprod<\cdot,\cdot>_{\Lp{2}{\mathcal{F},\mu}}$
	が実数値として確定している.また内積の公理を満たすことも次のように示される.
	\begin{description}
		\item[正値性] 
			任意の$[f] \in \Lp{2}{\Omega,\mathcal{F},\mu}$に対して,ノルム$\Norm{[f]}{\Lp{2}{\mathcal{F},\mu}}$との対応から
			$\inprod<[f],[f]>_{\Lp{2}{\mathcal{F},\mu}} \geq 0$と
			$\inprod<[f],[f]>_{\Lp{2}{\mathcal{F},\mu}} = 0 \Leftrightarrow [f] = [0]$が成り立つ.
		\item[対称性] 
			任意の$[f],[g] \in \Lp{2}{\Omega,\mathcal{F},\mu}$に対して
			\begin{align}
				\inprod<[f],[g]>_{\Lp{2}{\mathcal{F},\mu}} = \int_{\Omega} f(x)g(x)\ \mu(dx)
				= \int_{\Omega} g(x)f(x)\ \mu(dx) = \inprod<[g],[f]>_{\Lp{2}{\mathcal{F},\mu}}
			\end{align}
			が成り立つ.
		\item[双線型性] 
			任意の$[f],[g],[h] \in \Lp{2}{\Omega,\mathcal{F},\mu}$と$a \in \R$に対して
			\begin{align}
				\inprod<[f],[g] + [h]>_{\Lp{2}{\mathcal{F},\mu}} &= \inprod<[f],[g + h]>_{\Lp{2}{\mathcal{F},\mu}} \\
				&= \int_{\Omega} f(x)(g(x) + h(x))\ \mu(dx) \\
				&= \int_{\Omega} f(x)g(x)\ \mu(dx) + \int_{\Omega} f(x)h(x)\ \mu(dx) && (\scriptsize\because \mbox{$fg$も$fh$も可積分である.}) \\
				&= \inprod<[f],[g]>_{\Lp{2}{\mathcal{F},\mu}} + \inprod<[f],[h]>_{\Lp{2}{\mathcal{F},\mu}}, \\
				\inprod<\alpha [f],[g]>_{\Lp{2}{\mathcal{F},\mu}} 
				&= \inprod<[\alpha f],[g]>_{\Lp{2}{\mathcal{F},\mu}} \\
				&= \int_{\Omega} \alpha f(x)g(x)\ \mu(dx) \\
				&= \alpha \int_{\Omega} f(x)g(x)\ \mu(dx) \\
				&= \alpha \inprod<[f],[g]>_{\Lp{2}{\mathcal{F},\mu}}
			\end{align}
			が成り立つことと対称性による.
	\end{description}
	そしてノルム空間としての完備性から$\Lp{2}{\Omega,\mathcal{F},\mu}$はHilbert空間となる.
	
	$\mathcal{G} \subset \mathcal{F}$を部分$\sigma$-加法族として
	別のHilbert空間$\Lp{2}{\Omega, \mathcal{G},\mu}$を考えれば,
	任意の$\inprod<g> \in \Lp{2}{\Omega, \mathcal{G},\mu}\ $(空間が違うことを意識するために元の表示を変えた)に対し$g$は可測$\mathcal{F}/\borel{\R}$
	であるから,対応する$\Lp{2}{\Omega, \mathcal{F},\mu}$の元$[g]$が存在する.$\mathcal{G} \subset \mathcal{F}$より
	$\inprod<g> \subset [g]$であって必ずしも$\inprod<g> = [g]$ではないが,単射
	\begin{align}
		\Lp{2}{\Omega, \mathcal{G},\mu} \ni \inprod<g> \longmapsto [g] \in \Lp{2}{\Omega, \mathcal{F},\mu}
	\end{align}
	によって$\Lp{2}{\Omega, \mathcal{G},\mu}$は$\Lp{2}{\Omega, \mathcal{F},\mu}$に等長に埋め込まれ,
	$\Lp{2}{\Omega, \mathcal{G},\mu}$の完備性から埋め込まれた部分集合は$\Lp{2}{\Omega, \mathcal{F},\mu}$の閉部分空間となる.
	この部分集合を$\Lp{2}{\Omega, \mathcal{G},\mu}$と同一視し,
	\begin{align}
		\Lp{2}{\Omega, \mathcal{G},\mu} \subset \Lp{2}{\Omega, \mathcal{F},\mu}
	\end{align}
	と考えることにする.以上の準備の下,以降では同値類と関数は表記上で区別することはせず,
	$f \in \Lp{2}{\Omega, \mathcal{F},\mu}$と表現することで,この$f$に同値類$[f]$としての意味と,代表元の関数$f$としての意味の両方を持たせる.
	射影定理により一意に定まる射影$g \in \Lp{2}{\Omega, \mathcal{G},\mu}$を
	\begin{align}
		g = \cexp{f}{\mathcal{G}}
	\end{align}
	と表現する.$\mathcal{G} = \{\emptyset, \Omega\}$のときは$\cexp{f}{\mathcal{G}}$を$\Exp{f}$と書いて$f$の期待値と呼ぶ.
	\begin{qst}\mbox{}\\
		Hilbert空間$\Lp{2}{\Omega, \mathcal{F},\mu}$における内積を$\inprod<\cdot,\cdot>_{\Lp{2}{\mathcal{F},\mu}}$,ノルムを$\Norm{\cdot}{\Lp{2}{\mathcal{F},\mu}}$と表示する.
		次のC1 $\sim$ C6 を示せ.
		\begin{description}
			\item[C1] $\forall f \in \Lp{2}{\Omega, \mathcal{F},\mu}$
				\begin{align}
					\Exp{f} = \int_{\Omega} f(x)\ \mu(dx)
				\end{align}
				
			\item[C2]	$\forall f \in \Lp{2}{\Omega, \mathcal{F},\mu},\ \forall h \in \Lp{2}{\Omega, \mathcal{G},\mu}$
				\begin{align}
					\int_{\Omega} f(x)h(x)\ \mu(dx) = \int_{\Omega} \cexp{f}{\mathcal{G}}(x)h(x)\ \mu(dx)
				\end{align}
				
			\item[C3]	$\forall f_1,f_2 \in \Lp{2}{\Omega, \mathcal{F},\mu}$
				\begin{align}
					\cexp{f_1 + f_2}{\mathcal{G}} = \cexp{f_1}{\mathcal{G}} + \cexp{f_2}{\mathcal{G}}
				\end{align}

			\item[C4]	$\forall f_1,f_2 \in \Lp{2}{\Omega, \mathcal{F},\mu}$
				\begin{align}
					f_1 \leq f_2 \quad \mathrm{a.s.} \quad \Rightarrow \quad \cexp{f_1}{\mathcal{G}} \leq \cexp{f_2}{\mathcal{G}} \quad \mathrm{a.s.}
				\end{align}
			
			\item[C5]	$\forall f \in \Lp{2}{\Omega, \mathcal{F},\mu},\ \forall g \in \Lp{\infty}{\Omega, \mathcal{G},\mu}$
				\begin{align}
					\cexp{gf}{\mathcal{G}} = g\cexp{f}{\mathcal{G}}
				\end{align}
			
			\item[C6]	$\mathcal{H}$が$\mathcal{G}$の部分$\sigma$-加法族ならば$\forall f \in \Lp{2}{\Omega, \mathcal{F},\mu}$
				\begin{align}
					\cexp{\cexp{f}{\mathcal{G}}}{\mathcal{H}} = \cexp{f}{\mathcal{H}}
				\end{align}
		\end{description}
	\end{qst}
	
	\begin{prf}
		\begin{description}
			\item[C1] $\mathcal{G} = \{\emptyset, \Omega\}$とすれば,
				$\Lp{2}{\Omega, \mathcal{G},\mu}$の元は$\mathcal{G}$-可測でなくてはならないから$\Omega$上の定数関数である.
				従って各$g \in \Lp{2}{\Omega, \mathcal{G},\mu}$には定数$\alpha \in \R$が対応して$g(x)=\alpha\ (\forall x \in \Omega)$と表せる.
				射影定理より任意の$f \in \Lp{2}{\Omega, \mathcal{F},\mu}$の$\Lp{2}{\Omega, \mathcal{G},\mu}$への射影$\cexp{f}{\mathcal{G}} = \Exp{f}$は
				ノルム$\Norm{f-g}{\Lp{2}{\mathcal{F},\mu}}$を最小にする$g \in \Lp{2}{\Omega, \mathcal{G},\mu}$である.
				$g(x)=\alpha\ (\forall x \in \Omega)$としてノルムを直接計算すれば,
				\begin{align}
					\Norm{f-g}{\Lp{2}{\mathcal{F},\mu}}^2 &= \int_{\Omega} |f(x) - \alpha|^2\ \mu(dx) \\
					&= \int_{\Omega} |f(x)|^2 - 2 \alpha f(x) + |\alpha|^2\ \mu(dx) \\
					&= \int_{\Omega} |f(x)|^2\ \mu(dx) - 2 \alpha \int_{\Omega} f(x)\ \mu(dx) + |\alpha|^2 \\
					&= \left| \alpha - \int_{\Omega} f(x)\ \mu(dx) \right|^2 - \left| \int_{\Omega} f(x)\ \mu(dx) \right|^2 + \int_{\Omega} |f(x)|^2\ \mu(dx) \\
					&= \left| \alpha - \int_{\Omega} f(x)\ \mu(dx) \right|^2 + \int_{\Omega} \left| f(x) - \beta \right|^2\ \mu(dx) \qquad (\beta \coloneqq \int_{\Omega} f(x)\ \mu(dx))
				\end{align}
				と表現できて最終式は$\alpha = \int_{\Omega} f(x)\ \mu(dx)$で最小となる.すなわち
				\begin{align}
					\Exp{f} = \cexp{f}{\mathcal{G}} = \int_{\Omega} f(x)\ \mu(dx).
				\end{align}
			
			\item[C2] 
				射影定理により,$f \in \Lp{2}{\Omega, \mathcal{F},\mu}$の$\Lp{2}{\Omega, \mathcal{G},\mu}$への射影$\cexp{f}{\mathcal{G}}$は
				\begin{align}
					\inprod<f - \cexp{f}{\mathcal{G}}, h>_{\Lp{2}{\mathcal{F},\mu}} = 0 \quad (\forall h \in \Lp{2}{\Omega, \mathcal{G},\mu})
				\end{align}
				を満たし,内積の線型性から
				\begin{align}
					\inprod<f, h>_{\Lp{2}{\mathcal{F},\mu}} = \inprod<\cexp{f}{\mathcal{G}}, h>_{\Lp{2}{\mathcal{F},\mu}} \quad (\forall h \in \Lp{2}{\Omega, \mathcal{G},\mu})
				\end{align}
				が成り立つ.積分の形式で表示することにより
				\begin{align}
					\int_{\Omega} f(x)h(x)\ \mu(dx) = \int_{\Omega} \cexp{f}{\mathcal{G}}(x)h(x)\ \mu(dx) \quad (\forall h \in \Lp{2}{\Omega, \mathcal{G},\mu})
				\end{align}
				が示された.
				
			\item[C3] 
				射影定理により任意の$h \in \Lp{2}{\Omega, \mathcal{G},\mu}$に対して
				\begin{align}
					&\inprod<(f_1 + f_2) - \cexp{f_1 + f_2}{\mathcal{G}}, h>_{\Lp{2}{\mathcal{F},\mu}} = 0, \\
					&\inprod<f_1 - \cexp{f_1}{\mathcal{G}}, h>_{\Lp{2}{\mathcal{F},\mu}} = 0, \\
					&\inprod<f_2 - \cexp{f_2}{\mathcal{G}}, h>_{\Lp{2}{\mathcal{F},\mu}} = 0
				\end{align}
				が成り立っている.従って任意の$h \in \Lp{2}{\Omega, \mathcal{G},\mu}$に対して
				\begin{align}
					0 &= \inprod<(f_1 + f_2) - \cexp{f_1 + f_2}{\mathcal{G}}, h>_{\Lp{2}{\mathcal{F},\mu}} - \inprod<f_1 - \cexp{f_1}{\mathcal{G}}, h>_{\Lp{2}{\mathcal{F},\mu}} - \inprod<f_2 - \cexp{f_2}{\mathcal{G}}, h>_{\Lp{2}{\mathcal{F},\mu}} \\
					&= \inprod<(f_1 + f_2) - \cexp{f_1 + f_2}{\mathcal{G}}, h>_{\Lp{2}{\mathcal{F},\mu}} - \inprod<(f_1 + f_2) - (\cexp{f_1}{\mathcal{G}} + \cexp{f_2}{\mathcal{G}}), h>_{\Lp{2}{\mathcal{F},\mu}} \\
					&= \inprod<\cexp{f_1}{\mathcal{G}} + \cexp{f_2}{\mathcal{G}} - \cexp{f_1 + f_2}{\mathcal{G}}, h>_{\Lp{2}{\mathcal{F},\mu}}
				\end{align}
				となり,特に$h = \cexp{f_1}{\mathcal{G}} + \cexp{f_2}{\mathcal{G}} - \cexp{f_1 + f_2}{\mathcal{G}} \in \Lp{2}{\Omega, \mathcal{G},\mu}$とすれば
				\begin{align}
					\Norm{\cexp{f_1}{\mathcal{G}} + \cexp{f_2}{\mathcal{G}} - \cexp{f_1 + f_2}{\mathcal{G}}}{\Lp{2}{\mathcal{F},\mu}}^2 = 0
				\end{align}
				が成り立つことになるから
				\begin{align}
					\cexp{f_1}{\mathcal{G}} + \cexp{f_2}{\mathcal{G}} = \cexp{f_1 + f_2}{\mathcal{G}}
				\end{align}
				が示された.
				
			\item[C4] 「任意の$f \in \Lp{2}{\Omega, \mathcal{F},\mu}$に対して,$f \geq 0\ $a.s.ならば$\cexp{f}{\mathcal{G}} \geq 0\ $a.s.」---(※)を示せばよい.
				これが示されれば$f_1,f_2 \in \Lp{2}{\Omega, \mathcal{F},\mu}$で$f_1 \leq f_2\ $a.s.となるものに対し
				\begin{align}
					0 \leq f_2 - f_1\ \mathrm{a.s.} \quad \Rightarrow 0 \leq \cexp{f_2 - f_1}{\mathcal{G}} = \cexp{f_2}{\mathcal{G}} - \cexp{f_1}{\mathcal{G}}\ \mathrm{a.s.}
				\end{align}
				が成り立つ.しかし,この場合本題に入る前に次の命題を証明する必要がある.これは等号$\cexp{f_2 - f_1}{\mathcal{G}} = \cexp{f_2}{\mathcal{G}} - \cexp{f_1}{\mathcal{G}}$が成り立つことを保証するためである.
				\begin{prp}
					考えている空間は今までと同じHilbert空間$\Lp{2}{\Omega, \mathcal{F},\mu},\ \Lp{2}{\Omega, \mathcal{G},\mu}$である.
					任意の実数$\alpha$と任意の$f \in \Lp{2}{\Omega, \mathcal{F},\mu}$に対して次が成立する:
					\begin{align}
						\cexp{\alpha f}{\mathcal{G}} = \alpha \cexp{f}{\mathcal{G}}.
					\end{align}
				\end{prp}
				\begin{prf}
					射影定理より
					\begin{align}
						\inprod<f - \cexp{f}{\mathcal{G}}, h>_{\Lp{2}{\mathcal{F},\mu}} = 0, \quad \inprod<\alpha f - \cexp{\alpha f}{\mathcal{G}}, h>_{\Lp{2}{\mathcal{F},\mu}} = 0 \quad (\forall h \in \Lp{2}{\Omega, \mathcal{G},\mu})
					\end{align}
					が成り立っているから
					\begin{align}
						\inprod<\cexp{\alpha f}{\mathcal{G}} - \alpha \cexp{f}{\mathcal{G}}, h>_{\Lp{2}{\mathcal{F},\mu}} 
						&= \inprod<\cexp{\alpha f}{\mathcal{G}} - \alpha f, h>_{\Lp{2}{\mathcal{F},\mu}} - \inprod<\alpha \cexp{f}{\mathcal{G}} - \alpha f, h>_{\Lp{2}{\mathcal{F},\mu}} \\
						&= \inprod<\cexp{\alpha f}{\mathcal{G}} - \alpha f, h>_{\Lp{2}{\mathcal{F},\mu}} - \alpha \inprod<f - \cexp{f}{\mathcal{G}}, h>_{\Lp{2}{\mathcal{F},\mu}} \\
						&= 0. \quad (\forall h \in \Lp{2}{\Omega, \mathcal{G},\mu})
					\end{align}
					特に$h = \cexp{\alpha f}{\mathcal{G}} - \alpha \cexp{f}{\mathcal{G}} \in \Lp{2}{\Omega, \mathcal{G},\mu}$として
					\begin{align}
						\Norm{\cexp{\alpha f}{\mathcal{G}} - \alpha \cexp{f}{\mathcal{G}}}{\Lp{2}{\mathcal{F},\mu}}^2 = 0
					\end{align}
					だから$\cexp{\alpha f}{\mathcal{G}} = \alpha \cexp{f}{\mathcal{G}}$が成り立つ.
					\QED
				\end{prf}
				次に(※)を示す.
				\begin{align}
					A &\coloneqq \{x \in \Omega\ |\quad f(x) < 0 \} && (\in \mathcal{F}), \\
					B &\coloneqq \{x \in \Omega\ |\quad \cexp{f}{\mathcal{G}}(x) < 0 \} && (\in \mathcal{G})
				\end{align}
				として$\mu(A)=0 \quad \Rightarrow \quad \mu(B)=0$が成り立つと言えばよく,$\mu(A) = 0$の下で$\mu(B) > 0$
				と仮定しては不合理であることを以下に記述する.
				
				$\mu(A) = 0,\ \mu(B) > 0$であるとする.$\Lp{2}{\Omega, \mathcal{G},\mu}$の元を
				\begin{align}
					h(x) \coloneqq
					\begin{cases}
						\cexp{f}{\mathcal{G}}(x) & (x \in B^c) \\
						0 & (x \in B)
					\end{cases}
				\end{align}
				として定義すると
				\begin{align}
					\Norm{f - h}{\Lp{2}{\mathcal{F},\mu}}^2 &= \int_{\Omega} |f(x) - h(x)|^2\ \mu(dx) \\
					&= \int_{A^c \cap B^c} |f(x) - h(x)|^2\ \mu(dx) + \int_{A^c \cap B} |f(x) - h(x)|^2\ \mu(dx) \\
					&= \int_{A^c \cap B^c} |f(x) - \cexp{f}{\mathcal{G}}(x)|^2\ \mu(dx) + \int_{A^c \cap B} |f(x)|^2\ \mu(dx) \\
					&< \int_{A^c \cap B^c} |f(x) - \cexp{f}{\mathcal{G}}(x)|^2\ \mu(dx) + \int_{A^c \cap B} |f(x) - \cexp{f}{\mathcal{G}}(x)|^2\ \mu(dx) \\
					&= \Norm{f - \cexp{f}{\mathcal{G}}}{\Lp{2}{\mathcal{F},\mu}}^2
				\end{align}
				が成り立つ.これは$\mu(A) = 0$であること,$\mu(A^c \cap B) = \mu(B) - \mu(A \cap B) = \mu(B) > 0$であること,
				それから$A^c \cap B$の上で
				\begin{align}
					|f(x) - \cexp{f}{\mathcal{G}}(x)|^2 &- |f(x)|^2 = \left(f(x) - \cexp{f}{\mathcal{G}}(x) + f(x)\right)\left(- \cexp{f}{\mathcal{G}}(x)\right) > 0 \\
					&(\because f(x) \geq 0,\ \cexp{f}{\mathcal{G}}(x) < 0, \quad \forall x \in A^c \cap B)
				\end{align}
				が成り立っていることによる.上の結果,すなわち
				\begin{align}
					\Norm{f - h}{\Lp{2}{\mathcal{F},\mu}} < \Norm{f - \cexp{f}{\mathcal{G}}}{\Lp{2}{\mathcal{F},\mu}}
				\end{align}
				を満たす$h \in \Lp{2}{\Omega, \mathcal{G},\mu}$が存在することは
				$\cexp{f}{\mathcal{G}}$が$f$の射影であることに違反している.以上より$\mu(B) = 0$でなくてはならず,(※)が示された.
			
			\item[C5] $\Norm{\cexp{gf}{\mathcal{G}} - g\cexp{f}{\mathcal{G}}}{\Lp{2}{\mathcal{F},\mu}} = 0$
				が成り立つことを示す.任意の$h \in \Lp{2}{\Omega, \mathcal{G},\mu}$に対して
				\begin{align}
					\inprod<\cexp{gf}{\mathcal{G}} - g\cexp{f}{\mathcal{G}}, h>_{\Lp{2}{\mathcal{F},\mu}} 
					= \inprod<\cexp{gf}{\mathcal{G}} - gf, h>_{\Lp{2}{\mathcal{F},\mu}} + \inprod<gf - g\cexp{f}{\mathcal{G}}, h>_{\Lp{2}{\mathcal{F},\mu}}
				\end{align}
				を考えると,右辺が0になることが次のように証明される.先ず右辺第一項について,
				$gf$は$\Lp{2}{\Omega, \mathcal{F},\mu}$に入る.
				$g$は或る$\mu$-零集合$E \in \mathcal{G}$を除いて有界であるから,或る正数$\alpha$によって$|g(x)| \leq \alpha \ (\forall x \in E^c)$と抑えられ,
				\begin{align}
					\int_{\Omega} |g(x)f(x)|^2\ \mu(dx) = \int_{E^c} |g(x)|^2|f(x)|^2\ \mu(dx) \leq \alpha^2 \int_{E^c} |f(x)|^2\ \mu(dx) = \alpha^2 \int_{\Omega} |f(x)|^2\ \mu(dx) < \infty
				\end{align}
				が成り立つからである.従って射影定理により
				\begin{align}
					\inprod<\cexp{gf}{\mathcal{G}} - gf, h>_{\Lp{2}{\mathcal{F},\mu}} = 0 \quad (\forall h \in \Lp{2}{\Omega, \mathcal{G},\mu}).
				\end{align}
				右辺第二項について,
				\begin{align}
					\inprod<gf - g\cexp{f}{\mathcal{G}}, h>_{\Lp{2}{\mathcal{F},\mu}} = \int_{\Omega} \left( f(x) - \cexp{f}{\mathcal{G}}(x) \right) g(x)h(x)\ \mu(dx)
					= \inprod<f - \cexp{f}{\mathcal{G}}, gh>_{\Lp{2}{\mathcal{F},\mu}}
				\end{align}
				であって,先と同様の理由で$gh \in \Lp{2}{\Omega, \mathcal{G},\mu} \ (\forall h \in \Lp{2}{\Omega, \mathcal{G},\mu})$が成り立つから
				射影定理より
				\begin{align}
					\inprod<gf - g\cexp{f}{\mathcal{G}}, h>_{\Lp{2}{\mathcal{F},\mu}} = 0  \quad (\forall h \in \Lp{2}{\Omega, \mathcal{G},\mu})
				\end{align}
				であると判明した.始めの式に戻れば
				\begin{align}
					\inprod<\cexp{gf}{\mathcal{G}} - g\cexp{f}{\mathcal{G}}, h>_{\Lp{2}{\mathcal{F},\mu}} = 0  \quad (\forall h \in \Lp{2}{\Omega, \mathcal{G},\mu})
				\end{align}
				が成り立つことになり,特に$h = \cexp{gf}{\mathcal{G}} - g\cexp{f}{\mathcal{G}} \in \Lp{2}{\Omega, \mathcal{G},\mu}$に対しては
				\begin{align}
					\Norm{\cexp{gf}{\mathcal{G}} - g\cexp{f}{\mathcal{G}}}{\Lp{2}{\mathcal{F},\mu}}^2 = 0
				\end{align}
				となることから$\cexp{gf}{\mathcal{G}} = g\cexp{f}{\mathcal{G}}$が示された.
			
			\item[C6] 任意の$h \in \Lp{2}{\Omega, \mathcal{H},\mu} \subset \Lp{2}{\Omega, \mathcal{G},\mu}$に対し,射影定理より
				\begin{align}
					&\inprod<\cexp{\cexp{f}{\mathcal{G}}}{\mathcal{H}} - \cexp{f}{\mathcal{H}}, h>_{\Lp{2}{\mathcal{F},\mu}} \\
					&\qquad = \inprod<\cexp{\cexp{f}{\mathcal{G}}}{\mathcal{H}} - \cexp{f}{\mathcal{G}}, h>_{\Lp{2}{\mathcal{F},\mu}} \\
						&\qquad \qquad + \inprod<\cexp{f}{\mathcal{G}} - f, h>_{\Lp{2}{\mathcal{F},\mu}} + \inprod<f - \cexp{f}{\mathcal{H}}, h>_{\Lp{2}{\mathcal{F},\mu}}
					= 0
				\end{align}
				が成り立つ.特に$h = \cexp{\cexp{f}{\mathcal{G}}}{\mathcal{H}} - \cexp{f}{\mathcal{H}} \in \Lp{2}{\Omega, \mathcal{H},\mu}$とすれば
				\begin{align}
					\Norm{\cexp{\cexp{f}{\mathcal{G}}}{\mathcal{H}} - \cexp{f}{\mathcal{H}}}{\Lp{2}{\mathcal{F},\mu}}^2 = 0
				\end{align}
				ということになるので$\cexp{\cexp{f}{\mathcal{G}}}{\mathcal{H}} = \cexp{f}{\mathcal{H}}$であることが示された.
		\end{description}
		\QED
	\end{prf}
	
	レポート問題2[C3]と[C4]中の命題より,条件付き期待値が$\Lp{2}{\Omega,\mathcal{F},\mu}$からその部分空間($\Lp{2}{\Omega,\mathcal{G},\mu}$と同一視)への
	線型作用素であることが示された.H\Ddot{o}lderの不等式の不等式により$f \in \Lp{2}{\Omega,\mathcal{F},\mu}$は(代表元の関数が)可積分関数であるから
	$f \in \Lp{1}{\Omega,\mathcal{F},\mu}$であり,すなわち$\Lp{2}{\Omega,\mathcal{F},\mu}$は$\Lp{1}{\Omega,\mathcal{F},\mu}$の部分空間であると判る.
	同様に$\Lp{2}{\Omega,\mathcal{G},\mu}$は$\Lp{1}{\Omega,\mathcal{G},\mu}$の部分空間であるから,条件付き期待値は
	$\Lp{1}{\Omega,\mathcal{F},\mu}$から$\Lp{1}{\Omega,\mathcal{G},\mu}\ $(埋め込まれた部分空間と同一視している)への線型作用素と見ることができる.
	次に考えることは,線型作用素として見た条件付き期待値の拡張である.
	条件付き期待値が$\Lp{1}{\Omega,\mathcal{F},\mu}$から$\Lp{1}{\Omega,\mathcal{G},\mu}$への作用素として有界であり,更に
	定義域$\Lp{2}{\Omega,\mathcal{F},\mu}$が$\Lp{1}{\Omega,\mathcal{F},\mu}$において稠密であるならば拡張は可能となる.
	
	\begin{lem}[条件付き期待値の有界性]
		条件付き期待値について次が成り立つ:
		\begin{align}
			\sup{\substack{f \in \Lp{2}{\Omega,\mathcal{G},\mu} \\ f \neq 0}}{\frac{ \Norm{\cexp{f}{\mathcal{G}}}{\Lp{1}{\mathcal{F},\mu}} }{ \Norm{f}{\Lp{1}{\mathcal{F},\mu}} }} \leq 1.
		\end{align}
		\label{lem:conditional_exp_bound}
	\end{lem}
	\begin{prf}
		$f,\  \cexp{f}{\mathcal{G}}$を代表元の関数として扱えば次のように計算できる.
		\begin{align}
			\Norm{\cexp{f}{\mathcal{G}}}{\Lp{1}{\mathcal{F},\mu}} &= \int_\Omega \cexp{f}{\mathcal{G}}(\omega)\ \mu(dx) \\
			&= \int_\Omega \cexp{f}{\mathcal{G}}(\omega) \defunc_{\left(\cexp{f}{\mathcal{G}} \geq 0\right)}(\omega) 
				+ \cexp{f}{\mathcal{G}}(\omega) \defunc_{\left(\cexp{f}{\mathcal{G}} < 0\right)}(\omega)\ \mu(dx) \\
			&= \int_\Omega f(\omega) \defunc_{\left(\cexp{f}{\mathcal{G}} \geq 0\right)}(\omega) + f(\omega) \defunc_{\left(\cexp{f}{\mathcal{G}} < 0\right)}(\omega)\ \mu(dx) 
				&& (\scriptsize\because \mbox{レポート問題2[C2]}) \\
			&\leq \int_\Omega |f(\omega)| \defunc_{\left(\cexp{f}{\mathcal{G}} \geq 0\right)}(\omega) + |f(\omega)| \defunc_{\left(\cexp{f}{\mathcal{G}} < 0\right)}(\omega)\ \mu(dx) \\
			&= \Norm{f}{\Lp{1}{\mathcal{F},\mu}}.
		\end{align}
		\QED
	\end{prf}
	
	\begin{thm}[条件付き期待値の拡張]
		定義域を$\Lp{2}{\Omega,\mathcal{F},\mu}$としている条件付き期待値を,$\Lp{1}{\Omega,\mathcal{F},\mu}$を定義域とする有界線型作用素に
		(作用素ノルムを変えずに)一意に拡張することができる.つまりこの拡張された作用素を$\tcexp{\cdot}{\mathcal{G}}$と表示すれば
		\begin{align}
			\Lp{1}{\Omega,\mathcal{F},\mu} \ni f \longmapsto \tcexp{f}{\mathcal{G}} \in \Lp{1}{\Omega,\mathcal{G},\mu}
		\end{align}
		が有界な線型作用素となり,更にレポート問題2の[C1]$\sim$[C6]が$\mathrm{L}^2$を$\mathrm{L}^1$に置き換えて成り立つ.
	\end{thm}
	$\mathcal{G} = \{\emptyset,\ \Omega\}$である場合は$\tcexp{f}{\mathcal{G}} = \tExp{f}$と表示することにする.
	\begin{prf}	
		定理の主張する拡張が可能であることを示すには,$\Lp{2}{\Omega,\mathcal{F},\mu}$が$\Lp{1}{\Omega,\mathcal{F},\mu}$で稠密なことをいえばよい.
		任意の$f \in \Lp{1}{\Omega,\mathcal{F},\mu}$に対して,
		\begin{align}
			f_n(x) \coloneqq f(x) \defunc_{|f| \leq n} (x) \quad (\forall x \in \Omega,\ n=1,2,3,\cdots)
		\end{align}
		とおけば$(f_n)_{n=1}^{\infty} \subset \Lp{2}{\Omega,\mathcal{F},\mu}$となり,
		関数列として$f$に各点収束しているから,Lebesgueの収束定理より
		\begin{align}
			\lim_{n \to \infty} \Norm{f_n - f}{\Lp{1}{\mathcal{F},\mu}} = 0
		\end{align}
		が成り立つ.これで$\Lp{2}{\Omega,\mathcal{F},\mu}$が$\Lp{1}{\Omega,\mathcal{F},\mu}$で稠密であることが示された.
		次にレポート問題2の[C1]$\sim$[C6]が$\mathrm{L}^2$が$\mathrm{L}^1$に置き換えても成り立つことを示す.
		以下に主張を書き直す.
		\begin{description}
			\item[$\tilde{\mathrm{C}}$1] $\forall f \in \Lp{1}{\Omega, \mathcal{F},\mu}$
				\begin{align}
					\tExp{f} = \int_{\Omega} f(x)\ \mu(dx)
				\end{align}
				
			\item[$\tilde{\mathrm{C}}$2]	$\forall f \in \Lp{1}{\Omega, \mathcal{F},\mu},\ \forall h \in \Lp{\infty}{\Omega, \mathcal{G},\mu}$
				\begin{align}
					\int_{\Omega} f(x)h(x)\ \mu(dx) = \int_{\Omega} \tcexp{f}{\mathcal{G}}(x)h(x)\ \mu(dx)
				\end{align}
				
			\item[$\tilde{\mathrm{C}}$3]	$\forall f_1,f_2 \in \Lp{1}{\Omega, \mathcal{F},\mu}$
				\begin{align}
					\tcexp{f_1 + f_2}{\mathcal{G}} = \tcexp{f_1}{\mathcal{G}} + \tcexp{f_2}{\mathcal{G}}
				\end{align}

			\item[$\tilde{\mathrm{C}}$4]	$\forall f_1,f_2 \in \Lp{1}{\Omega, \mathcal{F},\mu}$
				\begin{align}
					f_1 \leq f_2 \quad \mathrm{a.s.} \quad \Rightarrow \quad \tcexp{f_1}{\mathcal{G}} \leq \tcexp{f_2}{\mathcal{G}} \quad \mathrm{a.s.}
				\end{align}
			
			\item[$\tilde{\mathrm{C}}$5]	$\forall f \in \Lp{1}{\Omega, \mathcal{F},\mu},\ \forall g \in \Lp{\infty}{\Omega, \mathcal{G},\mu}$
				\begin{align}
					\tcexp{gf}{\mathcal{G}} = g\tcexp{f}{\mathcal{G}}
				\end{align}
			
			\item[$\tilde{\mathrm{C}}$6]	$\mathcal{H}$が$\mathcal{G}$の部分$\sigma$-加法族ならば$\forall f \in \Lp{1}{\Omega, \mathcal{F},\mu}$
				\begin{align}
					\tcexp{\tcexp{f}{\mathcal{G}}}{\mathcal{H}} = \tcexp{f}{\mathcal{H}}
				\end{align}
		\end{description}
		一つ一つ証明していく.
		\begin{description}
			\item[$\tilde{\mathrm{C}}$1]
				$f$に対して先と同じ関数列$(f_n)_{n=1}^{\infty} \subset \Lp{2}{\Omega,\mathcal{F},\mu}$を作る.
				C1により全ての$n \in \N$に対して
				\begin{align}
					\tExp{f_n} = \int_{\Omega} f_n(x)\ \mu(dx)
				\end{align}
				が成り立っているから,Lebesgueの収束定理と作用素の有界性により
				\begin{align}
					\left| \tExp{f} - \int_{\Omega} f(x)\ \mu(dx) \right|
					\leq \left| \tExp{f} - \tExp{f_n} \right| + \left| \int_{\Omega} f_n(x)\ \mu(dx) - \int_{\Omega} f(x)\ \mu(dx) \right|
					\longrightarrow 0\ (n \longrightarrow \infty)
				\end{align}
				が成り立つ.ゆえに
				\begin{align}
					\tExp{f} = \int_{\Omega} f(x)\ \mu(dx)
				\end{align}
				が示された.
				
			\item[$\tilde{\mathrm{C}}$2]	
				$f$に対して先と同じ関数列$(f_n)_{n=1}^{\infty} \subset \Lp{2}{\Omega,\mathcal{F},\mu}$を作る.
				$h \in \Lp{\infty}{\Omega,\mathcal{F},\mu}$であることに注意すれば,
				C2により全ての$n \in \N$に対して
				\begin{align}
					\int_{\Omega} f_n(x)h(x)\ \mu(dx) = \int_{\Omega} \tcexp{f_n}{\mathcal{G}}(x)h(x)\ \mu(dx)
				\end{align}
				が成り立つ.
				\begin{align}
					A \coloneqq \left\{\ x \in \Omega\quad |\quad |h(x)| > \Norm{h}{\Lp{\infty}{\mathcal{F},\mu}} \right\}
				\end{align}
				とおけば$\mu(A) = 0$であり,拡張が作用素ノルムを変えないことと補助定理\ref{lem:conditional_exp_bound}の結果より
				\begin{align}
					&\left| \int_{\Omega} f(x)h(x)\ \mu(dx) - \int_{\Omega} \tcexp{f}{\mathcal{G}}(x)h(x)\ \mu(dx) \right| \\
					&\qquad \leq \left| \int_{\Omega} f(x)h(x)\ \mu(dx) - \int_{\Omega} f_n(x)h(x)\ \mu(dx) \right| \\
						&\qquad \qquad+ \left| \int_{\Omega} \tcexp{f_n}{\mathcal{G}}(x)h(x)\ \mu(dx) - \int_{\Omega} \tcexp{f}{\mathcal{G}}(x)h(x)\ \mu(dx) \right| \\
					%&\qquad = \left| \int_{\Omega \backslash A} f(x)h(x)\ \mu(dx) - \int_{\Omega \backslash A} f_n(x)h(x)\ \mu(dx) \right| 
					%	+ \left| \int_{\Omega \backslash A} \tcexp{f_n}{\mathcal{G}}(x)h(x)\ \mu(dx) - \int_{\Omega \backslash A} \tcexp{f}{\mathcal{G}}(x)h(x)\ \mu(dx) \right| \\
					&\qquad \leq \Norm{h}{\Lp{\infty}{\mathcal{F},\mu}} \int_{\Omega \backslash A} |f(x) - f_n(x)|\ \mu(dx) 
						+ \Norm{h}{\Lp{\infty}{\mathcal{F},\mu}} \int_{\Omega \backslash A} \left| \tcexp{f_n}{\mathcal{G}}(x) - \tcexp{f}{\mathcal{G}}(x) \right|\ \mu(dx) \\
					&\qquad = \Norm{h}{\Lp{\infty}{\mathcal{F},\mu}} \Norm{f - f_n}{\Lp{1}{\mathcal{F},\mu}}
						+ \Norm{h}{\Lp{\infty}{\mathcal{F},\mu}} \Norm{\tcexp{f}{\mathcal{G}} - \tcexp{f_n}{\mathcal{G}}}{\Lp{1}{\mathcal{F},\mu}} \\
					&\qquad \leq 2 \Norm{h}{\Lp{\infty}{\mathcal{F},\mu}} \Norm{f - f_n}{\Lp{1}{\mathcal{F},\mu}} \quad (\scriptsize\because \mbox{補助定理}\ref{lem:conditional_exp_bound})
				\end{align}
				が成り立つ.$(f_n)_{n=1}^{\infty}$の作り方からLebesgueの収束定理が適用されて
				\begin{align}
					\Norm{f - f_n}{\Lp{1}{\mathcal{F},\mu}} \longrightarrow 0\ (n \longrightarrow \infty)
				\end{align}
				となるから
				\begin{align}
					\int_{\Omega} f(x)h(x)\ \mu(dx) = \int_{\Omega} \tcexp{f}{\mathcal{G}}(x)h(x)\ \mu(dx)
				\end{align}
				が示された.
				
			\item[$\tilde{\mathrm{C}}$3]	
				作用素$\tcexp{\cdot}{\mathcal{G}}$の線型性による.

			\item[$\tilde{\mathrm{C}}$4]	$\forall f_1,f_2 \in \Lp{1}{\Omega, \mathcal{F},\mu}$
				\begin{align}
					f_1 \leq f_2 \quad \mathrm{a.s.} \quad \Rightarrow \quad \tcexp{f_1}{\mathcal{G}} \leq \tcexp{f_2}{\mathcal{G}} \quad \mathrm{a.s.}
				\end{align}
			
			\item[$\tilde{\mathrm{C}}$5]	$\forall f \in \Lp{1}{\Omega, \mathcal{F},\mu},\ \forall g \in \Lp{\infty}{\Omega, \mathcal{G},\mu}$
				\begin{align}
					\tcexp{gf}{\mathcal{G}} = g\tcexp{f}{\mathcal{G}}
				\end{align}
			
			\item[$\tilde{\mathrm{C}}$6]	$\mathcal{H}$が$\mathcal{G}$の部分$\sigma$-加法族ならば$\forall f \in \Lp{1}{\Omega, \mathcal{F},\mu}$
				\begin{align}
					\tcexp{\tcexp{f}{\mathcal{G}}}{\mathcal{H}} = \tcexp{f}{\mathcal{H}}
				\end{align}
		\end{description}
	\end{prf}
\section{10/11}
	基礎の確率空間を$(\Omega,\mathcal{F},\mu)$とする.
	$\mathcal{G} \subset \mathcal{F}$を部分$\sigma$-加法族とし,
	Hilbert空間$\Lp{2}{\Omega,\mathcal{F},\mu}$とその閉部分空間
	$\Lp{2}{\Omega, \mathcal{G},\mu}$を考える.
	任意の$f \in \Lp{2}{\Omega, \mathcal{F},\mu}$に対して,
	射影定理により一意に定まる射影$g \in \Lp{2}{\Omega, \mathcal{G},\mu}$を
	\begin{align}
		g = \cexp{f}{\mathcal{G}}
	\end{align}
	と表現する.$\mathcal{G} = \{\emptyset, \Omega\}$のときは$\cexp{f}{\mathcal{G}}$を$\Exp{f}$と書いて$f$の期待値と呼ぶ.
	\begin{qst}\mbox{}\\
		Hilbert空間$\Lp{2}{\Omega, \mathcal{F},\mu}$における内積を$\inprod<\cdot,\cdot>$,ノルムを$\Norm{\cdot}{\Lp{2}{\Omega, \mathcal{F},\mu}}$と表示する.
		次のC1 $\sim$ C6 を示せ.扱う関数は全て$\R$値と考える.(係数体を実数体としてHilbert空間を扱う.)
		\begin{description}
			\item[C1] $\forall f \in \Lp{2}{\Omega, \mathcal{F},\mu}$
				\begin{align}
					\Exp{f} = \int_{\Omega} f(x)\ \mu(dx)
				\end{align}
				
			\item[C2]	$\forall f \in \Lp{2}{\Omega, \mathcal{F},\mu},\ \forall h \in \Lp{2}{\Omega, \mathcal{G},\mu}$
				\begin{align}
					\int_{\Omega} f(x)h(x)\ \mu(dx) = \int_{\Omega} \cexp{f}{\mathcal{G}}(x)h(x)\ \mu(dx)
				\end{align}
				
			\item[C3]	$\forall f_1,f_2 \in \Lp{2}{\Omega, \mathcal{F},\mu}$
				\begin{align}
					\cexp{f_1 + f_2}{\mathcal{G}} = \cexp{f_1}{\mathcal{G}} + \cexp{f_2}{\mathcal{G}}
				\end{align}

			\item[C4]	$\forall f_1,f_2 \in \Lp{2}{\Omega, \mathcal{F},\mu}$
				\begin{align}
					f_1 \leq f_2 \quad \mathrm{a.s.} \quad \Rightarrow \quad \cexp{f_1}{\mathcal{G}} \leq \cexp{f_2}{\mathcal{G}} \quad \mathrm{a.s.}
				\end{align}
			
			\item[C5]	$\forall f \in \Lp{2}{\Omega, \mathcal{F},\mu},\ \forall g \in \Lp{\infty}{\Omega, \mathcal{G},\mu}$
				\begin{align}
					\cexp{gf}{\mathcal{G}} = g\cexp{f}{\mathcal{G}}
				\end{align}
			
			\item[C6]	$\mathcal{H}$が$\mathcal{G}$の部分$\sigma$-加法族ならば$\forall f \in \Lp{2}{\Omega, \mathcal{F},\mu}$
				\begin{align}
					\cexp{\cexp{f}{\mathcal{G}}}{\mathcal{H}} = \cexp{f}{\mathcal{H}}
				\end{align}
		\end{description}
	\end{qst}
	
	\begin{prf}
		\begin{description}
			\item[C1] $\mathcal{G} = \{\emptyset, \Omega\}$とすれば,
				$\Lp{2}{\Omega, \mathcal{G},\mu}$の元は$\mathcal{G}$-可測でなくてはならないから$\Omega$上の定数関数である.
				従って任意の$g \in \Lp{2}{\Omega, \mathcal{G},\mu}$に或る定数$\alpha \in \R$が対応して$g(x)=\alpha\ (\forall x \in \Omega)$と表せる.
				Hilbert空間$\Lp{2}{\Omega, \mathcal{F},\mu}$におけるノルムを$\Norm{\cdot}{\Lp{2}{\Omega, \mathcal{F},\mu}}$と表示すれば,
				射影定理より任意の$f \in \Lp{2}{\Omega, \mathcal{F},\mu}$の$\Lp{2}{\Omega, \mathcal{G},\mu}$への射影$\cexp{f}{\mathcal{G}} = \Exp{f}$は
				ノルム$\Norm{f-g}{\Lp{2}{\Omega, \mathcal{F},\mu}}$を最小にする$g \in \Lp{2}{\Omega, \mathcal{G},\mu}$である.
				$g(x)=\alpha\ (\forall x \in \Omega)$としてノルムを直接計算すれば,
				\begin{align}
					\Norm{f-g}{\Lp{2}{\Omega, \mathcal{F},\mu}}^2 &= \int_{\Omega} |f(x) - \alpha|^2\ \mu(dx) \\
					&= \int_{\Omega} |f(x)|^2 - 2 \alpha f(x) + |\alpha|^2\ \mu(dx) \\
					&= \int_{\Omega} |f(x)|^2\ \mu(dx) - 2 \alpha \int_{\Omega} f(x)\ \mu(dx) + |\alpha|^2 \\
					&= \left| \alpha - \int_{\Omega} f(x)\ \mu(dx) \right|^2 - \left| \int_{\Omega} f(x)\ \mu(dx) \right|^2 + \int_{\Omega} |f(x)|^2\ \mu(dx) \\
					&= \left| \alpha - \int_{\Omega} f(x)\ \mu(dx) \right|^2 + \int_{\Omega} \left| f(x) - \beta \right|^2\ \mu(dx) & (\beta \coloneqq \int_{\Omega} f(x)\ \mu(dx))
				\end{align}
				と表現できて最終式は$\alpha = \int_{\Omega} f(x)\ \mu(dx)$となることで最小となる.すなわち
				\begin{align}
					\Exp{f} = \cexp{f}{\mathcal{G}} = \int_{\Omega} f(x)\ \mu(dx).
				\end{align}
			
			\item[C2] 
				射影定理により,$f \in \Lp{2}{\Omega, \mathcal{F},\mu}$の$\Lp{2}{\Omega, \mathcal{G},\mu}$への射影$\cexp{f}{\mathcal{G}}$は
				\begin{align}
					\inprod<f - \cexp{f}{\mathcal{G}}, h> = 0 \quad (\forall h \in \Lp{2}{\Omega, \mathcal{G},\mu})
				\end{align}
				を満たし,内積の線型性から
				\begin{align}
					\inprod<f, h> = \inprod<\cexp{f}{\mathcal{G}}, h> \quad (\forall h \in \Lp{2}{\Omega, \mathcal{G},\mu})
				\end{align}
				が成り立つ.積分の形式で表示することにより
				\begin{align}
					\int_{\Omega} f(x)h(x)\ \mu(dx) = \int_{\Omega} \cexp{f}{\mathcal{G}}(x)h(x)\ \mu(dx) \quad (\forall h \in \Lp{2}{\Omega, \mathcal{G},\mu})
				\end{align}
				が示された.
				
			\item[C3] 
				射影定理により任意の$h \in \Lp{2}{\Omega, \mathcal{G},\mu}$に対して
				\begin{align}
					\inprod<(f_1 + f_2) - \cexp{f_1 + f_2}{\mathcal{G}}, h> = 0,\ \inprod<f_1 - \cexp{f_1}{\mathcal{G}}, h> = 0,\ \inprod<f_2 - \cexp{f_2}{\mathcal{G}}, h> = 0
				\end{align}
				が成り立っている.従って任意の$h \in \Lp{2}{\Omega, \mathcal{G},\mu}$に対して
				\begin{align}
					0 &= \inprod<(f_1 + f_2) - \cexp{f_1 + f_2}{\mathcal{G}}, h> - \inprod<f_1 - \cexp{f_1}{\mathcal{G}}, h> - \inprod<f_2 - \cexp{f_2}{\mathcal{G}}, h> \\
					&= \inprod<(f_1 + f_2) - \cexp{f_1 + f_2}{\mathcal{G}}, h> - \inprod<(f_1 + f_2) - (\cexp{f_1}{\mathcal{G}} + \cexp{f_2}{\mathcal{G}}), h> \\
					&= \inprod<\cexp{f_1}{\mathcal{G}} + \cexp{f_2}{\mathcal{G}} - \cexp{f_1 + f_2}{\mathcal{G}}, h>
				\end{align}
				となり,特に$h = \cexp{f_1}{\mathcal{G}} + \cexp{f_2}{\mathcal{G}} - \cexp{f_1 + f_2}{\mathcal{G}} \in \Lp{2}{\Omega, \mathcal{G},\mu}$とすれば
				\begin{align}
					\Norm{\cexp{f_1}{\mathcal{G}} + \cexp{f_2}{\mathcal{G}} - \cexp{f_1 + f_2}{\mathcal{G}}}{\Lp{2}{\Omega, \mathcal{F},\mu}}^2 = 0
				\end{align}
				が成り立つことになるから
				\begin{align}
					\cexp{f_1}{\mathcal{G}} + \cexp{f_2}{\mathcal{G}} = \cexp{f_1 + f_2}{\mathcal{G}}
				\end{align}
				が示された.
				
			\item[C4] 「任意の$f \in \Lp{2}{\Omega, \mathcal{F},\mu}$に対して,$f \geq 0\ $a.s.ならば$\cexp{f}{\mathcal{G}} \geq 0\ $a.s.」---(※)を示せばよい.
				これが示されれば$f_1,f_2 \in \Lp{2}{\Omega, \mathcal{F},\mu}$で$f_1 \leq f_2\ $a.s.となるものに対し
				\begin{align}
					0 \leq f_2 - f_1\ \mathrm{a.s.} \quad \Rightarrow 0 \leq \cexp{f_2 - f_1}{\mathcal{G}} = \cexp{f_2}{\mathcal{G}} - \cexp{f_1}{\mathcal{G}}\ \mathrm{a.s.}
				\end{align}
				が成り立つ.しかし,この場合本題に入る前に次の命題を証明する必要がある.これは等号$\cexp{f_2 - f_1}{\mathcal{G}} = \cexp{f_2}{\mathcal{G}} - \cexp{f_1}{\mathcal{G}}$が成り立つことを保証するためである.
				\begin{prp}
					考えている空間は今までと同じHilbert空間$\Lp{2}{\Omega, \mathcal{F},\mu},\ \Lp{2}{\Omega, \mathcal{G},\mu}$である.
					任意の実数$\alpha$と任意の$f \in \Lp{2}{\Omega, \mathcal{F},\mu}$に対して次が成立する:
					\begin{align}
						\cexp{\alpha f}{\mathcal{G}} = \alpha \cexp{f}{\mathcal{G}}.
					\end{align}
				\end{prp}
				\begin{prf}
					射影定理より
					\begin{align}
						\inprod<f - \cexp{f}{\mathcal{G}}, h> = 0, \quad \inprod<\alpha f - \cexp{\alpha f}{\mathcal{G}}, h> = 0 \quad (\forall h \in \Lp{2}{\Omega, \mathcal{G},\mu})
					\end{align}
					が成り立っているから
					\begin{align}
						\inprod<\cexp{\alpha f}{\mathcal{G}} - \alpha \cexp{f}{\mathcal{G}}, h> 
						&= \inprod<\cexp{\alpha f}{\mathcal{G}} - \alpha f, h> - \inprod<\alpha \cexp{f}{\mathcal{G}} - \alpha f, h> \\
						&= \inprod<\cexp{\alpha f}{\mathcal{G}} - \alpha f, h> - \alpha \inprod<f - \cexp{f}{\mathcal{G}}, h> = 0. \quad (\forall h \in \Lp{2}{\Omega, \mathcal{G},\mu})
					\end{align}
					特に$h = \cexp{\alpha f}{\mathcal{G}} - \alpha \cexp{f}{\mathcal{G}} \in \Lp{2}{\Omega, \mathcal{G},\mu}$として
					\begin{align}
						\Norm{\cexp{\alpha f}{\mathcal{G}} - \alpha \cexp{f}{\mathcal{G}}}{\Lp{2}{\Omega, \mathcal{F},\mu}}^2 = 0
					\end{align}
					だから$\cexp{\alpha f}{\mathcal{G}} = \alpha \cexp{f}{\mathcal{G}}$が成り立つ.
					\QED
				\end{prf}
				次に(※)を示す.
				\begin{align}
					A &\coloneqq \{x \in \Omega\ |\quad f(x) < 0 \} && (\in \mathcal{F}), \\
					B &\coloneqq \{x \in \Omega\ |\quad \cexp{f}{\mathcal{G}}(x) < 0 \} && (\in \mathcal{G})
				\end{align}
				として$\mu(A)=0 \quad \Rightarrow \quad \mu(B)=0$が成り立つと言えればよく,$\mu(A) = 0$の下で$\mu(B) > 0$
				と仮定しては不合理であることを以下に記述する.
		\end{description}
	\end{prf}
\begin{dfn}[Brown運動の講義における定義(講義資料引用)]
	$\mu$を$\R^N$上の分布(i.e.Borel確率測度)とする.確率空間$(\Omega, \mathfrak{F}, \operatorname{P})$上の
	$\R^N$-値確率過程$B = (B_t)_{t \geq 0}$で以下を満たすものを,初期分布$\mu$の$N$次元Brown運動という.とくに,
	$\mu$が$x \in \R^N$のDirac測度$\delta_x$のとき,$B$は$x$から出発する$N$次元Brown運動と呼ばれる.
	\begin{description}
		\item[\rm{(i)}] 任意の$\omega \in \Omega$に対して,$[0,\infty) \ni t \longmapsto B_t(\omega) \in \R^N$は連続.
		\item[\rm{(ii)}] 任意の$0 \leq s < t$に対して$B_t - B_s$は$\mathfrak{F}_s^B = \sigma(B_u\ :\ u \leq s)$と独立.
		\item[\rm{(iii)}] 任意の$0 \leq s < t$に対して$B_t - B_s$は平均ベクトル0,共分散行列$(t-s)I_N$の$N$次元Gauss型確率変数
			である.ここで$I_N$は$N$次元単位行列を表す.
		\item[\rm{(iv)}] $\operatorname{P}_{B_0} = \mu$.
	\end{description}
\end{dfn}

\section{レポート課題その1}


定理3.8のBrown運動の性質(1), (2), (3)を示せ.

\begin{description}
	\item[(1)](回転不変性) 任意の$R \in O(N)$に対して$RB = (RB_t)$は原点から出発するBrown運動である.ただし,
		$O(N)$は$N$次直交行列全体で$Rx$はベクトル$x$に左から行列$R$をかけることいを意味する.
	\item[(2)](スケール則) 任意の$c > 0$に対して$((1/\sqrt{c})B_{ct})$は原点から出発するBrown運動である.
	\item[(3)] 任意の$h > 0$に対して$(B_{t+h} - B_h)$は原点から出発するBrown運動である.
\end{description}

\begin{prf}
\begin{description}\mbox{}
	\item[(1)] 
		上に載せた定義の番号の順番に照合していく.(i)について,
		任意の$N$次直交行列$R$は,通常のEuclidノルムの入ったノルム空間$\R^N$(通常の位相空間としての$\R^N$に同じ)において
		$\R^N \rightarrow \R^N$の有界な線型作用素である.即ち$\R^N \rightarrow \R^N$の連続写像であり,
		連続写像の合成である
		\begin{align}
			[0,\infty) \ni t \longmapsto RB_t(\omega) \in \R^N, \quad(\forall \omega \in \Omega)
		\end{align}
		もまた連続写像であるから,(i)は満たされている.
		次に(ii)を示す.$\borel{\R^N}$は$\R^N$のBorel集合族を表すとする.まずは任意の$t \leq 0$に対して
		\begin{align}
			\left\{(RB_t)^{-1}(E)\ \left|\ E \in \borel{\R^N} \right.\right\} 
			= \left\{B_t^{-1}(E)\ \left|\ E \in \borel{\R^N} \right.\right\} \label{eq:stoc_proc_Q1_1}
		\end{align}
		が成り立つことを示す.これは次の理由による.
		任意の$N$次直交行列$R$は,通常のEuclidノルムの入ったノルム空間$\R^N$において
		$\R^N \rightarrow \R^N$の有界な線型作用素である.即ち$\R^N \rightarrow \R^N$の連続写像であり,
		任意のBorel集合$E \in \borel{\R^N}$を$\R^N$のBorel集合に引き戻す.また
		$R$が$\R^N \rightarrow \R^N$の全単射であること(全射,単射であることは$R$の正則性により示される,つまり
		任意の$y \in \R^N$に対して$Rx = y$を満たすような$x$は$R^{-1}y$であり,$Rx=Ry$ならば$R(x-y)=0$の両辺に$R^{-1}$をかけて$x=y$が出る.)
		と$\R^N$の完備性により関数解析の値域定理が適用され,
		$R$の逆写像$R^{-1}$もまた$\R^N \rightarrow \R^N$の有界な線型作用素である.
		従って任意のBorel集合の$R$による像は$\R^N$のBorel集合となる.
		以上より任意のBorel集合$A \in \borel{\R^N}$に対して
		\begin{align}
			&(RB_t)^{-1}(A) = B_t^{-1}\left(R^{-1}(A)\right) \in \left\{B_t^{-1}(E)\ \left|\ E \in \borel{\R^N} \right.\right\}, \\
			&B_t^{-1}(A) = B_t^{-1}\left(R^{-1}(R(A))\right) \in \left\{(RB_t)^{-1}(E)\ \left|\ E \in \borel{\R^N} \right.\right\}
		\end{align}
		が示され,式(\refeq{eq:stoc_proc_Q1_1})が成り立つと判る.従って
		\begin{align}
			\sigma(B_u\ :\ u \leq s) = \bigvee_{u \leq s} \left\{B_u^{-1}(E)\ \left|\ E \in \borel{\R^N} \right.\right\}
			= \bigvee_{u \leq s} \left\{(RB_u)^{-1}(E)\ \left|\ E \in \borel{\R^N} \right.\right\} 
			= \sigma(RB_u\ :\ u \leq s)
		\end{align}
		が成り立つ.任意の$0 \leq s < t$に対して$B_t - B_s$は$\mathfrak{F}_s^B = \sigma(B_u\ :\ u \leq s)$と独立であるから,
		任意の$A \in \borel{\R^N}$と$F \in \sigma(RB_u\ :\ u \leq s) = \sigma(B_u\ :\ u \leq s)$に対して,
		$R^{-1}(A) \in \borel{\R^N}$に注意すれば
		\begin{align}
			\operatorname{P}\left(\{RB_t - RB_s \in A\} \cap F \right) 
			&= \operatorname{P}\left(\{R(B_t - B_s) \in A\} \cap F \right) \\
			&= \operatorname{P}\left(\{B_t - B_s \in R^{-1}(A)\} \cap F \right) \\
			&= \operatorname{P}\left(B_t - B_s \in R^{-1}(A)\right)\operatorname{P}(F) \\
			&= \operatorname{P}\left(R(B_t - B_s) \in A\right)\operatorname{P}(F)
			= \operatorname{P}\left(RB_t - RB_s \in A\right)\operatorname{P}(F)
		\end{align}
		が成り立つ.これは任意の$0 \leq s < t$に対して$RB_t - RB_s$と$\sigma(RB_u\ :\ u \leq s)$とが独立であることを表しているから,
		(ii)も示されたことになる.(iii)について,行列式$\Det{R}$が$\pm 1$になることに注意すれば,任意の$A \in \borel{\R^N}$に対して
		\begin{align}
			\operatorname{P}(RB_t - RB_s \in A) &= \operatorname{P}\left(B_t - B_s \in R^{-1}(A)\right) \\
			&= (2\pi(t-s))^{-\frac{N}{2}} \int_{R^{-1}(A)} \mathrm{exp}\left( -\frac{|x|^2}{2(t-s)} \right)\, dx \\
			&= (2\pi(t-s))^{-\frac{N}{2}} \int_{A} \mathrm{exp}\left( -\frac{|y|^2}{2(t-s)} \right)\, dy & \left(y = Rx\mbox{として変数変換}\right)
		\end{align}
		が成り立つことにより,任意の$0 \leq s < t$に対して$RB_t - RB_s$もまた平均ベクトル0,共分散行列$(t-s)I_N$の$N$次元Gauss型確率変数
		であることが示された.最後に(iv)が満たされていることを確認する.今,$\operatorname{P}_{B_0} = \delta_0$を仮定している.
		全単射線型写像$R$について$0 \in A \Leftrightarrow 0 \in R^{-1}(A)\ (\forall A \in \borel{\R^N})$であることに注意すれば
		\begin{align}
			\operatorname{P}_{RB_0}(A) = \operatorname{P}_{B_0}(R^{-1}(A)) 
			= \begin{cases}
				1 & 0 \in R^{-1}(A) \\
				0 & 0 \notin R^{-1}(A)
			\end{cases}
			= \begin{cases}
				1 & 0 \in A \\
				0 & 0 \notin A
			\end{cases}
			= \operatorname{P}_{B_0}(A)
		\end{align}
		となり,$\operatorname{P}_{RB_0}$と$\operatorname{P}_{B_0}$は$\borel{\R^N}$の上で一致する.
		\QED
	
	\item[(2)] 上に載せた定義の番号の順番に照合していく.(i)について,
		これも連続写像の合成
		\begin{align}
			[0,\infty) \ni t \longmapsto ct \longmapsto \tfrac{1}{\sqrt{c}} B_{ct}(\omega) \in \R^N, \quad(\forall \omega \in \Omega)
		\end{align}
		と見做せばよい.(ii)について,(i)と同様に考えればよい.写像$\R^N \ni x \longmapsto x/\sqrt{c} \in \R^N$は
		$\R^N \rightarrow \R^N$の連続な全単射であり,明らかに逆写像$\R^N \ni x \longmapsto \sqrt{c} \in \R^N$もまた
		連続な全単射である.従って任意の$A \in \borel{\R^N}$に対して
		\begin{align}
			\tfrac{1}{\sqrt{c}} A \coloneqq \left\{ x/\sqrt{c}\ \left|\ x \in A \right.\right\} \in \borel{\R^N}, \quad
			\sqrt{c} A \coloneqq \left\{ \sqrt{c}x\ \left|\ x \in A \right.\right\} \in \borel{\R^N}
		\end{align}
		が成り立つから,任意の$t \geq 0$に対して
		\begin{align}
			\left\{\left. \tfrac{1}{\sqrt{c}}B_{ct} \in A\ \right|\ A \in \borel{\R^N} \right\}
			= \left\{\left. B_{ct} \in A\ \right|\ A \in \borel{\R^N} \right\}
		\end{align}
		が成り立つ.即ち任意の$s \geq 0$に対して
		\begin{align}
			\sigma\left(\tfrac{1}{\sqrt{c}}B_{cu}\ :\ u \leq s\right)
			\coloneqq \bigvee_{u \leq s} \left\{ \tfrac{1}{\sqrt{c}}B_{cu} \in A\ \left|\ A \in \borel{\R^N} \right.\right\}
			= \bigvee_{u \leq s} \left\{ B_{cu} \in A\ \left|\ A \in \borel{\R^N} \right.\right\}
			= \sigma(B_{cu}\ :\ u \leq s)
		\end{align}
		となっていて,さらに設問の仮定により任意の$0 \leq s < t$に対して$B_{ct} - B_{cs}$は$\sigma(B_{cu}\ :\ u \leq s)$と独立である.
		以上より,任意の$A \in \borel{\R^N}$と$F \in \sigma\left(\tfrac{1}{\sqrt{c}}B_{cu}\ :\ u \leq s\right) = \sigma(B_{cu}\ :\ u \leq s)$に対して
		\begin{align}
			\operatorname{P}\left( \left\{ \tfrac{1}{\sqrt{c}}B_{ct} - \tfrac{1}{\sqrt{c}}B_{cs} \in A \right\} \cap F \right)
			&= \operatorname{P}\left( \left\{ B_{ct} - B_{cs} \in \sqrt{c}A \right\} \cap F \right) \\
			&= \operatorname{P}\left(B_{ct} - B_{cs} \in \sqrt{c}A \right)  \operatorname{P}(F) \\
			&= \operatorname{P}\left(\tfrac{1}{\sqrt{c}}B_{ct} - \tfrac{1}{\sqrt{c}}B_{cs} \in A \right)  \operatorname{P}(F)
		\end{align}
		が成り立つから,任意の$0 \leq s < t$に対して$\tfrac{1}{\sqrt{c}}B_{ct} - \tfrac{1}{\sqrt{c}}B_{cs}$は
		$\sigma\left(\tfrac{1}{\sqrt{c}}B_{cu}\ :\ u \leq s\right)$と独立であると示された.
		(iii)について,これもヤコビアンが$\sqrt{c}$になることに注意すれば
		\begin{align}
			\operatorname{P}\left( \tfrac{1}{\sqrt{c}}B_{ct} - \tfrac{1}{\sqrt{c}}B_{cs} \in A \right)
			&= \operatorname{P}\left( B_{ct} - B_{cs} \in \sqrt{c}A \right) \\
			&= (2\pi(ct-cs))^{-\frac{N}{2}} \int_{\sqrt{c}A} \mathrm{exp}\left( -\frac{|x|^2}{2(ct-cs)} \right)\, dx \\
			&= (2\pi(t-s))^{-\frac{N}{2}} \int_{A} \mathrm{exp}\left( -\frac{|y|^2}{2(t-s)} \right)\, dy 
				& \left( y = \tfrac{1}{\sqrt{c}}x\mbox{と変数変換}\right) \\
		\end{align}
		が成り立つことにより,任意の$0 \leq s < t$に対して$\tfrac{1}{\sqrt{c}}B_{ct} - \tfrac{1}{\sqrt{c}}B_{cs}$
		は平均ベクトル0,共分散行列$(t-s)I_N$の$N$次元Gauss型確率変数である.最後に(iv)を示す.
		任意の$A \in \borel{\R^N}$に対して
		$0 \in A \Leftrightarrow 0 \in \sqrt{c}A$であることに注意すれば,
		\begin{align}
			\operatorname{P}_{\tfrac{1}{\sqrt{c}}B_0}(A) = \operatorname{P}_{B_0}(\sqrt{c}A) 
			= \begin{cases}
				1 & 0 \in \sqrt{c}A \\
				0 & 0 \notin \sqrt{c}A
			\end{cases}
			= \begin{cases}
				1 & 0 \in A \\
				0 & 0 \notin A
			\end{cases}
			= \operatorname{P}_{B_0}(A)
		\end{align}
		が成り立つから$\operatorname{P}_{\tfrac{1}{\sqrt{c}}B_0}$と$\operatorname{P}_{B_0}$は$\borel{\R^N}$の上で一致する.
\end{description}
\end{prf}

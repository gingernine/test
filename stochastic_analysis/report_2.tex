

\section{レポート課題その1}

講義資料定義3.1に定義されるBrown運動が存在する確率空間$(\Omega, \mathcal{F}, \operatorname{P})$を基礎に考える.

定理3.8のBrown運動の性質(1), (2), (3)を示せ.
\begin{thm}[定理3.8]
$B = (B_t)$を原点から出発する$N$-次元Brown運動とするとき,いかが成り立つ.
\begin{description}
	\item[(1)](回転不変性) 任意の$R \in O(N)$に対して$RB = (RB_t)$は原点から出発するBrown運動である.ただし,
		$O(N)$は$N$次直交行列全体で$Rx$はベクトル$x$に左から行列$R$をかけることいを意味する.
	\item[(2)](スケール則) 任意の$c > 0$に対して$((1/\sqrt{c})B_{ct})$は原点から出発するBrown運動である.
	\item[(3)] 任意の$h > 0$に対して$(B_{t+h} - B_h)$は原点から出発するBrown運動である.
	\begin{comment}
	\item[(4)] $B$の各座標が定める実数値確率過程を$B_i = (B_i(t))\ (i = 1,2,\cdots,N)$とすると,
		$B_i\ (i=1,2,\cdots,N)$は独立な1次元Brown運動である.逆に,独立な1次元Brown運動
		$B_i\ (i=1,2,\cdots,N)$が与えられたとき$B(t) = {}^t(B_1(t),\cdots,B_N(t))$とおくと
		$(B(t))$は$N$次元Brown運動である.
	\end{comment}
\end{description}
\end{thm}

\begin{prf}
\begin{description}\mbox{}
	\item[(1)] 
		講義資料定義3.1の番号(i)(ii)(iii)の順番に照合していくが,その前に$RB = (RB_t)$
		が確率空間$(\Omega, \mathcal{F}, \operatorname{P})$上の確率過程となっていることを確認する.
		任意の$N$次直交行列$R$は$\R^N \rightarrow \R^N$の連続な全単射線型作用素であるから可測
		$\borel{\R^N}/\borel{\R^N}$である.従って任意の$t \geq 0$に対して$RB_t$は可測$\mathcal{F}/\borel{\R^N}$
		となるから$RB = (RB_t)$は確率過程である.講義資料定義3.1の(i)について,
		連続写像の合成である
		\begin{align}
			[0,\infty) \ni t \longmapsto RB_t(\omega) \in \R^N, \quad(\forall \omega \in \Omega)
		\end{align}
		もまた連続写像であるから,(i)は満たされている.
		次に(ii)を示す.まずは任意の$t \leq 0$に対して
		\begin{align}
			\left\{(RB_t)^{-1}(E)\ \left|\ E \in \borel{\R^N} \right.\right\} 
			= \left\{B_t^{-1}(E)\ \left|\ E \in \borel{\R^N} \right.\right\} \label{eq:stoc_proc_Q1_1}
		\end{align}
		が成り立つことを示す.これは次の理由による.
		任意の$N$次直交行列$R$は$\R^N \rightarrow \R^N$の連続写像であるから
		任意のBorel集合$E \in \borel{\R^N}$を$\R^N$のBorel集合に引き戻す.また
		$R$の逆写像$R^{-1}$もまた$O(N)$の元であるから,
		任意のBorel集合の$R$による像,即ち$R^{-1}$によって引き戻した集合は$\R^N$のBorel集合となる.
		従って任意のBorel集合$A \in \borel{\R^N}$に対して
		\begin{align}
			&(RB_t)^{-1}(A) = B_t^{-1}\left(R^{-1}A\right) \in \left\{B_t^{-1}(E)\ \left|\ E \in \borel{\R^N} \right.\right\}, \\
			&B_t^{-1}(A) = B_t^{-1}\left(R^{-1}RA\right) \in \left\{(RB_t)^{-1}(E)\ \left|\ E \in \borel{\R^N} \right.\right\}
		\end{align}
		が示され,式(\refeq{eq:stoc_proc_Q1_1})が成り立つと判る.従って
		\begin{align}
			\sigma(B_u\ :\ u \leq s) = \bigvee_{u \leq s} \left\{B_u^{-1}(E)\ \left|\ E \in \borel{\R^N} \right.\right\}
			= \bigvee_{u \leq s} \left\{(RB_u)^{-1}(E)\ \left|\ E \in \borel{\R^N} \right.\right\} 
			= \sigma(RB_u\ :\ u \leq s)
		\end{align}
		が任意の$s \geq 0$で成り立つ.任意の$0 \leq s < t$に対して$B_t - B_s$は$\mathcal{F}_s^B = \sigma(B_u\ :\ u \leq s)$と独立であるから,
		任意の$A \in \borel{\R^N}$と$F \in \sigma(RB_u\ :\ u \leq s) = \sigma(B_u\ :\ u \leq s)$に対して
		\begin{align}
			\operatorname{P}\left(\{RB_t - RB_s \in A\} \cap F \right) 
			&= \operatorname{P}\left(\{R(B_t - B_s) \in A\} \cap F \right) \\
			&= \operatorname{P}\left(\{B_t - B_s \in R^{-1}A\} \cap F \right) \\
			&= \operatorname{P}\left(B_t - B_s \in R^{-1}A\right)\operatorname{P}(F) \\
			&= \operatorname{P}\left(R(B_t - B_s) \in A\right)\operatorname{P}(F)
			= \operatorname{P}\left(RB_t - RB_s \in A\right)\operatorname{P}(F)
		\end{align}
		が成り立つ.これは任意の$0 \leq s < t$に対して$RB_t - RB_s$と$\sigma(RB_u\ :\ u \leq s)$とが独立であることを表しているから,
		(ii)も示されたことになる.(iii)について,行列式$\Det{R}$が$\pm 1$になることに注意すれば,任意の$A \in \borel{\R^N}$に対して
		\begin{align}
			\operatorname{P}(RB_t - RB_s \in A) &= \operatorname{P}\left(B_t - B_s \in R^{-1}A\right) \\
			&= (2\pi(t-s))^{-\frac{N}{2}} \int_{R^{-1}A} \mathrm{exp}\left( -\frac{|x|^2}{2(t-s)} \right)\, dx \\
			&= (2\pi(t-s))^{-\frac{N}{2}} \int_{A} \mathrm{exp}\left( -\frac{|y|^2}{2(t-s)} \right)\, dy & \left(y = Rx\mbox{として変数変換}\right)
		\end{align}
		が成り立つことにより,任意の$0 \leq s < t$に対して$RB_t - RB_s$もまた平均ベクトル0,共分散行列$(t-s)I_N$の$N$次元Gauss型確率変数
		であることが示された.最後に(iv)が満たされていることを確認する.
		全単射線型写像$R$について$0 \in A \Leftrightarrow 0 \in R^{-1}A\ (\forall A \in \borel{\R^N})$であることに注意すれば
		\begin{align}
			\operatorname{P}_{RB_0}(A) 
			= \operatorname{P}\left(B_0^{-1}\left(R^{-1}A\right)\right) 
			= \begin{cases}
				1 & 0 \in R^{-1}(A) \\
				0 & 0 \notin R^{-1}(A)
			\end{cases}
			= \begin{cases}
				1 & 0 \in A \\
				0 & 0 \notin A
			\end{cases}
		\end{align}
		となり,$\operatorname{P}_{RB_0}$と$\delta_0$は$\borel{\R^N}$の上で一致する.
	
	\item[(2)] 講義資料定義3.1の番号(i)(ii)(iii)の順番に照合していく. その前に,$((1/\sqrt{c})B_{ct})$
		が確率空間$(\Omega, \mathcal{F}, \operatorname{P})$上の確率過程となっていることは,$(1/\sqrt{c})I_N$\ ($I_N$は$N$次単位行列)が
		$\R^N \rightarrow \R^N$の連続写像であることと(1)の証明から確認できる.
		(i)について,$[0,\infty) \ni t \longmapsto B_{ct}(\omega) \in \R^N, \quad(\forall \omega \in \Omega)$
		が連続であるから$[0,\infty) \ni t \longmapsto (1/\sqrt{c}) B_{ct}(\omega) \in \R^N, \quad(\forall \omega \in \Omega)$の連続性も明らかである.(ii)について,写像$\R^N \ni x \longmapsto x/\sqrt{c} \in \R^N$は
		$\R^N \rightarrow \R^N$の連続な全単射であり,明らかに逆写像$\R^N \ni x \longmapsto \sqrt{c}x \in \R^N$もまた
		連続な全単射である.従って任意の$A \in \borel{\R^N}$に対して
		\begin{align}
			\tfrac{1}{\sqrt{c}} A \coloneqq \left\{\left. \tfrac{1}{\sqrt{c}}x\ \right|\ x \in A \right\} \in \borel{\R^N}, \quad
			\sqrt{c} A \coloneqq \left\{\left. \sqrt{c}x\ \right|\ x \in A \right\} \in \borel{\R^N}
		\end{align}
		が成り立つ.よって任意の$t \geq 0$に対して
		\begin{align}
			\left\{\left. \left(\tfrac{1}{\sqrt{c}}B_{ct}\right)^{-1}(A)\ \right|\ A \in \borel{\R^N} \right\}
			= \left\{\left. B_{ct}^{-1}(A)\ \right|\ A \in \borel{\R^N} \right\}
		\end{align}
		が成り立つから,任意の$s \geq 0$に対して
		\begin{align}
			\sigma\left(\tfrac{1}{\sqrt{c}}B_{cu}\ :\ u \leq s\right)
			\coloneqq \bigvee_{u \leq s} \left\{ \left(\tfrac{1}{\sqrt{c}}B_{cu}\right)^{-1}(A)\ \left|\ A \in \borel{\R^N} \right.\right\}
			= \bigvee_{u \leq s} \left\{ B_{cu}^{-1}(A)\ \left|\ A \in \borel{\R^N} \right.\right\}
			= \sigma(B_{cu}\ :\ u \leq s)
		\end{align}
		となり,任意の$0 \leq s < t$に対して$B_{ct} - B_{cs}$は$\sigma(B_{cu}\ :\ u \leq s)$と独立であるから,
		任意の$A \in \borel{\R^N}$と$F \in \sigma\left(\tfrac{1}{\sqrt{c}}B_{cu}\ :\ u \leq s\right) = \sigma(B_{cu}\ :\ u \leq s)$に対して
		\begin{align}
			\operatorname{P}\left( \left\{ \tfrac{1}{\sqrt{c}}B_{ct} - \tfrac{1}{\sqrt{c}}B_{cs} \in A \right\} \cap F \right)
			&= \operatorname{P}\left( \left\{ B_{ct} - B_{cs} \in \sqrt{c}A \right\} \cap F \right) \\
			&= \operatorname{P}\left(B_{ct} - B_{cs} \in \sqrt{c}A \right)  \operatorname{P}(F) \\
			&= \operatorname{P}\left(\tfrac{1}{\sqrt{c}}B_{ct} - \tfrac{1}{\sqrt{c}}B_{cs} \in A \right)  \operatorname{P}(F)
		\end{align}
		が成り立つ.これは任意の$0 \leq s < t$に対して$\tfrac{1}{\sqrt{c}}B_{ct} - \tfrac{1}{\sqrt{c}}B_{cs}$が
		$\sigma\left(\tfrac{1}{\sqrt{c}}B_{cu}\ :\ u \leq s\right)$と独立であることを示す.
		(iii)について,これもヤコビアンが$\left(\sqrt{c}\right)^{N}$になることに注意すれば
		\begin{align}
			\operatorname{P}\left( \tfrac{1}{\sqrt{c}}B_{ct} - \tfrac{1}{\sqrt{c}}B_{cs} \in A \right)
			&= \operatorname{P}\left( B_{ct} - B_{cs} \in \sqrt{c}A \right) \\
			&= (2\pi(ct-cs))^{-\frac{N}{2}} \int_{\sqrt{c}A} \mathrm{exp}\left( -\frac{|x|^2}{2(ct-cs)} \right)\, dx \\
			&= (2\pi(t-s))^{-\frac{N}{2}} \int_{A} \mathrm{exp}\left( -\frac{|y|^2}{2(t-s)} \right)\, dy 
				& \left( y = \tfrac{1}{\sqrt{c}}x\mbox{と変数変換}\right)
		\end{align}
		が成り立つことにより,任意の$0 \leq s < t$に対して$\tfrac{1}{\sqrt{c}}B_{ct} - \tfrac{1}{\sqrt{c}}B_{cs}$
		は平均ベクトル0,共分散行列$(t-s)I_N$の$N$次元Gauss型確率変数であると判る.最後に(iv)を示す.
		任意の$A \in \borel{\R^N}$に対して
		$0 \in A \Leftrightarrow 0 \in \sqrt{c}A$であることに注意すれば,
		\begin{align}
			\operatorname{P}_{\tfrac{1}{\sqrt{c}}B_0}(A) 
			= \operatorname{P}\left(B_0^{-1}\left(\sqrt{c}A\right)\right)
			= \begin{cases}
				1 & 0 \in \sqrt{c}A \\
				0 & 0 \notin \sqrt{c}A
			\end{cases}
			= \begin{cases}
				1 & 0 \in A \\
				0 & 0 \notin A
			\end{cases}
		\end{align}
		が成り立つから$\operatorname{P}_{\tfrac{1}{\sqrt{c}}B_0}$と$\delta_0$は$\borel{\R^N}$の上で一致する.
		
	\item[(3)] 形から$(B_{t+h} - B_h)$が確率過程であることは明らかである.
		講義資料定義3.1の(i)から見ていく.$B = (B_t)$がBrown運動であるから
		\begin{align}
			[0,\infty) \ni t \longmapsto B_{t+h}(\omega) - B_h(\omega) \in \R^N, \quad(\forall \omega \in \Omega)
		\end{align}
		が連続写像であることは明らかである.次に(ii)を確認する.任意の$t \geq 0$に対して
		$B_{t+h}$は可測$\mathcal{F}_{t+h}^B/\borel{\R^N}$,$B_h$は可測$\mathcal{F}_h^B/\borel{\R^N}$であって,
		$\mathcal{F}_{h}^B \subset \mathcal{F}_{t+h}^B$により$B_{t+h} - B_h$は可測$\mathcal{F}_{t+h}^B/\borel{\R^N}$
		である.つまり
		\begin{align}
			\left\{ \left. \left(B_{t+h} - B_h\right)^{-1}(A)\ \right|\ A \in \borel{\R^N} \right\}
			\subset \mathcal{F}_{t+h}^B %= \sigma(B_{u}\ :\ u \leq t+h)
		\end{align}
		が成り立っているから
		\begin{align}
			\sigma(B_{u+h} - B_h\ :\ u \leq s)
			\coloneqq \bigvee_{u \leq s} \left\{ \left. \left(B_{u+h} - B_h\right)^{-1}(A)\ \right|\ A \in \borel{\R^N} \right\}
			\subset \mathcal{F}_{s+h}^B 
			%= \sigma(B_{u}\ :\ u \leq s+h)
		\end{align}
		の関係がわかる.従って任意の$A \in \borel{\R^N}$と$F \in \sigma(B_{u+h} - B_h\ :\ u \leq s) \subset \mathcal{F}_{s+h}^B $
		に対して,
		\begin{align}
			&\operatorname{P}\left( \left\{ (B_{t+h} - B_h) - (B_{s+h} - B_h) \in A \right\} \cap F \right) \\
			&\quad= \operatorname{P}\left( \left\{ B_{t+h} - B_{s+h} \in A \right\} \cap F \right) \\
			&\quad= \operatorname{P}\left( B_{t+h} - B_{s+h} \in A \right) \operatorname{P}(F) & (\mbox{$B_{t+h} - B_{s+h}$と$\mathcal{F}_{s+h}^B$が独立}) \\
			&\quad= \operatorname{P}\left( (B_{t+h} - B_h) - (B_{s+h} - B_h) \in A \right) \operatorname{P}(F)
		\end{align}
		となるから,任意の$0 \leq s < t$に対して$(B_{t+h} - B_h) - (B_{s+h} - B_h)$は$\sigma(B_{u+h} - B_h\ :\ u \leq s)$と独立であることが示された.
		(iii)について,任意の$A \in \borel{\R^N}$に対して
		\begin{align}
			\operatorname{P}\left( (B_{t+h} - B_h) - (B_{s+h} - B_h) \in A \right)
			&= \operatorname{P}\left( B_{t+h} - B_{s+h} \in A \right) \\
			&= (2\pi((t+h)-(s+h)))^{-\frac{N}{2}} \int_{A} \mathrm{exp}\left( -\frac{|x|^2}{2((t+h)-(s+h))} \right)\, dx \\
			&= (2\pi(t-s))^{-\frac{N}{2}} \int_{A} \mathrm{exp}\left( -\frac{|x|^2}{2(t-s)} \right)\, dx
		\end{align}
		が成り立つから,任意の$0 \leq s < t$に対して$(B_{t+h} - B_h) - (B_{s+h} - B_h) $
		は平均ベクトル0,共分散行列$(t-s)I_N$の$N$次元Gauss型確率変数であると示された.最後に(iv)を確認する.
		便宜上$Y_t \coloneqq B_{t+h} - B_h\ (\forall t \geq 0)$と表記する.
		$t=0$の場合$\Omega$上で$Y_0 = 0$が成り立っているから,任意の$A \in \borel{\R^N}$に対して
		\begin{align}
			\operatorname{P}_{Y_0}(A) 
			= \operatorname{P}\left(Y_0^{-1}(A)\right) 
			= \begin{cases}
				1 & 0 \in A \\
				0 & 0 \notin A
			\end{cases}
		\end{align}
		となり,$Y_0$の分布は$\borel{\R^N}$上で$\delta_0$に一致する.
		\QED
	
	\begin{comment}
	\item[(4)] 
		一般に$N$次元実確率変数$X = (X_1, \cdots, X_N)$の各成分も1次元実確率変数である.これは射影を考えればよい.
		$B = (B_i(t))\ (i=1,\cdots,N)$が$N$次元Brown運動であるとする.
		写像$pr_i : \R^N \longmapsto \R^1\ (i=1,\cdots,N)$を$i$射影とすると,
		各$i=1,2,\cdots,N$について,任意の$0 \leq s < t$と$A \in \borel{\R^1}$に対して,Fubiniの定理を適用し
		\begin{align}
			\operatorname{P}(B_i(t) - B_i(s) \in A) 
			&= \operatorname{P}(pr_i \circ (B(t) - B(s)) \in A) \\
			&= \operatorname{P}(B(t) - B(s) \in pr_i^{-1}(A)) \\
			&= (2 \pi (t-s))^{-\frac{N}{2}} \int_{pr_i^{-1}(A)} \mathrm{exp} \left( -\frac{|x|^2}{2(t-s)} \right)\, dx \\
			&= \frac{1}{\sqrt{2 \pi (t-s)}} \int_{A} \mathrm{exp} \left( -\frac{x_i^2}{2(t-s)} \right)\, dx_i 
				\ \prod_{j \neq i} \left(\frac{1}{\sqrt{2 \pi (t-s)}} \int_{\R^1} \mathrm{exp} \left( -\frac{x_j^2}{2(t-s)} \right)\, dx_j \right) \\
			&= \frac{1}{\sqrt{2 \pi (t-s)}} \int_{A} \mathrm{exp} \left( -\frac{x_i^2}{2(t-s)} \right)\, dx_i 
		\end{align}
		が成り立つ.これは$B_i(t)\ (i=1,2,\cdots,N)$がそれぞれ平均0,分散$t-s$の1次元Gauss型確率変数であることを示している.
		代表として$(B_1(t))_{t \geq 0}$が1次元Brown運動であることを確認する.
		写像$[0,\infty) \ni t \longmapsto B_1(t) \in \R^1$が連続写像であることは明らかである.
		(ii)について,
		\begin{align}
			\mathcal{F}_s^1 \coloneqq \bigvee_{u \leq s} \left\{ (B_1(u) \in A)\ \left|\ A \in \borel{\R^1} \right.\right\}
		\end{align}
		と$\mathcal{F}_s^1$を定義すれば,任意の$u \leq s$と$A \in \borel{\R^1}$ついて
		\begin{align}
			(B_1(u) \in A) = \left(B(u) \in pr_1^{-1}(A)\right), \quad pr_1^{-1}(A) \in \borel{\R^N}
		\end{align}
		であることにより
		\begin{align}
			\mathcal{F}_s^1 \subset \bigvee_{u \leq s} \left\{ (B(u) \in E)\ \left|\ E \in \borel{\R^N} \right.\right\} = \mathcal{F}_s^B
		\end{align}
		が成り立つ.従って任意の$0 \leq s < t$,$A \in \borel{\R^1}$と$F \in \mathcal{F}_s^1$に対して
		\begin{align}
			\operatorname{P}(\{B_1(t) - B_1(s) \in A\} \cap F)
			&= \operatorname{P}\left( \left\{ B(t) - B(s) \in pr_1^{-1}(A)\right\} \cap F\right) \\
			&= \operatorname{P}\left( B(t) - B(s) \in pr_1^{-1}(A)\right) \operatorname{P}(F) \\
			&= \operatorname{P}\left( B_1(t) - B_1(s) \in A\right) \operatorname{P}(F)
		\end{align}
		が成り立ち,任意の$0 \leq s < t$について$(B_1(t) - B_1(s))$が$\mathcal{F}_s^1$と独立であることが示される.
		(iv)を確認する.任意の$A \in \borel{\R^1}$に対して$0 \in A \Leftrightarrow 0 \in pr_1^{-1}(A)$(両辺で
		0の次元は違う)に注意すれば
		\begin{align}
			\operatorname{P}(B_1(0) \in A) = \operatorname{P}\left(B(0) \in pr_1^{-1}(A)\right)
			= \begin{cases}
				1 & 0 \in pr_1^{-1}(A) \\
				0 & 0 \notin pr_1^{-1}(A)
			\end{cases}
			= \begin{cases}
				1 & 0 \in A \\
				0 & 0 \notin A
			\end{cases}
		\end{align}
		が成り立つ.これは$B_1(0)$の分布と1次元Dirac分布$\delta_0$が$\borel{\R^1}$上で一致することを示す.
		最後に$(B_i(t))_{i=1}^{N}$が独立な確率変数の族であることを示す.
		$0 \leq i_1 < i_2 < \cdots < i_n \leq N$として任意に$n$個($n$も任意)の確率変数$\left(B_{i_j}(t)\right)_{j=1}^{n}$を取る.
		$F_{i_j} \in \mathcal{F}_t^{i_j}\ (j=1,2,\cdots,n)$を取れば,
		各$F_{i_j}$は或る$A_{i_j} \in \borel{\R^1}$によって$F_{i_j} = \left(B_{i_j}(t) \in A_{i_j} \right)$と表現される
		から,
		\begin{align}
			\operatorname{P}\left(F_{i_1} \cap F_{i_2} \cap \cdots \cap F_{i_n}\right)
			&= \operatorname{P}\left( \left(B_{i_1}(t) \in A_{i_1} \right) \cap \cdots \cap \left(B_{i_n}(t) \in A_{i_n} \right)\right) \\
			&= \operatorname{P}\left( B(t) \in pr_{i_1}^{-1}(A_{i_1}) \cap \cdots \cap pr_{i_n}^{-1}(A_{i_n})\right) \\
			&= \prod_{j=1}^{n} \frac{1}{\sqrt{2 \pi (t-s)}} \int_{A_{i_j}} \mathrm{exp} \left( -\frac{x_{i_j}^2}{2(t-s)} \right)\, dx_{i_j} \\
			&= \operatorname{P}\left(F_{i_1}\right) \operatorname{P}\left(F_{i_2}\right) \cdots \operatorname{P}\left(F_{i_n}\right)
		\end{align}
		によって$(B_i(t))_{i=1}^{N}$の独立性が示された.
		次に
	\end{comment}
\end{description}
\end{prf}

\documentclass[11pt,a4paper]{jsarticle}
%
\usepackage{amsmath,amssymb}
\usepackage{amsthm}
\usepackage{makeidx}
\usepackage{txfonts}
\usepackage{mathrsfs} %花文字
\usepackage{mathtools} %参照式のみ式番号表示
\usepackage{latexsym} %qed
\usepackage{ascmac}
\usepackage{color}

\newtheoremstyle{mystyle}% % Name
	{20pt}%                      % Space above
	{20pt}%                      % Space below
	{\rm}%           % Body font
	{}%                      % Indent amount
	{\gt}%             % Theorem head font
	{.}%                      % Punctuation after theorem head
	{10pt}%                     % Space after theorem head, ' ', or \newline
	{}%                      % Theorem head spec (can be left empty, meaning `normal')
\theoremstyle{mystyle}

\allowdisplaybreaks[1]
\newcommand{\bhline}[1]{\noalign {\hrule height #1}} %表の罫線を太くする.
\newcommand{\bvline}[1]{\vrule width #1} %表の罫線を太くする.
\newtheorem{Prop}{$Proposition.$}
\newtheorem{Proof}{$Proof.$}
\newcommand{\QED}{% %証明終了
	\relax\ifmmode
		\eqno{%
		\setlength{\fboxsep}{2pt}\setlength{\fboxrule}{0.3pt}
		\fcolorbox{black}{black}{\rule[2pt]{0pt}{1ex}}}
	\else
		\begingroup
		\setlength{\fboxsep}{2pt}\setlength{\fboxrule}{0.3pt}
		\hfill\fcolorbox{black}{black}{\rule[2pt]{0pt}{1ex}}
		\endgroup
	\fi}
\newtheorem*{thm}{}
\newtheorem*{dfn}{定義}
\newtheorem*{prf}{証明}
\def\Box#1{$(\mbox{#1})$} %丸括弧つきコメント
\def\DEF{\overset{\mathrm{def}}{\Leftrightarrow}} %定義記号
\def\max#1#2{\operatorname*{max}_{#1} #2 } %最大
\def\min#1#2{\operatorname*{min}_{#1} #2 } %最小
\def\sin#1#2{\operatorname{sin}^{#2} #1} %sin
\def\cos#1#2{\operatorname{cos}^{#2} #1} %cos
\def\tan#1#2{\operatorname{tan}^{#2} #1} %tan
\def\inprod<#1>{\langle #1 \rangle} %内積
\def\sup#1#2{\operatorname*{sup}_{#1} #2 } %上限
\def\inf#1#2{\operatorname*{inf}_{#1} #2 } %下限
\def\Vector#1{\mbox{\boldmath $#1$}} %ベクトルを太字表示
\def\Norm#1#2{\left\|\, #1\, \right\|_{#2}} %ノルム
\def\Log#1{\operatorname{log} #1} %log
\def\Det#1{\operatorname{det} ( #1 )} %行列式
\def\Diag#1{\operatorname{diag} \left( #1 \right)} %行列の対角成分
\def\Tmat#1{#1^\mathrm{T}} %転置行列
\def\Exp#1{\operatorname{E} \left[ #1 \right]} %期待値
\def\Var#1{\operatorname{V} \left[ #1 \right]} %分散
\def\Cov#1#2{\operatorname{Cov} \left[ #1,\ #2 \right]} %共分散
\def\exp#1{e^{#1}} %指数関数
\def\N{\mathbb{N}} %自然数全体
\def\Q{\mathbb{Q}} %有理数全体
\def\R{\mathbb{R}} %実数全体
\def\C{\mathbb{C}} %複素数全体
\def\borel#1{\mathcal{B}(#1)} %Borel集合族
\def\open#1{\mathfrak{O}(#1)} %位相空間 #1 の位相
\def\close#1{\mathfrak{U}(#1)} %%位相空間 #1 の閉集合系
\def\rapid#1{\mathfrak{S}(#1)} %急減少空間
\def\c#1{C(#1)} %有界実連続関数
\def\cbound#1{C_{b} (#1)} %有界実連続関数
\def\Lp#1#2{\operatorname{L}^{#1} \left(#2\right)} %L^p
\def\cinf#1{C^{\infty} (#1)} %無限回連続微分可能関数
\def\sgmalg#1{\sigma \left[#1\right]} %#1が生成するσ加法族
\def\ball#1#2{\operatorname{B} \left(#1\, ;\, #2 \right)} %開球
\def\prob#1{\operatorname{P} \left(#1\right)} %確率
\def\cprob#1#2{\operatorname{P} \left(\left\{ #1 \ \middle|\ #2 \right\}\right)} %条件付確率
\def\cexp#1#2{\operatorname{E} \left[ #1 \ \middle|\ #2 \right]} %条件付期待値
%\renewcommand{\contentsname}{\bm Index}
%
\makeindex
%
\setlength{\textwidth}{\fullwidth}
\setlength{\textheight}{40\baselineskip}
\addtolength{\textheight}{\topskip}
\setlength{\voffset}{-0.2in}
\setlength{\topmargin}{0pt}
\setlength{\headheight}{0pt}
\setlength{\headsep}{0pt}
%
\title{確率解析レポート}
\author{基礎工学研究科システム創成専攻修士1年\\学籍番号29C17095\\百合川尚学}
\date{\today}

\begin{document}
%
%

\mathtoolsset{showonlyrefs = true}
\maketitle

%
\begin{dfn}[Brown運動の講義における定義(講義資料引用)]
	$\mu$を$\R^N$上の分布(i.e.Borel確率測度)とする.確率空間$(\Omega, \mathfrak{F}, \operatorname{P})$上の
	$\R^N$-値確率過程$B = (B_t)_{t \geq 0}$で以下を満たすものを,初期分布$\mu$の$N$次元Brown運動という.とくに,
	$\mu$が$x \in \R^N$のDirac測度$\delta_x$のとき,$B$は$x$から出発する$N$次元Brown運動と呼ばれる.
	\begin{description}
		\item[\rm{(i)}] 任意の$\omega \in \Omega$に対して,$[0,\infty) \ni t \longmapsto B_t(\omega) \in \R^N$は連続.
		\item[\rm{(ii)}] 任意の$0 \leq s < t$に対して$B_t - B_s$は$\mathfrak{F}_s^B = \sigma(B_u\ :\ u \leq s)$と独立.
		\item[\rm{(iii)}] 任意の$0 \leq s < t$に対して$B_t - B_s$は平均ベクトル0,共分散行列$(t-s)I_N$の$N$次元Gauss型確率変数
			である.ここで$I_N$は$N$次元単位行列を表す.
		\item[\rm{(iv)}] $\operatorname{P}_{B_0} = \mu$.
	\end{description}
\end{dfn}

\section{レポート課題その1}


定理3.8のBrown運動の性質(1), (2), (3)を示せ.

\begin{description}
	\item[(1)](回転不変性) 任意の$R \in O(N)$に対して$RB = (RB_t)$は原点から出発するBrown運動である.ただし,
		$O(N)$は$N$次直交行列全体で$Rx$はベクトル$x$に左から行列$R$をかけることいを意味する.
	\item[(2)](スケール則) 任意の$c > 0$に対して$((1/\sqrt{c})B_{ct})$は原点から出発するBrown運動である.
	\item[(3)] 任意の$h > 0$に対して$(B_{t+h} - B_h)$は原点から出発するBrown運動である.
\end{description}

\begin{prf}
\begin{description}\mbox{}
	\item[(1)] 
		上に載せた定義の番号の順番に照合していく.(i)について,
		任意の$N$次直交行列$R$は,通常のEuclidノルムの入ったノルム空間$\R^N$(通常の位相空間としての$\R^N$に同じ)において
		$\R^N \rightarrow \R^N$の有界な線型作用素である.即ち$\R^N \rightarrow \R^N$の連続写像であり,
		連続写像の合成である
		\begin{align}
			[0,\infty) \ni t \longmapsto RB_t(\omega) \in \R^N, \quad(\forall \omega \in \Omega)
		\end{align}
		もまた連続写像であるから,(i)は満たされている.
		次に(ii)を示す.$\borel{\R^N}$は$\R^N$のBorel集合族を表すとする.まずは任意の$t \leq 0$に対して
		\begin{align}
			\left\{(RB_t)^{-1}(E)\ \left|\ E \in \borel{\R^N} \right.\right\} 
			= \left\{B_t^{-1}(E)\ \left|\ E \in \borel{\R^N} \right.\right\} \label{eq:stoc_proc_Q1_1}
		\end{align}
		が成り立つことを示す.これは次の理由による.
		任意の$N$次直交行列$R$は,通常のEuclidノルムの入ったノルム空間$\R^N$において
		$\R^N \rightarrow \R^N$の有界な線型作用素である.即ち$\R^N \rightarrow \R^N$の連続写像であり,
		任意のBorel集合$E \in \borel{\R^N}$を$\R^N$のBorel集合に引き戻す.また
		$R$が$\R^N \rightarrow \R^N$の全単射であること(全射,単射であることは$R$の正則性により示される,つまり
		任意の$y \in \R^N$に対して$Rx = y$を満たすような$x$は$R^{-1}y$であり,$Rx=Ry$ならば$R(x-y)=0$の両辺に$R^{-1}$をかけて$x=y$が出る.)
		と$\R^N$の完備性により関数解析の値域定理が適用され,
		$R$の逆写像$R^{-1}$もまた$\R^N \rightarrow \R^N$の有界な線型作用素である.
		従って任意のBorel集合の$R$による像は$\R^N$のBorel集合となる.
		以上より任意のBorel集合$A \in \borel{\R^N}$に対して
		\begin{align}
			&(RB_t)^{-1}(A) = B_t^{-1}\left(R^{-1}(A)\right) \in \left\{B_t^{-1}(E)\ \left|\ E \in \borel{\R^N} \right.\right\}, \\
			&B_t^{-1}(A) = B_t^{-1}\left(R^{-1}(R(A))\right) \in \left\{(RB_t)^{-1}(E)\ \left|\ E \in \borel{\R^N} \right.\right\}
		\end{align}
		が示され,式(\refeq{eq:stoc_proc_Q1_1})が成り立つと判る.従って
		\begin{align}
			\sigma(B_u\ :\ u \leq s) = \bigvee_{u \leq s} \left\{B_u^{-1}(E)\ \left|\ E \in \borel{\R^N} \right.\right\}
			= \bigvee_{u \leq s} \left\{(RB_u)^{-1}(E)\ \left|\ E \in \borel{\R^N} \right.\right\} 
			= \sigma(RB_u\ :\ u \leq s)
		\end{align}
		が成り立つ.任意の$0 \leq s < t$に対して$B_t - B_s$は$\mathfrak{F}_s^B = \sigma(B_u\ :\ u \leq s)$と独立であるから,
		任意の$A \in \borel{\R^N}$と$F \in \sigma(RB_u\ :\ u \leq s) = \sigma(B_u\ :\ u \leq s)$に対して,
		$R^{-1}(A) \in \borel{\R^N}$に注意すれば
		\begin{align}
			\operatorname{P}\left(\{RB_t - RB_s \in A\} \cap F \right) 
			&= \operatorname{P}\left(\{R(B_t - B_s) \in A\} \cap F \right) \\
			&= \operatorname{P}\left(\{B_t - B_s \in R^{-1}(A)\} \cap F \right) \\
			&= \operatorname{P}\left(B_t - B_s \in R^{-1}(A)\right)\operatorname{P}(F) \\
			&= \operatorname{P}\left(R(B_t - B_s) \in A\right)\operatorname{P}(F)
			= \operatorname{P}\left(RB_t - RB_s \in A\right)\operatorname{P}(F)
		\end{align}
		が成り立つ.これは任意の$0 \leq s < t$に対して$RB_t - RB_s$と$\sigma(RB_u\ :\ u \leq s)$とが独立であることを表しているから,
		(ii)も示されたことになる.(iii)について,行列式$\Det{R}$が$\pm 1$になることに注意すれば,任意の$A \in \borel{\R^N}$に対して
		\begin{align}
			\operatorname{P}(RB_t - RB_s \in A) &= \operatorname{P}\left(B_t - B_s \in R^{-1}(A)\right) \\
			&= (2\pi(t-s))^{-\frac{N}{2}} \int_{R^{-1}(A)} \mathrm{exp}\left( -\frac{|x|^2}{2(t-s)} \right)\, dx \\
			&= (2\pi(t-s))^{-\frac{N}{2}} \int_{A} \mathrm{exp}\left( -\frac{|y|^2}{2(t-s)} \right)\, dy & \left(y = Rx\mbox{として変数変換}\right)
		\end{align}
		が成り立つことにより,任意の$0 \leq s < t$に対して$RB_t - RB_s$もまた平均ベクトル0,共分散行列$(t-s)I_N$の$N$次元Gauss型確率変数
		であることが示された.最後に(iv)が満たされていることを確認する.今,$\operatorname{P}_{B_0} = \delta_0$を仮定している.
		全単射線型写像$R$について$0 \in A \Leftrightarrow 0 \in R^{-1}(A)\ (\forall A \in \borel{\R^N})$であることに注意すれば
		\begin{align}
			\operatorname{P}_{RB_0}(A) = \operatorname{P}_{B_0}(R^{-1}(A)) 
			= \begin{cases}
				1 & 0 \in R^{-1}(A) \\
				0 & 0 \notin R^{-1}(A)
			\end{cases}
			= \begin{cases}
				1 & 0 \in A \\
				0 & 0 \notin A
			\end{cases}
			= \operatorname{P}_{B_0}(A)
		\end{align}
		となり,$\operatorname{P}_{RB_0}$と$\operatorname{P}_{B_0}$は$\borel{\R^N}$の上で一致する.
		\QED
	
	\item[(2)] 上に載せた定義の番号の順番に照合していく.(i)について,
		これも連続写像の合成
		\begin{align}
			[0,\infty) \ni t \longmapsto ct \longmapsto \tfrac{1}{\sqrt{c}} B_{ct}(\omega) \in \R^N, \quad(\forall \omega \in \Omega)
		\end{align}
		と見做せばよい.(ii)について,(i)と同様に考えればよい.写像$\R^N \ni x \longmapsto x/\sqrt{c} \in \R^N$は
		$\R^N \rightarrow \R^N$の連続な全単射であり,明らかに逆写像$\R^N \ni x \longmapsto \sqrt{c} \in \R^N$もまた
		連続な全単射である.従って任意の$A \in \borel{\R^N}$に対して
		\begin{align}
			\tfrac{1}{\sqrt{c}} A \coloneqq \left\{ x/\sqrt{c}\ \left|\ x \in A \right.\right\} \in \borel{\R^N}, \quad
			\sqrt{c} A \coloneqq \left\{ \sqrt{c}x\ \left|\ x \in A \right.\right\} \in \borel{\R^N}
		\end{align}
		が成り立つから,任意の$t \geq 0$に対して
		\begin{align}
			\left\{\left. \tfrac{1}{\sqrt{c}}B_{ct} \in A\ \right|\ A \in \borel{\R^N} \right\}
			= \left\{\left. B_{ct} \in A\ \right|\ A \in \borel{\R^N} \right\}
		\end{align}
		が成り立つ.即ち任意の$s \geq 0$に対して
		\begin{align}
			\sigma\left(\tfrac{1}{\sqrt{c}}B_{cu}\ :\ u \leq s\right)
			\coloneqq \bigvee_{u \leq s} \left\{ \tfrac{1}{\sqrt{c}}B_{cu} \in A\ \left|\ A \in \borel{\R^N} \right.\right\}
			= \bigvee_{u \leq s} \left\{ B_{cu} \in A\ \left|\ A \in \borel{\R^N} \right.\right\}
			= \sigma(B_{cu}\ :\ u \leq s)
		\end{align}
		となっていて,さらに設問の仮定により任意の$0 \leq s < t$に対して$B_{ct} - B_{cs}$は$\sigma(B_{cu}\ :\ u \leq s)$と独立である.
		以上より,任意の$A \in \borel{\R^N}$と$F \in \sigma\left(\tfrac{1}{\sqrt{c}}B_{cu}\ :\ u \leq s\right) = \sigma(B_{cu}\ :\ u \leq s)$に対して
		\begin{align}
			\operatorname{P}\left( \left\{ \tfrac{1}{\sqrt{c}}B_{ct} - \tfrac{1}{\sqrt{c}}B_{cs} \in A \right\} \cap F \right)
			&= \operatorname{P}\left( \left\{ B_{ct} - B_{cs} \in \sqrt{c}A \right\} \cap F \right) \\
			&= \operatorname{P}\left(B_{ct} - B_{cs} \in \sqrt{c}A \right)  \operatorname{P}(F) \\
			&= \operatorname{P}\left(\tfrac{1}{\sqrt{c}}B_{ct} - \tfrac{1}{\sqrt{c}}B_{cs} \in A \right)  \operatorname{P}(F)
		\end{align}
		が成り立つから,任意の$0 \leq s < t$に対して$\tfrac{1}{\sqrt{c}}B_{ct} - \tfrac{1}{\sqrt{c}}B_{cs}$は
		$\sigma\left(\tfrac{1}{\sqrt{c}}B_{cu}\ :\ u \leq s\right)$と独立であると示された.
		(iii)について,これもヤコビアンが$\sqrt{c}$になることに注意すれば
		\begin{align}
			\operatorname{P}\left( \tfrac{1}{\sqrt{c}}B_{ct} - \tfrac{1}{\sqrt{c}}B_{cs} \in A \right)
			&= \operatorname{P}\left( B_{ct} - B_{cs} \in \sqrt{c}A \right) \\
			&= (2\pi(ct-cs))^{-\frac{N}{2}} \int_{\sqrt{c}A} \mathrm{exp}\left( -\frac{|x|^2}{2(ct-cs)} \right)\, dx \\
			&= (2\pi(t-s))^{-\frac{N}{2}} \int_{A} \mathrm{exp}\left( -\frac{|y|^2}{2(t-s)} \right)\, dy 
				& \left( y = \tfrac{1}{\sqrt{c}}x\mbox{と変数変換}\right) \\
		\end{align}
		が成り立つことにより,任意の$0 \leq s < t$に対して$\tfrac{1}{\sqrt{c}}B_{ct} - \tfrac{1}{\sqrt{c}}B_{cs}$
		は平均ベクトル0,共分散行列$(t-s)I_N$の$N$次元Gauss型確率変数である.最後に(iv)を示す.
		任意の$A \in \borel{\R^N}$に対して
		$0 \in A \Leftrightarrow 0 \in \sqrt{c}A$であることに注意すれば,
		\begin{align}
			\operatorname{P}_{\tfrac{1}{\sqrt{c}}B_0}(A) = \operatorname{P}_{B_0}(\sqrt{c}A) 
			= \begin{cases}
				1 & 0 \in \sqrt{c}A \\
				0 & 0 \notin \sqrt{c}A
			\end{cases}
			= \begin{cases}
				1 & 0 \in A \\
				0 & 0 \notin A
			\end{cases}
			= \operatorname{P}_{B_0}(A)
		\end{align}
		が成り立つから$\operatorname{P}_{\tfrac{1}{\sqrt{c}}B_0}$と$\operatorname{P}_{B_0}$は$\borel{\R^N}$の上で一致する.
\end{description}
\end{prf}

\section{レポート課題その2}

勝手で申し訳ございませんが,レポート問題ではなくても問題を解く際に必要になる部分をメモとしてここに載せることにいたします.
\begin{dfn}[$(\mathcal{F}_t)$-Brown運動(講義資料引用)]
	$\mu$を$\R^N$上の分布(i.e.Borel確率測度)とする.フィルター付き確率空間$(\Omega, \mathcal{F}, \operatorname{P}, (\mathcal{F}_t))$上の
	$(\mathcal{F}_t)$-適合$\R^N$-値確率過程$B = (B_t)_{t \geq 0}$で以下をみたすものを,初期分布$\mu$の$N$次元$(\mathcal{F}_t)$-Brown運動という.
	とくに,$\mu$が$x \in \R^N$のDirac測度$\delta_x$のとき,$B$は$x$から出発する$(\mathcal{F}_t)$-Brown運動と呼ばれる.
	\begin{description}
		\item[\rm{(i)}] 任意の$\omega \in \Omega$に対して,$[0,\infty) \ni t \longmapsto B_t(\omega) \in \R^N$は連続.
		\item[\rm{(ii)}] 任意の$0 \leq s < t$に対して$B_t - B_s$は$\mathcal{F}_s$と独立.
		\item[\rm{(iii)}] 任意の$0 \leq s < t$に対して$B_t - B_s$は平均ベクトル0,共分散行列$(t-s)I_N$の$N$次元Gauss型確率変数
			である.ここで$I_N$は$N$次元単位行列を表す.
		\item[\rm{(iv)}] $\operatorname{P}_{B_0} = \mu$.
	\end{description}
\end{dfn}

\begin{thm}[命題3.9'\ :\ 命題3.9を点$x$出発の$(\mathcal{F}_t)$-Brown運動で考えたもの]
$B = (B_t)$を点$x \in \R^1$から出発する1次元$(\mathcal{F}_t)$-Brown運動とするとき,以下の事実を確かめることができる.
\begin{description}
	\item[(1)] $s,t \geq 0$に対して$\Exp{B(t)B(s)} = t \wedge s + x^2$.
	\item[(2)] $t \geq 0$と正整数$n$に対して
		\begin{align}
			\Exp{(B(t)-B(0))^n} = \begin{cases} 
				0 & \mbox{$n$が奇数} \\ 
				(n-1)!!t^{n/2} & \mbox{$n$が偶数} 
			\end{cases},
		\end{align}
		ただし$(2n-1)!! = (2n - 1)(2n - 3)\cdots3 \cdot 1$.
	\item[(3)] 
\end{description}
\end{thm}

\begin{prf}
\begin{description}\mbox{}
	\item[(1)] 
		$t = s = 0$の場合,
		\begin{align}
			\Exp{B(0)^2} = x^2.
		\end{align}
		$t = s > 0$の場合,
		\begin{align}
			\Exp{B(t)^2} = \Exp{(B(t) - B(0) + B(0))^2} = \Exp{(B(t) - B(0))^2} + 2\Exp{(B(t) - B(0))B(0)} + \Exp{B(0)^2} = t + x^2.
		\end{align}
		$t > s \geq 0$の場合,
		\begin{align}
			\Exp{B(t)B(s)} = \Exp{(B(t) - B(s) + B(s))B(s)} = \Exp{(B(t) - B(s))B(s)} + \Exp{B(s)^2} = \Exp{B(s)^2} =  s + x^2.
		\end{align}
\end{description}
\end{prf}

$N$を正整数,$x = {}^t(x_1,\cdots,x_N) \in \R^N$に対して,$|x| = \sqrt{x_1^2 + \cdots + x_N^2}$
でEuclidのノルムを定義する.$B^x = \left(B^x(t)\right)_{t \geq 0}$を$x$から出発する$N$-次元$(\mathcal{F}_t)$-Brown運動
とし,$\sigma$を$(\mathcal{F}_t)$-停止時間とする,このとき,次の(1), (2), (3)に回答せよ.
\begin{description}
	\item[(1)] $|B_x|^2 = \left(\left|B^x(t)\right|^2\right)_{t \geq 0}$はクラス(DL)に属するSbMGであることを示し,
		そのDoob-Meyer分解を求めよ.
	\item[(2)] 任意の$t \geq 0$に対して$\Exp{|B^x(\sigma \wedge t)|^2} = N\Exp{\sigma \wedge t} + |x|^2$が成り立つことを示せ.
	\item[(3)] $D$を$\R^N$の有界領域とし,$x \in D$とする.$\sigma_D$を領域$D$からの脱出時間
		$\sigma_D = \inf{}{\{ t > 0\ :\ B^x(t) \in \R^N \backslash D \}}$とするとき,$\prob{\sigma_D < \infty} = 1$が成り立つことを示せ.
\end{description}

\begin{prf}
\begin{description}\mbox{}
	\item[(1)] 命題3.9により$B_i^{x_i}\ (i = 1,2,\cdots,N)$が$(\mathcal{F}_t)$-マルチンゲールであるとわかっているから,
		凸関数$|\cdot|^2$で変換することにより$\left|B_i^{x_i}\right|^2\ (i = 1,2,\cdots,N)$は$(\mathcal{F}_t)$-劣マルチンゲール
		である.従ってその有限和で表される$|B^x|^2 = \left(\left|B^x(t)\right|^2\right)_{t \geq 0}$も
		$(\mathcal{F}_t)$-劣マルチンゲールである.実際,$B_i^{x_i}\ (i = 1,2,\cdots,N)$が$(\mathcal{F}_t)$-適合過程で命題3.9より任意の$t \geq 0$
		で二乗可積分であることから,$|B^x|^2$についても$(\mathcal{F}_t)$-適合で任意の$t \geq 0$で可積分であることが従い,
		また任意の$0 \leq s < t$に対して$A \in \mathcal{F}_s$を任意に取れば
		\begin{align}
			\int_{A} |B^x(t,\omega)|^2\, \prob{d\omega} 
			= \sum_{i=1}^{N} \int_{A} |B_i^{x_i}(t,\omega)|^2\, \prob{d\omega} 
			\geq \sum_{i=1}^{N} \int_{A} |B_i^{x_i}(s,\omega)|^2\, \prob{d\omega} 
			= \int_{A} |B^x(s,\omega)|^2\, \prob{d\omega} 
		\end{align}
		が成り立つから$|B^x|^2$が$(\mathcal{F}_t)$-劣マルチンゲールであると判る.次に$|B^x|^2$がクラス(DL)に属することを示す.
		任意に$a > 0$を固定する.講義資料に倣い$\bf{S}_a$を$\sigma(\omega) \leq a\ (\forall \omega \in \Omega)$
		を満たす$(\Omega, \mathcal{F}, \operatorname{P}, (\mathcal{F}_t)_{t \geq 0})$上の停止時刻$\sigma$全体を表すとする.
		任意抽出定理(講義資料定理2.21)を適用すれば,任意の$\sigma \in \bf{S}_a$と$c > 0$に対して
		\begin{align}
			\int_{|B^x(\sigma)|^2 \geq c} |B^x(\sigma(\omega), \omega)|^2\, \prob{d\omega}
			\leq \int_{|B^x(\sigma)|^2 \geq c} |B^x(a, \omega)|^2\, \prob{d\omega}
		\end{align}
		が成り立つ.Chebyshevの不等式により
		\begin{align}
			\prob{|B^x(\sigma)|^2 \geq c} \leq \frac{1}{c} \int_{\Omega} |B^x(a, \omega)|^2\, \prob{d\omega} \label{eq:stoc_report_Q2}
		\end{align}
		も成り立ち,右辺が可積分であるから$\sigma$によらずに$c$の値のみで右辺をいくらでも小さくできる.可積分関数$|B^x(\sigma)|^2$について,
		任意の$\epsilon > 0$に対して或る$\delta > 0$が存在し,$\prob{A} < \delta$なる任意の$A \in \mathcal{F}$上での積分は$< \epsilon$となる.
		従って(\refeq{eq:stoc_report_Q2})の右辺を$ < \delta$となるような$c > 0$を選べば,全ての$c' > c$に対して
		\begin{align}
			\sup{\sigma \in \bf{S}_a}{\int_{|B^x(\sigma)|^2 \geq c'} |B^x(\sigma(\omega), \omega)|^2\, \prob{d\omega}} < \epsilon
		\end{align}
		が成り立つ.これは確率変数の族$\left(|B^x(\sigma)|^2\right)_{\sigma \in \bf{S}_a}$が一様可積分であることを表している.
		最後に$|B^x(\sigma)|^2$のDoob-Meyer分解を求める.命題3.9により$\left(|B_i^{x_i}(t)|^2 - t\right)_{t \geq 0}$が
		$(\mathcal{F}_t)$-マルチンゲールであるとわかっているから,$\left(|B^x(t)|^2 - Nt\right)_{t \geq 0}$もまた
		$(\mathcal{F}_t)$-マルチンゲールである.実際,$(\mathcal{F}_t)$-適合であることと可積分性は上に書いた理由で問題なく,
		任意の$0 \leq s < t$と$A \in \mathcal{F}_s$に対して
		\begin{align}
			\int_{A} |B^x(t, \omega)|^2 - Nt\, \prob{d\omega} 
			&= \int_{A} \sum_{i=1}^{N} |B_i^{x_i}(t,\omega)|^2 - Nt\, \prob{d\omega} \\
			&= \sum_{i=1}^{N} \int_{A} |B_i^{x_i}(t,\omega)|^2 - t\, \prob{d\omega} \\
			&= \sum_{i=1}^{N} \int_{A} |B_i^{x_i}(s,\omega)|^2 - s\, \prob{d\omega} \\
			&= \int_{A} |B^x(s, \omega)|^2 - Ns\, \prob{d\omega} 
		\end{align}
		も成り立つと確認された.これが求めるDoob-Meyer分解になっていることを確認する.講義資料の定理2.25に則れば,
		まず$\left(|B^x(t)|^2\right)_{t \geq 0}$がクラス(DL)に属している$(\mathcal{F}_t)$-劣マルチンゲールであり
		$\left(|B^x(t)|^2 - Nt\right)_{t \geq 0}$は$(\mathcal{F}_t)$-マルチンゲールであるから,
		あとは$(Nt)_{t \geq 0}$が予測可能な可積分増加過程であれば良い.
		$Nt$は明らかに左連続であって,更に$(\mathcal{F}_t)$-適合過程の差で表現できるから$(\mathcal{F}_t)$-適合過程で,
		従ってこれは予測可能である.また$\omega \in \Omega$に無関係に$N 0 = 0$,
		$Nt$は$t$の右連続な単調増加関数であって,全ての$t \geq 0$で$\Exp{Nt} = Nt < +\infty$が成り立っていることにより,これは可積分増加過程でもある.
		\QED
	
	\item[(2)] 一般にマルチンゲールを停止時間で停めた過程もまたマルチンゲールとなることをいえばよい.
		フィルター付き確率空間$(\Omega, \mathcal{F}, \operatorname{P}, (\mathcal{F}_t))$上の$\R^1$値確率過程$(X_t)_{t \geq 0}$
		が$(\mathcal{F}_t)$マルチンゲールであるとする.この確率空間上の停止時間$\sigma$を任意に取り$(X_{\sigma \wedge t})_{t \geq 0}$を考える.
		$(x_t)_{t \geq 0}$が連続で$(\mathcal{F}_t)$-適合であることから$(\mathcal{F}_t)$-発展的可測となり,講義資料命題2.20により
		全ての$t$で$X_{\sigma \wedge t}\ \left(= X_{\sigma \wedge t}I_{(\sigma \wedge t < +\infty)}\right)$
		は可測$\mathcal{F}_{\sigma \wedge t}/\borel{\R^1}$となる.$\mathcal{F}_{\sigma \wedge t}
		\subset \mathcal{F}_t$により$(X_{\sigma \wedge t})_{t \geq 0}$もまた$(\mathcal{F}_t)$-適合であると判る.
		全ての$t$で$X_{\sigma \wedge t}$が可積分となることは,任意抽出定理(講義資料定理2.21)により
		\begin{align}
			\Exp{X_t\ |\ \mathcal{F}_{\sigma \wedge t}} = X_{\sigma \wedge t}, \quad a.s.
		\end{align}
		となることより従う.マルチンゲール性の三つ目の性質が満たされるかを確認する.
		任意の時間$0 \leq s < t$に対して$A \in \mathcal{F}_s$を任意に取る.このとき
		\begin{align}
			A \cap \{ \sigma \wedge t > s \} \cap \{\sigma \leq u\} = \begin{cases}
				A \cap \{s < \sigma \leq u\} \in \mathcal{F}_u & (u \geq s) \\
				\emptyset \in \mathcal{F}_u & (u < s)
			\end{cases}, \quad \forall u \in [0,\infty)
		\end{align}
		が成り立つことから$A \cap \{ \sigma \wedge t > s \} \in \mathcal{F}_{\sigma}$である.$\sigma \wedge t$が停止時間であるから
		$A \cap \{ \sigma \wedge t > s \} \in \mathcal{F}_s$でもあり,従って
		\begin{align}
			A \cap \{ \sigma \wedge t > s \} \in \mathcal{F}_s \cap \mathcal{F}_{\sigma} = \mathcal{F}_{\sigma \wedge s}
		\end{align}
		が成り立つ.任意抽出定理(講義資料定理2.21)を適用すれば
		\begin{align}
			\int_{A \cap \{ \sigma \wedge t > s \}} X(\sigma(\omega) \wedge t, \omega)\, \prob{d\omega}
			= \int_{A \cap \{ \sigma \wedge t > s \}} X(\sigma(\omega) \wedge s, \omega)\, \prob{d\omega}
		\end{align}
		と表すことができる.一方で$A \cap \{ \sigma \wedge t \leq s \}$上の積分も考えると,$s < t$としているから
		この集合の上で$\sigma \wedge t = \sigma \wedge s = \sigma$が成り立っていることに注意して
		\begin{align}
			\int_{A \cap \{ \sigma \wedge t \leq s \}} X(\sigma(\omega) \wedge t, \omega)\, \prob{d\omega}
			= \int_{A \cap \{ \sigma \wedge t \leq s \}} X(\sigma(\omega) \wedge s, \omega)\, \prob{d\omega}
		\end{align}
		が成り立つ.二つの積分を併せれば
		\begin{align}
			\int_{A} X(\sigma(\omega) \wedge t, \omega)\, \prob{d\omega}
			= \int_{A} X(\sigma(\omega) \wedge s, \omega)\, \prob{d\omega}
		\end{align}
		が成り立つ.時間$0 \leq s < t$と$A \in \mathcal{F}_s$は任意であったから,
		$\sigma$で停めた過程$(X_{\sigma \wedge t})_{t \geq 0}$もまた$(\mathcal{F}_t)$-マルチンゲールであると示された.
		以上の結果を用いれば,(1)における$(\mathcal{F}_t)$-マルチンゲール$\left(|B^x(t)|^2 - Nt\right)_{t \geq 0}$
		に対して
		\begin{align}
			&\int_{\Omega} |B^x(\sigma(\omega) \wedge t,\omega)|^2 - N(\sigma(\omega) \wedge t)\, \prob{d\omega} \\
			&\qquad= \int_{\Omega} |B^x(0,\omega)|^2\, \prob{d\omega}
			= \sum_{i=1}^{N} \int_{\Omega} B_i^{x_i}(0,\omega)^2\, \prob{d\omega}
			= \sum_{i=1}^{N} x_i^2
			= |x|^2
		\end{align}
		が成り立つ.右辺の変形は上に乗せた命題3.9'の(1)による.左辺の被積分関数はどちらも可積分関数であるから,以上で任意の$t \geq 0$に対して
		\begin{align}
			\Exp{|B^x(\sigma \wedge t)|^2} = N\Exp{\sigma \wedge t} + |x|^2
		\end{align}
		が成り立つことが示された.
		
	\item[(3)] 講義資料定義2.8により$\sigma_D$は広義停止時間であるが,同資料仮定2.11によりフィルトレーションは右連続であるから,命題2.7により
		$\sigma_D$は停止時間として扱うことができる.
		(2)の結果により任意の$t \geq 0$に対して
		\begin{align}
			\Exp{|B^x(\sigma_D \wedge t)|^2} = N\Exp{\sigma_D \wedge t} + |x|^2 \label{eq:stoc_report_Q2_3}
		\end{align}
		が成り立つ.ここで左辺が$t$に関して一様に有界であることを証明する.各$\omega \in \Omega$ごとに,写像$[0,+\infty) \ni t \longmapsto B^x(t, \omega)$
		が連続であることと$D$が開集合であることにより$0 \leq s \leq \sigma_D(\omega)$であるような$s$に対して$B^x(s, \omega) \in \overline{D}$となる.
		ここで$\overline{D}$は$D$の閉包を表すとする.$D$が$\R^N$の有界領域であるから
		(つまり十分大きな$n \in \N$に対して$D$は原点中心半径$n$の閉球に含まれる.)
		$\overline{D}$も$\R^N$の有界閉集合となる.ここで
		\begin{align}
			d \coloneqq \sup{}{\left\{|x - y|\ :\ x,y \in \overline{D} \right\}}
		\end{align}
		とおく.等式(\refeq{eq:stoc_report_Q2_3})の左辺の被積分関数について,時刻の部分は$\sigma_D(\omega) \wedge t \leq \sigma_D(\omega)$
		が全ての$\omega \in \Omega$で成立しているから,$\omega$ごとに$B^x(\sigma_D(\omega) \wedge t, \omega)$は$\overline{D}$に属している.
		従って$|B^x(\sigma_D(\omega) \wedge t, \omega) - x| \leq d \ (\forall \omega \in \Omega)$で抑えられるから
		\begin{align}
			\Exp{|B^x(\sigma_D \wedge t)|^2} 
			&= \int_{\Omega} |B^x(\sigma_D(\omega) \wedge t, \omega)|^2\, \prob{d\omega} \\
			&= \int_{\Omega} |B^x(\sigma_D(\omega) \wedge t, \omega) - x + x|^2\, \prob{d\omega} \\
			&= \int_{\Omega} |B^x(\sigma_D(\omega) \wedge t, \omega) - x|^2 + 2\inprod<B^x(\sigma_D(\omega) \wedge t, \omega) - x,\ x> + |x|^2\, \prob{d\omega} \\
			&\leq d^2 + 2d|x| + |x|^2
		\end{align}
		が成り立つ.ここで$\inprod<\cdot, \cdot>$は$\R^N$の標準的内積を表し(つまり$\inprod<x, y> = \sum_{i=1}^{N}x_iy_i$.),最後の式変形でSchwarzの不等式を使った.
		等式(\refeq{eq:stoc_report_Q2_3})にこの結果を適用すれば
		\begin{align}
			\Exp{\sigma_D \wedge t} \leq \frac{d^2 + 2d|x|}{N}
		\end{align}
		が$t \geq 0$に依らずに成り立つ.これにより$\prob{\sigma_D = +\infty} = 0$が示される.
		もし$\alpha \coloneqq \prob{\sigma_D = +\infty} > 0$であるとすれば,集合$\{ \sigma_D = +\infty \}$
		の上では$\sigma_D$の値をいくらでも大きくできるから
		$t > (d^2 + 2d|x|)/(\alpha N)$となる$t$に対して
		\begin{align}
			\frac{d^2 + 2d|x|}{\alpha N} \prob{\sigma_D = +\infty} 
			< \int_{\{\sigma_D = +\infty\}} \sigma_D(\omega) \wedge t\, \prob{d\omega}
			\leq \int_{\Omega} \sigma_D(\omega) \wedge t\, \prob{d\omega}
			\leq \frac{d^2 + 2d|x|}{N}
		\end{align}
		が成り立ち矛盾ができるからである.ゆえに$\prob{\sigma_D < +\infty} = 1$が示された.
		\QED
\end{description}
\end{prf}
%
%
%\appendix
\newpage
\printindex
%
%
\end{document}
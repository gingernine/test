\section{Markov連鎖}
	基礎となる確率空間$(\Omega, \mathcal{F}, \operatorname{P})$.
	\begin{itemize}
		\item $E$:\ 集合,
		\item $(E, \mathcal{E})$:\ 可測空間,
		\item $(X_n)_{n=1}^{+\infty}$:\ $E$-値確率過程.
	\end{itemize}
	\begin{rem}
		\ref{sec:first}章 $\sim$ \ref{sec:ergodic}章は$E$が高々可算集合であるとして考える.
	\end{rem}
\section{Markov連鎖}
	\label{sec:first}
	\begin{dfn}[Markov性]
		$\forall n \in \N,\ i_0, i_1, \cdots,i_n \in E,$
		\begin{align}
			 &\cprob{X_n = i_n}{X_0 = i_0,\ X_1 = i_1,\ \cdots, X_{n-1} = i_{n-1}} \\
			 &\qquad= \cprob{X_n = i_n}{X_{n-1} = i_{n-1}}.
		\end{align}
	\end{dfn}
	$(X_n)_{n=1}^{+\infty}$がMarkov性を持つ場合,これをMarkov連鎖という.
	以後$(X_n)_{n=1}^{+\infty}$はMarkov連鎖.

\section{Markov行列}
	\begin{dfn}[Markov 行列]
		$(i,j)$成分$(\forall i,j \in E)$を$\cprob{X_1 = j}{X_0 = i}$とする確率行列.
		行列を$P$,$(i,j)$成分を$[P]_{ij}$と表記.計算規則は以下.
		\begin{align}
			P^0 &= I, && (\mbox{$I$:恒等写像}),\\
			[P^n]_{ij} &= \sum_{k \in E} [P^{n-1}]_{ik} [P]_{kj}, && (\forall i,j \in E,\ n \in \N). 
		\end{align}
	\end{dfn}
	定義から次が成立
	\begin{align}
		[P^n]_{ij} = \cprob{X_n = j}{X_0 = i}, \ (\forall n \in \N,\ i,j \in E).
	\end{align}

\section{Chapman-Kolmogorov方程式}
	\begin{prp}[Chapman-Kolmogorov方程式]
		任意の$n,m = 0,1,2,\cdots$と$i,j \in E$に対し次が成立.
		\begin{align}
			[P^{n+m}]_{ij} = \sum_{k \in E}[P^n]_{ik}[P^m]_{kj}. 
		\end{align}
	\end{prp}
	
	この命題は以降の命題の証明において基礎的である.

\section{既約性・再帰性}
	\begin{dfn}[既約性]
		$P$が既約である
		\begin{align}
			\DEF\ \forall i, j \in E,\ \exists n \in \N,\ \mathrm{s.t. \quad} [P^n]_{ij} > 0.
		\end{align}
	\end{dfn}
	\begin{dfn}[再帰性]
		$P$が再帰的である
		\begin{align}
			\DEF\ \cprob{\exists n \geq 1,\ X_n = i}{X_0 = i} = 1 \quad(\forall i \in E).
		\end{align}
		$P$が非再帰的である
		\begin{align}
			\DEF\ \cprob{\forall n \geq 1,\ X_n \neq i}{X_0 = i} > 0 \quad(\forall i \in E).
		\end{align}
	\end{dfn}
	
\section{離散空間上のMarkov連鎖}
	\begin{dfn}[到達時刻と到達回数]
		$\forall i \in E,\ \omega \in \Omega,$
		\begin{description}
			\item[到達時刻] $\tau_i(\omega) \coloneqq \inf{}{\left\{ n \geq 1\ \left|\ X_n(\omega) = i \right.\right\}},$
			\item[到達回数] $\eta_i(\omega) \coloneqq \sum_{n=1}^{+\infty} \defunc_{(X_n = i)}(\omega).$
		\end{description}
	\end{dfn}
	$p_{ij} \coloneqq \cprob{\tau_j < \infty}{X_0 = i}, \quad (\forall i,j \in E)$\\
	と表記すれば次が成立:
	\begin{align}
		p_{ii} &= \cprob{\exists n \geq 1,\ X_n = j}{X_0 = i}, \\
		p_{ii} &< 1 \Leftrightarrow \cexp{\eta_i}{X_0 = i} < +\infty, \quad (\forall i \in E).
	\end{align}

\section{正再帰性}
	\begin{dfn}[不変確率測度]
		$E$上の確率測度$\pi = ([\pi]_i)_{i \in E},\ (\sum_{i \in E} [\pi]_i = 1)$が$P$に対して不変確率測度である
		\begin{align}
			\DEF\ [\pi]_i = [\pi P]_i ( = \sum_{j \in E}[\pi]_j [P]_{ji} ), \quad (\forall i \in E).
		\end{align}
	\end{dfn}
	\begin{dfn}[正再帰性]
		$P$は正再帰的 \\
		$\quad\DEF\quad$ $P$が既約かつ不変確率測度が存在. 
	\end{dfn}
\section{再帰性の諸命題}
	\begin{prp}
		$P$が既約の下,(i) $\sim$ (iv)が順に示される:
		\begin{description}
			\item[\rm{(i)}] $P$が再帰的 $\Leftrightarrow \cexp{\eta_i}{X_0 = i} = +\infty, \ (\forall i \in E)$.
			\item[\rm{(ii)}] $P$は再帰的であるか非再帰的のどちらか.特に$E$が有限集合なら$P$は再帰的.
			\item[\rm{(iii)}] $P$が正再帰的 $\Rightarrow$ $P$は再帰的.
			\item[\rm{(iv)}] $E$が有限集合なら$P$は正再帰的.
		\end{description}
	\end{prp}

\section{周期}
	\begin{dfn}[$i \in E$の周期]
		$\mathcal{N}_i \coloneqq \left\{n \geq 1\ \left|\ [p^n]_{ii} > 0 \right.\right\}$の最大公約数を
		$i \in E$の周期といい$d_i$と表す.
	\end{dfn}
	\begin{prp}[既約なら周期はunique]
		$P$が既約ならば$d_i = d_j\ (\forall i,j \in E)$.
		この場合$d_i$を$P$の周期という.
	\end{prp}
	\begin{dfn}[非周期性]
		$P$が既約の下,
		\begin{align}
			\mbox{$P$は非周期的 $\quad\DEF\quad$ $P$の周期が1}.
		\end{align}
	\end{dfn}

\section{Ergodicity}
	\label{sec:ergodic}
	\begin{prp}[周期に関する一命題]
		$P$:既約,非周期的,
		\begin{align}
			\forall i,j \in E,\ \exists n_{ij} \in \N,\ \mathrm{s.t.}\ [P^n]_{ij} > 0 \ (\forall n \geq n_{ij}).
		\end{align}
	\end{prp}
	
	\begin{thm}[Ergodicity]
		$P$が既約で非周期的かつ正再帰的であるとする.$P$の不変確率測度を$\pi$で表すとき次が成立.
		\begin{align}
			\lim_{n \to +\infty}[P^n]_{ij} = [\pi]_j, \quad (\forall i,j \in E).
		\end{align}
	\end{thm}
	\begin{prf}\mbox{}
	\begin{description}
	\item[第一段]
		直積空間$E \times E$上のMarkov連鎖を考える.$E \times E$のMarkov行列を$Q$と表し
		\begin{align}
			[Q]_{ik,jl} \coloneqq [P]_{ij}[P]_{kl}, \quad (\forall (i,k), (j,l) \in E \times E)
		\end{align}
		と定義する.
		$(\Omega,\mathcal{F},\operatorname{P})$上の$E$-値確率過程$(X_n)_{n=1}^{+\infty}, (Y_n)_{n=1}^{+\infty}$がそれぞれ
		Markov行列$P$を持つ独立なMarkov連鎖であるとすれば,$Z_n=(X_n,Y_n)\ (n=1,2,\cdots)$は$E \times E$上のMarkov連鎖で
		Markov行列$Q$を持つ.なぜならば任意の$n \in \N$と$(i,j),(i_0,j_0),\cdots,(i_{n-1},j_{n-1}) \in E \times E$に対して
		\begin{align}
			&\cprob{Z_n = (i,j)}{Z_0=(i_0,j_0),Z_1=(i_1,j_1), \cdots, Z_{n-1} = (i_{n-1},j_{n-1})} \\
			&= \frac{\prob{Z_n = (i,j), Z_0=(i_0,j_0),\cdots, Z_{n-1} = (i_{n-1},j_{n-1})}}{\prob{Z_0=(i_0,j_0),\cdots, Z_{n-1} = (i_{n-1},j_{n-1})}} \\
			&= \frac{\prob{(X_n=i,X_0=i_0,\cdots, X_{n-1} = i_{n-1})\cap(Y_n=j,Y_0=j_0,\cdots, Y_{n-1}=j_{n-1})}}{\prob{(X_0=i_0,\cdots, X_{n-1} = i_{n-1})\cap(Y_0=j_0,\cdots, Y_{n-1}=j_{n-1})}} \\
			&= \frac{\prob{X_n=i,X_0=i_0,\cdots, X_{n-1} = i_{n-1}}\prob{Y_n=j,Y_0=j_0,\cdots, Y_{n-1}=j_{n-1}}}{\prob{X_0=i_0,\cdots, X_{n-1}=i_{n-1}}\prob{Y_0=j_0,\cdots, Y_{n-1}=j_{n-1}}} \\
			&= \cprob{X_n = i}{X_{n-1}=i_{n-1}}\cprob{Y_n = j}{Y_{n-1}=j_{n-1}} \label{eq:ergodic_discrete_1}\\
			&= \frac{\prob{X_n=i,Y_n=j,X_{n-1}=i_{n-1},Y_{n-1}=j_{n-1}}}{\prob{X_{n-1}=i_{n-1},Y_{n-1}=j_{n-1}}} \\
			&= \cprob{Z_n=(i,j)}{Z_{n-1}=(i_{n-1},j_{n-1})}
		\end{align}
		が成立するからである.また$Q$は既約かつ再帰的である.$P$が既約であるから,前命題により任意の$(i,k),(j,l) \in E \times E$に対して
		或る$n_{ij},n_{kl} \in \N$が存在し$[P^n]_{ij} > 0\ (\forall n \geq n_{ij})$と$[P^n]_{kl} > 0\ (\forall n \geq n_{kl})$
		が成立する.従って$\forall n \geq \max{}{\left\{n_{ij},n_{kl}\right\}}$に対して
		\begin{align}
			[Q^n]_{ik,jl} \coloneqq [P^n]_{ij}[P^n]_{kl} > 0
		\end{align}
		が成立するから$Q$は既約である.
		\begin{rem}
			先の式変形と同様に,$(X_n)_{n=1}^{+\infty}, (Y_n)_{n=1}^{+\infty}$の独立性から任意の$n \in \N,\ (i,k),(j,l) \in E \times E$に対して
			\begin{align}
				[Q^n]_{ik,jl} = \cprob{Z_n = (j,l)}{Z_0=(i,k)} = \cprob{X_n = j}{X_0=i}\cprob{Y_n = l}{Y_0=k} = [P^n]_{ij}[P^n]_{kl}
			\end{align}
			が導かれる.
		\end{rem}
		次に再帰性を示す.これには$Q$に対して$E \times E$上の不変確率測度が存在することを言えばよい.
		\begin{align}
			[\mu]_{ik} = [\pi]_i[\pi]_k \quad (\forall (i,k) \in E \times E)
		\end{align}
		として$\mu = ([\mu]_{ik})_{i,k \in E}$を定義すればこれは$E \times E$上の確率測度であり,任意の$(j,l) \in E \times E$に対して
		\begin{align}
			[\mu Q]_{jl} &= \sum_{(i,k) \in E \times E} [\mu]_{ik} [Q]_{ik,jl} 
			= \sum_{i,k \in E} [\pi]_i[\pi]_k [P]_{ij}[P]_{jl} 
			= \sum_{i \in E} [\pi]_i[P]_{ij}\sum_{k \in E} [\pi]_k[P]_{kl} 
			= [\pi]_j[\pi]_l
			= [\mu]_{jl}
		\end{align}
		が成り立つから$\mu$が$Q$の不変確率測度であることが判る.ゆえに$Q$は正再帰的で既約すなわち再帰的である.
	\item[第二段]
		\begin{align}
			\lim_{n \to +\infty} \left| [P^n]_{ij} - [P^n]_{kj} \right| = 0 \quad (\forall i,j,k \in E) \label{eq:discrete_ergodic}
		\end{align}
		を示す.$(Z_n)_{n=1}^{+\infty} = ((X_n, Y_n))_{n=1}^{+\infty}$に対しても同様に
		\begin{align}
			\tau_{ik}(\omega) \coloneqq \inf{}{\left\{ n \geq 1\ \left|\ Z_n(\omega) = (i,k) \right.\right\}} \quad (\forall i,k \in E,\ \omega \in \Omega)
		\end{align}
		として到達時刻を定義する.$(X_n)_{n=1}^{+\infty}, (Y_n)_{n=1}^{+\infty}$の独立性から
		\begin{align}
			[P^n]_{ij} &= \frac{\prob{X_n=j,X_0=i}}{\prob{X_0=i}} = \frac{\prob{X_n=j,X_0=i,Y_0=k}}{\prob{X_0=i,Y_0=k}} = \cprob{X_n=j}{(X_0,Y_0)=(i,k)},\\
			[P^n]_{kj} &= \frac{\prob{Y_n=j,Y_0=k}}{\prob{Y_0=k}} = \frac{\prob{Y_n=j,X_0=i,Y_0=k}}{\prob{X_0=i,Y_0=k}} = \cprob{Y_n=j}{(X_0,Y_0)=(i,k)}
		\end{align}
		が成立する.
		\begin{align}
			\cprob{X_n=j}{(X_0,Y_0)=(i,k)} &= \cprob{X_n=j, \tau_{jj} > n}{(X_0,Y_0)=(i,k)} + \cprob{X_n=j, \tau_{jj} \leq n}{(X_0,Y_0)=(i,k)}, \\
			\cprob{Y_n=j}{(X_0,Y_0)=(i,k)} &= \cprob{Y_n=j, \tau_{jj} > n}{(X_0,Y_0)=(i,k)} + \cprob{Y_n=j, \tau_{jj} \leq n}{(X_0,Y_0)=(i,k)}
		\end{align}
		と分解できるが,
		\begin{align}
			&\cprob{X_n=j, \tau_{jj} \leq n}{(X_0,Y_0)=(i,k)} \\
			&= \sum_{m=1}^{n} \cprob{X_n=j, \tau_{jj} = m}{(X_0,Y_0)=(i,k)} \\
			&= \sum_{m=1}^{n} \cprob{X_n=j, (X_m,Y_m)=(j,j), (X_{m-1},Y_{m-1}) \cdots (X_1,Y_1) \neq (j,j)}{(X_0,Y_0)=(i,k)} \\
			&= \sum_{m=1}^{n} \frac{\prob{X_n=j,X_m=j,X_{m-1},\cdots,X_1 \neq j}}{\prob{X_0=i}}
				\frac{\prob{Y_m=j,Y_{m-1},\cdots,Y_1 \neq j}}{\prob{Y_0=k}} \\
			&= \sum_{m=1}^{n} \frac{\prob{X_n=j,X_m=j,X_{m-1},\cdots,X_1 \neq j}}{\prob{X_m=j,X_{m-1},\cdots,X_1 \neq j}}
				\frac{\prob{X_m=j,X_{m-1},\cdots,X_1 \neq j}}{\prob{X_0=i}}\frac{\prob{Y_m=j,Y_{m-1},\cdots,Y_1 \neq j}}{\prob{Y_0=k}} \\
			&= \sum_{m=1}^{n} [P^{n-m}]_{jj} \frac{\prob{(X_m=j,X_{m-1},\cdots,X_1 \neq j)\cap(Y_m=j,Y_{m-1},\cdots,Y_1 \neq j)}}{\prob{(X_0=i)\cap(Y_0=k)}} \\
			&= \sum_{m=1}^{n} [P^{n-m}]_{jj} \cprob{\tau_{jj} = m}{(X_0,Y_0)=(i,k)} \\
			&= \cprob{Y_n=j, \tau_{jj} \leq n}{(X_0,Y_0)=(i,k)}
		\end{align}
		が成立することと,既約性($\cprob{\tau_{jj} < +\infty}{(X_0,Y_0)=(i,k)} = 1$)により
		\begin{align}
			&\left| \cprob{X_n=j, \tau_{jj} > n}{(X_0,Y_0)=(i,k)} - \cprob{Y_n=j, \tau_{jj} > n}{(X_0,Y_0)=(i,k)} \right| \\
			&\leq 2\cprob{\tau_{jj} > n}{(X_0,Y_0)=(i,k)} \\
			&= 2 \left( 1 - \cprob{\tau_{jj} \leq n}{(X_0,Y_0)=(i,k)} \right) \\
			&\longrightarrow 0 \quad (n \longrightarrow +\infty)
		\end{align}
		が成り立つことから式(\refeq{eq:discrete_ergodic})が成立する.
	\item[第三段]
		$\sum$を測度空間$(E,\mathcal{E},\pi)$上の積分と見做してLebesgueの収束定理を使う.
		\begin{align}
			\left| [P^n]_{ij} - [\pi]_j \right| &= \left| [P^n]_{ij} - [\pi P^n]_j \right| \\
			&= \left| [P^n]_{ij} - \sum_{k \in E}[\pi]_k[P^n]_{kj} \right| \\
			&= \left| \sum_{k \in E}[\pi]_k[P^n]_{ij} - \sum_{k \in E}[\pi]_k[P^n]_{kj} \right| \\
			&\leq \sum_{k \in E}[\pi]_k\left| [P^n]_{ij} - [P^n]_{kj} \right| \\
			&\longrightarrow 0 \quad (n \longrightarrow +\infty).
		\end{align}
		以上で命題の主張が示された.
		\QED
	\end{description}
	\end{prf}
	
	
	
	
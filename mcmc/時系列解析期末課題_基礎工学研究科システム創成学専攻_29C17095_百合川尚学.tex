\documentclass[a4j,papersize,disablejfam,slide,20pt]{jsarticle}
\usepackage[dvipdfmx]{graphicx}
\usepackage{xcolor}
\usepackage{lastpage}
\usepackage{fancyhdr}
\renewcommand{\headrulewidth}{0.0pt}
\pagestyle{fancy}
\lhead{}
\chead{}
\rhead{}
\lfoot{}
\rfoot{\thepage{}/{}\pageref{LastPage}}
\rfoot{}
\usepackage[T1]{fontenc}
\usepackage{textcomp}
\usepackage[utf8]{inputenc}
\usepackage{bm}
%%----------- documentclass から17行はコピペ.意味はわからず.-----------%
% 参考:
% 	「何かを書き留める何か LaTeX + jsarticle + slide でスライドを作る」
% 	url: http://xaro.hatenablog.jp/entry/2013/09/26/004920
%------------------------------------------------------------------------%
\usepackage{amsmath,amssymb}
\usepackage{amsthm}
\usepackage{makeidx}
\usepackage{txfonts}
\usepackage{mathrsfs} %花文字
\usepackage{mathtools} %参照式のみ式番号表示
\usepackage{latexsym} %qed
\usepackage{ascmac}
\usepackage{color}
\usepackage{comment}

\newtheoremstyle{mystyle}% % Name
	{10pt}%                      % Space above
	{10pt}%                      % Space below
	{\rm}%           % Body font
	{}%                      % Indent amount
	{\gt}%             % Theorem head font
	{.}%                      % Punctuation after theorem head
	{ }%                     % Space after theorem head, ' ', or \newline
	{}%                      % Theorem head spec (can be left empty, meaning `normal')
\theoremstyle{mystyle}

\allowdisplaybreaks[1]
\newcommand{\bhline}[1]{\noalign {\hrule height #1}} %表の罫線を太くする.
\newcommand{\bvline}[1]{\vrule width #1} %表の罫線を太くする.
\newtheorem{Prop}{$Proposition.$}
\newtheorem{Proof}{$Proof.$}
\newcommand{\QED}{% %証明終了
	\relax\ifmmode
		\eqno{%
		\setlength{\fboxsep}{2pt}\setlength{\fboxrule}{0.3pt}
		\fcolorbox{black}{black}{\rule[2pt]{0pt}{1ex}}}
	\else
		\begingroup
		\setlength{\fboxsep}{2pt}\setlength{\fboxrule}{0.3pt}
		\hfill\fcolorbox{black}{black}{\rule[2pt]{0pt}{1ex}}
		\endgroup
	\fi}
\newtheorem*{thm}{定理}
\newtheorem*{dfn}{定義}
\newtheorem*{prp}{命題}
\newtheorem*{lem}{補助定理}
\newtheorem*{prf}{証明}
\newtheorem*{rem}{注意}
\newcommand{\defunc}{\mbox{1}\hspace{-0.25em}\mbox{l}} %定義関数
\newcommand{\wlim}{\mbox{w-}\lim} %弱収束
\def\Box#1{$(\mbox{#1})$} %丸括弧つきコメント
\def\Hat#1{$\hat{\mathrm{#1}}$} %文中ハット
\def\Ddot#1{$\ddot{\mathrm{#1}}$} %文中ddot
\def\DEF{\overset{\mathrm{def}}{\Leftrightarrow}} %定義記号
\def\max#1#2{\operatorname*{max}_{#1} #2 } %最大
\def\min#1#2{\operatorname*{min}_{#1} #2 } %最小
\def\sin#1#2{\operatorname{sin}^{#2} #1} %sin
\def\cos#1#2{\operatorname{cos}^{#2} #1} %cos
\def\tan#1#2{\operatorname{tan}^{#2} #1} %tan
\def\inprod<#1>{\langle #1 \rangle} %内積
\def\sup#1#2{\operatorname*{sup}_{#1} #2 } %上限
\def\inf#1#2{\operatorname*{inf}_{#1} #2 } %下限
\def\Vector#1{\mbox{\boldmath $#1$}} %ベクトルを太字表示
\def\Norm#1#2{\left\|\, #1\, \right\|_{#2}} %ノルム
\def\Log#1{\operatorname{log} #1} %log
\def\Det#1{\operatorname{det} ( #1 )} %行列式
\def\Diag#1{\operatorname{diag} \left( #1 \right)} %行列の対角成分
\def\Tmat#1{#1^\mathrm{T}} %転置行列
\def\Exp#1{\operatorname{E} \left[ #1 \right]} %期待値
\def\Var#1{\operatorname{V} \left[ #1 \right]} %分散
\def\Cov#1#2{\operatorname{Cov} \left[ #1,\ #2 \right]} %共分散
\def\exp#1{e^{#1}} %指数関数
\def\N{\mathbb{N}} %自然数全体
\def\Q{\mathbb{Q}} %有理数全体
\def\R{\mathbb{R}} %実数全体
\def\C{\mathbb{C}} %複素数全体
\def\borel#1{\mathfrak{B}(#1)} %Borel集合族
\def\open#1{\mathfrak{O}(#1)} %位相空間 #1 の位相
\def\close#1{\mathfrak{A}(#1)} %%位相空間 #1 の閉集合系
\def\closure#1{\left[#1\right]^a}
\def\rapid#1{\mathfrak{S}(#1)} %急減少空間
\def\c#1{C(#1)} %有界実連続関数
\def\cbound#1{C_{b} (#1)} %有界実連続関数
\def\Lp#1#2{\operatorname{L}^{#1} \left(#2\right)} %L^p
\def\cinf#1{C^{\infty} (#1)} %無限回連続微分可能関数
\def\sgmalg#1{\sigma \left[#1\right]} %#1が生成するσ加法族
\def\ball#1#2{\operatorname{B} \left(#1\, ;\, #2 \right)} %開球
\def\prob#1{\operatorname{P} \left(#1\right)} %確率
\def\cprob#1#2{\operatorname{P} \left( #1 \ \middle|\ #2 \right)} %条件付確率
\def\cexp#1#2{\operatorname{E} \left[ #1 \ \middle|\ #2 \right]} %条件付期待値
%\renewcommand{\contentsname}{\bm Index}
%
\makeindex

\title{時系列解析\ 期末課題}
\author{基礎工学研究科システム創成専攻修士1年\\学籍番号29C17095\\百合川尚学}
\date{\today}

\begin{document}

\mathtoolsset{showonlyrefs = true}
\maketitle

\section{Markov連鎖}
	基礎となる確率空間$(\Omega, \mathcal{F}, \operatorname{P})$.
	\begin{itemize}
		\item $E$:\ 集合,
		\item $(E, \mathcal{E})$:\ 可測空間,
		\item $(X_n)_{n=1}^{+\infty}$:\ $E$-値確率過程.
	\end{itemize}
	\begin{rem}
		\ref{sec:first}章 $\sim$ \ref{sec:ergodic}章は$E$が高々可算集合であるとして考える.
	\end{rem}
\section{Markov連鎖}
	\label{sec:first}
	\begin{dfn}[Markov性]
		$\forall n \in \N,\ i_0, i_1, \cdots,i_n \in E,$
		\begin{align}
			 &\cprob{X_n = i_n}{X_0 = i_0,\ X_1 = i_1,\ \cdots, X_{n-1} = i_{n-1}} \\
			 &\qquad= \cprob{X_n = i_n}{X_{n-1} = i_{n-1}}.
		\end{align}
	\end{dfn}
	$(X_n)_{n=1}^{+\infty}$がMarkov性を持つ場合,これをMarkov連鎖という.
	以後$(X_n)_{n=1}^{+\infty}$はMarkov連鎖.

\section{Markov行列}
	\begin{dfn}[Markov 行列]
		$(i,j)$成分$(\forall i,j \in E)$を$\cprob{X_1 = j}{X_0 = i}$とする確率行列.
		行列を$P$,$(i,j)$成分を$[P]_{ij}$と表記.計算規則は以下.
		\begin{align}
			P^0 &= I, && (\mbox{$I$:恒等写像}),\\
			[P^n]_{ij} &= \sum_{k \in E} [P^{n-1}]_{ik} [P]_{kj}, && (\forall i,j \in E,\ n \in \N). 
		\end{align}
	\end{dfn}
	定義から次が成立
	\begin{align}
		[P^n]_{ij} = \cprob{X_n = j}{X_0 = i}, \ (\forall n \in \N,\ i,j \in E).
	\end{align}

\section{Chapman-Kolmogorov方程式}
	\begin{prp}[Chapman-Kolmogorov方程式]
		任意の$n,m = 0,1,2,\cdots$と$i,j \in E$に対し次が成立.
		\begin{align}
			[P^{n+m}]_{ij} = \sum_{k \in E}[P^n]_{ik}[P^m]_{kj}. 
		\end{align}
	\end{prp}

\section{既約性・再帰性}
	\begin{dfn}[既約性]
		$P$が既約である
		\begin{align}
			\DEF\ \forall i, j \in E,\ \exists n \in \N,\ \mathrm{s.t. \quad} [P^n]_{ij} > 0.
		\end{align}
	\end{dfn}
	\begin{dfn}[再帰性]
		$P$が再帰的である
		\begin{align}
			\DEF\ \cprob{\exists n \geq 1,\ X_n = i}{X_0 = i} = 1 \quad(\forall i \in E).
		\end{align}
		$P$が非再帰的である
		\begin{align}
			\DEF\ \cprob{\forall n \geq 1,\ X_n \neq i}{X_0 = i} > 0 \quad(\forall i \in E).
		\end{align}
	\end{dfn}
	
\section{離散空間上のMarkov連鎖}
	\begin{dfn}[到達時刻と到達回数]
		$\forall i \in E,\ \omega \in \Omega,$
		\begin{description}
			\item[到達時刻] $\tau_i(\omega) \coloneqq \inf{}{\left\{ n \geq 1\ \left|\ X_n(\omega) = i \right.\right\}},$
			\item[到達回数] $\eta_i(\omega) \coloneqq \sum_{n=1}^{+\infty} \defunc_{(X_n = i)}(\omega).$
		\end{description}
	\end{dfn}
	$p_{ij} \coloneqq \cprob{\tau_j < \infty}{X_0 = i}, \quad (\forall i,j \in E)$\\
	と表記すれば次が成立:
	\begin{align}
		p_{ii} &= \cprob{\exists n \geq 1,\ X_n = j}{X_0 = i}, \\
		p_{ii} &< 1 \Leftrightarrow \cexp{\eta_i}{X_0 = i} < +\infty, \quad (\forall i \in E).
	\end{align}

\section{正再帰性}
	\begin{dfn}[不変確率測度]
		$E$上の確率測度$\pi = ([\pi]_i)_{i \in E},\ (\sum_{i \in E} [\pi]_i = 1)$が$P$に対して不変確率測度である
		\begin{align}
			\DEF\ [\pi]_i = [\pi P]_i ( = \sum_{j \in E}[\pi]_j [P]_{ji} ), \quad (\forall i \in E).
		\end{align}
	\end{dfn}
	\begin{dfn}[正再帰性]
		$P$は正再帰的 \\
		$\quad\DEF\quad$ $P$が既約かつ不変確率測度が存在. 
	\end{dfn}
\section{再帰性の諸命題}
	\begin{prp}
		$P$が既約の下,(i) $\sim$ (iv)が順に示される:
		\begin{description}
			\item[\rm{(i)}] $P$が再帰的 $\Leftrightarrow \cexp{\eta_i}{X_0 = i} = +\infty, \ (\forall i \in E),$
			\item[\rm{(ii)}] $P$は再帰的であるか非再帰的のどちらか.特に$E$が有限集合なら$P$は再帰的.
			\item[\rm{(iii)}] $P$が正再帰的 $\Rightarrow$ $P$は再帰的.
			\item[\rm{(iv)}] $E$が有限集合なら$P$は正再帰的.
		\end{description}
	\end{prp}

\section{周期}
	\begin{dfn}[$i \in E$の周期]
		$\mathcal{N}_i \coloneqq \left\{n \geq 1\ \left|\ [p^n]_{ii} > 0 \right.\right\}$の最大公約数を
		$i \in E$の周期といい$d_i$と表す.
	\end{dfn}
	\begin{prp}[既約なら周期はunique]
		$P$が既約ならば$d_i = d_j\ (\forall i,j \in E)$.
		この場合$d_i$を$P$の周期という.
	\end{prp}
	\begin{dfn}[非周期性]
		$P$が既約の下,
		\begin{align}
			\mbox{$P$は非周期的 $\quad\DEF\quad$ $P$の周期が1}.
		\end{align}
	\end{dfn}

\section{Ergodicity}
	\label{sec:ergodic}
	\begin{prp}[周期に関する一命題]
		$P$:既約,非周期的,
		\begin{align}
			\forall i,j \in E,\ \exists n_{ij} \in \N,\ \mathrm{s.t.}\ [P^n]_{ij} > 0 \ (\forall n \geq n_{ij}).
		\end{align}
	\end{prp}
	\begin{thm}[Ergodicity]
		$P$が既約で非周期的かつ正再帰的であるとする.$P$の不変確率測度を$\pi$で表すとき次が成立.
		\begin{align}
			\lim_{n \to +\infty}[P^n]_{ij} = [\pi]_j, \quad (\forall i,j \in E).
		\end{align}
	\end{thm}
	
	
	
	
	
\newpage
\printindex
%
%
\end{document}
\documentclass[twocolumn,10pt]{jsarticle}
\setlength{\columnsep}{3zw}
\addtolength{\textheight}{1cm}

\usepackage{amsmath,amssymb}
\usepackage{amsthm}
\usepackage{makeidx}
\makeindex
\usepackage{newpxmath,newpxtext}
\usepackage{mathrsfs} %花文字
\usepackage{mathtools} %参照式のみ式番号表示
\usepackage{latexsym} %qed
\usepackage{ascmac}
\usepackage{here} %表を記述位置に強制出力する
\usepackage{tabularx}
\usepackage{bussproofs} %証明図
\usepackage{centernot} %\centernot\arrow
\usepackage[dvipdfmx]{graphicx}
\usepackage{tikz} %描画
\usepackage{color}
\usepackage{relsize}
\usepackage{comment}
\usepackage{url}
\usepackage[normalem]{ulem} %訂正線
\usepackage[dvipdfm,colorlinks=true,linkcolor=blue,filecolor=blue,urlcolor=blue]{hyperref} %文書内リンク
\usepackage{pxjahyper} %%hyperref読み込みの直後に
\setcounter{tocdepth}{3} %table of contents subsection表示
\newtheoremstyle{mystyle}% % Name
	{20pt}%                      % Space above
	{20pt}%                      % Space below
	{\rm}%           % Body font
	{}%                      % Indent amount
	{\gt}%             % Theorem head font
	{.}%                      % Punctuation after theorem head
	{10pt}%                     % Space after theorem head, ' ', or \newline
	{}%                      % Theorem head spec (can be left empty, meaning `normal')
\theoremstyle{mystyle}

\allowdisplaybreaks[1]
\newcommand{\bhline}[1]{\noalign {\hrule height #1}} %表の罫線を太くする.
\newcommand{\bvline}[1]{\vrule width #1} %表の罫線を太くする.
\newcommand{\QED}{% %証明終了
	\relax\ifmmode
		\eqno{%
		\setlength{\fboxsep}{2pt}\setlength{\fboxrule}{0.3pt}
		\fcolorbox{black}{black}{\rule[2pt]{0pt}{1ex}}}
	\else
		\begingroup
		\setlength{\fboxsep}{2pt}\setlength{\fboxrule}{0.3pt}
		\hfill\fcolorbox{black}{black}{\rule[2pt]{0pt}{1ex}}
		\endgroup
	\fi}

\definecolor{DarkMidnightBlue}{rgb}{0.0, 0.2, 0.4}
\definecolor{PakistanGreen}{rgb}{0.0, 0.4, 0.0}
\definecolor{Mahogany}{rgb}{0.65,0.10,0.10}
\definecolor{darkgray}{rgb}{0.21, 0.21, 0.21}
\definecolor{CarolinaBlue}{rgb}{0.6, 0.73, 0.89}

\newtheorem{thm}{\color{DarkMidnightBlue}{定理}}[section]
\newtheorem{dfn}[thm]{\color{PakistanGreen}{定義}}
\newtheorem{axm}[thm]{\color{Mahogany}{公理}}
\newtheorem{schema}[thm]{{公理図式}}
\newtheorem{logicalrule}[thm]{\color{Mahogany}{推論規則}}
\newtheorem{logicalaxm}[thm]{\color{Mahogany}{論理的公理}}
\newtheorem{logicalthm}[thm]{\color{DarkMidnightBlue}{論理的定理}}
\newtheorem{metadfn}[thm]{\color{PakistanGreen}{メタ定義}}
\newtheorem{metaaxm}[thm]{\color{Mahogany}{メタ公理}}
\newtheorem{metathm}[thm]{\color{DarkMidnightBlue}{メタ定理}}
\newtheorem{prp}[thm]{命題}
\newtheorem{cor}[thm]{系}
\newtheorem{lem}[thm]{補題}
\newtheorem*{prf}{証明}
\newtheorem*{metaprf}{メタ証明}
\newtheorem*{sketch}{略証}
\newtheorem{rem}[thm]{注意}
\newtheorem{e.g.}[thm]{例}
\newcommand{\defunc}{\mbox{1}\hspace{-0.25em}\mbox{l}} %定義関数
\newcommand*{\sgn}[1]{\operatorname{sgn}\left( #1 \right)} %signal関数
\newcommand{\monologue}[1]{
	{\color{CarolinaBlue}\hspace{-10.5pt}\mask{\hspace{21pt}\vbox{
		\hsize 445pt
		\normalcolor{\vskip 7pt \noindent #1 \vskip 7pt}
	}\hspace{21pt}}{E}}
}

\def\Ddot#1{$\ddot{\mathrm{#1}}$} %文中ddot

%論理
\newcommand{\lang}[1]{\mathcal{L}_{\scalebox{1.2}{$#1$}}} %言語
\newcommand{\Set}[2]{\{\, #1 \mid #2\, \}} %論理式の対象化
\newcommand{\defeq}{\overset{\mathrm{def}}{=\joinrel=}} %\scalebox{3}[1]{=}}} %定義記号=(=\joinrel=も使える)
\newcommand{\defarrow}{\ \overset{\mathrm{def}}{\longleftrightarrow}\ } %定義記号↔
\newcommand{\provable}[1]{\vdash_{{\scriptsize #1}}} %証明可能
\newcommand{\negation}{\rightharpoondown\hspace{-0.25em}} %否定
\newcommand{\rarrow}{\hspace{0.25em}\rightarrow\hspace{0.25em}} %右矢印
\newcommand{\lrarrow}{\hspace{0.25em}\leftrightarrow\hspace{0.25em}} %左右矢印

%集合
\newcommand{\EXTAX}{\mbox{{\bf EXT}}} %外延性公理
\newcommand{\EQAX}{\mbox{{\bf EQ}}} %相等性公理
%\newcommand{\EQAXEP}{\mbox{{\bf EQ}}_{\scalebox{1.2}{$\varepsilon$}}} %ε項の相等性公理
\newcommand{\COMAX}{\mbox{\bf COM}} %内包性公理
\newcommand{\ELEAX}{\mbox{{\bf ELE}}} %要素の公理
\newcommand{\REPAX}{\mbox{{\bf REP}}} %置換公理
\newcommand{\PAIAX}{\mbox{{\bf PAI}}} %対集合公理
\newcommand{\UNIAX}{\mbox{{\bf UNI}}} %合併の公理
\newcommand{\POWAX}{\mbox{{\bf POW}}} %冪集合公理
\newcommand{\INFAX}{\mbox{{\bf INF}}} %無限公理
\newcommand{\REGAX}{\mbox{{\bf REG}}} %正則性公理
\newcommand{\AC}{\mbox{{\bf CHOICE}}} %選択公理

\newcommand{\Univ}{\mathbf{V}} %宇宙
\newcommand{\set}[1]{\operatorname*{set}\hspace{0.15em}(#1)} %集合であることの論理式
\newcommand{\power}[1]{\operatorname*{P}\hspace{0.15em}(#1)} %冪集合
\newcommand{\rel}[1]{\operatorname*{rel}\hspace{0.15em}(#1)} %関係
\newcommand{\dom}[1]{\operatorname*{dom}\hspace{0.15em}(#1)} %類の定義域
\newcommand{\ran}[1]{\operatorname*{ran}\hspace{0.15em}(#1)} %類の値域
\newcommand{\sing}[1]{\operatorname*{sing}\hspace{0.15em}(#1)} %single-valuedの定義式
\newcommand{\fnc}[1]{\operatorname*{fnc}\hspace{0.15em}(#1)} %写像の定義式
\newcommand{\fon}{\operatorname*{:on}} %〇上の写像
\newcommand{\inj}{\overset{\mathrm{1:1}}{\longrightarrow}} %単射
\newcommand{\srj}{\overset{\mathrm{onto}}{\longrightarrow}} %全射
\newcommand{\bij}{\underset{\mathrm{onto}}{\overset{\mathrm{1:1}}{\longrightarrow}}} %全単射
\newcommand{\inv}[1]{{#1}^{-1}} %^{\operatorname{inv}}} %集合の反転
\newcommand{\rest}[2]{#1\hspace{-0.25em}\upharpoonright\hspace{-0.25em}{#2}} %制限写像
\newcommand{\tran}[1]{\operatorname*{tran}\hspace{0.15em}(#1)} %推移的類の定義式
\newcommand{\ord}[1]{\operatorname*{ord}\hspace{0.15em}(#1)} %順序数の定義式
\newcommand{\ON}{\mathrm{ON}} %順序数全体
\newcommand{\limo}[1]{\mathrm{lim.o}\hspace{0.15em}(#1)} %極限数の式
%\newcommand{\Natural}{{\boldsymbol \omega}} %自然数全体
\newcommand{\Natural}{\mathbf{N}} %自然数全体

%基数
\newcommand{\eqp}{\approx} %集合の対等
\newcommand{\card}[1]{\# #1} %濃度
%\newcommand{\card}[1]{\operatorname{card} #1} %濃度
%\newcommand{\card}[1]{\operatorname*{card} \left(#1\right)} %濃度
\newcommand{\CN}{\mathrm{CN}} %基数全体
\newcommand{\InfCN}{\mathrm{ICN}} %無限基数全体
\newcommand{\Fin}[1]{\operatorname*{Fin}\hspace{0.15em}(#1)} %有限集合の定義式
\newcommand{\Inf}[1]{\operatorname*{Inf}\hspace{0.15em}(#1)} %無限集合の定義式
\newcommand{\cof}[2]{\operatorname*{cof}\hspace{0.15em}(#1,#2)} %共終写像が存在する
\newcommand{\cf}[1]{cf(#1)} %共終数

\title{\vspace{-3cm}$\varepsilon$計算とクラスの導入による具体的で直観的な集合論の構築}
\author{関根深澤研修士二年 百合川尚学 \\ 学籍番号:29C17095}
\date{\today}

\begin{document}
\mathtoolsset{showonlyrefs = true}
\maketitle
	本論文では類(class)を扱うための{\bf ZF}集合論の一つの拡張を提示したが,
	そこでの主要な定理は,{\bf ZF}集合論のどの命題に対しても「{\bf ZF}集合論で証明可能」ならば
	「本論文の集合論で証明可能」であり,逆に「本論文の集合論で証明可能」ならば
	「{\bf ZF}集合論で証明可能」であるということである.これを精密に言い直せば,
	{\bf ZF}集合論の任意の命題$\psi$に対して
	「$\Gamma$から$\psi$への{\bf HK}の証明で$\lang{\in}$の式の列であるものが取れる」
	ことと「$\Sigma$から$\psi$への{\bf HE}の証明で$\mathcal{L}$の文の列であるものが取れる」
	ことが同値であるということになる.以下で記号を解説する.
	
	$\lang{\in}$とは{\bf ZF}集合論の言語のことである.
	本論文ではもう二つの言語$\lang{\varepsilon}$と$\mathcal{L}$があり,
	項および式の形成規則はそれぞれの言語の中で指定される.
	$\Gamma$とは$\lang{\in}$の文で書かれた{\bf ZF}集合論の公理系
	(外延性・相等性・置換・対・合併・冪・正則性・無限)であり,
	ここで文とは変項の自由な出現が無い式を指す.
	{\bf HK}とは古典論理のHilbert流証明体系のことであり,
	「$\Gamma$からの{\bf HK}の証明で$\lang{\in}$の式の列であるもの」とは,
	$\lang{\in}$の式の列$\varphi_{1},\cdots,\varphi_{n}$で,
	各$\varphi_{i}$について以下のいずれかが満たされるものである:
	(1) {\bf HK}の公理である.(2) $\Gamma$の公理である.(3) 列の前の式から三段論法で得られる.
	(4) 列の前の式から汎化で得られる.
	%つまりその場合は$\varphi_{j}$は
	%$\xi(x/a)$なる式で$\varphi_{i}$は$\forall x \xi$なる式である.ここで
	%$\xi(x/a)$とは$\xi$に自由に現れる$x$に変項$a$を代入した式であり,
	%この$a$はこの汎化の固有変項(eigenvariable)と呼ばれる.
	
	一方で$\mathcal{L}$も$\Sigma$も{\bf HE}も本論文特有のものである.
	$\mathcal{L}$とは$\lang{\in}$の語彙を拡張した言語であり,拡張の中間に
	$\lang{\varepsilon}$がある.$\lang{\varepsilon}$は
	$\lang{\in}$に$\varepsilon$を追加した言語であるが,
	この$\varepsilon$とは数論の無矛盾性の考察過程でHilbert\cite{Hilbert}が発案したものである.
	$\mathcal{L}$には$\Set{x}{\varphi}$の形の項(内包項)を追加し,正式に類が扱えるようになる.
	$\Sigma$とは本論文における集合論の公理系であり,
	$\Gamma$の「外延性」と「相等性」が類に対する言明に変更され,
	また「内包性」と「要素」の公理が新たに追加される.内包性公理は
	\begin{align}
		\forall u\, (\, u \in \Set{x}{\varphi(x)} \lrarrow \varphi(u)\, )
	\end{align}
	なる式を指し,$\Set{x}{\varphi(x)}$に対して「$\varphi$である$x$の全体」の意味を与える.
	要素の公理は
	\begin{align}
		a \in b \rarrow \exists x\, (\, a = x\, )
	\end{align}
	なる式を指し,これによって要素となりうるものは集合に限られる.
	右辺の$\exists x\, (\, a = x\, )$は「$a$は集合である」の意味の式であり,
	竹内\cite{TakeuchiSet}の集合の定義を引用したものである.
	{\bf HE}とは{\bf HK}の量化の公理を改造した証明体系である.
	%両者に共通するのは$\varphi(x/\tau) \rarrow \exists x \varphi$
	%と$\forall x \varphi \rarrow \varphi(x/\tau)$であるが,
	{\bf HE}では{\bf HK}の公理である$\forall y\, (\, \psi \rarrow \varphi(x/y)\, )
	\rarrow (\, \psi \rarrow \forall x \varphi\, )$と
	$\forall y\, (\, \varphi(x/y) \rarrow \psi\, )
	\rarrow (\, \exists x \varphi \rarrow \psi\, )$が削除され,代わりに
	\begin{align}
		\begin{gathered}
			\negation \forall x \varphi \rarrow \exists x \negation \varphi, \\
			\exists x \varphi \rarrow \varphi(x/\varepsilon x \varphi)
		\end{gathered}
	\end{align}
	が公理となる.{\bf HE}の証明は全て文で行う.
	「$\Sigma$からの{\bf HE}の証明で$\mathcal{L}$の文の列であるもの」とは,
	$\mathcal{L}$の文の列$\varphi_{1},\cdots,\varphi_{n}$で,
	各$\varphi_{i}$について以下のいずれかが満たされるものである:
	(1) {\bf HE}の公理である.(2) $\Sigma$の公理である.
	(3) 列の前の式から三段論法で得られる.
	$\varepsilon$項の作用によって{\bf HE}では汎化は不要になる.
	
	なお,$\varepsilon x \varphi$なる形の項を$\varepsilon$項と呼ぶが,
	特に$\varphi$に$x$のみ自由に現れている場合は主要$\varepsilon$項と呼ぶ.
	また$\Set{x}{\varphi}$なる内包項に関しては$\varphi$に$x$が自由に現れている場合に
	正則内包項と呼ぶ.そして扱う$\mathcal{L}$の式には次の制限を付ける:
	(1) 式に現れる$\varepsilon$項は主要$\varepsilon$項である.
	(2) 式に現れる内包項は正則内包項である.
	{\bf HE}の量化公理と外延性公理および集合の定義式によって,
	主要$\varepsilon$項は全て集合となり,また集合である類は
	いずれかの主要$\varepsilon$項と等しいものに限られる.
	
	\begin{thebibliography}{数字}
		\bibitem{Hilbert} D. ヒルベルト and P. ベルナイス, 数学の基礎(吉田夏彦, 渕野昌訳), 丸善出版株式会社, 2012, pp. 23-63, ISBN 978-4-621-06405-4.
	
		\bibitem{TakeuchiSet} 竹内外史, 現代集合論入門, 増強版, 日本評論社, 2016, pp. 138-183, ISBN 978-4-535-60116-1.
	\end{thebibliography}
\end{document}
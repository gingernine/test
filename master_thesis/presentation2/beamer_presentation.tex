\documentclass[dvipdfmx,11pt,notheorems]{beamer}
%%%% 和文用 %%%%%
\usepackage{bxdpx-beamer}
\usepackage{pxjahyper}
\usepackage{minijs}%和文用
\renewcommand{\kanjifamilydefault}{\gtdefault}%和文用

%%%% スライドの見た目 %%%%%
\usetheme{Madrid}
\usefonttheme{professionalfonts}
\setbeamertemplate{frametitle}[default][center]
\setbeamertemplate{navigation symbols}{}
\setbeamercovered{transparent}%好みに応じてどうぞ)
\setbeamertemplate{footline}[page number]
\setbeamerfont{footline}{size=\normalsize,series=\bfseries}
\setbeamercolor{footline}{fg=black,bg=black}
%%%%
\usepackage{mathtools} %参照式のみ式番号表示
%%%% 定義環境 %%%%%
\usepackage{amsmath,amssymb}
\usepackage{amsthm}
\theoremstyle{definition}
\newtheorem{theorem}{定理}
\newtheorem{definition}{定義}
\newtheorem{proposition}{命題}
\newtheorem{lemma}{補題}
\newtheorem{corollary}{系}
\newtheorem{conjecture}{予想}
\newtheorem*{remark}{Remark}
\renewcommand{\proofname}{}
%%%%%%%%%

%%%%% フォント基本設定 %%%%%
\usepackage[T1]{fontenc}%8bit フォント
\usepackage{textcomp}%欧文フォントの追加
\usepackage[utf8]{inputenc}%文字コードをUTF-8
\usepackage{otf}%otfパッケージ
\usepackage{pxfonts}%数式・英文ローマン体を Lxfont にする
\usepackage{bm}%数式太字
%%%%%%%%%%

%論理
\newcommand{\lang}[1]{\mathcal{L}_{\scalebox{1.2}{$#1$}}} %言語
\newcommand{\Set}[2]{\left\{\, #1 \mid #2\, \right\}} %論理式の対象化
\newcommand{\defeq}{\overset{\mathrm{def}}{=\joinrel=}} %\scalebox{3}[1]{=}}} %定義記号=(=\joinrel=も使える)
\newcommand{\defarrow}{\overset{\mathrm{def}}{\longleftrightarrow}} %定義記号↔
\newcommand{\provable}[1]{\vdash_{{\scriptsize #1}}} %証明可能
\newcommand{\negation}{\rightharpoondown\hspace{-0.25em}} %否定
\newcommand{\rarrow}{\hspace{0.25em}\rightarrow\hspace{0.25em}} %右矢印
\newcommand{\lrarrow}{\hspace{0.25em}\leftrightarrow\hspace{0.25em}} %左右矢印

%集合
\newcommand{\EXTAX}{\mbox{{\bf EXT}}} %外延性公理
\newcommand{\EQAX}{\mbox{{\bf EQ}}} %相等性公理
\newcommand{\EQAXEP}{\mbox{{\bf EQ}}_{\scalebox{1.2}{$\varepsilon$}}} %ε項の相等性公理
\newcommand{\COMAX}{\mbox{\bf COM}} %内包性公理
\newcommand{\ELEAX}{\mbox{{\bf ELE}}} %要素の公理
\newcommand{\REPAX}{\mbox{{\bf REP}}} %置換公理
\newcommand{\POWAX}{\mbox{{\bf POW}}} %冪集合公理
\newcommand{\PAIAX}{\mbox{{\bf PAI}}} %対集合公理
\newcommand{\INFAX}{\mbox{{\bf INF}}} %無限公理
\newcommand{\REGAX}{\mbox{{\bf REG}}} %正則性公理
\newcommand{\AC}{\mbox{{\bf CHOICE}}} %選択公理

\newcommand{\Univ}{\mathbf{V}} %宇宙
\newcommand{\set}[1]{\operatorname*{set} (#1)} %集合であることの論理式
\newcommand{\power}[1]{\operatorname*{P} (#1)} %冪集合
\newcommand{\rel}[1]{\operatorname*{rel} (#1)} %関係
\newcommand{\dom}[1]{\operatorname*{dom} (#1)} %類の定義域
\newcommand{\ran}[1]{\operatorname*{ran} (#1)} %類の値域
\newcommand{\sing}[1]{\operatorname*{sing} (#1)} %single-valuedの定義式
\newcommand{\fnc}[1]{\operatorname*{fnc} (#1)} %写像の定義式
\newcommand{\fon}{\operatorname*{:on}} %〇上の写像
\newcommand{\inj}{\overset{\mathrm{1:1}}{\longrightarrow}} %単射
\newcommand{\srj}{\overset{\mathrm{onto}}{\longrightarrow}} %全射
\newcommand{\bij}{\underset{\mathrm{onto}}{\overset{\mathrm{1:1}}{\longrightarrow}}} %全単射
\newcommand{\inv}[1]{{#1}^{-1}} %^{\operatorname{inv}}} %集合の反転
\newcommand{\rest}[2]{#1\hspace{-0.25em}\upharpoonright\hspace{-0.25em}{#2}} %制限写像
\newcommand{\tran}[1]{\operatorname*{tran} \left(#1\right)} %推移的類の定義式
\newcommand{\ord}[1]{\operatorname*{ord} \left(#1\right)} %順序数の定義式
\newcommand{\ON}{\mathrm{ON}} %順序数全体
\newcommand{\limo}[1]{\mathrm{lim.o}\left(#1\right)} %極限数の式
%\newcommand{\Natural}{{\boldsymbol \omega}} %自然数全体
\newcommand{\Natural}{\mathbf{N}} %自然数全体
 
\title{$\varepsilon$計算とクラスの導入による具体的で直観的な集合論の構築}%[略タイトル]{タイトル}
\author{関根深澤研 百合川尚学 \\ 学籍番号:29C17095}%[略名前]{名前}
\institute{}%[略所属]{所属}
\date{\today}%日付

\begin{document}
\mathtoolsset{showonlyrefs = true}

\begin{frame}[plain]\frametitle{}
\titlepage %表紙
\end{frame}

\begin{frame}\frametitle{Contents}
\tableofcontents %目次
\end{frame}

\section{導入}

\begin{frame}\frametitle{導入 $\varepsilon$について}
	\begin{itemize}
	%\setlength{\itemsep}{10pt}
		\item 量化$\exists,\forall$を使う証明を命題論理の証明に埋め込むためにHilbertが開始.
		
		\item 式$\varphi(x)$に対して
			\begin{align}
				\varepsilon x \varphi(x)
			\end{align}
			という形のオブジェクトを作り,$\varepsilon$項と呼ぶ.また
			命題論理の証明に埋め込む際には,$\exists$や$\forall$の付いた式を
			\begin{align}
				\varphi(x/\varepsilon x \varphi(x)) &\defarrow \exists x \varphi(x), \\
				\varphi(x/\varepsilon x \negation x \varphi(x)) &\defarrow \forall x \varphi(x)
			\end{align}
			によって変換すればよい.
			
		\item 今回$\varepsilon$項を導入したのは
			\textcolor{red}{「存在」と「実在」を同義}とするため.
			
		\item Hilbertの$\varepsilon$計算ではなく,$\varepsilon$項を用いて
			一種のHenkin拡大を行う.
	\end{itemize}
\end{frame}

\begin{frame}\frametitle{導入 $\varepsilon$について}
	\begin{itemize}
	\setlength{\itemsep}{10pt}
		\item {\bf ZF}集合論では\textcolor{red}{集合というオブジェクトが用意されていない}ため,
			「存在」は「実在」ではない.たとえば
			\begin{align}
				\exists x\, \forall y\, (\, y \notin x\, )
			\end{align}
			は定理であり「空集合は存在する」と読むが,空集合を
			\underline{``実際に取ってくる''}ことは不可能.
			
		\item $\varepsilon$項を使えば,$\exists$の公理と空集合の存在定理によって
			\begin{align}
				\forall y\, (\, y \notin \color{red}{\varepsilon x\, \forall y\, (\, y \notin x\, )}\, \color{black}{)}
			\end{align}
			が成り立つ
	\end{itemize}
	
	\begin{block}{$\varepsilon$項を使うメリット}
		\begin{itemize}
			\item 証明で用いる推論規則は三段論法のみで済む.
			\item 証明が容易になる場合がある.
		\end{itemize}
	\end{block}
\end{frame}

\begin{frame}\frametitle{導入 クラスについて}
	\begin{itemize}
	%\setlength{\itemsep}{10pt}
		\item ブルバキ\cite{}や島内\cite{}でも$\varepsilon$項を使った集合論を展開.
		
		\item ところで,「$\varphi$である集合の全体」の意味の
			\begin{align}
				\Set{x}{\varphi(x)}
			\end{align}
			というオブジェクトも取り入れたい.
		
		\item {\bf ZF}集合論では定義による拡大 or インフォーマルな導入.
		
		\item ブルバキ\cite{}や島内\cite{}では
			\begin{align}
				\Set{x}{\varphi(x)} \defeq \varepsilon y\, \forall x\, 
				(\, \varphi(x) \lrarrow x \in y\, )
			\end{align}
			と定めるが,
			\begin{align}
				\exists y\, \forall x\, (\, \varphi(x) \lrarrow x \in y\, )
			\end{align}
			が成立しない場合は「$\varphi$である集合の全体」という意味を持たない.
			
		\item \textcolor{red}{式$\varphi$から直接$\Set{x}{\varphi(x)}$の形のオブジェクトを作ればよい.}
	\end{itemize}
\end{frame}

\begin{frame}\frametitle{導入 クラスについて}
	\begin{exampleblock}{クラス}
		式$\varphi$に$x$のみが自由に現れているとき,$\varepsilon x \varphi(x),
		\quad \Set{x}{\varphi(x)}$の形のオブジェクトを{\bf クラス(class)}と呼ぶ.
	\end{exampleblock}
	
	\begin{itemize}
		\item クラスである$\varepsilon$項は集合である.
		\item 集合でないクラスもある.たとえば$\Set{x}{x = x}$や$\Set{x}{x \notin x}$
			は集合ではない.
	\end{itemize}
	
	集合の定義は竹内\cite{}に倣う.定義により\textcolor{red}{集合はクラスである}.
	\begin{exampleblock}{集合}
		クラス$c$が
		\begin{align}
			\exists x\, (\, c = x\, )
		\end{align}
		を満たすとき$c$を{\bf 集合(set)}と呼び,そうでない場合は
		{\bf 真クラス(proper class)}と呼ぶ.
	\end{exampleblock}
\end{frame}

\begin{frame}\frametitle{言語}
	\begin{itemize}
	%\setlength{\itemsep}{10pt}
		\item クラスという新しいオブジェクトを導入したら,
			この導入操作が``妥当''であるかどうかが問題になる.
		
		\item 妥当性は,{\bf ZF}集合論の命題$\varphi$に対して
			\begin{align}
				\mbox{{\bf ZF}集合論で$\varphi$が証明可能}
				\Longleftrightarrow
				\mbox{新しい集合論で$\varphi$が証明可能}
			\end{align}
			が成り立つかどうかで検証する.
		
		\item 集合論の\textcolor{red}{言語}と
			\textcolor{red}{証明のルール}を明らかにしなくてはならない.
		
		\item 言語とは「\textcolor{red}{変項}」,「\textcolor{red}{述語記号}」,
			「\textcolor{red}{論理記号}」とその他もろもろの記号からなる.
			そして「\textcolor{red}{式(formula)}」は言語の記号を用いて作られる.
			式を作るためには「\textcolor{red}{項(term)}」が必要であり,文字は最もよく使われる項である.
			たとえば
			\begin{align}
				s \in t
			\end{align}
			と書けば一つの式が出来上がる.
		
		\item まず{\bf ZF}集合論の言語$\lang{\in}$を明示する.
	\end{itemize}
\end{frame}

\begin{frame}\frametitle{言語$\lang{\in}$}
	\begin{exampleblock}{言語$\lang{\in}$}
		\begin{description}
			\item[矛盾記号] $\bot$
			\item[論理記号] $\negation,\ \vee,\ \wedge,\ \rarrow$
			\item[量化子] $\forall,\ \exists$
			\item[述語記号] $=,\ \in$
			\item[変項] $x,y,z,\cdots$.
		\end{description}
	\end{exampleblock}
\end{frame}

\begin{frame}\frametitle{言語$\lang{\in}$の項と式}
	$\lang{\in}$の項と式は次の規則で生成する.
	
	\begin{exampleblock}{$\lang{\in}$の項と式}
		\begin{description}
			\item[項] 変項は項であり,またこれらのみが項である.
				
			\item[式] 
				\begin{itemize}
					\item $\bot$は式である.
					\item 項$\tau$と項$\sigma$に対して
						$\tau \in \sigma$と$\tau = \sigma$は式である.
					\item 式$\varphi$に対して$\negation \varphi$は式である.
					\item 式$\varphi$と式$\psi$に対して$\varphi \vee \psi$と
						$\varphi \wedge \psi$と$\varphi \rarrow \psi$
						はいずれも式である.
					\item 式$\varphi$と項$x$に対して$\exists x \varphi$と
						$\forall x \varphi$は式である.
					\item これらのみが式である.
				\end{itemize}
		\end{description}
	\end{exampleblock}
\end{frame}

\begin{frame}\frametitle{言語の拡張}
	\begin{itemize}
	%\setlength{\itemsep}{10pt}
		\item クラスを正式に導入するには言語を拡張しなくてはならない.
		\item 拡張は二段階に分けて行う.
			始めに$\varepsilon$項のために拡張し,
			次に$\Set{x}{\varphi(x)}$の形の項のために拡張する.
			
		\item 始めの拡張により得る言語を$\lang{\varepsilon}$と名付ける.
	\end{itemize}

	\begin{exampleblock}{言語$\lang{\varepsilon}$}
		\begin{description}
			\item[矛盾記号] $\bot$
			\item[論理記号] $\negation,\ \vee,\ \wedge,\ \rarrow$
			\item[量化子] $\forall,\ \exists,\ \varepsilon$
			\item[述語記号] $=,\ \in$
			\item[変項] $x,y,z,\cdots$.
		\end{description}
	\end{exampleblock}
\end{frame}

\begin{frame}{$\lang{\varepsilon}$の項と式}
	\begin{exampleblock}{$\lang{\varepsilon}$の項と式の定義}
		\begin{itemize}
			\item 変項は項である.
			\item $\bot$は式である.
			\item 項$\tau$と項$\sigma$に対して
				$\tau \in \sigma$と$\tau = \sigma$は式である.
			\item 式$\varphi$に対して$\negation \varphi$は式である.
			\item 式$\varphi$と式$\psi$に対して$\varphi \vee \psi$と
				$\varphi \wedge \psi$と$\varphi \rarrow \psi$
				はいずれも式である.
			\item 式$\varphi$と変項$x$に対して$\exists x \varphi$と
				$\forall x \varphi$は式である.
			\item \textcolor{red}{式$\varphi$と変項$x$に対して$\varepsilon x \varphi$は項である.}
			\item これらのみが項と式である.
		\end{itemize}
	\end{exampleblock}
	
	\begin{itemize}
	%\setlength{\itemsep}{10pt}
		\item $\lang{\in}$との大きな違いは
		 \textcolor{red}{項と式の定義が循環している}点.
		
		\item$\lang{\varepsilon}$の式が$\lang{\varepsilon}$の項を用いて
			作られるのは当然ながら,その逆に$\lang{\varepsilon}$の項もまた
			$\lang{\varepsilon}$の式から作られる.
			
		\item $\lang{\in}$の式は$\lang{\varepsilon}$の式でもある.
	\end{itemize}
\end{frame}

\begin{frame}\frametitle{言語$\mathcal{L}$}
	\begin{exampleblock}{言語$\mathcal{L}$}
		\begin{description}
			\item[矛盾記号] $\bot$
			\item[論理記号] $\negation,\ \vee,\ \wedge,\ \rarrow$
			\item[量化子] $\forall,\ \exists$
			\item[述語記号] $=,\ \in$
			\item[変項] $x,y,z,\cdots$.
			\item[補助記号] $\{,\ |,\ \}$
		\end{description}
	\end{exampleblock}
\end{frame}

\begin{frame}\frametitle{$\mathcal{L}$の項と式}
	\begin{exampleblock}{$\mathcal{L}$の項と式の定義}
		\begin{description}
			\item[項] 
				\begin{itemize}
					\item 変項は項である.
					\item $\lang{\varepsilon}$の項は項である.
					\item $x$を変項とし,$\varphi$を
						$\lang{\varepsilon}$の式とするとき,
						$\Set{x}{\varphi}$なる記号列は項である.
					\item これらのみが項である.
				\end{itemize}
			
			\item[式] 
				\begin{itemize}
					\item $\bot$は式である.
					\item 項$\tau$と項$\sigma$に対して
						$\tau \in \sigma$と$\tau = \sigma$は式である.
					\item 式$\varphi$に対して$\negation \varphi$は式である.
					\item 式$\varphi$と式$\psi$に対して$\varphi \vee \psi$と
						$\varphi \wedge \psi$と$\varphi \rarrow \psi$
						はいずれも式である.
					\item 式$\varphi$と変項$x$に対して$\exists x \varphi$と
						$\forall x \varphi$は式である.
					\item これらのみが式である.
				\end{itemize}
		\end{description}
	\end{exampleblock}
\end{frame}

\begin{frame}\frametitle{いろんなブロック}
\begin{block}{ブロック}
これは普通のブロックです
\end{block}

\begin{alertblock}{警告ブロック}
警告!これは警告ブロックだ!
\end{alertblock}

\begin{exampleblock}{例ブロック}
例えば、こんなブロックです。
\end{exampleblock}
\end{frame}



%\setcounter{framenumber}{\value{finalframe}}
\end{document}
\documentclass[dvipdfmx,10pt,notheorems]{beamer}
%%%% 和文用 %%%%%
\usepackage{bxdpx-beamer}
\usepackage{pxjahyper}
\usepackage{minijs}%和文用
\renewcommand{\kanjifamilydefault}{\gtdefault}%和文用

%%%% スライドの見た目 %%%%%
\usetheme{Darmstadt} %Frankfurt, AnnArbor, Antibes, Berlin, Berkeley, Bergen, Boadilla, boxes, CambridgeUS, Copenhagen
\usefonttheme{professionalfonts}
\setbeamertemplate{frametitle}[default][center]
\setbeamertemplate{navigation symbols}{}
\setbeamercovered{transparent}%好みに応じてどうぞ)
\setbeamertemplate{footline}[page number]
\setbeamerfont{footline}{size=\normalsize,series=\bfseries}
\setbeamercolor{footline}{fg=black,bg=black}
%%%%
\usepackage{mathtools} %参照式のみ式番号表示
%%%% 定義環境 %%%%%
\usepackage{amsmath,amssymb}
\usepackage{amsthm}
\theoremstyle{definition}
\newtheorem{theorem}{定理}
\newtheorem{definition}{定義}
\newtheorem{proposition}{命題}
\newtheorem{lemma}{補題}
\newtheorem{corollary}{系}
\newtheorem{conjecture}{予想}
\newtheorem*{remark}{Remark}
\renewcommand{\proofname}{}
%%%%%%%%%

%%%%% フォント基本設定 %%%%%
\usepackage[T1]{fontenc}%8bit フォント
\usepackage{textcomp}%欧文フォントの追加
\usepackage[utf8]{inputenc}%文字コードをUTF-8
%\usepackage{otf}%otfパッケージ
\usepackage{pxfonts}%数式・英文ローマン体を Lxfont にする
\usepackage{bm}%数式太字
%%%%%%%%%%

%論理
\newcommand{\lang}[1]{\mathcal{L}_{\scalebox{1.2}{$#1$}}} %言語
\newcommand{\Set}[2]{\left\{\, #1 \mid #2\, \right\}} %論理式の対象化
\newcommand{\defeq}{\overset{\mathrm{def}}{=\joinrel=}} %\scalebox{3}[1]{=}}} %定義記号=(=\joinrel=も使える)
\newcommand{\defarrow}{\overset{\mathrm{def}}{\longleftrightarrow}} %定義記号↔
\newcommand{\provable}[1]{\vdash_{{\scriptsize #1}}} %証明可能
\newcommand{\negation}{\rightharpoondown\hspace{-0.25em}} %否定
\newcommand{\rarrow}{\hspace{0.25em}\rightarrow\hspace{0.25em}} %右矢印
\newcommand{\lrarrow}{\hspace{0.25em}\leftrightarrow\hspace{0.25em}} %左右矢印

%集合
\newcommand{\EXTAX}{\mbox{{\bf EXT}}} %外延性公理
\newcommand{\EQAX}{\mbox{{\bf EQ}}} %相等性公理
\newcommand{\EQAXEP}{\mbox{{\bf EQ}}_{\scalebox{1.2}{$\varepsilon$}}} %ε項の相等性公理
\newcommand{\COMAX}{\mbox{\bf COM}} %内包性公理
\newcommand{\ELEAX}{\mbox{{\bf ELE}}} %要素の公理
\newcommand{\REPAX}{\mbox{{\bf REP}}} %置換公理
\newcommand{\POWAX}{\mbox{{\bf POW}}} %冪集合公理
\newcommand{\PAIAX}{\mbox{{\bf PAI}}} %対集合公理
\newcommand{\INFAX}{\mbox{{\bf INF}}} %無限公理
\newcommand{\REGAX}{\mbox{{\bf REG}}} %正則性公理
\newcommand{\AC}{\mbox{{\bf CHOICE}}} %選択公理

\newcommand{\Univ}{\mathbf{V}} %宇宙
\newcommand{\set}[1]{\operatorname*{set} (#1)} %集合であることの論理式
\newcommand{\power}[1]{\operatorname*{P} (#1)} %冪集合
\newcommand{\rel}[1]{\operatorname*{rel} (#1)} %関係
\newcommand{\dom}[1]{\operatorname*{dom} (#1)} %類の定義域
\newcommand{\ran}[1]{\operatorname*{ran} (#1)} %類の値域
\newcommand{\sing}[1]{\operatorname*{sing} (#1)} %single-valuedの定義式
\newcommand{\fnc}[1]{\operatorname*{fnc} (#1)} %写像の定義式
\newcommand{\fon}{\operatorname*{:on}} %〇上の写像
\newcommand{\inj}{\overset{\mathrm{1:1}}{\longrightarrow}} %単射
\newcommand{\srj}{\overset{\mathrm{onto}}{\longrightarrow}} %全射
\newcommand{\bij}{\underset{\mathrm{onto}}{\overset{\mathrm{1:1}}{\longrightarrow}}} %全単射
\newcommand{\inv}[1]{{#1}^{-1}} %^{\operatorname{inv}}} %集合の反転
\newcommand{\rest}[2]{#1\hspace{-0.25em}\upharpoonright\hspace{-0.25em}{#2}} %制限写像
\newcommand{\tran}[1]{\operatorname*{tran} \left(#1\right)} %推移的類の定義式
\newcommand{\ord}[1]{\operatorname*{ord} \left(#1\right)} %順序数の定義式
\newcommand{\ON}{\mathrm{ON}} %順序数全体
\newcommand{\limo}[1]{\mathrm{lim.o}\left(#1\right)} %極限数の式
%\newcommand{\Natural}{{\boldsymbol \omega}} %自然数全体
\newcommand{\Natural}{\mathbf{N}} %自然数全体
 
\title{$\varepsilon$計算とクラスの導入による具体的で直観的な集合論の構築}%[略タイトル]{タイトル}
\author{関根深澤研 百合川尚学 \\ 学籍番号:29C17095}%[略名前]{名前}
\institute{}%[略所属]{所属}
\date{\today}%日付

\begin{document}
\mathtoolsset{showonlyrefs = true}

\begin{frame}[plain]\frametitle{}
\titlepage %表紙
\end{frame}

\begin{frame}\frametitle{Contents}
\tableofcontents %目次
\end{frame}

\section{導入}

\begin{frame}\frametitle{$\varepsilon$について}
	\begin{itemize}
	%\setlength{\itemsep}{10pt}
		\item $\varepsilon$計算は数論の無矛盾性証明のためにHilbert\cite{Hilbert}が開発.
		
		\item $\varepsilon$によって$\exists,\forall$を使う証明を命題論理の証明に埋め込める.
		
		\item 式$\varphi(x)$に対して
			\begin{align}
				\varepsilon x \varphi(x)
			\end{align}
			という形のモノを作り,$\varepsilon$項と呼ぶ.
			命題論理の証明に埋め込む際には,$\exists$や$\forall$の付いた式を
			\begin{align}
				\varphi(x/\varepsilon x \varphi(x)) &\defarrow \exists x \varphi(x), \\
				\varphi(x/\varepsilon x \negation x \varphi(x)) &\defarrow \forall x \varphi(x)
			\end{align}
			によって変換する.
	\end{itemize}
\end{frame}

\begin{frame}\frametitle{$\varepsilon$について}
	\begin{itemize}
		\item 今回$\varepsilon$項を導入したのは\textcolor{red}{「存在」と「実在」を同義}とするため.
		
		%\item Hilbertの$\varepsilon$計算ではなく,$\varepsilon$項を用いてHenkin拡大を行う. 
			
		\item つまり,導入の意図は\textcolor{red}{存在文}に対して\textcolor{red}{証人}を与えること(Henkin拡大):
			\begin{align}
				\exists x \varphi(x) \rarrow \varphi(\varepsilon x \varphi(x)).
			\end{align}
			この式は$\exists$に関する主要な公理.
			
		\item 「$\varphi$である集合が存在すれば,その一つは$\varepsilon x \varphi(x)$である.」
		
		\item 「$\negation \forall x \varphi(x) \rarrow \exists x \negation \varphi(x)$」と組み合わせると
			\begin{align}
				\varphi(\varepsilon x \negation \varphi(x)) \rarrow \forall x \varphi(x)
			\end{align}
			が出る.
	\end{itemize}
\end{frame}

\begin{frame}\frametitle{$\varepsilon$について}
	\begin{itemize}
	\setlength{\itemsep}{10pt}
		\item {\bf ZF}集合論では\textcolor{red}{集合というモノが用意されていない}ため,
			「存在」は「実在」ではない.たとえば
			\begin{align}
				\exists x\, \forall y\, (\, y \notin x\, )
			\end{align}
			は定理であり「空集合は存在する」と読むが,空集合を
			\underline{``実際に取ってくる''}ことは不可能.
		
		\item 空集合を手に入れる一つの方法は「定義による拡大」.
			
		\item $\varepsilon$項を使えば,$\exists$の公理と空集合の存在定理によって次が成立:
			\begin{align}
				\forall y\, (\, y \notin \textcolor{red}{\varepsilon x\, \forall y\, (\, y \notin x\, )}\, ).
			\end{align}
	\end{itemize}
	
	\begin{block}{$\varepsilon$項を使う他のメリット}
		\begin{itemize}
			\item 証明で用いる推論規則は三段論法のみで済む.
			\item 量化の範囲が具体的になる.
			\item 証明が容易になる場合がある.
		\end{itemize}
	\end{block}
\end{frame}

\begin{frame}\frametitle{クラスについて}
	\begin{itemize}
	%\setlength{\itemsep}{10pt}
		\item Bourbaki\cite{Bourbaki}や島内\cite{Shimauchi}でも$\varepsilon$項を使った集合論を展開.
		
		\item ところで,「$\varphi$である集合の全体」の意味の
			\begin{align}
				\Set{x}{\varphi(x)}
			\end{align}
			というモノも取り入れたい.
		
		\item Bourbaki\cite{Bourbaki}や島内\cite{Shimauchi}では
			\begin{align}
				\Set{x}{\varphi(x)} \defeq \varepsilon y\, \forall x\, 
				(\, \varphi(x) \lrarrow x \in y\, )
			\end{align}
			と定めるが,
			\begin{align}
				\exists y\, \forall x\, (\, \varphi(x) \lrarrow x \in y\, )
			\end{align}
			が成立しない場合は「$\varphi$である集合の全体」という意味を持たない.
		
		\item {\bf ZF}集合論では「定義による拡大」 or インフォーマルな導入.
			
		\item \textcolor{red}{式$\varphi$から直接$\Set{x}{\varphi(x)}$の形のモノを作ればよい
			(Bernays\cite{Bernays},竹内\cite{TakeuchiSet}).}
	\end{itemize}
\end{frame}

\begin{frame}\frametitle{クラスについて}
	\begin{exampleblock}{クラス}
		式$\varphi$に$x$のみが自由に現れているとき,$\varepsilon x \varphi(x),
		\quad \Set{x}{\varphi(x)}$の形のモノを{\bf クラス(class)}と呼ぶ.
	\end{exampleblock}
	
	\begin{itemize}
		\item クラスである$\varepsilon$項は集合である.
		\item 集合でないクラスもある.たとえば$\Set{x}{x = x}$や$\Set{x}{x \notin x}$
			は集合ではない.
	\end{itemize}
	
	集合の定義は竹内\cite{TakeuchiSet}に倣う.
	\begin{exampleblock}{集合}
		クラス$c$が
		\begin{align}
			\exists x\, (\, c = x\, )
		\end{align}
		を満たすとき$c$を{\bf 集合(set)}と呼ぶ.$\negation \exists x\, (\, c = x\, )$である場合は
		{\bf 真クラス(proper class)}と呼ぶ.定義により\textcolor{red}{集合はクラスである}.
	\end{exampleblock}
\end{frame}

\section{言語}
\begin{frame}\frametitle{主結果}
	\begin{itemize}
	%\setlength{\itemsep}{10pt}
		\item $\varepsilon$計算とクラスの直接的な導入を組み合わせたが,
			これが{\bf ZF}集合論の``妥当''な拡張であるかどうかが問題になる.
		
		\item 妥当性は,{\bf ZF}集合論の命題$\psi$に対して
			\begin{align}
				\mbox{{\bf ZF}集合論で$\psi$が証明可能}
				\Longleftrightarrow
				\mbox{新しい集合論で$\psi$が証明可能}
			\end{align}
			が成り立つかどうかで検証する.より精しく書くと,
	\end{itemize}
	
	\begin{block}{主結果}
		$\lang{\in}$の任意の文(自由な変項が現れない式) $\psi$に対して,
		「$\Gamma$から$\psi$への{\bf HK}の証明で$\lang{\in}$の式の列であるものが取れる」ことと
		「$\Sigma$から$\psi$への{\bf HE}の証明で$\mathcal{L}$の文の列であるものが取れる」ことは同値.
	\end{block}
	
	ここで,
	\begin{itemize}
		\item $\Gamma$は$\lang{\in}$の文で書かれた{\bf ZF}集合論の公理系.
		\item $\Sigma$は$\mathcal{L}$の文で書かれた本論文の公理系.
		\item {\bf HK}と{\bf HE}は証明体系(論理的公理+推論規則).
	\end{itemize}
	以下詳細.
	
	%\begin{itemize}
	%	\item 集合論の\textcolor{red}{言語}と
	%		\textcolor{red}{証明のルール}を明らかにしなくてはならない.
	%	
	%	\item 言語(の語彙)とは「\textcolor{red}{変項}」,「\textcolor{red}{述語記号}」,
	%		「\textcolor{red}{論理記号}」とその他もろもろの記号からなる.
	%		「\textcolor{red}{式(formula)}」は言語の語彙を用いて作られる.
	%		名詞の役を担うのが「\textcolor{red}{項(term)}」であり,文字は最もよく使われる項である.
	%		たとえば
	%		\begin{align}
	%			s \in t
	%		\end{align}
	%		と書けば一つの式が出来上がる.
	%	
	%	\item まず{\bf ZF}集合論の言語$\lang{\in}$を明示する.
	%\end{itemize}
\end{frame}

\begin{frame}\frametitle{言語$\lang{\in}$}
	%\begin{itemize}
		%\item 集合論の\textcolor{red}{言語}と\textcolor{red}{証明のルール}を明らかにしなくてはならない.
		
		%\item まずは{\bf ZF}集合論の言語$\lang{\in}$から.
	%\end{itemize}
	
	$\lang{\in}$とは{\bf ZF}集合論の言語である.
	
	\begin{exampleblock}{言語$\lang{\in}$の語彙(参考:菊池\cite{Kikuchi})}
		\begin{description}
			\item[矛盾記号] $\bot$
			\item[論理記号] $\negation,\ \vee,\ \wedge,\ \rarrow$
			\item[量化子] $\forall,\ \exists$
			\item[述語記号] $=,\ \in$
			\item[変項] $x,y,z,\cdots$.
		\end{description}
	\end{exampleblock}
\end{frame}

\begin{frame}\frametitle{言語$\lang{\in}$の項と式}
	$\lang{\in}$の項と式は次の規則で生成する.
	
	\begin{exampleblock}{$\lang{\in}$の項と式}
		\begin{description}
			\item[項] 変項は項であり,またこれらのみが項である.
				
			\item[式] 
				\begin{itemize}
					\item $\bot$は式である.
					\item 項$\tau$と項$\sigma$に対して
						$\tau \in \sigma$と$\tau = \sigma$は式である.
					\item 式$\varphi$に対して$\negation \varphi$は式である.
					\item 式$\varphi$と式$\psi$に対して$\varphi \vee \psi$と
						$\varphi \wedge \psi$と$\varphi \rarrow \psi$
						はいずれも式である.
					\item 式$\varphi$と項$x$に対して$\exists x \varphi$と
						$\forall x \varphi$は式である.
					\item これらのみが式である.
				\end{itemize}
		\end{description}
	\end{exampleblock}
\end{frame}

\begin{frame}\frametitle{言語の拡張}
	\begin{itemize}
	%\setlength{\itemsep}{10pt}
		\item クラスを正式に導入するために言語を拡張する.
		\item 拡張は二段階に分けて行う.
			始めに$\varepsilon$項のために拡張し,
			次に$\Set{x}{\varphi(x)}$の形の項のために拡張する.
			
		\item 始めの拡張で作る言語を$\lang{\varepsilon}$と名付ける.
	\end{itemize}

	\begin{exampleblock}{言語$\lang{\varepsilon}$の語彙(参考:島内\cite{Shimauchi})}
		\begin{description}
			\item[矛盾記号] $\bot$
			\item[論理記号] $\negation,\ \vee,\ \wedge,\ \rarrow$
			\item[量化子] $\forall,\ \exists,\ \varepsilon$
			\item[述語記号] $=,\ \in$
			\item[変項] $x,y,z,\cdots$.
		\end{description}
	\end{exampleblock}
\end{frame}

\begin{frame}{$\lang{\varepsilon}$の項と式}
	\begin{exampleblock}{$\lang{\varepsilon}$の項と式の定義(参考:Moser$\&$Zach\cite{Moser_Zach})}
		\begin{itemize}
			\item 変項は項である.
			\item $\bot$は式である.
			\item 項$\tau$と項$\sigma$に対して
				$\tau \in \sigma$と$\tau = \sigma$は式である.
			\item 式$\varphi$に対して$\negation \varphi$は式である.
			\item 式$\varphi$と式$\psi$に対して$\varphi \vee \psi$と
				$\varphi \wedge \psi$と$\varphi \rarrow \psi$
				はいずれも式である.
			\item 式$\varphi$と変項$x$に対して$\exists x \varphi$と
				$\forall x \varphi$は式である.
			\item \textcolor{red}{式$\varphi$と変項$x$に対して$\varepsilon x \varphi$は項である.}
			\item これらのみが項と式である.
		\end{itemize}
	\end{exampleblock}
	
	\begin{itemize}
	%\setlength{\itemsep}{10pt}
		\item $\lang{\in}$との大きな違いは
		 \textcolor{red}{項と式の生成が循環している}点.
		
		\item$\lang{\varepsilon}$の式が$\lang{\varepsilon}$の項を用いて
			作られるのは当然ながら,その逆に$\lang{\varepsilon}$の項もまた
			$\lang{\varepsilon}$の式から作られる.
			
		\item $\lang{\in}$の式は$\lang{\varepsilon}$の式でもある.
	\end{itemize}
\end{frame}

\begin{frame}\frametitle{言語$\mathcal{L}$}
	\begin{itemize}
		\item $\lang{\varepsilon}$の式$\varphi$と変項$x$で作られる
			$\varepsilon x \varphi$なる項を\textcolor{red}{$\varepsilon$項(epsilon term)}という.
			
		\item $\lang{\varepsilon}$の式$\varphi$と変項$x$で作られる
			$\Set{x}{\varphi}$なる項を\textcolor{red}{内包項}ということにする.
	\end{itemize}
	
	言語$\mathcal{L}$は本論文特有の言語である.内包項の導入は竹内\cite{TakeuchiSet}を参考にしているが,
	$\lang{\varepsilon}$の式を使って入れるという点が本論文の特徴である.
	
	\begin{exampleblock}{言語$\mathcal{L}$の語彙}
		\begin{description}
			\item[矛盾記号] $\bot$
			\item[論理記号] $\negation,\ \vee,\ \wedge,\ \rarrow$
			\item[量化子] $\forall,\ \exists$
			\item[述語記号] $=,\ \in$
			\item[変項] $x,y,z,\cdots$.
			\item[$\varepsilon$項と内包項] 上記のもの 
		\end{description}
	\end{exampleblock}
\end{frame}

\begin{frame}\frametitle{$\mathcal{L}$の項と式}

	\begin{exampleblock}{$\mathcal{L}$の項と式の定義}
		\begin{description}
			\item[項] 変項,$\varepsilon$項,内包項は項である.またこれらのみが項である.
			
			\item[式] 
				\begin{itemize}
					\item $\bot$は式である.
					\item 項$\tau$と項$\sigma$に対して
						$\tau \in \sigma$と$\tau = \sigma$は式である.
					\item 式$\varphi$に対して$\negation \varphi$は式である.
					\item 式$\varphi$と式$\psi$に対して$\varphi \vee \psi$と
						$\varphi \wedge \psi$と$\varphi \rarrow \psi$
						はいずれも式である.
					\item 式$\varphi$と変項$x$に対して$\exists x \varphi$と
						$\forall x \varphi$は式である.
					\item これらのみが式である.
				\end{itemize}
		\end{description}
	\end{exampleblock}
	
	言語$\mathcal{L}$こそが本論文の標準言語である.
	
\end{frame}

\begin{frame}\frametitle{扱う式の制限}
	上で作った項や式の中には
	\begin{align}
		\varepsilon x\, (\, y = y\, ),\quad \Set{x}{z \neq z},\quad \forall x\, (\, u \in v\, )
	\end{align}
	のような意味の通らないものが氾濫しているので,排除する.
	
	\begin{itemize}
		\item $\varepsilon x \varphi$なる形の$\varepsilon$項は,$\varphi$に$x$ ``のみ''自由に現れているとき
			\textcolor{red}{主要$\varepsilon$項}と呼ぶことにする.
			
		\item $\Set{x}{\varphi}$なる形の内包項は,$\varphi$に$x$ ``が''自由に現れているとき,
			\textcolor{red}{正則内包項}と呼ぶことにする.
			
		\item 以降扱う式に現れる$\varepsilon$項は全て主要$\varepsilon$項,内包項は全て正則内包項であるとし,
			$\forall x \varphi$や$\exists x \varphi$なる式は$\varphi$に$x$が自由に現れているとする.
	\end{itemize}
\end{frame}

\begin{frame}\frametitle{クラス}
	
	主要ε項と同様に,$\Set{x}{\varphi}$なる形の内包項は,$\varphi$に$x$ ``のみ''自由に現れているとき,
	\textcolor{red}{主要内包項}と呼ぶことにする.
			
	\begin{exampleblock}{クラス}
		主要ε項と主要内包項をクラス(class)と呼ぶ.またこれらのみがクラスである.
	\end{exampleblock}
	
	主要$\varepsilon$項は実際は集合である(後述).
	
\end{frame}

\section{式の書き換え}
\begin{frame}\frametitle{$\mathcal{L}$の式を$\lang{\varepsilon}$の式に書き換える}
	\begin{itemize}
		\item $\varepsilon$項を導入したのは,\underline{存在文に対して証人を付けるため}:
			\begin{align}
				\exists x \varphi(x) \rarrow \varphi(\varepsilon x \varphi(x)).
			\end{align}
			
		\item しかし$\varphi$に内包項が使われているとき,$\varepsilon x \varphi(x)$は使えない(作られていない).
		
		\item そのときは,$\varphi$を``同値''な$\lang{\varepsilon}$の式$\hat{\varphi}$に書き換えて
			\begin{align}
				\exists x \varphi(x) \rarrow \varphi(\varepsilon x \hat{\varphi}(x))
			\end{align}
			を公理とすればよい.
	\end{itemize}
\end{frame}

\begin{frame}\frametitle{式の書き換え}
	$\varphi$の部分式のうち原子式であるところを表に従って直したものを「$\varphi$の\textcolor{red}{書き換え}」と呼ぶ.
	%また書き換えた式の中で束縛されている変項を名前替えした式も書き換えとする.
	
	\begin{table}[H]
		\begin{center}
		\begin{tabular}{c|c|c}
			 & 元の式 & 書き換え後 \\ \hline \hline
			(1) & $a = \Set{z}{\psi}$ & $\forall v\, (\, v \in a \lrarrow \psi(z/v)\, )$ \\ \hline
			(2) & $\Set{y}{\varphi} = b$ & $\forall u\, (\, \varphi(y/u) \lrarrow u \in b\, )$ \\ \hline
			(3) & $\Set{y}{\varphi} = \Set{z}{\psi}$ & $\forall u\, (\, \varphi(y/u) \lrarrow \psi(z/u)\, )$ \\ \hline
			(4) & $a \in \Set{z}{\psi}$ & $\psi(z/a)$ \\ \hline
			(5) & $\Set{y}{\varphi} \in b$ & $\exists s\, (\, \forall u\, (\, \varphi(y/u) \lrarrow u \in s\, ) \wedge s \in b\, )$ \\ \hline
			(6) & $\Set{y}{\varphi} \in \Set{z}{\psi}$ & $\exists s\, (\, \forall u\, (\, \varphi(y/u) \lrarrow u \in s\, ) \wedge \psi(z/s)\, )$ \\ \hline
		\end{tabular}
		\end{center}
	\end{table}
	
	ここで,
	\begin{itemize}
		\item $a,b$は変項か主要$\varepsilon$項.
		\item $\psi(z/v)$は$\psi$に自由に現れている$z$に$v$を代入した式.
	\end{itemize}
\end{frame}

\section{証明}
%\begin{frame}\frametitle{主結果}
%	本論文の主結果は,{\bf ZF}集合論の任意の命題$\psi$に対して
%	\begin{align}
%		\mbox{{\bf ZF}集合論で$\psi$が証明可能}
%		\Longleftrightarrow
%		\mbox{本論文の集合論で$\psi$が証明可能}
%	\end{align}
%	が成り立つということであるが,より精密に書くと
%	\begin{block}{主結果}
%		$\lang{\in}$の任意の文(自由な変項が現れない式) $\psi$に対して,
%		「$\Gamma$から$\psi$への{\bf HK}の証明で$\lang{\in}$の式の列であるものが取れる」ことと
%		「$\Sigma$から$\psi$への{\bf HE}の証明で$\mathcal{L}$の文の列であるものが取れる」ことは同値.
%	\end{block}
%	
%	ここで,
%	\begin{itemize}
%		\item $\Gamma$は$\lang{\in}$の文で書かれた{\bf ZF}集合論の公理系.
%		\item $\Sigma$は$\mathcal{L}$の文で書かれた本論文の公理系.
%		\item {\bf HK}と{\bf HE}は証明体系(論理的公理+推論規則).
%	\end{itemize}
%	以下詳細.
%\end{frame}

\begin{frame}\frametitle{{\bf ZF}の公理系}
	\begin{alertblock}{$\Gamma$の公理(参考:Kunen\cite{Kunen})}
		\begin{description}
			\item[外延性] 「同一の要素を持つ集合同士は等しい」
				\begin{align}
					\forall x\, \forall y\, (\, \forall z\, (\, z \in x \lrarrow z \in y\, ) \rarrow x = y\, ).
				\end{align}
			\item[相等性] 「等しい集合同士の服属関係は一致する」
				\begin{align}
					&\forall x\, \forall y\, (\, x = y \rarrow y = x\, ), \\
					&\forall x\, \forall y\, \forall z\, 
					(\, x = y \rarrow (\, x \in z \rarrow y \in z\, )\, ), \\
					&\forall x\, \forall y\, \forall z\, 
					(\, x = y \rarrow (\, z \in x \rarrow z \in y\, )\, ).
				\end{align}
			\item[置換] 「集合を写像で写した像は集合」次の式の全称閉包: 
				\begin{align}
					&\forall x\, \forall y\, \forall z\, 
					(\, \varphi(x,y) \wedge \varphi(x,z) \rarrow y = z\, ) \\
					&\rarrow \forall a\, \exists z\, \forall y\,
					(\, y \in z \lrarrow \exists x\, (\, x \in a \wedge \varphi(x,y)\, )\, ).
				\end{align}
		\end{description}	
	\end{alertblock}
	置換公理は式$\varphi$ごとに公理となるので\textcolor{red}{図式(schema)}と呼ばれる.
\end{frame}

\begin{frame}\frametitle{{\bf ZF}の公理系}
	\begin{alertblock}{$\Gamma$の公理}
		\begin{description}
			\item[対] 「対集合が存在する」
				\begin{align}
					\forall x\, \forall y\, \exists p\, \forall z\, (\, x = z \vee y = z \lrarrow z \in p\, ).
				\end{align}
			\item[合併] 「合併集合が存在する」
				\begin{align}
					\forall x\, \exists u\, \forall y\, (\, \exists z\, (\, z \in x \wedge y \in z\, ) \lrarrow y \in u\, ).
				\end{align}
			\item[冪] 「冪集合が存在する」
				\begin{align}
					\forall x\, \exists p\, \forall y\, (\, \forall z\, (\, z \in y \rarrow z \in x\, ) 
					\lrarrow y \in p\, ).
				\end{align}
		\end{description}
	\end{alertblock}
	これらの公理によって既存の集合から新しい集合が作られる.
\end{frame}

\begin{frame}\frametitle{{\bf ZF}の公理系}
	\begin{alertblock}{$\Gamma$の公理}
		\begin{description}
			\item[正則性] 「空でない集合は自分自身と交わらない要素を持つ」
				\begin{align}
					\forall r\, (\, \exists x\, (\, x \in r\, ) \rarrow
					\exists y\, (\, y \in r \wedge \forall z\, (\, z \in r \rarrow
					z \notin y\, )\, )\, ).
				\end{align}
			\item[無限] 「自然数の全体を含む集合が存在する」
				\begin{align}
					&\exists x\, (\, \exists s\, (\, \forall t\, (\, t \notin s\, ) \wedge s \in x\, ) 
					\wedge \forall y\, (\, y \in x \rarrow \\
					&\quad \exists u\, (\, \forall v\, (\, v \in u \lrarrow v \in y \vee v = y\, ) 
					\wedge u \in x\, )\, )\, ).
				\end{align}
		\end{description}
	\end{alertblock}
	正則性公理によって集合の範囲が決定する(整礎集合).また無限公理は唯一「集合の存在」に言及している.
\end{frame}

\begin{frame}\frametitle{古典論理}
	{\bf HK}とは古典論理(classical logic)のHilbert流証明体系である.
	\begin{alertblock}{{\bf HK}の論理的公理(命題論理)(参考:戸次\cite{Bekki})}
		\begin{description}
			\item[含意の分配] $(\, \varphi \rarrow (\, \psi \rarrow \chi\, )\, ) 
				\rarrow (\, (\, \varphi \rarrow \psi\, )
				\rarrow (\, \varphi \rarrow \chi\, )\, ).$
			\item[含意の導入] $\varphi \rarrow (\, \psi \rarrow \varphi\, ).$
			\item[矛盾の導入] $\varphi \rarrow (\, \negation \varphi \rarrow \bot\, ),
				\quad \negation \varphi \rarrow (\, \varphi \rarrow \bot\, ).$
			\item[否定の導入] $(\, \varphi \rarrow \bot\, ) \rarrow\ \negation \varphi.$
			\item[論理和の導入] $\varphi \rarrow \varphi \vee \psi,
				\quad \psi \rarrow \varphi \vee \psi.$
			\item[論理和の除去] $(\, \varphi \rarrow \chi\, ) \rarrow 
					(\, (\, \psi \rarrow \chi\, ) 
					\rarrow (\, \varphi \vee \psi \rarrow \chi\, )\, ).$
			\item[論理積の導入] $\varphi \rarrow (\, \psi \rarrow (\, \varphi \wedge \psi\, )\, ).$
			\item[論理積の除去] $\varphi \wedge \psi \rarrow \varphi,
				\quad \varphi \wedge \psi \rarrow \psi.$
			\item[二重否定の除去] $\negation \negation \varphi \rarrow \varphi$.
		\end{description}
	\end{alertblock}
\end{frame}

\begin{frame}\frametitle{古典論理}
	\begin{alertblock}{{\bf HK}の論理的公理(量化)(参考:戸次\cite{Bekki})}
		\begin{description}
			\item[全称の導入] $\forall y\, (\, \psi \rarrow \varphi(x/y)\, ) 
				\rarrow (\, \psi \rarrow \forall x \varphi\, ).$
				
			\item[全称の除去] $\forall x \varphi \rarrow \varphi(x/t).$
				
			\item[存在の導入] $\varphi(x/t) \rarrow \exists x \varphi.$
				
			\item[存在の除去] $\forall y\, (\, \varphi(x/y) \rarrow \psi\, )
				\rarrow (\, \exists x \varphi \rarrow \psi\, ).$
		\end{description}
	\end{alertblock}
	
	\begin{exampleblock}{{\bf HK}の証明(参考:戸次\cite{Bekki})}
		「$\Gamma$からの{\bf HK}の証明で$\lang{\in}$の式の列であるもの」とは,
		$\lang{\in}$の式の列$\varphi_{1},\cdots,\varphi_{n}$で,各$\varphi_{i}$が次のいずれかであるもの:
		\begin{itemize}
			\item {\bf HK}の公理である
			\item $\Gamma$の公理である
			\item $\varphi_{j},\varphi_{k}\ (j,k < i)$から三段論法で得られる
			\item $\varphi_{j}\ (j < i)$から汎化で得られる.
		\end{itemize}
	\end{exampleblock}
\end{frame}

\begin{frame}\frametitle{$\Sigma$の公理}
	\begin{itemize}
		\item $\Sigma$は「対」「合併」「冪」「正則性」「無限」は$\Gamma$と共通.
		\item $\Sigma$の「置換」は,二つの変項が現れる式に対しての言明に替わる.
		\item 新しく「内包性」と「要素」の公理が追加.
	\end{itemize}
	
	\begin{alertblock}{$\Sigma$の公理(参考:竹内\cite{TakeuchiSet})}
		$a,b,c$をクラスとするとき
		\begin{description}
			\item[外延性] 「同一の要素を持つクラス同士は等しい」
				\begin{align}
					\forall z\, (\, z \in a \lrarrow z \in b\, ) \rarrow a = b.
				\end{align}
			\item[相等性] 「等しい集合同士の服属関係は一致する」
				\begin{align}
					&a = b \rarrow b = a, \\
					&a = b \rarrow (\, a \in c \rarrow b \in c\, ), \\
					&a = b \rarrow (\, c \in a \rarrow c \in b\, ).
				\end{align}
		\end{description}	
	\end{alertblock}
\end{frame}

\begin{frame}\frametitle{$\Sigma$の公理}
	
	\begin{alertblock}{$\Sigma$の公理}
		$a,b$をクラスとするとき
		\begin{description}
			\item[内包性] 「$\Set{y}{\varphi(y)}$は$\varphi$であるモノの全体」
				\begin{align}
					\forall x\, (\, x \in \Set{y}{\varphi(y)} \lrarrow \varphi(x)\, ).
				\end{align}
				ただし$\Set{y}{\varphi(y)}$は主要内包項.
				
			\item[要素] 「要素となりうるものは集合に限る」
				\begin{align}
					a \in b \rarrow \exists x\, (\, a = x\, ).
				\end{align}
		\end{description}	
	\end{alertblock}
	
	\begin{itemize}	
		\item クラスは量化しないのでこれらの公理は図式(schema)である.	
		\item 要素の公理はG$\ddot{\mbox{o}}$del\cite{Godel}の引用である.
	\end{itemize}
\end{frame}

\begin{frame}\frametitle{{\bf HE}の公理}
	{\bf HE}は本論文特有の証明体系である.命題論理の論理的公理は{\bf HK}と共通するが,量化公理が違う.
	
	\begin{alertblock}{{\bf HE}の論理的公理(量化)}
		\begin{description}
			\item[De Morganの法則] $\negation \forall x \varphi(x) \rarrow \exists x \negation \varphi(x).$
				
			\item[全称の除去] $\forall x \varphi \rarrow \varphi(x/\tau).$
				
			\item[存在の導入] $\varphi(x/\tau) \rarrow \exists x \varphi.$
				
			\item[存在の除去] $\exists x \varphi(x) \rarrow \varphi(\varepsilon x \hat{\varphi}(x)).$
		\end{description}
		$\hat{\varphi}$とは,$\varphi$が$\lang{\varepsilon}$の式でない場合に書き換えたもの.
		$\varphi$が$\lang{\varepsilon}$の式ならば$\hat{\varphi}$は$\varphi$とする.
		また$\tau$は主要$\varepsilon$項とする.
	\end{alertblock}
	
	{\bf HE}の公理により,\underline{量化$\forall x, \exists x$の亘る範囲は主要$\varepsilon$項の上}となる.
\end{frame}

\begin{frame}\frametitle{{\bf HE}の証明}
	{\bf HK}と違い,{\bf HE}の証明は文で行う.
	
	\begin{exampleblock}{{\bf HE}の証明}
		「$\Sigma$からの{\bf HE}の証明で$\mathcal{L}$の文の列であるもの」とは,
		$\mathcal{L}$の文の列$\varphi_{1},\cdots,\varphi_{n}$で,各$\varphi_{i}$が次のいずれかであるもの:
		\begin{itemize}
			\item {\bf HE}の公理である
			\item $\Sigma$の公理である
			\item $\varphi_{j},\varphi_{k}\ (j,k < i)$から三段論法で得られる
		\end{itemize}
	\end{exampleblock}
	
	「$\Sigma$から$\psi$への{\bf HE}の証明で$\mathcal{L}$の文の列であるもの」が取れることを
	\begin{align}
		\Sigma \vdash \psi
	\end{align}
	と書く.
\end{frame}

\section{保存拡大}
\begin{frame}\frametitle{主結果の証明方針}
	次の3ステップに分割する:
	\begin{description}
		\item[step1]  「$\Sigma$から$\psi$への{\bf HE}の証明で$\lang{\varepsilon}$の文の列
			であるものが取れる」ならば「$\Gamma$から$\psi$への{\bf HK}の証明で$\lang{\in}$の
			式の列であるものが取れる」ことを示す.
		
		\item[step2] 「$\Gamma$から$\psi$への{\bf HK}の証明で$\lang{\in}$の
			式の列であるものが取れる」ならば「$\Sigma$から$\psi$への{\bf HE}の証明で
			$\lang{\varepsilon}$の文の列であるものが取れる」ことを示す.
		
		\item[step3]  「$\Sigma$から$\psi$への{\bf HE}の証明で$\mathcal{L}$の文の列
			であるものが取れる」ならば 「$\Sigma$から$\psi$への{\bf HE}の証明で
			$\lang{\varepsilon}$の文の列であるものが取れる」ことを示す.
	\end{description}
\end{frame}

\section{いくつかの性質}
\begin{frame}\frametitle{集合}
	\begin{exampleblock}{集合}
	クラス$a$が集合であるとは,
	\begin{align}
		\Sigma \vdash \exists x\, (\, a = x\, )
	\end{align}
	となること.$\Sigma \vdash\ \negation \exists x\, (\, a = x\, )$なら$a$は真クラス(proper class).
	\end{exampleblock}
	
	\begin{block}{主要$\varepsilon$項は集合}
		任意の主要$\varepsilon$項$\tau$に対して$\Sigma \vdash \exists x\, (\, \tau = x\, )$.
	\end{block}
	実際,外延性公理より$\tau = \tau$となり,また
	\begin{align}
		\tau = \tau \rarrow \exists x\, (\, \tau = x\, )
	\end{align}
	は{\bf HE}の量化公理なので,三段論法で$\exists x\, (\, \tau = x\, )$が出る.
\end{frame}

\begin{frame}\frametitle{全称式の導出}
	\begin{block}{全称式の導出}
		$\varphi$を,$x$のみが自由に現れる$\mathcal{L}$の式とするとき,
		\begin{align}
			\vdash \varphi(\varepsilon x \negation \hat{\varphi}(x)) \rarrow \forall x \varphi(x).
		\end{align}
		ただし$\hat{\varphi}$は必要に応じて$\varphi$を$\lang{\varepsilon}$の式に書き換えたもの.
	\end{block}
	実際,
	\begin{align}
		\negation \forall x \varphi(x) &\rarrow \exists x \negation \varphi(x), \\
		\exists x \negation \varphi(x) &\rarrow\ \negation \varphi(\varepsilon x \negation \hat{\varphi}(x))
	\end{align}
	は{\bf HE}の量化公理であり,
	\begin{align}
		\negation \forall x \varphi(x) \rarrow\ \negation \varphi(\varepsilon x \negation \hat{\varphi}(x))
	\end{align}
	が導かれ,対偶律より
	\begin{align}
		\varphi(\varepsilon x \negation \hat{\varphi}(x)) \rarrow \forall x \varphi(x).
	\end{align}
\end{frame}

\begin{frame}\frametitle{内包項の$\varepsilon$項表現}
	\begin{block}{集合である主要内包項は$\varepsilon$項で書ける}
		$\varphi$を,$x$のみが自由に現れる$\mathcal{L}$の式とするとき,
		\begin{align}
			\exists s\, (\, \Set{x}{\varphi(x)} = s\, )
			\vdash \Set{x}{\varphi(x)} = 
			\varepsilon s\, \forall x\, (\, \varphi(x) \lrarrow x \in s\, ).
		\end{align}
	\end{block}
	
	$\forall x\, (\, \varphi(x) \lrarrow x \in s\, )$は$\Set{x}{\varphi(x)} = s$の書き換えなので,
	\begin{align}
		\exists s\, (\, \Set{x}{\varphi(x)} = s\, ) \rarrow 
		\Set{x}{\varphi(x)} = \varepsilon s\, \forall x\, (\, \varphi(x) \lrarrow x \in s\, )
	\end{align}
	は{\bf HE}の量化公理である.従って$\exists s\, (\, \Set{x}{\varphi(x)} = s\, )$を公理とすれば
	\begin{align}
		\Set{x}{\varphi(x)} = \varepsilon s\, \forall x\, (\, \varphi(x) \lrarrow x \in s\, )
	\end{align}
	が定理として出る.
\end{frame}

\begin{frame}\frametitle{書き換えの同値性}
	\begin{block}{書き換えは同値}
		$\varphi$を$\lang{\varepsilon}$の文ではない$\mathcal{L}$の文とするとき,
		\begin{align}
			\Sigma \vdash \varphi \lrarrow \hat{\varphi}.
		\end{align}
		ただし$\hat{\varphi}$は$\varphi$の書き換え.
	\end{block}
	
	内包性公理と要素の公理はこの同値性を得るためにある.
	
\end{frame}

\section{証明が簡単になる例}
\begin{frame}\frametitle{$\exists x \varphi(x) \rarrow \exists y \varphi(y)$の証明}
	\begin{block}{}
		$\varphi$は$\lang{\in}$の式で,$x$のみ自由に現れているとし,$y$は$x$への代入について自由であるとするとき,
		\begin{align}
			\vdash \exists x \varphi(x) \rarrow \exists y \varphi(y).
		\end{align}
	\end{block}
	{\bf HE}で証明すると,
	\begin{align}
		\exists x \varphi(x) &\rarrow \varphi(\varepsilon x \varphi(x)), \\
		\varphi(\varepsilon x \varphi(x)) &\rarrow \exists y \varphi(y)
	\end{align}
	が共に{\bf HE}の公理なので
	\begin{align}
		\exists x \varphi(x) \rarrow \exists y \varphi(y)
	\end{align}
	が従う.
\end{frame}

\begin{frame}\frametitle{$\exists x \varphi(x) \rarrow \exists y \varphi(y)$の証明}
	一方で{\bf HK}で証明すると,
	\begin{align}
		\varphi(x) \rarrow \exists y \varphi(y)
	\end{align}
	は{\bf HK}の公理であり,汎化によって
	\begin{align}
		\forall x\, (\, \varphi(x) \rarrow \exists y \varphi(y)\, )
	\end{align}
	が得られる.
	\begin{align}
		\forall x\, (\, \varphi(x) \rarrow \exists y \varphi(y)\, ) \rarrow 
		(\, \exists x \varphi(x) \rarrow \exists y \varphi(y)\, )
	\end{align}
	が{\bf HK}の公理なので,三段論法で
	\begin{align}
		\exists x \varphi(x) \rarrow \exists y \varphi(y)
	\end{align}
	が出る.
\end{frame}

\begin{frame}\frametitle{$\exists x \varphi(x) \rarrow \exists y \varphi(y)$の証明}
	{\bf HE}で証明した際,$A$を$\exists x \varphi(x)$,$B$を$\varphi(\varepsilon x \varphi(x))$, 
	$C$を$\exists y \varphi(y)$として
	\begin{align}
		&A \rarrow B, \\
		&B \rarrow C, \\
		&(\, B \rarrow C\, ) \rarrow (\, A \rarrow (\, B \rarrow C\, )\, ), \\
		&A \rarrow (\, B \rarrow C\, ), \\
		&(\, A \rarrow (\, B \rarrow C\, )\, ) \rarrow (\, (\, A \rarrow B\, ) \rarrow (\, A \rarrow C\, )\, ), \\
		&(\, A \rarrow B\, ) \rarrow (\, A \rarrow C\, ), \\
		&A \rarrow C
	\end{align}
	を追加しなくては証明とならないが,証明の組み立ては{\bf HK}よりも直観的である.
\end{frame}

\begin{frame}\frametitle{$\exists y\, (\, \exists x \varphi(x) \rarrow \varphi(y)\, )$の証明}
	\begin{block}{}
		$\varphi$は$\lang{\in}$の式で,$x$のみ自由に現れているとし,$y$は$x$への代入について自由であるとするとき,
		\begin{align}
			\vdash \exists y\, (\, \exists x \varphi(x) \rarrow \varphi(y)\, ).
		\end{align}
	\end{block}
	{\bf HE}で証明すると,
	\begin{align}
		\exists x \varphi(x) \rarrow \varphi(\varepsilon x \varphi(x))
	\end{align}
	は{\bf HE}の公理であり,
	\begin{align}
		&(\, \exists x \varphi(x) \rarrow \varphi(\varepsilon x \varphi(x))\, ) \\
		&\rarrow \exists y\, (\, \exists x \varphi(x) \rarrow \varphi(y)\, )
	\end{align}
	も{\bf HE}の公理なので,三段論法で
	\begin{align}
		\exists y\, (\, \exists x \varphi(x) \rarrow \varphi(y)\, )
	\end{align}
	が従う.
\end{frame}

\begin{frame}\frametitle{$\exists y\, (\, \exists x \varphi(x) \rarrow \varphi(y)\, )$の証明}
	一方で{\bf HK}で証明すると,
	\begin{align}
		\exists x \varphi(x) \rarrow \varphi(\varepsilon x \varphi(x))
	\end{align}
	と
	\begin{align}
		(\, \exists x \varphi(x) \rarrow \exists y \varphi(y) \, )
		&\rarrow (\, \negation \exists x \varphi(x) \vee \exists y \varphi(y) \, ), \\
		&\rarrow \exists y\, (\, \negation \exists x \varphi(x) \vee \varphi(y) \, ), \\
		&\rarrow \exists y\, (\, \exists x \varphi(x) \rarrow \varphi(y) \, )
	\end{align}
	の証明が必要になる.明らかに数行で終わる証明ではないし,証明の方針も直観とはずれる.
\end{frame}

\begin{thebibliography}{数字}
	\bibitem{key1} Moser, G. and Zach, R., ``The Epsilon Calculus and Herbrand Complexity'',
		Studia Logica 82, 133-155 (2006)
	
	\bibitem{key2} 高橋優太, ``1階述語論理に対する$\varepsilon$計算'', \\
		http://www2.kobe-u.ac.jp/~mkikuchi/ss2018files/takahashi1.pdf 
		
	\bibitem{key3} キューネン数学基礎論講義
	
	\bibitem{key5} ブルバキ, 数学原論 集合論 1, 
	
	\bibitem{key4} 竹内外史, 現代集合論入門, 増強版第5刷, 日本評論社, 2016, pp. 138-183, ISBN 978-4-535-60116-1
	
	\bibitem{key6} 島内剛一, 数学の基礎, 第1版第10刷, 日本評論社, 2016, ISBN 978-4-535-60106-2
	
	\bibitem{key7} 戸次大介, 数理論理学, 第2刷, 東京大学出版会, 2016, pp. 148-166, ISBN 978-4-13-062915-7
	
	\bibitem{key8} K. G$\ddot{\mbox{o}}$del, $The\ Consistency\ of\ the\ Continuum\ Hypothesis$, 8th printing, Princeton University Press 1970, p. 3, ISBN 0-691-07927-7.
	
	\bibitem{key9} 菊地誠, 不完全性定理, 初版3刷, 共立出版株式会社, 2017, pp. 86-91, ISBN 978-4-320-11096-0
	
	\bibitem{key10} 前原昭二, 記号論理入門, 新装版第8刷, 日本評論社, 2018, pp. 106-115, ISBN 4-535-60144-5
	
	\bibitem{key11} Kenji Miyamoto and Georg Moser, The Epsilon Calculus with Equality and Herbrand Complexity
\end{thebibliography}
\end{document}
\section{成り立つこと}
	次の定理は他の公理および構造的帰納法と併せて示される.
	
	\begin{screen}
		\begin{thm}[書き換えの同値性]
			$\varphi$を$\mathcal{L}$の文するとき,
			\begin{align}
				\varphi \lrarrow \hat{\varphi}.
			\end{align}
		\end{thm}
	\end{screen}
	
\newpage
	\begin{itembox}[l]{証明が容易になる例}
		$\varphi$を$x$のみ自由に現れる式とし,
		$y$を$\varphi$の中で$x$への代入について自由である変項とするとき,
		\begin{align}
			\exists y\, (\, \exists x \varphi(x) \rarrow \varphi(y)\, ).
		\end{align}
	\end{itembox}
	
	\begin{sketch}
		\begin{align}
			\psi(y) \defarrow \exists x \varphi(x) \rarrow \varphi(\varepsilon x \varphi(x))
		\end{align}
		とおけば
		\begin{align}
			\exists x \varphi(x) \rarrow \varphi(\varepsilon x \varphi(x))
		\end{align}
		より
		\begin{align}
			\psi(\varepsilon x \varphi(x))
		\end{align}
		が成り立つ.ゆえに
		\begin{align}
			\exists y \psi(y).
		\end{align}
	\end{sketch}
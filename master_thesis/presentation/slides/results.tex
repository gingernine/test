\section{成り立つこと}
	次の定理は他の公理および構造的帰納法と併せて示される.
	
	\begin{screen}
		\begin{thm}[書き換えの同値性]
			$\varphi$を$\mathcal{L}$の文するとき,
			\begin{align}
				\varphi \lrarrow \hat{\varphi}.
			\end{align}
		\end{thm}
	\end{screen}

\newpage
	\begin{itembox}[l]{証明が容易になる例}
		$\varphi$を$x$のみ自由に現れる式とし,
		$y$を$\varphi$の中で$x$への代入について自由である変項とするとき,
		\begin{align}
			\vdash \exists x \varphi(x) \rarrow \exists y \varphi(y).
		\end{align}
	\end{itembox}
	
	\begin{sketch}
		公理と演繹定理より
		\begin{align}
			\exists x \varphi(x) \vdash \varphi(\varepsilon x \varphi(x))
		\end{align}
		となり,また公理より
		\begin{align}
			\vdash \varphi(\varepsilon x \varphi(x))
			\rarrow \exists y \varphi(y)
		\end{align}
		も成り立つので,三段論法より
		\begin{align}
			\exists x \varphi(x) \vdash \exists y \varphi(y)
		\end{align}
		が出る.
		\QED
	\end{sketch}

\newpage
	$\exists x \varphi(x) \rarrow \exists y \varphi(y)$を{\bf HK}で証明すると,
	
	公理より
	\begin{align}
		\provable{\mbox{{\bf HK}}} \varphi(x) \rarrow \exists y \varphi(y)
	\end{align}
	が成り立つので,汎化によって
	\begin{align}
		\provable{\mbox{{\bf HK}}} \forall x\, (\, \varphi(x) \rarrow \exists y \varphi(y)\, )
	\end{align}
	となり,公理
	\begin{align}
		\provable{\mbox{{\bf HK}}} \forall x\, (\, \varphi(x) \rarrow \exists y \varphi(y)\, )
		\rarrow (\, \exists x \varphi(x) \rarrow \exists y \varphi(y)\, )
	\end{align}
	との三段論法より
	\begin{align}
		\provable{\mbox{{\bf HK}}} \exists x \varphi(x) \rarrow \exists y \varphi(y)
	\end{align}
	が出る.

\newpage
	\begin{itembox}[l]{証明が容易になる例}
		$\varphi$を$x$のみ自由に現れる式とし,
		$y$を$\varphi$の中で$x$への代入について自由である変項とするとき,
		\begin{align}
			\vdash \exists y\, (\, \exists x \varphi(x) \rarrow \varphi(y)\, ).
		\end{align}
	\end{itembox}
	
	\begin{sketch}
		公理より
		\begin{align}
			\vdash \exists x \varphi(x) \rarrow \varphi(\varepsilon x \varphi(x))
		\end{align}
		が成り立つので,公理より
		\begin{align}
			\vdash \exists y\, (\, \exists x \varphi(x) \rarrow \varphi(y)\, )
		\end{align}
		となる.
		\QED
	\end{sketch}

\newpage
	$\exists y\, (\, \exists x \varphi(x) \rarrow \varphi(y)\, )$を{\bf HK}で証明すると
	
	\begin{align}
		\exists x \varphi(x) \rarrow \exists y \varphi(y)
	\end{align}
	と
	\begin{align}
		(\, \exists x \varphi(x) \rarrow \exists y \varphi(y) \, )
		&\rarrow (\, \negation \exists x \varphi(x) \vee \exists y \varphi(y) \, ), \\
		&\rarrow \exists y\, (\, \negation \exists x \varphi(x) \vee \varphi(y) \, ), \\
		&\rarrow \exists y\, (\, \exists x \varphi(x) \rarrow \varphi(y) \, )
	\end{align}
	から示される.
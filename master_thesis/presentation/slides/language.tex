\section{導入}
	\begin{itemize}
		\item 集合論の言語$\mathcal{L}_{\in} = \{\in\}$の自然な拡張によりクラスを導入することは容易い:
			\begin{align}
				\Set{x}{\varphi(x)} \quad (\mbox{$\varphi$は$\mathcal{L}_{\in}$の式})
			\end{align}
			なるオブジェクトを取り入れればよい.
			
		\item これが持つ意味は
			\begin{align}
				\forall u\, \left(\, u \in \Set{x}{\varphi(x)}
				\Longleftrightarrow \varphi(u)\, \right)
			\end{align}
			を満たすモノ.
			
\newpage
		\item $\mathcal{L}_{\in}$においては無定義概念であった集合が
			\begin{align}
				\exists x\, (\, x = a\, )
			\end{align}
			を満たすクラス$a$のことであると\textcolor{red}{定義できる}.
		
			\begin{itembox}[l]{留意点}
				\begin{description}
					\item[※] $\varphi(x)$と書いたら,$\varphi$には変項$x$が自由に現れていて,
						また$x$の他に自由な変項は無い.
						
					\item[※] $\varphi(u)$など$x$を他の項で置換する際は,
						項は束縛による障害を受けないように選ばれている.
				\end{description}
			\end{itembox}
			
\newpage
		\item $\exists$とは?
			
		\item $\exists$に形式的な意味を付ける方法として
			\textcolor{red}{Hilbertの$\varepsilon$項}がある:
			
			式$\varphi(x)$に対して
			\begin{align}
				\varepsilon x \varphi(x).
			\end{align}
		
		\item これが持つ意味は
			\begin{align}
				\exists x \varphi(x) \Longleftrightarrow \varphi\left(\varepsilon x \varphi(x)\right)
			\end{align}
			を満たすモノ.
			
		\item 島内では$\varepsilon$項,ブルバキでは$\tau$項.
			
		\item しかし式$\varphi(x)$に対して$\varepsilon x \varphi(x)$なるオブジェクトを項とすると
			\textcolor{red}{項と式の定義が入れ子になってしまう}.
			
\newpage
		\item $\varepsilon$項を活用しつつ入れ子の問題を解消し,
			またクラスの自然な導入により具体的で直観的な集合論を構築.
		
		\item この言語の拡張がZFCの単純な保存拡大ではないのでZFCと厳密にどう関係しているかは未だ不明
			(ZFCで示せることは示せるはず.逆に本稿の集合論で示せることがZFCから示せるかは不明).
	\end{itemize}
	
\section{言語}
	\begin{itemize}
		\item 本稿で使う言語は,論理学的に書けば
			\begin{align}
				\mathcal{L}_{\in} = \{\in,\natural\}
			\end{align}
			及びその拡張言語$\mathcal{L}$.
			
		\item $\natural$とは何か?通常は$\mathcal{L}_{\in} = \{\in\}$.
		
		\item そもそも述語論理では可算個の{\bf 変項}{(variable)}として
			\begin{align}
				v_{0},\ v_{1},\ v_{2},\ \cdots
			\end{align}
			を用意していたりする.集合論の解説書も同様の記号列を変項としている...
			
\newpage
		\item でも実際の式に$v_{0},v_{1},v_{2},\cdots$なんて現れず,通常は文字
			\begin{align}
				a,b,c,\cdots, \quad x,y,z,\cdots, \quad \alpha,\beta,\gamma,\cdots.
			\end{align}
		
		\item \textcolor{red}{文字は項である}と約束する.
			
		\item ただし文字だけだと足りないので,
			
			\begin{itembox}[l]{項の生成規則}
				$\tau$と$\sigma$を項とするとき,
				\begin{align}
					\natural \tau \sigma
				\end{align}
				も項である(ポーランド記法).
			\end{itembox}
	
\newpage
		\item $\natural$を使うことの利点:
			\begin{itemize}
				\item 添え字の数字や「可算個」という言葉を用いることなく,
					\textcolor{red}{実質的に可算無限個の変項を用意できる}.
					\begin{align}
						\natural xx,\ \natural \natural xxx,\ \natural \natural \natural xxxx,\ 
						\natural \natural \natural \natural xxxxx,\ \cdots
					\end{align}
					のように,$\natural$と$x$だけで十分(極端). 
			\end{itemize}
			
			数字や可算の概念は「集合論の中で定義されるもの」と
			「感覚として持っているもの」の二つがあるが,
			字面では同じなのであまり使いたくない.
	\end{itemize}

\newpage
	\begin{itembox}[l]{$\mathcal{L}_{\in}$の項と式の定義}
		\begin{description}
			\item[項] 
				\begin{itemize}
					\item 文字は項である.
					\item 項$\tau$と項$\sigma$に対して
						$\natural \tau \sigma$は項である.
					\item これらのみが項である.
				\end{itemize}
			
			\item[式] 
				\begin{itemize}
					\item 項$\tau$と項$\sigma$に対して
						$\in \tau \sigma$と$= \tau \sigma$は式である.
					\item 式$\varphi$に対して$\rightharpoondown \varphi$は式である.
					\item 式$\varphi$と式$\psi$に対して$\vee \varphi \psi$と
						$\wedge \varphi \psi$と$\Longrightarrow \varphi \psi$
						はいずれも式である.
					\item 式$\varphi$と項$x$に対して$\exists x \varphi$と
						$\forall x \varphi$は式である.
					\item これらのみが式である.
				\end{itemize}
		\end{description}
	\end{itembox}
	
\section{言語の拡張}
	\begin{itembox}[l]{拡張の動機}
		$\mathcal{L}_{\in}$の式$\varphi(x)$に対して
		\begin{align}
			\Set{x}{\varphi(x)}
		\end{align}
		や
		\begin{align}
			\varepsilon x \varphi(x)
		\end{align}
		の形のオブジェクトを\textcolor{red}{項として}式に組み込みたい.
	\end{itembox}
	
\newpage
	$\Set{x}{\varphi(x)}$の形の項を\textcolor{red}{内包項},
	$\varepsilon x \varphi(x)$の形の項を\textcolor{red}{$\varepsilon$項}と呼ぶことにして,
	$\mathcal{L}_{\in}$に内包項と$\varepsilon$項を追加した言語を
	\begin{align}
		\mathcal{L}
	\end{align}
	と名付ける.また$\mathcal{L}_{\in}$の項は\textcolor{red}{($\mathcal{L}$の)変項}と呼ぶ.
	
	
	\begin{itembox}[l]{$\mathcal{L}$の項と式の定義}
		\begin{description}
			\item[項] 内包項と$\varepsilon$項と変項は項である.これらのみが項である.
			\item[式] 式の生成規則は$\mathcal{L}_{\in}$と殆ど同じであるが,
				\begin{itemize}
					\item 式$\varphi$と\textcolor{red}{変項}$x$に対して
						$\exists x \varphi$と$\forall x \varphi$は式である.
				\end{itemize}
				の箇所のみ変える.
		\end{description}
	\end{itembox}
\section{導入}
	\begin{itemize}
		\item 集合論の言語$\mathcal{L}_{\in} = \{\in\}$の自然な拡張によりクラスを導入することは容易い:
			\begin{align}
				\Set{x}{\varphi(x)} \quad (\mbox{$\varphi$は$\mathcal{L}_{\in}$の式})
			\end{align}
			なるオブジェクトを取り入れればよい.
		
		\item 言語を拡張する利点は,$\mathcal{L}_{\in}$においては無定義概念であった集合が
			\begin{align}
				\exists x\, (\, x = a\, )
			\end{align}
			を満たすクラス$a$のことであると\textcolor{red}{定義できる}こと.

\newpage
		\item $\exists$とはどういう意味を持つか?
			
		\item $\exists$に形式的な意味を付ける方法としてHilbertの$\varepsilon$項がある:
			式$\varphi(x)$に対して
			\begin{align}
				\varepsilon x \varphi(x).
			\end{align}
		
		\item 存在意義は
			\begin{align}
				\exists x \varphi(x) \Longleftrightarrow \varphi\left(\varepsilon x \varphi(x)\right)
			\end{align}
			を満たすモノ.論理学では\textcolor{red}{証人}と呼ばれる.
			
		\item たとえば,島内の$\varepsilon$項,ブルバキの$\tau$項.
			
		\item しかし式$\varphi(x)$に対して$\varepsilon x \varphi(x)$なるオブジェクトを項とすると
			\textcolor{red}{項と式の定義が入れ子になってしまう}.
			
\newpage
		\item 言語の適切な拡張により入れ子の問題を解決しつつ,$\varepsilon$項の良さを活かし,
			またクラスの導入により直観的な集合論を構築した.
	\end{itemize}
	
\section{言語}
	\begin{itemize}
		\item 本稿で紹介する言語は,論理学的に書けば
			\begin{align}
				\mathcal{L}_{\in} = \{\in,\natural\}
			\end{align}
			及びその拡張言語$\mathcal{L}$.
			
		\item $\natural$とは何か?通常は集合論の言語は$\mathcal{L}_{\in} = \{\in\}$.
		
		\item 導入の意図の前に,そもそも述語論理では可算個の{\bf 変項}{\bf (variable)}として
			\begin{align}
				v_{0},\ v_{1},\ v_{2},\ \cdots
			\end{align}
			を用意していたりする.集合論の解説書も同様の記号列を変項としている...
			
\newpage
		\item でも実際の式に$v_{0},v_{1},v_{2},\cdots$なんて現れず,通常は文字(アルファベット)が使われる.
		
		\item だったら始めから文字を変項とすれば良い.
		
		\item ということで,本稿では\textcolor{red}{文字は変項である}と約束する.
			
		\item ただし変項が文字だけだと足りないので,
			\begin{center}
				\begin{quote}
					\textcolor{red}{$\tau$と$\sigma$を変項とするとき,
					\begin{align}
						\natural \tau \sigma
					\end{align}
					も変項である(ポーランド記法)}
				\end{quote}
			\end{center}
			とも約束しておく.
	
\newpage
		\item $\natural$を使うことの利点:
			\begin{itemize}
				\item 文字そのものを変項としているので\textcolor{red}{自然}.
				\item 添え字に数字や``可算個''という言葉を用いることなく,\textcolor{red}{実質的に可算個の変項を用意できる}.
					\begin{align}
						\natural xx,\ \natural \natural xxx,\ \natural \natural \natural xxxx,\ 
						\natural \natural \natural \natural xxxxx,\ \cdots
					\end{align}
					のように,$\natural$と$x$だけで何個でも変項を作れる.
				\item $\uparrow$ 数字や``可算''の概念は集合論の中で定義されるものと現実に我々が感覚として持っているものの二つがあるが,
					字面では同じなのであまり使いたくない.
			\end{itemize}
	\end{itemize}
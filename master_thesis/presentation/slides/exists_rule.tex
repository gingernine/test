\section{$\exists$の規則}
	\begin{screen}
		\begin{logicalaxm}[$\exists$の導入]
			$\mathcal{L}$の式$\varphi(x)$と$\varepsilon$項$\tau$に対して
			\begin{align}
				\varphi(\tau) \vdash \exists x \varphi(x).
			\end{align}
		\end{logicalaxm}
	\end{screen}
	
	とくに,任意の$\varepsilon$項$\tau$に対して
	\begin{align}
		\tau = \tau
	\end{align}
	だから
	\begin{align}
		\exists x\, (\, x = \tau\, )
	\end{align}
	が成り立つ.つまり\textcolor{red}{$\varepsilon$項はすべて集合}.
	
\newpage
	\begin{screen}
		\begin{logicalaxm}[$\exists$の除去(NG版)]
			$\mathcal{L}$の式$\varphi(x)$に対して
			\begin{align}
				\exists x \varphi(x) \vdash \varphi(\varepsilon x \varphi(x)).
			\end{align}
		\end{logicalaxm}
	\end{screen}
	
	$\varphi$に内包項や$\varepsilon$項が現れる場合
	\begin{align}
		\varepsilon x \varphi(x)
	\end{align}
	なる項は無い(無理矢理つくると入れ子問題).
	
	
	\begin{itembox}[l]{解決法}
		$\mathcal{L}$の式を$\mathcal{L}_{\in}$の式に書き換える手順を用意する.
	\end{itembox}
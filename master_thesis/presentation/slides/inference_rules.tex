\section{推論の公理}
\subsection{$\exists$の導入}
	$\lang{\varepsilon}$の式$\varphi$に$x$のみが自由に現れているとき,
	$\varepsilon x \varphi$を\textcolor{red}{主要$\varepsilon$項}と呼ぶ.
	
	\begin{screen}
		\begin{logicalaxm}[$\exists$の導入]
			$\varphi$を$\mathcal{L}$の式とし,$x$を変項とし,
			$\varphi$には$x$のみが自由に現れているとし,
			$\tau$を主要$\varepsilon$項とするとき,
			\begin{align}
				\varphi(\tau) \rarrow \exists x \varphi(x).
			\end{align}
		\end{logicalaxm}
	\end{screen}
	
	とくに,任意の$\varepsilon$項$\tau$に対して
	\begin{align}
		\tau = \tau.
	\end{align}
	だから
	\begin{align}
		\exists x\, (\, x = \tau\, )
	\end{align}
	が成り立つ.つまり\textcolor{red}{$\varepsilon$項はすべて集合}.
	
\newpage
\subsection{$\exists$の除去}
	\begin{screen}
		\begin{logicalaxm}[$\exists$の除去(NG版)]
			$\varphi$を$\mathcal{L}$の式とし,$x$を変項とし,
			$\varphi$には$x$のみが自由に現れているとするとき,
			\begin{align}
				\exists x \varphi(x) \vdash \varphi(\varepsilon x \varphi(x)).
			\end{align}
		\end{logicalaxm}
	\end{screen}
	
	$\varphi$が$\lang{\varepsilon}$の式でない場合
	\begin{align}
		\varepsilon x \varphi(x)
	\end{align}
	なる項は無い.
	
	\begin{itembox}[l]{解決法}
		$\mathcal{L}$の式を$\lang{\varepsilon}$の式に書き換える手順を用意する.
	\end{itembox}

\newpage
\subsection{式の書き換え}
	$\mathcal{L}$の式はすべて$\lang{\varepsilon}$の式に書き換え可能(構造的帰納法による).
	
	\begin{table}[H]
		\begin{center}
		\begin{tabular}{c|c}
			元の式 & 書き換え後 \\ \hline \hline
			$a = \Set{z}{\psi}$ & $\forall v\, (\, v \in a \lrarrow \psi(z/v)\, )$ \\ \hline
			$\Set{y}{\varphi} = b$ & $\forall u\, (\, \varphi(y/u) \lrarrow u \in b\, )$ \\ \hline
			$\Set{y}{\varphi} = \Set{z}{\psi}$ & $\forall u\, (\, \varphi(y/u) \lrarrow \psi(z/u)\, )$ \\ \hline
			$a \in \Set{z}{\psi}$ & $\psi(z/a)$ \\ \hline
			$\Set{y}{\varphi} \in b$ & $\exists s\, (\, \forall u\, (\, \varphi(y/u) \lrarrow u \in s\, ) \wedge s \in b\, )$ \\ \hline
			$\Set{y}{\varphi} \in \Set{z}{\psi}$ & $\exists s\, (\, \forall u\, (\, \varphi(y/u) \lrarrow u \in s\, ) \wedge \psi(z/s)\, )$ \\ \hline
		\end{tabular}
		\end{center}
	\end{table}
	
	ただし上の記号に課している条件は
	\begin{itemize}
		\item $a,b$は$\lang{\varepsilon}$の項である.
		\item $\varphi,\psi$は$\lang{\varepsilon}$の式である.
		\item $\varphi$には$y$が自由に現れ,$\psi$には$z$が自由に現れている.
		\item $u$は$\varphi$の中で$y$への代入について自由であり,
			$v$は$\psi$の中で$z$への代入について自由である.
	\end{itemize}
	
\newpage
	$\mathcal{L}$の式$\varphi$を$\lang{\varepsilon}$の式に書き換えたものを$\hat{\varphi}$と書く.
	
	\begin{screen}
		\begin{logicalaxm}[$\exists$の除去]
			$\varphi$を$\mathcal{L}$の式とし,$x$を変項とし,
			$\varphi$には$x$のみが自由に現れているとするとき,
			\begin{align}
				\exists x \varphi(x) \rarrow 
				\varphi\left(\varepsilon x \hat{\varphi}(x)\right).
			\end{align}
		\end{logicalaxm}
	\end{screen}
	
	\begin{screen}
		\begin{thm}[集合は主要$\varepsilon$項に等しい]
			$\varphi$を$\mathcal{L}$の式とし,$x$を変項とし,
			$\varphi$には$x$のみが自由に現れているとするとき,
			\begin{align}
				\exists s\, (\, \Set{x}{\varphi(x)} = s\, )
				\vdash \Set{x}{\varphi(x)} = 
				\varepsilon s\, \forall x\, (\, \varphi(x) \lrarrow x \in s\, ).
			\end{align}
		\end{thm}
	\end{screen}
	
	\begin{sketch}
		$\exists s\, \left(\, \Set{x}{\varphi(x)} = s\, \right)$を
		$\mathcal{L}_{\in}$の式に書き直せば
		\begin{align}
			\exists s\, \forall x\, (\, \varphi(x) \lrarrow x \in s\, ).
		\end{align}
		存在記号の規則より結論が従う.
		\QED
	\end{sketch}
	
	\begin{comment}
	\begin{screen}
		\begin{thm}
			$\varphi$を$\mathcal{L}$の式とし,$x$を変項とし,
			$\varphi$には$x$のみが自由に現れているとするとき,
			\begin{align}
				\exists x \varphi(x) \lrarrow \varphi(\varepsilon x \hat{\varphi}(x)).
			\end{align}
		\end{thm}
	\end{screen}
	\end{comment}

\newpage
\subsection{$\forall$の導入}
	\begin{screen}
		\begin{logicalaxm}[$\forall$の導入]
			$\varphi$を$\mathcal{L}$の式とし,$x$を変項とし,
			$\varphi$には$x$のみが自由に現れているとするとき,
			\begin{align}
				\varphi(\varepsilon x \negation \hat{\varphi}(x))
				\rarrow \forall x \varphi(x).
			\end{align}
		\end{logicalaxm}
	\end{screen}
	
	\begin{screen}
		\begin{logicalaxm}[$\forall$の除去]
			$\varphi$を$\mathcal{L}$の式とし,$x$を変項とし,
			$\varphi$には$x$のみが自由に現れているとし,
			$\tau$を主要$\varepsilon$項とするとき,
			\begin{align}
				\forall x \varphi(x) \rarrow \varphi(\tau).
			\end{align}
		\end{logicalaxm}
	\end{screen}
	
	主要$\varepsilon$項は集合であるから,\textcolor{red}{量化の亘る範囲は集合の上だけ}.
	
	\begin{comment}
	\begin{screen}
		\begin{thm}
			甲種式$\varphi(x)$に対して
			\begin{align}
				\forall x \varphi(x) \Longleftrightarrow
				\varphi\left(\varepsilon x \rightharpoondown \hat{\varphi}(x)\right).
			\end{align}
		\end{thm}
	\end{screen}
	\end{comment}

\newpage
\subsection{その他の公理}
	\begin{screen}
		\begin{logicalaxm}
			$\varphi,\psi,\chi$を$\mathcal{L}$の文とするとき,
			\begin{itemize}
				\item $(\, \varphi \rarrow (\, \psi \rarrow \chi\, )\, ) 
					\rarrow (\, (\, \varphi \rarrow \psi\, )
					\rarrow (\, \varphi \rarrow \chi\, )\, ).$
				\item $\varphi \rarrow (\, \psi \rarrow \varphi\, ).$
				\item $\varphi \rarrow (\, \negation \varphi \rarrow \bot\, ).$
				\item $\negation \varphi \rarrow (\, \varphi \rarrow \bot\, ).$
				\item $(\, \varphi \rarrow \bot\, ) \rarrow\ \negation \varphi.$
				\item $\varphi \rarrow (\, \varphi \vee \psi\, ).$
				\item $\psi \rarrow (\, \varphi \vee \psi\, ).$
				\item $(\, \varphi \rarrow \chi\, ) \rarrow 
					(\, (\, \psi \rarrow \chi\, ) 
					\rarrow (\, (\, \varphi \vee \psi) \rarrow \chi\, )\, ).$
				\item $\varphi \rarrow (\, \psi \rarrow (\, \varphi \wedge \psi\, )\, ).$
				\item $(\, \varphi \wedge \psi\, ) \rarrow \varphi.$
				\item $(\, \varphi \wedge \psi\, ) \rarrow \psi.$
				\item $\negation \negation \varphi \rarrow \varphi$.
			\end{itemize}
		\end{logicalaxm}
	\end{screen}
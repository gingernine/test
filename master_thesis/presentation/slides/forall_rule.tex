\section{$\forall$の導入}
	\begin{screen}
		\begin{logicalaxm}[$\forall$の導入]
			式$\varphi(x)$に対し,すべての$\varepsilon$項$\tau$で
			$\varphi(\tau)$が成り立つなら
			\begin{align}
				\forall x \varphi(x).
			\end{align}
		\end{logicalaxm}
	\end{screen}
	
	\begin{screen}
		\begin{logicalaxm}[$\forall$の除去]
			式$\varphi(x)$と$\varepsilon$項$\tau$に対して
			\begin{align}
				\forall x \varphi(x) \vdash \varphi(\tau).
			\end{align}
		\end{logicalaxm}
	\end{screen}
	
	$\varepsilon$項は集合であるから,\textcolor{red}{量化の亘る範囲は集合の上だけ}.
	
\newpage
	\begin{screen}
		\begin{thm}
			甲種式$\varphi(x)$に対して
			\begin{align}
				\forall x \varphi(x) \Longleftrightarrow
				\varphi\left(\varepsilon x \rightharpoondown \hat{\varphi}(x)\right).
			\end{align}
		\end{thm}
	\end{screen}
	
	次の定理は他の公理および構造的帰納法と併せて示される.
	
	\begin{screen}
		\begin{thm}[書き換えの同値性]
			甲種式$\varphi(x)$に対して
			\begin{align}
				\forall x\,
				\left(\, \varphi(x) \Longleftrightarrow \hat{\varphi}(x)\, \right).
			\end{align}
		\end{thm}
	\end{screen}
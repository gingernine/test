\section{$\exists$の除去規則}
	甲種式$\varphi$を$\mathcal{L}_{\in}$の式に書き換えたものを$\hat{\varphi}$と書く.
	
	\begin{screen}
		\begin{logicalaxm}[$\exists$の除去]
			甲種式$\varphi(x)$に対して
			\begin{align}
				\exists x \varphi(x) \vdash 
				\varphi\left(\varepsilon x \hat{\varphi}(x)\right).
			\end{align}
		\end{logicalaxm}
	\end{screen}

	\begin{screen}
		\begin{thm}
			甲種式$\varphi(x)$に対して
			\begin{align}
				\exists x \varphi(x) \Longleftrightarrow
				\varphi\left(\varepsilon x \hat{\varphi}(x)\right).
			\end{align}
		\end{thm}
	\end{screen}
	
	\begin{sketch}
		$\Longrightarrow$は$\exists$の除去規則,
		$\Longleftarrow$は$\exists$の導入規則.
		\QED
	\end{sketch}
\section{導入}
\subsection{$\varepsilon$計算について}
	\begin{itemize}
		\item 量化$\exists,\forall$を使う証明を命題論理の証明に埋め込むためにHilbertが開始.
		
		\vspace{5pt}
		
		\item 式$\varphi(x)$に対して
			\begin{align}
				\varepsilon x \varphi(x)
			\end{align}
			という形のオブジェクトを作り,$\varepsilon$項と呼ぶ.また
			\begin{align}
				\exists x \varphi(x) &\lrarrow \varphi(x/\varepsilon x \varphi(x)), \\
				\forall x \varphi(x) &\lrarrow \varphi(x/\varepsilon \negation x \varphi(x))
			\end{align}
			を公理とする.
			
		\item 命題論理の証明に埋め込む際には$\exists$や$\forall$の付いた式を$\varepsilon$項を
			代入した式に変換すればよい.
			
		\item ただし,今回$\varepsilon$項を導入したのは埋め込むためではなく
			\textcolor{red}{集合を「具体化」}するため.
	
\newpage
		\item ``生の''集合論では\textcolor{red}{集合というオブジェクトが用意されていない}ため,
			「存在」は「実在」ではない.たとえば
			\begin{align}
				\exists x\, (\, x = x\, )
			\end{align}
			は公理であり「集合は存在する」と読むが,集合を``実際に取ってくる''ことはできない.
			
		\item $\varepsilon$項を使えば,$\exists$の公理と集合の存在公理によって
			\begin{align}
				\varepsilon x\, (\, x = x\, ) = \varepsilon x\, (\, x = x\, )
			\end{align}
			が成り立つ.つまり$\varepsilon$項は「存在」を「実在」に変える
			(ある種の$\varepsilon$項は集合である).
	\end{itemize}
	
	\begin{itembox}[l]{$\varepsilon$項のメリット}
		\begin{itemize}
			\item 「存在」と「実在」が同じになる.
			\item ある種の$\varepsilon$項は集合であり,集合を具体的なオブジェクトとして扱える.
			\item 証明で用いる推論規則は三段論法のみで済む.
			\item 証明は全て閉じた式で行える.
		\end{itemize}
	\end{itembox}
	
\newpage
\subsection{クラスについて}
	\begin{itemize}
		\item ブルバキ\cite{}や島内\cite{}でも$\varepsilon$項を使った集合論を展開.
		
		\item ところで,「$\varphi(x)$を満たす集合$x$の全体」の意味の
			\begin{align}
				\Set{x}{\varphi(x)}
			\end{align}
			というオブジェクトも取り入れたい.
		
		\item ``生の''集合論では``インフォーマル''な導入.
		
		\item ブルバキ\cite{}や島内\cite{}では
			\begin{align}
				\Set{x}{\varphi(x)} \defeq \varepsilon x\, \forall u\, 
				(\, \varphi(u) \lrarrow u \in x\, )
			\end{align}
			と定める.これは欠点がある.
			\begin{align}
				\exists x\, \forall u\, (\, \varphi(u) \lrarrow u \in x\, )
			\end{align}
			が成立しない場合は「$\varphi(x)$を満たす集合$x$の全体」という意味を持たない.
			
		\item 式$\varphi$から直接$\Set{x}{\varphi(x)}$の形のオブジェクトを作ればよい.
	\end{itemize}
	
\newpage
	\begin{screen}
		\begin{dfn}[クラス]
			式$\varphi$に$x$が自由に現れていて,かつ自由に現れているのは$x$のみであるとき,
			\begin{align}
				\varepsilon x \varphi(x), \quad \Set{x}{\varphi(x)}
			\end{align}
			の形のオブジェクトを{\bf クラス(class)}と呼ぶ.
		\end{dfn}
	\end{screen}
	
	\begin{itemize}
		\item 集合はクラスである.
		\item クラスである$\varepsilon$項は集合である.
		\item 集合でないクラスもある.たとえば$\Set{x}{x = x}$や$\Set{x}{x \notin x}$
			は集合ではない.
	\end{itemize}
	
	集合の定義は竹内\cite{}に倣う.
	\begin{screen}
		\begin{dfn}[集合]
			\begin{align}
				\exists x\, (\, c = x\, )
			\end{align}
			を満たすクラス$c$を{\bf 集合(set)}と呼ぶ.
		\end{dfn}
	\end{screen}
	
	\begin{description}
		\item[NBG集合論] クラスの概念を取り入れたNBG集合論というものがあるが,
			こちらのクラスは「実在」しない.
	\end{description}
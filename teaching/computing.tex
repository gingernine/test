\documentclass[a4j,10.5pt,oneside,openany]{jsbook}
%
\usepackage{amsmath,amssymb}
\usepackage{amsthm}
\usepackage{makeidx}
\usepackage{txfonts}
\usepackage{mathrsfs} %花文字
\usepackage{mathtools} %参照式のみ式番号表示
\usepackage{latexsym} %qed
\usepackage{ascmac}
\usepackage{color}
\usepackage{relsize}
\usepackage{comment}
\usepackage{url}
\newtheoremstyle{mystyle}% % Name
	{20pt}%                      % Space above
	{20pt}%                      % Space below
	{\rm}%           % Body font
	{}%                      % Indent amount
	{\gt}%             % Theorem head font
	{.}%                      % Punctuation after theorem head
	{10pt}%                     % Space after theorem head, ' ', or \newline
	{}%                      % Theorem head spec (can be left empty, meaning `normal')
\theoremstyle{mystyle}

\allowdisplaybreaks[1]
\newcommand{\bhline}[1]{\noalign {\hrule height #1}} %表の罫線を太くする.
\newcommand{\bvline}[1]{\vrule width #1} %表の罫線を太くする.
\newtheorem{Prop}{$Proposition.$}
\newtheorem{Proof}{$Proof.$}
\newcommand{\QED}{% %証明終了
	\relax\ifmmode
		\eqno{%
		\setlength{\fboxsep}{2pt}\setlength{\fboxrule}{0.3pt}
		\fcolorbox{black}{black}{\rule[2pt]{0pt}{1ex}}}
	\else
		\begingroup
		\setlength{\fboxsep}{2pt}\setlength{\fboxrule}{0.3pt}
		\hfill\fcolorbox{black}{black}{\rule[2pt]{0pt}{1ex}}
		\endgroup
	\fi}
\newtheorem{thm}{定理}[section]
\newtheorem{dfn}[thm]{定義}
\newtheorem{prp}[thm]{命題}
\newtheorem{cor}[thm]{系}
\newtheorem{lem}[thm]{補題}
\newtheorem*{prf}{解}
\newtheorem{rem}[thm]{注意}
\newtheorem{e.g.}[thm]{例}
%
\makeindex
%
\setlength{\textwidth}{\fullwidth}
\setlength{\textheight}{40\baselineskip}
\addtolength{\textheight}{\topskip}
%\setlength{\voffset}{-0.55in}
%
%
\title{計算数理A 試験問題案}
\author{大阪大学大学院基礎工学研究科システム創成専攻 \\ 百合川尚学}
\date{\today}

\begin{document}
%
%

\mathtoolsset{showonlyrefs = true}
\maketitle
%
%
%
\mainmatter
%
%本文
\begin{screen}
	\begin{description}
		\item[(1)] 次の定積分の値を手計算によって導け。
			\begin{align}
				\int_0^{2\pi} \frac{1}{1-2 a \operatorname{cos}\theta + a^2}\ d\theta, \quad (0 < a < 1).
			\end{align}
		\item[(2)] (1)の結果が正しいかどうか,Mathematicaを用いて$a=1/2,1/3,1/4$の場合で検証せよ.
	\end{description}
\end{screen}
	
\begin{prf}
	(1)について,$z = e^{i\theta}$とすれば,この積分は複素平面の単位円周$|z|=1$上の左回り線積分と書き直せる:
	\begin{align}
		\int_0^{2\pi} \frac{1}{1-2 a \operatorname{cos}\theta + a^2}\ d\theta
		= i \oint_{|z|=1} \frac{1}{az^2- (a^2+1) z + a}\ dz.
	\end{align}
	方程式$az^2- (a^2+1) z + a = 0$は二つの実数解$\alpha,\beta\ (\alpha < \beta)$をもち,
	$0 < \alpha < 1 <\beta$であるから,
	\begin{align}
		f(z) \coloneqq \frac{1}{az^2- (a^2+1) z + a}= \frac{1}{a(z-\alpha)(z-\beta)}
	\end{align}
	とおけば$\alpha$は$f$の$|z| < 1$での一位の極になっていて,留数定理より
	\begin{align}
		\oint_{|z|=1} f(z)\ dz = 2\pi i Res(\alpha,f)
		= 2\pi i \lim_{z \to \alpha} \frac{1}{a(z-\beta)}
		= \frac{2\pi i}{a(\alpha - \beta)} 
	\end{align}
	が従う.
	\begin{align}
		\alpha + \beta = \frac{a^2+1}{a},
		\quad \alpha \beta = 1
	\end{align}
	より
	\begin{align}
		\alpha - \beta = -\frac{1-a^2}{a}
	\end{align}
	となるから,
	\begin{align}
		\int_0^{2\pi} \frac{1}{1-2 a \operatorname{cos}\theta + a^2}\ d\theta = \frac{2\pi}{1-a^2} 
	\end{align}
	が出る.
	\QED
\end{prf}
\newpage
\printindex
%
%
\end{document}
\documentclass[11pt,a4paper]{jsreport}
%
\usepackage{amsmath,amssymb}
\usepackage{amsthm}
\usepackage{makeidx}
\usepackage{txfonts}
\usepackage{mathrsfs} %花文字
\usepackage{mathtools} %参照式のみ式番号表示
\usepackage{latexsym} %qed
\usepackage{ascmac}
\usepackage{color}
\usepackage{comment}

\newtheoremstyle{mystyle}% % Name
	{20pt}%                      % Space above
	{20pt}%                      % Space below
	{\rm}%           % Body font
	{}%                      % Indent amount
	{\gt}%             % Theorem head font
	{.}%                      % Punctuation after theorem head
	{10pt}%                     % Space after theorem head, ' ', or \newline
	{}%                      % Theorem head spec (can be left empty, meaning `normal')
\theoremstyle{mystyle}

\allowdisplaybreaks[1]

\newcommand{\bhline}[1]{\noalign {\hrule height #1}} %表の罫線を太くする.
\newcommand{\bvline}[1]{\vrule width #1} %表の罫線を太くする.
\newtheorem{Prop}{$Proposition.$}
\newtheorem{Proof}{$Proof.$}
\newcommand{\QED}{% %証明終了
	\relax\ifmmode
		\eqno{%
		\setlength{\fboxsep}{2pt}\setlength{\fboxrule}{0.3pt}
		\fcolorbox{black}{black}{\rule[2pt]{0pt}{1ex}}}
	\else
		\begingroup
		\setlength{\fboxsep}{2pt}\setlength{\fboxrule}{0.3pt}
		\hfill\fcolorbox{black}{black}{\rule[2pt]{0pt}{1ex}}
		\endgroup
	\fi}
\newtheorem{thm}{定理}[section]
\newtheorem{dfn}[thm]{定義}
\newtheorem{prp}[thm]{命題}
\newtheorem{lem}[thm]{補題}
\newtheorem*{prf}{証明}
\newtheorem{qst}{レポート問題}
\newtheorem*{bcs}{なぜならば}
\newtheorem{rem}[thm]{注意}
\newcommand{\defunc}{\mbox{1}\hspace{-0.25em}\mbox{l}} %定義関数
\newcommand{\wlim}{\mbox{w-}\lim}
\def\Set#1#2{\left\{\ #1\ \, ; \quad #2\ \right\}} %集合の書き方
\def\Box#1{$(\mbox{#1})$} %丸括弧つきコメント
\def\Hat#1{$\hat{\mathrm{#1}}$} %文中ハット
\def\Ddot#1{$\ddot{\mathrm{#1}}$} %文中ddot
\def\DEF{\overset{\mathrm{def}}{\Leftrightarrow}} %定義記号
\def\eqqcolon{=\mathrel{\mathop:}} %定義=:
\def\max#1#2{\operatorname*{max}_{#1} #2 } %最大
\def\min#1#2{\operatorname*{min}_{#1} #2 } %最小
\def\sin#1#2{\operatorname{sin}^{#2} #1} %sin
\def\cos#1#2{\operatorname{cos}^{#2} #1} %cos
\def\tan#1#2{\operatorname{tan}^{#2} #1} %tan
\def\inprod<#1>{\left\langle #1 \right\rangle} %内積
\def\sup#1#2{\operatorname*{sup}_{#1} #2 } %上限
\def\inf#1#2{\operatorname*{inf}_{#1} #2 } %下限
\def\Vector#1{\mbox{\boldmath $#1$}} %ベクトルを太字表示
\def\Norm#1#2{\left\|\, #1\, \right\|_{#2}} %ノルム
\def\Log#1{\operatorname{log} #1} %log
\def\Det#1{\operatorname{det} ( #1 )} %行列式
\def\Diag#1{\operatorname{diag} \left( #1 \right)} %行列の対角成分
\def\Tmat#1{#1^\mathrm{T}} %転置行列
\def\Exp#1{\operatorname{E} \left[ #1 \right]} %期待値
\def\Var#1{\operatorname{V} \left[ #1 \right]} %分散
\def\Cov#1#2{\operatorname{Cov} \left[ #1,\ #2 \right]} %共分散
\def\exp#1{e^{#1}} %指数関数
\def\N{\mathbb{N}} %自然数全体
\def\Q{\mathbb{Q}} %有理数全体
\def\R{\mathbb{R}} %実数全体
\def\C{\mathbb{C}} %複素数全体
\def\K{\mathbb{K}} %係数体K
\def\conj#1{\overline{#1}} %共役複素数
\def\Re#1{\mathfrak{Re}#1} %実部
\def\Im#1{\mathfrak{Im}#1} %虚部
\def\borel#1{\mathfrak{B}(#1)} %Borel集合族
\def\open#1{\mathfrak{O}(#1)} %位相空間 #1 の位相
\def\close#1{\mathfrak{A}(#1)} %%位相空間 #1 の閉集合系
\def\closure#1{\left[#1\right]^a}
\def\rapid#1{\mathfrak{S}(#1)} %急減少空間
\def\c#1{C(#1)} %有界実連続関数
\def\cbound#1{C_{b} (#1)} %有界実連続関数
\def\semiLp#1#2{\mathscr{L}^{#1} \left(#2\right)} %ノルム空間L^p
\def\Lp#1#2{\operatorname{L}^{#1} \left(#2\right)} %ノルム空間L^p
\def\cinf#1{C^{\infty} (#1)} %無限回連続微分可能関数
\def\sgmalg#1{\sigma \left[#1\right]} %#1が生成するσ加法族
\def\ball#1#2{\operatorname{B} \left(#1\, ;\, #2 \right)} %開球
\def\prob#1{\operatorname{P} \left(#1\right)} %確率
\def\cprob#1#2{\operatorname{P} \left(\left\{ #1 \ \middle|\ #2 \right\}\right)} %条件付確率
\def\cexp#1#2{\operatorname{E} \left[ #1 \ \middle|\ #2 \right]} %条件付期待値
\def\tExp#1{\tilde{\operatorname{E}} \left[ #1 \right]} %拡張期待値
\def\tcexp#1#2{\tilde{\operatorname{E}} \left[ #1 \ \middle|\ #2 \right]} %拡張条件付期待値
%\renewcommand{\contentsname}{\bm Index}
%
\makeindex
%
\setlength{\textwidth}{\fullwidth}
\setlength{\textheight}{40\baselineskip}
\addtolength{\textheight}{\topskip}
\setlength{\voffset}{-0.2in}
\setlength{\topmargin}{0pt}
\setlength{\headheight}{0pt}
\setlength{\headsep}{0pt}
%
\title{確率微分方程式講義録}
\author{基礎工学研究科システム創成専攻修士1年\\学籍番号29C17095\\百合川尚学}
\date{\today}

\begin{document}
%
%

\mathtoolsset{showonlyrefs = true}
\maketitle

\newpage
\tableofcontents
%
\chapter{関数解析}
	\section{10/4講義ノート}
\begin{qst}
係数体を$\K$,$\K = \R$或は$\K = \C$と考える.測度空間を$(X,\mathcal{F},m)$とし,
可測$\mathcal{F}/\borel{\K}$関数$f$に対して
\begin{align}
	\Norm{f}{\mathscr{L}^p} \coloneqq
	\begin{cases}
		\inf{}{\{\ r \in \R\quad |\quad |f(x)| \leq r,\ \mathrm{a.e.}x \in X\ \}} & (p = \infty) \\
		\left( \int_{X} |f(x)|^p\ m(dx) \right)^{\frac{1}{p}} & (0 < p < \infty)
	\end{cases}
\end{align}
と定め,
\begin{align}
	\semiLp{p}{X,\mathcal{F},m} \coloneqq \{\ f:X \rightarrow \K \quad |\quad f:\mbox{可測}\mathcal{F}/\borel{\K},\ \Norm{f}{\mathscr{L}^p} < \infty \ \} \quad (1 \leq p \leq \infty)
\end{align}
として空間$\semiLp{p}{X,\mathcal{F},m}$を定義する.この空間は$\K$上の線形空間となるが,そのことを保証するために
次の二つの不等式が成り立つことを証明する.
\begin{thm}[H\Ddot{o}lderの不等式]
	$1 \leq p, q \leq \infty$,$p + q = pq\ (p = \infty$なら$q = 1)$とする.このとき
	任意の可測$\mathcal{F}/\borel{\K}$関数$f,g$に対して次が成り立つ:
	\begin{align}
		\int_{X} |f(x)g(x)|\ m(dx) \leq \Norm{f}{\mathscr{L}^p} \Norm{g}{\mathscr{L}^q}. \label{ineq:holder}
	\end{align} 
\end{thm}
\begin{prf}
	まず次の補助定理を証明する.
	\begin{lem}
		$f \in \semiLp{\infty}{X,\mathcal{F},m}$ならば
		\begin{align}
			|f(x)| \leq \Norm{f}{\mathscr{L}^\infty} \quad (\mathrm{a.e.}x \in X).
		\end{align}
	\end{lem}
	\begin{prf}
		$\semiLp{\infty}{X,\mathcal{F},m}$の定義により,任意の実数$\alpha > \Norm{f}{\mathscr{L}^\infty}$に対して
		\begin{align}
			m(\{\ x \in X\quad |\quad |f(x)| > \alpha\ \}) = 0
		\end{align}
		である.これにより
		\begin{align}
			\{\ x \in X\quad |\quad |f(x)| > \Norm{f}{\mathscr{L}^\infty}\ \} = \bigcup_{n =1}^{\infty} \{\ x \in X\quad |\quad |f(x)| > \Norm{f}{\mathscr{L}^\infty} + 1/n\ \}
		\end{align}
		の右辺は$m$-零集合となり補題が証明された.
		\QED
	\end{prf}
	
	定理の証明に入る.\mbox{}\\
	\begin{description}
		\item[$p = \infty,\ q = 1$の場合]
			$\Norm{f}{\mathscr{L}^\infty} = \infty$又は$\Norm{g}{\mathscr{L}^1} = \infty$の場合は明らかに不等式(\refeq{ineq:holder})
			が成り立つから,$\Norm{f}{\mathscr{L}^\infty} < \infty$かつ$\Norm{g}{\mathscr{L}^1} < \infty$の場合を考える.
			補助定理により,或る$m$-零集合$A \in \mathcal{F}$を除いて$|f(x)| \leq \Norm{f}{\mathscr{L}^\infty}$が成り立つから,
			\begin{align}
				|f(x)g(x)| \leq \Norm{f}{\mathscr{L}^\infty}|g(x)| \quad (\forall x \in X \backslash A).
			\end{align}
			従って
			\begin{align}
				\int_{X} |f(x)g(x)|\ m(dx) = \int_{X \backslash A} |f(x)g(x)|\ m(dx) \leq \Norm{f}{\mathscr{L}^\infty} \int_{X \backslash A} |g(x)|\ m(dx) 
				= \Norm{f}{\mathscr{L}^\infty} \Norm{g}{\mathscr{L}^1}
			\end{align}
			となり不等式(\refeq{ineq:holder})が成り立つ.
		
		\item[$1 < p,q < \infty$の場合]
			$\Norm{f}{\mathscr{L}^p} = \infty$又は$\Norm{g}{\mathscr{L}^q} = \infty$の場合は明らかに不等式(\refeq{ineq:holder})
			が成り立つから,$\Norm{f}{\mathscr{L}^p} < \infty$かつ$\Norm{g}{\mathscr{L}^q} < \infty$の場合を考える.
			$\Norm{f}{\mathscr{L}^p} = 0$であるとすると
			\begin{align}
				B \coloneqq \{\ x \in X\quad |\quad |f(x)| > 0\ \}
			\end{align}
			は$m$-零集合となるから,
			\begin{align}
				\int_{X} |f(x)g(x)|\ m(dx) = \int_{B} |f(x)g(x)|\ m(dx) + \int_{X \backslash B} |f(x)g(x)|\ m(dx) = 0
			\end{align}
			となり不等式(\refeq{ineq:holder})が成り立つ.$\Norm{g}{\mathscr{L}^q} = 0$の場合も同じである.
			
			最後に$0 < \Norm{f}{\mathscr{L}^p},\Norm{g}{\mathscr{L}^q} < \infty$の場合を示す.
			$-\Log{t} \quad (t > 0)$は凸関数であるから,$1/p + 1/q = 1$に対して
			\begin{align}
				-\Log{\left( \frac{s}{p} + \frac{t}{q} \right)} \leq \frac{1}{p}(-\Log{s}) + \frac{1}{q}(-\Log{t}) \quad (\forall s,t > 0)
			\end{align}
			が成り立ち,従って
			\begin{align}
				s^{1/p}t^{1/q} \leq \frac{s}{p} + \frac{t}{q} \quad (\forall s,t > 0)
			\end{align}
			が成り立つ.この不等式を用いれば
			\begin{align}
				F(x) \coloneqq |f(x)|^p/ \Norm{f}{\mathscr{L}^p}^p,\quad G(x) \coloneqq |g(x)|^q/ \Norm{g}{\mathscr{L}^q}^q \quad (\forall x \in X)
			\end{align}
			とした$F,G$に対し
			\begin{align}
				F(x)^{1/p}G(x)^{1/q} \leq \frac{1}{p}F(x) + \frac{1}{q}G(x) \quad (\forall x \in X)
			\end{align}
			となり,両辺を積分して
			\begin{align}
				\int_{X} F(x)^{1/p}G(x)^{1/q}\ m(dx) &\leq \frac{1}{p} \int_{X} F(x)\ m(dx) + \frac{1}{q} \int_{X} G(x)\ m(dx) \\
				&= \frac{1}{p} \frac{1}{\Norm{f}{\mathscr{L}^p}^p} \int_{X} |f(x)|^p\ m(dx) + \frac{1}{q} \frac{1}{\Norm{g}{\mathscr{L}^q}^q} \int_{X} |g(x)|^q\ m(dx) \\
				&= \frac{1}{p} + \frac{1}{q} = 1
			\end{align}
			が成り立つ.最左辺と最右辺を比べて
			\begin{align}
				1 \geq \int_{X} F(x)^{1/p}G(x)^{1/q}\ m(dx) = \int_{X} \frac{|f(x)|}{\Norm{f}{\mathscr{L}^p}} \frac{|g(x)|}{\Norm{g}{\mathscr{L}^q}}\ m(dx)
			\end{align}
			から不等式
			\begin{align}
				\int_{X} |f(x)g(x)|\ m(dx) \leq \Norm{f}{\mathscr{L}^p}\Norm{g}{\mathscr{L}^q}
			\end{align}
			が示された.
			\QED
	\end{description}
\end{prf}

\begin{thm}[Minkowskiの不等式]
	$1 \leq p \leq \infty$とする.このとき
	任意の可測$\mathcal{F}/\borel{\K}$関数$f,g$に対して次が成り立つ:
	\begin{align}
		\Norm{f+g}{\mathscr{L}^p} \leq \Norm{f}{\mathscr{L}^p} + \Norm{g}{\mathscr{L}^p}. \label{ineq:minkowski}
	\end{align}
\end{thm}
\begin{prf}
	\begin{description}\mbox{}\\
		\item[$p = \infty$の場合]
			\begin{align}
				|f(x) + g(x)| \leq |f(x)| + |g(x)| \quad (\forall x \in X)
			\end{align}
			である.従って$\Norm{f}{\mathscr{L}^\infty} = \infty$又は$\Norm{g}{\mathscr{L}^\infty} = \infty$の場合に不等式
			(\refeq{ineq:minkowski})が成り立つことは明らかである.$\Norm{f}{\mathscr{L}^\infty} < \infty$かつ$\Norm{g}{\mathscr{L}^\infty} < \infty$
			の場合は
			\begin{align}
				C \coloneqq \{\ x \in X\quad |\quad |f(x)| > \Norm{f}{\mathscr{L}^\infty}\ \} \bigcup \{\ x \in X\quad |\quad |g(x)| > \Norm{g}{\mathscr{L}^\infty}\ \}
			\end{align}
			が$m$-零集合となり,$\Norm{\cdot}{\mathscr{L}^\infty}$の定義と
			\begin{align}
				|f(x) + g(x)| \leq \Norm{f}{\mathscr{L}^\infty} + \Norm{g}{\mathscr{L}^\infty} \quad (\forall x \in X \backslash C)
			\end{align}
			の関係により不等式(\refeq{ineq:minkowski})が成り立つ.
		
		\item[$p = 1$の場合]
			\begin{align}
				|f(x) + g(x)| \leq |f(x)| + |g(x)| \quad (\forall x \in X)
			\end{align}
			の両辺を積分することにより不等式(\refeq{ineq:minkowski})が成り立つ.
		
		\item[$1 < p < \infty$の場合]
			$p + q = pq$が成り立つように$q > 1$を取る.
			\begin{align}
				|f(x) + g(x)|^p = |f(x) + g(x)||f(x) + g(x)|^{p-1} \leq |f(x)||f(x) + g(x)|^{p-1} + |g(x)||f(x) + g(x)|^{p-1}
			\end{align}
			の両辺を積分すれば,H\Ddot{o}lderの不等式により
			\begin{align}
				\Norm{f+g}{\mathscr{L}^p}^p &= \int_{X} |f(x) + g(x)|^p\ m(dx) \\
				&\leq \int_{X} |f(x)||f(x) + g(x)|^{p-1}\ m(dx) + \int_{X} |g(x)||f(x) + g(x)|^{p-1}\ m(dx) \\
				&\leq \left( \int_{X} |f(x)|^p\ m(dx) \right)^{1/p} \left( \int_{X} |f(x) + g(x)|^{q(p-1)}\ m(dx) \right)^{1/q} \\
					&\qquad + \left( \int_{X} |g(x)|^p\ m(dx) \right)^{1/p} \left( \int_{X} |f(x) + g(x)|^{q(p-1)}\ m(dx) \right)^{1/q} \\
				&= \left( \int_{X} |f(x)|^p\ m(dx) \right)^{1/p} \left( \int_{X} |f(x) + g(x)|^p\ m(dx) \right)^{1/q} \\
					&\qquad + \left( \int_{X} |g(x)|^p\ m(dx) \right)^{1/p} \left( \int_{X} |f(x) + g(x)|^p\ m(dx) \right)^{1/q} \\
				&= \Norm{f}{\mathscr{L}^p}\Norm{f+g}{\mathscr{L}^p}^{p/q} + \Norm{g}{\mathscr{L}^p}\Norm{f+g}{\mathscr{L}^p}^{p/q} \\
				&= \Norm{f}{\mathscr{L}^p}\Norm{f+g}{\mathscr{L}^p}^{p-1} + \Norm{g}{\mathscr{L}^p}\Norm{f+g}{\mathscr{L}^p}^{p-1}
			\end{align}
			が成り立つ.$\Norm{f+g}{\mathscr{L}^p} = 0$の場合は明らかに不等式(\refeq{ineq:minkowski})が成り立つ.
			$\Norm{f+g}{\mathscr{L}^p} = \infty$の場合,
			\begin{align}
				|f(x) + g(x)| \leq |f(x)| + |g(x)| \leq 2 \max{}{(|f(x)|,|g(x)|)} \quad (\forall x \in X)
			\end{align}
			より
			\begin{align}
				|f(x) + g(x)|^p \leq 2^p \max{}{\left( |f(x)|^p,|g(x)|^p \right)} \leq 2^p \left( |f(x)|^p + |g(x)|^p \right) \quad (\forall x \in X)
			\end{align}
			から両辺を積分して
			\begin{align}
				\Norm{f+g}{\mathscr{L}^p}^p \leq 2^p \left( \Norm{f}{\mathscr{L}^p}^p + \Norm{g}{\mathscr{L}^p}^p \right)
			\end{align}
			という関係が出るから,上式右辺も$\infty$となり不等式(\refeq{ineq:minkowski})が成り立つ.
			$0 < \Norm{f+g}{\mathscr{L}^p} < \infty$の場合,$\Norm{f}{\mathscr{L}^p} + \Norm{g}{\mathscr{L}^p} = \infty$
			なら不等式(\refeq{ineq:minkowski})は明らかに成り立ち,$\Norm{f}{\mathscr{L}^p} + \Norm{g}{\mathscr{L}^p} < \infty$
			の場合は
			\begin{align}
				\Norm{f+g}{\mathscr{L}^p}^p \leq \Norm{f}{\mathscr{L}^p}\Norm{f+g}{\mathscr{L}^p}^{p-1} + \Norm{g}{\mathscr{L}^p}\Norm{f+g}{\mathscr{L}^p}^{p-1}
			\end{align}
			の両辺を$\Norm{f+g}{\mathscr{L}^p}^{p-1}$で割って不等式(\refeq{ineq:minkowski})が成り立つと判る.
			\QED
	\end{description}
\end{prf}

以上の結果より$\semiLp{p}{X,\mathcal{F},m}$が線形空間をなすことが判る.加法について閉じていることはMinkowskiの不等式により従い,加法について可換群となり
スカラ倍について(閉じていてかつ)分配的であることは積分の性質から従うからである.以下にこの線形空間に関する重要な性質を載せる.

\begin{lem}[$\mathscr{L}^p$のセミノルムについて]
	$\Norm{\cdot}{\mathscr{L}^p}$は線形空間$\semiLp{p}{X,\mathcal{F},m}$においてセミノルムとなる.
\end{lem}
\begin{prf}
	\begin{description}
	\item[正値性] これは明らかである.
	\item[同次性] 
		\begin{align}
			\left( \int_{X} |\alpha f(x)|^p\ m(dx) \right)^{1/p} = \left( |\alpha|^p \int_{X} |f(x)|^p\ m(dx) \right)^{1/p} 
			= |\alpha| \left( \int_{X} |f(x)|^p\ m(dx) \right)^{1/p} \quad (1 \leq p < \infty)
		\end{align}
		と
		\begin{align}
			\inf{}{\{\ r \in \R\quad |\quad |\alpha f(x)| \leq r,\ \mathrm{a.e.}x \in X\ \}} = |\alpha|\inf{}{\{\ r \in \R\quad |\quad |f(x)| \leq r,\ \mathrm{a.e.}x \in X\ \}}
		\end{align}
		により,任意の$\alpha \in \K$と任意の$f \in \semiLp{p}{X,\mathcal{F},m}\ (1 \leq p \leq \infty)$に対して
		\begin{align}
			\Norm{\alpha f}{\mathscr{L}^p} = |\alpha|\Norm{f}{\mathscr{L}^p}
		\end{align}
		が成り立つ.
	\item[三角不等式] Minkowskiの不等式による.
	\end{description}
	\QED
\end{prf}

しかし$\Norm{\cdot}{\mathscr{L}^p}$は$\semiLp{p}{X,\mathcal{F},m}$のノルムとはならない.$\Norm{f}{\mathscr{L}^p} = 0$であっても
$f(x) = 0 \ (\forall x \in X)$とは限らず,$m$-零集合の上で
$1 \in \K$を取るような関数$g$でも$\Norm{g}{\mathscr{L}^p} = 0$を満たすからである.
ここで次のものを考える.可測関数の集合を
\begin{align}
	\mathcal{M} \coloneqq \{\ f:X \rightarrow \K\quad |\quad f:\mbox{可測}\mathcal{F}/\borel{\K} \}
\end{align}
と表すことにする.
\begin{align}
	f,g \in \mathcal{M},\quad f \sim g \DEF f(x) = g(x)\quad \mathrm{a.e.}x \in X
\end{align}
と定義した関係$\sim$は$\mathcal{M}$における同値関係となり,この関係で$\mathcal{M}$を割った商を$ M \coloneqq \mathcal{M}/\sim$と表す.
$M$の元を$[f]\ $($f$は同値類の代表元)と表し,$M$における加法とスカラ倍を次のように定義すれば$M$は$\K$上の線形空間となる:
\begin{align}
	&[f] + [g] \coloneqq [f+g] && (\forall [f],[g] \in M),\\
	&\alpha [f] \coloneqq [\alpha f] && (\forall [f] \in M,\ \alpha \in \K).
\end{align}
そしてこの表現はwell-definedである.つまり代表元に依らずに値がただ一つに定まる.

\begin{prf}\mbox{}\\
	任意の$f' \in [f]$と$g' \in [g]$に対して,$[f'] = [f],\ [g'] = [g]$
	であるから
	\begin{align}
		[f + g] = [f' + g'],\quad [\alpha f'] = [\alpha f]
	\end{align}
	をいえばよい.
	\begin{align}
		(f \neq g) \coloneqq \{\ x \in X\quad |\quad f(x) \neq g(x)\ \}
	\end{align}
	と簡略した表記を使えば
	\begin{align}
		&(f+g \neq f'+g') \subset (f \neq f') \cup (g \neq g'), \\
		&(\alpha f \neq \alpha f') = (f \neq f')
	\end{align}
	であり,どちらも右辺は$m$-零集合であるから$[f + g] = [f' + g'],\ [\alpha f'] = [\alpha f]$である.
	\QED
\end{prf}

次に商空間$M$におけるノルムを定義する.
\begin{lem}[商空間$\mathrm{L}^p$におけるノルムの定義]
	\begin{align}
		\Norm{[f]}{\mathrm{L}^p} \coloneqq \Norm{f}{\mathscr{L}^p} \quad (1 \leq p \leq \infty)
	\end{align}
	として$\Norm{\cdot}{\mathrm{L}^p}$を定義すればこれはwell-definedである.つまり代表元に依らずに値がただ一つに定まる.
\end{lem}
\begin{prf}
	$f \in \semiLp{p}{X,\mathcal{F},m}$とし,任意に$g \in [f]$で$f \neq g$となるものを選ぶ.
	示すことは$\Norm{f}{\mathscr{L}^p}^p = \Norm{g}{\mathscr{L}^p}^p$が成り立つことである.
	\begin{align}
		A \coloneqq \{\ x \in X\quad |\quad f(x) \neq g(x)\ \} \quad \in \mathcal{F}
	\end{align}
	とおけば,$f,g$は同じ同値類の元同士であるから$m(A)=0$である.
	\begin{description}
		\item[$p = \infty$の場合]
			$A^c$の上で$f(x)=g(x)$となるから
			\begin{align}
				\left\{\ x \in X\quad |\quad |g(x)| > \Norm{f}{\mathscr{L}^\infty}\ \right\} 
				&\subset A + A^c \cap \left\{\ x \in X\quad |\quad |g(x)| > \Norm{f}{\mathscr{L}^\infty}\ \right\} \\
				&= A + A^c \cap \left\{\ x \in X\quad |\quad |f(x)| > \Norm{f}{\mathscr{L}^\infty}\ \right\} \\
				&\subset A + \left\{\ x \in X\quad |\quad |f(x)| > \Norm{f}{\mathscr{L}^\infty}\ \right\}
			\end{align}
			が成り立ち,最右辺は2項とも$m$-零集合であるから最左辺も$m$-零集合となる.すなわち$\Norm{g}{\mathscr{L}^\infty} \leq \Norm{f}{\mathscr{L}^\infty}$が示された.
			逆向きの不等号も同様に示されるから$\Norm{g}{\mathscr{L}^\infty} = \Norm{f}{\mathscr{L}^\infty}$となる.
		\item[$1 \leq p < \infty$の場合]
			$m(A)=0$により
			\begin{align}
				\Norm{f}{\mathscr{L}^p}^p = \int_X f(x)\ m(dx) = \int_{A^c} f(x)\ m(dx) = \int_{A^c} g(x)\ m(dx) = \int_X g(x)\ m(dx) = \Norm{g}{\mathscr{L}^p}^p
			\end{align}
			が成り立つ.
	\end{description}
	\QED
\end{prf}

\begin{lem}[商空間$\mathrm{L}^p$はノルム空間となる]
	\begin{align}
		\Lp{p}{X,\mathcal{F},m} \coloneqq \{\ [f] \in M \quad |\quad \Norm{[f]}{\mathrm{L}^p} < \infty\ \} \quad (1 \leq p \leq \infty)
	\end{align}
	を定義すると,$\Lp{p}{X,\mathcal{F},m}$は$\Norm{\cdot}{\mathrm{L}^p}$をノルムとしてノルム空間となる.
\end{lem}
\begin{prf}
	任意の$[f],[g] \in \Lp{p}{X,\mathcal{F},m}$と$\alpha \in \K$に対して,
	$\Norm{[f]}{\mathrm{L}^p} \geq 0$であることは$\Norm{\cdot}{\mathscr{L}^p}$の正値性による.
	また関数が0でない$x \in X$の集合の測度が正となるとノルムは正となるから,
	$\Norm{[f]}{\mathrm{L}^p} = 0$であるなら$[f]$は零写像(これを0と表す)の同値類(線形空間の零元),
	つまり$[f] = [0]$である.逆に$[f] = [0]$なら$\Norm{[f]}{\mathrm{L}^p} = 0$である.
	$\Norm{\cdot}{\mathscr{L}^p}$の同次性とMinkowskiの不等式から
	\begin{align}
		&\Norm{\alpha[f]}{\mathrm{L}^p} = \Norm{[\alpha f]}{\mathrm{L}^p} = \Norm{\alpha f}{\mathscr{L}^p} = |\alpha|\Norm{f}{\mathscr{L}^p} = |\alpha|\Norm{[f]}{\mathrm{L}^p} \\
		&\Norm{[f] + [g]}{\mathrm{L}^p} = \Norm{[f + g]}{\mathrm{L}^p} = \Norm{f + g}{\mathscr{L}^p} \leq \Norm{f}{\mathscr{L}^p} + \Norm{g}{\mathscr{L}^p} = \Norm{[f]}{\mathrm{L}^p} + \Norm{[g]}{\mathrm{L}^p}
	\end{align}
	も成り立つ.以上より$\Norm{\cdot}{\mathrm{L}^p}$は$\Lp{p}{X,\mathcal{F},m}$におけるノルムとなる.
	\QED
\end{prf}

\begin{prp}[$\mathrm{L}^p$の完備性]
	上で定義したノルム空間$\Lp{p}{X,\mathcal{F},m}$はBanach空間である.$(1 \leq p \leq \infty)$
\end{prp}
\begin{prf}
	任意に$\Lp{p}{X,\mathcal{F},m}$のCauchy列$[f_n] \in \Lp{p}{X,\mathcal{F},m}\ (n=1,2,3,\cdots)$を取る.
	Cauchy列であるから$1/2$に対して或る$N_1 \in \N$が取れて,$n>m \geq N_1$ならば
	$\Norm{[f_n]-[f_m]}{\mathrm{L}^p} = \Norm{[f_n - f_m]}{\mathrm{L}^p} < 1/2$となる.
	ここで$m = n_1$と表記することにする.
	同様に$1/2^2$に対して或る$N_2 \in \N\ (N_2 > N_1)$が取れて,$n'>m' \geq N_2$ならば
	$\Norm{[f_{n'} - f_{m'}]}{\mathrm{L}^p} < 1/2^2$となる.
	先ほどの$n$について,$n > N_2$となるように取れるからこれを$n = n_2$と表記し,更に$m' = n_2$ともしておく.今のところ
	\begin{align}
		\Norm{[f_{n_1} - f_{n_2}]}{\mathrm{L}^p} < 1/2
	\end{align}
	と表示できる.再び同様に$1/2^3$に対して或る$N_3 \in \N\ (N_3 > N_2)$が取れて,$n''>m'' \geq N_2$ならば
	$\Norm{[f_{n''} - f_{m''}]}{\mathrm{L}^p} < 1/2^3$となる.
	先ほどの$n'$について$n' > N_3$となるように取れるからこれを$n' = n_3$と表記し,更に$m'' = n_3$ともしておく.今までのところで
	\begin{align}
		&\Norm{[f_{n_1} - f_{n_2}]}{\mathrm{L}^p} < 1/2 \\
		&\Norm{[f_{n_2} - f_{n_3}]}{\mathrm{L}^p} < 1/2^2
	\end{align}
	が成り立っている.数学的帰納法により
	\begin{align}
		\Norm{[f_{n_k} - f_{n_{k+1}}]}{\mathrm{L}^p} < 1/2^k \quad (n_{k+1} > n_k,\ k=1,2,3,\cdots) \label{ineq:Lp_banach_2}
	\end{align}
	が成り立つように自然数の部分列$(n_k)_{k=1}^{\infty}$を取ることができる.
	\begin{description}
		\item[$p = \infty$の場合]\mbox{}\\
			$[f_{n_k}]$の代表元$f_{n_k}$について,
			\begin{align}
				A_k &\coloneqq \left\{\ x \in X\quad |\quad |f_{n_k}(x)| > \Norm{f_{n_k}}{\mathscr{L}^\infty} \right\}, \\
				A^k &\coloneqq \left\{\ x \in X\quad |\quad |f_{n_k}(x) - f_{n_{k+1}}(x)| > \Norm{f_{n_k} - f_{n_{k+1}}}{\mathscr{L}^\infty} \right\}
			\end{align}
			とおけばH\Ddot{o}lderの不等式の証明中の補助定理より$m(A_k) = m(A^k) = 0$であり,
			\begin{align}
				A \coloneqq \left( \bigcup_{k=1}^{\infty} A_k \right) \bigcup \left( \bigcup_{k=1}^{\infty} A^k \right)
			\end{align}
			として$m$-零集合を定め
			\begin{align}
				\hat{f}_{n_k}(x) =
				\begin{cases}
					f_{n_k}(x) & (x \notin A) \\
					0 & (x \in A)
				\end{cases}
				\quad (\forall x \in X)
			\end{align}
			と定義した$\hat{f}_{n_k}$もまた$[f_{n_k}]$の元となる.代表元を$f_{n_k}$に替えて$\hat{f}_{n_k}$とすれば,
			$\hat{f}_{n_k}$は$X$上の有界可測関数であり
			\begin{align}
				\Norm{\hat{f}_{n_k} - \hat{f}_{n_{k+1}}}{\mathscr{L}^\infty} = \sup{x \in X}{|\hat{f}_{n_k}(x) - \hat{f}_{n_{k+1}}(x)|} < 1/2^k \quad (k=1,2,3,\cdots) 
				\label{ineq:Lp_banach_1}
			\end{align}
			が成り立っていることになるから,各点$x \in X$で$\left( \hat{f}_{n_k}(x) \right)_{k=1}^{\infty}$は$\K$のCauchy列となる.
			(これは$\sum_{k > N} 1/2^k = 1/2^N \longrightarrow 0\ (N \longrightarrow \infty)$による.)従って各点$x \in X$
			で極限が存在するからこれを$\hat{f}(x)$として表す.一般に距離空間に値を取る可測関数列の各点収束の極限関数は可測関数であるから
			$\hat{f}$もまた可測$\mathcal{F}/\borel{\K}$である.また$\hat{f}$は有界である.これは次のように示される.
			式(\refeq{ineq:Lp_banach_1})から任意の$l > k$に対し
			\begin{align}
				|\hat{f}_{n_k}(x) - \hat{f}_{n_l}(x)| \leq \sum_{j=k}^{l-1} |\hat{f}_{n_{j}}(x) - \hat{f}_{n_{j+1}}(x)| 
				\leq \sum_{j=k}^{l-1} \sup{x \in X}{|\hat{f}_{n_j}(x) - \hat{f}_{n_{j+1}}(x)|} < 1/2^{k-1}
			\end{align}
			が成り立つから,極限関数$\hat{f}(x)$も
			\begin{align}
				\sup{x \in X}{|\hat{f}_{n_k}(x) - \hat{f}(x)|} \leq 1/2^{k-1} \label{ineq:Lp_banach_3}
			\end{align}
			を満たすことになる.なぜなら,もし或る$x \in X$で$\alpha \coloneqq |\hat{f}_{n_k}(x) - \hat{f}(x)| > 1/2^{k-1}$となる場合,
			任意の$l > k$に対し
			\begin{align}
				0 < \alpha - 1/2^{k-1} < |\hat{f}_{n_k}(x) - \hat{f}(x)| - |\hat{f}_{n_k}(x) - \hat{f}_{n_l}(x)| \leq |\hat{f}_{n_l}(x) - \hat{f}(x)|
			\end{align}
			となり各点収束に反するからである.不等式(\refeq{ineq:Lp_banach_3})により任意の$x \in X$において
			\begin{align}
				|\hat{f}(x)| < |\hat{f}_{n_k}(x)| + 1/2^{k-1} \leq \Norm{\hat{f}_{n_k}}{\mathscr{L}^\infty} + 1/2^{k-1}
			\end{align}
			が成り立ち$\hat{f}$の有界性が判る.以上で極限関数$\hat{f}$が有界可測関数であると示された.
			$\hat{f}$を代表元とする$[\hat{f}] \in \Lp{\infty}{X,\mathcal{F},m}$に対し,不等式(\refeq{ineq:Lp_banach_3})により
			\begin{align}
				\Norm{[f_{n_k}] - [\hat{f}]}{\mathrm{L}^\infty} = \Norm{\hat{f}_{n_k} - \hat{f}}{\mathscr{L}^\infty} 
				= \sup{x \in X}{|\hat{f}_{n_k}(x) - \hat{f}(x)|}
				\longrightarrow 0 \ (k \longrightarrow \infty)
			\end{align}
			が成り立つから,Cauchy列$\left( [f_{n}] \right)_{n=1}^{\infty}$の部分列$\left( [f_{n_k}] \right)_{k=1}^{\infty}$が$[\hat{f}]$に収束すると示された.
			Cauchy列の部分列が収束すれば,元のCauchy列はその部分列と同じ収束先に収束するから$\Lp{\infty}{X,\mathcal{F},m}$はBanach空間である.
			
		\item[$1 \leq p < \infty$の場合]\mbox{}\\
			$[f_{n_k}]$の代表元$f_{n_k}$に対して
			\begin{align}	
				f_{n_k}(x) &\coloneqq f_{n_1}(x) + \sum_{j=1}^{k}(f_{n_j}(x) - f_{n_{j-1}}(x)) \label{eq:Lp_banach_3}
			\end{align}
			と表現できるから,これに対して
			\begin{align}
				g_k(x) &\coloneqq |f_{n_1}(x)| + \sum_{j=1}^{k}|f_{n_j}(x) - f_{n_{j-1}}(x)|
			\end{align}
			として可測関数列$(g_k)_{k=1}^{\infty}$を用意する.Minkowskiの不等式と式(\refeq{ineq:Lp_banach_2})より
			\begin{align}
				\Norm{g_k}{\mathscr{L}^p} \leq \Norm{f_{n_1}}{\mathscr{L}^p} + \sum_{j=1}^{k}\Norm{f_{n_j} - f_{n_{j-1}}}{\mathscr{L}^p}
				< \Norm{f_{n_1}}{\mathscr{L}^p} + \sum_{j=1}^{k} 1/2^j < \Norm{f_{n_1}}{\mathscr{L}^p} + 1 < \infty
			\end{align}
			が成り立つ.各点$x \in X$で$g_k(x)$は$k$について単調増大であるから,単調収束定理より
			\begin{align}
				\Norm{g}{\mathscr{L}^p}^p = \lim_{k \to \infty} \Norm{g_k}{\mathscr{L}^p}^p < \Norm{f_{n_1}}{\mathscr{L}^p} + 1 < \infty
			\end{align}
			となるので$g \in \Lp{p}{X,\mathcal{F},m}$である.従って
			\begin{align}
				B_n &\coloneqq \{\ x \in X\quad |\quad g(x) \leq n\ \} \in \mathcal{F}, \\
				B &\coloneqq \bigcup_{n=1}^{\infty} B_n
			\end{align}
			とおけば$m(X \backslash B) = 0$であり,式(\refeq{eq:Lp_banach_3})の級数は$B$上で絶対収束する(各点).
			\begin{align}
				f(x) \coloneqq
				\begin{cases}
					\lim\limits_{k \to \infty} f_{n_k}(x) & (x \in B) \\
					0 & (x \in X \backslash B)
				\end{cases}
			\end{align}
			として可測$\mathcal{F}/\borel{\R}$関数$f$を定義すれば,$|f(x)| \leq g(x)\ (\forall x \in X)$と
			$g^p$が可積分であることから$f$を代表元とする同値類$[f]$は$\Lp{p}{X,\mathcal{F},m}$の元となる.
			関数列$(\left( f_{n_k} \right)_{k=1}^{\infty})$は$f$に概収束し,
			$|f_{n_k}(x) - f(x)|^p \leq 2^p(|f_{n_k}(x)|^p + |f(x)|^p) \leq 2^{p+1} g(x)^p\ (\forall x \in X)$となるから
			Lebesgueの収束定理により
			\begin{align}
				\lim_{k \to \infty}\Norm{[f_{n_k}] - [f]}{\mathrm{L}^p}^p
				= \lim_{k \to \infty}\Norm{f_{n_k} - f}{\mathscr{L}^p}^p
				= \lim_{k \to \infty} \int_X |f_{n_k}(x) - f(x)|^p\ m(dx) = 0
			\end{align}
			が成り立ち,Cauchy列$\left( [f_{n}] \right)_{n=1}^{\infty}$の部分列$\left( [f_{n_k}] \right)_{k=1}^{\infty}$が$[f]$に収束すると示された.
			Cauchy列の部分列が収束すれば,元のCauchy列はその部分列と同じ収束先に収束するから$\Lp{p}{X,\mathcal{F},m}$はBanach空間である.
	\end{description}
	\QED
\end{prf}

次節への準備として,ノルム空間における線型作用素の拡張定理とHilbert空間における射影定理を載せておく.
\begin{thm}[線型作用素の拡張]
	係数体を$\K$とする.$X,Y$をBanach空間とし,ノルムをそれぞれ$\Norm{\cdot}{X},\ \Norm{\cdot}{Y}$と表記する.
	$X$の部分空間$X_0$が$X$で稠密なら,$X$から$Y$への任意の有界線型作用素$T\ $($T$の定義域は$X_0$)に対し,
	作用素ノルムを変えない$T$の拡張$\tilde{T}\ $(定義域$X$)で,$X$から$Y$への有界線型作用素となるものが一意に存在する.
\end{thm}
\begin{prf}
		作用素ノルムは$\Norm{\cdot}{}$と表記する.$X_0$が$X$で稠密であるということにより,任意の$x \in X$に対して
		$x_n \in X_0 \ (n=1,2,\cdots)$で$\Norm{x_n - x}{X} \longrightarrow 0\ (k \longrightarrow +\infty)$
		となるものを取ることができる.任意の$m,n \in \N$に対して
		\begin{align}
			\Norm{Tx_m - Tx_n}{Y} \leq \Norm{T}{} \Norm{x_m - x_n}{X}
		\end{align}
		が成り立つから,右辺が$X_0$のCauchy列をなすことにより$(Tx_n)_{n=1}^{+\infty}$も$Y$のCauchy列となる.
		$Y$の完備性から$(Tx_n)_{n=1}^{+\infty}$は或る$y \in Y$に収束し,$y$は$x \in X$に対して一意に定まる.
		なぜならば,$x$への別の収束列$z_n \in X_0 \ (n=1,2,\cdots)$を取った場合の$(Tz_n)_{n=1}^{+\infty}$の収束先が
		$u \in \C$であるとして,任意の$n,m \in \N$に対して
		\begin{align}
			\Norm{y - u}{Y} &= \Norm{y - Tx_n + Tx_n - Tz_m + Tz_m - u}{Y} \\
			&\leq \Norm{y - Tx_n}{Y} + \Norm{Tx_n - Tz_m}{Y} + \Norm{Tz_m - u}{Y} \\
			&\leq \Norm{y - Tx_n}{Y} + \Norm{T}{}\Norm{x_n - z_m}{X} + \Norm{Tz_m - u}{Y} \\
			&\leq \Norm{y - Tx_n}{Y} + \Norm{T}{}\left(\Norm{x_n - x}{X} + \Norm{x - z_m}{X}\right)+ \Norm{Tz_m - u}{Y}
		\end{align}
		となるから$n,m \longrightarrow +\infty$で右辺は0に収束し,$y = u$が示されるためである.
		つまり$x$に$y$を対応させる関係は$X \longmapsto Y$の写像となり,この写像を$\tilde{T}$と表すことにする.$T$の線型性も次のように示される.
		任意の$x,\ z \in X,\ \alpha,\ \beta \in \K$に対して,$x,z$への収束列$(x_n)_{n=1}^{+\infty},\ (z_n)_{n=1}^{+\infty} \subset X_0$
		を取れば$(\alpha x_n + \beta z_n)_{n=1}^{+\infty}$が$\alpha x+ \beta z$への収束列となるから
		\begin{align}
			\Norm{\tilde{T}(\alpha x + \beta z) - \alpha \tilde{T}x - \beta \tilde{T}z}{Y}
			&= \Norm{\tilde{T}(\alpha x + \beta z) - T(\alpha x_n + \beta z_n) + \alpha Tx_n + \beta Tz_n - \alpha \tilde{T}x - \beta \tilde{T}z}{Y} \\
			&\leq \Norm{\tilde{T}(\alpha x + \beta z) - T(\alpha x_n + \beta z_n)}{Y}
				+ \Norm{\alpha Tx_n - \alpha \tilde{T}x}{Y} + \Norm{\beta Tz_n - \beta \tilde{T}z}{Y} \\
			&\leq \Norm{\tilde{T}(\alpha x + \beta z) - T(\alpha x_n + \beta z_n)}{Y}
				+ |\alpha| \Norm{Tx_n - \tilde{T}x}{Y} + |\beta| \Norm{Tz_n - \tilde{T}z}{Y} \\
			&\longrightarrow 0\quad (n \longrightarrow +\infty)
		\end{align}
		が成り立つ.ゆえに$\tilde{T}(\alpha x + \beta z) = \alpha \tilde{T}x + \beta \tilde{T}z\ (\forall x,\ z \in X,\ \alpha,\ \beta \in \K)$である.
		また$\tilde{T}$は有界な線型作用素である.なぜなら,任意に$x \in X$と$x$への収束列$x_n \in X_0\ (n = 1,2,\cdots)$を取れば,
		任意の$\epsilon > 0$に対し或る$K \in \N$が存在して全ての$k > K$について
		\begin{align}
			\Norm{\tilde{T}x}{Y} < \Norm{Tx_n}{Y} + \epsilon, \quad \Norm{x}{X} < \Norm{x_n}{X} + \epsilon/\Norm{T}{}
		\end{align}
		が成り立つようにできるから,この下で
		\begin{align}
			\Norm{\tilde{T}x}{Y} < \Norm{Tx}{Y} + \epsilon < \Norm{T}{} \Norm{x}{X} + 2\epsilon
		\end{align}
		となり$\Norm{\tilde{T}}{} \leq \Norm{T}{}$が判るからである.さらに
		\begin{align}
			\Norm{\tilde{T}}{} = \sup{\substack{x \in X \\ \Norm{x}{X} = 1}}{\Norm{\tilde{T}x}{Y}} 
			\geq \sup{\substack{x \in X_0 \\ \Norm{x}{X} = 1}}{\Norm{\tilde{T}x}{Y}} 
			= \sup{\substack{x \in X_0 \\ \Norm{x}{X} = 1}}{\Norm{Tx}{Y}} = \Norm{T}{}
		\end{align}
		も成り立つから結局$\Norm{\tilde{T}}{} = \Norm{T}{}$であると判る.以上より
		任意の有界線型作用素$T$がノルムを変えないまま或る有界線型作用素$\tilde{T}$に拡張されることが示された.
		拡張が一意であることは$X_0$が$X$で稠密であることと$T$の連続性による.
		\QED
\end{prf}

\begin{thm}[射影定理]
\end{thm}

\begin{prf}\mbox{}\\
	\begin{description}
	\item[射影の存在]
	$f \in H \backslash C$として
	\begin{align}
		\delta \coloneqq \inf{h \in C}{\Norm{f - h}{}}
	\end{align}
	とおく.$C$が閉集合で$f$が$C$の外にあるから$\delta > 0$となる.
	下限の性質から$h_n \in C\ (n = 1,2,3,\cdots)$を取って
	\begin{align}
		\delta = \lim_{n \to \infty}\Norm{f - h_n}{}
	\end{align}
	となるようにできるから,任意の$\epsilon > 0$に対して或る$N \in \N$が存在して
	$n> N$ならば$\Norm{f - h_n}{}^2 < \delta + \epsilon/4$が成り立つ.この$N$に対し
	$n,m > N$ならば,内積空間の中線定理と$(h_n + h_m)/2 \in C$であることにより
	\begin{align}
		\Norm{h_n - h_m}{}^2 &= 2\left( \Norm{f - h_m}{}^2 + \Norm{f - h_n}{}^2 \right) - \Norm{2f - (h_n + h_m)}{}^2 \\
		&= 2\left( \Norm{f - h_m}{}^2 + \Norm{f - h_n}{}^2 \right) - 4\Norm{f - \frac{h_n + h_m}{2}}{}^2 \\
		&< 2\delta + \epsilon - 4\delta = \epsilon
	\end{align}
	とできるから$(h_n)_{n=1}^{\infty}$は$C$のCauchy列であると判る.$H$がHilbert空間であり$C$が$H$で閉だから,
	$(h_n)_{n=1}^{\infty}$の極限$y \in H$が存在し$y \in C$である.
	\begin{align}
		\left| \delta - \Norm{f - y}{} \right| 
		\leq \left| \delta - \Norm{f - h_n}{} \right| + \left| \Norm{f - h_n}{} - \Norm{f - y}{} \right|
		\leq \left| \delta - \Norm{f - h_n}{} \right| + \Norm{h_n - y}{}
		\longrightarrow 0 \quad (n \longrightarrow \infty)
	\end{align}
	によって$\delta = \Norm{f - y}{}$が成り立つこと,すなわち射影の存在が示された.
	$f \in C$の場合は$f$が自身の射影である.

	\item[射影の一意性]
		$z \in C$もまた$\delta = \Norm{f - z}{}$を満たすとすれば,$C$の凸性により
		\begin{align}
			2 \delta \leq 2\Norm{f - \frac{y + z}{2}}{} \leq \Norm{f - y}{} + \Norm{f - z}{} = 2\delta
		\end{align}
		が成り立つから,中線定理より
		\begin{align}
			\Norm{y - z}{}^2 = 2\left( \Norm{f - z}{}^2 + \Norm{f - y}{}^2 \right) - 4\Norm{f - \frac{y + z}{2}}{}^2 = 0
		\end{align}
		となって$y = z$が判る.すなわち$f$の射影はただ一つに決まる.
	
	\item[$C$が閉部分空間の場合]
		$f \in H \backslash C$に対して$f$の$C$への射影を$y \in C\ $(存在は$C$が凸の場合と全く同様に示される.)
		とする.($f \in C$の場合は$y = f$である.)或る$h \in C$に対して
		\begin{align}
			\inprod<f -y,\ h> \neq 0
		\end{align}
		となると仮定すれば($f \neq y$より$h \neq 0$),$C$の元$\hat{y} \coloneqq y + \left(\inprod<f - g,h>/\Norm{h}{}^2\right)h$
		に対して
		\begin{align}
			\Norm{f - \hat{y}}{}^2 
			&= \inprod<f - y - \frac{\inprod<f - g,h>}{\Norm{h}{}^2}h,\ f - y - \frac{\inprod<f - g,h>}{\Norm{h}{}^2}h> \\
			&= \Norm{f-y}{}^2 - \frac{|\inprod<f - g,h>|^2}{\Norm{h}{}^2}
			&< \Norm{f-y}{}^2
		\end{align}
		が成り立つから$y$が射影であることに反する.従って射影$y$に対しては
		\begin{align}
			\inprod<f -y,\ h> \neq 0 \quad (\forall h \in C) \label{eq:projection}
		\end{align}
		が成り立つ.逆に$y \in C$に対して式(\refeq{eq:projection})が成り立っているとすれば
		$y$が$f$の射影であることも示される.任意の$h \in C$に対して
		\begin{align}
			\Norm{f - h}{}^2 &= \inprod<f - y + y - h,\ f - y + y - h> \\
			&= \Norm{f - y}{}^2 + 2 \Re{\inprod<f - y,\ y - h>} + \Norm{y - h}{}^2 \\
			&= \Norm{f - y}{}^2 + \Norm{y - h}{}^2 \\
			&\geq \Norm{f - y}{}^2
		\end{align}
		となることにより$\Norm{f - y}{} = \inf{h \in C}{\Norm{f - h}{}}$であることが示された.
	\end{description}
\end{prf}
\end{qst}
	
次節への準備として,ノルム空間における線型作用素の拡張定理とHilbert空間における射影定理を載せておく.
\begin{itembox}[l]{}
	\begin{thm}[線型作用素の拡張]
		係数体を$\K$とする.$X,Y$をBanach空間とし,ノルムをそれぞれ$\Norm{\cdot}{X},\ \Norm{\cdot}{Y}$と表記する.
		$X$の部分空間$X_0$が$X$で稠密なら,$X$から$Y$への任意の有界線型作用素$T\ $($T$の定義域は$X_0$)に対し,
		作用素ノルムを変えない$T$の拡張$\tilde{T}\ $(定義域$X$)で,$X$から$Y$への有界線型作用素となるものが一意に存在する.
	\end{thm}
\end{itembox}
\begin{prf}
		作用素ノルムは$\Norm{\cdot}{}$と表記する.$X_0$が$X$で稠密であるということにより,任意の$x \in X$に対して
		$x_n \in X_0 \ (n=1,2,\cdots)$で$\Norm{x_n - x}{X} \longrightarrow 0\ (k \longrightarrow +\infty)$
		となるものを取ることができる.任意の$m,n \in \N$に対して
		\begin{align}
			\Norm{Tx_m - Tx_n}{Y} \leq \Norm{T}{} \Norm{x_m - x_n}{X}
		\end{align}
		が成り立つから,右辺が$X_0$のCauchy列をなすことにより$(Tx_n)_{n=1}^{+\infty}$も$Y$のCauchy列となる.
		$Y$の完備性から$(Tx_n)_{n=1}^{+\infty}$は或る$y \in Y$に収束し,$y$は$x \in X$に対して一意に定まる.
		なぜならば,$x$への別の収束列$z_n \in X_0 \ (n=1,2,\cdots)$を取った場合の$(Tz_n)_{n=1}^{+\infty}$の収束先が
		$u \in \C$であるとして,任意の$n,m \in \N$に対して
		\begin{align}
			\Norm{y - u}{Y} &= \Norm{y - Tx_n + Tx_n - Tz_m + Tz_m - u}{Y} \\
			&\leq \Norm{y - Tx_n}{Y} + \Norm{Tx_n - Tz_m}{Y} + \Norm{Tz_m - u}{Y} \\
			&\leq \Norm{y - Tx_n}{Y} + \Norm{T}{}\Norm{x_n - z_m}{X} + \Norm{Tz_m - u}{Y} \\
			&\leq \Norm{y - Tx_n}{Y} + \Norm{T}{}\left(\Norm{x_n - x}{X} + \Norm{x - z_m}{X}\right)+ \Norm{Tz_m - u}{Y}
		\end{align}
		となるから$n,m \longrightarrow +\infty$で右辺は0に収束し,$y = u$が示されるためである.
		つまり$x$に$y$を対応させる関係は$X \longmapsto Y$の写像となり,この写像を$\tilde{T}$と表すことにする.$T$の線型性も次のように示される.
		任意の$x,\ z \in X,\ \alpha,\ \beta \in \K$に対して,$x,z$への収束列$(x_n)_{n=1}^{+\infty},\ (z_n)_{n=1}^{+\infty} \subset X_0$
		を取れば$(\alpha x_n + \beta z_n)_{n=1}^{+\infty}$が$\alpha x+ \beta z$への収束列となるから
		\begin{align}
			\Norm{\tilde{T}(\alpha x + \beta z) - \alpha \tilde{T}x - \beta \tilde{T}z}{Y}
			&= \Norm{\tilde{T}(\alpha x + \beta z) - T(\alpha x_n + \beta z_n) + \alpha Tx_n + \beta Tz_n - \alpha \tilde{T}x - \beta \tilde{T}z}{Y} \\
			&\leq \Norm{\tilde{T}(\alpha x + \beta z) - T(\alpha x_n + \beta z_n)}{Y}
				+ \Norm{\alpha Tx_n - \alpha \tilde{T}x}{Y} + \Norm{\beta Tz_n - \beta \tilde{T}z}{Y} \\
			&\leq \Norm{\tilde{T}(\alpha x + \beta z) - T(\alpha x_n + \beta z_n)}{Y}
				+ |\alpha| \Norm{Tx_n - \tilde{T}x}{Y} + |\beta| \Norm{Tz_n - \tilde{T}z}{Y} \\
			&\longrightarrow 0\quad (n \longrightarrow +\infty)
		\end{align}
		が成り立つ.ゆえに$\tilde{T}(\alpha x + \beta z) = \alpha \tilde{T}x + \beta \tilde{T}z\ (\forall x,\ z \in X,\ \alpha,\ \beta \in \K)$である.
		また$\tilde{T}$は有界な線型作用素である.なぜなら,任意に$x \in X$と$x$への収束列$x_n \in X_0\ (n = 1,2,\cdots)$を取れば,
		任意の$\epsilon > 0$に対し或る$K \in \N$が存在して全ての$k > K$について
		\begin{align}
			\Norm{\tilde{T}x}{Y} < \Norm{Tx_n}{Y} + \epsilon, \quad \Norm{x}{X} < \Norm{x_n}{X} + \epsilon/\Norm{T}{}
		\end{align}
		が成り立つようにできるから,この下で
		\begin{align}
			\Norm{\tilde{T}x}{Y} < \Norm{Tx}{Y} + \epsilon < \Norm{T}{} \Norm{x}{X} + 2\epsilon
		\end{align}
		となり$\Norm{\tilde{T}}{} \leq \Norm{T}{}$が判るからである.さらに
		\begin{align}
			\Norm{\tilde{T}}{} = \sup{\substack{x \in X \\ \Norm{x}{X} = 1}}{\Norm{\tilde{T}x}{Y}} 
			\geq \sup{\substack{x \in X_0 \\ \Norm{x}{X} = 1}}{\Norm{\tilde{T}x}{Y}} 
			= \sup{\substack{x \in X_0 \\ \Norm{x}{X} = 1}}{\Norm{Tx}{Y}} = \Norm{T}{}
		\end{align}
		も成り立つから結局$\Norm{\tilde{T}}{} = \Norm{T}{}$であると判る.以上より
		任意の有界線型作用素$T$がノルムを変えないまま或る有界線型作用素$\tilde{T}$に拡張されることが示された.
		拡張が一意であることは$X_0$が$X$で稠密であることと$T$の連続性による.
		\QED
\end{prf}

\begin{itembox}[l]{}
	\begin{thm}[射影定理]
	\end{thm}
\end{itembox}
\begin{prf}\mbox{}\\
	\begin{description}
	\item[射影の存在]
	$f \in H \backslash C$として
	\begin{align}
		\delta \coloneqq \inf{h \in C}{\Norm{f - h}{}}
	\end{align}
	とおく.$C$が閉集合で$f$が$C$の外にあるから$\delta > 0$となる.
	下限の性質から$h_n \in C\ (n = 1,2,3,\cdots)$を取って
	\begin{align}
		\delta = \lim_{n \to \infty}\Norm{f - h_n}{}
	\end{align}
	となるようにできるから,任意の$\epsilon > 0$に対して或る$N \in \N$が存在して
	$n> N$ならば$\Norm{f - h_n}{}^2 < \delta + \epsilon/4$が成り立つ.この$N$に対し
	$n,m > N$ならば,内積空間の中線定理と$(h_n + h_m)/2 \in C$であることにより
	\begin{align}
		\Norm{h_n - h_m}{}^2 &= 2\left( \Norm{f - h_m}{}^2 + \Norm{f - h_n}{}^2 \right) - \Norm{2f - (h_n + h_m)}{}^2 \\
		&= 2\left( \Norm{f - h_m}{}^2 + \Norm{f - h_n}{}^2 \right) - 4\Norm{f - \frac{h_n + h_m}{2}}{}^2 \\
		&< 2\delta + \epsilon - 4\delta = \epsilon
	\end{align}
	とできるから$(h_n)_{n=1}^{\infty}$は$C$のCauchy列であると判る.$H$がHilbert空間であり$C$が$H$で閉だから,
	$(h_n)_{n=1}^{\infty}$の極限$y \in H$が存在し$y \in C$である.
	\begin{align}
		\left| \delta - \Norm{f - y}{} \right| 
		\leq \left| \delta - \Norm{f - h_n}{} \right| + \left| \Norm{f - h_n}{} - \Norm{f - y}{} \right|
		\leq \left| \delta - \Norm{f - h_n}{} \right| + \Norm{h_n - y}{}
		\longrightarrow 0 \quad (n \longrightarrow \infty)
	\end{align}
	によって$\delta = \Norm{f - y}{}$が成り立つこと,すなわち射影の存在が示された.
	$f \in C$の場合は$f$が自身の射影である.

	\item[射影の一意性]
		$z \in C$もまた$\delta = \Norm{f - z}{}$を満たすとすれば,$C$の凸性により
		\begin{align}
			2 \delta \leq 2\Norm{f - \frac{y + z}{2}}{} \leq \Norm{f - y}{} + \Norm{f - z}{} = 2\delta
		\end{align}
		が成り立つから,中線定理より
		\begin{align}
			\Norm{y - z}{}^2 = 2\left( \Norm{f - z}{}^2 + \Norm{f - y}{}^2 \right) - 4\Norm{f - \frac{y + z}{2}}{}^2 = 0
		\end{align}
		となって$y = z$が判る.すなわち$f$の射影はただ一つに決まる.
	
	\item[$C$が閉部分空間の場合]
		$f \in H \backslash C$に対して$f$の$C$への射影を$y \in C\ $(存在は$C$が凸の場合と全く同様に示される.)
		とする.($f \in C$の場合は$y = f$である.)或る$h \in C$に対して
		\begin{align}
			\inprod<f -y,\ h> \neq 0
		\end{align}
		となると仮定すれば($f \neq y$より$h \neq 0$),$C$の元$\hat{y} \coloneqq y + \left(\inprod<f - g,h>/\Norm{h}{}^2\right)h$
		に対して
		\begin{align}
			\Norm{f - \hat{y}}{}^2 
			&= \inprod<f - y - \frac{\inprod<f - g,h>}{\Norm{h}{}^2}h,\ f - y - \frac{\inprod<f - g,h>}{\Norm{h}{}^2}h> \\
			&= \Norm{f-y}{}^2 - \frac{|\inprod<f - g,h>|^2}{\Norm{h}{}^2}
			&< \Norm{f-y}{}^2
		\end{align}
		が成り立つから$y$が射影であることに反する.従って射影$y$に対しては
		\begin{align}
			\inprod<f -y,\ h> \neq 0 \quad (\forall h \in C) \label{eq:projection}
		\end{align}
		が成り立つ.逆に$y \in C$に対して式(\refeq{eq:projection})が成り立っているとすれば
		$y$が$f$の射影であることも示される.任意の$h \in C$に対して
		\begin{align}
			\Norm{f - h}{}^2 &= \inprod<f - y + y - h,\ f - y + y - h> \\
			&= \Norm{f - y}{}^2 + 2 \Re{\inprod<f - y,\ y - h>} + \Norm{y - h}{}^2 \\
			&= \Norm{f - y}{}^2 + \Norm{y - h}{}^2 \\
			&\geq \Norm{f - y}{}^2
		\end{align}
		となることにより$\Norm{f - y}{} = \inf{h \in C}{\Norm{f - h}{}}$であることが示された.
	\end{description}
\end{prf}

	\section{Sobolev空間について}
係数体を$\K$,$\K = \R$或は$\K = \C$と考える.測度空間を$(X,\mathcal{F},m)$とする.

\begin{dfn}[絶対連続関数]
	$I \coloneqq [a,b]$を$\R$の区間とする.$I$上の関数$f:I \longrightarrow \K$が絶対連続であるとは,
	任意の$a \leq a_1 < b_1 \leq a_2 < b_2 \leq \cdots \leq a_n < b_n \leq b,\ (n = 1,2,3,\cdots)$
	と任意の$\epsilon > 0$に対し,或る$\delta > 0$が存在して
	\begin{align}
		\sum_{i = 1}^{n}(b_i - a_i) < \delta \quad \Rightarrow \quad 
		\sum_{i = 1}^{n}|f(b_i) - f(a_i)| < \epsilon
	\end{align}
	が成り立つことをいう.
\end{dfn}

\begin{thm}[絶対連続の同値条件]
	測度空間を$(X,\mathcal{F},m)$とする.
\end{thm}

\begin{dfn}[Sobolev空間]
\end{dfn}
\chapter{条件付き期待値作用素}
	\section{10/11}
	基礎の確率空間を$(\Omega,\mathcal{F},\mu)$とする.
	$\mathcal{G} \subset \mathcal{F}$を部分$\sigma$-加法族とし,
	Hilbert空間$\Lp{2}{\Omega,\mathcal{F},\mu}$とその閉部分空間
	$\Lp{2}{\Omega, \mathcal{G},\mu}$を考える.
	任意の$f \in \Lp{2}{\Omega, \mathcal{F},\mu}$に対して,
	射影定理により一意に定まる射影$g \in \Lp{2}{\Omega, \mathcal{G},\mu}$を
	\begin{align}
		g = \cexp{f}{\mathcal{G}}
	\end{align}
	と表現する.$\mathcal{G} = \{\emptyset, \Omega\}$のときは$\cexp{f}{\mathcal{G}}$を$\Exp{f}$と書いて$f$の期待値と呼ぶ.
	\begin{qst}\mbox{}\\
		扱う関数は全て$\R$値と考える.
		\begin{description}
			\item[C1] $\forall f \in \Lp{2}{\Omega, \mathcal{F},\mu}$
				\begin{align}
					\Exp{f} = \int_{\Omega} f(x)\ \mu(dx)
				\end{align}
				
			\item[C2]	$\forall f \in \Lp{2}{\Omega, \mathcal{F},\mu},\ \forall h \in \Lp{2}{\Omega, \mathcal{G},\mu}$
				\begin{align}
					\int_{\Omega} f(x)h(x)\ \mu(dx) = \int_{\Omega} \cexp{f}{\mathcal{G}}(x)h(x)\ \mu(dx)
				\end{align}
				
			\item[C3]	$\forall f_1,f_2 \in \Lp{2}{\Omega, \mathcal{F},\mu}$
				\begin{align}
					\cexp{f_1 + f_2}{\mathcal{G}} = \cexp{f_1}{\mathcal{G}} + \cexp{f_2}{\mathcal{G}}
				\end{align}

			\item[C4]	$\forall f_1,f_2 \in \Lp{2}{\Omega, \mathcal{F},\mu}$
				\begin{align}
					f_1 \leq f_2 \quad \mathrm{a.s.} \quad \Rightarrow \quad \cexp{f_1}{\mathcal{G}} \leq \cexp{f_2}{\mathcal{G}} \quad \mathrm{a.s.}
				\end{align}
			
			\item[C5]	$\forall f \in \Lp{2}{\Omega, \mathcal{F},\mu},\ \forall g \in \Lp{\infty}{\Omega, \mathcal{G},\mu}$
				\begin{align}
					\cexp{gf}{\mathcal{G}} = g\cexp{f}{\mathcal{G}}
				\end{align}
			
			\item[C6]	$\mathcal{H}$が$\mathcal{G}$の部分$\sigma$-加法族ならば$\forall f \in \Lp{2}{\Omega, \mathcal{F},\mu}$
				\begin{align}
					\cexp{\cexp{f}{\mathcal{G}}}{\mathcal{H}} = \cexp{f}{\mathcal{H}}
				\end{align}
		\end{description}
	\end{qst}
	
	\begin{prf}
		\begin{description}
			\item[C1] $\mathcal{G} = \{\emptyset, \Omega\}$とすれば,
				$\Lp{2}{\Omega, \mathcal{G},\mu}$の元は$\mathcal{G}$可測でなくてはならないから$\Omega$上の定数関数である.
				従って任意の$g \in \Lp{2}{\Omega, \mathcal{G},\mu}$に或る定数$\alpha \in \R$が対応して$g(x)=\alpha\ (\forall x \in \Omega)$と表せる.
				Hilbert空間$\Lp{2}{\Omega, \mathcal{F},\mu}$におけるノルムを$\Norm{\cdot}{\Lp{2}{\Omega, \mathcal{F},\mu}}$と表示すれば,
				射影定理より任意の$f \in \Lp{2}{\Omega, \mathcal{F},\mu}$の$\Lp{2}{\Omega, \mathcal{G},\mu}$への射影$\cexp{f}{\mathcal{G}} = \Exp{f}$は
				ノルム$\Norm{f-g}{\Lp{2}{\Omega, \mathcal{F},\mu}}$を最小にする$g \in \Lp{2}{\Omega, \mathcal{G},\mu}$である.
				$g(x)=\alpha\ (\forall x \in \Omega)$としてノルムを直接計算すれば,
				\begin{align}
					\Norm{f-g}{\Lp{2}{\Omega, \mathcal{F},\mu}}^2 &= \int_{\Omega} |f(x) - \alpha|^2\ \mu(dx) \\
					&= \int_{\Omega} |f(x)|^2 - 2 \alpha f(x) + |\alpha|^2\ \mu(dx) \\
					&= \int_{\Omega} |f(x)|^2\ \mu(dx) - 2 \alpha \int_{\Omega} f(x)\ \mu(dx) + |\alpha|^2 \\
					&= \left| \alpha - \int_{\Omega} f(x)\ \mu(dx) \right|^2 - \left| \int_{\Omega} f(x)\ \mu(dx) \right|^2 + \int_{\Omega} |f(x)|^2\ \mu(dx) \\
					&= \left| \alpha - \int_{\Omega} f(x)\ \mu(dx) \right|^2 + \int_{\Omega} \left| f(x) - \beta \right|^2\ \mu(dx) & (\beta \coloneqq \int_{\Omega} f(x)\ \mu(dx))
				\end{align}
				と表現できて最終式は$\alpha = \int_{\Omega} f(x)\ \mu(dx)$となることで最小となる.すなわち
				\begin{align}
					\Exp{f} = \cexp{f}{\mathcal{G}} = \int_{\Omega} f(x)\ \mu(dx).
				\end{align}
			
			\item[C2] Hilbert空間$\Lp{2}{\Omega, \mathcal{F},\mu}$における内積を$\inprod<\cdot,\cdot>$と表示する.
				講義中の射影定理により,$f \in \Lp{2}{\Omega, \mathcal{F},\mu}$の$\Lp{2}{\Omega, \mathcal{G},\mu}$への射影$\cexp{f}{\mathcal{G}}$は
				\begin{align}
					\inprod<f - \cexp{f}{\mathcal{G}}, h> = 0 \quad (\forall h \in \Lp{2}{\Omega, \mathcal{G},\mu})
				\end{align}
				を満たし,内積の線型性から
				\begin{align}
					\inprod<f, h> = \inprod<\cexp{f}{\mathcal{G}}, h> \quad (\forall h \in \Lp{2}{\Omega, \mathcal{G},\mu})
				\end{align}
				が成り立つ.積分の形式で表示することにより
				\begin{align}
					\int_{\Omega} f(x)h(x)\ \mu(dx) = \int_{\Omega} \cexp{f}{\mathcal{G}}(x)h(x)\ \mu(dx) \quad (\forall h \in \Lp{2}{\Omega, \mathcal{G},\mu})
				\end{align}
				が示された.
				
			\item[C3] 
		\end{description}
	\end{prf}
	\section{独立性}
	任意の有界実連続関数$h:\R \rightarrow \R$と$\mathcal{F}/\borel{\R}$-可測関数$X$に対して
	その合成$h(X)$は可積分であるから,条件付き期待値を作用させることができる.これを用いて独立性を次で定義する.
	
	\begin{screen}
		\begin{dfn}[独立性]
			Xを$\mathcal{F}/\borel{\R}$-可測関数,$\mathcal{G}$を$\mathcal{F}$の部分$\sigma$-加法族とする.
			任意の有界実連続関数$h:\R \rightarrow \R$に対し次が成り立つとき,$X$と$\mathcal{G}$は独立である(independent)と定める:
			\begin{align}
				\cexp{h(X)}{\mathcal{G}}(\omega) = \int_\Omega h(X(x))\ \mu(dx)
				\quad (\mbox{$\mu$-a.s.}\omega \in \Omega).
			\end{align}
		\end{dfn}
	\end{screen}

	\begin{screen}
		\begin{thm}[独立性の同値条件]
			任意の有界実連続関数$h:\R \rightarrow \R$に対し
			\begin{align}
				\cexp{h(X)}{\mathcal{G}}(\omega) = \int_\Omega h(X(x))\ \mu(dx)
				\quad (\mbox{$\mu$-a.s.}\omega \in \Omega)
				\label{eq:prp_equivalent_condition_of_independence_1}
			\end{align}
			が成り立つことと
			\begin{align}
				\mu\left( X^{-1}(E) \cap A \right) = \mu\left( X^{-1}(E) \right)\mu(A) \quad (\forall E \in \borel{\R},\ A \in \mathcal{G})
				\label{eq:prp_equivalent_condition_of_independence_2}
			\end{align}
			が成り立つことは同値である.
			\label{prp:equivalent_condition_of_independence}
		\end{thm}
	\end{screen}
	
	\begin{prf}\mbox{}
		\begin{description}
			\item[(\refeq{eq:prp_equivalent_condition_of_independence_1})$\Rightarrow$(\refeq{eq:prp_equivalent_condition_of_independence_2})] 
				(\refeq{eq:prp_equivalent_condition_of_independence_1})を仮定して
				\begin{align}
					\borel{\R} = \Set{E \in \borel{\R}}{\mu\left( X^{-1}(E) \cap A \right) = \mu\left( X^{-1}(E) \right) \mu(A),\quad \forall A \in \mathcal{G}}
					\label{eq:prp_equivalent_condition_of_independence_3}
				\end{align}
				が成り立つことを示す.まず(\refeq{eq:prp_equivalent_condition_of_independence_3})の右辺がDynkin族であることを示し,
				次に右辺が$\R$の閉集合系を含むことを示す.
				これが示されればDynkin族定理により(\refeq{eq:prp_equivalent_condition_of_independence_3})が得られる.
				\begin{align}
					\mathscr{D} \coloneqq \Set{E \in \borel{\R}}{\mu\left( X^{-1}(E) \cap A \right) = \mu\left( X^{-1}(E) \right) \mu(A),\quad \forall A \in \mathcal{G}}
				\end{align}
				とおけば,$\mathscr{D}$は次の(1)(2)(3)を満たすからDynkin族である:
				\begin{description}
					\item[(1)] $\R \in \mathscr{D}$,
					\item[(2)] $D_1,D_2 \in \mathscr{D},\ D_1 \subset D_2\quad \Rightarrow\quad D_2 \backslash D_1 \in \mathscr{D}$,
					\item[(3)] $D_n \in \mathscr{D},\ D_n \cap D_m = \emptyset\ (n \neq m)\quad \Rightarrow\quad \sum_{n=1}^{\infty} D_n \in \mathscr{D}$.
				\end{description}
				実際,$\Omega = X^{-1}(\R)$により任意の$A \in \mathcal{G}$に対して
				$\mu\left( X^{-1}(\R) \cap A \right) = \mu\left( X^{-1}(\R) \right) \mu(A)$が成り立つから(1)が従い,
				また$D_1 \subset D_2$ならば$X^{-1}(D_2 \backslash D_1) = X^{-1}(D_2) \backslash X^{-1}(D_1)$が成り立つから
				\begin{align}
					&\mu\left( X^{-1}(D_2 \backslash D_1) \cap A \right) = \mu \left( X^{-1}(D_2) \cap A \right) - \mu\left( X^{-1}(D_1) \cap A \right) \\
					&\qquad = \left\{ \mu\left( X^{-1}(D_2) \right) - \mu \left( X^{-1}(D_1) \right) \right\}\mu(A)
					= \mu\left( X^{-1}(D_2 - D_1) \right) \mu(A)\quad (\forall A \in \mathcal{G})
				\end{align}
				により(2)が従う.(3)も
				\begin{align}
					&\mu\Biggl( X^{-1}\biggl( \sum_{n=1}^{\infty} D_n \biggr) \cap A \Biggr) 
					= \mu\Biggl( \sum_{n=1}^{\infty} X^{-1}(D_n) \cap A \Biggr) 
					= \sum_{n=1}^{\infty} \mu\left( X^{-1}(D_n) \cap A \right) \\
					&\qquad = \sum_{n=1}^{\infty} \mu\left( X^{-1}(D_n) \right) \mu(A) 
					= \mu\Biggl( X^{-1}\biggl( \sum_{n=1}^{\infty} D_n \biggr) \Biggr) \mu(A)\quad (\forall A \in \mathcal{G})
				\end{align}
				により従う.次に$\R$の任意の閉集合が
				$\mathscr{D}$に属することを示す.$E$を$\R$の閉集合として
				\begin{align}
					d(\cdot,E):\R \ni x \longmapsto \inf{}{\Set{|x-y|}{y \in E}}
				\end{align}
				とおき
				\begin{align}
					h_n:\R \ni x \longmapsto \frac{1}{1 + nd(x,E)} \quad (n=1,2,3,\cdots)
				\end{align}
				により有界実連続関数列$(h_n)_{n=1}^{\infty}$を定める.
				$E$が閉集合であるから
				\begin{align}
					\lim_{n \to \infty} h_n(x) = \defunc_E(x)
					\quad (\forall x \in X)
				\end{align}
				が成り立ち,また$h_n$の有界連続性と(\refeq{eq:prp_equivalent_condition_of_independence_1})の仮定により
				\begin{align}
					\cexp{h_n(X)}{\mathcal{G}}(\omega) = \int_\Omega h_n(X(x))\ \mu(dx)
					\quad (\mbox{$\mu$-a.s.}\omega \in \Omega,\ n=1,2,\cdots)
				\end{align}
				が満たされているから,任意に$A \in \mathcal{G}$を取れば
				\begin{align}
					\mu\left( X^{-1}(E) \cap A \right) 
					&= \int_A \defunc_E(X(\omega))\ \mu(d\omega) \\
					&= \lim_{n \to \infty} \int_A h_n(X(\omega))\ \mu(d\omega) \\
					&= \lim_{n \to \infty} \int_A \cexp{h_n(X)}{\mathcal{G}}(\omega)\ \mu(d\omega) \\
					&= \mu(A)  \lim_{n \to \infty} \int_{\Omega} h_n(X(\omega))\ \mu(d\omega) \\
					&= \mu(A) \int_{\Omega} \defunc_E(X(\omega))\ \mu(d\omega) \\
					&= \mu\left( X^{-1}(E) \right) \mu(A)
				\end{align}
				が成り立ち$E \in \mathscr{D}$が従う.
			
			\item[(\refeq{eq:prp_equivalent_condition_of_independence_2})$\Rightarrow$(\refeq{eq:prp_equivalent_condition_of_independence_1})]
				(\refeq{eq:prp_equivalent_condition_of_independence_2})を仮定して
				\begin{align}
					\int_A \cexp{h(X)}{\mathcal{G}}(\omega)\ \mu(d\omega) 
					= \mu(A) \int_\Omega h(X(\omega))\ \mu(d\omega)
					\quad (\forall A \in \mathcal{G})
				\end{align}
				成り立つことを示す.有界実連続関数$h:\R \rightarrow \R$に対して
				$|h_n| \leq |h|$を満たすように単関数近似列$(h_n)_{n=1}^{\infty}$を取る.
				各$h_n$は$\alpha_i^n \in \R,\ E_i^n \in \borel{\R}\ (i=1,\cdots,N_n),\ \sum_{i=1}^{N_n}E_i^n = \R$を用いて
				\begin{align}
					h_n = \sum_{i=1}^{N_n} \alpha_i^n \defunc_{E_i^n}
				\end{align}
				の形で表現できるから,(\refeq{eq:prp_equivalent_condition_of_independence_2})の仮定の下では任意の$A \in \mathcal{G}$に対して
				\begin{align}
					&\int_A h_n(X(\omega))\ \mu(d\omega)
					= \sum_{i=1}^{N_n} \alpha_i^n \int_A \defunc_{X^{-1}(E_i^n)}(\omega)\ \mu(d\omega)
					= \sum_{i=1}^{N_n} \alpha_i^n \mu\left( X^{-1}(E_i^n) \cap A \right) \\
					&\qquad = \mu(A) \sum_{i=1}^{N_n} \alpha_i^n \mu\left( X^{-1}(E_i^n) \right)
					= \sum_{i=1}^{N_n} \alpha_i^n \int_\Omega \defunc_{X^{-1}(E_i^n)}(\omega)\ \mu(d\omega)
					= \mu(A) \int_\Omega h_n(X(\omega))\ \mu(d\omega)
				\end{align}
				が成り立ち,Lebesgueの収束定理より
				\begin{align}
					&\int_A \cexp{h(X)}{\mathcal{G}}(\omega)\ \mu(d\omega)
					= \int_A h(X(\omega))\ \mu(d\omega)
					= \lim_{n \to \infty} \int_A h_n(X(\omega))\ \mu(d\omega) \\
					&\qquad = \mu(A) \lim_{n \to \infty} \int_{\Omega} h_n(X(\omega))\ \mu(d\omega)
					= \mu(A) \int_{\Omega} h(X(\omega))\ \mu(d\omega)
				\end{align}
				が従う.
				\QED
		\end{description}
	\end{prf}

\begin{comment}	
	\begin{screen}
		\begin{thm}
			$A,B \in \mathcal{F}$に対し,$X = \defunc_A$,$\mathcal{G} = \{ \emptyset,\ \Omega,\ B,\ B^c\}$とすれば次が成り立つ:
			\begin{align}
				\mbox{$X$と$\mathcal{G}$が独立}\quad \Leftrightarrow\quad \mu(A \cap B) = \mu(A)\mu(B).
			\end{align}
			\label{thm:report_3}
		\end{thm}
	\end{screen}
	
	\begin{prf}
		$\Rightarrow$については定理\ref{prp:equivalent_condition_of_independence}より従う.
		また任意の$E \in \borel{\R}$に対して
		\begin{align}
			X^{-1}(E) =
			\begin{cases}
				\emptyset & (0,1 \notin E) \\
				A & (0 \notin E,\ 1 \in E) \\
				A^c & (0 \in E,\ 1 \notin E) \\
				\Omega & (0, 1 \in E)
			\end{cases}
		\end{align}
		が成り立つから,
		\begin{align}
			\mu(A \cap B) = \mu(A)\mu(B)
		\end{align}
		の下で
		\begin{align}
			\mu\left( X^{-1}(E) \cap B \right) = \mu\left( X^{-1}(E) \right)\mu(B)
			\quad \left( \forall E \in \borel{\R} \right)
		\end{align}
		が従い
		\footnote{
			$X^{-1}(E) = A^c$の場合
			\begin{align}
				\mu(A^c \cap B) = \mu(B) - \mu(A)\mu(B)  = (1 - \mu(A))\mu(B) = \mu(A^c)\mu(B)
			\end{align}
			が成り立つ.
		}
		,定理\ref{prp:equivalent_condition_of_independence}により$\Leftarrow$が成り立つ.
		\QED
	\end{prf}

\end{comment}
\chapter{停止時刻}
	\section{停止時刻}
	確率空間を$(\Omega,\mathcal{F},\mu)$とする.集合$I$によって確率過程の時点を表現し,
	以降でこれは$[0,\infty)$や$\{0,,1,\cdots,n\}$など実数の区間や高々可算集合を指すものと考え,
	$I$が高々可算集合の場合は離散位相,$\R$の区間の場合は相対位相を考える.また扱う確率変数は全て実数値で考える.
	\begin{itembox}[l]{}
		\begin{dfn}[フィルトレーション]
			$\mathcal{F}$の部分$\sigma$-加法族の部分系$\Set{\mathcal{F}_\alpha}{\alpha \in I}$
			がフィルトレーション(filtration)であるとは,任意の$\alpha,\beta \in I$に対して$\alpha \leq \beta$ならば
			$\mathcal{F}_\alpha \subset \mathcal{F}_\beta$の関係をもつことで定義する.
		\end{dfn}
	\end{itembox}
	
	\begin{itembox}[l]{}
		\begin{dfn}[停止時刻]
			$\Omega$上の関数で次を満たすものを($(\mathcal{F}_\alpha)$-)停止時刻(stopping time)という:
			\begin{align}
				\tau:\Omega \longrightarrow I\quad \mathrm{s.t.}\quad \forall \alpha \in I,\ \{ \tau \leq \alpha \} \in \mathcal{F}_\alpha.
			\end{align}
		\end{dfn}
	\end{itembox}
	
	\begin{itembox}[l]{}
		\begin{rem}[停止時刻は可測]
			上で定義した$\tau$は可測$\mathcal{F}/\borel{I}$である.
		\end{rem}
	\end{itembox}
	\begin{prf}\mbox{}
		\begin{description}
			\item[$I$が$\R$の区間である場合]
				任意の$\alpha \in I$に対して$I_\alpha \coloneqq (-\infty,\alpha) \cap I$は$I$における(相対の)開集合であり
				$\tau^{-1}(I_\alpha) = \{ \tau \leq \alpha \} \in \mathcal{F}_\alpha \subset \mathcal{F}$が成り立つ.
				つまり
				\begin{align}
					\Set{I_\alpha}{\alpha \in I} \subset \Set{A \in \borel{I}}{\tau^{-1}(A) \in \mathcal{F}}
					\label{eq:stopping_time_mble}
				\end{align}
				が成り立ち,左辺の$I_\alpha$の形の全体は$\borel{I}$を生成するから$\tau$の可測性が証明された.
				
			\item[$I$が高々可算集合である場合]
				先ず$\alpha \in I$に対して$\{ \tau < \alpha \}$が$\mathcal{F}_\alpha$に属することを示す.
				$\alpha$に対して直前の元$\beta \in I$が存在するか$\alpha$が$I$の最小限である場合,前者なら$\{ \tau < \alpha \} = \{ \tau \leq \beta \}$
				となり後者なら$\{ \tau < \alpha \} = \emptyset$となるからどちらも$\mathcal{F}_\alpha$に属する.
				そうでない場合は$\alpha - 1/n < x < \alpha$を満たす点列$x_n \in I\ (n=1,2,3,\cdots)$を取れば,
				$\{ \tau < \alpha \} = \cap_{n=1}^{\infty}\{ \tau \leq \alpha - 1/n \}$
				により$\{ \tau < \alpha \} \in \mathcal{F}_\alpha$が判る.以上の準備の下で
				任意の$\alpha \in I$に対して$\tau^{-1}(\{\alpha\}) = \{ \tau \leq \alpha \} - \{ \tau < \alpha \} \in \mathcal{F}_\alpha$が成り立ち,
				更に可算集合$I$には離散位相が入っているから任意の$A \in \borel{I}$は
				一点集合の可算和で表現できて,$\tau^{-1}(A) \in \mathcal{F}$であると証明された.		
		\end{description}
		\QED
	\end{prf}
	
	\begin{itembox}[l]{}
		\begin{dfn}[停止時刻の再定義]
			今$\tau$の終集合は$I$であるが,$I \rightarrow \R$の恒等写像$i$を用いて$\tau^* \coloneqq i \circ \tau$とすれば,
			\begin{align}
				\borel{I}=\Set{A \cap I}{A \in \borel{\R}} = \Set{i^{-1}(A)}{A \in \borel{\R}}
			\end{align}
			により($i$が可測$\borel{I}/\borel{\R}$であるから)合成写像$\tau^*$は可測$\mathcal{F}/\borel{\R}$となる.以降は
			この$\tau^*$を停止時刻$\tau$と表記して扱うことにする.
		\end{dfn}
	\end{itembox}
	
	定数関数は停止時刻となる.$\tau$が$\Omega$上の定数関数なら$(\tau \leq \alpha)$は空集合か全体集合にしかならないからである.
	また$\sigma,\tau$を$I$に値を取る停止時刻とすると
	$\sigma \vee \tau$と$\sigma \wedge \tau$も停止時刻となる.実際
	\begin{align}
		\begin{cases}
			\{ \sigma \wedge \tau \leq \alpha \} = \{ \sigma \leq \alpha \} \cup \{ \tau \leq \alpha \}, \\
			\{ \sigma \vee \tau \leq \alpha \} = \{ \sigma \leq \alpha \} \cap \{ \tau \leq \alpha \}
		\end{cases}
		\quad ,(\forall \alpha \in I)
	\end{align}
	が成り立つからである.
	
	\begin{itembox}[l]{}
		\begin{dfn}[停止時刻の前に決まっている事象系]
			$\tau$を$I$に値を取る停止時刻とする.$\tau$に対し次の集合系を定義する.
			\begin{align}
				\mathcal{F}_\tau \coloneqq \Set{A \in \mathcal{F}}{\{ \tau \leq \alpha \} \cap A \in \mathcal{F}_\alpha,\ \forall \alpha \in I}.
			\end{align}
		\end{dfn}
	\end{itembox}
	\begin{itembox}[l]{}
		\begin{prp}[停止時刻の性質]
			$I \subset \R$に値を取る停止時刻$\sigma, \tau$に対し次が成り立つ.
			\begin{description}
				\item[(1)] $\mathcal{F}_\tau$は$\sigma$-加法族である.
				\item[(2)] 或る$\alpha \in I$に対して$\tau(\omega) = \alpha\ (\forall \omega \in \Omega)$なら$\mathcal{F}_\alpha = \mathcal{F}_\tau$.
				\item[(3)] $\sigma(\omega) \leq \tau(\omega)\ (\forall \omega \in \Omega)$ならば$\mathcal{F}_\sigma \subset \mathcal{F}_\tau$.
				\item[(4)] $\mathcal{F}_{\sigma \wedge \tau} = \mathcal{F}_\sigma \cap \mathcal{F}_\tau$.
				\item[(5)] $\mathcal{F}_{\sigma \vee \tau} = \mathcal{F}_\sigma \vee \mathcal{F}_\tau$.
			\end{description}
		\end{prp}
	\end{itembox}
	
	\begin{prf}\mbox{}
		\begin{description}
			\item[(1)] 停止時刻の定義より$\Omega \in \mathcal{F}_\tau$である.また$A \in \mathcal{F}_\tau$なら
				$A^c \cap \{ \tau \leq \alpha \} = \{ \tau \leq \alpha \} - A \cap \{ \tau \leq \alpha \} \in \mathcal{F}_\alpha$より
				$A^c \in \mathcal{F}_\tau$となる.可算個の$A_n \in \mathcal{F}_\tau$については
				$\cup_{n=1}^{\infty} A_n \cap \{ \tau \leq \alpha \} = \cup_{n=1}^{\infty} \left( A_n \cap \{ \tau \leq \alpha \} \right) \in \mathcal{F}_\alpha$
				により$\cup_{n=1}^{\infty} A_n \in \mathcal{F}_\tau$が成り立つ.
			
			\item[(2)] $A \in \mathcal{F}_\alpha$なら任意の$\beta \in I$に対して
				\begin{align}
					A \cap \{ \tau \leq \beta \} =
					\begin{cases}
						A & \alpha \leq \beta \\
						\emptyset & \alpha > \beta
					\end{cases}
				\end{align}
				が成り立つから,いずれの場合も$A \in \mathcal{F}_\beta$となり$A \subset \mathcal{F}_\tau$が成り立つ.
				逆に$A \in \mathcal{F}_\tau$のとき,$A = A \cap \{ \tau \leq \alpha \} \in \mathcal{F}_\alpha$
				が成り立ち$\mathcal{F}_\alpha = \mathcal{F}_\tau$が示された.
				
			\item[(3)] $A \in \mathcal{F}_\sigma$なら任意の$\alpha \in I$に対して
				\begin{align}
					A \cap \{ \tau \leq \alpha \} = A \cap \{ \sigma \leq \alpha \} \cap \{ \tau \leq \alpha \} \in \mathcal{F}_\alpha
				\end{align}
				が成り立つから$A \in \mathcal{F}_\tau$となる.
			
			\item[(4)] $\sigma \wedge \tau$が停止時刻であることと(3)より
				$\mathcal{F}_{\sigma \wedge \tau} \subset \mathcal{F}_\sigma$と$\mathcal{F}_{\sigma \wedge \tau} \subset \mathcal{F}_\tau$が判る.
				また$A \in \mathcal{F}_\sigma \cap \mathcal{F}_\tau$に対し
				\begin{align}
					A \cap \{ \sigma \wedge \tau \leq \alpha \} 
					= \left( A \cap \{ \sigma \leq \alpha \} \right) \cup \left( A \cap \{ \tau \leq \alpha \} \right) \in \mathcal{F}_\alpha \quad (\forall \alpha \in I)
				\end{align}
				より$A \in \mathcal{F}_{\sigma \wedge \tau}$も成り立つ.
			
			\item[(5)] 
				先ず$\sigma \vee \tau$が停止時刻であることと(3)より
				$\mathcal{F}_\sigma \subset \mathcal{F}_{\sigma \wedge \tau}$と$\mathcal{F}_\tau \subset \mathcal{F}_{\sigma \wedge \tau}$が判る.
				逆に$A \in \mathcal{F}_{\sigma \wedge \tau}$に対して
				
		\end{description}
	\end{prf}
	
	\begin{itembox}[l]{}
		\begin{prp}[停止時刻と条件付き期待値]
			$X \in \Lp{1}{\mathcal{F},\operatorname{P}}$と
			$I$に値を取る停止時刻$\sigma, \tau$に対し以下が成立する.
			\begin{description}
				\item[(1)] $\cexp{\defunc_{(\sigma > \tau)} X}{\mathcal{F}_\tau} = \cexp{\defunc_{(\sigma > \tau)} X}{\mathcal{F}_{\sigma \wedge \tau}}$.
				\item[(2)] $\cexp{\defunc_{(\sigma \geq \tau)} X}{\mathcal{F}_\tau} = \cexp{\defunc_{(\sigma \geq \tau)} X}{\mathcal{F}_{\sigma \wedge \tau}}$.
				\item[(3)] $\cexp{\cexp{X}{\mathcal{F}_\tau}}{\mathcal{F}_\sigma} = \cexp{X}{\mathcal{F}_{\sigma \wedge \tau}}$.
			\end{description}
			\label{prp:stopping_time_and_conditional_expectation}
		\end{prp}
	\end{itembox}
	
	\begin{prf}\mbox{}
		\begin{description}
			\item[第一段] $\defunc_{(\sigma > \tau)}$が可測$\mathcal{F}_{\sigma \wedge \tau}/\borel{\R}$であることを示す.
				$A \in \borel{\R}$に対し
				\begin{align}
					\defunc_{(\sigma > \tau)}^{-1}(A) = 
					\begin{cases}
						\Omega & (0 \in A,\ 1 \in A) \\
						(\sigma > \tau) & (0 \notin A,\ 1 \in A) \\
						(\sigma > \tau)^c & (0 \in A,\ 1 \notin A) \\
						\emptyset & (0 \notin A,\ 1 \notin A)
					\end{cases}
				\end{align}
				と表現できるから,示すことは任意の$\alpha \in I$に対して
				\begin{align}
					(\sigma > \tau) \cap (\sigma \wedge \tau \leq \alpha) \in \mathcal{F}_\alpha
				\end{align}
				が成立することである.これが示されれば
				\begin{align}
					(\sigma > \tau)^c \cap (\sigma \wedge \tau \leq \alpha)
					= (\sigma \wedge \tau \leq \alpha) \backslash \left[(\sigma > \tau) \cap (\sigma \wedge \tau \leq \alpha)\right] \in \mathcal{F}_\alpha
				\end{align}
				も成り立ち,更に$(\sigma > \tau)^c = (\sigma \leq \tau)$であることと$\sigma,\tau$の対等性により$\defunc_{(\sigma \geq \tau)}$もまた
				可測$\mathcal{F}_{\sigma \wedge \tau}/\borel{\R}$であることが判る.目的の式は次が成り立つことにより示される.
				\begin{align}
					(\sigma > \tau) \cap (\sigma \wedge \tau \leq \alpha)
					&= (\sigma > \tau) \cap (\sigma \leq \alpha) + (\sigma > \tau) \cap (\sigma > \alpha) \cap (\tau \leq \alpha) \\
					&= \left[\bigcup_{\substack{\beta \in \Q \cap I \\ \beta \leq \alpha}} (\sigma > \beta)\cap(\tau \leq \beta)\right]\cap(\sigma \leq \alpha) + (\sigma > \alpha) \cap (\tau \leq \alpha)
					\label{eq:stopping_time_conditional_expectation_1} \\
					&\in \mathcal{F}_\alpha.
				\end{align}
			
			\item[第二段] 一般の実確率変数$Y$と停止時刻$\tau$に対して
				\begin{itemize}
					\item $Y$が可測$\mathcal{F}_\tau/\borel{\R}$$\quad \Leftrightarrow \quad$任意の$\alpha \in I$に対し$Y \defunc_{\tau \leq \alpha}$が可測$\mathcal{F}_\alpha/\borel{\R}$
				\end{itemize}
				が成り立つことを示す.
				\begin{description}
					\item[$\Rightarrow$について]
						$Y$の単関数近似列$(Y_n)_{n=1}^{\infty}$の一つ一つは$Y_n = \sum_{j=1}^{N_n}a_{j,n}\defunc_{A_{j,n}}\ (A_{j,n} \in \mathcal{F}_\tau)$
						の形で表現できる.$\alpha \in I$と$A \in \mathcal{F}_\tau$の指示関数$\defunc_{A}$に対し
						\begin{align}
							\left( \defunc_A\defunc_{(\tau \leq \alpha)} \right)^{-1}(E) =
							\begin{cases}
								\Omega & (0 \in E,\ 1 \in E) \\
								A \cap (\tau \leq \alpha) & (0 \notin E,\ 1 \in E) \\
								[A \cap (\tau \leq \alpha)]^c & (0 \in E,\ 1 \notin E) \\
								\emptyset & (0 \notin E,\ 1 \notin E)
							\end{cases}
							\quad (\forall E \in \borel{\R})
						\end{align}
						となり,$A \cap (\tau \leq \alpha) \in \mathcal{F}_\alpha$より$\defunc_A\defunc_{(\tau \leq \alpha)}$が
						可測$\mathcal{F}_\alpha/\borel{\R}$であると判る.$(Y_n)_{n=1}^{\infty}$は$Y$に各点収束していくから
						$Y$も可測$\mathcal{F}_\alpha/\borel{\R}$となり,$\alpha \in I$の任意性から''$\Rightarrow$''が示された.
						
					\item[$\Leftarrow$について]
						任意の$E \in \borel{\R}$に対して
						\begin{align}
							\left\{\ \omega \in \Omega\quad |\quad Y(\omega)\defunc_{(\tau \leq \alpha)}(\omega) \in E\ \right\}
							= \begin{cases}
								Y^{-1}(E) \cap (\tau \leq \alpha) & (0 \notin E) \\
								Y^{-1}(E) \cap (\tau \leq \alpha) + (\tau \leq \alpha)^c & (0 \in E)
							\end{cases}
						\end{align}
						がいずれも$\mathcal{F}_\alpha$に属する.特に下段について$(\tau \leq \alpha)^c \in \mathcal{F}_\alpha$
						より$Y^{-1}(E) \cap (\tau \leq \alpha) \in \mathcal{F}_\alpha$となるから,結局
						$Y^{-1}(E) \cap (\tau \leq \alpha) \in \mathcal{F}_\alpha \ (\forall E \in \borel{\R})$が成り立つ.
						$\alpha \in I$の任意性から$Y^{-1}(E) \in \mathcal{F}_\tau\ (\forall E \in \borel{\R})$が示された.
				\end{description}
			
			\item[第三段]
				(1)の式を示す.第一段と性質$\tilde{\mathrm{C}}$5より
				\begin{align}
					\cexp{\defunc_{(\sigma > \tau)}X}{\mathcal{F}_\tau} = \defunc_{(\sigma > \tau)} \cexp{X}{\mathcal{F}_\tau}
				\end{align}
				が成り立つから,あとは右辺が(関数とみて)可測$\mathcal{F}_{\sigma \wedge \tau}/\borel{\R}$であればよく,このためには
				第二段の結果より任意の$\alpha \in I$に対して$\cexp{X}{\mathcal{F}_\tau}\defunc_{(\sigma > \tau)}\defunc_{(\sigma \wedge \tau \leq \alpha)}$が
				可測$\mathcal{F}_\alpha/\borel{\R}$であることを示せばよい.
				式(\refeq{eq:stopping_time_conditional_expectation_1})を使えば
				\begin{align}
					\defunc_{(\sigma > \tau)}\defunc_{(\sigma \wedge \tau \leq \alpha)}
					= \sup{\substack{\beta \in \Q \cap I \\ \beta \leq \alpha}}{\defunc_{(\sigma > \beta)}\defunc_{(\tau \leq \beta)}\defunc_{(\sigma \leq \alpha)}}
						+ \defunc_{(\sigma > \alpha)} \defunc_{(\tau \leq \alpha)}
				\end{align}
				が成り立つ.$\beta \leq \alpha$ならば,$\cexp{X}{\mathcal{F}_\tau}$が可測$\mathcal{F}_\tau/\borel{\R}$であることと第二段の結果より
				$\cexp{X}{\mathcal{F}_\tau}\defunc_{(\tau \leq \beta)}$が可測$\mathcal{F}_\beta/\borel{\R}$すなわち可測$\mathcal{F}_\alpha/\borel{\R}$
				となるから,これで$\cexp{X}{\mathcal{F}_\tau}\defunc_{(\sigma > \tau)}\defunc_{(\sigma \wedge \tau \leq \alpha)}$が可測$\mathcal{F}_\alpha/\borel{\R}$
				であると判り$\defunc_{(\sigma > \tau)} \cexp{X}{\mathcal{F}_\tau}$が可測$\mathcal{F}_{\sigma \wedge \tau}/\borel{\R}$であることが示された.
				以上で
				\begin{align}
					\cexp{\cexp{\defunc_{(\sigma > \tau)}X}{\mathcal{F}_\tau}}{\mathcal{F}_{\sigma \wedge \tau}}
					= \cexp{\defunc_{(\sigma > \tau)} \cexp{X}{\mathcal{F}_\tau}}{\mathcal{F}_{\sigma \wedge \tau}}
					= \defunc_{(\sigma > \tau)} \cexp{X}{\mathcal{F}_\tau}
					= \cexp{\defunc_{(\sigma > \tau)}X}{\mathcal{F}_\tau}
				\end{align}
				が成り立ち,
				\begin{align}
					\cexp{\cexp{\defunc_{(\sigma > \tau)}X}{\mathcal{F}_\tau}}{\mathcal{F}_{\sigma \wedge \tau}}
					= \defunc_{(\sigma > \tau)} \cexp{X}{\mathcal{F}_{\sigma \wedge \tau}}
					= \cexp{\defunc_{(\sigma > \tau)}X}{\mathcal{F}_{\sigma \wedge \tau}}
				\end{align}
				と併せて(1)の式を得る.(2)の式も以上と同じ理由で成り立つ.
				
			\item[第四段]
				(3)の式を示す.
				\begin{align}
					\cexp{\cexp{X}{\mathcal{F}_\tau}}{\mathcal{F}_\sigma}
					&= \cexp{\cexp{X}{\mathcal{F}_\tau}\defunc_{(\sigma > \tau)}}{\mathcal{F}_\sigma}
						+ \cexp{\cexp{X}{\mathcal{F}_\tau}\defunc_{(\sigma \leq \tau)}}{\mathcal{F}_\sigma} \\
					&= \cexp{\cexp{X}{\mathcal{F}_{\sigma \wedge \tau}}\defunc_{(\sigma > \tau)}}{\mathcal{F}_\sigma}
						+ \cexp{\cexp{X}{\mathcal{F}_\tau}\defunc_{(\sigma \leq \tau)}}{\mathcal{F}_\sigma} && (\scriptsize\because\mbox{(1)}) \\
					&= \cexp{X}{\mathcal{F}_{\sigma \wedge \tau}}\defunc_{(\sigma > \tau)}
						+ \cexp{\cexp{X}{\mathcal{F}_\tau}\defunc_{(\sigma \leq \tau)}}{\mathcal{F}_{\sigma \wedge \tau}} && (\scriptsize\because\mbox{(2)}) \\
					&= \cexp{X}{\mathcal{F}_{\sigma \wedge \tau}}\defunc_{(\sigma > \tau)} + \cexp{X}{\mathcal{F}_{\sigma \wedge \tau}}\defunc_{(\sigma \leq \tau)} \\
					&= \cexp{X}{\mathcal{F}_{\sigma \wedge \tau}}.
				\end{align}
				(2)式を使った箇所では$X$を$\cexp{X}{\mathcal{F}_\tau}$に置き換え$\tau$と$\sigma$を入れ替えて適用した.
		\end{description}
		\QED
	\end{prf}
	
	\begin{itembox}[l]{}
		\begin{thm}[停止時刻との合成写像の可測性]
			$I = [0,T]$,フィルトレーションを$(\mathcal{F}_t)_{t \in I}$,$\tau$を停止時刻とし,
			$M$を$I \times \Omega$上の$\R$値関数とする.全ての$\omega \in \Omega$に対し
			$I \ni t \longmapsto M(t,\omega)$が右連続でかつ$(\mathcal{F}_t)$-適合ならば,
			写像$\omega \longmapsto M(\tau(\omega),\omega)$は可測$\mathcal{F}_\tau/\borel{\R}$となる.
			\label{thm:measurability_of_stopping_time}
		\end{thm}
	\end{itembox}
	
	\begin{prf}
		任意に$t \in I$を取り$t_j^n \coloneqq jt/2^n\ (j=0,1,\cdots,2^n,\ n=1,2,\cdots)$とおくと,
		右連続性により任意の$s \in [0,t]$に対して
		\begin{align}
			M(s,\omega) = \lim_{n \to \infty} \sum_{j=1}^{2^n} M_{t_j^n}(\omega) \defunc_{[t_{j-1}^n,t_j^n)}(s) + M_t(\omega) \defunc_{\{t\}}(s) \quad (\omega \in \Omega)
			\label{eq:stopping_time_measurability}
		\end{align}
		が成り立つ.右辺は各$n$で可測$\borel{[0,t]} \times \mathcal{F}_t/\borel{\R}$であるから
		$M$も可測$\borel{[0,t]} \times \mathcal{F}_t/\borel{\R}$となる.($t$の任意性から$M$は$(\mathcal{F}_t)$-発展的可測である.)
		一方停止時刻$\tau$について,$\tau \wedge t$が可測$\mathcal{F}_t/\borel{\R}$であるから
		\begin{align}
			\Omega \ni \omega \longmapsto (\tau(\omega) \wedge t, \omega) \in [0,t] \times \Omega
		\end{align}
		は可測$\mathcal{F}_t/\borel{[0,t]} \times \mathcal{F}_t$である.従って合成写像
		\begin{align}
			\Omega \ni \omega \longmapsto M(\tau(\omega) \wedge t,\omega) \in \R
		\end{align}
		は可測$\mathcal{F}_t/\borel{\R}$となる.任意の$A \in \borel{\R}$に対して
		\begin{align}
			\Set{\omega \in \Omega}{M(\tau(\omega),\omega) \in A} \cap \left\{ \tau \leq t \right\}
			= \Set{\omega \in \Omega}{M(\tau(\omega) \wedge t,\omega) \in A} \cap \left\{ \tau \leq t \right\}
			\in \mathcal{F}_t
		\end{align}
		が成り立つから,写像$\omega \longmapsto M(\tau(\omega),\omega)$は可測$\mathcal{F}_\tau/\borel{\R}$である
		\footnote{
			写像$\omega \longmapsto M(\tau(\omega),\omega)$が可測$\mathcal{F}/\borel{\R}$となっていないことにはこの結論が従わない.
			この点を確認すれば,式(\refeq{eq:stopping_time_measurability})より$M$が可測$\borel{I} \times \mathcal{F}/\borel{\R}$
			であることは慥かであるから,$\omega \longmapsto (\tau(\omega),\omega)$が可測$\mathcal{F}/\borel{I}\times\mathcal{F}$であることと
			併せて写像$\omega \longmapsto M(\tau(\omega),\omega)$が可測$\mathcal{F}/\borel{\R}$であることが判明する.
		}.
		\QED
	\end{prf}
	
	\begin{itembox}[l]{}
		\begin{thm}[閉集合と停止時刻]
			$I = [0,T] \subset \R$,$(E,\rho)$を距離空間,$(X_t)_{t \in I}$を$E$値確率変数の族とし,
			$\mathcal{F}_0$が$\mu$-零集合を全て含んでいると仮定する.
			$\mu$-零集合$N$を除いて$I \ni t \longmapsto X_t(\omega)$が右連続で,
			かつ$(X_t)_{t \in I}$が$(\mathcal{F}_t)$-適合であるなら,任意の閉集合$F \subset E$に対し
			\begin{align}
				\tau(\omega) \coloneqq
				\begin{cases}
					0 & (\omega \in N) \\
					\inf{}{\Set{t \in I}{X_t(\omega) \in F}} \wedge T & (\omega \in \Omega \backslash N)
				\end{cases}
			\end{align}
			として$\tau:\Omega \rightarrow \R$を定めれば$\tau$は停止時刻となる.
			また$N'\ (N \subset N')$を除いて$I \ni t \longmapsto X_t(\omega)$が連続であるなら
			\begin{align}
				X_{t \wedge \tau(\omega)}(\omega) \in \{ X_0(\omega) \} \cup F^{ic} \quad (\forall \omega \in N',\ t \in I)
			\end{align}
			が成り立つ.ただし$F^i$は$F$の内核を表し$F^{ic}$は$F^i$の補集合を表す.
			\label{thm:closed_set_stopping_time}
		\end{thm}
	\end{itembox}
	確率空間が完備である場合は
	\begin{align}
		\tau(\omega) \coloneqq \inf{}{\Set{t \in I}{X_t(\omega) \in F}} \wedge T
		\quad (\forall \omega \in \Omega)
	\end{align}
	として$\tau$は停止時刻となる.実際任意の$t \in I$に対して,
	\begin{align}
		\{\tau \leq t\} = \{\tau \leq t\} \cap N + \Set{\omega \in \Omega \backslash N}{\tau(\omega) \leq t}
	\end{align}
	の右辺第一項は完備性より$\mu$-零集合,第二項は以下で$\mathcal{F}_t$に属すると証明される.

	\begin{prf}
		\begin{align}
			D_t(\omega) \coloneqq 
			\begin{cases}
				1 & (\omega \in N) \\
				\inf{}{\Set{\rho(X_r(\omega),F)}{r \in ([0,t] \cap \Q) \cup \{t\}}} & (\omega \in \Omega \backslash N)
			\end{cases}
		\end{align}
		とおけば
		\footnote{
			$\rho(X_r(\omega),F) = \inf{y \in F}{\rho(X_r(\omega),y)}$である.
		},
		$D_t$は可測$\mathcal{F}_t/\borel{\R}$となる
		\footnote{
			写像$E \ni x \longmapsto \rho(x,F) \in \R$は連続であるから,合成写像
			\begin{align}
				\Omega \ni \omega \longmapsto \rho(X_t(\omega),F)
			\end{align}
			は可測$\mathcal{F}_t/\borel{\R}$となる.任意の$\lambda \in \R$に対し
			\begin{align}
				\left\{ \inf{}{\Set{\rho(X_r,F)}{r \in ([0,t] \cap \Q) \cup \{t\}}} \geq \lambda \right\}
				= \bigcap_{r \in ([0,t] \cap \Q) \cup \{t\}} \left\{ \rho(X_r,F) \geq \lambda \right\}
			\end{align}
			となり右辺の各集合は$\in \mathcal{F}_t$であるから
			写像$\Omega \ni \omega \longmapsto \inf{}{\Set{\rho(X_r(\omega),F)}{r \in ([0,t] \cap \Q) \cup \{t\}}}$
			も可測$\mathcal{F}_t/\borel{\R}$となる.任意の$A \in \borel{\R}$に対し
			\begin{align}
				D_t^{-1}(A) = 
				\begin{cases}
					N \cup \left\{ \inf{}{\Set{\rho(X_r,F)}{r \in ([0,t] \cap \Q) \cup \{t\}}} \in A \right\} & (1 \in A) \\
					\left\{ \inf{}{\Set{\rho(X_r,F)}{r \in ([0,t] \cap \Q) \cup \{t\}}} \in A \right\} & (1 \notin A)
				\end{cases}
			\end{align}
			となるが,$N \in \mathcal{F}_0$であるから$D_t$もまた可測$\mathcal{F}_t/\borel{\R}$となる.
		}.
		ここでは任意の$t \in [0,T)$に対して
		\begin{align}
			\Set{\omega \in \Omega \backslash N}{\tau(\omega) \leq t} = \Set{\omega \in \Omega \backslash N}{D_t(\omega) = 0}
			\label{eq:closed_set_stopping_time_1}
		\end{align}
		が成り立つことを示す.実際これが示されれば任意の$t \in I$に対し
		\begin{align}
			\{\tau \leq t\} =
			\begin{cases}
 				\Omega & (t = T) \\
				N + \Set{\omega \in \Omega \backslash N}{D_t(\omega) = 0} & (t < T)
 			\end{cases}
		\end{align}
		となるから$\tau$は停止時刻となる.式(\refeq{eq:closed_set_stopping_time_1})が成立することを示すには
		包含関係$\subset,\supset$のそれぞれを満たすことを確認すればよい.
		\begin{description}
			\item[$\subset$について]
				任意に$t \in [0,T)$を固定する.$\tau(\omega) \leq t$となる$\omega \in \Omega \backslash N$に対し
				$s \coloneqq \tau(\omega)$とおくと$X_s(\omega) \in F$となる.もし$X_s(\omega) \notin F$であるとすれば,
				$F$が閉集合であることと$s \longmapsto X_s(\omega)$の右連続性から
				或る$\delta > 0$が存在し,任意の$0 < h < \delta$に対して
				$X_{s+h}(\omega) \notin F$となり$s = \tau(\omega)$であることに矛盾する.
				今$\rho(X_s(\omega),F) = 0$が示されたが,$D_t(\omega) = 0$も成り立っている.
				実際もし$D_t(\omega) > 0$であるとすれば,$a \coloneqq D_t(\omega)$に対して
				\begin{align}
					\rho(X_s(\omega), X_r(\omega)) < a/2
				\end{align}
				を満たす$r \in ((s,t] \cap \Q) \cup \{t\}$が存在するから
				\begin{align}
					\rho(X_s(\omega),F) \geq \rho(X_r(\omega),F) - \rho(X_s(\omega), X_r(\omega)) > a - a/2 = a/2
				\end{align}
				となり矛盾が生じてしまう.
			
			\item[$\supset$について]
				任意に$t \in [0,T)$を固定する.$D_t(\omega) = 0$となる$\omega \in \Omega \backslash N$について
				\begin{align}
					\rho(X_{s_n}(\omega),F) < 1/n
				\end{align}
				となるように$s_n \in ([0,t] \cap \Q) \cup \{t\}$を取ることができる.$(s_n)_{n=1}^{\infty}$は
				$[0,t]$に集積点$s$を持ち,$F$が閉であるから$X_s(\omega) \in F$となる.従って
				$\tau(\omega) \leq s \leq t$が成り立つ.
		\end{description}
		以上で$\tau$が停止時刻であることが示されたから,次に定理の後半の主張を示す.
		$N'$を除いて$I \ni t \longmapsto X_t(\omega)$が連続である場合,
		$\tau(\omega) = 0$なら$X_0(\omega) \in F$である.実際
		$F$が閉集合であることとパスの連続性により$0$のある近傍内においても$X_t(\omega) \notin F$となる.
		従って$X_{t \wedge \tau(\omega)}(\omega) \in F^{ic}$であるとは限らない.
		$\tau(\omega) > 0$のとき,$t < \tau(\omega)\ (\omega \in \Omega \backslash N')$に対しては$X_t(\omega) \in F^c$が成り立っているから,示せばよいのは
		\begin{align}
			X_{\tau(\omega)}(\omega) \in F^{ic}
			\label{eq:closed_set_stopping_time_2}
		\end{align}
		が成り立つことである.$s = \tau(\omega)$とおく.もし$X_s(\omega) \in F^i$であるとすれば
		連続性から或る$\delta$が存在し,$0 < h < \delta$を満たす任意の$h$に対し
		\begin{align}
			X_{s - h}(\omega) \in F^i
		\end{align}
		となるが,
		\begin{align}
			s > s - h \geq \inf{}{\Set{t \in I}{X_t(\omega) \in F}}
		\end{align}
		が従い矛盾が生じる.よって(\refeq{eq:closed_set_stopping_time_2})が示された.
		\QED
	\end{prf}
	
\chapter{マルチンゲール}
	\section{11/1}
	確率空間を$(\Omega,\mathcal{F},\mu)$とする.
	
	\begin{itembox}[l]{}
		\begin{thm}[Doobの不等式(1)]
			$I=\{0,1,\cdots,n\}$,
			$(\mathcal{F}_t)_{t \in I}$をフィルトレーション,
			$(M_t)_{t \in I}$を$\mathrm{L}^1$-劣マルチンゲールとし,
			$M^* \coloneqq \max{t \in I}{M_t}$とおく.$(M_t)_{t \in I}$が非負値なら次が成り立つ:
			\begin{description}
				\item[(1)] 任意の$\lambda > 0$に対して
					\begin{align}
						\mu(M^* \geq \lambda) \leq \frac{1}{\lambda} \int_{\left\{\ M^* \geq \lambda\ \right\}} M_n(\omega)\ \mu(d\omega)
						\leq \frac{1}{\lambda} \Norm{M_n}{\mathscr{L}^1}.
					\end{align}
				\item[(2)] 任意の$p > 1$に対して$M_t\ (\forall t \in I)$が$p$乗可積分なら
					\begin{align}
						\Norm{M^*}{\mathscr{L}^p} \leq \frac{p}{p-1} \Norm{M_n}{\mathscr{L}^p}.
					\end{align}
			\end{description}
		\end{thm}
	\end{itembox}
	
	\begin{prf}
		\begin{align}
			\tau(\omega) \coloneqq \min{}{\Set{ i \in I}{M_i(\omega) \geq \lambda}} 
			\quad (\forall \omega \in \Omega)
		\end{align}
		とおけば$\tau$は$I$に値を取る停止時刻となる.ただし全ての$i \in I$で$M_i(\omega) < \lambda$となるような$\omega$については
		$\tau(\omega) = n$とする.実際停止時刻となることは
		\begin{align}
			\left\{\ \tau = i\ \right\} &= \bigcap_{j=0}^{i-1} \left\{\ M_j < \lambda\ \right\} \cap \left\{\ M_i \geq \lambda\ \right\} \in \mathcal{F}_i
			,\quad (i=0,1,\cdots,n-1), \\
			\left\{\ \tau = n\ \right\} &= \bigcap_{j=0}^{n-1} \left\{\ M_j < \lambda\ \right\} \in \mathcal{F}_n
		\end{align}
		により判る.任意抽出定理より
		\begin{align}
			\cexp{M_n}{\mathcal{F}_\tau} \geq M_{n \wedge \tau} = M_\tau \quad (\because \tau \leq n)
		\end{align}
		が成り立つから,期待値を取って
		\begin{align}
			\int_{\Omega} M_n(\omega)\ \mu(d\omega)
			&\geq \int_{\Omega} M_\tau(\omega)\ \mu(d\omega) \footnotemark \\
			&= \int_{\left\{\ M^* \geq \lambda\ \right\}} M_\tau(\omega)\ \mu(d\omega) 
				+ \int_{\left\{\ M^* < \lambda\ \right\}} M_\tau(\omega)\ \mu(d\omega) \\
			&\geq \lambda \mu( M^* \geq \lambda ) \footnotemark
				+ \int_{\left\{\ M^* < \lambda\ \right\}} M_n(\omega)\ \mu(d\omega) 
				&& (\scriptsize \because \mbox{$M^*(\omega) < \lambda$ならば$\tau(\omega) = n$である.})
		\end{align}
	\end{prf}
	\footnotetext{
		性質$\tilde{\mathrm{C}}2$より
		\begin{align}
			\int_{\Omega} M_n(\omega)\ \prob{d\omega} = \int_{\Omega} \cexp{M_n}{\mathcal{F}_\tau}(\omega)\ \mu(d\omega)
			\geq \int_{\Omega} M_\tau(\omega)\ \mu(d\omega)
		\end{align}
		が成り立つ.
	}
	が成り立つ.
	\footnotetext{
		最後の不等式は次の理由で成り立つ:
		\begin{align}
			M_\tau \defunc_{\{ M^* \geq \lambda \}}  = \sum_{i=0}^{n-1}M_i \defunc_{\{ \tau = i \}} + M_n \defunc_{\{ \tau = n \}}\defunc_{\{ M^* \geq \lambda \}} \geq \lambda.
		\end{align}
	}
	従って
	\begin{align}
		\lambda \mu( M^* \geq \lambda ) \leq 
		\int_{\left\{\ M^* \geq \lambda\ \right\}} M_n(\omega)\ \mu(d\omega) \leq \Norm{M_n}{\mathscr{L}^1} \label{Doob_ineq_1}
	\end{align}
	を得る.これは
	\begin{align}
		\mu( M^* > \lambda ) \leq \int_{\left\{\ M^* > \lambda\ \right\}} M_n(\omega)\ \mu(d\omega) \label{Doob_ineq_2}
	\end{align}
	としても成り立つ
	\footnote{
		式(\refeq{Doob_ineq_1})により任意の$n \in \N$で
		\begin{align}
			\mu( M^* \geq \lambda+1/n ) \leq \int_{\left\{\ M^* \geq \lambda+1/n\ \right\}} M_n(\omega)\ \mu(d\omega)
		\end{align}
		が成り立っているから,$n \longrightarrow \infty$とすればよい.
	}.
	次に(2)を示す.$K \in \N$とする.
	\begin{align}
		\Norm{M^* \wedge K}{\mathscr{L}^p}^p &= \int_{\Omega} \left|M^*(\omega) \wedge K\right|^p\ \mu(d\omega) \\
		&= p \int_{\Omega} \int_0^{M^*(\omega) \wedge K} t^{p-1}\ dt\ \mu(d\omega) \\
		&= p \int_{\Omega} \int_0^K t^{p-1} \defunc_{\left\{ M^*(\omega) > t \right\}}\ dt\ \mu(d\omega) \footnotemark \\
		&= p \int_0^K t^{p-1} \int_{\Omega} \defunc_{\left\{ M^*(\omega) > t \right\}}\ \mu(d\omega)\ dt & (\scriptsize\because \mbox{Fubiniの定理より}) \\
		&= p \int_0^K t^{p-1} \mu( M^* > t )\ dt \\
		&\leq p \int_0^K t^{p-2} \int_{\left\{\ M^* > t\ \right\}} M_n(\omega)\ \mu(d\omega) & (\scriptsize\because \mbox{式(\refeq{Doob_ineq_2})より}) \\
		&= p \int_\Omega M_n(\omega) \int_0^K t^{p-2} \defunc_{\left\{ M^*(\omega) > t \right\}}\ dt\ \mu(d\omega) \\
		&= \frac{p}{p-1} \int_\Omega M_n(\omega) \left| M^*(\omega) \wedge K \right|^{p-1}\ \mu(d\omega) \\
		&\leq \frac{p}{p-1} \Norm{M_n}{\mathscr{L}^p} \Norm{M^*(\omega) \wedge K}{\mathscr{L}^p}^{p-1} 
	\end{align}
	となるから,
	\begin{align}
		\Norm{M^* \wedge K}{\mathscr{L}^p} \leq \frac{p}{p-1} \Norm{M_n}{\mathscr{L}^p}
	\end{align}
	が成り立つ.$K \longrightarrow \infty$として単調収束定理より
	\begin{align}
		\Norm{M^*}{\mathscr{L}^p} \leq \frac{p}{p-1} \Norm{M_n}{\mathscr{L}^p}
	\end{align}
	を得る.
	\QED
	\footnotetext{
		写像$[0,K) \times \Omega \ni (t,\omega) \longmapsto \defunc_{\left\{ M^*(\omega) > t \right\}}$は可測$\borel{[0,K)}\times\mathcal{F}/\borel{\R}$である.
		実際,
		\begin{align}
			f(t,\omega) \coloneqq \defunc_{\left\{ M^*(\omega) > t \right\}},
			\quad f_n(t,\omega) \coloneqq \defunc_{\left\{ M^*(\omega) > (j+1)/2^n \right\}} \quad (t \in \left[ \tfrac{j}{2^n},\tfrac{j+1}{2^n} \right),\ j=0,1,\cdots,K2^n-1)
		\end{align}
		とおけば,任意の$A \in \borel{\R}$に対して
		\begin{align}
			f_n^{-1}(A) = \begin{cases}
				\emptyset & (0 \notin A,\ 1 \notin A) \\
				\bigcup_{j=0}^{K2^n-1} \left[ \tfrac{j}{2^n},\tfrac{j+1}{2^n} \right) \times \Set{\omega}{M^*(\omega) > \tfrac{j+1}{2^n}} & (0 \notin A,\ 1 \in A) \\
				\bigcup_{j=0}^{K2^n-1} \left[ \tfrac{j}{2^n},\tfrac{j+1}{2^n} \right) \times \Set{\omega}{M^*(\omega) \leq \tfrac{j+1}{2^n}} & (0 \in A,\ 1 \notin A) \\
				[0,n] \times \Omega & (0 \in A,\ 1 \in A)
			\end{cases}
		\end{align}
		が成り立つから$f_n$は可測$\borel{[0,K)}\times\mathcal{F}/\borel{\R}$である.また各点$(t,\omega) \in [0,K) \times \Omega$において
		\begin{align}
			f(t,\omega) - f_n(t,\omega) = \defunc_{\left\{ t < M^*(\omega) \leq (j+1)/2^n \right\}} \longrightarrow 0 \quad (n \longrightarrow \infty)
		\end{align}
		となり$f_n$は$f$に各点収束するから,可測性は保存され$f$も可測$\borel{[0,K)}\times\mathcal{F}/\borel{\R}$となる.
		$t^{p-1}$も2変数関数として$g(t,\omega) \coloneqq t^{p-1}\defunc_{\Omega}(\omega)$と見做せば可測$\borel{[0,K)}\times\mathcal{F}/\borel{\R}$で,
		よって$gf$に対しFubiniの定理を適用できる.
	}
	
	$I = [0,T] \subset \R\ (T > 0)$を考える.$t \longmapsto M_t$は右連続であるから$\sup{t \in I}{M_t}$は確率変数となる.これは
	\begin{align}
		\sup{t \in I}{M_t(\omega)} = \sup{n \in \N}{\max{j=0,1,\cdots,2^n}{M_{\tfrac{j}{2^n}T}(\omega)}} \quad (\forall \omega \in \Omega)
	\end{align}
	が成り立つからである.実際各点$\omega \in \Omega$で
	\begin{align}
		\alpha = \alpha(\omega) \coloneqq \sup{t \in I}{M_t(\omega)},
		\quad \beta = \beta(\omega) \coloneqq \sup{n \in \N}{\max{j=0,1,\cdots,2^n}{M_{\tfrac{j}{2^n}T}(\omega)}}
	\end{align}
	とおけば,$\alpha$の方が上限を取る範囲が広いから$\alpha \geq \beta$は成り立つ.
	だがもし$\alpha > \beta$とすれば,或る$s \in I$が存在して
	\begin{align}
		M_s(\omega) > \frac{\alpha + \beta}{2}
	\end{align}
	を満たすから,右連続性により$s$の近傍から$jT/2^n$の形の点を取ることができて
	\begin{align}
		(\beta \geq)\ M_{\tfrac{j}{2^n}T}(\omega) > \frac{\alpha + \beta}{2}
	\end{align}
	となりこれは矛盾である.
	
	\begin{itembox}[l]{}
		\begin{thm}[Doobの不等式(2)]
			$I=[0,T]$,$(\mathcal{F}_t)_{t \in I}$をフィルトレーション,
			$(M_t)_{t \in I}$を$\mathrm{L}^p$-劣マルチンゲールとし,
			$M^* \coloneqq \sup{t \in I}{M_t}$とおく.$(M_t)_{t \in I}$が非負値なら次が成り立つ:
			\begin{description}
				\item[(1)] 任意の$\lambda > 0$に対して
					\begin{align}
						\mu(M^* \geq \lambda) \leq \frac{1}{\lambda^p} \Norm{M_T}{\mathscr{L}^p}^p.
					\end{align}
				\item[(2)] $p > 1$なら
					\begin{align}
						\Norm{M^*}{\mathscr{L}^p} \leq \frac{p}{p-1} \Norm{M_T}{\mathscr{L}^p}.
					\end{align}
			\end{description}
		\label{thm:Doob_inequality_2}	
		\end{thm}
	\end{itembox}
	
	\begin{prf}
		\begin{align}
			D_n \coloneqq \Set{\tfrac{j}{2^n}T}{j=0,1,\cdots,2^n}
		\end{align}
		とおく.Jensenの不等式より,任意の$0 \leq s < t \leq T$に対して
		\begin{align}
			\cexp{M_t^p}{\mathcal{F}_s} \geq \cexp{M_t}{\mathcal{F}_s}^p \leq M_s^p
		\end{align}
		が成り立つ.従って$(M_t^p)_{t \in I}$は$\mathrm{L}^1$-劣マルチンゲールであり,前定理の結果を使えば
		\begin{align}
			\mu(\max{r \in D_n}{M_r^p} \geq \lambda^p) \leq \frac{1}{\lambda^p} \Norm{M_T}{\mathscr{L}^p}^p
		\end{align}
		が任意の$n \in \N$で成り立つ.非負性から$\max{r \in D_n}{M_r^p} = (\max{r \in D_n}{M_r})^p$となり
		\begin{align}
			\mu(\max{r \in D_n}{M_r} \geq \lambda) \leq \frac{1}{\lambda^p} \Norm{M_T}{\mathscr{L}^p}^p
		\end{align}
		と書き直すことができて,
		\begin{align}
			\mu(M^* \geq \lambda) 
			= \mu(\sup{n \in \N}{\max{r \in D_n}{M_r}} \geq \lambda)
			= \lim_{n \to \infty} \mu(\max{r \in D_n}{M_r} \geq \lambda) \leq \frac{1}{\lambda^p} \Norm{M_T}{\mathscr{L}^p}^p
		\end{align}
		が成り立つ.同じく前定理\footnote{$\mathrm{L}^p$-劣マルチンゲールなら$\mathrm{L}^1$-劣マルチンゲールであるから前定理の結果を適用できる.}を適用し,
		\begin{align}
			\Norm{\max{r \in D_n}{M_r}}{\mathscr{L}^p} \leq \frac{p}{p-1} \Norm{M_T}{\mathscr{L}^p}
		\end{align}
		を保って$n \longrightarrow \infty$とすれば単調収束定理より(2)を得る.
		\QED
	\end{prf}
	
	\begin{itembox}[l]{}
		\begin{thm}[停止時刻との合成写像の可測性]
			$I = [0,T]$,フィルトレーションを$(\mathcal{F}_t)_{t \in I}$,$\tau$を停止時刻とし,
			$M$を$I \times \Omega$上の$\R$値関数とする.$M$について,全ての$\omega \in \Omega$に対し
			$I \ni t \longmapsto M(t,\omega)$が右連続でかつ$(\mathcal{F}_t)$-適合ならば,
			写像$\omega \longmapsto M(\tau(\omega),\omega)$は可測$\mathcal{F}_\tau/\borel{\R}$となる.
			\label{thm:measurability_of_stopping_time}
		\end{thm}
	\end{itembox}
	
	\begin{prf}
		任意に$t \in I$を取り$t_j^n \coloneqq jt/2^n\ (j=0,1,\cdots,2^n,\ n=1,2,\cdots)$とおくと,
		右連続性により任意の$s \in [0,t]$に対して
		\begin{align}
			M(s,\omega) = \lim_{n \to \infty} \sum_{j=1}^{2^n} M_{t_j^n}(\omega) \defunc_{(t_{j-1}^n,t_j^n]}(s) \quad (\omega \in \Omega)
			\label{eq:stopping_time_measurability}
		\end{align}
		が成り立つ.右辺は各$n$で可測$\borel{[0,t]} \times \mathcal{F}_t/\borel{\R}$であるから
		$M$も可測$\borel{[0,t]} \times \mathcal{F}_t/\borel{\R}$となる.($t$の任意性から$M$は$(\mathcal{F}_t)$-発展的可測である.)
		一方停止時刻$\tau$について,$\tau \wedge t$が可測$\mathcal{F}_t/\borel{\R}$であるから
		\begin{align}
			\Omega \ni \omega \longmapsto (\tau(\omega) \wedge t, \omega) \in [0,t] \times \Omega
		\end{align}
		は可測$\mathcal{F}_t/\borel{[0,t]} \times \mathcal{F}_t$である.従って合成写像
		\begin{align}
			\Omega \ni \omega \longmapsto M(\tau(\omega) \wedge t,\omega) \in \R
		\end{align}
		は可測$\mathcal{F}_t/\borel{\R}$となる.任意の$A \in \borel{\R}$に対して
		\begin{align}
			\Set{\omega \in \Omega}{M(\tau(\omega),\omega) \in A} \cap \left\{ \tau \leq t \right\}
			= \Set{\omega \in \Omega}{M(\tau(\omega) \wedge t,\omega) \in A} \cap \left\{ \tau \leq t \right\}
			\in \mathcal{F}_t
		\end{align}
		が成り立つから,写像$\omega \longmapsto M(\tau(\omega),\omega)$は可測$\mathcal{F}_\tau/\borel{\R}$である
		\footnote{
			写像$\omega \longmapsto M(\tau(\omega),\omega)$が可測$\mathcal{F}/\borel{\R}$となっていないことにはこの結論が従わない.
			この点を確認すれば,式(\refeq{eq:stopping_time_measurability})より$M$が可測$\borel{I} \times \mathcal{F}/\borel{\R}$
			であることは慥かであるから,$\omega \longmapsto (\tau(\omega),\omega)$が可測$\mathcal{F}/\borel{I}\times\mathcal{F}$であることと
			併せて写像$\omega \longmapsto M(\tau(\omega),\omega)$が可測$\mathcal{F}/\borel{\R}$であることが判明する.
		}.
		\QED
	\end{prf}
	
	\begin{itembox}[l]{}
		\begin{thm}[任意抽出定理(2)]
			$I = [0,T]$,$p > 1$,$(M_t)_{t \in I}$を$\mathrm{L}^p$-マルチンゲールとする.
			このとき$I$に値を取る任意の停止時刻$\tau,\sigma$に対し次が成り立つ:
			\begin{align}
				\cexp{M_\tau}{\mathcal{F}_\sigma} = M_{\tau \wedge \sigma}.
			\end{align}
			\label{thm:optional_sampling_theorem_2}
		\end{thm}
	\end{itembox}
	
	\begin{prf}
		\begin{align}
			\tau_n \coloneqq \min{}{\left\{\ T, \frac{1+[2^n \tau]}{2^n}\ \right\}},
			\quad \sigma_n \coloneqq \min{}{\left\{\ T, \frac{1+[2^n \sigma]}{2^n}\ \right\}},
			\quad (n=1,2,\cdots)
		\end{align}
		とおく.このとき$\tau_n,\sigma_n$は停止時刻で$\mathcal{F}_\sigma \subset \mathcal{F}_{\sigma_n}\ (n=1,2,\cdots)$
		が成り立つ.実際任意の$0 \leq t < T$に対して
		\begin{align}
			\left\{ \tau_n \leq t \right\} = \left\{ 1 + [2^n \tau] \leq 2^n t \right\} = \left\{ \tau_n \leq [2^n t]/2^n \right\} \in \mathcal{F}_t
		\end{align}
		となり,$t = T$の時も
		\begin{align}
			\left\{ \tau_n \leq T \right\} = \left\{ 1 + [2^n \tau] > 2^n T \right\} + \left\{ 1 + [2^n \tau] \leq 2^n T \right\} \in \mathcal{F}_T
		\end{align}
		が成り立つから$\tau_n$は停止時刻
		\footnote{
			もとより$\tau_n$は可測関数である.$\R \ni x \longmapsto [x] \in \R$は可測$\borel{\R}/\borel{\R}$であるから
			$[2^n \tau]$は可測$\mathcal{F}/\borel{\R}$であり,従って$\tau_n$も可測$\mathcal{F}/\borel{\R}$となっている.
		}で,
		\begin{align}
			2^n \sigma_n \leq 1 + [2^n \sigma] 
			\Rightarrow \sigma < \sigma_n
		\end{align}
		により$\mathcal{F}_\sigma \subset \mathcal{F}_{\sigma_n}$となる.前定理により任意の$A \in \mathcal{F}_\sigma$に対して
		\begin{align}
			\int_A M_{\tau_n(\omega)}(\omega)\ \mu(d\omega) = \int_A M_{\tau_n(\omega)\wedge \sigma_n(\omega)}(\omega)\ \mu(d\omega) 
		\end{align}
		が成り立ち,$(|M_t|)_{t \in I}$が$\mathrm{L}^p$-劣マルチンゲールであることからDoobの不等式により
		$\sup{t \in I}{M_t}$は可積分である
		\footnote{
			$\sup{t \in I}{|M_t|^p} = \left( \sup{t \in I}{|M_t|} \right)^p$である.
		}.
		従って$\lim_{n \to \infty} \tau_n = \tau$と$M$の右連続性から,Lebesgueの収束定理より
		\begin{align}
			&\int_A M_{\tau(\omega)}(\omega)\ \mu(d\omega) = \lim_{n \to \infty} \int_A M_{\tau_n(\omega)}(\omega)\ \mu(d\omega) \\
			&\quad = \lim_{n \to \infty} \int_A M_{\tau_n(\omega)\wedge \sigma_n(\omega)}(\omega)\ \mu(d\omega)
			= \int_A M_{\tau(\omega)\wedge \sigma(\omega)}(\omega)\ \mu(d\omega)
		\end{align}
		が成り立つ.
		\QED
	\end{prf}
	
	
	\begin{itembox}[l]{}
		\begin{thm}[閉集合と停止時刻]
			$I = [0,T] \subset \R$,$(X_t)_{t \in I}$を$d$次元確率変数の族とし,
			$\mathcal{F}_0$がP-零集合を全て含んでいると仮定する.
			P-零集合$N$を除いて$I \ni t \longmapsto X_t(\omega)$が右連続で,
			かつ$(X_t)_{t \in I}$が$(\mathcal{F}_t)$-適合であるなら,任意の閉集合$F \subset \R^d$に対し
			\begin{align}
				\tau(\omega) \coloneqq
				\begin{cases}
					0 & (\omega \in N) \\
					\inf{}{\Set{t \in I}{X_t(\omega) \in F}} \wedge T & (\omega \in \Omega \backslash N)
				\end{cases}
			\end{align}
			として$\tau:\Omega \rightarrow \R$を定めれば$\tau$は停止時刻となる.
			また$N' (\subset N)$を除いて$I \ni t \longmapsto X_t(\omega)$が連続であるなら
			\begin{align}
				X_{t \wedge \tau(\omega)}(\omega) \in F^{ic} \quad (\forall \omega \in N',\ t \in I)
			\end{align}
			が成り立つ.ただし$F^i$は$F$の内核を表し$F^{ic}$は$F^i$の補集合を表す.
			\label{thm:closed_set_stopping_time}
		\end{thm}
	\end{itembox}
	確率空間が完備である場合は
	\begin{align}
		\tau(\omega) \coloneqq \inf{}{\Set{t \in I}{X_t(\omega) \in F}} \wedge T
		\quad (\forall \omega \in \Omega)
	\end{align}
	として$\tau$は停止時刻となる.実際任意の$t \in I$に対して,
	\begin{align}
		\{\tau \leq t\} = \{\tau \leq t\} \cap N + \Set{\omega \in \Omega \backslash N}{\tau(\omega) \leq t}
	\end{align}
	の右辺第一項は完備性よりP-零集合,第二項は以下で$\mathcal{F}_t$に属すると証明される.

	\begin{prf}
		$d:\R^d \times \R^d \rightarrow \R$をEuclid距離関数として
		\begin{align}
			D_t(\omega) \coloneqq 
			\begin{cases}
				1 & (\omega \in N) \\
				\inf{}{\Set{d(X_r(\omega),F)}{r \in ([0,t] \cap \Q) \cup \{t\}}} & (\omega \in \Omega \backslash N)
			\end{cases}
		\end{align}
		とおけば
		\footnote{
			$d(X_r(\omega),F) = \inf{y \in F}{d(X_r(\omega),y)}$である.
		},
		$D_t$は可測$\mathcal{F}_t/\borel{\R}$となる
		\footnote{
			写像$\R^d \ni x \longmapsto d(x,F) \in \R$は連続であるから,合成写像
			\begin{align}
				\Omega \ni \omega \longmapsto d(X_t(\omega),F)
			\end{align}
			は可測$\mathcal{F}_t/\borel{\R}$となる.任意の$\lambda \in \R$に対し
			\begin{align}
				\left\{ \inf{}{\Set{d(X_r,F)}{r \in ([0,t] \cap \Q) \cup \{t\}}} \geq \lambda \right\}
				= \bigcap_{r \in ([0,t] \cap \Q) \cup \{t\}} \left\{ d(X_r,F) \geq \lambda \right\}
			\end{align}
			となり右辺の各集合は$\in \mathcal{F}_t$であるから
			写像$\Omega \ni \omega \longmapsto \inf{}{\Set{d(X_r(\omega),F)}{r \in ([0,t] \cap \Q) \cup \{t\}}}$
			も可測$\mathcal{F}_t/\borel{\R}$となる.任意の$A \in \borel{\R}$に対し
			\begin{align}
				D_t^{-1}(A) = 
				\begin{cases}
					N \cup \left\{ \inf{}{\Set{d(X_r,F)}{r \in ([0,t] \cap \Q) \cup \{t\}}} \in A \right\} & (1 \in A) \\
					\left\{ \inf{}{\Set{d(X_r,F)}{r \in ([0,t] \cap \Q) \cup \{t\}}} \in A \right\} & (1 \notin A)
				\end{cases}
			\end{align}
			となるが,$N \in \mathcal{F}_0$であるから$D_t$もまた可測$\mathcal{F}_t/\borel{\R}$となる.
		}.
		ここでは任意の$t \in [0,T)$に対して
		\begin{align}
			\Set{\omega \in \Omega \backslash N}{\tau(\omega) \leq t} = \Set{\omega \in \Omega \backslash N}{D_t(\omega) = 0}
			\label{eq:closed_set_stopping_time_1}
		\end{align}
		が成り立つことを示す.実際これが示されれば任意の$t \in I$に対し
		\begin{align}
			\{\tau \leq t\} =
			\begin{cases}
 				\Omega & (t = T) \\
				N + \Set{\omega \in \Omega \backslash N}{D_t(\omega) = 0} & (t < T)
 			\end{cases}
		\end{align}
		となるから$\tau$は停止時刻となる.式(\refeq{eq:closed_set_stopping_time_1})が成立することを示すには
		包含関係$\subset,\supset$のそれぞれを満たすことを確認すればよい.
		\begin{description}
			\item[$\subset$について]
				任意に$t \in [0,T)$を固定する.$\tau(\omega) \leq t$となる$\omega \in \Omega \backslash N$に対し
				$s \coloneqq \tau(\omega)$とおくと$X_s(\omega) \in F$となる.もし$X_s(\omega) \notin F$であるとすれば,
				$F$が閉集合であることと$s \longmapsto X_s(\omega)$の右連続性から
				或る$\delta > 0$が存在し,任意の$0 < h < \delta$に対して
				$X_{s+h}(\omega) \notin F$となり$s = \tau(\omega)$であることに矛盾する.
				今$d(X_s(\omega),F) = 0$が示されたが,$D_t(\omega) = 0$も成り立っている.
				実際もし$D_t(\omega) > 0$であるとすれば,$a \coloneqq D_t(\omega)$に対して
				\begin{align}
					d(X_s(\omega), X_r(\omega)) < a/2
				\end{align}
				を満たす$r \in ((s,t] \cap \Q) \cup \{t\}$が存在するから
				\begin{align}
					d(X_s(\omega),F) \geq d(X_r(\omega),F) - d(X_s(\omega), X_r(\omega)) > a - a/2 = a/2
				\end{align}
				となり矛盾が生じてしまう.
			
			\item[$\supset$について]
				任意に$t \in [0,T)$を固定する.$D_t(\omega) = 0$となる$\omega \in \Omega \backslash N$について
				\begin{align}
					d(X_{s_n}(\omega),F) < 1/n
				\end{align}
				となるように$s_n \in ([0,t] \cap \Q) \cup \{t\}$を取ることができる.$(s_n)_{n=1}^{\infty}$は
				$[0,t]$に集積点$s$を持ち,$F$が閉であるから$X_s(\omega) \in F$となる.従って
				$\tau(\omega) \leq s \leq t$が成り立つ.
		\end{description}
		以上で$\tau$が停止時刻であることが示されたから,次に定理の後半の主張を示す.
		P-零集合$N' \subset N$を除いて$I \ni t \longmapsto X_t(\omega)$が連続である場合,
		$t < \tau(\omega)\ (\omega \in \Omega \backslash N')$に対しては$X_t(\omega) \in F^c$が成り立っているから,示せばよいのは
		\begin{align}
			X_{\tau(\omega)}(\omega) \in F^{ic}
			\label{eq:closed_set_stopping_time_2}
		\end{align}
		が成り立つことである.$s = \tau(\omega)$とおく.もし$X_s(\omega) \in F^i$であるとすれば
		連続性から或る$\delta$が存在し,$0 < h < \delta$を満たす任意の$h$に対し
		\begin{align}
			X_{s - h}(\omega) \in F^i
		\end{align}
		となるが,$s > s - h \geq \tau(\omega)$が従い矛盾が生じる.よって(\refeq{eq:closed_set_stopping_time_2})が示された.
		\QED
	\end{prf}
	
	上の証明では$\R^d$を一般の距離空間$(E,\rho)$として問題なく,定理の主張は少し一般化できる.
	\section{二次変分}
	確率空間を$(\Omega,\mathcal{F},\mu)$と表し,この空間は完備であると仮定する.
	$I \coloneqq [0,T]\ (T>0)$とし,$(\mathcal{F}_t)_{t \in I}$をフィルトレーションとする.
	このフィルトレーションは次の仮定を満たすものとする.
	\begin{align}
		\mathcal{F}_0 \supset \mathcal{N} \coloneqq \left\{\ N \in \mathcal{F}\quad |\quad \mu(N) = 0 \ \right\}
	\end{align}
	
	以下,いくつか集合を定義する.
	\begin{description}
		\item[$\mathrm{(1)}\ \mathcal{A}^+$] 
			$\mathcal{A}^+$は以下を満たす$(\Omega,\mathcal{F},\mu)$上の可測関数族$A = (A_t)_{t \in I}$の全体である.
			\begin{description}
				\item[適合性] 任意の$t \in I$に対し,写像$\Omega \ni \omega \longmapsto A_t(\omega) \in \R$は可測$\mathcal{F}_t/\borel{\R}$である.
				\item[連続性] $A$に対し或る$\mu$-零集合$N$が存在し,$\omega \in \Omega \backslash N$については写像$I \ni t \longmapsto A_t(\omega) \in \R$が連続である.
				\item[単調非減少性] $A$に対し或る$\mu$-零集合$N'$が存在し,$\omega \in \Omega \backslash N'$については写像$I \ni t \longmapsto A_t(\omega) \in \R$が単調非減少である.
			\end{description}
		
		\item[$\mathrm{(2)}\ \mathcal{A}$]
			$\mathcal{A} \coloneqq \left\{\ A^1 - A^2\quad |\quad A^1,A^2 \in \mathcal{A}^+\ \right\}$
			と定義する.$A^1 - A^2 \in \mathcal{A}$に対し或る$\mu$-零集合$N_1,N_2$が存在して,$\omega \in \Omega \backslash (N_1 \cup N_2)$
			なら写像$t \longmapsto A^1_t(\omega)$と$t \longmapsto A^2_t(\omega)$が連続かつ単調非減少となる.
			すなわちこの$\omega$について写像$t \longmapsto A^1_t(\omega) - A^2_t(\omega)$は有界連続となっている.
			
		\item[$\mathrm{(3)}\ \mathcal{M}_{p,c}\ (p \geq 1)$]
			$\mathcal{M}_{p,c}$は以下を満たす可測関数族$M = (M_t)_{t \in I} \subset \semiLp{p}{\mathcal{F},\mu}$の全体である.
			\begin{description}
				\item[$\mathrm{L}^p$-マルチンゲール] $M = (M_t)_{t \in I}$は$\mathrm{L}^p$-マルチンゲールである.
				\item[連続性] $M$に対し或る$\mu$-零集合$N$が存在し,$\omega \in \Omega \backslash N$については写像$I \ni t \longmapsto M_t(\omega) \in \R$が連続である.
			\end{description}
		
		\item[$\mathrm{(4)}\ \mathcal{M}_{b,c}$]
			$\mathcal{M}_{b,c}$はa.s.に連続で一様有界な$\mathrm{L}^1$-マルチンゲールの全体とする.つまり
			\begin{align}
				\mathcal{M}_{b,c} \coloneqq \left\{\ M = (M_t)_{t \in I} \in \mathcal{M}_{1,c}\quad |\quad \sup{t \in I}{\Norm{M_t}{\mathscr{L}^\infty}} < \infty\ \right\}
			\end{align}
			として定義されている.
			
		\item[$\mathrm{(5)}\ \mathcal{T}$]
			$\mathcal{T}$は以下を満たすような,$I$に値を取る停止時刻の列$(\tau_j)_{j=1}^{\infty}$の全体とする.
			\begin{description}
				\item[a)] $(\tau_j)_{j=1}^{\infty}$に対し或る$\mu$-零集合$N_0$が存在し,$\tau_0(\omega) = 0\ (\forall \omega \notin N_0)$となる.
				\item[b)] $(\tau_j)_{j=1}^{\infty}$の各$j$に対し或る$\mu$-零集合$N_j$が存在し,$\tau_j(\omega) \leq \tau_{j+1}(\omega)\ (\forall \omega \notin N_j)$となる.
				\item[c)] $(\tau_j)_{j=1}^{\infty}$に対し或る$\mu$-零集合$N_T$が存在し,任意の$\omega \in \Omega \backslash N_T$に或る$n = n(\omega) \in \N$が存在して$\tau_n(\omega)=T$が成り立つ.
			\end{description}
			例えば$\tau_j = jT/2^n$なら$(\tau_j)_{j=1}^{\infty} \in \mathcal{T}$となる.
			上の条件において$N \coloneqq N_0 \cup N_T \cup (\cup_{j=1}^{\infty}N_j)$とすればこれも$\mu$-零集合で,$\omega \in \Omega \backslash N$なら
			\begin{align}
				&\tau_0(\omega) = 0,\qquad \tau_j(\omega) \leq \tau_{j+1}(\omega)\ (j=1,2,\cdots),\\
				&\tau_{n_\omega}(\omega) = T\ (\exists n_\omega \in \N)
			\end{align}
			が成立することになる.
			
		\item[$\mathrm{(4)}\ \mathcal{M}_{c,loc}$]
			$\mathcal{M}_{c,loc} \coloneqq 
			\left\{\ M = (M_t)_{t \in I} \subset \semiLp{1}{\mathcal{F},\mu} \quad |\quad \exists (\tau_j)_{j=1}^{\infty} \in \mathcal{T}\ \mathrm{s.t.}\ M^j = (M_{\tau_j \wedge t})_{t \in I} \in \mathcal{M}_{b,c}\ (\forall j \in \N) \ \right\}$
			として定義される.(連続な局所マルチンゲールの全体)
	\end{description}
	
	以下で$\mathcal{M}_{2,c}$に適当な処置を施してこれがHilbert空間と見做せるようにする.
	次の手順に沿う.
	\begin{description}
		\item[$\mathrm{(i)}$] $\mathcal{M}_{p,c}$に線型演算を定義して線形空間(係数体は$\R$)となることを示す.
		\item[$\mathrm{(ii)}$] $\mathcal{M}_{p,c}$の或る同値関係により商空間を定義する.
		\item[$\mathrm{(iii)}$] 特に$p=2$のとき,$\mathcal{M}_{2,c}$の商空間に内積を導入してHilbert空間となることを示す.
	\end{description}
	
	\begin{description}
		\item[$\mathrm{(i)}$について] 
			任意の$M=(M_t)_{t \in I},\ N=(N_t)_{t \in I} \in \mathcal{M}_{p,c}$と$\alpha \in \R$に対して,加法とスカラ倍を
			\begin{align}
				M + N \coloneqq (M_t + N_t)_{t \in I}, \qquad \alpha M \coloneqq (\alpha M_t)_{t \in I}
			\end{align}
			として定義し,零元を$0$\footnote{全ての$t,\omega$に対し$0 \in \R$を取るもの.}と表す.また二元$M$と$N$が等しいということを
			\begin{align}
				M_t(\omega) = N_t(\omega) \quad (\forall t \in I,\ \omega \in \Omega)
			\end{align}
			が成り立っているということで定義する.$\mathcal{M}_{p,c}$が上の演算について閉じていることが示されれば,
			線形空間であるための条件を満たすことは$t \longmapsto M_t(\omega) (\forall \omega \in \Omega)$が全て実数値であることにより判ることである.
			加法とスカラ倍について閉じていることを示す.
			\begin{description}
				\item[加法について]
					任意の$0 \leq s \leq t \leq T$に対し,条件付き期待値の線型性(性質$\tilde{\mathrm{C}}3$)により
					\begin{align}
						\cexp{M_t + N_t}{\mathcal{F}_s} = \cexp{M_t}{\mathcal{F}_s} + \cexp{N_t}{\mathcal{F}_s} = M_s + N_s
					\end{align}
					が成り立つ\footnote{式の$M_t$などは代表元を$M_t$とする関数類の意味で使っている.}.
					また$M+N$は各$t \in I$について和を取っただけであるから,$(\mathcal{F}_t)$-適合であること,そして任意の$\omega \in \Omega$について
					左極限が存在しかつ右連続となっていることが判り,更にMinkowskiの不等式から各$t \in I$について$M_t + N_t \in \semiLp{p}{\mathcal{F},\mu}$
					となる.以上で$M+N = (M_t + N_t)_{t \in I}$もまた$\mathrm{L}^p$-マルチンゲールであることが示された.
					写像$I \ni t \longmapsto M_t(\omega) + N_t(\omega) \in \R$の連続性については,
					$M,N$それぞれに対して或る$\mu$-零集合$E_1,E_2$が存在して,$\omega \notin E_1$なら$t \longmapsto M_t(\omega)$は連続,
					$\omega \notin E_2$なら$t \longmapsto N_t(\omega)$は連続となるのだから,従って$\omega \notin E_1 \cup E_2$なら
					$t \longmapsto M_t(\omega) + N_t(\omega)$が連続($\mu$-a.s.に連続)となる.以上で$M+N \in \mathcal{M}_{p,c}$が示された.
				
				\item[スカラ倍について]
					任意の$0 \leq s \leq t \leq T$に対し,条件付き期待値の線型性(性質$\tilde{\mathrm{C}}3$)により
					\begin{align}
						\cexp{\alpha M_t}{\mathcal{F}_s} = \alpha \cexp{M_t}{\mathcal{F}_s} = \alpha M_s
					\end{align}
					が成り立つ.定数倍しているだけであるから,$(\alpha M_t)_{t \in I}$が$(\mathcal{F}_t)$-適合であること,そして任意の$\omega \in \Omega$について
					左極限が存在しかつ右連続となっていることが判り,更に各$t \in I$について$\alpha M_t \in \semiLp{p}{\mathcal{F},\mu}$
					となる.以上で$\alpha M = (\alpha M_t)_{t \in I}$もまた$\mathrm{L}^p$-マルチンゲールであることが示された.
					連続性については,写像$t \longmapsto M_t(\omega)$が連続となる$\omega$ならば
					写像$t \longmapsto \alpha M_t(\omega)$も連続($\mu$-a.s.に連続)となる.
					以上で$\alpha M \in \mathcal{M}_{p,c}$が示された.
			\end{description}
		
		\item[$\mathrm{(ii)}$について]
			任意の$M=(M_t)_{t \in I},\ N=(N_t)_{t \in I} \in \mathcal{M}_{p,c}$に対して,関係$R$を
			\begin{align}
				M\ R\ N &\coloneqq \left\{\ \omega \in \Omega\quad|\quad \sup{r \in (I \cap \Q) \cup \{ T \}}{\left|M_r(\omega) - N_r(\omega)\right| > 0}\ \right\}\mbox{が$\mu$-零集合} \\
				&\Leftrightarrow \bigcup_{r \in (I \cap \Q) \cup \{ T \}}{\left\{\ \omega \in \Omega\quad|\quad \left|M_r(\omega) - N_r(\omega)\right| > 0\ \right\}}\mbox{が$\mu$-零集合}
			\end{align}
			として定義すれば,関係$R$は同値関係となる.反射律と対称律は上式より成立しているから推移律について確認する.$M,N$とは別に$U=(U_t)_{t \in I} \in \mathcal{M}_{p,c}$
			を取って$M\ R\ N$かつ$N\ R\ U$となっているとすれば,各$r \in (I \cap \Q) \cup \{ T \}$にて
			\begin{align}
				\left(\ \left|M_r - U_r\right| > 0\ \right)\ \subset\ 
				\left(\ \left|M_r - N_r\right| > 0\ \right) \cup \left(\ \left|N_r - U_r\right| > 0\ \right)
			\end{align}
			の関係が成り立っているから$M\ R\ U$が従う.
			\footnote{$\left(\ \left|M_r - N_r\right| > 0\ \right) = \left\{\ \omega \in \Omega\quad|\quad \left|M_r(\omega) - N_r(\omega)\right| > 0\ \right\}.$}
			
			$M \in \mathcal{M}_{p,c}$の関係$R$による同値類を$\overline{M}$と表記し,商空間を$\mathfrak{M}_{p,c} \coloneqq \mathcal{M}_{p,c}/R$と表記すれば
			$\mathfrak{M}_{p,c}$において
			\begin{align}
				\overline{M} + \overline{N} \coloneqq \overline{M+N}, \quad \alpha \overline{M} \coloneqq \overline{\alpha M} \label{eq:mart_linear_arithmetic}
			\end{align}
			として演算を定義すれば,これは代表元の選び方に依らない(well-defined).つまり$M' \in \overline{M},\ N' \in \overline{N}$に対して
			\begin{align}
				\overline{M+N} = \overline{M'+N'}, \quad \overline{\alpha M} = \overline{\alpha M'}
			\end{align}
			が成り立つ.これは
			\begin{align}
				\left(\ \left|M_r + N_r - M'_r - N'_r \right| > 0\ \right) &\subset \left(\ \left|M_r - M'_r \right| > 0\ \right) \cup \left(\ \left|N_r - N'_r \right| > 0\ \right) \\
				\left(\ \left|\alpha M_r - \alpha M'_r \right| > 0\ \right) &= \left(\ \left|M_r - M'_r \right| > 0\ \right)
			\end{align}
			により$(M+N)\ R\ (M'+N'),\ (\alpha M)\ R\ (\alpha M')$が成り立つからである.以上の事柄に注意すれば,(\refeq{eq:mart_linear_arithmetic})で定義した算法を加法とスカラ倍として
			$\mathfrak{M}_{p,c}$は$\R$上の線形空間となる.
		
		\item[$\mathrm{(iii)}$について]
			先ずは$\mathfrak{M}_{2,c}$において内積を定義し,それから$\mathfrak{M}_{2,c}$がその内積によってHilbert空間となることを示す.
			\begin{itembox}[l]{内積の定義}
				$\mathfrak{M}_{2,c} \times \mathfrak{M}_{2,c}$上の実数値写像$\inprod<\cdot,\cdot>$を次で定義すれば,これは$\mathfrak{M}_{2,c}$において内積となる.
				\begin{align}
					\inprod<\overline{M},\overline{N}> \coloneqq \Exp{M_TN_T} = \int_{\Omega} M_T(\omega)N_T(\omega)\ \mu(d\omega), \quad (\overline{M},\overline{N} \in \mathfrak{M}_{2,c}).
				\end{align}
			\end{itembox}
	\end{description}
	
	\begin{itembox}[l]{}
		\begin{thm}[]
		\end{thm}
	\end{itembox}

%\newpage
%\printindex
%
%
\end{document}
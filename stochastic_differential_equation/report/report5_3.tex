	
	\begin{screen}
		\begin{prp}
			任意の$p \geq 1$に対し,$\mathcal{M}_{b,c} \subset \mathcal{M}_{p,c} \subset \mathcal{M}_{c,loc}$が成り立つ.
			\label{prp:M_pc_M_cloc}
		\end{prp}
	\end{screen}
	この証明には次の補題を使う.
	
	\begin{screen}
		\begin{lem}[右連続で左極限を持つ関数は閉区間上で有界]\mbox{}\\
			$(E,\rho)$を距離空間,$J = [a,b] \subset \R$とする.$f:J \rightarrow E$が各点で
			右連続且つ左極限を持つなら$f$は$J$上で有界である.
			ただし左端点では左極限を考えず,右端点では右連続性を考えない.
			\label{lem:rcll_bounded}
		\end{lem}
	\end{screen}
	
	\begin{prf}[補題\ref{lem:rcll_bounded}]
		任意に$\epsilon > 0$を取り固定する.$f$は各点$x \in [a,b)$で右連続であるから,$0 < \delta_x < b-x$を
		$0 < \forall h < \delta_x$が$\rho(f(x), f(x+h)) < \epsilon$を満たすように取り,
		\begin{align}
			V_x \coloneqq [x,x+\delta_x) \quad (\forall x \in [a,b))
		\end{align}
		とおく.また$f$は各点$x \in (a,b]$で左極限も持つから,左極限を$f(x-)$と表して
		$0 < \gamma_x < x-a$を$0 < \forall h < \gamma_x$が$\rho(f(x-),f(x-h)) < \epsilon$を満たすように取り,
		\begin{align}
			U_x \coloneqq (x-\gamma_x,x] \quad (\forall x \in (a,b])
		\end{align}
		とおく.特に$U_a \coloneqq (-\infty,a],\ V_b \coloneqq [b,\infty)$とおけば
		\begin{align}
			J \subset \bigcup_{x \in J}U_x \cup V_x
		\end{align}
		が成り立つが,$J$は$\R$のコンパクト部分集合であるから,このうち有限個を選び
		\begin{align}
			J = \bigcup_{i=1}^n \left( U_{x_i} \cup V_{x_i}\right) \cap J
		\end{align}
		とできる.$U_{x_i} \cap J,\ V_{x_i} \cap J$での$f$の挙動の振れ幅は$2\epsilon$で抑えられるから
		$J$全体での挙動の振れ幅は$2n\epsilon$より小さい
		\footnote{
			$x_1 < x_2 < \cdots < x_n$と仮定し,区間$U_{x_i} \cup V_{x_i}$と$U_{x_{i+1}} \cup V_{x_{i+1}}$の共通点を一つ取り$z_i$と表す.
			$\rho(f(x),f(y))\ (x,y \in J)$の上界を知りたいから$x \in U_{x_1} \cup V_{x_1},\ y \in U_{x_n} \cup V_{x_n}$の場合を調べればよい.このとき
			\begin{align}
				\rho(f(x),f(y)) &\leq \rho(f(x),f(x_1)) + \rho(f(x_1),f(x_2)) + \cdots + \rho(f(x_{n-1}),f(x_n)) + \rho(f(x_n),f(y)) \\
				&\leq \rho(f(x),f(x_1)) + \rho(f(x_1),f(z_1)) + \rho(f(z_1),f(x_2)) + \cdots + \rho(f(z_{n-1}),f(x_n)) + \rho(f(x_n),f(y)) \\
				& < 2n\epsilon
			\end{align}
			が成り立つ.
		}.
		ゆえに有界である.
		\QED
	\end{prf}
	
	\begin{screen}
		\begin{lem}[右連続関数で導入する停止時刻の単調性]
			或る零集合$N$が存在し,全ての$\omega \in \Omega \backslash N$に対して
			$I \ni t \longmapsto X_t(\omega)$が右連続なら,
			\begin{align}
				\tau_n(\omega) \coloneqq
				\begin{cases}
					0 & (\omega \in N) \\
					\inf{}{\Set{t \in I}{|X_t(\omega)| \geq n}} \wedge T & (\omega \in \Omega \backslash N) \footnotemark
				\end{cases}
				\quad (n=0,1,2,\cdots)
			\end{align}
			で定める停止時刻は$0 = \tau_0 \leq \tau_1 \leq \cdots$を満たし,
			特に,$\omega \in \Omega \backslash N$に対して
			$I \ni t \longmapsto X_t(\omega)$が各点で左極限を持つなら,或る$n = n(\omega) \in \N$が存在して
			$\tau_n(\omega) = T$を満たす.
			\label{lem:right_continuous_stopping_time_monotonous}
		\end{lem}
	\end{screen}
	\footnotetext{
		定理\ref{thm:closed_set_stopping_time}の脚注に書いた通り
		$\Set{t \in I}{|X_t(\omega)| \geq n} = \emptyset$の場合は$\tau_n(\omega) = T$とする.
	}
	\begin{prf}[補題\ref{lem:right_continuous_stopping_time_monotonous}]
		定理\ref{thm:closed_set_stopping_time}より$\tau_n$は停止時刻である.
		絶対値の非負性より$\tau_0 = 0$が従い,
		また
		\begin{align}
			\tau_{n}(\omega) \leq \inf{}{\Set{t \in I}{|X_t(\omega)| \geq n+1}} = \tau_{n+1}(\omega)
			\quad (\forall \omega \in \Omega \backslash N,\ n=0,1,2,\cdots)
		\end{align}
		が成り立つから$0 = \tau_0 \leq \tau_1 \leq \cdots$が得られる.
		或る$\omega \in \Omega \backslash N$に対してパスが各点で左極限を持つ場合,
		補題\ref{lem:rcll_bounded}より$n > \sup{t \in I}{|X_t(\omega)|}$を満たす
		$n = n(\omega)$が存在して$\tau_n(\omega) = T$が得られる.
		\QED
	\end{prf}
	
	\begin{prf}[命題\ref{prp:M_pc_M_cloc}]
		a.s.に有界な関数は$p$乗して可積分であるから$\mathcal{M}_{b,c} \subset \mathcal{M}_{p,c}$を得る.
		$\mathcal{M}_{p,c} \subset \mathcal{M}_{c,loc}$を示す.任意に$M \in \mathcal{M}_{p,c}$を取れば,
		或る零集合$E$が存在して
		\begin{align}
			M_0(\omega) = 0 \quad (\forall \omega \in \Omega \backslash E)
		\end{align}
		を満たし,$t \longmapsto M_t(\omega)$が連続となる.
		全てのパスは右連続で左極限を持つから,補題\ref{lem:right_continuous_stopping_time_monotonous}より
		\begin{align}
			\tau_j(\omega) \coloneqq \inf{}{\Set{t \in I}{|M_t(\omega)| \geq j}} \wedge T \quad (\forall \omega \in \Omega,\ j=0,1,2,\cdots)
		\end{align}
		で定める停止時刻列は$(\tau_j)_{j=0}^{\infty} \in \mathcal{T}$を満たす.
		Doobの不等式(定理\ref{thm:Doob_inequality_2})より
		$\sup{t \in I}{|M_t|}$が$p$乗可積分となるから$M_t^{\tau_j} = M_{t \wedge \tau_j}\ (\forall t \in I)$は可積分である.
		また$\omega \in \Omega \backslash E$については$t \longmapsto M_t(\omega)$の連続性から
		$I \ni t \longmapsto M_t^{\tau_j}(\omega)$の連続性が従い,かつ定理\ref{thm:closed_set_stopping_time}より
		\begin{align}
			\sup{t \in I}{\left| M_{t \wedge \tau_j(\omega)}(\omega) \right|} \leq j
			\quad (\forall \omega \in \Omega \backslash E)
			\label{eq:M_pc_M_cloc}
		\end{align}
		が成り立ち
		\begin{align}
			\Norm{M_t^{\tau_j}}{\mathscr{L}^\infty} \leq j \quad (\forall t \in I)
		\end{align}
		を得る.そして任意抽出定理(定理\ref{thm:optional_sampling_theorem_2})より
		\begin{align}
			\cexp{M_t^{\tau_j}}{\mathcal{F}_s} = M_{t \wedge \tau_j \wedge s} = M_s^{\tau_j} \quad (s,t \in I,\ s < t)
		\end{align}
		が成り立つから.$M^{\tau_j} \in \mathcal{M}_{b,c}\ (j=1,2,\cdots)$,すなわち$M \in \mathcal{M}_{c,loc}$となる.
		\QED
	\end{prf}
	
	\begin{screen}
		\begin{lem}
			$X \in \mathcal{M}_{2,c}$と停止時刻$\tau \geq \sigma$に対し次が成り立つ:
			\begin{description}
				\item[(1)] $\Exp{(X_{\tau} - X_{\sigma})^2} = \Exp{X_{\tau}^2 - X_{\sigma}^2}$,
				\item[(2)] $\cexp{(X_{\tau} - X_{\sigma})^2}{\mathcal{F}_\sigma} = \cexp{X_{\tau}^2 - X_{\sigma}^2}{\mathcal{F}_\sigma}$.
			\end{description}
			\label{lem:stopping_time_telescopic_sum}
		\end{lem}
	\end{screen}
	
	\begin{prf}\mbox{}
	\begin{description}
		\item[(1)] 
			$X \in \mathcal{M}_{2,c}$であるからDoobの不等式(定理\ref{thm:Doob_inequality_2})により$X_\tau,X_\sigma$は二乗可積分である.
			また定理\ref{thm:measurability_of_stopping_time}より$X_\sigma$は可測$\mathcal{F}_\sigma/\borel{\R}$であり,
			任意抽出定理(定理\ref{thm:optional_sampling_theorem_2})と$\tau \geq \sigma$の仮定より
			\begin{align}
				\cexp{X_{\tau}}{\mathcal{F}_\sigma} = X_{\tau \wedge \sigma} = X_\sigma
			\end{align}
			も成り立つ.命題\ref{prp:properties_of_expanded_conditional_expectation}の$\tilde{\mathrm{C}}$2と$\tilde{\mathrm{C}}$6を併せて
			\begin{align}
				\Exp{(X_{\tau} - X_{\sigma})^2}
				&= \Exp{X_{\tau}^2 + X_{\sigma}^2} - 2\Exp{X_{\tau}X_{\sigma}} \\
				&= \Exp{X_{\tau}^2 + X_{\sigma}^2} - 2\Exp{\cexp{X_{\tau}X_{\sigma}}{\mathcal{F}_\sigma}} \\
				&= \Exp{X_{\tau}^2 + X_{\sigma}^2} - 2\Exp{X_{\sigma}\cexp{X_{\tau}}{\mathcal{F}_\sigma}} \\
				&= \Exp{X_{\tau}^2 - X_{\sigma}^2}
			\end{align}
			を得る.
			
		\item[(2)]
			(1)と同様にして
			\begin{align}
				\cexp{(X_{\tau} - X_{\sigma})^2}{\mathcal{F}_\sigma}
				&= \cexp{X_{\tau}^2 + X_{\sigma}^2}{\mathcal{F}_\sigma} - 2\cexp{X_{\tau}X_{\sigma}}{\mathcal{F}_\sigma} \\
				&= \cexp{X_{\tau}^2 + X_{\sigma}^2}{\mathcal{F}_\sigma} - 2X_{\sigma}^2 \\
				&= \cexp{X_{\tau}^2 - X_{\sigma}^2}{\mathcal{F}_\sigma}
			\end{align}
			が成り立つ.
		\end{description}
		\QED
	\end{prf}
	
	\begin{screen}
		\begin{prp}[有界変動な連続二乗可積分マルチンゲールのパスは定数]\mbox{}\\
			$A \in \mathcal{A} \cap \mathcal{M}_{2,c}$に対し$A_t = 0\ (\forall t \in I)\quad \mbox{$\mu$-a.s.}$が成り立つ.
			\label{prp:bounded_continuous_M_2c_path}
		\end{prp}
	\end{screen}
	
	\begin{prf}
		$A$に対し或る$A^{(1)},A^{(2)} \in \mathcal{A}^+$が存在して
		\begin{align}
			A = A^{(1)} - A^{(2)}
		\end{align}
		と表現できる.また或る$\mu$-零集合$E$が存在し,全ての$\omega \in \Omega \backslash E$に対して
		\begin{align}
			A_0^{(1)}(\omega) = A_0^{(2)}(\omega) = 0
		\end{align}
		且つ$I \ni t \longmapsto A_t^{(1)}(\omega)$と$I \ni t \longmapsto A_t^{(2)}(\omega)$が共に連続,単調非減少となるから,
		\begin{align}
			\tau_m(\omega) \coloneqq
			\begin{cases}
				0 & (\omega \in E) \\
				\inf{}{\Set{t \in I}{A_t^{(1)}(\omega) \vee A_t^{(2)}(\omega) \geq m}} \wedge T & (\omega \in \Omega \backslash E)
			\end{cases}
			\quad (m=1,2,\cdots)
		\end{align}
		は定理\ref{thm:closed_set_stopping_time}より停止時刻となる.そして
		補題\ref{lem:right_continuous_stopping_time_monotonous}より
		\begin{align}
			0 = \tau_0(\omega) \leq \tau_1(\omega) \leq \cdots \leq \tau_i(\omega) = T \quad (\exists i = i(\omega) \in \N,\ \forall \omega \in \Omega \backslash E)
		\end{align}
		が満たされている.今任意に$t \in I,\ n,m \in \N$を取り固定する.
		\begin{align}
			\sigma_j^n \coloneqq \tau_m \wedge \frac{tj}{2^n} \quad (j = 0,1,\cdots, 2^n)
		\end{align}
		とおけば,$\sigma_j^n \leq \sigma_{j+1}^n\ (j = 0,1,\cdots, 2^n-1)$が成り立つから
		補題\ref{lem:stopping_time_telescopic_sum}により
		\begin{align}
			\Exp{\sum_{j=0}^{2^n-1} \left( A_{\sigma_{j+1}^n} - A_{\sigma_j^n} \right)^2}
			= \sum_{j=0}^{2^n-1} \Exp{A_{\sigma_{j+1}^n}^2 - A_{\sigma_j^n}^2}
			= \Exp{A_{\tau_m \wedge t}^2 - A_{0}^2}
			= \Exp{\left( A_{\tau_m \wedge t} - A_{0} \right)^2}
		\end{align}
		を得る.左辺の中の式は
		\begin{align}
			\sum_{j=0}^{2^n-1} \left( A_{\sigma_{j+1}^n} - A_{\sigma_j^n} \right)^2
			\leq \sup{j}{\left| A_{\sigma_{j+1}^n} - A_{\sigma_j^n} \right|} \sum_{j=0}^{2^n-1} \left| A_{\sigma_{j+1}^n} - A_{\sigma_j^n} \right|
		\end{align}
		となり,特に$\omega \in \Omega \backslash E$に対しては$I \ni t \longmapsto A_t(\omega)$の連続性から
		\begin{align}
			\sup{j}{\left| A_{\sigma_{j+1}^n}(\omega) - A_{\sigma_j^n}(\omega) \right|} \longrightarrow 0 \quad (n \longrightarrow \infty)
		\end{align}
		(概収束)と
		\begin{align}
			\sum_{j=0}^{2^n-1} \left| A_{\sigma_{j+1}^n}(\omega) - A_{\sigma_j^n}(\omega) \right|
			&\leq \sum_{j=0}^{2^n-1} \left( A^{(1)}_{\sigma_{j+1}^n}(\omega) - A^{(1)}_{\sigma_j^n}(\omega) + A^{(2)}_{\sigma_{j+1}^n}(\omega) - A^{(2)}_{\sigma_j^n}(\omega) \right) \\
			&= A^{(1)}_{\tau_m \wedge t}(\omega) + A^{(2)}_{\tau_m \wedge t}(\omega) \leq 2m
		\end{align}
		が満たされるから,Lebesgueの収束定理を適用できて
		\begin{align}
			\int_\Omega \sum_{j=0}^{2^n-1} \left( A_{\sigma_{j+1}^n(\omega)}(\omega) - A_{\sigma_j^n(\omega)}(\omega) \right)^2\ \mu(d\omega) \longrightarrow 0 \quad (n \longrightarrow \infty)
		\end{align}
		が従う.ゆえに
		\begin{align}
			\int_\Omega \left( A_{\tau_m(\omega) \wedge t}(\omega) - A_{0}(\omega) \right)^2\ \mu(d\omega) = 0 \quad (m=1,2,\cdots)
		\end{align}
		が成り立ち,Doobの不等式より$|A_{\tau_m \wedge t} - A_{0}| \leq \sup{t \in I}{|A_t - A_{0}|} \in \mathscr{L}^2$であるからLebesgueの収束定理を適用して
		\begin{align}
			\int_\Omega \left( A_t(\omega) - A_{0}(\omega) \right)^2\ \mu(d\omega) 
			= \int_{\Omega \backslash E} \left( A_t(\omega) \right)^2\ \mu(d\omega)
			= 0
		\end{align}
		を得る.$t \in I$は任意に取っていたから,
		\begin{align}
			A_t = 0 \quad \mbox{$\mu$-a.s.} \quad (\forall t \in I)
		\end{align}
		が従い,また$\omega \in \Omega \backslash E$に対する$A$のパスの連続性により
		\begin{align}
			\Set{\omega \in \Omega \backslash E}{A_t(\omega) = 0\ (\forall t \in I)}
			= \bigcap_{r \in I \cap \Q} \Set{\omega \in \Omega \backslash E}{A_r(\omega) = 0}
		\end{align}
		と表すことができる.従って
		\begin{align}
			\Set{\omega \in \Omega}{A_t(\omega) \neq 0\ (\exists t \in I)}
			\subset E + \bigcup_{r \in I \cap \Q} \Set{\omega \in \Omega \backslash E}{A_r(\omega) \neq 0}
		\end{align}
		が成り立ち,右辺が零集合であるから$\mu$-a.s.に$A_t = 0\ (\forall t \in I)$となる.
		\QED
	\end{prf}
	
	\begin{screen}
		\begin{lem}[二次変分補題]
			$n \in \N$と$M \in \mathcal{M}_{b,c}$を取り,$\tau_j^n = jT/2^n\ (j=0,1,\cdots,2^n)$に対し
			\begin{align}
				Q_t^n \coloneqq \sum_{j=0}^{2^n-1} \left( M_{t \wedge \tau_{j+1}^n} - M_{t \wedge \tau_j^n} \right)^2 \quad (\forall t \in I)
				\label{eq:lem_quadratic_variation_0}
			\end{align}
			とおけば,$M^2 - Q^n \in \mathcal{M}_{b,c}$かつ次が成り立つ:
			\begin{align}
				\Norm{M_T^2 - M_0^2 - Q_T^n}{\mathscr{L}^2} \leq 2 \sup{t \in I}{\Norm{M_t}{\mathscr{L}^\infty}} \Norm{M_T - M_0}{\mathscr{L}^2}.
			\end{align}
			\label{lem:quadratic_variation}
		\end{lem}
	\end{screen}
	
	\begin{prf}
		任意の停止時刻$\tau$に対し$M_{t \wedge \tau}$が可測$\mathcal{F}_t/\borel{\R}$であるから
		$Q^n$は$\mathcal{F}_t$-適合である.
		今任意に$s,t \in I,\ (s < t)$を取り固定する.$\tau_k^n \leq s < \tau_{k+1}^n$となる$k$を選べば,
		補題\ref{lem:stopping_time_telescopic_sum}と任意抽出定理\ref{thm:optional_sampling_theorem_2}より
		\begin{align}
			\cexp{Q_t^n - Q_s^n}{\mathcal{F}_s} 
			&= \cexp{\sum_{j=k}^{2^n-1}\left\{ \left( M_{t\wedge\tau_{j+1}^n} - M_{t\wedge\tau_j^n} \right)^2 - \left( M_{s\wedge\tau_{j+1}^n} - M_{s\wedge\tau_j^n} \right)^2 \right\}}{\mathcal{F}_s} \\
			&= \sum_{j=k+1}^{2^n-1} \cexp{ M_{t\wedge\tau_{j+1}^n}^2 - M_{t\wedge\tau_j^n}^2}{\mathcal{F}_s}
				+ \cexp{\left( M_{t\wedge\tau_{k+1}^n} - M_{\tau_k^n} \right)^2}{\mathcal{F}_s} - \left( M_s - M_{\tau_k^n} \right)^2 \\
			&= \cexp{ M_t^2 - M_{t\wedge\tau_{k+1}^n}^2}{\mathcal{F}_s} + \cexp{\left( M_{t\wedge\tau_{k+1}^n} - M_{\tau_k^n} \right)^2}{\mathcal{F}_s} - \left( M_s - M_{\tau_k^n} \right)^2 \\
			&= \cexp{M_t^2}{\mathcal{F}_s} - 2\cexp{M_{t\wedge\tau_{k+1}^n}M_{\tau_k^n}}{\mathcal{F}_s} + \cexp{M_{\tau_k^n}^2}{\mathcal{F}_s} - M_s^2 + 2M_sM_{\tau_k^n} - M_{\tau_k^n}^2 \\
			&= \cexp{M_t^2}{\mathcal{F}_s} - 2M_{\tau_k^n}\cexp{M_{t\wedge\tau_{k+1}^n}}{\mathcal{F}_s} + M_{\tau_k^n}^2 - M_s^2 + 2M_sM_{\tau_k^n} - M_{\tau_k^n}^2 \\
			&= \cexp{M_t^2}{\mathcal{F}_s} - M_s^2.
		\end{align}
		が成り立ち次を得る:
		\begin{align}
			\cexp{M_t^2 - Q_t^n}{\mathcal{F}_s} = M_s^2 - Q_s^n, \quad (\forall 0 \leq s < t \leq T).
			\label{eq:lem_quadratic_variation_1}
		\end{align}
		ここで
		\begin{align}
			N \coloneqq M^2 - Q^n
		\end{align}
		とおけば$N \in \mathcal{M}_{b,c}$であり
		\footnote{
			$M \in \mathcal{M}_{b,c}$より全ての$\omega \in \Omega$において写像$t \longmapsto M_t(\omega)$は各点で右連続かつ左極限を持つ.
			$Q^n$についてもGauss記号を用いて$Q_t^n = \sum_{j=0}^{[2^nt]/T} \left( M_t^n - M_{\tau_j^n} \right)^2$
			と表せば,$t \longmapsto Q_t^n(\omega)\ (\forall \omega \in \Omega)$が各点で右連続かつ左極限を持つことが明確になる.
			よって全ての$\omega \in \Omega$において$t \longmapsto N_t(\omega)$は各点で右連続かつ左極限を持つ.
			また同じ理由で$t \longmapsto M_t(\omega)$が連続となる点で$t \longmapsto N_t(\omega)$も連続となるから
			つまり$\mu$-a.s.に$t \longmapsto N_t$は連続.
			一様有界性については,$\sup{t \in I}{\Norm{M_t}{\mathscr{L}^\infty}} < \infty$であるから,任意の$t \in I$に対し
			或る零集合$E_t$が存在して$\omega \notin E_t$なら$|M_t(\omega)| \leq \sup{t \in I}{\Norm{M_t}{\mathscr{L}^\infty}}$が成り立つ.
			同様に$Q_t^n$についても$\omega \notin E_t \cup \bigcup_{j=0}^{[2^nt]/T}E_{\tau_j^n}$なら
			\begin{align}
				\left| Q_t^n(\omega) \right| \leq \sum_{j=0}^{[2^nt]/T} \left( 2\sup{t \in I}{\Norm{M_t}{\mathscr{L}^\infty}} \right)^2 \leq 2^{n+1} \sup{t \in I}{\Norm{M_t}{\mathscr{L}^\infty}^2}.
			\end{align}
			ゆえに
			\begin{align}
				\left| N_t(\omega) \right| \leq \left|{M_t(\omega)}^2\right| + \left|Q_t^n(\omega)\right| \leq \left( 2^{n+1}+1 \right) \sup{t \in I}{\Norm{M_t}{\mathscr{L}^\infty}^2}
				,\quad \left( \forall \omega \notin E_t \cup \cup_{j=0}^{[2^nt]/T}E_{\tau_j^n} \right).
			\end{align}
			この右辺は
			$t$に依らないから
			\begin{align}
				\sup{t \in I}{\Norm{N_t}{\mathscr{L}^\infty}} \leq \left( 2^{n+1}+1 \right) \sup{t \in I}{\Norm{M_t}{\mathscr{L}^\infty}^2}
			\end{align}
			を得る.以上の結果と(\refeq{eq:lem_quadratic_variation_1})を併せて$N \in \mathcal{M}_{b,c}$となる.
		},
		\begin{align}
			\Exp{(N_T - N_0)^2} = \Exp{N_T^2 - N_0^2} 
			&= \Exp{\sum_{j=0}^{2^n-1}\left( N_{\tau_{j+1}^n}^2 - N_{\tau_j^n}^2 \right)} \\
			&= \sum_{j=0}^{2^n-1}\Exp{\left\{ M_{\tau_{j+1}^n}^2 - M_{\tau_j^n}^2 - \left( Q_{\tau_{j+1}^n}^n - Q_{\tau_j^n}^n \right) \right\}^2} \\
			&= \sum_{j=0}^{2^n-1}\Exp{\left\{ M_{\tau_{j+1}^n}^2 - M_{\tau_j^n}^2 - \left( M_{\tau_{j+1}^n} - M_{\tau_j^n} \right)^2 \right\}^2} \\
			&\leq 4 \sup{t \in I}{\Norm{M_t}{\mathscr{L}^\infty}^2} \Exp{\sum_{j=0}^{2^n-1} \left( M_{\tau_{j+1}^n} - M_{\tau_j^n} \right)^2 } \\
			&= 4 \sup{t \in I}{\Norm{M_t}{\mathscr{L}^\infty}^2} \Exp{M_T^2 - M_0^2}
		\end{align}
		が成り立つ.
		\QED
	\end{prf}
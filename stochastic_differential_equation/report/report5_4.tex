	\begin{screen}
		\begin{thm}[二次変分の存在]
			任意の$M \in \mathcal{M}_{c,loc}$に対し或る$A \in \mathcal{A}^+$が存在して
			次を満たす:
			\begin{align}
				A_0 = 0\quad \mbox{$\mu$-a.s.},
				\quad M^2 - A \in \mathcal{M}_{c,loc}.
			\end{align}
			$A' \in \mathcal{A}^+$もまた上の主張を満たすときは$A$と$A'$は$\mu$-a.s.にパスが一致し,
			逆に$A,A' \in \mathcal{A}^+$が$\mu$-a.s.にパスが一致するならば$A'$も主張を満たす.
			\label{thm:existence_of_quadratic_variation}
		\end{thm}
	\end{screen}
	
	\begin{prf}
		まず$M \in \mathcal{M}_{b,c}$に対し$A$の存在を証明し,次にその結果を$\mathcal{M}_{c,loc}$に拡張する.
		\begin{description}
			\item[第一段]
				$M \in \mathcal{M}_{b,c}$とする.(\refeq{eq:lem_quadratic_variation_0})の$Q^n$を構成し
				$N^n \coloneqq M^2 - Q^n \in \mathcal{M}_{b,c}$とおけば
				\begin{align}
					\Norm{N_T^n}{\mathscr{L}^2} \leq 2 \sup{t \in I}{\Norm{M_t}{\mathscr{L}^\infty}} \Norm{M_T}{\mathscr{L}^2} \quad (n=1,2,\cdots)
				\end{align}
				が成り立つから,$N^n$の同値類
				\footnote{
					補題\ref{lem:M_2c_hilbert}で導入した同値関係$R$による同値類.
				}
				$\overline{N^n}$の列$(\overline{N^n})_{n=1}^{\infty}$はHilbert空間$\mathfrak{M}_{2,c}$において有界列となる.
				Kolmosの補題より$\overline{N^n}$の或る線型結合の列$\hat{\overline{N^n}}\ (n=1,2,\cdots)$が$\mathfrak{M}_{2,c}$においてCauchy列をなすから,
				その極限を$\overline{N} \in \mathfrak{M}_{2,c}$と表す.線型結合を
				\begin{align}
					\hat{\overline{N^n}} = \sum_{j=0}^{\infty} c^n_j \overline{N^{n+j}}, \quad
					\hat{N}^n \coloneqq \sum_{j=0}^{\infty} c^n_j N^{n+j}, \quad
					\hat{Q}^n \coloneqq \sum_{j=0}^{\infty} c^n_j Q^{n+j}
				\end{align}
				と表せば$\hat{N}^n = M^2 - \hat{Q}^n$を満たし
				\footnote{
					各$n \in \N$に対して$(c^n_j)_{j=0}^{\infty}$は
					$\sum_{j=0}^{\infty} c^n_j = 1$を満たし,且つ$\neq 0$であるのは有限個である.
				}
				,$N \in \overline{N}$を一つ取り
				\begin{align}
					A \coloneqq M^2 - N \label{eq:thm_quadratic_variation_0}
				\end{align}
				とおけば,Doobの不等式(定理\ref{thm:Doob_inequality_2})により
				\begin{align}
					&\Norm{\sup{t \in I}{\left| \hat{Q}_t^n - A_t \right|}}{\mathscr{L}^2}
					= \Norm{\sup{t \in I}{\left| N_t - \hat{N}_t^n \right|}}{\mathscr{L}^2} \\
					&\qquad \leq \Norm{N_T - \hat{N}_T^n}{\mathscr{L}^2}
					= \Norm{\overline{N} - \hat{\overline{N^n}}}{\mathfrak{M}_{2,c}} \longrightarrow 0 \quad (n \longrightarrow \infty) 
				\end{align}
				が成り立つ.
				\footnote{
					$\Norm{\cdot}{\mathfrak{M}_{2,c}}$は(\refeq{eq:M_2c_inner_product})で定義される内積により導入されるノルムを表す.
				}
				\begin{align}
					\Norm{\sup{t \in I}{\left| \hat{Q}_t^{n_k} - A_t \right|}}{\mathscr{L}^2} < \frac{1}{4^k} \quad (k=1,2,\cdots)
				\end{align}
				を満たすように部分列$(n_k)_{k=1}^{\infty}$を取り
				\begin{align}
					E_k \coloneqq \left\{\, \sup{t \in I}{\left| \hat{Q}_t^{n_k} - A_t \right|} \geq \frac{1}{2^k} \, \right\} \quad (k=1,2,\cdots)
				\end{align}
				とすると,Chebyshevの不等式より$\mu(E_k) < 1/2^k$を得る.
				\begin{align}
					E \coloneqq \bigcap_{N \in \N} \bigcup_{k \geq N} E_k
				\end{align}
				と定めればBorel-Cantelliの補題より$E$は$\mu$-零集合であり,
				\begin{align}
					\lim_{k \to \infty}\sup{t \in I}{\left| \hat{Q}_t^{n_k}(\omega) - A_t(\omega) \right|} = 0 
					\quad (\forall \omega \in \Omega \backslash E)
					\label{eq:thm_quadratic_variation_1}
				\end{align}
				が成り立ち,$A_0(\omega) = 0 \ (\forall \omega \in \Omega \backslash E)$が従う.
				\begin{align}
					D_k \coloneqq \Set{\frac{j}{2^{n_k}}T}{ j = 0,1,\cdots,2^{n_k} } \quad (k=1,2,\cdots)
				\end{align}
				とおけば,(\refeq{eq:lem_quadratic_variation_0})より全ての$v \geq k,\ \omega \in \Omega$に対して
				$t \longmapsto Q_t^{n_v}(\omega)$は$D_k$上で単調非減少となるから,
				その線型結合である$\hat{Q}_t^{n_v}$も$D_k$上で単調非減少となり,
				(\refeq{eq:thm_quadratic_variation_1})より
				$\omega \in \Omega \backslash E$に対しては$t \longmapsto A_t(\omega)$も$D_k$上で単調非減少となる
				\footnote{
					或る$j$と$u \in \Omega \backslash E$で$A_{\frac{j}{2^{n_k}}T}(u) > A_{\frac{j+1}{2^{n_k}}T}(u)$が成り立っているとする.
					式(\refeq{eq:thm_quadratic_variation_1})により,
					\begin{align}
						\alpha \coloneqq A_{\frac{j}{2^n}T}(u),
						\quad \beta \coloneqq A_{\frac{j+1}{2^n}T}(u)
					\end{align}
					とおけば或る$\nu \geq 1$が存在して
					\begin{align}
						\left| \hat{Q}_{\frac{j}{2^{n_k}}T}^{n_\nu}(u) - A_{\frac{j}{2^{n_k}}T}(u) \right| < \frac{\alpha - \beta}{2},
						\quad \left| \hat{Q}_{\frac{j+1}{2^{n_k}}T}^{n_\nu}(u) - A_{\frac{j+1}{2^{n_k}}T}(u) \right| < \frac{\alpha - \beta}{2}
					\end{align}
					を同時に満たすが,
					\begin{align}
						\hat{Q}_{\frac{j}{2^{n_k}}T}^{n_\nu}(u) > \frac{\alpha + \beta}{2} > \hat{Q}_{\frac{j+1}{2^{n_k}}T}^{n_\nu}(u)
					\end{align}
					が従うので$t \longmapsto \hat{Q}_t^{n_\nu}(u)$の単調増大性に矛盾する.
				}.
				$D \coloneqq \cup_{k=1}^{\infty} D_{n_k}$とおけば$D$は$I$で稠密であり,更に$A$は或る零集合$E'$を除いてパスが連続となるから
				\footnote{
					$M \in \mathcal{M}_{b,c},\ N \in \mathcal{M}_{2,c}$より$M,N$のパスが連続でない$\omega$の全体は或る零集合に含まれる.それを$E'$とおけばよい.
				}
				,写像$I \ni t \longmapsto A_t(\omega)\ (\forall \omega \in \Omega \backslash (E \cup E'))$は連続且つ単調非減少である.
				また(\refeq{eq:thm_quadratic_variation_0})より$A$は$(\mathcal{F}_t)$-適合でもあるから,以上より$A \in \mathcal{A}^+$である.
				$N = M^2 - A \in \mathcal{M}_{b,c} \subset \mathcal{M}_{c,loc}$(命題\ref{prp:M_pc_M_cloc})
				より$A$は定理の主張を満たす.
				存在の一意性は命題\ref{prp:bounded_continuous_M_2c_path}による.
				今$A' \in \mathcal{A}^+$もまた定理の主張を満たしているなら,
				$N' = M^2 - A'$として,$A - A' \in \mathcal{A}$かつ
				\begin{align}
					A - A' = N' - N \in \mathcal{M}_{2,c}
				\end{align}
				となるから$A_t - A'_t = 0\ (\forall t \in I)\quad \mbox{$\mu$-a.s.}$が従う.
				
			\item[第二段]
				$M \in \mathcal{M}_{c,loc}$を任意に取る.或る$(\tau_j)_{j=1}^{\infty} \in \mathcal{T}$が存在して
				$M^n \in \mathcal{M}_{b,c}\ (\forall t \in I,\ M_t^n = M_{t \wedge \tau_n})$となるから,
				前段の結果より或る$A^n \in \mathcal{A}^+$が存在して
				\begin{align}
					N^n \coloneqq (M^n)^2 - A^n \in \mathcal{M}_{b,c}
				\end{align}
				を満たす.或る$\mu$-零集合$E_1$が存在して
				\footnote{
					或る零集合$E_1^{(1)}$があり$\tau_0(\omega) = 0\ (\forall \omega \in \Omega \backslash E_1^{(1)})$,
					また或る零集合$E_1^j$があり$\tau_j(\omega) \leq \tau_{j+1}(\omega)\ (\forall \omega \in \Omega \backslash E_1^j)$,
					更に或る零集合$E_1^{(T)}$を取れば,各$\omega \in \Omega \backslash E_1^{(T)}$について
					或る$n(\omega)$番目以降は$\tau_n(\omega) = T\ (\forall n \geq n(\omega))$
					が成り立つ.
					\begin{align}
						E_1 = \left( \cup_{j=1}^{\infty} E_1^j \right) \cup E_1^{(1)} \cup E_1^{(T)}
					\end{align}
					とおけばよい.
				}
				全ての$\omega \in \Omega \backslash E_1$で$(\tau_n(\omega))_{n=1}^{\infty}$は$0$出発,単調非減少かつ或る$n(\omega)$番目以降は$T$に一致する.
				今任意に$n \leq m,\ n,m \in \N$を取って固定する.$\omega \in \Omega \backslash E_1$に対しては
				\begin{align}
					M_{t \wedge \tau_n(\omega)}^m(\omega) = M_{t \wedge \tau_n(\omega) \wedge \tau_m(\omega)}(\omega) = M_t^n(\omega) \quad (\forall t \in I)
				\end{align}
				となり関数類として$[M_{t \wedge \tau_n}^m] = [M_t^n]\ (\forall t \in I)$が成り立つから,任意抽出定理(定理\ref{thm:optional_sampling_theorem_2})より
				\begin{align}
					\cexp{(M_t^m)^2 - A_t^m}{\mathcal{F}_{\tau_n}} = (M_{t \wedge \tau_n}^m)^2 - A_{t \wedge \tau_n}^m = (M_t^n)^2 - A_{t \wedge \tau_n}^m
					\quad (\forall t \in I)
				\end{align}
				を得る.一方で$N^m \in \mathcal{M}_{2,c}$であるから
				\begin{align}
					\cexp{(M_t^m)^2 - A_t^m}{\mathcal{F}_{\tau_n}} = \cexp{N_t^m}{\mathcal{F}_{\tau_n}} = N_{t \wedge \tau_n}^m \quad (\forall t \in I)
				\end{align}
				も成り立ち,任意抽出定理(定理\ref{thm:optional_sampling_theorem_2})より$\tau_n$で停めた過程も二乗可積分マルチンゲールとなる.
				関数類としての意味を強調すれば$[(M_t^n)^2 - A_{t \wedge \tau_n}^m] = [N_{t \wedge \tau_n}^m]\ (\forall t \in I)$が成り立っているから
				$\left( (M_t^n)^2 - A_{t \wedge \tau_n}^m \right)_{t \in I}$が二乗可積分マルチンゲールであることになり,前段の一意性から
				或る$\mu$-零集合$E^{n,m}$が存在して
				\begin{align}
					A_{t \wedge \tau_n(\omega)}^m(\omega) = A_t^n(\omega) \quad (\forall t \in I,\ \omega \in \Omega \backslash E^{n,m})
				\end{align}
				が従う.特に$\omega \in \Omega \backslash (E_1 \cup E^{n,m})$なら
				\begin{align}
					A_t^n(\omega) = A_{t \wedge \tau_n(\omega)}^m(\omega) = A_t^m(\omega) \quad (t \leq \tau_n(\omega))
				\end{align}
				となる.
				\begin{align}
					E_2 \coloneqq \bigcup_{n \in \N} E^{n,n+1}
				\end{align}
				とおき
				\begin{align}
					A_t(\omega) \coloneqq
					\begin{cases}
						\lim_{n \to \infty} A_t^n(\omega) & (\omega \in \Omega \backslash (E_1 \cup E_2)) \\
						0 & (\omega \in E_1 \cup E_2)
					\end{cases}
					\quad (\forall t \in I)
				\end{align}
				として$A$を定めれば$A \in \mathcal{A}^+$を満たす
				\footnote{
					\begin{description}
						\item[連続性・単調非減少性]
							$\omega \in \Omega \backslash (E_1 \cup E_2)$の場合に確認する.任意に$s,u \in I,\ (s < u)$を取れば
							$u \leq \tau_n(\omega)$となる$n$が存在し$A_t(\omega) = A_t^n(\omega)\ (\forall t \leq \tau_n(\omega))$を満たす.
							写像$I \ni t \longmapsto A_t^n(\omega)$は連続且つ単調非減少であるから
							$t \longmapsto A_t(\omega)$も$t = s,u$において連続であり,且つ$A_s(\omega) = A_s^n(\omega) \leq A_u^n(\omega) = A_u(\omega)$
							により単調非減少である.
						
						\item[適合性]
							各$n \in \N$に対し$A^n$は$(\mathcal{F}_t)$-適合である.$t \in I$を固定し
							\begin{align}
								\tilde{\mathcal{F}}_t \coloneqq \Set{B \cap (E_1 \cup E_2)^c}{B \in \mathcal{F}_t}
							\end{align}
							とおく.写像$\Omega \ni \omega \longmapsto A_t^n(\omega)$を
							$\Omega \backslash (E_1 \cup E_2)$に制限した$\tilde{A}_t^n \coloneqq A_t^n|_{\Omega \backslash (E_1 \cup E_2)}$は可測$\tilde{\mathcal{F}}_t/\borel{\R}$
							であり,各点収束先の$\tilde{A}_t \coloneqq A_t|_{\Omega \backslash (E_1 \cup E_2)}$もまた可測$\tilde{\mathcal{F}}_t/\borel{\R}$となる.
							任意の$C \in \borel{\R}$に対して
							\begin{align}
								A_t^{-1}(C) =
								\begin{cases}
									\tilde{A}_t^{-1}(C) & (0 \notin C) \\
									(E_1 \cup E_2) \cup \tilde{A}_t^{-1}(C) & (0 \in C)
								\end{cases}
							\end{align}
							となり,$E_1 \cup E_2 \in \mathcal{F}_0$により$\tilde{\mathcal{F}}_t \subset \mathcal{F}_t$であるから
							$A_t$は可測$\mathcal{F}_t/\borel{\R}$となる.
					\end{description}
				}
				.そして$N \coloneqq M^2 - A$とおけば$\Omega \backslash (E_1 \cup E_2)$上で
				\begin{align}
					N_{t \wedge \tau_n} = M_{t \wedge \tau_n}^2 - A_{t \wedge \tau_n} 
					= \left( M_{t \wedge \tau_n}^n \right)^2 - A_{t \wedge \tau_n}^n
					= N_{t \wedge \tau_n}^n \quad (\forall t \in I,\ n \in \N)
				\end{align}
				が成り立つから$N \in \mathcal{M}_{c,loc}$となる.$A$の一意性について,
		\end{description}
	\end{prf}
	
	\begin{screen}
		\begin{dfn}[二次変分]
			$M \in \mathcal{M}_{c,loc}$に対して定理\ref{thm:existence_of_quadratic_variation}より
			存在する$A \in \mathcal{A}^+$のうち,全てのパスが$0$出発,連続,単調非減少であるものを
			$M$の二次変分(quadratic variation)と呼び$\inprod<M>$と表す.また
			$M,N \in \mathcal{M}_{c,loc}$に対して
			\begin{align}
				\inprod<M,N> \coloneqq \frac{1}{4} (\inprod<M+N> - \inprod<M-N>)
			\end{align}
			と定義して$M,N$の共変分と呼ぶ.
		\end{dfn}
	\end{screen}
	
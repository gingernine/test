	\begin{screen}
		\begin{thm}[二次変分の存在]
			任意の$M \in \mathcal{M}_{c,loc}$に対し或る$A \in \mathcal{A}^+$が存在して
			次を満たす:
			\begin{align}
				A_0 = 0\quad \mbox{$\mu$-a.s.},
				\quad M^2 - A \in \mathcal{M}_{c,loc}.
			\end{align}
			$A' \in \mathcal{A}^+$もまた上の主張を満たすときは$A$と$A'$は$\mu$-a.s.にパスが一致し,
			逆に$A,A' \in \mathcal{A}^+$が$\mu$-a.s.にパスが一致するならば$A'$も主張を満たす.
			また$M,M' \in \mathcal{M}_{c,loc}$が$\mu$-a.s.にパスが一致するなら,
			それぞれに対応する$A,A' \in \mathcal{A}^+$も$\mu$-a.s.にパスが一致する.
			特に$M \in \mathcal{M}_{p,c}\ (p \geq 2)$に対しては,対応する$A \in \mathcal{A}^+$は可積分である.
			\label{thm:existence_of_quadratic_variation}
		\end{thm}
	\end{screen}
	
	\begin{prf}
		まず$M \in \mathcal{M}_{b,c}$に対し$A$の存在を証明し,次にその結果を$\mathcal{M}_{c,loc}$に拡張する.
		\begin{description}
			\item[第一段]
				$M \in \mathcal{M}_{b,c}$とする.(\refeq{eq:lem_quadratic_variation_0})の$Q^n$を構成し
				$N^n \coloneqq M^2 - Q^n \in \mathcal{M}_{b,c}$とおけば
				\begin{align}
					\Norm{N_T^n}{\mathscr{L}^2} \leq 2 \sup{t \in I}{\Norm{M_t}{\mathscr{L}^\infty}} \Norm{M_T}{\mathscr{L}^2} \quad (n=1,2,\cdots)
				\end{align}
				が成り立つから,$N^n$の同値類
				\footnote{
					補題\ref{lem:M_2c_hilbert}で導入した同値関係$R$による同値類.
				}
				$\overline{N^n}$の列$(\overline{N^n})_{n=1}^{\infty}$はHilbert空間$\mathfrak{M}_{2,c}$において有界列となる.
				Kolmosの補題より$\overline{N^n}$の或る線型結合の列$\hat{\overline{N^n}}\ (n=1,2,\cdots)$が$\mathfrak{M}_{2,c}$においてCauchy列をなすから,
				その極限を$\overline{N} \in \mathfrak{M}_{2,c}$と表す.凸結合を
				\begin{align}
					\hat{\overline{N^n}} = \sum_{j=0}^{\infty} c^n_j \overline{N^{n+j}}, \quad
					\hat{N}^n \coloneqq \sum_{j=0}^{\infty} c^n_j N^{n+j}, \quad
					\hat{Q}^n \coloneqq \sum_{j=0}^{\infty} c^n_j Q^{n+j}
				\end{align}
				と表せば$\hat{N}^n = M^2 - \hat{Q}^n$を満たし
				\footnote{
					各$n \in \N$に対して$(c^n_j)_{j=0}^{\infty}$は
					$\sum_{j=0}^{\infty} c^n_j = 1$を満たし,且つ$\neq 0$であるのは有限個である.
				}
				,$N \in \overline{N}$を一つ取り
				\begin{align}
					A \coloneqq M^2 - N \label{eq:thm_quadratic_variation_0}
				\end{align}
				とおけば,Doobの不等式(定理\ref{thm:Doob_inequality_2})により
				\begin{align}
					&\Norm{\sup{t \in I}{\left| \hat{Q}_t^n - A_t \right|}}{\mathscr{L}^2}
					= \Norm{\sup{t \in I}{\left| N_t - \hat{N}_t^n \right|}}{\mathscr{L}^2} \\
					&\qquad \leq \Norm{N_T - \hat{N}_T^n}{\mathscr{L}^2}
					= \Norm{\overline{N} - \hat{\overline{N^n}}}{\mathfrak{M}_{2,c}} \longrightarrow 0 \quad (n \longrightarrow \infty) 
				\end{align}
				が成り立つ.
				\footnote{
					$\Norm{\cdot}{\mathfrak{M}_{2,c}}$は(\refeq{eq:M_2c_inner_product})で定義される内積により導入されるノルムを表す.
				}
				\begin{align}
					\Norm{\sup{t \in I}{\left| \hat{Q}_t^{n_k} - A_t \right|}}{\mathscr{L}^2} < \frac{1}{4^k} \quad (k=1,2,\cdots)
				\end{align}
				を満たすように部分列$(n_k)_{k=1}^{\infty}$を取り
				\begin{align}
					E_k \coloneqq \left\{\, \sup{t \in I}{\left| \hat{Q}_t^{n_k} - A_t \right|} \geq \frac{1}{2^k} \, \right\} \quad (k=1,2,\cdots)
				\end{align}
				とすると,Chebyshevの不等式より$\mu(E_k) < 1/2^k$を得る.
				\begin{align}
					E \coloneqq \bigcap_{N \in \N} \bigcup_{k \geq N} E_k
				\end{align}
				と定めればBorel-Cantelliの補題より$E$は$\mu$-零集合であり,
				\begin{align}
					\lim_{k \to \infty}\sup{t \in I}{\left| \hat{Q}_t^{n_k}(\omega) - A_t(\omega) \right|} = 0 
					\quad (\forall \omega \in \Omega \backslash E)
					\label{eq:thm_quadratic_variation_1}
				\end{align}
				が成り立ち,$A_0(\omega) = 0 \ (\forall \omega \in \Omega \backslash E)$が従う.
				\begin{align}
					D_k \coloneqq \Set{\frac{j}{2^{n_k}}T}{ j = 0,1,\cdots,2^{n_k} } \quad (k=1,2,\cdots)
				\end{align}
				とおけば,(\refeq{eq:lem_quadratic_variation_0})より全ての$v \geq k,\ \omega \in \Omega$に対して
				$t \longmapsto Q_t^{n_v}(\omega)$は$D_k$上で単調非減少となるから,
				その線型結合である$\hat{Q}_t^{n_v}$も$D_k$上で単調非減少となり,
				(\refeq{eq:thm_quadratic_variation_1})より
				$\omega \in \Omega \backslash E$に対しては$t \longmapsto A_t(\omega)$も$D_k$上で単調非減少となる
				\footnote{
					或る$j$と$u \in \Omega \backslash E$で$A_{\frac{j}{2^{n_k}}T}(u) > A_{\frac{j+1}{2^{n_k}}T}(u)$が成り立っているとする.
					式(\refeq{eq:thm_quadratic_variation_1})により,
					\begin{align}
						\alpha \coloneqq A_{\frac{j}{2^n}T}(u),
						\quad \beta \coloneqq A_{\frac{j+1}{2^n}T}(u)
					\end{align}
					とおけば或る$\nu \geq 1$が存在して
					\begin{align}
						\left| \hat{Q}_{\frac{j}{2^{n_k}}T}^{n_\nu}(u) - A_{\frac{j}{2^{n_k}}T}(u) \right| < \frac{\alpha - \beta}{2},
						\quad \left| \hat{Q}_{\frac{j+1}{2^{n_k}}T}^{n_\nu}(u) - A_{\frac{j+1}{2^{n_k}}T}(u) \right| < \frac{\alpha - \beta}{2}
					\end{align}
					を同時に満たすが,
					\begin{align}
						\hat{Q}_{\frac{j}{2^{n_k}}T}^{n_\nu}(u) > \frac{\alpha + \beta}{2} > \hat{Q}_{\frac{j+1}{2^{n_k}}T}^{n_\nu}(u)
					\end{align}
					が従うので$t \longmapsto \hat{Q}_t^{n_\nu}(u)$の単調増大性に矛盾する.
				}.
				$D \coloneqq \cup_{k=1}^{\infty} D_{n_k}$とおけば$D$は$I$で稠密であり,更に$A$は或る零集合$E'$を除いてパスが連続となるから
				\footnote{
					$M \in \mathcal{M}_{b,c},\ N \in \mathcal{M}_{2,c}$より$M,N$のパスが連続でない$\omega$の全体は或る零集合に含まれる.それを$E'$とおけばよい.
				}
				,写像$I \ni t \longmapsto A_t(\omega)\ (\forall \omega \in \Omega \backslash (E \cup E'))$は連続且つ単調非減少である.
				また(\refeq{eq:thm_quadratic_variation_0})より$A$は$(\mathcal{F}_t)$-適合でもあるから,以上より$A \in \mathcal{A}^+$である.
				$N = M^2 - A \in \mathcal{M}_{2,c} \subset \mathcal{M}_{c,loc}$(命題\ref{prp:M_pc_M_cloc})
				より$A$は定理の主張を満たす.
				存在の一意性は命題\ref{prp:bounded_continuous_M_2c_path}による.
				今$A' \in \mathcal{A}^+$もまた定理の主張を満たしているなら,
				$N' = M^2 - A'$として,$A - A' \in \mathcal{A}$かつ
				\begin{align}
					A - A' = N' - N \in \mathcal{M}_{2,c}
				\end{align}
				となるから$A_t - A'_t = 0\ (\forall t \in I)\quad \mbox{$\mu$-a.s.}$が従う.
				
			\item[第二段]
				$M \in \mathcal{M}_{c,loc}$を任意に取る.或る$(\tau_j)_{j=1}^{\infty} \in \mathcal{T}$が存在して
				$M^{\tau_j} \in \mathcal{M}_{b,c}$を満たすから,
				前段の結果より或る$A^j \in \mathcal{A}^+$が存在して
				\begin{align}
					N^j \coloneqq \left( M^{\tau_j} \right)^2 - A^j \in \mathcal{M}_{2,c}
				\end{align}
				が成り立つ.或る$\mu$-零集合$E_1$が存在して
				\footnote{
					或る零集合$E_1^{(1)}$があり$\tau_0(\omega) = 0\ (\forall \omega \in \Omega \backslash E_1^{(1)})$,
					また或る零集合$E_1^j$があり$\tau_j(\omega) \leq \tau_{j+1}(\omega)\ (\forall \omega \in \Omega \backslash E_1^j)$,
					更に或る零集合$E_1^{(T)}$を取れば,各$\omega \in \Omega \backslash E_1^{(T)}$について
					或る$J(\omega)$番目以降は$\tau_j(\omega) = T\ (\forall j \geq J(\omega))$
					が成り立つ.よって
					\begin{align}
						E_1 = \left( \cup_{j=1}^{\infty} E_1^j \right) \cup E_1^{(1)} \cup E_1^{(T)}
					\end{align}
					とおけばよい.
				}
				,全ての$\omega \in \Omega \backslash E_1$に対し$(\tau_j(\omega))_{j=1}^{\infty}$は$0$出発,
				単調非減少かつ或る$J=J(\omega)$番目以降は$\tau_j(\omega) = T\ (\forall j \geq J)$を満たす.
				$n \leq m$となるように任意に$n,m \in \N$を取って固定すれば,全ての$\omega \in \Omega \backslash E_1$に対して
				\begin{align}
					M_{t \wedge \tau_n(\omega)}^{\tau_m}(\omega) = M_{t \wedge \tau_n(\omega) \wedge \tau_m(\omega)}(\omega) = M_t^{\tau_n}(\omega) 
					\quad (\forall t \in I)
				\end{align}
				が従うから,各$t \in I$で$M_{t \wedge \tau_n}^m$と$M_t^n$の関数類が一致し,
				任意抽出定理(定理\ref{thm:optional_sampling_theorem_2})より
				\begin{align}
					\cexp{\left(M_t^{\tau_m}\right)^2 - A_t^m}{\mathcal{F}_{\tau_n}} 
					= \left(M_{t \wedge \tau_n}^{\tau_m}\right)^2 - A_{t \wedge \tau_n}^m 
					= \left(M_t^{\tau_n}\right)^2 - A_{t \wedge \tau_n}^m
					\quad (\forall t \in I)
				\end{align}
				が得られる.定理\ref{thm:stopped_process_martingale}より
				$\left(N_{t \wedge \tau_n}^m \right)_{t \in I} \in \mathcal{M}_{2,c}$
				となるから$\left(M^{\tau_n}\right)^2 - (A^m)^{\tau_n} \in \mathcal{M}_{2,c}$
				が従い,一方$N^n = \left(M^{\tau_n}\right)^2 - A^n \in \mathcal{M}_{2,c}$であるから,
				前段の考察より或る$\mu$-零集合$E^{n,m}$が存在して
				\begin{align}
					A_t^n(\omega) = A_{t \wedge \tau_n(\omega)}^m(\omega) \quad (\forall t \in I,\ \omega \in \Omega \backslash E^{n,m})
				\end{align}
				が得られ,特に
				\begin{align}
					A_{t \wedge \tau_n(\omega)}^n(\omega) = A_{t \wedge \tau_n(\omega)}^m(\omega) \quad (\forall t \in I,\ \omega \in \Omega \backslash E^{n,m})
				\end{align}
				が成り立つ.
				\begin{align}
					E_2 \coloneqq \bigcup_{\substack{n,m \in \N \\ n \leq m}} E^{n,m}
				\end{align}
				に対し
				\begin{align}
					A_t(\omega) \coloneqq
					\begin{cases}
						\lim_{n \to \infty} A_{t \wedge \tau_n(\omega)}^n(\omega) & (\omega \in \Omega \backslash (E_1 \cup E_2)) \\
						0 & (\omega \in E_1 \cup E_2)
					\end{cases}
					\quad (\forall t \in I)
				\end{align}
				として$A$を定めれば$A \in \mathcal{A}^+$を満たす
				\footnote{
					\begin{description}
						\item[連続性・単調非減少性]
							$\omega \in \Omega \backslash (E_1 \cup E_2)$の場合に確認する.任意に$s,u \in I,\ (s < u)$を取れば
							$u \leq \tau_n(\omega)$となる$n$が存在し$A_t(\omega) = A_t^n(\omega)\ (\forall t \leq \tau_n(\omega))$を満たす.
							写像$I \ni t \longmapsto A_t^n(\omega)$は連続且つ単調非減少であるから
							$t \longmapsto A_t(\omega)$も$t = s,u$において連続であり,且つ$A_s(\omega) = A_s^n(\omega) \leq A_u^n(\omega) = A_u(\omega)$
							により単調非減少である.
						
						\item[適合性]
							各$n \in \N$に対し$A^n$は$(\mathcal{F}_t)$-適合である.$t \in I$を固定し
							\begin{align}
								\tilde{\mathcal{F}}_t \coloneqq \Set{B \cap (E_1 \cup E_2)^c}{B \in \mathcal{F}_t}
							\end{align}
							とおく.写像$\Omega \ni \omega \longmapsto A_t^n(\omega)$を
							$\Omega \backslash (E_1 \cup E_2)$に制限した$\tilde{A}_t^n \coloneqq A_t^n|_{\Omega \backslash (E_1 \cup E_2)}$は可測$\tilde{\mathcal{F}}_t/\borel{\R}$
							であり,各点収束先の$\tilde{A}_t \coloneqq A_t|_{\Omega \backslash (E_1 \cup E_2)}$もまた可測$\tilde{\mathcal{F}}_t/\borel{\R}$となる.
							任意の$C \in \borel{\R}$に対して
							\begin{align}
								A_t^{-1}(C) =
								\begin{cases}
									\tilde{A}_t^{-1}(C) & (0 \notin C) \\
									(E_1 \cup E_2) \cup \tilde{A}_t^{-1}(C) & (0 \in C)
								\end{cases}
							\end{align}
							となり,$E_1 \cup E_2 \in \mathcal{F}_0$により$\tilde{\mathcal{F}}_t \subset \mathcal{F}_t$であるから
							$A_t$は可測$\mathcal{F}_t/\borel{\R}$となる.
					\end{description}
				}
				.そして$N \coloneqq M^2 - A$とおけば$\Omega \backslash (E_1 \cup E_2)$上で
				\begin{align}
					N_{t \wedge \tau_n} = M_{t \wedge \tau_n}^2 - A_{t \wedge \tau_n} 
					= \left( M_t^{\tau_n} \right)^2 - A_{t \wedge \tau_n}^n
					= N_{t \wedge \tau_n}^n \quad (\forall t \in I,\ n \in \N)
				\end{align}
				が成り立つから$N \in \mathcal{M}_{c,loc}$となる.$A$の一意性について,
				
			\item[第三段]
				後半の主張を示す.$M \in \mathcal{M}_{p,c}\ (p \geq 2)$の場合,
				命題\ref{prp:M_pc_M_cloc}より$M \in \mathcal{M}_{c,loc}$であるから,
				或る$A \in \mathcal{A}^+$が存在して$M^2 - A \in \mathcal{M}_{c,loc}$となる.
				従って或る$(\tau_j)_{j=0}^{\infty} \in \mathcal{T}$が存在して
				$\left(M^{\tau_j} \right)^2 - A^{\tau_j} \in \mathcal{M}_{b,c}\ (j=0,1,\cdots)$
				を満たすから,任意に$t \in I$を固定すれば
				\begin{align}
					\int_{\Omega} \left( M_{t \wedge \tau_j(\omega)}(\omega) \right)^2\ \mu(d\omega)
					= \int_{\Omega} A_{t \wedge \tau_j(\omega)}(\omega)\ \mu(d\omega)
					\quad (\forall j=0,1,\cdots)
				\end{align}
				が成り立つ.或る零集合$E$が存在して,$\omega \in \Omega \backslash E$なら
				$A_0(\omega) = 0$,$I \ni t \longmapsto A_t(\omega)$は連続且つ単調非減少,
				更に$0 = \tau_0(\omega) \leq \tau_1(\omega) \leq \cdots \leq \tau_{J}(\omega) = T\ (\exists J = J(\omega))$
				が満たされるから,$\left(A_{t \wedge \tau_j(\omega)}(\omega)\right)_{j=0}^{\infty}$は単調増大列である.
				またDoobの不等式により
				$\left| M_{t \wedge \tau_j} \right| \leq \sup{t \in I}{|M_t|} \in \mathscr{L}^p \subset \mathscr{L}^2$
				も満たされているから,Lebesgueの収束定理と単調収束定理より
				\begin{align}
					\int_{\Omega} \left( M_t(\omega) \right)^2\ \mu(d\omega)
					= \int_{\Omega} A_t(\omega)\ \mu(d\omega) < \infty
				\end{align}
				が得られる.
		\end{description}
	\end{prf}
	
	\begin{screen}
		\begin{dfn}[二次変分]
			$M \in \mathcal{M}_{c,loc}$に対して定理\ref{thm:existence_of_quadratic_variation}より
			存在する$A \in \mathcal{A}^+$のうち,全てのパスが$0$出発,連続,単調非減少であるものを
			$M$の二次変分(quadratic variation)と呼び$\inprod<M>$と表す.また
			$M,N \in \mathcal{M}_{c,loc}$に対して
			\begin{align}
				\inprod<M,N> \coloneqq \frac{1}{4} (\inprod<M+N> - \inprod<M-N>)
			\end{align}
			と定義して$M,N$の共変分と呼ぶ.
		\end{dfn}
	\end{screen}
	
	\begin{screen}
		\begin{thm}[二次変分が有界な局所マルチンゲールは二乗可積分マルチンゲール]
			$M \in \mathcal{M}_{c,loc}$かつ$\Norm{\inprod<M>_T}{\mathscr{L}^{\infty}} < \infty$
			であるならば,$M \in \mathcal{M}_{2,c}$が成り立つ.
			\label{thm:quadratic_variation_bounded_then_M_2c}
		\end{thm}
	\end{screen}
	
	\begin{prf} マルチンゲール性の定義に従い,以下三段階に分けて証明する.
		\begin{description}
			\item[第一段] 任意の$t \in I$に対し$M_t$が二乗可積分であることを示す.
				定理\ref{thm:existence_of_quadratic_variation}より,或る$(\tau_j)_{j=0}^{\infty} \in \mathcal{T}$が存在して
				$\left(M^{\tau_j} \right)^2 - \inprod<M>^{\tau_j} \in \mathcal{M}_{b,c}\ (j=0,1,\cdots)$を満たす.
				そして$M^2_0 - \inprod<M>_0 = 0\ \mu$-a.s.も満たされているから,マルチンゲール性より任意の$t \in I,\ j=0,1,\cdots$に対し
				\begin{align}
					\int_\Omega \left( M_{t \wedge \tau_j(\omega)}(\omega) \right)^2\ \mu(d\omega)
					= \int_\Omega \inprod<M>_{t \wedge \tau_j(\omega)}(\omega)\ \mu(d\omega)
					\label{thm:thm_quadratic_variation_bounded_then_M_2c_1}
				\end{align}
				が成り立つ.$(\tau_j)_{j=0}^{\infty}$に対して或る零集合$E$が存在し,$\omega \in \Omega \backslash E$なら
				$0 = \tau_0(\omega) \leq \tau_1(\omega) \leq \cdots \leq \tau_{J}(\omega) = T = \tau_{J+1}(\omega) = \tau_{J+2}(\omega) = \cdots\ (\exists J = J(\omega))$
				が満たされる.また全ての$\omega \in \Omega$に対し$I \ni t \longmapsto \inprod<M>_t(\omega)$が
				非負・連続・単調非減少であるから,任意の$t \in I$に対し
				\begin{align}
					\inprod<M>_{t \wedge \tau_j(\omega)}(\omega) \leq \inprod<M>_T(\omega)
					\quad (\forall \omega \in \Omega \backslash E,\ j=0,1,\cdots)
				\end{align}
				が成り立つ.$\inprod<M>_T$が可積分であるからLebesgueの収束定理より
				\begin{align}
					\int_\Omega \inprod<M>_t(\omega)\ \mu(d\omega)
					= \lim_{j \to \infty} \int_\Omega \inprod<M>_{t \wedge \tau_j(\omega)}(\omega)\ \mu(d\omega)
				\end{align}
				が成り立ち,(\refeq{thm:thm_quadratic_variation_bounded_then_M_2c_1})においてFatouの補題を使えば
				\begin{align}
					&\int_\Omega \left( M_t(\omega) \right)^2\ \mu(d\omega)
					= \int_\Omega \liminf_{j \to \infty} \left( M_{t \wedge \tau_j(\omega)}(\omega) \right)^2\ \mu(d\omega) \\
					&\qquad \leq \liminf_{j \to \infty} \int_\Omega \inprod<M>_{t \wedge \tau_j(\omega)}(\omega)\ \mu(d\omega)
					= \int_\Omega \inprod<M>_t(\omega)\ \mu(d\omega)
					\leq \Norm{\inprod<M>_T}{\mathscr{L}^{\infty}}
				\end{align}
				が得られ$M_t$の二乗可積分性が従う.
			
			\item[第二段] $I \ni t \longmapsto M_t(\omega)$について,定義より
				\footnote{
					講義中の$\mathcal{M}_{c,loc}$定義には,「全ての$\omega \in \Omega$に対し
					$I \ni t \longmapsto M_t(\omega)$が
					各点$t$で右連続且つ左極限を持つ」とは定められていませんでしたが,
					マルチンゲールの定義にはパスが右連続且つ左極限を持つことが含まれています.
					講義中の$\mathcal{M}_{c,loc}$の定義だとこのことが示せなかったので,
					勝手に自分の方で定義に入れました.
				}
				任意の$\omega \in \Omega$に対し
				各点$t$で右連続且つ左極限を持つから,以降は$\mbox{$\mu$-a.s.}\omega \in \Omega$に対し連続且つ0出発であることを示す.
				$M$に対し或る$\left( \sigma_k \right)_{k=0}^{\infty} \in \mathcal{T}$が存在して
				$M^{\sigma_k} \in \mathcal{M}_{b,c}\ (k=0,1,\cdots)$を満たすから,
				先ず$M_0 = M^{\sigma_k}_0 = 0\ \mu$-a.s.が従う.
				また或る零集合$E'$が存在して$\omega \in \Omega \backslash E'$なら
				$0 = \sigma_0(\omega) \leq \sigma_1(\omega) \leq \cdots \leq \sigma_{K}(\omega) = T = \sigma_{K+1}(\omega) = \sigma_{K+2}(\omega) = \cdots\ (\exists K = K(\omega))$
				が成り立ち,一方で各$k =0,1,\cdots$に対し或る零集合$E_k^{''}$が存在し,$\omega \in \Omega \backslash E_k^{''}$ならば
				$I \longmapsto M^{\hat{\sigma}_k}_t(\omega)$が連続となる.
				\begin{align}
					E^{''} \coloneqq \bigcup_{k=0}^{\infty} E_k^{''}
				\end{align}
				とおけば,$\omega \in \Omega \backslash (E' \cup E'')$ならば
				必ず或る$K = K(\omega)$が存在して$\sigma_K(\omega) = T$を満たし,
				$I \ni t \longmapsto M_t(\omega)$は$I \ni t \longmapsto M^{\sigma_K}_t(\omega)$
				に一致する.$I \ni t \longmapsto M^{\sigma_K}_t(\omega)$が連続であるから
				$I \ni t \longmapsto M_t(\omega)$の連続性が従い,
				よって$\mu$-a.s.に$I \ni t \longmapsto M_t$は連続である.
				
			\item[第三段] 任意の$s,t \in I\ (s < t)$に対し
				\begin{align}
					\int_A M_t(\omega)\ \mu(d\omega) = \int_A M_s(\omega)\ \mu(d\omega)
					\quad (\forall A \in \mathcal{F}_s)
					\label{thm:thm_quadratic_variation_bounded_then_M_2c_2}
				\end{align}
				が成り立つことを示す.前段の$\left( \sigma_k \right)_{k=0}^{\infty} \in \mathcal{T}$を取れば
				$M_{t \wedge \sigma_k} \longrightarrow M_t\ (k \longrightarrow \infty,\ \mbox{$\mu$-a.s.})$かつ
				$M^{\sigma_k} \in \mathcal{M}_{b,c}\ (k=0,1,\cdots)$が満たされるから,任意の$k =0,1,\cdots$に対して
				\begin{align}
					\int_A M_{t \wedge \sigma_k(\omega)}(\omega)\ \mu(d\omega) 
					= \int_A M_{s \wedge \sigma_k(\omega)}(\omega)\ \mu(d\omega)
					\quad (\forall A \in \mathcal{F}_s)
				\end{align}
				が成り立つ.第一段の結果とDoobの不等式より
				$\left| M_{t \wedge \sigma_k} \right| \leq \sup{t \in I}{\left| M_t \right|} \in \mathscr{L}^2 \subset \mathscr{L}^1$
				が満たされるから,
				Lebesgueの収束定理より(\refeq{thm:thm_quadratic_variation_bounded_then_M_2c_2})が得られる.
				\QED
		\end{description}
	\end{prf}
	\begin{screen}
		\begin{thm}[二次変分の存在]\mbox{}
			\begin{description}
				\item[(1)]
					任意の$M \in \mathcal{M}_{c,loc}$に対し或る$A \in \mathcal{A}^+$が存在して
					次を満たす:
					\begin{align}
						A_0 = 0\quad \mbox{$\mu$-a.s.},
						\quad M^2 - A \in \mathcal{M}_{c,loc}.
						\label{eq:thm_existence_of_quadratic_variation_0}
					\end{align}
				\item[(2)]
					$A' \in \mathcal{A}^+$も(\refeq{eq:thm_existence_of_quadratic_variation_0})を満たすなら$A$と$A'$は$\mu$-a.s.にパスが一致し,
					逆に$A,A' \in \mathcal{A}^+$が$\mu$-a.s.にパスが一致するなら$A'$も(\refeq{eq:thm_existence_of_quadratic_variation_0})を満たす.
					また$M,M' \in \mathcal{M}_{c,loc}$が$\mu$-a.s.にパスが一致するなら,
					各々に対し(\refeq{eq:thm_existence_of_quadratic_variation_0})を満たす$A,A' \in \mathcal{A}^+$も$\mu$-a.s.にパスが一致する.
				\item[(3)]
					$M \in \mathcal{M}_{p,c}\ (p \geq 2)$の場合,(\refeq{eq:thm_existence_of_quadratic_variation_0})を満たす$A \in \mathcal{A}^+$は可積分である.
				
				\item[(4)]
					$M \in \mathcal{M}_{b,c}$の場合,(\refeq{eq:thm_existence_of_quadratic_variation_0})を満たす$A \in \mathcal{A}^+$により次の不等式が成り立つ:
					\begin{align}
						\Norm{M_T^2 - A_T}{\mathscr{L}^2} \leq 2 \sup{t \in I}{\Norm{M_t}{\mathscr{L}^\infty}} \Norm{M_T}{\mathscr{L}^2}.
					\end{align}
			\end{description}
			\label{thm:existence_of_quadratic_variation}
		\end{thm}
	\end{screen}
	
	\begin{prf}
		まず$M \in \mathcal{M}_{b,c}$に対し$A$の存在を証明し,次にその結果を$\mathcal{M}_{c,loc}$に拡張する.
		\begin{description}
			\item[第一段]
				$M \in \mathcal{M}_{b,c}$とする.(\refeq{eq:lem_quadratic_variation_0})の$Q^n$を構成し
				$N^n \coloneqq M^2 - Q^n \in \mathcal{M}_{b,c}$とおけば
				\begin{align}
					\Norm{N_T^n}{\mathscr{L}^2} \leq 2 \sup{t \in I}{\Norm{M_t}{\mathscr{L}^\infty}} \Norm{M_T}{\mathscr{L}^2} \quad (n=1,2,\cdots)
					\label{eq:thm_existence_of_quadratic_variation_3}
				\end{align}
				が成り立つから,$N^n$の$\mathcal{M}_{2,c}$における同値類
				$\equiv{N^n}{2,c}$の列$\left( \equiv{N^n}{2,c} \right)_{n=1}^{\infty}$はHilbert空間$\mathfrak{M}_{2,c}$において有界列である.
				従ってKolmosの補題より$\equiv{N^n}{2,c}$の或る凸結合の列$\hat{\equiv{N^n}{2,c}}\ (n=1,2,\cdots)$が$\mathfrak{M}_{2,c}$においてCauchy列をなし,
				極限$\equiv{N}{2,c} \in \mathfrak{M}_{2,c}$が存在する.凸結合を
				\begin{align}
					\hat{\equiv{N^n}{2,c}} = \sum_{j=0}^{\infty} c^n_j \equiv{N^{n+j}}{2,c}, \quad
					\hat{N}^n \coloneqq \sum_{j=0}^{\infty} c^n_j N^{n+j}, \quad
					\hat{Q}^n \coloneqq \sum_{j=0}^{\infty} c^n_j Q^{n+j}
				\end{align}
				と表せば$\hat{N}^n = M^2 - \hat{Q}^n$を満たし
				\footnote{
					各$n \in \N$に対して$(c^n_j)_{j=0}^{\infty}$は
					$\sum_{j=0}^{\infty} c^n_j = 1$を満たし,且つ$\neq 0$であるのは有限個である.
				}
				,$N \in \equiv{N}{2,c}$を一つ取り
				\begin{align}
					A \coloneqq M^2 - N \label{eq:thm_quadratic_variation_0}
				\end{align}
				とおけば,Doobの不等式(定理\ref{thm:Doob_inequality_2})により
				\begin{align}
					&\Norm{\sup{t \in I}{\left| \hat{Q}_t^n - A_t \right|}}{\mathscr{L}^2}
					= \Norm{\sup{t \in I}{\left| N_t - \hat{N}_t^n \right|}}{\mathscr{L}^2} \\
					&\qquad \leq \Norm{N_T - \hat{N}_T^n}{\mathscr{L}^2}
					= \Norm{\equiv{N}{2,c} - \hat{\equiv{N^n}{2,c}}}{\mathfrak{M}_{2,c}} \longrightarrow 0 \quad (n \longrightarrow \infty) 
				\end{align}
				が成り立つ.
				\begin{align}
					\Norm{\sup{t \in I}{\left| \hat{Q}_t^{n_k} - A_t \right|}}{\mathscr{L}^2} < \frac{1}{4^k} \quad (k=1,2,\cdots)
				\end{align}
				を満たすように部分列$(n_k)_{k=1}^{\infty}$を取り
				\begin{align}
					E_k \coloneqq \left\{\, \sup{t \in I}{\left| \hat{Q}_t^{n_k} - A_t \right|} \geq \frac{1}{2^k} \, \right\} \quad (k=1,2,\cdots)
				\end{align}
				とすると,Chebyshevの不等式より$\mu(E_k) < 1/2^k$を得る.
				\begin{align}
					E \coloneqq \bigcap_{N \in \N} \bigcup_{k \geq N} E_k
				\end{align}
				と定めればBorel-Cantelliの補題より$E$は$\mu$-零集合であり,
				\begin{align}
					\lim_{k \to \infty}\sup{t \in I}{\left| \hat{Q}_t^{n_k}(\omega) - A_t(\omega) \right|} = 0 
					\quad (\forall \omega \in \Omega \backslash E)
					\label{eq:thm_quadratic_variation_1}
				\end{align}
				が成り立つから$A_0(\omega) = 0 \ (\forall \omega \in \Omega \backslash E)$が従う.
				\begin{align}
					D_k \coloneqq \Set{\frac{j}{2^{n_k}}T}{ j = 0,1,\cdots,2^{n_k} } \quad (k=1,2,\cdots)
				\end{align}
				とおけば,(\refeq{eq:lem_quadratic_variation_0})より全ての$v \geq k,\ \omega \in \Omega$に対して
				$t \longmapsto Q_t^{n_v}(\omega)$は$D_k$上で単調非減少であるから$\hat{Q}_t^{n_v}$も$D_k$上で単調非減少であり,
				(\refeq{eq:thm_quadratic_variation_1})より
				$\omega \in \Omega \backslash E$に対するパス$t \longmapsto A_t(\omega)$も$D_k$上で単調非減少である.
				実際,或る$j$と$u \in \Omega \backslash E$で$A_{\frac{j}{2^{n_k}}T}(u) > A_{\frac{j+1}{2^{n_k}}T}(u)$が成り立つとすると,
				$\alpha \coloneqq A_{\frac{j}{2^n}T}(u),\ \beta \coloneqq A_{\frac{j+1}{2^n}T}(u)$とおけば
				式(\refeq{eq:thm_quadratic_variation_1})より或る$\nu \geq 1$が存在して
				\begin{align}
					\left| \hat{Q}_{\frac{j}{2^{n_k}}T}^{n_\nu}(u) - A_{\frac{j}{2^{n_k}}T}(u) \right| < \frac{\alpha - \beta}{2},
					\quad \left| \hat{Q}_{\frac{j+1}{2^{n_k}}T}^{n_\nu}(u) - A_{\frac{j+1}{2^{n_k}}T}(u) \right| < \frac{\alpha - \beta}{2}
				\end{align}
				を同時に満たすが,
				\begin{align}
					\hat{Q}_{\frac{j}{2^{n_k}}T}^{n_\nu}(u) > \frac{\alpha + \beta}{2} > \hat{Q}_{\frac{j+1}{2^{n_k}}T}^{n_\nu}(u)
				\end{align}
				が従い$t \longmapsto \hat{Q}_t^{n_\nu}(u)$の単調増大性に矛盾する.
				$D \coloneqq \cup_{k=1}^{\infty} D_{n_k}$とおけば$D$は$I$で稠密であり,更に$A$は或る零集合$E'$を除いてパスが連続であるから
				\footnote{
					$M \in \mathcal{M}_{b,c},\ N \in \mathcal{M}_{2,c}$より$M,N$のパスが連続でない$\omega$の全体は或る零集合に含まれる.それを$E'$とおけばよい.
				}
				,任意の$\omega \in \Omega \backslash (E \cup E')$に対し$I \ni t \longmapsto A_t(\omega)$は連続且つ単調非減少である.
				また(\refeq{eq:thm_quadratic_variation_0})より$A$の適合性が出るから$A \in \mathcal{A}^+$が従う.
				そして命題\ref{prp:M_pc_M_cloc}より$N = M^2 - A \in \mathcal{M}_{2,c} \subset \mathcal{M}_{c,loc}$が成り立つから
				$A$は(\refeq{eq:thm_existence_of_quadratic_variation_0})を満たす.
				$A' \in \mathcal{A}^+$もまた(\refeq{eq:thm_existence_of_quadratic_variation_0})を満たすなら,$N' = M^2 - A'$として
				\begin{align}
					\mathcal{A} \ni A - A' = N' - N \in \mathcal{M}_{2,c}
				\end{align}
				が成り立ち命題\ref{prp:bounded_continuous_M_2c_path}により$A_t - A'_t = 0\ (\forall t \in I)\quad \mbox{$\mu$-a.s.}$が従う.
				
			\item[第二段]
				$M \in \mathcal{M}_{c,loc}$を取れば,或る$(\tau_j)_{j=0}^{\infty} \in \mathcal{T}$と
				(\refeq{eq:thm_existence_of_quadratic_variation_0})を満たす
				$\left( A^j \right)_{j=0}^{\infty} \subset \mathcal{A}^+$が存在して
				\begin{align}
					N^j \coloneqq \left( M^{\tau_j} \right)^2 - A^j \in \mathcal{M}_{2,c}
					\quad (j=0,1,2,\cdots)
				\end{align}
				を満たす.ここで$A^j$のパスは零集合$C_j$を除いて0出発で連続単調非減少であるとする.
				或る$\mu$-零集合$E_1$があり,任意の$\omega \in \Omega \backslash E_1$に対し或る$J = J(\omega) \in \N$が存在して
				\begin{align}
					0 = \tau_0(\omega) \leq \tau_1(\omega) \leq \cdots \leq \tau_J(\omega) = T
					\label{eq:thm_existence_of_quadratic_variation_1}
				\end{align}
				を満たすから,任意に$n,m \in \N\ (n \leq m)$を取り固定すれば全ての$\omega \in \Omega \backslash E_1$に対して
				\begin{align}
					M_{t \wedge \tau_n(\omega)}^{\tau_m}(\omega) = M_{t \wedge \tau_n(\omega) \wedge \tau_m(\omega)}(\omega) = M_t^{\tau_n}(\omega) 
					\quad (\forall t \in I)
				\end{align}
				が従い,任意抽出定理(定理\ref{thm:optional_sampling_theorem_2})より
				\begin{align}
					\cexp{\left(M_t^{\tau_m}\right)^2 - A_t^m}{\mathcal{F}_{\tau_n}} 
					= \left(M_{t \wedge \tau_n}^{\tau_m}\right)^2 - A_{t \wedge \tau_n}^m 
					= \left(M_t^{\tau_n}\right)^2 - A_{t \wedge \tau_n}^m
					\quad (\forall t \in I)
				\end{align}
				が得られる.定理\ref{thm:stopped_process_martingale}より
				$\left(N^m \right)^{\tau_n} = \left(M^{\tau_n}\right)^2 - (A^m)^{\tau_n} \in \mathcal{M}_{2,c}$
				が従い,一方で$N^n = \left(M^{\tau_n}\right)^2 - A^n \in \mathcal{M}_{2,c}$も満たされているから
				前段の考察より或る$\mu$-零集合$E^{n,m}$が存在して
				\begin{align}
					A_t^n(\omega) = A_{t \wedge \tau_n(\omega)}^m(\omega) \quad (\forall t \in I,\ \omega \in \Omega \backslash E^{n,m})
				\end{align}
				が成り立ち,特に両辺を$\tau_n$で停めれば
				\begin{align}
					A_{t \wedge \tau_n(\omega)}^n(\omega) = A_{t \wedge \tau_n(\omega)}^m(\omega) \quad (\forall t \in I,\ \omega \in \Omega \backslash E^{n,m})
				\end{align}
				が得られる.ここで
				\begin{align}
					E_2 \coloneqq \bigcup_{\substack{n,m \in \N \\ n \leq m}} E^{n,m},
					\quad E_3 \coloneqq \bigcup_{j \in \N_0} C_j
				\end{align}
				とおいて
				\begin{align}
					A_t(\omega) \coloneqq
					\begin{cases}
						\lim_{n \to \infty} A_{t \wedge \tau_n(\omega)}^n(\omega) & (\omega \in \Omega \backslash (E_1 \cup E_2 \cup E_3)) \\
						0 & (\omega \in E_1 \cup E_2 \cup E_3)
					\end{cases}
					\quad (\forall t \in I)
				\end{align}
				により$A$を定めれば$A \in \mathcal{A}^+$を満たす:
				\begin{description}
					\item[連続性・単調非減少性]
						$\omega \in \Omega \backslash (E_1 \cup E_2 \cup E_3)$の場合,(\refeq{eq:thm_existence_of_quadratic_variation_1})より
						或る$n \in \N$が存在して
						$I \ni t \longmapsto A_t(\omega)$と$I \ni t \longmapsto A_t^n(\omega)$は一致するから
						0出発且つ連続単調非減少である.
						
					\item[適合性]
						各$n \in \N$に対し$A^n$は$(\mathcal{F}_t)$-適合である.$t \in I$を固定し
						\begin{align}
							\tilde{\mathcal{F}}_t \coloneqq \Set{B \cap (E_1 \cup E_2 \cup E_3)^c}{B \in \mathcal{F}_t}
						\end{align}
						とおく.写像$\Omega \ni \omega \longmapsto A_t^n(\omega)$を
						$\Omega \backslash (E_1 \cup E_2 \cup E_3)$に制限した
						$\tilde{A}_t^n \coloneqq A_t^n|_{\Omega \backslash (E_1 \cup E_2 \cup E_3)}$は可測$\tilde{\mathcal{F}}_t/\borel{\R}$
						であり,$\tau_n$で停めた過程$\tilde{A}_{t \wedge \tau_n}^n$も可測$\tilde{\mathcal{F}}_t/\borel{\R}$である.
						その各点収束で定めた$\tilde{A}_t \coloneqq A_t|_{\Omega \backslash (E_1 \cup E_2 \cup E_3)}$もまた$\tilde{\mathcal{F}}_t/\borel{\R}$-可測性を持つ.
						任意の$C \in \borel{\R}$に対して
						\begin{align}
							A_t^{-1}(C) =
							\begin{cases}
								\tilde{A}_t^{-1}(C) & (0 \notin C) \\
								(E_1 \cup E_2) \cup \tilde{A}_t^{-1}(C) & (0 \in C)
							\end{cases}
						\end{align}
						が成り立ち,$\tilde{\mathcal{F}}_t \subset \mathcal{F}_t$が満たされているから
						$A_t$の$\mathcal{F}_t/\borel{\R}$-可測性が出る.
				\end{description}
				$N \coloneqq M^2 - A$は或る零集合$E_4$を除いてパスが連続であるから,命題(\ref{prp:M_pc_M_cloc})より
				\begin{align}
					\sigma_j(\omega) \coloneqq
					\begin{cases}
						0 & (\omega \in E_4), \\
						\inf{}{\Set{t \in I}{|N_t(\omega)| \geq j}} \wedge T & (\omega \in \Omega \backslash E_4)
					\end{cases}
					\quad (j=0,1,2,\cdots)
				\end{align}
				として$(\sigma_j)_{j=0}^{\infty} \in \mathcal{T}$を定めれば
				\begin{align}
					\sup{t \in I}{ \Norm{N_{t \wedge \tau_j \wedge \sigma_j}^j}{\mathscr{L}^\infty} } \leq j
					\quad (j = 0,1,2,\cdots)
				\end{align}
				が満たされる.また$\Omega \backslash (E_1 \cup E_2 \cup E_3)$上で
				\begin{align}
					N_t^{\tau_j}(\omega) = \left( M_t^{\tau_j} \right)^2(\omega) - A_t^{\tau_j}(\omega) 
				= \left( M_t^{\tau_j} \right)^2(\omega) - \left( A^j \right)_t^{\tau_j}(\omega)
					= \left( N^j \right)_t^{\tau_j}(\omega) \quad (\forall t \in I,\ j \in \N_0)
				\end{align}
				が成り立つから,
				\begin{align}
					\gamma_j \coloneqq \tau_j \wedge \sigma_j
					\quad (j=0,1,2,\cdots)
				\end{align}
				により$(\gamma_j)_{j=0}^{\infty} \in \mathcal{T}$を定めれば
				$N^{\gamma_j} \in \mathcal{M}_{b,c}$が満たされ
				$N \in \mathcal{M}_{c,loc}$が従う.$A$の一意性について,
				$A' \in \mathcal{A}^+$もまた(\refeq{eq:thm_existence_of_quadratic_variation_0})を満たしているとき,
				或る停止時刻$\left( \gamma'_j \right)_{j=0}^{\infty} \in \mathcal{T}$が存在して
				$\left( M^{\gamma'_j} \right)^2 - {A'}^{\gamma'_j} \in \mathcal{M}_{b,c}\ (j=0,1,2,\cdots)$が成り立つ.
				\begin{align}
					\zeta_j \coloneqq \gamma_j \wedge \gamma'_j
					\quad (j=0,1,2,\cdots)
				\end{align}
				とおけば
				\begin{align}
					\left( M^{\zeta_j} \right)^2 - A^{\zeta_j},\ \left( M^{\zeta_j} \right)^2 - {A'}^{\zeta_j} \in \mathcal{M}_{b,c}
					\quad (j=0,1,2,\cdots)
				\end{align}
				が満たされるから
				\begin{align}
					A_t^{\zeta_j} (\omega) = {A'}_t^{\zeta_j} (\omega)
				\end{align}
				が従い$A$と$A'$は$\mu$-a.s.にパスが一致する.
				
			\item[第三段]
				(3)を示す.$M \in \mathcal{M}_{p,c}\ (p \geq 2)$の場合,
				命題\ref{prp:M_pc_M_cloc}より$M \in \mathcal{M}_{c,loc}$であるから,
				或る$A \in \mathcal{A}^+$が存在して$M^2 - A \in \mathcal{M}_{c,loc}$となる.
				従って或る$(\tau_j)_{j=0}^{\infty} \in \mathcal{T}$が存在して
				$\left(M^{\tau_j} \right)^2 - A^{\tau_j} \in \mathcal{M}_{b,c}\ (j=0,1,\cdots)$
				を満たすから,任意に$t \in I$を固定すれば
				\begin{align}
					\int_{\Omega} \left( M_{t \wedge \tau_j(\omega)}(\omega) \right)^2\ \mu(d\omega)
					= \int_{\Omega} A_{t \wedge \tau_j(\omega)}(\omega)\ \mu(d\omega)
					\quad (\forall j=0,1,\cdots)
				\end{align}
				が成り立つ.或る零集合$E$が存在して,$\omega \in \Omega \backslash E$なら
				$A_0(\omega) = 0$,$I \ni t \longmapsto A_t(\omega)$は連続且つ単調非減少,
				更に$0 = \tau_0(\omega) \leq \tau_1(\omega) \leq \cdots \leq \tau_{J}(\omega) = T\ (\exists J = J(\omega))$
				が満たされるから,$\left(A_{t \wedge \tau_j(\omega)}(\omega)\right)_{j=0}^{\infty}$は単調増大列である.
				またDoobの不等式により
				$\left| M_{t \wedge \tau_j} \right| \leq \sup{t \in I}{|M_t|} \in \mathscr{L}^p \subset \mathscr{L}^2$
				も満たされているから,Lebesgueの収束定理と単調収束定理より
				\begin{align}
					\int_{\Omega} \left( M_t(\omega) \right)^2\ \mu(d\omega)
					= \int_{\Omega} A_t(\omega)\ \mu(d\omega) < \infty
				\end{align}
				が従う.
			
			\item[第四段]
				(4)を示す.第一段において,(\refeq{eq:thm_quadratic_variation_1})とFatouの補題,
				Minkowskiの不等式及び(\refeq{eq:thm_existence_of_quadratic_variation_3})より
				\begin{align}
					&\Norm{M_T^2 - A_T}{\mathscr{L}^2}
					\leq \liminf_{k \to \infty} \Norm{M_T^2 - \hat{Q}_t^{n_k}}{\mathscr{L}^2} \\
					&\qquad \leq \liminf_{k \to \infty} \sum_{j=0}^{\infty} c_j^{n_k} \Norm{M_T^2 - Q^{n_k+j}}{\mathscr{L}^2}
					\leq 2 \sup{t \in I}{\Norm{M_t}{\mathscr{L}^\infty}} \Norm{M_T}{\mathscr{L}^2}
				\end{align}
				が成立する.
				\QED
		\end{description}
	\end{prf}
	
	\begin{screen}
		\begin{dfn}[二次変分]
			$M \in \mathcal{M}_{c,loc}$に対して(\refeq{eq:thm_existence_of_quadratic_variation_0})を満たす
			$A \in \mathcal{A}^+$のうち,全てのパスが$0$出発,連続,単調非減少であるものを取り,これを
			$M$の二次変分(quadratic variation)と呼び$\inprod<M>$と表す.また
			$M,N \in \mathcal{M}_{c,loc}$に対して
			\begin{align}
				\inprod<M,N> \coloneqq \frac{1}{4} (\inprod<M+N> - \inprod<M-N>)
			\end{align}
			と定義して$M,N$の共変分と呼ぶ.
		\end{dfn}
	\end{screen}

	\begin{screen}
		\begin{thm}[二次変分が有界な連続局所マルチンゲールは連続な二乗可積分マルチンゲール]
			$M \in \mathcal{M}_{c,loc}$かつ$\Norm{\inprod<M>_T}{\mathscr{L}^{\infty}} < \infty$
			であるならば,$M \in \mathcal{M}_{2,c}$が成り立つ.
			\label{thm:quadratic_variation_bounded_then_M_2c}
		\end{thm}
	\end{screen}
	
	\begin{prf} マルチンゲール性の定義に従い,以下三段階に分けて証明する.
		\begin{description}
			\item[第一段] 任意の$t \in I$に対し$M_t$が二乗可積分であることを示す.
				%定理\ref{thm:existence_of_quadratic_variation}より,
				或る$(\tau_j)_{j=0}^{\infty} \in \mathcal{T}$が存在して
				$\left(M^{\tau_j} \right)^2 - \inprod<M>^{\tau_j} \in \mathcal{M}_{b,c}\ (j=0,1,\cdots)$を満たし,
				そして$M^2_0 - \inprod<M>_0 = 0\ \mu$-a.s.も満たされているから,マルチンゲール性により任意の$t \in I,\ j=0,1,\cdots$に対し
				\begin{align}
					\int_\Omega \left( M_{t \wedge \tau_j(\omega)}(\omega) \right)^2\ \mu(d\omega)
					= \int_\Omega \inprod<M>_{t \wedge \tau_j(\omega)}(\omega)\ \mu(d\omega)
					\label{thm:thm_quadratic_variation_bounded_then_M_2c_1}
				\end{align}
				が成り立つ.また$(\tau_j)_{j=0}^{\infty}$に対して或る零集合$E$が存在し,任意の$\omega \in \Omega \backslash E$に対し
				或る$J = J(\omega) \in \N$が存在して
				$0 = \tau_0(\omega) \leq \tau_1(\omega) \leq \cdots \leq \tau_{J}(\omega) = T$
				が満たされる.全ての$\omega \in \Omega$に対し$I \ni t \longmapsto \inprod<M>_t(\omega)$が
				非負・連続・単調非減少であるから,任意の$t \in I$に対し
				\begin{align}
					\inprod<M>_{t \wedge \tau_j(\omega)}(\omega) \leq \inprod<M>_T(\omega)
					\quad (\forall \omega \in \Omega \backslash E,\ j=0,1,\cdots)
				\end{align}
				が成り立ち,更に仮定より$\inprod<M>_T$が可積分であるから,Lebesgueの収束定理により
				\begin{align}
					\int_\Omega \inprod<M>_t(\omega)\ \mu(d\omega)
					= \lim_{j \to \infty} \int_\Omega \inprod<M>_{t \wedge \tau_j(\omega)}(\omega)\ \mu(d\omega)
				\end{align}
				が成立する.よって(\refeq{thm:thm_quadratic_variation_bounded_then_M_2c_1})においてFatouの補題を使えば
				\begin{align}
					&\int_\Omega \left( M_t(\omega) \right)^2\ \mu(d\omega)
					= \int_\Omega \liminf_{j \to \infty} \left( M_{t \wedge \tau_j(\omega)}(\omega) \right)^2\ \mu(d\omega) \\
					&\qquad \leq \liminf_{j \to \infty} \int_\Omega \inprod<M>_{t \wedge \tau_j(\omega)}(\omega)\ \mu(d\omega)
					= \int_\Omega \inprod<M>_t(\omega)\ \mu(d\omega)
					\leq \Norm{\inprod<M>_T}{\mathscr{L}^{\infty}}
				\end{align}
				が得られ$M_t$の二乗可積分性が従う.
			
			\item[第二段] $I \ni t \longmapsto M_t(\omega)$について,定義より
				任意の$\omega \in \Omega$に対し
				各点$t$で右連続且つ左極限を持つから,以降は$\mbox{$\mu$-a.s.}\omega \in \Omega$に対し連続且つ0出発であることを示す.
				$M$に対し或る$\left( \sigma_k \right)_{k=0}^{\infty} \in \mathcal{T}$が存在して
				$M^{\sigma_k} \in \mathcal{M}_{b,c}\ (k=0,1,\cdots)$を満たすから,
				先ず$M_0 = M^{\sigma_k}_0 = 0\ \mu$-a.s.が従う.
				また或る零集合$E'$が存在して$\omega \in \Omega \backslash E'$なら
				$0 = \sigma_0(\omega) \leq \sigma_1(\omega) \leq \cdots \leq \sigma_{K}(\omega) = T = \sigma_{K+1}(\omega) = \sigma_{K+2}(\omega) = \cdots\ (\exists K = K(\omega))$
				が成り立ち,一方で各$k =0,1,\cdots$に対し或る零集合$E_k^{''}$が存在し,$\omega \in \Omega \backslash E_k^{''}$ならば
				$I \ni t \longmapsto M^{\hat{\sigma}_k}_t(\omega)$が連続となる.
				\begin{align}
					E^{''} \coloneqq \bigcup_{k=0}^{\infty} E_k^{''}
				\end{align}
				とおけば,$\omega \in \Omega \backslash (E' \cup E'')$に対しては
				或る$K = K(\omega)$が存在して$\sigma_K(\omega) = T$を満たし,
				$I \ni t \longmapsto M_t(\omega)$は$I \ni t \longmapsto M^{\sigma_K}_t(\omega)$
				に一致する.$I \ni t \longmapsto M^{\sigma_K}_t(\omega)$が連続であるから
				$I \ni t \longmapsto M_t(\omega)$の連続性が従い,
				よって$\mu$-a.s.に$I \ni t \longmapsto M_t$は連続である.
				
			\item[第三段] 任意の$s,t \in I\ (s < t)$に対し
				\begin{align}
					\int_A M_t(\omega)\ \mu(d\omega) = \int_A M_s(\omega)\ \mu(d\omega)
					\quad (\forall A \in \mathcal{F}_s)
					\label{thm:thm_quadratic_variation_bounded_then_M_2c_2}
				\end{align}
				が成り立つことを示す.前段の$\left( \sigma_k \right)_{k=0}^{\infty} \in \mathcal{T}$を取れば
				$M_{t \wedge \sigma_k} \longrightarrow M_t\ (k \longrightarrow \infty,\ \mbox{$\mu$-a.s.})$
				が満たされ,かつ
				$M^{\sigma_k} \in \mathcal{M}_{b,c}\ (k=0,1,\cdots)$であるから任意の$k =0,1,\cdots$に対して
				\begin{align}
					\int_A M_{t \wedge \sigma_k(\omega)}(\omega)\ \mu(d\omega) 
					= \int_A M_{s \wedge \sigma_k(\omega)}(\omega)\ \mu(d\omega)
					\quad (\forall A \in \mathcal{F}_s)
				\end{align}
				が成り立つ.第一段の結果とDoobの不等式より
				$\left| M_{t \wedge \sigma_k} \right| \leq \sup{t \in I}{\left| M_t \right|} \in \mathscr{L}^2 \subset \mathscr{L}^1$
				が満たされるから,
				Lebesgueの収束定理より(\refeq{thm:thm_quadratic_variation_bounded_then_M_2c_2})が得られる.
				\QED
		\end{description}
	\end{prf}
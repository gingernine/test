\section{Doobの不等式・任意抽出定理}
	$I$を$[0,\infty)$の部分集合として基礎に置フィルター付き確率空間$(\Omega,\mathcal{F},\mu,(\mathcal{F}_t)_{t \in I})$と表す.
	\begin{screen}
		\begin{dfn}[マルチンゲール]
			$p \geq 1$とする.関数$M:I \times \Omega \rightarrow \R$が次の五つの条件を満たすとき,
			$M$を$\mathrm{L}^p$-劣マルチンゲール($\mathrm{L}^p$-submartingale)と呼ぶ:
			\begin{description}
				\item[(M.1)] $M_t \in \semiLp{p}{\mathcal{F},\mu}\ (\forall t \in I)$.
				\item[(M.2)] $\forall t \in I$に対し$M_t$は可測$\mathcal{F}_t/\borel{\R}$である.
				\item[(M.3)] 任意の$s \leq t\ (s,t \in I)$に対し$\cexp{M_t}{\mathcal{F}_s} \geq M_s$が成り立つ.
				\item[(M.4)] 任意の$\omega \in \Omega$に対し,各点$t \in I$で左極限が存在する:$\exists \lim\limits_{s \uparrow t} M_s(\omega) \in \R.$
				\item[(M.5)] 任意の$\omega \in \Omega$に対し,各点$t \in I$で右連続である:$M_t(\omega) = \lim\limits_{s \downarrow t} M_s(\omega).$
			\end{description}
			条件(M.2)の不等号が逆向き''$\leq$''の場合,$M$を$\mathrm{L}^p$-優マルチンゲール($\mathrm{L}^p$-supermartingale)と呼び,
			劣かつ優マルチンゲールであるものをマルチンゲールと呼ぶ.
			\label{dfn:martingale}
		\end{dfn}
	\end{screen}
	
	$\tau$を停止時刻,$M$を関数$I \times \Omega \rightarrow \R$とする.$\tau(\Omega)$が高々可算集合である場合
	\begin{align}
		M_\tau = \sum_{t \in \tau(\Omega)}M_t \defunc_{\{\, \tau = t\, \}}
	\end{align}
	と表現できる.このとき全ての$t \in \tau(\Omega)$について$M_t$が可測$\mathcal{F}_t/\borel{\R}$であるなら,
	$M_\tau$は可測$\mathcal{F}_\tau/\borel{\R}$である.実際任意の$t \in I$と$A \in \borel{\R}$に対して
	次が成り立つ:
	\begin{align}
		\{\, M_\tau \in A\, \} \cap \{\, \tau \leq t\, \}
		= \bigcup_{\substack{s \in \tau(\Omega) \\ s \leq t}}\{\, M_s \in A\, \} \cap \{\, \tau = s\, \} \in \mathcal{F}_t.
	\end{align}
	
	\begin{screen}
		\begin{thm}[任意抽出定理(1)]
			$I = \{\ 0,1,\cdots,n\ \}$とし,$M$を$\mathrm{L}^p$-劣マルチンゲールとする.
			このとき$I$値停止時刻$\sigma$と$\tau$に対し次が成立する:
			\begin{align}
				\cexp{M_\tau}{\mathcal{F}_\sigma}(\omega) \geq M_{\sigma(\omega) \wedge \tau(\omega)}(\omega) \quad \mbox{$\mu$-a.s.}\omega \in \Omega.
				\label{eq:thm_optional_sampling_theorem_1}
			\end{align}
			$M$が$\mathrm{L}^p$-マルチンゲールなら(\refeq{eq:thm_optional_sampling_theorem_1})は等号で成立する.
			\label{thm:optional_sampling_theorem_1}
		\end{thm}
	\end{screen}
	
	\begin{prf}\mbox{}
		\begin{description}
			\item[$\sigma \leq \tau$の場合]
				$F_t \coloneqq \defunc_{\{\, \sigma < t \leq \tau\, \}}\ (t \in I \backslash \{0\})$とおけば,
				\begin{align}
					\{\, \sigma < t \leq \tau\, \} = \{\, \sigma < t\, \} \cap \{\, t \leq \tau\, \} = \{\, \sigma \leq t-1\, \} \cap \{\, \tau \leq t-1\, \}^c
				\end{align}
				より$F_t$は可測$\mathcal{F}_{t-1}/\borel{\R}$である.また
				\begin{align}
					N_0 = 0,\quad N_t \coloneqq \sum_{s=0}^{t-1} F_{s+1}(M_{s+1} - M_s) \quad (t \in I \backslash \{0\})
				\end{align}
				として$N$を定めれば,$N$は$\mathrm{L}^p$-劣マルチンゲールである.実際
				$N_t$の右辺の各項が可測$\mathcal{F}_t/\borel{\R}$であるから
				$N_t$も可測$\mathcal{F}_t/\borel{\R}$であり,また$N_t$は
				$(M_t)_{t \in I} \subset \semiLp{p}{\mathcal{F},\mu}$の有限個の線型結合で表現されるから$p$乗可積分である.
				更に$s \leq t\ (s,t \in I)$に対して,命題\ref{prp:properties_of_expanded_conditional_expectation}より
				\begin{align}
					&\cexp{N_t - N_s}{\mathcal{F}_s}
					= \sum_{u=s}^{t-1} \cexp{F_{u+1}(M_{u+1} - M_u)}{\mathcal{F}_s} \\
					&\qquad= \sum_{u=s}^{t-1} \cexp{F_{u+1}\cexp{M_{u+1} - M_u}{\mathcal{F}_u}}{\mathcal{F}_s}
					\geq 0
				\end{align}
				が成り立つ.特に
				\begin{align}
					N_n = \sum_{s=\sigma}^{\tau-1} (M_{s+1} - M_s) = M_\tau - M_\sigma
				\end{align}
				と表せるから
				\begin{align}
					\Exp{M_\tau - M_\sigma} = \Exp{N_n} = \cexp{\cexp{N_n}{\mathcal{F}_0}}{\mathcal{G}} \geq \Exp{N_0} = 0
					\label{eq:thm_optional_sampling_theorem_1_1}
				\end{align}
				が成立する.ここで任意に$E \in \mathcal{F}_\sigma$を取り
				\begin{align}
					\sigma_E(\omega) \coloneqq
					\begin{cases}
						\sigma(\omega) & (\omega \in E) \\
						n & (\omega \in \Omega \backslash E) 
					\end{cases}
					,\quad \tau_E(\omega) \coloneqq
					\begin{cases}
						\tau(\omega) & (\omega \in E) \\
						n & (\omega \in \Omega \backslash E) 
					\end{cases}
				\end{align}
				として停止時刻を定めれば
				\footnote{
					任意の$t \in I$に対し
					\begin{align}
						\{\, \tau_E \leq t\, \} 
						= E \cap \{\, \tau_E \leq t\, \} + (\Omega \backslash E) \cap \{\, \tau_E \leq t\, \} 
						=\begin{cases}
							E \cap \{\, \tau \leq t\, \} & (t < n) \\
							\Omega & (t = n)
						\end{cases}
					\end{align}
					が成り立つ.$\mathcal{F}_\sigma$の定義より$E \cap \{\, \tau \leq t\, \} \in \mathcal{F}_t$となるから$\tau_E$は停止時刻である.
				}
				,(\refeq{eq:thm_optional_sampling_theorem_1_1})により
				\begin{align}
					\int_\Omega M(\tau_E(\omega),\omega)\ \mu(d\omega) \geq \int_\Omega M(\sigma_E(\omega),\omega)\ \mu(d\omega)
				\end{align}
				が成り立つ.$\tau_E,\sigma_E$の定め方より
				\begin{align}
					&\int_E M(\tau(\omega),\omega)\ \mu(d\omega) 
					= \int_\Omega M(\tau_E(\omega),\omega)\ \mu(d\omega) - \int_{\Omega \backslash E} M(n,\omega)\ \mu(d\omega) \\
					&\qquad \geq \int_\Omega M(\sigma_E(\omega),\omega)\ \mu(d\omega) - \int_{\Omega \backslash E} M(n,\omega)\ \mu(d\omega)
					= \int_E M(\sigma(\omega),\omega)\ \mu(d\omega)
				\end{align}
				が従い,命題\ref{prp:properties_of_expanded_conditional_expectation}より
				\begin{align}
					\int_E \cexp{M_\tau}{\mathcal{F}_\sigma}(\omega)\ \mu(d\omega) \geq \int_E M(\sigma(\omega),\omega)\ \mu(d\omega)
				\end{align}
				となるから,$E \in \mathcal{F}_\sigma$の任意性により
				\begin{align}
					\cexp{M_\tau}{\mathcal{F}_\sigma} \geq M_\sigma = M_{\sigma \wedge \tau}
				\end{align}
				が得られる.
			
			\item[一般の場合]
				$\sigma,\tau$を一般の停止時刻とするとき,任意の$A \in \mathcal{F}_\sigma$に対して
				\begin{align}
					&\int_A \cexp{M_\tau}{\mathcal{F}_\sigma}(\omega)\ \mu(d\omega) \\
					&\qquad = \int_A \cexp{\defunc_{\{\, \tau > \sigma \, \}} M_\tau}{\mathcal{F}_\sigma}(\omega)\ \mu(d\omega)
						+ \int_A \cexp{\defunc_{\{\, \tau \leq \sigma \, \}} M_\tau}{\mathcal{F}_\sigma}(\omega)\ \mu(d\omega)
					\label{eq:thm_optional_sampling_theorem_1_2}
				\end{align}
				とすれば,右辺第一項については命題\ref{prp:stopping_time_and_conditional_expectation}より
				\begin{align}
					\int_A \cexp{\defunc_{\{\, \tau > \sigma \, \}} M_\tau}{\mathcal{F}_\sigma}(\omega)\ \mu(d\omega)
					&= \int_A \cexp{\defunc_{\{\, \tau > \sigma \, \}} M_{\tau \vee \sigma}}{\mathcal{F}_\sigma}(\omega)\ \mu(d\omega) \\
					&= \int_A \defunc_{\{\, \tau > \sigma \, \}}(\omega) \cexp{M_{\tau \vee \sigma}}{\mathcal{F}_{\sigma \wedge \tau}}(\omega)\ \mu(d\omega)
				\end{align}
				が成り立ち,右辺第二項については命題\ref{prp:properties_of_expanded_conditional_expectation}より
				\begin{align}
					\int_A \cexp{\defunc_{\{\, \tau \leq \sigma \, \}} M_\tau}{\mathcal{F}_\sigma}(\omega)\ \mu(d\omega)
					&= \int_A \cexp{\defunc_{\{\, \tau \leq \sigma \, \}} M_{\sigma \wedge \tau}}{\mathcal{F}_\sigma}(\omega)\ \mu(d\omega) \\
					&= \int_A \defunc_{\{\, \tau \leq \sigma \, \}}(\omega) M_{\sigma(\omega) \wedge \tau(\omega)}(\omega)\ \mu(d\omega)
				\end{align}
				が成り立つから,前段の結果を適用して
				\begin{align}
					\mbox{(\refeq{eq:thm_optional_sampling_theorem_1_2})}
					&= \int_A \defunc_{\{\, \tau > \sigma \, \}}(\omega) \cexp{M_{\tau \vee \sigma}}{\mathcal{F}_{\sigma \wedge \tau}}(\omega)\ \mu(d\omega)
						+ \int_A \defunc_{\{\, \tau \leq \sigma \, \}}(\omega) M_{\sigma(\omega) \wedge \tau(\omega)}(\omega)\ \mu(d\omega) \\
					&\geq \int_A \defunc_{\{\, \tau > \sigma \, \}}(\omega) M_{\sigma(\omega) \wedge \tau(\omega)}(\omega)\ \mu(d\omega)
						+ \int_A \defunc_{\{\, \tau \leq \sigma \, \}}(\omega) M_{\sigma(\omega) \wedge \tau(\omega)}(\omega)\ \mu(d\omega) \\
					&= \int_A M_{\sigma(\omega) \wedge \tau(\omega)}(\omega)\ \mu(d\omega)
				\end{align}
				を得る.$A$の任意性より(\refeq{eq:thm_optional_sampling_theorem_1})が出る.
				\QED
		\end{description}
	\end{prf}

	\begin{screen}
		\begin{thm}[Doobの不等式(1)]
			$I=\{0,1,\cdots,n\}$,
			$(\mathcal{F}_t)_{t \in I}$をフィルトレーション,
			$M$を非負値$\mathrm{L}^1$-劣マルチンゲールとし,
			$M_* \coloneqq \max{t \in I}{M_t}$とおく.
			\begin{description}
				\item[(1)] 任意の$\lambda > 0$に対して次が成り立つ:
					\begin{align}
						\mu(M_* \geq \lambda) \leq \frac{1}{\lambda} \int_{\left\{\ M_* \geq \lambda\ \right\}} M_n(\omega)\ \mu(d\omega)
						\leq \frac{1}{\lambda} \Norm{M_n}{\mathscr{L}^1}.
					\end{align}
				\item[(2)] 任意の$p > 1$に対して$M_t\ (\forall t \in I)$が$p$乗可積分なら次が成り立つ:
					\begin{align}
						\Norm{M_*}{\mathscr{L}^p} \leq \frac{p}{p-1} \Norm{M_n}{\mathscr{L}^p}.
						\label{eq:thm_Doob_ineq_1_0}
					\end{align}
			\end{description}
			\label{thm:Doob_inequality_1}
		\end{thm}
	\end{screen}
	
	\begin{prf}\mbox{}
		\begin{description}
			\item[(1)] 
				\begin{align}
					\tau(\omega) \coloneqq \min{}{\Set{t \in I}{M_t(\omega) \geq \lambda}} \wedge n \quad (\forall \omega \in \Omega)
				\end{align}
				により$I$値停止時刻を定めれば
				\footnote{
					実際次が成り立つから$\tau$は停止時刻である:
					\begin{align}
						\left\{\ \tau = t\ \right\} = \bigcap_{j=0}^{t-1} \left\{\ M_j < \lambda\ \right\} \cap \left\{\ M_t \geq \lambda\ \right\} \in \mathcal{F}_t
						\quad (t=0,1,\cdots,n-1), \quad
						\left\{\ \tau = n\ \right\} = \bigcap_{j=0}^{n-1} \left\{\ M_j < \lambda\ \right\} \in \mathcal{F}_n.
					\end{align}
				}
				,任意抽出定理より
				\begin{align}
					\int_{\Omega} M_n(\omega)\ \mu(d\omega)
					\geq \int_{\Omega} M_\tau(\omega)\ \mu(d\omega)
					= \int_{\left\{M_* \geq \lambda\right\}} M_\tau(\omega)\ \mu(d\omega) 
						+ \int_{\left\{M_* < \lambda\right\}} M_\tau(\omega)\ \mu(d\omega)
				\end{align}
				が成り立つ.右辺第一項は$\tau$の定め方より
				\begin{align}
					\int_{\left\{M_* \geq \lambda\right\}} M_\tau(\omega)\ \mu(d\omega) 
					= \sum_{t=0}^{n} \int_{\left\{M_* \geq \lambda\right\} \cap \left\{\tau = t\right\}} M_t(\omega)\ \mu(d\omega)
					\geq \lambda \mu( M_* \geq \lambda )
				\end{align}
				が従い,一方右辺第二項については,$M_*(\omega) < \lambda$ならば$\tau(\omega) = n$であるから
				\begin{align}
					\int_{\left\{M_* < \lambda\right\}} M_\tau(\omega)\ \mu(d\omega) = \int_{\left\{M_* < \lambda\right\}} M_n(\omega)\ \mu(d\omega) 
				\end{align}
				と表されて
				\begin{align}
					\lambda \mu( M_* \geq \lambda ) \leq 
					\int_{\left\{M_* \geq \lambda\right\}} M_n(\omega)\ \mu(d\omega) \leq \Norm{M_n}{\mathscr{L}^1} \label{Doob_ineq_1}
				\end{align}
				が得られる.特に(\refeq{Doob_ineq_1})の$\lambda$を$\lambda - 1/m\ (m \in \N)$に置き換えて$m \longrightarrow \infty$とすれば
				\begin{align}
					\lambda \mu( M_* > \lambda ) \leq \int_{\left\{M_* > \lambda\right\}} M_n(\omega)\ \mu(d\omega) \label{Doob_ineq_2}
				\end{align}
				が出る.
			
			\item[(2)] 先ず任意の$K \in \N$に対して
				写像$[0,K) \times \Omega \ni (t,\omega) \longmapsto \defunc_{\left\{ M_*(\omega) > t \right\}}$が可測$\borel{[0,K)}\times\mathcal{F}/\borel{\R}$であることを示す.
				実際$t_j \coloneqq jK/2^n\ (j=0,1,\cdots,2^n-1)$とおいて
				\begin{align}
					f(t,\omega) \coloneqq \defunc_{\left\{ M_*(\omega) > t \right\}},
					\quad f_n(t,\omega) \coloneqq \sum_{j=0}^{2^n-1} \defunc_{\left[t_j,t_{j+1}\right)}(t) f(t_j,\omega)
					\quad (\forall (t,\omega) \in I \times \Omega)
				\end{align}
				により$f,(f_n)_{n=1}^{\infty}$を定めれば,任意の$A \in \borel{\R}$に対し
				\begin{align}
					f_n^{-1}(A) = 
					\begin{cases}
						\emptyset & (0 \notin A,\ 1 \notin A) \\
						\bigcup_{j=0}^{2^n-1} \left[ t_j,t_{j+1} \right) \times \Set{\omega}{M_*(\omega) > t_{j+1}} & (0 \notin A,\ 1 \in A) \\
						\bigcup_{j=0}^{2^n-1} \left[ t_j,t_{j+1} \right) \times \Set{\omega}{M_*(\omega) \leq t_{j+1}} & (0 \in A,\ 1 \notin A) \\
						[0,n] \times \Omega & (0 \in A,\ 1 \in A)
					\end{cases}
				\end{align}
				が成り立つから$f_n$は可測$\borel{[0,K)}\times\mathcal{F}/\borel{\R}$であり,
				$[0,K) \ni t \longmapsto f(t,\omega)\ (\forall \omega \in \Omega)$の右連続性より
				$f_n$は$f$に各点収束するから$f$の$\borel{[0,K)}\times\mathcal{F}/\borel{\R}$-可測性が従う.
				これによりFubiniの定理が適用でき,(\refeq{Doob_ineq_2})及びH\Ddot{o}lderの不等式と併せて
				\begin{align}
					\Norm{M_* \wedge K}{\mathscr{L}^p}^p &= \int_{\Omega} \left|M_*(\omega) \wedge K\right|^p\ \mu(d\omega) \\
					&= p \int_{\Omega} \int_0^K t^{p-1} \defunc_{\left\{M_*(\omega) > t\right\}}\ dt\ \mu(d\omega) \\
					&= p \int_0^K t^{p-1} \mu\left(M_* > t\right)\ dt \\
					&\leq p \int_0^K t^{p-2} \int_\Omega M_n(\omega) \defunc_{\left\{M_*(\omega) > t\right\}}\ \mu(d\omega)\ dt \\
					&= \frac{p}{p-1} \int_\Omega M_n(\omega) \left| M_*(\omega) \wedge K \right|^{p-1}\ \mu(d\omega) \\
					&\leq \frac{p}{p-1} \Norm{M_n}{\mathscr{L}^p} \Norm{M_*(\omega) \wedge K}{\mathscr{L}^p}^{p-1} 
				\end{align}
				が成立し
				\begin{align}
					\Norm{M_* \wedge K}{\mathscr{L}^p} \leq \frac{p}{p-1} \Norm{M_n}{\mathscr{L}^p}
				\end{align}
				が従う.$K \longrightarrow \infty$として単調収束定理より(\refeq{eq:thm_Doob_ineq_1_0})を得る.
				\QED
		\end{description}
		
	\end{prf}
	
	$T > 0$に対し$I = [0,T]$とおき$M$を$\mathrm{L}^p$-劣マルチンゲールとする.このとき
	\begin{align}
		\sup{t \in I}{M_t(\omega)} = \sup{n \in \N}{\max{j=0,1,\cdots,2^n}{M_{\tfrac{j}{2^n}T}(\omega)}} \quad (\forall \omega \in \Omega)
	\end{align}
	が成り立ち$\sup{t \in I}{M_t}$の$\mathcal{F}/\borel{\R}$-可測性が出る.実際,各$\omega \in \Omega$に対し
	\begin{align}
		\alpha(\omega) \coloneqq \sup{t \in I}{M_t(\omega)},
		\quad \beta(\omega) \coloneqq \sup{n \in \N}{\max{j=0,1,\cdots,2^n}{M_{\tfrac{j}{2^n}T}(\omega)}}
	\end{align}
	とおけば$\alpha \geq \beta$が満たされるが,
	もし或る$\omega \in \Omega$で$\alpha(\omega) > \beta(\omega)$となると,
	\begin{align}
		M_s(\omega) > \frac{\alpha(\omega) + \beta(\omega)}{2}
	\end{align}
	を満たす$s \in [0,T)$が存在し
	\footnote{
		$s = T$の場合$\beta(\omega) \leq M_T(\omega)$が満たされているから$M_T(\omega) > \alpha(\omega) + \beta(\omega)/2$となりえない.
	}
	,$I \ni t \longmapsto M_t(\omega)\ (\forall \omega \in \Omega)$の右連続性により或る$j$で
	\begin{align}
		M_{\tfrac{j}{2^n}T}(\omega) > \frac{\alpha(\omega) + \beta(\omega)}{2}
	\end{align}
	が成り立ち$M_{jT/2^n} \leq \beta$に矛盾する.
	
	\begin{screen}
		\begin{thm}[Doobの不等式(2)]
			$I=[0,T]$,$(\mathcal{F}_t)_{t \in I}$をフィルトレーション,
			$M$を非負値$\mathrm{L}^p$-劣マルチンゲールとし,
			$M_* \coloneqq \sup{t \in I}{M_t}$とおく.
			\begin{description}
				\item[(1)] 任意の$\lambda > 0$に対して次が成り立つ:
					\begin{align}
						\mu(M_* \geq \lambda) \leq \frac{1}{\lambda^p} \Norm{M_T}{\mathscr{L}^p}^p.
					\end{align}
				\item[(2)] $p > 1$なら次が成り立つ:
					\begin{align}
						\Norm{M_*}{\mathscr{L}^p} \leq \frac{p}{p-1} \Norm{M_T}{\mathscr{L}^p}.
					\end{align}
			\end{description}
		\label{thm:Doob_inequality_2}	
		\end{thm}
	\end{screen}
	
	\begin{prf}
		任意に$n \in \N$を取り固定し
		\begin{align}
			D_n \coloneqq \Set{\tfrac{j}{2^n}T}{j=0,1,\cdots,2^n}
		\end{align}
		とおく.Jensenの不等式より,任意の$s < t\ (s,t \in I)$に対して
		\begin{align}
			\cexp{M_t^p}{\mathcal{F}_s} \geq \cexp{M_t}{\mathcal{F}_s}^p \geq M_s^p
		\end{align}
		が成り立つから$(M_t^p)_{t \in I}$は$\mathrm{L}^1$-劣マルチンゲールであり,定理\ref{thm:Doob_inequality_1}より
		\begin{align}
			\mu\left(\max{r \in D_n}{M_r} \geq \lambda\right)
			= \mu\left(\max{r \in D_n}{M_r^p} \geq \lambda^p\right) 
			\leq \frac{1}{\lambda^p} \Norm{M_T}{\mathscr{L}^p}^p
		\end{align}
		従う.$n \in \N$は任意に取っていたから
		\begin{align}
			\mu(M_* \geq \lambda) 
			= \mu\left(\sup{n \in \N}{\max{r \in D_n}{M_r}} \geq \lambda\right)
			= \lim_{n \to \infty} \mu\left(\max{r \in D_n}{M_r} \geq \lambda\right) 
			\leq \frac{1}{\lambda^p} \Norm{M_T}{\mathscr{L}^p}^p
		\end{align}
		となり(1)の主張が得られる.同じく定理\ref{thm:Doob_inequality_1}より
		\begin{align}
			\Norm{\max{r \in D_n}{M_r}}{\mathscr{L}^p} \leq \frac{p}{p-1} \Norm{M_T}{\mathscr{L}^p}
		\end{align}
		が成り立つから,$n \longrightarrow \infty$とすれば単調収束定理より(2)の主張が得られる.
		\QED
	\end{prf}
	
	\begin{screen}
		\begin{thm}[任意抽出定理(2)]
			$I = [0,T]$,$M$を$\mathrm{L}^p$-マルチンゲール($p > 1$)とする.
			このとき任意の$I$値停止時刻$\tau,\sigma$に対し次が成り立つ:
			\begin{align}
				\cexp{M_\tau}{\mathcal{F}_\sigma} = M_{\sigma \wedge \tau}.
				\label{eq:thm_optional_sampling_theorem_2}
			\end{align}
			\label{thm:optional_sampling_theorem_2}
		\end{thm}
	\end{screen}
	
	\begin{prf}
		任意の$n \in \N$に対して
		\begin{align}
			\tau_n \coloneqq \frac{1+[2^n \tau]}{2^n} \wedge T,
			\quad \sigma_n \coloneqq \frac{1+[2^n \sigma]}{2^n} \wedge T
		\end{align}
		とおけば,$\tau_n,\sigma_n$は停止時刻であり$\tau \leq \tau_n$かつ$\tau_n \longrightarrow \tau\ (n \longrightarrow \infty)$を満たす.
		実際
		\begin{align}
			\begin{cases}
				\left\{ \tau_n \leq t \right\} = \left\{ 1 + [2^n \tau] \leq 2^n t \right\} = \left\{ \tau < t \right\} & (t \in [0,T)), \\
				\left\{ \tau_n \leq t \right\} = \left\{ 1 + [2^n \tau] > 2^n t \right\} + \left\{ 1 + [2^n \tau] \leq 2^n t \right\} & (t = T)
			\end{cases}
		\end{align}
		が成り立つから$\tau_n$は停止時刻であり,また
		\begin{align}
				\tau_n(\omega) - \tau(\omega) &= \frac{1 + [2^n \tau(\omega)] - 2^n \tau(\omega)}{2^n} \leq \frac{1}{2^n}, && \left( \tau(\omega) < T - \frac{1}{2^n} \right), \\
				\tau_n(\omega) - \tau(\omega) &= T - \tau(\omega) \leq \frac{1}{2^n}, && \left( T - \frac{1}{2^n} \leq \tau(\omega) \leq T \right)
		\end{align}
		より$\tau \leq \tau_n$かつ$\tau_n \longrightarrow \tau\ (n \longrightarrow \infty)$となる.
		従って$M$の右連続性より
		\begin{align}
			M_{\tau_n(\omega)}(\omega) \longrightarrow M_{\tau(\omega)}(\omega)
			\quad (n \longrightarrow \infty,\ \forall \omega \in \Omega)
		\end{align}
		が成り立ち,また定理\ref{thm:optional_sampling_theorem_1}より任意の$A \in \mathcal{F}_\sigma \subset \mathcal{F}_{\sigma_n}$に対して
		\begin{align}
			\int_A M_{\tau_n(\omega)}(\omega)\ \mu(d\omega) 
			= \int_A \cexp{M_{\tau_n}}{\mathcal{F}_{\sigma_n}}(\omega)\ \mu(d\omega) 
			= \int_A M_{\tau_n(\omega)\wedge \sigma_n(\omega)}(\omega)\ \mu(d\omega) 
		\end{align}
		が満たされる.Jensenの不等式により$(|M_t|)_{t \in I}$は$\mathrm{L}^p$-劣マルチンゲールであり,Doobの不等式より
		$\sup{t \in I}{|M_t|}$の可積分性が得られるから,Lebesgueの収束定理を適用して
		\begin{align}
			&\int_A M_{\tau(\omega)}(\omega)\ \mu(d\omega) = \lim_{n \to \infty} \int_A M_{\tau_n(\omega)}(\omega)\ \mu(d\omega) \\
			&\quad = \lim_{n \to \infty} \int_A M_{\tau_n(\omega)\wedge \sigma_n(\omega)}(\omega)\ \mu(d\omega)
			= \int_A M_{\tau(\omega)\wedge \sigma(\omega)}(\omega)\ \mu(d\omega)
		\end{align}
		が成立する.
		\QED
	\end{prf}
	
\section{11/1}
	確率空間を$(\Omega,\mathcal{F},\mu)$とする.
	
	\begin{itembox}[l]{}
		\begin{thm}[Doobの不等式(1)]
			$I=\{0,1,\cdots,n\}$,
			$(\mathcal{F}_t)_{t \in I}$をフィルトレーション,
			$(M_t)_{t \in I}$を$\mathrm{L}^1$-劣マルチンゲールとし,
			$M^* \coloneqq \max{t \in I}{M_t}$とおく.$(M_t)_{t \in I}$が非負値なら次が成り立つ:
			\begin{description}
				\item[(1)] 任意の$\lambda > 0$に対して
					\begin{align}
						\mu(M^* \geq \lambda) \leq \frac{1}{\lambda} \int_{\left\{\ M^* \geq \lambda\ \right\}} M_n(\omega)\ \mu(d\omega)
						\leq \frac{1}{\lambda} \Norm{M_n}{\mathscr{L}^1}.
					\end{align}
				\item[(2)] 任意の$p > 1$に対して$M_t\ (\forall t \in I)$が$p$乗可積分なら
					\begin{align}
						\Norm{M^*}{\mathscr{L}^p} \leq \frac{p}{p-1} \Norm{M_n}{\mathscr{L}^p}.
					\end{align}
			\end{description}
		\end{thm}
	\end{itembox}
	
	\begin{prf}
		\begin{align}
			\tau(\omega) \coloneqq \min{}{\Set{ i \in I}{M_i(\omega) \geq \lambda}} 
			\quad (\forall \omega \in \Omega)
		\end{align}
		とおけば$\tau$は$I$に値を取る停止時刻となる.ただし全ての$i \in I$で$M_i(\omega) < \lambda$となるような$\omega$については
		$\tau(\omega) = n$とする.実際停止時刻となることは
		\begin{align}
			\left\{\ \tau = i\ \right\} &= \bigcap_{j=0}^{i-1} \left\{\ M_j < \lambda\ \right\} \cap \left\{\ M_i \geq \lambda\ \right\} \in \mathcal{F}_i
			,\quad (i=0,1,\cdots,n-1), \\
			\left\{\ \tau = n\ \right\} &= \bigcap_{j=0}^{n-1} \left\{\ M_j < \lambda\ \right\} \in \mathcal{F}_n
		\end{align}
		により判る.任意抽出定理より
		\begin{align}
			\cexp{M_n}{\mathcal{F}_\tau} \geq M_{n \wedge \tau} = M_\tau \quad (\because \tau \leq n)
		\end{align}
		が成り立つから,期待値を取って
		\begin{align}
			\int_{\Omega} M_n(\omega)\ \mu(d\omega)
			&\geq \int_{\Omega} M_\tau(\omega)\ \mu(d\omega) \footnotemark \\
			&= \int_{\left\{\ M^* \geq \lambda\ \right\}} M_\tau(\omega)\ \mu(d\omega) 
				+ \int_{\left\{\ M^* < \lambda\ \right\}} M_\tau(\omega)\ \mu(d\omega) \\
			&\geq \lambda \mu( M^* \geq \lambda ) \footnotemark
				+ \int_{\left\{\ M^* < \lambda\ \right\}} M_n(\omega)\ \mu(d\omega) 
				&& (\scriptsize \because \mbox{$M^*(\omega) < \lambda$ならば$\tau(\omega) = n$である.})
		\end{align}
	\end{prf}
	\footnotetext{
		性質$\tilde{\mathrm{C}}2$より
		\begin{align}
			\int_{\Omega} M_n(\omega)\ \prob{d\omega} = \int_{\Omega} \cexp{M_n}{\mathcal{F}_\tau}(\omega)\ \mu(d\omega)
			\geq \int_{\Omega} M_\tau(\omega)\ \mu(d\omega)
		\end{align}
		が成り立つ.
	}
	が成り立つ.
	\footnotetext{
		最後の不等式は次の理由で成り立つ:
		\begin{align}
			M_\tau \defunc_{\{ M^* \geq \lambda \}}  = \sum_{i=0}^{n-1}M_i \defunc_{\{ \tau = i \}} + M_n \defunc_{\{ \tau = n \}}\defunc_{\{ M^* \geq \lambda \}} \geq \lambda.
		\end{align}
	}
	従って
	\begin{align}
		\lambda \mu( M^* \geq \lambda ) \leq 
		\int_{\left\{\ M^* \geq \lambda\ \right\}} M_n(\omega)\ \mu(d\omega) \leq \Norm{M_n}{\mathscr{L}^1} \label{Doob_ineq_1}
	\end{align}
	を得る.これは
	\begin{align}
		\mu( M^* > \lambda ) \leq \int_{\left\{\ M^* > \lambda\ \right\}} M_n(\omega)\ \mu(d\omega) \label{Doob_ineq_2}
	\end{align}
	としても成り立つ
	\footnote{
		式(\refeq{Doob_ineq_1})により任意の$n \in \N$で
		\begin{align}
			\mu( M^* \geq \lambda+1/n ) \leq \int_{\left\{\ M^* \geq \lambda+1/n\ \right\}} M_n(\omega)\ \mu(d\omega)
		\end{align}
		が成り立っているから,$n \longrightarrow \infty$とすればよい.
	}.
	次に(2)を示す.$K \in \N$とする.
	\begin{align}
		\Norm{M^* \wedge K}{\mathscr{L}^p}^p &= \int_{\Omega} \left|M^*(\omega) \wedge K\right|^p\ \mu(d\omega) \\
		&= p \int_{\Omega} \int_0^{M^*(\omega) \wedge K} t^{p-1}\ dt\ \mu(d\omega) \\
		&= p \int_{\Omega} \int_0^K t^{p-1} \defunc_{\left\{ M^*(\omega) > t \right\}}\ dt\ \mu(d\omega) \footnotemark \\
		&= p \int_0^K t^{p-1} \int_{\Omega} \defunc_{\left\{ M^*(\omega) > t \right\}}\ \mu(d\omega)\ dt & (\scriptsize\because \mbox{Fubiniの定理より}) \\
		&= p \int_0^K t^{p-1} \mu( M^* > t )\ dt \\
		&\leq p \int_0^K t^{p-2} \int_{\left\{\ M^* > t\ \right\}} M_n(\omega)\ \mu(d\omega) & (\scriptsize\because \mbox{式(\refeq{Doob_ineq_2})より}) \\
		&= p \int_\Omega M_n(\omega) \int_0^K t^{p-2} \defunc_{\left\{ M^*(\omega) > t \right\}}\ dt\ \mu(d\omega) \\
		&= \frac{p}{p-1} \int_\Omega M_n(\omega) \left| M^*(\omega) \wedge K \right|^{p-1}\ \mu(d\omega) \\
		&\leq \frac{p}{p-1} \Norm{M_n}{\mathscr{L}^p} \Norm{M^*(\omega) \wedge K}{\mathscr{L}^p}^{p-1} 
	\end{align}
	となるから,
	\begin{align}
		\Norm{M^* \wedge K}{\mathscr{L}^p} \leq \frac{p}{p-1} \Norm{M_n}{\mathscr{L}^p}
	\end{align}
	が成り立つ.$K \longrightarrow \infty$として単調収束定理より
	\begin{align}
		\Norm{M^*}{\mathscr{L}^p} \leq \frac{p}{p-1} \Norm{M_n}{\mathscr{L}^p}
	\end{align}
	を得る.
	\QED
	\footnotetext{
		写像$[0,K) \times \Omega \ni (t,\omega) \longmapsto \defunc_{\left\{ M^*(\omega) > t \right\}}$は可測$\borel{[0,K)}\times\mathcal{F}/\borel{\R}$である.
		実際,
		\begin{align}
			f(t,\omega) \coloneqq \defunc_{\left\{ M^*(\omega) > t \right\}},
			\quad f_n(t,\omega) \coloneqq \defunc_{\left\{ M^*(\omega) > (j+1)/2^n \right\}} \quad (t \in \left[ \tfrac{j}{2^n},\tfrac{j+1}{2^n} \right),\ j=0,1,\cdots,K2^n-1)
		\end{align}
		とおけば,任意の$A \in \borel{\R}$に対して
		\begin{align}
			f_n^{-1}(A) = \begin{cases}
				\emptyset & (0 \notin A,\ 1 \notin A) \\
				\bigcup_{j=0}^{K2^n-1} \left[ \tfrac{j}{2^n},\tfrac{j+1}{2^n} \right) \times \Set{\omega}{M^*(\omega) > \tfrac{j+1}{2^n}} & (0 \notin A,\ 1 \in A) \\
				\bigcup_{j=0}^{K2^n-1} \left[ \tfrac{j}{2^n},\tfrac{j+1}{2^n} \right) \times \Set{\omega}{M^*(\omega) \leq \tfrac{j+1}{2^n}} & (0 \in A,\ 1 \notin A) \\
				[0,n] \times \Omega & (0 \in A,\ 1 \in A)
			\end{cases}
		\end{align}
		が成り立つから$f_n$は可測$\borel{[0,K)}\times\mathcal{F}/\borel{\R}$である.また各点$(t,\omega) \in [0,K) \times \Omega$において
		\begin{align}
			f(t,\omega) - f_n(t,\omega) = \defunc_{\left\{ t < M^*(\omega) \leq (j+1)/2^n \right\}} \longrightarrow 0 \quad (n \longrightarrow \infty)
		\end{align}
		となり$f_n$は$f$に各点収束するから,可測性は保存され$f$も可測$\borel{[0,K)}\times\mathcal{F}/\borel{\R}$となる.
		$t^{p-1}$も2変数関数として$g(t,\omega) \coloneqq t^{p-1}\defunc_{\Omega}(\omega)$と見做せば可測$\borel{[0,K)}\times\mathcal{F}/\borel{\R}$で,
		よって$gf$に対しFubiniの定理を適用できる.
	}
	
	$I = [0,T] \subset \R\ (T > 0)$を考える.$t \longmapsto M_t$は右連続であるから$\sup{t \in I}{M_t}$は確率変数となる.これは
	\begin{align}
		\sup{t \in I}{M_t(\omega)} = \sup{n \in \N}{\max{j=0,1,\cdots,2^n}{M_{\tfrac{j}{2^n}T}(\omega)}} \quad (\forall \omega \in \Omega)
	\end{align}
	が成り立つからである.実際各点$\omega \in \Omega$で
	\begin{align}
		\alpha = \alpha(\omega) \coloneqq \sup{t \in I}{M_t(\omega)},
		\quad \beta = \beta(\omega) \coloneqq \sup{n \in \N}{\max{j=0,1,\cdots,2^n}{M_{\tfrac{j}{2^n}T}(\omega)}}
	\end{align}
	とおけば,$\alpha$の方が上限を取る範囲が広いから$\alpha \geq \beta$は成り立つ.
	だがもし$\alpha > \beta$とすれば,或る$s \in I$が存在して
	\begin{align}
		M_s(\omega) > \frac{\alpha + \beta}{2}
	\end{align}
	を満たすから,右連続性により$s$の近傍から$jT/2^n$の形の点を取ることができて
	\begin{align}
		(\beta \geq)\ M_{\tfrac{j}{2^n}T}(\omega) > \frac{\alpha + \beta}{2}
	\end{align}
	となりこれは矛盾である.
	
	\begin{itembox}[l]{}
		\begin{thm}[Doobの不等式(2)]
			$I=[0,T]$,$(\mathcal{F}_t)_{t \in I}$をフィルトレーション,
			$(M_t)_{t \in I}$を$\mathrm{L}^p$-劣マルチンゲールとし,
			$M^* \coloneqq \sup{t \in I}{M_t}$とおく.$(M_t)_{t \in I}$が非負値なら次が成り立つ:
			\begin{description}
				\item[(1)] 任意の$\lambda > 0$に対して
					\begin{align}
						\mu(M^* \geq \lambda) \leq \frac{1}{\lambda^p} \Norm{M_T}{\mathscr{L}^p}^p.
					\end{align}
				\item[(2)] $p > 1$なら
					\begin{align}
						\Norm{M^*}{\mathscr{L}^p} \leq \frac{p}{p-1} \Norm{M_T}{\mathscr{L}^p}.
					\end{align}
			\end{description}
		\label{thm:Doob_inequality_2}	
		\end{thm}
	\end{itembox}
	
	\begin{prf}
		\begin{align}
			D_n \coloneqq \Set{\tfrac{j}{2^n}T}{j=0,1,\cdots,2^n}
		\end{align}
		とおく.Jensenの不等式より,任意の$0 \leq s < t \leq T$に対して
		\begin{align}
			\cexp{M_t^p}{\mathcal{F}_s} \geq \cexp{M_t}{\mathcal{F}_s}^p \leq M_s^p
		\end{align}
		が成り立つ.従って$(M_t^p)_{t \in I}$は$\mathrm{L}^1$-劣マルチンゲールであり,前定理の結果を使えば
		\begin{align}
			\mu(\max{r \in D_n}{M_r^p} \geq \lambda^p) \leq \frac{1}{\lambda^p} \Norm{M_T}{\mathscr{L}^p}^p
		\end{align}
		が任意の$n \in \N$で成り立つ.非負性から$\max{r \in D_n}{M_r^p} = (\max{r \in D_n}{M_r})^p$となり
		\begin{align}
			\mu(\max{r \in D_n}{M_r} \geq \lambda) \leq \frac{1}{\lambda^p} \Norm{M_T}{\mathscr{L}^p}^p
		\end{align}
		と書き直すことができて,
		\begin{align}
			\mu(M^* \geq \lambda) 
			= \mu(\sup{n \in \N}{\max{r \in D_n}{M_r}} \geq \lambda)
			= \lim_{n \to \infty} \mu(\max{r \in D_n}{M_r} \geq \lambda) \leq \frac{1}{\lambda^p} \Norm{M_T}{\mathscr{L}^p}^p
		\end{align}
		が成り立つ.同じく前定理\footnote{$\mathrm{L}^p$-劣マルチンゲールなら$\mathrm{L}^1$-劣マルチンゲールであるから前定理の結果を適用できる.}を適用し,
		\begin{align}
			\Norm{\max{r \in D_n}{M_r}}{\mathscr{L}^p} \leq \frac{p}{p-1} \Norm{M_T}{\mathscr{L}^p}
		\end{align}
		を保って$n \longrightarrow \infty$とすれば単調収束定理より(2)を得る.
		\QED
	\end{prf}
	
	\begin{itembox}[l]{}
		\begin{thm}[停止時刻との合成写像の可測性]
			$I = [0,T]$,フィルトレーションを$(\mathcal{F}_t)_{t \in I}$,$\tau$を停止時刻とし,
			$M$を$I \times \Omega$上の$\R$値関数とする.$M$について,全ての$\omega \in \Omega$に対し
			$I \ni t \longmapsto M(t,\omega)$が右連続でかつ$(\mathcal{F}_t)$-適合ならば,
			写像$\omega \longmapsto M(\tau(\omega),\omega)$は可測$\mathcal{F}_\tau/\borel{\R}$となる.
			\label{thm:measurability_of_stopping_time}
		\end{thm}
	\end{itembox}
	
	\begin{prf}
		任意に$t \in I$を取り$t_j^n \coloneqq jt/2^n\ (j=0,1,\cdots,2^n,\ n=1,2,\cdots)$とおくと,
		右連続性により任意の$s \in [0,t]$に対して
		\begin{align}
			M(s,\omega) = \lim_{n \to \infty} \sum_{j=1}^{2^n} M_{t_j^n}(\omega) \defunc_{(t_{j-1}^n,t_j^n]}(s) \quad (\omega \in \Omega)
			\label{eq:stopping_time_measurability}
		\end{align}
		が成り立つ.右辺は各$n$で可測$\borel{[0,t]} \times \mathcal{F}_t/\borel{\R}$であるから
		$M$も可測$\borel{[0,t]} \times \mathcal{F}_t/\borel{\R}$となる.($t$の任意性から$M$は$(\mathcal{F}_t)$-発展的可測である.)
		一方停止時刻$\tau$について,$\tau \wedge t$が可測$\mathcal{F}_t/\borel{\R}$であるから
		\begin{align}
			\Omega \ni \omega \longmapsto (\tau(\omega) \wedge t, \omega) \in [0,t] \times \Omega
		\end{align}
		は可測$\mathcal{F}_t/\borel{[0,t]} \times \mathcal{F}_t$である.従って合成写像
		\begin{align}
			\Omega \ni \omega \longmapsto M(\tau(\omega) \wedge t,\omega) \in \R
		\end{align}
		は可測$\mathcal{F}_t/\borel{\R}$となる.任意の$A \in \borel{\R}$に対して
		\begin{align}
			\Set{\omega \in \Omega}{M(\tau(\omega),\omega) \in A} \cap \left\{ \tau \leq t \right\}
			= \Set{\omega \in \Omega}{M(\tau(\omega) \wedge t,\omega) \in A} \cap \left\{ \tau \leq t \right\}
			\in \mathcal{F}_t
		\end{align}
		が成り立つから,写像$\omega \longmapsto M(\tau(\omega),\omega)$は可測$\mathcal{F}_\tau/\borel{\R}$である
		\footnote{
			写像$\omega \longmapsto M(\tau(\omega),\omega)$が可測$\mathcal{F}/\borel{\R}$となっていないことにはこの結論が従わない.
			この点を確認すれば,式(\refeq{eq:stopping_time_measurability})より$M$が可測$\borel{I} \times \mathcal{F}/\borel{\R}$
			であることは慥かであるから,$\omega \longmapsto (\tau(\omega),\omega)$が可測$\mathcal{F}/\borel{I}\times\mathcal{F}$であることと
			併せて写像$\omega \longmapsto M(\tau(\omega),\omega)$が可測$\mathcal{F}/\borel{\R}$であることが判明する.
		}.
		\QED
	\end{prf}
	
	\begin{itembox}[l]{}
		\begin{thm}[任意抽出定理(2)]
			$I = [0,T]$,$p > 1$,$(M_t)_{t \in I}$を$\mathrm{L}^p$-マルチンゲールとする.
			このとき$I$に値を取る任意の停止時刻$\tau,\sigma$に対し次が成り立つ:
			\begin{align}
				\cexp{M_\tau}{\mathcal{F}_\sigma} = M_{\tau \wedge \sigma}.
			\end{align}
			\label{thm:optional_sampling_theorem_2}
		\end{thm}
	\end{itembox}
	
	\begin{prf}
		\begin{align}
			\tau_n \coloneqq \min{}{\left\{\ T, \frac{1+[2^n \tau]}{2^n}\ \right\}},
			\quad \sigma_n \coloneqq \min{}{\left\{\ T, \frac{1+[2^n \sigma]}{2^n}\ \right\}},
			\quad (n=1,2,\cdots)
		\end{align}
		とおく.このとき$\tau_n,\sigma_n$は停止時刻で$\mathcal{F}_\sigma \subset \mathcal{F}_{\sigma_n}\ (n=1,2,\cdots)$
		が成り立つ.実際任意の$0 \leq t < T$に対して
		\begin{align}
			\left\{ \tau_n \leq t \right\} = \left\{ 1 + [2^n \tau] \leq 2^n t \right\} = \left\{ \tau_n \leq [2^n t]/2^n \right\} \in \mathcal{F}_t
		\end{align}
		となり,$t = T$の時も
		\begin{align}
			\left\{ \tau_n \leq T \right\} = \left\{ 1 + [2^n \tau] > 2^n T \right\} + \left\{ 1 + [2^n \tau] \leq 2^n T \right\} \in \mathcal{F}_T
		\end{align}
		が成り立つから$\tau_n$は停止時刻
		\footnote{
			もとより$\tau_n$は可測関数である.$\R \ni x \longmapsto [x] \in \R$は可測$\borel{\R}/\borel{\R}$であるから
			$[2^n \tau]$は可測$\mathcal{F}/\borel{\R}$であり,従って$\tau_n$も可測$\mathcal{F}/\borel{\R}$となっている.
		}で,
		\begin{align}
			2^n \sigma_n \leq 1 + [2^n \sigma] 
			\Rightarrow \sigma < \sigma_n
		\end{align}
		により$\mathcal{F}_\sigma \subset \mathcal{F}_{\sigma_n}$となる.前定理により任意の$A \in \mathcal{F}_\sigma$に対して
		\begin{align}
			\int_A M_{\tau_n(\omega)}(\omega)\ \mu(d\omega) = \int_A M_{\tau_n(\omega)\wedge \sigma_n(\omega)}(\omega)\ \mu(d\omega) 
		\end{align}
		が成り立ち,$(|M_t|)_{t \in I}$が$\mathrm{L}^p$-劣マルチンゲールであることからDoobの不等式により
		$\sup{t \in I}{M_t}$は可積分である
		\footnote{
			$\sup{t \in I}{|M_t|^p} = \left( \sup{t \in I}{|M_t|} \right)^p$である.
		}.
		従って$\lim_{n \to \infty} \tau_n = \tau$と$M$の右連続性から,Lebesgueの収束定理より
		\begin{align}
			&\int_A M_{\tau(\omega)}(\omega)\ \mu(d\omega) = \lim_{n \to \infty} \int_A M_{\tau_n(\omega)}(\omega)\ \mu(d\omega) \\
			&\quad = \lim_{n \to \infty} \int_A M_{\tau_n(\omega)\wedge \sigma_n(\omega)}(\omega)\ \mu(d\omega)
			= \int_A M_{\tau(\omega)\wedge \sigma(\omega)}(\omega)\ \mu(d\omega)
		\end{align}
		が成り立つ.
		\QED
	\end{prf}
	
	
	\begin{itembox}[l]{}
		\begin{thm}[閉集合と停止時刻]
			$I = [0,T] \subset \R$,$(X_t)_{t \in I}$を$d$次元確率変数の族とし,
			$\mathcal{F}_0$がP-零集合を全て含んでいると仮定する.
			P-零集合$N$を除いて$I \ni t \longmapsto X_t(\omega)$が右連続で,
			かつ$(X_t)_{t \in I}$が$(\mathcal{F}_t)$-適合であるなら,任意の閉集合$F \subset \R^d$に対し
			\begin{align}
				\tau(\omega) \coloneqq
				\begin{cases}
					0 & (\omega \in N) \\
					\inf{}{\Set{t \in I}{X_t(\omega) \in F}} \wedge T & (\omega \in \Omega \backslash N)
				\end{cases}
			\end{align}
			として$\tau:\Omega \rightarrow \R$を定めれば$\tau$は停止時刻となる.
			また$N' (\subset N)$を除いて$I \ni t \longmapsto X_t(\omega)$が連続であるなら
			\begin{align}
				X_{t \wedge \tau(\omega)}(\omega) \in F^{ic} \quad (\forall \omega \in N',\ t \in I)
			\end{align}
			が成り立つ.ただし$F^i$は$F$の内核を表し$F^{ic}$は$F^i$の補集合を表す.
			\label{thm:closed_set_stopping_time}
		\end{thm}
	\end{itembox}
	確率空間が完備である場合は
	\begin{align}
		\tau(\omega) \coloneqq \inf{}{\Set{t \in I}{X_t(\omega) \in F}} \wedge T
		\quad (\forall \omega \in \Omega)
	\end{align}
	として$\tau$は停止時刻となる.実際任意の$t \in I$に対して,
	\begin{align}
		\{\tau \leq t\} = \{\tau \leq t\} \cap N + \Set{\omega \in \Omega \backslash N}{\tau(\omega) \leq t}
	\end{align}
	の右辺第一項は完備性よりP-零集合,第二項は以下で$\mathcal{F}_t$に属すると証明される.

	\begin{prf}
		$d:\R^d \times \R^d \rightarrow \R$をEuclid距離関数として
		\begin{align}
			D_t(\omega) \coloneqq 
			\begin{cases}
				1 & (\omega \in N) \\
				\inf{}{\Set{d(X_r(\omega),F)}{r \in ([0,t] \cap \Q) \cup \{t\}}} & (\omega \in \Omega \backslash N)
			\end{cases}
		\end{align}
		とおけば
		\footnote{
			$d(X_r(\omega),F) = \inf{y \in F}{d(X_r(\omega),y)}$である.
		},
		$D_t$は可測$\mathcal{F}_t/\borel{\R}$となる
		\footnote{
			写像$\R^d \ni x \longmapsto d(x,F) \in \R$は連続であるから,合成写像
			\begin{align}
				\Omega \ni \omega \longmapsto d(X_t(\omega),F)
			\end{align}
			は可測$\mathcal{F}_t/\borel{\R}$となる.任意の$\lambda \in \R$に対し
			\begin{align}
				\left\{ \inf{}{\Set{d(X_r,F)}{r \in ([0,t] \cap \Q) \cup \{t\}}} \geq \lambda \right\}
				= \bigcap_{r \in ([0,t] \cap \Q) \cup \{t\}} \left\{ d(X_r,F) \geq \lambda \right\}
			\end{align}
			となり右辺の各集合は$\in \mathcal{F}_t$であるから
			写像$\Omega \ni \omega \longmapsto \inf{}{\Set{d(X_r(\omega),F)}{r \in ([0,t] \cap \Q) \cup \{t\}}}$
			も可測$\mathcal{F}_t/\borel{\R}$となる.任意の$A \in \borel{\R}$に対し
			\begin{align}
				D_t^{-1}(A) = 
				\begin{cases}
					N \cup \left\{ \inf{}{\Set{d(X_r,F)}{r \in ([0,t] \cap \Q) \cup \{t\}}} \in A \right\} & (1 \in A) \\
					\left\{ \inf{}{\Set{d(X_r,F)}{r \in ([0,t] \cap \Q) \cup \{t\}}} \in A \right\} & (1 \notin A)
				\end{cases}
			\end{align}
			となるが,$N \in \mathcal{F}_0$であるから$D_t$もまた可測$\mathcal{F}_t/\borel{\R}$となる.
		}.
		ここでは任意の$t \in [0,T)$に対して
		\begin{align}
			\Set{\omega \in \Omega \backslash N}{\tau(\omega) \leq t} = \Set{\omega \in \Omega \backslash N}{D_t(\omega) = 0}
			\label{eq:closed_set_stopping_time_1}
		\end{align}
		が成り立つことを示す.実際これが示されれば任意の$t \in I$に対し
		\begin{align}
			\{\tau \leq t\} =
			\begin{cases}
 				\Omega & (t = T) \\
				N + \Set{\omega \in \Omega \backslash N}{D_t(\omega) = 0} & (t < T)
 			\end{cases}
		\end{align}
		となるから$\tau$は停止時刻となる.式(\refeq{eq:closed_set_stopping_time_1})が成立することを示すには
		包含関係$\subset,\supset$のそれぞれを満たすことを確認すればよい.
		\begin{description}
			\item[$\subset$について]
				任意に$t \in [0,T)$を固定する.$\tau(\omega) \leq t$となる$\omega \in \Omega \backslash N$に対し
				$s \coloneqq \tau(\omega)$とおくと$X_s(\omega) \in F$となる.もし$X_s(\omega) \notin F$であるとすれば,
				$F$が閉集合であることと$s \longmapsto X_s(\omega)$の右連続性から
				或る$\delta > 0$が存在し,任意の$0 < h < \delta$に対して
				$X_{s+h}(\omega) \notin F$となり$s = \tau(\omega)$であることに矛盾する.
				今$d(X_s(\omega),F) = 0$が示されたが,$D_t(\omega) = 0$も成り立っている.
				実際もし$D_t(\omega) > 0$であるとすれば,$a \coloneqq D_t(\omega)$に対して
				\begin{align}
					d(X_s(\omega), X_r(\omega)) < a/2
				\end{align}
				を満たす$r \in ((s,t] \cap \Q) \cup \{t\}$が存在するから
				\begin{align}
					d(X_s(\omega),F) \geq d(X_r(\omega),F) - d(X_s(\omega), X_r(\omega)) > a - a/2 = a/2
				\end{align}
				となり矛盾が生じてしまう.
			
			\item[$\supset$について]
				任意に$t \in [0,T)$を固定する.$D_t(\omega) = 0$となる$\omega \in \Omega \backslash N$について
				\begin{align}
					d(X_{s_n}(\omega),F) < 1/n
				\end{align}
				となるように$s_n \in ([0,t] \cap \Q) \cup \{t\}$を取ることができる.$(s_n)_{n=1}^{\infty}$は
				$[0,t]$に集積点$s$を持ち,$F$が閉であるから$X_s(\omega) \in F$となる.従って
				$\tau(\omega) \leq s \leq t$が成り立つ.
		\end{description}
		以上で$\tau$が停止時刻であることが示されたから,次に定理の後半の主張を示す.
		P-零集合$N' \subset N$を除いて$I \ni t \longmapsto X_t(\omega)$が連続である場合,
		$t < \tau(\omega)\ (\omega \in \Omega \backslash N')$に対しては$X_t(\omega) \in F^c$が成り立っているから,示せばよいのは
		\begin{align}
			X_{\tau(\omega)}(\omega) \in F^{ic}
			\label{eq:closed_set_stopping_time_2}
		\end{align}
		が成り立つことである.$s = \tau(\omega)$とおく.もし$X_s(\omega) \in F^i$であるとすれば
		連続性から或る$\delta$が存在し,$0 < h < \delta$を満たす任意の$h$に対し
		\begin{align}
			X_{s - h}(\omega) \in F^i
		\end{align}
		となるが,$s > s - h \geq \tau(\omega)$が従い矛盾が生じる.よって(\refeq{eq:closed_set_stopping_time_2})が示された.
		\QED
	\end{prf}
	
	上の証明では$\R^d$を一般の距離空間$(E,\rho)$として問題なく,定理の主張は少し一般化できる.
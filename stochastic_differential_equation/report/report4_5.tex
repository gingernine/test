\section{Doobの不等式・任意抽出定理}
	
	$I$を$[0,\infty)$の部分集合として基礎に置フィルター付き確率空間$(\Omega,\mathcal{F},\mu,(\mathcal{F}_\alpha)_{\alpha \in I})$と表す.
	
	\begin{itembox}[l]{}
		\begin{dfn}[マルチンゲール]
			フィルター付き確率空間$(\Omega,\mathcal{F},\mu,(\mathcal{F}_\alpha)_{\alpha \in I})$上の実確率変数の族
			$(M_\alpha)_{\alpha \in I} \subset \semiLp{p}{\mathcal{F},\mu}\ (p \geq 1)$が
			次の四条件を満たすとき,これを$\mathrm{L}^p$-劣マルチンゲール($\mathrm{L}^p$-submartingale)という.
			\begin{description}
				\item[(M.1)] $\forall \alpha \in I$に対し$M_\alpha$は可測$\mathcal{F}_\alpha/\borel{\R}$である.
				\item[(M.2)] 任意の$\alpha \leq \beta\ (\alpha,\beta \in I)$に対し$\cexp{M_\beta}{\mathcal{F}_\alpha} \geq M_\alpha\ $({\scriptsize 同値類に対する順序関係})が成り立つ.
				\item[(M.3)] 各$\omega \in \Omega$において,任意の$\alpha \in I$で左極限が存在する:$\exists \lim\limits_{\beta \uparrow \alpha} M_\beta(\omega) \in \R.$
				\item[(M.4)] 各$\omega \in \Omega$において,任意の$\alpha \in I$で右連続である:$M_\alpha(\omega) = \lim\limits_{\beta \downarrow \alpha} M_\beta(\omega).$
			\end{description}
			条件(M.2)の不等号が逆向き''$\leq$''の場合,$(M_\alpha)_{\alpha \in I}$を$\mathrm{L}^p$-優マルチンゲール($\mathrm{L}^p$-supermartingale)といい,
			劣かつ優マルチンゲールであるものをマルチンゲールという.
			\label{dfn:martingale}
		\end{dfn}
	\end{itembox}
	
	$\tau$を停止時刻とし,$\tau(\Omega)$が高々可算集合である場合,実確率変数の族$(M_\alpha)_{\alpha \in I}$
	に対して
	\begin{align}
		M_\tau \coloneqq \sum_{\alpha \in \tau(\Omega)}M_\alpha
	\end{align}
	とおく.全ての$\alpha \in \tau(\Omega)$について$M_\alpha$が可測$\mathcal{F}_\alpha/\borel{\R}$であるとき,
	$M_\tau$は可測$\mathcal{F}_\tau/\borel{\R}$となる.なぜならば任意の$\alpha \in I$と$A \in \borel{\R}$に対して
	\begin{align}
		(M_\tau \in A) \cap (\tau \leq \alpha)
		= \bigcup_{\substack{\beta \in \tau(\Omega) \\ \beta \leq \alpha}}(M_\beta \in A) \cap (\tau = \beta) \in \mathcal{F}_\alpha 
	\end{align}
	が成り立つからである.$M_\tau$の可測性を確認したところで次の定理を証明する.
	
	\begin{itembox}[l]{}
		\begin{thm}[任意抽出定理(その1)]
			$I = \{\ 1,2,\cdots,n\ \}$とし,実確率変数の族$(M_\alpha)_{\alpha \in I}$が$\mathrm{L}^p$-劣マルチンゲールであるとする.
			このとき$I$に値を取る停止時刻$\sigma$と$\tau$について次が成立する:
			\begin{align}
				\cexp{M_\tau}{\mathcal{F}_\sigma} \geq M_{\sigma \wedge \tau}.
			\end{align}
		\end{thm}
	\end{itembox}
	
	\begin{prf}\mbox{}
		\begin{description}
			\item[$\sigma \leq \tau$の場合]
				$F_\alpha \coloneqq \defunc_{\sigma < \alpha \leq \tau}\ (\alpha \in I)$とおくと,
				\begin{align}
					(\sigma < \alpha \leq \tau) = (\sigma < \alpha) \cap (\alpha \leq \tau) = (\sigma \leq \alpha-1) \cap (\tau \leq \alpha-1)^c
				\end{align}
				より$F_\alpha$は可測$\mathcal{F}_{\alpha-1}/\borel{\R}$となる.$F_\alpha$を用いて
				\begin{align}
					N_\beta \coloneqq \sum_{\alpha=0}^{\beta-1} F_{\alpha+1}(M_{\alpha+1} - M_\alpha) \quad (\beta \in I)
				\end{align}
				として$(N_\beta)_{\beta \in I}$を定義すれば,これもまた$\mathrm{L}^p$-劣マルチンゲールとなる.今$I$は有限集合であるから
				定義\ref{dfn:martingale}の条件(M.1)(M.2)を満たすことを確認すればよい.
				\begin{description}
					\item[(M.1)] 先ず$N_\beta$が可測$\mathcal{F}_\beta/\borel{\R}$であることを示す.
						$N_\beta$を構成する級数の項のうち最も可測性が厳しいものは最終項$F_{\beta}(M_{\beta} - M_{\beta-1})$であり,
						$M_\beta$も$F_{\beta}$も可測$\mathcal{F}_\beta/\borel{\R}$であるから$N_\beta$の可測性も判明する.可積分性については,
						$M_\alpha\ (\alpha \in I)$が$p$乗可積分であるからその有限個の結合で表現される$(N_\beta)_{\beta \in I}$もまた$p$乗可積分となる.
					\item[(M.2)]	
						$\alpha \leq \beta\ (\alpha,\beta \in I)$に対して
						\begin{align}
							N_{\beta} - N_{\alpha} = \sum_{\gamma=\alpha}^{\beta-1} F_{\gamma+1}(M_{\gamma+1} - M_\gamma)
						\end{align}
						と表せるから,(関数の同値類を同様に表記して)$\mathcal{F}_{\alpha}$で条件付ければ,
						性質$\tilde{\mathrm{C}}$5,$\tilde{\mathrm{C}}$6と$F_\alpha$の可測性事情,そして$(M_\alpha)_{\alpha \in I}$が劣マルチンゲールであることにより
						\begin{align}
							\cexp{N_{\beta} - N_{\alpha}}{\mathcal{F}_\alpha}
							&= \sum_{\gamma=\alpha}^{\beta-1} \cexp{F_{\gamma+1}(M_{\gamma+1} - M_\gamma)}{\mathcal{F}_\alpha} \\
							&= \sum_{\gamma=\alpha}^{\beta-1} \cexp{\cexp{F_{\gamma+1}(M_{\gamma+1} - M_\gamma)}{\mathcal{F}_\gamma}}{\mathcal{F}_\alpha} \\
							&= \sum_{\gamma=\alpha}^{\beta-1} \cexp{F_{\gamma+1}\cexp{M_{\gamma+1} - M_\gamma}{\mathcal{F}_\gamma}}{\mathcal{F}_\alpha}
							\geq 0\quad (\mbox{P-a.s.})
						\end{align}
						が成り立つ\footnote{同値類ではなく代表元の関数と見做している.}.
						従って$\cexp{N_{\beta}}{\mathcal{F}_\alpha} \geq \cexp{N_{\alpha}}{\mathcal{F}_\alpha} = N_\alpha$
						\footnote{こちらは同値類に対する順序記号を使っている.等号は性質$\tilde{\mathrm{C}}$5による.}が成り立つ.
				\end{description}
				
		\end{description}
	\end{prf}

	\begin{itembox}[l]{}
		\begin{thm}[Doobの不等式(1)]
			$I=\{0,1,\cdots,n\}$,
			$(\mathcal{F}_t)_{t \in I}$をフィルトレーション,
			$(M_t)_{t \in I}$を$\mathrm{L}^1$-劣マルチンゲールとし,
			$M^* \coloneqq \max{t \in I}{M_t}$とおく.$(M_t)_{t \in I}$が非負値なら次が成り立つ:
			\begin{description}
				\item[(1)] 任意の$\lambda > 0$に対して
					\begin{align}
						\mu(M^* \geq \lambda) \leq \frac{1}{\lambda} \int_{\left\{\ M^* \geq \lambda\ \right\}} M_n(\omega)\ \mu(d\omega)
						\leq \frac{1}{\lambda} \Norm{M_n}{\mathscr{L}^1}.
					\end{align}
				\item[(2)] 任意の$p > 1$に対して$M_t\ (\forall t \in I)$が$p$乗可積分なら
					\begin{align}
						\Norm{M^*}{\mathscr{L}^p} \leq \frac{p}{p-1} \Norm{M_n}{\mathscr{L}^p}.
					\end{align}
			\end{description}
		\end{thm}
	\end{itembox}
	
	\begin{prf}
		\begin{align}
			\tau(\omega) \coloneqq \min{}{\Set{ i \in I}{M_i(\omega) \geq \lambda}} 
			\quad (\forall \omega \in \Omega)
		\end{align}
		とおけば$\tau$は$I$に値を取る停止時刻となる.ただし全ての$i \in I$で$M_i(\omega) < \lambda$となるような$\omega$については
		$\tau(\omega) = n$とする.実際停止時刻となることは
		\begin{align}
			\left\{\ \tau = i\ \right\} &= \bigcap_{j=0}^{i-1} \left\{\ M_j < \lambda\ \right\} \cap \left\{\ M_i \geq \lambda\ \right\} \in \mathcal{F}_i
			,\quad (i=0,1,\cdots,n-1), \\
			\left\{\ \tau = n\ \right\} &= \bigcap_{j=0}^{n-1} \left\{\ M_j < \lambda\ \right\} \in \mathcal{F}_n
		\end{align}
		により判る.任意抽出定理より
		\begin{align}
			\cexp{M_n}{\mathcal{F}_\tau} \geq M_{n \wedge \tau} = M_\tau \quad (\because \tau \leq n)
		\end{align}
		が成り立つから,期待値を取って
		\begin{align}
			\int_{\Omega} M_n(\omega)\ \mu(d\omega)
			&\geq \int_{\Omega} M_\tau(\omega)\ \mu(d\omega) \footnotemark \\
			&= \int_{\left\{\ M^* \geq \lambda\ \right\}} M_\tau(\omega)\ \mu(d\omega) 
				+ \int_{\left\{\ M^* < \lambda\ \right\}} M_\tau(\omega)\ \mu(d\omega) \\
			&\geq \lambda \mu( M^* \geq \lambda ) \footnotemark
				+ \int_{\left\{\ M^* < \lambda\ \right\}} M_n(\omega)\ \mu(d\omega) 
				&& (\scriptsize \because \mbox{$M^*(\omega) < \lambda$ならば$\tau(\omega) = n$である.})
		\end{align}
	\end{prf}
	\footnotetext{
		性質$\tilde{\mathrm{C}}2$より
		\begin{align}
			\int_{\Omega} M_n(\omega)\ \prob{d\omega} = \int_{\Omega} \cexp{M_n}{\mathcal{F}_\tau}(\omega)\ \mu(d\omega)
			\geq \int_{\Omega} M_\tau(\omega)\ \mu(d\omega)
		\end{align}
		が成り立つ.
	}
	が成り立つ.
	\footnotetext{
		最後の不等式は次の理由で成り立つ:
		\begin{align}
			M_\tau \defunc_{\{ M^* \geq \lambda \}}  = \sum_{i=0}^{n-1}M_i \defunc_{\{ \tau = i \}} + M_n \defunc_{\{ \tau = n \}}\defunc_{\{ M^* \geq \lambda \}} \geq \lambda.
		\end{align}
	}
	従って
	\begin{align}
		\lambda \mu( M^* \geq \lambda ) \leq 
		\int_{\left\{\ M^* \geq \lambda\ \right\}} M_n(\omega)\ \mu(d\omega) \leq \Norm{M_n}{\mathscr{L}^1} \label{Doob_ineq_1}
	\end{align}
	を得る.これは
	\begin{align}
		\mu( M^* > \lambda ) \leq \int_{\left\{\ M^* > \lambda\ \right\}} M_n(\omega)\ \mu(d\omega) \label{Doob_ineq_2}
	\end{align}
	としても成り立つ
	\footnote{
		式(\refeq{Doob_ineq_1})により任意の$n \in \N$で
		\begin{align}
			\mu( M^* \geq \lambda+1/n ) \leq \int_{\left\{\ M^* \geq \lambda+1/n\ \right\}} M_n(\omega)\ \mu(d\omega)
		\end{align}
		が成り立っているから,$n \longrightarrow \infty$とすればよい.
	}.
	次に(2)を示す.$K \in \N$とする.
	\begin{align}
		\Norm{M^* \wedge K}{\mathscr{L}^p}^p &= \int_{\Omega} \left|M^*(\omega) \wedge K\right|^p\ \mu(d\omega) \\
		&= p \int_{\Omega} \int_0^{M^*(\omega) \wedge K} t^{p-1}\ dt\ \mu(d\omega) \\
		&= p \int_{\Omega} \int_0^K t^{p-1} \defunc_{\left\{ M^*(\omega) > t \right\}}\ dt\ \mu(d\omega) \footnotemark \\
		&= p \int_0^K t^{p-1} \int_{\Omega} \defunc_{\left\{ M^*(\omega) > t \right\}}\ \mu(d\omega)\ dt & (\scriptsize\because \mbox{Fubiniの定理より}) \\
		&= p \int_0^K t^{p-1} \mu( M^* > t )\ dt \\
		&\leq p \int_0^K t^{p-2} \int_{\left\{\ M^* > t\ \right\}} M_n(\omega)\ \mu(d\omega) & (\scriptsize\because \mbox{式(\refeq{Doob_ineq_2})より}) \\
		&= p \int_\Omega M_n(\omega) \int_0^K t^{p-2} \defunc_{\left\{ M^*(\omega) > t \right\}}\ dt\ \mu(d\omega) \\
		&= \frac{p}{p-1} \int_\Omega M_n(\omega) \left| M^*(\omega) \wedge K \right|^{p-1}\ \mu(d\omega) \\
		&\leq \frac{p}{p-1} \Norm{M_n}{\mathscr{L}^p} \Norm{M^*(\omega) \wedge K}{\mathscr{L}^p}^{p-1} 
	\end{align}
	となるから,
	\begin{align}
		\Norm{M^* \wedge K}{\mathscr{L}^p} \leq \frac{p}{p-1} \Norm{M_n}{\mathscr{L}^p}
	\end{align}
	が成り立つ.$K \longrightarrow \infty$として単調収束定理より
	\begin{align}
		\Norm{M^*}{\mathscr{L}^p} \leq \frac{p}{p-1} \Norm{M_n}{\mathscr{L}^p}
	\end{align}
	を得る.
	\QED
	\footnotetext{
		写像$[0,K) \times \Omega \ni (t,\omega) \longmapsto \defunc_{\left\{ M^*(\omega) > t \right\}}$は可測$\borel{[0,K)}\times\mathcal{F}/\borel{\R}$である.
		実際,
		\begin{align}
			f(t,\omega) \coloneqq \defunc_{\left\{ M^*(\omega) > t \right\}},
			\quad f_n(t,\omega) \coloneqq \defunc_{\left\{ M^*(\omega) > (j+1)/2^n \right\}} \quad (t \in \left[ \tfrac{j}{2^n},\tfrac{j+1}{2^n} \right),\ j=0,1,\cdots,K2^n-1)
		\end{align}
		とおけば,任意の$A \in \borel{\R}$に対して
		\begin{align}
			f_n^{-1}(A) = \begin{cases}
				\emptyset & (0 \notin A,\ 1 \notin A) \\
				\bigcup_{j=0}^{K2^n-1} \left[ \tfrac{j}{2^n},\tfrac{j+1}{2^n} \right) \times \Set{\omega}{M^*(\omega) > \tfrac{j+1}{2^n}} & (0 \notin A,\ 1 \in A) \\
				\bigcup_{j=0}^{K2^n-1} \left[ \tfrac{j}{2^n},\tfrac{j+1}{2^n} \right) \times \Set{\omega}{M^*(\omega) \leq \tfrac{j+1}{2^n}} & (0 \in A,\ 1 \notin A) \\
				[0,n] \times \Omega & (0 \in A,\ 1 \in A)
			\end{cases}
		\end{align}
		が成り立つから$f_n$は可測$\borel{[0,K)}\times\mathcal{F}/\borel{\R}$である.また各点$(t,\omega) \in [0,K) \times \Omega$において
		\begin{align}
			f(t,\omega) - f_n(t,\omega) = \defunc_{\left\{ t < M^*(\omega) \leq (j+1)/2^n \right\}} \longrightarrow 0 \quad (n \longrightarrow \infty)
		\end{align}
		となり$f_n$は$f$に各点収束するから,可測性は保存され$f$も可測$\borel{[0,K)}\times\mathcal{F}/\borel{\R}$となる.
		$t^{p-1}$も2変数関数として$g(t,\omega) \coloneqq t^{p-1}\defunc_{\Omega}(\omega)$と見做せば可測$\borel{[0,K)}\times\mathcal{F}/\borel{\R}$で,
		よって$gf$に対しFubiniの定理を適用できる.
	}
	
	$I = [0,T] \subset \R\ (T > 0)$を考える.$t \longmapsto M_t$は右連続であるから$\sup{t \in I}{M_t}$は確率変数となる.これは
	\begin{align}
		\sup{t \in I}{M_t(\omega)} = \sup{n \in \N}{\max{j=0,1,\cdots,2^n}{M_{\tfrac{j}{2^n}T}(\omega)}} \quad (\forall \omega \in \Omega)
	\end{align}
	が成り立つからである.実際各点$\omega \in \Omega$で
	\begin{align}
		\alpha = \alpha(\omega) \coloneqq \sup{t \in I}{M_t(\omega)},
		\quad \beta = \beta(\omega) \coloneqq \sup{n \in \N}{\max{j=0,1,\cdots,2^n}{M_{\tfrac{j}{2^n}T}(\omega)}}
	\end{align}
	とおけば,$\alpha$の方が上限を取る範囲が広いから$\alpha \geq \beta$は成り立つ.
	だがもし$\alpha > \beta$とすれば,或る$s \in I$が存在して
	\begin{align}
		M_s(\omega) > \frac{\alpha + \beta}{2}
	\end{align}
	を満たすから,右連続性により$s$の近傍から$jT/2^n$の形の点を取ることができて
	\begin{align}
		(\beta \geq)\ M_{\tfrac{j}{2^n}T}(\omega) > \frac{\alpha + \beta}{2}
	\end{align}
	となりこれは矛盾である.
	
	\begin{itembox}[l]{}
		\begin{thm}[Doobの不等式(2)]
			$I=[0,T]$,$(\mathcal{F}_t)_{t \in I}$をフィルトレーション,
			$(M_t)_{t \in I}$を$\mathrm{L}^p$-劣マルチンゲールとし,
			$M^* \coloneqq \sup{t \in I}{M_t}$とおく.$(M_t)_{t \in I}$が非負値なら次が成り立つ:
			\begin{description}
				\item[(1)] 任意の$\lambda > 0$に対して
					\begin{align}
						\mu(M^* \geq \lambda) \leq \frac{1}{\lambda^p} \Norm{M_T}{\mathscr{L}^p}^p.
					\end{align}
				\item[(2)] $p > 1$なら
					\begin{align}
						\Norm{M^*}{\mathscr{L}^p} \leq \frac{p}{p-1} \Norm{M_T}{\mathscr{L}^p}.
					\end{align}
			\end{description}
		\label{thm:Doob_inequality_2}	
		\end{thm}
	\end{itembox}
	
	\begin{prf}
		\begin{align}
			D_n \coloneqq \Set{\tfrac{j}{2^n}T}{j=0,1,\cdots,2^n}
		\end{align}
		とおく.Jensenの不等式より,任意の$0 \leq s < t \leq T$に対して
		\begin{align}
			\cexp{M_t^p}{\mathcal{F}_s} \geq \cexp{M_t}{\mathcal{F}_s}^p \leq M_s^p
		\end{align}
		が成り立つ.従って$(M_t^p)_{t \in I}$は$\mathrm{L}^1$-劣マルチンゲールであり,前定理の結果を使えば
		\begin{align}
			\mu(\max{r \in D_n}{M_r^p} \geq \lambda^p) \leq \frac{1}{\lambda^p} \Norm{M_T}{\mathscr{L}^p}^p
		\end{align}
		が任意の$n \in \N$で成り立つ.非負性から$\max{r \in D_n}{M_r^p} = (\max{r \in D_n}{M_r})^p$となり
		\begin{align}
			\mu(\max{r \in D_n}{M_r} \geq \lambda) \leq \frac{1}{\lambda^p} \Norm{M_T}{\mathscr{L}^p}^p
		\end{align}
		と書き直すことができて,
		\begin{align}
			\mu(M^* \geq \lambda) 
			= \mu(\sup{n \in \N}{\max{r \in D_n}{M_r}} \geq \lambda)
			= \lim_{n \to \infty} \mu(\max{r \in D_n}{M_r} \geq \lambda) \leq \frac{1}{\lambda^p} \Norm{M_T}{\mathscr{L}^p}^p
		\end{align}
		が成り立つ.同じく前定理\footnote{$\mathrm{L}^p$-劣マルチンゲールなら$\mathrm{L}^1$-劣マルチンゲールであるから前定理の結果を適用できる.}を適用し,
		\begin{align}
			\Norm{\max{r \in D_n}{M_r}}{\mathscr{L}^p} \leq \frac{p}{p-1} \Norm{M_T}{\mathscr{L}^p}
		\end{align}
		を保って$n \longrightarrow \infty$とすれば単調収束定理より(2)を得る.
		\QED
	\end{prf}
	
	\begin{itembox}[l]{}
		\begin{thm}[任意抽出定理(2)]
			$I = [0,T]$,$p > 1$,$(M_t)_{t \in I}$を$\mathrm{L}^p$-マルチンゲールとする.
			このとき$I$に値を取る任意の停止時刻$\tau,\sigma$に対し次が成り立つ:
			\begin{align}
				\cexp{M_\tau}{\mathcal{F}_\sigma} = M_{\tau \wedge \sigma}.
			\end{align}
			\label{thm:optional_sampling_theorem_2}
		\end{thm}
	\end{itembox}
	
	\begin{prf}
		\begin{align}
			\tau_n \coloneqq \min{}{\left\{\ T, \frac{1+[2^n \tau]}{2^n}\ \right\}},
			\quad \sigma_n \coloneqq \min{}{\left\{\ T, \frac{1+[2^n \sigma]}{2^n}\ \right\}},
			\quad (n=1,2,\cdots)
		\end{align}
		とおく.このとき$\tau_n,\sigma_n$は停止時刻で$\mathcal{F}_\sigma \subset \mathcal{F}_{\sigma_n}\ (n=1,2,\cdots)$
		が成り立つ.実際任意の$0 \leq t < T$に対して
		\begin{align}
			\left\{ \tau_n \leq t \right\} = \left\{ 1 + [2^n \tau] \leq 2^n t \right\} = \left\{ \tau_n \leq [2^n t]/2^n \right\} \in \mathcal{F}_t
		\end{align}
		となり,$t = T$の時も
		\begin{align}
			\left\{ \tau_n \leq T \right\} = \left\{ 1 + [2^n \tau] > 2^n T \right\} + \left\{ 1 + [2^n \tau] \leq 2^n T \right\} \in \mathcal{F}_T
		\end{align}
		が成り立つから$\tau_n$は停止時刻
		\footnote{
			もとより$\tau_n$は可測関数である.$\R \ni x \longmapsto [x] \in \R$は可測$\borel{\R}/\borel{\R}$であるから
			$[2^n \tau]$は可測$\mathcal{F}/\borel{\R}$であり,従って$\tau_n$も可測$\mathcal{F}/\borel{\R}$となっている.
		}で,
		\begin{align}
			2^n \sigma_n \leq 1 + [2^n \sigma] 
			\Rightarrow \sigma < \sigma_n
		\end{align}
		により$\mathcal{F}_\sigma \subset \mathcal{F}_{\sigma_n}$となる.前定理により任意の$A \in \mathcal{F}_\sigma$に対して
		\begin{align}
			\int_A M_{\tau_n(\omega)}(\omega)\ \mu(d\omega) = \int_A M_{\tau_n(\omega)\wedge \sigma_n(\omega)}(\omega)\ \mu(d\omega) 
		\end{align}
		が成り立ち,$(|M_t|)_{t \in I}$が$\mathrm{L}^p$-劣マルチンゲールであることからDoobの不等式により
		$\sup{t \in I}{M_t}$は可積分である
		\footnote{
			$\sup{t \in I}{|M_t|^p} = \left( \sup{t \in I}{|M_t|} \right)^p$である.
		}.
		従って$\lim_{n \to \infty} \tau_n = \tau$と$M$の右連続性から,Lebesgueの収束定理より
		\begin{align}
			&\int_A M_{\tau(\omega)}(\omega)\ \mu(d\omega) = \lim_{n \to \infty} \int_A M_{\tau_n(\omega)}(\omega)\ \mu(d\omega) \\
			&\quad = \lim_{n \to \infty} \int_A M_{\tau_n(\omega)\wedge \sigma_n(\omega)}(\omega)\ \mu(d\omega)
			= \int_A M_{\tau(\omega)\wedge \sigma(\omega)}(\omega)\ \mu(d\omega)
		\end{align}
		が成り立つ.
		\QED
	\end{prf}
	
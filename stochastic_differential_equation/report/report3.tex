\section{独立性}
	任意の有界実連続関数$h:\R \rightarrow \R$と$\mathcal{F}/\borel{\R}$-可測関数$X$に対して
	その合成$h(X)$は可積分であるから,条件付き期待値を作用させることができる.これを用いて独立性を次で定義する.
	
	\begin{screen}
		\begin{dfn}[独立性]
			Xを$\mathcal{F}/\borel{\R}$-可測関数,$\mathcal{G}$を$\mathcal{F}$の部分$\sigma$-加法族とする.
			任意の有界実連続関数$h:\R \rightarrow \R$に対し次が成り立つとき,$X$と$\mathcal{G}$は独立である(independent)と定める:
			\begin{align}
				\cexp{h(X)}{\mathcal{G}}(\omega) = \int_\Omega h(X(x))\ \mu(dx)
				\quad (\mbox{$\mu$-a.s.}\omega \in \Omega).
			\end{align}
		\end{dfn}
	\end{screen}

	\begin{screen}
		\begin{thm}[独立性の同値条件]
			任意の有界実連続関数$h:\R \rightarrow \R$に対し
			\begin{align}
				\cexp{h(X)}{\mathcal{G}}(\omega) = \int_\Omega h(X(x))\ \mu(dx)
				\quad (\mbox{$\mu$-a.s.}\omega \in \Omega)
				\label{eq:prp_equivalent_condition_of_independence_1}
			\end{align}
			が成り立つことと
			\begin{align}
				\mu\left( X^{-1}(E) \cap A \right) = \mu\left( X^{-1}(E) \right)\mu(A) \quad (\forall E \in \borel{\R},\ A \in \mathcal{G})
				\label{eq:prp_equivalent_condition_of_independence_2}
			\end{align}
			が成り立つことは同値である.
			\label{prp:equivalent_condition_of_independence}
		\end{thm}
	\end{screen}
	
	\begin{prf}\mbox{}
		\begin{description}
			\item[(\refeq{eq:prp_equivalent_condition_of_independence_1})$\Rightarrow$(\refeq{eq:prp_equivalent_condition_of_independence_2})] 
				(\refeq{eq:prp_equivalent_condition_of_independence_1})を仮定して
				\begin{align}
					\borel{\R} = \Set{E \in \borel{\R}}{\mu\left( X^{-1}(E) \cap A \right) = \mu\left( X^{-1}(E) \right) \mu(A),\quad \forall A \in \mathcal{G}}
					\label{eq:prp_equivalent_condition_of_independence_3}
				\end{align}
				が成り立つことを示す.まず(\refeq{eq:prp_equivalent_condition_of_independence_3})の右辺がDynkin族であることを示し,
				次に右辺が$\R$の閉集合系を含むことを示す.
				これが示されればDynkin族定理により(\refeq{eq:prp_equivalent_condition_of_independence_3})が得られる.
				\begin{align}
					\mathscr{D} \coloneqq \Set{E \in \borel{\R}}{\mu\left( X^{-1}(E) \cap A \right) = \mu\left( X^{-1}(E) \right) \mu(A),\quad \forall A \in \mathcal{G}}
				\end{align}
				とおけば,$\mathscr{D}$は次の(1)(2)(3)を満たすからDynkin族である:
				\begin{description}
					\item[(1)] $\R \in \mathscr{D}$,
					\item[(2)] $D_1,D_2 \in \mathscr{D},\ D_1 \subset D_2\quad \Rightarrow\quad D_2 \backslash D_1 \in \mathscr{D}$,
					\item[(3)] $D_n \in \mathscr{D},\ D_n \cap D_m = \emptyset\ (n \neq m)\quad \Rightarrow\quad \sum_{n=1}^{\infty} D_n \in \mathscr{D}$.
				\end{description}
				実際,$\Omega = X^{-1}(\R)$により任意の$A \in \mathcal{G}$に対して
				$\mu\left( X^{-1}(\R) \cap A \right) = \mu\left( X^{-1}(\R) \right) \mu(A)$が成り立つから(1)が従い,
				また$D_1 \subset D_2$ならば$X^{-1}(D_2 \backslash D_1) = X^{-1}(D_2) \backslash X^{-1}(D_1)$が成り立つから
				\begin{align}
					&\mu\left( X^{-1}(D_2 \backslash D_1) \cap A \right) = \mu \left( X^{-1}(D_2) \cap A \right) - \mu\left( X^{-1}(D_1) \cap A \right) \\
					&\qquad = \left\{ \mu\left( X^{-1}(D_2) \right) - \mu \left( X^{-1}(D_1) \right) \right\}\mu(A)
					= \mu\left( X^{-1}(D_2 - D_1) \right) \mu(A)\quad (\forall A \in \mathcal{G})
				\end{align}
				により(2)が従う.(3)も
				\begin{align}
					&\mu\Biggl( X^{-1}\biggl( \sum_{n=1}^{\infty} D_n \biggr) \cap A \Biggr) 
					= \mu\Biggl( \sum_{n=1}^{\infty} X^{-1}(D_n) \cap A \Biggr) 
					= \sum_{n=1}^{\infty} \mu\left( X^{-1}(D_n) \cap A \right) \\
					&\qquad = \sum_{n=1}^{\infty} \mu\left( X^{-1}(D_n) \right) \mu(A) 
					= \mu\Biggl( X^{-1}\biggl( \sum_{n=1}^{\infty} D_n \biggr) \Biggr) \mu(A)\quad (\forall A \in \mathcal{G})
				\end{align}
				により従う.次に$\R$の任意の閉集合が
				$\mathscr{D}$に属することを示す.$E$を$\R$の閉集合として
				\begin{align}
					d(\cdot,E):\R \ni x \longmapsto \inf{}{\Set{|x-y|}{y \in E}}
				\end{align}
				とおき
				\begin{align}
					h_n:\R \ni x \longmapsto \frac{1}{1 + nd(x,E)} \quad (n=1,2,3,\cdots)
				\end{align}
				により有界実連続関数列$(h_n)_{n=1}^{\infty}$を定める.
				$E$が閉集合であるから
				\begin{align}
					\lim_{n \to \infty} h_n(x) = \defunc_E(x)
					\quad (\forall x \in X)
				\end{align}
				が成り立ち,また$h_n$の有界連続性と(\refeq{eq:prp_equivalent_condition_of_independence_1})の仮定により
				\begin{align}
					\cexp{h_n(X)}{\mathcal{G}}(\omega) = \int_\Omega h_n(X(x))\ \mu(dx)
					\quad (\mbox{$\mu$-a.s.}\omega \in \Omega,\ n=1,2,\cdots)
				\end{align}
				が満たされているから,任意に$A \in \mathcal{G}$を取れば
				\begin{align}
					\mu\left( X^{-1}(E) \cap A \right) 
					&= \int_A \defunc_E(X(\omega))\ \mu(d\omega) \\
					&= \lim_{n \to \infty} \int_A h_n(X(\omega))\ \mu(d\omega) \\
					&= \lim_{n \to \infty} \int_A \cexp{h_n(X)}{\mathcal{G}}(\omega)\ \mu(d\omega) \\
					&= \mu(A)  \lim_{n \to \infty} \int_{\Omega} h_n(X(\omega))\ \mu(d\omega) \\
					&= \mu(A) \int_{\Omega} \defunc_E(X(\omega))\ \mu(d\omega) \\
					&= \mu\left( X^{-1}(E) \right) \mu(A)
				\end{align}
				が成り立ち$E \in \mathscr{D}$が従う.
			
			\item[(\refeq{eq:prp_equivalent_condition_of_independence_2})$\Rightarrow$(\refeq{eq:prp_equivalent_condition_of_independence_1})]
				(\refeq{eq:prp_equivalent_condition_of_independence_2})を仮定して
				\begin{align}
					\int_A \cexp{h(X)}{\mathcal{G}}(\omega)\ \mu(d\omega) 
					= \mu(A) \int_\Omega h(X(\omega))\ \mu(d\omega)
					\quad (\forall A \in \mathcal{G})
				\end{align}
				成り立つことを示す.有界実連続関数$h:\R \rightarrow \R$に対して
				$|h_n| \leq |h|$を満たすように単関数近似列$(h_n)_{n=1}^{\infty}$を取る.
				各$h_n$は$\alpha_i^n \in \R,\ E_i^n \in \borel{\R}\ (i=1,\cdots,N_n),\ \sum_{i=1}^{N_n}E_i^n = \R$を用いて
				\begin{align}
					h_n = \sum_{i=1}^{N_n} \alpha_i^n \defunc_{E_i^n}
				\end{align}
				の形で表現できるから,(\refeq{eq:prp_equivalent_condition_of_independence_2})の仮定の下では任意の$A \in \mathcal{G}$に対して
				\begin{align}
					&\int_A h_n(X(\omega))\ \mu(d\omega)
					= \sum_{i=1}^{N_n} \alpha_i^n \int_A \defunc_{X^{-1}(E_i^n)}(\omega)\ \mu(d\omega)
					= \sum_{i=1}^{N_n} \alpha_i^n \mu\left( X^{-1}(E_i^n) \cap A \right) \\
					&\qquad = \mu(A) \sum_{i=1}^{N_n} \alpha_i^n \mu\left( X^{-1}(E_i^n) \right)
					= \sum_{i=1}^{N_n} \alpha_i^n \int_\Omega \defunc_{X^{-1}(E_i^n)}(\omega)\ \mu(d\omega)
					= \mu(A) \int_\Omega h_n(X(\omega))\ \mu(d\omega)
				\end{align}
				が成り立ち,Lebesgueの収束定理より
				\begin{align}
					&\int_A \cexp{h(X)}{\mathcal{G}}(\omega)\ \mu(d\omega)
					= \int_A h(X(\omega))\ \mu(d\omega)
					= \lim_{n \to \infty} \int_A h_n(X(\omega))\ \mu(d\omega) \\
					&\qquad = \mu(A) \lim_{n \to \infty} \int_{\Omega} h_n(X(\omega))\ \mu(d\omega)
					= \mu(A) \int_{\Omega} h(X(\omega))\ \mu(d\omega)
				\end{align}
				が従う.
				\QED
		\end{description}
	\end{prf}

\begin{comment}	
	\begin{screen}
		\begin{thm}
			$A,B \in \mathcal{F}$に対し,$X = \defunc_A$,$\mathcal{G} = \{ \emptyset,\ \Omega,\ B,\ B^c\}$とすれば次が成り立つ:
			\begin{align}
				\mbox{$X$と$\mathcal{G}$が独立}\quad \Leftrightarrow\quad \mu(A \cap B) = \mu(A)\mu(B).
			\end{align}
			\label{thm:report_3}
		\end{thm}
	\end{screen}
	
	\begin{prf}
		$\Rightarrow$については定理\ref{prp:equivalent_condition_of_independence}より従う.
		また任意の$E \in \borel{\R}$に対して
		\begin{align}
			X^{-1}(E) =
			\begin{cases}
				\emptyset & (0,1 \notin E) \\
				A & (0 \notin E,\ 1 \in E) \\
				A^c & (0 \in E,\ 1 \notin E) \\
				\Omega & (0, 1 \in E)
			\end{cases}
		\end{align}
		が成り立つから,
		\begin{align}
			\mu(A \cap B) = \mu(A)\mu(B)
		\end{align}
		の下で
		\begin{align}
			\mu\left( X^{-1}(E) \cap B \right) = \mu\left( X^{-1}(E) \right)\mu(B)
			\quad \left( \forall E \in \borel{\R} \right)
		\end{align}
		が従い
		\footnote{
			$X^{-1}(E) = A^c$の場合
			\begin{align}
				\mu(A^c \cap B) = \mu(B) - \mu(A)\mu(B)  = (1 - \mu(A))\mu(B) = \mu(A^c)\mu(B)
			\end{align}
			が成り立つ.
		}
		,定理\ref{prp:equivalent_condition_of_independence}により$\Leftarrow$が成り立つ.
		\QED
	\end{prf}

\end{comment}
	
\section{条件付き期待値作用素の拡張}
	命題\ref{prp:L2_conditional_expectation}のC3で示した通り,
	(\refeq{eq:dfn_L2_conditional_expectation})で定めた条件付き期待値は$\Lp{2}{\Omega,\mathcal{F},\mu}$から$J\Lp{2}{\Omega,\mathcal{G},\mu}$への
	線型作用素(写像)である.$\Lp{2}{\Omega,\mathcal{F},\mu}$は$\Lp{1}{\Omega,\mathcal{F},\mu}$の部分空間であり,
	同様に$\Lp{2}{\Omega,\mathcal{G},\mu}$も$\Lp{1}{\Omega,\mathcal{G},\mu}$の部分空間であるから,条件付き期待値は
	$\Lp{1}{\Omega,\mathcal{F},\mu}$から$\Lp{1}{\Omega,\mathcal{G},\mu}$への線型作用素でもある.
	条件付き期待値が有界で,且つ
	定義域$\Lp{2}{\Omega,\mathcal{F},\mu}$が$\Lp{1}{\Omega,\mathcal{F},\mu}$において稠密ならば,
	定理\ref{thm:linear_operator_expansion}より条件付き期待値作用素は$\Lp{1}{\Omega,\mathcal{F},\mu}$上の線型写像に拡張可能となる.
	
	\begin{screen}
		\begin{lem}[条件付き期待値作用素の有界性]\mbox{}\\
			$\Lp{1}{\Omega,\mathcal{F},\mu}$から$\Lp{1}{\Omega,\mathcal{G},\mu}$への線型作用素$\cexp{\cdot}{\mathcal{G}}$の
			作用素ノルムは1以下である:
			\begin{align}
				\sup{\substack{f \in \Lp{2}{\Omega,\mathcal{F},\mu} \\ f \neq 0}}{\frac{ \Norm{\cexp{f}{\mathcal{G}}}{\Lp{1}{\mathcal{G}}} }{ \Norm{f}{\Lp{1}{\mathcal{F}}} }} \leq 1.
			\end{align}
			\label{lem:conditional_exp_bound}
		\end{lem}
	\end{screen}
	
	\begin{prf}
		命題\ref{prp:L2_conditional_expectation}のC2より,任意の$0 \neq f \in \Lp{1}{\Omega,\mathcal{F},\mu}$に対して
		\begin{align}
			\Norm{\cexp{f}{\mathcal{G}}}{\Lp{1}{\mathcal{G}}} 
			&= \int_\Omega \left| \cexp{f}{\mathcal{G}}(x) \right|\ \mu(dx) \\
			&= \int_\Omega \cexp{f}{\mathcal{G}}(x) \defunc_{\left\{ \cexp{f}{\mathcal{G}} \geq 0\right \}}(x) 
				- \cexp{f}{\mathcal{G}}(x) \defunc_{\left\{ \cexp{f}{\mathcal{G}} < 0\right\} }(x)\ \mu(dx) \\
			&= \int_\Omega f(x) \defunc_{\left\{ \cexp{f}{\mathcal{G}} \geq 0\right\} }(x) - f(x) \defunc_{\left\{ \cexp{f}{\mathcal{G}} < 0\right\} }(x)\ \mu(dx) \\
			&\leq \int_\Omega |f(x)| \defunc_{\left\{ \cexp{f}{\mathcal{G}} \geq 0\right\} }(x) + |f(x)| \defunc_{\left\{ \cexp{f}{\mathcal{G}} < 0\right\} }(x)\ \mu(dx) \\
			&= \Norm{f}{\Lp{1}{\mathcal{F}}}.
		\end{align}
		が成り立つ.
		\QED
	\end{prf}
	
	\begin{screen}
		\begin{thm}[条件付き期待値作用素の拡張]
			(\refeq{eq:dfn_L2_conditional_expectation})で定めた条件付き期待値$\cexp{\cdot}{\mathcal{G}}$に対して,
			作用素ノルムを変えない拡張線型作用素
			\begin{align}
				\tcexp{\cdot}{\mathcal{G}}:\Lp{1}{\Omega,\mathcal{F},\mu} \ni f \longmapsto \tcexp{f}{\mathcal{G}} \in \Lp{1}{\Omega,\mathcal{G},\mu}
			\end{align}
			が唯一つ存在する.ただし$\mathcal{G} = \{\emptyset,\ \Omega\}$の場合は特別に$\tExp{\cdot} \coloneqq \tcexp{\cdot}{\mathcal{G}}$と表示する.
			\label{thm:conditional_exp_expansion}
		\end{thm}
	\end{screen}
	
	\begin{prf}	
		定理\ref{thm:linear_operator_expansion}より,$\Lp{2}{\Omega,\mathcal{F},\mu}$が$\Lp{1}{\Omega,\mathcal{F},\mu}$で稠密ならば
		補題\ref{lem:conditional_exp_bound}と併せて拡張可能となる.
		今任意に$f \in \Lp{1}{\Omega,\mathcal{F},\mu}$を取り
		\begin{align}
			f_n(x) \coloneqq f(x) \defunc_{|f| \leq n} (x) \quad (\forall x \in \Omega,\ n=1,2,3,\cdots) \label{conditional_exp_expansion}
		\end{align}
		とおけば,$(f_n)_{n=1}^{\infty} \subset \Lp{2}{\Omega,\mathcal{F},\mu}$であり,かつLebesgueの収束定理より
		\begin{align}
			\lim_{n \to \infty} \Norm{f_n - f}{\Lp{1}{\mathcal{F}}} = 0
		\end{align}
		が成り立つ.
		\QED
	\end{prf}
	
	\begin{screen}
		\begin{lem}[凸関数は片側微分可能]
			$\R$から$\R$への凸関数は各点で左右の微係数が存在する.
			\label{lem:convex_function_measurability}
		\end{lem}
	\end{screen}
	
	\begin{prf}
		$\varphi:\R \rightarrow \R$を凸関数とする.先ず凸性から
		\begin{align}
			\frac{\varphi(y) - \varphi(x)}{y - x} 
			\leq \frac{\varphi(z) - \varphi(x)}{z - x}
			\leq \frac{\varphi(z) - \varphi(y)}{z - y}
			\quad (\forall x < y < z)
			\label{ineq:lem:convex_function_measurability_1}
		\end{align}
		を得る.今任意に$x$を取り固定する.$x$に単調減少で近づく点列$(x_n)_{n=1}^{\infty}$を一つ取れば,
		(\refeq{ineq:lem:convex_function_measurability_1})より
		 \begin{align}
		 	\left(\frac{f(x_n)-f(x)}{x_n-x}\right)_{n=1}^{\infty} 
		 	\label{seq:lem:convex_function_measurability_2}
		 \end{align}
		 は下に有界な単調減少列であり極限が存在する.$x$に単調減少で近づく別の点列$(y_k)_{k=1}^{\infty}$を取れば
		 \begin{align}
		 	\inf{k \in \N}{\frac{f(y_k)-f(x)}{y_k-x}} \leq \frac{f(x_n)-f(x)}{x_n-x} \quad (n=1,2,\cdots)
		 \end{align}
		 となるから
		 \begin{align}
		 	\inf{k \in \N}{\frac{f(y_k)-f(x)}{y_k-x}} \leq \inf{n \in \N}{\frac{f(x_n)-f(x)}{x_n-x}}
		 \end{align}
		 が成り立つ.逆向きの不等号も同様に成り立つから,極限は取る点列に依らず確定し$\varphi$は$x$で右側微係数を持つ.
		 同様の理由で左側微係数も存在し,特に$\varphi$の連続性及びBorel可測性が従う.
		 \QED
	\end{prf}
	
	\begin{screen}
		\begin{prp}[拡張条件付き期待値の性質]
		$\mathcal{G},\mathcal{H}$を$\mathcal{F}$の部分$\sigma$-加法族とする.
		\begin{description}
			\item[$\tilde{\mathrm{C}}$1] 任意の$f \in \Lp{1}{\Omega, \mathcal{F},\mu}$に対して次が成り立つ:
				\begin{align}
					\tExp{f} = \int_{\Omega} f(x)\ \mu(dx).
				\end{align}
				
			\item[$\tilde{\mathrm{C}}$2]	任意の$f \in \Lp{1}{\Omega, \mathcal{F},\mu}$と$h \in \Lp{\infty}{\Omega, \mathcal{G},\mu}$に対して次が成り立つ:
				\begin{align}
					\int_{\Omega} f(x)h(x)\ \mu(dx) = \int_{\Omega} \tcexp{f}{\mathcal{G}}(x)h(x)\ \mu(dx).
				\end{align}
				
			\item[$\tilde{\mathrm{C}}$3]	任意の$f,f_1,f_2 \in \Lp{1}{\Omega, \mathcal{F},\mu}$と$\alpha \in \R$に対して次が成り立つ:
				\begin{align}
					\tcexp{f_1 + f_2}{\mathcal{G}} = \tcexp{f_1}{\mathcal{G}} + \tcexp{f_2}{\mathcal{G}},
					\quad \tcexp{\alpha f}{\mathcal{G}} = \alpha \tcexp{f}{\mathcal{G}}.
				\end{align}

			\item[$\tilde{\mathrm{C}}$4]	任意の$f_1,f_2 \in \Lp{1}{\Omega, \mathcal{F},\mu}$に対して次が成り立つ:
				\begin{align}
					f_1 \leq f_2 \quad \mathrm{a.s.} \quad \Rightarrow \quad \tcexp{f_1}{\mathcal{G}} \leq \tcexp{f_2}{\mathcal{G}} \quad \mathrm{a.s.}
				\end{align}
			
			\item[$\tilde{\mathrm{C}}$5] $\varphi:\R \rightarrow \R$を凸関数とする.$f,\varphi(f) \in \Lp{1}{\Omega, \mathcal{F},\mu}$ならば次が成り立つ:
				\begin{align}
					\varphi\left( \cexp{f}{\mathcal{G}} \right) \leq \cexp{\varphi(f)}{\mathcal{G}}.
				\end{align}
				この不等式をJensenの不等式と呼ぶ.
				
			\item[$\tilde{\mathrm{C}}$6]	任意の$f \in \Lp{p}{\Omega, \mathcal{F},\mu}$と
				$g \in \Lp{q}{\Omega, \mathcal{G},\mu}$に対して次が成り立つ:
				\begin{align}
					\tcexp{gf}{\mathcal{G}} = g\tcexp{f}{\mathcal{G}}.
				\end{align}
				ただし$p,q$は$1/p + 1/q = 1,\ (1 \leq p,q \leq \infty)$を満たし,$p = 1$ならば$q = \infty$とする.
				
			\item[$\tilde{\mathrm{C}}$7]	$\mathcal{H}$が$\mathcal{G}$の部分$\sigma$-加法族ならば,任意の$f \in \Lp{1}{\Omega, \mathcal{F},\mu}$に対して次が成り立つ:
				\begin{align}
					\tcexp{\tcexp{f}{\mathcal{G}}}{\mathcal{H}} = \tcexp{f}{\mathcal{H}}.
				\end{align}
		\end{description}
		\label{prp:properties_of_expanded_conditional_expectation}
		\end{prp}
	\end{screen}
	
	\begin{prf}\mbox{}
		\begin{description}
			\item[$\tilde{\mathrm{C}}$1]
				$f$に対して,(\ref{conditional_exp_expansion})と同じように$(f_n)_{n=1}^{\infty} \subset \Lp{2}{\Omega,\mathcal{F},\mu}$を作る.
				命題\ref{prp:L2_conditional_expectation}のC1により
				\begin{align}
					\tExp{f_n} = \int_{\Omega} f_n(x)\ \mu(dx) \quad (\forall n = 1,2,\cdots)
				\end{align}
				が満たされるから,$\tExp{\cdot}$の有界性とLebesgueの収束定理により
				\begin{align}
					\left| \tExp{f} - \int_{\Omega} f(x)\ \mu(dx) \right|
					\leq \Norm{f - f_n}{\Lp{1}{\mathcal{F}}} + \Norm{f - f_n}{\Lp{1}{\mathcal{F}}}
					\longrightarrow 0\ (n \longrightarrow \infty)
				\end{align}
				が成り立つ.
				
			\item[$\tilde{\mathrm{C}}$2]	
				$f$に対して,(\ref{conditional_exp_expansion})と同じように$(f_n)_{n=1}^{\infty} \subset \Lp{2}{\Omega,\mathcal{F},\mu}$を作る.
				命題\ref{prp:L2_conditional_expectation}のC2により
				\begin{align}
					\int_{\Omega} f_n(x)h(x)\ \mu(dx) = \int_{\Omega} \tcexp{f_n}{\mathcal{G}}(x)h(x)\ \mu(dx)
					\quad (\forall n = 1,2,\cdots)
				\end{align}
				が成り立つ.補助定理\ref{lem:conditional_exp_bound}とLebesgueの収束定理より
				\begin{align}
					&\left| \int_{\Omega} f(x)h(x)\ \mu(dx) - \int_{\Omega} \tcexp{f}{\mathcal{G}}(x)h(x)\ \mu(dx) \right| \\
					&\qquad \leq \left| \int_{\Omega} f(x)h(x)\ \mu(dx) - \int_{\Omega} f_n(x)h(x)\ \mu(dx) \right| \\
						&\qquad \qquad+ \left| \int_{\Omega} \tcexp{f_n}{\mathcal{G}}(x)h(x)\ \mu(dx) - \int_{\Omega} \tcexp{f}{\mathcal{G}}(x)h(x)\ \mu(dx) \right| \\
					&\qquad \leq 2 \Norm{h}{\Lp{\infty}{\mathcal{F},\mu}} \Norm{f - f_n}{\Lp{1}{\mathcal{F}}} 
					\longrightarrow 0 \quad (n \longrightarrow \infty)
				\end{align}
				が成り立ち
				\begin{align}
					\int_{\Omega} f(x)h(x)\ \mu(dx) = \int_{\Omega} \tcexp{f}{\mathcal{G}}(x)h(x)\ \mu(dx)
				\end{align}
				が従う.
				
			\item[$\tilde{\mathrm{C}}$3]	
				定理\ref{thm:conditional_exp_expansion}より$\tcexp{\cdot}{\mathcal{G}}$は線型作用素である.

			\item[$\tilde{\mathrm{C}}$4]	
				$\tcexp{\cdot}{\mathcal{G}}$が線型であるから,命題\ref{prp:L2_conditional_expectation}のC4と同様に,
				任意の$f \in \Lp{1}{\Omega, \mathcal{F},\mu}$に対して$f \geq 0\ $ならば$\tcexp{f}{\mathcal{G}} \geq 0$となることを示せばよい.
				$f \in \Lp{1}{\Omega, \mathcal{F},\mu}$を取り
				\begin{align}
					A \coloneqq \Set{x \in \Omega}{f(x) < 0},\quad
					B \coloneqq \Set{x \in \Omega}{\tcexp{f}{\mathcal{G}}(x) < 0}
				\end{align}
				とおき,$\mu(A) = 0$の下で$\mu(B) = 0$が成り立つことを示す.
				$f$に対して,(\ref{conditional_exp_expansion})と同じように$(f_n)_{n=1}^{\infty} \subset \Lp{2}{\Omega,\mathcal{F},\mu}$を作る.
				\begin{align}
					\bigcup_{n=1}^{\infty}\Set{x \in \Omega}{f_n(x) < 0} \subset \Set{x \in \Omega}{f(x) < 0}
				\end{align}
				が成り立つから,$\mu(A)=0$の仮定とC4により
				\begin{align}
					\tcexp{f_n}{\mathcal{G}} \geq 0 \quad (n = 1,2,3,\cdots)
				\end{align}
				が従う.
				\begin{align}
					C_n \coloneqq \Set{x \in \Omega}{\tcexp{f_n}{\mathcal{G}}(x) < 0} \quad (n=1,2,3,\cdots), \quad
					C \coloneqq \bigcup_{n=1}^{\infty} C_n
				\end{align}
				とおけば$\mu(C) = 0$となり
				\begin{align}
					\mu(B \cap C^c) = \mu(B) - \mu(B \cap C) = \mu(B)
				\end{align}
				が成り立つから,$\mu(B \cap C^c) = 0$を示せばよい.
				$B \cap C^c$の上では
				\begin{align}
					\left| \tcexp{f}{\mathcal{G}}(x) - \tcexp{f_n}{\mathcal{G}}(x) \right| > 0
					\quad (n=1,2,\cdots)
				\end{align}
				より,
				\begin{align}
					D_k \coloneqq \Set{x \in B \cap A^c}{\left| \tcexp{f}{\mathcal{G}}(x)\right| > 1/k} \quad (k = 1,2,3,\cdots)
				\end{align}
				とおけば
				\begin{align}
					B \cap C^c = \bigcup_{k=1}^{\infty} D_k
				\end{align}
				が満たされる.全ての$k \in \N$に対して
				\begin{align}
					\Norm{\tcexp{f}{\mathcal{G}} - \tcexp{f_n}{\mathcal{G}}}{\Lp{1}{\mathcal{G}}}
					\geq \int_{C_k} \left| \tcexp{f}{\mathcal{G}}(x) - \tcexp{f_n}{\mathcal{G}}(x) \right|\ \mu(dx)
					> \frac{\mu(C_k)}{k}
				\end{align}
				が成り立ち,左辺は$n \longrightarrow \infty$で0に収束するから
				$\mu(C_k) = 0 \ (k = 1,2,3,\cdots)$が得られ
				\begin{align}
					\mu(B) = \mu(B \cap C^c) \leq \sum_{k=1}^{\infty} \mu(D_k) = 0
				\end{align}
				が従う.
				
			\item[$\tilde{\mathrm{C}}$5]
				補題\ref{lem:convex_function_measurability}より$\varphi$は各点$x \in \R$で右側接線を持つから,
				それを$t \longmapsto a_x t + b_x$と表せば
				\begin{align}
					\varphi(x) = \sup{r \in \Q}{\left\{ a_r x + b_r \right\}} \quad (\forall x \in \R)
					\label{eq:prp_properties_of_expanded_conditional_expectation_1}
				\end{align}
				が成り立つ.実際或る点$x_0 \in \R$で
				\begin{align}
					\varphi(x_0) > \epsilon \coloneqq \sup{r \in \Q}{\left\{ a_r x_0 + b_r \right\}}
					\label{eq:prp_properties_of_expanded_conditional_expectation_2}
				\end{align}
				が成り立つとする.
				任意に$\delta_1 > 0$を取り
				\begin{align}
					\alpha \coloneqq \left| a_{x_0 - \delta_1} \right| \vee \left| a_{x_0 + \delta_1} \right|
				\end{align}
				とおけば,(\refeq{ineq:lem:convex_function_measurability_1})より$x \longmapsto a_x$は単調非減少であるから,
				$|x_0 - r| < \delta_1$を満たす$r \in \Q$に対し
				\begin{align}
					|a_r| \leq \alpha
				\end{align}
				が成り立つ.また$\varphi$の連続性より或る$\delta_2 > 0$が存在して,$|x_0 - r| < \delta_2$となる限り
				\begin{align}
					\left| \varphi(x_0) - \varphi(r) \right| < \frac{\epsilon}{2}
				\end{align}
				が満たされる.従って
				\begin{align}
					\delta_3 \coloneqq \delta_1 \wedge \delta_2 \wedge \frac{\epsilon}{2 \alpha}\ \footnotemark
				\end{align}
				\footnotetext{
					$\alpha = 0$の場合は$\delta_3 = \delta_1 \wedge \delta_2$とおけばよい.
				}
				とおけば,$|x_0 - r| < \delta_3$を満たす$r \in \Q$に対して
				\begin{align}
					\left| \varphi(x_0) - \left( a_r x_0 + b_r \right) \right| 
					\leq \left| \varphi(x_0) - \varphi(r) \right| + |a_r| \left| x_0 - r \right| < \epsilon
				\end{align}
				となり(\refeq{eq:prp_properties_of_expanded_conditional_expectation_2})に矛盾する.
				今,(\refeq{eq:prp_properties_of_expanded_conditional_expectation_1})より任意の$a_r,b_r\ (r \in \Q)$に対して
				\begin{align}
					\varphi(f(\omega)) \geq a_r f(\omega) + b_r \quad (\forall \omega \in \Omega)
				\end{align}
				が成り立つから,$\tilde{\mathrm{C}}$3と$\tilde{\mathrm{C}}$4より
				\begin{align}
					\cexp{\varphi(f)}{\mathcal{G}} \geq a_r \cexp{f}{\mathcal{G}} + b_r \quad (\forall r \in \Q)
				\end{align}
				が従い
				\begin{align}
					\cexp{\varphi(f)}{\mathcal{G}} \geq \varphi\left( \cexp{f}{\mathcal{G}} \right)
				\end{align}
				を得る.
				
			\item[$\tilde{\mathrm{C}}$6]
				$f,g$に対して,(\ref{conditional_exp_expansion})と同じように$(f_n)_{n=1}^{\infty},(g_n)_{n=1}^{\infty}$を作る.
				\begin{description}
					\item[$p=1$の場合]
						命題\ref{prp:L2_conditional_expectation}のC5より
						\begin{align}
							\tcexp{gf_n}{\mathcal{G}} = g\tcexp{f_n}{\mathcal{G}}
							\quad (n=1,2,\cdots) \label{eq:conditional_exp_L1}
						\end{align}
						が成り立っている.
						補助定理\ref{lem:conditional_exp_bound}とLebesgueの収束定理より
						\begin{align}
							\Norm{\tcexp{gf}{\mathcal{G}} - \tcexp{gf_n}{\mathcal{G}}}{\Lp{1}{\mathcal{G}}}
							&\leq \Norm{gf - gf_n}{\Lp{1}{\mathcal{F}}} \\
							&\leq \Norm{g}{\Lp{\infty}{\mathcal{F},\mu}}\Norm{f - f_n}{\Lp{1}{\mathcal{F}}}
							\longrightarrow 0 \quad (n \longrightarrow \infty)
						\end{align}
						が成り立ち,同様にして
						\begin{align}
							\Norm{g\tcexp{f}{\mathcal{G}} - g\tcexp{f_n}{\mathcal{G}}}{\Lp{1}{\mathcal{G}}}
							\leq \Norm{g}{\Lp{\infty}{\mathcal{F}}} \Norm{f - f_n}{\Lp{1}{\mathcal{F}}}
							\longrightarrow 0 \quad (n \longrightarrow \infty)
						\end{align}
						も成り立つから,式(\refeq{eq:conditional_exp_L1})と併せて
						\begin{align}
							&\Norm{\tcexp{gf}{\mathcal{G}} - g\tcexp{f}{\mathcal{G}}}{\Lp{1}{\mathcal{G}}} \\
							&\quad \leq \Norm{\tcexp{gf}{\mathcal{G}} - \tcexp{gf_n}{\mathcal{G}}}{\Lp{1}{\mathcal{G}}}
								+ \Norm{g\tcexp{f_n}{\mathcal{G}} - g\tcexp{f}{\mathcal{G}}}{\Lp{1}{\mathcal{G}}} \\
							&\quad \longrightarrow 0 \quad (n \longrightarrow \infty)
						\end{align}
						が従う.
						
					\item[$1 < p < \infty$の場合]
						$f \in \Lp{p}{\mathcal{F}} \subset \Lp{1}{\mathcal{F}},g_n \in \Lp{\infty}{\mathcal{G}}$である
						から,前段の結果より
						\begin{align}
							\tcexp{g_nf}{\mathcal{G}} = g_n\tcexp{f}{\mathcal{G}}
							\quad (n = 1,2,\cdots)
							\label{eq:conditional_exp_L1_3}
						\end{align}
						が成り立つ.Jensenの不等式とH\Ddot{o}lderの不等式より
						\begin{align}
							\Norm{\tcexp{gf}{\mathcal{G}} - \tcexp{g_nf}{\mathcal{G}}}{\Lp{1}{\mathcal{G}}}
							&\leq \Norm{gf - g_nf}{\Lp{1}{\mathcal{F}}}
							\leq \Norm{g - g_n}{\Lp{q}{\mathcal{G}}} \Norm{f}{\Lp{p}{\mathcal{F}}}, \\
							\Norm{g_n \tcexp{f}{\mathcal{G}} -g \tcexp{f}{\mathcal{G}}}{\Lp{1}{\mathcal{G}}}
							&\leq \Norm{g_n - g}{\Lp{q}{\mathcal{G}}} \Norm{\tcexp{f}{\mathcal{G}}}{\Lp{p}{\mathcal{F}}}
							\leq \Norm{g_n - g}{\Lp{q}{\mathcal{G}}} \Norm{f}{\Lp{p}{\mathcal{F}}}
						\end{align}
						が成り立つから,(\refeq{eq:conditional_exp_L1_3})とLebesgueの収束定理より
						\begin{align}
							&\Norm{\tcexp{gf}{\mathcal{G}} - g\tcexp{f}{\mathcal{G}}}{\Lp{1}{\mathcal{G}}} \\
							&\qquad \leq \Norm{\tcexp{gf}{\mathcal{G}} - \tcexp{g_nf}{\mathcal{G}}}{\Lp{1}{\mathcal{G}}}
								+ \Norm{g_n \tcexp{f}{\mathcal{G}} - g \tcexp{f}{\mathcal{G}}}{\Lp{1}{\mathcal{G}}} \\
							&\qquad \longrightarrow 0 \quad (n \longrightarrow \infty)
						\end{align}
						が従う.
				\end{description}
				
			\item[$\tilde{\mathrm{C}}$7]
				$f$に対して,(\ref{conditional_exp_expansion})と同じように$(f_n)_{n=1}^{\infty} \subset \Lp{2}{\Omega,\mathcal{F},\mu}$を作る.
				命題\ref{prp:L2_conditional_expectation}のC6より
				\begin{align}
					\tcexp{\tcexp{f_n}{\mathcal{G}}}{\mathcal{H}} = \tcexp{f_n}{\mathcal{H}}
					\quad (n=1,2,\cdots)
					\label{eq:conditional_exp_L1_2}
				\end{align}
				が成り立っている.補助定理\ref{lem:conditional_exp_bound}とLebesgueの収束定理より
				\begin{align}
					\Norm{ \tcexp{f}{\mathcal{H}} -  \tcexp{f_n}{\mathcal{H}}}{\Lp{1}{\mathcal{H}}}
					&\leq \Norm{f -  f_n}{\Lp{1}{\mathcal{F}}}
					\longrightarrow 0 \quad (n \longrightarrow \infty), \\
					\Norm{ \tcexp{\tcexp{f}{\mathcal{G}}}{\mathcal{H}} -  \tcexp{\tcexp{f_n}{\mathcal{G}}}{\mathcal{H}}}{\Lp{1}{\mathcal{H}}}
					&\leq \Norm{f -  f_n}{\Lp{1}{\mathcal{F}}}
					\longrightarrow 0 \quad (n \longrightarrow \infty)
				\end{align}
				が成り立つから,(\refeq{eq:conditional_exp_L1_2})と併せて
				\begin{align}
					&\Norm{ \tcexp{\tcexp{f}{\mathcal{G}}}{\mathcal{H}} - \tcexp{f}{\mathcal{H}} }{\Lp{1}{\mathcal{H}}} \\
					&\qquad\leq \Norm{ \tcexp{\tcexp{f}{\mathcal{G}}}{\mathcal{H}} -  \tcexp{\tcexp{f_n}{\mathcal{G}}}{\mathcal{H}}}{\Lp{1}{\mathcal{H}}}
						+ \Norm{ \tcexp{f}{\mathcal{H}} -  \tcexp{f_n}{\mathcal{H}}}{\Lp{1}{\mathcal{H}}} \\
					&\qquad \leq 2 \Norm{f -  f_n}{\Lp{1}{\mathcal{F}}} \longrightarrow 0 \quad (n \longrightarrow \infty)
				\end{align}
				が従う.
				\QED
		\end{description}
	\end{prf}
	
	\begin{screen}
		\begin{dfn}[条件付き期待値の再定義]
			定理\ref{thm:conditional_exp_expansion}で定義した
			有界線型作用素$\tcexp{\cdot}{\mathcal{G}}$
			を$\cexp{\cdot}{\mathcal{G}}$と表記し直し,$\mathcal{G}$で条件付けた条件付き期待値と呼ぶ.
			$\mathcal{G} = \{\emptyset, \Omega\}$の場合は特別に$\Exp{\cdot} \coloneqq \cexp{\cdot}{\mathcal{G}}$
			と書いて期待値と呼ぶ.
		\end{dfn}
	\end{screen}
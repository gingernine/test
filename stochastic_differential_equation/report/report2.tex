基礎におく確率空間を$(\Omega,\mathcal{F},\mu)$と表し,係数体を$\R$とする.

\section{$\mathrm{L}^2$における条件付き期待値}
	\begin{description}
	\item[$\mathrm{L}^2$における内積]
		ノルム空間$\Lp{2}{\Omega,\mathcal{F},\mu}$は次の$\inprod<\cdot,\cdot>_{\Lp{2}{\mathcal{F}}}$を内積としてHilbert空間となる:
		\begin{align}
			\inprod<[f],[g]>_{\Lp{2}{\mathcal{F}}} \coloneqq \int_{\Omega} f(x)g(x)\ \mu(dx) \quad \left(\forall [f],[g] \in \Lp{2}{\Omega,\mathcal{F},\mu} \right).
			\label{eq:L2_inner_product}
		\end{align}
		(\refeq{eq:L2_inner_product})で定める$\inprod<\cdot,\cdot>_{\Lp{2}{\mathcal{F}}}$は
		代表の取り方に依らない実数値で確定する.
		実際H\Ddot{o}lderの不等式より右辺は実数値として確定し,
		$\equiv{f'}{} = \equiv{f}{},\ \equiv{g'}{} = \equiv{g}{}$を満たす
		$f',g'$に対しては
		\begin{align}
			E \coloneqq \Set{x \in \Omega}{f(x) \neq f'(x)}, \quad
			F \coloneqq \Set{x \in \Omega}{g(x) \neq g'(x)}
		\end{align}
		が$\mu$-零集合であるから
		\begin{align}
			\int_{\Omega} f(x)g(x)\ \mu(dx) = \int_{\Omega} f'(x)g'(x)\ \mu(dx)
		\end{align}
		が成り立つ.また次に示すように内積の公理が満たされる:
		\begin{description}
			\item[正値性] 
				ノルムとの対応
				$\Norm{[f]}{\Lp{2}{\mathcal{F}}}^2 = \inprod<[f],[f]>_{\Lp{2}{\mathcal{F}}}$
				により従う.
			\item[対称性] 
				$\int_\Omega fg\ d\mu = \int_\Omega gf\ d\mu$により従う.
			\item[双線型性] 
				片側の線型性を示す.今,任意に$[f],[g],[h] \in \Lp{2}{\Omega,\mathcal{F},\mu}$と$a \in \R$を取れば
				\begin{align}
					&\inprod<[f],[g] + [h]>_{\Lp{2}{\mathcal{F}}} 
					= \inprod<[f],[g + h]>_{\Lp{2}{\mathcal{F}}}
					= \int_{\Omega} f(x)(g(x) + h(x))\ \mu(dx) \\
					&\qquad= \int_{\Omega} f(x)g(x)\ \mu(dx) + \int_{\Omega} f(x)h(x)\ \mu(dx)
					= \inprod<[f],[g]>_{\Lp{2}{\mathcal{F}}} + \inprod<[f],[h]>_{\Lp{2}{\mathcal{F}}}, \\
					&\inprod<[f],\alpha [g]>_{\Lp{2}{\mathcal{F}}} 
						= \inprod<[f],[\alpha g]>_{\Lp{2}{\mathcal{F}}} \\
					&\qquad= \int_{\Omega} \alpha f(x)g(x)\ \mu(dx)
					= \alpha \int_{\Omega} f(x)g(x)\ \mu(dx)
					= \alpha \inprod<[f],[g]>_{\Lp{2}{\mathcal{F}}}
				\end{align}
				が成り立つ.対称性と併せれば双線型性が従う.
		\end{description}
		ノルム空間としての完備性により$\Lp{2}{\Omega,\mathcal{F},\mu}$はHilbert空間となる.
	
	\item[条件付き期待値の存在]
		$\mathcal{G}$を$\mathcal{F}$の部分$\sigma$-加法族とする.
		任意の$[g]_{\mathcal{G}} \in \Lp{2}{\Omega, \mathcal{G},\mu}$
		\footnote{
			$\Lp{2}{\Omega, \mathcal{G},\mu}$は$\Lp{2}{\Omega, \mathcal{F},\mu}$
			とは空間が違うから関数類の表示を変えた.
		}
		に対し$g$は可測$\mathcal{F}/\borel{\R}$
		であるから,$g$を代表とする$[g]_{\mathcal{F}} \in \Lp{2}{\Omega, \mathcal{F},\mu}$が唯一つ存在する.
		\begin{align}
			J_{\mathcal{G}}:\Lp{0}{\Omega, \mathcal{G},\mu} \ni [g]_{\mathcal{G}} \longmapsto [g]_{\mathcal{F}} \in \Lp{0}{\Omega, \mathcal{F},\mu}
			\label{eq:Lp_sp_embedding}
		\end{align}
		として定める線型単射により$\Lp{2}{\Omega, \mathcal{G},\mu}$は$\Lp{2}{\Omega, \mathcal{F},\mu}$に等長に埋め込まれ
		\footnote{
			任意の$1 \leq p \leq \infty$に対し$\mathrm{L}^{p}$は$\mathrm{L}^{0}$の部分集合であるから,
			$J_{\mathcal{G}}$は$\Lp{p}{\Omega, \mathcal{G},\mu}$を$\Lp{p}{\Omega, \mathcal{F},\mu}$に等長に埋め込む.
		},
		$\Lp{2}{\Omega, \mathcal{G},\mu}$の完備性より$J_{\mathcal{G}} \Lp{2}{\Omega, \mathcal{G},\mu}$は閉部分空間となる.
		$\Lp{2}{\Omega, \mathcal{F},\mu}$から$J_{\mathcal{G}} \Lp{2}{\Omega, \mathcal{G},\mu}$への射影作用素を$P_\mathcal{G}$,
		$J_{\mathcal{G}}$の値域を$J_{\mathcal{G}} \Lp{2}{\Omega, \mathcal{G},\mu}$に制限した全単射を$J'_{\mathcal{G}}$と表し,
		\begin{align}
			{J'_{\mathcal{G}}}^{-1} P_{\mathcal{G}}:\Lp{2}{\Omega, \mathcal{F},\mu} \longrightarrow \Lp{2}{\Omega, \mathcal{G},\mu}
			\label{eq:composition_L2_conditional_expectation}
		\end{align}
		により定める線型作用素${J'_{\mathcal{G}}}^{-1} P_{\mathcal{G}}$を条件付き期待値とする.
		また$J_{\mathcal{G}}$は環準同型としての性質も供えている.実際任意の
		$[g]_{\mathcal{G}}, [h]_{\mathcal{G}} \in \Lp{2}{\Omega, \mathcal{G},\mu}$に対して次を満たす:
		\begin{align}
			J_{\mathcal{G}}\left( [g]_{\mathcal{G}} [h]_{\mathcal{G}} \right)
			= J_{\mathcal{G}} [g h]_{\mathcal{G}}
			= [g h]_{\mathcal{F}}
			= [g]_{\mathcal{F}} [h]_{\mathcal{F}}
			= J_{\mathcal{G}} [g]_{\mathcal{G}} J_{\mathcal{G}} [h]_{\mathcal{G}}.
		\end{align}
		
	\item[関数類と関数の表記]
		$\mathrm{L}^p,\mathscr{L}^p$の元について,
		以降は関数類と関数は表記上で区別することはあまりせず,
		状況に応じて$f$を関数類$[f]$,或は関数$f$の意味で扱う.
		なぜならば,例えば後述の命題\ref{prp:L2_conditional_expectation}のC5において,
		$\cexp{gf}{\mathcal{G}} = g\cexp{f}{\mathcal{G}}$と書いているところは本来
		\begin{align}
			\cexp{J_{\mathcal{G}}\equiv{g}{\mathcal{G}}\equiv{f}{\mathcal{F}}}{\mathcal{G}} 
			= \equiv{g}{\mathcal{G}}\cexp{\equiv{f}{\mathcal{F}}}{\mathcal{G}}
		\end{align}
		と表記されるが,これでは非常にややこしいからである.
		ただし主に$f(x)$と表記しているときは関数として,
		条件付き期待値を作用させる場合は関数類として扱っている.
	\end{description}
	
	\begin{screen}
		\begin{dfn}[$\mathrm{L}^2$における条件付き期待値]
			(\refeq{eq:composition_L2_conditional_expectation})で定めた線型作用素を
			\begin{align}
				\cexp{\cdot}{\mathcal{G}}:\Lp{2}{\Omega, \mathcal{F},\mu} \ni f \longmapsto \cexp{f}{\mathcal{G}} \in \Lp{2}{\Omega, \mathcal{G},\mu}
				\label{eq:dfn_L2_conditional_expectation}
			\end{align}
			と表記し直し,$\mathcal{G}$の下での条件付き期待値(conditional expectation)と呼ぶ.
			$\mathcal{G} = \{\emptyset, \Omega\}$の場合は特別に
			$\Exp{\cdot} \coloneqq \cexp{\cdot}{\mathcal{G}}$
			と書いて期待値と呼ぶ.
		\end{dfn}
	\end{screen}
	
	\begin{screen}
	\begin{thm}[条件付き期待値の性質]
		Hilbert空間$\Lp{2}{\Omega, \mathcal{F},\mu}$における内積を$\inprod<\cdot,\cdot>_{\Lp{2}{\mathcal{F}}}$,ノルムを$\Norm{\cdot}{\Lp{2}{\mathcal{F}}}$と表記し,
		$\mathcal{G},\mathcal{H}$を$\mathcal{F}$の部分$\sigma$-加法族とする.
		\begin{description}
			\item[C1] 任意の$f \in \Lp{2}{\Omega, \mathcal{F},\mu}$に対して次が成り立つ:
				\begin{align}
					\Exp{f} = \int_{\Omega} f(x)\ \mu(dx)\ \footnotemark.
				\end{align}
				
			\item[C2]	任意の$f \in \Lp{2}{\Omega, \mathcal{F},\mu}$と$h \in \Lp{2}{\Omega, \mathcal{G},\mu}$に対して次が成り立つ:
				\begin{align}
					\int_{\Omega} f(x)h(x)\ \mu(dx) = \int_{\Omega} \cexp{f}{\mathcal{G}}(x)h(x)\ \mu(dx).
				\end{align}
				
			\item[C3]	任意の$f,f_1,f_2 \in \Lp{2}{\Omega, \mathcal{F},\mu}$と$\alpha \in \R$に対して次が成り立つ:
				\begin{align}
					\cexp{f_1 + f_2}{\mathcal{G}} = \cexp{f_1}{\mathcal{G}} + \cexp{f_2}{\mathcal{G}},
					\quad \cexp{\alpha f}{\mathcal{G}} = \alpha \cexp{f}{\mathcal{G}}.
				\end{align}

			\item[C4]	任意の$f_1,f_2 \in \Lp{2}{\Omega, \mathcal{F},\mu}$に対して次が成り立つ:
				\begin{align}
					f_1 \leq f_2 \quad \Rightarrow \quad \cexp{f_1}{\mathcal{G}} \leq \cexp{f_2}{\mathcal{G}} \ \footnotemark
				\end{align}
			
			\item[C5]	任意の$f \in \Lp{2}{\Omega, \mathcal{F},\mu}$と$g \in \Lp{\infty}{\Omega, \mathcal{G},\mu}$に対して次が成り立つ:
				\begin{align}
					\cexp{gf}{\mathcal{G}} = g\cexp{f}{\mathcal{G}}.
				\end{align}
			
			\item[C6]	$\mathcal{H}$が$\mathcal{G}$の部分$\sigma$-加法族ならば,任意の$f \in \Lp{2}{\Omega, \mathcal{F},\mu}$に対して次が成り立つ:
				\begin{align}
					\cexp{\cexp{f}{\mathcal{G}}}{\mathcal{H}} = \cexp{f}{\mathcal{H}}.
				\end{align}
		\end{description}
		\label{prp:L2_conditional_expectation}
	\end{thm}
	\end{screen}
	
	\footnotetext{
		本来は,恒等的に$\int_{\Omega} f(x)\ \mu(dx)$のみを取る定数関数を代表とする関数類が$\Exp{f}$である.
	}
				
	\footnotetext{
		関数類に対する順序(式(\refeq{dfn:equiv_class_order}))を表している.
	}
	
	\begin{prf}\mbox{}
		\begin{description}
			\item[C1] 
				(\refeq{eq:Lp_sp_embedding})で定めた単射$J_{\mathcal{G}}$を$J$と表す.
				$\mathcal{G} = \{\emptyset, \Omega\}$とすれば,
				$\Lp{2}{\Omega, \mathcal{G},\mu}$の元の代表は$\mathcal{G}$-可測でなくてはならないから定数関数である.
				従って各$g \in \Lp{2}{\Omega, \mathcal{G},\mu}$にはただ一つの定数$\alpha \in \R$が対応して$g(x)=\alpha\ (\forall x \in \Omega)$と表すことができ,
				かつ関数$g$を代表とする$\Lp{2}{\Omega, \mathcal{G},\mu}$の関数類は関数$g$のみからなる.
				射影定理よりノルム$\Norm{f-J g}{\Lp{2}{\mathcal{F}}}$を
				最小にする$g \in \Lp{2}{\Omega, \mathcal{G},\mu}$が$\Exp{f}$である.
				今$g(x)=\alpha\ (\forall x \in \Omega)$であるとして
				\begin{align}
					\beta \coloneqq \int_{\Omega} f(x)\ \mu(dx)
				\end{align}
				とおけば,
				\begin{align}
					\Norm{f-J g}{\Lp{2}{\mathcal{F}}}^2 
					&= \int_{\Omega} |f(x) - \alpha|^2\ \mu(dx) \\
					&= \int_{\Omega} |f(x)|^2\ \mu(dx) - 2 \alpha \beta + |\alpha|^2 \\
					&= \left| \alpha - \beta \right|^2 + \int_{\Omega} \left| f(x) - \beta \right|^2\ \mu(dx)
				\end{align}
				が成り立ち,最終式は$\alpha = \int_{\Omega} f(x)\ \mu(dx)$のとき最小となる.従って
				\begin{align}
					\Exp{f} = \int_{\Omega} f(x)\ \mu(dx)
				\end{align}
				を得る.
				
			\item[C2] 
				(\refeq{eq:Lp_sp_embedding})で定めた単射$J_{\mathcal{G}}$を$J$と表す.
				射影定理により,$f \in \Lp{2}{\Omega, \mathcal{F},\mu}$に対する$\cexp{f}{\mathcal{G}}$は
				\begin{align}
					\inprod<f - J \cexp{f}{\mathcal{G}}, h>_{\Lp{2}{\mathcal{F}}} = 0 \quad (\forall h \in J \Lp{2}{\Omega, \mathcal{G},\mu})
				\end{align}
				を満たすから,内積の線型性より任意の$h \in J \Lp{2}{\Omega, \mathcal{G},\mu}$に対して
				\begin{align}
					&\int_{\Omega} f(x)h(x)\ \mu(dx) = \inprod<f, h>_{\Lp{2}{\mathcal{F}}} \\
					&\qquad = \inprod<J \cexp{f}{\mathcal{G}}, h>_{\Lp{2}{\mathcal{F}}} = \int_{\Omega} \cexp{f}{\mathcal{G}}(x)h(x)\ \mu(dx)
				\end{align}
				が成り立つ.
				
			\item[C3] (\refeq{eq:composition_L2_conditional_expectation})における合成作用素は線型作用素である.
				\begin{comment}
				(\refeq{eq:Lp_sp_embedding})で定めた単射$J_{\mathcal{G}}$を$J$と表す.
				\begin{description}
					\item[加法について]
						射影定理により任意の$h \in J \Lp{2}{\Omega, \mathcal{G},\mu}$に対して
						\begin{align}
							&\inprod<(f_1 + f_2) - J\cexp{f_1 + f_2}{\mathcal{G}}, h>_{\Lp{2}{\mathcal{F}}} = 0, \\
							&\inprod<f_1 - J\cexp{f_1}{\mathcal{G}}, h>_{\Lp{2}{\mathcal{F}}} = 0
							,\quad \inprod<f_2 - J\cexp{f_2}{\mathcal{G}}, h>_{\Lp{2}{\mathcal{F}}} = 0
						\end{align}
						が成り立つから,任意の$h \in J\Lp{2}{\Omega, \mathcal{G},\mu}$に対して
						\begin{align}
							0 &= \inprod<(f_1 + f_2) - J\cexp{f_1 + f_2}{\mathcal{G}}, h>_{\Lp{2}{\mathcal{F}}} \\
								&\qquad- \inprod<f_1 - J\cexp{f_1}{\mathcal{G}}, h>_{\Lp{2}{\mathcal{F}}} - \inprod<f_2 - J\cexp{f_2}{\mathcal{G}}, h>_{\Lp{2}{\mathcal{F}}} \\
							&= \inprod<J\cexp{f_1}{\mathcal{G}} + J\cexp{f_2}{\mathcal{G}} - J\cexp{f_1 + f_2}{\mathcal{G}}, h>_{\Lp{2}{\mathcal{F}}}
						\end{align}
						となり,特に$h = J\cexp{f_1}{\mathcal{G}} + J\cexp{f_2}{\mathcal{G}} - J\cexp{f_1 + f_2}{\mathcal{G}} \in J\Lp{2}{\Omega, \mathcal{G},\mu}$として
						\begin{align}
							\Norm{J\cexp{f_1}{\mathcal{G}} + J\cexp{f_2}{\mathcal{G}} - J\cexp{f_1 + f_2}{\mathcal{G}}}{\Lp{2}{\mathcal{F}}}^2 = 0
						\end{align}
						を得る.$J$の線型単射性により
						\begin{align}
							\cexp{f_1}{\mathcal{G}} + \cexp{f_2}{\mathcal{G}} = \cexp{f_1 + f_2}{\mathcal{G}}
						\end{align}
						が従う.
						
					\item[スカラ倍について]
							射影定理より任意の$h \in J\Lp{2}{\Omega, \mathcal{G},\mu}$に対して
							\begin{align}
								\inprod<f - J\cexp{f}{\mathcal{G}}, h>_{\Lp{2}{\mathcal{F}}} = 0, \quad \inprod<\alpha f - J\cexp{\alpha f}{\mathcal{G}}, h>_{\Lp{2}{\mathcal{F}}} = 0
							\end{align}
							が成り立つから
							\begin{align}
								&\inprod<J\cexp{\alpha f}{\mathcal{G}} - \alpha J\cexp{f}{\mathcal{G}}, h>_{\Lp{2}{\mathcal{F}}} \\
								&\qquad = \inprod<J\cexp{\alpha f}{\mathcal{G}} - \alpha f, h>_{\Lp{2}{\mathcal{F}}} - \alpha \inprod<J\cexp{f}{\mathcal{G}} - f, h>_{\Lp{2}{\mathcal{F}}}
								= 0
							\end{align}
							となる.特に$h = J\cexp{\alpha f}{\mathcal{G}} - \alpha J\cexp{f}{\mathcal{G}} \in J\Lp{2}{\Omega, \mathcal{G},\mu}$として
							\begin{align}
								\Norm{J\cexp{\alpha f}{\mathcal{G}} - \alpha J\cexp{f}{\mathcal{G}}}{\Lp{2}{\mathcal{F}}}^2 = 0
							\end{align}
							が得られ,$J$の線型単射性より$\cexp{\alpha f}{\mathcal{G}} = \alpha \cexp{f}{\mathcal{G}}$が従う.
				\end{description}
				\end{comment}
				
			\item[C4] (\refeq{eq:Lp_sp_embedding})で定めた単射$J_{\mathcal{G}}$を$J$と表す.
				任意の$f \in \Lp{2}{\Omega, \mathcal{F},\mu}$に対して$f \geq 0$ならば$\cexp{f}{\mathcal{G}} \geq 0$が成り立つことを示せばよい.
				実際これが示されれば$f_1,f_2 \in \Lp{2}{\Omega, \mathcal{F},\mu}$に対して
				\begin{align}
					0 \leq f_2 - f_1 \quad \Rightarrow \quad 0 \leq \cexp{f_2 - f_1}{\mathcal{G}}
				\end{align}
				が従い,C3と併せて主張を得る.証明は背理法を使う.
				\begin{align}
					A \coloneqq \Set{x \in \Omega}{f(x) < 0}, \quad
					B \coloneqq \Set{x \in \Omega}{\cexp{f}{\mathcal{G}}(x) < 0}
				\end{align}
				とおき,$\mu(A)=0$の下で$\mu(B) > 0$が成り立つと仮定して矛盾を導く.
				\begin{align}
					h(x) \coloneqq
					\begin{cases}
						\cexp{f}{\mathcal{G}}(x) & (x \in \Omega \backslash B) \\
						0 & (x \in B)
					\end{cases}
				\end{align}
				として$\mathcal{G}/\borel{\R}$-可測関数を定義すれば
				\begin{align}
					\Norm{f - Jh}{\Lp{2}{\mathcal{F}}}^2 
					&= \int_{\Omega} |f(x) - h(x)|^2\ \mu(dx) \\
					&= \int_{A^c \cap B^c} \left|f(x) - \cexp{f}{\mathcal{G}}(x)\right|^2\ \mu(dx) + \int_{A^c \cap B} |f(x)|^2\ \mu(dx) \\
					&< \int_{A^c \cap B^c} \left|f(x) - \cexp{f}{\mathcal{G}}(x)\right|^2\ \mu(dx) + \int_{A^c \cap B} \left|f(x) - \cexp{f}{\mathcal{G}}(x)\right|^2\ \mu(dx) \\
					&= \Norm{f - J\cexp{f}{\mathcal{G}}}{\Lp{2}{\mathcal{F}}}^2
				\end{align}
				が成り立つ.途中の不等号$<$は,$\mu(A^c \cap B) = \mu(B) - \mu(A \cap B) = \mu(B) > 0$であることと
				\begin{align}
					0 \leq f(x) < f(x) - \cexp{f}{\mathcal{G}}(x) \quad (\forall x \in A^c \cap B)
				\end{align}
				による.しかし
				\begin{align}
					\Norm{f - Jh}{\Lp{2}{\mathcal{F}}} < \Norm{f - J\cexp{f}{\mathcal{G}}}{\Lp{2}{\mathcal{F}}}
				\end{align}
				を満たす$h \in \Lp{2}{\Omega, \mathcal{G},\mu}$が存在することは
				$J\cexp{f}{\mathcal{G}}$が$f$の射影であることに矛盾する.よって$\mu(A) = 0$の下では$\mu(B) = 0$でなくてはならず,冒頭の主張が従う.
			
			\item[C5] (\refeq{eq:Lp_sp_embedding})で定めた単射$J_{\mathcal{G}}$を$J$と表し,
				任意の$h \in J\Lp{2}{\Omega, \mathcal{G},\mu}$に対して
				\begin{align}
					&\inprod<J\cexp{gf}{\mathcal{G}} - Jg J\cexp{f}{\mathcal{G}}, h>_{\Lp{2}{\mathcal{F}}} \\
					&\qquad = \inprod<J\cexp{gf}{\mathcal{G}} - Jg f, h>_{\Lp{2}{\mathcal{F}}} + \inprod<Jg f - Jg J\cexp{f}{\mathcal{G}}, h>_{\Lp{2}{\mathcal{F}}}
					= 0
					\label{eq:prp_L2_conditional_expectation_C5}
				\end{align}
				が成り立つことを示す.$g$の代表の有界性より$Jg f \in \Lp{2}{\Omega, \mathcal{F},\mu}$が満たされているから,
				$Jg f$に対して条件付き期待値を作用させることができ,射影定理により
				\begin{align}
					\inprod<J\cexp{gf}{\mathcal{G}} - Jg f, h>_{\Lp{2}{\mathcal{F}}} = 0
				\end{align}
				が従う.一方右辺第二項についても,
				\begin{align}
					&\inprod<Jg f - Jg J\cexp{f}{\mathcal{G}}, h>_{\Lp{2}{\mathcal{F}}} \\
					&\qquad = \int_{\Omega} \left( f(x) - \cexp{f}{\mathcal{G}}(x) \right) g(x)h(x)\ \mu(dx)
					= \inprod<f - J\cexp{f}{\mathcal{G}}, Jg h>_{\Lp{2}{\mathcal{F}}}
				\end{align}
				が成り立ち,$Jg h \in J\Lp{2}{\Omega, \mathcal{G},\mu}$が満たされているから射影定理により
				\begin{align}
					\inprod<Jg f - Jg J\cexp{f}{\mathcal{G}}, h>_{\Lp{2}{\mathcal{F}}} 
					= \inprod<f - J\cexp{f}{\mathcal{G}}, Jg h>_{\Lp{2}{\mathcal{F}}} = 0
				\end{align}
				が従い(\refeq{eq:prp_L2_conditional_expectation_C5})が得られる.
				特に$h = J\cexp{gf}{\mathcal{G}} - Jg J\cexp{f}{\mathcal{G}} \in J\Lp{2}{\Omega, \mathcal{G},\mu}$として
				\begin{align}
					\Norm{J\cexp{gf}{\mathcal{G}} - Jg J\cexp{f}{\mathcal{G}}}{\Lp{2}{\mathcal{F}}}^2 = 0
				\end{align}
				が成り立つから,環準同型$J$の線型等長性により$\cexp{gf}{\mathcal{G}} = g\cexp{f}{\mathcal{G}}$が得られる.
			
			\item[C6] 
				(\refeq{eq:Lp_sp_embedding})で定めた単射$J_{\mathcal{G}}$を$J_1$と表し,同様に$J_{\mathcal{H}}$を$J_2$と表す.今
				\begin{align}
					J_2 \Lp{2}{\Omega, \mathcal{H},\mu} \subset J_1 \Lp{2}{\Omega, \mathcal{G},\mu}
				\end{align}
				が満たされているから,
				射影定理より任意の$h \in J_2 \Lp{2}{\Omega, \mathcal{H},\mu}$に対して
				\begin{align}
					&\inprod<J_2 \cexp{\cexp{f}{\mathcal{G}}}{\mathcal{H}} - J_2 \cexp{f}{\mathcal{H}}, h>_{\Lp{2}{\mathcal{F}}} \\
					&\qquad = \inprod<J_2 \cexp{\cexp{f}{\mathcal{G}}}{\mathcal{H}} - J_1 \cexp{f}{\mathcal{G}}, h>_{\Lp{2}{\mathcal{F}}} \\
						&\qquad \qquad + \inprod<J_1 \cexp{f}{\mathcal{G}} - f, h>_{\Lp{2}{\mathcal{F}}} + \inprod<f - J_2 \cexp{f}{\mathcal{H}}, h>_{\Lp{2}{\mathcal{F}}}
					= 0
				\end{align}
				が成り立つ.特に$h = J_2 \cexp{\cexp{f}{\mathcal{G}}}{\mathcal{H}} - J_2 \cexp{f}{\mathcal{H}} \in J_2 \Lp{2}{\Omega, \mathcal{H},\mu}$とすれば
				\begin{align}
					\Norm{J_2 \cexp{\cexp{f}{\mathcal{G}}}{\mathcal{H}} - J_2 \cexp{f}{\mathcal{H}}}{\Lp{2}{\mathcal{F}}}^2 = 0
				\end{align}
				が得られ,$J_2$の線型等長性より$\cexp{\cexp{f}{\mathcal{G}}}{\mathcal{H}} = \cexp{f}{\mathcal{H}}$が従う.
				\QED
		\end{description}
	\end{prf}

	$M \in \mathcal{M}_{2,c},\ X \in \semiLp{2}{\mathcal{P},\mu_M}$に対して
	定義した伊藤積分を更に拡張する.
	
	\begin{screen}
		\begin{dfn}[局所有界過程]
			$(\Omega,\mathcal{F},\mu)$上の確率過程$X$に対し或る$(\tau_j)_{j=0}^{\infty} \in \mathcal{T}$が存在して
			\begin{align}
				\sup{t \in I}{\Norm{X_{t \wedge \tau_j}}{\mathscr{L}^\infty}} < \infty
				\quad (j=0,1,\cdots)
			\end{align}
			が満たされているとき,$X$を局所有界過程(locally bounded process)という.
		\end{dfn}
	\end{screen}

\newpage
	\begin{screen}
		\begin{thm}[局所マルチンゲールと左連続局所有界適合過程に対する伊藤積分]
			$X$を左連続且つ局所有界な適合過程,$M \in \mathcal{M}_{c,loc}$とする.
			このとき確率積分
			\begin{align}
				\int_0^t X_s\ dM_s \quad (t \in I)
			\end{align}
			が定義される.
			\label{thm:Ito_integral_on_M_c_loc_and_left_cont_locally_bounded}
		\end{thm}
	\end{screen}
	
	\begin{prf}\mbox{}
		\begin{description}
			\item[第一段] $\sup{t \in I}{\Norm{X_t}{\semiLp{\infty}{\mu}}} < \infty$かつ$\Norm{\inprod<M>_T}{\semiLp{\infty}{\mu}} < \infty$
				が満たされている場合$X \in \semiLp{2}{\mathcal{P},\mu_M}$が成り立つことを示す.%すなわち,このとき$I_M(X)$が定義される.
				$X$は左連続であるから定理\ref{thm:left_continuous_adapted_then_predictable}により可測$\mathcal{P}/\borel{\R}$である.
				また定理\ref{thm:quadratic_variation_bounded_then_M_2c}より$M \in \mathcal{M}_{2,c}$も満たされている.
				$X$に対し(\refeq{eq:thm_left_continuous_adapted_then_predictable})で定義される単純可予測過程の列$(X^n)_{n=1}^{\infty}$を取れば,
				$X$の有界性及び$\mu_M$の有限性によりLebesgueの収束定理を適用できて
				\begin{align}
					\Norm{X - X^n}{\semiLp{2}{\mu_M}} \longrightarrow 0 
					\quad (n \longrightarrow \infty)
				\end{align}
				が成り立つ.
				補題\ref{lem:properties_of_simple_predictable_processes}より$\mathcal{S}$は
				$\semiLp{2}{\mathcal{P},\mu_M}$で稠密であるから
				$X \in \semiLp{2}{\mathcal{P},\mu_M}$が従う.
				
			\item[第二段]
				前段の仮定を外す.$X$が局所有界過程であるから,或る$(\tau_j)_{j=0}^{\infty} \in \mathcal{T}$が存在して
				\begin{align}
					\sup{t \in I}{\Norm{X_{t \wedge \tau_j}}{\semiLp{\infty}{\mu}}} < \infty
					\quad (j=0,1,\cdots)
				\end{align}
				が満たされる.また
				\begin{align}
					\hat{\tau}_j(\omega) \coloneqq
					\inf{}{\Set{t \in I}{\inprod<M>_t(\omega) \geq j}} \wedge T\ \footnotemark
					\quad (\forall \omega \in \Omega,\ j=0,1,\cdots)
				\end{align}
				\footnotetext{
					$\Set{t \in I}{|\inprod<M>_t(\omega)| \geq j} = \emptyset$の場合$\sigma_j(\omega) = T$とする.
				}
				として$\left( \hat{\tau}_j \right)_{j=0}^{\infty} \in \mathcal{T}$を定め
				\begin{align}
					\sigma_j \coloneqq \tau_j \wedge \hat{\tau}_j
					\quad (j=0,1,\cdots)
				\end{align}
				とおけば,$(\sigma_j)_{j=0}^{\infty} \in \mathcal{T}$且つ
				\begin{align}
					\Norm{X^{\sigma_j}_t}{\semiLp{\infty}{\mu}} \leq \Norm{X^{\tau_j}_t}{\semiLp{\infty}{\mu}},
					\quad \Norm{\inprod<M>^{\sigma_j}_t}{\semiLp{\infty}{\mu}} \leq j
					\quad (\forall t \in I,\ j=0,1,\cdots)
				\end{align}
				が成り立つ.従って前段の結果より$I_{M^{\sigma_j}}(X^{\sigma_j})\ (j=0,1,\cdots)$が定義される.
				
			\item[第三段]
				次の極限が$\mu$-a.s.に確定することを示す:
				\begin{align}
					\lim_{j \to \infty} I_{M^{\sigma_j}}(X^{\sigma_j})_{t \wedge \sigma_j}
					\quad (\forall t \in I).
				\end{align}
				$(\sigma_j)_{j=0}^{\infty} \in \mathcal{T}$より或る$\mu$-零集合$E$が存在して,$\omega \in \Omega \backslash E$なら
				$0 = \sigma_0(\omega) \leq \sigma_1(\omega) \leq \cdots$且つ,或る$J = J(\omega) \in \N$が存在して
				$\sigma_j(\omega) = T\ (j \geq J)$が満たされる.今,$j \leq k$を満たす$j,k \in \N_0$を任意に取り固定する
				\footnote{
					$\N_0 = \N \cup \{0\}.$
				}
				.任意に$Y \in \mathcal{S}$を取り,$Y$が時点$0=t_0 < t_1 < \cdots < t_n = T$と
				$F \in \semiLp{\infty}{\Omega,\mathcal{F}_0,\mu},F_i \in \semiLp{\infty}{\Omega,\mathcal{F}_{t_i},\mu}\ (i=0,1,\cdots,n-1)$によって
				\begin{align}
					Y_t = F \defunc_{\{0\}}(t) + \sum_{i=0}^{n-1} F_i \defunc_{\left(t_i,t_{i+1}\right]}(t)
					\quad (t \in I)
				\end{align}
				と表現されているとすれば,$\mathcal{M}_{2,c}$上の確率積分の定義より
				\begin{align}
					I_{M^{\sigma_j}}(Y)_t = \sum_{i=0}^{n-1} F_i \left( M^{\sigma_j}_{t \wedge t_{i+1}} - M^{\sigma_j}_{t \wedge t_i} \right)
					\quad (\forall t \in I,\ \mbox{$\mu$-a.s.})
				\end{align}
				が成り立つ.特に両辺を$\sigma_j$で停めても等号は保たれ
				\begin{align}
					I_{M^{\sigma_j}}(Y)^{\sigma_j}_t = \sum_{i=0}^{n-1} F_i \left( M^{\sigma_j}_{t \wedge t_{i+1}} - M^{\sigma_j}_{t \wedge t_i} \right)
					\quad (\forall t \in I,\ \mbox{$\mu$-a.s.})
					\label{eq:thm_Ito_integral_on_M_c_loc_and_left_cont_locally_bounded_1}
				\end{align}
				を得る.$\sigma_{j+k}$についても同様に
				\begin{align}
					I_{M^{\sigma_{j+k}}}(Y)_t = \sum_{i=0}^{n-1} F_i \left( M^{\sigma_{j+k}}_{t \wedge t_{i+1}} - M^{\sigma_{j+k}}_{t \wedge t_i} \right)
					\quad (\forall t \in I,\ \mbox{$\mu$-a.s.})
				\end{align}
				が成り立ち,特に$\Omega \backslash E$上では$\sigma_j \leq \sigma_{j+k}$が満たされるから,両辺を$\sigma_j$で停めて
				\begin{align}
					I_{M^{\sigma_{j+k}}}(Y)^{\sigma_j}_t = \sum_{i=0}^{n-1} F_i \left( M^{\sigma_j}_{t \wedge t_{i+1}} - M^{\sigma_j}_{t \wedge t_i} \right)
					\quad (\forall t \in I,\ \mbox{$\mu$-a.s.})
					\label{eq:thm_Ito_integral_on_M_c_loc_and_left_cont_locally_bounded_2}
				\end{align}
				が得られ,(\refeq{eq:thm_Ito_integral_on_M_c_loc_and_left_cont_locally_bounded_1})と
				(\refeq{eq:thm_Ito_integral_on_M_c_loc_and_left_cont_locally_bounded_2})を併せれば
				\begin{align}
					I_{M^{\sigma_j}}(Y)^{\sigma_j}_t = I_{M^{\sigma_{j+k}}}(Y)^{\sigma_j}_t
					\quad (\forall t \in I,\ \mbox{$\mu$-a.s.})
				\end{align}
				が従う.一般の$Y \in \semiLp{2}{\mathcal{P},\mu_{M^{\sigma_j}}} \cap \semiLp{2}{\mathcal{P},\mu_{M^{\sigma_{j+k}}}}$に対しては
				或る$Y_n \in \mathcal{S}\ (n=1,2,\cdots)$が存在して
				\begin{align}
					\int_{I \times \Omega} \left| Y(t,\omega) - Y_n(t,\omega) \right|^2\ \mu_{M^{\sigma_j}}(dtd\omega)
					\longrightarrow 0
					\quad (n \longrightarrow \infty)
				\end{align}
				を満たすから
				\begin{align}
					&\Norm{I_{M^{\sigma_j}}(Y)^{\sigma_j} - I_{M^{\sigma_{j+k}}}(Y)^{\sigma_j}}{\mathcal{M}_{2,c}} \\
					&\qquad \leq \Norm{I_{M^{\sigma_j}}(Y)^{\sigma_j} - I_{M^{\sigma_j}}(Y_n)^{\sigma_j}}{\mathcal{M}_{2,c}}
						+ \Norm{I_{M^{\sigma_{j+k}}}(Y_n)^{\sigma_j} - I_{M^{\sigma_{j+k}}}(Y)^{\sigma_j}}{\mathcal{M}_{2,c}} \\
					&\qquad \leq \Norm{Y - Y_n}{\semiLp{2}{\mu_{M^{\sigma_j}}}} + \Norm{Y - Y_n}{\semiLp{2}{\mu_{M^{\sigma_{j+k}}}}}
					\quad \longrightarrow 0 \quad (n \longrightarrow \infty)
				\end{align}
				が成り立ち
				\begin{align}
					I_{M^{\sigma_j}}(Y)^{\sigma_j}_t = I_{M^{\sigma_{j+k}}}(Y)^{\sigma_j}_t
					\quad (\forall t \in I,\ \mbox{$\mu$-a.s.})
					\label{eq:thm_Ito_integral_on_M_c_loc_and_left_cont_locally_bounded_3}
				\end{align}
				が従う.特に$X^{\sigma_j}$は$\Norm{\cdot}{\semiLp{\infty}{\mu}}$について一様有界であるから
				第一段より$X^{\sigma_j} \in \semiLp{2}{\mathcal{P},\mu_{M^{\sigma_j}}} \cap \semiLp{2}{\mathcal{P},\mu_{M^{\sigma_{j+k}}}}$
				が満たされ,(\refeq{eq:thm_Ito_integral_on_M_c_loc_and_left_cont_locally_bounded_3})より
				或る$\mu$-零集合$A_{j,k}$が存在して
				\begin{align}
					I_{M^{\sigma_j}}(X^{\sigma_j})^{\sigma_j}_t(\omega) = I_{M^{\sigma_{j+k}}}(X^{\sigma_j})^{\sigma_j}_t(\omega)
					\quad (\forall t \in I,\ \omega \in \Omega \backslash A_{j,k})
					\label{eq:thm_Ito_integral_on_M_c_loc_and_left_cont_locally_bounded_4}
				\end{align}
				が成り立つ.一方で定理\ref{thm:stopped_Ito_integral}より
				\begin{align}
					&\Norm{I_{M^{\sigma_{j+k}}}(X^{\sigma_j})^{\sigma_j} - I_{M^{\sigma_{j+k}}}(X^{\sigma_{j+k}})^{\sigma_j}}{\mathcal{M}_{2,c}}^2 \\
					&\qquad = \int_\Omega \int_I \left| X^{\sigma_j}(t,\omega) - X^{\sigma_{j+k}}(t,\omega) \right|^2 
						\defunc_{\left[0,\sigma_j(\omega)\right]}(t)\ \inprod<M^{\sigma_{j+k}}>(dt,\omega)\ \mu(d\omega)
				\end{align}
				が成り立つが,$\Omega \backslash E$上では$\sigma_j \leq \sigma_{j+k}$により
				$X^{\sigma_j}(t) \defunc_{\left[0,\sigma_j\right]}(t) = X^{\sigma_{j+k}}(t) \defunc_{\left[0,\sigma_j\right]}(t)$
				が満たされているから右辺の積分は0であり,或る$\mu$-零集合$B_{j,k}$が存在して
				\begin{align}
					I_{M^{\sigma_{j+k}}}(X^{\sigma_j})^{\sigma_j}_t(\omega) = I_{M^{\sigma_{j+k}}}(X^{\sigma_{j+k}})^{\sigma_j}_t(\omega)
					\quad (\forall t \in I,\ \omega \in \Omega \backslash B_{j,k})
					\label{eq:thm_Ito_integral_on_M_c_loc_and_left_cont_locally_bounded_5}
				\end{align}
				が成り立つ.(\refeq{eq:thm_Ito_integral_on_M_c_loc_and_left_cont_locally_bounded_4})と
				(\refeq{eq:thm_Ito_integral_on_M_c_loc_and_left_cont_locally_bounded_5})を併せれば
				\begin{align}
					I_{M^{\sigma_j}}(X^{\sigma_j})^{\sigma_j}_t(\omega) = I_{M^{\sigma_{j+k}}}(X^{\sigma_{j+k}})^{\sigma_j}_t(\omega)
					\quad \left( \forall t \in I,\ \omega \in \Omega \backslash \left( A_{j,k} \cup B_{j,k} \right) \right)
				\end{align}
				が従う.$j,k$は任意に選んでいたから,
				\begin{align}
					A \coloneqq \bigcup_{\substack{j,k \in \N_0\\j \leq k}} A_{j,k},
					\quad B \coloneqq \bigcup_{\substack{j,k \in \N_0\\j \leq k}} B_{j,k},
					\quad C \coloneqq A \cup B \cup E
				\end{align}
				により定める$C$は$\mu$-零集合である.
				任意の$\omega \in \Omega \backslash C$に対し或る$J = J(\omega)$が存在して
				$0 = \sigma_0(\omega) \leq \sigma_1(\omega) \leq \cdots \leq \sigma_J(\omega) = T$が満たされ,
				且つ任意の$j,k \in \N_0,\ j \leq k$に対し
				\begin{align}
					I_{M^{\sigma_j}}(X^{\sigma_j})^{\sigma_j}_t(\omega) = I_{M^{\sigma_k}}(X^{\sigma_k})^{\sigma_j}_t(\omega)
					\quad (\forall t \in I,\ \omega \in \Omega \backslash C)
				\end{align}
				も成り立つから,$\Omega \backslash C$上で$\lim_{j \to \infty} I_{M^{\sigma_j}}(X^{\sigma_j})^{\sigma_j}_t\ (\forall t \in I)$
				が確定する.
				\begin{align}
					I_M(X)(t,\omega) \coloneqq 
					\begin{cases}
						\lim\limits_{j \to \infty} I_{M^{\sigma_j}}(X^{\sigma_j})^{\sigma_j}(t,\omega) & (\omega \in \Omega \backslash C) \\
						0 & (\omega \in C)
					\end{cases}
					\quad (t \in I)
				\end{align}
				により$I_M(X)$を定めれば,$I_M(X)$は適合過程であり,
				且つ任意の$\omega \in \Omega$に対し$I \ni t \longmapsto I_M(X)_t(\omega)$は各点$t$で右連続かつ左極限を持つ.
				そして任意の$j \in \N_0$に対し
				\begin{align}
					I_M(X)^{\sigma_j}(t,\omega) = I_{M^{\sigma_j}}(X^{\sigma_j})^{\sigma_j}(t,\omega)
					\quad (\forall t \in I,\ \omega \in \Omega \backslash C)
				\end{align}
				が満たされる.
				\QED
		\end{description}
	\end{prf}
	
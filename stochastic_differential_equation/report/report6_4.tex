	$M \in \mathcal{M}_{2,c},\ X \in \semiLp{2}{I \times \Omega,\mathcal{P},\mu_M}$に対して
	定義した伊藤積分を更に拡張する.
	
	\begin{screen}
		\begin{dfn}[局所有界過程]
			$(\Omega,\mathcal{F},\mu)$上の確率過程$X$に対し或る$(\tau_j)_{j=0}^{\infty} \in \mathcal{T}$が存在して
			\begin{align}
				\sup{t \in I}{\Norm{X_{t \wedge \tau_j}}{\mathscr{L}^\infty}} < \infty
				\quad (j=0,1,\cdots)
			\end{align}
			が満たされているとき,$X$を局所有界過程(locally bounded process)という.
		\end{dfn}
	\end{screen}
	
	\begin{screen}
		\begin{thm}[局所マルチンゲールと左連続局所有界適合過程に対する伊藤積分]
			$X$を左連続且つ局所有界な適合過程,$M \in \mathcal{M}_{c,loc}$とする.
			このとき確率積分
			\begin{align}
				\int_0^t X_s\ dM_s \quad (t \in I)
			\end{align}
			が定義される.
		\end{thm}
	\end{screen}
	
	\begin{prf}\mbox{}
		\begin{description}
			\item[第一段] $\sup{t \in I}{\Norm{X_t}{\mathscr{L}^\infty}} < \infty$かつ$\Norm{\inprod<M>_T}{\semiLp{\infty}{\mu}} < \infty$
				なら$X \in \semiLp{2}{I \times \Omega,\mathcal{P},\mu_M}$が成り立つことを示す.すなわち,このとき$I_M(X)$が定義される.
				$X$は左連続であるから定理\ref{thm:left_continuous_adapted_then_predictable}により可測$\mathcal{P}/\borel{\R}$である.
				また定理\ref{thm:quadratic_variation_bounded_then_M_2c}より$M \in \mathcal{M}_{2,c}$も満たされている.
				従って$\mathcal{P}/\borel{\R}$-可測関数の関数類を$\equiv{\cdot}{\Lp{0}{\mu_M}}$と表せば,
				(\refeq{eq:lem_properties_of_simple_predictable_processes_0})より
				\begin{align}
					\Norm{\equiv{X}{\Lp{0}{\mu_M}} - \equiv{X^n}{\Lp{0}{\mu_M}}}{\Lp{2}{\mu_M}}^2
					= \int_\Omega \int_I \left| X(t,\omega) - X^n(t,\omega) \right|^2\ \inprod<M>(dt,\omega)\ \mu(d\omega)
				\end{align}
				が成り立ち,Lebesgueの収束定理より右辺は$n \longrightarrow \infty$で0に収束する.
				補題\ref{lem:properties_of_simple_predictable_processes}より$\mathfrak{S}$は
				$\Lp{2}{I \times \Omega,\mathcal{P},\mu_M}$で稠密であるから
				$\equiv{X}{\Lp{0}{\mu_M}} \in \Lp{2}{I \times \Omega,\mathcal{P},\mu_M}$が従う.
				
			\item[第二段]
				前段の仮定を外す.$X$が局所有界過程であるから,或る$(\tau_j)_{j=0}^{\infty} \in \mathcal{T}$が存在して
				\begin{align}
					\sup{t \in I}{\Norm{X_{t \wedge \tau_j}}{\mathscr{L}^\infty}} < \infty
					\quad (j=0,1,\cdots)
				\end{align}
				が満たされる.また
				\begin{align}
					\hat{\tau}_j(\omega) \coloneqq
					\inf{}{\Set{t \in I}{\inprod<M>_t(\omega) \geq j}} \wedge T\ \footnotemark
					\quad (\forall \omega \in \Omega,\ j=0,1,\cdots)
				\end{align}
				\footnotetext{
					$\Set{t \in I}{|\inprod<M>_t(\omega)| \geq j} = \emptyset$の場合$\sigma_j(\omega) = T$とする.
				}
				として$\left( \hat{\tau}_j \right)_{j=0}^{\infty} \in \mathcal{T}$を定め
				\begin{align}
					\sigma_j \coloneqq \tau_j \wedge \hat{\tau}_j
					\quad (j=0,1,\cdots)
				\end{align}
				とおけば,$(\sigma_j)_{j=0}^{\infty} \in \mathcal{T}$且つ
				\begin{align}
					\Norm{X_{t \wedge \sigma_j}}{\mathscr{L}^\infty} \leq \Norm{X_{t \wedge \tau_j}}{\mathscr{L}^\infty},
					\quad \Norm{\inprod<M>_{t \wedge \sigma_j}}{\mathscr{L}^\infty} \leq j
					\quad (\forall t \in I,\ j=0,1,\cdots)
				\end{align}
				が成り立つ.従って前段の結果より$I_{M^{\sigma_j}}(X^{\sigma_j})\ (j=0,1,\cdots)$が定義される.
				
			\item[第三段]
				次の極限が$\mu$-a.s.に確定することを示す:
				\begin{align}
					\lim_{j \to \infty} \lim_{k \to \infty} I_{M^{\sigma_k}}(X^{\sigma_k})_{t \wedge \sigma_j}.
				\end{align}
				$(\sigma_j)_{j=0}^{\infty} \in \mathcal{T}$であるから,或る零集合$E$が存在して,$\omega \in \Omega \backslash E$なら
				$0 = \sigma_0(\omega) \leq \sigma_1(\omega) \leq \cdots$且つ,或る$J = J(\omega) \in \N$が存在して
				$\sigma_j(\omega) = T\ (j \geq J)$となる.今,任意に$j \in \N_0,k \in \N$を取り固定する
				\footnote{
					$\N_0 = \N \cup \{0\}.$
				}
				.任意の$Y \in \mathcal{S}$に対し,$Y$が時点$0=t_0 < t_1 < \cdots < t_n = T$と
				$F \in \semiLp{\infty}{\Omega,\mathcal{F}_0,\mu},F_i \in \semiLp{\infty}{\Omega,\mathcal{F}_{t_i},\mu}\ (i=0,1,\cdots,n-1)$によって
				\begin{align}
					Y_t = F \defunc_{\{0\}}(t) + \sum_{i=0}^{n-1} F_i \defunc_{\left(t_i,t_{i+1}\right]}(t)
					\quad (t \in I)
				\end{align}
				と表現されているとき,$\mathcal{M}_{2,c}$上の確率積分の定義より或る零集合$E_j$が存在して
				\begin{align}
					I_{M^{\sigma_j}}(Y)_t(\omega) = \sum_{i=0}^{n-1} F_i(\omega) \left( M^{\sigma_j}_{t \wedge t_{i+1}}(\omega) - M^{\sigma_j}_{t \wedge t_i}(\omega) \right)
					\quad (\forall t \in I,\ \omega \in \Omega \backslash E_j)
				\end{align}
				が成り立つ.特に両辺を$\sigma_j$で停めても等号は保たれ
				\begin{align}
					I_{M^{\sigma_j}}(Y)_{t \wedge \sigma_j(\omega)}(\omega) = \sum_{i=0}^{n-1} F_i(\omega) \left( M^{\sigma_j}_{t \wedge t_{i+1}}(\omega) - M^{\sigma_j}_{t \wedge t_i}(\omega) \right)
					\quad (\forall t \in I,\ \omega \in \Omega \backslash E_j)
				\end{align}
				を得る.一方$\sigma_{j+k}$についても或る零集合$E_{j+k}$が存在して
				\begin{align}
					I_{M^{\sigma_{j+k}}}(Y)_t(\omega) = \sum_{i=0}^{n-1} F_i(\omega) \left( M^{\sigma_{j+k}}_{t \wedge t_{i+1}}(\omega) - M^{\sigma_{j+k}}_{t \wedge t_i}(\omega) \right)
					\quad (\forall t \in I,\ \omega \in \Omega \backslash E_{j+k})
				\end{align}
				が成り立ち,特に$\omega \in \Omega \backslash E$については$\sigma_j(\omega) \leq \sigma_{j+k}(\omega)$が満たされるから
				\begin{align}
					I_{M^{\sigma_{j+k}}}(Y)_{t \wedge \sigma_j(\omega)}(\omega) = \sum_{i=0}^{n-1} F_i(\omega) \left( M^{\sigma_j}_{t \wedge t_{i+1}}(\omega) - M^{\sigma_j}_{t \wedge t_i}(\omega) \right)
					\quad (\forall t \in I,\ \omega \in \Omega \backslash (E_{j+k} \cup E))
				\end{align}
				が従う.ゆえに
				\begin{align}
					I_{M^{\sigma_j}}(Y)_{t \wedge \sigma_j(\omega)}(\omega) = I_{M^{\sigma_{j+k}}}(Y)_{t \wedge \sigma_j(\omega)}(\omega)
					\quad (\forall t \in I,\ \omega \in \Omega \backslash (E_j \cup E_{j+k} \cup E))
				\end{align}
				が得られる.
		\end{description}
		\QED
	\end{prf}
	
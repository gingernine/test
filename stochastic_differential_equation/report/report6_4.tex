	$M \in \mathcal{M}_{2,c},\ X \in \semiLp{2}{\mathcal{P},\mu_M}$に対して
	定義した伊藤積分を更に拡張する.
	
	\begin{screen}
		\begin{dfn}[局所有界過程]
			$(\Omega,\mathcal{F},\mu)$上の確率過程$X$に対し或る$(\tau_j)_{j=0}^{\infty} \in \mathcal{T}$が存在して
			\begin{align}
				\sup{t \in I}{\Norm{X_{t \wedge \tau_j}}{\mathscr{L}^\infty}} < \infty
				\quad (j=0,1,\cdots)
			\end{align}
			が満たされているとき,$X$を局所有界過程(locally bounded process)という.
		\end{dfn}
	\end{screen}

	\begin{screen}
		\begin{thm}[局所マルチンゲールと左連続局所有界適合過程に対する伊藤積分]
			$X$を左連続且つ局所有界な適合過程,$M \in \mathcal{M}_{c,loc}$とする.
			このとき確率積分
			\begin{align}
				\int_0^t X_s\ dM_s \quad (t \in I)
			\end{align}
			が定義される.
			\label{thm:Ito_integral_on_M_c_loc_and_left_cont_locally_bounded}
		\end{thm}
	\end{screen}
	
\newpage
	\begin{itembox}[l]{レポート問題6}
	$I = [a,b]$を$I = [0,T]$に替えて示す.
	局所有界な左連続適合過程$X$と$M \in \mathcal{M}_{c,loc}$に対して,
	$X$が連続適合過程である時と同様に確率積分
	\begin{align}
		\int_0^t X_s\ dM_s,
		\quad t \in I
	\end{align}
	が定義できることを示せ.
	\end{itembox}
	
	\begin{prf}\mbox{}
		\begin{description}
			\item[第一段] $\sup{t \in I}{\Norm{X_t}{\semiLp{\infty}{\mu}}} < \infty$かつ$\Norm{\inprod<M>_T}{\semiLp{\infty}{\mu}} < \infty$
				が満たされている場合,$X$は可測$\mathcal{P}/\borel{\R}$であり$X \in \semiLp{2}{\mathcal{P},\mu_M}$を満たすことを示す.実際,%すなわち,このとき$I_M(X)$が定義される.
				\begin{align}
					X^n_t \coloneqq X_0 \defunc_{\{0\}} + \sum_{k=0}^{2^n-1} X_{\frac{kT}{2^n}} \defunc_{\left( \frac{kT}{2^n}, \frac{(k+1)T}{2^n}\right]}(t)
					\quad (\forall t \in I)
					\label{eq:thm_left_continuous_adapted_then_predictable}
				\end{align}
				として$\left( X^n \right)_{n=1}^{\infty}$を構成すれば
				$\sup{t \in I}{\Norm{X_t}{\mathscr{L}^\infty}} < \infty$より
				$\left( X^n \right)_{n=1}^{\infty} \subset \mathcal{S}$が満たされるから,
				$X^n$は全て可測$\mathcal{P}/\borel{\R}$である.
				また$I \ni t \longmapsto X_t(\omega)\ (\forall \omega \in \Omega)$の左連続性から
				\begin{align}
					\left| X^n(t,\omega) - X(t,\omega) \right| \longrightarrow 0
					\quad (n \longrightarrow \infty,\ \forall (t,\omega) \in I \times \Omega)
				\end{align}
				が従い,各点収束先も可測性は保たれるから$X$は可測$\mathcal{P}/\borel{\R}$である.
				そして$X$の有界性及び$\mu_M$の有限性により,Lebesgueの収束定理を適用して
				\begin{align}
					\Norm{X - X^n}{\semiLp{2}{\mu_M}} \longrightarrow 0 
					\quad (n \longrightarrow \infty)
				\end{align}
				が得られ,$\mathcal{S}$は
				$\semiLp{2}{\mathcal{P},\mu_M}$で稠密であるから
				$X \in \semiLp{2}{\mathcal{P},\mu_M}$が従う.
				また定理\ref{thm:quadratic_variation_bounded_then_M_2c}より$M \in \mathcal{M}_{2,c}$も満たされている.
				ゆえに確率積分$I_M(X)$が定義される.
				
			\item[第二段]
				前段の仮定を外す.$X$が局所有界過程であるから,或る$(\tau_j)_{j=0}^{\infty} \in \mathcal{T}$が存在して
				\begin{align}
					\sup{t \in I}{\Norm{X_{t \wedge \tau_j}}{\semiLp{\infty}{\mu}}} < \infty
					\quad (j=0,1,\cdots)
				\end{align}
				が満たされる.また
				\begin{align}
					\hat{\tau}_j(\omega) \coloneqq
					\inf{}{\Set{t \in I}{\inprod<M>_t(\omega) \geq j}} \wedge T\ \footnotemark
					\quad (\forall \omega \in \Omega,\ j=0,1,\cdots)
				\end{align}
				\footnotetext{
					$\Set{t \in I}{|\inprod<M>_t(\omega)| \geq j} = \emptyset$の場合$\sigma_j(\omega) = T$とする.
				}
				として$\left( \hat{\tau}_j \right)_{j=0}^{\infty} \in \mathcal{T}$を定め
				\begin{align}
					\sigma_j \coloneqq \tau_j \wedge \hat{\tau}_j
					\quad (j=0,1,\cdots)
					\label{eq:Ito-Integral_on_M_c_loc_stopping_time}
				\end{align}
				とおけば,$(\sigma_j)_{j=0}^{\infty} \in \mathcal{T}$且つ
				\begin{align}
					\Norm{X^{\sigma_j}_t}{\semiLp{\infty}{\mu}} \leq \sup{s \in I}{\Norm{X^{\tau_j}_s}{\semiLp{\infty}{\mu}}},
					\quad \Norm{\inprod<M>^{\sigma_j}_t}{\semiLp{\infty}{\mu}} \leq j
					\quad (\forall t \in I,\ j=0,1,\cdots)
				\end{align}
				が成り立つ.従って前段の結果より$I_{M^{\sigma_j}}(X^{\sigma_j})\ (j=0,1,\cdots)$が定義される.
				
			\item[第三段]
				次の極限が$\mu$-a.s.に確定することを示す:
				\begin{align}
					\lim_{j \to \infty} I_{M^{\sigma_j}}(X^{\sigma_j})_{t \wedge \sigma_j}
					\quad (\forall t \in I).
				\end{align}
				$(\sigma_j)_{j=0}^{\infty} \in \mathcal{T}$より或る$\mu$-零集合$E$が存在して,$\omega \in \Omega \backslash E$なら
				$0 = \sigma_0(\omega) \leq \sigma_1(\omega) \leq \cdots$且つ,或る$J = J(\omega) \in \N$が存在して
				$\sigma_j(\omega) = T\ (j \geq J)$が満たされる.今,$j \leq k$を満たす$j,k \in \N_0$を任意に取り固定する.
				任意に$Y \in \mathcal{S}$を取り,$Y$が時点$0=t_0 < t_1 < \cdots < t_n = T$と
				$F \in \semiLp{\infty}{\Omega,\mathcal{F}_0,\mu},F_i \in \semiLp{\infty}{\Omega,\mathcal{F}_{t_i},\mu}\ (i=0,1,\cdots,n-1)$によって
				\begin{align}
					Y_t = F \defunc_{\{0\}}(t) + \sum_{i=0}^{n-1} F_i \defunc_{\left(t_i,t_{i+1}\right]}(t)
					\quad (t \in I)
				\end{align}
				と表現されているとすれば,$\mathcal{M}_{2,c}$上の確率積分の定義より
				\begin{align}
					I_{M^{\sigma_j}}(Y)_t = \sum_{i=0}^{n-1} F_i \left( M^{\sigma_j}_{t \wedge t_{i+1}} - M^{\sigma_j}_{t \wedge t_i} \right)
					\quad (\forall t \in I,\ \mbox{$\mu$-a.s.})
				\end{align}
				が成り立つ.特に両辺を$\sigma_j$で停めても等号は保たれ
				\begin{align}
					I_{M^{\sigma_j}}(Y)^{\sigma_j}_t = \sum_{i=0}^{n-1} F_i \left( M^{\sigma_j}_{t \wedge t_{i+1}} - M^{\sigma_j}_{t \wedge t_i} \right)
					\quad (\forall t \in I,\ \mbox{$\mu$-a.s.})
					\label{eq:thm_Ito_integral_on_M_c_loc_and_left_cont_locally_bounded_1}
				\end{align}
				を得る.$\sigma_{j+k}$についても同様に
				\begin{align}
					I_{M^{\sigma_{j+k}}}(Y)_t = \sum_{i=0}^{n-1} F_i \left( M^{\sigma_{j+k}}_{t \wedge t_{i+1}} - M^{\sigma_{j+k}}_{t \wedge t_i} \right)
					\quad (\forall t \in I,\ \mbox{$\mu$-a.s.})
				\end{align}
				が成り立ち,特に$\Omega \backslash E$上では$\sigma_j \leq \sigma_{j+k}$が満たされるから,両辺を$\sigma_j$で停めて
				\begin{align}
					I_{M^{\sigma_{j+k}}}(Y)^{\sigma_j}_t = \sum_{i=0}^{n-1} F_i \left( M^{\sigma_j}_{t \wedge t_{i+1}} - M^{\sigma_j}_{t \wedge t_i} \right)
					\quad (\forall t \in I,\ \mbox{$\mu$-a.s.})
					\label{eq:thm_Ito_integral_on_M_c_loc_and_left_cont_locally_bounded_2}
				\end{align}
				が得られ,(\refeq{eq:thm_Ito_integral_on_M_c_loc_and_left_cont_locally_bounded_1})と
				(\refeq{eq:thm_Ito_integral_on_M_c_loc_and_left_cont_locally_bounded_2})を併せれば
				\begin{align}
					I_{M^{\sigma_j}}(Y)^{\sigma_j}_t = I_{M^{\sigma_{j+k}}}(Y)^{\sigma_j}_t
					\quad (\forall t \in I,\ \mbox{$\mu$-a.s.})
				\end{align}
				が従う.一般の$Y \in \semiLp{2}{\mathcal{P},\mu_{M^{\sigma_j}}} \cap \semiLp{2}{\mathcal{P},\mu_{M^{\sigma_{j+k}}}}$に対しては
				或る$(Y_n)_{n=1}^{\infty} \in \mathcal{S}$が存在して
				\begin{align}
					&\int_{I \times \Omega} \left| Y(t,\omega) - Y_n(t,\omega) \right|^2\ \mu_{M^{\sigma_{j+k}}}(dtd\omega) \\
					&\qquad = \int_{\Omega} \int_I \left| Y(t,\omega) - Y_n(t,\omega) \right|^2\ \inprod<M^{\sigma_{j+k}}>(dt,\omega)\ \mu(d\omega)
					\longrightarrow 0
					\quad (n \longrightarrow \infty)
				\end{align}
				を満たすから,$\sigma_j \leq \sigma_{j+k}\ (\mu$-a.s.)により$\Norm{Y - Y_n}{\semiLp{2}{\mu_{M^{\sigma_j}}}} \longrightarrow 0$も従い
				\begin{align}
					&\Norm{I_{M^{\sigma_j}}(Y)^{\sigma_j} - I_{M^{\sigma_{j+k}}}(Y)^{\sigma_j}}{\mathcal{M}_{2,c}}\ \footnotemark\\
					&\qquad \leq \Norm{I_{M^{\sigma_j}}(Y)^{\sigma_j} - I_{M^{\sigma_j}}(Y_n)^{\sigma_j}}{\mathcal{M}_{2,c}}
						+ \Norm{I_{M^{\sigma_{j+k}}}(Y_n)^{\sigma_j} - I_{M^{\sigma_{j+k}}}(Y)^{\sigma_j}}{\mathcal{M}_{2,c}} \\
					&\qquad \leq \Norm{Y - Y_n}{\semiLp{2}{\mu_{M^{\sigma_j}}}} + \Norm{Y - Y_n}{\semiLp{2}{\mu_{M^{\sigma_{j+k}}}}}
					\quad \longrightarrow 0 \quad (n \longrightarrow \infty)
				\end{align}
				が成り立つ.
				\footnotetext{
					$\Norm{Y}{\mathcal{M}_{2,c}}^2 \coloneqq \int_\Omega |Y_T(\omega)|^2\ \mu(d\omega)
						\quad (\forall Y \in \mathcal{M}_{2,c}).$
				}
				以上より
				\begin{align}
					I_{M^{\sigma_j}}(Y)^{\sigma_j}_t = I_{M^{\sigma_{j+k}}}(Y)^{\sigma_j}_t
					\quad (\forall t \in I,\ \mbox{$\mu$-a.s.})
					\label{eq:thm_Ito_integral_on_M_c_loc_and_left_cont_locally_bounded_3}
				\end{align}
				を得る.特に$X^{\sigma_j}$は$\Norm{\cdot}{\semiLp{\infty}{\mu}}$について一様有界であるから
				第一段より$X^{\sigma_j} \in \semiLp{2}{\mathcal{P},\mu_{M^{\sigma_j}}} \cap \semiLp{2}{\mathcal{P},\mu_{M^{\sigma_{j+k}}}}$
				が満たされ,(\refeq{eq:thm_Ito_integral_on_M_c_loc_and_left_cont_locally_bounded_3})より
				或る$\mu$-零集合$A_{j,k}$が存在して
				\begin{align}
					I_{M^{\sigma_j}}(X^{\sigma_j})^{\sigma_j}_t(\omega) = I_{M^{\sigma_{j+k}}}(X^{\sigma_j})^{\sigma_j}_t(\omega)
					\quad (\forall t \in I,\ \omega \in \Omega \backslash A_{j,k})
					\label{eq:thm_Ito_integral_on_M_c_loc_and_left_cont_locally_bounded_4}
				\end{align}
				が成り立つ.一方で定理\ref{thm:stopped_Ito_integral}より
				\begin{align}
					&\Norm{I_{M^{\sigma_{j+k}}}(X^{\sigma_j})^{\sigma_j} - I_{M^{\sigma_{j+k}}}(X^{\sigma_{j+k}})^{\sigma_j}}{\mathcal{M}_{2,c}}^2 \\
					&\qquad = \int_\Omega \int_I \left| X^{\sigma_j}(t,\omega) - X^{\sigma_{j+k}}(t,\omega) \right|^2 
						\defunc_{\left[0,\sigma_j(\omega)\right]}(t)\ \inprod<M^{\sigma_{j+k}}>(dt,\omega)\ \mu(d\omega)
				\end{align}
				が成り立つが,$\Omega \backslash E$上では$\sigma_j \leq \sigma_{j+k}$により
				$X^{\sigma_j}(t) \defunc_{\left[0,\sigma_j\right]}(t) = X^{\sigma_{j+k}}(t) \defunc_{\left[0,\sigma_j\right]}(t)$
				が満たされているから右辺の積分は0であり,或る$\mu$-零集合$B_{j,k}$が存在して
				\begin{align}
					I_{M^{\sigma_{j+k}}}(X^{\sigma_j})^{\sigma_j}_t(\omega) = I_{M^{\sigma_{j+k}}}(X^{\sigma_{j+k}})^{\sigma_j}_t(\omega)
					\quad (\forall t \in I,\ \omega \in \Omega \backslash B_{j,k})
					\label{eq:thm_Ito_integral_on_M_c_loc_and_left_cont_locally_bounded_5}
				\end{align}
				が成り立つ.(\refeq{eq:thm_Ito_integral_on_M_c_loc_and_left_cont_locally_bounded_4})と
				(\refeq{eq:thm_Ito_integral_on_M_c_loc_and_left_cont_locally_bounded_5})を併せれば
				\begin{align}
					I_{M^{\sigma_j}}(X^{\sigma_j})^{\sigma_j}_t(\omega) = I_{M^{\sigma_{j+k}}}(X^{\sigma_{j+k}})^{\sigma_j}_t(\omega)
					\quad \left( \forall t \in I,\ \omega \in \Omega \backslash \left( A_{j,k} \cup B_{j,k} \right) \right)
				\end{align}
				が従う.$j,k$は任意に選んでいたから,
				\begin{align}
					A \coloneqq \bigcup_{\substack{j,k \in \N_0\\j \leq k}} A_{j,k},
					\quad B \coloneqq \bigcup_{\substack{j,k \in \N_0\\j \leq k}} B_{j,k},
					\quad C \coloneqq A \cup B \cup E
				\end{align}
				により定める$C$は$\mu$-零集合である.
				任意の$\omega \in \Omega \backslash C$に対し或る$J = J(\omega)$が存在して
				$0 = \sigma_0(\omega) \leq \sigma_1(\omega) \leq \cdots \leq \sigma_J(\omega) = T$が満たされ,
				且つ任意の$j,k \in \N_0,\ j \leq k$に対し
				\begin{align}
					I_{M^{\sigma_j}}(X^{\sigma_j})^{\sigma_j}_t(\omega) = I_{M^{\sigma_k}}(X^{\sigma_k})^{\sigma_j}_t(\omega)
					\quad (\forall t \in I,\ \omega \in \Omega \backslash C)
				\end{align}
				も成り立つから,$\Omega \backslash C$上で$\lim_{j \to \infty} I_{M^{\sigma_j}}(X^{\sigma_j})^{\sigma_j}_t\ (\forall t \in I)$
				が確定する.
				\begin{align}
					N(t,\omega) \coloneqq 
					\begin{cases}
						\lim\limits_{j \to \infty} I_{M^{\sigma_j}}(X^{\sigma_j})^{\sigma_j}(t,\omega) & (\omega \in \Omega \backslash C) \\
						0 & (\omega \in C)
					\end{cases}
					\quad (t \in I)
					\label{eq:Ito-Integral_on_M_c_loc_limit}
				\end{align}
				により$N$を定め,これを確率積分$\int_0^t X_s\ dM_s$と表す.
				\QED
		\end{description}
	\end{prf}
	
	\newpage
	\begin{itembox}[l]{レポート問題7}
		$I = [a,b]$を$I = [0,T]$に替えて示す.
		$X,Y$を局所有界な左連続適合過程とし,$M \in \mathcal{M}_{c,loc}$とする.
		\begin{align}
			N_t = \int_0^t X_s\ dM_s,
			\quad t \in I
		\end{align}
		とする時,$N \in \mathcal{M}_{c,loc}$であり,
		\begin{align}
			\int_0^t Y_s\ dN_s = \int_0^t Y_s X_s\ dM_s,
			\quad t \in I
			\label{eq:Ito-Integral_on_M_c_loc_report_7_0}
		\end{align}
		であることを示せ.
	\end{itembox}

	\begin{prf}[レポート問題6の続き] \mbox{}
		\begin{description}
			\item[第一段] $N \in \mathcal{M}_{c,loc}$を示す.
				(\refeq{eq:Ito-Integral_on_M_c_loc_limit})により定めた$N$は適合過程であり,
				且つ任意の$\omega \in \Omega$に対し$I \ni t \longmapsto N_t(\omega)$は各点$t$で右連続かつ左極限を持つ.
				特に$\mu$-a.s.に$I \ni t \longmapsto N_t$は連続である.
				そして任意の$\sigma_j\ (j \in \N_0)$について次が満たされる:
				\begin{align}
					N^{\sigma_j}(t,\omega) = I_{M^{\sigma_j}}(X^{\sigma_j})^{\sigma_j}(t,\omega)
					\quad (\forall t \in I,\ \omega \in \Omega \backslash C).
					\label{eq:Ito-Integral_on_M_c_loc_localize}
				\end{align}
				\begin{description}
					\item[(\refeq{eq:Ito-Integral_on_M_c_loc_localize})について]
						定め方(\refeq{eq:Ito-Integral_on_M_c_loc_limit})による.
					\item[適合性について]
						$I_{M^{\sigma_j}}(X^{\sigma_j})^{\sigma_j}$は$\mathcal{M}_{2,c}$に属するから適合過程である.
						その各点収束極限で$N$が定義されるから$N$の適合性が出る.
					\item[(右)連続性について] $I_{M^{\sigma_j}}(X^{\sigma_j})^{\sigma_j} \in \mathcal{M}_{2,c}$であることと$N$の定め方による.
					\item[左極限について] $I_{M^{\sigma_j}}(X^{\sigma_j})^{\sigma_j} \in \mathcal{M}_{2,c}$であることと$N$の定め方による.
				\end{description}
				従って
				\begin{align}
					\tau_j \coloneqq \inf{}{\Set{t \in I}{|N_t| \geq j}} \wedge T
				\end{align}
				により$(\tau_j)_{j=0}^{\infty} \in \mathcal{T}$を定めることができる.
				\begin{align}
					\hat{\sigma}_j \coloneqq \sigma_j \wedge \tau_j
				\end{align}
				として$\left( \hat{\sigma}_j \right)_{j=0}^{\infty} \in \mathcal{T}$を構成すれば任意の$t \in I$に対して
				\begin{align}
					\left| N^{\hat{\sigma}_j}_t \right| \leq j \quad \mbox{$\mu$-a.s.}
				\end{align}
				が満たされ,且つ$\mathcal{M}_{2,c}$を停めた過程であるから$N^{\hat{\sigma}_j} \in \mathcal{M}_{b,c}$となり$N \in \mathcal{M}_{c,loc}$が従う.
			
			\item[第二段]
				(\refeq{eq:Ito-Integral_on_M_c_loc_stopping_time})で構成した通り,或る停止時刻の列
				$(\tau_j)_{j=0}^{\infty},(\hat{\tau}_j)_{j=0}^{\infty},(\sigma_j)_{j=0}^{\infty} \in \mathcal{T}$が存在して,
				\begin{align}
					&Y^{\tau_j},X^{\hat{\tau}_j},(YX)^{\sigma_j} \in \semiLp{2}{\mathcal{P},\mu_{M}}
					\quad (\forall M \in \mathcal{M}_{2,c},\ j=0,1,2,\cdots), \\
					&\sup{t \in I}{\Norm{Y^{\tau_j}_t}{\semiLp{\infty}{\mu}}} < \infty,
					\quad \sup{t \in I}{\Norm{X^{\hat{\tau}_j}_t}{\semiLp{\infty}{\mu}}} < \infty,
					\quad \sup{t \in I}{\Norm{(YX)^{\sigma_j}_t}{\semiLp{\infty}{\mu}}} < \infty
					\label{eq:Ito-Integral_on_M_c_loc_report_7_6}
				\end{align}
				かつ(\refeq{eq:Ito-Integral_on_M_c_loc_limit})の収束の意味で
				\begin{align}
					I_{N^{\tau_j}}(Y^{\tau_j})^{\tau_j} \longrightarrow \int_0^t Y_s\ dN_s,
					\quad 
					I_{M^{\sigma_j}}(Y^{\sigma_j}X^{\sigma_j})^{\sigma_j}_t \longrightarrow \int_0^t Y_s X_s\ dM_s
					\quad (t \in I)
					\label{eq:Ito-Integral_on_M_c_loc_report_7_1}
				\end{align}
				を満たす.よって
				\begin{align}
					\upsilon_j \coloneqq \sigma_j \wedge \tau_j \wedge \hat{\tau}_j
					\quad (j=0,1,2,\cdots)
				\end{align}
				により定める$(\upsilon_j)_{j=0}^{\infty} \in \mathcal{T}$に対して
				\begin{align}
					I_{N^{\upsilon_j}} (Y^{\upsilon_j})^{\upsilon_j}_t
					= I_{M^{\upsilon_j}} (Y^{\upsilon_j}X^{\upsilon_j})^{\upsilon_j}_t
					\quad (\forall t \in I,\ \mbox{$\mu$-a.s.},\ j=0,1,2,\cdots)
					\label{eq:Ito-Integral_on_M_c_loc_report_7_2}
				\end{align}
				が満たされることを示せば,(\refeq{eq:Ito-Integral_on_M_c_loc_report_7_1})より(\refeq{eq:Ito-Integral_on_M_c_loc_report_7_0})が従う.
				今,任意に$j \in \N_0$を取り固定する.(\refeq{eq:thm_left_continuous_adapted_then_predictable})と同様に
				$X^{\upsilon_j}$に各点収束する$X^j_n \in \mathcal{S}\ (n=1,2,\cdots)$を構成し,同様に$Y^{\upsilon_j}$に対しても
				$Y^j_n \in \mathcal{S}\ (n=1,2,\cdots)$を取れば,
				Lebesgueの収束定理及びH\Ddot{o}lderの不等式により
				\begin{align}
					&\Norm{Y^{\upsilon_j} - Y^j_n}{\semiLp{2}{\mu_{N^{\upsilon_j}}}} \longrightarrow 0, 
					\quad \Norm{X^{\upsilon_j} - X^j_n}{\semiLp{2}{\mu_{M^{\upsilon_j}}}} \longrightarrow 0, \label{eq:Ito-Integral_on_M_c_loc_report_7_5} \\
					&\Norm{Y^{\upsilon_j}X^{\upsilon_j} - Y^j_n X^j_n}{\semiLp{2}{\mu_{M^{\upsilon_j}}}} \longrightarrow 0, 
				\end{align}
				が成立する.従って,もし全ての$n \in \N$に対して
				\begin{align}
					\Norm{I_{N^{\upsilon_j}} (Y_n^j)^{\upsilon_j} - I_{M^{\upsilon_j}} (Y^j_n X^j_n)^{\upsilon_j}}{\mathcal{M}_{2,c}}
					= 0
					\label{eq:Ito-Integral_on_M_c_loc_report_7_3}
				\end{align}
				が満たされているなら,伊藤積分の線型等長性より
				\begin{align}
					&\Norm{I_{N^{\upsilon_j}} (Y^{\upsilon_j})^{\upsilon_j} - I_{M^{\upsilon_j}} (Y^{\upsilon_j}X^{\upsilon_j})^{\upsilon_j}}{\mathcal{M}_{2,c}} \\
					&\qquad \leq \Norm{I_{N^{\upsilon_j}} (Y^{\upsilon_j})^{\upsilon_j} - I_{N^{\upsilon_j}} (Y^j_n)^{\upsilon_j}}{\mathcal{M}_{2,c}}
						+ \Norm{I_{M^{\upsilon_j}} (Y_n^j X_n^j)^{\upsilon_j} - I_{M^{\upsilon_j}} (Y^{\upsilon_j}X^{\upsilon_j})^{\upsilon_j}}{\mathcal{M}_{2,c}} \\
					&\qquad \longrightarrow 0 \quad (n \longrightarrow \infty)
				\end{align}
				が従い(\refeq{eq:Ito-Integral_on_M_c_loc_report_7_2})を得る.
			
			\item[第三段]
				任意に$n \in \N$を取り固定し,(\refeq{eq:Ito-Integral_on_M_c_loc_report_7_3})を示す.
				\begin{align}
					Y_n \coloneqq Y_n^j,
					\quad X_n \coloneqq X_n^j,
					\quad N \coloneqq N^{\upsilon_j},
					\quad M \coloneqq M^{\upsilon_j}
				\end{align}
				と表示しなおし,
				\begin{align}
					N^n_t \coloneqq \int_0^t X_n(s)\ d M_s
					\quad (t \in I)
				\end{align}
				とおく.また$F^n \coloneqq X^{\upsilon_j}_0,\ G^n \coloneqq Y^{\upsilon_j}_0,\ 
				F^n_k \coloneqq X^{\upsilon_j}_{kT/2^n},\ G^n_k \coloneqq Y^{\upsilon_j}_{kT/2^n},\ t_k \coloneqq kT/2^n$とおけば
				\begin{align}
					X_n = F^n \defunc_{\{0\}} + \sum_{k=0}^{2^n-1} F^n_k \defunc_{\left( t_k, t_{k+1} \right]},
					\quad Y_n = G^n \defunc_{\{0\}} + \sum_{k=0}^{2^n-1} G^n_k \defunc_{\left( t_k, t_{k+1} \right]}
				\end{align}
				と表現でき,
				\begin{align}
					&I_{N} (Y_n)_t = \sum_{k=0}^{2^n-1} G^n_k \left( N_{t \wedge t_{k+1}} - N_{t \wedge t_k} \right) \quad (\forall t \in I,\ \mbox{$\mu$-a.s.}), \\
					&I_{N^n} (Y_n)_t = \sum_{k=0}^{2^n-1} G^n_k \left( N^n_{t \wedge t_{k+1}} - N^n_{t \wedge t_k} \right) \quad (\forall t \in I,\ \mbox{$\mu$-a.s.}), \\
					&I_{M} (Y_n X_n)_t = \sum_{k=0}^{2^n-1} F^n_k G^n_k \left( M_{t \wedge t_{k+1}} - M_{t \wedge t_k} \right) \quad (\forall t \in I,\ \mbox{$\mu$-a.s.})
				\end{align}
				と書ける.特に
				\begin{align}
					N^n_t = \sum_{k=0}^{2^n-1} F^n_k \left( M_{t \wedge t_{k+1}} - M_{t \wedge t_k} \right) \quad (\forall t \in I,\ \mbox{$\mu$-a.s.})
				\end{align}
				が成り立っているから
				\begin{align}
					I_{N^n} (Y_n)_t = I_{M} (Y_n X_n)_t \quad (\forall t \in I,\ \mbox{$\mu$-a.s.})
					\label{eq:Ito-Integral_on_M_c_loc_report_7_4}
				\end{align}
				が従う.一方で
				\begin{align}
					I_{N} (Y_n)_T - I_{N^n} (Y_n)_T
					= \sum_{k=0}^{2^n-1} G^n_k \left( N_{t_{k+1}} - N^n_{t_{k+1}} \right) -  \sum_{k=0}^{2^n-1} G^n_k \left( N_{t_k} - N^n_{t_k} \right) 
					\quad \mbox{$\mu$-a.s.}
				\end{align}
				が成り立ち,また(\refeq{eq:Ito-Integral_on_M_c_loc_report_7_6})と$\upsilon_j$の構成法より
				$a \coloneqq \sup{t \in I}{\Norm{Y^{\upsilon_j}_t}{\semiLp{\infty}{\mu}}} < \infty$が満たされているから,
				$G^n_k\ (\forall k,n)$は$\mu$-a.s.に$a$で抑えられる.従って伊藤積分の線型等長性より
				\begin{align}
					\Norm{I_{N} (Y_n) - I_{N^n} (Y_n)}{\mathcal{M}_{2,c}}
					&\leq \sum_{k=0}^{2^n-1} \left\{ a\Norm{N_{t_{k+1}} - N^n_{t_{k+1}}}{\semiLp{2}{\mu}} + a\Norm{N_{t_k} - N^n_{t_k}}{\semiLp{2}{\mu}} \right\} \\
					&= \sum_{k=0}^{2^n-1} \left\{ a\Norm{I_M(X^{\upsilon_j} - X_n)_{t_{k+1}}}{\semiLp{2}{\mu}} + a\Norm{I_M(X^{\upsilon_j} - X_n)_{t_k}}{\semiLp{2}{\mu}} \right\} \\
					&\leq \sum_{k=0}^{2^n-1} 2a \Norm{X^{\upsilon_j} - X_n}{\semiLp{2}{\mathcal{P},\mu_M}}
					\longrightarrow 0 \quad (n \longrightarrow \infty)
				\end{align}
				が成り立つ.$n$の任意性と(\refeq{eq:Ito-Integral_on_M_c_loc_report_7_5})及び(\refeq{eq:Ito-Integral_on_M_c_loc_report_7_4})
				を併せて(\refeq{eq:Ito-Integral_on_M_c_loc_report_7_3})が成立する.
				\QED
		\end{description}
	\end{prf}
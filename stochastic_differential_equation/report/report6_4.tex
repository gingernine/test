	$M \in \mathcal{M}_{2,c},\ X \in \semiLp{2}{I \times \Omega,\mathcal{P},\mu_M}$に対して
	定義した伊藤積分を更に拡張する.
	
	\begin{screen}
		\begin{dfn}[局所有界過程]
			$(\Omega,\mathcal{F},\mu)$上の確率過程$X$に対し或る$(\tau_j)_{j=0}^{\infty} \in \mathcal{T}$が存在して
			\begin{align}
				\sup{t \in I}{\Norm{X_{t \wedge \tau_j}}{\mathscr{L}^\infty}} < \infty
				\quad (j=0,1,\cdots)
			\end{align}
			が満たされているとき,$X$を局所有界過程(locally bounded process)という.
		\end{dfn}
	\end{screen}
	
	\begin{screen}
		\begin{thm}[局所マルチンゲールと左連続局所有界適合過程に対する伊藤積分]
			$X$を左連続且つ局所有界な適合過程,$M \in \mathcal{M}_{c,loc}$とする.
			このとき確率積分
			\begin{align}
				\int_0^t X_s\ dM_s \quad (t \in I)
			\end{align}
			が定義される.
		\end{thm}
	\end{screen}
	
	\begin{prf}
		\begin{description}
			\item[第一段] $\sup{t \in I}{\Norm{X_t}{\mathscr{L}^\infty}} < \infty$かつ$\Norm{\inprod<M>_T}{\semiLp{\infty}{\mu}} < \infty$
				なら$X \in \semiLp{2}{I \times \Omega,\mathcal{P},\mu_M}$が成り立つことを示す.すなわち,このとき$I_M(X)$が定義される.
				\begin{align}
					X^n_t \coloneqq X_0 \defunc_{\{0\}} + \sum_{k=0}^{2^n-1} X_{\frac{kT}{2^n}} \defunc_{\left( \frac{kT}{2^n}, \frac{(k+1)T}{2^n}\right]}(t)
					\quad (\forall t \in I)
				\end{align}
				として単純過程の列$\left( X^n \right)_{n=1}^{\infty}$を構成する.
				この表現より$X^n \in \mathcal{S}\ (n=1,2,\cdots)$が満たされ,また全ての$\omega \in \Omega$について
				$I \ni t \longmapsto X_t(\omega)$が左連続であるから,任意の$t \in I,\omega \in \Omega$に対して
				\begin{align}
					\left| X^n_t(\omega) - X_t(\omega) \right| \longrightarrow 0
					\quad (n \longrightarrow \infty)
				\end{align}
				が成り立つ.補題\ref{lem:properties_of_simple_predictable_processes}より
				$X^n\ (n \in \N)$は全て可測$\mathcal{P}/\borel{\R}$であるから,その各点収束先の
				$X$もまた可測$\mathcal{P}/\borel{\R}$である.
				また定理\ref{thm:quadratic_variation_bounded_then_M_2c}より$M \in \mathcal{M}_{2,c}$も満たされている.
				従って$\mathcal{P}/\borel{\R}$-可測関数の関数類を$\equiv{\cdot}{\Lp{0}{\mu_M}}$と表せば,
				(\refeq{eq:lem_properties_of_simple_predictable_processes_0})より
				\begin{align}
					\Norm{\equiv{X}{\Lp{0}{\mu_M}} - \equiv{X^n}{\Lp{0}{\mu_M}}}{\Lp{2}{\mu_M}}^2
					= \int_\Omega \int_I \left| X(t,\omega) - X^n(t,\omega) \right|^2\ \inprod<M>(dt,\omega)\ \mu(d\omega)
				\end{align}
				が成り立ち,Lebesgueの収束定理より右辺は$n \longrightarrow \infty$で0に収束する.
				補題\ref{lem:properties_of_simple_predictable_processes}より$\mathfrak{S}$は
				$\Lp{2}{I \times \Omega,\mathcal{P},\mu_M}$で稠密であるから
				$\equiv{X}{\Lp{0}{\mu_M}} \in \Lp{2}{I \times \Omega,\mathcal{P},\mu_M}$が従う.
				
			\item[第二段]
				前段の仮定を外す.$X$が局所有界過程であるから,或る$(\tau_j)_{j=0}^{\infty} \in \mathcal{T}$が存在して
				\begin{align}
					\sup{t \in I}{\Norm{X_{t \wedge \tau_j}}{\mathscr{L}^\infty}} < \infty
					\quad (j=0,1,\cdots)
				\end{align}
				が満たされる.また
				\begin{align}
					\hat{\tau}_j(\omega) \coloneqq
					\inf{}{\Set{t \in I}{|\inprod<M>_t(\omega)| \geq j}} \wedge T\ \footnotemark
					\quad (\forall \omega \in \Omega,\ j=0,1,\cdots)
				\end{align}
				\footnotetext{
					$\Set{t \in I}{|\inprod<M>_t(\omega)| \geq j} = \emptyset$の場合$\sigma_j(\omega) = T$とする.
				}
				として$\left( \hat{\tau}_j \right)_{j=0}^{\infty} \in \mathcal{T}$を定め
				\begin{align}
					\sigma_j \coloneqq \tau_j \wedge \hat{\tau}_j
					\quad (j=0,1,\cdots)
				\end{align}
				とおけば,$(\sigma_j)_{j=0}^{\infty} \in \mathcal{T}$且つ
				\begin{align}
					\Norm{X_{t \wedge \sigma_j}}{\mathscr{L}^\infty} < \Norm{X_{t \wedge \tau_j}}{\mathscr{L}^\infty},
					\quad \Norm{\inprod<M>_{t \wedge \sigma_j}}{\mathscr{L}^\infty} \leq j
					\quad (\forall t \in I,\ j=0,1,\cdots)
				\end{align}
				が成り立つ.従って前段の結果より$I_{M^{\sigma_j}(X^{\sigma_j})}\ (j=0,1,\cdots)$が定義される.
		\end{description}
		\QED
	\end{prf}
	
係数体を$\R$,測度空間を$(X,\mathcal{F},\mu)$とする.

\section{H\Ddot{o}lderの不等式とMinkowskiの不等式}
$\mathcal{F}/\borel{\R}$-可測関数$f$に対して
\begin{align}
	\Norm{f}{\mathscr{L}^p} \coloneqq
	\begin{cases}
		\inf{}{\Set{r \in \R}{|f(x)| \leq r\quad \mbox{$\mu$-a.e.}x \in X}} & (p = \infty) \\
		\left( \int_{X} |f(x)|^p\ \mu(dx) \right)^{\frac{1}{p}} & (0 < p < \infty)
	\end{cases}
\end{align}
により$\Norm{\cdot}{\mathscr{L}^p}$を定め,以降$\Norm{\cdot}{\semiLp{p}{\Omega,\mathcal{F},\mu}}$
或は$\Norm{\cdot}{\semiLp{p}{\mu}}$と表記することもある.
\begin{align}
	\semiLp{p}{X,\mathcal{F},\mu} \coloneqq \Set{f:X \rightarrow \R}{f:\mbox{可測}\mathcal{F}/\borel{\R},\ \Norm{f}{\mathscr{L}^p} < \infty} \quad (1 \leq p \leq \infty)
\end{align}
として空間$\semiLp{p}{X,\mathcal{F},\mu}$を定義し,$\semiLp{p}{\mu},\semiLp{p}{\mathcal{F},\mu},
\semiLp{p}{\mathcal{F}}$などと略記することもある.以下に示す不等式により
$\semiLp{p}{X,\mathcal{F},\mu}$は$\R$上の線形空間となる.

\begin{screen}
	\begin{lem}
		任意の$f \in \semiLp{\infty}{X,\mathcal{F},\mu}$に対して次が成り立つ:
		\begin{align}
			|f(x)| \leq \Norm{f}{\mathscr{L}^\infty} \quad \mbox{$\mu$-a.e.}x \in X.
		\end{align}
		\label{lem:holder_inequality}
	\end{lem}
\end{screen}

\begin{prf}
	$\semiLp{\infty}{X,\mathcal{F},\mu}$の定義より任意の実数$\alpha > \Norm{f}{\mathscr{L}^\infty}$に対して
	\begin{align}
		\mu\left( \Set{x \in X}{|f(x)| > \alpha} \right) = 0
	\end{align}
	が成り立つから,
	\begin{align}
		\Set{x \in X}{|f(x)| > \Norm{f}{\mathscr{L}^\infty}} = \bigcup_{n =1}^{\infty} \Set{x \in X}{|f(x)| > \Norm{f}{\mathscr{L}^\infty} + 1/n}
	\end{align}
	の右辺は$\mu$-零集合であり主張が従う.
	\QED
\end{prf}

\begin{screen}
	\begin{thm}[H\Ddot{o}lderの不等式]
		$1 \leq p, q \leq \infty$,$p + q = pq\ (p = \infty$なら$q = 1)$とする.このとき
		任意の$\mathcal{F}/\borel{\R}$-可測関数$f,g$に対して次が成り立つ:
		\begin{align}
			\int_{X} |f(x)g(x)|\ \mu(dx) \leq \Norm{f}{\mathscr{L}^p} \Norm{g}{\mathscr{L}^q}. \label{ineq:holder}
		\end{align}
		\label{thm:holder_inequality}
	\end{thm}
\end{screen}

\begin{prf}
	$\Norm{f}{\mathscr{L}^p} = \infty$又は$\Norm{g}{\mathscr{L}^q} = \infty$のとき不等式(\refeq{ineq:holder})
		は成り立つから,以下では$\Norm{f}{\mathscr{L}^p} < \infty$かつ$\Norm{g}{\mathscr{L}^q} < \infty$の場合を考える.
	\begin{description}
		\item[$p = \infty,\ q = 1$の場合]
			補題\ref{lem:holder_inequality}により或る零集合$A$が存在して
			\begin{align}
				|f(x)g(x)| \leq \Norm{f}{\mathscr{L}^\infty}|g(x)| \quad (\forall x \in X \backslash A).
			\end{align}
			が成り立つから,
			\begin{align}
				&\int_{X} |f(x)g(x)|\ \mu(dx) = \int_{X \backslash A} |f(x)g(x)|\ \mu(dx) \\
				&\qquad \leq \Norm{f}{\mathscr{L}^\infty} \int_{X \backslash A} |g(x)|\ \mu(dx) 
				= \Norm{f}{\mathscr{L}^\infty} \Norm{g}{\mathscr{L}^1}
			\end{align}
			が従い不等式(\refeq{ineq:holder})を得る.
		
		\item[$1 < p,q < \infty$の場合]
			$\Norm{f}{\mathscr{L}^p} = 0$のとき
			\begin{align}
				B \coloneqq \Set{x \in X}{|f(x)| > 0}
			\end{align}
			は零集合であるから,
			\begin{align}
				\int_{X} |f(x)g(x)|\ \mu(dx) = \int_{X \backslash B} |f(x)g(x)|\ \mu(dx) = 0
			\end{align}
			となり(\refeq{ineq:holder})を得る.$\Norm{g}{\mathscr{L}^q} = 0$の場合も同じである.
			次に$0 < \Norm{f}{\mathscr{L}^p},\Norm{g}{\mathscr{L}^q} < \infty$の場合を示す.
			実数値対数関数$(0,\infty) \ni t \longmapsto -\Log{t}$は凸であるから,$1/p + 1/q = 1$に対して
			\begin{align}
				-\Log{\left( \frac{s}{p} + \frac{t}{q} \right)} \leq \frac{1}{p}(-\Log{s}) + \frac{1}{q}(-\Log{t}) \quad (\forall s,t > 0)
			\end{align}
			を満たし
			\begin{align}
				s^{\frac{1}{p}}t^{\frac{1}{q}} \leq \frac{s}{p} + \frac{t}{q} \quad (\forall s,t > 0)
			\end{align}
			が従う.ここで
			\begin{align}
				F(x) \coloneqq \frac{|f(x)|^p}{\Norm{f}{\mathscr{L}^p}^p},
				\quad G(x) \coloneqq \frac{|g(x)|^q}{\Norm{g}{\mathscr{L}^q}^q} \quad (\forall x \in X)
			\end{align}
			により可積分関数$F,G$を定めれば,
			\begin{align}
				F(x)^{\frac{1}{p}}G(x)^{\frac{1}{q}} \leq \frac{1}{p}F(x) + \frac{1}{q}G(x) \quad (\forall x \in X)
			\end{align}
			が成り立つから
			\begin{align}
				&\int_{X} \frac{|f(x)|}{\Norm{f}{\mathscr{L}^p}} \frac{|g(x)|}{\Norm{g}{\mathscr{L}^q}}\ \mu(dx)
				= \int_{X} F(x)^{\frac{1}{p}}G(x)^{\frac{1}{q}}\ \mu(dx) \\
				&\qquad \leq \frac{1}{p} \int_{X} F(x)\ \mu(dx) + \frac{1}{q} \int_{X} G(x)\ \mu(dx)
				= \frac{1}{p} + \frac{1}{q} = 1
			\end{align}
			が従い,$\Norm{f}{\mathscr{L}^p}\Norm{g}{\mathscr{L}^q}$を移項して不等式(\refeq{ineq:holder})を得る.
			\QED
	\end{description}
\end{prf}

\begin{screen}
	\begin{thm}[Minkowskiの不等式]
		$1 \leq p \leq \infty$とする.このとき
		任意の$\mathcal{F}/\borel{\R}$-可測関数$f,g$に対して次が成り立つ:
		\begin{align}
			\Norm{f+g}{\mathscr{L}^p} \leq \Norm{f}{\mathscr{L}^p} + \Norm{g}{\mathscr{L}^p}. \label{ineq:minkowski}
		\end{align}
	\end{thm}
\end{screen}

\begin{prf}
	$\Norm{f+g}{\mathscr{L}^p} = 0,\ \Norm{f}{\mathscr{L}^p} = \infty,\ \Norm{g}{\mathscr{L}^p} = \infty$
	のいずれかが満たされているとき不等式(\refeq{ineq:minkowski})は成り立つから,以下では
	$\Norm{f+g}{\mathscr{L}^p} > 0,\ \Norm{f}{\mathscr{L}^p} < \infty,\ \Norm{g}{\mathscr{L}^p} < \infty$
	の場合を考える.
	\begin{description}
		\item[$p = \infty$の場合]
			補題\ref{lem:holder_inequality}により
			\begin{align}
				C \coloneqq \Set{x \in X}{|f(x)| > \Norm{f}{\mathscr{L}^\infty}} \cup \Set{x \in X}{|g(x)| > \Norm{g}{\mathscr{L}^\infty}}
			\end{align}
			は零集合であり,
			\begin{align}
				|f(x) + g(x)| \leq |f(x)| + |g(x)| \leq \Norm{f}{\mathscr{L}^\infty} + \Norm{g}{\mathscr{L}^\infty} \quad (\forall x \in X \backslash C)
			\end{align}
			が成り立つ.$\Norm{\cdot}{\mathscr{L}^\infty}$の定義より不等式(\refeq{ineq:minkowski})を得る.
		
		\item[$p = 1$の場合]
			\begin{align}
				|f(x) + g(x)| \leq |f(x)| + |g(x)| \quad (\forall x \in X)
			\end{align}
			の両辺を積分して不等式(\refeq{ineq:minkowski})を得る.
		
		\item[$1 < p < \infty$の場合]
			$p + q = pq$が成り立つように$q > 1$を取る.各点$x \in X$で
			\begin{align}
				|f(x) + g(x)|^p = |f(x) + g(x)||f(x) + g(x)|^{p-1} \leq |f(x)||f(x) + g(x)|^{p-1} + |g(x)||f(x) + g(x)|^{p-1}
			\end{align}
			が成り立つから,両辺を積分すればH\Ddot{o}lderの不等式により
			\begin{align}
				\Norm{f+g}{\mathscr{L}^p}^p &= \int_{X} |f(x) + g(x)|^p\ \mu(dx) \\
				&\leq \int_{X} |f(x)||f(x) + g(x)|^{p-1}\ \mu(dx) + \int_{X} |g(x)||f(x) + g(x)|^{p-1}\ \mu(dx) \\
				&\leq \Norm{f}{\mathscr{L}^p}\Norm{f+g}{\mathscr{L}^p}^{p-1} + \Norm{g}{\mathscr{L}^p}\Norm{f+g}{\mathscr{L}^p}^{p-1}
				\label{Minkowski_1}
			\end{align}
			が得られる.また$|f|^p,|g|^p$の可積性と
			\begin{align}
				|f(x) + g(x)|^p \leq \left(|f(x)| + |g(x)|\right)^p \leq 2^p \left( |f(x)|^p + |g(x)|^p \right) \quad (\forall x \in X)
			\end{align}
			により$\Norm{f+g}{\mathscr{L}^p} < \infty$が従うから,
			(\refeq{Minkowski_1})の両辺を$\Norm{f+g}{\mathscr{L}^p}^{p-1}$で割って(\refeq{ineq:minkowski})を得る.
			\QED
	\end{description}
\end{prf}

以上の結果より$\semiLp{p}{X,\mathcal{F},\mu}$は線形空間となる.実際線型演算は
\begin{align}
	(f+g)(x) \coloneqq f(x) + g(x), \quad (\alpha f)(x) \coloneqq \alpha f(x), \quad (\forall x \in X,\ f,g \in \semiLp{p}{X,\mathcal{F},\mu},\ \alpha \in \C)
\end{align}
により定義され,Minkowskiの不等式により加法について閉じている.

\begin{itembox}[l]{}
	\begin{lem}
		$1 \leq p \leq \infty$に対し,$\Norm{\cdot}{\mathscr{L}^p}$は線形空間$\semiLp{p}{X,\mathcal{F},\mu}$のセミノルムである.
	\end{lem}
\end{itembox}
\begin{prf}\mbox{}
	\begin{description}
	\item[半正値性] $\Norm{\cdot}{\mathscr{L}^p}$が正値であることは定義による.
		しかし$\Norm{f}{\mathscr{L}^p} = 0$であっても$f$が零写像であるとは限らず,実際$\mu$-零集合$E$を取り
		\begin{align}
			f(x) \coloneqq
			\begin{cases}
				1 & (x \in E) \\
				0 & (x \in \Omega \backslash E)
			\end{cases}
		\end{align}
		により$f$を定めれば$\Norm{f}{\mathscr{L}^p} = 0$が成り立つ.
		
	\item[同次性] 
		任意に$\alpha \in \R,\ f \in \semiLp{p}{X,\mathcal{F},\mu}$を取る.
		$1 \leq p < \infty$の場合は
		\begin{align}
			\left( \int_{X} |\alpha f(x)|^p\ \mu(dx) \right)^{1/p} = \left( |\alpha|^p \int_{X} |f(x)|^p\ \mu(dx) \right)^{1/p} 
			= |\alpha| \left( \int_{X} |f(x)|^p\ \mu(dx) \right)^{1/p}
		\end{align}
		により,$p = \infty$の場合は
		\begin{align}
			\inf{}{\Set{r \in \R}{|\alpha f(x)| \leq r \quad \mbox{$\mu$-a.e.}x \in X}} = |\alpha|\inf{}{\Set{r \in \R}{|f(x)|  \leq r \quad \mbox{$\mu$-a.e.}x \in X}}
		\end{align}
		により$\Norm{\alpha f}{\mathscr{L}^p} = |\alpha|\Norm{f}{\mathscr{L}^p}$が成り立つ.
		
	\item[三角不等式] Minkowskiの不等式による.
	\QED
	\end{description}
\end{prf}

\section{空間$\mathrm{L}^p$}
\begin{description}
	\item[可測関数全体の商集合]
		$\mathcal{F}/\borel{R}$-可測関数全体の集合を
		\begin{align}
			\semiLp{0}{X,\mathcal{F},\mu} \coloneqq \Set{f:X \rightarrow \R}{f:\mbox{可測}\mathcal{F}/\borel{\R}}
		\end{align}
		とおく.二元$f,g \in \semiLp{0}{X,\mathcal{F},\mu}$に対し
		\begin{align}
			 f \sim g \quad \DEF \quad f = g \quad \mbox{$\mu$-a.e.}
		\end{align}
		により定める$\sim$は同値関係であり,$\sim$による$\semiLp{0}{X,\mathcal{F},\mu}$の商集合を$\Lp{0}{X,\mathcal{F},\mu}$と表す.
	
	\item[商集合における算法]
		$\Lp{0}{X,\mathcal{F},\mu}$の元である関数類(同値類)を$[f]\ $($f$は関数類の代表)と表す.
		$\Lp{0}{X,\mathcal{F},\mu}$における線型演算を次で定義すれば,$\Lp{0}{X,\mathcal{F},\mu}$は$\R$上の線形空間となる:
		\begin{align}
			&[f] + [g] \coloneqq [f+g] && (\forall [f],[g] \in \Lp{0}{X,\mathcal{F},\mu}),\\
			&\alpha [f] \coloneqq [\alpha f] && (\forall [f] \in \Lp{0}{X,\mathcal{F},\mu},\ \alpha \in \R).
		\end{align}
		この演算はwell-definedである.実際任意の$f' \in [f]$と$g' \in [g]$に対して
		\begin{align}
			\{ f+g \neq f'+g' \} \subset \{ f \neq f' \} \cup \{ g \neq g' \}, \quad
			\{ \alpha f \neq \alpha f' \} = \{ f \neq f' \}
		\end{align}
		が成り立ち
		\footnote{
			$\{ f \neq g \} \coloneqq \Set{x \in X}{f(x) \neq g(x)}.$
		}
		,どちらの右辺も$\mu$-零集合であるから$[f + g] = [f' + g'],\ [\alpha f'] = [\alpha f]$が従う.
		更に乗法を次により定義すれば$\Lp{0}{X,\mathcal{F},\mu}$は可換環にもなる:
		\begin{align}
			[f][g] \coloneqq [fg] \quad (\forall [f],[g] \in \Lp{0}{X,\mathcal{F},\mu}).
		\end{align}
		$\Lp{0}{X,\mathcal{F},\mu}$における零元は零写像の関数類でありこれを[0]と表す.また
		単位元は恒等的に$1$を取る関数の関数類でありこれを[1]と表す.
		また減法を
		\begin{align}
			[f] - [g] \coloneqq [f] + (-[g]) = [f] + [-g] = [f - g]
		\end{align}
		により定める.
	
	\item[関数類の順序]
		$[f],[g] \in \Lp{0}{X,\mathcal{F},\mu}$に対して次の関係$<(>)$を定める:
		\begin{align}
			[f] < [g]\ \left( [g] > [f] \right) \quad \DEF \quad f < g \quad \mbox{$\mu$-a.s.} \label{dfn:equiv_class_order}
		\end{align}
		この定義はwell-definedである.実際任意の$f' \in [f],g' \in [g]$に対して
		\begin{align}
			\left\{ f' \geq g' \right\} \subset \left\{ f \neq f' \right\} \cup \left\{ f \geq g \right\} \cup \left\{ g \neq g' \right\}
		\end{align}
		の右辺は零集合であるから
		\begin{align}
			[f] < [g] \Leftrightarrow [f'] < [g']
		\end{align}
		が従う.$<$または$=$であることを$\leq$と表し,同様に$\geq$を定める.
		この関係$\leq$は次に示す規則を満たし,$\Lp{0}{X,\mathcal{F},\mu}$における順序となる:
		\ 任意の$[f],[g],[h] \in \Lp{0}{X,\mathcal{F},\mu}$に対し,
		\begin{itemize}
			\item $[f] \leq [f]$が成り立つ.
			\item $[f] \leq [g]$かつ$[g] \leq [f]$ならば$[f] = [g]$が成り立つ.
			\item $[f] \leq [g],\ [g] \leq [h]$ならば$[f] \leq [h]$が成り立つ.
		\end{itemize}
		
	\item[関数類の冪・絶対値]
		$[f] \in \Lp{0}{X,\mathcal{F},\mu}$が$[f] \geq [0]$を満たすとき,$p \geq 0$に対し
		$[f]^p \coloneqq [f^p]$により冪乗を定める.また$|[f]| \coloneqq [|f|]$として絶対値$|\cdot|$を定める.
\end{description}

\begin{itembox}[l]{}
	\begin{lem}[商空間におけるノルムの定義]
		\begin{align}
			\Norm{[f]}{\mathrm{L}^p} \coloneqq \Norm{f}{\mathscr{L}^p} \quad (1 \leq p \leq \infty,\ f \in \semiLp{p}{X,\mathcal{F},m})
		\end{align}
		として$\Norm{\cdot}{\mathrm{L}^p}:\Lp{0}{X,\mathcal{F},m} \rightarrow \R$を定義すればこれはwell-definedである.つまり代表元に依らずに値がただ一つに定まる.
		更に次で定義する空間
		\begin{align}
			\Lp{p}{X,\mathcal{F},m} \coloneqq \Set{[f] \in \Lp{0}{X,\mathcal{F},m}}{\Norm{[f]}{\mathrm{L}^p} < \infty} \quad (1 \leq p \leq \infty)
		\end{align}
		は$\Norm{\cdot}{\mathrm{L}^p}$をノルムとしてノルム空間となる.
	\end{lem}
\end{itembox}
\begin{prf}\mbox{}
	\begin{description}
		\item[well-definedであること]
			$f \in \semiLp{p}{X,\mathcal{F},m}$に対し,任意に$g \in [f]$を選ぶ.
			示すことは$\Norm{f}{\mathscr{L}^p}^p = \Norm{g}{\mathscr{L}^p}^p$が成り立つことである.
			\begin{align}
				A \coloneqq \Set{x \in X}{f(x) \neq g(x)} \quad \in \mathcal{F}
			\end{align}
			とおく.$A$は$m(A)=0$を満たす.
			\begin{description}
				\item[$p = \infty$の場合]
					$A^c$の上で$f(x)=g(x)$となるから
					\begin{align}
						\Set{x \in X}{|g(x)| > \Norm{f}{\mathscr{L}^\infty}} 
						&\subset A + A^c \cap \Set{x \in X}{|g(x)| > \Norm{f}{\mathscr{L}^\infty}} \\
						&= A + A^c \cap \Set{x \in X}{|f(x)| > \Norm{f}{\mathscr{L}^\infty}} \\
						&\subset A + \Set{x \in X}{|f(x)| > \Norm{f}{\mathscr{L}^\infty}}
					\end{align}
					が成り立ち,最右辺は2項とも$m$-零集合であるから$\Norm{g}{\mathscr{L}^\infty} \leq \Norm{f}{\mathscr{L}^\infty}$が従う.
					逆向きの不等号も同様に示されて$\Norm{g}{\mathscr{L}^\infty} = \Norm{f}{\mathscr{L}^\infty}$を得る.
				\item[$1 \leq p < \infty$の場合]
					$m(A)=0$により
					\begin{align}
						\Norm{f}{\mathscr{L}^p}^p = \int_X f(x)\ m(dx) = \int_{A^c} f(x)\ m(dx) = \int_{A^c} g(x)\ m(dx) = \int_X g(x)\ m(dx) = \Norm{g}{\mathscr{L}^p}^p
					\end{align}
					が成り立つ.
			\end{description}
		
		\item[ノルムとなること]
			任意に$[f],[g] \in \Lp{p}{X,\mathcal{F},m}$と$\alpha \in \K$を取る.
			$\Norm{[f]}{\mathrm{L}^p}$の正値性は$\Norm{\cdot}{\mathscr{L}^p}$の正値性から従う.
			また$m(f \neq 0) > 0$なら$\Norm{f}{\mathscr{L}^p} > 0$となるから,
			対偶により$\Norm{[f]}{\mathrm{L}^p} = 0$なら$[f]$は零元$[0]$,
			逆に$[f] = [0]$なら$\Norm{[f]}{\mathrm{L}^p} = \Norm{0}{\mathscr{L}^p} = 0$となる.
			$\Norm{\cdot}{\mathscr{L}^p}$の同次性とMinkowskiの不等式から
			\begin{align}
				&\Norm{\alpha[f]}{\mathrm{L}^p} = \Norm{[\alpha f]}{\mathrm{L}^p} = \Norm{\alpha f}{\mathscr{L}^p} = |\alpha|\Norm{f}{\mathscr{L}^p} = |\alpha|\Norm{[f]}{\mathrm{L}^p} \\
				&\Norm{[f] + [g]}{\mathrm{L}^p} = \Norm{[f + g]}{\mathrm{L}^p} = \Norm{f + g}{\mathscr{L}^p} \leq \Norm{f}{\mathscr{L}^p} + \Norm{g}{\mathscr{L}^p} = \Norm{[f]}{\mathrm{L}^p} + \Norm{[g]}{\mathrm{L}^p}
			\end{align}
			も成り立ち,$\Norm{\cdot}{\mathrm{L}^p}$が$\Lp{p}{X,\mathcal{F},m}$においてノルムとなると示された.
			\QED
	\end{description}
\end{prf}

\begin{itembox}[l]{}
	\begin{prp}[$\mathrm{L}^p$の完備性]
		ノルム空間$\Lp{p}{X,\mathcal{F},m}\ (1 \leq p \leq \infty)$はBanach空間である.
	\end{prp}
\end{itembox}
\begin{prf}
	任意に$\Lp{p}{X,\mathcal{F},m}$のCauchy列$[f_n] \in \Lp{p}{X,\mathcal{F},m}\ (n=1,2,3,\cdots)$を取る.
	Cauchy列であるから$1/2$に対して或る$N_1 \in \N$が取れて,$n>m \geq N_1$ならば
	$\Norm{[f_n]-[f_m]}{\mathrm{L}^p} = \Norm{[f_n - f_m]}{\mathrm{L}^p} < 1/2$となる.
	ここで$m = n_1$と表記することにする.
	同様に$1/2^2$に対して或る$N_2 \in \N\ (N_2 > N_1)$が取れて,$n'>m' \geq N_2$ならば
	$\Norm{[f_{n'} - f_{m'}]}{\mathrm{L}^p} < 1/2^2$となる.
	先ほどの$n$について,$n > N_2$となるように取れるからこれを$n = n_2$と表記し,更に$m' = n_2$ともしておく.今のところ
	\begin{align}
		\Norm{[f_{n_1} - f_{n_2}]}{\mathrm{L}^p} < 1/2
	\end{align}
	と表示できる.再び同様に$1/2^3$に対して或る$N_3 \in \N\ (N_3 > N_2)$が取れて,$n''>m'' \geq N_2$ならば
	$\Norm{[f_{n''} - f_{m''}]}{\mathrm{L}^p} < 1/2^3$となる.
	先ほどの$n'$について$n' > N_3$となるように取れるからこれを$n' = n_3$と表記し,更に$m'' = n_3$ともしておく.今までのところで
	\begin{align}
		&\Norm{[f_{n_1} - f_{n_2}]}{\mathrm{L}^p} < 1/2 \\
		&\Norm{[f_{n_2} - f_{n_3}]}{\mathrm{L}^p} < 1/2^2
	\end{align}
	が成り立っている.数学的帰納法により
	\begin{align}
		\Norm{[f_{n_k} - f_{n_{k+1}}]}{\mathrm{L}^p} < 1/2^k \quad (n_{k+1} > n_k,\ k=1,2,3,\cdots) \label{ineq:Lp_banach_2}
	\end{align}
	が成り立つように自然数の部分列$(n_k)_{k=1}^{\infty}$を取ることができる.
	\begin{description}
		\item[$p = \infty$の場合]\mbox{}\\
			$[f_{n_k}]$の代表元$f_{n_k}$について,
			\begin{align}
				A_k &\coloneqq \Set{x \in X}{|f_{n_k}(x)| > \Norm{f_{n_k}}{\mathscr{L}^\infty}}, \\
				A^k &\coloneqq \Set{x \in X}{|f_{n_k}(x) - f_{n_{k+1}}(x)| > \Norm{f_{n_k} - f_{n_{k+1}}}{\mathscr{L}^\infty}}
			\end{align}
			とおけばH\Ddot{o}lderの不等式の証明中の補助定理より$m(A_k) = m(A^k) = 0$であり,
			\begin{align}
				A \coloneqq \left( \cup_{k=1}^{\infty} A_k \right) \cup \left( \cup_{k=1}^{\infty} A^k \right)
			\end{align}
			として$m$-零集合を定め
			\begin{align}
				\hat{f}_{n_k}(x) =
				\begin{cases}
					f_{n_k}(x) & (x \notin A) \\
					0 & (x \in A)
				\end{cases}
				\quad (\forall x \in X)
			\end{align}
			と定義した$\hat{f}_{n_k}$もまた$[f_{n_k}]$の元となる.
			$\hat{f}_{n_k}$は$X$上の有界可測関数であり
			\begin{align}
				\Norm{\hat{f}_{n_k} - \hat{f}_{n_{k+1}}}{\mathscr{L}^\infty} = \sup{x \in X}{\left|\hat{f}_{n_k}(x) - \hat{f}_{n_{l}}(x)\right|} < 1/2^k \quad (k=1,2,3,\cdots) 
				\label{ineq:Lp_banach_1}
			\end{align}
			が成り立つから,各点$x \in X$で$\left( \hat{f}_{n_k}(x) \right)_{k=1}^{\infty}$は$\K$のCauchy列となり
			\footnote{
				任意の$\epsilon > 0$に対し$1/2^N < \epsilon$を満たす$N \in \N$を取れば,全ての$l > k > N$に対して
				\begin{align}
					\left|\hat{f}_{n_k}(x) - \hat{f}_{n_{l}}(x)\right| \leq \sum_{j=k}^{l-1}\left|\hat{f}_{n_j}(x) - \hat{f}_{n_{j+1}}(x)\right| < \sum_{k > N} 1/2^k = 1/2^N < \epsilon
				\end{align}
				が成り立つ.
			}
			各点$x \in X$での極限を$\hat{f}(x)$と表す.一般に距離空間に値を取る可測関数列の各点収束の極限関数は可測関数であるから
			$\hat{f}$もまた可測$\mathcal{F}/\borel{\K}$である.また$\hat{f}$は有界である.実際
			式(\refeq{ineq:Lp_banach_1})から任意の$l > k$に対し
			\begin{align}
				|\hat{f}_{n_k}(x) - \hat{f}_{n_l}(x)| \leq \sum_{j=k}^{l-1} |\hat{f}_{n_{j}}(x) - \hat{f}_{n_{j+1}}(x)| 
				\leq \sum_{j=k}^{l-1} \sup{x \in X}{|\hat{f}_{n_j}(x) - \hat{f}_{n_{j+1}}(x)|} < 1/2^{k-1}
			\end{align}
			が成り立つから,極限関数$\hat{f}(x)$も
			\begin{align}
				\sup{x \in X}{|\hat{f}_{n_k}(x) - \hat{f}(x)|} \leq 1/2^{k-1} \label{ineq:Lp_banach_3}
			\end{align}
			を満たすことになる.なぜなら,もし或る$x \in X$で$\alpha \coloneqq |\hat{f}_{n_k}(x) - \hat{f}(x)| > 1/2^{k-1}$となる場合,
			任意の$l > k$に対し
			\begin{align}
				0 < \alpha - 1/2^{k-1} < |\hat{f}_{n_k}(x) - \hat{f}(x)| - |\hat{f}_{n_k}(x) - \hat{f}_{n_l}(x)| \leq |\hat{f}_{n_l}(x) - \hat{f}(x)|
			\end{align}
			となり各点収束に反するからである.不等式(\refeq{ineq:Lp_banach_3})により任意の$x \in X$において
			\begin{align}
				|\hat{f}(x)| < |\hat{f}_{n_k}(x)| + 1/2^{k-1} \leq \Norm{\hat{f}_{n_k}}{\mathscr{L}^\infty} + 1/2^{k-1}
			\end{align}
			が成り立ち$\hat{f}$の有界性が判る.以上で極限関数$\hat{f}$が有界可測関数であると示された.
			$\hat{f}$を代表元とする$[\hat{f}] \in \Lp{\infty}{X,\mathcal{F},m}$に対し,不等式(\refeq{ineq:Lp_banach_3})により
			\begin{align}
				\Norm{[f_{n_k}] - [\hat{f}]}{\mathrm{L}^\infty} = \Norm{\hat{f}_{n_k} - \hat{f}}{\mathscr{L}^\infty} 
				= \sup{x \in X}{|\hat{f}_{n_k}(x) - \hat{f}(x)|}
				\longrightarrow 0 \ (k \longrightarrow \infty)
			\end{align}
			となりCauchy列$\left( [f_{n}] \right)_{n=1}^{\infty}$の部分列$\left( [f_{n_k}] \right)_{k=1}^{\infty}$は$[\hat{f}]$に収束する.
			元のCauchy列も$[\hat{f}]$に収束するから$\Lp{\infty}{X,\mathcal{F},m}$はBanach空間である.
			
		\item[$1 \leq p < \infty$の場合]\mbox{}\\
			$[f_{n_k}]$の代表元$f_{n_k}$に対して
			\begin{align}	
				f_{n_k}(x) &\coloneqq f_{n_1}(x) + \sum_{j=1}^{k}(f_{n_j}(x) - f_{n_{j-1}}(x)) \label{eq:Lp_banach_3}
			\end{align}
			と表現できるから,これに対して
			\begin{align}
				g_k(x) &\coloneqq |f_{n_1}(x)| + \sum_{j=1}^{k}|f_{n_j}(x) - f_{n_{j-1}}(x)|
			\end{align}
			として可測関数列$(g_k)_{k=1}^{\infty}$を定める.Minkowskiの不等式と式(\refeq{ineq:Lp_banach_2})より
			\begin{align}
				\Norm{g_k}{\mathscr{L}^p} \leq \Norm{f_{n_1}}{\mathscr{L}^p} + \sum_{j=1}^{k}\Norm{f_{n_j} - f_{n_{j-1}}}{\mathscr{L}^p}
				< \Norm{f_{n_1}}{\mathscr{L}^p} + \sum_{j=1}^{k} 1/2^j < \Norm{f_{n_1}}{\mathscr{L}^p} + 1 < \infty
			\end{align}
			が成り立つ.各点$x \in X$で$g_k(x)$は$k$について単調増大であるから,単調収束定理より
			\begin{align}
				\Norm{g}{\mathscr{L}^p}^p = \lim_{k \to \infty} \Norm{g_k}{\mathscr{L}^p}^p < \Norm{f_{n_1}}{\mathscr{L}^p} + 1 < \infty
			\end{align}
			となるので$g \in \Lp{p}{X,\mathcal{F},m}$である.従って
			\begin{align}
				B_n &\coloneqq \Set{x \in X}{g(x) \leq n} \in \mathcal{F}, \\
				B &\coloneqq \bigcup_{n=1}^{\infty} B_n
			\end{align}
			とおけば$m(X \backslash B) = 0$であり,式(\refeq{eq:Lp_banach_3})の級数は$B$上で絶対収束する(各点).
			\begin{align}
				f(x) \coloneqq
				\begin{cases}
					\lim\limits_{k \to \infty} f_{n_k}(x) & (x \in B) \\
					0 & (x \in X \backslash B)
				\end{cases}
			\end{align}
			として可測$\mathcal{F}/\borel{\R}$関数$f$を定義すれば,$|f(x)| \leq g(x)\ (\forall x \in X)$と
			$g^p$の可積性から$f$の関数類$[f]$は$\Lp{p}{X,\mathcal{F},m}$の元となる.
			関数列$(\left( f_{n_k} \right)_{k=1}^{\infty})$は$f$に概収束し,
			また$|f_{n_k}(x) - f(x)|^p \leq 2^p(|f_{n_k}(x)|^p + |f(x)|^p) \leq 2^{p+1} g(x)^p\ (\forall x \in X)$となるから,
			Lebesgueの収束定理により
			\begin{align}
				\lim_{k \to \infty}\Norm{[f_{n_k}] - [f]}{\mathrm{L}^p}^p
				= \lim_{k \to \infty}\Norm{f_{n_k} - f}{\mathscr{L}^p}^p
				= \lim_{k \to \infty} \int_X |f_{n_k}(x) - f(x)|^p\ m(dx) = 0
			\end{align}
			が成り立ち,Cauchy列$\left( [f_{n}] \right)_{n=1}^{\infty}$の部分列$\left( [f_{n_k}] \right)_{k=1}^{\infty}$が$[f]$に収束すると示された.
			従って元のCauchy列も$[f]$に収束するから$\Lp{p}{X,\mathcal{F},m}$はBanach空間である.
			\QED
	\end{description}
\end{prf}

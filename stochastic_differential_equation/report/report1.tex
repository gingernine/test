\section{H\Ddot{o}lderの不等式とMinkowskiの不等式}
$\K$を$\K = \R$或は$\K = \C$とおく.測度空間を$(X,\mathcal{F},m)$とし,
可測$\mathcal{F}/\borel{\K}$関数$f$に対して
\begin{align}
	\Norm{f}{\mathscr{L}^p} \coloneqq
	\begin{cases}
		\inf{}{\Set{r \in \R}{|f(x)| \leq r,\ \mathrm{a.e.}x \in X}} & (p = \infty) \\
		\left( \int_{X} |f(x)|^p\ m(dx) \right)^{\frac{1}{p}} & (0 < p < \infty)
	\end{cases}
\end{align}
と定め,
\begin{align}
	\semiLp{p}{X,\mathcal{F},m} \coloneqq \Set{f:X \rightarrow \K}{f:\mbox{可測}\mathcal{F}/\borel{\K},\ \Norm{f}{\mathscr{L}^p} < \infty} \quad (1 \leq p \leq \infty)
\end{align}
として空間$\semiLp{p}{X,\mathcal{F},m}$を定義する.この空間は$\K$上の線形空間となるが,そのことを保証するために
次の二つの不等式が成り立つことを証明する.

\begin{itembox}[l]{}
	\begin{thm}[H\Ddot{o}lderの不等式]
		$1 \leq p, q \leq \infty$,$p + q = pq\ (p = \infty$なら$q = 1)$とする.このとき
		任意の可測$\mathcal{F}/\borel{\K}$関数$f,g$に対して次が成り立つ:
		\begin{align}
			\int_{X} |f(x)g(x)|\ m(dx) \leq \Norm{f}{\mathscr{L}^p} \Norm{g}{\mathscr{L}^q}. \label{ineq:holder}
		\end{align}
		\label{thm:holder_inequality}
	\end{thm}
\end{itembox}

まず次の補助定理を証明する.
\begin{itembox}[l]{}
	\begin{lem}
		$f \in \semiLp{\infty}{X,\mathcal{F},m}$ならば
		\begin{align}
			|f(x)| \leq \Norm{f}{\mathscr{L}^\infty} \quad (\mathrm{a.e.}x \in X).
		\end{align}
		\label{lem:holder_inequality}
	\end{lem}
\end{itembox}
\begin{prf}[補題\ref{lem:holder_inequality}]
	$\semiLp{\infty}{X,\mathcal{F},m}$の定義により任意の実数$\alpha > \Norm{f}{\mathscr{L}^\infty}$に対して
	\begin{align}
		m(\Set{x \in X}{|f(x)| > \alpha}) = 0
	\end{align}
	が成り立つから,
	\begin{align}
		\Set{x \in X}{|f(x)| > \Norm{f}{\mathscr{L}^\infty}} = \bigcup_{n =1}^{\infty} \Set{x \in X}{|f(x)| > \Norm{f}{\mathscr{L}^\infty} + 1/n}
	\end{align}
	の右辺は$m$-零集合となる.
	\QED
\end{prf}
\begin{prf}[定理\ref{thm:holder_inequality}]	
	定理の証明に入る.
	\begin{description}
		\item[$p = \infty,\ q = 1$の場合]
			$\Norm{f}{\mathscr{L}^\infty} = \infty$又は$\Norm{g}{\mathscr{L}^1} = \infty$なら不等式(\refeq{ineq:holder})
			は成り立つから,以下では$\Norm{f}{\mathscr{L}^\infty} < \infty$かつ$\Norm{g}{\mathscr{L}^1} < \infty$の場合を考える.
			補助定理により或る$m$-零集合$A \in \mathcal{F}$を除いて$|f(x)| \leq \Norm{f}{\mathscr{L}^\infty}$が成り立つから
			\begin{align}
				|f(x)g(x)| \leq \Norm{f}{\mathscr{L}^\infty}|g(x)| \quad (\forall x \in X \backslash A).
			\end{align}
			とでき,従って
			\begin{align}
				\int_{X} |f(x)g(x)|\ m(dx) = \int_{X \backslash A} |f(x)g(x)|\ m(dx) \leq \Norm{f}{\mathscr{L}^\infty} \int_{X \backslash A} |g(x)|\ m(dx) 
				= \Norm{f}{\mathscr{L}^\infty} \Norm{g}{\mathscr{L}^1}
			\end{align}
			となり不等式(\refeq{ineq:holder})が成り立つ.
		
		\item[$1 < p,q < \infty$の場合]
			$\Norm{f}{\mathscr{L}^p} = \infty$又は$\Norm{g}{\mathscr{L}^q} = \infty$なら不等式(\refeq{ineq:holder})
			は成り立つから,以下では$\Norm{f}{\mathscr{L}^p} < \infty$かつ$\Norm{g}{\mathscr{L}^q} < \infty$の場合を考える.
			$\Norm{f}{\mathscr{L}^p} = 0$であるとすると
			\begin{align}
				B \coloneqq \Set{x \in X}{|f(x)| > 0}
			\end{align}
			は$m$-零集合となるから,
			\begin{align}
				\int_{X} |f(x)g(x)|\ m(dx) = \int_{B} |f(x)g(x)|\ m(dx) + \int_{X \backslash B} |f(x)g(x)|\ m(dx) = 0
			\end{align}
			となり不等式(\refeq{ineq:holder})が成り立つ.$\Norm{g}{\mathscr{L}^q} = 0$の場合も同じである.
			
			最後に$0 < \Norm{f}{\mathscr{L}^p},\Norm{g}{\mathscr{L}^q} < \infty$の場合を示す.
			$-\Log{t} \quad (t > 0)$は凸関数であるから,$1/p + 1/q = 1$に対して
			\begin{align}
				-\Log{\left( \frac{s}{p} + \frac{t}{q} \right)} \leq \frac{1}{p}(-\Log{s}) + \frac{1}{q}(-\Log{t}) \quad (\forall s,t > 0)
			\end{align}
			が成り立ち,従って
			\begin{align}
				s^{1/p}t^{1/q} \leq \frac{s}{p} + \frac{t}{q} \quad (\forall s,t > 0)
			\end{align}
			を得る.この不等式を用いれば
			\begin{align}
				F(x) \coloneqq |f(x)|^p/ \Norm{f}{\mathscr{L}^p}^p,\quad G(x) \coloneqq |g(x)|^q/ \Norm{g}{\mathscr{L}^q}^q \quad (\forall x \in X)
			\end{align}
			とした可積分関数$F,G$に対し
			\begin{align}
				F(x)^{1/p}G(x)^{1/q} \leq \frac{1}{p}F(x) + \frac{1}{q}G(x) \quad (\forall x \in X)
			\end{align}
			となるから,両辺を積分して
			\begin{align}
				\int_{X} F(x)^{1/p}G(x)^{1/q}\ m(dx) &\leq \frac{1}{p} \int_{X} F(x)\ m(dx) + \frac{1}{q} \int_{X} G(x)\ m(dx) \\
				&= \frac{1}{p} \frac{1}{\Norm{f}{\mathscr{L}^p}^p} \int_{X} |f(x)|^p\ m(dx) + \frac{1}{q} \frac{1}{\Norm{g}{\mathscr{L}^q}^q} \int_{X} |g(x)|^q\ m(dx) \\
				&= \frac{1}{p} + \frac{1}{q} = 1
			\end{align}
			が成り立つ.最左辺と最右辺を比べて
			\begin{align}
				\int_{X} \frac{|f(x)|}{\Norm{f}{\mathscr{L}^p}} \frac{|g(x)|}{\Norm{g}{\mathscr{L}^q}}\ m(dx) = \int_{X} F(x)^{1/p}G(x)^{1/q}\ m(dx) \leq 1
			\end{align}
			となるから不等式
			\begin{align}
				\int_{X} |f(x)g(x)|\ m(dx) \leq \Norm{f}{\mathscr{L}^p}\Norm{g}{\mathscr{L}^q}
			\end{align}
			が示された.
			\QED
	\end{description}
\end{prf}

\begin{itembox}[l]{}
	\begin{thm}[Minkowskiの不等式]
		$1 \leq p \leq \infty$とする.このとき
		任意の可測$\mathcal{F}/\borel{\K}$関数$f,g$に対して次が成り立つ:
		\begin{align}
			\Norm{f+g}{\mathscr{L}^p} \leq \Norm{f}{\mathscr{L}^p} + \Norm{g}{\mathscr{L}^p}. \label{ineq:minkowski}
		\end{align}
	\end{thm}
\end{itembox}
\begin{prf}\mbox{}
	\begin{description}
		\item[$p = \infty$の場合]
			各点$x \in X$で
			\begin{align}
				|f(x) + g(x)| \leq |f(x)| + |g(x)|
			\end{align}
			となるから,$\Norm{f}{\mathscr{L}^\infty} = \infty$又は$\Norm{g}{\mathscr{L}^\infty} = \infty$なら不等式
			(\refeq{ineq:minkowski})は成り立つ.$\Norm{f}{\mathscr{L}^\infty} < \infty$かつ$\Norm{g}{\mathscr{L}^\infty} < \infty$
			の場合は
			\begin{align}
				C \coloneqq \Set{x \in X}{|f(x)| > \Norm{f}{\mathscr{L}^\infty}} \bigcup \Set{x \in X}{|g(x)| > \Norm{g}{\mathscr{L}^\infty}}
			\end{align}
			が$m$-零集合となり
			\begin{align}
				|f(x) + g(x)| \leq \Norm{f}{\mathscr{L}^\infty} + \Norm{g}{\mathscr{L}^\infty} \quad (\forall x \in X \backslash C)
			\end{align}
			を得て$\Norm{\cdot}{\mathscr{L}^\infty}$の定義と併せて不等式(\refeq{ineq:minkowski})が成り立つ.
		
		\item[$p = 1$の場合]
			\begin{align}
				|f(x) + g(x)| \leq |f(x)| + |g(x)| \quad (\forall x \in X)
			\end{align}
			の両辺を積分することにより不等式(\refeq{ineq:minkowski})が成り立つ.
		
		\item[$1 < p < \infty$の場合]
			$p + q = pq$が成り立つように$q > 1$を取る.
			\begin{align}
				|f(x) + g(x)|^p = |f(x) + g(x)||f(x) + g(x)|^{p-1} \leq |f(x)||f(x) + g(x)|^{p-1} + |g(x)||f(x) + g(x)|^{p-1}
			\end{align}
			の両辺を積分すれば,H\Ddot{o}lderの不等式により
			\begin{align}
				\Norm{f+g}{\mathscr{L}^p}^p &= \int_{X} |f(x) + g(x)|^p\ m(dx) \\
				&\leq \int_{X} |f(x)||f(x) + g(x)|^{p-1}\ m(dx) + \int_{X} |g(x)||f(x) + g(x)|^{p-1}\ m(dx) \\
				&\leq \left( \int_{X} |f(x)|^p\ m(dx) \right)^{1/p} \left( \int_{X} |f(x) + g(x)|^{q(p-1)}\ m(dx) \right)^{1/q} \\
					&\qquad + \left( \int_{X} |g(x)|^p\ m(dx) \right)^{1/p} \left( \int_{X} |f(x) + g(x)|^{q(p-1)}\ m(dx) \right)^{1/q} \\
				&= \left( \int_{X} |f(x)|^p\ m(dx) \right)^{1/p} \left( \int_{X} |f(x) + g(x)|^p\ m(dx) \right)^{1/q} \\
					&\qquad + \left( \int_{X} |g(x)|^p\ m(dx) \right)^{1/p} \left( \int_{X} |f(x) + g(x)|^p\ m(dx) \right)^{1/q} \\
				&= \Norm{f}{\mathscr{L}^p}\Norm{f+g}{\mathscr{L}^p}^{p/q} + \Norm{g}{\mathscr{L}^p}\Norm{f+g}{\mathscr{L}^p}^{p/q} \\
				&= \Norm{f}{\mathscr{L}^p}\Norm{f+g}{\mathscr{L}^p}^{p-1} + \Norm{g}{\mathscr{L}^p}\Norm{f+g}{\mathscr{L}^p}^{p-1}
				\label{Minkowski_1}
			\end{align}
			となる.$\Norm{f+g}{\mathscr{L}^p} = 0$の場合は不等式(\refeq{ineq:minkowski})が成り立つからそうでない場合を考える.
			$\Norm{f+g}{\mathscr{L}^p} = \infty$の場合,
			\begin{align}
				|f(x) + g(x)| \leq |f(x)| + |g(x)| \leq 2 \max{}{\left\{ |f(x)|,|g(x)| \right\}} \quad (\forall x \in X)
			\end{align}
			より
			\begin{align}
				|f(x) + g(x)|^p \leq 2^p \max{}{\left\{ |f(x)|,|g(x)| \right\}} \leq 2^p \left( |f(x)|^p + |g(x)|^p \right) \quad (\forall x \in X)
			\end{align}
			の両辺を積分して
			\begin{align}
				\Norm{f+g}{\mathscr{L}^p}^p \leq 2^p \left( \Norm{f}{\mathscr{L}^p}^p + \Norm{g}{\mathscr{L}^p}^p \right)
			\end{align}
			という関係が出るから,$\Norm{f}{\mathscr{L}^p}$か$\Norm{g}{\mathscr{L}^p}$の一方は$\infty$となり不等式(\refeq{ineq:minkowski})が成り立つ.
			$0 < \Norm{f+g}{\mathscr{L}^p} < \infty$の場合,$\Norm{f}{\mathscr{L}^p} + \Norm{g}{\mathscr{L}^p} = \infty$
			なら不等式(\refeq{ineq:minkowski})は成り立ち,$\Norm{f}{\mathscr{L}^p} + \Norm{g}{\mathscr{L}^p} < \infty$
			の場合でも上式(\refeq{Minkowski_1})より得る
			\begin{align}
				\Norm{f+g}{\mathscr{L}^p}^p \leq \Norm{f}{\mathscr{L}^p}\Norm{f+g}{\mathscr{L}^p}^{p-1} + \Norm{g}{\mathscr{L}^p}\Norm{f+g}{\mathscr{L}^p}^{p-1}
			\end{align}
			の両辺を$\Norm{f+g}{\mathscr{L}^p}^{p-1}$で割って不等式(\refeq{ineq:minkowski})が成り立つ.		
	\end{description}
	\QED
\end{prf}

以上の結果より$\semiLp{p}{X,\mathcal{F},m}$は線形空間となる.実際加法とスカラ倍は
\begin{align}
	(f+g)(x) \coloneqq f(x) + g(x), \quad (\alpha f)(x) \coloneqq \alpha f(x), \quad (\forall x \in X,\ f,g \in \semiLp{p}{X,\mathcal{F},m},\ \alpha \in \C)
\end{align}
により定義され,Minkowskiの不等式により加法について閉じている.

\begin{itembox}[l]{}
	\begin{lem}[$\mathscr{L}^p$のセミノルム]
		$\Norm{\cdot}{\mathscr{L}^p}$は線形空間$\semiLp{p}{X,\mathcal{F},m}$においてセミノルムとなる.
	\end{lem}
\end{itembox}
\begin{prf}\mbox{}
	\begin{description}
	\item[正値性] 定義の仕方による.
	\item[同次性] 
		$1 \leq p < \infty$なら
		\begin{align}
			\left( \int_{X} |\alpha f(x)|^p\ m(dx) \right)^{1/p} = \left( |\alpha|^p \int_{X} |f(x)|^p\ m(dx) \right)^{1/p} 
			= |\alpha| \left( \int_{X} |f(x)|^p\ m(dx) \right)^{1/p}
		\end{align}
		により,また$p = \infty$なら
		\begin{align}
			\inf{}{\Set{r \in \R}{|\alpha f(x)| \leq r,\ \mathrm{a.e.}x \in X}} = |\alpha|\inf{}{\Set{r \in \R}{|f(x)| \leq r,\ \mathrm{a.e.}x \in X}}
		\end{align}
		により,任意の$\alpha \in \K$と任意の$f \in \semiLp{p}{X,\mathcal{F},m}\ (1 \leq p \leq \infty)$に対して
		\begin{align}
			\Norm{\alpha f}{\mathscr{L}^p} = |\alpha|\Norm{f}{\mathscr{L}^p}
		\end{align}
		が成り立つ.
	\item[三角不等式] Minkowskiの不等式による.
	\end{description}
	\QED
\end{prf}

\section{空間$\mathrm{L}^p$}
$\Norm{\cdot}{\mathscr{L}^p}$は$\semiLp{p}{X,\mathcal{F},m}$のノルムとはならない.$\Norm{f}{\mathscr{L}^p} = 0$であっても
$f$が零写像であるとは限らず,実際$m$-零集合の上で
$1 \in \K$を取るような関数$g$でも$\Norm{g}{\mathscr{L}^p} = 0$を満たすからである.
ここで可測関数全体の集合を
\begin{align}
	\mathscr{L}^0 \coloneqq \Set{f:X \rightarrow \K}{f:\mbox{可測}\mathcal{F}/\borel{\K}}
\end{align}
とおく.
\begin{align}
	f,g \in \mathscr{L}^0 ,\quad f \sim g \DEF f(x) = g(x)\quad \mathrm{a.e.}x \in X
\end{align}
と定義した関係$\sim$は$\mathscr{L}^0 $における同値関係となり,この関係で$\mathscr{L}^0 $を割った商集合を$\mathrm{L}^0  \coloneqq \mathscr{L}^0 /\sim$と表す.
$\mathrm{L}^0$の元である関数類(同値類)を$[f]\ $($f$は関数類の代表元)と表し,$\mathrm{L}^0$における加法とスカラ倍を次のように定義すれば$\mathrm{L}^0$は$\K$上の線形空間となる
\footnote{
	この表現はwell-defined,つまり代表元に依らずに値がただ一つに定まる.
	任意の$f' \in [f]$と$g' \in [g]$に対して
	\begin{align}
		[f + g] = [f' + g'],\quad [\alpha f'] = [\alpha f]
	\end{align}
	となるからである.実際
	\begin{align}
		\{ f \neq g \} \coloneqq \Set{x \in X}{f(x) \neq g(x)}
	\end{align}
	と簡略した表記を使えば
	\begin{align}
		&\{ f+g \neq f'+g' \} \subset \{ f \neq f' \} \cup \{ g \neq g' \}, \\
		&\{ \alpha f \neq \alpha f' \} = \{ f \neq f' \}
	\end{align}
	と表現でき,どれも右辺は$m$-零集合であるから$[f + g] = [f' + g'],\ [\alpha f'] = [\alpha f]$が成り立つ.
}:
\begin{align}
	&[f] + [g] \coloneqq [f+g] && (\forall [f],[g] \in \mathrm{L}^0),\\
	&\alpha [f] \coloneqq [\alpha f] && (\forall [f] \in \mathrm{L}^0,\ \alpha \in \K).
\end{align}
また乗法も次のように定義できる:
\begin{align}
	[f][g] \coloneqq [fg] \quad (\forall [f],[g] \in \mathrm{L}^0).
\end{align}
この乗法は可換であり,加法と併せて$\mathrm{L}^0$は可換環をなす.従って関数類に対して加減乗算が自由にできるようになった
\footnote{
	加法乗法が可換であることと結合則は代表元の関数の空間が加法乗法について可換であることと結合則を満たすことによる.
	また分配測は
	\begin{align}
		&([f]+[g])[h] = [f+g][h] = [(f+g)h] = [fh + gh] = [fh] + [gh] = [f][h] + [g][h], \\
		&[h]([f]+[g]) = [h][f+g] = [h(f+g)] = [hf + hg] = [hf] + [hg] = [h][f] + [h][g]
	\end{align}
	により成り立つ.そして零元は零写像の関数類で[0]と表し,単位元は恒等的に$1$を取る関数の関数類で[1]と表す.
	また$-[f] = [-f]$であるから減法は
	\begin{align}
		[f] - [g] \coloneqq [f] + (-[g]) = [f] + [-g] = [f - g]
	\end{align}
	で定義される.
}.
また$[f][f]=[f^2]$を$[f]^2$と表記し,$[f]^n \coloneqq [f]^{n-1}[f]$と帰納的に定義すれば$[f]^n=[f^n]$となる.

次に$\mathrm{L}^0$における順序を定める.
$[f],[g] \in \mathrm{L}^0$に対して
\begin{align}
	[f] \leq [g] \quad \DEF \quad f \leq g \quad \mbox{$m$-a.s.} 
\end{align}
として関係''$\leq$''(記号は$\mathscr{L}^0$におけるものと同じであるが)を定義すればこれは$\mathrm{L}^0$において順序となる.
この定義が代表元の取り方に依存しないことは
$\left\{ f' > g' \right\} \subset \left\{ f \neq f' \right\} \cup \left\{ f > g \right\} \cup \left\{ g \neq g' \right\}$
\footnote{
	$\left\{ f > g \right\} \coloneqq \Set{x \in X}{f(x) > g(x)}$
}
の右辺が零集合であることにより明確であり,また順序関係としての定義を満たしていることは以下で判る:
\begin{itemize}
	\item $f=f$により$[f] \leq [f]$,
	\item $[f] \leq [g]$かつ$[g] \leq [f]$なら$\{ f > g \}$と$\{ g > f \}$は$m$-零集合だから$[f]=[g]$,
	\item $[f] \leq [g]$かつ$[g] \leq [h]$なら$\{ f > h \} \subset \{ f > g \} \cup \{ g > h \}$により$[f]\leq[h]$.
\end{itemize}
	
次に商線型空間$\mathrm{L}^0$におけるノルムを定義する.
\begin{itembox}[l]{}
	\begin{lem}[商空間$\mathrm{L}^p$におけるノルムの定義]
		\begin{align}
			\Norm{[f]}{\mathrm{L}^p} \coloneqq \Norm{f}{\mathscr{L}^p} \quad (1 \leq p \leq \infty,\ f \in \semiLp{p}{X,\mathcal{F},m})
		\end{align}
		として$\Norm{\cdot}{\mathrm{L}^p}:\mathrm{L}^0 \rightarrow \R$を定義すればこれはwell-definedである.つまり代表元に依らずに値がただ一つに定まる.
		更に次で定義する空間
		\begin{align}
			\Lp{p}{X,\mathcal{F},m} \coloneqq \Set{[f] \in \mathrm{L}^0}{\Norm{[f]}{\mathrm{L}^p} < \infty} \quad (1 \leq p \leq \infty)
		\end{align}
		は$\Norm{\cdot}{\mathrm{L}^p}$をノルムとしてノルム空間となる.
	\end{lem}
\end{itembox}
\begin{prf}\mbox{}
	\begin{description}
		\item[well-definedであること]
			$f \in \semiLp{p}{X,\mathcal{F},m}$に対し,任意に$g \in [f]$を選ぶ.
			示すことは$\Norm{f}{\mathscr{L}^p}^p = \Norm{g}{\mathscr{L}^p}^p$が成り立つことである.
			\begin{align}
				A \coloneqq \Set{x \in X}{f(x) \neq g(x)} \quad \in \mathcal{F}
			\end{align}
			として$m$-零集合を用意する.
			\begin{description}
				\item[$p = \infty$の場合]
					$A^c$の上で$f(x)=g(x)$となるから
					\begin{align}
						\Set{x \in X}{|g(x)| > \Norm{f}{\mathscr{L}^\infty}} 
						&\subset A + A^c \cap \Set{x \in X}{|g(x)| > \Norm{f}{\mathscr{L}^\infty}} \\
						&= A + A^c \cap \Set{x \in X}{|f(x)| > \Norm{f}{\mathscr{L}^\infty}} \\
						&\subset A + \Set{x \in X}{|f(x)| > \Norm{f}{\mathscr{L}^\infty}}
					\end{align}
					が成り立ち,最右辺は2項とも$m$-零集合であるから$\Norm{g}{\mathscr{L}^\infty} \leq \Norm{f}{\mathscr{L}^\infty}$が従う.
					逆向きの不等号も同様に示されて$\Norm{g}{\mathscr{L}^\infty} = \Norm{f}{\mathscr{L}^\infty}$を得る.
				\item[$1 \leq p < \infty$の場合]
					$m(A)=0$により
					\begin{align}
						\Norm{f}{\mathscr{L}^p}^p = \int_X f(x)\ m(dx) = \int_{A^c} f(x)\ m(dx) = \int_{A^c} g(x)\ m(dx) = \int_X g(x)\ m(dx) = \Norm{g}{\mathscr{L}^p}^p
					\end{align}
					が成り立つ.
			\end{description}
		
		\item[ノルムとなること]
			任意に$[f],[g] \in \Lp{p}{X,\mathcal{F},m}$と$\alpha \in \K$を取る.
			$\Norm{[f]}{\mathrm{L}^p}$の正値性は$\Norm{\cdot}{\mathscr{L}^p}$の正値性から従う.
			また$m(f \neq 0) > 0$なら$\Norm{f}{\mathscr{L}^p} > 0$となるから,
			対偶により$\Norm{[f]}{\mathrm{L}^p} = 0$なら$[f]$は零元$[0]$,
			逆に$[f] = [0]$なら$\Norm{[f]}{\mathrm{L}^p} = \Norm{0}{\mathscr{L}^p} = 0$となる.
			$\Norm{\cdot}{\mathscr{L}^p}$の同次性とMinkowskiの不等式から
			\begin{align}
				&\Norm{\alpha[f]}{\mathrm{L}^p} = \Norm{[\alpha f]}{\mathrm{L}^p} = \Norm{\alpha f}{\mathscr{L}^p} = |\alpha|\Norm{f}{\mathscr{L}^p} = |\alpha|\Norm{[f]}{\mathrm{L}^p} \\
				&\Norm{[f] + [g]}{\mathrm{L}^p} = \Norm{[f + g]}{\mathrm{L}^p} = \Norm{f + g}{\mathscr{L}^p} \leq \Norm{f}{\mathscr{L}^p} + \Norm{g}{\mathscr{L}^p} = \Norm{[f]}{\mathrm{L}^p} + \Norm{[g]}{\mathrm{L}^p}
			\end{align}
			も成り立ち,$\Norm{\cdot}{\mathrm{L}^p}$が$\Lp{p}{X,\mathcal{F},m}$においてノルムとなると示された.
	\end{description}
	\QED
\end{prf}

\begin{itembox}[l]{}
	\begin{prp}[$\mathrm{L}^p$の完備性]
		上で定義したノルム空間$\Lp{p}{X,\mathcal{F},m}\ (1 \leq p \leq \infty)$はBanach空間である.
	\end{prp}
\end{itembox}
\begin{prf}
	任意に$\Lp{p}{X,\mathcal{F},m}$のCauchy列$[f_n] \in \Lp{p}{X,\mathcal{F},m}\ (n=1,2,3,\cdots)$を取る.
	Cauchy列であるから$1/2$に対して或る$N_1 \in \N$が取れて,$n>m \geq N_1$ならば
	$\Norm{[f_n]-[f_m]}{\mathrm{L}^p} = \Norm{[f_n - f_m]}{\mathrm{L}^p} < 1/2$となる.
	ここで$m = n_1$と表記することにする.
	同様に$1/2^2$に対して或る$N_2 \in \N\ (N_2 > N_1)$が取れて,$n'>m' \geq N_2$ならば
	$\Norm{[f_{n'} - f_{m'}]}{\mathrm{L}^p} < 1/2^2$となる.
	先ほどの$n$について,$n > N_2$となるように取れるからこれを$n = n_2$と表記し,更に$m' = n_2$ともしておく.今のところ
	\begin{align}
		\Norm{[f_{n_1} - f_{n_2}]}{\mathrm{L}^p} < 1/2
	\end{align}
	と表示できる.再び同様に$1/2^3$に対して或る$N_3 \in \N\ (N_3 > N_2)$が取れて,$n''>m'' \geq N_2$ならば
	$\Norm{[f_{n''} - f_{m''}]}{\mathrm{L}^p} < 1/2^3$となる.
	先ほどの$n'$について$n' > N_3$となるように取れるからこれを$n' = n_3$と表記し,更に$m'' = n_3$ともしておく.今までのところで
	\begin{align}
		&\Norm{[f_{n_1} - f_{n_2}]}{\mathrm{L}^p} < 1/2 \\
		&\Norm{[f_{n_2} - f_{n_3}]}{\mathrm{L}^p} < 1/2^2
	\end{align}
	が成り立っている.数学的帰納法により
	\begin{align}
		\Norm{[f_{n_k} - f_{n_{k+1}}]}{\mathrm{L}^p} < 1/2^k \quad (n_{k+1} > n_k,\ k=1,2,3,\cdots) \label{ineq:Lp_banach_2}
	\end{align}
	が成り立つように自然数の部分列$(n_k)_{k=1}^{\infty}$を取ることができる.
	\begin{description}
		\item[$p = \infty$の場合]\mbox{}\\
			$[f_{n_k}]$の代表元$f_{n_k}$について,
			\begin{align}
				A_k &\coloneqq \Set{x \in X}{|f_{n_k}(x)| > \Norm{f_{n_k}}{\mathscr{L}^\infty}}, \\
				A^k &\coloneqq \Set{x \in X}{|f_{n_k}(x) - f_{n_{k+1}}(x)| > \Norm{f_{n_k} - f_{n_{k+1}}}{\mathscr{L}^\infty}}
			\end{align}
			とおけばH\Ddot{o}lderの不等式の証明中の補助定理より$m(A_k) = m(A^k) = 0$であり,
			\begin{align}
				A \coloneqq \left( \bigcup_{k=1}^{\infty} A_k \right) \bigcup \left( \bigcup_{k=1}^{\infty} A^k \right)
			\end{align}
			として$m$-零集合を定め
			\begin{align}
				\hat{f}_{n_k}(x) =
				\begin{cases}
					f_{n_k}(x) & (x \notin A) \\
					0 & (x \in A)
				\end{cases}
				\quad (\forall x \in X)
			\end{align}
			と定義した$\hat{f}_{n_k}$もまた$[f_{n_k}]$の元となる.代表元を$f_{n_k}$に替えて$\hat{f}_{n_k}$とすれば,
			$\hat{f}_{n_k}$は$X$上の有界可測関数であり
			\begin{align}
				\Norm{\hat{f}_{n_k} - \hat{f}_{n_{k+1}}}{\mathscr{L}^\infty} = \sup{x \in X}{|\hat{f}_{n_k}(x) - \hat{f}_{n_{k+1}}(x)|} < 1/2^k \quad (k=1,2,3,\cdots) 
				\label{ineq:Lp_banach_1}
			\end{align}
			が成り立っていることになるから,各点$x \in X$で$\left( \hat{f}_{n_k}(x) \right)_{k=1}^{\infty}$は$\K$のCauchy列となる.
			(これは$\sum_{k > N} 1/2^k = 1/2^N \longrightarrow 0\ (N \longrightarrow \infty)$による.)従って各点$x \in X$
			で極限が存在するからこれを$\hat{f}(x)$として表す.一般に距離空間に値を取る可測関数列の各点収束の極限関数は可測関数であるから
			$\hat{f}$もまた可測$\mathcal{F}/\borel{\K}$である.また$\hat{f}$は有界である.これは次のように示される.
			式(\refeq{ineq:Lp_banach_1})から任意の$l > k$に対し
			\begin{align}
				|\hat{f}_{n_k}(x) - \hat{f}_{n_l}(x)| \leq \sum_{j=k}^{l-1} |\hat{f}_{n_{j}}(x) - \hat{f}_{n_{j+1}}(x)| 
				\leq \sum_{j=k}^{l-1} \sup{x \in X}{|\hat{f}_{n_j}(x) - \hat{f}_{n_{j+1}}(x)|} < 1/2^{k-1}
			\end{align}
			が成り立つから,極限関数$\hat{f}(x)$も
			\begin{align}
				\sup{x \in X}{|\hat{f}_{n_k}(x) - \hat{f}(x)|} \leq 1/2^{k-1} \label{ineq:Lp_banach_3}
			\end{align}
			を満たすことになる.なぜなら,もし或る$x \in X$で$\alpha \coloneqq |\hat{f}_{n_k}(x) - \hat{f}(x)| > 1/2^{k-1}$となる場合,
			任意の$l > k$に対し
			\begin{align}
				0 < \alpha - 1/2^{k-1} < |\hat{f}_{n_k}(x) - \hat{f}(x)| - |\hat{f}_{n_k}(x) - \hat{f}_{n_l}(x)| \leq |\hat{f}_{n_l}(x) - \hat{f}(x)|
			\end{align}
			となり各点収束に反するからである.不等式(\refeq{ineq:Lp_banach_3})により任意の$x \in X$において
			\begin{align}
				|\hat{f}(x)| < |\hat{f}_{n_k}(x)| + 1/2^{k-1} \leq \Norm{\hat{f}_{n_k}}{\mathscr{L}^\infty} + 1/2^{k-1}
			\end{align}
			が成り立ち$\hat{f}$の有界性が判る.以上で極限関数$\hat{f}$が有界可測関数であると示された.
			$\hat{f}$を代表元とする$[\hat{f}] \in \Lp{\infty}{X,\mathcal{F},m}$に対し,不等式(\refeq{ineq:Lp_banach_3})により
			\begin{align}
				\Norm{[f_{n_k}] - [\hat{f}]}{\mathrm{L}^\infty} = \Norm{\hat{f}_{n_k} - \hat{f}}{\mathscr{L}^\infty} 
				= \sup{x \in X}{|\hat{f}_{n_k}(x) - \hat{f}(x)|}
				\longrightarrow 0 \ (k \longrightarrow \infty)
			\end{align}
			が成り立つから,Cauchy列$\left( [f_{n}] \right)_{n=1}^{\infty}$の部分列$\left( [f_{n_k}] \right)_{k=1}^{\infty}$が$[\hat{f}]$に収束すると示された.
			Cauchy列の部分列が収束すれば,元のCauchy列はその部分列と同じ収束先に収束するから$\Lp{\infty}{X,\mathcal{F},m}$はBanach空間である.
			
		\item[$1 \leq p < \infty$の場合]\mbox{}\\
			$[f_{n_k}]$の代表元$f_{n_k}$に対して
			\begin{align}	
				f_{n_k}(x) &\coloneqq f_{n_1}(x) + \sum_{j=1}^{k}(f_{n_j}(x) - f_{n_{j-1}}(x)) \label{eq:Lp_banach_3}
			\end{align}
			と表現できるから,これに対して
			\begin{align}
				g_k(x) &\coloneqq |f_{n_1}(x)| + \sum_{j=1}^{k}|f_{n_j}(x) - f_{n_{j-1}}(x)|
			\end{align}
			として可測関数列$(g_k)_{k=1}^{\infty}$を用意する.Minkowskiの不等式と式(\refeq{ineq:Lp_banach_2})より
			\begin{align}
				\Norm{g_k}{\mathscr{L}^p} \leq \Norm{f_{n_1}}{\mathscr{L}^p} + \sum_{j=1}^{k}\Norm{f_{n_j} - f_{n_{j-1}}}{\mathscr{L}^p}
				< \Norm{f_{n_1}}{\mathscr{L}^p} + \sum_{j=1}^{k} 1/2^j < \Norm{f_{n_1}}{\mathscr{L}^p} + 1 < \infty
			\end{align}
			が成り立つ.各点$x \in X$で$g_k(x)$は$k$について単調増大であるから,単調収束定理より
			\begin{align}
				\Norm{g}{\mathscr{L}^p}^p = \lim_{k \to \infty} \Norm{g_k}{\mathscr{L}^p}^p < \Norm{f_{n_1}}{\mathscr{L}^p} + 1 < \infty
			\end{align}
			となるので$g \in \Lp{p}{X,\mathcal{F},m}$である.従って
			\begin{align}
				B_n &\coloneqq \Set{x \in X}{g(x) \leq n} \in \mathcal{F}, \\
				B &\coloneqq \bigcup_{n=1}^{\infty} B_n
			\end{align}
			とおけば$m(X \backslash B) = 0$であり,式(\refeq{eq:Lp_banach_3})の級数は$B$上で絶対収束する(各点).
			\begin{align}
				f(x) \coloneqq
				\begin{cases}
					\lim\limits_{k \to \infty} f_{n_k}(x) & (x \in B) \\
					0 & (x \in X \backslash B)
				\end{cases}
			\end{align}
			として可測$\mathcal{F}/\borel{\R}$関数$f$を定義すれば,$|f(x)| \leq g(x)\ (\forall x \in X)$と
			$g^p$が可積分であることから$f$を代表元とする関数類$[f]$は$\Lp{p}{X,\mathcal{F},m}$の元となる.
			関数列$(\left( f_{n_k} \right)_{k=1}^{\infty})$は$f$に概収束し,
			$|f_{n_k}(x) - f(x)|^p \leq 2^p(|f_{n_k}(x)|^p + |f(x)|^p) \leq 2^{p+1} g(x)^p\ (\forall x \in X)$となるから
			Lebesgueの収束定理により
			\begin{align}
				\lim_{k \to \infty}\Norm{[f_{n_k}] - [f]}{\mathrm{L}^p}^p
				= \lim_{k \to \infty}\Norm{f_{n_k} - f}{\mathscr{L}^p}^p
				= \lim_{k \to \infty} \int_X |f_{n_k}(x) - f(x)|^p\ m(dx) = 0
			\end{align}
			が成り立ち,Cauchy列$\left( [f_{n}] \right)_{n=1}^{\infty}$の部分列$\left( [f_{n_k}] \right)_{k=1}^{\infty}$が$[f]$に収束すると示された.
			Cauchy列の部分列が収束すれば,元のCauchy列はその部分列と同じ収束先に収束するから$\Lp{p}{X,\mathcal{F},m}$はBanach空間である.
	\end{description}
	\QED
\end{prf}

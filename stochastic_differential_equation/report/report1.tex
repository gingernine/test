係数体を$\R$,測度空間を$(X,\mathcal{F},\mu)$とする.

\section{H\Ddot{o}lderの不等式とMinkowskiの不等式}
$\mathcal{F}/\borel{\R}$-可測関数$f$に対して
\begin{align}
	\Norm{f}{\mathscr{L}^p} \coloneqq
	\begin{cases}
		\inf{}{\Set{r \in \R}{|f(x)| \leq r\quad \mbox{$\mu$-a.e.}x \in X}} & (p = \infty) \\
		\left( \int_{X} |f(x)|^p\ \mu(dx) \right)^{\frac{1}{p}} & (0 < p < \infty)
	\end{cases}
\end{align}
により$\Norm{\cdot}{\mathscr{L}^p}$を定める.以後は$\Norm{\cdot}{\semiLp{p}{\Omega,\mathcal{F},\mu}}$
或は$\Norm{\cdot}{\semiLp{p}{\mu}}$と表記することもある.
\begin{align}
	\semiLp{p}{X,\mathcal{F},\mu} \coloneqq \Set{f:X \rightarrow \R}{f:\mbox{可測}\mathcal{F}/\borel{\R},\ \Norm{f}{\mathscr{L}^p} < \infty} \quad (1 \leq p \leq \infty)
\end{align}
として空間$\semiLp{p}{X,\mathcal{F},\mu}$を定義し,これも以後は$\semiLp{p}{\mu},\semiLp{p}{\mathcal{F},\mu},
\semiLp{p}{\mathcal{F}}$などと略記することもある.以下に示す不等式により
$\semiLp{p}{X,\mathcal{F},\mu}$は$\R$上の線形空間となる.

\begin{screen}
	\begin{lem}
		任意の$f \in \semiLp{\infty}{X,\mathcal{F},\mu}$に対して次が成り立つ:
		\begin{align}
			|f(x)| \leq \Norm{f}{\mathscr{L}^\infty} \quad \mbox{$\mu$-a.e.}x \in X.
		\end{align}
		\label{lem:holder_inequality}
	\end{lem}
\end{screen}

\begin{prf}
	$\semiLp{\infty}{X,\mathcal{F},\mu}$の定義より任意の実数$\alpha > \Norm{f}{\mathscr{L}^\infty}$に対して
	\begin{align}
		\mu\left( \Set{x \in X}{|f(x)| > \alpha} \right) = 0
	\end{align}
	が成り立つから,
	\begin{align}
		\Set{x \in X}{|f(x)| > \Norm{f}{\mathscr{L}^\infty}} = \bigcup_{n =1}^{\infty} \Set{x \in X}{|f(x)| > \Norm{f}{\mathscr{L}^\infty} + 1/n}
	\end{align}
	の右辺は$\mu$-零集合であり主張が従う.
	\QED
\end{prf}

\begin{screen}
	\begin{thm}[H\Ddot{o}lderの不等式]
		$1 \leq p, q \leq \infty$,$p + q = pq\ (p = \infty$なら$q = 1)$とする.このとき
		任意の$\mathcal{F}/\borel{\R}$-可測関数$f,g$に対して次が成り立つ:
		\begin{align}
			\int_{X} |f(x)g(x)|\ \mu(dx) \leq \Norm{f}{\mathscr{L}^p} \Norm{g}{\mathscr{L}^q}. \label{ineq:holder}
		\end{align}
		\label{thm:holder_inequality}
	\end{thm}
\end{screen}

\begin{prf}
	$\Norm{f}{\mathscr{L}^p} = \infty$又は$\Norm{g}{\mathscr{L}^q} = \infty$のとき不等式(\refeq{ineq:holder})
		は成り立つから,以下では$\Norm{f}{\mathscr{L}^p} < \infty$かつ$\Norm{g}{\mathscr{L}^q} < \infty$の場合を考える.
	\begin{description}
		\item[$p = \infty,\ q = 1$の場合]
			補題\ref{lem:holder_inequality}により或る零集合$A$が存在して
			\begin{align}
				|f(x)g(x)| \leq \Norm{f}{\mathscr{L}^\infty}|g(x)| \quad (\forall x \in X \backslash A).
			\end{align}
			が成り立つから,
			\begin{align}
				&\int_{X} |f(x)g(x)|\ \mu(dx) = \int_{X \backslash A} |f(x)g(x)|\ \mu(dx) \\
				&\qquad \leq \Norm{f}{\mathscr{L}^\infty} \int_{X \backslash A} |g(x)|\ \mu(dx) 
				= \Norm{f}{\mathscr{L}^\infty} \Norm{g}{\mathscr{L}^1}
			\end{align}
			が従い不等式(\refeq{ineq:holder})を得る.
		
		\item[$1 < p,q < \infty$の場合]
			$\Norm{f}{\mathscr{L}^p} = 0$のとき
			\begin{align}
				B \coloneqq \Set{x \in X}{|f(x)| > 0}
			\end{align}
			は零集合であるから,
			\begin{align}
				\int_{X} |f(x)g(x)|\ \mu(dx) = \int_{X \backslash B} |f(x)g(x)|\ \mu(dx) = 0
			\end{align}
			となり(\refeq{ineq:holder})を得る.$\Norm{g}{\mathscr{L}^q} = 0$の場合も同じである.
			次に$0 < \Norm{f}{\mathscr{L}^p},\Norm{g}{\mathscr{L}^q} < \infty$の場合を示す.
			実数値対数関数$(0,\infty) \ni t \longmapsto -\Log{t}$は凸であるから,$1/p + 1/q = 1$に対して
			\begin{align}
				-\Log{\left( \frac{s}{p} + \frac{t}{q} \right)} \leq \frac{1}{p}(-\Log{s}) + \frac{1}{q}(-\Log{t}) \quad (\forall s,t > 0)
			\end{align}
			を満たし
			\begin{align}
				s^{\frac{1}{p}}t^{\frac{1}{q}} \leq \frac{s}{p} + \frac{t}{q} \quad (\forall s,t > 0)
			\end{align}
			が従う.ここで
			\begin{align}
				F(x) \coloneqq \frac{|f(x)|^p}{\Norm{f}{\mathscr{L}^p}^p},
				\quad G(x) \coloneqq \frac{|g(x)|^q}{\Norm{g}{\mathscr{L}^q}^q} \quad (\forall x \in X)
			\end{align}
			により可積分関数$F,G$を定めれば,
			\begin{align}
				F(x)^{\frac{1}{p}}G(x)^{\frac{1}{q}} \leq \frac{1}{p}F(x) + \frac{1}{q}G(x) \quad (\forall x \in X)
			\end{align}
			が成り立つから
			\begin{align}
				&\int_{X} \frac{|f(x)|}{\Norm{f}{\mathscr{L}^p}} \frac{|g(x)|}{\Norm{g}{\mathscr{L}^q}}\ \mu(dx)
				= \int_{X} F(x)^{\frac{1}{p}}G(x)^{\frac{1}{q}}\ \mu(dx) \\
				&\qquad \leq \frac{1}{p} \int_{X} F(x)\ \mu(dx) + \frac{1}{q} \int_{X} G(x)\ \mu(dx)
				= \frac{1}{p} + \frac{1}{q} = 1
			\end{align}
			が従い,$\Norm{f}{\mathscr{L}^p}\Norm{g}{\mathscr{L}^q}$を移項して不等式(\refeq{ineq:holder})を得る.
			\QED
	\end{description}
\end{prf}

\begin{screen}
	\begin{thm}[Minkowskiの不等式]
		$1 \leq p \leq \infty$とする.このとき
		任意の$\mathcal{F}/\borel{\R}$-可測関数$f,g$に対して次が成り立つ:
		\begin{align}
			\Norm{f+g}{\mathscr{L}^p} \leq \Norm{f}{\mathscr{L}^p} + \Norm{g}{\mathscr{L}^p}. \label{ineq:minkowski}
		\end{align}
		\label{thm:minkowski_inequality}
	\end{thm}
\end{screen}

\begin{prf}
	$\Norm{f+g}{\mathscr{L}^p} = 0,\ \Norm{f}{\mathscr{L}^p} = \infty,\ \Norm{g}{\mathscr{L}^p} = \infty$
	のいずれかが満たされているとき不等式(\refeq{ineq:minkowski})は成り立つから,以下では
	$\Norm{f+g}{\mathscr{L}^p} > 0,\ \Norm{f}{\mathscr{L}^p} < \infty,\ \Norm{g}{\mathscr{L}^p} < \infty$
	の場合を考える.
	\begin{description}
		\item[$p = \infty$の場合]
			補題\ref{lem:holder_inequality}により
			\begin{align}
				C \coloneqq \Set{x \in X}{|f(x)| > \Norm{f}{\mathscr{L}^\infty}} \cup \Set{x \in X}{|g(x)| > \Norm{g}{\mathscr{L}^\infty}}
			\end{align}
			は零集合であり,
			\begin{align}
				|f(x) + g(x)| \leq |f(x)| + |g(x)| \leq \Norm{f}{\mathscr{L}^\infty} + \Norm{g}{\mathscr{L}^\infty} \quad (\forall x \in X \backslash C)
			\end{align}
			が成り立つ.$\Norm{\cdot}{\mathscr{L}^\infty}$の定義より不等式(\refeq{ineq:minkowski})を得る.
		
		\item[$p = 1$の場合]
			\begin{align}
				|f(x) + g(x)| \leq |f(x)| + |g(x)| \quad (\forall x \in X)
			\end{align}
			の両辺を積分して不等式(\refeq{ineq:minkowski})を得る.
		
		\item[$1 < p < \infty$の場合]
			$p + q = pq$が成り立つように$q > 1$を取る.各点$x \in X$で
			\begin{align}
				|f(x) + g(x)|^p = |f(x) + g(x)||f(x) + g(x)|^{p-1} \leq |f(x)||f(x) + g(x)|^{p-1} + |g(x)||f(x) + g(x)|^{p-1}
			\end{align}
			が成り立つから,両辺を積分すればH\Ddot{o}lderの不等式により
			\begin{align}
				\Norm{f+g}{\mathscr{L}^p}^p &= \int_{X} |f(x) + g(x)|^p\ \mu(dx) \\
				&\leq \int_{X} |f(x)||f(x) + g(x)|^{p-1}\ \mu(dx) + \int_{X} |g(x)||f(x) + g(x)|^{p-1}\ \mu(dx) \\
				&\leq \Norm{f}{\mathscr{L}^p}\Norm{f+g}{\mathscr{L}^p}^{p-1} + \Norm{g}{\mathscr{L}^p}\Norm{f+g}{\mathscr{L}^p}^{p-1}
				\label{Minkowski_1}
			\end{align}
			が得られる.また$|f|^p,|g|^p$の可積性と
			\begin{align}
				|f(x) + g(x)|^p \leq \left(|f(x)| + |g(x)|\right)^p \leq 2^p \left( |f(x)|^p + |g(x)|^p \right) \quad (\forall x \in X)
			\end{align}
			により$\Norm{f+g}{\mathscr{L}^p} < \infty$が従うから,
			(\refeq{Minkowski_1})の両辺を$\Norm{f+g}{\mathscr{L}^p}^{p-1}$で割って(\refeq{ineq:minkowski})を得る.
			\QED
	\end{description}
\end{prf}

以上の結果より$\semiLp{p}{X,\mathcal{F},\mu}$は線形空間となる.実際線型演算は
\begin{align}
	(f+g)(x) \coloneqq f(x) + g(x), \quad (\alpha f)(x) \coloneqq \alpha f(x), \quad (\forall x \in X,\ f,g \in \semiLp{p}{X,\mathcal{F},\mu},\ \alpha \in \C)
\end{align}
により定義され,Minkowskiの不等式により加法について閉じている.

\begin{screen}
	\begin{lem}
		$1 \leq p \leq \infty$に対し,$\Norm{\cdot}{\mathscr{L}^p}$は線形空間$\semiLp{p}{X,\mathcal{F},\mu}$のセミノルムである.
	\end{lem}
\end{screen}

\begin{prf}\mbox{}
	\begin{description}
	\item[半正値性] $\Norm{\cdot}{\mathscr{L}^p}$が正値であることは定義による.
		しかし$\Norm{f}{\mathscr{L}^p} = 0$であっても$f$が零写像であるとは限らず,実際$\mu$-零集合$E$を取り
		\begin{align}
			f(x) \coloneqq
			\begin{cases}
				1 & (x \in E) \\
				0 & (x \in \Omega \backslash E)
			\end{cases}
		\end{align}
		により$f$を定めれば$\Norm{f}{\mathscr{L}^p} = 0$が成り立つ.
		
	\item[同次性] 
		任意に$\alpha \in \R,\ f \in \semiLp{p}{X,\mathcal{F},\mu}$を取る.
		$1 \leq p < \infty$の場合は
		\begin{align}
			\left( \int_{X} |\alpha f(x)|^p\ \mu(dx) \right)^{1/p} = \left( |\alpha|^p \int_{X} |f(x)|^p\ \mu(dx) \right)^{1/p} 
			= |\alpha| \left( \int_{X} |f(x)|^p\ \mu(dx) \right)^{1/p}
		\end{align}
		により,$p = \infty$の場合は
		\begin{align}
			\inf{}{\Set{r \in \R}{|\alpha f(x)| \leq r \quad \mbox{$\mu$-a.e.}x \in X}} = |\alpha|\inf{}{\Set{r \in \R}{|f(x)|  \leq r \quad \mbox{$\mu$-a.e.}x \in X}}
		\end{align}
		により$\Norm{\alpha f}{\mathscr{L}^p} = |\alpha|\Norm{f}{\mathscr{L}^p}$が成り立つ.
		
	\item[三角不等式] Minkowskiの不等式による.
	\QED
	\end{description}
\end{prf}

\section{空間$\mathrm{L}^p$}
\begin{description}
	\item[可測関数全体の商集合]
		$\mathcal{F}/\borel{R}$-可測関数全体の集合を
		\begin{align}
			\semiLp{0}{X,\mathcal{F},\mu} \coloneqq \Set{f:X \rightarrow \R}{f:\mbox{可測}\mathcal{F}/\borel{\R}}
		\end{align}
		とおく.二元$f,g \in \semiLp{0}{X,\mathcal{F},\mu}$に対し
		\begin{align}
			 f \sim g \quad \DEF \quad f = g \quad \mbox{$\mu$-a.e.}
		\end{align}
		により定める$\sim$は同値関係であり,$\sim$による$\semiLp{0}{X,\mathcal{F},\mu}$の商集合を$\Lp{0}{X,\mathcal{F},\mu}$と表す.
	
	\item[商集合における算法]
		$\Lp{0}{X,\mathcal{F},\mu}$の元である関数類(同値類)を$[f]\ $($f$は関数類の代表)と表す.
		$\Lp{0}{X,\mathcal{F},\mu}$における線型演算を次で定義すれば,$\Lp{0}{X,\mathcal{F},\mu}$は$\R$上の線形空間となる:
		\begin{align}
			&[f] + [g] \coloneqq [f+g] && (\forall [f],[g] \in \Lp{0}{X,\mathcal{F},\mu}),\\
			&\alpha [f] \coloneqq [\alpha f] && (\forall [f] \in \Lp{0}{X,\mathcal{F},\mu},\ \alpha \in \R).
		\end{align}
		この演算はwell-definedである.実際任意の$f' \in [f]$と$g' \in [g]$に対して
		\begin{align}
			\{ f+g \neq f'+g' \} \subset \{ f \neq f' \} \cup \{ g \neq g' \}, \quad
			\{ \alpha f \neq \alpha f' \} = \{ f \neq f' \}
		\end{align}
		が成り立ち
		\footnote{
			$\{ f \neq g \} \coloneqq \Set{x \in X}{f(x) \neq g(x)}.$
		}
		,どちらの右辺も$\mu$-零集合であるから$[f + g] = [f' + g'],\ [\alpha f'] = [\alpha f]$が従う.
		更に乗法を次により定義すれば$\Lp{0}{X,\mathcal{F},\mu}$は多元環になる:
		\begin{align}
			[f][g] \coloneqq [fg] \quad \left( \forall [f],[g] \in \Lp{0}{X,\mathcal{F},\mu} \right).
		\end{align}
		$\Lp{0}{X,\mathcal{F},\mu}$における零元は零写像の関数類でありこれを[0]と表す.また
		単位元は恒等的に$1$を取る関数の関数類でありこれを[1]と表す.
		また減法を
		\begin{align}
			[f] - [g] \coloneqq [f] + (-[g]) = [f] + [-g] = [f - g]
		\end{align}
		により定める.
	
	\item[関数類の順序]
		$[f],[g] \in \Lp{0}{X,\mathcal{F},\mu}$に対して次の関係$<(>)$を定める:
		\begin{align}
			[f] < [g]\ \left( [g] > [f] \right) \quad \DEF \quad f < g \quad \mbox{$\mu$-a.s.} \label{dfn:equiv_class_order}
		\end{align}
		この定義はwell-definedである.実際任意の$f' \in [f],g' \in [g]$に対して
		\begin{align}
			\left\{ f' \geq g' \right\} \subset \left\{ f \neq f' \right\} \cup \left\{ f \geq g \right\} \cup \left\{ g \neq g' \right\}
		\end{align}
		の右辺は零集合であるから
		\begin{align}
			[f] < [g] \Leftrightarrow [f'] < [g']
		\end{align}
		が従う.$<$または$=$であることを$\leq$と表し,同様に$\geq$を定める.
		この関係$\leq$は次に示す規則を満たし,$\Lp{0}{X,\mathcal{F},\mu}$における順序となる:
		\ 任意の$[f],[g],[h] \in \Lp{0}{X,\mathcal{F},\mu}$に対し,
		\begin{itemize}
			\item $[f] \leq [f]$が成り立つ.
			\item $[f] \leq [g]$かつ$[g] \leq [f]$ならば$[f] = [g]$が成り立つ.
			\item $[f] \leq [g],\ [g] \leq [h]$ならば$[f] \leq [h]$が成り立つ.
		\end{itemize}
		
	\item[関数類の冪・絶対値]
		$[f] \in \Lp{0}{X,\mathcal{F},\mu}$が$[f] \geq [0]$を満たすとき,$p \geq 0$に対し
		$[f]^p \coloneqq [f^p]$により冪乗を定める.また$|[f]| \coloneqq [|f|]$として絶対値$|\cdot|$を定める.
\end{description}

\begin{screen}
	\begin{lem}[商空間におけるノルムの定義]
		\begin{align}
			\Norm{[f]}{\mathrm{L}^p} \coloneqq \Norm{f}{\mathscr{L}^p} \quad (f \in \semiLp{p}{X,\mathcal{F},\mu},\ 1 \leq p \leq \infty)
		\end{align}
		により定める$\Norm{\cdot}{\mathrm{L}^p}:\Lp{0}{X,\mathcal{F},\mu} \rightarrow \R$は関数類の代表に依らずに値が確定する.
		そして
		\begin{align}
			\Lp{p}{X,\mathcal{F},\mu} \coloneqq \Set{[f] \in \Lp{0}{X,\mathcal{F},\mu}}{\Norm{[f]}{\mathrm{L}^p} < \infty} \quad (1 \leq p \leq \infty)
		\end{align}
		として定める空間は$\Norm{\cdot}{\mathrm{L}^p}$をノルムとしてノルム空間となる.
	\end{lem}
\end{screen}

\begin{prf}\mbox{}
	\begin{description}
		\item[第一段]
			任意の$f,g \in \semiLp{p}{X,\mathcal{F},\mu}$に対し,$[f] = [g]$なら
			$\Norm{f}{\mathscr{L}^p}^p = \Norm{g}{\mathscr{L}^p}^p$となることを示す.
			\begin{align}
				A \coloneqq \Set{x \in X}{f(x) \neq g(x)}
			\end{align}
			として零集合を定める.
			\begin{description}
				\item[$p = \infty$の場合]
					$A^c$の上では$f(x)=g(x)$が満たされているから
					\begin{align}
						&\Set{x \in X}{|g(x)| > \Norm{f}{\mathscr{L}^\infty}} \\
						&\qquad = A \cap \Set{x \in X}{|g(x)| > \Norm{f}{\mathscr{L}^\infty}} + A^c \cap \Set{x \in X}{|f(x)| > \Norm{f}{\mathscr{L}^\infty}}
					\end{align}
					が成り立ち,右辺はどちらも零集合であるから$\Norm{g}{\mathscr{L}^\infty} \leq \Norm{f}{\mathscr{L}^\infty}$が従う.
					$f,g$を入れ替えれば$\Norm{f}{\mathscr{L}^\infty} \leq \Norm{g}{\mathscr{L}^\infty}$も示されて
					$\Norm{g}{\mathscr{L}^\infty} = \Norm{f}{\mathscr{L}^\infty}$を得る.
					
				\item[$1 \leq p < \infty$の場合]
					\begin{align}
						\Norm{f}{\mathscr{L}^p}^p = \int_{X \backslash A} |f(x)|^p\ \mu(dx) 
						= \int_{X \backslash A} |g(x)|^p\ \mu(dx) = \Norm{g}{\mathscr{L}^p}^p
					\end{align}
					が成り立つ.
			\end{description}
		
		\item[第二段]
			$\Norm{\cdot}{\mathrm{L}^p}$がノルムの公理を満たすことを示す.
			任意に$[f],[g] \in \Lp{p}{X,\mathcal{F},\mu}$と$\alpha \in \R$を取る.
			$\Norm{[f]}{\mathrm{L}^p}$が正値であることは$\Norm{\cdot}{\mathscr{L}^p}$が正値であることから従う.
			また$\mu(f \neq 0) > 0 \Rightarrow \Norm{f}{\mathscr{L}^p} > 0$の
			対偶により$\Norm{[f]}{\mathrm{L}^p} = 0 \Rightarrow [f] = [0]$が成り立ち,
			逆に$[f] = [0]$ならば$\Norm{[f]}{\mathrm{L}^p} = \Norm{0}{\mathscr{L}^p} = 0$が成り立つ.
			更に$\Norm{\cdot}{\mathscr{L}^p}$の同次性とMinkowskiの不等式より
			\begin{align}
				&\Norm{\alpha[f]}{\mathrm{L}^p} = \Norm{\alpha f}{\mathscr{L}^p} = |\alpha|\Norm{f}{\mathscr{L}^p} = |\alpha|\Norm{[f]}{\mathrm{L}^p}, \\
				&\Norm{[f] + [g]}{\mathrm{L}^p} = \Norm{f + g}{\mathscr{L}^p} \leq \Norm{f}{\mathscr{L}^p} + \Norm{g}{\mathscr{L}^p} = \Norm{[f]}{\mathrm{L}^p} + \Norm{[g]}{\mathrm{L}^p}
			\end{align}
			が得られ,$\Norm{\cdot}{\mathrm{L}^p}$の同次性と劣加法性が導かれる.
			\QED
	\end{description}
\end{prf}

	\begin{screen}
		\begin{lem}[距離空間値可測関数列の各点極限は可測]
			$(S,d)$を距離空間とする.
			$\mathcal{F}/\borel{S}$-可測関数列$(f_n)_{n=1}^{\infty}$が
			各点収束すれば,
			$f \coloneqq \lim_{n \to \infty} f_n$で定める関数$f$も可測$\mathcal{F}/\borel{S}$となる.
		\end{lem}
	\end{screen}
	
	\begin{prf}
		任意に$S$の閉集合$A$を取り,閉集合の系$(A_m)_{m=1}^{\infty}$を次で定める:
		\begin{align}
			A_m \coloneqq \Set{y \in S}{d(y,A) \leq \frac{1}{m}}, \quad (m=1,2,\cdots).\ \footnotemark
		\end{align}
		\footnotetext{
			$S \ni y \longmapsto d(y,A) \in [0,\infty)$は連続であるから,
			閉集合$[0,1/m]$は$S$の閉集合に引き戻される.
		}
		$f(x) \in A$ならば,各点収束の仮定より任意の$m \in \N$に対し或る$N \in \N$が存在して
		\begin{align}
			d\left( f_n(x),A \right) \leq d\left( f_n(x),f(x) \right) < \frac{1}{m}
			\quad (\forall n \geq N)
		\end{align}
		が成り立ち
		\begin{align}
			f^{-1}(A) \subset \bigcap_{m \geq 1} \bigcup_{N \in \N} \bigcap_{n \geq N} f_n^{-1}(A_m)
			\label{eq:lem_measurability_metric_space}
		\end{align}
		が従う.一方$f(x) \notin A$については,$0 < \epsilon < d(f(x),A)$を満たす$\epsilon$に対し
		或る$N' \in \N$が存在して
		\begin{align}
			d\left( f_n(x), f(x) \right) < \epsilon
			\quad (\forall n \geq N')
		\end{align}
		が成り立つから,$1/m < d(f(x),A) - \epsilon$を満たす$m \in \N$を取れば
		\begin{align}
			\frac{1}{m} < d(f(x),A) - d(f(x),f_n(x)) \leq d(f_n(x),A)
			\quad (\forall n \geq N')
		\end{align}
		が従い
		\begin{align}
			f^{-1}(A^c) \subset \bigcup_{m \geq 1} \bigcup_{N \in \N} \bigcap_{n \geq N} f_n^{-1}(A_m^c)
			\subset \bigcup_{m \geq 1} \bigcap_{N \in \N} \bigcup_{n \geq N} f_n^{-1}(A_m^c)
		\end{align}
		が出る.そして(\refeq{eq:lem_measurability_metric_space})と併せれば次を得る:
		\begin{align}
			f^{-1}(A) = \bigcap_{m \geq 1} \bigcup_{N \in \N} \bigcap_{n \geq N} f_n^{-1}(A_m).
		\end{align}
		これにより$S$の閉集合は$f$により$\mathcal{F}$の元に引き戻されるから$f$は可測$\mathcal{F}/\borel{S}$である.
		\QED
	\end{prf}

\begin{screen}
	\begin{thm}[$\mathrm{L}^p$の完備性]
		ノルム空間$\Lp{p}{X,\mathcal{F},\mu}\ (1 \leq p \leq \infty)$はBanach空間である.
		\label{prp:Lp_banach}
	\end{thm}
\end{screen}

\begin{prf}
	任意にCauchy列$[f_n] \in \Lp{p}{X,\mathcal{F},\mu}\ (n=1,2,3,\cdots)$を取れば,
	或る$N_1 \in \N$が存在して
	\begin{align}
		\Norm{[f_n]-[f_m]}{\mathrm{L}^p} < \frac{1}{2}
		\quad (\forall n > m \geq N_1)
	\end{align}
	を満たす.ここで$m > N_1$を一つ選び$n_1$とおく.
	同様に$N_2 > N_1$を満たす$N_2 \in \N$が存在して
	\begin{align}
		\Norm{[f_n]-[f_m]}{\mathrm{L}^p} < \frac{1}{2^2}
		\quad (\forall n > m \geq N_2)
	\end{align}
	を満たすから,$m > N_2$を一つ選び$n_2$とおけば
	\begin{align}
		\Norm{\equiv{f_{n_1}}{} - \equiv{f_{n_2}}{}}{\mathrm{L}^p} < \frac{1}{2}
	\end{align}
	が成り立つ.同様の操作を繰り返して
	\begin{align}
		\Norm{\equiv{f_{n_k}}{} - \equiv{f_{n_{k+1}}}{}}{\mathrm{L}^p} < \frac{1}{2^k} 
		\quad (n_k < n_{k+1},\ k=1,2,3,\cdots) \label{ineq:Lp_banach_2}
	\end{align}
	を満たす部分添数列$(n_k)_{k=1}^{\infty}$を構成する.
	\begin{description}
		\item[$p = \infty$の場合]
			$\equiv{f_{n_k}}{}$の代表$f_{n_k}\ (k=1,2,\cdots)$に対して
			\begin{align}
				A_k &\coloneqq \Set{x \in X}{\left| f_{n_k}(x) \right| > \Norm{f_{n_k}}{\mathscr{L}^\infty}}, \\
				A^k &\coloneqq \Set{x \in X}{\left| f_{n_k}(x) - f_{n_{k+1}}(x) \right| > \Norm{f_{n_k} - f_{n_{k+1}}}{\mathscr{L}^\infty}}
			\end{align}
			とおけば,補題\ref{lem:holder_inequality}より$\mu(A_k) = \mu(A^k) = 0\ (k=1,2,\cdots)$が成り立つ.
			\begin{align}
				A_\circ \coloneqq \bigcup_{k=1}^{\infty} A_k,
				\quad A^\circ \coloneqq \bigcup_{k=1}^{\infty}A^k,
				\quad A \coloneqq A_\circ \cup A^\circ
			\end{align}
			として$\mu$-零集合$A$を定めて
			\begin{align}
				\hat{f}_{n_k}(x) \coloneqq
				\begin{cases}
					f_{n_k}(x) & (x \in X \backslash A) \\
					0 & (x \in A)
				\end{cases}
				\quad (\forall x \in X,\ k=1,2,\cdots)
			\end{align}
			により$\left( \hat{f}_{n_k} \right)_{k=1}^{\infty}$を構成すれば,
			各$\hat{f}_{n_k}$は$\equiv{\hat{f}_{n_k}}{} = \equiv{f_{n_k}}{}$を満たす有界可測関数であり,
			(\refeq{ineq:Lp_banach_2})より
			\begin{align}
				\sup{x \in X}{\left|\hat{f}_{n_k}(x) - \hat{f}_{n_{k+1}}(x)\right|}
				\leq \Norm{\hat{f}_{n_k} - \hat{f}_{n_{k+1}}}{\mathscr{L}^\infty} < \frac{1}{2^k} \quad (k=1,2,3,\cdots) 
				\label{ineq:Lp_banach_1}
			\end{align}
			が成り立つ.従って各点$x \in X$に対し$\left( \hat{f}_{n_k}(x) \right)_{k=1}^{\infty}$は$\R$のCauchy列をなし
			\footnote{
				任意の$\epsilon > 0$に対し$1/2^N < \epsilon$を満たす$N \in \N$を取れば,全ての$\ell > k > N$に対して
				\begin{align}
					\left|\hat{f}_{n_k}(x) - \hat{f}_{n_{\ell}}(x)\right| 
					\leq \sum_{j=k}^{\ell-1}\left|\hat{f}_{n_j}(x) - \hat{f}_{n_{j+1}}(x)\right| 
					< \sum_{k > N} \frac{1}{2^k} = \frac{1}{2^N} < \epsilon
					\quad (\forall x \in X)
				\end{align}
				が成り立つ.
			}
			収束する.
			\begin{align}
				\hat{f}(x) \coloneqq \lim_{k \to \infty} \hat{f}_{n_k}(x)
				\quad (\forall x \in X)
			\end{align}
			として$\hat{f}$を定めれば,$\hat{f}$は可測$\mathcal{F}/\borel{\R}$であり,且つ任意に$k \in \N$を取れば
			\begin{align}
				\sup{x \in X}{|\hat{f}_{n_k}(x) - \hat{f}(x)|} \leq 1/2^{k-1} \label{ineq:Lp_banach_3}
			\end{align}
			を満たす.実際或る$y \in X$で$\alpha \coloneqq |\hat{f}_{n_k}(y) - \hat{f}(y)| > 1/2^{k-1}$が成り立つと仮定すれば,
			\begin{align}
				\left| \hat{f}_{n_k}(y) - \hat{f}_{n_\ell}(y) \right|
				\leq \sum_{j=k}^{\ell-1} \sup{x \in X}{\left|\hat{f}_{n_j}(x) - \hat{f}_{n_{j+1}}(x)\right|}
				< \sum_{j=k}^{\infty} \frac{1}{2^j}
				= \frac{1}{2^{k-1}}
				\quad (\forall \ell > k)
			\end{align}
			より
			\begin{align}
				0 < \alpha - \frac{1}{2^{k-1}} < \left| \hat{f}_{n_k}(y) - \hat{f}(y) \right| - \left| \hat{f}_{n_k}(y) - \hat{f}_{n_\ell}(y) \right|
				\leq \left| \hat{f}(y) - \hat{f}_{n_\ell}(y) \right|
				\quad (\forall \ell > k)
			\end{align}
			が従い各点収束に反する.不等式(\refeq{ineq:Lp_banach_3})により
			\begin{align}
				\sup{x \in X}{\left| \hat{f}(x) \right|} 
				< \sup{x \in X}{\left| \hat{f}(x) - \hat{f}_{n_k}(x) \right|} + \sup{x \in X}{\left| \hat{f}_{n_k}(x) \right|} 
				\leq \frac{1}{2^{k-1}} + \Norm{\hat{f}_{n_k}}{\mathscr{L}^\infty}
			\end{align}
			が成り立つから$\equiv{\hat{f}}{} \in \Lp{\infty}{\Omega,\mathcal{F},\mu}$が従い,
			\begin{align}
				\Norm{\equiv{f_{n_k}}{} - \equiv{\hat{f}}{}}{\mathrm{L}^\infty}
				= \Norm{\equiv{\hat{f}_{n_k}}{} - \equiv{\hat{f}}{}}{\mathrm{L}^\infty}
				\leq \sup{x \in X}{|\hat{f}_{n_k}(x) - \hat{f}(x)|}
				\longrightarrow 0 \quad (k \longrightarrow \infty)
			\end{align}
			により部分列$\left( \equiv{f_{n_k}}{} \right)_{k=1}^{\infty}$が$\equiv{\hat{f}}{}$に収束するから
			元のCauchy列も$\equiv{\hat{f}}{}$に収束する.
			
		\item[$1 \leq p < \infty$の場合]
			$\equiv{f_{n_k}}{}$の代表$f_{n_k}\ (k=1,2,\cdots)$は
			\begin{align}	
				f_{n_k}(x) = f_{n_1}(x) + \sum_{j=1}^{k}\left( f_{n_j}(x) - f_{n_{j-1}}(x) \right) \quad (\forall x \in X)
				\label{eq:Lp_banach_3}
			\end{align}
			を満たし,これに対して
			\begin{align}
				g_k(x) &\coloneqq \left| f_{n_1}(x) \right| + \sum_{j=1}^{k} \left| f_{n_j}(x) - f_{n_{j-1}}(x) \right|
				\quad (\forall x \in X,\ k=1,2,\cdots)
			\end{align}
			により単調非減少な可測関数列$(g_k)_{k=1}^{\infty}$を定めれば,Minkowskiの不等式と(\refeq{ineq:Lp_banach_2})により
			\begin{align}
				\Norm{g_k}{\mathscr{L}^p} \leq \Norm{f_{n_1}}{\mathscr{L}^p} + \sum_{j=1}^{k}\Norm{f_{n_j} - f_{n_{j-1}}}{\mathscr{L}^p}
				< \Norm{f_{n_1}}{\mathscr{L}^p} + 1 < \infty
				\quad (k = 1,2,\cdots)
				\label{eq:thm_Lp_banach_1}
			\end{align}
			が成り立つ.ここで
			\begin{align}
				B_N \coloneqq \bigcap_{k=1}^{\infty} \Set{x \in X}{g_k(x) \leq N},
				\quad B \coloneqq \bigcup_{N=1}^{\infty} B_N
			\end{align}
			とおけば$(g_k)_{k=1}^{\infty}$は$B$上で各点収束し$X \backslash B$上では発散するが,
			$X \backslash B$は零集合である.実際
			\begin{align}
				\int_X |g_k(x)|^p\ \mu(dx)
				&= \int_B |g_k(x)|^p\ \mu(dx) + \int_{X \backslash B} |g_k(x)|^p\ \mu(dx) \\
				&\leq \left( \Norm{f_{n_1}}{\mathscr{L}^p} + 1 \right)^p
				\quad (k=1,2,\cdots)
			\end{align}
			が満たされているから,単調収束定理より
			\begin{align}
				\int_B \lim_{k \to \infty}|g_k(x)|^p\ \mu(dx) + \int_{X \backslash B} \lim_{k \to \infty}|g_k(x)|^p\ \mu(dx)
				\leq \left( \Norm{f_{n_1}}{\mathscr{L}^p} + 1 \right)^p
			\end{align}
			が成り立ち$\mu(X \backslash B) = 0$が従う.
			\begin{align}
				g(x) \coloneqq
				\begin{cases}
					\lim\limits_{k \to \infty} g_k(x) & (x \in B) \\
					0 & (x \in X \backslash B)
				\end{cases}
				,\quad
				f(x) \coloneqq
				\begin{cases}
					\lim\limits_{k \to \infty} f_{n_k}(x) & (x \in B) \\
					0 & (x \in X \backslash B)
				\end{cases}
			\end{align}
			として$g,f$を定義すれば$g,f$は共に$\mathcal{F}/\borel{\R}$-可測であり,また$|f(x)| \leq g(x)\ (\forall x \in X)$と
			$g^p$の可積分性により$\equiv{f}{} \in \Lp{p}{X,\mathcal{F},\mu}$が成り立つ.
			今$\left|f_{n_k}(x) - f(x)\right|^p \leq 2^p g(x)^p\ (\forall x \in B,\ k=1,2,\cdots)$が満たされているから,
			Lebesgueの収束定理により
			\begin{align}
				\lim_{k \to \infty}\Norm{\equiv{f_{n_k}}{} - \equiv{f}{}}{\mathrm{L}^p}^p
				= \lim_{k \to \infty} \int_X \left| f_{n_k}(x) - f(x) \right|^p\ \mu(dx) = 0
			\end{align}
			が得られ,部分列の収束により元のCauchy列も$\equiv{f}{}$に収束する.
			\QED
	\end{description}
\end{prf}

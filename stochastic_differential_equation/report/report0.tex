\newpage
\begin{itembox}[l]{レポート問題1}
	\begin{description}
		\item[(1)]
			$p \geq 1, q \in (1,\infty],p+q = p q$とする.このとき任意の可測関数$f,g$に対して,
			H\Ddot{o}lderの不等式
			\begin{align}
				\Norm{fg}{1} \leq \Norm{f}{p} \Norm{g}{q},
			\end{align}
			またMinkowskiの不等式
			\begin{align}
				\Norm{f + g}{p} \leq \Norm{f}{p} + \Norm{g}{q}
			\end{align}
			が成立する.
	
		\item[(2)]
			$\Lp{p}{\mathcal{F},m}$はBanach空間である.
	
		\item[(3)]
			$\Lp{p}{\mathcal{F},m}$のCauchy列は概収束する部分列を持つ.
	\end{description}
\end{itembox}

\begin{prf}
	(1)はレポート中の定理\ref{thm:holder_inequality},
	\refeq{thm:minkowski_inequality}で,
	(2)はレポート中の定理\ref{prp:Lp_banach}で証明した.
	(3)は定理\ref{prp:Lp_banach}の証明中の部分列の収束を指す.
	\QED
\end{prf}

\newpage
\begin{itembox}[l]{レポート問題2}
	補題1.3を用いてC1-6を示せ.
\end{itembox}

\begin{prf}
	レポート中の定理\ref{prp:L2_conditional_expectation}
	で証明した.
	\QED
\end{prf}

\newpage
\begin{itembox}[l]{レポート問題3}
	確率変数$X$として集合$A$の定義関数$\defunc_A$,
	$\mathcal{G}$として$\mathcal{G} = \{\emptyset,B,B^c,\Omega \}$
	を考えると,$X$が$\mathcal{G}$と独立であることは
	$m(A \cap B) = m(A)m(B)$が成立することと同値である.この同値性を示せ.
\end{itembox}

\begin{prf}
	レポート中の定理\ref{thm:report_3}で証明した.
	\QED
\end{prf}

\newpage
\begin{itembox}[l]{レポート問題4}
	定理4.1を証明せよ.
\end{itembox}

\begin{prf}
	レポート中の定理\ref{prp:M_2_c_hilbert}で証明した.
	\QED
\end{prf}

\newpage
\begin{itembox}[l]{レポート問題5}
	$I = [a,b]$を$I = [0,T]$に替えて示す.
	$M \in \mathcal{M}_{c,loc}$かつ$\Norm{\inprod<M>_T} < \infty$であるならば
	$M \in \mathcal{M}_{2,c}$となることを示せ.
\end{itembox}

\begin{prf}
	レポート中の定理\ref{thm:quadratic_variation_bounded_then_M_2c}で証明した.
	\QED
\end{prf}

\newpage
\begin{itembox}[l]{レポート問題6}
	$I = [a,b]$を$I = [0,T]$に替えて示す.
	局所有界な左連続適合過程$X$と$M \in \mathcal{M}_{c,loc}$に対して,
	$X$が連続適合過程である時と同様に確率積分
	\begin{align}
		\int_0^t X_s\ dM_s,
		\quad t \in I
	\end{align}
	が定義できることを示せ.
\end{itembox}

\begin{prf}
	レポート中の定理\ref{thm:quadratic_variation_bounded_then_M_2c}で証明した.
	\QED
\end{prf}

\newpage
\begin{itembox}[l]{レポート問題7}
	$I = [a,b]$を$I = [0,T]$に替えて示す.
	$X,Y$を局所有界な左連続適合過程とし,$M \in \mathcal{M}_{c,loc}$とする.
	\begin{align}
		N_t = \int_0^t X_s\ dM_s,
		\quad t \in I
	\end{align}
	とする時,$N \in \mathcal{M}_{c,loc}$であり,
	\begin{align}
		\int_0^t Y_s\ dN_s = \int_0^t Y_s X_s\ dM_s,
		\quad t \in I
	\end{align}
	であることを示せ.
\end{itembox}

\begin{prf}
	レポート中の定理\ref{thm:quadratic_variation_bounded_then_M_2c}で証明した.
	\QED
\end{prf}
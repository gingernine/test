\section{停止時刻}
	確率空間を$(\Omega,\mathcal{F},\mu)$とする.集合$I$によって確率過程の時点を表現し,
	以降でこれは$[0,\infty)$や$\{0,,1,\cdots,n\}$など実数の区間や高々可算集合を指すものと考え,
	$I$が高々可算集合の場合は離散位相,$\R$の区間の場合は相対位相を考える.また扱う確率変数は全て実数値で考える.
	\begin{itembox}[l]{}
		\begin{dfn}[フィルトレーション]
			$\mathcal{F}$の部分$\sigma$-加法族の部分系$\Set{\mathcal{F}_\alpha}{\alpha \in I}$
			がフィルトレーション(filtration)であるとは,任意の$\alpha,\beta \in I$に対して$\alpha \leq \beta$ならば
			$\mathcal{F}_\alpha \subset \mathcal{F}_\beta$の関係をもつことで定義する.
		\end{dfn}
	\end{itembox}
	
	\begin{itembox}[l]{}
		\begin{dfn}[停止時刻]
			$\Omega$上の関数で次を満たすものを($(\mathcal{F}_\alpha)$-)停止時刻(stopping time)という:
			\begin{align}
				\tau:\Omega \longrightarrow I\quad \mathrm{s.t.}\quad \forall \alpha \in I,\ \{ \tau \leq \alpha \} \in \mathcal{F}_\alpha.
			\end{align}
		\end{dfn}
	\end{itembox}
	
	\begin{itembox}[l]{}
		\begin{rem}[停止時刻は可測]
			上で定義した$\tau$は可測$\mathcal{F}/\borel{I}$である.
		\end{rem}
	\end{itembox}
	\begin{prf}\mbox{}
		\begin{description}
			\item[$I$が$\R$の区間である場合]
				任意の$\alpha \in I$に対して$I_\alpha \coloneqq (-\infty,\alpha) \cap I$は$I$における(相対の)開集合であり
				$\tau^{-1}(I_\alpha) = \{ \tau \leq \alpha \} \in \mathcal{F}_\alpha \subset \mathcal{F}$が成り立つ.
				つまり
				\begin{align}
					\Set{I_\alpha}{\alpha \in I} \subset \Set{A \in \borel{I}}{\tau^{-1}(A) \in \mathcal{F}}
					\label{eq:stopping_time_mble}
				\end{align}
				が成り立ち,左辺の$I_\alpha$の形の全体は$\borel{I}$を生成するから$\tau$の可測性が証明された.
				
			\item[$I$が高々可算集合である場合]
				先ず$\alpha \in I$に対して$\{ \tau < \alpha \}$が$\mathcal{F}_\alpha$に属することを示す.
				$\alpha$に対して直前の元$\beta \in I$が存在するか$\alpha$が$I$の最小限である場合,前者なら$\{ \tau < \alpha \} = \{ \tau \leq \beta \}$
				となり後者なら$\{ \tau < \alpha \} = \emptyset$となるからどちらも$\mathcal{F}_\alpha$に属する.
				そうでない場合は$\alpha - 1/n < x < \alpha$を満たす点列$x_n \in I\ (n=1,2,3,\cdots)$を取れば,
				$\{ \tau < \alpha \} = \cap_{n=1}^{\infty}\{ \tau \leq \alpha - 1/n \}$
				により$\{ \tau < \alpha \} \in \mathcal{F}_\alpha$が判る.以上の準備の下で
				任意の$\alpha \in I$に対して$\tau^{-1}(\{\alpha\}) = \{ \tau \leq \alpha \} - \{ \tau < \alpha \} \in \mathcal{F}_\alpha$が成り立ち,
				更に可算集合$I$には離散位相が入っているから任意の$A \in \borel{I}$は
				一点集合の可算和で表現できて,$\tau^{-1}(A) \in \mathcal{F}$であると証明された.		
		\end{description}
		\QED
	\end{prf}
	
	\begin{itembox}[l]{}
		\begin{dfn}[停止時刻の再定義]
			今$\tau$の終集合は$I$であるが,$I \rightarrow \R$の恒等写像$i$を用いて$\tau^* \coloneqq i \circ \tau$とすれば,
			\begin{align}
				\borel{I}=\Set{A \cap I}{A \in \borel{\R}} = \Set{i^{-1}(A)}{A \in \borel{\R}}
			\end{align}
			により($i$が可測$\borel{I}/\borel{\R}$であるから)合成写像$\tau^*$は可測$\mathcal{F}/\borel{\R}$となる.以降は
			この$\tau^*$を停止時刻$\tau$と表記して扱うことにする.
		\end{dfn}
	\end{itembox}
	
	定数関数は停止時刻となる.$\tau$が$\Omega$上の定数関数なら$(\tau \leq \alpha)$は空集合か全体集合にしかならないからである.
	また$\sigma,\tau$を$I$に値を取る停止時刻とすると
	$\sigma \vee \tau$と$\sigma \wedge \tau$も停止時刻となる.実際
	\begin{align}
		\begin{cases}
			\{ \sigma \wedge \tau \leq \alpha \} = \{ \sigma \leq \alpha \} \cup \{ \tau \leq \alpha \}, \\
			\{ \sigma \vee \tau \leq \alpha \} = \{ \sigma \leq \alpha \} \cap \{ \tau \leq \alpha \}
		\end{cases}
		\quad ,(\forall \alpha \in I)
	\end{align}
	が成り立つからである.
	
	\begin{itembox}[l]{}
		\begin{dfn}[停止時刻の前に決まっている事象系]
			$\tau$を$I$に値を取る停止時刻とする.$\tau$に対し次の集合系を定義する.
			\begin{align}
				\mathcal{F}_\tau \coloneqq \Set{A \in \mathcal{F}}{\{ \tau \leq \alpha \} \cap A \in \mathcal{F}_\alpha,\ \forall \alpha \in I}.
			\end{align}
		\end{dfn}
	\end{itembox}
	\begin{itembox}[l]{}
		\begin{prp}[停止時刻の性質]
			$I \subset \R$に値を取る停止時刻$\sigma, \tau$に対し次が成り立つ.
			\begin{description}
				\item[(1)] $\mathcal{F}_\tau$は$\sigma$-加法族である.
				\item[(2)] 或る$\alpha \in I$に対して$\tau(\omega) = \alpha\ (\forall \omega \in \Omega)$なら$\mathcal{F}_\alpha = \mathcal{F}_\tau$.
				\item[(3)] $\sigma(\omega) \leq \tau(\omega)\ (\forall \omega \in \Omega)$ならば$\mathcal{F}_\sigma \subset \mathcal{F}_\tau$.
				\item[(4)] $\mathcal{F}_{\sigma \wedge \tau} = \mathcal{F}_\sigma \cap \mathcal{F}_\tau$.
				\item[(5)] $\mathcal{F}_{\sigma \vee \tau} = \mathcal{F}_\sigma \vee \mathcal{F}_\tau$.
			\end{description}
		\end{prp}
	\end{itembox}
	
	\begin{prf}\mbox{}
		\begin{description}
			\item[(1)] 停止時刻の定義より$\Omega \in \mathcal{F}_\tau$である.また$A \in \mathcal{F}_\tau$なら
				$A^c \cap \{ \tau \leq \alpha \} = \{ \tau \leq \alpha \} - A \cap \{ \tau \leq \alpha \} \in \mathcal{F}_\alpha$より
				$A^c \in \mathcal{F}_\tau$となる.可算個の$A_n \in \mathcal{F}_\tau$については
				$\cup_{n=1}^{\infty} A_n \cap \{ \tau \leq \alpha \} = \cup_{n=1}^{\infty} \left( A_n \cap \{ \tau \leq \alpha \} \right) \in \mathcal{F}_\alpha$
				により$\cup_{n=1}^{\infty} A_n \in \mathcal{F}_\tau$が成り立つ.
			
			\item[(2)] $A \in \mathcal{F}_\alpha$なら任意の$\beta \in I$に対して
				\begin{align}
					A \cap \{ \tau \leq \beta \} =
					\begin{cases}
						A & \alpha \leq \beta \\
						\emptyset & \alpha > \beta
					\end{cases}
				\end{align}
				が成り立つから,いずれの場合も$A \in \mathcal{F}_\beta$となり$A \subset \mathcal{F}_\tau$が成り立つ.
				逆に$A \in \mathcal{F}_\tau$のとき,$A = A \cap \{ \tau \leq \alpha \} \in \mathcal{F}_\alpha$
				が成り立ち$\mathcal{F}_\alpha = \mathcal{F}_\tau$が示された.
				
			\item[(3)] $A \in \mathcal{F}_\sigma$なら任意の$\alpha \in I$に対して
				\begin{align}
					A \cap \{ \tau \leq \alpha \} = A \cap \{ \sigma \leq \alpha \} \cap \{ \tau \leq \alpha \} \in \mathcal{F}_\alpha
				\end{align}
				が成り立つから$A \in \mathcal{F}_\tau$となる.
			
			\item[(4)] $\sigma \wedge \tau$が停止時刻であることと(3)より
				$\mathcal{F}_{\sigma \wedge \tau} \subset \mathcal{F}_\sigma$と$\mathcal{F}_{\sigma \wedge \tau} \subset \mathcal{F}_\tau$が判る.
				また$A \in \mathcal{F}_\sigma \cap \mathcal{F}_\tau$に対し
				\begin{align}
					A \cap \{ \sigma \wedge \tau \leq \alpha \} 
					= \left( A \cap \{ \sigma \leq \alpha \} \right) \cup \left( A \cap \{ \tau \leq \alpha \} \right) \in \mathcal{F}_\alpha \quad (\forall \alpha \in I)
				\end{align}
				より$A \in \mathcal{F}_{\sigma \wedge \tau}$も成り立つ.
			
			\item[(5)] 
				先ず$\sigma \vee \tau$が停止時刻であることと(3)より
				$\mathcal{F}_\sigma \subset \mathcal{F}_{\sigma \wedge \tau}$と$\mathcal{F}_\tau \subset \mathcal{F}_{\sigma \wedge \tau}$が判る.
				逆に$A \in \mathcal{F}_{\sigma \wedge \tau}$に対して
				
		\end{description}
	\end{prf}
	
	\begin{itembox}[l]{}
		\begin{prp}[停止時刻と条件付き期待値]
			$X \in \Lp{1}{\mathcal{F},\operatorname{P}}$と
			$I$に値を取る停止時刻$\sigma, \tau$に対し以下が成立する.
			\begin{description}
				\item[(1)] $\cexp{\defunc_{(\sigma > \tau)} X}{\mathcal{F}_\tau} = \cexp{\defunc_{(\sigma > \tau)} X}{\mathcal{F}_{\sigma \wedge \tau}}$.
				\item[(2)] $\cexp{\defunc_{(\sigma \geq \tau)} X}{\mathcal{F}_\tau} = \cexp{\defunc_{(\sigma \geq \tau)} X}{\mathcal{F}_{\sigma \wedge \tau}}$.
				\item[(3)] $\cexp{\cexp{X}{\mathcal{F}_\tau}}{\mathcal{F}_\sigma} = \cexp{X}{\mathcal{F}_{\sigma \wedge \tau}}$.
			\end{description}
			\label{prp:stopping_time_and_conditional_expectation}
		\end{prp}
	\end{itembox}
	
	\begin{prf}\mbox{}
		\begin{description}
			\item[第一段] $\defunc_{(\sigma > \tau)}$が可測$\mathcal{F}_{\sigma \wedge \tau}/\borel{\R}$であることを示す.
				$A \in \borel{\R}$に対し
				\begin{align}
					\defunc_{(\sigma > \tau)}^{-1}(A) = 
					\begin{cases}
						\Omega & (0 \in A,\ 1 \in A) \\
						(\sigma > \tau) & (0 \notin A,\ 1 \in A) \\
						(\sigma > \tau)^c & (0 \in A,\ 1 \notin A) \\
						\emptyset & (0 \notin A,\ 1 \notin A)
					\end{cases}
				\end{align}
				と表現できるから,示すことは任意の$\alpha \in I$に対して
				\begin{align}
					(\sigma > \tau) \cap (\sigma \wedge \tau \leq \alpha) \in \mathcal{F}_\alpha
				\end{align}
				が成立することである.これが示されれば
				\begin{align}
					(\sigma > \tau)^c \cap (\sigma \wedge \tau \leq \alpha)
					= (\sigma \wedge \tau \leq \alpha) \backslash \left[(\sigma > \tau) \cap (\sigma \wedge \tau \leq \alpha)\right] \in \mathcal{F}_\alpha
				\end{align}
				も成り立ち,更に$(\sigma > \tau)^c = (\sigma \leq \tau)$であることと$\sigma,\tau$の対等性により$\defunc_{(\sigma \geq \tau)}$もまた
				可測$\mathcal{F}_{\sigma \wedge \tau}/\borel{\R}$であることが判る.目的の式は次が成り立つことにより示される.
				\begin{align}
					(\sigma > \tau) \cap (\sigma \wedge \tau \leq \alpha)
					&= (\sigma > \tau) \cap (\sigma \leq \alpha) + (\sigma > \tau) \cap (\sigma > \alpha) \cap (\tau \leq \alpha) \\
					&= \left[\bigcup_{\substack{\beta \in \Q \cap I \\ \beta \leq \alpha}} (\sigma > \beta)\cap(\tau \leq \beta)\right]\cap(\sigma \leq \alpha) + (\sigma > \alpha) \cap (\tau \leq \alpha)
					\label{eq:stopping_time_conditional_expectation_1} \\
					&\in \mathcal{F}_\alpha.
				\end{align}
			
			\item[第二段] 一般の実確率変数$Y$と停止時刻$\tau$に対して
				\begin{itemize}
					\item $Y$が可測$\mathcal{F}_\tau/\borel{\R}$$\quad \Leftrightarrow \quad$任意の$\alpha \in I$に対し$Y \defunc_{\tau \leq \alpha}$が可測$\mathcal{F}_\alpha/\borel{\R}$
				\end{itemize}
				が成り立つことを示す.
				\begin{description}
					\item[$\Rightarrow$について]
						$Y$の単関数近似列$(Y_n)_{n=1}^{\infty}$の一つ一つは$Y_n = \sum_{j=1}^{N_n}a_{j,n}\defunc_{A_{j,n}}\ (A_{j,n} \in \mathcal{F}_\tau)$
						の形で表現できる.$\alpha \in I$と$A \in \mathcal{F}_\tau$の指示関数$\defunc_{A}$に対し
						\begin{align}
							\left( \defunc_A\defunc_{(\tau \leq \alpha)} \right)^{-1}(E) =
							\begin{cases}
								\Omega & (0 \in E,\ 1 \in E) \\
								A \cap (\tau \leq \alpha) & (0 \notin E,\ 1 \in E) \\
								[A \cap (\tau \leq \alpha)]^c & (0 \in E,\ 1 \notin E) \\
								\emptyset & (0 \notin E,\ 1 \notin E)
							\end{cases}
							\quad (\forall E \in \borel{\R})
						\end{align}
						となり,$A \cap (\tau \leq \alpha) \in \mathcal{F}_\alpha$より$\defunc_A\defunc_{(\tau \leq \alpha)}$が
						可測$\mathcal{F}_\alpha/\borel{\R}$であると判る.$(Y_n)_{n=1}^{\infty}$は$Y$に各点収束していくから
						$Y$も可測$\mathcal{F}_\alpha/\borel{\R}$となり,$\alpha \in I$の任意性から''$\Rightarrow$''が示された.
						
					\item[$\Leftarrow$について]
						任意の$E \in \borel{\R}$に対して
						\begin{align}
							\left\{\ \omega \in \Omega\quad |\quad Y(\omega)\defunc_{(\tau \leq \alpha)}(\omega) \in E\ \right\}
							= \begin{cases}
								Y^{-1}(E) \cap (\tau \leq \alpha) & (0 \notin E) \\
								Y^{-1}(E) \cap (\tau \leq \alpha) + (\tau \leq \alpha)^c & (0 \in E)
							\end{cases}
						\end{align}
						がいずれも$\mathcal{F}_\alpha$に属する.特に下段について$(\tau \leq \alpha)^c \in \mathcal{F}_\alpha$
						より$Y^{-1}(E) \cap (\tau \leq \alpha) \in \mathcal{F}_\alpha$となるから,結局
						$Y^{-1}(E) \cap (\tau \leq \alpha) \in \mathcal{F}_\alpha \ (\forall E \in \borel{\R})$が成り立つ.
						$\alpha \in I$の任意性から$Y^{-1}(E) \in \mathcal{F}_\tau\ (\forall E \in \borel{\R})$が示された.
				\end{description}
			
			\item[第三段]
				(1)の式を示す.第一段と性質$\tilde{\mathrm{C}}$5より
				\begin{align}
					\cexp{\defunc_{(\sigma > \tau)}X}{\mathcal{F}_\tau} = \defunc_{(\sigma > \tau)} \cexp{X}{\mathcal{F}_\tau}
				\end{align}
				が成り立つから,あとは右辺が(関数とみて)可測$\mathcal{F}_{\sigma \wedge \tau}/\borel{\R}$であればよく,このためには
				第二段の結果より任意の$\alpha \in I$に対して$\cexp{X}{\mathcal{F}_\tau}\defunc_{(\sigma > \tau)}\defunc_{(\sigma \wedge \tau \leq \alpha)}$が
				可測$\mathcal{F}_\alpha/\borel{\R}$であることを示せばよい.
				式(\refeq{eq:stopping_time_conditional_expectation_1})を使えば
				\begin{align}
					\defunc_{(\sigma > \tau)}\defunc_{(\sigma \wedge \tau \leq \alpha)}
					= \sup{\substack{\beta \in \Q \cap I \\ \beta \leq \alpha}}{\defunc_{(\sigma > \beta)}\defunc_{(\tau \leq \beta)}\defunc_{(\sigma \leq \alpha)}}
						+ \defunc_{(\sigma > \alpha)} \defunc_{(\tau \leq \alpha)}
				\end{align}
				が成り立つ.$\beta \leq \alpha$ならば,$\cexp{X}{\mathcal{F}_\tau}$が可測$\mathcal{F}_\tau/\borel{\R}$であることと第二段の結果より
				$\cexp{X}{\mathcal{F}_\tau}\defunc_{(\tau \leq \beta)}$が可測$\mathcal{F}_\beta/\borel{\R}$すなわち可測$\mathcal{F}_\alpha/\borel{\R}$
				となるから,これで$\cexp{X}{\mathcal{F}_\tau}\defunc_{(\sigma > \tau)}\defunc_{(\sigma \wedge \tau \leq \alpha)}$が可測$\mathcal{F}_\alpha/\borel{\R}$
				であると判り$\defunc_{(\sigma > \tau)} \cexp{X}{\mathcal{F}_\tau}$が可測$\mathcal{F}_{\sigma \wedge \tau}/\borel{\R}$であることが示された.
				以上で
				\begin{align}
					\cexp{\cexp{\defunc_{(\sigma > \tau)}X}{\mathcal{F}_\tau}}{\mathcal{F}_{\sigma \wedge \tau}}
					= \cexp{\defunc_{(\sigma > \tau)} \cexp{X}{\mathcal{F}_\tau}}{\mathcal{F}_{\sigma \wedge \tau}}
					= \defunc_{(\sigma > \tau)} \cexp{X}{\mathcal{F}_\tau}
					= \cexp{\defunc_{(\sigma > \tau)}X}{\mathcal{F}_\tau}
				\end{align}
				が成り立ち,
				\begin{align}
					\cexp{\cexp{\defunc_{(\sigma > \tau)}X}{\mathcal{F}_\tau}}{\mathcal{F}_{\sigma \wedge \tau}}
					= \defunc_{(\sigma > \tau)} \cexp{X}{\mathcal{F}_{\sigma \wedge \tau}}
					= \cexp{\defunc_{(\sigma > \tau)}X}{\mathcal{F}_{\sigma \wedge \tau}}
				\end{align}
				と併せて(1)の式を得る.(2)の式も以上と同じ理由で成り立つ.
				
			\item[第四段]
				(3)の式を示す.
				\begin{align}
					\cexp{\cexp{X}{\mathcal{F}_\tau}}{\mathcal{F}_\sigma}
					&= \cexp{\cexp{X}{\mathcal{F}_\tau}\defunc_{(\sigma > \tau)}}{\mathcal{F}_\sigma}
						+ \cexp{\cexp{X}{\mathcal{F}_\tau}\defunc_{(\sigma \leq \tau)}}{\mathcal{F}_\sigma} \\
					&= \cexp{\cexp{X}{\mathcal{F}_{\sigma \wedge \tau}}\defunc_{(\sigma > \tau)}}{\mathcal{F}_\sigma}
						+ \cexp{\cexp{X}{\mathcal{F}_\tau}\defunc_{(\sigma \leq \tau)}}{\mathcal{F}_\sigma} && (\scriptsize\because\mbox{(1)}) \\
					&= \cexp{X}{\mathcal{F}_{\sigma \wedge \tau}}\defunc_{(\sigma > \tau)}
						+ \cexp{\cexp{X}{\mathcal{F}_\tau}\defunc_{(\sigma \leq \tau)}}{\mathcal{F}_{\sigma \wedge \tau}} && (\scriptsize\because\mbox{(2)}) \\
					&= \cexp{X}{\mathcal{F}_{\sigma \wedge \tau}}\defunc_{(\sigma > \tau)} + \cexp{X}{\mathcal{F}_{\sigma \wedge \tau}}\defunc_{(\sigma \leq \tau)} \\
					&= \cexp{X}{\mathcal{F}_{\sigma \wedge \tau}}.
				\end{align}
				(2)式を使った箇所では$X$を$\cexp{X}{\mathcal{F}_\tau}$に置き換え$\tau$と$\sigma$を入れ替えて適用した.
		\end{description}
		\QED
	\end{prf}
	
	\begin{itembox}[l]{}
		\begin{thm}[停止時刻との合成写像の可測性]
			$I = [0,T]$,フィルトレーションを$(\mathcal{F}_t)_{t \in I}$,$\tau$を停止時刻とし,
			$M$を$I \times \Omega$上の$\R$値関数とする.全ての$\omega \in \Omega$に対し
			$I \ni t \longmapsto M(t,\omega)$が右連続でかつ$(\mathcal{F}_t)$-適合ならば,
			写像$\omega \longmapsto M(\tau(\omega),\omega)$は可測$\mathcal{F}_\tau/\borel{\R}$となる.
			\label{thm:measurability_of_stopping_time}
		\end{thm}
	\end{itembox}
	
	\begin{prf}
		任意に$t \in I$を取り$t_j^n \coloneqq jt/2^n\ (j=0,1,\cdots,2^n,\ n=1,2,\cdots)$とおくと,
		右連続性により任意の$s \in [0,t]$に対して
		\begin{align}
			M(s,\omega) = \lim_{n \to \infty} \sum_{j=1}^{2^n} M_{t_j^n}(\omega) \defunc_{[t_{j-1}^n,t_j^n)}(s) + M_t(\omega) \defunc_{\{t\}}(s) \quad (\omega \in \Omega)
			\label{eq:stopping_time_measurability}
		\end{align}
		が成り立つ.右辺は各$n$で可測$\borel{[0,t]} \times \mathcal{F}_t/\borel{\R}$であるから
		$M$も可測$\borel{[0,t]} \times \mathcal{F}_t/\borel{\R}$となる.($t$の任意性から$M$は$(\mathcal{F}_t)$-発展的可測である.)
		一方停止時刻$\tau$について,$\tau \wedge t$が可測$\mathcal{F}_t/\borel{\R}$であるから
		\begin{align}
			\Omega \ni \omega \longmapsto (\tau(\omega) \wedge t, \omega) \in [0,t] \times \Omega
		\end{align}
		は可測$\mathcal{F}_t/\borel{[0,t]} \times \mathcal{F}_t$である.従って合成写像
		\begin{align}
			\Omega \ni \omega \longmapsto M(\tau(\omega) \wedge t,\omega) \in \R
		\end{align}
		は可測$\mathcal{F}_t/\borel{\R}$となる.任意の$A \in \borel{\R}$に対して
		\begin{align}
			\Set{\omega \in \Omega}{M(\tau(\omega),\omega) \in A} \cap \left\{ \tau \leq t \right\}
			= \Set{\omega \in \Omega}{M(\tau(\omega) \wedge t,\omega) \in A} \cap \left\{ \tau \leq t \right\}
			\in \mathcal{F}_t
		\end{align}
		が成り立つから,写像$\omega \longmapsto M(\tau(\omega),\omega)$は可測$\mathcal{F}_\tau/\borel{\R}$である
		\footnote{
			写像$\omega \longmapsto M(\tau(\omega),\omega)$が可測$\mathcal{F}/\borel{\R}$となっていないことにはこの結論が従わない.
			この点を確認すれば,式(\refeq{eq:stopping_time_measurability})より$M$が可測$\borel{I} \times \mathcal{F}/\borel{\R}$
			であることは慥かであるから,$\omega \longmapsto (\tau(\omega),\omega)$が可測$\mathcal{F}/\borel{I}\times\mathcal{F}$であることと
			併せて写像$\omega \longmapsto M(\tau(\omega),\omega)$が可測$\mathcal{F}/\borel{\R}$であることが判明する.
		}.
		\QED
	\end{prf}
	
	\begin{itembox}[l]{}
		\begin{thm}[閉集合と停止時刻]
			$I = [0,T] \subset \R$,$(E,\rho)$を距離空間,$(X_t)_{t \in I}$を$E$値確率変数の族とし,
			$\mathcal{F}_0$が$\mu$-零集合を全て含んでいると仮定する.
			$\mu$-零集合$N$を除いて$I \ni t \longmapsto X_t(\omega)$が右連続で,
			かつ$(X_t)_{t \in I}$が$(\mathcal{F}_t)$-適合であるなら,任意の閉集合$F \subset E$に対し
			\begin{align}
				\tau(\omega) \coloneqq
				\begin{cases}
					0 & (\omega \in N) \\
					\inf{}{\Set{t \in I}{X_t(\omega) \in F}} \wedge T & (\omega \in \Omega \backslash N)
				\end{cases}
			\end{align}
			として$\tau:\Omega \rightarrow \R$を定めれば$\tau$は停止時刻となる.
			また$N'\ (N \subset N')$を除いて$I \ni t \longmapsto X_t(\omega)$が連続であるなら
			\begin{align}
				X_{t \wedge \tau(\omega)}(\omega) \in \{ X_0(\omega) \} \cup F^{ic} \quad (\forall \omega \in N',\ t \in I)
			\end{align}
			が成り立つ.ただし$F^i$は$F$の内核を表し$F^{ic}$は$F^i$の補集合を表す.
			\label{thm:closed_set_stopping_time}
		\end{thm}
	\end{itembox}
	確率空間が完備である場合は
	\begin{align}
		\tau(\omega) \coloneqq \inf{}{\Set{t \in I}{X_t(\omega) \in F}} \wedge T
		\quad (\forall \omega \in \Omega)
	\end{align}
	として$\tau$は停止時刻となる.実際任意の$t \in I$に対して,
	\begin{align}
		\{\tau \leq t\} = \{\tau \leq t\} \cap N + \Set{\omega \in \Omega \backslash N}{\tau(\omega) \leq t}
	\end{align}
	の右辺第一項は完備性より$\mu$-零集合,第二項は以下で$\mathcal{F}_t$に属すると証明される.

	\begin{prf}
		\begin{align}
			D_t(\omega) \coloneqq 
			\begin{cases}
				1 & (\omega \in N) \\
				\inf{}{\Set{\rho(X_r(\omega),F)}{r \in ([0,t] \cap \Q) \cup \{t\}}} & (\omega \in \Omega \backslash N)
			\end{cases}
		\end{align}
		とおけば
		\footnote{
			$\rho(X_r(\omega),F) = \inf{y \in F}{\rho(X_r(\omega),y)}$である.
		},
		$D_t$は可測$\mathcal{F}_t/\borel{\R}$となる
		\footnote{
			写像$E \ni x \longmapsto \rho(x,F) \in \R$は連続であるから,合成写像
			\begin{align}
				\Omega \ni \omega \longmapsto \rho(X_t(\omega),F)
			\end{align}
			は可測$\mathcal{F}_t/\borel{\R}$となる.任意の$\lambda \in \R$に対し
			\begin{align}
				\left\{ \inf{}{\Set{\rho(X_r,F)}{r \in ([0,t] \cap \Q) \cup \{t\}}} \geq \lambda \right\}
				= \bigcap_{r \in ([0,t] \cap \Q) \cup \{t\}} \left\{ \rho(X_r,F) \geq \lambda \right\}
			\end{align}
			となり右辺の各集合は$\in \mathcal{F}_t$であるから
			写像$\Omega \ni \omega \longmapsto \inf{}{\Set{\rho(X_r(\omega),F)}{r \in ([0,t] \cap \Q) \cup \{t\}}}$
			も可測$\mathcal{F}_t/\borel{\R}$となる.任意の$A \in \borel{\R}$に対し
			\begin{align}
				D_t^{-1}(A) = 
				\begin{cases}
					N \cup \left\{ \inf{}{\Set{\rho(X_r,F)}{r \in ([0,t] \cap \Q) \cup \{t\}}} \in A \right\} & (1 \in A) \\
					\left\{ \inf{}{\Set{\rho(X_r,F)}{r \in ([0,t] \cap \Q) \cup \{t\}}} \in A \right\} & (1 \notin A)
				\end{cases}
			\end{align}
			となるが,$N \in \mathcal{F}_0$であるから$D_t$もまた可測$\mathcal{F}_t/\borel{\R}$となる.
		}.
		ここでは任意の$t \in [0,T)$に対して
		\begin{align}
			\Set{\omega \in \Omega \backslash N}{\tau(\omega) \leq t} = \Set{\omega \in \Omega \backslash N}{D_t(\omega) = 0}
			\label{eq:closed_set_stopping_time_1}
		\end{align}
		が成り立つことを示す.実際これが示されれば任意の$t \in I$に対し
		\begin{align}
			\{\tau \leq t\} =
			\begin{cases}
 				\Omega & (t = T) \\
				N + \Set{\omega \in \Omega \backslash N}{D_t(\omega) = 0} & (t < T)
 			\end{cases}
		\end{align}
		となるから$\tau$は停止時刻となる.式(\refeq{eq:closed_set_stopping_time_1})が成立することを示すには
		包含関係$\subset,\supset$のそれぞれを満たすことを確認すればよい.
		\begin{description}
			\item[$\subset$について]
				任意に$t \in [0,T)$を固定する.$\tau(\omega) \leq t$となる$\omega \in \Omega \backslash N$に対し
				$s \coloneqq \tau(\omega)$とおくと$X_s(\omega) \in F$となる.もし$X_s(\omega) \notin F$であるとすれば,
				$F$が閉集合であることと$s \longmapsto X_s(\omega)$の右連続性から
				或る$\delta > 0$が存在し,任意の$0 < h < \delta$に対して
				$X_{s+h}(\omega) \notin F$となり$s = \tau(\omega)$であることに矛盾する.
				今$\rho(X_s(\omega),F) = 0$が示されたが,$D_t(\omega) = 0$も成り立っている.
				実際もし$D_t(\omega) > 0$であるとすれば,$a \coloneqq D_t(\omega)$に対して
				\begin{align}
					\rho(X_s(\omega), X_r(\omega)) < a/2
				\end{align}
				を満たす$r \in ((s,t] \cap \Q) \cup \{t\}$が存在するから
				\begin{align}
					\rho(X_s(\omega),F) \geq \rho(X_r(\omega),F) - \rho(X_s(\omega), X_r(\omega)) > a - a/2 = a/2
				\end{align}
				となり矛盾が生じてしまう.
			
			\item[$\supset$について]
				任意に$t \in [0,T)$を固定する.$D_t(\omega) = 0$となる$\omega \in \Omega \backslash N$について
				\begin{align}
					\rho(X_{s_n}(\omega),F) < 1/n
				\end{align}
				となるように$s_n \in ([0,t] \cap \Q) \cup \{t\}$を取ることができる.$(s_n)_{n=1}^{\infty}$は
				$[0,t]$に集積点$s$を持ち,$F$が閉であるから$X_s(\omega) \in F$となる.従って
				$\tau(\omega) \leq s \leq t$が成り立つ.
		\end{description}
		以上で$\tau$が停止時刻であることが示されたから,次に定理の後半の主張を示す.
		$N'$を除いて$I \ni t \longmapsto X_t(\omega)$が連続である場合,
		$\tau(\omega) = 0$なら$X_0(\omega) \in F$である.実際
		$F$が閉集合であることとパスの連続性により$0$のある近傍内においても$X_t(\omega) \notin F$となる.
		従って$X_{t \wedge \tau(\omega)}(\omega) \in F^{ic}$であるとは限らない.
		$\tau(\omega) > 0$のとき,$t < \tau(\omega)\ (\omega \in \Omega \backslash N')$に対しては$X_t(\omega) \in F^c$が成り立っているから,示せばよいのは
		\begin{align}
			X_{\tau(\omega)}(\omega) \in F^{ic}
			\label{eq:closed_set_stopping_time_2}
		\end{align}
		が成り立つことである.$s = \tau(\omega)$とおく.もし$X_s(\omega) \in F^i$であるとすれば
		連続性から或る$\delta$が存在し,$0 < h < \delta$を満たす任意の$h$に対し
		\begin{align}
			X_{s - h}(\omega) \in F^i
		\end{align}
		となるが,
		\begin{align}
			s > s - h \geq \inf{}{\Set{t \in I}{X_t(\omega) \in F}}
		\end{align}
		が従い矛盾が生じる.よって(\refeq{eq:closed_set_stopping_time_2})が示された.
		\QED
	\end{prf}
	
	
	\begin{itembox}[l]{}
		\begin{prp}
			任意の$p \geq 1$に対し,$M \in \mathcal{M}_{p,c}$が
			$\Norm{M_0}{\mathscr{L}^\infty} < \infty$を満たすなら$M \in \mathcal{M}_{c,loc}$が成り立つ.
			\label{prp:M_pc_M_cloc}
		\end{prp}
	\end{itembox}
	この証明には次の補題を使う.
	
	\begin{itembox}[l]{}
		\begin{lem}[各点で右連続であり左極限を持つ関数は閉区間上で有界]\mbox{}\\
			$(E,\rho)$を距離空間,$J = [a,b] \subset \R$とする.$f:J \rightarrow E$が各点$x \in J$で
			右連続且つ左極限を持つ\footnotemark
			なら$f$は$J$上で有界である.
			\label{lem:rcll_bounded}
		\end{lem}
	\end{itembox}
	\footnotetext{
		左端点では左極限を考えず,右端点では右連続性を考えない.
	}
	\begin{prf}[補題]
		任意に$\epsilon > 0$を取り固定する.$f$は各点$x \in [a,b)$で右連続であるから,$0 < \delta_x < b-x$を
		$0 < \forall h < \delta_x$が$\rho(f(x), f(x+h)) < \epsilon$を満たすように取り,
		\begin{align}
			V_x \coloneqq [x,x+\delta_x) \quad (\forall x \in [a,b))
		\end{align}
		とおく.また$f$は各点$x \in (a,b]$で左極限も持つから,左極限を$f(x-)$と表して
		$0 < \gamma_x < x-a$を$0 < \forall h < \gamma_x$が$\rho(f(x-),f(x-h)) < \epsilon$を満たすように取り,
		\begin{align}
			U_x \coloneqq (x-\gamma_x,x] \quad (\forall x \in (a,b])
		\end{align}
		とおく.特に$U_a \coloneqq (-\infty,a],\ V_b \coloneqq [b,\infty)$とおけば
		\begin{align}
			J \subset \bigcup_{x \in J}U_x \cup V_x
		\end{align}
		が成り立つが,$J$は$\R$のコンパクト部分集合であるから,このうち有限個を選び
		\begin{align}
			J = \bigcup_{i=1}^n \left( U_{x_i} \cup V_{x_i}\right) \cap J
		\end{align}
		とできる.$U_{x_i} \cap J,\ V_{x_i} \cap J$での$f$の挙動の振れ幅は$2\epsilon$で抑えられるから
		$J$全体での挙動の振れ幅は$2n\epsilon$より小さい
		\footnote{
			$x_1 < x_2 < \cdots < x_n$と仮定し,区間$U_{x_i} \cup V_{x_i}$と$U_{x_{i+1}} \cup V_{x_{i+1}}$の共通点を一つ取り$z_i$と表す.
			$\rho(f(x),f(y))\ (x,y \in J)$の上界を知りたいから$x \in U_{x_1} \cup V_{x_1},\ y \in U_{x_n} \cup V_{x_n}$の場合を調べればよい.このとき
			\begin{align}
				\rho(f(x),f(y)) &\leq \rho(f(x),f(x_1)) + \rho(f(x_1),f(x_2)) + \cdots + \rho(f(x_{n-1}),f(x_n)) + \rho(f(x_n),f(y)) \\
				&\leq \rho(f(x),f(x_1)) + \rho(f(x_1),f(z_1)) + \rho(f(z_1),f(x_2)) + \cdots + \rho(f(z_{n-1}),f(x_n)) + \rho(f(x_n),f(y)) \\
				& < 2n\epsilon
			\end{align}
			が成り立つ.
		}.
		ゆえに有界である.
		\QED
	\end{prf}
	
	\begin{prf}[命題\ref{prp:M_pc_M_cloc}]
		$M \in \mathcal{M}_{p,c}$が
		\begin{align}
			K \coloneqq \Norm{M_0}{\mathscr{L}^\infty} < \infty
		\end{align}
		を満たすと仮定する.全ての$\omega \in \Omega$に対し写像$I \ni t \longmapsto M_t(\omega)$
		は右連続且つ左極限を持つから,定理\ref{thm:closed_set_stopping_time}より
		\begin{align}
			\tau_j(\omega) \coloneqq \inf{}{\Set{t \in I}{|M_t(\omega)| \geq j}} \wedge T \quad (\forall \omega \in \Omega,\ j=1,2,\cdots)
		\end{align}
		として$\tau_j$は停止時刻となり,かつ$t \longmapsto M_t(\omega)$が連続となる$\omega$に対しては
		\begin{align}
			\sup{t \in I}{\left| M_{t \wedge \tau_j(\omega)}(\omega) \right|} \leq j \vee K
			\label{eq:M_pc_M_cloc}
		\end{align}
		が成り立つ.また$t \longmapsto M_t(\omega)$の右連続性から$\tau_j \leq \tau_{j+1}\ (j=1,2,\cdots)$となり,
		更に補題\ref{lem:rcll_bounded}より$\sup{t \in I}{|M_t(\omega)|} < \infty\ (\forall \omega \in \Omega)$も成り立つから
		$j_\omega > \sup{t \in I}{|M_t(\omega)|}$となるような$j_\omega$に対し$\tau_j(\omega) = T\ (\forall j \geq j_\omega)$を満たす.従って$(\tau_j)_{j=1}^{\infty} \in \mathcal{T}$である.
		後は$M^j$が$\mathcal{M}_{b,c}$に属することを示せばよい.先ずDoobの不等式(定理\ref{thm:Doob_inequality_2})より
		$\sup{t \in I}{|M_t|}$が$p$乗可積分となることから$M_t^j = M_{t \wedge \tau_j}$の可積分性が従う.
		また式(\refeq{eq:M_pc_M_cloc})より
		\begin{align}
			\Norm{M_t^j}{\mathscr{L}^\infty} \leq j \vee K \quad (\forall t \in I)
		\end{align}
		が成り立ち,$t \longmapsto M_t(\omega)$が連続となる$\omega$に対しては$I \ni t \longmapsto M_t^j(\omega)$もまた連続,そして
		任意抽出定理(定理\ref{thm:optional_sampling_theorem_2})より
		\begin{align}
			\cexp{M_t^j}{\mathcal{F}_s} = M_{t \wedge \tau_j \wedge s} = M_s^j \quad (\forall 0 \leq s < t \leq T)
		\end{align}
		を得る.以上より$M^j \in \mathcal{M}_{b,c}\ (j=1,2,\cdots)$,すなわち$M \in \mathcal{M}_{c,loc}$である.
		\QED
	\end{prf}
	
	\begin{itembox}[l]{}
		\begin{lem}
			$X \in \mathcal{M}_{2,c}$と停止時刻$\tau \geq \sigma$に対し,$\Norm{M_\sigma}{\mathscr{L}^\infty} < \infty$ならば次が成り立つ:
			\begin{description}
				\item[(1)] $\Exp{(X_{\tau} - X_{\sigma})^2} = \Exp{X_{\tau}^2 - X_{\sigma}^2}$,
				\item[(2)] $\cexp{(X_{\tau} - X_{\sigma})^2}{\mathcal{F}_\sigma} = \cexp{X_{\tau}^2 - X_{\sigma}^2}{\mathcal{F}_\sigma}$.
			\end{description}
			\label{lem:stopping_time_telescopic_sum}
		\end{lem}
	\end{itembox}
	
	\begin{prf}\mbox{}
		以下の式中では関数ではなく関数類を扱う.
		\begin{description}
			\item[(1)] 
				$\mathcal{G} = \{\Omega,\emptyset\}$とおく.条件付き期待値の性質(定理\ref{thm:conditional_exp_expansion})を使えば
				\begin{align}
					\Exp{(X_{\tau} - X_{\sigma})^2} &= \cexp{(X_{\tau} - X_{\sigma})^2}{\mathcal{G}} \\
					&= \cexp{X_{\tau}^2 + X_{\sigma}^2}{\mathcal{G}} - 2\cexp{X_{\tau}X_{\sigma}}{\mathcal{G}} \\
					&= \cexp{X_{\tau}^2 + X_{\sigma}^2}{\mathcal{G}} - 2\cexp{\cexp{X_{\tau}X_{\sigma}}{\mathcal{F}_\sigma}}{\mathcal{G}} \\
					&= \cexp{X_{\tau}^2 + X_{\sigma}^2}{\mathcal{G}} - 2\cexp{X_{\sigma}\cexp{X_{\tau}}{\mathcal{F}_\sigma}}{\mathcal{G}} 
						&& (\because\mbox{\scriptsize 定理\ref{thm:measurability_of_stopping_time}}) \\
					&= \cexp{X_{\tau}^2 + X_{\sigma}^2}{\mathcal{G}} - 2\cexp{X_{\sigma}^2}{\mathcal{G}} 
						&& (\because\mbox{\scriptsize 定理\ref{thm:optional_sampling_theorem_2}}) \\
					&= \cexp{X_{\tau}^2 - X_{\sigma}^2}{\mathcal{G}} \\
					&= \Exp{X_{\tau}^2 - X_{\sigma}^2}
				\end{align}
				が成り立つ.
			
			\item[(2)]
				(1)と同様に
				\begin{align}
					\cexp{(X_{\tau} - X_{\sigma})^2}{\mathcal{F}_\sigma}
					&= \cexp{X_{\tau}^2 + X_{\sigma}^2}{\mathcal{F}_\sigma} - 2\cexp{X_{\tau}X_{\sigma}}{\mathcal{F}_\sigma} \\
					&= \cexp{X_{\tau}^2 + X_{\sigma}^2}{\mathcal{F}_\sigma} - 2X_{\sigma}^2 \\
					&= \cexp{X_{\tau}^2 + X_{\sigma}^2}{\mathcal{F}_\sigma} - 2\cexp{X_{\sigma}^2}{\mathcal{F}_\sigma} \\
					&= \cexp{X_{\tau}^2 - X_{\sigma}^2}{\mathcal{F}_\sigma}
				\end{align}
				が成り立つ.
		\end{description}
		\QED
	\end{prf}
	
	\begin{itembox}[l]{}
		\begin{prp}[有界変動かつ連続な二乗可積分マルチンゲールのパスは定数となる]\mbox{}\\
			$A \in \mathcal{A} \cap \mathcal{M}_{2,c}$に対し,$\Norm{A_0}{\mathscr{L}^\infty} < \infty$ならば$A_t = A_0\ (\forall t \in I)\ $$\mu$-a.s.が成り立つ.
			\label{prp:bounded_continuous_M_2c_path}
		\end{prp}
	\end{itembox}
	
	\begin{prf}
		$A \in \mathcal{A}$であるから,$A$に対し或る$A^{(1)},A^{(2)} \in \mathcal{A}^+$が存在して
		\begin{align}
			A = A^{(1)} - A^{(2)}
		\end{align}
		と表現できる.或る$\mu$-零集合$E$を取れば,全ての$\omega \in \Omega \backslash E$に対し写像$I \ni t \longmapsto A_t^{(1)}(\omega)$と
		$I \ni t \longmapsto A_t^{(2)}(\omega)$が連続且つ単調非減少となるようにできるから,
		\begin{align}
			\tau_m(\omega) \coloneqq
			\begin{cases}
				0 & (\omega \in E) \\
				\inf{}{\Set{t \in I}{\left( A_t^{(1)}(\omega) - A_0^{(1)} \right) \vee \left( A_t^{(2)}(\omega) - A_0^{(2)} \right) \geq m}} & (\omega \in \Omega \backslash E)
			\end{cases}
			\quad (m=1,2,\cdots)
		\end{align}
		と定義すれば,定理\ref{thm:closed_set_stopping_time}よりこれは停止時刻となる.また連続性から$\tau_m \leq \tau_{m+1}$となり
		更に$\lim_{m \to \infty}\tau_m(\omega) = T\ (\forall \omega \in \Omega \backslash E)も成り立つ.
		$今$t \in I,\ n,m \in \N$を任意に取って固定する.
		\begin{align}
			\sigma_j^n \coloneqq \tau_m \wedge \frac{tj}{2^n} \quad (j = 0,1,\cdots, 2^n)
		\end{align}
		とおけば,補題\ref{lem:stopping_time_telescopic_sum}により
		\footnote{
			補題の有界性の仮定を満たしていることを確認する.任意の$j \in \N$番目の$\sigma_j^n$を取る.
			$\omega \in \Omega \backslash E$に対し写像$t \longmapsto A_t(\omega)$は連続であるから
			\begin{align}
				\left| A_{\sigma_j^n(\omega)}(\omega) - A_0(\omega) \right| 
				\leq \left| A^{(1)}_{\tau_m(\omega)\wedge \frac{tj}{2^n}}(\omega) - A^{(1)}_0(\omega) \right| + \left| A^{(2)}_{\tau_m(\omega)\wedge \frac{tj}{2^n}}(\omega) - A^{(2)}_0(\omega) \right|
				\leq 2m
			\end{align}
			が成り立つ.また補題\ref{lem:holder_inequality}より或る零集合$E'$を除いて$|A_0| \leq \Norm{A_0}{\mathscr{L}^\infty}$が成り立っているから,
			\begin{align}
				\left| A_{\sigma_j^n(\omega)}(\omega) \right| \leq 2m + \Norm{A_0}{\mathscr{L}^\infty} \quad (\forall \omega \in \Omega \backslash (E \cup E'))
			\end{align}
			となり$\Norm{A_{\sigma_j^n}}{\mathscr{L}^\infty} \leq 2m + \Norm{A_0}{\mathscr{L}^\infty}$であると判る.
		}
		\begin{align}
			\Exp{\sum_{j=0}^{2^n-1} \left( A_{\sigma_{j+1}^n} - A_{\sigma_j^n} \right)^2}
			= \sum_{j=0}^{2^n-1} \Exp{A_{\sigma_{j+1}^n}^2 - A_{\sigma_j^n}^2}
			= \Exp{A_{\tau_m \wedge t}^2 - A_{0}^2}
			= \Exp{\left( A_{\tau_m \wedge t} - A_{0} \right)^2}
		\end{align}
		が成り立つ.左辺の中の式は
		\begin{align}
			\sum_{j=0}^{2^n-1} \left( A_{\sigma_{j+1}^n} - A_{\sigma_j^n} \right)^2
			\leq \sup{j}{\left| A_{\sigma_{j+1}^n} - A_{\sigma_j^n} \right|} \sum_{j=0}^{2^n-1} \left| A_{\sigma_{j+1}^n} - A_{\sigma_j^n} \right|
		\end{align}
		となり,全ての$\omega \in \Omega \backslash E$に対して$I \ni t \longmapsto A_t(\omega)$は(一様)連続だから
		\begin{align}
			\sup{j}{\left| A_{\sigma_{j+1}^n}(\omega) - A_{\sigma_j^n}(\omega) \right|} \longrightarrow 0 \quad (n \longrightarrow \infty).
		\end{align}
		また定理\ref{thm:closed_set_stopping_time}より全ての$\omega \in \Omega \backslash E$と$t \in I$に対して
		\begin{align}
			\left( A^{(1)}_{\tau_m \wedge t}(\omega) - A^{(1)}_0(\omega) \right) \vee \left( A^{(2)}_{\tau_m \wedge t}(\omega) - A^{(2)}_0(\omega) \right) \leq m
		\end{align}
		が成り立つから
		\begin{align}
			\sum_{j=0}^{2^n-1} \left| A_{\sigma_{j+1}^n}(\omega) - A_{\sigma_j^n}(\omega) \right|
			&\leq \sum_{j=0}^{2^n-1} \left( A^{(1)}_{\sigma_{j+1}^n}(\omega) - A^{(1)}_{\sigma_j^n}(\omega) + A^{(2)}_{\sigma_{j+1}^n}(\omega) - A^{(2)}_{\sigma_j^n}(\omega) \right) \\
			&= \left( A^{(1)}_{\tau_m \wedge t}(\omega) - A^{(1)}_0(\omega) \right) + \left( A^{(2)}_{\tau_m \wedge t}(\omega) - A^{(2)}_0(\omega) \right) \leq 2m
		\end{align}
		となり,Lebesgueの収束定理により
		\begin{align}
			\int_\Omega \sum_{j=0}^{2^n-1} \left( A_{\sigma_{j+1}^n(\omega)}(\omega) - A_{\sigma_j^n(\omega)}(\omega) \right)^2\ \mu(d\omega) \longrightarrow 0 \quad (n \longrightarrow \infty)
		\end{align}
		を得る.ゆえに
		\begin{align}
			\int_\Omega \left( A_{\tau_m(\omega) \wedge t}(\omega) - A_{0}(\omega) \right)^2\ \mu(d\omega) = 0 \quad (m=1,2,\cdots)
		\end{align}
		が成り立ち,更にDoobの不等式より$|A_{\tau_m \wedge t} - A_{0}| \leq \sup{t \in I}{|A_t - A_{0}|} \in \mathscr{L}^2$であるから,
		再びLebesgueの収束定理を適用して
		\begin{align}
			\int_\Omega \left( A_{\tau_m(\omega) \wedge t}(\omega) - A_{0}(\omega) \right)^2\ \mu(d\omega)
			\longrightarrow \int_\Omega \left( A_t(\omega) - A_{0}(\omega) \right)^2\ \mu(d\omega) \quad (n \longrightarrow \infty)
		\end{align}
		を得る.$t \in I$は任意に取っていたからつまり
		\begin{align}
			A_t = A_0 \quad \mbox{$\mu$-a.s.} \quad (\forall t \in I)
		\end{align}
		が示されたが,実際$\omega \in \Omega \backslash E$に対しパスは連続であるから
		\begin{align}
			\Set{\omega \in \Omega \backslash E}{A_t(\omega) = A_0(\omega)\ (\forall t \in I)}
			= \bigcap_{r \in I \cap \Q} \Set{\omega \in \Omega \backslash E}{A_r(\omega) = A_0(\omega)}
		\end{align}
		と表せる.
		\begin{align}
			\Set{\omega \in \Omega \backslash E}{A_t(\omega) \neq A_0(\omega)\ (\exists t \in I)}
			\subset E + \bigcup_{r \in I \cap \Q} \Set{\omega \in \Omega \backslash E}{A_r(\omega) \neq A_0(\omega)}
		\end{align}
		の右辺は零集合であるから$A_t = A_0\ (\forall t \in I)\ $$\mu$-a.s.が成り立つ.
		\QED
	\end{prf}
	
	\begin{itembox}[l]{}
		\begin{lem}[二次変分補題]
			任意に$n \in \N$と$M \in \mathcal{M}_{b,c}$を取る.$\tau_j^n = jT/2^n\ (j=0,1,\cdots,2^n)$に対し
			\begin{align}
				Q_t^n \coloneqq \sum_{j=0}^{2^n-1} \left( M_{t \wedge \tau_{j+1}^n} - M_{t \wedge \tau_j^n} \right)^2 \quad (\forall t \in I)
				\label{eq:lem_quadratic_variation_0}
			\end{align}
			とおけば$M^2 - Q^n \in \mathcal{M}_{b,c}$となり,さらに次が成り立つ:
			\begin{align}
				\Norm{M_T - M_0 - Q_T^n}{\mathscr{L}^2} \leq 2 \sup{t \in I}{\Norm{M_t}{\mathscr{L}^\infty}} \Norm{M_T - M_0}{\mathscr{L}^2}.
			\end{align}
			\label{lem:quadratic_variation}
		\end{lem}
	\end{itembox}
	
	\begin{prf}
		先ず$Q^n$は$\mathcal{F}_t$-適合である.これは任意の停止時刻$\tau$に対し$M_{t \wedge \tau}$が可測$\mathcal{F}_t/\borel{\R}$であることによる.
		今任意に$0 \leq s < t \leq T$を取り固定する.$\tau_k^n \leq s < \tau_{k+1}^n$となる$k$を選べば,
		補題\ref{lem:stopping_time_telescopic_sum}と任意抽出定理\ref{thm:optional_sampling_theorem_2}を使って次のように式変形できる:
		\begin{align}
			\cexp{Q_t^n - Q_s^n}{\mathcal{F}_s} 
			&= \cexp{\sum_{j=0}^{2^n-1}\left( M_{t\wedge\tau_{j+1}^n} - M_{t\wedge\tau_j^n} \right)^2 - \sum_{j=0}^{2^n-1}\left( M_{s\wedge\tau_{j+1}^n} - M_{s\wedge\tau_j^n} \right)^2}{\mathcal{F}_s} \\
			&= \cexp{\sum_{j=k}^{2^n-1}\left\{ \left( M_{t\wedge\tau_{j+1}^n} - M_{t\wedge\tau_j^n} \right)^2 - \left( M_{s\wedge\tau_{j+1}^n} - M_{s\wedge\tau_j^n} \right)^2 \right\}}{\mathcal{F}_s} \\
			&= \sum_{j=k+1}^{2^n-1} \cexp{\left( M_{t\wedge\tau_{j+1}^n} - M_{t\wedge\tau_j^n} \right)^2}{\mathcal{F}_s}
				+ \cexp{\left( M_{t\wedge\tau_{k+1}^n} - M_{t\wedge\tau_k^n} \right)^2}{\mathcal{F}_s} - \cexp{\left( M_s - M_{\tau_k^n} \right)^2}{\mathcal{F}_s} \\
			&= \sum_{j=k+1}^{2^n-1} \cexp{ M_{t\wedge\tau_{j+1}^n}^2 - M_{t\wedge\tau_j^n}^2}{\mathcal{F}_s}
				+ \cexp{\left( M_{t\wedge\tau_{k+1}^n} - M_{\tau_k^n} \right)^2}{\mathcal{F}_s} - \left( M_s - M_{\tau_k^n} \right)^2 \\
			&= \cexp{ M_t^2 - M_{t\wedge\tau_{k+1}^n}^2}{\mathcal{F}_s} + \cexp{\left( M_{t\wedge\tau_{k+1}^n} - M_{\tau_k^n} \right)^2}{\mathcal{F}_s} - \left( M_s - M_{\tau_k^n} \right)^2 \\
			&= \cexp{M_t^2}{\mathcal{F}_s} - 2\cexp{M_{t\wedge\tau_{k+1}^n}M_{\tau_k^n}}{\mathcal{F}_s} + \cexp{M_{\tau_k^n}^2}{\mathcal{F}_s} - M_s^2 + 2M_sM_{\tau_k^n} - M_{\tau_k^n}^2 \\
			&= \cexp{M_t^2}{\mathcal{F}_s} - 2M_{\tau_k^n}\cexp{M_{t\wedge\tau_{k+1}^n}}{\mathcal{F}_s} + M_{\tau_k^n}^2 - M_s^2 + 2M_sM_{\tau_k^n} - M_{\tau_k^n}^2 \\
			&= \cexp{M_t^2}{\mathcal{F}_s} - M_s^2.
		\end{align}
		従って次を得た:
		\begin{align}
			\cexp{M_t^2 - Q_t^n}{\mathcal{F}_s} = M_s^2 - Q_s^n, \quad (\forall 0 \leq s < t \leq T).
			\label{eq:lem_quadratic_variation_1}
		\end{align}
		ここで
		\begin{align}
			N \coloneqq M^2 - Q^n
		\end{align}
		とおけば$N \in \mathcal{M}_{b,c}$であり
		\footnote{
			$M \in \mathcal{M}_{b,c}$より全ての$\omega \in \Omega$において写像$t \longmapsto M_t(\omega)$は各点で右連続かつ左極限を持つ.
			$Q^n$についてもGauss記号を用いて$Q_t^n = \sum_{j=0}^{[2^nt]/T} \left( M_t^n - M_{\tau_j^n} \right)^2$
			と表せば,$t \longmapsto Q_t^n(\omega)\ (\forall \omega \in \Omega)$が各点で右連続かつ左極限を持つことが明確になる.
			よって全ての$\omega \in \Omega$において$t \longmapsto N_t(\omega)$は各点で右連続かつ左極限を持つ.
			また同じ理由で$t \longmapsto M_t(\omega)$が連続となる点で$t \longmapsto N_t(\omega)$も連続となるから
			つまり$\mu$-a.s.に$t \longmapsto N_t$は連続.
			一様有界性については,$\sup{t \in I}{\Norm{M_t}{\mathscr{L}^\infty}} < \infty$であるから,任意の$t \in I$に対し
			或る零集合$E_t$が存在して$\omega \notin E_t$なら$|M_t(\omega)| \leq \sup{t \in I}{\Norm{M_t}{\mathscr{L}^\infty}}$が成り立つ.
			同様に$Q_t^n$についても$\omega \notin E_t \cup \bigcup_{j=0}^{[2^nt]/T}E_{\tau_j^n}$なら
			\begin{align}
				\left| Q_t^n(\omega) \right| \leq \sum_{j=0}^{[2^nt]/T} \left( 2\sup{t \in I}{\Norm{M_t}{\mathscr{L}^\infty}} \right)^2 \leq 2^{n+1} \sup{t \in I}{\Norm{M_t}{\mathscr{L}^\infty}^2}.
			\end{align}
			ゆえに
			\begin{align}
				\left| N_t(\omega) \right| \leq \left|{M_t(\omega)}^2\right| + \left|Q_t^n(\omega)\right| \leq \left( 2^{n+1}+1 \right) \sup{t \in I}{\Norm{M_t}{\mathscr{L}^\infty}^2}
				,\quad \left( \forall \omega \notin E_t \cup \cup_{j=0}^{[2^nt]/T}E_{\tau_j^n} \right).
			\end{align}
			この右辺は
			$t$に依らないから
			\begin{align}
				\sup{t \in I}{\Norm{N_t}{\mathscr{L}^\infty}} \leq \left( 2^{n+1}+1 \right) \sup{t \in I}{\Norm{M_t}{\mathscr{L}^\infty}^2}
			\end{align}
			を得る.以上の結果と(\refeq{eq:lem_quadratic_variation_1})を併せて$N \in \mathcal{M}_{b,c}$となる.
		},
		\begin{align}
			\Exp{(N_T - N_0)^2} = \Exp{N_T^2 - N_0^2} &= \Exp{\sum_{j=0}^{2^n-1}\left( N_{\tau_{j+1}^n}^2 - N_{\tau_j^n}^2 \right)} \\
			&= \sum_{j=0}^{2^n-1}\Exp{\left( N_{\tau_{j+1}^n} - N_{\tau_j^n} \right)^2} \\
			&= \sum_{j=0}^{2^n-1}\Exp{\left\{ M_{\tau_{j+1}^n}^2 - M_{\tau_j^n}^2 - \left( Q_{\tau_{j+1}^n}^n - Q_{\tau_j^n}^n \right) \right\}^2} \\
			&= \sum_{j=0}^{2^n-1}\Exp{\left\{ M_{\tau_{j+1}^n}^2 - M_{\tau_j^n}^2 - \left( M_{\tau_{j+1}^n} - M_{\tau_j^n} \right)^2 \right\}^2} \\
			&= \sum_{j=0}^{2^n-1}\Exp{\left\{ -2M_{\tau_j^n} \left( M_{\tau_{j+1}^n} - M_{\tau_j^n} \right) \right\}^2} \\
			&= 4 \Exp{ \sum_{j=0}^{2^n-1} M_{\tau_j^n}^2 \left( M_{\tau_{j+1}^n} - M_{\tau_j^n} \right)^2 } \\
			&\leq 4 \sup{t \in I}{\Norm{M_t}{\mathscr{L}^\infty}^2} \Exp{\sum_{j=0}^{2^n-1} \left( M_{\tau_{j+1}^n} - M_{\tau_j^n} \right)^2 } \\
			&= 4 \sup{t \in I}{\Norm{M_t}{\mathscr{L}^\infty}^2} \Exp{M_T^2 - M_0^2}
		\end{align}
		が成り立つ.これより
		\begin{align}
			\Norm{M_T - M_0 - Q_T^n}{\mathscr{L}^2} \leq 2 \sup{t \in I}{\Norm{M_t}{\mathscr{L}^\infty}} \Norm{M_T - M_0}{\mathscr{L}^2}.
		\end{align}
		\QED
	\end{prf}
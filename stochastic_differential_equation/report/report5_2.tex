\section{二次変分}
	以降では$I \coloneqq [0,T]\ (T>0)$とし,
	このフィルトレーション$(\mathcal{F}_t)_{t \in I}$が次の仮定を満たす:
	\begin{align}
		\mathcal{N} \coloneqq \Set{N \in \mathcal{F}}{\mu(N) = 0}
		\subset \mathcal{F}_0.
	\end{align}
	
	\begin{screen}
		\begin{dfn}[停止時刻で停めた過程]
			任意の停止時刻$\tau$と確率過程$M$に対し
			\begin{align}
				M^\tau_t \coloneqq M_{t \wedge \tau}
				\quad (\forall t \in I)
			\end{align}
			として定義する$M^\tau$を,停止時刻$\tau$で停めた過程という.
			
		\end{dfn}
	\end{screen}
	
	\begin{screen}
		\begin{prp}[停めた過程の適合性]
			確率過程$M$の全てのパスが右連続且つ$(\mathcal{F}_t)$-適合のとき,任意の停止時刻$\tau$に対し
			$M^\tau$もまた右連続且つ$(\mathcal{F}_t)$-適合である.
		\end{prp}
	\end{screen}
	
	\begin{prf}
		任意に$\omega \in \Omega$を取り固定する.
		写像$t \longmapsto M_t^\tau(\omega) = M_{t \wedge \tau(\omega)}(\omega)$
		について,$t < \tau(\omega)$なら$t \longmapsto M_t(\omega)$の右連続性により,
		$t \geq \tau(\omega)$なら右側で定数関数となるから右連続性が従う.また
		定理\ref{thm:measurability_of_stopping_time}より
		$M_t^{\tau}$は可測$\mathcal{F}_{t \wedge \tau}/\borel{\R}$
		であるから,$\mathcal{F}_{t\wedge \tau} \subset \mathcal{F}_t$より$M^\tau$の適合性が従う.
		\QED
	\end{prf}
	
	以下,いくつか集合を定義する.
	\begin{description}
		\item[$\mathrm{(1)}\ \mathcal{A}^+$] 
			$\mathcal{A}^+$は以下を満たす$(\Omega,\mathcal{F},\mu)$上の可測関数族$A = (A_t)_{t \in I}$の全体として定める:
			\begin{description}
				\item[適合性] 任意の$t \in I$に対し,写像$\Omega \ni \omega \longmapsto A_t(\omega) \in \R$は可測$\mathcal{F}_t/\borel{\R}$である.
				\item[連続性] $\mu$-a.s.に写像$I \ni t \longmapsto A_t(\omega) \in \R$が連続である.
				\item[単調非減少性] $\mu$-a.s.に写像$I \ni t \longmapsto A_t(\omega) \in \R$が単調非減少である.
			\end{description}
		
		\item[$\mathrm{(2)}\ \mathcal{A}$]
			$\mathcal{A}$を$\mathcal{A} \coloneqq \Set{A^1 - A^2}{A^1,A^2 \in \mathcal{A}^+}$
			により定める.任意の$A = A_1 - A_2 \in \mathcal{A}$に対し,$A$は適合過程であり
			$\mu$-a.s.に$I \ni t \longmapsto A_t$は連続且つ有界変動である.
			
		\item[$\mathrm{(3)}\ \mathcal{M}_{p,c}\ (p \geq 1)$]
			$\mathcal{M}_{p,c}$を以下を満たす可測関数族$M = (M_t)_{t \in I} \subset \semiLp{p}{\mathcal{F},\mu}$の全体として定める.
			\begin{description}
				\item[0出発] $M_0 = 0 \quad \mbox{$\mu$-a.s.}$を満たす.
				\item[$\mathrm{L}^p$-マルチンゲール] $M = (M_t)_{t \in I}$は$\mathrm{L}^p$-マルチンゲールである.
				\item[連続性] $\mu$-a.s.$\omega$に対し写像$I \ni t \longmapsto M_t(\omega) \in \R$が連続である.
			\end{description}
		
		\item[$\mathrm{(4)}\ \mathcal{M}_{b,c}$]
			$\mathcal{M}_{b,c}$を一様有界な$\mathrm{L}^1$-マルチンゲールの全体として定める:
			\begin{align}
				\mathcal{M}_{b,c} \coloneqq \Set{M = (M_t)_{t \in I} \in \mathcal{M}_{1,c}}{\sup{t \in I}{\Norm{M_t}{\mathscr{L}^\infty}} < \infty}.
			\end{align}
			
		\item[$\mathrm{(5)}\ \mathcal{T}$]
			$\mathcal{T}$を以下を満たすような,$I$に値を取る停止時刻の列$(\tau_j)_{j=0}^{\infty}$の全体として定める.
			\begin{description}
				\item[a)] $\tau_0 = 0 \quad \mbox{$\mu$-a.s.}$
				\item[b)] $\tau_j \leq \tau_{j+1} \quad \mbox{$\mu$-a.s.}\ (j=1,2,\cdots).$
				\item[c)] $(\tau_j)_{j=1}^{\infty}$に対し或る$\mu$-零集合$N_T$が存在し,任意の$\omega \in \Omega \backslash N_T$に対し或る$n = n(\omega) \in \N$が存在して$\tau_n(\omega)=T$が成り立つ.
			\end{description}
			例えば$\tau_j = jT/2^n$なら$(\tau_j)_{j=0}^{\infty} \in \mathcal{T}$となる.
			上の条件において,$a)$が零集合$N_0$を除いて成立し,$b)$が各$j$について零集合$N_j$を除いて成立するとき,
			$N \coloneqq N_0 \cup N_T \cup (\cup_{j=0}^{\infty}N_j)$とすればこれも$\mu$-零集合であり,任意の$\omega \in \Omega \backslash N$に対して
			\begin{align}
				\tau_0(\omega) = 0,\quad \tau_j(\omega) \leq \tau_{j+1}(\omega)\ (j=1,2,\cdots),\quad
				\tau_n(\omega) = T\ (\exists n = n(\omega) \in \N)
			\end{align}
			が成立する.
			
		\item[$\mathrm{(4)}\ \mathcal{M}_{c,loc}$]
			$\mathcal{M}_{c,loc}$を次で定める.$\mathcal{M}_{c,loc}$の元を「連続な局所マルチンゲール」という:
			\begin{align}
				\mathcal{M}_{c,loc} \coloneqq 
				\Set{M = (M_t)_{t \in I} \subset \semiLp{1}{\mathcal{F},\mu}}{\substack{\mbox{全ての$\omega \in \Omega$に対し,$I \ni t \longmapsto M_t(\omega)$が} \\ \mbox{各点$t$で右連続且つ左極限を持ち,} \\ \mbox{或る$(\tau_j)_{j=0}^{\infty} \in \mathcal{T}$が存在して} \\ \mbox{$M^{\tau_j} \in \mathcal{M}_{b,c}\ (\forall j =0,1,\cdots)$を満たす.}}}.
			\end{align}
	\end{description}
	
	\begin{screen}
		\begin{thm}[有界なマルチンゲールを停止時刻で停めた過程の有界性]
			$\tau$を任意の停止時刻とする.任意の$M \in \mathcal{M}_{b,c}$に対し
			$\sup{t \in I}{\Norm{M_t^\tau}{\mathscr{L}^\infty}} \leq \sup{t \in I}{\Norm{M_t}{\mathscr{L}^\infty}}$
			が成り立つ.
			\label{thm:boundedness_of_stopped_process_of_bounded_martingale}
		\end{thm}
	\end{screen}
	
	\begin{prf}
		任意に$s \in I$を取り固定する.
		$|M_{s \wedge \tau(\omega)}(\omega)| \leq \sup{t \in I}{|M_t(\omega)|}\ (\forall \omega \in \Omega)$より
		\begin{align}
			\sup{t \in I}{|M_t|} \leq \sup{t \in I}{\Norm{M_t}{\mathscr{L}^\infty}}
			\quad \mbox{$\mu$-a.s.}
		\end{align}
		が成り立つことを示せばよい.$M \in \mathcal{M}_{b,c}$であるから,
		或る零集合$A$が存在して全ての$\omega \in \Omega \backslash A$に対し
		$I \ni t \longmapsto M_t(\omega)$が連続である.
		また補題\ref{lem:holder_inequality}より
		\begin{align}
			B_r \coloneqq \Set{\omega \in \Omega}{|M_r(\omega)| > \Norm{M_r}{\mathscr{L}^\infty}},
			\quad (r \in \Q \cap I)
		\end{align}
		は全て零集合であるから,
		\begin{align}
			B \coloneqq \bigcup_{r \in \Q \cap I} B_r
		\end{align}
		に対し$C \coloneqq A \cup B$として零集合を定める.
		任意の$\omega \in \Omega \backslash C$に対し
		$t \longmapsto M_t(\omega)$が連続であるから
		\begin{align}
			|M_u(\omega)| \leq \sup{t \in I}{\Norm{M_t}{\mathscr{L}^\infty}}
			\quad (\forall u \in I)
		\end{align}
		が成り立ち
		\begin{align}
			\sup{t \in I}{|M_t(\omega)|} \leq \sup{t \in I}{\Norm{M_t}{\mathscr{L}^\infty}}
			\quad (\forall \omega \in \Omega \backslash C)
		\end{align}
		が従う.
		\QED
	\end{prf}
	
	\begin{screen}
		\begin{thm}[停止時刻で停めてもマルチンゲール]
			$p > 1$とする.任意の$M \in \mathcal{M}_{p,c}$と停止時刻$\tau$に対し,
			$M$を$\tau$で停めた過程について$M^\tau \in \mathcal{M}_{p,c}$が成り立つ.
			\label{thm:stopped_process_martingale}
		\end{thm}
	\end{screen}
	
	\begin{prf}
		任意の$\omega \in \Omega$に対し$I \ni t \longmapsto M_t(\omega)$は各点で右連続且つ左極限を持つから
		$I \ni t \longmapsto M^{\tau}_t(\omega)$も各点で右連続且つ左極限を持ち,
		特に$I \ni t \longmapsto M_t(\omega)$が連続となる$\omega$に対しては
		$I \ni t \longmapsto M^{\tau}_t(\omega)$も連続である.
		また$M_0 = M^{\tau}_0$より$M^{\tau}_0 = 0\ \mu$-a.s.が従う.あとは$M^{\tau}$の
		$p$乗可積分性と$\cexp{M^\tau_t}{\mathcal{F}_s} = M^\tau_s\ (s < t)$を示せばよい.
		実際Doobの不等式(定理\ref{thm:Doob_inequality_2})より$\sup{t \in I}{|M_t|^p}$
		が可積分であるから,全ての$t \in I$に対し$\left| M^\tau_t \right|$は$p$乗可積分であり,
		更に任意抽出定理(定理\ref{thm:optional_sampling_theorem_2})より
		\begin{align}
			\cexp{M^\tau_t}{\mathcal{F}_s} = M_{t \wedge \tau \wedge s} = M^\tau_s \quad (s < t)
		\end{align}
		が従う.
		\QED
	\end{prf}
	
	\begin{screen}
		\begin{lem}[$\mathcal{M}_{p,c}$は線形空間]
			任意の$M,N \in \mathcal{M}_{p,c}$と$\alpha \in \R$に対して線型演算を
			\begin{align}
				(M + N)_t \coloneqq (M_t + N_t)_{t \in I}, 
				\quad (\alpha M)_t \coloneqq (\alpha M_t)_{t \in I}
				\label{eq:mart_linear_arithmetic_0}
			\end{align}
			として定義すれば,$\mathcal{M}_{p,c}$は$\R$上の線形空間となる.
		\end{lem}
	\end{screen}
	
	\footnotetext{
		全ての$t,\omega$に対し$0 \in \R$を取るもの.
	}
	
	\begin{prf}
		$\mathcal{M}_{p,c}$が(\refeq{eq:mart_linear_arithmetic_0})の演算で閉じていることを示す.
		任意に$M,N \in \mathcal{M}_{p,c}$と$\alpha, \beta \in \R$を取る.
		\begin{description}
			\item[第一段]
				先ず$\alpha M + \beta N$が$\mathrm{L}^p$-マルチンゲールであることを示す.
				各$t \in I$に対し$M_t,N_t$が$\mathcal{F}_t$-可測であるから
				$\alpha M_t + \beta N_t$も$\mathcal{F}_t$-可測であり$\alpha M + \beta N$の適合性が従う.
				またMinkowskiの不等式より$\alpha M_t + \beta N_t \in \semiLp{p}{\mu}\ (\forall t \in I)$も従う.
				そして全ての$\omega \in \Omega$に対し$t \longmapsto M_t(\omega)$と
				$t \longmapsto N_t(\omega)$は各点で右連続且つ左極限を持つから
				$t \longmapsto \alpha M_t(\omega) + \beta N_t(\omega)$も各点で右連続且つ左極限を持ち,
				更に任意の$s,t \in I,\ s \leq t$に対して,条件付き期待値の線型性により
				\begin{align}
					\cexp{\alpha M_t + \beta M_t}{\mathcal{F}_s} 
					= \alpha \cexp{M_t}{\mathcal{F}_s} + \beta \cexp{N_t}{\mathcal{F}_s} 
					= \alpha M_s + \beta N_s
				\end{align}
				も成り立つ.以上より$\left( (\alpha M + \beta N)_t \right)_{t \in I}$は$\mathrm{L}^p$-マルチンゲールである.
			
			\item[第二段]
				$\alpha M + \beta N \in \mathcal{M}_{p,c}$を示す.
				実際,$\mu$-a.s.に$M_0 = N_0 = 0$であるから$\alpha M_0 + \beta N_0 = 0\ \mu$-a.s.が従い,
				同様に$\mu$-a.s.に$I \ni t \longmapsto M_t,\ I \ni t \longmapsto N_t$が連続であるから
				$\mu$-a.s.に$I \ni t \longmapsto (\alpha M + \beta N)_t$も連続である.
				第一段の結果と併せて$\alpha M + \beta N \in \mathcal{M}_{p,c}$を得る.
				\QED
		\end{description}		
	\end{prf}
	
	\begin{screen}
		\begin{lem}[$\mathcal{M}_{p,c}$における同値関係]
			任意の$M,N \in \mathcal{M}_{p,c}\ (p \geq 1)$に対して,関係$R$を
			\begin{align}
				M\ R\ N \DEF M_t = N_t \quad (\forall t \in I),\quad \mbox{$\mu$-a.s.}
			\end{align}
			により定めれば,これは$\mathcal{M}_{p,c}$における同値関係である.また関係$R$は次を満たす:
			\begin{align}
				M\ R\ N \Leftrightarrow \mu\left( \Set{\omega \in \Omega}{\sup{r \in (I \cap \Q) \cup \{ T \}}{\left|M_r(\omega) - N_r(\omega)\right| > 0}}\right) = 0.
			\end{align}
			\label{lem:M_2c_hilbert}
		\end{lem}
	\end{screen}
	
	\begin{prf}
		反射律と対称律は$R$の定義より従うから推移律について示す.$M,N,U \in \mathcal{M}_{p,c}$
		が$M\ R\ N$かつ$N\ R\ U$を満たしているなら,$M,N$のパスが一致しない$\omega$の全体$E_1$と
		$N,U$のパスが一致しない$\omega$の全体$E_2$は零集合に含まれるから$M\ R\ U$が従う.
		後半の主張を示す.
		$M,N$に対し或る零集合$E$が存在して,$\Omega \backslash E$上で
		$I \ni t \longmapsto M_t(\omega)$と$I \ni t \longmapsto N_t(\omega)$は共に連続である.
		\begin{align}
			F \coloneqq \Set{\omega \in \Omega}{\sup{r \in (I \cap \Q) \cup \{ T \}}{\left|M_r(\omega) - N_r(\omega)\right| > 0}}
		\end{align}
		とおけば,連続性により$F^c \cap E^c$上で$M$と$N$のパスが一致するから,$\mu(F) = 0$ならば
		$M\ R\ N$が成り立つ.逆に$M\ R\ N$ならば,或る零集合$G$が存在して$G^c$上で$M$と$N$のパスが一致する.
		\begin{align}
			G^c \subset F^c
		\end{align}
		が満たされているから$F \subset G$が従い$\mu(F) = 0$が成り立つ.
		\QED
	\end{prf}
	
	\begin{screen}
		\begin{dfn}[$\mathcal{M}_{p,c}$の商空間]
			補題\ref{lem:M_2c_hilbert}で導入した同値関係$R$による$\mathcal{M}_{p,c}$の商集合を$\mathfrak{M}_{p,c}$と表す.
			$M \in \mathcal{M}_{p,c}$の関係$R$による同値類を$\equiv{M}{2,c}$と表記し,
			$\mathfrak{M}_{p,c}$において
			\begin{align}
				\equiv{M}{2,c} + \equiv{N}{2,c} \coloneqq \equiv{M+N}{2,c}, 
				\quad \alpha \equiv{M}{2,c} \coloneqq \equiv{\alpha M}{2,c} \label{eq:mart_linear_arithmetic}
			\end{align}
			として線型演算を定めれば,これは代表の選び方に依存せず,
			$\mathfrak{M}_{p,c}$は$\R$上の線形空間となる.
		\end{dfn}
	\end{screen}
	
	\begin{screen}
		\begin{lem}[$\mathfrak{M}_{2,c}$における内積の定義]
			\begin{align}
				\inprod<\equiv{M}{2,c},\equiv{N}{2,c}>_{2,c} \coloneqq \int_{\Omega} M_T(\omega)N_T(\omega)\ \mu(d\omega), \quad (\equiv{M}{2,c},\equiv{N}{2,c} \in \mathfrak{M}_{2,c}).
				\label{eq:M_2c_inner_product}
			\end{align}
			により$\inprod<\cdot,\cdot>_{2,c}$を定めれば,右辺は代表に無関係に確定し,
			$\mathfrak{M}_{2,c}$において内積となる.
			\label{lem:M_2c_hilbert_inner_product}
		\end{lem}
	\end{screen}
			
	\footnotetext{
		
	}
			
	\begin{prf}\mbox{}
		\begin{description}
			\item[第一段]
				先ず$\inprod<\cdot,\cdot>_{2,c}$の定義が代表の取り方に依らないことを示す.
				任意の$\equiv{M}{2,c},\equiv{N}{2,c} \in \mathfrak{M}_{2,c}$に対し
				(\refeq{eq:M_2c_inner_product})の積分が実数値で確定することは,
				$M_T,N_T$の二乗可積分性とH\Ddot{o}lderの不等式による.
				また任意の$M' \in \equiv{M}{2,c}$と$N' \in \equiv{N}{2,c}$に対しては,
				$M_T = M'_T,\ N_T = N'_T\ \mu$-a.s.により
				\begin{align}
					\int_{\Omega} M_T(\omega)N_T(\omega)\ \mu(d\omega) = \int_{\Omega} M'_T(\omega)N'_T(\omega)\ \mu(d\omega)
				\end{align}
				が成り立つから,$\inprod<\equiv{M}{2,c},\equiv{N}{2,c}>_{2,c}$は代表に無関係に確定する.
	
			\item[第二段]
				$\inprod<\cdot,\cdot>_{2,c}:\mathfrak{M}_{2,c} \times \mathfrak{M}_{2,c} \rightarrow \R$が内積であることを示す.
				先ず任意の$\equiv{M}{2,c} \in \mathfrak{M}_{2,c}$に対して(\refeq{eq:M_2c_inner_product})より
				$\inprod<\equiv{M}{2,c},\equiv{M}{2,c}>_{2,c} \geq 0$が満たされる.
				また$\inprod<\equiv{M}{2,c},\equiv{M}{2,c}>_{2,c} = 0 \Leftrightarrow \equiv{M}{2,c} = \equiv{0}{2,c}$
				も成り立つ.実際$\Leftarrow$は(\refeq{eq:M_2c_inner_product})より従う.$\Rightarrow$については,
				$M$が$\mathrm{L}^2$-マルチンゲールであるからJensenの不等式より
				$(|M_t|)_{t \in I}$は$\mathrm{L}^2$-劣マルチンゲールであり,Doobの不等式により
				\begin{align}
					\int_{\Omega} \left( \sup{t \in I}{|M_t(\omega)|} \right)^2\ \mu(d\omega) \leq 4 \int_{\Omega} {M_T(\omega)}^2\ \mu(d\omega) = 0
				\end{align}
				が成り立つから
				%\begin{align}
				%	\left\{\ \sup{t \in I}{|M_t|} > 0\ \right\} = \left\{\ \sup{t \in I}{|M_t|^2} > 0\ \right\} = \left\{\ \left(\sup{t \in I}{|M_t(\omega)|}\right)^2 > 0\ \right\}
				%\end{align}
				%により
				\begin{align}
					\mu\left( \sup{t \in I}{|M_t|} > 0 \right) = 0
				\end{align}
				が従い$\equiv{M}{2,c} = \equiv{0}{2,c}$が得られる.
				$\inprod<\cdot,\cdot>_{2,c}$の双線型性は積分の線型性より従う.
				\QED
		\end{description}
	\end{prf}
		
	\begin{screen}
		\begin{prp}[$\mathfrak{M}_{2,c}$はHilbert空間]
			$\mathfrak{M}_{2,c}$は$\inprod<\cdot,\cdot>_{2,c}$を内積としてHilbert空間となる.
			\label{prp:M_2_c_hilbert}
		\end{prp}
	\end{screen}
			
	\begin{prf}
			内積$\inprod<\cdot,\cdot>_{2,c}$により導入されるノルムを$\Norm{\cdot}{\mathfrak{M}_{2,c}}$と表記する.
			任意にCauchy列$\equiv{M^{(n)}}{2,c} \in \mathfrak{M}_{2,c}\ (n=1,2,\cdots)$を取れば,
			各代表$M^{(n)}$に対し或る$\mu$-零集合$E_n$が存在して,$\Omega \backslash E_n$上で
			$I \ni t \longmapsto M^{(n)}_t(\omega) \in \R$が連続となる.
			$E \coloneqq \bigcup_{n=1}^{\infty} E_n$により零集合を定め
			\begin{align}
				N^{(n)}_t(\omega) \coloneqq
				\begin{cases}
					M^{(n)}_t(\omega) & (\omega \in \Omega \backslash E) \\
					0 & (\omega \in E)
				\end{cases}
				,\quad (\forall t \in I,\ n = 1,2,\cdots)
			\end{align}
		として$\left( N^{(n)} \right)_{n=1}^{\infty}$を定義すれば,$N^{(n)}$は全てのパスが連続な$\mathrm{L}^2$-マルチンゲールであるから
		\footnote{
			パスの連続性より右連続性と左極限の存在は満たされる.
			また$E \in \mathcal{F}_0$であることと
			$M^{(n)}$の適合性から$N^{(n)}$の適合性が従う.
			更に$N^{(n)}_t = M^{(n)}_t\ \mbox{$\mu$-a.s.}\ (\forall t \in I)$が満たされているから,
			任意の$s,t \in I,\ s \leq t$に対して
			\begin{align}
				\cexp{N^{(n)}_t}{\mathcal{F}_s} = \cexp{M^{(n)}_t}{\mathcal{F}_s} = M^{(n)}_s = N^{(n)}_s
			\end{align}
			が成り立つ.$N^{(n)}_t$の二乗可積分性は$M^{(n)}_t$の二乗可積分性から従う.
		}
		$N^{(n)} \in \mathcal{M}_{2,c}$を満たし,かつ
		$\equiv{N^{(n)}}{2,c} = \equiv{M^{(n)}}{2,c}\ (n=1,2,\cdots)$も満たすから任意の$n,m \in \N$に対し
		\begin{align}
			\Norm{\equiv{M^{(n)}}{2,c} - \equiv{M^{(m)}}{2,c}}{\mathfrak{M}_{2,c}}^2 
			= \Norm{\equiv{N^{(n)}}{2,c} - \equiv{N^{(m)}}{2,c}}{\mathfrak{M}_{2,c}}^2
			= \int_{\Omega} \left| N^{(n)}_T(\omega) - N^{(m)}_T(\omega) \right|^2\ \mu(d\omega)
			\label{eq:prp_M_2_c_hilbert_1}
		\end{align}
		が成り立つ.一方,$\left(\left|N^{(n)}_t - N^{(m)}_t\right|\right)_{t \in T}$は連続な$\mathrm{L}^2$-劣マルチンゲールであるから
		Doobの不等式より
		\begin{align}
			\lambda^2 \mu\left(\sup{t \in I}{\left| N^{(n)}_t - N^{(m)}_t \right| \geq \lambda}\right) 
			\leq \int_{\Omega} \left| N^{(n)}_T(\omega) - N^{(m)}_T(\omega) \right|^2\ \mu(d\omega)
			\quad (\forall \lambda > 0)
			\label{eq:prp_M_2_c_hilbert_2}
		\end{align}
		が成り立つ.$\left(\equiv{M^{(n)}}{2,c}\right)_{n=1}^{\infty}$はCauchy列であるから
		\begin{align}
			\Norm{\equiv{M^{(n_k)}}{2,c} - \equiv{M^{(n_{k+1})}}{2,c}}{} < \frac{1}{4^k}, \quad (k = 1,2,\cdots) \label{eq:mart_hilbert_1}
		\end{align}
		を満たす部分添数列$(n_k)_{k=1}^{\infty}$が存在して,(\refeq{eq:prp_M_2_c_hilbert_1})(\refeq{eq:prp_M_2_c_hilbert_2})より
		\begin{align}
			\mu\left(\sup{t \in I}{\left| N^{(n_k)}_t - N^{(n_{k+1})}_t \right| \geq \frac{1}{2^k}}\right) < \frac{1}{2^k}, \quad (k=1,2,\cdots)
		\end{align}
		が成立する.
		\begin{align}
			F \coloneqq \bigcup_{N=1}^{\infty} \bigcap_{k \geq N} 
				\Set{\omega \in \Omega}{\sup{t \in I}{\left| N^{(n_k)}_t(\omega) - N^{(n_{k+1})}_t(\omega) \right|} < \frac{1}{2^k}}
		\end{align}
		とおけば,Borel-Cantelliの補題より$F^c$は$\mu$-零集合であり,
		任意の$\omega \in F$については全ての$t \in I$に対して数列
		$\left(N^{(n_k)}_t(\omega)\right)_{k=1}^{\infty}$が$\R$のCauchy列となる.
		数列$\left(N^{(n_k)}_t(\omega)\right)_{k=1}^{\infty}\ (\omega \in F)$の極限を$N^*_t(\omega)$と表せば,
		数列は$t$に関して一様に収束するから
		\footnote{
			$\left| N^{(n_k)}_t(\omega) - N^*_t(\omega) \right| \leq \sum_{j=k}^{\infty} \left| N^{(n_j)}_t(\omega) - N^{(n_{j+1})}_t(\omega) \right|
			\leq \sum_{j=k}^{\infty} \sup{t \in I}{\left| N^{(n_j)}_t(\omega) - N^{(n_{j+1})}_t(\omega) \right|} < 1/2^k, \quad (\forall t \in T)$
			による.
		}
		$I \ni t \longmapsto N^*_t(\omega)$は連続であり,
		\begin{align}
			N_t(\omega) \coloneqq 
			\begin{cases}
				N^*_t(\omega) & (\omega \in F) \\
				0 & (\omega \in \Omega \backslash F)
			\end{cases}
		\end{align}
		により$N$を定義すれば$N \in \mathcal{M}_{2,c}$が成り立つ:
		$N$は全てのパスが連続で$N_0 = 0$を満たすから,以下では$N$が$\mathrm{L}^2$-マルチンゲールであることを示す.
		全てのパスが各点で右連続性かつ左極限を持つことはパスの連続性により満たされているから,
		(1)適合性(2)二乗可積分性及び
		\begin{align}
			\mbox{(3)}:\cexp{N_t}{\mathcal{F}_s} = N_s
			\quad (\forall s \leq t,\ s,t \in I)
		\end{align}
		を証明すればよい.
		\begin{description}
			\item[(1)]
				任意に$t \in I$を取り固定する.
				$N^{(n_k)}_t$の定義域を$F$に制限した写像を$N^{F(k)}_t$として
				\begin{align}
					\mathcal{F}^F_t \coloneqq \Set{F \cap B}{B \in \mathcal{F}_t}
				\end{align}
				とおけば,$N^{F(k)}_t$は$\mathcal{F}^F_t/\borel{\R}$-可測であるから
				各点収束先の$N^*_t$も可測$\mathcal{F}^F_t/\borel{\R}$である.
				\begin{align}
					N^{-1}_t(C) = 
					\begin{cases}
						(\Omega \backslash F) \cup {N^*}^{-1}_t(C) & (0 \in C) \\
						{N^*}^{-1}_t(C) & (0 \notin C)
					\end{cases}
					\quad (\forall C \in \borel{\R})
				\end{align}
				が成り立ち,また$F \in \mathcal{F}_0$により$\mathcal{F}^F_t \subset \mathcal{F}_t$が従うから,
				$N_t$は$\mathcal{F}_t/\borel{\R}$-可測である.
			
			\item[(2)]
				任意に$t \in I$を取り固定する.(\refeq{eq:mart_hilbert_1})とDoobの不等式より
				\begin{align}
					\Norm{\sup{t \in I}{\left|N^{(n_k)}_t - N^{(n_{k+1})}_t\right|}}{\mathscr{L}^2} 
					\leq 2 \Norm{N^{(n_k)}_T - N^{(n_{k+1})}_T}{\mathscr{L}^2} < \frac{2}{4^k} \leq \frac{1}{2^k}
					\quad (k=1,2,\cdots)
				\end{align}
				が成り立つから$\Norm{N^{(n_k)}_t - N^{(n_{k+1})}_t}{\mathscr{L}^2} < 1/2^k \ (k=1,2,\cdots)$が従う.
				特に$j \in \N$を固定すれば全ての$k > j$に対して$\Norm{N^{(n_j)}_t - N^{(n_k)}_t}{\mathscr{L}^2} < 1/2^j$が満たされるから,Fatouの補題により
				\begin{align}
					\Norm{N^{(n_j)}_t - N_t}{\mathscr{L}^2}^2 = \int_{\Omega \backslash F} \lim_{k \to \infty} \left| N^{(n_j)}_t(\omega) - N^{(n_k)}_t(\omega) \right|^2\ \mu(d\omega)
					< \frac{1}{4^j}
					\label{eq:M_c2_hilbert_2}
				\end{align}
				が得られ,Minkowskiの不等式より
				\begin{align}
					\Norm{N_t}{\mathscr{L}^2} \leq \Norm{N_t - N^{(n_j)}_t}{\mathscr{L}^2} + \Norm{N^{(n_j)}_t}{\mathscr{L}^2} < \infty
				\end{align}
				が成り立つ.
			
			\item[(3)]
				各$t \in I,\ k \in \N$について$(M^{(n_k)}_t)_{t \in I}$が$\mathrm{L}^2$-マルチンゲールであるということを利用すればよい.
				任意の$0 \leq s \leq t \leq T$と$A \in \mathcal{F}_s$に対して
				\begin{align}
					\int_{A} \cexp{N^{(n_k)}_t}{\mathcal{F}_s}(\omega)\ \mu(d\omega) &= \int_{A} \cexp{M^{(n_k)}_t}{\mathcal{F}_s}(\omega)\ \mu(d\omega) \\
					&= \int_{A} M^{(n_k)}_s(\omega)\ \mu(d\omega) = \int_{A} N^{(n_k)}_s(\omega)\ \mu(d\omega)
				\end{align}
				が全ての$k = 1,2,\cdots$で成り立つから,H\Ddot{o}lderの不等式及び(\refeq{eq:M_c2_hilbert_2})より
				\begin{align}
					&\left| \int_{A} \cexp{N_t}{\mathcal{F}_s}(\omega)\ \mu(d\omega) - \int_{A} N_s(\omega)\ \mu(d\omega) \right| \\
					&\leq \left| \int_{A} \cexp{N_t}{\mathcal{F}_s}(\omega)\ \mu(d\omega) - \int_{A} \cexp{N^{(n_k)}_t}{\mathcal{F}_s}(\omega)\ \mu(d\omega) \right| \\
						&\qquad+ \left| \int_{A} N^{(n_k)}_s(\omega)\ \mu(d\omega) - \int_{A} N_s(\omega)\ \mu(d\omega) \right| \\
					&= \left| \int_{A} N_t(\omega) - N^{(n_k)}_t(\omega)\ \mu(d\omega) \right|
						+ \left| \int_{A} N^{(n_k)}_s(\omega) - N_s(\omega)\ \mu(d\omega) \right| \\
					&\leq \int_{A} \left| N_t(\omega) - N^{(n_k)}_t(\omega) \right|\ \mu(d\omega)
						+ \int_{A} \left| N^{(n_k)}_s(\omega) - N_s(\omega) \right|\ \mu(d\omega) \\
					&\leq \Norm{N_t - N^{(n_k)}_t}{\mathscr{L}^2} + \Norm{N^{(n_k)}_s - N_s}{\mathscr{L}^2} \\
					&\leq 1/2^{k-1}
				\end{align}
				が全ての$k = 1,2,\cdots$で成り立つ.$k$の任意性から
				\begin{align}
					\int_{A} \cexp{N_t}{\mathcal{F}_s}(\omega)\ \mu(d\omega)
					= \int_{A} N_s(\omega)\ \mu(d\omega)
				\end{align}
				が従い,$\cexp{N_t}{\mathcal{F}_s} = N_s \quad \mbox{in $\Lp{2}{\mathcal{F},\mu}$}$となる.
		\end{description}
	
		最後に,$N$の$\mathfrak{M}_{2,c}$における同値類$\overline{N}$がCauchy列$\left(\overline{M^{(n)}}\right)_{n=1}^{\infty}$の極限であるということを明示して証明を完全に終える.
		部分列$\left(\overline{M^{(n_k)}}\right)_{k=1}^{\infty}$に対して,(\refeq{eq:M_c2_hilbert_2})より
		\begin{align}
				\Norm{\overline{N} - \overline{M^{(n_k)}}}{} 
				= \Norm{\overline{N} - \overline{N^{(n_k)}}}{}
				= \Norm{N_T - N^{(n_k)}_T}{\mathscr{L}^2} \longrightarrow 0 \quad (k \longrightarrow \infty)
		\end{align}
		が成り立つ.部分列が収束することはCauchy列が収束することになるから$\Norm{\overline{N} - \overline{M^{(n)}}}{} \longrightarrow 0$が従い,
		$\mathfrak{M}_{2,c}$がHilbert空間であることが証明された.
		\QED
	\end{prf}

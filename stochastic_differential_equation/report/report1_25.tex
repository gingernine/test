\section{補助定理}
以後の準備として,線型作用素の拡張定理,射影定理,kolmosの補題を証明する.
\begin{itembox}[l]{}
	\begin{thm}[線型作用素の拡張]
		$X,Y$を$\K$上のBanach空間とし,ノルムをそれぞれ$\Norm{\cdot}{X},\ \Norm{\cdot}{Y}$と表記する.
		$X$の部分空間$X_0$が$X$で稠密なら,$X_0$を定義域とする任意の有界線型作用素$T:X \rightarrow Y$に対し,
		定義域を$X$とする$T$の拡張$\tilde{T}:X \rightarrow Y$で,作用素ノルムを変えず,かつ有界線型作用素となるものが一意に存在する.
		\label{thm:linear_operator_expansion}
	\end{thm}
\end{itembox}
\begin{prf}
		作用素ノルムを$\Norm{\cdot}{}$と表記する.$X_0$が$X$で稠密であるから,任意の$x \in X$に対して
		$\Norm{x_n - x}{X} \longrightarrow 0\ (k \longrightarrow +\infty)$
		を満たす$x_n \in X_0 \ (n=1,2,\cdots)$を取れる.任意の$m,n \in \N$に対して
		\begin{align}
			\Norm{Tx_m - Tx_n}{Y} \leq \Norm{T}{} \Norm{x_m - x_n}{X}
		\end{align}
		が成り立ち,右辺が$X_0$のCauchy列をなしているから$(Tx_n)_{n=1}^{+\infty}$も$Y$のCauchy列となる.
		$Y$の完備性から$(Tx_n)_{n=1}^{+\infty}$は或る$y \in Y$に収束し,$y$は$x \in X$に対して一意に定まる.
		実際$x$への別の収束列$z_n \in X_0 \ (n=1,2,\cdots)$を取った場合,$(Tz_n)_{n=1}^{+\infty}$の収束先を
		$u \in \C$として,任意の$n,m \in \N$に対して
		\begin{align}
			\Norm{y - u}{Y} &= \Norm{y - Tx_n + Tx_n - Tz_m + Tz_m - u}{Y} \\
			&\leq \Norm{y - Tx_n}{Y} + \Norm{Tx_n - Tz_m}{Y} + \Norm{Tz_m - u}{Y} \\
			&\leq \Norm{y - Tx_n}{Y} + \Norm{T}{}\Norm{x_n - z_m}{X} + \Norm{Tz_m - u}{Y} \\
			&\leq \Norm{y - Tx_n}{Y} + \Norm{T}{}\left(\Norm{x_n - x}{X} + \Norm{x - z_m}{X}\right)+ \Norm{Tz_m - u}{Y}
		\end{align}
		となるから$n,m \longrightarrow +\infty$で右辺は0に収束し$y = u$が成り立つ.
		ゆえに$x$に$y$を対応させる関係は$X \longmapsto Y$の写像となり,この写像を$\tilde{T}$と表す.この$\tilde{T}$が有界な線型作用素であることを示す.
		\begin{description}
			\item[線型性]
				任意に$x,\ z \in X,\ \alpha,\ \beta \in \K$と$x,z$への収束列$(x_n)_{n=1}^{+\infty},\ (z_n)_{n=1}^{+\infty} \subset X_0$
				を取れば$(\alpha x_n + \beta z_n)_{n=1}^{+\infty}$が$\alpha x+ \beta z$への収束列となるから
				\begin{align}
					\Norm{\tilde{T}(\alpha x + \beta z) - \alpha \tilde{T}x - \beta \tilde{T}z}{Y}
					&= \Norm{\tilde{T}(\alpha x + \beta z) - T(\alpha x_n + \beta z_n) + \alpha Tx_n + \beta Tz_n - \alpha \tilde{T}x - \beta \tilde{T}z}{Y} \\
					&\leq \Norm{\tilde{T}(\alpha x + \beta z) - T(\alpha x_n + \beta z_n)}{Y}
						+ \Norm{\alpha Tx_n - \alpha \tilde{T}x}{Y} + \Norm{\beta Tz_n - \beta \tilde{T}z}{Y} \\
					&\leq \Norm{\tilde{T}(\alpha x + \beta z) - T(\alpha x_n + \beta z_n)}{Y}
						+ |\alpha| \Norm{Tx_n - \tilde{T}x}{Y} + |\beta| \Norm{Tz_n - \tilde{T}z}{Y} \\
					&\longrightarrow 0\quad (n \longrightarrow +\infty)
				\end{align}
				が成り立つから
				\begin{align}
					\tilde{T}(\alpha x + \beta z) = \alpha \tilde{T}x + \beta \tilde{T}z \quad (\forall x,\ z \in X,\ \alpha,\ \beta \in \K)
				\end{align}
				が従う.
			\item[有界性] 任意に$x \in X$と$x$への収束列$x_n \in X_0\ (n = 1,2,\cdots)$を取る.
				任意の$\epsilon > 0$に対し或る$K \in \N$が存在して全ての$k > K$について
				\begin{align}
					\Norm{\tilde{T}x}{Y} < \Norm{Tx_n}{Y} + \epsilon, \quad \Norm{x}{X} < \Norm{x_n}{X} + \epsilon/\Norm{T}{}
				\end{align}
				が成り立つようにできるから
				\begin{align}
					\Norm{\tilde{T}x}{Y} < \Norm{Tx}{Y} + \epsilon < \Norm{T}{} \Norm{x}{X} + 2\epsilon
				\end{align}
				となり$\Norm{\tilde{T}}{} \leq \Norm{T}{}$が成り立つ.
		\end{description}
		さらに
		\begin{align}
			\Norm{\tilde{T}}{} = \sup{\substack{x \in X \\ \Norm{x}{X} = 1}}{\Norm{\tilde{T}x}{Y}} 
			\geq \sup{\substack{x \in X_0 \\ \Norm{x}{X} = 1}}{\Norm{\tilde{T}x}{Y}} 
			= \sup{\substack{x \in X_0 \\ \Norm{x}{X} = 1}}{\Norm{Tx}{Y}} = \Norm{T}{}
		\end{align}
		も成り立つから$\Norm{\tilde{T}}{} = \Norm{T}{}$が従い,$\tilde{T}$が作用素ノルムを変えない拡張であると示された.
		拡張が一意であることは$X_0$が$X$で稠密であることと有界作用素の連続性による.
		\QED
\end{prf}

\begin{itembox}[l]{}
	\begin{thm}[射影定理]
		$H$を$\K$上のHilbert空間とし,内積とノルムをそれぞれ
		$\inprod<\cdot,\cdot>,\Norm{\cdot}{}$と表記する.
		$C \subset H$が閉凸集合なら,$f \in H$に対し
		或る$y \in C$がただ一つ存在して
		\begin{align}
			\Norm{f - y}{} = \inf{h \in C}{\Norm{f - h}{}}
		\end{align}
		を満たす.また$C$が$H$の閉部分空間なら,
		$y \in C$が$f$の射影であることと
		\begin{align}
			\inprod<f-y,h> = 0 \quad (\forall h \in C)
		\end{align}
		が成り立つことは同値になる.
	\end{thm}
\end{itembox}
\begin{prf}\mbox{}
	\begin{description}
	\item[射影の存在]
	$C$が凸集合であるとする.$f \in H \backslash C$として
	\begin{align}
		\delta \coloneqq \inf{h \in C}{\Norm{f - h}{}}
	\end{align}
	とおけば,$C$が閉であるから$\delta > 0$となる.
	$h_n \in C\ (n = 1,2,3,\cdots)$を
	\begin{align}
		\delta = \lim_{n \to \infty}\Norm{f - h_n}{}
	\end{align}
	となるように取れば,任意の$\epsilon > 0$に対して或る$N \in \N$が存在し
	\begin{align}
		\Norm{f - h_n}{}^2 < \delta^2 + \epsilon/4 \quad (\forall n > N)
	\end{align}
	が成り立つ.$n,m > N$ならば,内積空間の中線定理と$(h_n + h_m)/2 \in C$により
	\begin{align}
		\Norm{h_n - h_m}{}^2 &= 2\left( \Norm{f - h_m}{}^2 + \Norm{f - h_n}{}^2 \right) - \Norm{2f - (h_n + h_m)}{}^2 \\
		&= 2\left( \Norm{f - h_m}{}^2 + \Norm{f - h_n}{}^2 \right) - 4\Norm{f - \frac{h_n + h_m}{2}}{}^2 \\
		&< 2\delta^2 + \epsilon - 4\delta^2 = \epsilon
	\end{align}
	が成り立ち$(h_n)_{n=1}^{\infty}$は$H$のCauchy列となり,$H$がHilbert空間で$C$が$H$で閉だから
	極限$y \in H$が存在して$y \in C$が従う.
	\begin{align}
		\left| \delta - \Norm{f - y}{} \right| 
		&\leq \left| \delta - \Norm{f - h_n}{} \right| + \left| \Norm{f - h_n}{} - \Norm{f - y}{} \right| \\
		&\leq \left| \delta - \Norm{f - h_n}{} \right| + \Norm{h_n - y}{}
		\longrightarrow 0 \quad (n \longrightarrow \infty)
	\end{align}
	より$\delta = \Norm{f - y}{}$となり射影の存在が示された.
	$f \in C$の場合は$f$が自身の射影である.

	\item[射影の一意性]
		$z \in C$もまた$\delta = \Norm{f - z}{}$を満たすとすれば,$C$の凸性により
		\begin{align}
			2 \delta \leq 2\Norm{f - \frac{y + z}{2}}{} \leq \Norm{f - y}{} + \Norm{f - z}{} = 2\delta
		\end{align}
		が成り立つから,中線定理より
		\begin{align}
			\Norm{y - z}{}^2 = 2\left( \Norm{f - z}{}^2 + \Norm{f - y}{}^2 \right) - 4\Norm{f - \frac{y + z}{2}}{}^2 = 0
		\end{align}
		となって$y = z$が従う.
	
	\item[$C$が閉部分空間の場合]
		$f \in H \backslash C$に対して$f$の$C$への射影を$y \in C$
		とする.($f \in C$の場合は$y = f$.)或る$h \in C$に対して
		\begin{align}
			\inprod<f -y,\ h> \neq 0
		\end{align}
		となると仮定すれば($f \neq y$より$h \neq 0$),
		\begin{align}
			\hat{y} \coloneqq y + \left( \frac{\inprod<f - g,h>}{\Norm{h}{}^2} \right)h \in C
		\end{align}
		に対して
		\footnote{
			$\hat{y} \in C$となるためには$C$が部分空間である必要がある.凸性だけではこれが成り立たない.
		}
		\begin{align}
			\Norm{f - \hat{y}}{}^2 
			&= \inprod<f - y - \frac{\inprod<f - g,h>}{\Norm{h}{}^2}h,\ f - y - \frac{\inprod<f - g,h>}{\Norm{h}{}^2}h> \\
			&= \Norm{f-y}{}^2 - \frac{|\inprod<f - g,h>|^2}{\Norm{h}{}^2} \\
			&< \Norm{f-y}{}^2
		\end{align}
		が成り立つから$y$が射影であることに反する.従って射影$y$は
		\begin{align}
			\inprod<f -y,\ h> = 0 \quad (\forall h \in C) \label{eq:projection}
		\end{align}
		を満たす.逆に$y \in C$に対して式(\refeq{eq:projection})が成り立っているとすれば
		$y$は$f$の射影となる.実際任意の$h \in C$に対して
		\begin{align}
			\Norm{f - h}{}^2 &= \inprod<f - y + y - h,\ f - y + y - h> \\
			&= \Norm{f - y}{}^2 + 2 \Re{\inprod<f - y,\ y - h>} + \Norm{y - h}{}^2 \\
			&= \Norm{f - y}{}^2 + \Norm{y - h}{}^2 \\
			&\geq \Norm{f - y}{}^2
		\end{align}
		となり
		\begin{align}
			\Norm{f - y}{} = \inf{h \in C}{\Norm{f - h}{}}
		\end{align}
		が成り立つ.
	\end{description}
	\QED
\end{prf}

\begin{itembox}[l]{}
	\begin{thm}[Kolmosの補題]
		$H$をHilbert空間,$\Norm{\cdot}{}$を$H$の内積により
		導入されるノルムとする.$f_n \in H\ (n=1,2,\cdots)$が$\sup{n \in \N}{\Norm{f_n}{}} < \infty$
		を満たすとき,
		\begin{align}
			\exists g_n \in \Conv{f_n,f_{n+1},f_{n+2} \cdots}\footnotemark \quad (n=1,2,\cdots)
		\end{align}
		が取れて$(g_n)_{n=1}^{\infty}$は$H$のCauchy列となる.
	\end{thm}
\end{itembox}

\footnotetext{
	$\Conv{f_n,f_{n+1},f_{n+2} \cdots}$は$f_n,f_{n+1},f_{n+2} \cdots$を含む最小の凸集合である.
	すなわち
	\begin{align}
		\Conv{f_n,f_{n+1},f_{n+2} \cdots} = \Set{\sum_{i=0}^{\infty}c_i f_{n+i}}{ 0 \leq c_i \leq 1,\ \exists j \in \N,\ c_i=0\ (\forall i \geq j),\ \sum_{i=0}^{\infty} c_i = 1}
	\end{align}
	で定義される.
}

\begin{prf}
	$F \coloneqq \sup{n \in \N}{\Norm{f_n}{}} < \infty$とし
	\begin{align}
		A \coloneqq \sup{n \geq 1}{\inf{}{\Set{\Norm{g}{}}{g \in \Conv{f_n,f_{n+1},f_{n+2} \cdots}}}}
	\end{align}
	とおけば$A \leq F$となる.実際各$n \in \N$について任意に$g \in \Conv{f_n,f_{n+1},f_{n+2} \cdots}$を取れば
	\begin{align}
		g = \sum_{i=0}^{N}c_i f_{n+i} \quad \left( 0 \leq c_i \leq 1,\ \sum_{i=0}^{N} c_i = 1 \right)
	\end{align}
	と表現できるから,
	\begin{align}
		\Norm{g}{} = \Norm{\sum_{i=0}^{N}c_i f_{n+i}}{} \leq \sum_{i=0}^{N}c_i \Norm{f_n}{} \leq F
	\end{align}
	が成り立ち,
	\begin{align}
		\inf{}{\Set{\Norm{g}{}}{g \in \Conv{f_n,f_{n+1},f_{n+2} \cdots}}} \leq F \quad (\forall n \in \N)
	\end{align}
	が従い$A \leq F$を得る.また全ての$n \in \N$に対して
	\begin{align}
		\inf{}{\Set{\Norm{g}{}}{g \in \Conv{f_n,f_{n+1},f_{n+2} \cdots}}} \leq A
	\end{align}
	が成り立つから,或る$g_n \in \Conv{f_n,f_{n+1},f_{n+2} \cdots}$を
	\begin{align}
		\Norm{g_n}{} \leq A + \frac{1}{n} \quad (n=1,2,\cdots) \label{eq:thm_kolmos_1}
	\end{align}
	を満たすよう取ることができる.一方で
	\begin{align}
		\left( \inf{}{\Set{\Norm{g}{}}{g \in \Conv{f_n,f_{n+1},f_{n+2} \cdots}}} \right)_{n=1}^\infty
	\end{align}
	が単調増大列である(下限を取る範囲が縮小していく)から,任意に$m \in \N$を取れば或る$N(m) \in \N\ (N(m) > m)$が存在し,
	全ての$n \geq N(m)$に対して
	\begin{align}
		\inf{}{\Set{\Norm{g}{}}{g \in \Conv{f_n,f_{n+1},f_{n+2} \cdots}}} \geq A - \frac{1}{m}
	\end{align}
	が成り立つ.任意の$l,k \geq N(m)$に対し$(g_l + g_k) / 2 \in \Conv{f_{N(m)},f_{N(m)+1},f_{N(m)+2} \cdots}$となるから
	\begin{align}
		\Norm{\frac{g_l + g_k}{2}}{} \geq A - \frac{1}{m}
	\end{align}
	が従い,中線定理と(\refeq{eq:thm_kolmos_1})を併せれば
	\begin{align}
		\Norm{g_l - g_k}{}^2 = 2\Norm{g_l}{} + 2\Norm{g_k}{} + \Norm{g_l + g_k}{}^2 \leq 4\left( A + \frac{1}{N(m)} \right)^2 - 4\left( A - \frac{1}{m} \right)^2 \leq \frac{16A}{m}
	\end{align}
	が成り立つ.$A \leq F < \infty$であったから,これは$(g_n)_{n=1}^{\infty}$がCauchy列をなしていることを表現している.
	\QED
\end{prf}
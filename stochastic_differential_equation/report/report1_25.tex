
次節への準備として,ノルム空間における線型作用素の拡張定理とHilbert空間における射影定理を載せておく.
\begin{itembox}[l]{}
	\begin{thm}[線型作用素の拡張]
		係数体を$\K$とする.$X,Y$をBanach空間とし,ノルムをそれぞれ$\Norm{\cdot}{X},\ \Norm{\cdot}{Y}$と表記する.
		$X$の部分空間$X_0$が$X$で稠密なら,$X$から$Y$への任意の有界線型作用素$T\ $($T$の定義域は$X_0$)に対し,
		作用素ノルムを変えない$T$の拡張$\tilde{T}\ $(定義域$X$)で,$X$から$Y$への有界線型作用素となるものが一意に存在する.
	\end{thm}
\end{itembox}
\begin{prf}
		作用素ノルムは$\Norm{\cdot}{}$と表記する.$X_0$が$X$で稠密であるということにより,任意の$x \in X$に対して
		$x_n \in X_0 \ (n=1,2,\cdots)$で$\Norm{x_n - x}{X} \longrightarrow 0\ (k \longrightarrow +\infty)$
		となるものを取ることができる.任意の$m,n \in \N$に対して
		\begin{align}
			\Norm{Tx_m - Tx_n}{Y} \leq \Norm{T}{} \Norm{x_m - x_n}{X}
		\end{align}
		が成り立つから,右辺が$X_0$のCauchy列をなすことにより$(Tx_n)_{n=1}^{+\infty}$も$Y$のCauchy列となる.
		$Y$の完備性から$(Tx_n)_{n=1}^{+\infty}$は或る$y \in Y$に収束し,$y$は$x \in X$に対して一意に定まる.
		なぜならば,$x$への別の収束列$z_n \in X_0 \ (n=1,2,\cdots)$を取った場合の$(Tz_n)_{n=1}^{+\infty}$の収束先が
		$u \in \C$であるとして,任意の$n,m \in \N$に対して
		\begin{align}
			\Norm{y - u}{Y} &= \Norm{y - Tx_n + Tx_n - Tz_m + Tz_m - u}{Y} \\
			&\leq \Norm{y - Tx_n}{Y} + \Norm{Tx_n - Tz_m}{Y} + \Norm{Tz_m - u}{Y} \\
			&\leq \Norm{y - Tx_n}{Y} + \Norm{T}{}\Norm{x_n - z_m}{X} + \Norm{Tz_m - u}{Y} \\
			&\leq \Norm{y - Tx_n}{Y} + \Norm{T}{}\left(\Norm{x_n - x}{X} + \Norm{x - z_m}{X}\right)+ \Norm{Tz_m - u}{Y}
		\end{align}
		となるから$n,m \longrightarrow +\infty$で右辺は0に収束し,$y = u$が示されるためである.
		つまり$x$に$y$を対応させる関係は$X \longmapsto Y$の写像となり,この写像を$\tilde{T}$と表すことにする.$T$の線型性も次のように示される.
		任意の$x,\ z \in X,\ \alpha,\ \beta \in \K$に対して,$x,z$への収束列$(x_n)_{n=1}^{+\infty},\ (z_n)_{n=1}^{+\infty} \subset X_0$
		を取れば$(\alpha x_n + \beta z_n)_{n=1}^{+\infty}$が$\alpha x+ \beta z$への収束列となるから
		\begin{align}
			\Norm{\tilde{T}(\alpha x + \beta z) - \alpha \tilde{T}x - \beta \tilde{T}z}{Y}
			&= \Norm{\tilde{T}(\alpha x + \beta z) - T(\alpha x_n + \beta z_n) + \alpha Tx_n + \beta Tz_n - \alpha \tilde{T}x - \beta \tilde{T}z}{Y} \\
			&\leq \Norm{\tilde{T}(\alpha x + \beta z) - T(\alpha x_n + \beta z_n)}{Y}
				+ \Norm{\alpha Tx_n - \alpha \tilde{T}x}{Y} + \Norm{\beta Tz_n - \beta \tilde{T}z}{Y} \\
			&\leq \Norm{\tilde{T}(\alpha x + \beta z) - T(\alpha x_n + \beta z_n)}{Y}
				+ |\alpha| \Norm{Tx_n - \tilde{T}x}{Y} + |\beta| \Norm{Tz_n - \tilde{T}z}{Y} \\
			&\longrightarrow 0\quad (n \longrightarrow +\infty)
		\end{align}
		が成り立つ.ゆえに$\tilde{T}(\alpha x + \beta z) = \alpha \tilde{T}x + \beta \tilde{T}z\ (\forall x,\ z \in X,\ \alpha,\ \beta \in \K)$である.
		また$\tilde{T}$は有界な線型作用素である.なぜなら,任意に$x \in X$と$x$への収束列$x_n \in X_0\ (n = 1,2,\cdots)$を取れば,
		任意の$\epsilon > 0$に対し或る$K \in \N$が存在して全ての$k > K$について
		\begin{align}
			\Norm{\tilde{T}x}{Y} < \Norm{Tx_n}{Y} + \epsilon, \quad \Norm{x}{X} < \Norm{x_n}{X} + \epsilon/\Norm{T}{}
		\end{align}
		が成り立つようにできるから,この下で
		\begin{align}
			\Norm{\tilde{T}x}{Y} < \Norm{Tx}{Y} + \epsilon < \Norm{T}{} \Norm{x}{X} + 2\epsilon
		\end{align}
		となり$\Norm{\tilde{T}}{} \leq \Norm{T}{}$が判るからである.さらに
		\begin{align}
			\Norm{\tilde{T}}{} = \sup{\substack{x \in X \\ \Norm{x}{X} = 1}}{\Norm{\tilde{T}x}{Y}} 
			\geq \sup{\substack{x \in X_0 \\ \Norm{x}{X} = 1}}{\Norm{\tilde{T}x}{Y}} 
			= \sup{\substack{x \in X_0 \\ \Norm{x}{X} = 1}}{\Norm{Tx}{Y}} = \Norm{T}{}
		\end{align}
		も成り立つから結局$\Norm{\tilde{T}}{} = \Norm{T}{}$であると判る.以上より
		任意の有界線型作用素$T$がノルムを変えないまま或る有界線型作用素$\tilde{T}$に拡張されることが示された.
		拡張が一意であることは$X_0$が$X$で稠密であることと$T$の連続性による.
		\QED
\end{prf}

\begin{itembox}[l]{}
	\begin{thm}[射影定理]
	\end{thm}
\end{itembox}
\begin{prf}\mbox{}\\
	\begin{description}
	\item[射影の存在]
	$f \in H \backslash C$として
	\begin{align}
		\delta \coloneqq \inf{h \in C}{\Norm{f - h}{}}
	\end{align}
	とおく.$C$が閉集合で$f$が$C$の外にあるから$\delta > 0$となる.
	下限の性質から$h_n \in C\ (n = 1,2,3,\cdots)$を取って
	\begin{align}
		\delta = \lim_{n \to \infty}\Norm{f - h_n}{}
	\end{align}
	となるようにできるから,任意の$\epsilon > 0$に対して或る$N \in \N$が存在して
	$n> N$ならば$\Norm{f - h_n}{}^2 < \delta + \epsilon/4$が成り立つ.この$N$に対し
	$n,m > N$ならば,内積空間の中線定理と$(h_n + h_m)/2 \in C$であることにより
	\begin{align}
		\Norm{h_n - h_m}{}^2 &= 2\left( \Norm{f - h_m}{}^2 + \Norm{f - h_n}{}^2 \right) - \Norm{2f - (h_n + h_m)}{}^2 \\
		&= 2\left( \Norm{f - h_m}{}^2 + \Norm{f - h_n}{}^2 \right) - 4\Norm{f - \frac{h_n + h_m}{2}}{}^2 \\
		&< 2\delta + \epsilon - 4\delta = \epsilon
	\end{align}
	とできるから$(h_n)_{n=1}^{\infty}$は$C$のCauchy列であると判る.$H$がHilbert空間であり$C$が$H$で閉だから,
	$(h_n)_{n=1}^{\infty}$の極限$y \in H$が存在し$y \in C$である.
	\begin{align}
		\left| \delta - \Norm{f - y}{} \right| 
		\leq \left| \delta - \Norm{f - h_n}{} \right| + \left| \Norm{f - h_n}{} - \Norm{f - y}{} \right|
		\leq \left| \delta - \Norm{f - h_n}{} \right| + \Norm{h_n - y}{}
		\longrightarrow 0 \quad (n \longrightarrow \infty)
	\end{align}
	によって$\delta = \Norm{f - y}{}$が成り立つこと,すなわち射影の存在が示された.
	$f \in C$の場合は$f$が自身の射影である.

	\item[射影の一意性]
		$z \in C$もまた$\delta = \Norm{f - z}{}$を満たすとすれば,$C$の凸性により
		\begin{align}
			2 \delta \leq 2\Norm{f - \frac{y + z}{2}}{} \leq \Norm{f - y}{} + \Norm{f - z}{} = 2\delta
		\end{align}
		が成り立つから,中線定理より
		\begin{align}
			\Norm{y - z}{}^2 = 2\left( \Norm{f - z}{}^2 + \Norm{f - y}{}^2 \right) - 4\Norm{f - \frac{y + z}{2}}{}^2 = 0
		\end{align}
		となって$y = z$が判る.すなわち$f$の射影はただ一つに決まる.
	
	\item[$C$が閉部分空間の場合]
		$f \in H \backslash C$に対して$f$の$C$への射影を$y \in C\ $(存在は$C$が凸の場合と全く同様に示される.)
		とする.($f \in C$の場合は$y = f$である.)或る$h \in C$に対して
		\begin{align}
			\inprod<f -y,\ h> \neq 0
		\end{align}
		となると仮定すれば($f \neq y$より$h \neq 0$),$C$の元$\hat{y} \coloneqq y + \left(\inprod<f - g,h>/\Norm{h}{}^2\right)h$
		に対して
		\begin{align}
			\Norm{f - \hat{y}}{}^2 
			&= \inprod<f - y - \frac{\inprod<f - g,h>}{\Norm{h}{}^2}h,\ f - y - \frac{\inprod<f - g,h>}{\Norm{h}{}^2}h> \\
			&= \Norm{f-y}{}^2 - \frac{|\inprod<f - g,h>|^2}{\Norm{h}{}^2}
			&< \Norm{f-y}{}^2
		\end{align}
		が成り立つから$y$が射影であることに反する.従って射影$y$に対しては
		\begin{align}
			\inprod<f -y,\ h> \neq 0 \quad (\forall h \in C) \label{eq:projection}
		\end{align}
		が成り立つ.逆に$y \in C$に対して式(\refeq{eq:projection})が成り立っているとすれば
		$y$が$f$の射影であることも示される.任意の$h \in C$に対して
		\begin{align}
			\Norm{f - h}{}^2 &= \inprod<f - y + y - h,\ f - y + y - h> \\
			&= \Norm{f - y}{}^2 + 2 \Re{\inprod<f - y,\ y - h>} + \Norm{y - h}{}^2 \\
			&= \Norm{f - y}{}^2 + \Norm{y - h}{}^2 \\
			&\geq \Norm{f - y}{}^2
		\end{align}
		となることにより$\Norm{f - y}{} = \inf{h \in C}{\Norm{f - h}{}}$であることが示された.
	\end{description}
\end{prf}

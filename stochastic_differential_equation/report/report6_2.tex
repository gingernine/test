\section{可予測過程}
	\begin{screen}
		\begin{dfn}[可予測$\sigma$-加法族・可予測過程]
			$\{0\} \times A\ (A \in \mathcal{F}_0)$の形,或は
			$(s,t] \times A\ \left( (s,t] \subset I,\ A \in \mathcal{F}_s \right)$
			の形の$I \times \Omega$の部分集合の全体を$\Pi$とおく.
			そして$\Pi$の生成する$\sigma$-加法族を$\mathcal{P}$と表し,
			これを可予測$\sigma$-加法族(predictable $\sigma$-algebra)と呼ぶ.
			また$\mathcal{P}/\borel{\R}$可測の関数を可予測過程(predictable process)という.
			\label{dfn:predictable_sigma_algebra}
		\end{dfn}
	\end{screen}
	
	\begin{screen}
		\begin{lem}[$\Pi$は乗法族]
			定義\ref{dfn:predictable_sigma_algebra}の$\Pi$は乗法族である.
			\label{lem:predictable_Pi_pi_system}
		\end{lem}
	\end{screen}
	
	\begin{prf}
		任意に$B_1,B_2 \in \Pi$を取り,
		$B_1 = (s_1,t_1] \times A_1,B_2 = (s_2,t_2] \times A_2\ (A_1 \in \mathcal{F}_{s_1},A_2 \in \mathcal{F}_{s_2},s_1 \leq s_2)$
		と仮定する.$t_1 \leq s_2$なら$B_1 \cap B_2 = \emptyset$であり,
		$s_2 < t_1$とすれば$A_1 \in \mathcal{F}_{s_2}$であるから
		\begin{align}
			B_1 \cap B_2 =  (s_2, t_1 \wedge t_2] \times \left(A_1 \cap A_2\right) \in \Pi
		\end{align}
		が成り立つ.$B_1 = \{0\} \times A_1\ (A_1 \in \mathcal{F}_0)$の場合,$B_2 = \{0\} \times A_2\ (A_2 \in \mathcal{F}_0)$の形であれば
		$B_1 \cap B_2 = \{0\} \times \left(A_1 \cap A_2\right)$,そうでなければ
		$B_1 \cap B_2 = \emptyset$となり,いずれの場合も交演算で閉じている.
		\QED
		\end{prf}
	
	\begin{screen}
		\begin{lem}[可予測単関数の時間に関する可測性]
			任意の$B \in \mathcal{P}$と任意の$\omega \in \Omega$に対し
			$f_B:I \ni t \longmapsto \defunc_B(t,\omega)$は可測$\borel{I}/\borel{\R}$である.
			\label{lem:measurability_of_predictable_simple_functions}
		\end{lem}
	\end{screen}
	
	\begin{prf}
		任意に$C \in \borel{\R}$を取り固定する.
		\begin{align}
			\mathcal{D} \coloneqq \Set{B \in \mathcal{P}}{\mbox{$f_B$が可測$\borel{I}/\borel{\R}$.}}
		\end{align}
		とおけば$\mathcal{D}$はDynkin族である.実際次が成り立つ:
		\begin{itemize}
			\item $B = I \times \Omega$なら$f_B^{-1}(C) = I$又は$\emptyset$であるから$I \times \Omega \in \mathcal{D}$である.
			\item $B_1,B_2 \in \mathcal{D},\ B_1 \subset B_2$に対して,$f_{B_2 \backslash B_1} = f_{B_2} - f_{B_1}$より$B_2 \backslash B_1 \in \mathcal{D}$が成り立つ.
			\item 互いに素な列$B_n \in \mathcal{D}\ (n\in\N)$に対して,$f_{\sum_{n=1}^{\infty} B_n} = \sum_{n=1}^{\infty} f_{B_n}$より$\sum_{n=1}^{\infty} B_n \in \mathcal{D}$が成り立つ.
		\end{itemize}
		補題\ref{lem:predictable_Pi_pi_system}より$\Pi$は乗法族であるから,
		$\Pi \subset \mathcal{D}$ならばDynkin族定理により補題の主張が従う.
		$B = (s,t] \times A\ \left( (s,t] \subset I,\ A \in \mathcal{F}_s \right)$と表されるとすれば,
		$\omega \in A$のとき
		\begin{align}
			f_B^{-1}(C) = 
			\begin{cases}
				I & (0 \in C,1 \in C) \\
				(s,t] & (0 \notin C,1 \in C) \\
				I \backslash (s,t] & (0 \in C,1 \notin C) \\
				\emptyset & (0 \notin C,1 \notin C) \\
			\end{cases}
		\end{align}
		が成り立ち,$\omega \notin A$のとき
		\begin{align}
			f_B^{-1}(C) = 
			\begin{cases}
				I & (0 \in C) \\
				\emptyset & (0 \notin C) \\
			\end{cases}
		\end{align}
		が成り立つ.$B = \{0\} \times A$の場合も同様であるから,いずれの場合も$f_B^{-1}(C) \in \borel{I}$を満たす.
		\QED
	\end{prf}
	
	\begin{screen}
		\begin{thm}[$(\Omega \times I,\mathcal{P})$における測度の構成]
			任意の$M \in \mathcal{M}_{2,c},B \in \mathcal{P}$に対して
			\begin{align}
				\mu_M(B) \coloneqq \int_\Omega \int_I \defunc_B(s,\omega)\ \inprod<M>(ds,\omega)\ \mu(d\omega)
				\label{eq:thm_composition_of_predictable_space_measure_0}
			\end{align}
			と定めれば,$\mu_M$は可測空間$(I \times \Omega,\mathcal{P})$上の測度となる.
			\label{thm:composition_of_predictable_space_measure}
		\end{thm}
	\end{screen}
	
	\begin{prf}\mbox{}
		\begin{description}
			\item[第一段]
				任意の$B \in \mathcal{P},t \in I$に対して
				\begin{align}
					\Omega \ni \omega \longmapsto \int_{[0,t]} \defunc_B(s,\omega)\ \inprod<M>(ds,\omega)
					\label{eq:thm_composition_of_predictable_space_measure_1}
				\end{align}
				が可測$\mathcal{F}_t/\borel{\R}$であることを示す
				\footnote{
					補題\ref{lem:measurability_of_predictable_simple_functions}より可測性が保証されているから,
					(\refeq{eq:thm_composition_of_predictable_space_measure_1})の積分はLebesgue-Stieltjes積分として定義される.
				}
				.
				今,$I \ni t \longmapsto \inprod<M>_t$の単調非減少性より
				\begin{align}
					0 &\leq \int_{[0,t]} \defunc_B(s,\omega)\ \inprod<M>(ds,\omega) \\
					&\leq \int_{[0,t]}\ \inprod<M>(ds,\omega) = \inprod<M>(t,\omega) < \infty
					\quad (\forall t \in I,B \in \mathcal{P},\omega \in \Omega)
					\label{eq:thm_composition_of_predictable_space_measure_2}
				\end{align}
				が成り立ち,いかなる場合も可積分性は保証される.
				\begin{align}
					\mathcal{D} \coloneqq \Set{B \in \mathcal{P}}{\mbox{任意の$t \in I$に対し(\refeq{eq:thm_composition_of_predictable_space_measure_1})が可測$\mathcal{F}_t/\borel{\R}$.}}
				\end{align}
				とおけば$\mathcal{D}$はDynkin族である.実際以下が成り立つ:
				\begin{itemize}
					\item $B = I \times \Omega$の場合,(\refeq{eq:thm_composition_of_predictable_space_measure_2})と
						$\inprod<M>$の適合性より(\refeq{eq:thm_composition_of_predictable_space_measure_1})は可測$\mathcal{F}_t/\borel{\R}$となる.
					\item $B_1,B_2 \in \mathcal{P},\ B_1 \subset B_2$に対して,(\refeq{eq:thm_composition_of_predictable_space_measure_1})の積分は可積分であるから線型性より
						\begin{align}
							&\int_{[0,t]} \defunc_{B_2 \backslash B_1}(s,\omega)\ \inprod<M>(ds,\omega) \\
							&\qquad= \int_{[0,t]} \defunc_{B_2}(s,\omega)\ \inprod<M>(ds,\omega) - \int_{[0,t]} \defunc_{B_1}(s,\omega)\ \inprod<M>(ds,\omega)
						\end{align}
						が成り立ち$B_2 \backslash B_1 \in \mathcal{P}$が従う.
					\item 互いに素な列$B_n \in \mathcal{P}\ (n \in \N)$に対して,単調収束定理より
						\begin{align}
							\int_{[0,t]} \defunc_{\sum_{n=1}^{\infty}B_n}(s,\omega)\ \inprod<M>(ds,\omega)
							= \sum_{n=1}^{\infty} \int_{[0,t]} \defunc_{B_n}(s,\omega)\ \inprod<M>(ds,\omega)
							\label{eq:thm_composition_of_predictable_space_measure_3}
						\end{align}
						が成り立つ.(\refeq{eq:thm_composition_of_predictable_space_measure_2})より$\omega$ごとに右辺の級数は有限確定し$\sum_{n=1}^{\infty}B_n \in \mathcal{P}$が従う.
				\end{itemize}
				補題\ref{lem:measurability_of_predictable_simple_functions}と同様に$\Pi \subset \mathcal{D}$となることを示せば,Dynkin族定理より第一段の主張が従う.
				$B = (\alpha,\beta] \times A\ (A \in \mathcal{F}_\alpha)$として,全ての$\omega \in \Omega$に対し
				\begin{align}
					\int_{[0,t]} \defunc_B(s,\omega)\ \inprod<M>(ds,\omega)
					&= \defunc_A(\omega) \int_{[0,t]} \defunc_{(\alpha,\beta]}(s)\ \inprod<M>(ds,\omega) \\
					&= \begin{cases}
						0 & (t \leq \alpha) \\
						\defunc_A(\omega)\left( \inprod<M>(t \wedge \beta,\omega) - \inprod<M>(\alpha,\omega) \right) & (t > \alpha)
					\end{cases}
				\end{align}
				が成り立つから,$A \in \mathcal{F}_t$であることと$\inprod<M>$の適合性から$(\alpha,\beta] \times A \in \mathcal{D}$が従う.
				$B = \{0\} \times A\ (A \in \mathcal{F}_0)$の場合は積分は常に0になる.
				
			\item[第二段]
				前段の結果より(\refeq{eq:thm_composition_of_predictable_space_measure_0})右辺の積分が定義される.
				この段では(\refeq{eq:thm_composition_of_predictable_space_measure_0})で定める$\mu_M$が測度となることを示す.
				先ず$\mu_M$の正値性は(\refeq{eq:thm_composition_of_predictable_space_measure_2})より従う.
				また互いに素な列$B_n \in \mathcal{P}$を取れば,(\refeq{eq:thm_composition_of_predictable_space_measure_3})と単調収束定理より
				\begin{align}
					\int_\Omega \int_I \defunc_{\sum_{n=1}^{\infty}B_n}(s,\omega)\ \inprod<M>(ds,\omega)\ \mu(d\omega)
					= \sum_{n=1}^{\infty} \int_\Omega \int_I \defunc_{B_n}(s,\omega)\ \inprod<M>(ds,\omega)\ \mu(d\omega)
				\end{align}
				が成り立ち$\mu_M$の完全加法性が従う.
				\QED
		\end{description}
	\end{prf}
	
	\begin{screen}
		\begin{dfn}[単純可予測過程]
			$n \in \N$,$0=t_0 < t_1 < \cdots < t_n = T$,
			$F \in \semiLp{\infty}{\Omega,\mathcal{F}_0,\mu},F_i \in \semiLp{\infty}{\Omega,\mathcal{F}_{t_i},\mu}\ (i=0,\cdots,n-1)$
			を任意に取り構成する次の過程
			\begin{align}
				X(t,\omega) \coloneqq F(\omega) \defunc_{\{0\}} (t) + \sum_{i=0}^{n-1} F_i(\omega) \defunc_{\left(t_i,t_{i+1}\right]}(t)
				\quad (\forall \omega \in \Omega,t \in I)
				\label{eq:dfn_simple_predictable_process}
			\end{align}
			を単純可予測過程(simple predictable process)という.単純可予測過程の全体を$\mathcal{S}$と表す.
			\label{dfn:predictable_simple_process}
		\end{dfn}
	\end{screen}
	
	\begin{screen}
		\begin{lem}[単純可予測過程の性質]\mbox{}
			\begin{description}
				\item[(1)] $\mathcal{S}$は$\R$上の線形空間をなす.
				\item[(2)] 任意の$X \in \mathcal{S}$は可測$\mathcal{P}/\borel{\R}$である.
				\item[(3)] $\mathcal{S}$は$\semiLp{2}{I \times \Omega,\mathcal{P},\mu_M}$の稠密な部分集合である.
			\end{description}
			\label{lem:properties_of_simple_predictable_processes}
		\end{lem}
	\end{screen}
	
	\begin{prf}\mbox{}
		\begin{description}
			\item[(1)] $\mathcal{S}$が線型演算について閉じていることを示す.
				\begin{description}
					\item[加法]
						$S_1,S_2 \in \mathcal{S}$を取れば,定義\ref{dfn:predictable_simple_process}に従って
						\begin{align}
							S_1 = F \defunc_{\{0\}} + \sum_{i=0}^{n-1} F_i \defunc_{\left(t_i,t_{i+1}\right]}, \quad
							S_2 = G \defunc_{\{0\}} + \sum_{j=0}^{m-1} G_j \defunc_{\left(s_j,s_{j+1}\right]}
						\end{align}
						と表現できる.今,時点の分点の合併が$0 = u_0 < u_1 < \cdots < u_r = T$であるとする.
						\begin{align}
							\tilde{F}_k &\coloneqq F_i \quad (t_i \leq u_k < t_{i+1},\ i=0,\cdots,n-1), \\
							\tilde{G}_k &\coloneqq G_j \quad (s_j \leq u_k < s_{j+1},\ j=0,\cdots,m-1)
						\end{align}
						とおけば,全ての$k = 0,\cdots,r-1$に対し$\tilde{F}_k,\tilde{G}_k \in \semiLp{\infty}{\Omega,\mathcal{F}_{u_k},\mu}$であり,また
						\begin{align}
							S_1 = F \defunc_{\{0\}} + \sum_{k=0}^{r-1} \tilde{F}_k \defunc_{\left(u_k,u_{k+1}\right]}, \quad
							S_2 = G \defunc_{\{0\}} + \sum_{k=0}^{r-1} \tilde{G}_k \defunc_{\left(u_k,u_{k+1}\right]}
						\end{align}
						と表現しなおせば
						\begin{align}
							S_1 + S_2 = (F+G) \defunc_{\{0\}} + \sum_{k=0}^{r-1} \left( \tilde{F}_k + \tilde{G}_k \right) \defunc_{\left(u_k,u_{k+1}\right]}
							\label{eq:lem_properties_of_simple_predictable_processes}
						\end{align}
						が成り立つ.$\mathscr{L}^{\infty}$の線型性より$S_1 + S_2 \in \mathcal{S}$が従う.
						
					\item[スカラ倍]
						$S \in \mathcal{S}$と$\alpha \in \R$を取れば,
						\begin{align}
							S = F \defunc_{\{0\}} + \sum_{i=0}^{n-1} F_i \defunc_{\left(t_i,t_{i+1}\right]}
						\end{align}
						と表されているとして
						\begin{align}
							\alpha S = \alpha F \defunc_{\{0\}} + \sum_{i=0}^{n-1} \alpha F_i \defunc_{\left(t_i,t_{i+1}\right]}
						\end{align}
						となる.$\mathscr{L}^{\infty}$の線型性より$\alpha S \in \mathcal{S}$が従う.
				\end{description}
				
			\item[(2)] 
				(\refeq{eq:dfn_simple_predictable_process})の各項が可測$\mathcal{P}/\borel{\R}$であることを示せばよい.
				$(s,t] \in I$と$F \in \semiLp{\infty}{\Omega,\mathcal{F}_s,\mu}$を取る.$F$の単関数近似列$(F_n)_{n=1}^{\infty}$
				を取れば,一つ一つは
				\begin{align}
					F_n = \sum_{j=1}^{N_n} \alpha_j^n \defunc_{A_j^n} \quad \left( \alpha_j^n \in \R,\ A_j^n \in \mathcal{F}_s,\ \mbox{$\sum_{j=1}^{N_n}A_j^n = \Omega$} \right)
				\end{align}
				と表現される.各$j=1,\cdots,N_n$について
				\begin{align}
					I \times \Omega \ni (u,\omega) \longmapsto \defunc_{A_j^n}(\omega) \defunc_{(s,t]}(u)
				\end{align}
				が可測$\mathcal{P}/\borel{\R}$であるから,$F_n \defunc_{(s,t]}$及び
				その各点極限の$F \defunc_{(s,t]}$も可測$\mathcal{P}/\borel{\R}$である.
			
			\item[(3)]
				まず$\mathcal{S} \subset \semiLp{2}{I \times \Omega,\mathcal{P},\mu_M}$であることを示す.
				(\refeq{eq:dfn_simple_predictable_process})の各項が二乗可積分であればよいから,任意に
				$(s,t] \in I$と$F \in \semiLp{\infty}{\Omega,\mathcal{F}_s,\mu}$を取り$F \defunc{(s,t]}$の二乗可積分性を示す.
				\begin{align}
					E \coloneqq \left\{\, \Norm{F}{\semiLp{\infty}{\mu}} < F\, \right\}
				\end{align}
				とおけば補題\ref{lem:holder_inequality}より$\mu(E) = 0$となるから,$\mu_M(I \times E) = 0$が従い
				\begin{align}
					\int_{I \times \Omega} |F(\omega)|^2 \defunc_{(s,t]}(u)\ \mu_M(d(u,\omega)) 
					\leq \Norm{F}{\semiLp{\infty}{\mu}}^2 \mu_M\left((s,t] \times (\Omega \backslash E)\right) 
				\end{align}
				が成り立つ.また$M \in \mathcal{M}_{2,c}$であるから$t$ごとに$\inprod<M>_t$は可積分であり
				\begin{align}
					\mu_M\left((s,t] \times (\Omega \backslash E)\right) = \int_{\Omega} \inprod<M>_t(\omega) - \inprod<M>_s(\omega)\ \mu(d\omega) < \infty
				\end{align}
				が得られる.
				次に$\mathcal{S}$が$\semiLp{2}{I \times \Omega,\mathcal{P},\mu_M}$で稠密であることを示す.
				\begin{align}
					\mathcal{D} \coloneqq \Set{B \in \mathcal{P}}{\mbox{$\forall \epsilon > 0,\ \exists S \in \mathcal{S},\ \mbox{s.t.}\ \Norm{\defunc_B - S}{\semiLp{2}{\mu_M}} < \epsilon.$}}
				\end{align}
				とおけば$\Pi \subset \mathcal{D}$を満たす.実際任意の$B \in \Pi$は$B = (s,t] \times A\ (A \in \mathcal{F}_s)$或は$B = \{0\} \times A\ (A \in \mathcal{F}_0)$
				と表現され,$\defunc_A \in \semiLp{\infty}{\mu}$を満たすから$\defunc_{(s,t] \times A} = \defunc_{(s,t]} \defunc_A \in \mathcal{S}$が成り立つ.
				そしてまた,以下に示すように$\mathcal{D}$はDynkin族である.
				\begin{itemize}
					\item $B = I \times \Omega$に対し$S = \defunc_{(0,T] \times \Omega}$とすれば
						\begin{align}
							\Norm{\defunc_B - S}{\semiLp{2}{\mu_M}}^2
							= \mu_M(\{0\} \times \Omega)
							= 0
						\end{align}
						が成り立ち$I \times \Omega \in \mathcal{D}$が従う.
					
					\item $B_1,B_2 \in \mathcal{D},\ B_1 \subset B_2$と$\epsilon > 0$を任意に取る.
						\begin{align}
							\Norm{\defunc_{B_1} - S_1}{\semiLp{2}{\mu_M}} < \frac{\epsilon}{2}, \quad
							\Norm{\defunc_{B_2} - S_2}{\semiLp{2}{\mu_M}} < \frac{\epsilon}{2}
						\end{align}
						を満たす$S_1,S_2 \in \mathcal{S}$を選べば,(1)より$S_2 - S_1 \in \mathcal{S}$であり,かつ
						\begin{align}
							\Norm{\defunc_{B_2} - \defunc_{B_1} - \left(S_2 - S_1\right)}{\semiLp{2}{\mu_M}} < \epsilon
						\end{align}
						が成り立つから$B_2 \backslash B_1 \in \mathcal{D}$が従う.
						
					\item 互いに素な列$B_n \in \mathcal{D}\ (n=1,2,\cdots)$と$\epsilon > 0$を任意に取る.
						各$B_n$に対し
						\begin{align}
							\Norm{\defunc_{B_n} - S_n}{\semiLp{2}{\mu_M}} < \frac{\epsilon}{2^n}
						\end{align}
						を満たす$S_n \in \mathcal{S}$を選ぶ.任意の$N \in \N$に対し
						\begin{align}
							\Norm{\defunc_{\sum_{n=1}^{\infty}B_n} - \sum_{n=1}^{N}S_n}{\semiLp{2}{\mu_M}}
							&= \Norm{\sum_{n=1}^{\infty}\defunc_{B_n} - \sum_{n=1}^{N}S_n}{\semiLp{2}{\mu_M}} \\
							&\leq \Norm{\sum_{n=N+1}^{\infty}\defunc_{B_n}}{\semiLp{2}{\mu_M}} + \Norm{\sum_{n=1}^{N}\defunc_{B_n} - \sum_{n=1}^{N}S_n}{\semiLp{2}{\mu_M}} \\
							&\leq \Norm{\sum_{n=N+1}^{\infty}\defunc_{B_n}}{\semiLp{2}{\mu_M}} + \sum_{n=1}^{N} \Norm{\defunc_{B_n} - S_n}{\semiLp{2}{\mu_M}} \\
							&< \Norm{\sum_{n=N+1}^{\infty}\defunc_{B_n}}{\semiLp{2}{\mu_M}} + \frac{\epsilon}{2}
						\end{align}
						が成り立つ.$\sum_{n=N+1}^{\infty}\defunc_{B_n}$は可積分関数1で抑えられ,かつ各点$(t,\omega) \in I \times \Omega$において
						\begin{align}
							\sum_{n=N+1}^{\infty}\defunc_{B_n}(t,\omega) \longrightarrow 0
							\quad (N \longrightarrow \infty)
						\end{align}
						となるからLebesgueの収束定理より
						\begin{align}
							\Norm{\sum_{n=N+1}^{\infty}\defunc_{B_n}}{\semiLp{2}{\mu_M}} \longrightarrow 0
							\quad (N \longrightarrow \infty)
						\end{align}
						が成り立つ.従って
						\begin{align}
							\Norm{\sum_{n=N+1}^{\infty}\defunc_{B_n}}{\semiLp{2}{\mu_M}} < \frac{\epsilon}{2}
						\end{align}
						を満たすように$N \in \N$を選べば,(1)より$\sum_{n=1}^{N}S_n \in \mathcal{S}$が成り立ち,かつ
						\begin{align}
							\Norm{\defunc_{\sum_{n=1}^{\infty}B_n} - \sum_{n=1}^{N}S_n}{\semiLp{2}{\mu_M}} < \epsilon
						\end{align}
						も得られ$\sum_{n=1}^{\infty} B_n \in \mathcal{D}$が従う.
				\end{itemize}
				以上で$\mathcal{P} = \mathcal{D}$が示された.つまり$S$は$\mathcal{P}$-可測の単関数を近似できるから
				主張が従う.
				\QED
		\end{description}
	\end{prf}
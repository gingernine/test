\section{伊藤積分}
	
	\begin{screen}
		\begin{dfn}[単純可予測過程に対する伊藤積分]
			任意の$X \in \mathcal{S}$は(\refeq{eq:dfn_simple_predictable_process})に倣って
			\begin{align}
				X = F \defunc_{\{0\}} + \sum_{i=0}^{n-1} F_i \defunc_{\left(t_i,t_{i+1}\right]}
				\label{eq:dfn_Ito_integral_simple_predictable_process}
			\end{align}
			と表現される.$X$に対し$M \in \mathcal{M}_{2,c}$を取り
			\begin{align}
				I_M(X)(t,\omega) \coloneqq \sum_{i=0}^{n-1} F_i(\omega) \left(M_{t \wedge t_{i+1}}(\omega) - M_{t \wedge t_i}(\omega)\right)
				\quad (\forall t \in I,\omega \in \Omega)
			\end{align}
			として定める$I_M$を単純可予測過程に対する伊藤積分(It\Hat{o} integral)とする.
			\label{dfn:Ito_integral_simple_predictable_process}
		\end{dfn}
	\end{screen}
	
	\begin{screen}
		\begin{thm}[伊藤積分の連続線型性]
			定義\ref{dfn:Ito_integral_simple_predictable_process}の$I_M$は
			$\mathcal{S}$から$\mathcal{M}_{2,c}$への線型作用素であり,次の意味で等長である:
			\begin{align}
				\Norm{X}{\Lp{2}{\mu_M}} = \Norm{I_M(X)}{\mathfrak{M}_{2,c}}
				\quad (X \in \mathcal{S}).
			\end{align}
		\end{thm}
	\end{screen}
	
	\begin{prf}\mbox{}
		\begin{description}
			\item[線型性]
				任意に$X_1,X_2 \in \mathcal{S}$と$\alpha \in \R$を取る.
				\begin{description}
					\item[加法]
						$X_1,X_2$の表示を
						\begin{align}
							X_1 = F \defunc_{\{0\}} + \sum_{i=0}^{n-1} F_i \defunc_{\left(t_i,t_{i+1}\right]},
							\quad X_2 = G \defunc_{\{0\}} + \sum_{j=0}^{m-1} G_j \defunc_{\left(s_j,s_{j+1}\right]}
						\end{align}
						として,時点の分点の合併を$0=u_0 < u_1 < \cdots < u_r = T$と表し
						\begin{align}
							\tilde{F}_k &\coloneqq F_i \quad (t_i \leq u_k < t_{i+1},\ i=0,\cdots,n-1), \\
							\tilde{G}_k &\coloneqq G_j \quad (s_j \leq u_k < s_{j+1},\ j=0,\cdots,m-1)
						\end{align}
						とおく.$\sum_{\alpha \leq u_k < \beta}^{(k)}$と表記して$\alpha \leq u_k < \beta$を満たす$k$についての総和記号を定めれば
						\begin{align}
							\sum_{k=0}^{r-1} \tilde{F}_k(\omega) \left( M_{t \wedge u_{k+1}}(\omega) - M_{t \wedge u_k}(\omega) \right)
							&= \sum_{i=0}^{n-1} \sum_{t_i \leq u_k < t_{i+1}}^{(k)} F_i(\omega) \left( M_{t \wedge u_{k+1}}(\omega) - M_{t \wedge u_k}(\omega) \right) \\
							&= \sum_{i=0}^{n-1} F_i(\omega) \left( M_{t \wedge t_{i+1}}(\omega) - M_{t \wedge t_i}(\omega) \right)
						\end{align}
						が成り立つから,(\refeq{eq:lem_properties_of_simple_predictable_processes})と併せて
						\begin{align}
							I_M(X_1 + X_2)(t,\omega)
							&= \sum_{k=0}^{r-1} \left( \tilde{F}_k(\omega) + \tilde{G}_k(\omega) \right) \left( M_{t \wedge u_{k+1}}(\omega) - M_{t \wedge u_k}(\omega) \right) \\
							&= I_M(X_1) + I_M(X_2)(t,\omega)
							\quad (\forall t \in I,\omega \in \Omega)
						\end{align}
						が得られる.
					\item[スカラ倍]
						$X_1$と$\alpha$に対して次が成り立つ:
						\begin{align}
							&I_M(\alpha X_1)(t,\omega)
							= \sum_{i=0}^{n-1} \alpha F_i(\omega) \left(M_{t \wedge t_{i+1}}(\omega) - M_{t \wedge t_i}(\omega)\right) \\
							&\qquad= \alpha \sum_{i=0}^{n-1} F_i(\omega) \left(M_{t \wedge t_{i+1}}(\omega) - M_{t \wedge t_i}(\omega)\right)
							= \alpha I_M(X_1)(t,\omega)
							\quad (\forall t \in I,\omega \in \Omega).
						\end{align}
					
				\end{description}
			\item[等長性]
				任意に$X \in \mathcal{S}$を取り,(\refeq{eq:dfn_Ito_integral_simple_predictable_process})を$X$の表示とする.
				\begin{align}
					d
				\end{align}
		\end{description}
	\end{prf}
\section{伊藤積分}
	
	\begin{screen}
		\begin{dfn}[単純可予測過程に対する伊藤積分]
			任意に$X \in \mathcal{S}$を取れば,(\refeq{eq:dfn_simple_predictable_process})に倣って
			\begin{align}
				X = F \defunc_{\{0\}} + \sum_{i=0}^{n-1} F_i \defunc_{\left(t_i,t_{i+1}\right]}
				\label{eq:dfn_Ito_integral_simple_predictable_process}
			\end{align}
			と表現される.$M \in \mathcal{M}_{2,c}$を用いて,単純可予測過程に対する伊藤積分(It\Hat{o} integral)$\ I_M$を
			\begin{align}
				I_M(X)(t,\omega) \coloneqq \sum_{i=0}^{n-1} F_i(\omega) \left(M_{t \wedge t_{i+1}}(\omega) - M_{t \wedge t_i}(\omega)\right)
				\quad (\forall t \in I,\omega \in \Omega)
			\end{align}
			として定める.またこれに並行して次を定める:
			\begin{align}
				\tilde{I}_M:\mathfrak{S} \ni \equiv{X}{\mathfrak{S}} \longmapsto \equiv{I_M(X)}{2,c} \in \mathfrak{M}_{2,c}.
			\end{align}
			\label{dfn:Ito_integral_simple_predictable_process}
		\end{dfn}
	\end{screen}
	
	\begin{screen}
		\begin{lem}[$\tilde{I}_M$はwell-defined]
			任意の$M \in \mathcal{M}_{2,c}$と
			$X_1,X_2 \in \equiv{X}{\mathfrak{S}} \in \mathfrak{S}$に対し,
			$\equiv{I_M(X_1)}{2,c} = \equiv{I_M(X_2)}{2,c}$が成り立つ.
		\end{lem}
	\end{screen}
	
	\begin{prf}
		$X_1,X_2$が,時点の列$0=u_0<u_1<\cdots<u_r = T$と集合の系$F,G \in \semiLp{\infty}{\mathcal{F}_0,\mu},
		F_k,G_k \in \semiLp{\infty}{\mathcal{F}_{u_k},\mu}\ (k=0,1,\cdots,r-1)$
		を用いて次で表示されていると仮定する:
		\begin{align}
			X_1 = F \defunc_{\{0\}} + \sum_{k=0}^{r-1} F_k \defunc_{\left(u_k,u_{k+1}\right]},
			\quad X_2 = G \defunc_{\{0\}} + \sum_{k=0}^{r-1} G_k \defunc_{\left(u_k,u_{k+1}\right]}.
		\end{align}
		ここで
		\begin{align}
			A \coloneqq \Set{\omega \in \Omega}{F(\omega) \neq G(\omega)},
			\quad A_k \coloneqq \Set{\omega \in \Omega}{F_k(\omega) \neq G_k(\omega)},
			\quad (k=0,1,\cdots,r-1)
		\end{align}
		とおけば
		\begin{align}
			\Set{(t,\omega) \in I \times \Omega}{X_1(t,\omega) \neq X_2(t,\omega)}
			= \{0\} \times A + \sum_{k=0}^{r-1} \left( u_k,u_{k+1} \right] \times A_k
		\end{align}
		と表され,$X_1 = X_2\ \mu_M$-a.s.であるから
		\begin{align}
			0 = \mu_M\left( \{0\} \times A \right) + \sum_{k=0}^{r-1} \mu_M\left( \left( u_k,u_{k+1} \right] \times A_k \right)
			= \sum_{k=0}^{r-1} \int_{A_k} \inprod<M>_{u_{k+1}}(\omega) - \inprod<M>_{u_k}(\omega)\ \mu(d\omega)
		\end{align}
		が成り立つ.従って任意の$t \in I$に対して
		\begin{align}
			\int_{A_k} \inprod<M>_{t \wedge u_{k+1}}(\omega) - \inprod<M>_{t \wedge u_k}(\omega)\ \mu(d\omega) = 0
			\quad (\forall k=0,1,\cdots,r-1)
		\end{align}
		が満たされる.$M^2 - \inprod<M>$のマルチンゲール性と$A_k \in \mathcal{F}_{u_k}$より,任意の$t \in I$に対して
		\begin{align}
			\int_{A_k} \left( M_{t \wedge u_{k+1}}(\omega) \right)^2 - \left( M_{t \wedge u_k}(\omega) \right)^2\ \mu(d\omega) = 0
			\quad (\forall k=0,1,\cdots,r-1)
		\end{align}
		が成り立つ.
	\end{prf}
	
	\begin{screen}
		\begin{thm}[伊藤積分の線型等長性]
			任意の$M \in \mathcal{M}_{2,c}$に対し$I_M$は線型であり,次を満たす:
			\begin{align}
				\int_{I \times \Omega} \left| X(t,\omega) \right|^2\ \mu_M(dtd\omega)
				= \int_\Omega \left| I_M(X)_T(\omega) \right|^2\ \mu(d\omega)
				\quad (X \in \mathcal{S}).
				\label{eq:thm_Ito_integral_linearity_isometric}
			\end{align}
			\label{thm:Ito_integral_linearity_isometric}
		\end{thm}
	\end{screen}
	
	\begin{prf}\mbox{}
		\begin{description}
			\item[線型性]
				任意に$X_1,X_2 \in \mathcal{S}$と$\alpha \in \R$を取る.
				\begin{description}
					\item[加法]
						$X_1,X_2$が,時点の列$0=t_0<t_1<\cdots<t_n = T$と集合の系$F,G \in \semiLp{\infty}{\mathcal{F}_0,\mu},
						F_k,G_k \in \semiLp{\infty}{\mathcal{F}_{t_k},\mu}\ (k=0,1,\cdots,n-1)$
						を用いて次で表示されていると仮定する:
						\begin{align}
							X_1 = F \defunc_{\{0\}} + \sum_{k=0}^{n-1} F_k \defunc_{\left(t_k,t_{k+1}\right]},
							\quad X_2 = G \defunc_{\{0\}} + \sum_{k=0}^{n-1} G_k \defunc_{\left(t_k,t_{k+1}\right]}.
						\end{align}
						このとき次が成り立つ:
						\begin{align}
							I_M(X_1 + X_2)(t,\omega)
							&= \sum_{k=0}^{n-1} \left( F_k(\omega) + G_k(\omega) \right) \left( M_{t \wedge t_{k+1}}(\omega) - M_{t \wedge t_k}(\omega) \right) \\
							&= I_M(X_1)(t,\omega) + I_M(X_2)(t,\omega)
							\quad (\forall t \in I,\omega \in \Omega).
						\end{align}
					\item[スカラ倍]
						$X_1$と$\alpha$に対して次が成り立つ:
						\begin{align}
							&I_M(\alpha X_1)(t,\omega)
							= \sum_{k=0}^{n-1} \alpha F_k(\omega) \left(M_{t \wedge t_{k+1}}(\omega) - M_{t \wedge t_k}(\omega)\right) \\
							&\qquad= \alpha \sum_{k=0}^{n-1} F_k(\omega) \left(M_{t \wedge t_{k+1}}(\omega) - M_{t \wedge t_k}(\omega)\right)
							= \alpha I_M(X_1)(t,\omega)
							\quad (\forall t \in I,\omega \in \Omega).
						\end{align}
				\end{description}
				
			\item[等長性]
				任意に$X \in \mathcal{S}$を取り,(\refeq{eq:dfn_Ito_integral_simple_predictable_process})を$X$の表示とする.
				\begin{align}
					d
				\end{align}
		\end{description}
	\end{prf}
	
	\begin{screen}
		\begin{thm}[$\tilde{I}_M$はwell-defined・$\tilde{I}_M$の線型等長性]
			任意の$M \in \mathcal{M}_{2,c}$と
			$X_1,X_2 \in \equiv{X}{\mathfrak{S}} \in \mathfrak{S}$に対し,
			$\equiv{I_M(X_1)}{2,c} = \equiv{I_M(X_2)}{2,c}$が成り立つ.
			これにより$\tilde{I}_M$は$\equiv{X}{\mathfrak{S}} \in \mathfrak{S}$に対し
			$\equiv{I_M(X)}{2,c}$のみを対応させ,更に線型性と次の意味での等長性を持つ:
			\begin{align}
				\Norm{\equiv{X}{\mathfrak{S}}}{\Lp{2}{\mu_M}} = \Norm{\equiv{I_M(X)}{2,c}}{\mathfrak{M}_{2,c}}
				\quad (\equiv{X}{\mathfrak{S}} \in \mathfrak{S}).
				\label{eq:thm_Ito_integral_equiv_class_linearity_isometric}
			\end{align}
			\label{thm:Ito_integral_equiv_class_linearity_isometric}
		\end{thm}
	\end{screen}
	
	\begin{prf}
		定理\ref{thm:Ito_integral_linearity_isometric}の$I_M$の線型性と(\refeq{eq:thm_Ito_integral_linearity_isometric})より,
		任意の$X_1,X_2 \in \mathcal{S}$に対して
		\begin{align}
			&\int_{I \times \Omega} \left| X_1(t,\omega) - X_2(t,\omega) \right|^2\ \mu_M(dtd\omega)
			= \int_{I \times \Omega} \left| (X_1 - X_2)(t,\omega) \right|^2\ \mu_M(dtd\omega) \\
			&\qquad = \int_\Omega \left| I_M(X_1 - X_2)_T(\omega) \right|^2\ \mu(d\omega)
			= \int_\Omega \left| I_M(X_1)_T(\omega) - I_M(X_2)_T(\omega) \right|^2\ \mu(d\omega)
		\end{align}
		が成り立つ.これにより$\equiv{X_1}{\mathfrak{S}} = \equiv{X_2}{\mathfrak{S}}$のとき
		$\equiv{I_M(X_1)}{2,c} = \equiv{I_M(X_2)}{2,c}$が従う.
		(\refeq{eq:thm_Ito_integral_equiv_class_linearity_isometric})は
		(\refeq{eq:thm_Ito_integral_linearity_isometric})より得られ,
		また任意に$\equiv{X_1}{\mathfrak{S}},\equiv{X_2}{\mathfrak{S}} \in \mathfrak{S},\ \alpha,\beta \in \R$を取れば,
		$I_M$及び$\equiv{\cdot}{\mathfrak{S}},\equiv{\cdot}{2,c}$の線型性より
		\begin{align}
			&\tilde{I}\left( \alpha \equiv{X_1}{\mathfrak{S}} + \beta \equiv{X_2}{\mathfrak{S}} \right)
			= \tilde{I}\left( \equiv{\alpha X_1 + \beta X_2}{\mathfrak{S}} \right)
			= \equiv{I(\alpha X_1 + \beta X_2)}{2,c} \\
			&\qquad = \equiv{\alpha I(X_1)}{2,c} + \equiv{\beta I(X_2)}{2,c}
			= \alpha \tilde{I}\left( \equiv{X_1}{\mathfrak{S}} \right) + \beta \tilde{I}\left( \equiv{X_2}{\mathfrak{S}} \right)
		\end{align}
		が成り立つ.
	\end{prf}
	
	\begin{screen}
		\begin{thm}[伊藤積分の拡張]
			定理\ref{thm:linear_operator_expansion}と補題\ref{lem:properties_of_simple_predictable_processes}及び
			定理\ref{thm:Ito_integral_equiv_class_linearity_isometric}により,
			$M \in \mathcal{M}_{2,c}$に対し$\tilde{I}_M$は$\Lp{2}{I \times \Omega,\mathcal{P},\mu_M}$上の線型作用素に
			拡張可能である.この拡張作用素も同様に$\tilde{I}_M$と表記し,伊藤積分$I_M$についても,
			任意の$X \in \semiLp{2}{I \times \Omega,\mathcal{P},\mu_M}$に対し
			或る$N \in \equiv{\tilde{I}_M\left( \equiv{X}{\Lp{2}{\mu_M}} \right)}{2,c}$を対応させるものとして拡張し,
			$N$を$I_M(X)$と表記しこれを伊藤積分として定義しなおす.
		\end{thm}
	\end{screen}
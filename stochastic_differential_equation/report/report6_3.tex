\section{伊藤積分}
	
	\begin{screen}
		\begin{dfn}[単純可予測過程に対する伊藤積分]
			任意に$X \in \mathcal{S}$を取れば,(\refeq{eq:dfn_simple_predictable_process})に倣って
			\begin{align}
				X = F \defunc_{\{0\}} + \sum_{i=0}^{n-1} F_i \defunc_{\left(t_i,t_{i+1}\right]}
				\label{eq:dfn_Ito_integral_simple_predictable_process}
			\end{align}
			と表現される.$M \in \mathcal{M}_{2,c}$に対し$I_M:\mathcal{S} \rightarrow \mathcal{M}_{2,c}$を
			\begin{align}
				I_M(X)(t,\omega) \coloneqq \sum_{i=0}^{n-1} F_i(\omega) \left(M_{t \wedge t_{i+1}}(\omega) - M_{t \wedge t_i}(\omega)\right)
				\quad (\forall t \in I,\omega \in \Omega)
			\end{align}
			により定め,単純可予測過程に対する伊藤積分(It\Hat{o} integral)とする.
			\label{dfn:Ito_integral_simple_predictable_process}
		\end{dfn}
	\end{screen}
	
	\begin{screen}
		\begin{thm}[単純可予測過程に対する伊藤積分の線型等長性]
			任意の$M \in \mathcal{M}_{2,c}$に対し$I_M$は$\mathcal{S}$から$\mathcal{M}_{2,c}$への線型作用素であり,
			任意の停止時刻$\tau$に対して次を満たす:
			\begin{align}
				\int_{I \times \Omega} \left| X(t,\omega) \right|^2 \defunc_{[0,\tau(\omega)]}(t)\ \mu_M(dtd\omega)
				= \int_\Omega \left| I_M(X)(\tau(\omega),\omega) \right|^2\ \mu(d\omega)
				\quad (X \in \mathcal{S}).
				\label{eq:thm_Ito_integral_linearity_isometric}
			\end{align}
			特に$\tau = T$の場合次を得る:
			\begin{align}
				\int_{I \times \Omega} \left| X(t,\omega) \right|^2\ \mu_M(dtd\omega)
				= \int_\Omega \left| I_M(X)_T(\omega) \right|^2\ \mu(d\omega)
				\quad (X \in \mathcal{S}).
				\label{eq:thm_Ito_integral_linearity_isometric_3}
			\end{align}
			\label{thm:Ito_integral_linearity_isometric}
		\end{thm}
	\end{screen}
	
	\begin{prf}\mbox{}
		\begin{description}
			\item[線型性]
				先ず任意の$X \in \mathcal{S}$に対し$I_M(X) \in \mathcal{M}_{2,c}$となることを示す.
				次に$I_M$の線型性を示す.任意に$X_1,X_2 \in \mathcal{S}$と$\alpha \in \R$を取る.
				\begin{description}
					\item[加法]
						$X_1,X_2$が,時点の列$0=t_0<t_1<\cdots<t_n = T$と集合の系$F,G \in \semiLp{\infty}{\mathcal{F}_0,\mu},
						F_k,G_k \in \semiLp{\infty}{\mathcal{F}_{t_k},\mu}\ (k=0,1,\cdots,n-1)$
						を用いて次で表示されていると仮定する:
						\begin{align}
							X_1 = F \defunc_{\{0\}} + \sum_{k=0}^{n-1} F_k \defunc_{\left(t_k,t_{k+1}\right]},
							\quad X_2 = G \defunc_{\{0\}} + \sum_{k=0}^{n-1} G_k \defunc_{\left(t_k,t_{k+1}\right]}.
						\end{align}
						このとき次が成り立つ:
						\begin{align}
							I_M(X_1 + X_2)(t,\omega)
							&= \sum_{k=0}^{n-1} \left( F_k(\omega) + G_k(\omega) \right) \left( M_{t \wedge t_{k+1}}(\omega) - M_{t \wedge t_k}(\omega) \right) \\
							&= I_M(X_1)(t,\omega) + I_M(X_2)(t,\omega)
							\quad (\forall t \in I,\omega \in \Omega).
						\end{align}
					\item[スカラ倍]
						$X_1$と$\alpha$に対して次が成り立つ:
						\begin{align}
							&I_M(\alpha X_1)(t,\omega)
							= \sum_{k=0}^{n-1} \alpha F_k(\omega) \left(M_{t \wedge t_{k+1}}(\omega) - M_{t \wedge t_k}(\omega)\right) \\
							&\qquad= \alpha \sum_{k=0}^{n-1} F_k(\omega) \left(M_{t \wedge t_{k+1}}(\omega) - M_{t \wedge t_k}(\omega)\right)
							= \alpha I_M(X_1)(t,\omega)
							\quad (\forall t \in I,\omega \in \Omega).
						\end{align}
				\end{description}
				
			\item[等長性]
				$B \coloneqq \Set{(t,\omega) \in I \times \Omega}{t \leq \tau(\omega)}$とおけば,
				$\defunc_B$は
				\begin{align}
					\defunc_{[0,\tau(\omega)]}(t) = \defunc_B(t,\omega)
					\quad (\forall (t,\omega) \in I \times \Omega)
				\end{align}
				を満たし,かつ左連続な適合過程であるから定理\ref{thm:left_continuous_adapted_then_predictable}により可測$\mathcal{P}/\borel{\R}$であるから
				(\refeq{eq:thm_Ito_integral_linearity_isometric})左辺の積分を考察できる.
				今,$X \in \mathcal{S}$が(\refeq{eq:dfn_simple_predictable_process})により表示されているとすれば,
				\begin{align}
					&\int_\Omega \left| I_M(X)(\tau(\omega),\omega) \right|^2\ \mu(d\omega) \\
					&\qquad = \int_\Omega \left| \sum_{i=0}^{n-1} F_i(\omega) \left( M^\tau_{t_{i+1}}(\omega) - M^\tau_{t_i}(\omega) \right) \right|^2\ \mu(d\omega) \\
					&\qquad = \sum_{i=0}^{n-1} \int_\Omega \left| F_i(\omega) \left( M^\tau_{t_{i+1}}(\omega) - M^\tau_{t_i}(\omega) \right) \right|^2\ \mu(d\omega) \\
						&\quad\qquad + 2 \sum_{i<j} \int_\Omega F_i(\omega) F_j(\omega) \left( M^\tau_{t_{i+1}}(\omega) - M^\tau_{t_i}(\omega) \right) 
							\left( M^\tau_{t_{j+1}}(\omega) - M^\tau_{t_j}(\omega) \right)\ \mu(d\omega)
					\label{eq:thm_Ito_integral_linearity_isometric_2}
				\end{align}
				が成り立つ.右辺第二項についてはマルチンゲール性より
				\begin{align}
					&\int_\Omega F_i(\omega) F_j(\omega) \left( M^\tau_{t_{i+1}}(\omega) - M^\tau_{t_i}(\omega) \right) 
							\left( M^\tau_{t_{j+1}}(\omega) - M^\tau_{t_j}(\omega) \right)\ \mu(d\omega) \\
					&= \int_\Omega F_i(\omega) F_j(\omega) \left( M^\tau_{t_{i+1}}(\omega) - M^\tau_{t_i}(\omega) \right) 
							\cexp{M^\tau_{t_{j+1}}(\omega) - M^\tau_{t_j}(\omega)}{\mathcal{F}_{t_j}}(\omega)\ \mu(d\omega)
							= 0
				\end{align}
				が従い,右辺第一項についても
				\begin{align}
					\int_\Omega \left| F_i(\omega) \left( M^\tau_{t_{i+1}}(\omega) - M^\tau_{t_i}(\omega) \right) \right|^2\ \mu(d\omega)
					= \int_\Omega |F_i(\omega)|^2 \left( \inprod<M^\tau>_{t_{i+1}}(\omega) - \inprod<M^\tau>_{t_i}(\omega) \right)\ \mu(d\omega)
				\end{align}
				が成り立つから
				\begin{align}
					\mbox{(\refeq{eq:thm_Ito_integral_linearity_isometric_2})}
					&= \sum_{i=0}^{n-1} \int_\Omega |F_i(\omega)|^2 \left( \inprod<M>^\tau_{t_{i+1}}(\omega) - \inprod<M>^\tau_{t_i}(\omega) \right)\ \mu(d\omega) \\
					&= \int_\Omega \int_I |X(t,\omega)|^2 \defunc_B(t,\omega)\ \inprod<M>(dt,\omega)\ \mu(d\omega)
				\end{align}
				が導かれ,(\refeq{eq:lem_properties_of_simple_predictable_processes_0})により主張を得る.
				\QED
		\end{description}
	\end{prf}
	
	\begin{screen}
		\begin{thm}[同値類に対する伊藤積分]
			$\tau$を停止時刻とし,$M \in \mathcal{M}_{2,c}$を取り
			\begin{align}
				B \coloneqq \Set{(t,\omega) \in I \times \Omega}{t \leq \tau(\omega)}
			\end{align}
			とおく.任意の$X_1,X_2 \in \equiv{X}{\mathfrak{S}} \in \mathfrak{S}$に対し,
			$\equiv{I_M(X_1)^\tau}{2,c} = \equiv{I_M(X_2)^\tau}{2,c}$が成り立つ.従って
			\begin{align}
				\tilde{I}^\tau_M:\mathfrak{S} \ni \equiv{X}{\mathfrak{S}} \longmapsto \equiv{I_M(X)^\tau}{2,c} \in \mathfrak{M}_{2,c}
			\end{align}
			により定める$\tilde{I}^\tau_M$はwell-definedであり,更に線型性と次の意味での等長性を持つ:
			\begin{align}
				\Norm{\equiv{X \defunc_B}{\mathfrak{S}}}{\Lp{2}{\mu_M}} = \Norm{\equiv{I_M(X)^\tau}{2,c}}{\mathfrak{M}_{2,c}}
				\quad (\forall X \in \mathcal{S}).
				\label{eq:thm_Ito_integral_equiv_class_linearity_isometric}
			\end{align}
			\label{thm:Ito_integral_equiv_class_linearity_isometric}
		\end{thm}
	\end{screen}
	
	\begin{prf}
		定理\ref{thm:Ito_integral_linearity_isometric}の$I_M$の線型性と(\refeq{eq:thm_Ito_integral_linearity_isometric})より,
		任意の$X_1,X_2 \in \mathcal{S}$に対して
		\begin{align}
			&\int_\Omega \left| I_M(X_1)^\tau(\omega) - I_M(X_2)^\tau(\omega) \right|^2\ \mu(d\omega)
			= \int_\Omega \left| I_M(X_1 - X_2)^\tau(\omega) \right|^2\ \mu(d\omega) \\
			&\qquad = \int_{I \times \Omega} \left| (X_1 - X_2)(t,\omega) \right|^2 \defunc_B(t,\omega)\ \mu_M(dtd\omega)
			= \int_{I \times \Omega} \left| X_1(t,\omega) - X_2(t,\omega) \right|^2 \defunc_B(t,\omega)\ \mu_M(dtd\omega)
		\end{align}
		が成り立つ.これにより$\equiv{X_1}{\mathfrak{S}} = \equiv{X_2}{\mathfrak{S}}$ならば
		$\equiv{I_M(X_1)}{2,c} = \equiv{I_M(X_2)}{2,c}$である.
		(\refeq{eq:thm_Ito_integral_equiv_class_linearity_isometric})は
		(\refeq{eq:thm_Ito_integral_linearity_isometric})より従い,
		また任意に$\equiv{X_1}{\mathfrak{S}},\equiv{X_2}{\mathfrak{S}} \in \mathfrak{S},\ \alpha,\beta \in \R$を取れば,
		$I_M$及び$\equiv{\cdot}{\mathfrak{S}},\equiv{\cdot}{2,c}$の線型性より
		\begin{align}
			&\tilde{I}^\tau \left( \alpha \equiv{X_1}{\mathfrak{S}} + \beta \equiv{X_2}{\mathfrak{S}} \right)
			= \tilde{I}^\tau \left( \equiv{\alpha X_1 + \beta X_2}{\mathfrak{S}} \right)
			= \equiv{I(\alpha X_1 + \beta X_2)^\tau}{2,c} \\
			&\qquad = \equiv{\alpha I(X_1)^\tau}{2,c} + \equiv{\beta I(X_2)^\tau}{2,c}
			= \alpha \tilde{I}^\tau \left( \equiv{X_1}{\mathfrak{S}} \right) + \beta \tilde{I}^\tau \left( \equiv{X_2}{\mathfrak{S}} \right)
		\end{align}
		が成り立ち$\tilde{I}^\tau_M$の線型性が得られる.
		\QED
	\end{prf}
	
	\begin{screen}
		\begin{thm}[同値類に対する伊藤積分の拡張]
			$M \in \mathcal{M}_{2,c}$とする.
			任意の停止時刻$\tau$に対し,定理\ref{thm:Ito_integral_equiv_class_linearity_isometric}で定めた
			$\tilde{I}^\tau_M$は$\Lp{2}{I \times \Omega,\mathcal{P},\mu_M}$上の等長線型作用素に
			拡張可能である.
			\label{thm:expansion_of_Ito_integral_equiv_class}
		\end{thm}
	\end{screen}
	
	\begin{prf}
		補題\ref{lem:properties_of_simple_predictable_processes}の等長単射
		\begin{align}
			J:\mathfrak{S} \ni \equiv{X}{\mathfrak{S}} \longmapsto \equiv{X}{\Lp{2}{\mu_M}} \in \Lp{2}{I \times \Omega,\mathcal{P},\mu_M}
		\end{align}
		に対し,値域を$J\mathfrak{S}$に制限した線形全単射を$\tilde{J}$と表す.
		定理\ref{thm:Ito_integral_equiv_class_linearity_isometric}より$\tilde{I}^\tau_M$も線型性を持つから
		\begin{align}
			\tilde{I}^\tau_M \circ \tilde{J}^{-1}: J \mathfrak{S} \ni \equiv{X}{\Lp{2}{\mu_M}} \longmapsto \equiv{I_M(X)^\tau}{2,c} \in \mathfrak{M}_{2,c}
		\end{align}
		もまた線型写像である.この$\tilde{I}^\tau_M \circ \tilde{J}^{-1}$を
		補題\ref{lem:properties_of_simple_predictable_processes}及び定理\ref{thm:linear_operator_expansion}により拡張すればよい.
		\QED
	\end{prf}
	
	\begin{screen}
		\begin{dfn}[伊藤積分の拡張]
			停止時刻$\tau$に対し
			定理\ref{thm:expansion_of_Ito_integral_equiv_class}で拡張した作用素もまた$\tilde{I}^\tau_M$と書く.
			$\tau = T$として,$X \in \semiLp{2}{\mathcal{P},\mu_M}$に対し
			或る$N \in \tilde{I}^\tau_M\left( \equiv{X}{\Lp{2}{\mu_M}} \right)$を対応させる関係を
			\begin{align}
				I_M: \semiLp{2}{\mathcal{P},\mu_M} \ni X \longmapsto N \in \tilde{I}^\tau_M\left( \equiv{X}{\Lp{2}{\mu_M}} \right), \quad
				I_M(X)_t = \int_0^t X_s\ d M_s \quad (t \in I)
			\end{align}
			と表記し,この$I_M$を伊藤積分として定義しなおす.
		\end{dfn}
	\end{screen}
	
	\begin{screen}
		\begin{lem}[停止時刻で停めた伊藤積分]
			任意の停止時刻$\tau$に対し
			\begin{align}
				\int_\Omega \left| I_M(X)(\tau(\omega),\omega) \right|^2\ \mu(d\omega)
				= \int_\Omega \int_I \left| X(t,\omega) \right|^2 \defunc_{[0,\tau(\omega)]}(t)\ \inprod<M>(dt,\omega)\ \mu(d\omega).
			\end{align}
		\end{lem}
	\end{screen}
	
	\begin{screen}
		\begin{prp}[伊藤積分の二次変分]
			$M \in \mathcal{M}_{2,c},\ X \in \semiLp{2}{I \times \Omega,\mathcal{P},\mu_M}$に対し次が成り立つ:
			\begin{align}
				\inprod<I_M(X)>_t = \int_0^t X_s^2\ \inprod<M>(ds)
				\quad (\forall t \in I,\ \mbox{$\mu$-a.s.})
			\end{align}
		\end{prp}
	\end{screen}
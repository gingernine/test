\section{二次変分}
	以降では$I \coloneqq [0,T]\ (T>0)$とし,
	このフィルトレーション$(\mathcal{F}_t)_{t \in I}$が次の仮定を満たすものとする:
	\begin{align}
		\mathcal{N} \coloneqq \Set{N \in \mathcal{F}}{\mu(N) = 0}
		\subset \mathcal{F}_0.
	\end{align}
	
	以下,いくつか集合を定義する.
	\begin{description}
		\item[$\mathrm{(1)}\ \mathcal{A}^+$] 
			$\mathcal{A}^+$は以下を満たす$(\Omega,\mathcal{F},\mu)$上の可測関数族$A = (A_t)_{t \in I}$の全体である.
			\begin{description}
				\item[適合性] 任意の$t \in I$に対し,写像$\Omega \ni \omega \longmapsto A_t(\omega) \in \R$は可測$\mathcal{F}_t/\borel{\R}$である.
				\item[連続性] $\mu$-a.s.に写像$I \ni t \longmapsto A_t(\omega) \in \R$が連続である.
				\item[単調非減少性] $\mu$-a.s.に写像$I \ni t \longmapsto A_t(\omega) \in \R$が単調非減少である.
			\end{description}
		
		\item[$\mathrm{(2)}\ \mathcal{A}$]
			$\mathcal{A} \coloneqq \Set{A^1 - A^2}{A^1,A^2 \in \mathcal{A}^+}$
			と定義する.$\mu$-a.s.に写像$t \longmapsto A^1_t(\omega)$と$t \longmapsto A^2_t(\omega)$が連続かつ単調非減少となるから
			すなわち$\mu$-a.s.に写像$t \longmapsto A^1_t(\omega) - A^2_t(\omega)$は有界連続となっている.
			
		\item[$\mathrm{(3)}\ \mathcal{M}_{p,c}\ (p \geq 1)$]
			$\mathcal{M}_{p,c}$は以下を満たす可測関数族$M = (M_t)_{t \in I} \subset \semiLp{p}{\mathcal{F},\mu}$の全体である.
			\begin{description}
				\item[$\mathrm{L}^p$-マルチンゲール] $M = (M_t)_{t \in I}$は$\mathrm{L}^p$-マルチンゲールである.
				\item[連続性] $\mu$-a.s.に写像$I \ni t \longmapsto M_t(\omega) \in \R$が連続である.
			\end{description}
		
		\item[$\mathrm{(4)}\ \mathcal{M}_{b,c}$]
			$\mathcal{M}_{b,c}$は$\mu$-a.s.に連続で一様有界な$\mathrm{L}^1$-マルチンゲールの全体とする.つまり
			\begin{align}
				\mathcal{M}_{b,c} \coloneqq \Set{M = (M_t)_{t \in I} \in \mathcal{M}_{1,c}}{\sup{t \in I}{\Norm{M_t}{\mathscr{L}^\infty}} < \infty}
			\end{align}
			として定義されている.
			
		\item[$\mathrm{(5)}\ \mathcal{T}$]
			$\mathcal{T}$は以下を満たすような,$I$に値を取る停止時刻の列$(\tau_j)_{j=1}^{\infty}$の全体とする.
			\begin{description}
				\item[a)] $\tau_0 = 0 \quad \mbox{$\mu$-a.s.}$
				\item[b)] $\tau_j \leq \tau_{j+1} \quad \mbox{$\mu$-a.s.}\ (j=1,2,\cdots).$
				\item[c)] $(\tau_j)_{j=1}^{\infty}$に対し或る$\mu$-零集合$N_T$が存在し,任意の$\omega \in \Omega \backslash N_T$に対し或る$n = n(\omega) \in \N$が存在して$\tau_n(\omega)=T$が成り立つ.
			\end{description}
			例えば$\tau_j = jT/2^n$なら$(\tau_j)_{j=1}^{\infty} \in \mathcal{T}$となる.
			上の条件において$N \coloneqq N_0 \cup N_T \cup (\cup_{j=1}^{\infty}N_j)$とすればこれも$\mu$-零集合で,$\omega \in \Omega \backslash N$なら
			\begin{align}
				&\tau_0(\omega) = 0,\qquad \tau_j(\omega) \leq \tau_{j+1}(\omega)\ (j=1,2,\cdots),\\
				&\tau_{n_\omega}(\omega) = T\ (\exists n_\omega \in \N)
			\end{align}
			が成立することになる.
			
		\item[$\mathrm{(4)}\ \mathcal{M}_{c,loc}$]
			$\mathcal{M}_{c,loc}$の元を「連続な局所マルチンゲール」という.$\mathcal{M}_{c,loc}$は次で定義される:
			\begin{align}
				\mathcal{M}_{c,loc} \coloneqq 
				\Set{M = (M_t)_{t \in I} \subset \semiLp{1}{\mathcal{F},\mu}}{\exists (\tau_j)_{j=1}^{\infty} \in \mathcal{T}\ \mathrm{s.t.}\ M^j = (M_{\tau_j \wedge t})_{t \in I} \in \mathcal{M}_{b,c}\ (\forall j \in \N)}.
			\end{align}
	\end{description}
	
	\begin{itembox}[l]{}
		\begin{lem}[$\mathcal{M}_{p,c}$は線形空間となる]\mbox{}\\
			任意の$M,N \in \mathcal{M}_{p,c}$と$\alpha \in \R$に対して線型演算を
			\begin{align}
				M + N \coloneqq (M_t + N_t)_{t \in I}, \qquad \alpha M \coloneqq (\alpha M_t)_{t \in I}
				\label{eq:mart_linear_arithmetic_0}
			\end{align}
			として定義すれば,$\mathcal{M}_{p,c}$は$\R$上の線形空間となる.
		\end{lem}
	\end{itembox}
	
	\footnotetext{
		全ての$t,\omega$に対し$0 \in \R$を取るもの.
	}
	
	\begin{prf}
		$\mathcal{M}_{p,c}$が(\refeq{eq:mart_linear_arithmetic_0})の演算について閉じていることを示す.
		\begin{description}
			\item[加法について]
				先ず$M+N$が$\mathrm{L}^p$-マルチンゲールの定義を満たすことを確認する.
				適合性について,各$t \in I$について$M_t,N_t$は$\mathcal{F}_t$-可測であるから
				$M_t + N_t$も$\mathcal{F}_t$-可測であり,またMinkowskiの不等式より$M_t + N_t \in \semiLp{p}{\mu}$であることも従う.
				任意の$\omega \in \Omega$でパス$I \ni t \longmapsto M_t(\omega) + N_t(\omega)$が右連続且つ左極限を持つ
				ことも$M,N$のパスが右連続且つ左極限を持つことにより従い,さらに
				任意に$0 \leq s \leq t \leq T$を取れば,条件付き期待値の線型性より
				\begin{align}
					\cexp{M_t + N_t}{\mathcal{F}_s} = \cexp{M_t}{\mathcal{F}_s} + \cexp{N_t}{\mathcal{F}_s} = M_s + N_s
				\end{align}
				も成り立つ.以上より$M+N = (M_t + N_t)_{t \in I}$は$\mathrm{L}^p$-マルチンゲールである.
				次に写像$I \ni t \longmapsto M_t(\omega) + N_t(\omega) \in \R$の連続性を示す.
				$M,N$に対して或る$\mu$-零集合$E$が存在し,$\omega \notin E$について
				$t \longmapsto M_t(\omega)$と$t \longmapsto N_t(\omega)$が共に連続となるから
				$t \longmapsto M_t(\omega) + N_t(\omega)$も連続となる.以上で$M+N \in \mathcal{M}_{p,c}$が示された.
			
			\item[スカラ倍について]
				任意の$0 \leq s \leq t \leq T$に対し,条件付き期待値の線型性(性質$\tilde{\mathrm{C}}3$)により
				\begin{align}
					\cexp{\alpha M_t}{\mathcal{F}_s} = \alpha \cexp{M_t}{\mathcal{F}_s} = \alpha M_s
				\end{align}
				が成り立つ.定数倍しているだけであるから,$\alpha M$が
				$\mathrm{L}^p$-マルチンゲールであるためのその他の条件,及び$\mu$-a.s.にパスが連続であることも成り立ち,
				$\alpha M \in \mathcal{M}_{p,c}$となる.
		\end{description}
		\QED
	\end{prf}
	
	\begin{itembox}[l]{}
		\begin{lem}[$\mathcal{M}_{p,c}$における同値関係の導入]\mbox{}\\
			任意の$M,N \in \mathcal{M}_{p,c}\ (p \geq 1)$に対して,関係$R$を
			\begin{align}
				M\ R\ N \DEF \Set{\omega \in \Omega}{\sup{r \in (I \cap \Q) \cup \{ T \}}{\left|M_r(\omega) - N_r(\omega)\right| > 0}}\mbox{が$\mu$-零集合}
			\end{align}
			として定義すれば,関係$R$は同値関係となる.そして$M\ R\ N$となることと$\mu$-a.s.にパスが一致することは同値である.
			\label{lem:M_2c_hilbert}
		\end{lem}
	\end{itembox}
	
	\begin{prf}
		反射律と対称律は$R$の定義式より判然しているから推移律について確認する.$M,N$とは別に$U=(U_t)_{t \in I} \in \mathcal{M}_{p,c}$
		を取って$M\ R\ N$かつ$N\ R\ U$となっているとすれば,各$r \in (I \cap \Q) \cup \{ T \}$にて
		\begin{align}
			\left\{\ \left|M_r - U_r\right| > 0\ \right\}\ \subset\ 
			\left\{\ \left|M_r - N_r\right| > 0\ \right\} \cup \left\{\ \left|N_r - U_r\right| > 0\ \right\}
		\end{align}
		の関係が成り立っているから$M\ R\ U$が従う.
		\footnote{
			$\left\{\ \left|M_r - N_r\right| > 0\ \right\} = \Set{\omega \in \Omega}{\left|M_r(\omega) - N_r(\omega)\right| > 0}.$
		}
		後半の主張を示す.
		$M,N$に対し或る零集合$E$が存在して,$\omega \in \Omega \backslash E$に対し
		$I \ni t \longmapsto M_t(\omega)$と$I \ni t \longmapsto N_t(\omega)$は共に連続写像となっている.
		今$M\ R\ N$であるとする.
		\begin{align}
			F \coloneqq \Set{\omega \in \Omega}{\sup{r \in (I \cap \Q) \cup \{ T \}}{\left|M_r(\omega) - N_r(\omega)\right| > 0}}
		\end{align}
		とおけば$F^c \cap E^c$上で$M$と$N$のパスは完全に一致し,また$F \cup E$が零集合であるから$\mu$-a.s.にパスが一致しているということになる.
		逆に$\mu$-a.s.にパスが一致しているとすれば,或る零集合$G$が存在して$G^c$上でパスが一致している.
		\begin{align}
			G^c \subset F^c
		\end{align}
		の関係から$F \subset G$となり$F$が零集合となるから$M\ R\ N$が従う.
		\QED
	\end{prf}
	
	\begin{itembox}[l]{}
		\begin{lem}[$\mathcal{M}_{p,c}$の商空間の定義]
			補題\ref{lem:M_2c_hilbert}で導入した同値関係$R$による$\mathcal{M}_{p,c}\ (p \geq 1)$の商集合を$\mathfrak{M}_{p,c}$と表記する.
			$M \in \mathcal{M}_{p,c}$の関係$R$による同値類を$\overline{M}$と表記し,
			$\mathfrak{M}_{p,c}$において
			\begin{align}
				\overline{M} + \overline{N} \coloneqq \overline{M+N}, \quad \alpha \overline{M} \coloneqq \overline{\alpha M} \label{eq:mart_linear_arithmetic}
			\end{align}
			として演算を定義すれば,これは代表元の選び方に依らない(well-defined).そして(\refeq{eq:mart_linear_arithmetic})で定義した算法を加法とスカラ倍として
			$\mathfrak{M}_{p,c}$は$\R$上の線形空間となる.
		\end{lem}
	\end{itembox}
	
	\begin{prf}
		$M' \in \overline{M},\ N' \in \overline{N}$を任意に選べば,
		\begin{align}
			\left\{\ \left|M_r + N_r - M'_r - N'_r \right| > 0\ \right\} &\subset \left\{\ \left|M_r - M'_r \right| > 0\ \right\} \cup \left\{\ \left|N_r - N'_r \right| > 0\ \right\} \\
			\left\{\ \left|\alpha M_r - \alpha M'_r \right| > 0\ \right\} &= \left\{\ \left|M_r - M'_r \right| > 0\ \right\}
		\end{align}
		により$(M+N)\ R\ (M'+N'),\ (\alpha M)\ R\ (\alpha M')$が成り立ち
		\begin{align}
			\overline{M+N} = \overline{M'+N'}, \quad \overline{\alpha M} = \overline{\alpha M'}
		\end{align}
		が従う.
		\QED
	\end{prf}
	
	\begin{itembox}[l]{}
		\begin{lem}[$\mathfrak{M}_{2,c}$における内積の定義]
			写像\footnotemark
			$\inprod<\cdot,\cdot>:\mathfrak{M}_{2,c} \times \mathfrak{M}_{2,c} \rightarrow \R$
			を次で定義すれば,これは$\mathfrak{M}_{2,c}$において内積となる:
			\begin{align}
				\inprod<\overline{M},\overline{N}> \coloneqq \int_{\Omega} M_T(\omega)N_T(\omega)\ \mu(d\omega), \quad (\overline{M},\overline{N} \in \mathfrak{M}_{2,c}).
				\label{eq:M_2c_inner_product}
			\end{align}
			\label{lem:M_2c_hilbert_inner_product}
		\end{lem}
	\end{itembox}
			
	\footnotetext{
		実数値として確定することは,$M_T,N_T$が共に二乗可積分であることとH\Ddot{o}lderの不等式による.
	}
			
	\begin{prf}\mbox{}
		\begin{description}
			\item[well-definedであること]
				先ずは上の$\inprod<\cdot,\cdot>$の定義が代表元の取り方に依らないことを確認する.
				$M' \in \overline{M}$と$N' \in \overline{N}$に対して,
				同値関係の定義から$\mu$-a.s.に$M'_T = M_T,\ N'_T = N_T$であり
				\begin{align}
					\int_{\Omega} M_T(\omega)N_T(\omega)\ \mu(d\omega) = \int_{\Omega} M'_T(\omega)N'_T(\omega)\ \mu(d\omega)
				\end{align}
				が成り立つから,$\inprod<\overline{M},\overline{N}>$は一つの値に確定している.
				次に$\inprod<\cdot,\cdot>$が内積であることを証明する.
	
			\item[正値性]
				先ず任意の$\overline{M} \in \mathfrak{M}_{2,c}$に対して$\inprod<\overline{M},\overline{M}> = \Norm{M_T}{\mathscr{L}^2}^2 \geq 0$が成り立つ.
				次に$\inprod<\overline{M},\overline{M}> = 0 \quad \Leftrightarrow \quad \overline{M} = \overline{0}$
				が成り立つことを示す.$\inprod<\cdot,\cdot>$の定義により$\Leftarrow$は判然しているから,$\Rightarrow$について示す.
				$M$は$\mathrm{L}^2$-マルチンゲールであるから,Jensenの不等式より
				$(|M_t|)_{t \in I}$が$\mathrm{L}^2$-劣マルチンゲールとなる.Doobの不等式を適用すれば
				\begin{align}
					\int_{\Omega} \left( \sup{t \in I}{|M_t(\omega)|} \right)^2\ \mu(d\omega) \leq 4 \int_{\Omega} {M_T(\omega)}^2\ \mu(d\omega) = 0
				\end{align}
				が成り立ち,
				\begin{align}
					\left\{\ \sup{t \in I}{|M_t|} > 0\ \right\} = \left\{\ \sup{t \in I}{|M_t|^2} > 0\ \right\} = \left\{\ \left(\sup{t \in I}{|M_t(\omega)|}\right)^2 > 0\ \right\}
				\end{align}
				%\footnote{
				%	$\left\{\ \sup{t \in I}{|M_t|} > 0\ \right\}$は$\Set{\omega \in \Omega}{\sup{t \in I}{|M_t(\omega)|} > 0}$の略記(他も同様)であるが,
				%	ここの等号は次の関係が成立することにより正当化される:
				%	\begin{align}
				%		[\sup{t \in I}{|M_t(\omega)|}]^2 = \sup{t \in I}{[M_t(\omega)]^2},\ (\forall \omega \in \Omega).
				%	\end{align}
				%	もし$[\sup{t \in I}{|M_t(\omega)|}]^2 > \sup{t \in I}{[M_t(\omega)]^2} \eqqcolon \beta$
				%	とすると,$\sup{t \in I}{|M_t(\omega)|} > \beta^{1/2}$より或る$s \in I$について
				%	$|M_s(\omega)| > \beta^{1/2}$が成り立つから,$[M_s(\omega)]^2 > \beta$となり$\beta = \sup{t \in I}{[M_t(\omega)]^2}$に矛盾する.
				%	逆の場合,つまり$\alpha \coloneqq [\sup{t \in I}{|M_t(\omega)|}]^2 < \sup{t \in I}{[M_t(\omega)]^2}$
				%	が成り立っているとしても,或る$z \in I$が存在して$\alpha^{1/2} < |M_z(\omega)| \leq \sup{t \in I}{|M_t(\omega)|}$が成り立ち,
				%	$\alpha < [\sup{t \in I}{|M_t(\omega)|}]^2 = \alpha$となり矛盾ができた.
				%}
				であるから
				\begin{align}
					\mu\left( \sup{t \in I}{|M_t|} > 0 \right) = 0
				\end{align}
				が従う.よって$\overline{M} = \overline{0}$となる.
	
			\item[双線型性]
				双線型性は積分の線型性による.
			\end{description}
		\QED
	\end{prf}
		
	\begin{itembox}[l]{}
		\begin{prp}[$\mathfrak{M}_{2,c}$はHilbert空間である]
			$\mathfrak{M}_{2,c}$は補題\ref{lem:M_2c_hilbert_inner_product}で導入した$\inprod<\cdot,\cdot>$を内積としてHilbert空間となる.
		\end{prp}
	\end{itembox}
			
	\begin{prf}
			内積$\inprod<\cdot,\cdot>$により導入されるノルムを$\Norm{\cdot}{}$と表記する.
			$\overline{M^{(n)}} \in \mathfrak{M}_{2,c}\ (n=1,2,\cdots)$をCauchy列として取れば,
			各代表元$M^{(n)}$に対し或る$\mu$-零集合$E_n$が存在して,$\omega \in \Omega \backslash E_n$なら
			写像$I \ni t \longmapsto M^{(n)}_t(\omega) \in \R$が連続となる.後で連続関数列の一様収束を扱うからここで次の処理を行う:
			\begin{align}
				E \coloneqq \bigcup_{n=1}^{\infty} E_n
			\end{align}
			として,$M^{(n)} = (M^{(n)}_t)_{t \in I}$を零集合$E$上で修正した過程$(N^{(n)}_t)_{t \in I}$を
			\begin{align}
				N^{(n)}_t(\omega) \coloneqq
				\begin{cases}
					M^{(n)}_t(\omega) & (\omega \in \Omega \backslash E) \\
					0 & (\omega \in E)
				\end{cases}
				,\quad (\forall n = 1,2,\cdots,\ t \in I)
			\end{align}
		として定義すれば,$N^{(n)}$は$\Omega$全体でパスが連続,かつ$\mathrm{L}^2$-マルチンゲールであるから
		\footnote{
			$\mathrm{L}^2$-マルチンゲールとなることを証明する.
			パスの右連続性と左極限の存在は連続性により成り立つことである.適合性については,フィルトレーションの仮定より$E \in \mathcal{F}_0$であることに注意すれば,
			$N^{(n)}_t = M^{(n)}_t \defunc_{\Omega \backslash E}$
			であることと$M^{(n)}_t$が適合過程であることから$N^{(n)}_t$も可測$\mathcal{F}_t/\borel{\R}$となる.
			また各$t \in I$に対し$N^{(n)}_t$と$M^{(n)}_t$の関数類は一致するから,任意に$0 \leq s \leq t \leq T$を取って
			\begin{align}
				\cexp{N^{(n)}_t}{\mathcal{F}_s} = \cexp{M^{(n)}_t}{\mathcal{F}_s} = M^{(n)}_s = N^{(n)}_s
			\end{align}
			が成り立つ.$N^{(n)}_t = M^{(n)}_t \defunc_{\Omega \backslash E}$の二乗可積分性は$M^{(n)}_t$の二乗可積分性から従う.
		}
		$\mathcal{M}_{2,c}$の元となる.また零集合$E$を除いて$M^{(n)}$とパスが一致するから,
		$\overline{N^{(n)}} = \overline{M^{(n)}}\ (n=1,2,\cdots)$が成立し
		\begin{align}
			\Norm{\overline{M^{(n)}} - \overline{M^{(m)}}}{}^2 = \Norm{\overline{N^{(n)}} - \overline{N^{(m)}}}{}^2 
			=  \Norm{\overline{N^{(n)} - N^{(m)}}}{}^2
			= \int_{\Omega} \left| N^{(n)}_T(\omega) - N^{(m)}_T(\omega) \right|^2\ \mu(d\omega)
		\end{align}
		と表現できる.
		任意の$n,m \in N$の組に対し,$\mathcal{M}_{2,c}$が線形空間であるから
		$\left(\left|N^{(n)}_t - N^{(m)}_t\right|\right)_{t \in T}$は連続な$\mathrm{L}^2$-劣マルチンゲールとなり,
		Doobの不等式を適用して
		\begin{align}
			\lambda^2 \mu\left(\sup{t \in I}{|N^{(n)}_t - N^{(m)}_t| > \lambda}\right) \leq \int_{\Omega} \left| N^{(n)}_T(\omega) - N^{(m)}_T(\omega) \right|^2\ \mu(d\omega)
			= \Norm{\overline{M^{(n)}} - \overline{M^{(m)}}}{}^2 \quad (\forall \lambda > 0)
		\end{align}
		が成り立つ.この不等式と$\left(\overline{M^{(n)}}\right)_{n=1}^{\infty}$がCauchy列であることを併せれば,
		\begin{align}
			\Norm{\overline{M^{(n_k)}} - \overline{M^{(n_{k+1})}}}{} < 1/4^k, \quad (k = 1,2,\cdots) \label{eq:mart_hilbert_1}
		\end{align}
		となるように添数の部分列$(n_k)_{k=1}^{\infty}$を抜き出して
		\begin{align}
			\mu\left(\sup{t \in I}{|N^{(n_k)}_t - N^{(n_{k+1})}_t| > 1/2^k}\right) < 1/2^k, \quad (k=1,2,\cdots)
		\end{align}
		が成り立つようにできる.
		\begin{align}
			F \coloneqq \bigcup_{N=1}^{\infty} \bigcap_{k \geq N} \Set{\omega \in \Omega}{\sup{t \in I}{|N^{(n_k)}_t(\omega) - N^{(n_{k+1})}_t(\omega)|} \leq 1/2^k}
		\end{align}
		とおけば,Borel-Cantelliの補題により$F^c$は$\mu$-零集合であって,$\omega \in F$なら全ての$t \in I$について数列$\left(N^{(n_k)}_t(\omega)\right)_{k=1}^{\infty}$はCauchy列となる.
		実数の完備性から数列$\left(N^{(n_k)}_t(\omega)\right)_{k=1}^{\infty}\ (\omega \in F)$に極限$N^*_t(\omega)$が存在し,
		この収束は$t$に関して一様である
		\footnote{
			$\left| N^{(n_k)}_t(\omega) - N^*_t(\omega) \right| \leq \sum_{j=k}^{\infty} \left| N^{(n_j)}_t(\omega) - N^{(n_{j+1})}_t(\omega) \right|
			\leq \sum_{j=k}^{\infty} \sup{t \in I}{\left| N^{(n_j)}_t(\omega) - N^{(n_{j+1})}_t(\omega) \right|} < 1/2^k, \quad (\forall t \in T)$
			による.
		}から写像$t \longmapsto N^*_t(\omega)$は連続で,
		\begin{align}
			N_t(\omega) \coloneqq 
			\begin{cases}
				N^*_t(\omega) & (\omega \in F) \\
				0 & (\omega \in \Omega \backslash F)
			\end{cases}
		\end{align}
		として$N$を定義すればこれは$\mathcal{M}_{2,c}$の元となる.$N$は全てのパスが連続であるから$\mathrm{L}^2$-マルチンゲールとなっていることを示す.
		マルチンゲールの定義の(M.3)(M.4)はパスの連続性により従うことであるから,後は(M.1)と(M.2)を証明すればよい.
		\begin{description}
			\item[(M.1)適合性について]
				今任意に$t \in I$を取り固定する.
				$N^{n_k}_t$の定義域を$F$に制限した写像を$N^{F(k)}_t$と表記し
				\begin{align}
					\mathcal{F}^F_t \coloneqq \Set{F \cap B}{B \in \mathcal{F}_t}
				\end{align}
				とおけば,$N^{F(k)}_t$は可測$\mathcal{F}^F_t/\borel{\R}$となる.従って各点収束先の関数である$N^*_t$もまた可測$\mathcal{F}^F_t/\borel{\R}$となる.
				任意の$C \in \borel{\R}$に対して
				\begin{align}
					N^{-1}_t(C) = 
					\begin{cases}
						(\Omega \backslash F) \cup {N^*}^{-1}_t(C) & (0 \in C) \\
						{N^*}^{-1}_t(C) & (0 \notin C)
					\end{cases}
				\end{align}
				が成り立ち,フィルトレーションの仮定から$F \in \mathcal{F}_0$であり$\mathcal{F}^F_t \subset \mathcal{F}_t$が従うから,
				$N_t$は可測$\mathcal{F}_t/\borel{\R}$である.
			
			\item[(M.1)二乗可積分性について]
				任意に$t \in I$を取り固定する.$N^{(n_k)}_t\ (k=1,2,\cdots)$は二乗可積分関数$M^{(n_k)}_t$と零集合$E$を除いて一致し,
				$N_t$に概収束する.また添数列$(n_k)_{k=1}^{\infty}$の抜き出し方(\refeq{eq:mart_hilbert_1})とDoobの不等式より
				\begin{align}
					\Norm{\sup{t \in I}{\left|N^{(n_k)}_t - N^{(n_{k+1})}_t\right|}}{\mathscr{L}^2} \leq 2 \Norm{\left|N^{(n_k)}_T - N^{(n_{k+1})}_T\right|}{\mathscr{L}^2} < 2/4^k
				\end{align}
				が成り立つから$\Norm{N^{(n_k)}_t - N^{(n_{k+1})}_t}{\mathscr{L}^2} < 2/4^k \leq 1/2^k \ (k=1,2,\cdots)$を得る.
				特に$j \in \N$を固定すれば全ての$k > j$に対して$\Norm{N^{(n_j)}_t - N^{(n_k)}_t}{\mathscr{L}^2} < 1/2^j$となるから,Fatouの補題より
				\begin{align}
					\Norm{N^{(n_j)}_t - N_t}{\mathscr{L}^2}^2 = \int_{\Omega \backslash F} \liminf_{k \to \infty} \left| N^{(n_j)}_t(\omega) - N^{(n_k)}_t(\omega) \right|^2\ \mu(d\omega)
					< 1/4^j
					\label{eq:M_c2_hilbert_2}
				\end{align}
				が従い,Minkowskiの不等式より
				\begin{align}
					\Norm{N_t}{\mathscr{L}^2} \leq \Norm{N_t - N^{(n_j)}_t}{\mathscr{L}^2} + \Norm{N^{(n_j)}_t}{\mathscr{L}^2} < \infty
				\end{align}
				が成り立つ.
			
			\item[(M.2)について]
				各$t \in I,\ k \in \N$について$(M^{(n_k)}_t)_{t \in I}$が$\mathrm{L}^2$-マルチンゲールであるということを利用すればよい.
				任意の$0 \leq s \leq t \leq T$と$A \in \mathcal{F}_s$に対して
				\begin{align}
					\int_{A} \cexp{N^{(n_k)}_t}{\mathcal{F}_s}(\omega)\ \mu(d\omega) &= \int_{A} \cexp{M^{(n_k)}_t}{\mathcal{F}_s}(\omega)\ \mu(d\omega) \\
					&= \int_{A} M^{(n_k)}_s(\omega)\ \mu(d\omega) = \int_{A} N^{(n_k)}_s(\omega)\ \mu(d\omega)
				\end{align}
				が全ての$k = 1,2,\cdots$で成り立つから,H\Ddot{o}lderの不等式及び(\refeq{eq:M_c2_hilbert_2})より
				\begin{align}
					&\left| \int_{A} \cexp{N_t}{\mathcal{F}_s}(\omega)\ \mu(d\omega) - \int_{A} N_s(\omega)\ \mu(d\omega) \right| \\
					&\leq \left| \int_{A} \cexp{N_t}{\mathcal{F}_s}(\omega)\ \mu(d\omega) - \int_{A} \cexp{N^{(n_k)}_t}{\mathcal{F}_s}(\omega)\ \mu(d\omega) \right| \\
						&\qquad+ \left| \int_{A} N^{(n_k)}_s(\omega)\ \mu(d\omega) - \int_{A} N_s(\omega)\ \mu(d\omega) \right| \\
					&= \left| \int_{A} N_t(\omega) - N^{(n_k)}_t(\omega)\ \mu(d\omega) \right|
						+ \left| \int_{A} N^{(n_k)}_s(\omega) - N_s(\omega)\ \mu(d\omega) \right| \\
					&\leq \int_{A} \left| N_t(\omega) - N^{(n_k)}_t(\omega) \right|\ \mu(d\omega)
						+ \int_{A} \left| N^{(n_k)}_s(\omega) - N_s(\omega) \right|\ \mu(d\omega) \\
					&\leq \Norm{N_t - N^{(n_k)}_t}{\mathscr{L}^2} + \Norm{N^{(n_k)}_s - N_s}{\mathscr{L}^2} \\
					&\leq 1/2^{k-1}
				\end{align}
				が全ての$k = 1,2,\cdots$で成り立つ.$k$の任意性から
				\begin{align}
					\int_{A} \cexp{N_t}{\mathcal{F}_s}(\omega)\ \mu(d\omega)
					= \int_{A} N_s(\omega)\ \mu(d\omega)
				\end{align}
				が従い,$\cexp{N_t}{\mathcal{F}_s} = N_s \quad \mbox{in $\Lp{2}{\mathcal{F},\mu}$}$となる.
		\end{description}
	
		最後に,$N$の$\mathfrak{M}_{2,c}$における同値類$\overline{N}$がCauchy列$\left(\overline{M^{(n)}}\right)_{n=1}^{\infty}$の極限であるということを明示して証明を完全に終える.
		部分列$\left(\overline{M^{(n_k)}}\right)_{k=1}^{\infty}$に対して,(\refeq{eq:M_c2_hilbert_2})より
		\begin{align}
				\Norm{\overline{N} - \overline{M^{(n_k)}}}{} 
				= \Norm{\overline{N} - \overline{N^{(n_k)}}}{}
				= \Norm{N_T - N^{(n_k)}_T}{\mathscr{L}^2} \longrightarrow 0 \quad (k \longrightarrow \infty)
		\end{align}
		が成り立つ.部分列が収束することはCauchy列が収束することになるから$\Norm{\overline{N} - \overline{M^{(n)}}}{} \longrightarrow 0$が従い,
		$\mathfrak{M}_{2,c}$がHilbert空間であることが証明された.
		\QED
	\end{prf}
	
	\begin{itembox}[l]{}
		\begin{prp}
			任意の$p \geq 1$に対し,$M \in \mathcal{M}_{p,c}$が
			$\Norm{M_0}{\mathscr{L}^\infty} < \infty$を満たすなら$M \in \mathcal{M}_{c,loc}$が成り立つ.
			\label{prp:M_pc_M_cloc}
		\end{prp}
	\end{itembox}
	この証明には次の補題を使う.
	
	\begin{itembox}[l]{}
		\begin{lem}[各点で右連続であり左極限を持つ関数は閉区間上で有界]\mbox{}\\
			$(E,\rho)$を距離空間,$J = [a,b] \subset \R$とする.$f:J \rightarrow E$が各点$x \in J$で
			右連続且つ左極限を持つ\footnotemark
			なら$f$は$J$上で有界である.
			\label{lem:rcll_bounded}
		\end{lem}
	\end{itembox}
	\footnotetext{
		左端点では左極限を考えず,右端点では右連続性を考えない.
	}
	\begin{prf}[補題]
		任意に$\epsilon > 0$を取り固定する.$f$は各点$x \in [a,b)$で右連続であるから,$0 < \delta_x < b-x$を
		$0 < \forall h < \delta_x$が$\rho(f(x), f(x+h)) < \epsilon$を満たすように取り,
		\begin{align}
			V_x \coloneqq [x,x+\delta_x) \quad (\forall x \in [a,b))
		\end{align}
		とおく.また$f$は各点$x \in (a,b]$で左極限も持つから,左極限を$f(x-)$と表して
		$0 < \gamma_x < x-a$を$0 < \forall h < \gamma_x$が$\rho(f(x-),f(x-h)) < \epsilon$を満たすように取り,
		\begin{align}
			U_x \coloneqq (x-\gamma_x,x] \quad (\forall x \in (a,b])
		\end{align}
		とおく.特に$U_a \coloneqq (-\infty,a],\ V_b \coloneqq [b,\infty)$とおけば
		\begin{align}
			J \subset \bigcup_{x \in J}U_x \cup V_x
		\end{align}
		が成り立つが,$J$は$\R$のコンパクト部分集合であるから,このうち有限個を選び
		\begin{align}
			J = \bigcup_{i=1}^n \left( U_{x_i} \cup V_{x_i}\right) \cap J
		\end{align}
		とできる.$U_{x_i} \cap J,\ V_{x_i} \cap J$での$f$の挙動の振れ幅は$2\epsilon$で抑えられるから
		$J$全体での挙動の振れ幅は$2n\epsilon$より小さい
		\footnote{
			$x_1 < x_2 < \cdots < x_n$と仮定し,区間$U_{x_i} \cup V_{x_i}$と$U_{x_{i+1}} \cup V_{x_{i+1}}$の共通点を一つ取り$z_i$と表す.
			$\rho(f(x),f(y))\ (x,y \in J)$の上界を知りたいから$x \in U_{x_1} \cup V_{x_1},\ y \in U_{x_n} \cup V_{x_n}$の場合を調べればよい.このとき
			\begin{align}
				\rho(f(x),f(y)) &\leq \rho(f(x),f(x_1)) + \rho(f(x_1),f(x_2)) + \cdots + \rho(f(x_{n-1}),f(x_n)) + \rho(f(x_n),f(y)) \\
				&\leq \rho(f(x),f(x_1)) + \rho(f(x_1),f(z_1)) + \rho(f(z_1),f(x_2)) + \cdots + \rho(f(z_{n-1}),f(x_n)) + \rho(f(x_n),f(y)) \\
				& < 2n\epsilon
			\end{align}
			が成り立つ.
		}.
		ゆえに有界である.
		\QED
	\end{prf}
	
	\begin{prf}[命題\ref{prp:M_pc_M_cloc}]
		$M \in \mathcal{M}_{p,c}$が
		\begin{align}
			K \coloneqq \Norm{M_0}{\mathscr{L}^\infty} < \infty
		\end{align}
		を満たすと仮定する.全ての$\omega \in \Omega$に対し写像$I \ni t \longmapsto M_t(\omega)$
		は右連続且つ左極限を持つから,定理\ref{thm:closed_set_stopping_time}より
		\begin{align}
			\tau_j(\omega) \coloneqq \inf{}{\Set{t \in I}{|M_t(\omega)| \geq j}} \wedge T \quad (\forall \omega \in \Omega,\ j=1,2,\cdots)
		\end{align}
		として$\tau_j$は停止時刻となり,かつ$t \longmapsto M_t(\omega)$が連続となる$\omega$に対しては
		\begin{align}
			\sup{t \in I}{\left| M_{t \wedge \tau_j(\omega)}(\omega) \right|} \leq j \vee K
			\label{eq:M_pc_M_cloc}
		\end{align}
		が成り立つ.また$t \longmapsto M_t(\omega)$の右連続性から$\tau_j \leq \tau_{j+1}\ (j=1,2,\cdots)$となり,
		更に補題\ref{lem:rcll_bounded}より$\sup{t \in I}{|M_t(\omega)|} < \infty\ (\forall \omega \in \Omega)$も成り立つから
		$j_\omega > \sup{t \in I}{|M_t(\omega)|}$となるような$j_\omega$に対し$\tau_j(\omega) = T\ (\forall j \geq j_\omega)$を満たす.従って$(\tau_j)_{j=1}^{\infty} \in \mathcal{T}$である.
		後は$M^j$が$\mathcal{M}_{b,c}$に属することを示せばよい.先ずDoobの不等式(定理\ref{thm:Doob_inequality_2})より
		$\sup{t \in I}{|M_t|}$が$p$乗可積分となることから$M_t^j = M_{t \wedge \tau_j}$の可積分性が従う.
		また式(\refeq{eq:M_pc_M_cloc})より
		\begin{align}
			\Norm{M_t^j}{\mathscr{L}^\infty} \leq j \vee K \quad (\forall t \in I)
		\end{align}
		が成り立ち,$t \longmapsto M_t(\omega)$が連続となる$\omega$に対しては$I \ni t \longmapsto M_t^j(\omega)$もまた連続,そして
		任意抽出定理(定理\ref{thm:optional_sampling_theorem_2})より
		\begin{align}
			\cexp{M_t^j}{\mathcal{F}_s} = M_{t \wedge \tau_j \wedge s} = M_s^j \quad (\forall 0 \leq s < t \leq T)
		\end{align}
		を得る.以上より$M^j \in \mathcal{M}_{b,c}\ (j=1,2,\cdots)$,すなわち$M \in \mathcal{M}_{c,loc}$である.
		\QED
	\end{prf}
	
	\begin{itembox}[l]{}
		\begin{lem}
			$X \in \mathcal{M}_{2,c}$と停止時刻$\tau \geq \sigma$に対し,$\Norm{M_\sigma}{\mathscr{L}^\infty} < \infty$ならば次が成り立つ:
			\begin{description}
				\item[(1)] $\Exp{(X_{\tau} - X_{\sigma})^2} = \Exp{X_{\tau}^2 - X_{\sigma}^2}$,
				\item[(2)] $\cexp{(X_{\tau} - X_{\sigma})^2}{\mathcal{F}_\sigma} = \cexp{X_{\tau}^2 - X_{\sigma}^2}{\mathcal{F}_\sigma}$.
			\end{description}
			\label{lem:stopping_time_telescopic_sum}
		\end{lem}
	\end{itembox}
	
	\begin{prf}\mbox{}
		以下の式中では関数ではなく関数類を扱う.
		\begin{description}
			\item[(1)] 
				$\mathcal{G} = \{\Omega,\emptyset\}$とおく.条件付き期待値の性質(定理\ref{thm:conditional_exp_expansion})を使えば
				\begin{align}
					\Exp{(X_{\tau} - X_{\sigma})^2} &= \cexp{(X_{\tau} - X_{\sigma})^2}{\mathcal{G}} \\
					&= \cexp{X_{\tau}^2 + X_{\sigma}^2}{\mathcal{G}} - 2\cexp{X_{\tau}X_{\sigma}}{\mathcal{G}} \\
					&= \cexp{X_{\tau}^2 + X_{\sigma}^2}{\mathcal{G}} - 2\cexp{\cexp{X_{\tau}X_{\sigma}}{\mathcal{F}_\sigma}}{\mathcal{G}} \\
					&= \cexp{X_{\tau}^2 + X_{\sigma}^2}{\mathcal{G}} - 2\cexp{X_{\sigma}\cexp{X_{\tau}}{\mathcal{F}_\sigma}}{\mathcal{G}} 
						&& (\because\mbox{\scriptsize 定理\ref{thm:measurability_of_stopping_time}}) \\
					&= \cexp{X_{\tau}^2 + X_{\sigma}^2}{\mathcal{G}} - 2\cexp{X_{\sigma}^2}{\mathcal{G}} 
						&& (\because\mbox{\scriptsize 定理\ref{thm:optional_sampling_theorem_2}}) \\
					&= \cexp{X_{\tau}^2 - X_{\sigma}^2}{\mathcal{G}} \\
					&= \Exp{X_{\tau}^2 - X_{\sigma}^2}
				\end{align}
				が成り立つ.
			
			\item[(2)]
				(1)と同様に
				\begin{align}
					\cexp{(X_{\tau} - X_{\sigma})^2}{\mathcal{F}_\sigma}
					&= \cexp{X_{\tau}^2 + X_{\sigma}^2}{\mathcal{F}_\sigma} - 2\cexp{X_{\tau}X_{\sigma}}{\mathcal{F}_\sigma} \\
					&= \cexp{X_{\tau}^2 + X_{\sigma}^2}{\mathcal{F}_\sigma} - 2X_{\sigma}^2 \\
					&= \cexp{X_{\tau}^2 + X_{\sigma}^2}{\mathcal{F}_\sigma} - 2\cexp{X_{\sigma}^2}{\mathcal{F}_\sigma} \\
					&= \cexp{X_{\tau}^2 - X_{\sigma}^2}{\mathcal{F}_\sigma}
				\end{align}
				が成り立つ.
		\end{description}
		\QED
	\end{prf}
	
	\begin{itembox}[l]{}
		\begin{prp}[有界変動かつ連続な二乗可積分マルチンゲールのパスは定数となる]\mbox{}\\
			$A \in \mathcal{A} \cap \mathcal{M}_{2,c}$に対し,$\Norm{A_0}{\mathscr{L}^\infty} < \infty$ならば$A_t = A_0\ (\forall t \in I)\ $$\mu$-a.s.が成り立つ.
			\label{prp:bounded_continuous_M_2c_path}
		\end{prp}
	\end{itembox}
	
	\begin{prf}
		$A \in \mathcal{A}$であるから,$A$に対し或る$A^{(1)},A^{(2)} \in \mathcal{A}^+$が存在して
		\begin{align}
			A = A^{(1)} - A^{(2)}
		\end{align}
		と表現できる.或る$\mu$-零集合$E$を取れば,全ての$\omega \in \Omega \backslash E$に対し写像$I \ni t \longmapsto A_t^{(1)}(\omega)$と
		$I \ni t \longmapsto A_t^{(2)}(\omega)$が連続且つ単調非減少となるようにできるから,
		\begin{align}
			\tau_m(\omega) \coloneqq
			\begin{cases}
				0 & (\omega \in E) \\
				\inf{}{\Set{t \in I}{\left( A_t^{(1)}(\omega) - A_0^{(1)} \right) \vee \left( A_t^{(2)}(\omega) - A_0^{(2)} \right) \geq m}} & (\omega \in \Omega \backslash E)
			\end{cases}
			\quad (m=1,2,\cdots)
		\end{align}
		と定義すれば,定理\ref{thm:closed_set_stopping_time}よりこれは停止時刻となる.また連続性から$\tau_m \leq \tau_{m+1}$となり
		更に$\lim_{m \to \infty}\tau_m(\omega) = T\ (\forall \omega \in \Omega \backslash E)も成り立つ.
		$今$t \in I,\ n,m \in \N$を任意に取って固定する.
		\begin{align}
			\sigma_j^n \coloneqq \tau_m \wedge \frac{tj}{2^n} \quad (j = 0,1,\cdots, 2^n)
		\end{align}
		とおけば,補題\ref{lem:stopping_time_telescopic_sum}により
		\footnote{
			補題の有界性の仮定を満たしていることを確認する.任意の$j \in \N$番目の$\sigma_j^n$を取る.
			$\omega \in \Omega \backslash E$に対し写像$t \longmapsto A_t(\omega)$は連続であるから
			\begin{align}
				\left| A_{\sigma_j^n(\omega)}(\omega) - A_0(\omega) \right| 
				\leq \left| A^{(1)}_{\tau_m(\omega)\wedge \frac{tj}{2^n}}(\omega) - A^{(1)}_0(\omega) \right| + \left| A^{(2)}_{\tau_m(\omega)\wedge \frac{tj}{2^n}}(\omega) - A^{(2)}_0(\omega) \right|
				\leq 2m
			\end{align}
			が成り立つ.また補題\ref{lem:holder_inequality}より或る零集合$E'$を除いて$|A_0| \leq \Norm{A_0}{\mathscr{L}^\infty}$が成り立っているから,
			\begin{align}
				\left| A_{\sigma_j^n(\omega)}(\omega) \right| \leq 2m + \Norm{A_0}{\mathscr{L}^\infty} \quad (\forall \omega \in \Omega \backslash (E \cup E'))
			\end{align}
			となり$\Norm{A_{\sigma_j^n}}{\mathscr{L}^\infty} \leq 2m + \Norm{A_0}{\mathscr{L}^\infty}$であると判る.
		}
		\begin{align}
			\Exp{\sum_{j=0}^{2^n-1} \left( A_{\sigma_{j+1}^n} - A_{\sigma_j^n} \right)^2}
			= \sum_{j=0}^{2^n-1} \Exp{A_{\sigma_{j+1}^n}^2 - A_{\sigma_j^n}^2}
			= \Exp{A_{\tau_m \wedge t}^2 - A_{0}^2}
			= \Exp{\left( A_{\tau_m \wedge t} - A_{0} \right)^2}
		\end{align}
		が成り立つ.左辺の中の式は
		\begin{align}
			\sum_{j=0}^{2^n-1} \left( A_{\sigma_{j+1}^n} - A_{\sigma_j^n} \right)^2
			\leq \sup{j}{\left| A_{\sigma_{j+1}^n} - A_{\sigma_j^n} \right|} \sum_{j=0}^{2^n-1} \left| A_{\sigma_{j+1}^n} - A_{\sigma_j^n} \right|
		\end{align}
		となり,全ての$\omega \in \Omega \backslash E$に対して$I \ni t \longmapsto A_t(\omega)$は(一様)連続だから
		\begin{align}
			\sup{j}{\left| A_{\sigma_{j+1}^n}(\omega) - A_{\sigma_j^n}(\omega) \right|} \longrightarrow 0 \quad (n \longrightarrow \infty).
		\end{align}
		また定理\ref{thm:closed_set_stopping_time}より全ての$\omega \in \Omega \backslash E$と$t \in I$に対して
		\begin{align}
			\left( A^{(1)}_{\tau_m \wedge t}(\omega) - A^{(1)}_0(\omega) \right) \vee \left( A^{(2)}_{\tau_m \wedge t}(\omega) - A^{(2)}_0(\omega) \right) \leq m
		\end{align}
		が成り立つから
		\begin{align}
			\sum_{j=0}^{2^n-1} \left| A_{\sigma_{j+1}^n}(\omega) - A_{\sigma_j^n}(\omega) \right|
			&\leq \sum_{j=0}^{2^n-1} \left( A^{(1)}_{\sigma_{j+1}^n}(\omega) - A^{(1)}_{\sigma_j^n}(\omega) + A^{(2)}_{\sigma_{j+1}^n}(\omega) - A^{(2)}_{\sigma_j^n}(\omega) \right) \\
			&= \left( A^{(1)}_{\tau_m \wedge t}(\omega) - A^{(1)}_0(\omega) \right) + \left( A^{(2)}_{\tau_m \wedge t}(\omega) - A^{(2)}_0(\omega) \right) \leq 2m
		\end{align}
		となり,Lebesgueの収束定理により
		\begin{align}
			\int_\Omega \sum_{j=0}^{2^n-1} \left( A_{\sigma_{j+1}^n(\omega)}(\omega) - A_{\sigma_j^n(\omega)}(\omega) \right)^2\ \mu(d\omega) \longrightarrow 0 \quad (n \longrightarrow \infty)
		\end{align}
		を得る.ゆえに
		\begin{align}
			\int_\Omega \left( A_{\tau_m(\omega) \wedge t}(\omega) - A_{0}(\omega) \right)^2\ \mu(d\omega) = 0 \quad (m=1,2,\cdots)
		\end{align}
		が成り立ち,更にDoobの不等式より$|A_{\tau_m \wedge t} - A_{0}| \leq \sup{t \in I}{|A_t - A_{0}|} \in \mathscr{L}^2$であるから,
		再びLebesgueの収束定理を適用して
		\begin{align}
			\int_\Omega \left( A_{\tau_m(\omega) \wedge t}(\omega) - A_{0}(\omega) \right)^2\ \mu(d\omega)
			\longrightarrow \int_\Omega \left( A_t(\omega) - A_{0}(\omega) \right)^2\ \mu(d\omega) \quad (n \longrightarrow \infty)
		\end{align}
		を得る.$t \in I$は任意に取っていたからつまり
		\begin{align}
			A_t = A_0 \quad \mbox{$\mu$-a.s.} \quad (\forall t \in I)
		\end{align}
		が示されたが,実際$\omega \in \Omega \backslash E$に対しパスは連続であるから
		\begin{align}
			\Set{\omega \in \Omega \backslash E}{A_t(\omega) = A_0(\omega)\ (\forall t \in I)}
			= \bigcap_{r \in I \cap \Q} \Set{\omega \in \Omega \backslash E}{A_r(\omega) = A_0(\omega)}
		\end{align}
		と表せる.
		\begin{align}
			\Set{\omega \in \Omega \backslash E}{A_t(\omega) \neq A_0(\omega)\ (\exists t \in I)}
			\subset E + \bigcup_{r \in I \cap \Q} \Set{\omega \in \Omega \backslash E}{A_r(\omega) \neq A_0(\omega)}
		\end{align}
		の右辺は零集合であるから$A_t = A_0\ (\forall t \in I)\ $$\mu$-a.s.が成り立つ.
		\QED
	\end{prf}
	
	\begin{itembox}[l]{}
		\begin{lem}[二次変分補題]
			任意に$n \in \N$と$M \in \mathcal{M}_{b,c}$を取る.$\tau_j^n = jT/2^n\ (j=0,1,\cdots,2^n)$に対し
			\begin{align}
				Q_t^n \coloneqq \sum_{j=0}^{2^n-1} \left( M_{t \wedge \tau_{j+1}^n} - M_{t \wedge \tau_j^n} \right)^2 \quad (\forall t \in I)
				\label{eq:lem_quadratic_variation_0}
			\end{align}
			とおけば$M^2 - Q^n \in \mathcal{M}_{b,c}$となり,さらに次が成り立つ:
			\begin{align}
				\Norm{M_T - M_0 - Q_T^n}{\mathscr{L}^2} \leq 2 \sup{t \in I}{\Norm{M_t}{\mathscr{L}^\infty}} \Norm{M_T - M_0}{\mathscr{L}^2}.
			\end{align}
			\label{lem:quadratic_variation}
		\end{lem}
	\end{itembox}
	
	\begin{prf}
		先ず$Q^n$は$\mathcal{F}_t$-適合である.これは任意の停止時刻$\tau$に対し$M_{t \wedge \tau}$が可測$\mathcal{F}_t/\borel{\R}$であることによる.
		今任意に$0 \leq s < t \leq T$を取り固定する.$\tau_k^n \leq s < \tau_{k+1}^n$となる$k$を選べば,
		補題\ref{lem:stopping_time_telescopic_sum}と任意抽出定理\ref{thm:optional_sampling_theorem_2}を使って次のように式変形できる:
		\begin{align}
			\cexp{Q_t^n - Q_s^n}{\mathcal{F}_s} 
			&= \cexp{\sum_{j=0}^{2^n-1}\left( M_{t\wedge\tau_{j+1}^n} - M_{t\wedge\tau_j^n} \right)^2 - \sum_{j=0}^{2^n-1}\left( M_{s\wedge\tau_{j+1}^n} - M_{s\wedge\tau_j^n} \right)^2}{\mathcal{F}_s} \\
			&= \cexp{\sum_{j=k}^{2^n-1}\left\{ \left( M_{t\wedge\tau_{j+1}^n} - M_{t\wedge\tau_j^n} \right)^2 - \left( M_{s\wedge\tau_{j+1}^n} - M_{s\wedge\tau_j^n} \right)^2 \right\}}{\mathcal{F}_s} \\
			&= \sum_{j=k+1}^{2^n-1} \cexp{\left( M_{t\wedge\tau_{j+1}^n} - M_{t\wedge\tau_j^n} \right)^2}{\mathcal{F}_s}
				+ \cexp{\left( M_{t\wedge\tau_{k+1}^n} - M_{t\wedge\tau_k^n} \right)^2}{\mathcal{F}_s} - \cexp{\left( M_s - M_{\tau_k^n} \right)^2}{\mathcal{F}_s} \\
			&= \sum_{j=k+1}^{2^n-1} \cexp{ M_{t\wedge\tau_{j+1}^n}^2 - M_{t\wedge\tau_j^n}^2}{\mathcal{F}_s}
				+ \cexp{\left( M_{t\wedge\tau_{k+1}^n} - M_{\tau_k^n} \right)^2}{\mathcal{F}_s} - \left( M_s - M_{\tau_k^n} \right)^2 \\
			&= \cexp{ M_t^2 - M_{t\wedge\tau_{k+1}^n}^2}{\mathcal{F}_s} + \cexp{\left( M_{t\wedge\tau_{k+1}^n} - M_{\tau_k^n} \right)^2}{\mathcal{F}_s} - \left( M_s - M_{\tau_k^n} \right)^2 \\
			&= \cexp{M_t^2}{\mathcal{F}_s} - 2\cexp{M_{t\wedge\tau_{k+1}^n}M_{\tau_k^n}}{\mathcal{F}_s} + \cexp{M_{\tau_k^n}^2}{\mathcal{F}_s} - M_s^2 + 2M_sM_{\tau_k^n} - M_{\tau_k^n}^2 \\
			&= \cexp{M_t^2}{\mathcal{F}_s} - 2M_{\tau_k^n}\cexp{M_{t\wedge\tau_{k+1}^n}}{\mathcal{F}_s} + M_{\tau_k^n}^2 - M_s^2 + 2M_sM_{\tau_k^n} - M_{\tau_k^n}^2 \\
			&= \cexp{M_t^2}{\mathcal{F}_s} - M_s^2.
		\end{align}
		従って次を得た:
		\begin{align}
			\cexp{M_t^2 - Q_t^n}{\mathcal{F}_s} = M_s^2 - Q_s^n, \quad (\forall 0 \leq s < t \leq T).
			\label{eq:lem_quadratic_variation_1}
		\end{align}
		ここで
		\begin{align}
			N \coloneqq M^2 - Q^n
		\end{align}
		とおけば$N \in \mathcal{M}_{b,c}$であり
		\footnote{
			$M \in \mathcal{M}_{b,c}$より全ての$\omega \in \Omega$において写像$t \longmapsto M_t(\omega)$は各点で右連続かつ左極限を持つ.
			$Q^n$についてもGauss記号を用いて$Q_t^n = \sum_{j=0}^{[2^nt]/T} \left( M_t^n - M_{\tau_j^n} \right)^2$
			と表せば,$t \longmapsto Q_t^n(\omega)\ (\forall \omega \in \Omega)$が各点で右連続かつ左極限を持つことが明確になる.
			よって全ての$\omega \in \Omega$において$t \longmapsto N_t(\omega)$は各点で右連続かつ左極限を持つ.
			また同じ理由で$t \longmapsto M_t(\omega)$が連続となる点で$t \longmapsto N_t(\omega)$も連続となるから
			つまり$\mu$-a.s.に$t \longmapsto N_t$は連続.
			一様有界性については,$\sup{t \in I}{\Norm{M_t}{\mathscr{L}^\infty}} < \infty$であるから,任意の$t \in I$に対し
			或る零集合$E_t$が存在して$\omega \notin E_t$なら$|M_t(\omega)| \leq \sup{t \in I}{\Norm{M_t}{\mathscr{L}^\infty}}$が成り立つ.
			同様に$Q_t^n$についても$\omega \notin E_t \cup \bigcup_{j=0}^{[2^nt]/T}E_{\tau_j^n}$なら
			\begin{align}
				\left| Q_t^n(\omega) \right| \leq \sum_{j=0}^{[2^nt]/T} \left( 2\sup{t \in I}{\Norm{M_t}{\mathscr{L}^\infty}} \right)^2 \leq 2^{n+1} \sup{t \in I}{\Norm{M_t}{\mathscr{L}^\infty}^2}.
			\end{align}
			ゆえに
			\begin{align}
				\left| N_t(\omega) \right| \leq \left|{M_t(\omega)}^2\right| + \left|Q_t^n(\omega)\right| \leq \left( 2^{n+1}+1 \right) \sup{t \in I}{\Norm{M_t}{\mathscr{L}^\infty}^2}
				,\quad \left( \forall \omega \notin E_t \cup \cup_{j=0}^{[2^nt]/T}E_{\tau_j^n} \right).
			\end{align}
			この右辺は
			$t$に依らないから
			\begin{align}
				\sup{t \in I}{\Norm{N_t}{\mathscr{L}^\infty}} \leq \left( 2^{n+1}+1 \right) \sup{t \in I}{\Norm{M_t}{\mathscr{L}^\infty}^2}
			\end{align}
			を得る.以上の結果と(\refeq{eq:lem_quadratic_variation_1})を併せて$N \in \mathcal{M}_{b,c}$となる.
		},
		\begin{align}
			\Exp{(N_T - N_0)^2} = \Exp{N_T^2 - N_0^2} &= \Exp{\sum_{j=0}^{2^n-1}\left( N_{\tau_{j+1}^n}^2 - N_{\tau_j^n}^2 \right)} \\
			&= \sum_{j=0}^{2^n-1}\Exp{\left( N_{\tau_{j+1}^n} - N_{\tau_j^n} \right)^2} \\
			&= \sum_{j=0}^{2^n-1}\Exp{\left\{ M_{\tau_{j+1}^n}^2 - M_{\tau_j^n}^2 - \left( Q_{\tau_{j+1}^n}^n - Q_{\tau_j^n}^n \right) \right\}^2} \\
			&= \sum_{j=0}^{2^n-1}\Exp{\left\{ M_{\tau_{j+1}^n}^2 - M_{\tau_j^n}^2 - \left( M_{\tau_{j+1}^n} - M_{\tau_j^n} \right)^2 \right\}^2} \\
			&= \sum_{j=0}^{2^n-1}\Exp{\left\{ -2M_{\tau_j^n} \left( M_{\tau_{j+1}^n} - M_{\tau_j^n} \right) \right\}^2} \\
			&= 4 \Exp{ \sum_{j=0}^{2^n-1} M_{\tau_j^n}^2 \left( M_{\tau_{j+1}^n} - M_{\tau_j^n} \right)^2 } \\
			&\leq 4 \sup{t \in I}{\Norm{M_t}{\mathscr{L}^\infty}^2} \Exp{\sum_{j=0}^{2^n-1} \left( M_{\tau_{j+1}^n} - M_{\tau_j^n} \right)^2 } \\
			&= 4 \sup{t \in I}{\Norm{M_t}{\mathscr{L}^\infty}^2} \Exp{M_T^2 - M_0^2}
		\end{align}
		が成り立つ.これより
		\begin{align}
			\Norm{M_T - M_0 - Q_T^n}{\mathscr{L}^2} \leq 2 \sup{t \in I}{\Norm{M_t}{\mathscr{L}^\infty}} \Norm{M_T - M_0}{\mathscr{L}^2}.
		\end{align}
		\QED
	\end{prf}
	
	\begin{itembox}[l]{}
		\begin{thm}[二次変分の存在]\mbox{}\\
			任意の$M \in \mathcal{M}_{c,loc}$に対し或る$A \in \mathcal{A}^+$が一意的に存在して\footnotemark
			次を満たす:
			\begin{itemize}
				\item $A_0 = 0\quad \mbox{$\mu$-a.s.}$
				\item $M^2 - A \in \mathcal{M}_{c,loc}.$
			\end{itemize}
		\end{thm}
	\end{itembox}
	
	\footnotetext{
		$A,A' \in \mathcal{A}^+$が主張の二条件を満たすときは$\mu$-a.s.にパスが一致する,という意味で存在が一意的である.
	}
	
	\begin{prf}
		証明は二段階ある.まず$M \in \mathcal{M}_{b,c}$に対し$A$の存在を証明し,その結果を$\mathcal{M}_{c,loc}$に拡張する.
		\begin{description}
			\item[第一段]
				補題\ref{lem:quadratic_variation}に従って$Q^n\ (n=1,2,\cdots)$を構成し
				$N^n \coloneqq M^2 - Q^n \in \mathcal{M}_{b,c}$とおけば
				\begin{align}
					\Norm{N_T^n - N_0^n}{\mathscr{L}^2} \leq 2 \sup{t \in I}{\Norm{M_t}{\mathscr{L}^\infty}} \Norm{M_T - M_0}{\mathscr{L}^2} \quad (n=1,2,\cdots)
				\end{align}
				を満たす.$Q^n$の構成方法(\refeq{eq:lem_quadratic_variation_0})より$N_0^n(\omega) = M^2_0(\omega)\ (\forall \omega \in \Omega,\ n=1,2,\cdots)$となるから
				\begin{align}
					\Norm{N_T^n}{\mathscr{L}^2} 
					\leq 2 \sup{t \in I}{\Norm{M_t}{\mathscr{L}^\infty}} \Norm{M_T - M_0}{\mathscr{L}^2} + \Norm{M^2_0}{\mathscr{L}^2} \quad (n=1,2,\cdots)
				\end{align}
				が成り立ち,従って$N^n$の同値類
				\footnote{
					補題\ref{lem:M_2c_hilbert}で導入した同値関係$R$による同値類.
				}
				$\overline{N^n}$の列$(\overline{N^n})_{n=1}^{\infty}$はHilbert空間$\mathfrak{M}_{2,c}$において有界列となる.
				Kolmosの補題より$\overline{N^n}$の線型結合の列$\hat{\overline{N^n}}\ (n=1,2,\cdots)$が存在して$\mathfrak{M}_{2,c}$においてCauchy列となるから,
				その極限を$\overline{N} \in \mathfrak{M}_{2,c}$と表す.線型結合を
				\begin{align}
					\hat{\overline{N^n}} = \sum_{j=0}^{\infty} c^n_j \overline{N^{n+j}}, \quad
					\hat{N}^n \coloneqq \sum_{j=0}^{\infty} c^n_j N^{n+j}, \quad
					\hat{Q}^n \coloneqq \sum_{j=0}^{\infty} c^n_j Q^{n+j}
				\end{align}
				と表せば$\hat{N}^n = M^2 - \hat{Q}^n$となり
				\footnote{
					任意の$n \in \N$に対して$(c^n_j)_{j=0}^{\infty}$は
					$\sum_{j=0}^{\infty} c^n_j = 1$を満たし,且つ$\neq 0$であるのは有限個である.
				}
				,任意に$N \in \overline{N}$を取り
				\begin{align}
					A \coloneqq M^2 - N \label{eq:thm_quadratic_variation_0}
				\end{align}
				とおけば,Doobの不等式(定理\ref{thm:Doob_inequality_2})により
				\begin{align}
					\Norm{\sup{t \in I}{\left| \hat{Q}_t^n - A_t \right|}}{\mathscr{L}^2}
					&= \Norm{\sup{t \in I}{\left| N_t - \hat{N}_t^n \right|}}{\mathscr{L}^2} \\
					&\leq \Norm{N_T - \hat{N}_T^n}{\mathscr{L}^2}
					= \Norm{\overline{N} - \hat{\overline{N^n}}}{\mathfrak{M}_{2,c}} \longrightarrow 0 \quad (n \longrightarrow \infty) 
				\end{align}
				が成り立つ
				\footnote{
					$\Norm{\cdot}{\mathfrak{M}_{2,c}}$は(\refeq{eq:M_2c_inner_product})で定義される内積により導入されるノルムを表す.
				}
				.これよりFatouの補題を使えば
				\begin{align}
					\Norm{\liminf_{n \to \infty}\sup{t \in I}{\left| \hat{Q}_t^n - A_t \right|}}{\mathscr{L}^2} = 0
				\end{align}
				が成り立つから,或る$\mu$-零集合$E$が存在して
				\begin{align}
					\liminf_{n \to \infty}\sup{t \in I}{\left| \hat{Q}_t^n(\omega) - A_t(\omega) \right|} = 0 
					\quad (\forall \omega \in \Omega \backslash E)
					\label{eq:thm_quadratic_variation_1}
				\end{align}
				となる.
				\begin{align}
					D_n \coloneqq \Set{\frac{j}{2^n}T}{ j = 0,1,\cdots,2^n } \quad (n=1,2,\cdots)
				\end{align}
				とおけば,(\refeq{eq:lem_quadratic_variation_0})より全ての$m \geq n,\ \omega \in \Omega$に対して
				$t \longmapsto Q_t^m(\omega)$は$D_n$上で単調非減少となるから,
				その線型結合である$\hat{Q}_t^m$も$D_n$上で単調非減少となり,
				(\refeq{eq:thm_quadratic_variation_1})より
				$\omega \in \Omega \backslash E$に対しては$t \longmapsto A_t(\omega)$も$D_n$上で単調非減少となる
				\footnote{
					或る$j$と$u \in \Omega \backslash E$で$A_{\frac{j}{2^n}T}(u) > A_{\frac{j+1}{2^n}T}(u)$が成り立っているとする.
					式(\refeq{eq:thm_quadratic_variation_1})を詳しく書けば
					\begin{align}
						\sup{n \geq 1}{\inf{\nu \geq n}{\sup{t \in I}{\left| \hat{Q}_t^\nu(\omega) - A_t(\omega) \right|}}} = 0 
						\quad (\forall \omega \in \Omega \backslash E)
					\end{align}
					ということになるから
					\begin{align}
						\inf{\nu \geq n}{\sup{t \in I}{\left| \hat{Q}_t^\nu(\omega) - A_t(\omega) \right|}} = 0 \quad (\forall n \in \N,\ \omega \in \Omega \backslash E)
					\end{align}
					を得る.従って任意の$\epsilon > 0$に対し必ず或る$\nu \geq 1$が存在して
					\begin{align}
						\sup{t \in I}{\left| \hat{Q}_t^\nu(\omega) - A_t(\omega) \right|} < \epsilon \quad (\forall \omega \in \Omega \backslash E)
						\label{eq:thm_quadratic_variation_2}
					\end{align}
					を満たすから,
					\begin{align}
						\alpha \coloneqq A_{\frac{j}{2^n}T}(u),
						\quad \beta \coloneqq A_{\frac{j+1}{2^n}T}(u)
					\end{align}
					とおけば或る$\nu \geq 1$に対し
					\begin{align}
						\left| \hat{Q}_{\frac{j}{2^n}T}^\nu(u) - A_{\frac{j}{2^n}T}(u) \right| < \frac{\alpha - \beta}{2},
						\quad \left| \hat{Q}_{\frac{j+1}{2^n}T}^\nu(u) - A_{\frac{j+1}{2^n}T}(u) \right| < \frac{\alpha - \beta}{2}
					\end{align}
					を同時に満たすが,
					\begin{align}
						\hat{Q}_{\frac{j}{2^n}T}^\nu(u) > \frac{\alpha + \beta}{2} > \hat{Q}_{\frac{j+1}{2^n}T}^\nu(u)
					\end{align}
					が従うので$t \longmapsto \hat{Q}_t^\nu(u)$の単調増大性に矛盾する.
					同様にして$A_0(\omega) = 0\ (\forall \omega \in \Omega \backslash E)$も成り立つ.
					或る$\upsilon \in \Omega \backslash E$で$A_0(\upsilon) \neq 0$であるとすれば,
					$Q^n_0(\upsilon) = 0\ (\forall n)$であるから(\refeq{eq:thm_quadratic_variation_2})に矛盾する.
					\label{footnote:thm_quadratic_variation}
				}.
				$D \coloneqq \cup_{n=1}^{\infty} D_n$とおけば$D$は$I$で稠密であり,更に$A$は或る零集合$E'$を除いてパスが連続となるから
				\footnote{
					$M \in \mathcal{M}_{b,c},\ N \in \mathcal{M}_{2,c}$より$M,N$のパスが連続でない$\omega$の全体は或る零集合に含まれる.それを$E'$とおけばよい.
				}
				,
				写像$I \ni t \longmapsto A_t(\omega)\ (\forall \omega \in \Omega \backslash (E \cup E'))$は連続且つ単調非減少である.
				$A$の適合性は(\refeq{eq:thm_quadratic_variation_0})より判明しているから,以上で$A \in \mathcal{A}^+$であることが示された.
				$A_0(\omega) = 0\ (\forall \omega \in \Omega \backslash E)$であることも脚注\ref{footnote:thm_quadratic_variation}
				で証明したから,$N = M^2 - A \in \mathcal{M}_{b,c} \subset \mathcal{M}_{c,loc}$(命題\ref{prp:M_pc_M_cloc})
				より定理の主張を満たす$A$の存在が言えた.
				
				存在の一意性は命題\ref{prp:bounded_continuous_M_2c_path}による.
				今$A' \in \mathcal{A}^+$もまた定理の主張を満たしているなら,
				$N' = M^2 - A'$として,$A - A' \in \mathcal{A}$かつ
				\begin{align}
					A - A' = N' - N \in \mathcal{M}_{2,c}
				\end{align}
				となり$A_t - A'_t = 0\ (\forall t \in I)\quad \mbox{$\mu$-a.s.}$が従う.
				
			\item[第二段]
				$M \in \mathcal{M}_{c,loc}$を任意に取る.或る$(\tau_j)_{j=1}^{\infty} \in \mathcal{T}$が存在して
				$M^n \in \mathcal{M}_{b,c}\ (\forall t \in I,\ M_t^n = M_{t \wedge \tau_n})$となるから,
				前段の結果より或る$A^n \in \mathcal{A}^+$が存在して
				\begin{align}
					N^n \coloneqq (M^n)^2 - A^n \in \mathcal{M}_{b,c}
				\end{align}
				を満たす.或る$\mu$-零集合$E_1$が存在して
				\footnote{
					或る零集合$E_1^{(1)}$があり$\tau_0(\omega) = 0\ (\forall \omega \in \Omega \backslash E_1^{(1)})$,
					また或る零集合$E_1^j$があり$\tau_j(\omega) \leq \tau_{j+1}(\omega)\ (\forall \omega \in \Omega \backslash E_1^j)$,
					更に或る零集合$E_1^{(T)}$を取れば,各$\omega \in \Omega \backslash E_1^{(T)}$について
					或る$n(\omega)$番目以降は$\tau_n(\omega) = T\ (\forall n \geq n(\omega))$
					が成り立つ.
					\begin{align}
						E_1 = \left( \cup_{j=1}^{\infty} E_1^j \right) \cup E_1^{(1)} \cup E_1^{(T)}
					\end{align}
					とおけばよい.
				}
				全ての$\omega \in \Omega \backslash E_1$で$(\tau_n(\omega))_{n=1}^{\infty}$は$0$出発,単調非減少かつ或る$n(\omega)$番目以降は$T$に一致する.
				今任意に$n \leq m,\ n,m \in \N$を取って固定する.$\omega \in \Omega \backslash E_1$に対しては
				\begin{align}
					M_{t \wedge \tau_n(\omega)}^m(\omega) = M_{t \wedge \tau_n(\omega) \wedge \tau_m(\omega)}(\omega) = M_t^n(\omega) \quad (\forall t \in I)
				\end{align}
				となり関数類として$[M_{t \wedge \tau_n}^m] = [M_t^n]\ (\forall t \in I)$が成り立つから,任意抽出定理(定理\ref{thm:optional_sampling_theorem_2})より
				\begin{align}
					\cexp{(M_t^m)^2 - A_t^m}{\mathcal{F}_{\tau_n}} = (M_{t \wedge \tau_n}^m)^2 - A_{t \wedge \tau_n}^m = (M_t^n)^2 - A_{t \wedge \tau_n}^m
					\quad (\forall t \in I)
				\end{align}
				を得る.一方で$N^m \in \mathcal{M}_{2,c}$であるから
				\begin{align}
					\cexp{(M_t^m)^2 - A_t^m}{\mathcal{F}_{\tau_n}} = \cexp{N_t^m}{\mathcal{F}_{\tau_n}} = N_{t \wedge \tau_n}^m \quad (\forall t \in I)
				\end{align}
				も成り立ち,任意抽出定理(定理\ref{thm:optional_sampling_theorem_2})より$\tau_n$で停めた過程も二乗可積分マルチンゲールとなる.
				関数類としての意味を強調すれば$[(M_t^n)^2 - A_{t \wedge \tau_n}^m] = [N_{t \wedge \tau_n}^m]\ (\forall t \in I)$が成り立っているから
				$\left( (M_t^n)^2 - A_{t \wedge \tau_n}^m \right)_{t \in I}$が二乗可積分マルチンゲールであることになり,前段の一意性から
				或る$\mu$-零集合$E^{n,m}$が存在して
				\begin{align}
					A_{t \wedge \tau_n(\omega)}^m(\omega) = A_t^n(\omega) \quad (\forall t \in I,\ \omega \in \Omega \backslash E^{n,m})
				\end{align}
				が従う.特に$\omega \in \Omega \backslash (E_1 \cup E^{n,m})$なら
				\begin{align}
					A_t^n(\omega) = A_{t \wedge \tau_n(\omega)}^m(\omega) = A_t^m(\omega) \quad (t \leq \tau_n(\omega))
				\end{align}
				となる.
				\begin{align}
					E_2 \coloneqq \bigcup_{n \in \N} E^{n,n+1}
				\end{align}
				とおき
				\begin{align}
					A_t(\omega) \coloneqq
					\begin{cases}
						\lim_{n \to \infty} A_t^n(\omega) & (\omega \in \Omega \backslash (E_1 \cup E_2)) \\
						0 & (\omega \in E_1 \cup E_2)
					\end{cases}
					\quad (\forall t \in I)
				\end{align}
				として$A$を定めれば$A \in \mathcal{A}^+$を満たす
				\footnote{
					\begin{description}
						\item[連続性・単調非減少性]
							$\omega \in \Omega \backslash (E_1 \cup E_2)$の場合に確認する.任意に$s,u \in I,\ (s < u)$を取れば
							$u \leq \tau_n(\omega)$となる$n$が存在し$A_t(\omega) = A_t^n(\omega)\ (\forall t \leq \tau_n(\omega))$を満たす.
							写像$I \ni t \longmapsto A_t^n(\omega)$は連続且つ単調非減少であるから
							$t \longmapsto A_t(\omega)$も$t = s,u$において連続であり,且つ$A_s(\omega) = A_s^n(\omega) \leq A_u^n(\omega) = A_u(\omega)$
							により単調非減少である.
						
						\item[適合性]
							各$n \in \N$に対し$A^n$は$(\mathcal{F}_t)$-適合である.$t \in I$を固定し
							\begin{align}
								\tilde{\mathcal{F}}_t \coloneqq \Set{B \cap (E_1 \cup E_2)^c}{B \in \mathcal{F}_t}
							\end{align}
							とおく.写像$\Omega \ni \omega \longmapsto A_t^n(\omega)$を
							$\Omega \backslash (E_1 \cup E_2)$に制限した$\tilde{A}_t^n \coloneqq A_t^n|_{\Omega \backslash (E_1 \cup E_2)}$は可測$\tilde{\mathcal{F}}_t/\borel{\R}$
							であり,各点収束先の$\tilde{A}_t \coloneqq A_t|_{\Omega \backslash (E_1 \cup E_2)}$もまた可測$\tilde{\mathcal{F}}_t/\borel{\R}$となる.
							任意の$C \in \borel{\R}$に対して
							\begin{align}
								A_t^{-1}(C) =
								\begin{cases}
									\tilde{A}_t^{-1}(C) & (0 \notin C) \\
									(E_1 \cup E_2) \cup \tilde{A}_t^{-1}(C) & (0 \in C)
								\end{cases}
							\end{align}
							となり,$E_1 \cup E_2 \in \mathcal{F}_0$により$\tilde{\mathcal{F}}_t \subset \mathcal{F}_t$であるから
							$A_t$は可測$\mathcal{F}_t/\borel{\R}$となる.
					\end{description}
				}
				.そして$N \coloneqq M^2 - A$とおけば$\Omega \backslash (E_1 \cup E_2)$上で
				\begin{align}
					N_{t \wedge \tau_n} = M_{t \wedge \tau_n}^2 - A_{t \wedge \tau_n} 
					= \left( M_{t \wedge \tau_n}^n \right)^2 - A_{t \wedge \tau_n}^n
					= N_{t \wedge \tau_n}^n \quad (\forall t \in I,\ n \in \N)
				\end{align}
				が成り立つから$N \in \mathcal{M}_{c,loc}$となる.$A$の一意性について,
		\end{description}
	\end{prf}
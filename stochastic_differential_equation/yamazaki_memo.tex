\chapter{山崎君講義メモ}
	\begin{screen}
		\begin{thm}[Yamazaki]
			$a:[0,\infty) \rightarrow \R$を$a(0) = 0$を満たす連続写像,
			$\kappa:[0,\infty) \rightarrow \R$を
			\begin{align}
				\kappa(0) = 0,\quad 
				\int_0^1 \frac{1}{\kappa(u)}\ du = \infty
			\end{align}
			を満たす単調非減少関数とする.
			$a,\kappa$が次を満たすとき,$a(t) = 0\ (\forall t \geq 0)$が成り立つ:
			\begin{align}
				|a(t) - a(s)| \leq \int_s^t \kappa(a(u))\ du
				\quad (\forall t \geq s \geq 0).
				\label{eq:thm_yamazaki_1}
			\end{align}
		\end{thm}
	\end{screen}
	
	\begin{prf}
		$A(t)\ (t \geq 0)$を$a$の$[0,t]$上の総変動とすれば
		\footnote{
			条件(\refeq{eq:thm_yamazaki_1})より$a$は任意の閉区間上で有界変動である.
		}
		,$t \longmapsto A(t)$は連続且つ非減少である.$A$により定めるStieltjes測度を$A$と表せば
		単調族定理より
		\begin{align}
			dA \leq \kappa(a(t))\ dt
		\end{align}
		が成り立つ.特に
		\begin{align}
			a(t) = a(t) - a(0) \leq A(t) \quad (\forall t \geq 0)
		\end{align}
		が満たされる.今,任意に$T > 0$を取り固定する.任意の$\epsilon > 0$に対し
		\begin{align}
			&\int_{A(0)}^{A(T)} \frac{1}{\kappa(u) + \epsilon}\ du
			= \int_0^T \frac{1}{\kappa(A(u)) + \epsilon}\ dA(u) \\
			&\qquad \leq \int_0^T \frac{1}{\kappa(a(u)) + \epsilon}\ dA(u)
			\leq \int_0^T \frac{1}{\kappa(a(u)) + \epsilon} \kappa(a(u))\ du
			\leq T
		\end{align}
		が成り立つ.$A(T) > 0$の場合,左辺は単調収束定理より$\epsilon \longrightarrow +0$で発散するから,
		不等式と整合性を保つためには$A(T) = 0$でなくてはならない.これより主張を得る.
		\QED
	\end{prf}
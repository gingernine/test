\documentclass[11pt,a4paper]{jsarticle}
%
\usepackage{amsmath,amssymb}
\usepackage{amsthm}
\usepackage{makeidx}
\usepackage{txfonts}
\usepackage{mathrsfs} %花文字
\usepackage{mathtools} %参照式のみ式番号表示
\usepackage{latexsym} %qed
\usepackage{ascmac}
\usepackage{color}

\usepackage{comment}
\newtheoremstyle{mystyle}% % Name
	{20pt}%                      % Space above
	{20pt}%                      % Space below
	{\rm}%           % Body font
	{}%                      % Indent amount
	{\gt}%             % Theorem head font
	{.}%                      % Punctuation after theorem head
	{10pt}%                     % Space after theorem head, ' ', or \newline
	{}%                      % Theorem head spec (can be left empty, meaning `normal')
\theoremstyle{mystyle}

\allowdisplaybreaks[1]
\newcommand{\bhline}[1]{\noalign {\hrule height #1}} %表の罫線を太くする.
\newcommand{\bvline}[1]{\vrule width #1} %表の罫線を太くする.
\newtheorem{Prop}{$Proposition.$}
\newtheorem{Proof}{$Proof.$}
\newcommand{\QED}{% %証明終了
	\relax\ifmmode
		\eqno{%
		\setlength{\fboxsep}{2pt}\setlength{\fboxrule}{0.3pt}
		\fcolorbox{black}{black}{\rule[2pt]{0pt}{1ex}}}
	\else
		\begingroup
		\setlength{\fboxsep}{2pt}\setlength{\fboxrule}{0.3pt}
		\hfill\fcolorbox{black}{black}{\rule[2pt]{0pt}{1ex}}
		\endgroup
	\fi}
\newtheorem*{thm}{定理}
\newtheorem*{prp}{命題}
\newtheorem*{dfn}{定義}
\newtheorem*{prf}{証明}
\newtheorem*{asm}{仮定}
\newtheorem*{ans}{答案}

%\renewcommand{\contentsname}{\bm Index}
%
\makeindex
%
\setlength{\textheight}{40\baselineskip}
\setlength{\fullwidth}{\textwidth}
\addtolength{\textheight}{\topskip}
\setlength{\voffset}{-0.2in}
\setlength{\topmargin}{0pt}
\setlength{\headheight}{0pt}
\setlength{\headsep}{0pt}
%
\title{金融確率解析レポート}
\author{基礎工学研究科システム創成専攻修士2年\\学籍番号29C17095\\百合川尚学\\選択問題\ 1)\ 5)\ 6)}
\date{\today}

\begin{document}
\mathtoolsset{showonlyrefs = true}
\maketitle

\begin{itembox}[l]{5)}
	定数$-1 \leq \rho \leq 1$を用いて
	\begin{align}
		\bar{W}_{t} \coloneqq \rho W^{1}_{t} + \sqrt{1-\rho^{2}} W^{2}_{t},
		\quad t \geq 0
	\end{align}
	を定義する.$(\bar{W}_{t})_{t \geq 0}$は標準ブラウン運動であることを示せ.また
	$W^{1}_{t}$と$\bar{W}_{t}$との共分散を求めよ.
\end{itembox}

\begin{ans}\mbox{}
	\begin{description}
		\item[連続性]
			連続関数の定数倍も連続関数同士の和も連続関数になるので
			$t \longmapsto \bar{W}_{t}$は連続である.
			
		\item[$0$出発]
			$W^{1}_{0} = 0$かつ$W^{2}_{0} = 0$なので$\bar{W}_{0} = 0$である.
			
		\item[独立増分性と分布]
			$(\bar{W}_{t})_{t \geq 0}$の独立増分性と
			任意の時点$0 \leq s < t$に対して$\bar{W}_{t} - \bar{W}_{s}$が
			$N(0,t-s)$に従うことを言うには,
			任意の自然数$n$に対して
			任意の$0 = t_{0} < t_{1} < \cdots < t_{n}$および
			任意の実数$\alpha_{1},\cdots,\alpha_{n}$を取って
			\begin{align}
				\operatorname{E}\left[\exp{\left(\sum_{j=1}^{n}\alpha_{j}(\bar{W}_{t_{j}} - \bar{W}_{t_{j-1}})\right)}\right]
				= \prod_{j=1}^{n} \exp{\left(\frac{1}{2}\alpha_{j}^{2}(t_{j}-t_{j-1})\right)}
			\end{align}
			を示せばよい.まず
			\begin{align}
				&\operatorname{E}\left[\exp{\left(\sum_{j=1}^{n}\alpha_{j}(\bar{W}_{t_{j}} - \bar{W}_{t_{j-1}})\right)}\right] \\
				&= \operatorname{E}\left[e^{\sum_{j=1}^{n}\rho\alpha_{j}(W^{1}_{t_{j}} - W^{1}_{t_{j-1}})}e^{\sum_{j=1}^{n}\sqrt{1-\rho^{2}}\alpha_{j}(W^{2}_{t_{j}} - W^{2}_{t_{j-1}})}\right] \\
				&=  \operatorname{E}\left[e^{\sum_{j=1}^{n}\rho\alpha_{j}(W^{1}_{t_{j}} - W^{1}_{t_{j-1}})}\right] \operatorname{E}\left[e^{\sum_{j=1}^{n}\sqrt{1-\rho^{2}}\alpha_{j}(W^{2}_{t_{j}} - W^{2}_{t_{j-1}})}\right] && \mbox{($W^{1}$と$W^{2}$は独立なので)}
			\end{align}
			となるが,ここで$W^{1}$の独立増分性より
			\begin{align}
				\operatorname{E}\left[e^{\sum_{j=1}^{n}\rho\alpha_{j}(W^{1}_{t_{j}} - W^{1}_{t_{j-1}})}\right] 
				= \prod_{j=1}^{n} \operatorname{E}\left[e^{\rho\alpha_{j}(W^{1}_{t_{j}} - W^{1}_{t_{j-1}})}\right]
			\end{align}
			が成り立つ.この右辺については
			\begin{align}
				&\operatorname{E}\left[e^{\rho\alpha_{j}(W^{1}_{t_{j}} - W^{1}_{t_{j-1}})}\right] \\
				&= \int_{-\infty}^{\infty} e^{\rho\alpha_{j}x}\frac{1}{\sqrt{2\pi(t_{j}-t_{j-1})}}e^{-\frac{x^{2}}{2(t_{j}-t_{j-1})}}\ dx \\
				&= \int_{-\infty}^{\infty} e^{\rho\alpha_{j}\sqrt{t_{j}-t_{j-1}}x}\frac{1}{\sqrt{2\pi}}e^{-\frac{x^{2}}{2}}\ dx \\
				&= e^{\frac{1}{2}\rho^{2}\alpha_{j}^{2}(t_{j}-t_{j-1})}\int_{-\infty}^{\infty} \frac{1}{\sqrt{2\pi}}e^{-\frac{(x-\rho\alpha_{j}\sqrt{t_{j}-t_{j-1}})^{2}}{2}}\ dx \\
				&= e^{\frac{1}{2}\rho^{2}\alpha_{j}^{2}(t_{j}-t_{j-1})}
			\end{align}
			が成り立つので
			\begin{align}
				\operatorname{E}\left[e^{\sum_{j=1}^{n}\rho\alpha_{j}(W^{1}_{t_{j}} - W^{1}_{t_{j-1}})}\right] 
				= \prod_{j=1}^{n} e^{\frac{1}{2}\rho^{2}\alpha_{j}^{2}(t_{j}-t_{j-1})}
			\end{align}
			となる.同様にして
			\begin{align}
				\operatorname{E}\left[e^{\sum_{j=1}^{n}\sqrt{1-\rho^{2}}\alpha_{j}(W^{2}_{t_{j}} - W^{2}_{t_{j-1}})}\right] 
				= \prod_{j=1}^{n} e^{\frac{1}{2}(1-\rho^{2})\alpha_{j}^{2}(t_{j}-t_{j-1})}
			\end{align}
			が成り立つので
			\begin{align}
				&\operatorname{E}\left[\exp{\left(\sum_{j=1}^{n}\alpha_{j}(\bar{W}_{t_{j}} - \bar{W}_{t_{j-1}})\right)}\right] \\
				&=  \operatorname{E}\left[e^{\sum_{j=1}^{n}\rho\alpha_{j}(W^{1}_{t_{j}} - W^{1}_{t_{j-1}})}\right] \operatorname{E}\left[e^{\sum_{j=1}^{n}\sqrt{1-\rho^{2}}\alpha_{j}(W^{2}_{t_{j}} - W^{2}_{t_{j-1}})}\right] \\
				&= \prod_{j=1}^{n} e^{\frac{1}{2}\rho^{2}\alpha_{j}^{2}(t_{j}-t_{j-1})}e^{\frac{1}{2}(1-\rho^{2})\alpha_{j}^{2}(t_{j}-t_{j-1})} \\
				&= \prod_{j=1}^{n} e^{\frac{1}{2}\alpha_{j}^{2}(t_{j}-t_{j-1})}
			\end{align}
			が従う.これで求める式を得た.
			\QED
	\end{description}
	
\begin{itembox}[l]{6)}
	$\mbox{USD}/\mbox{JPY}$為替レート過程を表す$S^{1}$と
	$\mbox{EUR}/\mbox{JPY}$為替レート過程を表す$S^{2}$が
	\begin{align}
		dS^{1}_{t} &= S^{1}_{t}(\sigma_{1}dW^{1}_{t} + \mu_{1}dt), \quad S^{1}_{0} > 0, \\
		dS^{2}_{t} &= S^{2}_{t}(\sigma_{2}dW^{2}_{t} + \mu_{2}dt), \quad S^{2}_{0} > 0, 
	\end{align}
	と与えられている(すなわち,時刻$t$で$1\mbox{USD}$が$S^{1}_{t}$円であり,
	$\mbox{1EUR}$が$S^{2}_{t}$円である).ただし$\sigma_{1},\sigma_{2} > 0,\ 
	\mu_{1},\mu_{2} \in \mathbb{R}$である.このとき$\mbox{USD}/\mbox{EUR}$為替レート過程を
	計算し,そのボラティリティと期待収益率を求めよ.
\end{itembox}

\begin{ans}
	$\mbox{USD}/\mbox{EUR}$為替レート過程は$S^{1}/S^{2}$を計算すればよい.
	\begin{align}
		d\begin{bmatrix}
			X^{1}_{t} \\
			X^{2}_{t} \\
		\end{bmatrix} = 
		\begin{bmatrix}
			S^{1}_{t} & 0 \\
			0 & S^{2}_{t} \\
		\end{bmatrix}
		\left(
		\begin{bmatrix}
			\sigma_{1} & 0 \\
			0 & \sigma_{2} \\
		\end{bmatrix}
		\begin{bmatrix}
			dW^{1}_{t} \\
			dW^{2}_{t} \\
		\end{bmatrix}
		+ 
		\begin{bmatrix}
			\mu_{1} \\
			\mu_{2} \\
		\end{bmatrix}
		dt
		\right)
	\end{align}
	なので,伊藤の公式より
	\begin{align}
		\frac{S^{1}_{t}}{S^{2}_{t}} 
		&= \frac{S^{1}_{0}}{S^{2}_{0}} + \int_{0}^{t} \frac{1}{S^{2}_{s}}\ dS^{1}_{s}
		+ \int_{0}^{t} \frac{-S^{1}_{s}}{(S^{2}_{s})^{2}}\ dS^{2}_{s} \\
		&\quad + \frac{1}{2}\int_{0}^{t} \mbox{tr}\left(
		\begin{bmatrix}
			0 & \frac{-1}{(S^{2}_{s})^{2}} \\
			\frac{-1}{(S^{2}_{s})^{2}} & \frac{2S^{1}_{s}}{(S^{2}_{s})^{3}} \\
		\end{bmatrix}
		\begin{bmatrix}
			S^{1}_{s}\sigma_{1} & 0 \\
			0 & S^{2}_{s}\sigma_{2} \\
		\end{bmatrix}
		\begin{bmatrix}
			S^{1}_{s}\sigma_{1} & 0 \\
			0 & S^{2}_{s}\sigma_{2} \\
		\end{bmatrix}
		\right)ds \\
		&= \frac{S^{1}_{0}}{S^{2}_{0}} + \int_{0}^{t} \frac{1}{S^{2}_{s}}\ dS^{1}_{s}
		+ \int_{0}^{t} \frac{-S^{1}_{s}}{(S^{2}_{s})^{2}}\ dS^{2}_{s}
		+ \int_{0}^{t} \frac{S^{1}_{s}}{S^{2}_{s}} \sigma_{2}^{2}\ ds \\
		&= \frac{S^{1}_{0}}{S^{2}_{0}} + 
		\int_{0}^{t} \frac{1}{S^{2}_{s}}\ S^{1}_{s}(\sigma_{1}dW^{1}_{s} + \mu_{1}ds) \\
		&\quad+ \int_{0}^{t} \frac{-S^{1}_{s}}{(S^{2}_{s})^{2}}\ S^{2}_{s}(\sigma_{2}dW^{2}_{s} + \mu_{2}ds) \\
		&\quad+ \int_{0}^{t} \frac{S^{1}_{s}}{S^{2}_{s}} \sigma_{2}^{2}\ ds \\
		&= \frac{S^{1}_{0}}{S^{2}_{0}} + 
		\int_{0}^{t}\frac{S^{1}_{s}}{S^{2}_{s}}\sigma_{1}\ dW^{1}_{s}
		+ \int_{0}^{t}\frac{-S^{1}_{s}}{S^{2}_{s}}\sigma_{2}\ dW^{2}_{s}
		+ \int_{0}^{t}\frac{S^{1}_{s}}{S^{2}_{s}}(\mu_{1}-\mu_{2}+\sigma_{2}^{2})\ ds
	\end{align}
	が成り立つ.これが$\mbox{USD}/\mbox{EUR}$為替レート過程の式である.
\end{ans}

\end{ans}
\end{document}
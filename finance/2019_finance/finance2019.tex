\documentclass[11pt,a4paper]{jsarticle}
%
\usepackage{amsmath,amssymb}
\usepackage{amsthm}
\usepackage{makeidx}
\usepackage{txfonts}
\usepackage{mathrsfs} %花文字
\usepackage{mathtools} %参照式のみ式番号表示
\usepackage{latexsym} %qed
\usepackage{ascmac}
\usepackage{color}

\usepackage{comment}
\newtheoremstyle{mystyle}% % Name
	{20pt}%                      % Space above
	{20pt}%                      % Space below
	{\rm}%           % Body font
	{}%                      % Indent amount
	{\gt}%             % Theorem head font
	{.}%                      % Punctuation after theorem head
	{10pt}%                     % Space after theorem head, ' ', or \newline
	{}%                      % Theorem head spec (can be left empty, meaning `normal')
\theoremstyle{mystyle}

\allowdisplaybreaks[1]
\newcommand{\bhline}[1]{\noalign {\hrule height #1}} %表の罫線を太くする.
\newcommand{\bvline}[1]{\vrule width #1} %表の罫線を太くする.
\newtheorem{Prop}{$Proposition.$}
\newtheorem{Proof}{$Proof.$}
\newcommand{\QED}{% %証明終了
	\relax\ifmmode
		\eqno{%
		\setlength{\fboxsep}{2pt}\setlength{\fboxrule}{0.3pt}
		\fcolorbox{black}{black}{\rule[2pt]{0pt}{1ex}}}
	\else
		\begingroup
		\setlength{\fboxsep}{2pt}\setlength{\fboxrule}{0.3pt}
		\hfill\fcolorbox{black}{black}{\rule[2pt]{0pt}{1ex}}
		\endgroup
	\fi}
\newtheorem*{thm}{定理}
\newtheorem*{prp}{命題}
\newtheorem*{dfn}{定義}
\newtheorem*{prf}{証明}
\newtheorem*{asm}{仮定}
\newtheorem*{ans}{答案}

%\renewcommand{\contentsname}{\bm Index}
%
\makeindex
%
\setlength{\textheight}{40\baselineskip}
\setlength{\fullwidth}{\textwidth}
\addtolength{\textheight}{\topskip}
\setlength{\voffset}{-0.2in}
\setlength{\topmargin}{0pt}
\setlength{\headheight}{0pt}
\setlength{\headsep}{0pt}
%
\title{金融確率解析レポート}
\author{基礎工学研究科システム創成専攻修士2年\\学籍番号29C17095 百合川尚学\\選択問題\ 3)\ 4)\ 5)\ 6)}
\date{\today}

\begin{document}
\mathtoolsset{showonlyrefs = true}
\maketitle

\begin{itembox}[l]{3)}
	コールオプションのベガ
	\begin{align}
		\mathcal{V}_{t} = \partial_{\sigma}C_{BS}(t,T,S_{t},K,r,\sigma)
	\end{align}
	を計算せよ(計算の途中経過も示せ).また,ボラティリティに関してコールオプション価格が単調増加であること,
	すなわち
	\begin{align}
		\sigma \longmapsto C_{BS}(t,T,x,K,r,\sigma)
	\end{align}
	が単調増加であることを示せ.
\end{itembox}

\begin{ans}
	$\tau \coloneqq T - t$として,まず$C_{BS}(t,T,x,K,r,\sigma)$を偏微分すれば
	\begin{align}
		\partial_{\sigma}C_{BS}
		= x \frac{1}{\sqrt{2\pi}}e^{-\frac{d_{1}^{2}}{2}}\partial_{\sigma}d_{1}
		- e^{-r\tau}K\frac{1}{\sqrt{2\pi}}e^{-\frac{d_{2}^{2}}{2}}\partial_{\sigma}d_{2}
		\label{formula:3_1}
	\end{align}
	となる.ここで
	\begin{align}
		\partial_{\sigma}d_{1} &= \frac{-1}{\sigma^{2}\sqrt{\tau}}
		\left\{\log{\left(\frac{x}{K}\right)}+\left(r+\frac{\sigma^{2}}{2}\right)\tau\right\} + \sqrt{\tau} \\
		&= \frac{-1}{\sigma^{2}\sqrt{\tau}}
		\left\{\log{\left(\frac{x}{K}\right)}+\left(r-\frac{\sigma^{2}}{2}\right)\tau\right\} \\
		&= \frac{-d_{2}}{\sigma}
	\end{align}
	および
	\begin{align}
		\partial_{\sigma}d_{2} &= \partial_{\sigma}d_{1}-\sqrt{\tau} \\
		&= -\frac{d_{2} + \sigma \sqrt{\tau}}{\sigma} \\
		&= \frac{-d_{1}}{\sigma}
	\end{align}
	が成り立つので,
	\begin{align}
		(\refeq{formula:3_1}) &= x \frac{1}{\sqrt{2\pi}}e^{-\frac{d_{1}^{2}}{2}}
		\frac{-d_{2}}{\sigma} + e^{-r\tau}K\frac{1}{\sqrt{2\pi}}e^{-\frac{d_{2}^{2}}{2}}\frac{d_{1}}{\sigma} \\
		&= \frac{1}{\sqrt{2\pi}}e^{-\frac{d_{1}^{2}}{2}}
		\left\{x\frac{-d_{2}}{\sigma} + e^{-r\tau}K\frac{d_{1}}{\sigma}e^{\frac{1}{2}(d_{1}^{2}-d_{2}^{2})}\right\} \\
		&= \Phi'(d_{1}) \left\{x\frac{-d_{2}}{\sigma} + e^{-r\tau}K\frac{d_{1}}{\sigma}e^{\frac{1}{2}(d_{1}^{2}-d_{2}^{2})}\right\}
		\label{formula:3_2}
	\end{align}
	となる.ここで
	\begin{align}
		d_{1}^{2}-d_{2}^{2}
		&= (d_{1}-d_{2})(d_{1}+d_{2}) \\
		&= \sigma \sqrt{\tau}(2d_{1}-\sigma \sqrt{\tau}) \\
		&= 2\log{\left(\frac{x}{K}\right)} + 2\left(r+\frac{\sigma^{2}}{2}\right)\tau
		- \sigma^{2}\tau \\
		&= 2\log{\left(\frac{x}{K}\right)} + 2r\tau
		\label{formula:3_3}
	\end{align}
	が成り立つので
	\begin{align}
		(\refeq{formula:3_2}) &= 
		\Phi'(d_{1}) \left\{x\frac{-d_{2}}{\sigma} + e^{-r\tau}K\frac{d_{1}}{\sigma}e^{\log{\left(\frac{x}{K}\right)}+r\tau}\right\} \\
		&= \Phi'(d_{1}) \left\{x\frac{-d_{2}}{\sigma} + x\frac{d_{1}}{\sigma}\right\} \\
		&= \Phi'(d_{1}) \left\{x \frac{d_{1}-d_{2}}{\sigma}\right\} \\
		&= \Phi'(d_{1}) x \sqrt{\tau}
	\end{align}
	が出る.以上より
	\begin{align}
		\mathcal{V}_{t} = S_{t} \sqrt{\tau} \Phi'(d_{1}(\tau,S_{t},K,r,\sigma))
		\label{formula:3_4}
	\end{align}
	である.また単調増大性については,$\Phi'(x) = e^{-x^{2}/2}/\sqrt{2\pi}$
	が非負関数のため($x$を正の数とすれば)
	\begin{align}
		\partial_{\sigma}C_{BS} = x \sqrt{\tau} \Phi'(d_{1}) > 0
	\end{align}
	となるからである.
\end{ans}

\begin{itembox}[l]{4)}
	コールオプションのガンマ
	\begin{align}
		\Gamma_{t} \coloneqq \partial_{xx}C_{BS}(t,T,S_{t},K,r,\sigma)
	\end{align}
	とベガの間に
	\begin{align}
		\mathcal{V}_{t} = \sigma S_{t}^{2}(T-t)\Gamma_{t}
	\end{align}
	が成り立つことを示せ.
\end{itembox}

\begin{ans}
	$\tau \coloneqq T - t$として,まず$C_{BS}(t,T,x,K,r,\sigma)$を一回偏微分すれば
	\begin{align}
		\partial_{x}C_{BS} = \Phi(d_{1}) + x \Phi'(d_{1}) \partial_{x}d_{1}
		- e^{-r\tau}K\Phi'(d_{2})\partial_{x}d_{2} 
	\end{align}
	となり,もう一度偏微分すれば
	\begin{align}
		\partial_{x}C_{BS} &= \Phi'(d_{1})\partial_{x}d_{1} 
		+ \Phi'(d_{1}) \partial_{x}d_{1} + x \Phi''(d_{1}) (\partial_{x} d_{1})^{2}
		+ x \Phi'(d_{1}) \partial_{xx} d_{1} \\
		&\quad - e^{-r\tau}K\left(\Phi''(d_{2})(\partial_{x}d_{2})^{2}
		+ \Phi'(d_{2})\partial_{xx}d_{2}\right)
		\label{formula:4_1}
	\end{align}
	となる.ここで
	\begin{align}
		\Phi''(z) &= \frac{-z}{\sqrt{2\pi}}e^{-\frac{z^{2}}{2}}, \\
		\partial_{x} d_{1} &= \frac{1}{\sigma\sqrt{\tau}} \frac{1}{x}, \\
		\partial_{xx} d_{1} &= \frac{1}{\sigma\sqrt{\tau}} \frac{-1}{x^{2}} \\
		\partial_{x} d_{2} &= \frac{1}{\sigma\sqrt{\tau}} \frac{1}{x}, \\
		\partial_{xx} d_{2} &= \frac{1}{\sigma\sqrt{\tau}} \frac{-1}{x^{2}}
	\end{align}
	であるから
	\begin{align}
		&\Phi'(d_{1})\partial_{x}d_{1} 
		+ \Phi'(d_{1}) \partial_{x}d_{1} + x \Phi''(d_{1}) (\partial_{x} d_{1})^{2}
		+ x \Phi'(d_{1}) \partial_{xx} d_{1}, \\
		&= 2\frac{1}{\sqrt{2\pi}}e^{-\frac{d_{1}^{2}}{2}} \frac{1}{\sigma\sqrt{\tau}}\frac{1}{x} 
		+ x\frac{-d_{1}}{\sqrt{2\pi}}e^{-\frac{d_{1}^{2}}{2}} 
		\frac{1}{\sigma^{2}\tau}\frac{1}{x^{2}}
		+ x\frac{1}{\sqrt{2\pi}}e^{-\frac{d_{1}^{2}}{2}} \frac{1}{\sigma\sqrt{\tau}}\frac{-1}{x^{2}} \\
		&= \frac{1}{\sqrt{2\pi}}e^{-\frac{d_{1}^{2}}{2}} \frac{1}{\sigma\sqrt{\tau}}\frac{1}{x} 
		+ \frac{1}{\sqrt{2\pi}}e^{-\frac{d_{1}^{2}}{2}} \frac{-d_{1}}{\sigma^{2}\tau} \frac{1}{x} \\
		&= \frac{1}{\sqrt{2\pi}}e^{-\frac{d_{1}^{2}}{2}} \frac{1}{x\sigma^{2}\tau}
		(\sigma \sqrt{\tau} - d_{1}) \\
		&= \frac{1}{\sqrt{2\pi}}e^{-\frac{d_{1}^{2}}{2}} \frac{-d_{2}}{x\sigma^{2}\tau}
	\end{align}
	および
	\begin{align}
		&\Phi''(d_{2})(\partial_{x}d_{2})^{2} + \Phi'(d_{2})\partial_{xx}d_{2} \\
		&= \frac{-d_{2}}{\sqrt{2\pi}}e^{-\frac{d_{2}^{2}}{2}} \frac{1}{\sigma^{2}\tau}
		\frac{1}{x^{2}} + \frac{1}{\sqrt{2\pi}}e^{-\frac{d_{2}^{2}}{2}}
		\frac{1}{\sigma \sqrt{\tau}}\frac{-1}{x^{2}} \\
		&= \frac{1}{\sqrt{2\pi}} e^{-\frac{d_{2}^{2}}{2}} \frac{1}{x^{2}\sigma^{2}\tau}
		(d_{2} + \sigma \sqrt{\tau}) \\
		&= \frac{1}{\sqrt{2\pi}} e^{-\frac{d_{2}^{2}}{2}} \frac{-d_{1}}{x^{2}\sigma^{2}\tau}
	\end{align}
	が成り立つ.従って
	\begin{align}
		(\refeq{formula:4_1}) 
		&= \frac{1}{\sqrt{2\pi}}e^{-\frac{d_{1}^{2}}{2}} \frac{-d_{2}}{x\sigma^{2}\tau}
		+ e^{-r\tau}K \frac{1}{\sqrt{2\pi}} e^{-\frac{d_{2}^{2}}{2}} \frac{d_{1}}{x^{2}\sigma^{2}\tau} \\
		&= \frac{1}{\sqrt{2\pi}}e^{-\frac{d_{1}^{2}}{2}} \frac{1}{x\sigma^{2}\tau}
		\left(-d_{2} + e^{-r\tau}K e^{\frac{1}{2}(d_{1}^{2} - d_{2}^{2})}
		\frac{d_{1}}{x}\right) \\
		&= \frac{1}{\sqrt{2\pi}}e^{-\frac{d_{1}^{2}}{2}} \frac{1}{x\sigma^{2}\tau}
		(-d_{2} + d_{1}) 
		&& (\mbox{(\refeq{formula:3_3})より$e^{\frac{1}{2}(d_{1}^{2} - d_{2}^{2})} = (x/K)e^{r\tau}$}) \\
		&= \frac{1}{\sqrt{2\pi}}e^{-\frac{d_{1}^{2}}{2}} \frac{1}{x\sigma\sqrt{\tau}}
	\end{align}
	が成立する.特に
	\begin{align}
		\Gamma_{t} = \frac{1}{S_{t}\sigma\sqrt{\tau}}\Phi'(d_{1}(\tau,S_{t},K,r,\sigma))
	\end{align}
	なので,(\refeq{formula:3_4})との関係式
	\begin{align}
		\mathcal{V}_{t} = \sigma S_{t}^{2} (T-t) \Gamma_{t}
	\end{align}
	が得られる.
	\QED
\end{ans}

\begin{itembox}[l]{5)}
	定数$-1 \leq \rho \leq 1$を用いて
	\begin{align}
		\bar{W}_{t} \coloneqq \rho W^{1}_{t} + \sqrt{1-\rho^{2}} W^{2}_{t},
		\quad t \geq 0
	\end{align}
	を定義する.$(\bar{W}_{t})_{t \geq 0}$は標準ブラウン運動であることを示せ.また
	$W^{1}_{t}$と$\bar{W}_{t}$との共分散を求めよ.
\end{itembox}

\begin{ans}\mbox{}
	\begin{description}
		\item[連続性]
			連続関数の定数倍も連続関数同士の和も連続関数になるので
			$t \longmapsto \bar{W}_{t}$は連続である.
			
		\item[$0$出発]
			$W^{1}_{0} = 0$かつ$W^{2}_{0} = 0$なので$\bar{W}_{0} = 0$である.
			
		\item[独立増分性と分布]
			$(\bar{W}_{t})_{t \geq 0}$の独立増分性と
			任意の時点$0 \leq s < t$に対して$\bar{W}_{t} - \bar{W}_{s}$が
			$N(0,t-s)$に従うことを言うには,
			任意の自然数$n$に対して
			任意の$0 = t_{0} < t_{1} < \cdots < t_{n}$および
			任意の実数$\alpha_{1},\cdots,\alpha_{n}$を取って
			\begin{align}
				\operatorname{E}\left[\exp{\left(\sum_{j=1}^{n}\alpha_{j}(\bar{W}_{t_{j}} - \bar{W}_{t_{j-1}})\right)}\right]
				= \prod_{j=1}^{n} \exp{\left(\frac{1}{2}\alpha_{j}^{2}(t_{j}-t_{j-1})\right)}
			\end{align}
			を示せばよい.まず
			\begin{align}
				&\operatorname{E}\left[\exp{\left(\sum_{j=1}^{n}\alpha_{j}(\bar{W}_{t_{j}} - \bar{W}_{t_{j-1}})\right)}\right] \\
				&= \operatorname{E}\left[e^{\sum_{j=1}^{n}\rho\alpha_{j}(W^{1}_{t_{j}} - W^{1}_{t_{j-1}})}e^{\sum_{j=1}^{n}\sqrt{1-\rho^{2}}\alpha_{j}(W^{2}_{t_{j}} - W^{2}_{t_{j-1}})}\right] \\
				&=  \operatorname{E}\left[e^{\sum_{j=1}^{n}\rho\alpha_{j}(W^{1}_{t_{j}} - W^{1}_{t_{j-1}})}\right] \operatorname{E}\left[e^{\sum_{j=1}^{n}\sqrt{1-\rho^{2}}\alpha_{j}(W^{2}_{t_{j}} - W^{2}_{t_{j-1}})}\right] && \mbox{($W^{1}$と$W^{2}$は独立なので)}
			\end{align}
			となるが,ここで$W^{1}$の独立増分性より
			\begin{align}
				\operatorname{E}\left[e^{\sum_{j=1}^{n}\rho\alpha_{j}(W^{1}_{t_{j}} - W^{1}_{t_{j-1}})}\right] 
				= \prod_{j=1}^{n} \operatorname{E}\left[e^{\rho\alpha_{j}(W^{1}_{t_{j}} - W^{1}_{t_{j-1}})}\right]
			\end{align}
			が成り立つ.この右辺については
			\begin{align}
				&\operatorname{E}\left[e^{\rho\alpha_{j}(W^{1}_{t_{j}} - W^{1}_{t_{j-1}})}\right] \\
				&= \int_{-\infty}^{\infty} e^{\rho\alpha_{j}x}\frac{1}{\sqrt{2\pi(t_{j}-t_{j-1})}}e^{-\frac{x^{2}}{2(t_{j}-t_{j-1})}}\ dx \\
				&= \int_{-\infty}^{\infty} e^{\rho\alpha_{j}\sqrt{t_{j}-t_{j-1}}x}\frac{1}{\sqrt{2\pi}}e^{-\frac{x^{2}}{2}}\ dx \\
				&= e^{\frac{1}{2}\rho^{2}\alpha_{j}^{2}(t_{j}-t_{j-1})}\int_{-\infty}^{\infty} \frac{1}{\sqrt{2\pi}}e^{-\frac{(x-\rho\alpha_{j}\sqrt{t_{j}-t_{j-1}})^{2}}{2}}\ dx \\
				&= e^{\frac{1}{2}\rho^{2}\alpha_{j}^{2}(t_{j}-t_{j-1})}
			\end{align}
			が成り立つので
			\begin{align}
				\operatorname{E}\left[e^{\sum_{j=1}^{n}\rho\alpha_{j}(W^{1}_{t_{j}} - W^{1}_{t_{j-1}})}\right] 
				= \prod_{j=1}^{n} e^{\frac{1}{2}\rho^{2}\alpha_{j}^{2}(t_{j}-t_{j-1})}
			\end{align}
			となる.同様にして
			\begin{align}
				\operatorname{E}\left[e^{\sum_{j=1}^{n}\sqrt{1-\rho^{2}}\alpha_{j}(W^{2}_{t_{j}} - W^{2}_{t_{j-1}})}\right] 
				= \prod_{j=1}^{n} e^{\frac{1}{2}(1-\rho^{2})\alpha_{j}^{2}(t_{j}-t_{j-1})}
			\end{align}
			が成り立つので
			\begin{align}
				&\operatorname{E}\left[\exp{\left(\sum_{j=1}^{n}\alpha_{j}(\bar{W}_{t_{j}} - \bar{W}_{t_{j-1}})\right)}\right] \\
				&=  \operatorname{E}\left[e^{\sum_{j=1}^{n}\rho\alpha_{j}(W^{1}_{t_{j}} - W^{1}_{t_{j-1}})}\right] \operatorname{E}\left[e^{\sum_{j=1}^{n}\sqrt{1-\rho^{2}}\alpha_{j}(W^{2}_{t_{j}} - W^{2}_{t_{j-1}})}\right] \\
				&= \prod_{j=1}^{n} e^{\frac{1}{2}\rho^{2}\alpha_{j}^{2}(t_{j}-t_{j-1})}e^{\frac{1}{2}(1-\rho^{2})\alpha_{j}^{2}(t_{j}-t_{j-1})} \\
				&= \prod_{j=1}^{n} e^{\frac{1}{2}\alpha_{j}^{2}(t_{j}-t_{j-1})}
			\end{align}
			が従う.これで求める式を得た.
			
		\item[$W^{1}_{t}$と$\bar{W}_{t}$との共分散]
			$W^{1}_{t}$も$\bar{W}_{t}$も平均は$0$なので
			\begin{align}
				\operatorname{E}\left[W^{1}_{t}\bar{W}_{t}\right]
				&= \operatorname{E}\left[W^{1}_{t}\left(\rho W^{1}_{t} + \sqrt{1-\rho^{2}}W^{2}_{t}\right)\right] \\
				&= \rho\operatorname{E}\left[{W^{1}_{t}}^{2}\right] +
				\sqrt{1-\rho^{2}}\operatorname{E}\left[W^{1}_{t}W^{2}_{t}\right] \\
				&= \rho t 
			\end{align}
			が共分散である.
			\QED
	\end{description}
	
\begin{itembox}[l]{6)}
	$\mbox{USD}/\mbox{JPY}$為替レート過程を表す$S^{1}$と
	$\mbox{EUR}/\mbox{JPY}$為替レート過程を表す$S^{2}$が
	\begin{align}
		dS^{1}_{t} &= S^{1}_{t}(\sigma_{1}dW^{1}_{t} + \mu_{1}dt), \quad S^{1}_{0} > 0, \\
		dS^{2}_{t} &= S^{2}_{t}(\sigma_{2}dW^{2}_{t} + \mu_{2}dt), \quad S^{2}_{0} > 0, 
	\end{align}
	と与えられている(すなわち,時刻$t$で$1\mbox{USD}$が$S^{1}_{t}$円であり,
	$\mbox{1EUR}$が$S^{2}_{t}$円である).ただし$\sigma_{1},\sigma_{2} > 0,\ 
	\mu_{1},\mu_{2} \in \mathbb{R}$である.このとき$\mbox{USD}/\mbox{EUR}$為替レート過程を
	計算し,そのボラティリティと期待収益率を求めよ.
\end{itembox}

\begin{ans}
	$\mbox{USD}/\mbox{EUR}$為替レート過程は$S^{1}/S^{2}$を計算すればよい.
	\begin{align}
		d\begin{bmatrix}
			X^{1}_{t} \\
			X^{2}_{t} \\
		\end{bmatrix} = 
		\begin{bmatrix}
			S^{1}_{t} & 0 \\
			0 & S^{2}_{t} \\
		\end{bmatrix}
		\left(
		\begin{bmatrix}
			\sigma_{1} & 0 \\
			0 & \sigma_{2} \\
		\end{bmatrix}
		\begin{bmatrix}
			dW^{1}_{t} \\
			dW^{2}_{t} \\
		\end{bmatrix}
		+ 
		\begin{bmatrix}
			\mu_{1} \\
			\mu_{2} \\
		\end{bmatrix}
		dt
		\right)
	\end{align}
	なので,伊藤の公式より
	\begin{align}
		\frac{S^{1}_{t}}{S^{2}_{t}} 
		&= \frac{S^{1}_{0}}{S^{2}_{0}} + \int_{0}^{t} \frac{1}{S^{2}_{s}}\ dS^{1}_{s}
		+ \int_{0}^{t} \frac{-S^{1}_{s}}{(S^{2}_{s})^{2}}\ dS^{2}_{s} \\
		&\quad + \frac{1}{2}\int_{0}^{t} \mbox{tr}\left(
		\begin{bmatrix}
			0 & \frac{-1}{(S^{2}_{s})^{2}} \\
			\frac{-1}{(S^{2}_{s})^{2}} & \frac{2S^{1}_{s}}{(S^{2}_{s})^{3}} \\
		\end{bmatrix}
		\begin{bmatrix}
			S^{1}_{s}\sigma_{1} & 0 \\
			0 & S^{2}_{s}\sigma_{2} \\
		\end{bmatrix}
		\begin{bmatrix}
			S^{1}_{s}\sigma_{1} & 0 \\
			0 & S^{2}_{s}\sigma_{2} \\
		\end{bmatrix}
		\right)ds \\
		&= \frac{S^{1}_{0}}{S^{2}_{0}} + \int_{0}^{t} \frac{1}{S^{2}_{s}}\ dS^{1}_{s}
		+ \int_{0}^{t} \frac{-S^{1}_{s}}{(S^{2}_{s})^{2}}\ dS^{2}_{s}
		+ \int_{0}^{t} \frac{S^{1}_{s}}{S^{2}_{s}} \sigma_{2}^{2}\ ds \\
		&= \frac{S^{1}_{0}}{S^{2}_{0}} + 
		\int_{0}^{t} \frac{1}{S^{2}_{s}}\ S^{1}_{s}(\sigma_{1}dW^{1}_{s} + \mu_{1}ds) \\
		&\quad+ \int_{0}^{t} \frac{-S^{1}_{s}}{(S^{2}_{s})^{2}}\ S^{2}_{s}(\sigma_{2}dW^{2}_{s} + \mu_{2}ds) \\
		&\quad+ \int_{0}^{t} \frac{S^{1}_{s}}{S^{2}_{s}} \sigma_{2}^{2}\ ds \\
		&= \frac{S^{1}_{0}}{S^{2}_{0}} + 
		\int_{0}^{t}\frac{S^{1}_{s}}{S^{2}_{s}}\sigma_{1}\ dW^{1}_{s}
		+ \int_{0}^{t}\frac{-S^{1}_{s}}{S^{2}_{s}}\sigma_{2}\ dW^{2}_{s}
		+ \int_{0}^{t}\frac{S^{1}_{s}}{S^{2}_{s}}(\mu_{1}-\mu_{2}+\sigma_{2}^{2})\ ds
	\end{align}
	が成り立つ.これが$\mbox{USD}/\mbox{EUR}$為替レート過程の式である.従って
	\begin{align}
		d\frac{S^{1}_{t}}{S^{2}_{t}} = 
		\frac{S^{1}_{0}}{S^{2}_{0}}\left\{\sigma_{1}dW^{1}_{t} - \sigma_{2}dW^{2}_{t} +
		(\mu_{1} - \mu_{2} + \sigma_{2}^{2})dt\right\}
	\end{align}
	となる.時点$t$での期待収益率は
	\begin{align}
		\operatorname{E}_{t}\left[d(S^{1}_{t}/S^{2}_{t})/(S^{1}_{t}/S^{2}_{t})\right]
		&= \operatorname{E}_{t}\left[\sigma_{1}dW^{1}_{t} - \sigma_{2}dW^{2}_{t} +
		(\mu_{1} - \mu_{2} + \sigma_{2}^{2})dt\right] \\
		&= (\mu_{1} - \mu_{2} + \sigma_{2}^{2})dt
	\end{align}
	である.また時点$t$でのボラティリティは
	\begin{align}
		&\operatorname{E}_{t}\left[\left\{d(S^{1}_{t}/S^{2}_{t})/(S^{1}_{t}/S^{2}_{t})
		- (\mu_{1} - \mu_{2} + \sigma_{2}^{2})dt\right\}^{2}\right] \\
		&= \operatorname{E}_{t}\left[\left\{\sigma_{1}dW^{1}_{t} - \sigma_{2}dW^{2}_{t} \right\}^{2}\right] \\
		&= \operatorname{E}_{t}\left[\sigma_{1}^{2}(dW^{1}_{t})^{2}\right]
		-2\sigma_{1}\sigma_{2} \operatorname{E}_{t}\left[dW^{1}_{t}dW^{2}_{t}\right]
		+ \operatorname{E}_{t}\left[\sigma_{2}^{2}(dW^{2}_{t})^{2}\right] \\
		&= (\sigma_{1}^{2} + \sigma_{2}^{2})dt
	\end{align}
	である($W^{1}$と$W^{2}$は独立なので
		$\operatorname{E}_{t}\left[dW^{1}_{t}dW^{2}_{t}\right]=
		\operatorname{E}_{t}\left[dW^{1}_{t}\right]
		\operatorname{E}_{t}\left[dW^{2}_{t}\right]=0$).
\end{ans}

\end{ans}
\end{document}
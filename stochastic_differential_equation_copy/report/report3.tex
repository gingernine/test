\section{独立性}
	任意の有界実連続関数$h:\R \rightarrow \R$と$\mathcal{F}/\borel{\R}$-可測関数$X$に対して
	その合成$h(X)$は可積分であるから,条件付き期待値を作用させることができる.これを用いて独立性を次で定義する.
	
	\begin{screen}
		\begin{dfn}[確率変数と$\sigma$-加法族との独立性]
			Xを$\mathcal{F}/\borel{\R}$-可測関数,$\mathcal{G}$を$\mathcal{F}$の部分$\sigma$-加法族とする.
			任意の有界実連続関数$h:\R \rightarrow \R$に対し次が成り立つとき,$X$と$\mathcal{G}$は独立である(independent)と定める:
			\begin{align}
				\cexp{h(X)}{\mathcal{G}} = \Exp{h(X)} \quad \mbox{$\mu$-a.s.}
			\end{align}
		\end{dfn}
	\end{screen}
	
	\begin{screen}
		\begin{prp}[独立性の同値条件]
			任意の有界実連続関数$h:\R \rightarrow \R$に対し
			\begin{align}
				\cexp{h(X)}{\mathcal{G}} = \Exp{h(X)} \quad \mbox{$\mu$-a.s.}
				\label{eq:prp_equivalent_condition_of_independence_1}
			\end{align}
			が成り立つことと
			\begin{align}
				\mu\left( X^{-1}(E) \cap A \right) = \mu\left( X^{-1}(E) \right)\mu(A) \quad (\forall E \in \borel{\R},\ A \in \mathcal{G})
				\label{eq:prp_equivalent_condition_of_independence_2}
			\end{align}
			が成り立つことは同値である.
		\end{prp}
	\end{screen}
	
	\begin{prf}\mbox{}\\
		\begin{description}
			\item[(\refeq{eq:prp_equivalent_condition_of_independence_1})$\Rightarrow$(\refeq{eq:prp_equivalent_condition_of_independence_2})] 
				示したいことは
				\begin{align}
					\borel{\R} = \left\{\ E \in \borel{\R}\quad |\quad \prob{X^{-1}(E) \cap A} = \prob{X^{-1}(E)}\prob{A}, \quad \forall A \in \mathcal{G}\ \right\}
				\end{align}
				である.証明の手順としてまず上式右辺がDynkin族となることを示し,次に$\R$の閉集合系が右辺に含まれることを示せば,
				Dynkin族定理より上式が成立することが判る.右辺を
				\begin{align}
					\mathscr{D} \coloneqq \left\{\ E \in \borel{\R}\quad |\quad \prob{X^{-1}(E) \cap A} = \prob{X^{-1}(E)}\prob{A}, \quad \forall A \in \mathcal{G}\ \right\}
				\end{align}
				とおいて表示を簡単にしておく.$\mathscr{D}$がDynkin族であることを示すには次の3条件を確認すればよい:
				\begin{description}
					\item[\rm{(i).}] $\R \in \mathscr{D}$,
					\item[\rm{(ii).}] $D_1,D_2 \in \mathscr{D},\ D_1 \subset D_2\quad \Rightarrow\quad D_2 - D_1 \in \mathscr{D}$,
					\item[\rm{(iii).}] $D_n \in \mathscr{D},\ D_n \cap D_m = \emptyset\ (n \neq m)\quad \Rightarrow\quad \sum_{n=1}^{\infty} D_n \in \mathscr{D}$
				\end{description}
				(i)について,$\Omega = X^{-1}(\R)$により$\prob{X^{-1}(\R) \cap A} = \prob{A} = \prob{X^{-1}(\R)}\prob{A}\ (\forall A \in \mathcal{G})$であるから$\R \in \mathscr{D}$となる.
				(ii)について,
				\begin{align}
					&\prob{X^{-1}(D_2 - D_1) \cap A} = \prob{X^{-1}(D_2) \cap A} - \prob{X^{-1}(D_1) \cap A} \\
					&\qquad = \left( \prob{X^{-1}(D_2)} - \prob{X^{-1}(D_1)} \right)\prob{A}
					= \prob{X^{-1}(D_2 - D_1)}\prob{A},\quad (\forall A \in \mathcal{G})
				\end{align}
				により$D_2 - D_1 \in \mathscr{D}$となる.(iii)について,
				\begin{align}
					&\prob{X^{-1}(\textstyle\sum_{n=1}^{\infty} D_n) \cap A} = \prob{\textstyle\sum_{n=1}^{\infty} X^{-1}(D_n) \cap A} = \textstyle\sum_{n=1}^{\infty} \prob{X^{-1}(D_n) \cap A} \\
					&\qquad = \textstyle\sum_{n=1}^{\infty} \prob{X^{-1}(D_n)}\prob{A} = \prob{X^{-1}(\textstyle\sum_{n=1}^{\infty} D_n)}\prob{A},\quad (\forall A \in \mathcal{G})
				\end{align}
				により$\sum_{n=1}^{\infty} D_n \in \mathscr{D}$となる.以上で$\mathscr{D}$がDynkin族であることが判った.次に$\R$の閉集合系が
				$\mathscr{D}$に含まれることを示す.$E$を$\R$の任意の閉集合とする.
				\begin{align}
					d(\cdot,E) : \R \ni x \longmapsto \inf{}{\{\ |x-y|\quad |\quad y \in E\ \}}
				\end{align}
				として集合との距離の関数を表せばこれは$\R \rightarrow \R$の実連続関数であり,
				\begin{align}
					h_n : \R \ni x \longmapsto \frac{1}{1 + nd(x,E)} \quad (n=1,2,3,\cdots)
				\end{align}
				は有界実連続関数となる.$E$が閉集合であるから$\lim\limits_{n \to \infty} h_n(x) = \defunc_E(x) \ (\forall x \in \R)$であり,また
				$h_n$の有界連続性から$h_n \circ X \in \Lp{1}{\mathcal{F},\mu}\ (n=1,2,3,\cdots)$であることに注意すれば,任意の$A \in \mathcal{G}$に対して
				\begin{align}
					\prob{X^{-1}(E) \cap A} &= \int_{\Omega} \defunc_E(X(\omega)) \defunc_A(\omega)\ \prob{d\omega} \\
					&= \lim_{n \to \infty} \int_{\Omega} h_n(X(\omega)) \defunc_A(\omega)\ \prob{d\omega} 
						&& (\scriptsize \because \mbox{Lebesgueの収束定理})\\
					&= \lim_{n \to \infty} \int_{\Omega} \cexp{h_n(X)}{\mathcal{G}}(\omega) \defunc_A(\omega)\ \prob{d\omega} 
						&& (\scriptsize\because \mbox{定理\ref{conditional_exp_expansion}[$\tilde{\rm{C}}$2]}) \\
					&= \lim_{n \to \infty} \Exp{h_n(X)} \int_{\Omega} \defunc_A(\omega)\ \prob{d\omega} 
						&& (\scriptsize\because \mbox{(1)より.$\cexp{h_n(X)}{\mathcal{G}}\neq\Exp{h_n(X)}$の部分は積分に影響しない.}) \\
					&= \prob{A}  \lim_{n \to \infty} \int_{\Omega} h_n(X(\omega))\ \prob{d\omega} 
						&& (\scriptsize\because \mbox{定理\ref{conditional_exp_expansion}[$\tilde{\rm{C}}$1]}) \\
					&= \prob{A} \int_{\Omega} \defunc_E(X(\omega))\ \prob{d\omega} 
						&& (\scriptsize\because \mbox{Lebesgueの収束定理})\\
					&= \prob{X^{-1}(E)}\prob{A}
				\end{align}
				が成り立ち,$\R$の閉集合系が$\mathscr{D}$に含まれることが示された.閉集合系は乗法族であり$\borel{\R}$を生成するから(2)が成り立つと判明する.
			
			\item[(\refeq{eq:prp_equivalent_condition_of_independence_2})$\Rightarrow$(\refeq{eq:prp_equivalent_condition_of_independence_1})]
				任意の$A \in \mathcal{G}$に対し
				\begin{align}
					\int_{\Omega} \defunc_A(\omega) \cexp{h(X)}{\mathcal{G}}(\omega)\ \prob{d\omega} = \prob{A} \Exp{h(X)} = \int_{\Omega} \defunc_A(\omega) \Exp{h(X)} \prob{d\omega}
				\end{align}
				が成立することをいえばよい.有界実連続関数$h:\R \rightarrow \R$が非負であるとして$h$の単関数近似を考えると,例えば
				\begin{align}
					&E_{n}^{j} \coloneqq \left\{\ x \in \R\quad |\quad \frac{j-1}{3^n} \leq h(x) < \frac{j}{3^n}\ \right\}, \quad (j = 1,2,\cdots,n3^n-1,\ n = 1,2,3,\cdots) \\
					&E_{n}^{n3^n} \coloneqq \left\{\ x \in \R\quad |\quad n \leq h(x)\ \right\}
				\end{align}
				として
				\begin{align}
					h_n(x) = \sum_{j=1}^{n3^n} \frac{j}{3^n} \defunc_{E_n^j} (x) \quad (\forall x \in \R,\ n = 1,2,3,\cdots)
				\end{align}
				とおけば,$h$が可測$\borel{\R}/\borel{\R}$であるから$E_n^j$は全てBorel集合であり,任意の$A \in \mathcal{G}$に対して
				\begin{align}
					\int_{\Omega} \defunc_A(\omega) \cexp{h(X)}{\mathcal{G}}(\omega)\ \prob{d\omega}
					&= \int_{\Omega} \defunc_A(\omega) h(X(\omega))\ \prob{d\omega} 
						&& (\scriptsize\because \mbox{定理\ref{conditional_exp_expansion}[$\tilde{\rm{C}}$2]}) \\
					&= \lim_{n \to \infty} \int_{\Omega} \defunc_A(\omega) h_n(X(\omega))\ \prob{d\omega} 
						&& (\scriptsize\because \mbox{積分の定義}) \\
					&= \lim_{n \to \infty} \sum_{j=1}^{n3^n} \int_{\Omega} \defunc_A(\omega) \defunc_{X^{-1}(E_n^j)}(\omega)\ \prob{d\omega} \\
					&= \lim_{n \to \infty} \sum_{j=1}^{n3^n} \prob{X^{-1}(E_n^j) \cap A} \\
					&= \lim_{n \to \infty} \sum_{j=1}^{n3^n} \prob{X^{-1}(E_n^j)}\prob{A} 
						&& (\scriptsize\because \mbox{(2)より.}) \\
					&= \prob{A} \lim_{n \to \infty} \sum_{j=1}^{n3^n} \int_{\Omega} \defunc_{X^{-1}(E_n^j)}(\omega)\ \prob{d\omega} \\
					&= \prob{A} \lim_{n \to \infty} \int_{\Omega} h_n(X(\omega))\ \prob{d\omega} \\
					&= \prob{A} \int_{\Omega} h(X(\omega))\ \prob{d\omega} \\
					&= \prob{A} \Exp{h(X)}
				\end{align}
				が成り立つ.一般の有界実連続関数$h$についても$\{\ x \in \R\quad |\quad h(x) \geq 0\ \}$の部分と$\{\ x \in \R\quad |\quad h(x) < 0\ \}$の部分に
				分解して上と同様に考えれば
				\begin{align}
					\int_{\Omega} \defunc_A(\omega) \cexp{h(X)}{\mathcal{G}}(\omega)\ \prob{d\omega} = \prob{A} \Exp{h(X)} \quad (\forall A \in \mathcal{G})
				\end{align}
				となる.従って$\mathcal{G}$の元$\{\ \omega \in \Omega\quad |\quad \cexp{h(X)}{\mathcal{G}}(\omega) > \Exp{h(X)}\ \}$も
				$\{\ \omega \in \Omega\quad |\quad \cexp{h(X)}{\mathcal{G}}(\omega) < \Exp{h(X)}\ \}$も$\operatorname{P}$-零集合でなくてはならず,
				\begin{align}
					\cexp{h(X)}{\mathcal{G}} = \Exp{h(X)} \quad \operatorname{P}\mathrm{-a.s.}
				\end{align}
				が示された.		
		\end{description}
		\QED
	\end{prf}
	
	\begin{qst}
		$A,B \in \mathcal{F}$に対し,$X = \defunc_A$,$\mathcal{G} = \{ \emptyset,\ \Omega,\ B,\ B^c\}$とする.この時
		\begin{align}
			\mbox{$X$と$\mathcal{G}$が独立}\quad \Leftrightarrow\quad \prob{A \cap B} = \prob{A}\prob{B}.
		\end{align}
	\end{qst}
	
	\begin{prf}
		$\forall E \in \borel{\R}$に対して
		\begin{align}
			X^{-1}(E) =
			\begin{cases}
				\emptyset & (0,1 \notin E) \\
				A & (0 \notin E,\ 1 \in E) \\
				A^c & (0 \in E,\ 1 \notin E) \\
				\Omega & (0, 1 \in E)
			\end{cases}
		\end{align}
		であることに注意する.$\Rightarrow$については前命題より成り立ち,$\Leftarrow$については
		\begin{align}
			\prob{A \cap B} = \prob{A}\prob{B}
		\end{align}
		ならば
		\begin{align}
			\prob{A^c \cap B} = \prob{B} -\prob{A}\prob{B}  = (1 - \prob{A})\prob{B} = \prob{A^c}\prob{B}
		\end{align}
		も成り立ち,$A$を$\Omega,\ \emptyset$にしても上の等式は成り立つから,前命題により$X$と$\mathcal{G}$が独立であると判る.
		\QED
	\end{prf}
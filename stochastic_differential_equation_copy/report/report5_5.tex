	\begin{screen}
		\begin{lem}[停止時刻で停めた二次変分]
			$\sigma,\tau$を停止時刻とし,$\sigma \geq \tau$を満たしていると仮定する.
			このとき任意の$M \in \mathcal{M}_{c,loc}$に対し$N \coloneqq M^\sigma - M^\tau$
			と定めれば,$N \in \mathcal{M}_{c,loc}$かつ
			\begin{align}
				\inprod<N>_t = \inprod<M>_{t\wedge\sigma} - \inprod<M>_{t \wedge \tau}
				\quad (\forall t \in I,\ \mbox{$\mu$-a.s.})
			\end{align}
			が成り立つ.特に$\tau = 0$の場合,任意の停止時刻$\sigma$に対し
			\begin{align}
				\inprod<M^\sigma>_t = \inprod<M>_{t \wedge \sigma}
				\quad (\forall t \in I,\ \mbox{$\mu$-a.s.})
			\end{align}
			の関係が得られる.
			\label{lem:stopped_quadratic_variation}
		\end{lem}
	\end{screen}
	
	\begin{prf}\mbox{}
		\begin{description}
			\item[第一段] $M \in \mathcal{M}_{b,c}$の場合を考える.
				まず$N \in \mathcal{M}_{b,c}$が成り立つことを示す.実際
				$M$がマルチンゲールであるから任意の$\omega \in \Omega$に対し
				$t \longmapsto N_t(\omega)$
				は各点で右連続且つ左極限を持ち,
				さらに定理\ref{thm:boundedness_of_stopped_process_of_bounded_martingale}より
				$|N_t| \leq |M_{t\wedge\sigma}| + |M_{t\wedge\tau}| \leq 2 \sup{t \in I}{\Norm{M_t}{\mathscr{L}^\infty}}\ \mu$-a.s.
				が成り立つから
				\begin{align}
					\sup{t \in I}{\Norm{N_t}{\mathscr{L}^\infty}}
					\leq 2 \sup{t \in I}{\Norm{M_t}{\mathscr{L}^\infty}} < \infty
				\end{align}
				が満たされる.そして任意に$s,t \in I\ (s < t)$を取れば,
				命題\ref{prp:properties_of_expanded_conditional_expectation}と
				任意抽出定理より
				\begin{align}
					\cexp{N_t}{\mathcal{F}_s}
					= \cexp{M_{t \wedge \sigma} - M_{t \wedge \tau}}{\mathcal{F}_s}
					= M_{s \wedge \sigma} - M_{s \wedge \tau}
					= N_s
				\end{align}
				が成り立ち,$\mu$-a.s.に$M = 0$であるから$N = 0\ \mu$-a.s.も従う.
				次に
				\begin{align}
					\inprod<N>_t = \inprod<M>_{t \wedge \sigma} - \inprod<M>_{t \wedge \sigma}
					\quad (\forall t \in I,\ \mbox{$\mu$-a.s.})
					\label{eq:lem_stopped_quadratic_variation}
				\end{align}
				が成り立つことを示す.
				
				補題\ref{lem:quadratic_variation}の$(\tau^n_j)_{j=0}^{2^n}$と
				任意の$\omega \in \Omega$に対し或る$i,k\ (i \leq k)$が存在して
				\begin{align}
					&\tau_i^n \leq \tau(\omega) \leq \tau_{i+1}^n \leq \sigma(\omega) \leq \tau_{i+2}^n, \label{eq:stopped_quadratic_variation_1} \\
					&\tau_i^n \leq \tau(\omega) \leq \tau_{i+1}^n < \tau^n_k \leq \sigma(\omega) \leq \tau_{k+1}^n, \label{eq:stopped_quadratic_variation_2} \\
					&\tau_i^n \leq \tau(\omega) \leq \sigma(\omega) \leq \tau_{i+1}^n, \label{eq:stopped_quadratic_variation_3}
				\end{align}
				のいずれかを満たす.
				\begin{align}
					Q^n_t(N) \coloneqq \sum_{j=0}^{2^n-1} \left( N_{t \wedge \tau_{j+1}^n} - N_{t \wedge \tau_j^n} \right)^2 \quad (\forall t \in I)
				\end{align}
				とおき,同様に$Q^n(M)$も定める.$\omega$が(\refeq{eq:stopped_quadratic_variation_1})を満たしている場合,
				\begin{align}
					Q^n_t(N)(\omega) &= \sum_{j=0}^{2^n-1} \left( N_{t \wedge \tau_{j+1}^n}(\omega) - N_{t \wedge \tau_j^n}(\omega) \right)^2 \\
					&= \sum_{j=0}^{2^n-1} \left( M_{t \wedge \sigma(\omega) \wedge \tau_{j+1}^n}(\omega) - M_{t \wedge \tau(\omega) \wedge \tau_{j+1}^n}(\omega) - M_{t \wedge \sigma(\omega) \wedge \tau_j^n}(\omega) + M_{t \wedge \tau(\omega) \wedge \tau_j^n}(\omega)  \right)^2 \\
					&= \left( M_{t \wedge \sigma(\omega)}(\omega) - M_{t \wedge \tau_{i+1}^n}(\omega) \right)^2
						+ \left( M_{t \wedge \tau_{i+1}^n}(\omega) - M_{t \wedge \tau(\omega)}(\omega) \right)^2, \\
					Q^n_{t \wedge \sigma(\omega)}(M)(\omega) &= \sum_{j=0}^{2^n-1} \left( M_{t \wedge \sigma(\omega) \wedge \tau_{j+1}^n}(\omega) - M_{t \wedge \sigma(\omega) \wedge \tau_j^n}(\omega) \right)^2 \\
					&= \sum_{j=0}^{i} \left( M_{t \wedge \tau_{j+1}^n}(\omega) - M_{t \wedge \tau_j^n}(\omega) \right)^2 + \left( M_{t \wedge \sigma(\omega)}(\omega) - M_{t \wedge \tau_{i+1}^n}(\omega) \right)^2, \\
					Q^n_{t \wedge \tau(\omega)}(M)(\omega) &= \sum_{j=0}^{2^n-1} \left( M_{t \wedge \tau(\omega) \wedge \tau_{j+1}^n}(\omega) - M_{t \wedge \tau(\omega) \wedge \tau_j^n}(\omega) \right)^2 \\
					&= \sum_{j=0}^{i-1} \left( M_{t \wedge \tau_{j+1}^n}(\omega) - M_{t \wedge \tau_j^n}(\omega) \right)^2 + \left( M_{t \wedge \tau(\omega)}(\omega) - M_{t \wedge \tau_{i}^n}(\omega) \right)^2
				\end{align}
				と表せるから
				\begin{align}
					&\left| Q^n_t(N)(\omega) - \left( Q^n_{t\wedge\sigma(\omega)}(M)(\omega) - Q^n_{t\wedge\tau(\omega)}(M)(\omega) \right) \right| \\
					&\qquad \leq 
						\left( M_{t \wedge \tau_{i+1}^n}(\omega) - M_{t \wedge \tau(\omega)}(\omega) \right)^2
						+ \left( M_{t \wedge \tau_{i+1}^n}(\omega) - M_{t \wedge \tau_i^n}(\omega) \right)^2 
						+ \left( M_{t \wedge \tau(\omega)}(\omega) - M_{t \wedge \tau_{i}^n}(\omega) \right)^2 \\
					&\qquad \leq 3 \sup{|t - s| \leq T/2^n}{|M_t(\omega) - M_s(\omega)|^2}
				\end{align}
				が成り立つ.同様にして$\omega$が(\refeq{eq:stopped_quadratic_variation_2})或は
				(\refeq{eq:stopped_quadratic_variation_3})を満たしている場合も
				\begin{align}
					\left| Q^n_t(N)(\omega) - \left( Q^n_{t\wedge\sigma(\omega)}(M)(\omega) - Q^n_{t\wedge\tau(\omega)}(M)(\omega) \right) \right|
					\leq 3 \sup{|t - s| \leq T/2^n}{|M_t(\omega) - M_s(\omega)|^2}
				\end{align}
				が成り立つから,つまり全ての$\omega \in \Omega$に対し
				\begin{align}
					\left| Q^n_t(N)(\omega) - \left( Q^n_{t\wedge\sigma(\omega)}(M)(\omega) - Q^n_{t\wedge\tau(\omega)}(M)(\omega) \right) \right|
					\leq 3 \sup{|t - s| \leq T/2^n}{|M_t(\omega) - M_s(\omega)|^2}
					\label{eq:stopped_quadratic_variation_4}
				\end{align}
				が満たされる.更に$t \longmapsto M_t$が$\mu$-a.s.に一様連続であるから,或る零集合$A_1$が存在して
				\begin{align}
					3 \sup{|t - s| \leq T/2^n}{|M_t(\omega) - M_s(\omega)|^2}
					\longrightarrow 0 \quad (n \longrightarrow \infty,\ \omega \in \Omega \backslash A_1)
					\label{eq:stopped_quadratic_variation_5}
				\end{align}
				が従う.また定理\ref{thm:existence_of_quadratic_variation}の証明中の
				(\refeq{eq:thm_quadratic_variation_1})より,$\left( Q^n(M) \right)_{n=1}^{\infty}$の或る凸結合列
				$\left( \hat{Q}^n(M) \right)_{n=1}^{\infty}$,或る零集合$A_2$及び或る部分添数列$(n_k)_{k=1}^{\infty}$
				が存在して,全ての$t \in I$と$\omega \in \Omega \backslash A_2$に対し
				\begin{align}
					\hat{Q}^{n_k}_t(M)(\omega) \longrightarrow \inprod<M>_t(\omega) \quad (k \longrightarrow \infty)
				\end{align}
				を満たす.従って全ての$t \in I,\ \omega \in \Omega \backslash A_2$に対し
				\begin{align}
					\hat{Q}^{n_k}_{t\wedge\sigma(\omega)}(M)(\omega) \longrightarrow \inprod<M>_{t\wedge\sigma(\omega)}(\omega),
					\quad \hat{Q}^{n_k}_{t\wedge\tau(\omega)}(M)(\omega) \longrightarrow \inprod<M>_{t\wedge\tau(\omega)}(\omega) 
					\quad (k \longrightarrow \infty)
					\label{eq:stopped_quadratic_variation_6}
				\end{align}
				も成り立つ.各$n \in \N$についての凸結合を
				\begin{align}
					\hat{Q}^n(M) = \sum_{j=0}^{\infty} c_j^n Q^{n+j}(M),
					\quad \left( \sum_{j=0}^{\infty} c_j^n = 1 \right)
				\end{align}
				と表し,
				\begin{align}
					\hat{Q}^n(N) \coloneqq \sum_{j=0}^{\infty} c_j^n Q^{n+j}(N)
				\end{align}
				とおけば,(\refeq{eq:stopped_quadratic_variation_4})より全ての$\omega \in \Omega$に対し
				\begin{align}
					&\left| \hat{Q}^n_t(N)(\omega) - \left( \hat{Q}^n_{t\wedge\sigma(\omega)}(M)(\omega) - \hat{Q}^n_{t\wedge\tau(\omega)}(M)(\omega) \right) \right| \\
					&\qquad = \left| \sum_{j=0}^{\infty} c_j^n Q^{n+j}_t(N)(\omega) - \left( Q^{n+j}_{t\wedge\sigma(\omega)}(M)(\omega) - Q^{n+j}_{t\wedge\tau(\omega)}(M)(\omega) \right) \right| \\
					&\qquad \leq \sum_{j=0}^{\infty} c_j^n \left| Q^{n+j}_t(N)(\omega) - \left( Q^{n+j}_{t\wedge\sigma(\omega)}(M)(\omega) - Q^{n+j}_{t\wedge\tau(\omega)}(M)(\omega) \right) \right| \\
					&\qquad \leq 3 \sup{|t - s| \leq T/2^n}{|M_t(\omega) - M_s(\omega)|^2}
					\quad (n=1,2,\cdots)
				\end{align}
				が成り立つから,(\refeq{eq:stopped_quadratic_variation_5})と(\refeq{eq:stopped_quadratic_variation_6})より
				全ての$t \in I,\ \omega \in \Omega \backslash (A_1 \cup A_2)$に対して
				\begin{align}
					\left| \inprod<M>_{t\wedge\sigma(\omega)}(\omega) - \inprod<M>_{t\wedge\tau(\omega)}(\omega) - \hat{Q}^{n_k}_t(N)(\omega) \right|
					\longrightarrow 0 \quad (n \longrightarrow \infty)
				\end{align}
				が成り立つ.このとき
				\begin{align}
					\inprod<N>_t = \inprod<M>_{t\wedge\sigma} - \inprod<M>_{t\wedge\tau}
					\quad (\forall t \in I,\ \mbox{$\mu$-a.s.})
				\end{align}
				が成立する.
			
				定理\ref{thm:existence_of_quadratic_variation}より$M$に対し
				二次変分$\inprod<M>$が存在して
				\begin{align}
					N_t - \left( \inprod<M>_{t \wedge \sigma} - \inprod<M>_{t \wedge \tau} \right)
					= \left( M - \inprod<M> \right)_{t \wedge \sigma} - \left( M - \inprod<M> \right)_{t \wedge \tau}
					\quad (\forall t \in I)
				\end{align}
				と表せる.$M - \inprod<M> \in \mathcal{M}_{b,c}$であるから,
				上と同様にして$\left( M - \inprod<M> \right)^\sigma - \left( M - \inprod<M> \right)^\tau \in \mathcal{M}_{b,c}$
				が成り立つ.また全ての$\omega \in \Omega$に対し
				$\inprod<M>_0(\omega) = 0$が満たされ,更に$\sigma \geq \tau$により全ての$\omega \in \Omega$に対し
				$I \ni t \longmapsto \inprod<M>_{t \wedge \sigma(\omega)}(\omega) - \inprod<M>_{t \wedge \tau(\omega)}(\omega)$
				は連続且つ単調非減少である.従って定理\ref{thm:existence_of_quadratic_variation}の意味での二次変分の一意性より
				(\refeq{eq:lem_stopped_quadratic_variation})が得られる.
				
			\item[第二段] 
				$M \in \mathcal{M}_{c,loc}$のとき,或る$(\tau_j)_{j=0}^{\infty} \in \mathcal{T}$が存在して
				$M^{\tau_j} \in \mathcal{M}_{b,c}$となる.全ての$j \in \N$で
				\begin{align}
					N^{\tau_j}_t = M^{\tau_j}_{t \wedge \sigma} - M^{\tau_j}_{t \wedge \tau}
					\quad (\forall t \in I)
				\end{align}
				が成り立つから,前段の考察より各$j$に対し$N^{\tau_j},\left( N^{\tau_j} \right)^2 - \inprod<N^{\tau_j}> \in \mathcal{M}_{b,c}$且つ
				\begin{align}
					\inprod<N^{\tau_j}>_t = \inprod<M^{\tau_j}>_{t \wedge \sigma} - \inprod<M^{\tau_j}>_{t \wedge \tau}
					\quad (\forall t \in I,\ \mbox{$\mu$-a.s.})
				\end{align}
				が満たされる.一方$N \in \mathcal{M}_{c,loc}$であるから,定理\ref{thm:existence_of_quadratic_variation}より
				$\inprod<N>$が存在して$\left( N^2 - \inprod<N> \right)^{\tau_j} 
				= \left( N^{\tau_j} \right)^2 - \inprod<N>^{\tau_j} \in \mathcal{M}_{b,c}\ (j=0,1,\cdots)$も満たされ,
				二次変分の一意性より各$j$について
				\begin{align}
					\inprod<N^{\tau_j}>_t = \inprod<N>_{t \wedge \tau_j}
					\quad (\forall t \in I,\ \mbox{$\mu$-a.s.})
				\end{align}
				が成り立ち,同じ理由で
				\begin{align}
					\inprod<M^{\tau_j}>_t = \inprod<M>_{t \wedge \tau_j}
					\quad (\forall t \in I,\ \mbox{$\mu$-a.s.})
				\end{align}
				も得られる.すなわち或る零集合$A_j,B_j,C_j\ (j=0,1,\cdots)$が存在して
				\begin{align}
					\inprod<N^{\tau_j}>_t(\omega) &= \inprod<M^{\tau_j}>_{t \wedge \sigma(\omega)}(\omega) - \inprod<M^{\tau_j}>_{t \wedge \tau(\omega)}(\omega)
					&& (\forall t \in I,\ \omega \in \Omega \backslash A_j), \\
					\inprod<N^{\tau_j}>_t(\omega) &= \inprod<N>_{t \wedge \tau_j(\omega)}(\omega)
					&& (\forall t \in I,\ \omega \in \Omega \backslash B_j), \\
					\inprod<M^{\tau_j}>_t(\omega) &= \inprod<M>_{t \wedge \tau_j(\omega)}(\omega)
					&& (\forall t \in I,\ \omega \in \Omega \backslash C_j)
				\end{align}
				が満たされるから,
				\begin{align}
					A \coloneqq \bigcup_{j=0}^{\infty} A_j,
					\quad B \coloneqq \bigcup_{j=0}^{\infty} B_j,
					\quad C \coloneqq \bigcup_{j=0}^{\infty} C_j,
					\quad E \coloneqq A \cup B \cup C
				\end{align}
				として零集合$E$を定めれば
				\begin{align}
					\inprod<N>_t(\omega) = \inprod<M>_{t\wedge\sigma(\omega)}(\omega) - \inprod<M>_{t \wedge \tau(\omega)}(\omega)
					\quad (\forall t \in I,\ \omega \in \Omega \backslash E)
				\end{align}
				が従い主張を得る.
				\QED
				
			
		\end{description}
	\end{prf}
	
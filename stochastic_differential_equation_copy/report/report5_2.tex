\section{二次変分}
	以降では$I \coloneqq [0,T]\ (T>0)$とし,
	このフィルトレーション$(\mathcal{F}_t)_{t \in I}$が次の仮定を満たす:
	\begin{align}
		\mathcal{N} \coloneqq \Set{N \in \mathcal{F}}{\mu(N) = 0}
		\subset \mathcal{F}_0.
	\end{align}
	
	\begin{screen}
		\begin{dfn}[停止時刻で停めた過程]
			任意の停止時刻$\tau$と確率過程$M$に対し
			\begin{align}
				M^\tau_t \coloneqq M_{t \wedge \tau}
				\quad (\forall t \in I)
			\end{align}
			として定義する$M^\tau$を,停止時刻$\tau$で停めた過程という.
			
		\end{dfn}
	\end{screen}
	
	\begin{screen}
		\begin{prp}[停めた過程の適合性]
			確率過程$M$の全てのパスが右連続且つ$(\mathcal{F}_t)$-適合のとき,任意の停止時刻$\tau$に対し
			$M^\tau$もまた右連続且つ$(\mathcal{F}_t)$-適合である.
		\end{prp}
	\end{screen}
	
	\begin{prf}
		任意に$\omega \in \Omega$を取り固定する.
		写像$t \longmapsto M_t^\tau(\omega) = M_{t \wedge \tau(\omega)}(\omega)$
		について,$t < \tau(\omega)$なら$t \longmapsto M_t(\omega)$の右連続性により,
		$t \geq \tau(\omega)$なら右側で定数関数となるから右連続性が従う.また
		定理\ref{thm:measurability_of_stopping_time}より
		$M_t^{\tau}$は可測$\mathcal{F}_{t \wedge \tau}/\borel{\R}$
		であるから,$\mathcal{F}_{t\wedge \tau} \subset \mathcal{F}_t$より$M^\tau$の適合性が従う.
		\QED
	\end{prf}
	
	以下,いくつか集合を定義する.
	\begin{description}
		\item[$\mathrm{(1)}\ \mathcal{A}^+$] 
			$\mathcal{A}^+$は以下を満たす$(\Omega,\mathcal{F},\mu)$上の可測関数族$A = (A_t)_{t \in I}$の全体である.
			\begin{description}
				\item[適合性] 任意の$t \in I$に対し,写像$\Omega \ni \omega \longmapsto A_t(\omega) \in \R$は可測$\mathcal{F}_t/\borel{\R}$である.
				\item[連続性] $\mu$-a.s.に写像$I \ni t \longmapsto A_t(\omega) \in \R$が連続である.
				\item[単調非減少性] $\mu$-a.s.に写像$I \ni t \longmapsto A_t(\omega) \in \R$が単調非減少である.
			\end{description}
		
		\item[$\mathrm{(2)}\ \mathcal{A}$]
			$\mathcal{A} \coloneqq \Set{A^1 - A^2}{A^1,A^2 \in \mathcal{A}^+}$
			と定義する.$\mu$-a.s.に写像$t \longmapsto A^1_t(\omega)$と$t \longmapsto A^2_t(\omega)$が連続かつ単調非減少となるから
			すなわち$\mu$-a.s.に写像$t \longmapsto A^1_t(\omega) - A^2_t(\omega)$は有界連続となっている.
			
		\item[$\mathrm{(3)}\ \mathcal{M}_{p,c}\ (p \geq 1)$]
			$\mathcal{M}_{p,c}$は以下を満たす可測関数族$M = (M_t)_{t \in I} \subset \semiLp{p}{\mathcal{F},\mu}$の全体である.
			\begin{description}
				\item[0出発] $M_0 = 0 \quad \mbox{$\mu$-a.s.}$を満たす.
				\item[$\mathrm{L}^p$-マルチンゲール] $M = (M_t)_{t \in I}$は$\mathrm{L}^p$-マルチンゲールである.
				\item[連続性] $\mu$-a.s.$\omega$に対し写像$I \ni t \longmapsto M_t(\omega) \in \R$が連続である.
			\end{description}
		
		\item[$\mathrm{(4)}\ \mathcal{M}_{b,c}$]
			$\mathcal{M}_{b,c}$は$\mu$-a.s.に連続で一様有界な$\mathrm{L}^1$-マルチンゲールの全体とする.つまり
			\begin{align}
				\mathcal{M}_{b,c} \coloneqq \Set{M = (M_t)_{t \in I} \in \mathcal{M}_{1,c}}{\sup{t \in I}{\Norm{M_t}{\mathscr{L}^\infty}} < \infty}
			\end{align}
			として定義されている.
			
		\item[$\mathrm{(5)}\ \mathcal{T}$]
			$\mathcal{T}$は以下を満たすような,$I$に値を取る停止時刻の列$(\tau_j)_{j=0}^{\infty}$の全体とする.
			\begin{description}
				\item[a)] $\tau_0 = 0 \quad \mbox{$\mu$-a.s.}$
				\item[b)] $\tau_j \leq \tau_{j+1} \quad \mbox{$\mu$-a.s.}\ (j=1,2,\cdots).$
				\item[c)] $(\tau_j)_{j=1}^{\infty}$に対し或る$\mu$-零集合$N_T$が存在し,任意の$\omega \in \Omega \backslash N_T$に対し或る$n = n(\omega) \in \N$が存在して$\tau_n(\omega)=T$が成り立つ.
			\end{description}
			例えば$\tau_j = jT/2^n$なら$(\tau_j)_{j=0}^{\infty} \in \mathcal{T}$となる.
			上の条件において,$a)$が零集合$N_0$を除いて成立し,$b)$が各$j$について零集合$N_j$を除いて成立するとき,
			$N \coloneqq N_0 \cup N_T \cup (\cup_{j=0}^{\infty}N_j)$とすればこれも$\mu$-零集合であり,任意の$\omega \in \Omega \backslash N$に対して
			\begin{align}
				\tau_0(\omega) = 0,\quad \tau_j(\omega) \leq \tau_{j+1}(\omega)\ (j=1,2,\cdots),\quad
				\tau_n(\omega) = T\ (\exists n = n(\omega) \in \N)
			\end{align}
			が成立する.
			
		\item[$\mathrm{(4)}\ \mathcal{M}_{c,loc}$]
			$\mathcal{M}_{c,loc}$の元を「連続な局所マルチンゲール」という.$\mathcal{M}_{c,loc}$は次で定義される:
			\begin{align}
				\mathcal{M}_{c,loc} \coloneqq 
				\Set{M = (M_t)_{t \in I} \subset \semiLp{1}{\mathcal{F},\mu}}{\substack{\mbox{全ての$\omega \in \Omega$に対し,$I \ni t \longmapsto M_t(\omega)$が} \\ \mbox{各点$t$で右連続且つ左極限を持ち,} \\ \mbox{或る$(\tau_j)_{j=0}^{\infty} \in \mathcal{T}$が存在して} \\ \mbox{$M^{\tau_j} \in \mathcal{M}_{b,c}\ (\forall j =0,1,\cdots)$を満たす.}}}.
			\end{align}
	\end{description}
	
	\begin{screen}
		\begin{thm}[有界なマルチンゲールを停止時刻で停めた過程の有界性]
			$\tau$を任意の停止時刻とする.任意の$M \in \mathcal{M}_{b,c}$に対し
			$\sup{t \in I}{\Norm{M_t^\tau}{\mathscr{L}^\infty}} \leq \sup{t \in I}{\Norm{M_t}{\mathscr{L}^\infty}}$
			が成り立つ.
			\label{thm:boundedness_of_stopped_process_of_bounded_martingale}
		\end{thm}
	\end{screen}
	
	\begin{prf}
		任意に$s \in I$を取り固定する.
		$|M_{s \wedge \tau(\omega)}(\omega)| \leq \sup{t \in I}{|M_t(\omega)|}\ (\forall \omega \in \Omega)$より
		\begin{align}
			\sup{t \in I}{|M_t|} \leq \sup{t \in I}{\Norm{M_t}{\mathscr{L}^\infty}}
			\quad \mbox{$\mu$-a.s.}
		\end{align}
		が成り立つことを示せばよい.$M \in \mathcal{M}_{b,c}$であるから,
		或る零集合$A$が存在して全ての$\omega \in \Omega \backslash A$に対し
		$I \ni t \longmapsto M_t(\omega)$が連続である.
		また補題\ref{lem:holder_inequality}より
		\begin{align}
			B_r \coloneqq \Set{\omega \in \Omega}{|M_r(\omega)| > \Norm{M_r}{\mathscr{L}^\infty}},
			\quad (r \in \Q \cap I)
		\end{align}
		は全て零集合であるから,
		\begin{align}
			B \coloneqq \bigcup_{r \in \Q \cap I} B_r
		\end{align}
		に対し$C \coloneqq A \cup B$として零集合を定める.
		任意の$\omega \in \Omega \backslash C$に対し
		$t \longmapsto M_t(\omega)$が連続であるから
		\begin{align}
			|M_u(\omega)| \leq \sup{t \in I}{\Norm{M_t}{\mathscr{L}^\infty}}
			\quad (\forall u \in I)
		\end{align}
		が成り立ち
		\begin{align}
			\sup{t \in I}{|M_t(\omega)|} \leq \sup{t \in I}{\Norm{M_t}{\mathscr{L}^\infty}}
			\quad (\forall \omega \in \Omega \backslash C)
		\end{align}
		が従う.
		\QED
	\end{prf}
	
	\begin{screen}
		\begin{thm}[停止時刻で停めてもマルチンゲール]
			$p > 1$とする.任意の$M \in \mathcal{M}_{p,c}$と停止時刻$\tau$に対し,
			$M$を$\tau$で停めた過程について$M^\tau \in \mathcal{M}_{p,c}$が成り立つ.
			\label{thm:stopped_process_martingale}
		\end{thm}
	\end{screen}
	
	\begin{prf}
		任意の$\omega \in \Omega$に対し$I \ni t \longmapsto M_t(\omega)$は各点で右連続且つ左極限を持つから
		$I \ni t \longmapsto M^{\tau}_t(\omega)$も各点で右連続且つ左極限を持ち,
		特に$I \ni t \longmapsto M_t(\omega)$が連続となる$\omega$に対しては
		$I \ni t \longmapsto M^{\tau}_t(\omega)$も連続である.
		また$M_0 = M^{\tau}_0$より$M^{\tau}_0 = 0\ \mu$-a.s.が従う.あとは$M^{\tau}$の
		$p$乗可積分性と$\cexp{M^\tau_t}{\mathcal{F}_s} = M^\tau_s\ (s < t)$を示せばよい.
		実際Doobの不等式(定理\ref{thm:Doob_inequality_2})より$\sup{t \in I}{|M_t|^p}$
		が可積分であるから,全ての$t \in I$に対し$\left| M^\tau_t \right|$は$p$乗可積分であり,
		更に任意抽出定理(定理\ref{thm:optional_sampling_theorem_2})より
		\begin{align}
			\cexp{M^\tau_t}{\mathcal{F}_s} = M_{t \wedge \tau \wedge s} = M^\tau_s \quad (s < t)
		\end{align}
		が従う.
		\QED
	\end{prf}
	
	\begin{screen}
		\begin{lem}[$\mathcal{M}_{p,c}$は線形空間となる]\mbox{}\\
			任意の$M,N \in \mathcal{M}_{p,c}$と$\alpha \in \R$に対して線型演算を
			\begin{align}
				(M + N)_t \coloneqq (M_t + N_t)_{t \in I}, 
				\quad (\alpha M)_t \coloneqq (\alpha M_t)_{t \in I}
				\label{eq:mart_linear_arithmetic_0}
			\end{align}
			として定義すれば,$\mathcal{M}_{p,c}$は$\R$上の線形空間となる.
		\end{lem}
	\end{screen}
	
	\footnotetext{
		全ての$t,\omega$に対し$0 \in \R$を取るもの.
	}
	
	\begin{prf}
		$\mathcal{M}_{p,c}$が(\refeq{eq:mart_linear_arithmetic_0})の演算について閉じていることを示す.
		\begin{description}
			\item[加法について]
				先ず$M+N$が$\mathrm{L}^p$-マルチンゲールの定義を満たすことを確認する.
				適合性について,各$t \in I$について$M_t,N_t$は$\mathcal{F}_t$-可測であるから
				$M_t + N_t$も$\mathcal{F}_t$-可測であり,またMinkowskiの不等式より$M_t + N_t \in \semiLp{p}{\mu}$であることも従う.
				任意の$\omega \in \Omega$でパス$I \ni t \longmapsto M_t(\omega) + N_t(\omega)$が右連続且つ左極限を持つ
				ことも$M,N$のパスが右連続且つ左極限を持つことにより従い,さらに
				任意に$0 \leq s \leq t \leq T$を取れば,条件付き期待値の線型性より
				\begin{align}
					\cexp{M_t + N_t}{\mathcal{F}_s} = \cexp{M_t}{\mathcal{F}_s} + \cexp{N_t}{\mathcal{F}_s} = M_s + N_s
				\end{align}
				も成り立つ.以上より$M+N = (M_t + N_t)_{t \in I}$は$\mathrm{L}^p$-マルチンゲールである.
				次に写像$I \ni t \longmapsto M_t(\omega) + N_t(\omega) \in \R$の連続性を示す.
				$M,N$に対して或る$\mu$-零集合$E$が存在し,$\omega \notin E$について
				$t \longmapsto M_t(\omega)$と$t \longmapsto N_t(\omega)$が共に連続となるから
				$t \longmapsto M_t(\omega) + N_t(\omega)$も連続となる.以上で$M+N \in \mathcal{M}_{p,c}$が示された.
			
			\item[スカラ倍について]
				任意の$0 \leq s \leq t \leq T$に対し,条件付き期待値の線型性(性質$\tilde{\mathrm{C}}3$)により
				\begin{align}
					\cexp{\alpha M_t}{\mathcal{F}_s} = \alpha \cexp{M_t}{\mathcal{F}_s} = \alpha M_s
				\end{align}
				が成り立つ.定数倍しているだけであるから,$\alpha M$が
				$\mathrm{L}^p$-マルチンゲールであるためのその他の条件,及び$\mu$-a.s.にパスが連続であることも成り立ち,
				$\alpha M \in \mathcal{M}_{p,c}$となる.
		\end{description}
		\QED
	\end{prf}
	
	\begin{screen}
		\begin{lem}[$\mathcal{M}_{p,c}$における同値関係の導入]\mbox{}\\
			任意の$M,N \in \mathcal{M}_{p,c}\ (p \geq 1)$に対して,関係$R$を
			\begin{align}
				M\ R\ N \DEF \Set{\omega \in \Omega}{\sup{r \in (I \cap \Q) \cup \{ T \}}{\left|M_r(\omega) - N_r(\omega)\right| > 0}}\mbox{が$\mu$-零集合}
			\end{align}
			として定義すれば,関係$R$は同値関係となる.そして$M\ R\ N$となることと$\mu$-a.s.にパスが一致することは同値である.
			\label{lem:M_2c_hilbert}
		\end{lem}
	\end{screen}
	
	\begin{prf}
		反射律と対称律は$R$の定義式より判然しているから推移律について確認する.$M,N$とは別に$U=(U_t)_{t \in I} \in \mathcal{M}_{p,c}$
		を取って$M\ R\ N$かつ$N\ R\ U$となっているとすれば,各$r \in (I \cap \Q) \cup \{ T \}$にて
		\begin{align}
			\left\{\ \left|M_r - U_r\right| > 0\ \right\}\ \subset\ 
			\left\{\ \left|M_r - N_r\right| > 0\ \right\} \cup \left\{\ \left|N_r - U_r\right| > 0\ \right\}
		\end{align}
		の関係が成り立っているから$M\ R\ U$が従う.
		\footnote{
			$\left\{\ \left|M_r - N_r\right| > 0\ \right\} = \Set{\omega \in \Omega}{\left|M_r(\omega) - N_r(\omega)\right| > 0}.$
		}
		後半の主張を示す.
		$M,N$に対し或る零集合$E$が存在して,$\omega \in \Omega \backslash E$に対し
		$I \ni t \longmapsto M_t(\omega)$と$I \ni t \longmapsto N_t(\omega)$は共に連続写像となっている.
		今$M\ R\ N$であるとする.
		\begin{align}
			F \coloneqq \Set{\omega \in \Omega}{\sup{r \in (I \cap \Q) \cup \{ T \}}{\left|M_r(\omega) - N_r(\omega)\right| > 0}}
		\end{align}
		とおけば$F^c \cap E^c$上で$M$と$N$のパスは完全に一致し,また$F \cup E$が零集合であるから$\mu$-a.s.にパスが一致しているということになる.
		逆に$\mu$-a.s.にパスが一致しているとすれば,或る零集合$G$が存在して$G^c$上でパスが一致している.
		\begin{align}
			G^c \subset F^c
		\end{align}
		の関係から$F \subset G$となり$F$が零集合となるから$M\ R\ N$が従う.
		\QED
	\end{prf}
	
	\begin{screen}
		\begin{lem}[$\mathcal{M}_{p,c}$の商空間の定義]
			補題\ref{lem:M_2c_hilbert}で導入した同値関係$R$による$\mathcal{M}_{p,c}\ (p \geq 1)$の商集合を$\mathfrak{M}_{p,c}$と表記する.
			$M \in \mathcal{M}_{p,c}$の関係$R$による同値類を$\overline{M}$と表記し,
			$\mathfrak{M}_{p,c}$において
			\begin{align}
				\overline{M} + \overline{N} \coloneqq \overline{M+N}, \quad \alpha \overline{M} \coloneqq \overline{\alpha M} \label{eq:mart_linear_arithmetic}
			\end{align}
			として演算を定義すれば,これは代表元の選び方に依らない(well-defined).そして(\refeq{eq:mart_linear_arithmetic})で定義した算法を加法とスカラ倍として
			$\mathfrak{M}_{p,c}$は$\R$上の線形空間となる.
		\end{lem}
	\end{screen}
	
	\begin{prf}
		$M' \in \overline{M},\ N' \in \overline{N}$を任意に選べば,
		\begin{align}
			\left\{\ \left|M_r + N_r - M'_r - N'_r \right| > 0\ \right\} &\subset \left\{\ \left|M_r - M'_r \right| > 0\ \right\} \cup \left\{\ \left|N_r - N'_r \right| > 0\ \right\} \\
			\left\{\ \left|\alpha M_r - \alpha M'_r \right| > 0\ \right\} &= \left\{\ \left|M_r - M'_r \right| > 0\ \right\}
		\end{align}
		により$(M+N)\ R\ (M'+N'),\ (\alpha M)\ R\ (\alpha M')$が成り立ち
		\begin{align}
			\overline{M+N} = \overline{M'+N'}, \quad \overline{\alpha M} = \overline{\alpha M'}
		\end{align}
		が従う.
		\QED
	\end{prf}
	
	\begin{screen}
		\begin{lem}[$\mathfrak{M}_{2,c}$における内積の定義]
			写像\footnotemark
			$\inprod<\cdot,\cdot>:\mathfrak{M}_{2,c} \times \mathfrak{M}_{2,c} \rightarrow \R$
			を次で定義すれば,これは$\mathfrak{M}_{2,c}$において内積となる:
			\begin{align}
				\inprod<\overline{M},\overline{N}> \coloneqq \int_{\Omega} M_T(\omega)N_T(\omega)\ \mu(d\omega), \quad (\overline{M},\overline{N} \in \mathfrak{M}_{2,c}).
				\label{eq:M_2c_inner_product}
			\end{align}
			\label{lem:M_2c_hilbert_inner_product}
		\end{lem}
	\end{screen}
			
	\footnotetext{
		実数値として確定することは,$M_T,N_T$が共に二乗可積分であることとH\Ddot{o}lderの不等式による.
	}
			
	\begin{prf}\mbox{}
		\begin{description}
			\item[well-definedであること]
				先ずは上の$\inprod<\cdot,\cdot>$の定義が代表元の取り方に依らないことを確認する.
				$M' \in \overline{M}$と$N' \in \overline{N}$に対して,
				同値関係の定義から$\mu$-a.s.に$M'_T = M_T,\ N'_T = N_T$であり
				\begin{align}
					\int_{\Omega} M_T(\omega)N_T(\omega)\ \mu(d\omega) = \int_{\Omega} M'_T(\omega)N'_T(\omega)\ \mu(d\omega)
				\end{align}
				が成り立つから,$\inprod<\overline{M},\overline{N}>$は一つの値に確定している.
				次に$\inprod<\cdot,\cdot>$が内積であることを証明する.
	
			\item[正値性]
				先ず任意の$\overline{M} \in \mathfrak{M}_{2,c}$に対して$\inprod<\overline{M},\overline{M}> = \Norm{M_T}{\mathscr{L}^2}^2 \geq 0$が成り立つ.
				次に$\inprod<\overline{M},\overline{M}> = 0 \quad \Leftrightarrow \quad \overline{M} = \overline{0}$
				が成り立つことを示す.$\inprod<\cdot,\cdot>$の定義により$\Leftarrow$は判然しているから,$\Rightarrow$について示す.
				$M$は$\mathrm{L}^2$-マルチンゲールであるから,Jensenの不等式より
				$(|M_t|)_{t \in I}$が$\mathrm{L}^2$-劣マルチンゲールとなる.Doobの不等式を適用すれば
				\begin{align}
					\int_{\Omega} \left( \sup{t \in I}{|M_t(\omega)|} \right)^2\ \mu(d\omega) \leq 4 \int_{\Omega} {M_T(\omega)}^2\ \mu(d\omega) = 0
				\end{align}
				が成り立ち,
				\begin{align}
					\left\{\ \sup{t \in I}{|M_t|} > 0\ \right\} = \left\{\ \sup{t \in I}{|M_t|^2} > 0\ \right\} = \left\{\ \left(\sup{t \in I}{|M_t(\omega)|}\right)^2 > 0\ \right\}
				\end{align}
				%\footnote{
				%	$\left\{\ \sup{t \in I}{|M_t|} > 0\ \right\}$は$\Set{\omega \in \Omega}{\sup{t \in I}{|M_t(\omega)|} > 0}$の略記(他も同様)であるが,
				%	ここの等号は次の関係が成立することにより正当化される:
				%	\begin{align}
				%		[\sup{t \in I}{|M_t(\omega)|}]^2 = \sup{t \in I}{[M_t(\omega)]^2},\ (\forall \omega \in \Omega).
				%	\end{align}
				%	もし$[\sup{t \in I}{|M_t(\omega)|}]^2 > \sup{t \in I}{[M_t(\omega)]^2} \eqqcolon \beta$
				%	とすると,$\sup{t \in I}{|M_t(\omega)|} > \beta^{1/2}$より或る$s \in I$について
				%	$|M_s(\omega)| > \beta^{1/2}$が成り立つから,$[M_s(\omega)]^2 > \beta$となり$\beta = \sup{t \in I}{[M_t(\omega)]^2}$に矛盾する.
				%	逆の場合,つまり$\alpha \coloneqq [\sup{t \in I}{|M_t(\omega)|}]^2 < \sup{t \in I}{[M_t(\omega)]^2}$
				%	が成り立っているとしても,或る$z \in I$が存在して$\alpha^{1/2} < |M_z(\omega)| \leq \sup{t \in I}{|M_t(\omega)|}$が成り立ち,
				%	$\alpha < [\sup{t \in I}{|M_t(\omega)|}]^2 = \alpha$となり矛盾ができた.
				%}
				であるから
				\begin{align}
					\mu\left( \sup{t \in I}{|M_t|} > 0 \right) = 0
				\end{align}
				が従う.よって$\overline{M} = \overline{0}$となる.
	
			\item[双線型性]
				双線型性は積分の線型性による.
			\end{description}
		\QED
	\end{prf}
		
	\begin{screen}
		\begin{prp}[$\mathfrak{M}_{2,c}$はHilbert空間である]
			$\mathfrak{M}_{2,c}$は補題\ref{lem:M_2c_hilbert_inner_product}で導入した$\inprod<\cdot,\cdot>$を内積としてHilbert空間となる.
		\end{prp}
	\end{screen}
			
	\begin{prf}
			内積$\inprod<\cdot,\cdot>$により導入されるノルムを$\Norm{\cdot}{}$と表記する.
			$\overline{M^{(n)}} \in \mathfrak{M}_{2,c}\ (n=1,2,\cdots)$をCauchy列として取れば,
			各代表元$M^{(n)}$に対し或る$\mu$-零集合$E_n$が存在して,$\omega \in \Omega \backslash E_n$なら
			写像$I \ni t \longmapsto M^{(n)}_t(\omega) \in \R$が連続となる.後で連続関数列の一様収束を扱うからここで次の処理を行う:
			\begin{align}
				E \coloneqq \bigcup_{n=1}^{\infty} E_n
			\end{align}
			として,$M^{(n)} = (M^{(n)}_t)_{t \in I}$を零集合$E$上で修正した過程$(N^{(n)}_t)_{t \in I}$を
			\begin{align}
				N^{(n)}_t(\omega) \coloneqq
				\begin{cases}
					M^{(n)}_t(\omega) & (\omega \in \Omega \backslash E) \\
					0 & (\omega \in E)
				\end{cases}
				,\quad (\forall n = 1,2,\cdots,\ t \in I)
			\end{align}
		として定義すれば,$N^{(n)}$は$\Omega$全体でパスが連続,かつ$\mathrm{L}^2$-マルチンゲールであるから
		\footnote{
			$\mathrm{L}^2$-マルチンゲールとなることを証明する.
			パスの右連続性と左極限の存在は連続性により成り立つことである.適合性については,フィルトレーションの仮定より$E \in \mathcal{F}_0$であることに注意すれば,
			$N^{(n)}_t = M^{(n)}_t \defunc_{\Omega \backslash E}$
			であることと$M^{(n)}_t$が適合過程であることから$N^{(n)}_t$も可測$\mathcal{F}_t/\borel{\R}$となる.
			また各$t \in I$に対し$N^{(n)}_t$と$M^{(n)}_t$の関数類は一致するから,任意に$0 \leq s \leq t \leq T$を取って
			\begin{align}
				\cexp{N^{(n)}_t}{\mathcal{F}_s} = \cexp{M^{(n)}_t}{\mathcal{F}_s} = M^{(n)}_s = N^{(n)}_s
			\end{align}
			が成り立つ.$N^{(n)}_t = M^{(n)}_t \defunc_{\Omega \backslash E}$の二乗可積分性は$M^{(n)}_t$の二乗可積分性から従う.
		}
		$\mathcal{M}_{2,c}$の元となる.また零集合$E$を除いて$M^{(n)}$とパスが一致するから,
		$\overline{N^{(n)}} = \overline{M^{(n)}}\ (n=1,2,\cdots)$が成立し
		\begin{align}
			\Norm{\overline{M^{(n)}} - \overline{M^{(m)}}}{}^2 = \Norm{\overline{N^{(n)}} - \overline{N^{(m)}}}{}^2 
			=  \Norm{\overline{N^{(n)} - N^{(m)}}}{}^2
			= \int_{\Omega} \left| N^{(n)}_T(\omega) - N^{(m)}_T(\omega) \right|^2\ \mu(d\omega)
		\end{align}
		と表現できる.
		任意の$n,m \in N$の組に対し,$\mathcal{M}_{2,c}$が線形空間であるから
		$\left(\left|N^{(n)}_t - N^{(m)}_t\right|\right)_{t \in T}$は連続な$\mathrm{L}^2$-劣マルチンゲールとなり,
		Doobの不等式を適用して
		\begin{align}
			\lambda^2 \mu\left(\sup{t \in I}{|N^{(n)}_t - N^{(m)}_t| > \lambda}\right) \leq \int_{\Omega} \left| N^{(n)}_T(\omega) - N^{(m)}_T(\omega) \right|^2\ \mu(d\omega)
			= \Norm{\overline{M^{(n)}} - \overline{M^{(m)}}}{}^2 \quad (\forall \lambda > 0)
		\end{align}
		が成り立つ.この不等式と$\left(\overline{M^{(n)}}\right)_{n=1}^{\infty}$がCauchy列であることを併せれば,
		\begin{align}
			\Norm{\overline{M^{(n_k)}} - \overline{M^{(n_{k+1})}}}{} < 1/4^k, \quad (k = 1,2,\cdots) \label{eq:mart_hilbert_1}
		\end{align}
		となるように添数の部分列$(n_k)_{k=1}^{\infty}$を抜き出して
		\begin{align}
			\mu\left(\sup{t \in I}{|N^{(n_k)}_t - N^{(n_{k+1})}_t| > 1/2^k}\right) < 1/2^k, \quad (k=1,2,\cdots)
		\end{align}
		が成り立つようにできる.
		\begin{align}
			F \coloneqq \bigcup_{N=1}^{\infty} \bigcap_{k \geq N} \Set{\omega \in \Omega}{\sup{t \in I}{|N^{(n_k)}_t(\omega) - N^{(n_{k+1})}_t(\omega)|} \leq 1/2^k}
		\end{align}
		とおけば,Borel-Cantelliの補題により$F^c$は$\mu$-零集合であって,$\omega \in F$なら全ての$t \in I$について数列$\left(N^{(n_k)}_t(\omega)\right)_{k=1}^{\infty}$はCauchy列となる.
		実数の完備性から数列$\left(N^{(n_k)}_t(\omega)\right)_{k=1}^{\infty}\ (\omega \in F)$に極限$N^*_t(\omega)$が存在し,
		この収束は$t$に関して一様である
		\footnote{
			$\left| N^{(n_k)}_t(\omega) - N^*_t(\omega) \right| \leq \sum_{j=k}^{\infty} \left| N^{(n_j)}_t(\omega) - N^{(n_{j+1})}_t(\omega) \right|
			\leq \sum_{j=k}^{\infty} \sup{t \in I}{\left| N^{(n_j)}_t(\omega) - N^{(n_{j+1})}_t(\omega) \right|} < 1/2^k, \quad (\forall t \in T)$
			による.
		}から写像$t \longmapsto N^*_t(\omega)$は連続で,
		\begin{align}
			N_t(\omega) \coloneqq 
			\begin{cases}
				N^*_t(\omega) & (\omega \in F) \\
				0 & (\omega \in \Omega \backslash F)
			\end{cases}
		\end{align}
		として$N$を定義すればこれは$\mathcal{M}_{2,c}$の元となる.$N$は全てのパスが連続であるから$\mathrm{L}^2$-マルチンゲールとなっていることを示す.
		マルチンゲールの定義の(M.3)(M.4)はパスの連続性により従うことであるから,後は(M.1)と(M.2)を証明すればよい.
		\begin{description}
			\item[(M.1)適合性について]
				今任意に$t \in I$を取り固定する.
				$N^{n_k}_t$の定義域を$F$に制限した写像を$N^{F(k)}_t$と表記し
				\begin{align}
					\mathcal{F}^F_t \coloneqq \Set{F \cap B}{B \in \mathcal{F}_t}
				\end{align}
				とおけば,$N^{F(k)}_t$は可測$\mathcal{F}^F_t/\borel{\R}$となる.従って各点収束先の関数である$N^*_t$もまた可測$\mathcal{F}^F_t/\borel{\R}$となる.
				任意の$C \in \borel{\R}$に対して
				\begin{align}
					N^{-1}_t(C) = 
					\begin{cases}
						(\Omega \backslash F) \cup {N^*}^{-1}_t(C) & (0 \in C) \\
						{N^*}^{-1}_t(C) & (0 \notin C)
					\end{cases}
				\end{align}
				が成り立ち,フィルトレーションの仮定から$F \in \mathcal{F}_0$であり$\mathcal{F}^F_t \subset \mathcal{F}_t$が従うから,
				$N_t$は可測$\mathcal{F}_t/\borel{\R}$である.
			
			\item[(M.1)二乗可積分性について]
				任意に$t \in I$を取り固定する.$N^{(n_k)}_t\ (k=1,2,\cdots)$は二乗可積分関数$M^{(n_k)}_t$と零集合$E$を除いて一致し,
				$N_t$に概収束する.また添数列$(n_k)_{k=1}^{\infty}$の抜き出し方(\refeq{eq:mart_hilbert_1})とDoobの不等式より
				\begin{align}
					\Norm{\sup{t \in I}{\left|N^{(n_k)}_t - N^{(n_{k+1})}_t\right|}}{\mathscr{L}^2} \leq 2 \Norm{\left|N^{(n_k)}_T - N^{(n_{k+1})}_T\right|}{\mathscr{L}^2} < 2/4^k
				\end{align}
				が成り立つから$\Norm{N^{(n_k)}_t - N^{(n_{k+1})}_t}{\mathscr{L}^2} < 2/4^k \leq 1/2^k \ (k=1,2,\cdots)$を得る.
				特に$j \in \N$を固定すれば全ての$k > j$に対して$\Norm{N^{(n_j)}_t - N^{(n_k)}_t}{\mathscr{L}^2} < 1/2^j$となるから,Fatouの補題より
				\begin{align}
					\Norm{N^{(n_j)}_t - N_t}{\mathscr{L}^2}^2 = \int_{\Omega \backslash F} \liminf_{k \to \infty} \left| N^{(n_j)}_t(\omega) - N^{(n_k)}_t(\omega) \right|^2\ \mu(d\omega)
					< 1/4^j
					\label{eq:M_c2_hilbert_2}
				\end{align}
				が従い,Minkowskiの不等式より
				\begin{align}
					\Norm{N_t}{\mathscr{L}^2} \leq \Norm{N_t - N^{(n_j)}_t}{\mathscr{L}^2} + \Norm{N^{(n_j)}_t}{\mathscr{L}^2} < \infty
				\end{align}
				が成り立つ.
			
			\item[(M.2)について]
				各$t \in I,\ k \in \N$について$(M^{(n_k)}_t)_{t \in I}$が$\mathrm{L}^2$-マルチンゲールであるということを利用すればよい.
				任意の$0 \leq s \leq t \leq T$と$A \in \mathcal{F}_s$に対して
				\begin{align}
					\int_{A} \cexp{N^{(n_k)}_t}{\mathcal{F}_s}(\omega)\ \mu(d\omega) &= \int_{A} \cexp{M^{(n_k)}_t}{\mathcal{F}_s}(\omega)\ \mu(d\omega) \\
					&= \int_{A} M^{(n_k)}_s(\omega)\ \mu(d\omega) = \int_{A} N^{(n_k)}_s(\omega)\ \mu(d\omega)
				\end{align}
				が全ての$k = 1,2,\cdots$で成り立つから,H\Ddot{o}lderの不等式及び(\refeq{eq:M_c2_hilbert_2})より
				\begin{align}
					&\left| \int_{A} \cexp{N_t}{\mathcal{F}_s}(\omega)\ \mu(d\omega) - \int_{A} N_s(\omega)\ \mu(d\omega) \right| \\
					&\leq \left| \int_{A} \cexp{N_t}{\mathcal{F}_s}(\omega)\ \mu(d\omega) - \int_{A} \cexp{N^{(n_k)}_t}{\mathcal{F}_s}(\omega)\ \mu(d\omega) \right| \\
						&\qquad+ \left| \int_{A} N^{(n_k)}_s(\omega)\ \mu(d\omega) - \int_{A} N_s(\omega)\ \mu(d\omega) \right| \\
					&= \left| \int_{A} N_t(\omega) - N^{(n_k)}_t(\omega)\ \mu(d\omega) \right|
						+ \left| \int_{A} N^{(n_k)}_s(\omega) - N_s(\omega)\ \mu(d\omega) \right| \\
					&\leq \int_{A} \left| N_t(\omega) - N^{(n_k)}_t(\omega) \right|\ \mu(d\omega)
						+ \int_{A} \left| N^{(n_k)}_s(\omega) - N_s(\omega) \right|\ \mu(d\omega) \\
					&\leq \Norm{N_t - N^{(n_k)}_t}{\mathscr{L}^2} + \Norm{N^{(n_k)}_s - N_s}{\mathscr{L}^2} \\
					&\leq 1/2^{k-1}
				\end{align}
				が全ての$k = 1,2,\cdots$で成り立つ.$k$の任意性から
				\begin{align}
					\int_{A} \cexp{N_t}{\mathcal{F}_s}(\omega)\ \mu(d\omega)
					= \int_{A} N_s(\omega)\ \mu(d\omega)
				\end{align}
				が従い,$\cexp{N_t}{\mathcal{F}_s} = N_s \quad \mbox{in $\Lp{2}{\mathcal{F},\mu}$}$となる.
		\end{description}
	
		最後に,$N$の$\mathfrak{M}_{2,c}$における同値類$\overline{N}$がCauchy列$\left(\overline{M^{(n)}}\right)_{n=1}^{\infty}$の極限であるということを明示して証明を完全に終える.
		部分列$\left(\overline{M^{(n_k)}}\right)_{k=1}^{\infty}$に対して,(\refeq{eq:M_c2_hilbert_2})より
		\begin{align}
				\Norm{\overline{N} - \overline{M^{(n_k)}}}{} 
				= \Norm{\overline{N} - \overline{N^{(n_k)}}}{}
				= \Norm{N_T - N^{(n_k)}_T}{\mathscr{L}^2} \longrightarrow 0 \quad (k \longrightarrow \infty)
		\end{align}
		が成り立つ.部分列が収束することはCauchy列が収束することになるから$\Norm{\overline{N} - \overline{M^{(n)}}}{} \longrightarrow 0$が従い,
		$\mathfrak{M}_{2,c}$がHilbert空間であることが証明された.
		\QED
	\end{prf}

$I \coloneqq [0,T]\ (T>0)$に対して,基礎に置くフィルター付き確率空間を$(\Omega,\mathcal{F},\mu,(\mathcal{F}_t)_{t \in I})$とする.
また次を仮定する:
\begin{align}
	\mathcal{N} \coloneqq \Set{N \in \mathcal{F}}{\mu(N) = 0}
	\subset \mathcal{F}_0.
\end{align}
以後,確率過程が連続(resp.右連続,左連続)であると書く場合は全てのパスが連続(resp.右連続,左連続)であることを指す.

\section{パスの変動を制限する停止時刻}
	\begin{screen}
		\begin{dfn}[パスの変動を制限する停止時刻]
			$X=(X^1,\cdots,X^d)$を連続な$\R^d$値適合過程\footnotemark
			とする.$X$と$\epsilon > 0$に対して
			\begin{align}
				\sum_{i=1}^{d} \Norm{X_{t \wedge \tau_{j+1}}^i - X_{t \wedge \tau_j}^i}{\mathscr{L}^\infty} \leq \epsilon,
				\quad (\forall t \in I,\ j=0,1,2,\cdots)
			\end{align}
			を満たす$(\tau_j)_{j=0}^{\infty} \in \mathcal{T}$の全体を$\mathcal{T}(X,\epsilon)$と表す.
		\end{dfn}
	\end{screen}
	
	\footnotetext{
		定理\ref{thm:measurability_of_stopping_time}により,任意の停止時刻$\tau$に対し
		写像$\omega \longmapsto X^i(\tau(\omega),\omega)\ (i=1,\cdots,d)$は可測$\mathcal{F}_\tau/\borel{\R}$となり,
		$\Norm{X_{t \wedge \tau_{j+1}}^i - X_{t \wedge \tau_j}^i}{\mathscr{L}^\infty}$を考えることができる.
	}
	
	\begin{screen}
		\begin{thm}[パスの変動を制限する停止時刻の存在]
			$X=(X^1,\cdots,X^d)$を連続な$\R^d$値適合過程とする.
			任意の$\epsilon > 0$に対して,$\tau_0 = 0$かつ
			\begin{align}
				\tau_{j+1}(\omega) \coloneqq
				\inf{}{\Set{t \in I}{\left| X_t(\omega) - X_{t \wedge \tau_j(\omega)}(\omega) \right| \geq \epsilon/d}} \wedge T,
				\quad (j=0,1,2,\cdots)
			\end{align}
			と定めれば$(\tau_j)_{j=0}^{\infty} \in \mathcal{T}(X,\epsilon)$となる.
		\end{thm}
	\end{screen}
	
	\begin{prf}
		数学的帰納法で示す.$\tau_0$は定義より停止時刻であるから,$\tau_j$が停止時刻であると仮定して
		$\tau_{j+1}$が停止時刻となることを示せばよい.
		\begin{description}
			\item[第一段]
				\begin{align}
					I \ni t \longmapsto \sup{s \in [0,t]}{\left| X_s - X_{s \wedge \tau_j} \right|}
				\end{align}
				は連続であり
				\begin{align}
					\Omega \ni \omega \longmapsto \sup{s \in [0,t]}{\left| X_s(\omega) - X_{s \wedge \tau_j(\omega)}(\omega) \right|}
				\end{align}
				は全ての$t \in I$に対し可測$\mathcal{F}_t/\borel{\R}$であることを示す.
				\begin{description}
					\item[連続性]
						任意に$t \in I$を取り$t$における連続性を調べる.表記を簡単にするため,任意の$\omega \in \Omega$に対し
						\begin{align}
							f_\omega(t) \coloneqq \left| X_t(\omega) - X_{t \wedge \tau_j(\omega)}(\omega) \right| \quad (t \in I)
						\end{align}
						とおく.$f_\omega(t) < \sup{s \in [0,t]}{ f_\omega(s) }$の場合,
						$s \longmapsto f_\omega(s)$の連続性より$t$の十分小さな近傍を取っても上限は変化しない.
						$f_\omega(t) = \sup{s \in [0,t]}{f_\omega(s)}$の場合,$s < t$なら
						\begin{align}
							\sup{u \in [0,t]}{f_\omega(u)} - \sup{u \in [0,s]}{f_\omega(u)} \leq f_\omega(t) - f_\omega(s)
						\end{align}
						となり,$s \longmapsto f_\omega(s)$が連続だから左側で連続である.
						同じく$s \longmapsto f_\omega(s)$の連続性より,任意の$\delta > 0$に対し十分小さな$h > 0$を取れば,全ての$t < u < t + h$に対し
						\begin{align}
							f_\omega(u) < f_\omega(t) + \delta
						\end{align}
						とできるから 
						\begin{align}
							\sup{u \in [0,s]}{f_\omega(u)} - f_\omega(t) < \delta \quad (t < \forall s < t + h)
						\end{align}
						が成り立つ.
						
					\item[可測性]
						定理\ref{thm:measurability_of_stopping_time}により$X_{s \wedge \tau_j}$は$\mathcal{F}_{s \wedge \tau_j}$-可測であるから,
						$X$の適合性及び絶対値の連続性と併せて写像$\left| X_s - X_{s \wedge \tau_j} \right|$
						は可測$\mathcal{F}_s/\borel{\R}$,すなわち可測$\mathcal{F}_t/\borel{\R}$となる.
						\begin{align}
							D_t^n \coloneqq \Set{\frac{jt}{2^n}}{j=0,1,\cdots,2^n}
							\quad (n=1,2,\cdots)
						\end{align}
						として,$X$のパスの連続性により
						\begin{align}
							\sup{s \in [0,t]}{\left| X_s(\omega) - X_{s \wedge \tau_j(\omega)}(\omega) \right|}
							= \lim_{n \to \infty} \max{s \in D_t^n}{\left| X_s(\omega) - X_{s \wedge \tau_j(\omega)}(\omega) \right|}
							\quad (\forall \omega \in \Omega)
						\end{align}
						が成り立つから$\sup{s \in [0,t]}{\left| X_s - X_{s \wedge \tau_j} \right|}$
						も可測$\mathcal{F}_t/\borel{\R}$となる.
				\end{description}
				
			\item[第二段]
				$\tau_j$が停止時刻であるとして$\tau_{j+1}$が停止時刻であることを示す.
				前段の結果より,任意の$t \in [0,T)$に対して
				\begin{align}
					\left\{\, \tau_{j+1} \leq t\, \right\}
					= \left\{\, \sup{s \in [0,t]}{\left| X_s - X_{s \wedge \tau_j} \right|} \geq \epsilon/d\, \right\}
					\label{eq:stopping_time_restricting_path_variation}
				\end{align}
				が成り立つことをいえばよい.$t = T$の場合は$\left\{\, \tau_{j+1} \leq T\, \right\} = \Omega$である.
				\begin{align}
					\sup{s \in [0,t]}{\left| X_s(\omega) - X_{s \wedge \tau_j(\omega)}(\omega) \right|} < \epsilon/d
				\end{align}
				を満たす$\omega$について,前段で示した連続性より,或る$h > 0$が取れて
				$[0,t+h]$が$\Set{u \in I}{\left| X_u(\omega) - X_{u \wedge \tau_j(\omega)}(\omega) \right| \geq \epsilon/d}$
				の下界の集合となるから$t < \tau_{j+1}(\omega)$が従う.逆に
				\begin{align}
					\sup{s \in [0,t]}{\left| X_s(\omega) - X_{s \wedge \tau_j(\omega)}(\omega) \right|} \geq \epsilon/d
				\end{align}
				を満たす$\omega$について,パスの連続性よりsupはmaxと一致するから,或る$u \in [0,t]$で
				\begin{align}
					\left| X_u(\omega) - X_{u \wedge \tau_j(\omega)}(\omega) \right| \geq \epsilon/d
				\end{align}
				が成り立ち$\tau_{j+1}(\omega) \leq u \leq t$が従う.
				以上で(\refeq{eq:stopping_time_restricting_path_variation})が示された.
				前段で示した可測性より
				\begin{align}
					\left\{\, \tau_{j+1} \leq t\, \right\}
					= \left\{\, \sup{s \in [0,t]}{\left| X_s - X_{s \wedge \tau_j} \right|} \geq \epsilon/d\, \right\} 
					\in \mathcal{F}_t
				\end{align}
				が任意の$t \in I$に対して成立するから$\tau_{j+1}$は停止時刻である.
			
			\item[第三段]
				$(\tau_j)_{j=0}^{\infty}$が$\mathcal{T}$の元であることを示す.
				今任意に$\omega \in \Omega$を取り固定する.先ず定義より$\tau_0(\omega) = 0$は満たされている.
				また,各$j \in \N_0$について$\tau_{j+1}(\omega)$は
				\begin{align}
					\left| X_t(\omega) - X_{t \wedge \tau_j(\omega)}(\omega) \right| \geq \epsilon/d
				\end{align}
				を満たす$t$の下限であり,$t \leq \tau_j(\omega)$のときは左辺が0となるから
				\begin{align}
					\tau_j(\omega) < \tau_{j+1}(\omega) \quad (j=0,1,2,\cdots)
				\end{align}
				を満たす.
				
			\item[第四段]
				$(\tau_j)_{j=0}^{\infty} \in \mathcal{T}(X,\epsilon)$を示す.
				任意に$\omega \in \Omega$と$t \in I$を取り固定する.
				$(\tau_j)_{j=0}^{\infty}$の作り方より
				\begin{align}
					\left| X_{t \wedge \tau_{j+1}(\omega)}(\omega) - X_{t \wedge \tau_j(\omega)}(\omega) \right| \leq \epsilon/d
					\quad (j=0,1,2,\cdots)
				\end{align}
				が満たされている.実際もし或る$j$で
				\begin{align}
					\left| X_{t \wedge \tau_{j+1}(\omega)}(\omega) - X_{t \wedge \tau_j(\omega)}(\omega) \right| > \epsilon/d
				\end{align}
				が成り立つとすれば,この場合は$t \geq \tau_j(\omega)$でなくてはならないが,
				$t \geq \tau_{j+1}(\omega)$ならば,パスの連続性より$\tau_{j+1}(\omega) > s$を満たす$s$についても
				\begin{align}
					\left| X_s(\omega) - X_{\tau_j(\omega)}(\omega) \right| > \epsilon/d
				\end{align}
				が成り立ち,下限の定義から$\tau_{j+1}(\omega) > s \geq \tau_{j+1}(\omega)$が従い矛盾が生じる.
				$t < \tau_{j+1}(\omega)$のときも$\tau_{j+1}(\omega) > t \geq \tau_{j+1}(\omega)$が従い矛盾が生じる.
				よって
				\begin{align}
					\left| X^i_{t \wedge \tau_{j+1}(\omega)}(\omega) - X^i_{t \wedge \tau_j(\omega)}(\omega) \right| \leq \epsilon/d
					\quad (i=1,\cdots,d,\ j=0,1,2,\cdots)
				\end{align}
				が成り立ち,$\omega,t$の任意性から
				\begin{align}
					\sum_{i=1}^d\Norm{X^i_{t \wedge \tau_{j+1}} - X^i_{t \wedge \tau_j}}{\mathscr{L}^\infty} \leq \epsilon
					\quad (\forall t \in I,\ j=0,1,2,\cdots)
				\end{align}
				が従う.
		\end{description}
		\QED
	\end{prf}
	
	\begin{screen}
		\begin{thm}[停止時刻によるパスの変動の制限]
			$X=(X^1,\cdots,X^d)$を連続な$\R^d$値適合過程とし,任意に$\epsilon > 0$を取る.
			このとき任意の$(\tau_j)_{j=0}^{\infty} \in \mathcal{T}(X,\epsilon)$に対して或る零集合$N$が存在し,
			任意の$\omega \in \Omega \backslash N$に対して
			\begin{align}
				\tau_0(\omega) = 0,\quad \tau_j(\omega) \leq \tau_{j+1}(\omega)\ (j=1,2,\cdots),\quad
				\tau_n(\omega) = T\ (\exists n = n(\omega) \in \N)
			\end{align}
			かつ
			\begin{align}
				\sum_{i=1}^{d}\left| X_{t \wedge \tau_{j+1}(\omega)}^i(\omega) - X_{t \wedge \tau_j(\omega)}^i(\omega) \right| \leq \epsilon,
				\quad (\forall t \in I,\ j=0,1,2,\cdots)
			\end{align}
			が満たされる.
		\end{thm}
	\end{screen}
	
	\begin{prf}
		任意に$t \in I$を取り固定する.
		\begin{align}
			A_{t,j}^i \coloneqq 
			\left\{\, \left| X_{t \wedge \tau_{j+1}}^i - X_{t \wedge \tau_j}^i \right| > \Norm{X_{t \wedge \tau_{j+1}}^i - X_{t \wedge \tau_j}^i}{\mathscr{L}^\infty}\, \right\}
			\quad (i=1,\cdots,d,\ j=0,1,2,\cdots)
		\end{align}
		とおけば,補題\ref{lem:holder_inequality}により$A_{t,j}^i$は$\mu$-零集合となる.また
		\begin{align}
			A_{t,j} \coloneqq \bigcup_{i=1}^{d} A_{t,j}^i, \quad
			A_t \coloneqq \bigcup_{j=0}^{\infty} A_{t,j}
		\end{align}
		とおけば,任意の$\omega \in \Omega \backslash A_t$に対して
		\begin{align}
			\sum_{i=1}^{d}\left| X_{t \wedge \tau_{j+1}(\omega)}^i(\omega) - X_{t \wedge \tau_j(\omega)}^i(\omega) \right| \leq \epsilon,
			\quad (j=0,1,2,\cdots)
		\end{align}
		が成り立つ.更に
		\begin{align}
			A \coloneqq \bigcup_{r \in \Q \cap I} A_r
		\end{align}
		とすれば,任意の$\omega \in \Omega \backslash A$に対して
		\begin{align}
			\sum_{i=1}^{d}\left| X_{t \wedge \tau_{j+1}(\omega)}^i(\omega) - X_{t \wedge \tau_j(\omega)}^i(\omega) \right| \leq \epsilon,
			\quad (\forall t \in I,\ j=0,1,2,\cdots)
		\end{align}
		が成り立つ.実際もし或る$t \in I,\ j \in \N_0$で
		\begin{align}
			\sum_{i=1}^{d}\left| X_{t \wedge \tau_{j+1}(\omega)}^i(\omega) - X_{t \wedge \tau_j(\omega)}^i(\omega) \right| > \epsilon
		\end{align}
		となると,$t \longmapsto X_t$の連続性より或る$r \in \Q \cap I$に対して
		\begin{align}
			\sum_{i=1}^{d}\left| X_{r \wedge \tau_{j+1}(\omega)}^i(\omega) - X_{r \wedge \tau_j(\omega)}^i(\omega) \right| > \epsilon
		\end{align}
		が従うがこれは矛盾である.
		$(\tau_j)_{j=0}^{\infty} \in \mathcal{T}$でもあるから,或る$\mu$-零集合$B$が存在し,任意の$\omega \in \Omega \backslash B$に対して
		\begin{align}
			\tau_0(\omega) = 0,\quad \tau_j(\omega) \leq \tau_{j+1}(\omega)\ (j=1,2,\cdots),\quad
			\tau_n(\omega) = T\ (\exists n = n(\omega) \in \N)
		\end{align}
		を満たす.従って$N \coloneqq A \cup B$とすればよい.
		\QED
	\end{prf}
	
	
\section{Sobolev空間について}
係数体を$\K$,$\K = \R$或は$\K = \C$と考える.測度空間を$(X,\mathcal{F},m)$とする.

\begin{dfn}[絶対連続関数]
	$I \coloneqq [a,b]$を$\R$の区間とする.$I$上の関数$f:I \longrightarrow \K$が絶対連続であるとは,
	任意の$a \leq a_1 < b_1 \leq a_2 < b_2 \leq \cdots \leq a_n < b_n \leq b,\ (n = 1,2,3,\cdots)$
	と任意の$\epsilon > 0$に対し,或る$\delta > 0$が存在して
	\begin{align}
		\sum_{i = 1}^{n}(b_i - a_i) < \delta \quad \Rightarrow \quad 
		\sum_{i = 1}^{n}|f(b_i) - f(a_i)| < \epsilon
	\end{align}
	が成り立つことをいう.
\end{dfn}

\begin{thm}[絶対連続の同値条件]
	測度空間を$(X,\mathcal{F},m)$とする.
\end{thm}

\begin{dfn}[Sobolev空間]
\end{dfn}
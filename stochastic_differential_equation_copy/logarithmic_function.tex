
\begin{screen}
	\begin{dfn}[複素対数]
		\begin{align}
			\exp{z} = 1 + z + \frac{z^2}{2!} + \frac{z^3}{3!} + \cdots \quad (\forall z \in \C)
		\end{align}
		として指数関数を定める.右辺の収束半径は$\infty$であるから$e$は$\C$の整関数であり,
		いかなる$z \in \C$に対しても$\exp{z}$は0を取りえない.
		$z \in \C \backslash \{0\}$の対数を,
		\begin{align}
			\exp{w} = z
			\label{eq:complex_logarithm}
		\end{align}
		を満たす$w$であると定義して
		\begin{align}
			w = \log{z}
		\end{align}
		と表記する.$e$が0を取らないから$z = 0$の対数は定義されない.
	\end{dfn}
\end{screen}

		$z \in \C \backslash \{0\}$に対して
		\begin{align}
			\exp{w} = z
		\end{align}
		を満たす$w \in \C$を一つ取る.$\R^2$と$\C$は対応$\C \ni z \longmapsto (\Re{z} , \Im{z}) \in \R^2$により位相同型
		であるから
		\begin{align}
			w = x + i y
		\end{align}
		を満たす$x,y \in \R$の組がただ一つ存在すし,これを(\refeq{eq:complex_logarithm})に代入すれば
		\begin{align}
			z = \exp{x + i y} = \exp{x}\exp{i y}
		\end{align}
		を満たし
		\begin{align}
			w = \log{|z|} + i y 
		\end{align}
		が従う.ただし$\log{|z|}$は正実数についての実数値対数関数を表す.Eulerの公式より
		\begin{align}
			z = \exp{w} = \exp{\log{|z|} + i y} = |z| \exp{i y} = |z| (\cos{y}{} + i \sin{y}{})
			\label{eq:argument}
		\end{align}
		が成り立つが,実際$y$について$2 \pi$の整数倍の違いを許しても右辺は同じ値を表現できる.
		$z$に対し(\refeq{eq:argument})を満たす$y$の全体(集合)を$\arg{z}$と表記し$z$の{\bf 偏角}と呼ぶ.つまり
		\begin{align}
			z \longmapsto \arg{z}
		\end{align}
		は無限多価(集合値)関数の意味を持ち,従って
		\begin{align}
			z \longmapsto \log{z}
		\end{align}
		も無限多価(集合値)関数を表し
		\begin{align}
			\log{z} = \log{|z|} + i \arg{z}
		\end{align}
		と表現される.$\log{z}$を一価の関数として扱うためには偏角を制限しなくてはならない.この操作を
		{\bf 対数の枝を取る}といい,例えば偏角を$(-\pi,\pi]$に制限した複素対数を{\bf 主値}と定めて
		特別に$\Log{z}$と表記したりする.また以降は主値$\Log{z}$の虚部を特別に$\Arg{z}$と書く.
		
		\begin{screen}
			\begin{thm}[対数の主値の正則性]
				上述の通りに主値を定め,また複素平面から$0$と負の実軸を除いた領域を$\Omega$と書く\footnotemark
				.このとき
				\begin{align}
					\Omega \ni z \longmapsto \Log{z}
				\end{align}
				は$\Omega$上で正則となり
				\begin{align}
					\frac{d}{dz} \Log{z} = \frac{1}{z}
				\end{align}
				が成り立つ.
			\end{thm}
		\end{screen}
		\footnotetext{
			主値自体は$\C \backslash \{0\}$において定義されているが,
			負の実軸上にある$z$に対し,
			$\Re{z} = \Re{w_n}$かつ$\Im{z} > \Im{w_n}$を満たす点列$w_n$を
			$z$に近づけても偏角は連続となりえない.$\Arg{z} = \pi$であるが$\Arg{w_n}$は$-\pi$の付近にあるためである.
			従ってこの$z$において主値は連続ではない.以上の理由で領域$\Omega$を定めている.
			また偏角の取り方により$\Omega$の定め方は変わる.
		}
		\begin{prf}
			\begin{align}
				\Omega \ni z \longmapsto \Log{z}
			\end{align}
			は一価であり
			\begin{align}
				\Omega \ni z \longmapsto \exp{z}
			\end{align}
			の逆写像となっている.指数関数は整関数であるから,主値の連続性が判れば
			主値の正則性が従う.
		\end{prf}



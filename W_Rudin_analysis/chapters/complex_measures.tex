\section{複素測度}

	\begin{screen}
		\begin{dfn}[複素測度]
			$(X,\mathcal{M})$を可測空間とする.
			$\lambda: \mathcal{M} \rightarrow \C$が
			任意の互いに素な集合列$E_i \in \mathcal{M}\ (i=1,2,\cdots)$と
			$E \coloneqq \sum_{i=1}^{\infty} E_i$に対して
			\begin{align}
				\lambda(E) = \sum_{i=1}^{\infty} \lambda(E_i)
				\label{eq:dfn_complex_measure}
			\end{align}
			を満たすとき,$\lambda$を複素測度(complex measure)という.
		\end{dfn}
	\end{screen}
	
	$\lambda$は複素数値であるから任意の$E \in \mathcal{M}$に対して
	$|\lambda(E)| < \infty$を満たす.
	従って(\refeq{eq:dfn_complex_measure})において右辺の級数は収束していなくてはならない.
	$\sigma:\N \rightarrow \N$を任意の並び替え
	\footnote{
		$\sigma$は$\N$から$\N$への全単射である.
	}とすれば
	\begin{align}
		E = \sum_{i=1}^{\infty}E_{\sigma(i)}
	\end{align}
	が成り立つから
	\begin{align}
		\lambda(E) = \sum_{i=1}^{\infty} \lambda(E_{\sigma(i)})
	\end{align}
	を得る.従って複素数列$\left( \lambda(E_i) \right)_{i=1}^{\infty}$
	は無条件に$\lambda(E)$に収束し,Riemannの級数定理より
	$\left( \lambda(E_i) \right)_{i=1}^{\infty}$は絶対収束する:
	\begin{align}
		\sum_{i=1}^{\infty} |\lambda(E_i)| < \infty.
	\end{align}
	今,$\lambda$を支配するような或る$\mathcal{M}$上の測度$\mu$,つまり
	\begin{align}
		|\lambda(E)| \leq \mu(E) \quad (\forall E \in \mathcal{M})
		\label{radon_nikodym_1}
	\end{align}
	を満たすものを,できるだけ小さいものとして取ろうと考える
	\footnote{
		つまり(\refeq{radon_nikodym_1})を満たす$\mu$のうちから,
		同様に(\refeq{radon_nikodym_1})を満たす任意の測度$\mu'$に対し
		\begin{align}
			\mu(E) \leq \mu'(E) \quad (\forall E \in \mathcal{M})
		\end{align}
		を満たすものを選べるかどうかを考える.
	}.
	このような$\mu$は次を満たすことになる:
	\begin{align}
		\sum_{i=1}^{\infty} |\lambda(E_i)| \leq \sum_{i=1}^{\infty} \mu(E_i) = \mu(E).
	\end{align}
	ゆえに
	\begin{align}
		\mu(E) \geq \sup{}{\sum_{i=1}^{\infty} |\lambda(E_i)|} 
		\label{radon_nikodym_2}
	\end{align}
	でなくてはならず(上限は$E$のあらゆる分割$E = \sum_{i}E_i$に対して取るものである),
	ここで$\mathcal{M}$上の関数を
	\begin{align}
		|\lambda|(E) \coloneqq \sup{}{\sum_{i=1}^{\infty} |\lambda(E_i)|} \quad (\forall E \in \mathcal{M})
		\label{radon_nikodym_3}
	\end{align}
	として定義してみれば,$E$自体が$E$の一つの分割であるから(\refeq{radon_nikodym_3})より$|\lambda|$は$\lambda$を支配し,
	更に,後述することであるが,$|\lambda|$は$\mathcal{M}$上の測度でもあり(\refeq{radon_nikodym_2})と併せて当座の問題の解となる.
	
	\begin{itembox}[l]{}
		\begin{dfn}[総変動・総変動測度]
			可測空間$(X,\mathcal{M})$上の複素測度$\lambda$に対し,上で定めた
			測度$|\lambda|:\mathcal{M} \longrightarrow [0,\infty)$を
			$\lambda$の総変動測度(total variation measure)といい,$|\lambda|(X)$を
			$\lambda$の総変動(total variation)という.
			特に$\lambda$が正値有限測度である場合は$\lambda = |\lambda|$が成り立つ.\footnotemark
		\end{dfn}
	\end{itembox}
	\footnotetext{
		複素測度の虚部が0であるものとして考えれば$0 \leq \lambda(E) \leq \lambda(X) < \infty\ (\forall E \in \mathcal{M})$が成り立つ.
		また実際任意の$E \in \mathcal{M}$とその分割$(E_i)_{i=1}^{\infty}$に対して
		\begin{align}
			|\lambda|(E) = \sup{}{\sum_{i=1}^{\infty} |\lambda(E_i)|} = \sup{}{\sum_{i=1}^{\infty} \lambda(E_i)} = \sup{}{\lambda(E)} = \lambda(E)
		\end{align}
		が成り立つ.
	}
	
	以降で$|\lambda|$の性質
	\begin{description}
		\item[(1)] $|\lambda|$は測度である.
		\item[(2)] $|\lambda|(X) < \infty$が成り立つ.
	\end{description}
	を証明する.特に(2)により任意の$E \in \mathcal{M}$に対し
	\begin{align}
		|\lambda(E)| \leq |\lambda|(E) \leq |\lambda|(X) < \infty
	\end{align}
	が従うから,複素測度は有界であると判明する.

	\begin{itembox}[l]{}
		\begin{thm}[$|\lambda|$は測度となる]
			可測空間$(X,\mathcal{M})$上の複素測度$\lambda$に対し(\refeq{radon_nikodym_3})で定義する$|\lambda|$は
			$(X,\mathcal{M})$において測度となる.
		\end{thm}
	\end{itembox}
	
	\begin{prf}
		(\refeq{radon_nikodym_3})により$|\lambda|$は正値であるから,ここで示すことは$|\lambda|$が完全加法的であるということである.
		任意に$\epsilon > 0$とどの二つも互いに素な集合列$E_i \in \mathcal{M}\ (i=1,2,\cdots)$を取る.
		示すことは$E \coloneqq \sum_{i=1}^{\infty} E_i$に対して
		\begin{align}
			|\lambda|(E) = \sum_{i=1}^{\infty} |\lambda|(E_i)
		\end{align}
		が成り立つことである.(\refeq{radon_nikodym_3})により$E_i$の分割$(A_{ij})_{j=1}^{\infty} \subset \mathcal{M}$を
		\begin{align}
			|\lambda|(E_i) \geq \sum_{j=1}^{\infty} |\lambda(A_{ij})| > |\lambda|(E_i) - \epsilon/2^i
		\end{align}
		となるように取ることができる.また$E = \sum_{i,j=1}^{\infty} A_{ij}$でもあるから
		\begin{align}
			|\lambda|(E) \geq \sum_{i,j=1}^{\infty} |\lambda(A_{ij})| \geq \sum_{i=1}^{\infty}\sum_{j=1}^{\infty} |\lambda(A_{ij})| > \sum_{i=1}^{\infty} |\lambda|(E_i) - \epsilon
		\end{align}
		が成り立つ.$\epsilon > 0$は任意であるから
		\begin{align}
			|\lambda|(E) \geq \sum_{j=1}^{\infty} |\lambda|(E_j)
		\end{align}
		が従う.逆向きの不等号について,$E$の任意の分割$(A_j)_{j=1}^{\infty} \subset \mathcal{M}$に対し
		\begin{align}
			\sum_{j=1}^{\infty} |\lambda(A_j)| 
			= \sum_{j=1}^{\infty} \left| \sum_{i=1}^{\infty} \lambda(A_j \cap E_i) \right|
			\leq \sum_{j=1}^{\infty} \sum_{i=1}^{\infty} |\lambda(A_j \cap E_i)|
			\leq \sum_{i=1}^{\infty} |\lambda|(E_i)
			\footnotemark
		\end{align}
		\footnotetext{
			正項級数は和の順序に依らないから
			\begin{align}
				\sum_{j=1}^{\infty} \sum_{i=1}^{\infty} |\lambda(A_j \cap E_i)|
				= \sum_{i=1}^{\infty} \sum_{j=1}^{\infty} |\lambda(A_j \cap E_i)|
			\end{align}
			が成り立つ.これと(\refeq{radon_nikodym_3})を併せれば最後の不等号が従う.
		}
		が成り立つから,左辺の上限を取って
		\begin{align}
			|\lambda|(E) \leq \sum_{i=1} |\lambda|(E_i)
		\end{align}
		を得る.
		\QED
	\end{prf}
	
	\begin{itembox}[l]{}	
		\begin{thm}[総変動測度は有界]
			可測空間$(X,\mathcal{M})$上の複素測度$\lambda$の総変動測度$|\lambda|$について次が成り立つ:
			\begin{align}
				|\lambda|(X) < \infty.
			\end{align}
			\label{thm:total_variation_measure_bounded}
		\end{thm}
	\end{itembox}
	
	先ずは次の補題を示す.
	
	\begin{itembox}[l]{}
		\begin{lem}
			$z_1,\cdots,z_N$を複素数とする.これらの添数集合の或る部分$S \subset \{1,\cdots,N\}$を抜き取れば次が成り立つ:
			\begin{align}
				\left| \sum_{k \in S} z_k \right| \geq \frac{1}{2\pi} \sum_{k=1}^{N} |z_k|.
			\end{align}
			\label{lem:total_variation_measure_bounded}
		\end{lem}
	\end{itembox}
	
	\begin{prf}[補題]
		$z_k = |z_k|\exp{i \alpha_k}\ (-\pi \leq \alpha_k < \pi,\ k=1,\cdots,N)$となるように$\alpha_1,\cdots,\alpha_N$を取る.
		ここで$i$は虚数単位である.また$-\pi \leq \theta \leq \pi$に対し
		\begin{align}
			S(\theta) \coloneqq \Set{k \in \{1,\cdots,N\}}{\cos{(\alpha_k - \theta)}{} > 0}
		\end{align}
		とおく.このとき
		\begin{align}
			\left| \sum_{k \in S(\theta)} z_k \right| &= |\exp{-i\theta}|\left| \sum_{k \in S(\theta)} z_k \right| = \left| \sum_{k \in S(\theta)} |z_k|\exp{i(\alpha_k - \theta)} \right| \\
			&\geq \Re{\sum_{k \in S(\theta)} |z_k|\exp{i(\alpha_k - \theta)}} = \sum_{k \in S(\theta)} |z_k| \cos{(\alpha_k - \theta)}{} = \sum_{k=1}^{N} |z_k| \cos{(\alpha_k - \theta)}{+}
			\footnotemark
		\end{align}
		\footnotetext{$\cos{x}{+} = 0 \vee \cos{x}{}\ (x \in \R)$である.}
		が成り立ち,最右辺は$\theta$に関して連続となるから$[-\pi,\pi]$上で式を最大にする$\theta_0$が存在する.$S \coloneqq S(\theta_0)$とおき,$\theta_0$と任意の$\theta \in [-\pi,\pi]$に対して
		\begin{align}
			\left| \sum_{k \in S} z_k \right| \geq \sum_{k=1}^{N} |z_k| \cos{(\alpha_k - \theta_0)}{+} \geq \sum_{k=1}^{N} |z_k| \cos{(\alpha_k - \theta)}{+}
		\end{align}
		が成り立つから,左辺右辺を積分して
		\begin{align}
			\left| \sum_{k \in S} z_k \right| \geq \sum_{k=1}^{N} |z_k| \frac{1}{2\pi} \int_{[-\pi,\pi]} \cos{(\alpha_k - \theta)}{+}\ d\theta
			= \frac{1}{2\pi} \sum_{k=1}^{N} |z_k|
		\end{align}
		が成り立つ\footnote{
			最後の積分について,実際三角関数の周期性を使えば任意の$\alpha \in \R$に対して
			\begin{align}
				\int_{[-\pi,\pi]} \cos{(\alpha - \theta)}{+}\ d\theta = \int_{[\alpha-\pi,\alpha+\pi]} \cos{\theta}{+}\ d\theta = \int_{[-\pi,\pi]} \cos{\theta}{+}\ d\theta = 1
			\end{align}
			が成り立つ.
		}.
		\QED
	\end{prf}
	
	\begin{prf}[定理\ref{thm:total_variation_measure_bounded}]\mbox{}
		\begin{description}
		\item[第一段]
			或る$E \in \mathcal{M}$に対し$|\lambda|(E) = \infty$が成り立っていると仮定する.
			$t \coloneqq 2\pi(1 + |\lambda(E)|)$とおけば({\scriptsize 複素測度であるから$|\lambda(E)| < \infty$})
			$|\lambda|(E) > t$となるから,(\refeq{radon_nikodym_3})より$E$の分割$(E_i)_{i=1}^{\infty}$を
			\begin{align}
				\sum_{i=1}^{\infty} |\lambda(E_i)| > t
			\end{align}
			となるように取ることができる.従って或る$N \in \N$を取れば
			\begin{align}
				\sum_{i=1}^{N} |\lambda(E_i)| > t
			\end{align}
			が成り立つ.$z_i \coloneqq \lambda(E_i)\ (i=1,\cdots,N)$として補題\ref{lem:total_variation_measure_bounded}を使えば,
			或る$S \subset \{1,\cdots,N\}$に対し
			\begin{align}
				\left| \sum_{k \in S} \lambda(E_k) \right| \geq \frac{1}{2\pi} \sum_{k=1}^{N} |\lambda(E_k)| > \frac{t}{2\pi} > 1
			\end{align}
			となる.$A \coloneqq \sum_{k \in S} E_k$とおいて$B \coloneqq E - A$とすれば
			\begin{align}
				|\lambda(B)| = |\lambda(E)-\lambda(A)| \geq |\lambda(A)| - |\lambda(E)| > \frac{t}{2\pi} - |\lambda(E)| = 1
			\end{align}
			が成り立つから,つまり$|\lambda|(E) = \infty$の場合,$E$の直和分割$A,B$で
			\begin{align}	
				|\lambda(A)| > 1, \quad |\lambda(B)| > 1
			\end{align}
			を満たすものが取れると示された.そして$|\lambda|$の加法性から
			\begin{align}
				|\lambda|(E) = |\lambda|(A) + |\lambda|(B)
			\end{align}
			も成り立つから,この場合右辺の少なくとも一方は$\infty$となる.
		
		\item[第二段]
			背理法により定理の主張することを証明する.今$|\lambda|(X) = \infty$と仮定すると,前段の結果より$X$の或る直和分割$A_1,B_1$で
			\begin{align}
				|\lambda|(B_1) = \infty, \quad |\lambda(A_1)| > 1, \quad |\lambda(B_1)| > 1
			\end{align}
			を満たすものが取れる.$B_1$についてもその直和分割$A_2,B_2$で
			\begin{align}
				|\lambda|(B_2) = \infty, \quad |\lambda(A_2)| > 1, \quad |\lambda(B_2)| > 1
			\end{align}
			を満たすものが取れる.この操作を繰り返せば,どの二つも互いに素な集合列$(A_j)_{j=1}^{\infty}$で$|\lambda(A_j)| > 1\ (j=1,2,\cdots)$
			を満たすものを構成できる.$A \coloneqq \sum_{j=1}^{\infty}$について,$|\lambda(A)| < \infty$でなくてはならないから,
			Riemannの級数定理より
			\begin{align}
				\lambda(A) = \sum_{j=1}^{\infty} \lambda(A_j)
			\end{align}
			の右辺は絶対収束する.従って$0 < \epsilon < 1$に対し或る$N \in \N$が存在して$n > N$なら$|\lambda(A_n)| < \epsilon$
			が成り立つはずであるが,これは$|\lambda(A_n)| > 1$であることに矛盾する.
			背理法により$|\lambda|(X) < \infty$であることが示された.
		\end{description}
		\QED
	\end{prf}
	
	\begin{itembox}[l]{}
		\begin{dfn}[複素測度の空間・ノルムの定義]
			可測空間$(X,\mathcal{M})$上の複素測度の全体を$\CM$と表す.$\lambda,\mu \in \CM,\ c \in \C,\ E \in \mathcal{M}$に対し
			\begin{align}
				&(\lambda + \mu)(E) \coloneqq \lambda(E) + \mu(E), \\
				&(c\lambda)(E) \coloneqq c\lambda(E)
				\label{complex_measure_linear}
			\end{align}
			を線型演算として$\CM$は線形空間となり,特に定理\ref{thm:total_variation_measure_bounded}により
			$\lambda \in \CM$に対して$|\lambda| \in \CM$が成り立つ.
			また$\Norm{\cdot}{}:\CM \rightarrow \R$を
			\begin{align}
				\Norm{\lambda}{} \coloneqq |\lambda|(X) \quad (\lambda \in \CM)
			\end{align}
			と定義すればこれは$\CM$においてノルムとなる.
		\end{dfn}
	\end{itembox}
	
	上で定義した$\Norm{\cdot}{}$がノルムとなることを証明する.総変動の正値性からノルムの正値性が従うから,以下示すのは
	同次性と三角不等式である.
	\begin{description}
		\item[同次性]
			総変動測度の定義(\refeq{radon_nikodym_3})とスカラ倍の定義(\refeq{complex_measure_linear})より,任意の$\lambda \in \CM$と$c \in \C$に対し
			\begin{align}
				\Norm{c\lambda}{} = \sup{}{\sum_{i}|(c\lambda)(E_i)|} = \sup{}{\sum_{i}|c\lambda(E_i)|} = |c|\sup{}{\sum_{i}|\lambda(E_i)|} = |c|\Norm{\lambda}{}
			\end{align}
			が成り立つ.
			
		\item[三角不等式]
			任意の$\lambda,\mu \in \CM$に対し
			\begin{align}
				\Norm{\lambda + \mu}{} = |\lambda + \mu|(X) = \sup{}{\sum_{i} |(\lambda + \mu)(E_i)|} = \sup{}{\sum_{i} |\lambda(E_i) + \mu(E_i)|}
			\end{align}
			となるが,ここで
			\begin{align}
				\sum_{i} |\lambda(E_i) + \mu(E_i)| \leq \sum_{i} |\lambda(E_i)| + \sum_{i} |\mu(E_i)| \leq \Norm{\lambda}{} + \Norm{\mu}{}
			\end{align}
			が成り立つから
			\begin{align}
				\Norm{\lambda + \mu}{} = \sup{}{\sum_{i} |\lambda(E_i) + \mu(E_i)|} \leq \Norm{\lambda}{} + \Norm{\mu}{}
			\end{align}
			が従う.
	\end{description}
	
	\begin{itembox}[l]{}
		\begin{dfn}[正変動と負変動・Jordanの分解]
			可測空間$(X,\mathcal{M})$上の複素測度の全体を$\CM$とし,
			実数値の$\mu \in \CM$を取る({\scriptsize このような$\mu$を符号付き測度(signed measure)という}).
			\begin{align}
				\mu^+ \coloneqq \frac{1}{2}(|\mu| + \mu) , \quad \mu^- \coloneqq \frac{1}{2}(|\mu| - \mu)
			\end{align}
			とおけば$\mu^+,\mu^-$はどちらも正値有限測度となる\footnotemark
			.$\mu^+$を$\mu$の正変動(positive variation)といい$\mu^-$を$\mu$の負変動(negative variation)という.
			また
			\begin{align}
				\mu = \mu^+ - \mu^-, \quad |\mu| = \mu^+ + \mu^-
			\end{align}
			が成り立ち,ここで示した符号付き測度の正変動と負変動による表現をJordanの分解という.
		\end{dfn}
	\end{itembox}
	
	\footnotetext{
		$\mathcal{M}$上で$|\mu|(E) \geq |\mu(E)|$であることと定理\ref{thm:total_variation_measure_bounded}による.
	}
	
	\begin{itembox}[l]{}
		\begin{dfn}[絶対連続・特異]
			$(X,\mathcal{M})$を可測空間,
			$\mu$を$\mathcal{M}$上の正値測度\footnotemark
			,$\lambda,\lambda_1,\lambda_2$を$\mathcal{M}$上の任意の測度({\scriptsize 正値測度或は複素測度})とする.
			\begin{itemize}
				\item $\lambda$が$\mu$に関して絶対連続である(absolutely continuous)ということを
					\begin{align}
						\lambda \ll \mu
					\end{align}
					と書き,その意味は,「$\mu(E)=0$となる全ての$E \in \mathcal{M}$について$\lambda(E)=0$」である.
				
				\item 或る$A \in \mathcal{M}$があって$\lambda(E) = \lambda(A \cap E)\ (\forall E \in \mathcal{M})$
					が成り立っているとき,$\lambda$は$A$に集中している(concentrated on A)という.
					$\lambda_1,\lambda_2$に対し或る$A_1,A_2 \in \mathcal{M}$があって,
					$\lambda_1$が$A_1$に集中,$\lambda_2$が$A_2$に集中しかつ
					$A_1 \cap A_2 = \emptyset$を満たしているとき,これを互いに特異である(mutually singular)といい
					\begin{align}
						\lambda_1 \perp \lambda_2
					\end{align}
					と書く.
			\end{itemize}
		\end{dfn}
	\end{itembox}
	
	\footnotetext{
		正値測度という場合は$\infty$も取りうる.従って正値測度は複素測度の範疇にはない.$\mu$として例えば$k$次元Lebesgue測度を想定している.
	}
	
	\begin{itembox}[l]{}
		\begin{prp}[絶対連続性と特異性に関する性質]
			$(X,\mathcal{M})$を可測空間,$\mu$を$\mathcal{M}$上の正値測度,
			$\lambda,\lambda_1,\lambda_2$を$\mathcal{M}$上の複素測度とする.このとき以下に羅列する事柄が成り立つ.
			\begin{description}
				\item[(1)] $\lambda$が$A \in \mathcal{M}$に集中しているなら$|\lambda|$も$A$に集中している.
				\item[(2)] $\lambda_1 \perp \lambda_2$ならば$|\lambda_1| \perp |\lambda_2|$.
				\item[(3)] $\lambda_1 \perp \mu$かつ$\lambda_2 \perp \mu$ならば$\lambda_1 + \lambda_2 \perp \mu$.
				\item[(4)] $\lambda_1 \ll \mu$かつ$\lambda_2 \ll \mu$ならば$\lambda_1 + \lambda_2 \ll \mu$.
				\item[(5)] $\lambda \ll \mu$ならば$|\lambda| \ll \mu$.
				\item[(6)] $\lambda_1 \ll \mu$かつ$\lambda_2 \perp \mu$ならば$\lambda_1 \perp \lambda_2$.
				\item[(7)] $\lambda \ll \mu$かつ$\lambda \perp \mu$ならば$\lambda = 0$.
			\end{description}
			\label{prp:absolute_continuous_singular}
		\end{prp}
	\end{itembox}
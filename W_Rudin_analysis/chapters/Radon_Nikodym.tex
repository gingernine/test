\section{Lebesgue-Radon-Nikodymの定理}
	\begin{itembox}[l]{}
		\begin{lem}
			$(X,\mathcal{M})$を可測空間,$\mu$を$\mathcal{M}$上の$\sigma$-有限な正値測度とする.
			この$\mu$に対し次を満たす$w \in \semiLp{1}{\mu}$\footnotemark
			が存在する:
			\begin{align}
				0 < w(x) < 1 \quad (\forall x \in X), \quad \int_X w(x)\ \mu(dx) < 1.
			\end{align}
			\label{lem:lebesgue-radon-nikodym}
		\end{lem}
	\end{itembox}
	
	\footnotetext{
		以後測度が定義されている空間は同じだから
		$\semiLp{1}{\mu} = \semiLp{1}{X,\mathcal{M},\mu}$と略記する.
		$\mathrm{L}^p$についても同様に略記する.
	}
	
	\begin{prf}
		$\mu$は$(X,\mathcal{M})$において$\sigma$-有限であるから,
		$\mu(E_n) < \infty$かつ$X = \cup_{n=1}^\infty E_n$を満たす
		集合の系$(E_n)_{n=1}^\infty \subset \mathcal{M}$が存在する.
		\begin{align}
			w_n(x) \coloneqq
			\begin{cases}
				0 & (x \in X - E_n) \\
				\frac{1}{2^n(1+\mu(E_n))} & (x \in E_n)
			\end{cases}
			\quad (n=1,2,\cdots)
		\end{align}
		として定義した関数$w_n$はどれも可測$\mathcal{M}/\borel{\R}$であり,そして
		\begin{align}
			w(x) \coloneqq \sum_{n=1}^{\infty} w_n(x) \quad (\forall x \in X)
		\end{align}
		として$w$を定義すれば,これもまた可測$\mathcal{M}/\borel{\R}$であり$0 < w < 1$を満たす.
		実際
		\begin{align}
			\sum_{n=1}^\infty w_n(x) < \sum_{n=1}^\infty \frac{1}{2^n} = 1
		\end{align}
		となるから級数は各点$x \in X$で収束し,従って$w$の可測性が従う.また任意の$x \in X$は
		或る$E_n$に属しているから$w(x) \geq w_n(x) > 0$となり,ゆえに$0 < w < 1$も従う.
		最後に$w$が$\mu$について可積分であることを示す.これは単調収束定理より
		\begin{align}
			\int_X w(x)\ \mu(dx) = \sum_{n=1}^{\infty} \int_X w_n(x)\ \mu(dx) = \sum_{n=1}^{\infty} \frac{\mu(E_n)}{2^n(1+\mu(E_n))} < 1
		\end{align}
		が成り立つことによる.
		\QED
	\end{prf}
	
	\begin{itembox}[l]{}
		\begin{thm}[Lebesgue-Radon-Nikodym]
			$(X,\mathcal{M})$を可測空間,$\lambda$を$\mathcal{M}$上の複素測度,
			そして$\mu$を$\mathcal{M}$上の$\sigma$-有限な正値測度とする.
			このとき次の二つの主張が成り立つ:
			\begin{description}
				\item[(1)Lebesgueの分解] $\lambda$に対して或る複素測度$\lambda_a,\lambda_s$が存在し,
					\begin{align}
						\lambda(E) = \lambda_a(E) + \lambda_s(E) \quad (\forall E \in \mathcal{M})
					\end{align}
					かつ$\lambda_a \ll \mu,\ \lambda_s \perp \mu$を満たす.これをLebesgueの分解といい,
					分解は一意である.
					
				\item[(2)密度関数の存在] (1)の$\lambda_a$に対し或る$[h] \in \Lp{1}{\mu}$がただ一つだけ存在し
					\begin{align}
						\lambda_a(E) = \int_E h(x)\ \mu(dx) \quad (\forall E \in \mathcal{M})
					\end{align}
					を満たす.$\lambda_a$に対する上の関数$h$をRadon-Nikodymの密度関数という.
			\end{description}
		\end{thm}
	\end{itembox}
	
	\begin{prf}\mbox{}
	\begin{description}
	\item[分解と密度関数の一意性]
		まずはLebesgue分解について証明する.$\lambda$の分解が$\lambda_a,\lambda_s$の組と$\tilde{\lambda}_a,\tilde{\lambda}_s$
		の組の二つあるとして($\lambda_a,\tilde{\lambda}_a \ll \mu,\ \lambda_s,\tilde{\lambda}_s \perp \mu$),
		\begin{align}
			\lambda_a - \tilde{\lambda}_a = \tilde{\lambda}_s - \lambda_s
		\end{align}
		とできるから,命題\ref{prp:absolute_continuous_singular}により両辺とも$\ll \mu$かつ$\perp \mu$を満たす.
		命題\ref{prp:absolute_continuous_singular}により絶対連続性と特異性を両方持つ場合は零写像となるのだから
		\begin{align}
			\lambda_a = \tilde{\lambda}_a, \quad \lambda_s = \tilde{\lambda}_s
		\end{align}
		となる.密度関数の一意性についても,$\lambda_a$に対し$[h],[\tilde{h}] \in \Lp{1}{\mu}$が存在して
		\begin{align}
			\int_E h\ d\mu = \int_E \tilde{h}\ d\mu \quad (\forall E \in \mathcal{M})
		\end{align}
		が成り立つなら,$\mathcal{M}$の元$\{h > \tilde{h}\},\{h < \tilde{h}\}$はいずれも$\mu$測度0であるから
		$h = \tilde{h}\ \mu$-a.e.が従い$[h] = [\tilde{h}]$となる.
	\item[分解と密度関数の存在]
		正値有限測度に対して示せば十分である
		\footnote{
			Lebesgueの分解と密度関数の存在の一意性を利用すれば,正値測度の場合に示すので十分である:
			\begin{align}
				\lambda_1(E) \coloneqq \Re{\lambda(E)}, \quad \lambda_2(E) \coloneqq \Im{\lambda(E)} \quad (\forall E \in \mathcal{M})
			\end{align}
			として$\lambda$の実部と虚部を表し,$\lambda_1,\lambda_2$をそれぞれ正変動$\lambda_1^+,\lambda_2^+$と
			負変動$\lambda_1^-,\lambda_2^-$にJordan分解する.
			$\lambda_1^+,\lambda_2^+,\lambda_1^-,\lambda_2^-$のそれぞれに対しLebesgue分解と密度関数が存在するとして,
			分解を$\lambda_{1,a}^+,\lambda_{1,s}^+,\lambda_{2,a}^+,\lambda_{2,s}^+,\lambda_{1,a}^-,\lambda_{1,s}^-,\lambda_{1,a}^-,\lambda_{1,s}^-$,
			密度関数を$h_1^+,h_2^+,h_1^-,h_2^-$と表す.$\lambda$の分解について,
			\begin{align}
				\lambda
				&= \lambda_1^+ - \lambda_1^- + i\left( \lambda_2^+ - \lambda_2^- \right) \\
				&= \lambda_{1,a}^+ - \lambda_{1,a}^- + \lambda_{1,s}^+ - \lambda_{1,s}^-
					+ i\left( \lambda_{2,a}^+ - \lambda_{2,a}^- + \lambda_{2,s}^+ - \lambda_{2,s}^- \right) \\
				&= \left\{ \lambda_{1,a}^+ - \lambda_{1,a}^- + i\left( \lambda_{2,a}^+ - \lambda_{2,a}^- \right) \right\}
					+ \left\{ \lambda_{1,s}^+ - \lambda_{1,s}^- + i\left( \lambda_{2,s}^+ - \lambda_{2,s}^- \right) \right\}
			\end{align}
			と表現できて,命題\ref{prp:absolute_continuous_singular}により
			$\lambda_{1,a}^+ - \lambda_{1,a}^- + i\left( \lambda_{2,a}^+ - \lambda_{2,a}^- \right) \ll \mu$かつ
			$\lambda_{1,s}^+ - \lambda_{1,s}^- + i\left( \lambda_{2,s}^+ - \lambda_{2,s}^- \right) \perp \mu$が成り立つから
			これは$\lambda$のLebesgue分解である.また密度関数に対しても
			\begin{align}
				h = h_1^+ - h_1^- + i \left( h_2^+ - h_2^- \right)
			\end{align}
			とおけば$[h] \in \Lp{1}{\mu}$であり,
			\begin{align}
				\int h\ d\mu = \int h_1^+ - h_1^- + i \left( h_2^+ - h_2^- \right)\ d\mu = \lambda_{1,a}^+ - \lambda_{1,a}^- + i \left( \lambda_{2,a}^+ - \lambda_{2,a}^- \right) = \lambda_a
			\end{align}
			が成り立つ.
		}.
		$\lambda$を正値有限測度とする.$\mu$に対し補題(\ref{lem:lebesgue-radon-nikodym})の通りに関数$w$を構成し,
		\begin{align}
			\varphi(E) \coloneqq \lambda(E) + \int_E w(x)\ \mu(dx) \quad (\forall E \in \mathcal{M})
		\end{align}
		として$(X,\mathcal{M})$上の正値有限測度$\varphi$を作る.このとき任意の可測$\mathcal{M}/\borel{\R}$関数$f \geq 0$に対し
		\begin{align}
			\int f(x)\ \varphi(dx) = \int f(x)\ \lambda(dx) + \int f(x)w(x)\ \mu(dx)
			\label{eq:lebesgue_radon_nikodym}
		\end{align}
		が成り立つから,特に$[f] \in \Lp{2}{\varphi}$に対してはH\Ddot{o}lderの不等式より
		\begin{align}
			\left| \int_X f(x)\ \lambda(dx) \right| \leq \int_X |f(x)|\ \lambda(dx) \leq \Norm{[f]}{\Lp{2}{\varphi}}^{\frac{1}{2}} \varphi(X)^{\frac{1}{2}} < \infty
		\end{align}
		となり,写像
		\begin{align}
			F:\Lp{2}{\varphi} \ni [f] \longmapsto \int_X f(x)\ \lambda(dx)
		\end{align}
		は連続線型である.従ってReiszの表現定理より$[f]$に対し唯一つの$[g] \in \Lp{2}{\varphi}$が存在して
		\begin{align}
			F([f]) = \int_X f(x)\ \lambda(dx) = \int_X f(x)g(x)\ \varphi(dx) \quad (\forall [f] \in \Lp{2}{\varphi})
			\label{eq:lebesgue_radon_nikodym_2}
		\end{align}
		を満たす.(\refeq{eq:lebesgue_radon_nikodym_2})より$f = \defunc_E\ (E \in \mathcal{M})$とおけば
		\begin{align}
			\lambda(E) = \int_E g(x)\ \varphi(dx)
		\end{align}
		となるから,$\varphi(E) > 0$となる$E \in \mathcal{M}$に対し
		\begin{align}
			0 \leq \frac{1}{\varphi(E)} \int_E g\ d\varphi = \frac{\lambda(E)}{\varphi(E)} \leq 1
		\end{align}
		が成り立つ.書き直せば
		\begin{align}
			0 \leq \int_E g\ d\varphi \leq \varphi(E) \quad (\forall E \in \mathcal{M})
		\end{align}
		が成り立つ
		\footnote{
			$g$の$\varphi$に関する可積分性より,この関係は$\varphi(E) > 0$の場合に限らず成り立つ.
		}.
		これより$0 \leq g \leq 1\ \varphi-1.a.e.$が成り立つから,
		$g$を零集合上で修正して$0 \leq g(x) \leq 1\ (\forall x \in X)$とする.
		
		ここで式(\refeq{eq:lebesgue_radon_nikodym_2})を式(\refeq{eq:lebesgue_radon_nikodym})に適用すれば
		\begin{align}
			&\int f(x)\ \varphi(dx) = \int f(x)g(x)\ \varphi(dx) + \int f(x)w(x)\ \mu(dx) \\
			\Rightarrow &\int f(x)(1-g(x))\ \varphi(dx) = \int f(x)w(x)\ \mu(dx)
		\end{align}
		が成り立ち,$f = \defunc_E\ (E \in \mathcal{M})$として
		\begin{align}
			\int_E 1-g(x)\ \varphi(dx) = \int_E w(x)\ \mu(dx)
		\end{align}
		と表せる.
		\begin{align}
			A \coloneqq \{0 \leq g < 1\}, \quad B \coloneqq \{g = 1\}
		\end{align}
		に対して
		\begin{align}
			\lambda_a(E) \coloneqq \lambda(E \cap A), \quad \lambda_s(E) \coloneqq \lambda(E \cap B), \quad (\forall E \in \mathcal{M})
		\end{align}
		として$\lambda_a,\lambda_s$を定義すると,$\lambda_a$は$A$に,$\lambda_s$は$B$に集中し,
		また$A+B=X$より$\lambda = \lambda_a + \lambda_s$となる.
		\begin{align}
			0 = \int_B 1-g(x)\ \varphi(dx) = \int_B w(x)\ \mu(dx)
		\end{align}
		であることと$w > 0$により$\mu(B) = 0$となり$\lambda_s \perp \mu$が従う.また
		\begin{align}
			h(x) \coloneqq
			\begin{cases}
				w(x)/(1-g(x)) & (x \in A) \\
				0 & (x \in B)
			\end{cases}
		\end{align}
		とおけば,Lebesgueの収束定理より任意の$E \in \mathcal{M}$に対して
		\begin{align}
			\lambda_a(E) = \lambda(E \cap A) = \lim_{n \to \infty} \int_{E \cap A} (1 - g^{n+1})\ d\varphi 
			= \lim_{n \to \infty} \int_{E \cap A} (1 + g + \cdots + g^n)w\ d\mu = \int_E h\ d\mu 
		\end{align}
		が成り立つ.$\lambda_a(X) < \infty$であるから$h$の可積分性が従い,さらに$\lambda_a \ll \mu$であることも判る.
	\end{description}
	\QED
	\end{prf}